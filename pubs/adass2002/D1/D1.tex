\documentstyle[11pt,adassconf]{article}

\begin{document}

\paperID{D1}

\title{Starlink Software Developments}

\author{Martin Bly, David Giaretta, Malcolm Currie}
\affil{Starlink, Space Science and Technology Department,
Rutherford Appleton Laboratory, Chilton, Didcot, Oxon, OX11 0QX, UK}

\author{Mark Taylor}
\affil{Astrophysics group, H.\ H.\ Wills Physics Laboratory,
Tyndall Avenue, Bristol University, Bristol, BS8 1TL, UK}

\contact{Mark Taylor}
\email{m.b.taylor@bristol.ac.uk}

\paindex{Bly, M. }
\aindex{Giaretta, D. }
\aindex{Currie, M. }
\aindex{Taylor, M. }

\keywords{Starlink, software, grid, Java, Treeview, VOTable}

\begin{abstract}
Some current and upcoming software developments from Starlink were demonstrated.
These included invoking traditional Starlink applications via
web services, the current version of the ORAC-DR reduction pipeline,
and some new Java-based tools including Treeview, an
interactive explorer of hierarchical data structures.
\end{abstract}

\section{Introduction}

Starlink is continuing to serve UK astronomers in a era of change;
as well as supporting our established range of reduction and analysis
software we are providing new tools and ways of controlling existing
applications which fit in with the grid/virtual observatory paradigm
while remaining compatible with existing data and ways of working.
Key technologies for this work are Java, XML and SOAP.

This paper summarises new features of some existing packages and
describes some new Java-based tools.
Control of existing applications via web services is presented
in a separate paper (Taylor et al.\ 2002) and not covered further here.

Up to date details of all Starlink software, including information for
obtaining it, can be found at 
\htmladdnormallinkfoot{the Starlink web site}{http://www.starlink.ac.uk/}.

\section{New Features of Existing Software}

A number of packages were on display with major new facilities,
including the following:
\begin{description}

\item[CCDPACK] --- CCD data reduction package
   \begin{itemize}
   \item Graphical display and interactive registration are greatly improved
   \item Integrated support for reduction and registration of frames 
         from mosaic cameras has been added
   \end{itemize}

\item[GAIA] --- multi-purpose interactive image display and analysis tool
   \begin{itemize}
   \item Multi-extension FITS files can be displayed
   \item Linear polarization maps can be displayed
   \item Positions can be recorded from an image
   \end{itemize}

\item[KAPPA] --- generic analysis and display package
   \begin{itemize}
   \item After many years, version 1.0 is finally released!
   \item All tasks now use the Starlink NDF structure instead of
         the old IMAGE format
   \item All applications are now WCS-aware
   \item All graphics are now produced using native PGPLOT
   \item Graphics applications are no longer restricted to 8-bit displays
   \end{itemize}

\item[ORAC-DR] --- general purpose automatic data reduction pipeline
   \begin{itemize}
   \item More instruments are supported (Cavanagh et al., 2002)
   \item Data error handling is more accurate
   % \item Object masking is more efficient
   \item A graphical front end is provided
   \end{itemize}

\item[POLPACK] --- polarimetry data reduction package
   \begin{itemize}
   \item Support is now provided for spectro-polarimetry data
   \end{itemize}

\end{description}
Additionally a new package DATACUBE consisting of A-tasks, shell scripts
and a cookbook for IFU data reduction has been introduced.


\section{New Tools}

New Starlink software development is being done using Java.
We are developing an array data access layer called HDX/NDX
(Giaretta et al., 2002)
which permits network-transparent and (largely) portable 
access to data stored in a variety of forms.  In this way the 
application software does not need to worry about the source
(local or remote) or the format (FITS, NDF, XML) of a data array.

New tools written using this framework, 
demonstrated at the conference, are described in the following sections.
All are undergoing active development and will offer new facilities
in the near future.

\subsection{Treeview}
\label{sec:treeview}

Treeview is an interactive graphical browser for hierarchical structures
which knows about astronomical data formats.
It resembles Windows Explorer in representing a hierarchical structure
as a tree of nodes which the user may expand or collapse, 
but as well as displaying directory trees
containing files, it can look inside files to display
their structure in a hierarchical fashion to whatever depth 
makes sense.  The GUI has a single window with two panels:
one displays the tree-like hierarchical structure being browsed,
and the other offers a variety of views of a selected node according
to what kind of object it is.  For instance you might open up a directory
containing a number of FITS files, open one FITS file to find that it
contains a BINTABLE and four IMAGE extensions, select one of the
images, and then examine its header cards or WCS information, 
or view it as an image or a table of pixel values.
Operation is extremely intuitive.
\begin{figure}
\plotone{D1_1.eps}
\caption{Treeview displaying various files in a directory}
\end{figure}

Treeview understands many kinds of structure, and more are being added.
The current list includes filesystems, Starlink NDF and HDS files, 
FITS files, zip/jar archives, XML documents, 
AST FrameSets, and Starlink NDX and HDX structures.
More are being added; in particular 
VOTable support will be available very soon.

It is easy to add new node types and new views of existing ones, 
and explicit support for users to extend functionality by local 
provision of class libraries is planned in the near future.

The various data views which Treeview can provide of a selected node
are confined to a single window and intended to be useful for a quick look.
More detailed analysis of data, for instance 
line fitting of spectra
or
photometry on images 
is best delegated to external applications.
For nodes of a suitable type, Treeview allows invocation of 
external applications to do this; currently 
SPLAT (section \ref{sec:splat}) for spectra
and
SoG (section \ref{sec:sog}) for images.
Communication between the applications is done via SOAP, 
serialising data array references using NDX.
This permits very flexible deployment of resources; for instance the 
machine running Treeview, the machine running the data analysis 
application and the machine hosting the data might all be 
geographically separated.

Treeview itself is written entirely in Java, but for WCS-related tasks
and display of Starlink HDS/NDF files it uses some platform-specific 
shared library code (JNI wraps of the Starlink AST and HDS
libraries respectively).  These are currently available for 
Linux, Solaris and Tru64 Unix, while Windows versions should be
available shortly.
However, if these libraries 
are not detected at runtime, Treeview will run happily with some 
features unavailable.  This means that it can
run anywhere that a J2SE1.4 Java Runtime Environment is installed, 
which makes it highly portable.
Installation is simply a case of unpacking a single tar or zip archive.

\subsection{SPLAT}
\label{sec:splat}

SPLAT (SPectraL Analysis Tool) is an interactive Java application for 
display, comparison and analysis of astronomical spectra.
Spectra can be viewed singly, together or as an animated sequence;
the display is highly configurable and publication quality plots
can be produced.
Facilities are provided for fitting spectral lines using a
variety of profile models, 
fitting background levels using polynomials,
and doing spectrum arithmetic.
The program can be controlled from the command line, or using sockets or 
via SOAP messages.
\begin{figure}
\plotone{D1_2.eps}
\caption{The SPLAT spectral analysis tool}
\end{figure}

\subsection{SoG}
\label{sec:sog}

SoG (``Son of Gaia'') is an extension of JSkycat which can display
NDX images, and is planned in due course to provide toolkits interfacing
with other software.
By taking advantage of grid technologies it will be able to provide
similar functionality to the existing Gaia tool, while offering
platform independence and network transparency.


\begin{references}
\reference Cavanagh, B., Hirst, P., Currie, M.\ J., Jenness, T., Economou, F.,
           Ryder, S. \& Todd, S.\ P. 2002, \adassxii, \paperref{P6.4}
% \reference Draper, P.\ W.\ 2000, \adassix, 615
% \reference Economou, F., Bridger, A., Wright, G.\ S., Jenness, T., 
%            Currie, M.\ J. \& Adamson, A.\ 1999, \adassviii
\reference Giaretta, D.\ L., Taylor, M.\ B., Draper, P.\ W., Gray, N.\ G. \&
           McIlwrath B.\ K.\ 2002, \adassxii, \paperref{O7.4}
\reference Taylor, M.\ B., Platon, R.\ T., Chipperfield, A.\ J.,
           Draper, P.\ W. \& Giaretta, D.\ L.\ 2002, \adassxii, \paperref{P2.5}
\end{references}

\end{document}
