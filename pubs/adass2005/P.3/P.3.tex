\documentclass[11pt,twoside]{article}  % Leave intact
\usepackage{adassconf}

% $Revision$

% If you have the old LaTeX 2.09, and not the current LaTeX2e, comment
% out the \documentclass and \usepackage lines above and uncomment
% the following:

%\documentstyle[11pt,twoside,adassconf]{article}

\begin{document}   % Leave intact

%-----------------------------------------------------------------------
%			    Paper ID Code
%-----------------------------------------------------------------------
% Enter the proper paper identification code.  The ID code for your
% paper is the session number associated with your presentation as
% published in the official conference proceedings.  You can           
% find this number locating your abstract in the printed proceedings
% that you received at the meeting or on-line at the conference web
% site; the ID code is the letter/number sequence proceeding the title 
% of your presentation. 
%
% This will not appear in your paper; however, it allows different
% papers in the proceedings to cross-reference each other.  Note that
% you should only have one \paperID, and it should not include a
% trailing period.
%
% EXAMPLE: \paperID{O4-1}
% EXAMPLE: \paperID{P7-7}
%

\paperID{P.3}

%-----------------------------------------------------------------------
%		            Paper Title 
%-----------------------------------------------------------------------
% Enter the title of the paper.
%
% EXAMPLE: \title{A Breakthrough in Astronomical Software Development}
% 
% If your title is so long as to fill the page header when you print it,
% then please supply a short form as a \titlemark.
%
% EXAMPLE: 
%  \title{Rapid Development for Distributed Computing, with Implications
%         for the Virtual Observatory}
%  \titlemark{Rapid Development for Distributed Computing}
%

\title{STILTS - A Package for Command-Line Processing of Tabular Data}
%\titlemark{ }

%-----------------------------------------------------------------------
%		          Authors of Paper
%-----------------------------------------------------------------------
% Enter the authors followed by their affiliations.  The \author and
% \affil commands may appear multiple times as necessary (see example
% below).  List each author by giving the first name or initials first
% followed by the last name.  Authors with the same affiliations
% should grouped together. 
%
% EXAMPLE: \author{Raymond Plante, Doug Roberts, 
%                  R.\ M.\ Crutcher\altaffilmark{1}}
%          \affil{National Center for Supercomputing Applications, 
%                 University of Illinois Urbana-Champaign, Urbana, IL
%                 61801}
%          \author{Tom Troland}
%          \affil{University of Kentucky}
%
%          \altaffiltext{1}{Astronomy Department, UIUC}
%
% In this example, the first three authors, "Plante", "Roberts", and
% "Crutcher" are affiliated with "NCSA".  "Crutcher" has an alternate 
% affiliation with the "Astronomy Department".  The fourth author,
% "Troland", is affiliated with "University of Kentucky"

\author{Mark B.\ Taylor}
\affil{H H Wills Physics Laboratory,
       Tyndall Avenue,
       Bristol University,
       Bristol, UK}

%-----------------------------------------------------------------------
%			 Contact Information
%-----------------------------------------------------------------------
% This information will not appear in the paper but will be used by
% the editors in case you need to be contacted concerning your
% submission.  Enter your name as the contact along with your email
% address.
% 
% EXAMPLE:  \contact{Dennis Crabtree}
%           \email{crabtree@cfht.hawaii.edu}
%

\contact{Mark Taylor}
\email{m.b.taylor@bristol.ac.uk}

%-----------------------------------------------------------------------
%		      Author Index Specification
%-----------------------------------------------------------------------
% Specify how each author name should appear in the author index.  The 
% \paindex{ } should be used to indicate the primary author, and the
% \aindex for all other co-authors.  You MUST use the following
% syntax: 
%
% SYNTAX:  \aindex{Lastname, F. M.}
% 
% where F is the first initial and M is the second initial (if
% used).  This guarantees that authors that appear in multiple papers
% will appear only once in the author index.  
%
% EXAMPLE: \paindex{Crabtree, D.}
%          \aindex{Manset, N.}        
%          \aindex{Veillet, C.}        
%
% NOTE: this information is also used to build the author list that
% appears in the table of contents.  Authors will be listed in the order
% of the \paindex and \aindex commmands.
%

\paindex{Taylor, M. B.}
%\aindex{ }     % Remove this line if there is only one author

%-----------------------------------------------------------------------
%		      Author list for page header	
%-----------------------------------------------------------------------
% Please supply a list of author last names for the page header. in
% one of these formats:
%
% EXAMPLES:
% \authormark{Lastname}
% \authormark{Lastname1 \& Lastname2}
% \authormark{Lastname1, Lastname2, ... \& LastnameN}
% \authormark{Lastname et al.}
%
% Use the "et al." form in the case of seven or more authors, or if
% the preferred form is too long to fit in the header.

\authormark{Taylor}

%-----------------------------------------------------------------------
%			Subject Index keywords
%-----------------------------------------------------------------------
% Enter a comma separated list of up to 6 keywords describing your
% paper.  These will NOT be printed as part of your paper; however,
% they will be used to generate the subject index for the proceedings.
% There is no standard list; however, you can consult the indices
% for past proceedings (http://adass.org/adass/proceedings/).
%
% EXAMPLE:  \keywords{visualization, astronomy: radio, parallel
%                     computing, AIPS++, Galactic Center}
%
% In this example, the author noticed that "radio astronomy" appeared
% in the ADASS VII Index as "astronomy" being the major keyword and
% "radio" as the minor keyword.  The colon is used to introduce another
% level into the index.

\keywords{table, catalogue, VOTable, virtual observatory, 
          software: applications}

%-----------------------------------------------------------------------
%			       Abstract
%-----------------------------------------------------------------------
% Type abstract in the space below.  Consult the User Guide and Latex
% Information file for a list of supported macros (e.g. for typesetting 
% special symbols). Do not leave a blank line between \begin{abstract} 
% and the start of your text.

\begin{abstract}          % Leave intact
STILTS, the STIL Tool Set, is a set of non-interactive tools for
manipulation of tables such as astronomical object catalogues.
It can read and write data in many formats, including VOTable, FITS,
relational databases and ASCII.
Facilities provided include 
table format conversion,
row selection and sorting,
column creation and rearrangement,
coordinate conversion,
metadata manipulation and display,
flexible cross-matching,
per-row and statistical calculations and
VOTable validation.
STILTS is based on the Starlink Tables Infrastructure Library,
which also underlies the interactive table-analysis tool TOPCAT,
and can be considered its non-interactive counterpart,
providing many of the same features in a form which is 
suitable for headless, batch or scripted environments.
Uses include data manipulation from the desktop
or as part of server-based workflows or query operations.
The package is portable (Java), open source, fully documented,
efficient and scalable; in particular it is designed for use with
large, and for many purposes arbitrarily large, tables.
\end{abstract}

%-----------------------------------------------------------------------
%			      Main Body
%-----------------------------------------------------------------------
% Place the text for the main body of the paper here.  You should use
% the \section command to label the various sections; use of
% \subsection is optional.  Significant words in section titles should
% be capitalized.  Sections and subsections will be numbered
% automatically. 
%
% EXAMPLE:  \section{Introduction}
%           ...
%           \subsection{Our View of the World}
%           ...
%           \section{A New Approach}
%
% It is recommended that you look at the sample papers, sample1.tex
% and sample2.tex, for examples for formatting references, footnotes,
% figures, equations, html links, lists, and other special features.  

\section{Introduction}

Tabular data, especially in the form of object catalogues, 
are common in astronomy, and much of the data and processing 
in the virtual observatory in particular is dedicated to tables
in various forms.  The IVOA-endorsed VOTable standard is important
in this context as defining a metadata-rich transport and storage 
format for tables; however it has not and will not displace
FITS files on the one hand, and relational databases on the
other, not to mention a host of ad-hoc ASCII-based formats,
as a repository for catalogues and other tables.

\htmladdnormallinkfoot{STILTS}{http://www.starlink.ac.uk/stilts/} 
is a new package of command-line tools for performing versatile
and powerful manipulations on tables in a number of formats including
those mentioned above.
It provides some tools for generic (format-independent) table
processing, and some specific to VOTables.

The package is based on the Starlink Tables Infrastructure Library 
(\htmladdnormallinkfoot{STIL}{http://www.starlink.ac.uk/stil/})
a table I/O and processing library whose first public release was
early in 2004.  Among the features of STIL are fully compliant 
VOTable parsing (it was the first parser to handle
{\sc binary} and {\sc fits} as well
as {\sc tabledata}-format VOTables),
a streaming model which supports efficient 
processing of arbitrary length tables
in limited memory
and a pluggable data handling architecture which is easily adapted for 
reading or writing new table formats.
STILTS benefits from these features.

STIL was originally developed as the table handling engine for
\htmladdnormallinkfoot{TOPCAT}{http://www.starlink.ac.uk/topcat/},
which is an interactive graphical viewing and analysis package
for tabular data.  TOPCAT is becoming widely used both within
and without the context of the Virtual Observatory, and is
good for interactive investigation of the properties of
one or more tables, providing facilities for data visualisation, 
row selection, column and statistical calculations, 
column rearrangements, data and metadata 
manipulation, sorting, cross-matching, joining and so on
(Taylor 2005).
However it is not scriptable or otherwise controllable except 
from within a GUI, and so STILTS has been developed primarily 
as a non-graphical counterpart to TOPCAT.  
Usage scenarios in which such non-GUI tools are required include
performing the same operation on multiple tables without user intervention,
and processing on a headless server, either as part of
a user-specified batch-type workflow or to produce data 
for delivery to the client of a web service.

STILTS, like STIL and TOPCAT, is written in pure Java and available
under the GNU General Public License.

\section{Generic Table Processing Tools}

STILTS provides four commands for generic table processing.
Each of these conceptually takes one or more tables as input and
produces a table as output.
The input tables can be in any of the supported formats.
The table generated as output can either be written
to a stream or to disk in one of the supported formats, or 
can be operated on in some other way, such as having statistics
calculated on it or sending it for display in TOPCAT.

The input/output tables can be operated on using `filter'-like 
operations such as sorting, various kinds of row selection and sampling,
column calculations, metadata manipulation and so on.
These filters operate analogously to the filters in a Unix pipeline, 
except that what is flowing from one to the next is a
stream of table data and metadata rather than a sequence of bytes.
The way it is implemented means that the table at each stage is 
`virtual', so that usually data not required for the 
output are never actually obtained from the input.  This means that
operation can be very efficient, and in particular much better
than writing the results of each intermediate processing stages to disk.

The individual commands are described in the following subsections.

\subsection{tpipe}

{\tt tpipe} performs general purpose table-processing pipeline operations.
Processing steps (`filters'), which can be combined freely,
include coordinate conversion, row selection, data sampling,
metadata display and editing, column calculations, row sorting
and blank value substitution amongst others.
A powerful but intuitive expression language
(based on Java) is provided for specifying algebraic expressions.

Here is an example which performs bad value substitution 
(999$\rightarrow${\sc null}) in some of the columns, 
selects only those rows in a given range of proper motions
and with a given value in one column, adds a new
column calculated from two existing ones, and sorts the
result according to redshift.
\begin{verbatim}
   stilts tpipe in=data.cat.gz ifmt=ascii out=data.fits \
      cmd='badval 999 *MAG' \
      cmd='select OBJ_CLASS==3 && PMRA*PMRA+PMDEC*PMDEC<0.5' \
      cmd='addcol I_V IMAG-VMAG' \
      cmd='sort -down Z_ABS'
\end{verbatim}


\subsection{tcopy}

{\tt tcopy} is used to convert between table formats.
This doesn't actually require any specialist processing beyond
the normal I/O facilities of STILTS, and in fact {\tt tcopy}
is just a simplified form of {\tt tpipe}, provided for
convenience.  Supported I/O formats
include FITS, VOTable, relational databases (via SQL queries),
plain ASCII, and comma-separated values.
Input can be from a disk file, a URL or a stream, 
and data compression (gzip, bip2, Unix compress)
is detected and handled automatically.

An example invocation might look like this:
\begin{verbatim}
   stilts tcopy ifmt=votable ofmt=fits in=cat.vot.gz out=cat.fit 
\end{verbatim}


\subsection{tmatch2}

{\tt tmatch2} is a crossmatching tool for finding rows which match
between two tables
(single-table and multi-table crossmatchers will be introduced 
in the future).
The algorithm is generic and extensible 
and can perform approximate or exact
matches in various different physical or notional parameter spaces
of arbitrary dimensionality.  The most common case for astronomers 
is matching on sky position with a global or per-row maximum angular
separation,
but there are other possibilities for the matching criteria,
for instance proximity in two- or three-dimensional Cartesian space,
or requiring proximity in flux value as well as sky position.
Either all matches or only the best match can be retained, and the
form of the joined table to be output is configurable.
Calculation time scales as approximately $O(N \ln N)$, where $N$ is
the sum of number of rows in both tables, making it
suitable for medium-sized data sets
(e.g.\ a sky match of two $10^5$-row tables takes of the order of one
minute on a current laptop).

The following locates all matches within 3 arcsec and 0.5 blue
magnitudes, returning only matched pairs:
\begin{verbatim}
   stilts tmatch2 in1=mgc.fits in2=6dfgs.xml out=matched.fits \
      join=1and2 find=all matcher='sky+1d' params='3 0.5' \
      values1='ra dec bmag' values2='RA2000 DE2000 B_MAG' 
\end{verbatim}

\subsection{tcat}

{\tt tcat} simply joins multiple input tables top to bottom, giving
one output table with a row count equal to the sum of the input table
row counts.
The input column types have to be compatible with each other;
if they're not you can introduce preprocessing 
filters in the command to make them so.

This example modifies two input tables so that they contain only 
identifier and galactic longitude and latitude columns 
before concatenating them to give a three-column output:
\begin{verbatim}
   stilts tcat in1=t1.fits in2=t2.fits out=t.fits \
      cmd1='addskycoords fk5 galactic RA DEC GLON GLAT' \
      cmd1='keepcols "ID GLON GLAT"' \
      cmd2='keepcols "index gal_long gal_lat"'
\end{verbatim}

\section{VOTable-Specific Tools}

Although the generic table tools listed in the previous section
can operate on VOTables as well as other formats, 
some of the detailed structure permitted in VOTable documents 
may be lost as a consequence of interpreting them in terms of
STIL's abstract model of what a table is (some per-table metadata,
some per-column metadata, and a sequence of rows).
The two tools listed below deal with VOTable document structures
directly.

\subsection{votcopy}

The VOTable standard defines three encodings in which the
data content of each table element can be represented:
{\sc tabledata}, {\sc binary} and {\sc fits}.
Much existing VOTable software is able to produce/consume only the
first of these, although being text-based it is less efficient 
than the others
for transmission or processing.
{\tt votcopy} can take a VOTable document and convert the data
encodings freely between these formats, leaving the rest of
the XML structure untouched.

\subsection{votlint}

It is easy to make mistakes when writing VOTable documents
and often hard to spot them except by seeing that software 
behaves unexpectedly;
many, perhaps most, VOTable documents at large
exhibit errors of varying degrees of seriousness.
{\tt votlint} is a VOTable validator designed to be used
by authors and users of VOTable-producing software to 
identify such errors and thereby to ensure that the documents they
publish conform to the relevant standards.
The tool simply examines a VOTable document and emits 
warnings about erroneous or questionable usages,
which it's then up to the user to address.
It provides much more rigorous tests than simply validating
against a DTD or schema can offer.

\acknowledgements

The bulk of the work for STILTS was performed under the now-terminated
Starlink Project.  Development and support is continuing,
funded by the UK Particle Physics and Astronomy Research Council.



%-----------------------------------------------------------------------
%			      References
%-----------------------------------------------------------------------
% List your references below within the reference environment
% (i.e. between the \begin{references} and \end{references} tags).
% Each new reference should begin with a \reference command which sets
% up the proper indentation.  Observe the following order when listing
% bibliographical information for each reference:  author name(s),
% publication year, journal name, volume, and page number for
% articles.  Note that many journal names are available as macros; see
% the User Guide listing "macro-ized" journals.   
%
% EXAMPLE:  \reference Hagiwara, K., \& Zeppenfeld, D.\  1986, 
%                Nucl.Phys., 274, 1
%           \reference H\'enon, M.\  1961, Ann.d'Ap., 24, 369
%           \reference King, I.\ R.\  1966, \aj, 71, 276
%           \reference King, I.\ R.\  1975, in Dynamics of Stellar 
%                Systems, ed.\ A.\ Hayli (Dordrecht: Reidel), 99
%           \reference Tody, D.\  1998, \adassvii, 146
%           \reference Zacharias, N.\ \& Zacharias, M.\ 2003,
%                \adassxii, \paperref{P7.6}
% 
% Note the following tricks used in the example above:
%
%   o  \& is used to format an ampersand symbol (&).
%   o  \'e puts an accent agu over the letter e.  See the User Guide
%      and the sample files for details on formatting special
%      characters.  
%   o  "\ " after a period prevents LaTeX from interpreting the period 
%      as an end of a sentence.
%   o  \aj is a macro that expands to "Astron. J."  See the User Guide
%      for a full list of journal macros
%   o  \adassvii is a macro that expands to the full title, editor,
%      and publishing information for the ADASS VII conference
%      proceedings.  Such macros are defined for ADASS conferences I
%      through XI.
%   o  When referencing a paper in the current volume, use the
%      \adassxii and \paperref macros.  The argument to \paperref is
%      the paper ID code for the paper you are referencing.  See the 
%      note in the "Paper ID Code" section above for details on how to 
%      determine the paper ID code for the paper you reference.  
%
\begin{references}
\reference Taylor, M.\ B. 2005, \adassxiv, \paperref{FM-3}
\end{references}

% Do not place any material after the references section

\end{document}  % Leave intact
