%ADASS_PROCEEDINGS_FORM%%%%%%%%%%%%%%%%%%%%%%%%%%%%%%%%%
%
% TEMPLATE.TEX -- ADASS Conference Proceedings template.
%
% Use this template to create your proceedings paper in LaTeX format
% by following the instructions given below.  Much of the input will
% be enclosed by braces (i.e., { }).  The percent sign, "%", denotes
% the start of a comment; text after it will be ignored by LaTeX.
% You might also notice in some of the examples below the use of "\ "
% after a period; this prevents LaTeX from interpreting the period as
% the end of a sentence and putting extra space after it.
%
% You should check your paper by processing it with LaTeX.  For
% details about how to run LaTeX as well as how to print out the User
% Guide, consult the README file.  You should also consult the sample
% LaTeX papers, sample1.tex and sample2.tex, for examples of including
% figures, html links, special symbols, and other advanced features.
%
%%%%%%%%%%%%%%%%%%%%%%%%%%%%%%%%%%%%%%%%%%%%%%%%%%
% Note that the primary style file is that from the ASP Conf. Series; ADASS style
% elements are included by an additional \usepackage. You may use other
% _standard_ packages if needed, such as lscape, psfig, epsf, and graphicx,
% although these packages may already be installed on your system.
%
\documentclass[11pt,twoside]{article}  % Leave intact
\usepackage{asp2006}
\usepackage{adassconf}

% Set counters for footnotes and sectioning, which is needed when
% constructing the full volume of all papers.
% DO NOT DELETE.
\setcounter{equation}{0}
\setcounter{figure}{0}
\setcounter{footnote}{0}
\setcounter{section}{0}
\setcounter{table}{0}

\begin{document}   % Leave intact

%-----------------------------------------------------------------------
%			    Paper ID Code
%-----------------------------------------------------------------------
% Enter the proper paper identification code.  The ID code for your paper
% is the session number associated with your presentation as published
% in the official conference proceedings.  You can find this number by
% locating your abstract in the printed proceedings that you received
% at the meeting, or on-line at the conference web site.
%
% This identifier will not appear in your paper; however, it allows different
% papers in the proceedings to cross-reference each other.  Note that
% you should only have one \paperID, and it should not include a
% trailing period.
%
% EXAMPLE: \paperID{O4.1}
% EXAMPLE: \paperID{P2.7}

\paperID{P2.7}

%-----------------------------------------------------------------------
%		            Paper Title
%-----------------------------------------------------------------------
% Enter the title of the paper.
%
% EXAMPLE: \title{A Breakthrough in Astronomical Software Development}

\title{GAIA-3D: Volume visualisation of data-cubes}

%-----------------------------------------------------------------------
%          Short Title & Author list for page headers
%-----------------------------------------------------------------------
% Please supply the author list and the title (abbreviated if necessary) as
% arguments to \markboth.
%
% The author last names for the page header must appear in one of
% these formats:
%
% EXAMPLES:
%     LASTNAME
%     LASTNAME1 and LASTNAME2
%     LASTNAME1, LASTNAME2, and LASTNAME3
%     LASTNAME et al.
%
% Use the "et al." form in the case of four or more authors.
%
% If the title is too long to fit in the header, shorten it:
%
% EXAMPLE: change
%    Rapid Development for Distributed Computing, with Implications for the Virtual Observatory
% to:
%    Rapid Development for Distributed Computing

\markboth{Draper, Berry, Jenness, Economou and Currie }
         {GAIA-3D: Volume visualisation of data-cubes}

%-----------------------------------------------------------------------
%		          Authors of Paper
%-----------------------------------------------------------------------
% Enter the authors followed by their affiliations.  The \author and
% \affil commands may appear multiple times as necessary.  List each
% author by giving the first name or initials first followed by the
% last name. Do not include street addresses and postal codes, but
% do include the country name or abbreviation.
%
% If the list of authors is lengthy and there are several institutional
% affiliations, you can save space by using the \altaffilmark and \altaffiltext
% commands in place of the \affil command.
%
% EXAMPLE:
%      \author{Raymond Plante, Doug Roberts,
%                  R.\ M.\ Crutcher\altaffilmark{1}}
%      \affil{National Center for Supercomputing Applications,
%                 University of Illinois, Urbana, IL, USA}
%      \author{Tom Troland}
%      \affil{University of Kentucky, Lexington, KY, USA}
%
%      \altaffiltext{1}{Astronomy Department, UIUC}
%
% In this example, the first three authors, "Plante", "Roberts", and
% "Crutcher" are affiliated with "NCSA".  "Crutcher" has an alternate
% affiliation with the "Astronomy Department".  The fourth author,
% "Troland", is affiliated with "University of Kentucky"

\author{Peter W.\ Draper}
\affil{Department of Physics, Durham University, Durham, UK}

\author{David S.\ Berry}
\affil{Centre for Astrophysics, University of Central Lancashire,
       Lancs, UK}

\author{Tim Jenness \& Frossie Economou}
\affil{Joint Astronomy Centre, Hilo, Hawaii, USA}

\author{Malcolm J.\ Currie}
\affil{Space Science and Technology Department, 
Rutherford Appleton Laboratory, Oxfordshire, UK}

%-----------------------------------------------------------------------
%			 Contact Information
%-----------------------------------------------------------------------
% This information will not appear in the paper but will be used by
% the editors in case you need to be contacted concerning your
% submission.  Enter your name as the contact along with your email
% address.
%
% EXAMPLE:  \contact{Dennis Crabtree}
%           \email{crabtree@cfht.hawaii.edu}

\contact{Peter Draper}
\email{p.w.draper@durham.ac.uk}

%-----------------------------------------------------------------------
%		      Author Index Specification
%-----------------------------------------------------------------------
% Specify how each author name should appear in the author index.  The
% \paindex{ } should be used to indicate the primary author, and the
% \aindex for all other co-authors.  You MUST use the following
% syntax:
%
% SYNTAX:  \aindex{Lastname, F.~M.}
%
% where F is the first initial and M is the second initial (if used). Please
% ensure that there are no extraneous spaces anywhere within the command
% argument. This guarantees that authors that appear in multiple papers
% will appear only once in the author index. Authors must be listed in the order
% of the \paindex and \aindex commmands.
%
% EXAMPLE: \paindex{Crabtree, D.}
%          \aindex{Manset, N.}
%          \aindex{Veillet, C.}

\paindex{Draper, P.~W.}
\aindex{Berry, D.~S.}
\aindex{Jenness, T.}
\aindex{Currie, M.~J.}
\aindex{Economou, F.}

%-----------------------------------------------------------------------
%			Subject Index keywords
%-----------------------------------------------------------------------
% Enter up to 6 keywords that are relevant to the topic of your paper.  These
% will NOT be printed as part of your paper; however, they will guide the creation
% of the subject index for the proceedings.  Please use entries from the
% standard list where possible, which can be found in the index for the
% ADASS XVI proceedings. Separate topics from sub-topics with an exclamation
% point (!).
%
% EXAMPLE:  \keywords{astronomy!radio, computing!grid, data management!workflows,
%     instrumentation!control}

\keywords{visualization, software!applications, software!interfaces,
          astronomy!sub-millimetre, interfaces!human-computer}

%-----------------------------------------------------------------------
%			       Abstract
%-----------------------------------------------------------------------
% Type abstract in the space below.  Consult the User Guide and Latex
% Information file for a list of supported macros (e.g. for typesetting
% special symbols). Do not leave a blank line between \begin{abstract}
% and the start of your text.

\begin{abstract}          % Leave intact
This paper shows some of the 3D visualisation extensions to the Starlink GAIA
application in operation.
\end{abstract}

%-----------------------------------------------------------------------
%			      Main Body
%-----------------------------------------------------------------------
% Place the text for the main body of the paper here.  You should use
% the \section command to label the various sections; use of
% \subsection is optional.  Significant words in section titles should
% be capitalized.  Sections and subsections will be numbered
% automatically.
%
% EXAMPLE:  \section{Introduction}
%           ...
%           \subsection{Our View of the World}
%           ...
%           \section{A New Approach}
%
% It is recommended that you look at the sample paper sample2.tex
% for examples of formatting references, footnotes, figures, equations,
% html links, lists, and other features.

\section{Introduction}
The Starlink \htmladdnormallink{GAIA}{http://www.starlink.ac.uk/gaia}
application (Draper et al., 2007) is being extended to provide 3D volume and
iso-surface visualisations of data-cubes. It is expected that this work will
result in a simple integrated system, delivered to astronomers desktops, that
will limit the initial knowledge necessary to exploit 3D visualisation for
data exploration and analysis.
The review by Leech \& Jenness (2005) presented the rationale for this work.
For a more-recent perspective see Goodman et al., (2008).
This paper shows some of the work done so far, using data from the JCMT
ACSIS/HARP instrument (Smith et al., 2003).

\section{Iso-surfaces and volume rendering}
Figure~\ref{P2.7-fig1} shows a new toolbox displaying a JCMT data-cube of the
central parts of the Orion nebula, using iso-surface contours at various
levels and with various opacities to render the volume. Using opacities allows
you to see different depths within the outer volumes. To increase efficiency
the data can be directly shared with GAIA unless replacement of bad/blank
values is required (or some other transformational processing). Interaction
with the 3D scene uses a series of mouse gestures, or the keyboard can be used
for finer control. The cardinal directions of the data's world coordinates (in
this case RA, Dec and Radio velocity) can be permanently displayed in the
current view.

\begin{figure}
\epsscale{0.75}\plotone{P2.7_1.eps}
\caption{Iso-surface toolbox with cardinal axes also displayed.}
\label{P2.7-fig1}
\end{figure}

Figure~\ref{P2.7-fig2} shows a second toolbox performing a volume rendering of
the same dataset. The colour transfer function is a simple mapping between two
selected colours for the given data range. As volume rendering is sensitive to
the presence of bad data values any bad/blank values have been replaced with
the value zero.

\begin{figure}
\epsscale{0.75}\plotone{P2.7_2.eps}
\caption{Volume rendering toolbox with cardinal axes also displayed. Bad
pixels have been replaced by zero and a simple two-colour transfer function is
used for colouring.}
\label{P2.7-fig2}
\end{figure}

\section{Interaction with GAIA}
GAIA currently provides cube slicing and extraction of a point spectrum. Both
these elements can now be visualised, as a plane and a line respectively, and
moved using the existing controls, or by interacting with their actors in the
visualised volume. When dragging over the plane a readout of the coordinates
and data value is provided. This is shown in Figure~\ref{P2.7-fig3}.

\begin{figure}
\epsscale{0.75}\plotone{P2.7_3.eps}
\caption{Iso-surface toolbox interacting with GAIA. Note the readout of RA,
Dec, Radio velocity and data value, and the rendering of the current image
plane and spectral extraction line.}
\label{P2.7-fig3}
\end{figure}

\section {Astronomy support}
GAIA-3D can process data in Starlink NDF and FITS formats. Coordinates are
handled by the \htmladdnormallink{AST}{http://www.starlink.ac.uk/ast} library
(Berry 2008), which supports all FITS-WCS celestial and spectral co-ordinate
systems. 

AST also provides functions for drawing coordinate grids and axes in 2 and
3D. These are used by GAIA-3D to draw annotated axes with labelling in the
familiar sexagesimal format. You can see one possible way of displaying these,
shown in a print rather than a screen capture, in Figure~\ref{P2.7-fig4}.

\begin{figure}
\epsscale{0.5}\plotone{P2.7_4.eps}
\caption{A print from the iso-surface toolbox showing the use of 
AST library annotated axes.}
\label{P2.7-fig4}
\end{figure}

\section{Future work}
It is expected that the eventual
\htmladdnormallink{release}{http://starlink.jach.hawaii.edu} of GAIA-3D will
include additional facilities like being able to compare data-cubes, to work on
subsets and display region extracted spectra. It is also an aim to visualise
and interact with the results of segmenting the volume into significant
clumps, as produced by the
\htmladdnormallink{CUPID}{http://www.starlink.ac.uk/~dsb/cupid} application
(Berry et al., 2007).

\section{Acknowledgements}
The 3D rendering uses the Visualization Toolkit,
\htmladdnormallink{VTK}{http://www.vtk.org}, and the Tool Command Language,
\htmladdnormallink{TCL}{http://www.tcl.tk}. Work on GAIA is supported by the
Science and Technology Facilities Council for the Joint Astronomy Centre,
Hawaii.

%-----------------------------------------------------------------------
%			      References
%-----------------------------------------------------------------------
% List your references below within the reference environment
% (i.e. between the \begin{references} and \end{references} tags).
% Each new reference should begin with a \reference command which sets
% up the proper indentation.
%    NOTE: all citations in the text _must_ have a corresponding entry in
%    the reference list, and all references must be cited in the text.
%
% Observe the following order when listing bibliographical
% information for each reference:  author name(s), publication
% year, journal name, volume, and page number for articles.
% URLs to the reference may be given either in-line, or as a footnote.
% Note that many journal names are available as macros; see
% the User Guide for a listing "macro-ized" journals.
%
% EXAMPLES:
% Reference to a Journal article:
%     \reference Cornwell, T.\ J.\ 1988, \aap, 202, 316
%
% Journal paper with more than 7 authors;
%     \reference Hanisch, R.\ et al.\ 2001, \aap, 376, 359
%
% Reference to an SPIE paper:
%     \reference Noordam, J.~E.\ 2004, Proc.\ SPIE, 5489, 817
%
% Reference to a contribution to a proceedings (not ADASS)
%     \reference Schmitz, M., Helou, G., Dubois, P., LaGue, C., Madore,B., Corwin, H.~G., Jr.,
%          \& Lesteven, S.\ 1995, in Information \& On-Line Data in Astronomy,
%          ed.\ D.\ Egret \& M.~A.\ Albrecht (Dordrecht: Kluwer Academic Publishers), 259
%
% Reference to a paper in an earlier ADASS proceedings:
%     \reference Kantor, J., et al.\ 2007, \adassvii, 3
%
% Reference to a paper in the current ADASS:
%     \reference Hanisch, R.~J.\ 2008, \adassxvii, \paperref{O1.3}
%
% Reference to a book:
%     \reference Jacobson, I.\ Booch, G., \& Rumbaugh, J.\ 1999,
%            The Unified Software Development Process (Reading, MA: Addison-Wesley)
%
% Reference to a thesis:
%     \reference Gering, D.\ 1999, Master's Thesis, Massachusetts Institute of Technology
%
% Reference to a purely on-line resource:
%     \reference Staveley-Smith, L.\ 2006, ATNF SKA Memo~6, http://www.atnf.csiro.au/ska
%
% Note the following tricks used in the example above:
%
%   o  \& is used to format an ampersand symbol (&).
%   o  \'e puts an accent agu over the letter e.  See the User Guide
%      and the sample files for details on formatting special
%      characters.
%   o  "\ " after a period prevents LaTeX from interpreting the period
%      as an end of a sentence.
%   o  \aj is a macro that expands to "Astron. J."  See the User Guide
%      for a full list of journal macros
%   o  \adassvii is a macro that expands to the full title, editor,
%      and publishing information for the ADASS VII conference
%      proceedings.  Such macros are defined for ADASS conferences I
%      through XVI.
%   o  When referencing a paper in the current volume, use the
%      \adassxvii and \paperref macros.  The argument to \paperref is
%      the paper ID code for the paper you are referencing.  See the
%      note in the "Paper ID Code" section above for details on how to
%      determine the paper ID code for the paper you reference.
%
\begin{references}
\reference Berry, D.S.\ 2008, \adassxvii, \paperref{P8.6}
\reference Berry, D.S., Reinhold, K., Jenness, T., Economou, F.\ 2007,
           \adassxvi, 425
\reference Draper, P.W.\ et al.\ 2007, \adassxvi, 695
\reference Goodman, A.\ et al.\ 2008 \adassxvii, \paperref{O7.1}
\reference Leech, J., Jenness, T.\ 2005 \adassxiv, 143
\reference Smith, H., et al.\ 2003, Proc.\ SPIE, 4855, 338
\end{references}

% Do not place any material after the references section

\end{document}  % Leave intact
