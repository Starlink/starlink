%ADASS_PROCEEDINGS_FORM%%%%%%%%%%%%%%%%%%%%%%%%%%%%%%%%%
%
% TEMPLATE.TEX -- ADASS Conference Proceedings template.
%
% Use this template to create your proceedings paper in LaTeX format
% by following the instructions given below.  Much of the input will
% be enclosed by braces (i.e., { }).  The percent sign, "%", denotes
% the start of a comment; text after it will be ignored by LaTeX.  
% You might also notice in some of the examples below the use of "\ "
% after a period; this prevents LaTeX from interpreting the period as
% the end of a sentence and putting extra space after it.  
% 
% You should check your paper by processing it with LaTeX.  For
% details about how to run LaTeX as well as how to print out the User
% Guide, consult the README file.  You should also consult the sample
% LaTeX papers, sample1.tex and sample2.tex, for examples of including
% figures, html links, special symbols, and other advanced features.
%
%%%%%%%%%%%%%%%%%%%%%%%%%%%%%%%%%%%%%%%%%%%%%%%%%%
% Note that the primary style file is that from the ASP Conf. Series; ADASS style 
% elements are included by an additional \usepackage. You may use other 
% _standard_ packages if needed, such as lscape, psfig, epsf, and graphicx, 
% although these packages may already be installed on your system. 
%
\documentclass[11pt,twoside]{article}  % Leave intact
\usepackage{asp2006}
\usepackage{adassconf}

% Set counters for footnotes and sectioning, which is needed when 
% constructing the full volume of all papers. 
% DO NOT DELETE. 
\setcounter{equation}{0}
\setcounter{figure}{0}
\setcounter{footnote}{0}
\setcounter{section}{0}
\setcounter{table}{0}

\begin{document}   % Leave intact

%-----------------------------------------------------------------------
%			    Paper ID Code
%-----------------------------------------------------------------------
% Enter the proper paper identification code.  The ID code for your paper 
% is the session number associated with your presentation as published 
% in the official conference proceedings.  You can find this number by 
% locating your abstract in the printed proceedings that you received 
% at the meeting, or on-line at the conference web site.
%
% This identifier will not appear in your paper; however, it allows different
% papers in the proceedings to cross-reference each other.  Note that
% you should only have one \paperID, and it should not include a
% trailing period.
%
% EXAMPLE: \paperID{O4.1}
% EXAMPLE: \paperID{P2.7}

\paperID{P8.11}

%-----------------------------------------------------------------------
%		            Paper Title 
%-----------------------------------------------------------------------
% Enter the title of the paper.
%
% EXAMPLE: \title{A Breakthrough in Astronomical Software Development}

\title{Starlink Software Developments}

%-----------------------------------------------------------------------
%          Short Title & Author list for page headers
%-----------------------------------------------------------------------
% Please supply the author list and the title (abbreviated if necessary) as 
% arguments to \markboth.
%
% The author last names for the page header must appear in one of 
% these formats:
%
% EXAMPLES:
%     LASTNAME
%     LASTNAME1 and LASTNAME2
%     LASTNAME1, LASTNAME2, and LASTNAME3
%     LASTNAME et al.
%
% Use the "et al." form in the case of four or more authors.
%
% If the title is too long to fit in the header, shorten it: 
%
% EXAMPLE: change
%    Rapid Development for Distributed Computing, with Implications for the Virtual Observatory
% to:
%    Rapid Development for Distributed Computing

\markboth{Currie et al.}{Starlink Software Developments}

%-----------------------------------------------------------------------
%		          Authors of Paper
%-----------------------------------------------------------------------
% Enter the authors followed by their affiliations.  The \author and
% \affil commands may appear multiple times as necessary.  List each
% author by giving the first name or initials first followed by the
% last name. Do not include street addresses and postal codes, but 
% do include the country name or abbreviation. 
%
% If the list of authors is lengthy and there are several institutional 
% affiliations, you can save space by using the \altaffilmark and \altaffiltext 
% commands in place of the \affil command.
%
% EXAMPLE: 
%      \author{Raymond Plante, Doug Roberts, 
%                  R.\ M.\ Crutcher\altaffilmark{1}}
%      \affil{National Center for Supercomputing Applications, 
%                 University of Illinois, Urbana, IL, USA}
%      \author{Tom Troland}
%      \affil{University of Kentucky, Lexington, KY, USA}
%
%      \altaffiltext{1}{Astronomy Department, UIUC}
%
% In this example, the first three authors, "Plante", "Roberts", and
% "Crutcher" are affiliated with "NCSA".  "Crutcher" has an alternate 
% affiliation with the "Astronomy Department".  The fourth author,
% "Troland", is affiliated with "University of Kentucky"

\author{Malcolm J. Currie}
\affil{SSTD, Rutherford Appleton Laboratory, Oxfordshire, UK}
\author{Peter W. Draper}
\affil{Department of Physics, Durham University, Durham, UK}
\author{David S. Berry}
\affil{Centre for Astrophysics, University of Central Lancashire, Lancs, UK}
\author{Tim Jenness, Brad Cavanagh, Frossie Economou}
\affil{Joint Astronomy Centre, Hilo, Hawaii, USA}

%-----------------------------------------------------------------------
%			 Contact Information
%-----------------------------------------------------------------------
% This information will not appear in the paper but will be used by
% the editors in case you need to be contacted concerning your
% submission.  Enter your name as the contact along with your email
% address.
% 
% EXAMPLE:  \contact{Dennis Crabtree}
%           \email{crabtree@cfht.hawaii.edu}

\contact{Malcolm Currie}
\email{mjc@star.rl.ac.uk}

%-----------------------------------------------------------------------
%		      Author Index Specification
%-----------------------------------------------------------------------
% Specify how each author name should appear in the author index.  The 
% \paindex{ } should be used to indicate the primary author, and the
% \aindex for all other co-authors.  You MUST use the following
% syntax: 
%
% SYNTAX:  \aindex{Lastname, F.~M.}
% 
% where F is the first initial and M is the second initial (if used). Please 
% ensure that there are no extraneous spaces anywhere within the command 
% argument. This guarantees that authors that appear in multiple papers
% will appear only once in the author index. Authors must be listed in the order
% of the \paindex and \aindex commmands.
%
% EXAMPLE: \paindex{Crabtree, D.}
%          \aindex{Manset, N.}        
%          \aindex{Veillet, C.}        

\paindex{Currie, M.~J.}
\aindex{Berry, D.~S.}     % Remove this line if there is only one author
\aindex{Draper, P.~W.}
\aindex{Cavanagh, B.}
\aindex{Jenness, T.}
\aindex{Economou, F.}

%-----------------------------------------------------------------------
%			Subject Index keywords
%-----------------------------------------------------------------------
% Enter up to 6 keywords that are relevant to the topic of your paper.  These 
% will NOT be printed as part of your paper; however, they will guide the creation 
% of the subject index for the proceedings.  Please use entries from the
% standard list where possible, which can be found in the index for the 
% ADASS XVI proceedings. Separate topics from sub-topics with an exclamation 
% point (!). 
%
% EXAMPLE:  \keywords{astronomy!radio, computing!grid, data management!workflows, 
%     instrumentation!control}

\keywords{Starlink, software|cube processing, software!image processing, 
software|spectral analysis, data analysis environments,
software|image analysis}

%-----------------------------------------------------------------------
%			       Abstract
%-----------------------------------------------------------------------
% Type abstract in the space below.  Consult the User Guide and Latex
% Information file for a list of supported macros (e.g. for typesetting 
% special symbols). Do not leave a blank line between \begin{abstract} 
% and the start of your text.

\begin{abstract}          % Leave intact
This paper summarises new and improved facilities to Starlink software
since mid-2005, with the emphasis on cube and spectral analysis. 
\end{abstract}

%-----------------------------------------------------------------------
%			      Main Body
%-----------------------------------------------------------------------
% Place the text for the main body of the paper here.  You should use
% the \section command to label the various sections; use of
% \subsection is optional.  Significant words in section titles should
% be capitalized.  Sections and subsections will be numbered
% automatically. 
%
% EXAMPLE:  \section{Introduction}
%           ...
%           \subsection{Our View of the World}
%           ...
%           \section{A New Approach}
%
% It is recommended that you look at the sample paper sample2.tex
% for examples of formatting references, footnotes, figures, equations, 
% html links, lists, and other features.  

\section{Introduction}
In 2005 Starlink software came under the auspices of the Joint
Astronomy Centre, Hawaii, where it plays a vital role in real-time
data-reduction pipelines.  The software has continued to be developed
for the data reduction and analysis for new instruments at JAC,
particularly for spectral cubes from the James Clerk Maxwell
Telescope.  This poster outlines the changes made since 2005, 
focusing on those not presented elsewhere in this volume or at 
earlier ADASS conferences. 

\section{KAPPA---Kernel Package}

There are new and improved applications for cube processing and
analysis.  {\footnotesize CHANMAP} creates a tiled channel map, while
{\footnotesize CLINPLOT} draws a spatial grid of line plots for a cube
axis (see Figure~\ref{fig:P8.11_1}).  {\footnotesize MFITTREND} fits
independent trends to data lines with optional automatic feature
masking.  {\footnotesize COLLAPSE} has many new estimators targeted at
cube analysis. Smoothing and rotation may now be applied to all planes
in a cube.

Other new tasks are {\footnotesize BEAMFIT} which fits up to five
Gaussians to beam features with various optional constraints such as
separation or width; {\footnotesize INTERLEAVE} which forms
higher-resolution data by interleaving a set of files; and
{\footnotesize WCSMOSAIC} which aligns and co-adds a group of NDFs
using WCS information.

There are improvements in statistics such as a cumulative-histogram 
option in {\footnotesize HISTOGRAM}, the reporting of extreme values'
locations in world co-ordinates, and more ways to compute the mode 
with {\footnotesize HISTAT}.

Other improvements include: optional flux conservation in
{\footnotesize REGRID} and {\footnotesize WCSALIGN}; support for AST
SpecFrames and FluxFrames; alignment of plots via any suitable
co-ordinate system; better surface fitting and fit-error estimation.

\begin{figure}
\plotfiddle{P8.11_1.eps}{61mm}{270}{40}{40}{-143.0}{202.0}
\caption{An example of {\footnotesize CLINPLOT}.  Each box shows a raw
spectrum (light curve) at the spatial position given by the outer
axes. The dark curve shows the trend removed automatically after
feature masking (gaps indicated above).  The annotations around the
lower-left spectrum are the axes for each line plot.}
\label{fig:P8.11_1}
\end{figure}

\section{SPLAT-VO---Spectral Analysis}

SPLAT-VO continues to evolve.  Some recent improvements are listed below.
\begin{itemize}
\item Fuller support for dual-sideband spectra (see Figure~\ref{fig:P8.11_2})
\item A synopsis of the properties of the current spectrum
\item More standard sub-millimetre lines
\item Interoperate with the Virtual Observatory, using SSAP Version 1 queries
\item PLASTIC support for sending and receiving spectra from other
   PLASTIC-enabled applications
\item Supports data units, e.g. transformations between common
   flux systems
\item Region statistics tool
\item New smoothing filters: Hann, Hamming, Welch and Barlett
\end{itemize}

\begin{figure}
\epsscale{0.62}
\plotone{P8.11_2.eps}
\caption{
SPLAT-VO displaying a dual-sideband (see Berry 2008) sub-millimetre spectrum
from the JCMT, with the two sideband velocity systems shown.  Standard
lines have been overlaid and corrected for the source velocity. Note
that the rest frequency can be trivially changed to that of one of the
lines, or a picked velocity, as can the co-ordinate system and sideband.}
\label{fig:P8.11_2}
\end{figure}

\section{GAIA---image display and analysis}
A major upgrade to support three-dimensional data was presented at a
Focus Demo'
in 2006.  That has continued into 3-D visualisation (Draper et al. 2008).
Some of the non-3D enhancements include the following.
\begin{itemize}
\item Display of FITS in-line compressed images
\item PLASTIC enabled so can send and receive images and receive catalogues
\item Various new default on-line catalogues available
\item Rotatable box item now available as a catalogue plotting symbol
\end{itemize}

\section{New Application Packages}

Three packages have been developed or added to the Starlink software.
\begin{description}
\item[{\footnotesize CUPID}] ClUmP IDentification in up to three
dimensions (Berry et al. 2007)
\item[{\footnotesize SMURF}] Sub-Millimetre User Reduction Facility
whose main task transforms raw time series into cubes (Jenness et al. 2008)
\item[{\footnotesize SOURCEPLOT}] plots astronomical objects in 
altitude  and azimuth for a supplied point on the Earth (originally written 
in 1998 at the JAC)
\end{description}

\section{DATACUBE---scripts for automating spectral cube analysis}

This underwent a major overhaul and update including documentation
reflecting the state of 3-D tools and scripts in early 2006.

\begin{itemize}
\item Generalised to work for arbitrary datacubes and sections thereof
\item All scripts improved and enhanced, notably {\footnotesize VELMAP}.
Improvements include the map output in one of four velocity systems, 
recognition of dual-sideband spectra and an optional velocity key
\item Addition {\footnotesize GRIDSPEC} averages and plots groups of 
neighbouring spectra
\item Addition {\footnotesize VELMOMENT} builds a velocity map from 
collapsed  intensity-weighted spectra, it also has optional spatial 
compression
\end{itemize}           

\section{Data System and Infrastructure Libraries}

\begin{itemize}
\item HDS supports files $>$2GB on 32-bit machines, and individual arrays 
$>$2GB on 64-bit systems; 64-bit files are created by default
\item Can include \texttt{".sdf"} file extension when supplying the name
of an HDS file
\item Scaled NDFs for compression
\item NDF sections can be specified using world co-ordinates, for example
\begin{verbatim}
       cube(12:34:56.7,-41:52:09,400.0;650.0)
\end{verbatim}
might extract a spectrum from a cube at the given sky position between
400 and 650 km/s, and
\begin{verbatim}
        ifu(1h34m10.1s:1h34m12.4s,2d35m:2d35.5m,656.3)
\end{verbatim}
could specify a region of sky 2.3s by 0.5 arcminutes at 656.3 nm.
\item Major developments in AST library (Berry 2008)
\item ARD can refer to regions within time or spectral co-ordinate systems
\item GRP, HDS, and NDG libraries have official C interfaces
\end{itemize}

\section{Obtaining the software} For further details of the changes,
and how to obtain the now GPL-licensed software, please visit
\htmladdURL{http://starlink.jach.hawaii.edu/}.  At the time of writing 
the {\em puana} release is available pre-built for the following operating 
systems.
\begin{itemize}
\item 32-bit Linux glibc 2.3.4 (434MB .tar.gz) and glibc 2.5 (438MB .tar.gz)
\item 64-bit Linux (420MB .tar.gz)
\item PPC OS X Tiger (348MB .tar.gz)
\item Intel OS X Tiger (335MB .tar.gz)
\end{itemize}
Other systems, such as Solaris, may be provided given sufficient demand.

The source is available from an SVN repository.  The build system is
based on GNU autotools and is known to work for a range of compiler   
options, including g77/GCC3, g95/GCC4, gfortran/GCC4.3, Solaris Studio~10/11 
and Intel.

\section{Acknowledgments}
This work was funded by the UK Science and Technology Facilities Council
and its predecessor (PPARC) for the Joint Astronomy Centre, Hawaii.

%-----------------------------------------------------------------------
%			      References
%-----------------------------------------------------------------------
% List your references below within the reference environment
% (i.e. between the \begin{references} and \end{references} tags).
% Each new reference should begin with a \reference command which sets
% up the proper indentation.  
%    NOTE: all citations in the text _must_ have a corresponding entry in 
%    the reference list, and all references must be cited in the text.
%
% Observe the following order when listing bibliographical 
% information for each reference:  author name(s), publication 
% year, journal name, volume, and page number for articles. 
% URLs to the reference may be given either in-line, or as a footnote. 
% Note that many journal names are available as macros; see
% the User Guide for a listing "macro-ized" journals. 
%
% EXAMPLES:  
% Reference to a Journal article:
%     \reference Cornwell, T.\ J.\ 1988, \aap, 202, 316
%
% Journal paper with more than 7 authors;
%     \reference Hanisch, R.\ et al.\ 2001, \aap, 376, 359
%
% Reference to an SPIE paper:
%     \reference Noordam, J.~E.\ 2004, Proc.\ SPIE, 5489, 817
%
% Reference to a contribution to a proceedings (not ADASS)
%     \reference Schmitz, M., Helou, G., Dubois, P., LaGue, C., Madore,B., Corwin, H.~G., Jr., 
%          \& Lesteven, S.\ 1995, in Information \& On-Line Data in Astronomy, 
%          ed.\ D.\ Egret \& M.~A.\ Albrecht (Dordrecht: Kluwer Academic Publishers), 259
%
% Reference to a paper in an earlier ADASS proceedings:
%     \reference Kantor, J., et al.\ 2007, \adassvii, 3
%
% Reference to a paper in the current ADASS:
%     \reference Hanisch, R.~J.\ 2008, \adassxvii, \paperref{O1.3}
% 
% Reference to a book:
%     \reference Jacobson, I.\ Booch, G., \& Rumbaugh, J.\ 1999, 
%            The Unified Software Development Process (Reading, MA: Addison-Wesley)
%
% Reference to a thesis:
%     \reference Gering, D.\ 1999, Master's Thesis, Massachusetts Institute of Technology
% 
% Reference to a purely on-line resource:
%     \reference Staveley-Smith, L.\ 2006, ATNF SKA Memo~6, http://www.atnf.csiro.au/ska
%
% Note the following tricks used in the example above:
%
%   o  \& is used to format an ampersand symbol (&).
%   o  \'e puts an accent agu over the letter e.  See the User Guide
%      and the sample files for details on formatting special
%      characters.  
%   o  "\ " after a period prevents LaTeX from interpreting the period 
%      as an end of a sentence.
%   o  \aj is a macro that expands to "Astron. J."  See the User Guide
%      for a full list of journal macros
%   o  \adassvii is a macro that expands to the full title, editor,
%      and publishing information for the ADASS VII conference
%      proceedings.  Such macros are defined for ADASS conferences I
%      through XVI.
%   o  When referencing a paper in the current volume, use the
%      \adassxvii and \paperref macros.  The argument to \paperref is
%      the paper ID code for the paper you are referencing.  See the 
%      note in the "Paper ID Code" section above for details on how to 
%      determine the paper ID code for the paper you reference.  
%
\begin{references}
\reference Berry, D.~S., Reinhold, K., Jenness, T., \& Economou, F.\ 2007, \adassxvi, 425
\reference Berry, D.~S.\ 2008, \adassxvii, \paperref{P8.6}
\reference Draper, P.~W., Berry, D.~S., Jenness, T., Economou, F., \& Currie, M.~J.\ 2008, \adassxvii, \paperref{P2.7}
\reference Jenness, T., Cavanagh, B., Economou, F., \& Berry, D.~S.\ 2008, \adassxvii, \paperref{P7.1}
\end{references}

% Do not place any material after the references section

\end{document}  % Leave intact
