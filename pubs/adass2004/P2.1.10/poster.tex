\documentclass[a0]{a0poster}
% You might find the 'draft' option to a0 poster useful if you have
% lots of graphics, because they can take some time to process and
% display. (\documentclass[a0,draft]{a0poster})

\pagestyle{empty}
\setcounter{secnumdepth}{0}

% The textpos package is necessary to position textblocks at arbitary 
% places on the page.
\usepackage[absolute]{textpos}

\usepackage{graphics,times,url}

% These colours are tried and tested for titles and headers. Don't
% over use color!
\usepackage{color}
\definecolor{DarkBlue}{rgb}{0.1,0.1,0.5}
\definecolor{Red}{rgb}{0.9,0.0,0.1}

% see documentation for a0poster class for the size options here
\let\Textsize\normalsize
\def\Head#1{\noindent\hbox to \hsize{\hfil{\LARGE\color{DarkBlue} #1}}\bigskip}
\def\LHead#1{\noindent{\LARGE\color{DarkBlue} #1}\bigskip}
\def\Subhead#1{\noindent{\large\color{DarkBlue} #1}\bigskip}
\def\Title#1{\noindent{\VeryHuge\color{Red} #1}}


\TPGrid[40mm,40mm]{23}{12}      % 3 cols of width 7, plus 2 gaps width 1

\parindent=0pt
\parskip=0.5\baselineskip

\begin{document}

\begin{textblock}{23}(0,0)
\Title{P2.1.10: The Starlink Software Collection vs.\ GNU Autoconf}
\end{textblock}



\begin{textblock}{23}(0,1.0)
\LHead{\strut
  Norman Gray$^1$, Tim Jenness$^2$, Alasdair Allan$^1$, David Berry$^1$,
  Malcolm Currie$^1$, Peter Draper$^1$, Mark Taylor$^1$ and Brad Cavanagh$^2$
\hfil\break
\textsl{%
  \strut 1: Starlink, UK\\
  \strut 2: Joint Astronomy Centre, Hawai'i, USA}}
\end{textblock}


%% The original abstract was:
%%
%% P2.1.10 Porting the Starlink Software Collection to GNU Autoconf
%% Norman Gray, Starlink, Tim Jenness, Joint Astronomy Centre, Alasdair
%% Allan, Starlink, David Berry, Starlink, Malcolm Currie, Starlink,
%% Peter Draper, Starlink, Mark Taylor, Starlink, Brad Cavanagh, Joint
%% Astronomy Centre
%% 
%% 
%% The Starlink Classic Software Collection (USSC) currently runs on
%% three different unix platforms and contains approximately 130 separate
%% software items, totaling over 6 million lines of code using a mix of
%% Fortran, C, Tcl and Perl. The proliferation of requests for ports to
%% new operating systems (including multiple variants of Linux), in
%% conjunction with a decrease in the level of support for the classic
%% software collection, has lead to a decision to modify the build system
%% from the current collection of makefiles with hard-wired OS
%% configurations to a scheme involving feature-discovery via GNU
%% Autoconf.
%% 
%% As a result of this work, we have already ported the USSC to Mac OSX
%% and Cygwin. This poster will present the issues involved in a
%% substantial reorganization of a large legacy code base, including the
%% difficulties in extending the autoconf system to properly support
%% Fortran.




\begin{textblock}{2}(18,-0.5)
\includegraphics{starlink_logo}
\end{textblock}

% Graphics
% In both cases, place the centre of the graphic at the centre of the
% lower region, extending from (0,2)..(23,12).
\begin{textblock}{7}[0.5,0.5](11.5,7)
\vbox to 0pt{\vss\hbox to 0pt{\hss\includegraphics{script-big}\hss}\vss}%
\vbox to 0pt{\vss\hbox to 0pt{\hss\includegraphics{script-little}\hss}\vss}%
\end{textblock}



\begin{textblock}{7}(0,2)
\hrule\medskip

\Head{Summary}

\slshape

The Starlink Classic Software Collection (the `USSC') currently runs
on three different unix platforms and contains approximately 130 separate
software items, totaling millions of lines of code, in a mix of
Fortran, C, Tcl, Perl and a few others. The proliferation of requests
for ports to
new operating systems (including multiple variants of Linux), in
conjunction with a decrease in the level of support for the classic
software collection, has lead to a decision to modify the build system
from the current collection of makefiles with hard-wired OS
configurations to a scheme involving feature-discovery via GNU
Autoconf.

As a result of this work, we have already ported the USSC to Mac OSX
and Cygwin. 

This had some unexpected benefits and costs, and a few valuable lessons.

\bigskip
\hrule
\end{textblock}


\begin{textblock}{7}(0,5)

\Head{The problems}

\begin{itemize}
\item It's Big!  About 130 items, with around 1700kSLOC of our
  legacy code (Fortran, C, C++ and others), 300kSLOC of Java (not included in
  this build system), plus another 600kSLOC of thirdparty code, some
  tweaked, and all built at the same time.

\item Just by the way\dots.  According to David A Wheeler's SLOCCount,
  that's \$110M worth.  And it works out larger than everything in the
  RH7.1 distribution except the kernel, Mozilla and XFree86
  \url{http://www.dwheeler.com/sloc/redhat71-v1/redhat71sloc.html}.
  Busy types, aren't we?

\item It's Crusty!  Some of that code dates back quite a long way, and
  some of it comes from community donations.  While the majority of it
  is well-written, there are a few gems in there.

\item It (was) Scattered!  For historical reasons, the master copies
  of all our code weren't kept in the one place.  Simply
  \emph{finding} all our code was slightly harder than we expected.

\item Did I mention funding problems?

\end{itemize}
\end{textblock}


\begin{textblock}{7}(0,8)

\Head{History}

Starlink was set up originally in the late seventies, as a way of
supplying UK astronomy with hardware (`astronomers will never need
more than six VAXes\dots'), naturally along with the data-analysis
software to go with them, and the system management to make it all
work smoothly.

We switched the hardware to Unix in the early nineties -- ending up
supporting specifically Solaris, Alpha and a Linux distribution.  We
also ported the software collection then.  That meant that some
software was dropped, and the rest preened quite extensively.
The build system we are now moving away from was designed in this
period.

In the late nineties, Starlink slimmed down, dropped the hardware and
management provision, and focused on maintaining and developing the
large legacy codebase.

Now we're concentrating on new software, we need to make the classic
software as accessible and maintainable as possible, since we expect
the community will have to take a larger role in its curation.

\end{textblock}


\begin{textblock}{7}(8,2)
\Head{The Outcome}

\begin{itemize}
\item As of October 2004, we have 2300kSLOC building unattended on
  Linux (RHEL), Solaris, Cygwin (incomplete), and OSX (almost).  Other
  platforms should be easy (but we can only cope with so much pain at
  one time).

\item All available in one place, via anonymous CVS (see `Contacts' box).

\item Unexpected but valuable preening, refactoring, bugfixing.

\item We extended the autoconf support for Fortran, and this will be
  offered back to the autoconf mainline.

\item GPLed wherever possible.

\item The project has always obsessed about coding standards and
  coding discipline.  This really did pay off here.

\end{itemize}  
\end{textblock}


\begin{textblock}{7}(8,9)
\Head{Patching autoconf and automake}

We have a number of project-specific code and installation
conventions, which we wanted to preserve.  This is what
\texttt{automake} is for.

Since we had to have autoconf, automake and libtool snapshots in our
repository in any case, it was a natural (though nerve-wracking) move
to start using patched versions.

This paid off, since \dots.  Was easier than we expected.
  
\end{textblock}


%\begin{textblock}{7}(16,4)
%\Head{Payoffs}
%
%\begin{itemize}
%\item The project has always obsessed about coding standards and
%  coding discipline.  This really did pay off here.
%\end{itemize}
%\end{textblock}


\begin{textblock}{7}(16,5)

\Head{Lessons and Warnings}

\begin{itemize}
\item The Fortran support in autoconf is rather slim.

\item The port to OSX was easier, in some ways, than we expected,
  partly because the OSX system compiler is a modified GCC.  We needed
  the Fink/OpenDarwin port of \texttt{g77}, and that causes lots of
  linking problems (the whole \texttt{restFP/saveFP} saga).  Be
  warned: the OSX linker has some very fixed ideas about things.

\item We couldn't automatically convert our old build system to the
  new one, because it was mostly hand-maintained.  But that wasn't a
  problem: disciplined coding in the past meant that packages could
  generally be ported to the new system very mechanically, and this
  turned out to be a small part of the effort.

\item The original plan was to autoconf everything with only necessary
  code changes.  However it was impossible to stop ourselves
  refactoring and tidying, fairly extensively in some cases.  This is
  both a warning to other projects that they won't be able to stop
  developers doing this, and a benefit, in the sense that a lot of
  code-hygiene tasks which have been too boring, confusing or risky in
  the past, become a lot less so when you're adjusting everything
  anyway.

\item We should have bowed to the inevitable, and started tweaking the
  autotools earlier.

\item GPLing was harder than we expected.  This is largely because of
  the donations in the past, of code written in gentler times, with
  bizarre `licences'.

\item It tool a \emph{lot} longer than we expected.  Around 6
  person-months to get the initial system up and running, and then
  another 6 to put the remainder into the system.

\end{itemize}

\end{textblock}

\begin{textblock}{7}(16,10)
\Head{Contacts}
\raggedright
\begin{itemize}
\item Norman Gray: \texttt{norman@astro.gla.ac.uk},
  \url{http://www.astro.gla.ac.uk/users/norman}
\item Tim Jenness: \texttt{t.jenness@jach.hawaii.edu},
  \url{http://www.jach.hawaii.edu/~timj/}
\item Starlink: \url{http://www.starlink.ac.uk} and
  \url{http://dev.starlink.ac.uk}
\item AnonCVS: access at
  \texttt{:pserver:anonymous@cvs.starlink.ac.uk:/cvs} 
  with password `starlink'
\item CVS browser: \url{http://cvsweb.starlink.ac.uk}
\end{itemize}
  
\end{textblock}


\end{document}
