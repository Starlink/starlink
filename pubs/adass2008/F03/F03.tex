%ADASS_PROCEEDINGS_FORM%%%%%%%%%%%%%%%%%%%%%%%%%%%%%%%%%
%
% TEMPLATE.TEX -- ADASS Conference Proceedings template.
%
% Use this template to create your proceedings paper in LaTeX format
% by following the instructions given below.  Much of the input will
% be enclosed by braces (i.e., { }).  The percent sign, "%", denotes
% the start of a comment; text after it will be ignored by LaTeX.
% You might also notice in some of the examples below the use of "\ "
% after a period; this prevents LaTeX from interpreting the period as
% the end of a sentence and putting extra space after it.
%
% You should check your paper by processing it with LaTeX.  For
% details about how to run LaTeX as well as how to print out the User
% Guide, consult the README file.  You should also consult the sample
% LaTeX papers, sample1.tex and sample2.tex, for examples of including
% figures, html links, special symbols, and other advanced features.
%
%%%%%%%%%%%%%%%%%%%%%%%%%%%%%%%%%%%%%%%%%%%%%%%%%%
% Note that the primary style file is that from the ASP Conf. Series; ADASS style
% elements are included by an additional \usepackage. You may use other
% _standard_ packages if needed, such as lscape, psfig, epsf, and graphicx,
% although these packages may already be installed on your system.
%
\documentclass[11pt,twoside]{article}  % Leave intact
\usepackage{asp2006}
\usepackage{adassconf}

% Set counters for footnotes and sectioning, which is needed when
% constructing the full volume of all papers.
% DO NOT DELETE.
\setcounter{equation}{0}
\setcounter{figure}{0}
\setcounter{footnote}{0}
\setcounter{section}{0}
\setcounter{table}{0}

\begin{document}   % Leave intact

%-----------------------------------------------------------------------
%			    Paper ID Code
%-----------------------------------------------------------------------
% Enter the proper paper identification code.  The ID code for your paper
% is the session number associated with your presentation as published
% in the official conference proceedings.  You can find this number by
% locating your abstract in the printed proceedings that you received
% at the meeting, or on-line at the conference web site.
%
% This identifier will not appear in your paper; however, it allows different
% papers in the proceedings to cross-reference each other.  Note that
% you should only have one \paperID, and it should not include a
% trailing period.
%
% EXAMPLE: \paperID{O4.1}
% EXAMPLE: \paperID{P2.7}

\paperID{F03}

%-----------------------------------------------------------------------
%		            Paper Title
%-----------------------------------------------------------------------
% Enter the title of the paper.
%
% EXAMPLE: \title{A Breakthrough in Astronomical Software Development}

\title{GAIA -- Version 4}

%-----------------------------------------------------------------------
%          Short Title & Author list for page headers
%-----------------------------------------------------------------------
% Please supply the author list and the title (abbreviated if necessary) as
% arguments to \markboth.
%
% The author last names for the page header must appear in one of
% these formats:
%
% EXAMPLES:
%     LASTNAME
%     LASTNAME1 and LASTNAME2
%     LASTNAME1, LASTNAME2, and LASTNAME3
%     LASTNAME et al.
%
% Use the "et al." form in the case of four or more authors.
%
% If the title is too long to fit in the header, shorten it:
%
% EXAMPLE: change
%    Rapid Development for Distributed Computing, with Implications for the Virtual Observatory
% to:
%    Rapid Development for Distributed Computing

\markboth{Draper, Berry, Jenness and Economou}
         {GAIA -- Version 4}

%-----------------------------------------------------------------------
%		          Authors of Paper
%-----------------------------------------------------------------------
% Enter the authors followed by their affiliations.  The \author and
% \affil commands may appear multiple times as necessary.  List each
% author by giving the first name or initials first followed by the
% last name. Do not include street addresses and postal codes, but
% do include the country name or abbreviation.
%
% If the list of authors is lengthy and there are several institutional
% affiliations, you can save space by using the \altaffilmark and \altaffiltext
% commands in place of the \affil command.
%
% EXAMPLE:
%      \author{Raymond Plante, Doug Roberts,
%                  R.\ M.\ Crutcher\altaffilmark{1}}
%      \affil{National Center for Supercomputing Applications,
%                 University of Illinois, Urbana, IL, USA}
%      \author{Tom Troland}
%      \affil{University of Kentucky, Lexington, KY, USA}
%
%      \altaffiltext{1}{Astronomy Department, UIUC}
%
% In this example, the first three authors, "Plante", "Roberts", and
% "Crutcher" are affiliated with "NCSA".  "Crutcher" has an alternate
% affiliation with the "Astronomy Department".  The fourth author,
% "Troland", is affiliated with "University of Kentucky"

\author{Peter W.\ Draper}
\affil{Department of Physics, Durham University, Durham, UK}

\author{David S.\ Berry}
\affil{Centre for Astrophysics, University of Central Lancashire, Lancs, UK}

\author{Tim Jenness \& Frossie Economou}
\affil{Joint Astronomy Centre, Hilo, Hawaii, USA}


%-----------------------------------------------------------------------
%			 Contact Information
%-----------------------------------------------------------------------
% This information will not appear in the paper but will be used by
% the editors in case you need to be contacted concerning your
% submission.  Enter your name as the contact along with your email
% address.
%
% EXAMPLE:  \contact{Dennis Crabtree}
%           \email{crabtree@cfht.hawaii.edu}

\contact{Peter W. Draper}
\email{p.w.draper@durham.ac.uk}

%-----------------------------------------------------------------------
%		      Author Index Specification
%-----------------------------------------------------------------------
% Specify how each author name should appear in the author index.  The
% \paindex{ } should be used to indicate the primary author, and the
% \aindex for all other co-authors.  You MUST use the following
% syntax:
%
% SYNTAX:  \aindex{Lastname, F.~M.}
%
% where F is the first initial and M is the second initial (if used). Please
% ensure that there are no extraneous spaces anywhere within the command
% argument. This guarantees that authors that appear in multiple papers
% will appear only once in the author index. Authors must be listed in the order
% of the \paindex and \aindex commmands.
%
% EXAMPLE: \paindex{Crabtree, D.}
%          \aindex{Manset, N.}
%          \aindex{Veillet, C.}

\paindex{Draper, P.~W.}
\aindex{Berry, D.~S.}
\aindex{Jenness, T.}
\aindex{Economou, F.}

%-----------------------------------------------------------------------
%			Subject Index keywords
%-----------------------------------------------------------------------
% Enter up to 6 keywords that are relevant to the topic of your paper.  These
% will NOT be printed as part of your paper; however, they will guide the creation
% of the subject index for the proceedings.  Please use entries from the
% standard list where possible, which can be found in the index for the
% ADASS XVI proceedings. Separate topics from sub-topics with an exclamation
% point (!).
%
% EXAMPLE:  \keywords{astronomy!radio, computing!grid, data management!workflows,
%     instrumentation!control}

\keywords{visualization, software!applications, software!interfaces,
          astronomy!sub-millimetre, interfaces!human-computer}

%-----------------------------------------------------------------------
%			       Abstract
%-----------------------------------------------------------------------
% Type abstract in the space below.  Consult the User Guide and Latex
% Information file for a list of supported macros (e.g. for typesetting
% special symbols). Do not leave a blank line between \begin{abstract}
% and the start of your text.

\begin{abstract}          % Leave intact
This paper describes recent developments to the Starlink GAIA image
display and analysis tool. These include the interactive 3D display of data
cubes as slices, extracted spectra and rendered volumes. We also present our 
plans for adding new VO features to GAIA.
\end{abstract}

%-----------------------------------------------------------------------
%			      Main Body
%-----------------------------------------------------------------------
% Place the text for the main body of the paper here.  You should use
% the \section command to label the various sections; use of
% \subsection is optional.  Significant words in section titles should
% be capitalized.  Sections and subsections will be numbered
% automatically.
%
% EXAMPLE:  \section{Introduction}
%           ...
%           \subsection{Our View of the World}
%           ...
%           \section{A New Approach}
%
% It is recommended that you look at the sample paper sample2.tex
% for examples of formatting references, footnotes, figures, equations,
% html links, lists, and other features.

\section{Introduction}

In recent years Starlink 
\htmladdnormallinkfoot{GAIA}{http://www.starlink.ac.uk/gaia}
(Draper et al.\ 2008; Draper et al.\ 2007) has been extended mainly in the
area of datacube handling. These changes have been introduced to support the
inspection and analysis of the large datacubes produced by the JCMT's
heterodyne instrumentation (Smith et al.\ 2003), but also have more general
application as they work on FITS and
\htmladdnormallinkfoot{Starlink}{http://www.starlink.ac.uk} NDF datacubes.

Support for datacubes is provided in two basic forms, the traditional movement
through slices, with interactive spectral extraction, plus full 3D
visualisation.

The most basic operation when handling datacubes is displaying image
slices. GAIA version 4 extended this existing feature so that moving between
slices is a very fast operation. Support for displaying the coordinates of the
extracted slice was also added, along with spectral extraction.

Using GAIA extracting and displaying a spectrum from a cube is very simple,
just click on an image position and drag around. The spectrum will be
interactively updated.  In addition to single-point extraction you can also
extract spectra that are averaged over regions: circles, rectangles, ellipses,
polygons and lines of various kinds. You can also mark an extracted spectrum
as the reference spectrum for simple point-to-point and region-to-region
comparisons. Some of these features are shown in Figure~\ref{F03-fig1}.

\begin{figure}
\epsscale{1.0}\plotone{F03_1.eps}
\caption{Cube handling toolbox displaying many of the recently added features.}
\label{F03-fig1}
\end{figure}

More recent changes support 3D visualisation, proving isophotal and volume
rendering, these are coupled to the more traditional cube handling, so the
slice and spectrum interact. Special features like astronomical coordinate
grids, the possibility to overlay isophotes from other cubes (in various
coordinate systems, conversion is handled automatically) are also provided.
Some of these features are shown in Figure~\ref{F03-fig2}.

\begin{figure}
\epsscale{1.0}\plotone{F03_2.eps}
\caption{Cube handling toolbox displaying many of the recently added features.}
\label{F03-fig2}
\end{figure}

\section{Adding VO facilities to GAIA}

The development of GAIA continues, as we plan to add VO support for accessing
the JCMT Science Archive (Gaudet et al.\ 2008), which will contain a variety
of data products. Progress so far includes VOTable support, querying
registries and access to images using the SIA protocol. We currently plan to
add cone search, before starting more ambitious work, in conjunction with
CADC, to support datacube queries since they are a primary product of our
heterodyne instrumentation.

Submillimetre images and cubes are typically full of amorphous emission, so we
would like to also extend GAIA to overlay the outlines of the clump emission,
in 2D and 3D using STC regions. The regions of emission will be detected by
the \htmladdnormallink{CUPID}{http://www.starlink.ac.uk/cupid} application
(Berry\ 2007), whose outputs will form part of the JCMT Science Archive.

\section{Acknowledgements}
GAIA is based on the ESO
\htmladdnormallinkfoot{SkyCat}{http://archive.eso.org/skycat} tool.
GAIA and SkyCat are both based on the scripting language
\htmladdnormallinkfoot{Tcl/Tk}{http://www.tcl.tk}.
They also make use of many other extensions and scripts developed by the Tcl
community. The 3D rendering uses the Visualization Toolkit,
\htmladdnormallinkfoot{VTK}{http://www.vtk.org}. 

Work on GAIA is supported by the 
\htmladdnormallinkfoot{Science and Technology Facilities Council}{http://www.scitech.ac.uk} 
for the 
\htmladdnormallinkfoot{Joint Astronomy Centre, Hawaii}{http://www.jach.hawaii.edu}.

\section{Obtaining GAIA}

GAIA is part of the Starlink JAC release (Jenness et al.\ 2009) which is
available from:
\begin{quote}
\begin{verbatim}
http://starlink.jach.hawaii.edu
\end{verbatim}
\end{quote}

%-----------------------------------------------------------------------
%			      References
%-----------------------------------------------------------------------
% List your references below within the reference environment
% (i.e. between the \begin{references} and \end{references} tags).
% Each new reference should begin with a \reference command which sets
% up the proper indentation.
%    NOTE: all citations in the text _must_ have a corresponding entry in
%    the reference list, and all references must be cited in the text.
%
% Observe the following order when listing bibliographical
% information for each reference:  author name(s), publication
% year, journal name, volume, and page number for articles.
% URLs to the reference may be given either in-line, or as a footnote.
% Note that many journal names are available as macros; see
% the User Guide for a listing "macro-ized" journals.
%
% EXAMPLES:
% Reference to a Journal article:
%     \reference Cornwell, T.\ J.\ 1988, \aap, 202, 316
%
% Journal paper with more than 7 authors;
%     \reference Hanisch, R.\ et al.\ 2001, \aap, 376, 359
%
% Reference to an SPIE paper:
%     \reference Noordam, J.~E.\ 2004, Proc.\ SPIE, 5489, 817
%
% Reference to a contribution to a proceedings (not ADASS)
%     \reference Schmitz, M., Helou, G., Dubois, P., LaGue, C., Madore,B., Corwin, H.~G., Jr.,
%          \& Lesteven, S.\ 1995, in Information \& On-Line Data in Astronomy,
%          ed.\ D.\ Egret \& M.~A.\ Albrecht (Dordrecht: Kluwer Academic Publishers), 259
%
% Reference to a paper in an earlier ADASS proceedings:
%     \reference Kantor, J., et al.\ 2007, \adassvii, 3
%
% Reference to a paper in the current ADASS:
%     \reference Hanisch, R.~J.\ 2008, \adassxvii, \paperref{O1.3}
%
% Reference to a book:
%     \reference Jacobson, I.\ Booch, G., \& Rumbaugh, J.\ 1999,
%            The Unified Software Development Process (Reading, MA: Addison-Wesley)
%
% Reference to a thesis:
%     \reference Gering, D.\ 1999, Master's Thesis, Massachusetts Institute of Technology
%
% Reference to a purely on-line resource:
%     \reference Staveley-Smith, L.\ 2006, ATNF SKA Memo~6, http://www.atnf.csiro.au/ska
%
% Note the following tricks used in the example above:
%
%   o  \& is used to format an ampersand symbol (&).
%   o  \'e puts an accent agu over the letter e.  See the User Guide
%      and the sample files for details on formatting special
%      characters.
%   o  "\ " after a period prevents LaTeX from interpreting the period
%      as an end of a sentence.
%   o  \aj is a macro that expands to "Astron. J."  See the User Guide
%      for a full list of journal macros
%   o  \adassvii is a macro that expands to the full title, editor,
%      and publishing information for the ADASS VII conference
%      proceedings.  Such macros are defined for ADASS conferences I
%      through XVI.
%   o  When referencing a paper in the current volume, use the
%      \adassxvii and \paperref macros.  The argument to \paperref is
%      the paper ID code for the paper you are referencing.  See the
%      note in the "Paper ID Code" section above for details on how to
%      determine the paper ID code for the paper you reference.
%
\begin{references}
\reference Berry, D.S., Reinhold, K., Jenness, T., Economou, F.\ 2007,
           \adassxvi, 425

\reference Draper, P.W., Berry, D.S., Jenness, T., Economou, F. 
\& Currie, M.J.\ 2008, \adassxvii, 339

\reference Draper, P.W., Currie, M.J., Jenness, T., Leech, J., Economou, F.,
Berry, D.S. \& Taylor, M.B.\ 2007, \adassxvi, 695

\reference Gaudet, S., Dowler, P., Goliath, S. \& Redman, R.\ 2008, 
\adassxvii, 135

\reference Jenness, T., Berry, D.S., Cavanagh, B., Currie, M.J.,
Draper, P.W. \& Economou F. \ 2009, \adassxviii, \paperref{E09}

\reference Smith, H., et al.\ 2003, Proc.\ SPIE, 4855, 338

\end{references}

% Do not place any material after the references section

\end{document}  % Leave intact
