%% This is file `elsarticle-template-2-harv.tex',
%%
%% Copyright 2009 Elsevier Ltd
%%
%% This file is part of the 'Elsarticle Bundle'.
%% ---------------------------------------------
%%
%% It may be distributed under the conditions of the LaTeX Project Public
%% License, either version 1.2 of this license or (at your option) any
%% later version.  The latest version of this license is in
%%    http://www.latex-project.org/lppl.txt
%% and version 1.2 or later is part of all distributions of LaTeX
%% version 1999/12/01 or later.
%%
%% The list of all files belonging to the 'Elsarticle Bundle' is
%% given in the file `manifest.txt'.
%%
%% Template article for Elsevier's document class `elsarticle'
%% with harvard style bibliographic references
%%
%% $Id: elsarticle-template-2-harv.tex 155 2009-10-08 05:35:05Z rishi $
%% $URL: http://lenova.river-valley.com/svn/elsbst/trunk/elsarticle-template-2-harv.tex $
%%

%%\documentclass[preprint,authoryear,12pt]{elsarticle}

%% Use the option review to obtain double line spacing
%% \documentclass[authoryear,preprint,review,12pt]{elsarticle}

%% Use the options 1p,twocolumn; 3p; 3p,twocolumn; 5p; or 5p,twocolumn
%% for a journal layout:

%% Astronomy & Computing uses 5p
%% \documentclass[final,authoryear,5p,times]{elsarticle}
\documentclass[final,authoryear,5p,times,twocolumn]{elsarticle}

%% if you use PostScript figures in your article
%% use the graphics package for simple commands
%% \usepackage{graphics}
%% or use the graphicx package for more complicated commands
%% or use the epsfig package if you prefer to use the old commands
%% \usepackage{epsfig}

\usepackage{graphicx}
\usepackage[labelformat=empty]{subfig}

%% The amssymb package provides various useful mathematical symbols
\usepackage{amssymb}
%% The amsthm package provides extended theorem environments
%% \usepackage{amsthm}

\usepackage[pdftex,pdfpagemode={UseOutlines},bookmarks,bookmarksopen,colorlinks,linkcolor={blue},citecolor={green},urlcolor={red}]{hyperref}
\usepackage{hypernat}

%% The lineno packages adds line numbers. Start line numbering with
%% \begin{linenumbers}, end it with \end{linenumbers}. Or switch it on
%% for the whole article with \linenumbers after \end{frontmatter}.
%% \usepackage{lineno}

%% natbib.sty is loaded by default. However, natbib options can be
%% provided with \biboptions{...} command. Following options are
%% valid:

%%   round  -  round parentheses are used (default)
%%   square -  square brackets are used   [option]
%%   curly  -  curly braces are used      {option}
%%   angle  -  angle brackets are used    <option>
%%   semicolon  -  multiple citations separated by semi-colon (default)
%%   colon  - same as semicolon, an earlier confusion
%%   comma  -  separated by comma
%%   authoryear - selects author-year citations (default)
%%   numbers-  selects numerical citations
%%   super  -  numerical citations as superscripts
%%   sort   -  sorts multiple citations according to order in ref. list
%%   sort&compress   -  like sort, but also compresses numerical citations
%%   compress - compresses without sorting
%%   longnamesfirst  -  makes first citation full author list
%%
%% \biboptions{longnamesfirst,comma}

% \biboptions{}

\journal{Astronomy \& Computing}

%% Make single quotes look right in verbatim mode
\usepackage{upquote}

\begin{document}

\begin{frontmatter}

%% Title, authors and addresses

%% use the tnoteref command within \title for footnotes;
%% use the tnotetext command for the associated footnote;
%% use the fnref command within \author or \address for footnotes;
%% use the fntext command for the associated footnote;
%% use the corref command within \author for corresponding author footnotes;
%% use the cortext command for the associated footnote;
%% use the ead command for the email address,
%% and the form \ead[url] for the home page:
%%
%% \title{Title\tnoteref{label1}}
%% \tnotetext[label1]{}
%% \author{Name\corref{cor1}\fnref{label2}}
%% \ead{email address}
%% \ead[url]{home page}
%% \fntext[label2]{}
%% \cortext[cor1]{}
%% \address{Address\fnref{label3}}
%% \fntext[label3]{}

\title{FellWalker - a Clump Identification Algorithm }

%% use optional labels to link authors explicitly to addresses:
%% \author[label1,label2]{<author name>}
%% \address[label1]{<address>}
%% \address[label2]{<address>}

\author[jac]{David S.\ Berry\corref{cor1}}
\ead{d.berry@jach.hawaii.edu}

\cortext[cor1]{Corresponding author}

\address[jac]{Joint Astronomy Centre, 660 N.\ A`oh\=ok\=u Place, Hilo, HI
  96720, USA}

\begin{abstract}
%% Text of abstract

This paper describes the FellWalker algorithm, which segments a 1-, 2- or
3-dimensional array of data values into a set of disjoint clumps of
emission, each containing a single significant peak. Pixels below a
nominated constant data level are assumed to be background pixels and are
not assigned to any clump. FellWalker is thus equivalent in purpose to
the CLUMPFIND algorithm. However, unlike CLUMPFIND, which segments the
array on the basis of a set of evenly-spaced contours and thus uses only
a small fraction of the available data  values, the FellWalker algorithm
is based on a gradient-tracing scheme which uses all available data values.

\end{abstract}

\begin{keyword}
%% keywords here, in the form: keyword \sep keyword

%% MSC codes here, in the form: \MSC code \sep code
%% or \MSC[2008] code \sep code (2000 is the default)

clump identification \sep
Starlink

\end{keyword}

\end{frontmatter}

% \linenumbers

%% Journal abbreviations
\newcommand{\mnras}{Mon Not R Astron Soc}
\newcommand{\aap}{Astron Astrophys}
\newcommand{\aaps}{Astron Astrophys Supp}
\newcommand{\pasp}{Pub Astron Soc Pacific}
\newcommand{\apj}{Astrophys J}
\newcommand{\apjs}{Astrophys J Supp}
\newcommand{\qjras}{Quart J R Astron Soc}
\newcommand{\an}{Astron.\ Nach.}
\newcommand{\ijimw}{Int.\ J.\ Infrared \& Millimeter Waves}
\newcommand{\procspie}{Proc.\ SPIE}
\newcommand{\aspconf}{ASP Conf. Ser.}

%% ASCL
\newcommand{\ascl}[1]{\href{http://www.ascl.net/#1}{ascl:#1}}

%% main text
\section{Introduction}
\label{sec:intro}

The CLUMPFIND algorithm \citep[][\ascl{1107.014}]{1994Williams} has been widely used for
decomposing 2- and 3-dimensional data into disjoint clumps of emission,
each associated with a single significant peak. It is based upon an
analysis of a set of evenly spaced contours derived from the data array
and has two main parameters - the lowest contour level, below which
data is ignored, and the interval between contours. However it has been
noted by \cite{2009Pineda} that the decomposition produced by CLUMPFIND
can be very sensitive to the specific value used for the contour interval,
particularly for 3-dimensional data and crowded fields. The choice of an
optimal contour interval is a compromise - real peaks may be missed if
the interval is too large, but noise spikes may be interpreted as real
peaks if the interval is too small.

The FellWalker algorithm attempts to circumvent these issues by avoiding
the use of contours altogether. Only a small fraction of the available
pixel values fall on the contour levels used by CLUMPFIND - the majority
fall \emph{between} these levels and so will have no effect on the
resulting decomposition. By contrast, FellWalker makes equal use of all
available pixel values above a stated threshold.

The name ``Fell Walker'' relates to the popular British pass-time of
walking up the hills and mountains of northern England, particularly
those of the Lake District
(\htmladdnormallink{http://en.wikipedia.org/wiki/Hillwalking}~), and was
chosen to reflect the way in which the algorithm proceeds iteratively by
following an upward path from a low-valued pixel to a significant summit
or peak in data-value. The following description of the algorithm uses
this fell-walking metaphor at frequent intervals.

\begin{figure}
\includegraphics[width=\columnwidth]{wasdale}
\caption{Wastwater and the Wasdale Fells, including Great Gable
(centre-left) and snow-covered Scafell Pike, the highest point in England
at 978 metres above sea level, just visible under cloud. Copyright: Nick
Thorne, www.lakedistrict.gov.uk/learning/freephotos\# }
\label{fig:wasdale}
\end{figure}

\begin{figure}
\includegraphics[width=\columnwidth]{fellwalking}
\caption{In 2-dimensions, peaks in data value are often reminiscent of the
fells of northern England. The FellWalker algorithm performs many walks
starting at various low-land pixels, and for each one follows a line of steepest ascent
until a significant summit is reached. All walks that terminate at the
same peak are assigned to the same clump, indicated by different colours
in the above figure.}
\label{fig:fellwalking}
\end{figure}

\section{The FellWalker Algorithm}

The core of the FellWalker algorithm consists of following many different
paths of steepest ascent in order to reach a significant summit, each of
which is associated with a clump, as illustrated in Fig.~\ref{fig:fellwalking}.
Every pixel with a data value above a user-specified threshold is used in
turn as the start of a ``walk''. A walk consists of a series of steps,
each of which takes the algorithm from the current pixel to an immediately
neighbouring pixel of higher value, until a pixel is found which is
higher than any of its immediate neighbours. When this happens, a search
for a higher pixel is made over a larger neighbourhood. If such a pixel
is found the walk jumps the gap and continues from this higher pixel. If no
higher pixel is found it is assumed that a new summit has been reached -
a new clump identifier is issued and all pixels visited on the walk are
assigned to the new clump. If at any point a walk encounters a pixel
which has already been assigned to a clump, then all pixels so far
visited on the walk are assigned to that same clump and the walk
terminates.

It is possible for this basic algorithm to fragment up-land plateau
regions into lots of small clumps which are well separated spatially but
have only minimal dips between them. The raw clumps identified by the above
process can be merged to avoid such fragmentation, on the basis of a
user-specified minimum dip between clumps\footnote{In common with other
parameters, this minimum dip parameter is specified as a multiple of the
noise level in the data.}. These merged clumps may,
optionally, be cleaned by smoothing their boundaries using a single step
of a cellular automaton.

Finally, each clump is characterised using a number of statistics, and a
catalogue of clumps statistics is created together with a pixel mask
identifying the clump to which each pixel is assigned.

The following sections give more detailed descriptions of each of these
phases in the FellWalker algorithm.

\subsection{Identifying Raw Clumps}
\label{sec:raw}
An array of integer values is first allocated, which is the same shape
and size as the supplied data array. This ``clump assignment array''
(CAA) is used to record the integer identifier of the clump, if any, to
which each pixel has been assigned. All clump identifiers are greater
than zero. An initial pass is made through the supplied data array to
identify pixels which have a data value above a user-specified threshold
value. Such pixels are assigned a value of zero in the CAA indicating
that the pixel is usable but has not yet been assigned to a clump, and
all other pixels are assigned a value of -1 indicating that they are
unusable and should never be assigned to a clump.

This initial CAA is then searched for any isolated individual pixels
above the threshold. Such pixels are set to -1 in the CAA, indicating
they should be ignored.

The main loop is then entered, which considers each pixel in turn as the
potential start of a walk to a peak. Pixels which have a non-zero value
in the CAA are skipped since they have either already been assigned to a
clump (if the CAA value is positive) or have been flagged as unusable (if
the CAA value is negative). A single walk consists of stepping from pixel
to pixel until a pixel is reached which is already known to be part of a
clump, or a significant isolated peak is encountered. The vector indicies of
the pixels visited along a walk are recorded in a temporary array so that
they  can be identified later.

At each step, the pixel values within a box of width three pixels are compared
to the central pixel to find the neighbouring pixel which give the highest
gradient\footnote{This gradient takes into account the fact that the
centres of the corner pixels are further away from the box centre than
are the centres of the mid-side pixels.}. Thus 2 neighbours are checked if
the data is 1-dimensional data, 8 are checked if the data is 2-dimensional
and 26 are checked if the data is 3-dimensional.

If the highest gradient found above is greater than zero - that is, if
there is an upward route out of the current pixel - the walk steps to the
selected neighbouring pixel. If this new pixel has already been assigned
to a clump (\emph{i.e.} if the CAA holds a positive value at the new
pixel), then the new walk has joined an older walk and so will eventually
end up at the same peak as the older walk. In this case, the existing
positive CAA value of the new pixel (\emph{i.e.} the clump index assigned
to the older walk) is copied into the CAA for all pixels visited so far
on the new walk, and a new walk from the next starting pixel is
initiated.

If the highest gradient found to any neighbouring pixel is less than or
equal to zero, then there is no upward route from the central pixel. This
could mean the walk has reached a significant peak, but it could also
mean it has merely reached a noise spike. To distinguish these two cases,
a search is made over a larger box\footnote{By default a box of width 9
pixels, but the user can specify a different size.}. If the maximum pixel
value in this larger box is smaller than the central pixel value, then
the central pixel is considered to be a significant peak. A new clump
identifier is issued for it and stored in the CAA at all pixels visited
on the walk. A new walk from the next starting pixel is then initiated.

If the maximum pixel value found in the larger box is greater than the
central pixel value, then the central pixel is considered to be a noise
spike. The walk then ``jumps across the gap'' and continues from the
highest pixel found in the box.

\begin{figure}
\includegraphics[width=\columnwidth]{sim}
\caption{A 50x50 array of artificial data used to illustrate the
FellWalker algorithm below.}
\label{fig:sim}
\end{figure}

\begin{figure}
\includegraphics[width=\columnwidth]{walks}
\caption{Two walks to a peak within the artificial data shown in
Fig.~\ref{fig:sim}. The contours show the data values themselves. The white
background pixels are below the nominated threshold, the grey pixels are
above the threshold but have not yet been assigned to a clump. The green
pixels trace the first walk that reached the left-hand peak. The blue pixels
trace a later walk to the same peak that was terminated when it met the first
walk. The green and blue pixels are all assigned to the same clump. These
walks follow the steepest line of ascent. Note the gap in the green line
near its start at the lowest contour - this is where a jump was made from
a noise spike to the highest value in a 9x9 box of neighbouring pixels.}

\label{fig:walks}
\end{figure}

The above process results in the CAA holding a clump identifier for every
usable pixel in the supplied data array. However, some of the walks
performed above may start with a section of very low gradient before any
significant ascent begins. The user is allowed to specify a minimum
gradient which must be achieved before a walk is considered to have
begun. Any section of the walk that occurs before the first such
``steep'' section is flagged as unusable in the CAA. For this test,
the gradient of a walk is averaged over four consecutive steps.

This process is illustrated in Fig.~\ref{fig:walks} which shows two
example up-hill walks produced by FellWalker for the artificial data shown
in Fig.~\ref{fig:sim}. The final CAA produced by the above process for
this data is shown in Fig.~\ref{fig:rawmask}.

\begin{figure}
\includegraphics[width=\columnwidth]{rawmask}
\caption{The raw clump mask created for the artificial data shown in
Fig.~\ref{fig:sim}. Two clumps are found, indicated by the black and grey
pixels.}
\label{fig:rawmask}
\end{figure}


\subsection{Merging Clumps}

The number of significant peaks found by the above process is determined
primarily by the maximum distance a walk can jump when searching for a
higher neighbouring pixel value. This parameter - known as \emph{MaxJump}
- defaults to 4 pixels. Using a larger value results in more local peaks
being interpreted as noise spikes rather than significant peaks, with a
corresponding reduction in the number of significant peaks found. Thus at
this point, peaks are discriminated simply on the basis of their spatial
separation.

This means it is possible for a clump with a wide, flat summit to be
fragmented into multiple clumps on the basis of noise spikes that are
separated by more than \emph{MaxJump} pixels.

To correct this, FellWalker merges adjacent clumps if the ``valley''
between the two adjacent peaks is very shallow.

Each clump (referred to as the ``central'' clump below) is considered in
turn to see if it should be merged with any of its neighbouring clumps.
The height of the ``col\footnote{The highest point on the boundary
between the two clumps.}'' between the central clump and each
neighbouring clump is found in turn, and the neighbouring clump with the
highest col is selected as a candidate for merging. If the peak value in
the central clump is less than a specified value,
\emph{MinDip}\footnote{The default is three times the noise level in the
data.}, above the col, the two clumps are merged into a single clump.

Once all central clumps have been checked in this way, the whole process
is repeated to see if any of the merged clumps should themselves be
merged. This process repeats until no further clumps can be merged.

\subsection{Cleaning Clump Outlines}
\label{sec:cleaning}

Once neighbouring clumps separated by shallow valleys have been merged,
there is an option to smooth the boundaries between adjacent clumps to
reduce the effects of noise. This is done using a specified number of
steps of a cellular automaton to modify the integer values in the CAA.

A single step of the cellular automaton creates a new CAA from the old
CAA. Each pixel in the new CAA is set to the most commonly occurring
clump index within a box of width 3 pixels centred on the corresponding
pixel within the old CAA. The output CAA from one step becomes the input
CAA to the next step. By default, only one step is performed.
Fig.~\ref{fig:cleanedmask} shows the effects of applying a single step to
the CAA shown in Fig.~\ref{fig:rawmask}.

\begin{figure}
\includegraphics[width=\columnwidth]{cleaned}
\caption{The smoothing effect of a single step of the cellular automaton
on the clump outlines shown in Fig.~\ref{fig:rawmask}.}
\label{fig:cleanedmask}
\end{figure}

\subsection{Removing Unusable Clumps}
Various criteria are available to select unusable clumps and exclude them
from the final data products. These include:

\begin{itemize}
\item Clumps that touch an edge of the supplied data array may be excluded.
\item Clumps that touch areas of missing (\emph{i.e.} blank) pixels may be excluded.
\item Clumps that have a peak value less than a given limit may be excluded.
\item Clumps that contain fewer than a given number of pixels may be excluded.
\end{itemize}

The number of clumps rejected for each of these reasons is reported.

\subsection{Characterising Each Clump}
The FellWalker algorithm is implemented within the {\tt findclumps}
command of the Starlink CUPID package (see section~\ref{sec:cupid}). This
command implements several other clump finding algorithms in addition to
FellWalker, and one of its design requirements was that each algorithm should characterise clumps in
the same way, so that results from different algorithms can be compared
directly. The results of each algorithm are presented in the following ways:

\begin{enumerate}

\item A pixel mask which is the same shape and size as the supplied data
array. Each pixel value is an integer which gives the index of the clump
to which the pixel has been assigned. Pixels that have not been assigned
to any clump are flagged with a special value.

\item A set of minimal cut-outs from the supplied data array. Each cut-out
holds the supplied pixels values corresponding to a single clump, with pixels
outside the clump set to a special ``blank'' value.

\item A table in which each row describes a single clump. The columns are:

\begin{description}
\item[Peak1] The position of the clump peak value on axis 1.
\item[Peak2] The position of the clump peak value on axis 2.
\item[Peak3] The position of the clump peak value on axis 3.
\item[Cen1] The position of the clump centroid on axis 1.
\item[Cen2] The position of the clump centroid on axis 2.
\item[Cen3] The position of the clump centroid on axis 3.
\item[Size1] The size of the clump along pixel axis 1.
\item[Size2] The size of the clump along pixel axis 2.
\item[Size3] The size of the clump along pixel axis 3.
\item[Sum] The total data sum in the clump.
\item[Peak] The peak value in the clump.
\item[Volume] The total number of pixels falling within the clump.
\item[Shape] An optional column containing an STC-S description
\citep{2007STCS,2007STC,2010ASPC..434..213B} of the spatial coverage of the clump.
\end{description}

\end{enumerate}

The values stored in the size columns of the output table are the RMS
deviation of each pixel centre from the clump centroid, where each pixel
is weighted by the corresponding pixel data value minus an estimate of
the background value in the clump\footnote{The minimum data value in the
clump is used as the background value.}. So for each axis, the size of
the clump on that axis is given by:

\[ size = \sqrt{ \frac{ \sum d_{i}.x_{i}^{2} }{ \sum d_{i} } -
\left( \frac{\sum d_{i}.x_{i} }{\sum d_{i}}  \right)^2 } \]

where $d_{i}$ is the data value of pixel $i$, and $x_{i}$ is the axis
value of pixel $i$. For a clump with a Gaussian profile, this
``size'' value is equal to the standard deviation of the Gaussian.

For observational data, the clumps will be blurred by the telescope beam.
FellWalker includes an option to remove this blurring if the beam size of
the telescope is known:

\[ size_{corrected} = \sqrt{ size^{2} - beam^{2} } \]

where $beam$ is the standard deviation of the Gaussian beam profile. A
corresponding correction is also applied to the peak values stored within
the table, on the assumption that the product of the peak value and the
clump volume should be unchanged by the instrumental blurring:

\[ peak_{corrected} = peak.(size/size_{corrected}) \]




\section{\label{sec:cupid}CUPID - an Implementation of FellWalker}
The Starlink CUPID package \citep[][\ascl{1311.007}]{CupidAdass,SUN255} provides
implementations of various clump-finding algorithms, including FellWalker,
Gaussclumps \citep[1406.018][\ascl{}]{1990ApJ...356..513S},
and CLUMPFIND. In common with the rest of the Starlink software \citep[][\ascl{1110.012}]{StarlinkAdass},
the source code for the CUPID package is open-source and is available on
\htmladdnormallinkfoot{Github}{https://github.com/Starlink}.

\section{Comparing FellWalker and CLUMPFIND}

\cite{2010Watson} made a detailed comparison of the performance of several
different clump finding algorithms, including FellWalker and CLUMPFIND.
This concluded that, under the conditions used in the study, FellWalker
is less likely than CLUMPFIND to split up large clumps, and is less
likely to create false clump detections. An independent illustration of
this result is presented in Fig.~\ref{fig:comp1}, which shows a field of
artificial Gaussian clumps, with the resulting clump assignment arrays
produced by FellWalker and CLUMPFIND. The input data image is an array of
500 x 500 pixels containing 200 circular clumps distributed randomly
across the image\footnote{Except none are allowed to touch an edge of the
image}. Each clump has a Gaussian profile with a Full Width at Half
Maximum of 15 pixels. The clump peak values are distributed uniformly
between 30 and 100, and Gaussian noise of standard deviation 15 is added
to the image. Both FellWalker and CLUMPFIND are run with the default
parameter values supplied by CUPID (\emph{FellWalker.MinDip}=3.RMS,
\emph{FellWalker.MaxJump}=4, \emph{ClumpFind.DeltaT}=2.RMS,
\emph{ClumpFind.TLow}=2.RMS).

\begin{figure}
\includegraphics[width=\columnwidth]{comp1}
\caption{Top: field of artificial Gaussian clumps. Lower left: the clump
assignment array produced by FellWalker. Lower right: the clump assignment
array produced by CLUMPFIND. Each colour indicates a different clump. }
\label{fig:comp1}
\end{figure}

It can be seen that the two algorithm have different problems; CLUMPFIND
is splitting each real clump into several parts, but FellWalker is
possibly failing to detect some of clumps that are visible by eye. FellWalker
detects 133 clumps\footnote{Since some of the 200 real sources will, by
chance, overlay each other very closely, we cannot expect all 200 clumps
to be detected.} and CLUMPFIND detects 1239.

This tendency for CLUMPFIND to split sources seems to be worse for low
signal-to-noise data. If we create a second field of artificial data with
a lower RMS noise level (3 instead of 15), the clump assignment arrays
shown in Fig.~\ref{fig:comp2} are created. CLUMPFIND is still tending to
fragment the edges of clumps into many tiny detections, but the bulk of
the interior of each clump is now left unfragmented.

\begin{figure}
\includegraphics[width=\columnwidth]{comp2}
\caption{The effect of reducing the noise level in the artificial data
by a factor of 5.}
\label{fig:comp2}
\end{figure}

If either algorithm splits clumps into several parts, not only will it
produce too many clumps, but the total data sum in each clump will on
average be too low. A useful tool for measuring the performance of these
algorithms is therefore the distribution of the measured total data sum
in each clump compared to the expected distribution, based on knowledge
of the clumps in the artificial data. In order to simplify such a
comparison, all the artificial clumps can be made identical (\emph{i.e.}
have the same peak amplitude and size), so that they all have the same
total data sum. In this case, the distribution of measured total data
sums should be peaked at the expected value, but will always have a tail
of higher-valued clumps due to the random positioning of clumps causing
some clumps to overlay each other. However an optimal clump-finding
algorithm should not produce any significant tail of lower-valued clumps.

The next sections describes the results of many such comparisons,
performed with a range of different signal-to-noise ratios, and with
different clump-finding parameter values.

\subsection{The Artificial Data}
The input data for each test was a two-dimensional image of 500x500
pixels containing 200 randomly positioned Gaussian clumps, all with the
same peak value of 100 (arbitrary units) and the same FWHM of 15 pixels.
Gaussian noise was then added, with a different noise level for each of
eight successive set of tests. The eight noise levels used were
spread evenly between 2 and 16.

\subsection{The Clump-finding Parameters}
For each of the eight different noise levels, the data was analysed
multiple times by both FellWalker and CLUMPFIND, using different values
for the main clump-finding parameters in each case. For CLUMPFIND, the
DeltaT parameter (the gap between contour levels) was varied from twenty
times the noise level down to one times the noise level. For FellWalker,
the MaxJump parameter (the minimum spatial separation between distinct
peaks) was varied between 2 and 14 pixels. In addition, the MinDip
parameter (the smallest dip in height allowed between distinct peaks) was
varied between 0 and 5 times the noise level.

The other main parameter common to both algorithms is the threshold
``sea-level'' below which all pixel values are ignored. This was fixed at
two times the noise level for all tests of both algorithms.

The default implementation of CLUMPFIND provided by the Starlink CUPID
package follows the description of the algorithm contained in
\cite{1994Williams}. The IDL version of CLUMPFIND distributed by Williams
includes some enhancements to the published algorithm. These are also
available in the CUPID version, but are disabled by default. They have,
however been enabled for the purposes of the current comparison, using
the following extra CUPID parameter settings:

\smallskip
\begin{tabular}{c c c}
ClumpFind.IDLAlg & = & 1 \\
ClumpFind.FwhmBeam & = & 0 \\
ClumpFind.MaxBad & = & 1.0 \\
ClumpFind.MinPix & = & 20 \\
ClumpFind.VeloRes & = & 0 \\
ClumpFind.AllowEdge & = & 1 \\
\end{tabular}
\smallskip

\subsection{Measuring Performance}
In each test, the artificial data consists of a collection of identical
clumps. So the initial expectation is that a reliable clump-finding
algorithm should produce a set of measured clumps that all have the same
total data sum, and that this total data sum should be the same as that
of the artificial clumps. However, this will not be the case in practice
for two different reasons:

\begin{enumerate}

\item Because of the random position of clumps, some clumps will have
spatial overlap to a greater or lesser extent. If the overlap is small,
then a good clump-finding algorithm should be able to resolve them. But
for larger overlaps, and larger noise levels, it becomes progressively
more difficult to resolve overlapping clumps. In the limiting case of
exactly co-incident clumps, it is clearly impossible for any algorithm
to resolve them. For this reason, we expect to see a tail of high-valued
clumps, although it is difficult to quantify the expected size of this
tail.

\item Each algorithm ignores pixel values below a threshold of two times the
noise level. This means that tests performed at higher noise levels will
set the threshold higher and so will detect clumps with smaller total
data sums.

\end{enumerate}

Whilst it is difficult to quantify the effects of the first of these two
issues, its impact can be reduced by comparing the expected clump data
sum with the median of the measured clump data sums rather than the mean.
This will reduce the influence of the tail of high-valued clump data sums.

The second issue can be taken into account by thresholding a single artificial
clump at the two sigma threshold before finding its data sum. So at each
noise level the sum of the data values in an artificial clump is found,
excluding pixel below the two sigma threshold. This value is then used as
the expected value for the total data sum in the clumps measured at the
same noise level. To simplify this comparison, the ratio of the median of
the measured clump data sums to the expected clump data sum is used below
as a measure of the performance of each algorithm. This is referred to as
the ``gain'' in the following figures.

\subsection{CLUMPFIND Results}
Fig.~\ref{fig:cf_results} shows the ratio of median clump data sum, as
measured by CLUMPFIND, to expected clump data sum at various values of
the DeltaT parameter and for various noise levels. Ideally , we would
hope for a gain of unity (\emph{i.e.} the median measured clump data sum
equalling the expected clump data sum). It can be seen that such a
condition is typically achieved at a DeltaT value around 10 times the
noise level, but that the relationship between DeltaT, Gain and RMS is
quite unpredictable.

At the DeltaT value of 2.0 recommended by \cite{1994Williams}, all noise
levels produce clumps which are well under the expected total data sum,
supporting the finding of \cite{2010Watson} that CLUMPFIND tends to
fragment clumps.

\begin{figure}
\includegraphics[width=\columnwidth]{comp4_cf}
\caption{The ratio of measured to expected data sum in clumps identified
using CLUMPFIND, as a function of the DelatT parameter value (the gap
between CLUMPFIND contour levels). The DeltaT values are multiples of the
noise level. The colours indicate the noise level, as shown in the colour
bar on the right of the figure.}
\label{fig:cf_results}
\end{figure}


\subsection{FellWalker Results}
Fig.~\ref{fig:fw_results} shows the ratio of median clump data sum, as
measured by FellWalker, to expected clump data sum at various values of
the MaxJump and MinDip parameters and for the same noise levels shown in
Fig.~\ref{fig:cf_results}. It can be seen that in these tests, FellWalker
produces clumps with data sums that are much more consistent and closer
to the expected data sum than CLUMPFIND.

It can be seen that at small MaxJump and small MinDip, FellWalker splits
clumps into small fragments. The small MaxJump value causes noise spikes
to be interpreted as peaks, and the small MinDip value then prevents these
``peaks'' from being merged. At the other extreme, large MaxJump values
allow the walk process to jump between real peaks, thus merging them
together into clumps that are larger than expected. This effect is worse
at larger MinDip values.

But between these extremes, values of MaxJump between 4 and 10, and MinDip
between 0 and 4 times the noise level produce predictable gain values that
are close to the expected value of unity, meaning that the median clump data
sum is close to the value expected on the basis of the known properties
of the artificial clumps.

\begin{figure}
\centering
\subfloat[][]{\includegraphics[trim=0 0 25mm 0,clip,height=0.5\columnwidth]{comp4_fw_0}}
\hspace*{10pt}
\subfloat[][]{\includegraphics[trim=20mm 0 25mm 0,clip,height=0.5\columnwidth]{comp4_fw_1}}\\
\subfloat[][]{\includegraphics[trim=0 0 25mm 0,clip,height=0.5\columnwidth]{comp4_fw_2}}
\hspace*{10pt}
\subfloat[][]{\includegraphics[trim=20mm 0 25mm 0,clip,height=0.5\columnwidth]{comp4_fw_3}}\\
\subfloat[][]{\includegraphics[trim=0 0 25mm 0,clip,height=0.5\columnwidth]{comp4_fw_4}}
\hspace*{10pt}
\subfloat[][]{\includegraphics[trim=20mm 0 25mm 0,clip,height=0.5\columnwidth]{comp4_fw_5}}
\caption{The ratio of measured to expected data sum in clumps identified
using FellWalker, as a function of the MaxJump parameter value (the
minimum spatial distance between distinct peaks, in pixels), and the
MinDip parameter value (the minimum dip between distinct peaks, as a
multiple of the noise level). The colours indicate the noise level, as
shown in Fig.~\ref{fig:cf_results}.}
\label{fig:fw_results}
\end{figure}

\section{Possible Enhancements to FellWalker}
Future work on the FellWalker algorithm is planned to address two small
problems with the current implementation:

\begin{itemize}
\item The cleaning process described in section~\ref{sec:cleaning} can
sometimes cause clumps to be split into two or more dis-contiguous parts.
This can occur for ``dog-bone'' shaped clumps that have a narrow
steep-sided central ridge that widens out at the two ends. In such cases,
the cleaning process can sometimes erode pixels from the central ridge
causing a gap to appear between the two wider end regions.

\item As described in section~\ref{sec:raw}, when a walk reaches a local
maximum, an attempt is made to find a higher pixel value within a small
box centred on the local maximum, and if found, the walk continues from this
higher pixel. However, no check is made that the higher pixel is on the
same ``island''\footnote{i.e. a contiguous group of
pixels that are all higher than the threshold level.} as the local
maximum. Thus it is possible that walks could jump from one island to
another, and so merge clumps together that are in fact distinct.
\end{itemize}

\section{Acknowledgements}

The Starlink software is currently maintained by the Joint Astronomy
Centre, Hawaii with support from the UK Science and Technology
Facilities Council.


%% The Appendices part is started with the command \appendix;
%% appendix sections are then done as normal sections
%% \appendix

%% \section{}
%% \label{}

%% References
%%
%% Following citation commands can be used in the body text:
%%
%%  \citet{key}  ==>>  Jones et al. (1990)
%%  \citep{key}  ==>>  (Jones et al., 1990)
%%
%% Multiple citations as normal:
%% \citep{key1,key2}         ==>> (Jones et al., 1990; Smith, 1989)
%%                            or  (Jones et al., 1990, 1991)
%%                            or  (Jones et al., 1990a,b)
%% \cite{key} is the equivalent of \citet{key} in author-year mode
%%
%% Full author lists may be forced with \citet* or \citep*, e.g.
%%   \citep*{key}            ==>> (Jones, Baker, and Williams, 1990)
%%
%% Optional notes as:
%%   \citep[chap. 2]{key}    ==>> (Jones et al., 1990, chap. 2)
%%   \citep[e.g.,][]{key}    ==>> (e.g., Jones et al., 1990)
%%   \citep[see][pg. 34]{key}==>> (see Jones et al., 1990, pg. 34)
%%  (Note: in standard LaTeX, only one note is allowed, after the ref.
%%   Here, one note is like the standard, two make pre- and post-notes.)
%%
%%   \citealt{key}          ==>> Jones et al. 1990
%%   \citealt*{key}         ==>> Jones, Baker, and Williams 1990
%%   \citealp{key}          ==>> Jones et al., 1990
%%   \citealp*{key}         ==>> Jones, Baker, and Williams, 1990
%%
%% Additional citation possibilities
%%   \citeauthor{key}       ==>> Jones et al.
%%   \citeauthor*{key}      ==>> Jones, Baker, and Williams
%%   \citeyear{key}         ==>> 1990
%%   \citeyearpar{key}      ==>> (1990)
%%   \citetext{priv. comm.} ==>> (priv. comm.)
%%   \citenum{key}          ==>> 11 [non-superscripted]
%% Note: full author lists depends on whether the bib style supports them;
%%       if not, the abbreviated list is printed even when full requested.
%%
%% For names like della Robbia at the start of a sentence, use
%%   \Citet{dRob98}         ==>> Della Robbia (1998)
%%   \Citep{dRob98}         ==>> (Della Robbia, 1998)
%%   \Citeauthor{dRob98}    ==>> Della Robbia


%% References with bibTeX database:

\bibliographystyle{model2-names-astronomy}
\bibliography{fellwalker}

%% Authors are advised to submit their bibtex database files. They are
%% requested to list a bibtex style file in the manuscript if they do
%% not want to use model2-names.bst.

%% References without bibTeX database:

% \begin{thebibliography}{00}

%% \bibitem must have one of the following forms:
%%   \bibitem[Jones et al.(1990)]{key}...
%%   \bibitem[Jones et al.(1990)Jones, Baker, and Williams]{key}...
%%   \bibitem[Jones et al., 1990]{key}...
%%   \bibitem[\protect\citeauthoryear{Jones, Baker, and Williams}{Jones
%%       et al.}{1990}]{key}...
%%   \bibitem[\protect\citeauthoryear{Jones et al.}{1990}]{key}...
%%   \bibitem[\protect\astroncite{Jones et al.}{1990}]{key}...
%%   \bibitem[\protect\citename{Jones et al., }1990]{key}...
%%   \harvarditem[Jones et al.]{Jones, Baker, and Williams}{1990}{key}...
%%

% \bibitem[ ()]{}

% \end{thebibliography}

\end{document}

%%
%% End of file `elsarticle-template-2-harv.tex'.
