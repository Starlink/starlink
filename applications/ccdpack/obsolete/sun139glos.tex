In the following section various terms which are used when describing
CCD datasets are explained, a little of the rationale for the
existence of the various CCD data types is also given. A pixel in
following context is one of the CCDs light sensitive elements and
should not be mistaken for a data pixel, although there is a one to
one correspondence between them.

\subsection{The bias level\xlabel{CCDglosbias}}


The bias level of a CCD frame is an artificially induced electronic offset
which ensures that the Analogue-to-Digital Converter (ADC) always receives a
positive signal. All CCD data has such an offset which must be removed if the
data values are to be truly representative of the counts recorded per pixel.

\subsection{Readout-noise\xlabel{CCDglosreadnoise}}

The readout-noise is the noise which is seen in the bias level. This is produced
by the on-chip amplifier and other sources of noise in the data transmission
before the signal is converted into a digital representation by the ADC.
Typically this can be represented by one value which is an estimate of the
standard deviation of the bias level values.

\subsection{Bias strips\xlabel{CCDglosbiasstrips}}

In order that the bias level of the CCD system can be constantly monitored (it
may at times move due to thermal changes and very occasionally, discontinuous
steps) values (columns or rows) are read from the CCD {\em without} moving any
charge into the output registers. These extra readouts are usually found at the
sides of the real data and are often referred to as bias strips or over-scan
regions (see Figure \ref{CCDPICCY}).

\subsection{ADC factor\xlabel{CCDglosADCfactor}}

The analogue to digital converter, samples the charge which is returned from
the CCD and returns a digital value (usually a 15 or 16 bit value). This value
does not equate to the actual number of electrons detected in the pixel in
question, but is proportional to it. Typically the proportionality constant is
determined by noise considerations --- the variance of the actual detected
electrons is poissonian, hence the variance in the output from the ADC should
equate to this (plus a few other terms such as the readout-noise), so the
constant ADC factor can be derived. The output from an ADC is measured in
analogue to digital units (ADUs). The ADC factor is multiplicative and converts
ADUs into detected electrons.

\subsection{Saturation\xlabel{CCDglossaturation}}

The capability of pixels to hold charge (charge is entered into a pixel
every time a photon is detected) is not infinite and after a certain
limit is exceeded the pixel then stops accumulating charge. When the
charge in such a pixel is clocked along the CCD (on route to the output
registers, from where it is amplified and transferred to the ADC) the
excess from it `bleeds' along the readout columns and sometimes even
across them. Before saturation slight non-linearities in
intensity occur. Data values which exceed this non-linearity limit
should be removed from the final datasets and generally cause no further
problems. However, because of charge bleeding, contamination may occur
around the vicinity and care should be taken when using such data.

\subsection{Dark current\xlabel{CCDglosdarkcurrent}}

All CCDs, at some level, exhibit the phenomenon of dark current. This is
basically charge which accumulates in the CCD pixels due to thermal
noise. The effect of dark current is to produce an additive quantity to
the electron count in each pixel. The reduction of dark current is the
main reason why all astronomical CCDs are cooled to liquid nitrogen
temperatures. Most modern CCDs only produce a few ADU (or less) counts
per pixel per hour and so this effect can generally be ignored. This,
however, is not the case for Infra-Red arrays.

\subsection{Pre-flashing\xlabel{CCDglospreflash}}

The transfer of charge between pixels (and hence along columns) suffers
from inefficiencies. Usually this amounts to a charge loss which is
never read out from the CCD well - this level is often referred to as
the `fat' or `skinny' zero to confuse matters; I refer to it as the
deferred charge value. When observing objects with low sky backgrounds
(and/or low counts themselves) this loss of charge may be significant
(at least in some older CCDs). To overcome this CCDs can be pre-flashed.
This amounts simply to illuminating the CCD with a uniform light flux
just prior to the actual object exposure. The object counts are then
simply added to this pre-flash level of charge in the CCD wells. Note,
however, that this method is of no use for very low counts as the signal
to noise level which is required after pre-flashing is higher than
before (the noise from the pre-flash photons adding to the noise of the
object photons). Correction of data for pre-flashing is achieved by
subtracting the pre-flash ADU count from the final data (before
flatfielding).

\subsection{Flatfielding\xlabel{CCDglosflatfielding}}

The sensitivity of a CCD to incident photon flux is not uniform across
the whole of its surface and before data can be said to be properly
relatively flux calibrated this needs to be corrected for. The
variations in CCD response can be on the large scale (one end of the CCD
to the other) and pixel-to-pixel. The relative flux levels on different
parts of the CCD are also vignetted by of the optics of the instrument
and telescope, this variation also needs correcting for and is performed
together with the CCD sensitivity corrections \footnote {An additional
effect of interest, which cannot be fully corrected, is the colour
sensitivity of the CCD pixels. Most pixels on a typical CCD frame are
exposed to the night sky which has a specific colour, this, however, may
not be the same colour as the object itself, so the best case response
is that the object and night sky colours mix to produce a response not
typical to the night sky dominated parts of the frame, if the object is
much brighter than the sky then its colour will dominate and ideally the
flatfield should be produced with a source mimicking this colour
response.}.

Flatfield calibration frames are usually taken of a photometrically flat
source using the same optical setup as that used to take the object
frames. In the past images of the interior of the telescope dome have
been used for this purpose, however, it now generally thought that
images of the twilight/dawn sky are more representative of a true
flatfield, having the same global illumination as the data and having a
good signal level (remember that calibration frames will be applied to
the object data at some stage and hence will introduce a noise
contribution to the final data values, it is therefore essential to get
a good set of calibration frames with lots of signal if this process is
to introduce the absolute minimum of noise, CCDPACK provides calibration
frame combination routines to produce `best bet' calibration frames with
very low noise levels), but these frames have a colour response which
may be not representative of the colour of the night time sky. If this
factor is important then specially taken night sky flatfields must be
produced. These can be taken of star free parts of the sky or produced
from many object frames whose (contaminating) objects are removed,
before median stacking to remove more spurious data values. Note in this
final case that the noise levels required to correct for small scale
variations are very time consuming to meet.

\subsection{Fringing\xlabel{CCDglosfringing}}

Some CCD data show an effect known as `fringing'. This usually has the
appearance of a series of `ripples' in the sky regions. Fringing is
caused by the multiple reflection and interference of the night-sky
emission lines in the CCD substrate. The effect is considerably enhanced
in CCDs whose substrates have been machined thinned to increase the blue
sensitivity, the thickness of the substrate being comparable to that of
the incident radiation, hence any deviations from a planar geometry
cause these `Newton Ring' like effects.

The fringe pattern is an additive effect and must be subtracted. To
de-fringe data it is necessary to get special exposures of an object
clear part of the night sky, or, alternatively, remove all the
contaminations (objects) from data frames with large areas of night sky.
These frames should then be combined to give complete spatial coverage
and to reduce the noise contribution. This `fringe-frame' should then be
scaled to the fringes present on the data frame (after normalisation
--- MAKEMOS) and {\em subtracted}.

