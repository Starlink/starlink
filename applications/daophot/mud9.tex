\def\today{\number\year\space \ifcase\month\or
  January\or February\or March\or April\or May\or June\or
  July\or August\or September\or October\or November\or December\fi
  \space\number\day}
\magnification = 1000
\baselineskip = 0.5 true cm
\pretolerance=3000
\tolerance=5000
\hsize=6.1 true in
\vsize=8.25 true in
\hoffset=1 true cm
\parskip=3pt plus 2 pt
\tenrm
\centerline{}
\vfill
\centerline {\bf User's Manual for}
\centerline {\bf DAOPHOT~II}
\bigskip
\bigskip

This manual is intended as a guide to the use of the digital stellar
photometry reduction program DAOPHOT~II: The Next Generation.  Brief
descriptions of the major routines are provided to aid the user in the
use of the program, and to serve as an introduction to the algorithms
for programmers called upon to make modifications.  A more complete
understanding of the methods used can best be obtained by direct
examination of the source code and the comment statements embedded
therein.

DAOPHOT~Classic was originally constructed within the framework of a
computer program given to Linda Stryker and the Dominion Astrophysical
Observatory by Jeremy Mould (Caltech).  The aperture-photometry portion
of the current program is still somewhat influenced by algorithms and
subroutines developed at Kitt Peak National Observatory, by way of the
POORMAN code developed by Jeremy Mould and Keith Shortridge at Caltech.
POORMAN was first made to run on the DAO VAX by Linda Stryker, and was
modified and generalized by Ed Olszewski.  Over the years, I have
replaced all of the subroutines with others of my own writing. In
DAOPHOT~II: The Next Generation all of the major algorithms and actual
FORTRAN coding are my own, with the exception of those Figaro, IRAF, or
Midas routines which are used for the image input and output.  I am
solely responsible for bugs and algorithm-related problems.  If any
such problems are found, please contact me; for major problems,
hard-copy documentation of the circumstances and nature of the
difficulty would be highly desirable.

\bigskip
\bigskip

\centerline{\bf ***** NOTE *****}

This manual has been modified to reflect the new VMS/Unix IRAF/Midas
compatible version of DAOPHOT~II: The Next Generation.  If you are
still running DAOPHOT~Classic you will find many places where this
manual is inappropriate.

\vfill
\indent \today
\vfill
\indent Peter B. Stetson
\bigskip
(604)363--0029$\qquad\qquad$stetson@dao.nrc.ca
\medskip
\indent Dominion Astrophysical Observatory

\indent Herzberg Institute of Astrophysics

\indent 5071 West Saanich Road

\indent Victoria, British Columbia V8X 4M6
\medskip
\indent Canada
\vfill
\eject
\centerline{\bf Contents}
\bigskip
{\noindent\obeylines\obeyspaces\frenchspacing\tt\baselineskip=0.5truecm
A. Introduction ......................................  3
B. A Typical Run of DAOPHOT~II .......................  6
C. Descriptions of Main Routines ..................... 11
\ \   I.     DAOPHOT~II itself ................. 11
\ \   II.    ATTACH ............................ 12
\ \   III.   OPTIONS ........................... 14
\ \   IV.    SKY ............................... 19
\ \   V.     FIND .............................. 20
\ \   VI.    PHOTOMETRY ........................ 25
\ \   VII.   PICK and PSF ...................... 29
\ \   VIII.  PEAK .............................. 34
\ \   IX.    GROUP ............................. 36
\ \   X.     NSTAR ............................. 37
\ \   XI.    SUBSTAR ........................... 29
D. Additional Commands ............................... 41
\ \   XIII.  MONITOR/NOMONITOR ................. 41
\ \   XIV.   SORT .............................. 42
\ \   XV.    SELECT ............................ 44
\ \   XVI.   OFFSET ............................ 45
\ \   XVII.  APPEND ............................ 46
\ \   XVIII. DUMP .............................. 47
\ \   XIX.   FUDGE ............................. 49
\ \   XX.    ADDSTAR ........................... 51
\ \   XXI.   LIST .............................. 53
\ \   XXII.  HELP .............................. 54
\ \   XXIII. EXIT .............................. 55
E. ALLSTAR ........................................... 56
Appendix I - Optional Parameters ..................... 60
Appendix II - The FIND Threshold ..................... 61
Appendix III - Deriving a PSF in a Crowded Field ..... 65
Appendix IV - Data Files ............................. 68
}
\vfill
\eject
\centerline{A.  Introduction}

DAOPHOT~II is a computer program for obtaining precise photometric
indices and astrometric positions for stellar objects in
two-dimensional digital images. It is intended to run as
non-interactively as possible and, furthermore, the possibility that
DAOPHOT~II would be used at other places than the DAO was kept in mind
as it was approaching its present form.  Therefore DAOPHOT~II performs
no operations related to the display or manipulation of the digital
image on an image-display system, even though at some stages in the
data reduction it is useful to be able to examine the picture
visually. Picture-display operations and some other steps in the
reduction procedure, such as editing intermediate data files or
combining results from different frames to obtain instrumental colors,
may be done outside of DAOPHOT~II using IRAF, Midas, or whatever
software you have handy or feel like writing.

It is assumed that (1)~before running DAOPHOT~II, the user will have
performed all necessary preparation of the images, such as
flat-fielding, bias-level subtraction, and trimming worthless rows and
columns from around the perimeter of the picture, and (2)~the
brightness data in the image are linearly related to true intensities.
The user is also assumed to have {\it a priori\/} knowledge of the
following pieces of information: (1)~the approximate size (full-width
at half-maximum) of unresolved stellar objects in the frame; (2)~the
number of photons corresponding to one analog-to-digital conversion
unit; (3)~the readout noise per pixel; and (4)~the maximum brightness
level (in analog-to-digital units) at which the detector still operates
linearly. These conditions being satisfied, DAOPHOT~II will perform the
following primary tasks:  (1)~find star-like objects above a certain
detection threshold, rejecting with a certain degree of reliability bad
pixels, rows, and columns, and avoiding multiple hits on individual
bright objects (although it continues to have some trouble with grossly
saturated objects.  I'm still thinking about it.); (2)~derive
concentric aperture photometry for these objects, estimating a local
sky brightness for each star from a surrounding annulus of pixels;
(3)~obtain a point-spread function for the frame from one star or from
the average of several stars, in an iterative procedure intended to fit
and subtract faint neighbor stars which contaminate the profile;
(4)~compute precise positions and magnitudes for the program stars by
fitting the point-spread function to each star, either individually or
in simultaneous multiple-profile fits for up to 60 stars at a time; and
(5)~erase stars from the picture by subtracting appropriately scaled
point-spread functions corresponding to the positions and magnitudes
derived for the stars during the photometric reductions.  ALLSTAR~II is
a separate stand-alone program which performs a much more sophisticated
multiple-profile fit to all the stars in a frame simultaneously.
Hereinafter I will include ALLSTAR~II under the generic heading of
DAOPHOT~II even though it is, as I said, a separate, stand-alone
program.  In addition to the aforementioned tasks, DAOPHOT~II contains
routines to perform some bookkeeping operations more easily than may be
the case with standard facilities: e.g., estimating an average sky
brightness for a frame, sorting the stars' output data according to
their positions in the frame or their apparent magnitudes, and dividing
the stars in the frame into natural groupings (for optimal
multiple-star reductions with NSTAR).  There is also a routine for
adding artificial stars to the picture, at random, so that the
effectiveness of the star-finding and profile-fitting routines can be
studied quantitatively in the particular circumstances of your own
picture. A few other global considerations of which you should be
aware:

\item{(1)} Although DAOPHOT~II is designed to be non-interactive, in
fact many of the operations run quickly enough that they are
conveniently executed directly from the terminal or workstation.  Only
the multiple-star profile fits take long enough that they are more
conveniently performed in batch mode: they may require anywhere from a
few CPU minutes to a few CPU hours per frame, depending upon the number
of stars to be reduced, the degree of crowding, and --- of course ---
the speed of your machine.  If computer time is expensive for you, you
may want to decide just how many stars must be reduced in order to
fulfill your basic mission.  For instance, if your goal is to locate
the principal sequences of some remote cluster, it may not be necessary
to reduce every last crowded star on the frame; instead, only a subset
consisting of the less-crowded stars might be reduced.

\item{(2)} The derivation of the point-spread function can also be
performed non-interactively with a reasonable degree of success,
but it may be to your advantage to check the quality of the profile
fits visually on an image display before accepting the final product.

\item{(3)} The shape of the point-spread function is assumed to
be spatially constant or to vary smoothly with position within the
frame; it is assumed {\it not\/} to depend at all on apparent
magnitude.  If these conditions are not met, systematic errors may
result.

\item{(4)} Although the star-finding algorithm is by itself not
sophisticated enough to separate badly blended images (two stars whose
centers are separated by significantly less than one FWHM), by
iteratively substracting the known stars and searching for fainter
companions, it is still possible to identify the separate stars in such
a case with a good degree of reliability:  First, one runs the
star-finding algorithm, derives aperture magnitudes and local sky
values for the objects just found, and obtains a point-spread function
in the manner described in an Appendix to this manual.  Second, one
performs a profile-fitting reduction run for these objects, and they
are subtracted from the data frame. This new picture, with the known
stars subtracted out, is then subjected to the star-finding procedure;
stars which were previously concealed in the profiles of brighter stars
stand out in this frame, and are picked up quite effectively by the
star-finding algorithm.  Sky values and aperture magnitudes for these
new stars are obtained from the original data frame, and the output
from this reduction is appended to the most recent photometry file for
the original star list.  This augmented set of stars is then run
through the profile-fitting code, and the entire list of fitted stars
can be subtracted from the original frame.  The process through this
point can {\it easily\/} be set up in a command procedure (or script,
or whatever your favorite rubrik is), and carried out in batch mode
while you are home sleeping or drinking beer or whatever.  Finally, if
absolute completeness is wanted, the star-subtracted picture can be
examined on an image display.  Any stars that were still undiscovered
by the program can be picked out by eye and added to the star list
manually.  Then one final reduction run may be performed.  Visual
examination is also a reasonable way to identify galaxies among the
program objects --- they are easily recognizable, with over-subtracted
centers surrounded by luminous fuzz.

\item{} My experience is that the number of stars found in the second
pass (the automatic star-finding on the first subtracted frame) amounts
to of order one-third the number of stars found in the first pass.  The
number of stars missed in the first two passes and later picked out by
eye is of order one to three percent of the total found in the first
two passes.  This procedure assumes that computer time is cheap for
you, and your own time is valuable.  If the converse is the case, you
may prefer to skip the second or even both automatic star-finding
passes, and go directly to interactive star identification.

\item{(5)} A principal source of photometric error for the faint stars
is the difficulty of defining what is meant by the term ``sky
brightness'' in crowded fields.  This is not simply the practical
difficulty of identifying ``contaminated'' pixels in the sky annulus so
that they can be omitted from the average, although certainly this is a
significant part of the problem.  There is also an underlying
philosophical ambiguity. For aperture photometry the term ``sky
brightness'' encompasses not only emission from the terrestrial night
sky, from diffuse interplanetary and interstellar material, and from
faint, unresolved stars and galaxies. It also includes the possibility
of a contribution of light from some bright star or galaxy.  That is to
say, for aperture photometry the relevant sky brightness is defined by
the answer to the question, ``If the {\it particular star\/} we are
interested in were not in the center of this aperture, what intensity
would we measure there from all other sources of brightness on or near
this line of sight?'' If there is available a set of pixels whose
brightness values are uncontaminated by the star we are trying to
measure, but which are subject to all other sources of emission
characteristic of that portion of the frame (including the possibility
of a contribution from individual bright objects nearby), then we may
answer the question: ``Most probably the intensity would be
such-and-such''.  The specific value ``such-and-such'' is well
predicted by the modal value of the brightnesses in the sample of sky
pixels. This is why DAOPHOT~II uses the mode of the intensities within
the sky annulus to define the value that should be subtracted from the
intensities inside the star aperture; not because it is a ``robust
estimator of the local diffuse sky brightness'', but because it is a
sort of maximum-likelihood estimator --- it yields the ``most probable
value of the brightness of a randomly-chosen pixel in this region of
the picture''. In the case of photometry from multiple simultaneous
profile fits, on the other hand, the sky brightness is defined by the
answer to a different question altogether:  ``If {\it none} of the
stars included in my star list were in this region of the picture, what
would the typical brightness be?'' This is a much harder question to
answer, partly because the answer cannot be obtained by an empirical
method as simple as finding the mode of an observed distribution, and
partly because the criteria for including a star in the star list are
hard to define absolutely.  For instance, a faint star is far easier to
see in a clear-sky zone, where its contamination can be identified and
then ignored in the sky-brightness estimate, than it would be if it
happened to lie near a much brighter star whose intrinsic brightness we
are trying to measure. Clearly then, when we detect a faint star in the
sample sky region, the decision whether to include or reject that
star's photons in the sky estimate becomes some function of the
magnitude of the star we are interested in measuring.  Further, a
serious attempt to estimate the sky brightness using probabilistic
methods would require an accurate model for the full noise spectrum of
the instrument, including the arrival rate and energy spectrum of
cosmic rays, the surface density of stars and galaxies on the sky,
their apparent luminosity functions, and the variations of these
quantities across the frame. Thus, a definitive answer to the question
``How bright is the sky here?'' is exceedingly hard to obtain with full
statistical rigor. For the present we must be content with a merely
adequate answer.

I have discussed the problem of sky determination in such philosophical
detail as a warning to the user not to regard DAOPHOT~II (or any other
program of which I am aware) as {\it the\/} final solution to the
problem of accurate stellar photometry in crowded fields.  As it stands
now, the aperture photometry routine is the only place in DAOPHOT~II
proper where sky brightness values are estimated; these estimates are
based on the modal values observed in annuli around the stars, and they
are carried along for use by PEAK and NSTAR.  ALLSTAR~II has the
capability (as an {\it option\/}) of iteratively {\it re\/}-determining
the sky brightness value for each star, defined as the median value
found in pixels surrounding the star {\it after\/} all known stars have
been subtracted from the frame using the current, provisional estimates
of their position and brightness.  These sky brightnesses are {\it
assumed\/} to be adequate for the purposes of defining and fitting
point-spread functions, but of course this is only approximately true.
The extent to which this false assumption affects the observational
results and the astrophysical conclusions which you derive from your
frames can best be estimated, at present, by the addition of artificial
stars of known properties to your pictures, with subsequent
identification and reduction by procedures identical to those used for
the program stars.

\vfill
\eject
\centerline{B.  A Typical Run of DAOPHOT~II}

Before you run DAOPHOT~II: The Next Generation you must arrange for
your pictures to exist on the computer's disk in a format acceptable to
the program.  On the DAO VAXen the standard format for images intended
for processing with DAOPHOT is the Caltech data-structure (.DST) file,
and routines exist for copying data from a number of different formats,
including FITS, to this type of file. On our Unix machines we use IRAF
for image manipulation and IRAF image files (*.imh and *.pix) for image
storage. At ESO there exists a version of DAOPHOT~II that operates on
Midas-format image files on a variety of hardwares. If you don't happen
to be at the DAO, then you will have to check with your local curator
of DAOPHOT~II to learn how to put your data into the proper format. In
all that follows, I shall assume that you are either at DAO, or are
using an unadulterated DAO/VMS/Unix version of DAOPHOT~II.  See your
local curator for changes that are specific to your facility.

I will now talk you quickly through the separate steps in reducing
a typical data frame, from beginning to end.  I suggest that you read
quickly through this section and the following chapters on the major
and minor routines in DAOPHOT~II, and then come back and reread this
section more carefully.  Words written in boldface {\bf CAPITALS} will
be DAOPHOT~II commands, which you may issue in response to a
``Command:'' prompt.

\item{I.} From a system prompt run DAOPHOT~II. Either read or don't
read the latest news (if it's even offered to you).  The values of
certain user-definable parameters will be typed out.  {\it Check their
values!  You might want to change some of them.}

\item{II.} Use {\bf OPTIONS} to change the values of any of the
optional reduction parameters. (This step is, itself, optional.)

\item{III.} Use {\bf ATTACH}  to tell the program which picture you
want to reduce. (In the VMS version, you do not need to include the
filename-extension, .DST, when you specify the filename, but you may if
you like.  In the Unix IRAF version, your life will be simpler if you
{\it do not\/} include the .imh extension.)

\item{IV.} You might want to use {\bf SKY} to obtain an estimate of the
average sky brightness in the picture.  {\it Write this number down in
your notes.\/}  This step is not really necessary, because {\bf FIND}
below will do it anyway.

\item{V.} Use {\bf FIND} to identify and compute approximate centroids
for small, luminous objects in the picture.  One of the
``user-definable optional parameters'' which you are permitted to
define is the significance level, in standard deviations, of a
luminosity enhancement in your image which is to be regarded as real.
Two other parameters which you {\it must\/} define are the readout
noise and gain in photons (or electrons) per data number which are
appropriate to a {\it single\/} exposure with your detector.  When you
run {\bf FIND}, it will ask you whether this particular image is the
average or the sum of several individual exposures.  From the available
information, {\bf FIND} will then compute the actual brightness
enhancement, in data numbers above the {\it local\/} sky brightness,
which corresponds to the significance level you have specified.  See
the section on {\bf FIND} and the Appendix on ``The {\bf FIND}
Threshold'' for further details.  According to a parameter set by the
user, {\bf FIND} will also compute a ``Lowest good data-value'':  any
pixel whose brightness value is less than some number of standard
deviations below the mean sky value will be regarded as bad, and will
be ignored by {\bf FIND} and by all subsequent reduction stages.

\item{VI.} Use {\bf PHOTOMETRY} to obtain sky values and concentric
aperture photometry for all objects found by the star-finding routine.

\item{VII.} Use {\bf PICK} to select a set of reasonable candidates for
PSF stars.  {\bf PICK} first sorts the stars by magnitude, and then
rejects any stars that are too close to the edge of the frame or a
brighter star.  It will then write a user-specified number of good
candidates to a disk file for use by {\bf PSF}.

\item{VIII.} Use {\bf PSF} to define a point-spread function for the
frame.  In crowded fields this is a subtle, iterative procedure
requiring an image processing system; it is outlined in detail in the
Appendix on ``Obtaining a Point-Spread Function''.  Consider, then,
that this step is a self-contained loop which you will go through
several times.

\item{IX.}  {\bf GROUP}, {\bf NSTAR}, and {\bf SUBSTAR}; {\it or\/}
ALLSTAR.  {\bf GROUP} divides the stars in the
aperture-photometry file created in step VI above into finely divided
``natural'' groups for reduction with the multiple-star PSF-fitting
algorithm, \hbox{\bf NSTAR}. {\bf NSTAR} will then produce improved
positions and instrumental magnitudes by means of multiple-profile
fits, and {\bf SUBSTAR} may then be used to subtract the fitted
profiles from the image, producing a new image containing the fitting
residuals.  Alternatively, you could feed the aperture-photometry file
directly to ALLSTAR, which will reduce all the stars in the image
simultaneously and produce the star-subtracted picture without further
ado.

\item{X.} Use {\bf ATTACH} to specify the star-subtracted picture
created in step IX as the one to work on.

\item{XI.} Use {\bf FIND} to locate new stars which have become visible
now that all the previously-known stars have been subtracted out.

\item{XII.} Use {\bf ATTACH} again, this time specifying the {\it
original} picture as the one to work with, and use {\bf PHOTOMETRY} to
obtain sky values and crude aperture photometry for the newly-found
stars, using the coordinates obtained in step XI.  (You are performing
this operation on the original picture so that the sky estimates will
be consistent with the sky estimates obtained for the original star
list.)

\item{XIII.} Use {\bf GROUP} on the new aperture-photometry file you
just created.  Use {\bf GROUP} again on the profile-fitting photometry
file created in step IX (this step is unfortunately necessary to put
both the old and new photometry into files with the same format, so
that you can $\ldots$).  Use {\bf APPEND} to combine the two group
files just created into one.

\item{XIV.} {\bf GROUP} + {\bf SELECT} + {\bf SELECT} + {\bf GROUP} +
{\bf SELECT} + {\bf SELECT} + $\ldots$ + {\bf NSTAR} + {\bf SUBSTAR},
{\it or\/} ALLSTAR:  If for some reason you prefer {\bf NSTAR} to
ALLSTAR (I sure don't), the file just created in step XIII needs to be
run through {\bf GROUP} once again to sort the combined starlist into
the best groupings for the next pass of {\bf NSTAR}.  Watch the table
of group sizes that gets created on your terminal very carefully.  The
multiple-PSF fitting routine is at present capable of fitting no more
than 60 stars at a time.  If any of the groups created by {\bf GROUP}
is larger than 60 stars, the SELECT command can be used to pick out
only those groups within a certain range of sizes.  You would run

\itemitem{(1)}~{\bf SELECT} once to pick out those groups containing
from 1 to 60 stars, putting them in their own file.  You could discard
all groups larger than 60 if you only wanted a representative, as
distinguished from a complete, sample.  Alternatively, you could run

\itemitem{(2)}~{\bf SELECT} again to pick out those groups containing
61 or more stars, putting them in their own file.  Then you would run

\itemitem{(3)}~{\bf GROUP} with a {\it larger\/} critical overlap on
the file created in (2), to produce a new group file with smaller
groups. {\it The photometry for these stars will be poorer than the
photometry for the less-crowded stars picked out in XIV-1.\/}

\item{}Return to (1) and repeat until (a)~all stars are in
groups containing less than or equal to 60 stars, or (b)~(preferred,
and cheaper) enough stars are in groups smaller than 60 that you feel
you can perform your basic astronomical mission.  {\it Then,\/}

\itemitem{(4)}~{\bf NSTAR} as many times as necessary to reduce the
group files just created, and

\itemitem{(5)}~{\bf SUBSTAR} as many times as necessary to subtract the
stars just reduced from the data frame.

\item{} {\it Or\/}, you could get around the whole thing just by
running the {\bf APPEND}ed group file through ALLSTAR.

\item{XV.} {\bf EXIT} from DAOPHOT~II.  Display the star-subtracted
picture created in step XIV on your image-display system. Look for
stars that have been missed, and for galaxies and bad pixels that have
been found and reduced as if they were stars.  If desired, create a
file containing the coordinates of stars you wish to add to the
solution and run {\bf PHOTOMETRY} on these coordinates.  To make one
more pass through the data, you should run this aperture photometry
file through {\bf GROUP}, run the previous {\bf NSTAR} or ALLSTAR
results through {\bf GROUP} (again, necessary in order to get both the
old and new stars into files with the same format), {\bf APPEND} these
files together, and return to step XIV.  {\it Repeat as many times as
you like, and have the time for.}

\item{XVI.}  {\bf EXIT} from DAOPHOT and examine your picture on the
image-display system.  Choose several minimally-crowded, bright,
unsaturated stars.  Make a copy of the output file from your very last
run of {\bf NSTAR} or ALLSTAR and, with a text editor, delete
from this file the data lines for those bright stars which you have
just chosen.  Run DAOPHOT, {\bf ATTACH} your original picture and
invoke {\bf SUBSTAR} to subtract from your picture all the stars
remaining in the edited data file.  With equal ease, you can also
create a file containing {\it only\/} the stars you want to retain ---
you could even use the file containing the list of PSF stars --- and
you can tell {\bf SUBSTAR} to subtract all the stars from the image
{\it except\/} the ones listed here.  In either case, the stars which
you have chosen will now be completely alone and uncrowded in this new
picture --- measure them with the aperture {\bf PHOTOMETRY} routine,
using apertures with a range of sizes up to very large. These data will
serve to establish the absolute photometric zero-point of your image.

\bigskip
\medskip
\centerline{***** NOTE *****}

This has been the reduction procedure for a program field, assumed
to contain some hundreds or thousands of stars, most of which you are
potentially interested in.  The reduction procedure for a standard-star
frame, which may contain only some tens of objects, only a few of which
you are interested in, may be different.  It may be that you will want
to run {\bf FIND} on these frames, and later on match up the stars
found with the ones you want.  Or perhaps you would rather examine
these frames on the image-display system, use the cursor to measure the
coordinates of the stars you are interested in, and create your own
coordinate file for input into {\bf PHOTOMETRY} (step VI).  In any
case, for your standard fields it is possible that you won't bother
with profile fits, but will just use the aperture photometry (employing
a growth-curve analysis) to define the stars' instrumental magnitudes.

\vfill
\eject
\centerline{C.  Descriptions of Main Routines}
\centerline{(In approximate order of use)}
\noindent I.  DAOPHOT~II itself

When you run DAOPHOT~II, the first thing that may occur is the typing
out on your terminal of a brief message, notifying you that some
information to your advantage is now available.  If this message has
changed since the last time you ran the program, answer the question
with a capital or lower-case ``Y$<$CR$>$''.  The program will then type
out the text of the entire message, a section at a time.  It will pause
at the end of each section of the message to allow you to read what
it's written before it rolls off the screen, or to escape to the main
program without reading the rest of the message.  When you reach the
main part of the program, the current values of the optional reduction
parameters will appear on your screen (see the Appendix on Optional
Parameters, and the {\bf OPTIONS} command below).  When you see the
``Command:'' prompt, the program is ready to accept the commands
described below.

\vfill
\eject
\noindent II.  {\bf ATTACH}

     If you want to work on a digital picture, the first thing you
should do is specify the disk filename of that picture with the ATTACH
command:

\bigskip
{\noindent\obeylines\obeyspaces\frenchspacing\tt\baselineskip=0.3truecm
=======================================================================
| COMPUTER TYPES:                                      YOU ENTER:     |
|                                                                     |
| Command:                                             AT filename    |
|                                                                     |
|       Your picture's header comment (if any)                        |
|                                                                     |
|                   Picture size:   nnn  nnn                          |
+---------------------------------------------------------------------+
| or,                                                                 |
|                                                                     |
| COMPUTER TYPES:                                      YOU ENTER:     |
|                                                                     |
| Command:                                             AT             |
|                                                                     |
|                Enter file name:                      filename       |
|                                                                     |
|       Your picture's header comment (if any)                        |
|                                                                     |
|                   Picture size:   nnn  nnn                          |
=======================================================================
}
\bigskip

\noindent Some commands, the ones which operate only on data files,
({\it e.g.} {\bf SORT}, {\bf OFFSET}, {\bf APPEND}), and others which
set the optional parameters ({\bf OPTIONS}, {\bf MONITOR}, {\bf
NOMONITOR}) may be issued to DAOPHOT~II without a prior {\bf ATTACH}
command having been given.  The program will refuse to let you tell it
to perform tasks requiring a picture file (e.g., {\bf FIND}, {\bf
PHOTOMETRY}, {\bf PSF}, {\bf GROUP}, {\bf NSTAR}) unless a picture has
been {\bf ATTACH}ed.

In all implementations of DAOPHOT~II of which I am aware, if the
extension part of your picture's filename is the standard one for that
image format ('.DST' for Caltech data structures, '.imh' for IRAF,
'.bdf' for Midas) it may be omitted.  If it is not the standard
extension (or, on VMS machines, if you wish to specify other than the
most recent version of the image file), the filename-extension must be
included in the {\bf ATTACH} command.

The {\bf ATTACH} command is the only one in DAOPHOT~II which allows the
user to include additional information ({\it viz}. a filename) on the
command line. All other commands are issued simply and without
modification --- the routines will then prompt the user for any
necessary input.

This is also a good time to point out that DAOPHOT~II commands are {\it
case insensitive\/}:  commands and parameter options (see below) may be
entered using either upper or lower case letters, even on Unix
machines.  On VMS machines filenames are also case insensitive:
FILE.EXT and file.ext refer to the same file.  On Unix machines
FILE.EXT, file.ext, FILE.ext, FiLe.ExT, etc.\ are all different.
Finally, for Unix afficionados, I have taught DAOPHOT~II to recognize a
string of characters terminated with a colon (``$\,$:$\,$'') at the
beginning of a filename as a {\it directory\/} name, \`a la VMS.  Thus,
if in your .cshrc file or some similar location, you have a statement
like

\medskip
\noindent setenv ccd
/scr/nebuchadnezzar/mountstromloandsidingspring/1989/ccd-data
\medskip

\noindent then while running DAOPHOT~II in some other directory, you
can refer to an image or other file, obs137, in {\it this\/} directory
as ccd:$\,$obs137.  In fact, I recommend that you do so, because all
file names used by DAOPHOT are limited to 30 characters.  Finally, on
the VMS side you can create multiple versions of the same filename ad
libitum, but Unix doesn't allow it.  Therefore, on the Unix side, if
you try to create a file with the same name as one that already exists,
DAOPHOT~II will type out a warning and give you an opportunity to enter
a new filename.  If you respond to the prompt with a simple carriage
return, the pre-existing file will be deleted before the new one is
created.  If you write command procedures to handle your DAOPHOT~II
reductions (I know I do), I recommend that you include in your
procedures Unix ``rm'' commands to remove any files you know you're
going to be creating.  Otherwise you run the risk of getting unexpected
prompts asking whether you want to overwrite pre-existing files, and
your list of commands and filenames can get out of synch with what the
program is asking for.  Chaos could result.

\vfill
\eject
\noindent III.  {\bf OPTIONS}

Numerous parameters which optimize the reduction code for the specific
properties of your picture may be altered.  The sigificance of each one
is described below in reference to the particular routines affected, and
a reference table is presented as an Appendix.  Here the options will
be simply enumerated, and the procedures available for changing the
parameters will be described.

At present the user is permitted to specify values for nineteen
parameters:

\item{(1)}  ``READ NOISE'':  The readout noise, in data numbers, of a
{\it single\/} exposure made with your detector.  Later on, the
software will allow you to specify whether a given data frame is in
fact the sum or the average of several individual exposures.  If you
have a separate subdirectory for data from a given night or observing
run, the readout noise needs to be specified only once (in the
DAOPHOT.OPT file, see below).

\item{(2)}  ``GAIN'':  The gain factor of your detector, in photons or
electrons per data number.  As with the readout noise, you want to
specify the gain corresponding to a {\it single\/} exposure; allowance
can be made later for frames that are in fact the averages or sums of
several exposures.

\item{}  {\it For both READ NOISE and GAIN the default values are
{\bf deliberately invalid}.  You must put correct values for these
parameters in a DAOPHOT.OPT file, or the program will hassle you
again, and again, and again, and $\ldots$}

\item{(3)}  ``LOW GOOD DATUM'':  The level, in standard deviations
below the frame's mean sky value, below which you want the program to
consider a pixel defective.  If the background is flat across the
frame, then you can set a tight limit:  maybe 5$\sigma$ or so.  If
there is a strong background gradient, you will need to set a more
generous limit --- maybe 10$\sigma$ or more --- to keep legitimate sky
pixels from being rejected in those parts of the frame where the
background is faint.  Intelligent use of the {\bf DUMP} command and/or
an image display will help you decide what to do.

\item{(4)}  ``HIGH GOOD DATUM'':  The level, in data numbers, above
which a pixel value is to be considered defective.  Note that this
differs from the ``LOW GOOD DATUM'' just discussed.  The ``LOW GOOD
DATUM'' is defined as a certain number of {\it standard deviations it
below a frame's mean sky value\/}. Thus, assuming that all your frames
have comparably flat backgrounds, it needs to be specified only once;
the actual numerical value used will ride up and down as frames with
different mean background levels are considered.  The ``HIGH GOOD
DATUM'' is specified as a single, fixed number which represents the
absolute level in data numbers at which the detector becomes non-linear
or saturates.  (Note that since your data have been bias-level
subtracted and flat-fielded, this number will {\it not\/} be 32767, but
will be somewhat lower.)

\item{(5)}  ``FWHM'':  The approximate FWHM, in pixels, of the objects
for which the {\bf FIND} algorithm is to be optimized.  This parameter
determines the width of the Gaussian function and the size of the array
with which your picture is numerically convolved by {\bf FIND} (see
detailed discussion under {\bf FIND} command below).  If conditions
during your run were {\it reasonably\/} constant, a single value should
be adequate for all your frames.

\item{(6)}  ``THRESHOLD'':  The significance level, in standard
deviations, that you want the program to use in deciding whether a
given positive brightness enhancement is real.  Normally, somewhere
around 4$\sigma$ is good, but you may want to set it a little higher
for the first pass.  Then again, maybe not.

\item{(7), (8)}  ``LOW'' and ``HIGH SHARPNESS CUTOFF'':  Minimum and
maximum values for the sharpness of a brightness enhancement which {\bf
FIND} is to regard as a real star (intended to eliminate bad pixels;
may also help to eliminate low-surface-brightness galaxies). In most
cases the default values given in the program are adequate, but if you
want to fine-tune them, here they are.

\item{(9), (10)}  ``LOW'' and ``HIGH ROUNDNESS CUTOFF'':  Minimum and
maximum values for the roundness of a brightness enhancement which {\bf
FIND} is to regard as a real star (intended to eliminate bad rows and
columns, may also reject some edge-on galaxies). Again, I think you'll
find that the program-assigned default values are adequate for all but
special cases.

\item{(11)}  ``WATCH PROGRESS'':  Whether to display results on the
computer terminal in real time as they are computed.  Displaying the
results may keep you entertained as the reductions proceed, but it may
slow down the execution time, and in batch mode, it will fill up your
logfile excessively.

\item{(12)}  ``FITTING RADIUS'':  Most obviously, this parameter
defines the circular area within which pixels will actually be used in
performing the profile fits in {\bf PEAK} and {\bf NSTAR}:  As the
point-spread function is shifted and scaled to determine the position
and brightness of each of your program stars, only those pixels within
one fitting radius of the centroid will actually be used in the fit.
More subtly, the same region is also used in fitting the analytic first
approximation to the point-spread function for the PSF stars. Moreover,
the parameter will also contribute in a minor way to the determination
of when stars overlap ``significantly.''  Under normal circumstances,
this radius should be of order twice the half-width at half-maximum of
a stellar image which is, obviously, the same as the FWHM.  When the
crowding is extremely severe, however, it may be advantageous to use a
value somewhat smaller than this.  On the other hand, if the
point-spread function is known to vary across the frame, then {\it
increasing} the fitting radius beyond the FWHM may improve the
photometric accuracy provided, of course, that the field is {\it not\/}
horribly crowded.

\item{(13)} ``PSF RADIUS'':  The radius, in pixels, of the circle
within which the point-spread function is to be defined.  This should
be somewhat larger than the actual radius of the brightest star you are
interested in, as you would measure it on your image display.  If,
toward the end of your reductions (see \S B above), you notice that the
subtracted images of your bright stars are surrounded by luminous halos
with sharp inner edges, then your PSF radius is too small. On the other
hand, the CPU time required for the profile-fitting reductions is a
strong function of the PSF radius, so it is counterproductive to make
this parameter too large.

\item{(14)} ``VARIABLE PSF'':  The degree of complexity with which the
point-spread function is to be modeled.  In its infancy, DAOPHOT
Classic allowed only one form for the model PSF:  a Gaussian analytic
first approximation, plus a look-up table of empirical corrections from
the approximate analytic model to the ``true'' PSF.  This now
corresponds to VARIABLE~PSF~=~0.  Later on, I added the possibility of
a point-spread function which varies linearly with position in the
frame; this is VARIABLE~PSF~=~1.  DAOPHOT~II now allows two more
possibilities:  a point-spread function which varies quadratically with
position in the frame (VARIABLE~PSF~=~2), and a purely analytic model
PSF, with no empirical lookup table of corrections, as in ROMAFOT
(VARIABLE~PSF~=~--1).  Probably best to leave it at 0.0 (=
``Constant'') until you are {\it sure\/} you know what you're doing.

\item{(15)} ``FRACTIONAL PIXEL EXPANSION'':  Not implemented.  Leave it
alone.

\item{(16)} ``ANALYTIC MODEL PSF'':  DAOPHOT Classic always used a
Gaussian function as an analytic first approximation to the
point-spread function.  DAOPHOT~II allows a number of alternatives,
which will be discussed below under the {\bf PSF} command.

\item{(17)}  ``EXTRA PSF CLEANING PASSES'':  DAOPHOT~II is now
empowered to recognize and reduce the weight of obviously discrepant
pixels while generating the average model point-spread function for the
frame --- cosmic rays, poorly subtracted neighbors and the like.  This
parameter specifies the number of times you want the program to go
through the data and reevaluate the degree to which any given pixel is
discordant and the amount by which its weight is to be reduced.  Set
this parameter to 0 if you want every pixel accepted at face value with
full weight.  The amount of time taken by the routine increases with
the number of extra passes, and in my experience the point-spread
function has usually converged to a stable final value within five
passes, so I guess that's a reasonable guess at the largest value you'd
want to use.

\item{(18)}  ``PERCENT ERROR'':  In computing the standard error
expected for the brightness value in any given pixel, the program
obviously uses the readout noise and the Poisson statistics of the
expected number of photons.  This parameter allows you to specify a
particular value for the uncertainty of the {\it fine-scale\/}
structure of the flat field.  The readout noise is a constant; the
Poisson error increases as the square root of the intensity; the
``PERCENT ERROR'' increases linearly with the intensity of (star +
sky).  You may think of it as the graininess of the inappropriateness
of the flat-field frame which you used to calibrate your program images
--- not just the photon statistics but also any fine structure (on a
scale smaller than a seeing disk) in the mismatch between the
illumination of your flat-field frame and your program images.

\item{(19)} ``PROFILE ERROR'':  In fitting the point-spread function to
actual stellar images, there will also be some error due to the fact
that the point-spread function is not known to infinite precision:  not
only will there be interpolation errors due to the finite sampling, but
the point-spread function may vary in a greater than quadratic fashion
with position in the frame, or with apparent magnitude, or with color,
or with something else.  This parameter defines the amplitude of this
further contribution to the noise model; the ``PROFILE ERROR''
increases linearly with the intensity of the star alone (no sky), and
inversely as the fourth power of the full-width at half-maximum.
Therefore, this error grows in importance relative to the PERCENT ERROR
as the seeing improves (since interpolation becomes harder as the data
become more undersampled). Leave parameters (18) and (19) alone until
much, much later.

\noindent There are four ways in which the user can supply values for
these parameters to the program:

\item{(1a)} Whenever you run DAOPHOT~II, the first thing the program
will do is look in your current default directory for a file named
DAOPHOT.OPT (daophot.opt on Unix machines).  If it finds such a file,
then it will read in the parameter specifications according to the
format described below.  When you run DAOPHOT~II the file DAOPHOT.OPT
acts pretty much the way LOGIN.COM does when you log onto a VMS VAX.
Note that DAOPHOT~II will only look for DAOPHOT.OPT in the directory
from which you are currently running the program, so you should have a
copy of DAOPHOT.OPT in each subdirectory where you are likely to work.
If you have one subdirectory for every set of matched frames you are
working on, it is easy to set up optimum parameter values for each set,
and have them invoked automatically whenever you run DAOPHOT from that
set's directory.  {\it At the very least, you should have a}
DAOPHOT.OPT {\it file specifying the READOUT NOISE, GAIN, FWHM, FITTING
RADIUS, and HIGH GOOD DATA VALUE in {\bf every} subdirectory where you
keep images for} DAOPHOT~II.

\item{(1b)}  If no file named DAOPHOT.OPT is found, then default values
for the parameters, as defined in the table in the Appendix, will be
supplied by the program.

\item{(2a)}  Whenever you have the `Command:' prompt, you may use the
{\bf OPTIONS} command.  The program will type on the terminal the
current values for all user-definable parameters, and then ask you for
an input filename.  You may then enter the name of a file containing
values (same format as used in a DAOPHOT.OPT file) you want to specify
for the parameters.

\item{(2b)}  When the {\bf OPTIONS} command asks you for a filename,
you may simply type a carriage return, and the program will then permit
you to enter parameter values from the keyboard.

\noindent The syntax for specifying a parameter value, either from
within a file or from the keyboard, is as follows.  The parameter you
wish to define is indicated by two alphanumeric characters; it doesn't
matter whether they are upper or lower case, and any spaces or
additional characters (except an equals sign and a number) after the
first two are optional.  The parameter identifier is followed by an
equals sign, and this is followed by a number.  The following commands
would all set the FWHM of the objects for which the search is to be
optimized to the value 3.0:

\indent\indent\indent FW=3.0

\indent\indent\indent fwhm = 3

\indent\indent\indent Fwied wice is nice = 3.

\noindent When inputing parameter values from a file, one parameter is
specified per line.  Note that only those parameters whose values you
want to change from the program-supplied defaults need be supplied by
the user, either in manner (1a) above, or in (2a) or (2b).

You exit from OPTIONS by responding to the OPT$>$ prompt with a
carriage return or a CTRL-Z (CTRL-D on some Unix machines?  Whatever
means END-OF-FILE on your system).

\vfill
\eject
\noindent EXAMPLE: CHANGING VALUES FROM THE KEYBOARD

To change the estimated full-width at half-maximum from 2.5 to 3.5,
and the high sharpness cutoff from 1.0 to 0.9:

\bigskip
{\noindent\obeylines\obeyspaces\frenchspacing\tt\baselineskip=0.3truecm
=============================================================================
| COMPUTER TYPES:                                       YOU ENTER:          |
|                                                                           |
| Command:                                              OP                  |
+---------------------------------------------------------------------------+
| COMPUTER TYPES:                                                           |
|                                                                           |
|  READ NOISE (ADU; 1 frame) =  5.00      GAIN (e-/ADU; 1 frame) =    10.00 |
|    LOW GOOD DATUM (sigmas) =  7.00    HIGH GOOD DATUM (in ADU) = 32766.50 |
|             FWHM OF OBJECT =  2.50       THRESHOLD (in sigmas) =     4.00 |
|  LS (LOW SHARPNESS CUTOFF) =  0.20  HS (HIGH SHARPNESS CUTOFF) =     1.00 |
|  LR (LOW ROUNDNESS CUTOFF) = -1.00  HR (HIGH ROUNDNESS CUTOFF) =     1.00 |
|             WATCH PROGRESS =  1.00              FITTING RADIUS =     2.00 |
|                 PSF RADIUS = 11.00                VARIABLE PSF =     0.00 |
| FRACTIONAL-PIXEL EXPANSION =  0.00          ANALYTIC MODEL PSF =     1.00 |
|  EXTRA PSF CLEANING PASSES =  5.00        PERCENT ERROR (in \%) =     0.75 |
|       PROFILE ERROR (in \%) =  5.00                                        |
+---------------------------------------------------------------------------+
| COMPUTER TYPES:                                       YOU ENTER:          |
|                                                                           |
|     Parameter file (default KEYBOARD INPUT):          <CR>                |
|                                                                           |
| OPT>                                                  FW=3.5              |
|                                                                           |
| OPT>                                                  HS=0.9              |
|                                                                           |
| OPT>                                                  <CR>                |
+---------------------------------------------------------------------------+
| COMPUTER TYPES:                                                           |
|                                                                           |
|  READ NOISE (ADU; 1 frame) =  5.00      GAIN (e-/ADU; 1 frame) =    10.00 |
|    LOW GOOD DATUM (sigmas) =  7.00    HIGH GOOD DATUM (in ADU) = 32766.50 |
|             FWHM OF OBJECT =  3.50       THRESHOLD (in sigmas) =     4.00 |
|  LS (LOW SHARPNESS CUTOFF) =  0.20  HS (HIGH SHARPNESS CUTOFF) =     0.90 |
|  LR (LOW ROUNDNESS CUTOFF) = -1.00  HR (HIGH ROUNDNESS CUTOFF) =     1.00 |
|             WATCH PROGRESS =  1.00              FITTING RADIUS =     2.00 |
|                 PSF RADIUS = 11.00                VARIABLE PSF =     0.00 |
| FRACTIONAL-PIXEL EXPANSION =  0.00          ANALYTIC MODEL PSF =     1.00 |
|  EXTRA PSF CLEANING PASSES =  5.00        PERCENT ERROR (in \%) =     0.75 |
|       PROFILE ERROR (in \%) =  5.00                                        |
=============================================================================
}
\vfill
\eject
\noindent IV.  {\bf SKY}

The first time you start to work on a new frame with DAOPHOT~II, you
might want to issue the {\bf SKY} command, which will return an
estimate of the typical sky brightness for the frame.

\bigskip
{\noindent\obeylines\obeyspaces\frenchspacing\tt\baselineskip=0.3truecm
=======================================================================
| COMPUTER TYPES:                                      YOU ENTER:     |
|                                                                     |
| Command:                                             SK             |
|                                                                     |
|     Approximate sky value for this frame =   156.8                  |
|     Standard deviation of sky brightness =     4.16                 |
|                                                                     |
|                  Clipped mean and median =   157.9     157.5        |
|       Number of pixels used (after clip) =  8673                    |
=======================================================================
}
\bigskip

\noindent The sky value returned is an estimate of the mode of the
intensity values in somewhat fewer than 10,000 pixels scattered
uniformly throughout the frame. That is, it picks 10,000 pixels, clips
the low and high tails after the fashion of Kitt Peak's Mountain
Photometry code, and computes the mean and median from what is left.
The mode is taken as three times the median minus twice the mean.  The
standard deviation is the one-sigma width of the peak of the
sky-brightness histogram about the {\it mean\/} sky brightness ---
({\it not\/} the mode or the median --- after clipping; for all but
horrendously crowded frames this distinction is negligible for our
present purposes).  If you don't want to run the {\bf SKY} command,
{\bf FIND} will do it for you anyhow.

\vfill
\eject
\noindent V.  {\bf FIND}

You are now ready to find stars.

\bigskip
{\noindent\obeylines\obeyspaces\frenchspacing\tt\baselineskip=0.3truecm
=======================================================================
| COMPUTER TYPES:                                      YOU ENTER:     |
|                                                                     |
| Command:                                             FI             |
|                                                                     |
|                                                                     |
|   Approximate sky value for this frame =   156.8                    |
|   Standard deviation of sky brightness =     4.16                   |
|                                                                     |
|                        Relative error = 1.14                        |
|                                                                     |
|        Number of frames averaged, summed:            5,1            |
|                                                                     |
|   File for the positions (default ?.COO):            <CR>           |
|                                                   or filename.ext   |
|                                                                     |
=======================================================================
}
\bigskip

\noindent The ``Number of frames averaged, summed'' question is in case
the frame you are about to reduce represents the average or the sum of
several independent readouts (readsout?) of the chip.  The program uses
this information to adjust the readout noise and the gain
appropriately.  In the example here, I have assumed that five
individual exposures were {\it averaged\/} together to make the frame
being reduced here; that is the five frames were added together and the
sum was divided by the scalar value 5.  If the frames had been added
together and {\it not\/} divided by five, I would have entered ``1,5''
instead of ``5,1''.  If I had taken six frames, summed them by pairs,
and then averaged the three different sums, I would have entered
``3,2'': meaning ``the average of three sums of two''.  If I had taken
six frames, averaged the first three, averaged the second three, and
then summed the two averages, I would also have entered ``3,2'':
``averages of three, sum two of them''.  One final nuance:  it happens
that from a statistical noise point of view, the median of three frames
is about as good as the average of two.  Therefore, if the frame you
are reducing represents the {\it median\/} of several independent
exposures (to remove cosmic rays or whatever) enter it as if it were
the average of two-thirds as many frames:  the median of three images
would be entered as ``2,1'', and the median of five would be entered as
``3.3,1''.

With this information, the routine will calculate a star-detection
threshold corresponding to the number of standard deviations which you
entered as the ``THRESHOLD'' option.  The derived value of this
threshold represents the minimum central height of a star image, in
ADU, above its {\it local\/} sky background, which is required for a
detection to be regarded as statistically significant.  The theory
behind star-detection is discussed briefly below, and the method used
for computing the optimum threshold is described in the Appendix on
``The {\bf FIND} Threshold.'' The routine also computes a lowest good
data-value corresponding to the number of standard deviations you
specified in the ``LOW GOOD DATUM'' option, relative to the {\it
average\/} sky brightness.  That is, having determined that the modal
sky level in this frame is 156.8, it determines that the brightness
7$\sigma$ below this is 136.8.  For this calculation, it uses the
specified readout noise, gain, and the fact that this is the average of
five frames, {\it not\/} the {\it observed\/} $\sigma$(sky) = 4.16,
because this latter value may have been puffed up by stars and
defects.  Then, any pixel {\it anywhere in the frame\/} which has an
observed brightness value less than 136.8 or greater than the ``HIGH
GOOD DATUM'' which you specified directly, will now and ever after be
discarded as ``defective.''  If you want to see what numerical values
the routine gave to the star-detection threshold and the lowest good
data value, you can find them in the second line of the output file
created by this routine.  If you want to {\it change\/} either of these
values for subsequent reduction steps, you {\it must\/} do it in the
file header lines; all other routines read these numbers from the input
files, {\it not\/} from the optional-parameter table.  The same goes
for the HIGH GOOD DATUM value.  Therefore, it is entirely in your own
best interests to provide the program with the most reasonable possible
estimates of the readout noise and gain, and of the minimum and maximum
valid data values; doing it correctly at the beginning will save you
hassles later.  For instance, if the sky background in your frame is
dead flat (a random Galactic star field, not in any cluster), a strict
value for the lowest good data-value might be, say, five sigma below
the average sky.  A five-sigma or greater deviation has a normal
probability of about $3 \times 10^{-7}$, so in a $300 \times 500$
image, there would be only about one chance in twenty that even one
legitimate datum would be unfairly rejected.  Of course, if the sky
background does vary significantly across the frame (in a globular
cluster, H~II region, or external galaxy), you would want to set the
minimum good data value maybe ten or more sigma below the average sky.

Finally, {\bf FIND} asks you to provide a name for the disk file where
it is to store the coordinates of the stars it finds.  Here, as you
will find nearly everywhere in DAOPHOT~II, when asking you for a
filename the program will offer you a default.  If you are satisfied
with the default filename, you need only type a carriage return and
this name will be used; if you want to change the filename but keep the
default filename extension, type in the filename you want, without any
extension or period, and the default filename's extension will be
tacked onto the filename you enter.  Similarly, if you like the
filename part but want to change the extension, simply type a period
and the new extension: the filename which was offered to you will be
retained, but with your extension replacing the one offered.  I
strongly suggest that you use the default filenames unless you have
some very good reason for not doing so --- it reduces the amount of
typing that you have to do (and thereby reduces the probability of an
unnoticed typographical error) and it helps you to keep your
bookkeeping straight.

If you have elected to watch the output of the program on your terminal
(either by using the option WATCH PROGRESS = 1.0 or by using the
MONITOR command, see below), you will now see the computer count
through the rows of your picture.  As it is doing this, it is
convolving your picture with a lowered truncated Gaussian function
whose FWHM is equal to the value set by the FWHM option (see the {\bf
OPTIONS} command above).  The Gaussian convolution includes
compensation for the local background level, and since the Gaussian is
symmetric, smooth gradients in the sky brightness also cancel out.
These properties enable the sky-finding algorithm to ignore smooth,
large-scale variations in the background level of your frame, such as
those caused by a sea of unresolved fainter stars --- the threshold
that you have specified represents the minimum central
brightness-enhancement {\it over the local background\/} which an
object must have in order to be detected.

After having performed the convolution the program will then go through
the convolved data looking for local maxima in the brightness
enhancement. As the program finds candidates, it computes a couple of
image-shape statistics (named SHARP and ROUND) which are designed to
weed out delta functions (bad pixels), and brightness enhancements that
are elongated in $x$ or $y$ (bad rows and columns).

SHARP is defined as the ratio of the height of the bivariate
delta-function which best fits the brightness peak in the original
image to the height of the bivariate Gaussian function (with the
user-supplied value of the FWHM) which best fits the peak.  If the
brightness enhancement which was found in the convolved data is due to
a single bright (``hot'') pixel, then the best-fitting delta-function
will have an amplitude equal to the height of this pixel above the mean
local background, while the amplitude of the best-fitting Gaussian will
be pulled down by the surrounding low-valued pixels, hence SHARP $>
1$.  On the other hand, where there is a cold pixel, that is to say,
where there is an isolated pixel whose brightness value is below the
local average (but still above the ``lowest good data-value'') in the
convolved data there will tend to be brightness-enhancements found
approximately 0.5 FWHM away from this pixel in all directions; in such
a case, the height of the delta function which best fits one of these
spurious maxima tends to be close to zero, while the height of the
best-fitting Gaussian is some small positive number: SHARP $\sim 0$.
To reject both types of bad pixel, the default acceptance region for
the SHARP parameter is:

$$
0.20 \leq \hbox{\rm SHARP} = {\hbox{\rm height of best-fitting delta function}
\over \hbox{\rm height of best-fitting Gaussian function}} \leq 1.00
$$

ROUND is computed from the data in the original picture by fitting
one-dimensional Gaussian functions to marginal sums of the data in $x$
and $y$. Specifically, for each brightness enhancement which passes the
SHARP test, if the height of either of these one-dimensional Gaussian
distributions happens to be negative (a local minimum in the brightness
distribution in that coordinate --- sometimes happens) or zero, the
object is rejected.  If both turn out to be positive, then the ROUND
parameter is computed:

$$
\hbox{\rm ROUND} =
{\hbox{\rm difference between the heights of the two one-dimensional Gaussians}
\over
\hbox{\rm average of the heights of the two one-dimensional Gaussians}}
$$

\noindent Thus, if the two heights are, say, 90 and 150 ADU, then the
average is 120 ADU and the difference is $\pm$60 ADU, so that ROUND would
be $\pm$0.5.  The sense of the difference is such that an object which is
elongated in the $x$-direction has ROUND $<0$ and one which is
elongated in the $y$-direction has ROUND $>0$.  If the brightness
enhancement which has been detected is really a charge-overflow column,
for instance, then the brightness distribution as a function of $x$
would be sharply peaked, while the distribution as a function of $y$
would be nearly flat; the height of the $x$-Gaussian function would
have some significant value, while that of the $y$-Gaussian would be
near zero.  In this case, ROUND would have a value near +2.0.  The
default acceptance limits for ROUND are

$$-1.0 \leq \hbox{\rm ROUND} \leq 1.0,$$

\noindent i.e., if the heights of the $x$- and $y$-Gaussian
distributions for a brightness enhancement were 60 and 180 ADU,
difference/average = $\pm$120/120 and the object would be right at the
limit of acceptance.  Note that ROUND discriminates only against
objects which are elongated along either rows or columns --- objects
which are elongated at an oblique angle will not be preferentially
rejected.

The numerical values for the limits of the acceptance interval for
SHARP and ROUND may be changed by the user (see the {\bf OPTIONS}
command above),  if normal stars in your frame should happen to have
unusual shapes due to optical aberrations or guiding problems.
However, I recommend that you take great care in deciding on new values
for these cutoffs.  It might be useful to perform a preliminary run of
{\bf FIND} with very generous limits on the acceptance regions, and
then to plot up both SHARP and ROUND as functions of magnitude for the
objects detected (see Stetson 1987, PASP, 99, 191, Fig.~2).  Observe
the mean values of SHARP and ROUND for well-exposed stars, observe the
magnitude fainter than which these indices blow up, and determine the
range of values of SHARP and ROUND spanned by stars above that
magnitude limit.  Having decided on the region of each of the two
diagrams occupied by worthwhile stars, you could then rerun {\bf FIND}
with a new threshold and new values for the limits on SHARP and ROUND.
Alternatively, you could write yourself a quickie program which goes
through the output file produced by {\bf FIND} and rejects those
objects which fall outside your new acceptance limits (which could even
be functions of magnitude if you so desired).

Back in {\bf FIND}:  Once a brightness enhancement passes muster
as a probable star, its centroid is computed (approximately).  When the
FIND routine has gone through your entire picture, it will ask

\bigskip
{\noindent\obeylines\obeyspaces\frenchspacing\tt\baselineskip=0.3truecm
=======================================================================
| COMPUTER TYPES:                                      YOU ENTER:     |
|                                                                     |
| Are you happy with this?                             Y              |
|                                                   or N              |
=======================================================================
}
\bigskip

\noindent If you answer ``Y'', the program will exit from {\bf FIND}
and return you to DAOPHOT ``Command:'' mode.  If you answer ``N'', it
will ask for a new threshold and output filename, and will then search
through the convolved picture again.

A couple more comments on star-finding:

\item{(1)} Stars found in the outermost few rows and columns of the
image, that is, where part of their profile hangs over the edge of the
frame, do not have their positions improved.  Instead, their positions
are returned only to the nearest pixel.  Furthermore, the ROUNDNESS
index is not computed for such stars, although the SHARPNESS test is
still performed.

\item{(2)} Please don't try to set the threshold low enough to pick up
every last one-sigma detection --- this will cause you nothing but
grief later on.  The CPU time for {\bf NSTAR} goes as something like
the fourth power of the surface density of detected objects in your
frame (ALLSTAR isn't quite so bad), so including a large number
of spurious detections greatly increases the reduction time.  Even
worse, as the iterative profile fits progress, if these fictitious
stars have no real brightness enhancements to anchor themselves to,
they can migrate around the frame causing real stars to be fit twice
(in {\bf PEAK} and {\bf NSTAR}, probably not in ALLSTAR) or
fitting themselves to noise peaks in the profiles of brighter stars
(all three routines).  This will add scatter to your photometry.  Try
to set a reasonable threshold. If you want, you can experiment by
repeatedly replying to ``Are you happy with this?'' with ``N'', and on
a piece of graph paper build yourself up a curve showing number of
detected objects as a function of threshold.  This curve will probably
have an elbow in it, with the number of detected objects taking off as
the threshold continues to be lowered. The best threshold value will be
somewhere near this elbow.  (This method of experimenting with
different thresholds is much faster than just running {\bf FIND} many
times, because by answering ``N'' and giving a new threshold, you avoid
the need to recompute the convolution of the picture.)

\item{(3)} The crude magnitudes which {\bf FIND} computes and includes
in your terminal and disk-file output are defined relative to the
threshold which you gave it --- a star with a {\bf FIND} magnitude of
0.000 is right at the detection limit. Since most stars are brighter
than the threshold {\bf FIND} will obviously give them negative
magnitudes.  For the faintest stars the magnitudes may be quantized,
since if the original image data are stored as integers, the convolved
image data will be, too.  Thus, if your {\bf FIND} threshold came out
to 20.0 ADU, the next brighter magnitude a star may have, after 0.000,
is --2.5 log (21/20) = --0.053.  When the image data are stored as
floating-point numbers, this quantization will not occur.

\vfill
\eject
\noindent VI.  {\bf PHOTOMETRY}

Before you can do concentric aperture photometry with DAOPHOT, you
need to have an aperture photometry parameter file.  At the DAO, a
prototype parameter file named DAO:PHOTO.OPT is available.  The
file looks something like this:

\bigskip
{\obeylines\obeyspaces\frenchspacing\tt\baselineskip=0.3truecm
\     =========
\      A1 = 3
\      A2 = 4
\      A3 = 5
\      A4 = 6
\      A5 = 7
\      A6 = 8
\      A7 = 10
\      IS = 20
\      OS = 35
\     =========
}
\bigskip

\noindent When you give DAOPHOT the {\bf PHOTOMETRY} command to invoke the
aperture photometry routine,

\bigskip
{\noindent\obeylines\obeyspaces\frenchspacing\tt\baselineskip=0.3truecm
=======================================================================
| COMPUTER TYPES:                                  YOU ENTER:         |
|                                                                     |
| Command:                                         PH                 |
|                                                                     |
|       Enter table name (default PHOTO.OPT):      <CR> or table name |
=======================================================================
}
\bigskip

\noindent and a table like this one will appear on your terminal screen:

\bigskip
{\noindent\obeylines\obeyspaces\frenchspacing\tt\baselineskip=0.3truecm
==============================================================================

\ A1  RADIUS OF APERTURE  1 =     3.00    A2  RADIUS OF APERTURE  2 =     4.00
\ A3  RADIUS OF APERTURE  3 =     5.00    A4  RADIUS OF APERTURE  4 =     6.00
\ A5  RADIUS OF APERTURE  5 =     7.00    A6  RADIUS OF APERTURE  6 =     8.00
\ A7  RADIUS OF APERTURE  7 =    10.00    A8  RADIUS OF APERTURE  8 =     0.00
\ A9  RADIUS OF APERTURE  9 =     0.00    AA  RADIUS OF APERTURE 10 =     0.00
\ AB  RADIUS OF APERTURE 11 =     0.00    AC  RADIUS OF APERTURE 12 =     0.00
\ IS       INNER SKY RADIUS =    20.00    OS       OUTER SKY RADIUS =    35.00
~~~~~~~~
\ PHO$>$

==============================================================================
}
\bigskip

When you have the ``PHO$>$'' prompt, you can alter any of the values
displayed in the table.  To do this, you first enter the two-character
identifier of the item that you want to change, followed by an
``$\,$=$\,$'' sign, and the new numerical value for that parameter.  It
works just like the {\bf OPTIONS} command:  any characters between the
first two and the equals sign will be ignored, and anything but a
legitimate decimal number after the equals sign will produce an error
message.  For instance, to change the radius of the first aperture to
2.5 pixels, and change the inner and outer radii of the sky annulus to
10 and 20 pixels, respectively:

\bigskip
{\noindent\obeylines\obeyspaces\frenchspacing\tt\baselineskip=0.3truecm
=======================================================================
| COMPUTER TYPES:                                  YOU ENTER:         |
|                                                                     |
| PHO>                                             A1=2.5             |
|                                                                     |
| PHO>                                             IS=10              |
|                                                                     |
| PHO>                                             OS=20              |
|                                                                     |
| PHO>                                             <CR>               |
|                                                                     |
=======================================================================
}
\bigskip

\noindent and the modified table will appear on the screen:

\bigskip
{\noindent\obeylines\obeyspaces\frenchspacing\tt\baselineskip=0.3truecm
==============================================================================

\ A1  RADIUS OF APERTURE  1 =     2.50    A2  RADIUS OF APERTURE  2 =     4.00
\ A3  RADIUS OF APERTURE  3 =     5.00    A4  RADIUS OF APERTURE  4 =     6.00
\ A5  RADIUS OF APERTURE  5 =     7.00    A6  RADIUS OF APERTURE  6 =     8.00
\ A7  RADIUS OF APERTURE  7 =    10.00    A8  RADIUS OF APERTURE  8 =     0.00
\ A9  RADIUS OF APERTURE  9 =     0.00    AA  RADIUS OF APERTURE 10 =     0.00
\ AB  RADIUS OF APERTURE 11 =     0.00    AC  RADIUS OF APERTURE 12 =     0.00
\ IS       INNER SKY RADIUS =    10.00    OS       OUTER SKY RADIUS =    20.00
~~~~~~~~
\         File with the positions (default ?.COO):

==============================================================================
}
\bigskip

\noindent Note that you are allowed to specify radii for up to twelve
concentric apertures.  These do not need to increase or decrease in any
particular order\footnote{$^*$}{If you plan to use DAOGROW, the
aperture radii must increase monotonically.}, except that only the
magnitude in the first aperture will appear on the screen of your
terminal as the reductions proceed, and the magnitude in the first
aperture will be used to define the zero point and to provide starting
guesses for the profile-fitting photometry. Photometric data for all
apertures will appear in the disk data file created by this routine.
The first zero or negative number appearing in the list of aperture
radii terminates the list.  Thus, the tables above both specify that
photometry through seven apertures is to be obtained, and that the
first aperture is to be 3 pixels in radius in the first instance, 2.5
pixels in radius in the second.  Items IS and OS are the inner and
outer radii, respectively, of a sky annulus centered on the position of
each star.

\vfill
\eject

Now you can proceed to do the photometry:

\bigskip
{\noindent\obeylines\obeyspaces\frenchspacing\tt\baselineskip=0.3truecm
=======================================================================
| COMPUTER TYPES:                                  YOU ENTER:         |
|                                                                     |
|    File with the positions (default ?.COO):      <CR> or filename   |
|                                                                     |
|     File for the magnitudes (default ?.AP):      <CR> or filename   |
|                                                                     |
=======================================================================
}
\bigskip

\noindent If you are monitoring the progress of the reductions on your
terminal, the computer will start spitting out the star ID's and
coordinates from the star-finding results, along with the instrumental
magnitudes from the first aperture and the sky brightness values for
all the stars.  When all the stars have been reduced, a line like the
following will appear:

$$\hbox{\rm Estimated magnitude limit (Aperture 1):  nn.nn +-- n.n per star.}$$

\noindent This is simply a very crude estimate of the apparent
instrumental magnitude of a star whose brightness enhancement was
exactly equal to the threshold you specified for the star-finding
algorithm.  It is intended for your general information only --- it is
not used elsewhere in DAOPHOT~II, and it has meaning only if all the
stars were found by the {\bf FIND} routine.  If you have entered some
of the coordinates by hand, or if you are redoing aperture photometry
from data files that have already undergone later stages of reduction,
the magnitude limit is not meaningfully defined.

Other things you should know about {\bf PHOTOMETRY}:

\item{(1)} The maximum outer radius for the sky annulus depends on how
much virtual memory your system manager will let the program use, and
on how big you set the inner radius.  If you set too large a value the
program will complain and tell you the largest radius you can get away
with.

\item{(2)} If any of the following conditions is true, {\bf PHOTOMETRY}
will assign an apparent magnitude of 99.999 $\pm$ 9.999 to your star:

\itemitem{(a)} if the star aperture extends outside the limits of the picture;

\itemitem{(b)} if there is a bad pixel (either above the ``High good
datum'' or below the ``Low good datum'') inside the star aperture;

\itemitem{(c)} if the star + sky is fainter than the sky; or

\itemitem{(d)} if for some reason the program couldn't determine a
reasonable modal sky value (very rare; this latter case will be flagged
by $\sigma_{\hbox{\it sky}} = -1$).

\item{(3)} {\it Very important\/}:  If, after the rejection of the
tails of the sky histogram, the sky-finding algorithm is left with
fewer than 20 pixels, {\bf PHOTOMETRY} will refuse to proceed, and will
return to to DAOPHOT~II command mode.  {\it Check your inner and outer
sky annulus radii, and your good data-value limits.\/}  In particular,
if you are reducing a short-exposure frame, where the mean sky
brightness is small compared to the readout noise, the lowest good
data-value (computed in {\bf FIND}, see above) should have been set to
some moderate-sized negative number.  If this is not the case there'll
be ``bad'' pixels all over the place and you'll never get any
photometry done.  As suggested above, I generally set the bad pixel
threshold somewhere around five to six sigma below the typical sky
brightness (obtained from the {\bf SKY} command, see above) unless the
background is significantly non-uniform, in which case I set it even
lower.  But, if you are absolutely sure that the sky background does
not change significantly across your frame and you want to be really
clever, you can set the threshold to something like 4.35 sigma below
the mean sky brightness:  the one-sided tail of the cumulative normal
distribution, $\hbox{\rm Prob}(X < -4.35 \sigma) = 7\times 10^{-6}$, or
about one legitimate random pixel being spuriously identified as
``bad'' in a $300 \times 500$ picture.

\vfill
\eject
\noindent VII.  {\bf PICK} and {\bf PSF}

In my opinion, obtaining a point-spread function in a crowded field is
still done best with at least a small modicum of common (as
distinguished from artificial) intelligence. One possible procedure for
performing this task is outlined in an Appendix. At this point I will
say only that in the beginning you may find it to your advantage to
perform this task interactively, while you are able to examine your
original picture and its offspring (with various of the stars
subtracted out by procedures described in the Appendix) on an image
display system.  After you find that you are performing exactly the
same operations in exactly the same sequence for all your frames, you
may choose to write command procedures to carry out the bulk of this
chore, with a visual check near the end.

DAOPHOT~Classic assumed that the point-spread function of a CCD image
could be adequately modeled by the sum of (a)~an analytic, bivariate
Gaussian function with some half-width in the $x$-direction and some
half-width in the $y$-direction, and (b)~an empirical look-up table
representing corrections from the best-fitting Gaussian to the actual
observed brightness values within the average profile of several stars
in the image.  This hybrid point-spread function seemed to offer both
adequate flexibility in modelling the complex point-spread functions
that occur in real telescopes, with some hope of reasonable
interpolations for critically sampled or slightly undersampled data.
This approximation has since turned out to be not terribly good for
significantly undersampled groundbased data, and it has turned out to
be particularly poor for images from the Hubble Space Telescope.  To
help meet these new requirements, DAOPHOT~II offers a wider range of
choices in modelling the point-spread function.  In particular,
different numerical models besides the bivariate Gaussian function may
be selected for the analytic first approximation.  These are selected
by the ``ANALYTIC MODEL PSF'' option, as I will explain later.

But to back up a bit, to aid you in your choice of PSF stars, I have
provided a very simple routine called {\bf PICK}, which does some
fairly obvious things.  First, it asks for an input file containing a
list of positions and magnitudes:  by no coincidence at all, the
aperture photometry file you just created with {\bf PHOTOMETRY} is
ideal:

\bigskip
{\noindent\obeylines\obeyspaces\frenchspacing\tt\baselineskip=0.3truecm
=======================================================================
| COMPUTER TYPES:                                  YOU ENTER:         |
|                                                                     |
| Command:                                         PICK               |
|                                                                     |
|             Input file name (default ?.AP):      <CR> or filename   |
|                                                                     |
|                Desired number of PSF stars:      <some number>      |
|                                                                     |
|           Output file name (default ?.LST):      <CR> or filename   |
|                                                                     |
|                                                                     |
|  <Some number of> suitable candidates were found.                   |
|                                                                     |
|                                                                     |
| Command:                                                            |
=======================================================================
}
\bigskip

You tell the thing how many PSF stars you'd like to have.  It then
sorts the input star list by apparent magnitude (note that if you have
set the HIGH GOOD DATUM parameter appropriately, any saturated stars
should have magnitudes of 99.999 and will appear at the {\it end\/} of
the list), and then it uses the FITTING RADIUS and the PSF RADIUS that
you have specified among the optional parameters to eliminate:
(a)~stars that are too close to the edge of the frame (within one
FITTING RADIUS), and (b)~stars that are too close to {\it brighter\/}
stars or possibly saturated stars (within one PSF RADIUS {\it plus\/}
one FITTING RADIUS).  Any stars that remain after this culling process
will be written out in order of increasing apparent magnitude, up to
the number that you have specified as a target.

With the ``.LST'' file containing the candidate PSF stars, you are
ready to run the {\bf PSF} routine:

\bigskip
{\noindent\obeylines\obeyspaces\frenchspacing\tt\baselineskip=0.3truecm
=======================================================================
| COMPUTER TYPES:                                  YOU ENTER:         |
|                                                                     |
| Command:                                         PSF                |
|                                                                     |
|  File with aperture results (default ?.AP):      <CR> or filename   |
|                                                                     |
|        File with PSF stars (default ?.LST):      <CR> or filename   |
|                                                                     |
|           File for the PSF (default ?.PSF):      <CR> or filename   |
|                                                                     |
=======================================================================
}
\bigskip

What happens next depends upon the ``WATCH PROGRESS'' option.  If
``WATCH PROGRESS'' has been set to 1 or 2, the program will produce
little pseudo-images of your PSF stars on your terminal, one by one.
These images are made up of standard alphanumeric characters, so it
should work whether you are on a graphics terminal or not.  After
showing you each picture, the routine will ask you whether you want
this star included in the average PSF.  If you answer ``y'' or ``Y'',
the star will be included, if you answer ``n'' or ``N'', it won't.

If the PSF routine discovers that a PSF star has a bad pixel (defined
as being below the low good data value or above the HIGH GOOD DATUM
value) {\it inside\/} one FITTING RADIUS of the star's centroid, it
will refuse to use the star.  If it discovers that the star has a bad
pixel {\it outside\/} one FITTING RADIUS but inside a radius equal to
(one PSF RADIUS plus two pixels), what it does again depends on the
WATCH PROGRESS option.  If WATCH PROGRESS = 0, 1, or 2, {\bf PSF} will
inform you of the existence of the bad pixels and ask whether you want
to use that star anyway.  If you answer ``y'' or ``Y'' you are counting
on the bad-pixel rejection scheme later on in {\bf PSF} to eliminate
the flaw, while extracting whatever information it can from the
remaining, uncontaminated part of the star's profile.  In my
experience, this has always worked acceptably well.  If you have set
WATCH PROGRESS to --2 or --1, {\bf PSF} will type out a message about
the bad pixel(s), but will go ahead and use the star anyway without
prompting for a response.

After {\bf PSF} has read your input list of PSF stars$\ldots$

\vfill
\eject
{\noindent\obeylines\obeyspaces\frenchspacing\tt\baselineskip=0.3truecm
=======================================================================
| COMPUTER TYPES:                                                     |
|                                                                     |
|     Chi      Parameters...                                          |
|   0.0282    0.71847   0.83218   0.30568                             |
|                                                                     |
|  Profile errors:                                                    |
|                                                                     |
|   180 0.023    555 0.026    122 0.026    531 0.023    725 0.026     |
|   821 0.026    759 0.029     82 0.063 ?   19 0.028    303 0.027     |
|   784 0.027    189 0.029    660 0.028    536 0.026    427 0.025     |
|    92 0.028    766 0.027    873 0.025    512 0.029    456 0.096 *   |
|   865 0.023    752 0.027    715 0.026    125 0.026    440 0.026     |
|   375 0.026    467 0.030     99 0.028    652 0.029                  |
|                                                                     |
| File with PSF stars' neighbors = ?.NEI                              |
|                                                                     |
=======================================================================
}
\bigskip

If WATCH PROGRESS $\geq$ --1 the numbers under the ``Chi
Parameters$\ldots$'' heading will dance about for a while.  (If you are
doing these calculations in a batch job which creates a logfile, you
should set WATCH PROGRESS = --2.  This will suppress the dancing, which
--- especially under Unix --- could cause a clog in the log.)  When the
dancing stops, the computer has fit the analytic function of your
choice to the (in this case) 29 PSF stars.  The value labeled ``Chi''
represents the root-mean-square residuals of the actual brightness
values contained within circles of radius one FITTING RADIUS about the
centroids of the PSF stars.  That is to say:  the routine has done the
best job it could of fitting the analytic function to the pixels within
one FITTING RADIUS of the PSF stars --- one function fitting all the
stars.  The ``Chi'' value is the root-mean-square of the residuals that
are left, expressed as a fraction of the peak height of the analytic
function.  In this case, the analytic first approximation matched the
observed stellar profiles to within about 2.8\%, root-mean-square, on
average.  Part of this number presumably represents the noise in the
stellar profiles, but most of it is due to the fact that the analytic
function does not accurately reflect the true stellar profile of the
frame.  It is this systematic difference between the true profile and
the analytic first approximation that is to go into the look-up table
of profile corrections.

The actual derived parameters of the best-fitting analytic function are
typed out next.  In this case, the analytic function chosen had three
free parameters.  For {\it all\/} the different analytic first
approximations, the first two parameters are {\it always\/} the
half-width at half-maximum in $x$ and $y$.  Any other parameters the
model may have differ from function to function, but the first two are,
as I say, always the half-width at half-maximum in $x$ and $y$.

Next, the routine types out the {\it individual\/} values for the
root-mean-square residual of the actual stellar profile from the
best-fitting analytic model, star by star.  Again, these values are
computed only from those pixels lying within a FITTING RADIUS of the
stars' centroids --- cosmic rays or companion stars in the PSF stars'
outer wings do not contribute to these numbers.  Any star with an
individual profile scatter greater than three times the average is
flagged with a ``$\,$*$\,$'';  a star showing scatter greater than
twice the average is flagged with ``$\,$?$\,$''.  The user may want to
consider deleting these stars from the .LST file and making another PSF
without them.  Apart from the two objects flagged, the scatter values
given in the sample above demonstrate that systematic differences
between the ``true'' stellar profile and the analytic first
approximation dominate over raw noise in the profiles:  the typical
root-mean-square residual does not increase very much from the
beginning of the list to the end --- I happen to know that these stars
were sorted by increasing apparent magnitude in the .LST file.

Finally, {\bf PSF} reminds you that it has created a file named
?.NEI which contains the PSF stars and their recognized neighbors
in the frame.  This file may be run through {\bf GROUP} and
{\bf NSTAR} or through ALLSTAR for a quickie profile fit, and
then the neighbors may be selectively subtracted from the original
image to help isolate the PSF stars for an improved second-generation
PSF.

Unlike DAOPHOT~Classic, DAOPHOT~II: The Next Generation {\it is\/}
able to use PSF stars that are within a PSF RADIUS of the edge of
the frame, provided that they are at least a FITTING RADIUS from
the edge.

Finally, the analytic first approximations.  At this particular point
in time (you like that verbose stuff?) there are six allowed options:

\item{(1)} A Gaussian function, having two free parameters:  half-width
at half-maximum in $x$ and $y$.  The Gaussian function may be elliptical,
but the axes are aligned with the $x$ and $y$ directions in the image.
This restriction allows for fast computation, since the two-dimensional
integral of the bivariate Gaussian over the area of any given pixel
may be evaluated as the product of two one-dimensional integrals.

\item{(2)} A Lorentz function, having three free parameters: half-width
at half-maximum in $x$ and $y$, and (effectively) a position angle for
the major axis of the ellipse.  Since it's necessary to compute the
two-dimensional integral anyway, we may as well let the ellipse be
inclined with respect to the cardinal directions.

\item{(3)} A Moffat function, having three free parameters: ditto.
In case you don't know it, a Moffat function is

$$ \propto{1\over{(1 + z^2)^\beta}}$$

\item{} where $z^2$ is something like ${x^2/\alpha_x^2} +
{y^2/\alpha_y^2} + {\alpha_{xy} x y}$ (Note: {\it not\/} $\ldots +
{xy/\alpha_{xy}}$ so $\alpha_{xy}$ can be zero).  In this case, $\beta
= 1.5$.

\item{(4)} A Moffat function, having the same three parameters free,
but with $\beta = 2.5$.

\item{(5)}  A ``Penny'' function: the sum of a Gaussian and a Lorentz
function, having four free parameters.  (As always) half-width at
half-maximum in $x$ and $y$; the fractional amplitude of the
Gaussian function at the peak of the stellar profile; and the position
angle of the tilted elliptical Gaussian.  The Lorentz function may
be elongated, too, but its long axis is parallel to the $x$ or
$y$ direction.

\item{(6)} A ``Penny'' function with five free parameters.  This time
the Lorentz function may also be tilted, in a different direction
from the Gaussian.

It is possible that these assignments will be changed or augmented in
the future.  If you are worried about the details, I suggest you
consult the source code:  the routine PROFIL in the file MATHSUBS.FOR.

\vfill
\eject
\noindent VIII.  {\bf PEAK}

{\bf PEAK} is a single-star profile-fitting algorithm.  Because it is
not to be trusted for the reduction of overlapping images, {\bf PEAK}
should never be used for the final photometric reduction in fields
where stars of interest may be blended. On the other hand, for
sparsely-populated frames aperture photometry is often fine, and {\bf
NSTAR} or ALLSTAR photometry is virtually as fast.  Thus, {\bf
PEAK} is largely a historical artifact, reminding us that once life was
simpler. For you archaeology buffs, here is how {\bf PEAK} used to be
used.  I now quote from the original DAOPHOT manual.

``Obviously, before PEAK can be run, you must have created a
point-spread function.  Assuming this to be the case, then

\bigskip
{\noindent\obeylines\obeyspaces\frenchspacing\tt\baselineskip=0.3truecm
=======================================================================
| COMPUTER TYPES:                                  YOU ENTER:         |
|                                                                     |
| Command:                                         PE                 |
|                                                                     |
|     File with aperture results (default ?.AP):   <CR> or filename   |
|                                                                     |
|             File with the PSF (default ?.PSF):   <CR> or filename   |
|                                                                     |
|          File for PEAK results (default ?.PK):   <CR> or filename   |
=======================================================================
}
\bigskip

\noindent If you have issued the {\bf MONITOR} command (see below) or
if $\hbox{\rm WATCH PROGRESS} = 1.0$ (see the {\bf OPTIONS} command
above), you will then see the results of the peak-fitting photometry
appear on your screen.  If $\hbox{\rm WATCH PROGRESS} = 2.0$, then you
will also see an eleven-level gray-scale image of the region around
each star as it is being reduced; since it is very time-consuming to
produce these pictures on your terminal screen, WATCH PROGRESS should
be set to 2.0 only when it is strictly necessary.

``The {\bf PEAK} algorithm also makes use of the optional parameter
FITTING RADIUS:  only pixels within a FITTING RADIUS of the centroid of
a star will be used in fitting the point-spread function.  Furthermore,
for reasons related to accelerating the convergence of the iterative
non-linear least-squares algorithm, the pixels within this FITTING
RADIUS are assigned weights which fall off from a value of unity at the
position of the centroid of the star to identically zero at the fitting
radius.  Since the least-squares fit determines three unknowns ($x$ and
$y$ position of the star's centroid, and the star's brightness) it is
absolutely essential that the fitting radius not be so small as to
include only a few pixels with non-zero weight.  FITTING RADII less
than 1.6 pixels (that would include as few as four pixels in the fit)
are explicitly forbidden by the program.  I suggest that a fitting
radius comparable to the FWHM be used as a general rule, but suit
yourself.

``In addition to an improved estimate of the $x,y$-position of the
centroid of each star, and of its magnitude and the standard error of
the magnitude, {\bf PEAK} also produces two image-peculiarity indices
which can be used to identify a disturbed image.  The first of these,
CHI, is essentially the ratio of the observed pixel-to-pixel scatter in
the fitting residuals to the expected scatter, based on the values of
readout noise and the photons per ADU which you specified in your
aperture photometry table.  If your values for the readout noise and
the photons per ADU are correct, then in a plot of CHI against derived
magnitude ({\it e.g.} Stetson and Harris 1988, {\it A.J.} {\bf 96},
909, Fig.~28), most stars should scatter around unity, with little or
no trend in CHI with magnitude (except at the very bright end, where
saturation effects may begin to set in).

``The second image-peculiarity statistic, SHARP, is vaguely related to
the {\it intrinsic\/} ({\it i.e.\/} outside the atmosphere) angular
size of the astronomical object:  effectively, SHARP is a zero-th order
estimate of the square of the quantity (actual one-sigma
half-characteristic-width of the astronomical object as it would be
measured outside the atmosphere, in pixels).

$$ \hbox{\rm SHARP}^2 \sim
\sigma^2(\hbox{\rm observed}) - \sigma^2(\hbox{\rm point-spread
function})$$

\noindent (This equation is reasonably valid provided SHARP is not
significantly larger than the square of the one-sigma Gaussian
half-width of the core of the PSF; see the .PSF file in Appendix IV.)
For an isolated star, SHARP should have a value close to zero, whereas
for semi-resolved galaxies and unrecognized blended doubles SHARP will
be significantly greater than zero, and for cosmic rays and some image
defects which have survived this far into the analysis SHARP will be
significantly less than zero.  SHARP is most easily interpreted when
plotted as a function of apparent magnitude for all objects reduced
({\it e.g.}, Stetson and Harris 1988, {\it A.J.} {\bf 96}, 909,
Fig.~27). Upper and lower envelopes bounding the region of single stars
may be drawn by eye or by some automatic scheme of your own devising.

``Finally, {\bf PEAK} tells you the number of times that the profile
fit had to be iterated.  The program gives up if the solution has been
iterated 50 times without achieving convergence, so stars for which the
number of iterations is 50 are inherently more suspect than the rest.
Frequently, however, the solution was oscillating by just a little bit
more than the convergence criterion (which is fairly strict: from one
iteration to the next the computed magnitude must change by less than
0.0001 mag. or 0.05 sigma, whichever is larger, and the $x$- and
$y$-coordinates of the centroid must change by less than 0.001 pixel
for the program to feel that convergence has been achieved).
Therefore, in many cases stars which have gone 50 iterations are still
moderately well measured.

``One other thing of which you should be aware:  {\bf PEAK} uses a
fairly conservative formula for automatically reducing the weight of a
``bad'' pixel (not a ``bad'' pixel as defined in {\bf FIND} --- {\bf
PEAK} ignores those --- but rather any pixel which refuses to approach
the model profile as the iterative fit proceeds).  This formula depends
in part on the random errors that the program expects, based on the
values for the readout noise and the photons per ADU which you, the
user, specified in the aperture photometry table.  It is therefore
distinctly to your advantage to see to it that the values supplied are
reasonably correct. ''

\vfill
\eject
\noindent IX.  {\bf GROUP}

Before performing multiple, simultaneous profile fits using the routine
{\bf NSTAR}, it is necessary to divide the stars in your frame up into
natural groups, each group to be reduced as a unit.  The principle is
this:  if two stars are close enough together that the light of one
will influence the profile-fit of another, they belong in the same
group.  That way, as the solution iterates to convergence, the
influence on each star of all of its relevant neighbors can be
explicitly accounted for.

\bigskip
{\noindent\obeylines\obeyspaces\frenchspacing\tt\baselineskip=0.3truecm
=======================================================================
| COMPUTER TYPES:                                  YOU ENTER:         |
|                                                                     |
| Command:                                         GR                 |
|                                                                     |
|     File with the photometry (default ?.AP):     <CR> or filename   |
|                                                                     |
|           File with the PSF (default ?.GRP):     <CR> or filename   |
|                                                                     |
|                            Critical overlap:     n.n                |
|                                                                     |
|     File for stellar groups (default ?.GRP):     <CR> or filename   |
=======================================================================
}
\bigskip

\noindent The ``critical overlap'' has the following significance.
When {\bf GROUP} is examining two stars to see whether they could
influence each others' fits, it first identifies the fainter of the two
stars.  Then, it calculates the brightness of the brighter star (using
the scaled PSF) at a distance of one fitting radius plus one pixel from
the centroid of the fainter. If this brightness is greater than
``critical overlap'' times the random error per pixel (calculated from
the known readout noise, sky brightness in ADU and number of photons
per ADU), then the brighter star is known to be capable of affecting
the photometry of the fainter, and the two are grouped together.  You
must determine what value of ``critical overlap'' is suitable for your
data frame by trial and error:  if a critical overlap = 0.1 divides all
of the stars up into groups smaller than 60, then you may be sure that
unavoidable random errors will dominate over crowding.  If critical
overlap = 1.0 works, then crowding will be no worse than the random
errors.  If critical overlap $\gg  1.0$ is needed, then in many cases
crowding will be a dominant source of error.

After {\bf GROUP} has divided the stars up and created an output disk
file containing these natural stellar groups, a little table will be
produced on your terminal showing the number of groups as a function of
their size.  If any group is larger than the maximum acceptable to {\bf
NSTAR} (currently 60 stars), then the critical overlap must be
increased, or the {\bf SELECT} command (see below) should be used to
cut the overly large groups out of the file.  When crowding conditions
vary across the frame, judicious use of {\bf GROUP} and {\bf SELECT}
will pick out regions of the data frame where different critical
overlaps will allow you to get the best possible photometry for stars
in all of the crowding regimes.

\vfill
\eject
\noindent X.  {\bf NSTAR}

{\bf NSTAR} is DAOPHOT's multiple-simultaneous-profile-fitting
photometry routine. It is used in very much the same way as PEAK:

\bigskip
{\noindent\obeylines\obeyspaces\frenchspacing\tt\baselineskip=0.3truecm
=======================================================================
| COMPUTER TYPES:                                  YOU ENTER:         |
|                                                                     |
| Command:                                         NS                 |
|                                                                     |
|             File with the PSF (default ?.PSF):   <CR> or filename   |
|                                                                     |
|      File with stellar groups (default ?.GRP):   <CR> or filename   |
|                                                                     |
|        File for NSTAR results (default ?.NST):   <CR> or filename   |
|                                                                     |
=======================================================================
}
\bigskip

\noindent The principal difference is that {\bf NSTAR} is much more
accurate than {\bf PEAK} in crowded regions, although at the cost of
requiring somewhat more time per star, depending on the degree of
crowding. {\bf NSTAR} automatically reduces the weight of bad pixels
just as {\bf PEAK} does, so it is highly advisable that your values for
the readout noise and the number of photons per ADU be just as correct
as you can make them.  {\bf NSTAR} also produces the same
image-peculiarity statistics CHI and SHARP, defined as they were in
{\bf PEAK}.  Plots of these against apparent magnitude are powerful
tools for seeing whether you have specified the correct values for the
readout noise and the photons per ADU (Do most stars have CHI near
unity?  Is there a strong trend of CHI with magnitude?), for
identifying stars for which the profile fits just haven't worked (they
will have values of CHI much larger than normal for stars of the same
derived magnitude), for identifying probable and possible galaxies
(they will have larger values of SHARP than most), for identifying bad
pixels and cosmic rays that made it through {\bf FIND} (large negative
values of SHARP), and for seeing whether your brightest stars are
saturated (the typical values of SHARP and CHI will tend to increase
for the very brightest stars).

The maximum number of iterations for {\bf NSTAR} is 50, but {\it every}
star in a group must individually satisfy the convergence criteria (see
{\bf PEAK}) before the program considers the group adequately reduced.

{\bf NSTAR} also has a slightly sophisticated star-rejection algorithm,
which is essential to its proper operation.  A star can be rejected for
several reasons:

\item{(1)}  If two stars in the same group have their centroids
separated by less than a critical distance (currently set more or less
arbitrarily to $0.37 \times$  the FWHM of the stellar core), they are
presumed to be the same star, their photocentric position and combined
magnitude is provisionally assigned to the brighter of the two and the
fainter is eliminated from the starlist before going into the next
iteration.

\item{(2)}  Any star which converges to more than 12.5 magnitudes
fainter than the point-spread function (one part in ten to the fifth;
e.g., central brightness $<$ 0.2~ADU/pixel if the first PSF star had a
central brightness of 20,000~ADU/pixel) is considered to be
non-existent and is eliminated from the starlist.

\item{(3a)} After iterations 5, 6, 7, 8, and 9, if the faintest star in
the group has a brightness less than one sigma above zero, it is
eliminated;

\item{(3b)} after iterations 10 -- 14, if the faintest star in the
group has a brightness less than 1.5 sigma above zero, it is
eliminated;

\item{(3c)} after iterations 15 -- 50, or when the solution thinks it
has converged, whichever comes first, if the faintest star in the group
has a brightness less than 2.0 sigma above zero, it is eliminated.

\item{(4a,b,c)}  Before iterations 5 -- 9, before iterations 10 -- 14,
and before iterations 15 -- 50, if two stars are separated by more than
$0.37 \times$  the FWHM and less than $1.0 \times$  the FWHM, and if
the fainter of the two is more uncertain than 1.0, 1.5, or 2.0 sigma
(respectively), the fainter one is eliminated.

\noindent Whenever a star is eliminated, the iteration counter is
backed up by one, and the reduction proceeds from that point with the
smaller set of stars. Backing up the iteration counter gives the second
least certain star in the group as much as two full iterations to
settle into the new model before it comes up for a tenure decision.
Since the star-rejection formula depends in part upon the
user-specified values for the readout noise and the number of photons
per ADU, it is once again important that the values you give for these
parameters be reasonable.

\vfill
\eject
\noindent XI.  {\bf SUBSTAR}

The {\bf SUBSTAR} command takes the point-spread function for a frame
and a data file containing a set of $x,y$ coordinates and apparent
magnitudes for a set of stars, shifts and scales the point-spread
function according to each position and magnitude, and then subtracts
it from your original frame.  In the process a new data frame is
produced (your original picture file is left inviolate). In principal,
this can be done using photometry from {\bf PHOTOMETRY} as well as from
{\bf PEAK}, {\bf NSTAR}, or ALLSTAR, but I can't imagine why on
earth you'd want to.  In general, this star subtraction is done after
the photometric reductions have been performed, as a check on the
quality of the profile fits, and to highlight non-stellar objects and
tight binaries.  Additional uses for {\bf SUBSTAR}:

\item{(1)} decrowding bright stars for the generation of an improved
point-spread function (see Appendix on constructing a PSF); and

\item{(2)} decrowding bright stars for establishing the magnitude
zero-point of the frame by means of aperture photometry through a
series of apertures.

Here's how it goes.

\bigskip
{\noindent\obeylines\obeyspaces\frenchspacing\tt\baselineskip=0.3truecm
=======================================================================
| COMPUTER TYPES:                                  YOU ENTER:         |
|                                                                     |
| Command:                                         SU                 |
|                                                                     |
|             File with the PSF (default ?.PSF):   <CR> or filename   |
|                                                                     |
|          File with photometry (default ?.NST):   <CR> or filename   |
|                                                                     |
|                Do you have stars to leave in?    N                  |
|                                                                     |
|    Name for subtracted image (default ?s.DST):   <CR> or filename   |
|                                                                     |
=======================================================================
}
\bigskip

\noindent This will subtract from the image that is currently {\bf
ATTACH}ed all the stars in the ``File with photometry.''  The picture
produced will have a format identical to your original picture --- it
will be acceptable as input to DAOPHOT, if you so desire.

\vfill
\eject
{\noindent\obeylines\obeyspaces\frenchspacing\tt\baselineskip=0.3truecm
=======================================================================
| COMPUTER TYPES:                                  YOU ENTER:         |
|                                                                     |
| Command:                                         SU                 |
|                                                                     |
|             File with the PSF (default ?.PSF):   <CR> or filename   |
|                                                                     |
|          File with photometry (default ?.NST):   <CR> or filename   |
|                                                                     |
|                Do you have stars to leave in?    Y                  |
|                                                                     |
|           File with star list (default ?.LST)    <CR> or filename   |
|                                                                     |
|    Name for subtracted image (default ?s.DST):   <CR> or filename   |
|                                                                     |
=======================================================================
}
\bigskip

This time {\bf SUBSTAR} will subtract from the image all the stars
in the ``File with photometry'' {\it except\/} those that appear
in the ``File with star list.''  This makes it easy to clean out stars
around your PSF stars or around the stars for which you wish to perform
concentric-aperture photometry.  Note, however, that stars are
cross-identified solely on the basis of their ID numbers in the various
files.  If you use the renumber option of the {\bf SORT} command or
if you {\bf APPEND} together files where ID numbers are duplicated,
you may find yourself leaving the wrong stars in the image.

\vfill
\eject
\centerline{D.  Additional Commands}
\noindent XIII.  {\bf MONITOR/NOMONITOR}

If you want simply to turn off the sending of the results to your
terminal screen, this can be accomplished without going through all the
foofarah of the {\bf OPTIONS} command, by using the {\bf NOMONITOR}
command:

\bigskip
{\noindent\obeylines\obeyspaces\frenchspacing\tt\baselineskip=0.3truecm
=======================================================================
| COMPUTER TYPES:                                  YOU ENTER:         |
|                                                                     |
| Command:                                         NO                 |
=======================================================================
}
\bigskip

\noindent Similarly, after the {\bf NOMONITOR} command or the WATCH
PROGRESS = 0 option, output to your terminal screen can be restored
with the {\bf MONITOR} command:

\bigskip
{\noindent\obeylines\obeyspaces\frenchspacing\tt\baselineskip=0.3truecm
=======================================================================
| COMPUTER TYPES:                                  YOU ENTER:         |
|                                                                     |
| Command:                                         MO                 |
=======================================================================
}
\bigskip

{\bf MONITOR} and {\bf NOMONITOR} set the WATCH PROGRESS parameter to 1
and 0, respectively.  If you want the special effects produced by
setting WATCH PROGRESS to --2, --1, or 2, you must set it explicitly in
your DAOPHOT.OPT file or with the {\bf OPTIONS} command.

\vfill
\eject
\noindent XIV.  {\bf SORT}

This routine will take in any stellar data file produced by DAOPHOT
({\it viz.} files produced by {\bf FIND}, {\bf PHOTOMETRY}, {\bf PEAK},
{\bf GROUP}, {\bf NSTAR}, ALLSTAR, or any of the other auxiliary
routines discussed below) and re-order the stars according to position
within the frame, apparent magnitude, identification number, or other
available datum (e.g., magnitude error, number of iterations, CHI, or
SHARP).  Of course, if a file produced by {\bf GROUP} or {\bf NSTAR} is
sorted, then the association of stars into groups will be destroyed.

\bigskip
{\noindent\obeylines\obeyspaces\frenchspacing\tt\baselineskip=0.3truecm
=======================================================================
| COMPUTER TYPES:                                      YOU ENTER:     |
|                                                                     |
| Command:                                             SO             |
|                                                                     |
|                                                                     |
|      The following sorts are currently possible:                    |
|                                                                     |
|   +/- 1  By increasing/decreasing star ID number                    |
|                                                                     |
|   +/- 2  By increasing/decreasing  X  coordinate                    |
|                                                                     |
|   +/- 3  By increasing/decreasing  Y  coordinate                    |
|                                                                     |
|   +/- 4  By increasing/decreasing magnitude                         |
|                                                                     |
|   +/- n  By increasing/decreasing OTHER (n <= 30)                   |
|                                                                     |
|                                                                     |
|                          Which do you want?          n              |
|                                                                     |
|                            Input file name:          filename       |
|                                                                     |
|               Output file name (default ?):          <CR>           |
|                                                   or filename       |
|                                                                     |
|           Do you want the stars renumbered?          Y or N         |
|                                                                     |
=======================================================================
}
\bigskip

\noindent If you answer the question, ``Which do you want?'' with
``4'', the stars will be reordered by increasing apparent magnitude; if
by ``--4'', they will be reordered by decreasing apparent magnitude,
and so forth.  If you say that you want the stars renumbered, the first
star in the new output file will be given the identification number
``1'', the second star ``2'', and so on.  The output file will contain
exactly the same data in exactly the same format as the input file ---
the stars will just be rearranged within the file. The ``+/-- n''
option permits you to sort according to any of the auxiliary
information in any output file --- stars can thus be reordered by their
sharpness or roundness indices, by their magnitudes in any aperture, by
their sky brightness, by the number of iterations they required to
converge in {\bf PEAK} or {\bf NSTAR}, or whatever.  The value of $n$
which should be entered to specify one of these items is that which is
given at the bottom of each sample output file in Appendix IV below.
(Note:  data on the second line for a star in an .AP file are always
designated by 16, 17, 18, $\ldots$, regardless of how many apertures
were used.  Thus, if two apertures were used, then the magnitude in
aperture 1 is number 4, the magnitude in aperture 2 is number 5, the
sky brightness is number 16, the standard deviation of the sky
brightness is number 17, the skewness of the sky brightness is number
18, the error in the first magnitude is 19, and the error in the second
magnitude is 20.  Numbers 6--15 and 21--30 are useless, in this
example.)

\vfill
\eject
\noindent XV.  {\bf SELECT}

When you run {\bf GROUP} on a photometry file to create a group file
suitable for input for {\bf NSTAR}, you may find that some groups are
larger than 60 stars, which is the current maximum group size allowed.
The {\bf SELECT} command allows you to cut out of the group file only
those groups within a certain range of sizes, and put them in their own
file.

\bigskip
{\noindent\obeylines\obeyspaces\frenchspacing\tt\baselineskip=0.3truecm
=======================================================================
| COMPUTER TYPES:                                  YOU ENTER:         |
|                                                                     |
| Command:                                         SE                 |
|                                                                     |
|                     Input group file:            filename           |
|                                                                     |
|          Minimum, maximum group size:            nn nn              |
|                                                                     |
|    Output group file (default ?.GRP):            <CR> or filename   |
|                                                                     |
|                                                                     |
|          nnn stars in nnn groups.                                   |
|                                                                     |
=======================================================================
}
\bigskip

\noindent If you want to try to reduce every star in the frame,
regardless of the errors, then you will need to run {\bf SELECT} at
least twice:  once with minimum, maximum group size = 1,60, and again
with the same input group file, a different output group file, and
minimum, maximum group size = 61,9999 (say).  This latter file would
then be run through GROUP again, with a larger critical overlap.  (Note:
You'd be better off using ALLSTAR.)

\vfill
\eject
\noindent XVI.  {\bf OFFSET}

If you have a set of coordinates for objects found in one frame,
and want to use these as centroids for aperture photometry in another
frame, and if there is some arbitrary translational shift between the
two frames, then {\bf OFFSET} can be used to add constants to all the
$x$- and $y$-coordinates in a stellar data file:

\bigskip
{\noindent\obeylines\obeyspaces\frenchspacing\tt\baselineskip=0.3truecm
=======================================================================
| COMPUTER TYPES:                               YOU ENTER:            |
|                                                                     |
| Command:                                      OF                    |
|                                                                     |
|                    Input file name:           filename              |
|                                                                     |
|  Additive offsets ID, DX, DY, DMAG:           n nn.nn nn.nn n.nnn   |
|                                            or n nn.nn nn.nn/        |
|                                                                     |
|   Output file name (default ?.OFF):           <CR> or filename      |
|                                                                     |
=======================================================================
}
\bigskip

\noindent As you can tell from the example, you can also use this
routine to add some (integer) constant to the ID numbers, or some
(real) constant to the magnitudes.  Why would you ever want to do these
things?  Well, if you've just run {\bf FIND} on a star-subtracted
image, you'll probably want to add 50000 or something to the ID numbers
before appending this new list to the original star list for the frame,
so there will be no doubt which star number 1 is referred to in the
.LST file.  If you have a list of artificial stars that you've just
added to one frame of a field, you may want to offset the positions and
instrumental magnitudes so that you can add exactly the same stars into
another frame of the same field.  Just for instance.

An unformatted READ is used to input these data, so the numbers may be
separated by spaces or commas, and the list can be terminated with a
slash (``/'') if you want all remaining numbers to be zero.

Use of this routine for transferring the starlist from one frame to
another is not recommended in crowded fields, particularly when the two
frames were taken in different photometric bandpasses.  You really need
to know about all of a star's neighbors in order to reduce it properly,
and different neighbors may be prominent in frames taken in greatly
different colors.  It would be better simply to run {\bf FIND} on each
frame, and to match up the stars you are interested in after the
photometry is done.

\vfill
\eject
\noindent XVII.  {\bf APPEND}

{\bf APPEND} provides a simple way for the user to concatenate any two
of the stellar data files which DAOPHOT has written to the disk.  A
special DAOPHOT command has been written to perform this function,
because if the user were to leave DAOPHOT and use the operating
system's COPY, MERGE, or APPEND command (or equivalent), it would then
be necessary to use an editor to remove the extra file header from the
middle of the newly created file. {\bf APPEND} does this for you
automatically.

\bigskip
{\noindent\obeylines\obeyspaces\frenchspacing\tt\baselineskip=0.3truecm
=======================================================================
| COMPUTER TYPES:                                  YOU ENTER:         |
|                                                                     |
| Command:                                         AP                 |
|                                                                     |
|                   First input file:              filename           |
|                                                                     |
|                  Second input file:              filename           |
|                                                                     |
|                                                                     |
|        Output file (default ?.CMB):              <CR> or filename   |
|                                                                     |
=======================================================================
}
\bigskip

\noindent Note that {\bf APPEND} does no checking to ensure that the
input files are of the same type ---  the user is perfectly able to
{\bf APPEND} output from {\bf FIND} onto output from {\bf PHOTOMETRY}.
The resulting hybrid file would be illegible to most other DAOPHOT
routines, and might cause a crash if DAOPHOT tried to read it in.  Note
further that the {\bf GROUP} command (above) makes a point of leaving a
blank line at the end of every group file, so that the groups will
remain distinct when {\bf APPEND}ed.  If in editing group files you
delete that last blank line, then when {\bf APPEND}ing those files the
last group of the first input file and the first group of the second
input file will be joined together in the output file.

\vfill
\eject
\noindent XVIII.  {\bf DUMP}

{\bf DUMP} permits you to display on your terminal screen the raw
values in a specified square subarray of your picture.  This is useful
for figuring out why DAOPHOT isn't working with your data (maybe all
the pixels have intensities of --32768?), or in deciding whether a
particular star has a central intensity above the maximum value where
your detector behaves linearly.

\bigskip
\noindent Example:
\bigskip
{\noindent\obeylines\obeyspaces\frenchspacing\tt\baselineskip=0.3truecm
=======================================================================
| COMPUTER TYPES:                                  YOU ENTER:         |
|                                                                     |
| Command:                                         DU                 |
|                                                                     |
|                                   Box size:      9                  |
|                                                                     |
|               Coordinates of central pixel:      201,378            |
|                                                                     |
+---------------------------------------------------------------------+
| COMPUTER TYPES:                                                     |
|                                                                     |
|               197   198   199   200   201   202   203   204   205   |
|           +------------------------------------------------------   |
|       383 |   543   556   600   633   643   630   589   538   514   |
|       382 |   581   644   760   884   930   865   732   623   570   |
|       381 |   651   864  1248  1823  2062  1657  1116   800   626   |
|       380 |   775  1303  2791  5995  7442  4802  2166  1096   732   |
|       379 |   916  1955  5933 16430 22029 11974  4104  1526   846   |
|       378 |   977  2259  6364 16623 23622 13658  4751  1762   933   |
|       377 |   936  1836  3878  7751 10436  7269  3380  1611   949   |
|       376 |   798  1217  1963  3179  3815  3109  2006  1273   861   |
|       375 |   656   847  1138  1563  1778  1621  1300  1003   790   |
|       374 |   602   682   798   962  1090  1073   959   824   698   |
|       373 |   550   587   653   729   801   807   782   705   638   |
|                                                                     |
|                   Minimum, median, maximum:    570   1248  23622    |
|                                                                     |
+---------------------------------------------------------------------+
|                                                                     |
| To get out ...                                                      |
|                                                                     |
| COMPUTER TYPES:                                      YOU ENTER:     |
|                                                                     |
|               Coordinates of central pixel:          0,0            |
|                                                   or CTRL-Z         |
|                                                                     |
=======================================================================
}
\bigskip

\noindent (I just happened to know that there is a bright star centered
near 201,378.) The user specifies a box size that will conveniently
fill his screen, without any wraparound; with the column and row ID's
across the top and down the left side, a box 12 pixels on a side is the
largest that can be accomodated on an 80-column terminal screen, while
a box 21 pixels on a side is the largest that can be fit into a $24
\times 132$ screen.

Responding to the ``Coordinates of central pixel:'' prompt with a
position that is outside the picture or with a CTRL-Z will return you
to DAOPHOT command mode.

\vfill
\eject
\noindent XIX.  {\bf FUDGE}

Although it is morally wrong, someday there may come a time when you
just {\it have\/} to fudge some of your image data.  Suppose, for
instance, that you are trying to derive a point-spread function from
the sole acceptable star in the frame, and way, way out in the corner
of the box wherein the point-spread function is to be defined there is
a cosmic-ray hit.  If you do nothing, then the cosmic ray will produce
a spike in the point-spread function which will generate a hole
whenever this PSF is used to subtract stars.  {\it In such a desperate
case it may be {\bf slightly} the lesser of two evils to fudge the bad
datum}, which DAOPHOT's {\bf FUDGE} routine will do.  Let us assume
that from your image display you have ascertained that the cosmic ray
occupies the two pixels (267,381) and (268,381); let us further suppose
that you see from the aperture photometry for this star that the sky
background brightness in its neighborhood is 893 ADU.  Then:

\bigskip
{\noindent\obeylines\obeyspaces\frenchspacing\tt\baselineskip=0.3truecm
=======================================================================
| COMPUTER TYPES:                                  YOU ENTER:         |
|                                                                     |
| Command:                                         FU                 |
|                                                                     |
|  Name for output picture (default ?f.DST):       <CR> or filename   |
|                                                                     |
|                           Border (pixels):       0                  |
|                                                                     |
|                 First, last column number:       267,268            |
|                                                                     |
|                    First, last row number:       381,381            |
|                                              or  381/               |
|                                                                     |
|                          Brightness value:       893                |
|                                                                     |
|                                                                     |
|                 First, last column number:       CTRL-Z             |
|                                                                     |
| Command:                                                            |
|                                                                     |
=======================================================================
}
\bigskip

As may be inferred from the example, {\bf FUDGE} allows you to insert
any constant brightness value into any rectangular subsection of an
otherwise exact copy of the image.  Alternatively, $\ldots$

\vfill
\eject
{\noindent\obeylines\obeyspaces\frenchspacing\tt\baselineskip=0.3truecm
=======================================================================
| COMPUTER TYPES:                                  YOU ENTER:         |
|                                                                     |
| Command:                                         FU                 |
|                                                                     |
|     Name for output picture (default ?f.DST):    <CR> or filename   |
|                                                                     |
|                              Border (pixels):    n                  |
|                                                                     |
| Polynomial order (0=constant, 1=plane, etc.):    n                  |
|                                                                     |
|                    First, last column number:    267,268            |
|                                                                     |
|                       First, last row number:    381,381            |
|                                               or 381/               |
|                                                                     |
|                                                                     |
|                    First, last column number:    CTRL-Z             |
|                                                                     |
| Command:                                                            |
|                                                                     |
=======================================================================
}
\bigskip

In this case you don't want to insert a single constant brightness
value and/or you don't know what value you would like to insert.
Instead you are asking the routine to consider a border $n$ pixels wide
around the rectangular area you have specified, and use a least-squares
polynomial surface to interpolate values into the fudged region.  Now
for the tricky bit.  The routine does not fit a single polynomial
surface to the border and then use that polynomial to predict values
for all the pixels in the fudged region.  Oh, no.  Instead, for each
pixel in the rectangle to be fudged it fits a {\it different\/}
polynomial surface to the border pixels, employing $1/r^4$ weights.
Thus, pixels in the corners and near the edges of the rectangular fudge
region will closely reflect the data values and gradients in the pixels
next to them and will be minimally affected by the gross gradient
across the gap; pixels in a long rectangular region will be affected by
the pixels next to them and less by the pixels at the far ends.  Thus,
even a ``constant'' or a ``plane'' polynomial order will produce some
complex surface that flows smoothly between the borders, and doesn't
have discontinuous values or gradients at the edges.  This is quite a
bit slower than a simple surface fit, so keep the border narrow ---
only a couple-three pixels --- but I think you'll be pleased with the
results.

Use your fudged image to generate a point-spread function free of the
cosmic-ray spike, {\it then delete it before anyone finds out what you
have done!}

\vfill
\eject
\noindent XX.  {\bf ADDSTAR}

This routine is used to add synthetic stars, either placed at
random by the computer, or in accordance with positions and magnitudes
specified by you, to your picture. They can then be found by {\bf
FIND}, reduced by {\bf PHOTOMETRY} and the rest, and the star-finding
efficiency and the photometric accuracy can be estimated by comparing
the output data for these stars to what was put in.

\bigskip

\noindent Example 1:  Random star placement

\bigskip
{\noindent\obeylines\obeyspaces\frenchspacing\tt\baselineskip=0.3truecm
=======================================================================
| COMPUTER TYPES:                                  YOU ENTER:         |
|                                                                     |
| Command:                                         AD                 |
|                                                                     |
|         File with the PSF (default ?.PSF):       <CR> or filename   |
|                                                                     |
|                        Seed (any integer):       n                  |
|                                                                     |
|                           Photons per ADU:       nn.n               |
|                                                                     |
|    Input data file (default RANDOM STARS):       <CR>               |
|                                                                     |
|                                                                     |
|               Magnitude of PSF star is nn.nnn                       |
|                                                                     |
|                                                                     |
|       Minimum, maximum magnitudes desired:       15.0 18.0          |
|                                                                     |
|      Number of stars to add to each frame:       5                  |
|                                                                     |
|                      Number of new frames:       100                |
|                                                                     |
|                            File-name stem:       FAKE               |
|                                                                     |
=======================================================================
}
\bigskip

\noindent This will produce five different, new data frames, each
containing 100 artificial stars with instrumental magnitudes between
15.00 and 18.00 mag. added to what was already there.  The five frames
will be named FAKE01.DST, $\ldots$, FAKE05.DST.  There will also be created
five files named FAKE01.ADD, $\ldots$, FAKE05.ADD containing the
x,y-positions and the magnitudes of the new stars.  I have you specify
a seed for the random number generator so that (a)~if you lose the
images, by specifying the same seed you can make them again, and
(b)~exactly the same artificial image will be created on any computer,
provided the same seed is used.

\vfill
\eject

\noindent Example 2:  Deliberate star placement

\bigskip
{\noindent\obeylines\obeyspaces\frenchspacing\tt\baselineskip=0.3truecm
=======================================================================
| COMPUTER TYPES:                                  YOU ENTER:         |
|                                                                     |
| Command:                                         AD                 |
|                                                                     |
|         File with the PSF (default ?.PSF):       <CR> or filename   |
|                                                                     |
|                        Seed (any integer):       n                  |
|                                                                     |
|                           Photons per ADU:       14.1               |
|                                                                     |
|    Input data file (default RANDOM STARS):       ARTSTAR.MAG        |
|                                                                     |
|    Output picture name (default ARTSTARa):       <CR>               |
|                                                                     |
=======================================================================
}
\bigskip

\noindent This presumes that by some means you have already created on
the disk a file named ``ARTSTAR.MAG'' (or whatever) which contains
centroid positions and instrumental magnitudes for the stars you want
added to the picture.  This permits you to become just as sophisticated
with your artificial-star tests as you want --- you can simulate any
color-magnitude diagram, luminosity function, and spatial distribution
in the frame that you want, just by writing yourself a program which
creates the necessary input files.

Note that in both examples the code asks for a number of photons per
ADU.  It uses this to add the appropriate Poisson noise to the star
images (the correct amount of readout noise already exists in the
frame).  For realistic tests you should specify the correct value for
this number.  If for some reason you would like to have the scaled PSF
added into the image {\it without\/} extra Poisson noise, just specify
some enormous number of photons per ADU, such as 99999.

     To avoid small number statistics in your artificial-star analysis,
you should create a number of different frames, each containing only a
few extra stars.  If you try to add too many stars at once your
synthetic frames will be significantly more crowded than your original
frame, making it difficult to apply the artificial-star conclusions to
your program-star results.

\vfill
\eject
\noindent XXI.  {\bf LIST}

If at your installation the standard disk format for digital images to
be reduced with DAOPHOT is the Caltech data-structure file (as it is on
the VMS machines at the DAO), then the {\bf LIST} command will enable
you to examine the contents of the image header.  For instance, let us
suppose that your images originated as FITS files from Kitt Peak or
Cerro Tololo, and that you want to learn the right ascension,
declination, sidereal time, and integration time of your image.  It
happens that all this information is contained in the OBS substructure
of the Caltech data structure, so$\ldots$

\bigskip
{\noindent\obeylines\obeyspaces\frenchspacing\tt\baselineskip=0.3truecm
=======================================================================
| COMPUTER TYPES:                                  YOU ENTER:         |
|                                                                     |
| Command:                                         LI                 |
|                                                                     |
|           File = ?.DST                                              |
|                                                                     |
|      Components: OBS                                                |
|                  Z                                                  |
|                                                                     |
| LIST>                                            OBS.RA             |
|                                                                     |
|           ' 9:23:41'  / right ascension                             |
|                                                                     |
| LIST>                                            OBS.DEC            |
|                                                                     |
|           '-77: 4:48' / declination                                 |
|                                                                     |
| LIST>                                            OBS.ST             |
|                                                                     |
|           '12:53:46'  / sideral time                                |
|                                                                     |
| LIST>                                            OBS.ITIME          |
|                                                                     |
|           150.0                                                     |
|                                                                     |
| LIST>                                            <CR> or CTRL-Z     |
|                                                                     |
| Command:                                                            |
|                                                                     |
=======================================================================
}
\bigskip

If you don't happen to remember the FITS keyword for the particular
information you want, respond to the ``LIST$>$'' prompt with just
``OBS'' and all the keywords will be typed to your terminal; just taken
note of the one you want as it flies past.

I have not yet gotten around to implementing this on the Unix side.
There, the {\bf LIST} command merely reminds you of the name of the
image you are working on and returns you to DAOPHOT Command:\ mode.

\vfill
\eject
\noindent XXII.  {\bf HELP}

This routine simply produces an alphabetical listing of the currently
defined commands on your computer terminal; it does not allow you to
obtain detailed information on any of the routines.  It is included
primarily as a memory-jogger in case you momentarily forget what some
routine is called.

\bigskip
{\noindent\obeylines\obeyspaces\frenchspacing\tt\baselineskip=0.3truecm
=======================================================================
| COMPUTER TYPES:                                  YOU ENTER:         |
|                                                                     |
| Command:                                         HE                 |
|                                                                     |
+---------------------------------------------------------------------+
| COMPUTER TYPES:                                                     |
|                                                                     |
| The commands currently recognized are:                              |
|                                                                     |
|     ADDSTAR      APPEND       ATTACH       DUMP         EXIT        |
|     FIND         FUDGE        GROUP        HELP         LIST        |
|     MONITOR      NOMONITOR    NSTAR        OFFSET       OPTION      |
|     PEAK         PHOTOMETRY   PICK         PSF          SELECT      |
|     SKY          SORT         SUBSTAR                               |
|                                                                     |
| Any command may be abbreviated down to its first two characters.    |
|                                                                     |
=======================================================================
}

\vfill
\eject
\noindent XXIII.  {\bf EXIT}

This command allows you to exit cleanly from DAOPHOT, with all
files properly closed and everything nice and neat.

\bigskip
{\noindent\obeylines\obeyspaces\frenchspacing\tt\baselineskip=0.3truecm
=======================================================================
| COMPUTER TYPES:                                      YOU ENTER:     |
|                                                                     |
| Command:                                             EX             |
|                                                                     |
| Good bye.                                                           |
|                                                                     |
=======================================================================
}

\vfill
\eject
\centerline {E. ALLSTAR}

Unlike everything else in this manual, ALLSTAR is not a
routine within DAOPHOT~II, which can be executed in response to a
``Command:'' prompt.  Rather, ALLSTAR is a separate stand-alone
program which one executes directly from the operating system.  I
have done it this way because that makes it easier to conserve
virtual memory, so that DAOPHOT and ALLSTAR can both operate
on the largest possible images.

In general, ALLSTAR works pretty much the same as the {\bf NSTAR}
routine in DAOPHOT, fitting multiple, overlapping point-spread functions
to star images in your CCD frames.  Input and output images and data
files are fully compatible with those produced by DAOPHOT.  Some of
the noteworthy differences between ALLSTAR and {\bf NSTAR}:

\item{1.} ALLSTAR reduces the entire starlist for a frame at once
(current maximum: 15,000 stars).  With every iteration, ALLSTAR
subtracts {\it all\/} the stars from a working copy of your image
according to the current best guesses of their positions and
magnitudes, computes increments to the positions and magnitudes from
examination of the subtraction residuals around each position, and then
checks each star to see whether it has converged or has become
insignificant.  When a star has converged, its results are written out
and the star is subtracted permanently from the working copy of the
image; when a star has disappeared it is discarded.  In either case,
the program has a smaller problem to operate on for the next
iteration.  Throughout this process, ALLSTAR maintains a noise map of
the image, including knowledge of the Poisson statistics of stars that
have previously converged and been permanently subtracted from the
working copy.  Since the entire star list is a ``group'' (in the sense
of {\bf NSTAR}), ALLSTAR does not require that the starlist be {\bf
GROUP}ed ahead of time, and is perfectly happy to run from your
.AP files.

\item{2.} In order to make the problem tractable (after all, we're
potentially dealing with a 45,000$\times$45,000 matrix with every
iteration), ALLSTAR does associate stars with one another in
order to make the big matrix block-diagonal, so it can be inverted in a
finite amount of time.  These associations are temporary, they are
reconsidered every iteration, and they do not compromise the full,
rigorous, simultaneous, least-squares solution of the entire
star-list.  ALLSTAR will do the best it can to reduce your data
frame regardless of the degree of crowding.  What happens is the
following:  during the process of each iteration, ALLSTAR automatically
associates all the stars into the frame into manageable bite-sizes, on
the basis of a critical separation which is calculated from the fitting
radius.  If you have the ``WATCH PROGRESS'' option equal to 1, then in
a densely-packed frame, you might notice ALLSTAR typing out messages
like

$$\hbox{\rm ``Group too large:  107 (2.50).''}$$

\item{} That means that a group of 107 stars resulted from use of the
nominal critical separation (2.50 pixels, in this case).  ALLSTAR will
attempt to break up this group by regrouping it with smaller and
smaller critical separations until it falls apart into subgroups each
containing fewer than some user-specified number of stars.  (Other
associations, which are already smaller than the maximum size, will
{\it not\/} be regrouped with the smaller critical separation.  Only
the ones that are too big will.)  If the group is so dense that the
critical separation drops to less than 1.2 pixels, the program will
arbitrarily delete the faintest star in the group, and proceed from
there.  By increasing the value of the optional parameter MAXIMUM GROUP
SIZE, the user may reduce the number of faint stars that get rejected
by this mechanism; on the other hand, these stars will be so poorly
measured that they may not be {\it worth\/} retaining.  Of course,
increasing the MAXIMUM GROUP SIZE will also greatly increase the amount
of time required for reducing the frame, since the reduction time goes
roughly as the cube of the size of the largest associations.

\item{3.} ALLSTAR will produce the star-subtracted image for you
directly, without need to run the DAOPHOT routine {\bf SUBSTAR}.  (If
you don't want the output picture produced, when the program asks you
for the star-subtracted image filename respond with a CTRL-Z or type
the words END OF FILE, in capitals.)

\item{4.} If you think that the stars' positions are very well known
ahead of time --- for instance because you have averaged their observed
positions on a number of frames, or because you can apply positions
derived from an excellent-seeing frame to poorer frames --- then it is
possible to tell ALLSTAR not to attempt to adjust the stellar
positions you give it, but just to solve for the magnitudes.  If you
are very, very careful, this can produce more accurate photometry for
your stars than otherwise.  But {\it please be careful!\/} Remember
that if you have frames taken at different airmasses or in different
photometric bandpasses, then stars' positions in the various frames
will not be related by simple zero-point offsets.  There may also be
slight rotations, scale changes, compression along the direction toward
the zenith, and shifts dependent upon the stars' intrinsic colors (due
to atmospheric dispersion). This, then, is an option to use only if you
are sure your predicted positions correctly include these effects.

\item{5.}  Unlike {\bf NSTAR}, ALLSTAR~II is now empowered to
redetermine sky values for the stars in the frame.  This is done as
follows:  the user sets the optional parameters OS (= ``Outer sky
radius'') $> 0$ and IS (= ``Inner sky radius'') $<$ OS, where OS is
large compared to the FITTING RADIUS and not large compared to the
spatial scales of the background-sky variations. Then before every
third iteration (starting with iteration 3), after all the stars have
been subtracted from the working copy of the image, the program
determines the median brightness value in an annulus bounded by these
two radii.  This is used as the sky estimate for the star for the next
three iterations.  If IS $>$ 0, then this sky determination will be
mostly independent of the zit produced by improper subtraction of the
star itself.  I do it this way rather than solving for the sky
brightness (or some analytic function representing the spatial
distribution of the sky brightness) as part of the least-squares
profile fits, because it is far, far easier and more precise to
determine the median of several hundred pixels than to fit a
least-squares surface to one or two dozen.  This is discussed at some
length in Stetson, 1987 PASP, 99, 101, \S III.D.2.a.  Perhaps in
ALLSTAR~II.v (decimal Roman numerals --- far out) I will include
optional fitting of a sky model in the least-squares problem.  If OS
$\leq$ IS, ALLSTAR will continue to use the sky-brightness values that
were in the input data file.

Like DAOPHOT, ALLSTAR begins by showing you a table of optional
parameter settings; unlike DAOPHOT, (but like {\bf PHOTOMETRY}) it
immediately gives you a chance to change any you don't like.

\bigskip
{\noindent\obeylines\obeyspaces\frenchspacing\tt\baselineskip=0.3truecm
========================================================================
\           FITTING RADIUS =  2.50       CE (CLIPPING EXPONENT) =  6.00
\    REDETERMINE CENTROIDS =  1.00          CR (CLIPPING RANGE) =  2.50
\           WATCH PROGRESS =  1.00           MAXIMUM GROUP SIZE = 50.00
\     PERCENT ERROR (in \%) =  0.75         PROFILE ERROR (in \%) =  5.00
\    IS (INNER SKY RADIUS) =  0.00        OS (OUTER SKY RADIUS) =  0.00

\ OPT>
========================================================================
}
\bigskip

The FITTING RADIUS and the WATCH PROGRESS options you are familiar with
from DAOPHOT:  the fitting radius is the size of the region around each
star which will actually be used in performing the profile fits.  The
WATCH PROGRESS option determines just how much garbage will be typed
onto your terminal screen or written into your batch job's logfile.  The
MAXIMUM GROUP SIZE option has been explained above.  The REDETERMINE
CENTROIDS, CLIPPING EXPONENT, and CLIPPING RANGE options are new.

The REDETERMINE CENTROIDS option is how you tell the program whether
to improve the stars' positions in the profile fits.  RE = 0 means
``no'', 1 means ``yes.''  If you enter ``no,'' the program will
assume that the positions of the stars are known with absolute
accuracy, and will redetermine only the stars' magnitudes (this is
one way of imposing a star list from a good frame onto a poor frame
of the same region); if ``yes,'' you will get the full-blown
astrometric and photometric reduction you are used to from {\bf NSTAR}.

The CLIPPING EXPONENT and CLIPPING RANGE options are explained in
Stetson, 1987 PASP, 99, 191, \S III.D.2.d, ``Resisting bad data''.  The
CLIPPING RANGE is variable $a$ in the formula given there, and the
CLIPPING EXPONENT is $b$.  The clipping exponent you specify will be
rounded to the {\it nearest integer\/}.  I have given the default
values RANGE = 2.5 (i.e., a 2.5--$\sigma$ residual gets half weight),
and EXPONENT = 6.0, because in my own experiments they seem to work
reasonably well. I do not have a profound understanding of some
fundamental way to obtain ``best'' values for these parameters; this
subject still needs {\it much\/} more experimentation (by me and, if
you want, by you).  Anyway, on the basis of whatever religious tenets
you hold, adopt values for these parameters. Experiment with them only
if you are prepared to burn up a {\it lot\/} of CPU time.  If you are
thoroughly conservative, setting CLIPPING EXPONENT to 0.0 turns the
clipping off altogether.

Your own default parameter values may be set by creating a file named
ALLSTAR.OPT (allstar.opt under Unix) in your directory.  It works
exactly the same as the DAOPHOT.OPT file does in DAOPHOT, except of
course that it should include only those ten parameters recognizable to
ALLSTAR.

\vfill
\eject
The rest of it's pretty darn trivial.

\bigskip
{\noindent\obeylines\obeyspaces\frenchspacing\tt\baselineskip=0.3truecm
=======================================================================
| COMPUTER TYPES:                                  YOU ENTER:         |
|                                                                     |
|                              Input image name:   filename           |
|                                                                     |
|        Object name from file header                                 |
|                                                                     |
|                    Picture size: nnn nnn                            |
|                                                                     |
|             File with the PSF (default ?.PSF):   <CR> or filename   |
|                                                                     |
|                     Input file (default ?.AP):   <CR> or filename   |
|                                                                     |
|              File for results (default ?.ALS):   <CR> or filename   |
|                                                                     |
|    Name for subtracted image (default ?s.DST):   <CR> or filename   |
|                                                                     |
=======================================================================
}
\bigskip

And away we go!  With each iteration, the program will keep you updated
on how many stars remain to be reduced, how many have disappeared due
to insignificance, and how many have converged and been written out.

Finally, I would like to reiterate that in the final analysis I wrote
ALLSTAR, like DAOPHOT, for {\it me\/}.  I make no promise to
leave it alone or to notify you of minor changes.  If I discover some
major bug that destroys the science, I may remember that I gave you a
copy and then again I may not.  As I get time to play, I will certainly
be attempting to make the program more powerful and reliable for {\it
my\/} applications.   Therefore, in some sense you use this copy of the
program (and of DAOPHOT~II) at your own risk.  Use it as long as you
are happy with the results.  If you don't like what you are getting,
stop using the program and complain to me.  Maybe I will have already
fixed your problem, or maybe your problem will be interesting or
important enough that I will want to fix it for you.  However, I get
bent all out of shape when somebody has a problem with my software and
publishes complaints in the literature, without ever coming to {\it
me\/} to give me a chance to fix it for them, or to explain some point
they may have misunderstood.

\vfill
\eject
\centerline{APPENDIX I}
\bigskip
\centerline{Optional parameters}
\bigskip
{\noindent\obeylines\obeyspaces\frenchspacing\tt\baselineskip=0.3truecm

=============================================================================
\                                          Routines      Permitted     Default
ID        Description (Note)              Affected        values       value
-----------------------------------------------------------------------------
RE  Readout noise, 1 exposure (ADU) (1)     FIND         positive        0

GA  Gain, 1 exposure (photons per ADU)      FIND         positive        0
\    (1)

LO  Low good datum (standard deviations)    FIND       non-negative      7
\    (1)

HI  High good datum (ADU) (1)               FIND       non-negative   32766.5

FW  FWHM of objects for which FIND          FIND        0.2 - 15.0      2.5
\    is to be optimized (in pixels)

TH  Significance threshold for detection    FIND       non-negative     4.0
\    (standard deviations)

LS  Low sharpness cutoff                    FIND         0.0 - 0.6      0.2

HS  High sharpness cutoff                   FIND         0.6 - 2.0      1.0

LR  Low roundness cutoff                    FIND        -2.0 - 0.0     -1.0

HR  High roundness cutoff                   FIND         0.0 - 2.0      1.0
\    for the profile fits.

WA  Watch progress of reductions on    FIND,PHOT,PEAK,    -2 - 2         1
\    terminal?                            PSF,NSTAR,
\                                        SUBSTAR,SORT

FI  The fitting radius (in pixels)        PSF,PEAK,      1.0 - 10.0     2.0
\                                         GROUP,NSTAR

PS  PSF radius: radius (in pixels)          PSF          1.0 - 35.0    11.0
\    within which the point-spread
\    function is to be defined. (2)

VA  Degree of variation in the PSF (2)      PSF           -1 - 2         0

FR                         ***** NOT IMPLEMENTED *****

AN  Which analytic formula for PSF (2)      PSF            1 - 6         1

EX  How many passes to clean discordant     PSF            0 - 9         0
\    pixels from the PSF table(s)

PE  Percent error (e.g. flat-field)      PEAK,NSTAR       0 - 100       0.75

PR  Profile error (inadequate PSF)       PEAK,NSTAR       0 - 100       5.0

=============================================================================

}

\noindent Notes:

\item{(1)}  {\bf FIND} is the only routine where these values are read
from the options table.  However, the file headers carry these numbers
along to other routines, which will also be affected.

\item{(2)} {\bf PSF} is the only routine where these values are read from
the options table.  However, the .PSF file will carry these numbers along
to other routines, which will then act accordingly.

\vfill
\eject
\centerline{APPENDIX II}
\bigskip
\centerline{The {\bf FIND} Threshold:  Theory and Practice}
\bigskip

Assume that a given frame has an average sky brightness of $s$ (in
units of ADU), a gain factor of $p$ photons per ADU, and a readout
noise per pixel of $r$ (in electrons or, equivalently, photons.  Note
that since ``electrons'' and ``photons'' are simply number counts, they
really have no physical dimensions --- this is inherent in the use of
Poisson statistics, where $\sigma (N) = \sqrt{N}$ would be meaningless
if any physical dimensions were involved.)  The expected random noise
per pixel may now be computed as follows:

\bigskip
\noindent IN UNITS OF PHOTONS:
\bigskip

If sky brightness in photons $= p \times s$(ADU), then for the$\ldots$

\noindent Poisson statistics of the sky:

$$\eqalignno{
\hbox{\rm Variance (sky)} &= \sigma^2(\hbox{\rm sky})\cr
&= \hbox{\rm number of photons}&\hbox{\rm [Poisson statistics]}\cr
&= p \times s &\hbox{\rm (variance is in dimensionless units.)}\cr
\noalign{\hbox{Readout noise:}}
\hbox{\rm Variance (readout)} &= \sigma^2(\hbox{\rm readout})\cr
&= r^2 &\hbox{\rm (dimensionless units) [by definition]}\cr
\noalign{\hbox{Total noise:}}
\hbox{\rm Variance(total)} &= \hbox{\rm variance(sky)}\cr
&\qquad + \hbox{\rm variance(readout)}&
\hbox{\rm [propagation of error]}\cr
&= p\times s + r^2 &\hbox{\rm (dimensionless units)}\cr
\hbox{\rm standard deviation} &= \sqrt{p \times s + r^2}
&\hbox{\rm (dimensionless units)}
}$$

\noindent (Note that in this equation $p$ has units of photons/ADU, $s$
has units of ADU, and $r$ has units of electrons or photons.  This is
very clumsy, but that's the way we usually think of these numbers.)

\bigskip

\noindent IN UNITS OF ADU:

$$\eqalignno{
\sigma(\hbox{\rm ADU}) &= {\sigma(\hbox{\rm photons})\over
\hbox{\rm (photons per ADU)}} &\hbox{\rm [propagation of error]}\cr
\noalign{\hbox{Let $R$ = readout noise in ADU $\equiv r/p$.  Then}}
\sigma(\hbox{\rm total}) &=
\sqrt{s/p + R^2}&\hbox{\rm (units are ADU).}\cr
}$$

\eject

Please note that if the frame you are working on is the average or the
sum of several raw data frames, the values of the gain factor (in
photons per ADU) and the readout noise will have to be adjusted
accordingly:

\bigskip
{\noindent\obeylines\obeyspaces\frenchspacing\tt\baselineskip=0.3truecm
\ ----------------------------------------------+------------------------------
\               If N frames were averaged       |  If N frames were summed
\ ----------------------------------------------+------------------------------
\                                               |
\ photons/ADU   p(N) = N*p(1)                   |  p(N) = p(1)
\                                               |
\ r-o noise     R(N) = R(1)/SQRT(N)             |  R(N) = R(1)*SQRT(N)
\                                               |
\ total         std. dev. (N) =                 |  std. dev. (N) =
\                  SQRT(s/[N*p(1)] + [R**2]/N)  |   SQRT(s/p(1) + N*[R(1)]**2)
\                                               |
\ ----------------------------------------------+------------------------------
}
\bigskip

\noindent The value of $s$ which {\bf FIND} uses in these equations is
the number you get when you issue the DAOPHOT command ``{\bf SKY}''.

Note that the expectation value of $s$ scales as follows:

\bigskip
{\noindent\obeylines\obeyspaces\frenchspacing\tt\baselineskip=0.3truecm
\ ---------------------------------------+-------------------------------------
\ If N frames were averaged              |  If N frames were summed
\ ---------------------------------------+-------------------------------------
\                                        |
\ s(N) = s(1)                            |  s(N) = N*s(1)
\                                        |
\ std. dev.(N) = SQRT                    |  std. dev.(N) = SQRT
\          (s(1)/[N*p(1)] + [R(1)]**2/N) |     (s(1)*N/p(1) + N*[R(1)/p(1)]**2)
\                                        |
\              = std. dev.(1)/SQRT(N)    |               = std. dev.(1)*SQRT(N)
\                                        |
\ ---------------------------------------+-------------------------------------
}
\bigskip

Therefore, to ensure that the statistics are being done properly, for
your frames {\bf FIND} takes the readout noise per frame IN UNITS OF
ADU, and the ratio of photons per ADU, and corrects them for the number
of frames that have been averaged or summed according to the first
table above.

\vfill
\eject
\noindent TO ESTABLISH A REASONABLE THRESHOLD FOR YOUR STAR-FINDING:
\medskip

\noindent {\bf FIND} computes the random error per pixel {\it in units
of ADU} from

$$
\hbox{\rm random error in 1 pixel} = \sqrt{ {s\over p_N } + R_N^2}
$$

\noindent where $s$ is the number you get from {\bf SKY} and
appropriate values of $p_N$ and $R_N$ have been computed according to
the table above.  When you then run {\bf FIND}, you will discover that
it also gives you a number called ``relative error''.  This is because
in trying to establish whether there is a star centered in a certain
pixel, {\bf FIND} operates on weighted sums and differences of several
adjacent pixels.  The ``relative error'' is merely a scaling parameter
--- it is the number that the standard error of one pixel must be
multiplied by to obtain the standard error of the smoothed/differenced
data in the convolved picture.  Therefore, to arrive at a reasonable
threshold, {\bf FIND}, computes the standard error per pixel as
described above, multiplies it by the ``relative error'' factor, and
sets the threshold at the multiple of this number which you have
specified with the ``THRESHOLD'' option, say,

$$\hbox{\rm Threshold} = 3.5 \times (\hbox{\rm relative error})
\times (\hbox{\rm standard error in one pixel}) $$

\noindent for 3.5-$\sigma$ detections.  (A +3.5-$\sigma$ excursion
occurs about 233 times per one million independent random events.  In
an otherwise empty $300 \times  500$ frame this would produce about 35
false detections.  In a frame some non-trivial fraction of whose area
was occupied by real stars, the false detections would be fewer.)

\vfill
\eject
\noindent TO SUMMARIZE:

\item{(1)} Ascertain the values of the readout noise (in electrons or
photons) and the number of photons per ADU for a {\it single frame} for the
detector you used.  Specify these as {\it options\/}.  {\bf FIND} will

\item{(2)} compute the readout noise in ADU:

$$
R(\hbox{\rm 1 frame; ADU}) = {r(\hbox{\rm 1 frame; photons}) \over
p(\hbox{\rm 1 frame; photons/ADU})};$$

\item{(3)} correct the ratio of photons per ADU and the readout noise
(in ADU) for the number of frames that were averaged or summed:

\bigskip
{\noindent\obeylines\obeyspaces\frenchspacing\tt\baselineskip=0.3truecm
\ ----------------------------------------+-----------------------------
\               If N frames were averaged |  If N frames were summed
\ ----------------------------------------+-----------------------------
\                                         |
\ photons/ADU   p(N) = N*p(1)             |  p(N) = p(1)
\                                         |
\ r-o noise     R(N) = R(1)/SQRT(N)       |  R(N) = R(1)*SQRT(N)
\                                         |
\ ----------------------------------------+-----------------------------
}
\bigskip

\item{(4)} determine the typical sky brightness, $s$, in your data
frame, by using the DAOPHOT command {\bf SKY};

\item{(5)} compute the random error per pixel:
$$ \hbox{\rm random error in 1 pixel} = \sqrt{s/p_N + R_N^2};$$

\item{(6)} multiply by the relative error defined above to arrive at the
random noise in the sky background of the {\it convolved\/} data frame:
$$
\hbox{\rm background noise} = (\hbox{\rm relative error}) \times
(\hbox{\rm random error per pixel})
$$

Some multiple (of order 3 -- 5, say) of this background noise, as
specified by the user) is used as your desired star-detection threshold.

\vfill
\eject
\centerline {APPENDIX III}
\bigskip
\centerline{DERIVING A POINT-SPREAD FUNCTION IN A CROWDED FIELD}
\bigskip

Obtaining a good point-spread function in a crowded field is a delicate
business, so please do not expect to do it quickly --- plan on spending
a couple of hours in the endeavor the first few times you try it.
After that it gets easier, and once you know what you're trying to do,
it's fairly easy to write command procedures to handle major parts of
the FIT--SUBTRACT--MAKE A NEW PSF--FIT AGAIN--$\ldots$ loop.  I
recommend that you proceed {\it approximately} as follows:

\noindent Invoke DAOPHOT and:

\item{(1)} Run {\bf FIND} on your frame.

\item{(2)} Run {\bf PHOTOMETRY} on your frame.

\item{(3)} {\bf SORT} the output from {\bf PHOTOMETRY} in order of
increasing apparent magnitude (decreasing stellar brightness), with the
renumbering feature.  This step is optional, but it can be more
convenient than not.

\item{(4)} {\bf PICK} to generate a set of likely PSF stars.  How many
stars you want to use is a function of the degree of variation you
expect, and the frequency with which stars are contaminated by cosmic
rays or neighbor stars.  I'd say you'd want a rock-bottom minimum of
three stars per degree of freedom, where the degrees of freedom are 1
(constant PSF), 3 (linearly varying PSF), and 6 (quadratically varying
PSF).  I'm referring here to the number of degrees of freedom you
expect you'll need {\it ultimately\/}.  PSF stars are weighted
according to their magnitudes, so it doesn't hurt to include many faint
(but good) stars along with a few bright (but good) ones. {\it The more
crowded the field, the more important it is to have many PSF stars, so
that the increased noise caused by the neighbor-subtraction procedure
can be beaten down.} Furthermore, if you intend to use the variable-PSF
option, it is vital that you have PSF stars spread over as much of the
frame as possible.  I don't feel particularly bad if I end up using as
many as 25 or 50 PSF stars in such a case, but maybe I'm too cautious.

\item{(5)} Run {\bf PSF}, tell it the name of your complete (sorted,
renumbered) aperture photometry file, the name of the file with the
list of PSF stars, and the name of the disk file you want the
point-spread function stored in (the default should be fine).

\itemitem{(a)} If you have the ``WATCH PROGRESS'' option equal to 1 or
2, {\bf PSF} will produce on your terminal a gray-scale plot of each
star and its environs, it will tell you the number of ADU in the
brightest pixel within the area displayed, and then it will
ask whether you wish to include this star in the point-spread
function, to which you should answer ``Y'' or ``N'', as appropriate.

\itemitem{(b)} If ``WATCH PROGRESS'' is 0, it will go ahead and use the
star, unless it finds a bad pixel more than one fitting radius and less
than (one PSF radius plus 2 pixels) from the star's center; in such a
case it will ask whether you want the star.

\itemitem{(c)} If ``WATCH PROGRESS'' = --1 or --2, it will use the
star, regardless, counting on the bad data rejection to clean out any
bad pixels.  It will {\it report\/} the presence of bad pixels, but it
will use the star anyway.

\item{} If the frame is crowded, it is probably worth your while to
generate the first PSF with the ``VARIABLE PSF'' option set to --1 ---
pure analytic PSF.  That way, the companions will not generate ghosts
in the model PSF that will come back to haunt you later.  You should
also have specified a reasonably generous fitting radius --- these
stars have been preselected to be as isolated as possible, and you want
the best fits you can get.  But remember to avoid letting neighbor
stars intrude within one fitting radius of the center of any PSF star.

\item{(6)} Run {\bf GROUP} and {\bf NSTAR} or ALLSTAR on your
.NEI file.  If your PSF stars have many neighbors this may take some
minutes of real time.  Please be patient (or submit it as a batch
job and perform steps 1 -- 5 on your next frame while you wait).

\item{(7)} After {\bf NSTAR} is finished, run {\bf SUBSTAR} to
subtract the stars in the output file from your original picture.
This step is unnecessary if you used ALLSTAR. (And why didn't
you?)

\item{(8)} {\bf EXIT} from DAOPHOT and send this new picture to the
image display. Examine each of the PSF stars and its environs.  Have
all of the PSF stars subtracted out more or less cleanly, or should
some of them be rejected from further use as PSF stars?  (If so, use
a text editor to delete these stars from the .LST file.)  Have the
neighbors mostly disappeared, or have they left behind big zits?  Have
you uncovered any faint companions that {\bf FIND} missed?  If the
latter, then

\itemitem{(a)} use the cursor on your image display to measure the
positions of the new companions;

\itemitem{(b)} use your system's text editor to create a .COO file
containing star ID numbers which you invent and the $(x,y)$ coordinates
of the new stars [FORMAT (1X, I5, 2F9.?)];

\itemitem{(c)} re-enter DAOPHOT, {\bf ATTACH} the original image, run
{\bf PHOTOMETRY} to get sky values and crude magnitudes for the faint
stars;

\itemitem{(d)} run {\bf PEAK}, {\bf GROUP} $+$ {\bf NSTAR}, or ALLSTAR
to get slightly improved positions and magnitudes for the stars; and

\itemitem{(e)} {\bf APPEND} this .PK, .NST, or .ALS file to the one
generated in step (6).

\itemitem{(f)} Using the original picture again, run {\bf GROUP} $+$
{\bf NSTAR} or ALLSTAR on the file created in step 8(e).

\item{(9)} Back in DAOPHOT~II, {\bf ATTACH} the original picture and
run {\bf SUBSTAR}, specifying the file created in step~6 or in step
8(f) as the stars to subtract, and the stars in the .LST file as the
stars to keep.  You have now created a new picture which has the PSF
stars still in it, but from which the known neighbors of these PSF
stars have been mostly removed.  If the PSF was made with
``VARIABLE PSF'' = --1, the neighbors are maybe only 90 or 95\%
removed, but still that's a big gain.

\item{(10)} {\bf ATTACH} the new star-subtracted frame and repeat step
(5) to derive a new point-spread function.  This time you should have
``VARIABLE PSF'' = 0 (unless the pure analytic PSF did a great job of
erasing the stars).  Specify the file which
you created in step~6 or 8(f) as the input file with the stellar photometry,
since this file has the most recent and best estimates for the
positions and magnitudes of the PSF stars and their neighbors. If the
zits left behind when some of the neighbor stars were subtracted
trigger the ``bad pixel'' detector, answer ``Y'' to the question
``Try this one anyway?'' and count on the bad-pixel rejection to fudge
them away (if ``EXTRA PSF CLEANING PASSES'' is greater than zero;
otherwise you're counting on the large number of PSF stars to beat the
zits down in the average profile).

\item{(11+...)} Run {\bf GROUP} $+$ {\bf NSTAR} or ALLSTAR on the
file you created in step 6 or 8(f); loop back to step (5) and iterate
as many times as necessary to get everything to subtract out as cleanly
as possible.  Remember that each time through the loop you should be
obtaining the new point-spread function from a frame in which the
neighbors (but {\it not\/} the PSF stars) have been subtracted, while
{\bf NSTAR} or ALLSTAR should be run on the original picture,
with all the stars still in it (except when you are trying to get crude
data for new stars you have just identified by hand and eye on the
image display).  Increase the ``VARIABLE PSF'' option by one per
iteration, if the neighbors are subtracting out relatively cleanly.
Once you have produced a frame in which the PSF stars and their
neighbors all subtract out cleanly, one more time through PSF should
produce a point-spread function you can be proud of.

\vfill
\eject
\centerline{APPENDIX  IV}
\bigskip
\centerline{DATA FILES}
\medskip

DAOPHOT~II writes its output data to disk in ordinary ASCII
sequential-access files which may be TYPEd, PRINTed, and EDITed with
your operating system's utilities, and may be shipped freely from one
machine to another.  Files produced by the different routines have some
things in common.  The first of these is a three-line file header
which, in its most elaborate form, looks like this:

\bigskip
{\noindent\obeylines\obeyspaces\frenchspacing\tt\baselineskip=0.3truecm

======================================================================
\ NL   NX   NY  LOWBAD HIGHBAD  THRESH     AP1  PH/ADU  RNOISE    FRAD
\  1  284  492   400.0 24000.0    20.0    3.00   20.00    6.50     2.0
~~~~~~~~~~~~
======================================================================
}
\bigskip

The purpose of this header is to provide a partial record of the
analysis which produced the file, and to supply some auxiliary
information to subsequent routines (so you don't have to).  As the data
reduction proceeds through {\bf FIND}, {\bf PHOTOMETRY}, and profile
fitting, each routine adds more information to the header. NL is a
historical artifact --- it started out meaning ``Number of lines'',
where NL=1 indicated that the file contained one line of data per star
(output from {\bf FIND} or {\bf PEAK}, for example), and NL=2 flagged
output from {\bf PHOTOMETRY}, where two lines of data are generated per
star.  NL has since ceased to have precisely this significance; now it
serves more as a flag to the program to tell it where in the data
record the sky brightness for each star may be found.  For files
produced by {\bf FIND}, {\bf PEAK}, {\bf NSTAR}, and ALLSTAR, NL=1; for
files produced by {\bf PHOTOMETRY}, NL=2; for files produced by {\bf
GROUP} (and {\bf SELECT}), NL=3.  {\bf SORT}, {\bf OFFSET}, and {\bf
APPEND} produce output files retaining the form of whatever file was
used for input.

Items NX and NY in the file header are the size of the picture in
pixels. LOWBAD and HIGHBAD are the good data limits, the former
calculated in {\bf FIND} and the latter defined as an optional
parameter.  THRESH is the threshold that was calculated in {\bf FIND}.
AP1 is the radius in pixels of the first aperture specified for {\bf
PHOTOMETRY}, PH/ADU and RNOISE are the numerical values of photons/ADU
and readout noise which were specified as options when {\bf FIND} was
run.  FRAD is the value of the fitting radius (user-alterable
optimizing parameter) which was in effect when the file was created.

One other thing which the output data files have in common is the
format of the first four numbers in the data for each star.  Always
they are:

\centerline{Star ID, X centroid, Y centroid, magnitude}

\noindent followed by other stuff, with format

\centerline{( 1X, I5, 3F9.3, nF9.? ) .}

\noindent In the output from PHOTOMETRY, alone, this is followed by
another data line per star.

In the pages that follow, a sample of each of the output formats is
shown.  An example of the disk file that contains the numerical
point-spread function is also provided.
\vfill
\eject
\centerline{Sample output from {\bf FIND} (a .COO file)}
\bigskip
{\noindent\obeylines\obeyspaces\frenchspacing\tt\baselineskip=0.3truecm

=============================================================
\ NL  NX  NY  LOWBAD HIGHBAD  THRESH
\  2 284 492   400.0 24000.0    20.0
~~~~~~~~~~~~~~~~~~~~~~~~
\  1    6.953    2.612   -0.053    0.333   -0.546
\  2  200.061    2.807   -3.833    0.597    0.000
\  3  156.254    5.171   -3.306    0.653   -0.144
\  4  168.911    5.318   -1.137    0.613    0.288
\  5  110.885    9.742   -7.080    0.590    0.040
\  6  147.949   10.208   -0.440    0.489    0.034
\  7   64.002   13.161   -2.427    0.643   -0.124
\  8  270.856   12.738   -2.304    0.602    0.074
\  9   14.925   13.842   -2.976    0.627   -0.045
\ 10   38.813   15.268   -1.491    0.643    0.065
\ 11   93.695   15.164   -3.687    0.625   -0.203
\ 12  139.798   15.715   -1.156    0.599    0.034
\ 13  207.929   16.209   -3.608    0.649   -0.062
\                           .
\                           .
\                           .
=============================================================
\ (1)    (2)      (3)      (4)      (5)      (6)
}
\item{(1)} Star ID number.
\item{(2)} X coordinate of stellar centroid.
\item{(3)} Y coordinate of stellar centroid.
\item{(4)} Star's magnitude, measured in magnitudes relative to star-finding
threshold (hence never positive, since stars fainter than the
threshold are rejected, obviously).
\item{(5)} Sharpness index (see {\bf FIND} above).
\item{(6)} Roundness index (see {\bf FIND} above).
\vfill
\eject
\centerline{Sample output from {\bf PHOTOMETRY} (a .AP file)}
\bigskip
{\noindent\obeylines\obeyspaces\frenchspacing\tt\baselineskip=0.3truecm

==============================================================
\ NL   NX   NY  LOWBAD HIGHBAD  THRESH     AP1  PH/ADU  RNOISE
\  2  284  492   400.0 24000.0    20.0    3.00   20.00    6.50
~~~~~~~~~~~~~~~~~~~
\     1    6.953    2.612   99.999   99.999   99.999   99.999 . . .
\      464.618  9.71  0.52   9.999    9.999    9.999    9.999 . . .
~~~~~~~~~~~~~~~~~~~
\     2  200.061    2.807   99.999   99.999   99.999   99.999
\      465.180  7.79  0.16   9.999    9.999    9.999    9.999
~~~~~~~~~~~~~~~~~~~
\     3  156.254    5.171   14.610   14.537   14.483   14.438
\      462.206  7.26  0.37   0.013    0.014    0.015    0.016
~~~~~~~~~~~~~~~~~~~
\     4  168.911    5.318   16.261   16.056   15.855   15.658
\      463.292  7.16  0.36   0.055    0.053    0.050    0.048
~~~~~~~~~~~~~~~~~~~
\     5  110.885    9.742   10.792   10.728   10.688   10.660
\      463.926  6.81  0.24   0.001    0.001    0.001    0.001
~~~~~~~~~~~~~~~~~~~
\     6  147.949   10.208   17.167   17.084   17.019   16.878
\      462.241  7.16  0.37   0.124    0.135    0.145    0.144
~~~~~~~~~~~~~~~~~~~
\     7   64.002   13.161   15.620   15.569   15.549   15.538
\      462.009  5.93  0.25   0.025    0.028    0.032    0.035
~~~~~~~~~~~~~~~~~~~
\     8  270.856   12.738   15.566   15.527   15.493   15.460
\      460.965  6.28  0.14   0.026    0.029    0.032    0.035
~~~~~~~~~~~~~~~~~~~
\     9   14.925   13.842   14.951   14.863   14.800   14.744
\      463.874  6.65  0.25   0.016    0.017    0.018    0.019
~~~~~~~~~~~~~~~~~~~
\    10   38.813   15.268   16.200   16.113   16.052   16.019
\      462.127  8.63  0.50   0.062    0.066    0.072    0.079
\                                   .
\                                   .
\                                   .
=================================================================
\    (1)    (2)      (3)      (4)      (5)      (6)      (7)  ...
\       (16)   (17)  (18)    (19)     (20)     (21)     (22) ...
}

\item{(1)} Star ID number.
\item{(2)} X coordinate of stellar centroid, same as in .COO file.
\item{(3)} Y coordinate of stellar centroid, same as in .COO file.
\item{(4)} Star's magnitude in aperture 1, measured in magnitudes relative to
a zero-point of 1 star ADU = 25.0 mag.
\item{(5)} Star's magnitude in aperture 2, ditto ditto.
\item{(6)} Star's magnitude in aperture 3, ditto ditto.
\item{(7)} Star's magnitude in aperture 4, ditto ditto.
\item{(8)-(15)} Ditto ditto.
\item{(16)} Estimated modal sky value for the star.
\item{(17)} Standard deviation of the sky values about the mean.
\item{(18)} Skewness of the sky values about the mean.
\item{(19)} Estimated standard error of the star's magnitude in aperture 1.
\item{(20)} Ditto ditto aperture 2.
\item{(21)} Ditto ditto aperture 3.
\item{(22)} Ditto ditto aperture 4.
\item{(23)-(30)} Ditto ditto.
\medskip
Magnitudes for a number of stars are 99.999 +-- 9.999, because the
aperture extends beyond the boundary of the frame.

\vfill
\eject

\centerline{Sample output from {\bf PEAK}, {\bf NSTAR}, or ALLSTAR}
\centerline {(a .PK, .NST, or .ALS file)}
\bigskip
{\noindent\obeylines\obeyspaces\frenchspacing\tt\baselineskip=0.3truecm

=============================================================================
\ NL   NX   NY  LOWBAD HIGHBAD  THRESH     AP1  PH/ADU  RNOISE    FRAD
\  1  284  492   400.0 24000.0    20.0    3.00   20.00    6.50     2.0
~~~~~~~~~~~~~~~~~~~
\    1    7.413    2.652   18.771    0.421  464.618      10.     0.71   -0.399
\    2  200.131    2.807   14.094    0.012  465.180       5.     0.92   -0.013
\    3  156.454    5.421   14.629    0.018  462.206       4.     1.09   -0.012
\    4  168.921    5.738   16.528    0.053  463.292       6.     0.63    0.056
\    5  110.865    9.732   10.793    0.007  463.926       3.     0.76   -0.015
\    6  147.759   10.478   17.330    0.128  462.241       9.     0.86    0.044
\    7   64.032   13.401   15.595    0.022  462.009       4.     0.54   -0.021
\    8  270.776   12.748   15.522    0.029  460.965       3.     0.93   -0.006
\    9   14.925   13.882   14.982    0.015  463.874       3.     0.58   -0.016
\   10   38.833   15.508   16.316    0.055  462.127       5.     0.94    0.075
\   11   93.625   15.354   14.188    0.015  462.560       4.     1.21    0.026
\   12  139.818   15.875   17.009    0.081  465.570       3.     0.63   -0.069
\   13  207.909   16.459   14.385    0.012  463.223       4.     0.74   -0.009
\                                       .
\                                       .
\                                       .
=============================================================================
\   (1)    (2)      (3)      (4)      (5)     (6)        (7)     (8)      (9)
}
\item{(1)} Star ID number.
\item{(2)} X coordinate of stellar centroid; a more accurate value than before.
\item{(3)} Y coordinate of stellar centroid; a more accurate value than before.
\item{(4)} Star's magnitude, measured in magnitudes relative to the magnitude
of the PSF star (see discussion of .PSF file below).
\item{(5)} Estimated standard error of the star's magnitude.
\item{(6)} Estimated modal sky value for the star (from {\bf PHOTOMETRY},
see above).
\item{(7)} Number of iterations required for the non-linear least-squares to
converge.
\item{(8)} CHI - a robust estimate of the ratio:  the observed pixel-to-pixel
scatter from the model image profile DIVIDED BY the expected
pixel-to-pixel scatter from the image profile.
\item{(9)} SHARP:  another image quality diagnostic (see {\bf PEAK}
command above).
SHARP is most easily interpreted by plotting it as a function of apparent
magnitude.  Objects with SHARP significantly greater than zero are
probably galaxies or unrecognized doubles;  objects with SHARP
significantly less than zero are probably bad pixels or cosmic rays that
made it past {\bf FIND}.
\vfill
\eject
\centerline{Sample of a point-spread function file (a .PSF file)}
\bigskip
{\noindent\obeylines\obeyspaces\frenchspacing\tt\baselineskip=0.3truecm

\  (1)      (2)   (3)  (4)  (5)   (6)           (7)        (8)      (9)
       (10)         (11)         (12)          (13)
    (14)...
============================================================================
\ PENNY1     51    4    3    0   12.033      33419.105    158.5    255.0
\   1.02691E+00  1.06132E+00  7.23460E-01  1.44567E-01
\  1.348391E-02 1.348391E-02 1.348391E-02 1.348391E-02 1.348391E-02 1.348391E-02
\  1.348391E-02 1.348391E-02 1.348391E-02 1.348391E-02 1.348391E-02 1.348391E-02
\  1.348391E-02 1.348391E-02 1.348391E-02 1.348391E-02 1.348391E-02 1.348391E-02
\ -6.484396E+01-6.400895E+01-6.459602E+01-6.713283E+01-6.921574E+01-6.974660E+01
\ -6.971574E+01-6.964603E+01-7.014906E+01-7.076376E+01-7.028298E+01-6.946377E+01
\ -6.923672E+01-6.868890E+01-6.699422E+01 1.348391E-02 1.348391E-02 1.348391E-02
\  1.348391E-02 1.348391E-02 1.348391E-02 1.348391E-02 1.348391E-02 1.348391E-02
\  1.348391E-02 1.348391E-02 1.348391E-02 1.348391E-02 1.348391E-02 1.348391E-02
\  1.348391E-02 1.348391E-02 1.348391E-02 1.348391E-02 1.348391E-02 1.348391E-02
\  1.348391E-02 1.348391E-02 1.348391E-02 1.348391E-02 1.348391E-02 1.348391E-02
\  1.348391E-02 1.348391E-02 1.348391E-02 1.348391E-02 1.348391E-02 1.348391E-02
\ -6.271802E+01-6.469427E+01-6.637348E+01-6.902106E+01-6.919862E+01-6.852924E+01
\ -6.951303E+01-7.280193E+01-7.551125E+01-7.588057E+01-7.524973E+01-7.459446E+01
\ -7.402506E+01-7.357459E+01-7.343834E+01-7.333043E+01-7.215495E+01-7.010623E+01
\ -6.792765E+01-6.512524E+01-6.251689E+01 1.348391E-02 1.348391E-02 1.348391E-02
\  1.348391E-02 1.348391E-02 1.348391E-02 1.348391E-02 1.348391E-02 1.348391E-02
\  1.348391E-02 1.348391E-02 1.348391E-02 1.348391E-02 1.348391E-02 1.348391E-02
\  1.348391E-02 1.348391E-02 1.348391E-02 1.348391E-02 1.348391E-02 1.348391E-02
\                                        .
\                                        .
\                                        .
==============================================================================
}
\item{(1)}  An alphanumeric string uniquely defining the module used to define
the analytic first approximation to the PSF.

\item{(2)} Size of the array containing the PSF look-up table.  Values are
tabulated at half-pixel intervals in a square array which extends
somewhat beyond the user-specified PSF radius:
$N = 2(2\cdot\hbox{\rm radius}+1)+1$. Here the PSF radius was 12.

\item{(3)} The number of shape parameters in the analytic function.

\item{(4)} The number of look-up tables in the PSF.  Here a linearly
varying PSF was used.

\item{(5)} Fractional-pixel expansion.  ***** NOT IMPLEMENTED *****

\item{(6)} The instrumental magnitude corresponding to the point-spread
function of unit normalization.

\item{(7)} Central height, in ADU, of the analytic function which is used as
the first-order approximation to the point-spread function in the stellar core.

\item{(8),(9)} X and Y coordinates of the center of the frame (used for
expanding the variable PSF).

\item{(10)-(13)}  The shape parameters of the analytic function.  In this
case there are four of them.  The first two are always the half-width at
half-maximum in x and y.

\item{(14)$\ldots$}  The look-up table of corrections from the analytic
first approximation to the ``true'' point-spread function.
\vfill
\end
