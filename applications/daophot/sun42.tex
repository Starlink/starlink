\documentclass[11pt,nolof]{starlink}


% -----------------------------------------------------------------------------
% ? Document identification
\stardoccategory    {Starlink User Note}
\stardocinitials    {SUN}
\stardocsource      {sun42.5}
\stardocnumber      {42.5}
\stardocauthors     {Nicholas Eaton and Grant Privett}
\stardocdate        {6 December 1996}
\stardoctitle     {DAOPHOT --- Stellar Photometry Package}
\stardocversion     {Version 1.2}
\stardocmanual      {Introduction}
\stardocabstract  {
DAOPHOT is a stellar photometry package designed by Peter Stetson at DAO to deal
with crowded fields. The package performs various tasks including
finding objects, aperture photometry, obtaining the point-spread function, and
profile-fitting photometry. Profile fitting in crowded regions is performed
iteratively, which improves the accuracy of the photometry.
}
% ? End of document identification

% -----------------------------------------------------------------------------

% ? Document specific \providecommand or \newenvironment commands.
% ? End of document specific commands
% -----------------------------------------------------------------------------
%  Title Page.
%  ===========
\begin{document}
\scfrontmatter

\section{\xlabel{INTRODUCTION}Introduction}
\label{sec:intro}

DAOPHOT is a stellar photometry package designed by Peter Stetson at DAO to deal
with crowded fields. The package performs various tasks including
finding objects, aperture photometry, obtaining the point-spread function, and
profile-fitting photometry. Profile fitting in crowded regions is performed
iteratively, which improves the accuracy of the photometry.

This document is an introduction to DAOPHOT II: The Next Generation. It replaces
the previous version which is known as DAOPHOT Classic. The main changes concern
the choice of the point-spread fitting function and the handling of under-sampled
data. This version uses image data written in NDF, the Starlink N-Dimensional
Data Format \xref{(SUN/33)}{sun33}{}, that contain a DATA\_ARRAY component
and have an extension \texttt{.sdf}.

A document written by Peter Stetson (DAOPHOT: A computer program for crowded-field
stellar photometry, Publ.astr.Soc.Pacific, \textbf{99}, 191-222, 1987)
gives the background to the algorithms used by DAOPHOT. A user's manual (MUD/9)
is also available through Starlink.

DAOPHOT does not directly use an image display, which is one of the reasons why the
package has been successfully ported round the world. Three additional routines
have therefore been supplied which allow results obtained with DAOPHOT to be
displayed on an image device.

The routine DAOGREY (see \S{\ref{sec:daogrey}})
will display a grey scale image of the data on a suitable
device. DAOPLOT (see \S{\ref{sec:daoplot}})
will indicate the positions of objects found with DAOPHOT on top of
the grey image. DAOCURS (see \S{\ref{sec:daocurs}})
will put up a cursor on the display to allow positions to
be measured from the screen. Although KAPPA (\xref{SUN/95}{sun95}{}) routines could
be used for the image display and cursor interaction, DAOPHOT does not comply with
the Starlink coordinate system convention (\xref{SUN/33}{sun33}{}) and thus any
coordinates would be half a pixel out.

DAOPHOT is copyright of Peter Stetson at the Dominion Astrophysical Observatory.
The algorithms should not be changed without permission.


\section{\xlabel{RUNNING}Running DAOPHOT}

\label{sec:running}

DAOPHOT is invoked by the simple command:

\begin{terminalv}
% daophot
\end{terminalv}

this should generate a response similar to:

\begin{terminalv}
DAOPHOT applications are now available -- (Version 1.2)
\end{terminalv}

Subsequent use of \texttt{daophot} will place you inside the DAOPHOT executable image,
which you can leave with the command:

\begin{terminalv}
Command: EXIT
\end{terminalv}

Inside the executable image you can identify the data to be analysed by DAOPHOT
using the command:

\begin{terminalv}
Command: ATTACH <filename>
\end{terminalv}

The file has to be in the NDF format of the Starlink Standard Data Structures
\xref{(SUN/33)}{sun33}{}. The default file extension is \texttt{.sdf}.

At present DAOPHOT can only handle NDF files that have the DATA\_ARRAY structure at
the top level. If the program fails to find the data structure in the container
file, then the message:

\begin{terminalv}
Failed to attach <filename>
\end{terminalv}

will be displayed. If you do not expect this message, then this is the moment to
seek help from your system manager. If you do not get this message, then you can
carry on with DAOPHOT.

The complete set of DAOPHOT commands can be found in the user's manual (MUD/9).

There are three options files which are used by DAOPHOT at various stages of the
reduction. There are examples of these in the directory specified by the
DAOPHOT\_DIR environmental variable. The file \texttt{daophot.opt} is used when the
program starts up. Parameter definitions can be added or removed from the options
file as required. An option is specified using a two letter mnemonic followed by
its value. If a copy of this file is not present in the default directory, or the
parameters have invalid values, then the program will prompt for those parameters
that it cannot continue without, namely the read-out noise and the gain. This is
signified by the message:

\begin{terminalv}
Value unacceptable --- please re-enter
\end{terminalv}

followed by the parameter prompt. The user has to give values for the read-out
noise and gain before the program will continue. Both parameters are used at
various stages of processing and the user guide keeps stressing that it is highly
advisable to have the most accurate values as possible for these parameters.
Although the program will execute with guesstimates for these, the results should
be treated with caution. The sections on the \textbf{PEAK} and \textbf{NSTAR} commands in
the user guide suggest a method of checking the validity of the values given.

The file \texttt{photo.opt} contains the list of aperture radii used by DAOPHOT when
performing aperture photometry.

An alternative to the iterative profile fitting routine NSTAR in DAOPHOT has been
written by Peter Stetson and is called ALLSTAR. It is described in the user manual
and can be executed by typing:

\begin{terminalv}
% allstar
\end{terminalv}

The file \texttt{allstar.opt} contains parameters for this program.

\section{\xlabel{DAOGREY}DAOGREY -- Display an image}
\label{sec:daogrey}

DAOPHOT does not directly use an image display, instead it produces image files of
intermediate results which can be displayed on any available display system. It is
often desirable to indicate the positions of stars on top of the displayed image.
To achieve this some special routines have been created.

DAOGREY is the first of these routines and this will display the image data on a
suitable device.
It is executed by typing:

\begin{terminalv}
% daogrey
\end{terminalv}

The following parameters are requested:

\begin{description}
\item[\mbox{}]\mbox{}
\begin{description}
\item[IN -- NDF containing input image]
The name of the file containing the image data.
\item[XSTART -- first index of subarray]
\item[XEND  -- second index of subarray]
\item[YSTART -- first index of subarray]
\item[YEND  -- second index of subarray]

These specify the limits in pixel coordinates of the area of the data array to
be plotted.

The choice of values allows subsections of the data to be viewed.

\item[LOW -- lowest data value to plot]
\item[HIGH -- highest data value to plot]

These specify the data values corresponding to black and white on the display.
The grey levels in between are set automatically.

\item[DEVICE -- Display device]

The name of the display device to be used.
The image is plotted in the centre of the display, best filling the display
area.

\end{description}
\end{description}

\section{\xlabel{DAOPLOT}DAOPLOT -- Overlay object positions on an image}
\label{sec:daoplot}

DAOPLOT will take a file produced by DAOPHOT and overlay the positions of objects
in the file on top of the grey image produced by DAOGREY. The positions are
indicated by a green cross. The file can be any of the types \texttt{.coo}, \texttt{.ap},
\texttt{.grp} or \texttt{.nst} produced by DAOPHOT. If the file is not suitable then
error messages will be given and the routine will end. If the filename was entered
wrongly, then just run the task again.

To run DAOPLOT simply type:

\begin{terminalv}
% daoplot
\end{terminalv}

The following parameter is requested:

\begin{description}
\item[\mbox{}]\mbox{}
\begin{description}
\item[PFILE -- File with object positions]
The name of the file containing the positions of the objects to be indicated.
The filename and extension should be given.
\end{description}
\end{description}

\section{\xlabel{DAOCURS}DAOCURS -- Overlay a cursor on an image}
\label{sec:daocurs}

DAOCURS will display a cursor over the grey image produced by DAOGREY so that
positions can be read off the screen. The cursor is positioned with the mouse and
the position recorded by pressing any of the keys, except the break key on the
mouse. The break key ends the cursor session. Positions are written to the terminal
as well as a file.

To run DAOCURS type:
\begin{terminalv}
% daocurs
\end{terminalv}

The following parameter is requested:

\begin{description}
\item[\mbox{}]\mbox{}
\begin{description}
\item[RFILE -- File for results]
The name of the file to receive the positions selected by the cursor operation.
\end{description}
\end{description}

\section{\xlabel{DATABASE}Database}
\label{sec:database}

DAOGREY, DAOPLOT and DAOCURS have been specially tailored round each other. DAOPLOT
and DAOCURS will not at present work on an image displayed with any routine other
than DAOGREY. DAOGREY uses the AGI database system (\xref{SUN/48}{sun48}{}) to
record the position of the image on the screen. DAOPLOT and DAOCURS use the
database file to set up the screen coordinates. When an entire session has been
completed the database file can be deleted.

\end{document}




