\documentclass[11pt,nolof]{starlink}

% -----------------------------------------------------------------------------
% ? Document identification
\stardoccategory    {Starlink User Note}
\stardocinitials    {SUN}
\stardocsource      {sun\stardocnumber}
\stardocnumber      {111.2}
\stardocauthors     {D L Terrett\\P M Allan}
\stardocdate        {1 February 1993}
\stardoctitle       {SPT --- Software Porting Tools}
\stardocabstract  {
SPT is an inhomogeneous set of
software tools designed to assist in the
conversion of code from one operating system to another (principally from
VMS to UNIX and vice-versa, and between different flavours of UNIX).
}
% ? End of document identification

% -----------------------------------------------------------------------------
% ? Document specific \providecommand or \newenvironment commands.
% ? End of document specific commands
% -----------------------------------------------------------------------------
%  Title Page.
%  ===========
\begin{document}
\scfrontmatter

\section{Introduction\xlabel{introduction}}

SPT is an inhomogeneous set of
software tools designed to assist in the
conversion of code from one operating system to another (principally from
VMS to UNIX and vice-versa, and between different flavours of UNIX).

The tools are:
\begin{description}
\item[FORCONV] Converts file name specifications in Fortran INCLUDE statements
and ``escapes'' backslash characters. The output file on VMS is also in
Stream\_lf format so can be read on a UNIX system using NFS.
\item[SOFTLINK] Manages the creation and removal of soft links on Unix systems.
These soft links will generally point to Fortran INCLUDE files, thereby
providing a machine-independent way of writing INCLUDE statements.
\end{description}

\section{Running the Tools\xlabel{running_the_tools}}

\subsection{On UNIX}

The programs are stored in \texttt{/star/bin} so all that is required is
that \texttt{/star/bin} is in your PATH. This is normally done when you log in.
\subsection{On VMS}
The tools must be defined as foreign DCL commands with the command:
\begin{terminalv}
$ SPT
\end{terminalv}
Note that softlink is only available on Unix. Its effect is achieved on VMS by
the use of logical names.

\section{FORCONV\xlabel{forconv}}

The principal job of FORCONV is to convert the file name specification
in Fortran include statements but it will also do some other routine tasks
that are necessary when converting a program from VMS to UNIX.
\begin{itemize}
\item Replace all backslash characters with two backslashes (UNIX
compilers interpret a backslash as an ``escape'' character by default
although on some systems there is a compiler switch to alter this
behaviour).

\item Generate a file containing a ``make'' dependency line for each
include statement processed. This file can be used as a starting point for a
make file.

\item Insert a statement to include \texttt{dat\_par} and \texttt{par\_par} after
every occurrence of the include file \texttt{sae\_par}. This compensates for an
incompatibility  between \texttt{sae\_par} on VMS and on UNIX.

\end{itemize}

The substitution of include file names is driven by a ``substitution
file'' which lists the VMS file specifications and their equivalent UNIX
file names, one pair per line. For example the file might contain:
\begin{terminalv}
PARS.FOR pars
INCLIB(COMMON) commons
\end{terminalv}
and an example of the source code to be converted might be:
\begin{terminalv}
      INCLUDE 'PARS.FOR'
      INCLUDE 'INCLIB(COMMON)'
\end{terminalv}
which, after processing, would be changed to:
\begin{terminalv}
      INCLUDE 'pars'
      INCLUDE 'commons'
\end{terminalv}
The case of the VMS file specification is ignored when trying to find a
match and spaces are ignored in the file specification in the INCLUDE
statement.

If a match for an include file specification is not found then the name
is folded to lower case and a prefix prepended; the default prefix is
\texttt{/star/include/} but this can be changed with the \texttt{-p} command line
option. So in the above example:
\begin{terminalv}
INCLUDE 'FRED'
\end{terminalv}
would be converted to:
\begin{terminalv}
INCLUDE '/star/include/fred'
\end{terminalv}
As a result of this,
when converting programs that use Starlink libraries, the substitution file
need only specify any ``program-specific'' include files; references to
Starlink include files will all be converted automatically. If the
program does not use Starlink libraries, then the prefix can be suppressed
entirely by specifying the \texttt{-p} option with an empty string.

The program will not recognise INCLUDE statements that are continued
on to a new line.

The command:
\begin{terminalv}
forconv
\end{terminalv}
Will read from the standard input stream and write to the standard output
stream, and will use \texttt{include.sub} as the name of the substitution file.
The program's behaviour can be modified with the following UNIX-style
command options:
\begin{description}

\item[\texttt{-i}\emph{filename}] Read from \emph{filename} instead of the
standard input stream. Note that there is no space between the \texttt{i}
and the file name in this or any other command options.

\item[\texttt{-o}\emph{filename}] Write to \emph{filename} instead of the
standard output stream.

\item[\texttt{-s}\emph{filename}] Use \emph{filename} as the substitution file.

\item[\texttt{-m}\emph{filename}] Generate a skeleton make file. If \emph{filename} is omitted a file called \texttt{makefile} will be written.

\item[\texttt{-b}\emph{name}] Use \emph{name} as the target in the dependency
rules written to the make file. In the absence of \texttt{-b} the name \texttt{module} is used.

\item[\texttt{-p}\emph{prefix}] Use \emph{prefix} as the string to be inserted
before all file specifications that did not match a file specification
in the substitution file. The default prefix is \texttt{/star/include/};
\texttt{-p} on its own specifies no prefix.

\item[\texttt{-h}] Add
\begin{terminalv}
INCLUDE '/star/include/dat_par'
\end{terminalv}
after every occurrence of \texttt{'/star/include/sae\_par'}. This
compensates for an incompatibility between the standard Starlink include
file \texttt{sae\_par} on VMS and on
UNIX and is required if the program being converted uses any \texttt{DAT\_\_}
constants.

\item[\texttt{-j}] Add
\begin{terminalv}
INCLUDE '/star/include/par_par'
\end{terminalv}
after every occurrence of \texttt{'/star/include/sae\_par'}. This
compensates for an incompatibility between the standard Starlink include
file \texttt{sae\_par} on VMS and on
UNIX and is required if the program being converted uses any \texttt{PAR\_\_}
constants.

\item[\texttt{-q}] Suppress all informational messages. By default the
program writes a message to the standard error stream each time it
process an INCLUDE statement.

\end{description}

\section{Softlink\xlabel{softlink}}

Softlink is a tool to provide easy, portable use of standard Starlink include
files on Unix systems. It is complementary to FORCONV and will be of general
use when developing Fortran programs on Unix as well as in porting code to
Unix.

\subsection{Logical names and soft links}

Starlink Fortran programs frequently make use of INCLUDE files to define
symbolic constants or common blocks. When the INCLUDE file is one supplied by
Starlink, a logical name is defined on VMS to point to the file. For example,
the logical name \texttt{SAE\_PAR} points to the file \texttt{LIB\_COMMON\_DIR:SAE\_PAR.FOR}. This means that the file can be included in a
Fortran program with the statement:

\begin{terminalv}
INCLUDE 'SAE_PAR'
\end{terminalv}

There is no need to refer to the location of the file, just to its name. This
makes the program more portable than referring to the explicit file.

Unix does not provide logical names, but you can achieve a similar effect by
the use of soft links. A soft link looks like a file in a directory, but is
actually just a pointer to another file. To use a soft link to refer to the
INCLUDE file in the above example, you would type:

\begin{terminalv}
% ln -s /star/include/sae_par SAE_PAR
\end{terminalv}

This command should be executed in the same directory as the source of the
Fortran program. If you are working on Fortran programs in different
directories, you will need the appropriate soft links in each directory. This
is unlike VMS, where logical names exist independently of the file system. Note
that the name of the file is in lower case, but the name in the include
statement is in upper case. This distinction is important as Unix is case
sensitive, unlike VMS.

\subsection{Overview of softlink}

Softlink is a Unix shell script that can be used to create and remove soft
links to files. The purpose of writing it was to automate the setting up of
soft links that point to standard Starlink Fortran include files.  Rather than
having programs contain lines like:

\begin{terminalv}
INCLUDE '/star/include/sae_par'
\end{terminalv}

they will contain lines like:

\begin{terminalv}
INCLUDE 'SAE_PAR'
\end{terminalv}

and a soft link to the file \texttt{/star/include/sae\_par} will be created in the
directory containing the source code. This makes the source code more portable;
it may well run on Unix and VMS systems with no changes.

Before compiling a program on a Unix system, the soft links to the include
files should be created. This need only be done once as the soft links are
permanent. For the above example, you would type:

\begin{terminalv}
% softlink /star/include/sae_par upper
\end{terminalv}

Note that the name of the file being pointed to must match the name of the file
in the Fortran INCLUDE statement for softlink to be effective. If the names are
different, then you should either set up the soft link by hand, or use FORCONV.

\subsection{Details}

Softlink accepts three arguments. Only the first one is mandatory. This
specifies the full path name of the file that the soft link will point to.
An example is \texttt{/star/include/sae\_par}.

The second (optional) argument specifies whether the name of the soft link is
in upper or lower case. If it is omitted (or given as the null string), then
the name of the soft link matches exactly the name of the file that it points
to. The value of the argument can be one of `\texttt{lower}', `\texttt{upper}', `\texttt{both}', `\texttt{all}' or `\texttt{remove}'. A value of `\texttt{lower}' specifies that
the name of the soft link will be in lower case and a value of `\texttt{upper}'
specifies that the name of the soft link will be in upper case. A value of
`\texttt{both}' specifies that two soft links are to be created, one in lower case
and one in upper case. A value of `\texttt{all}' specifies that links will be
created in lower case, upper case and with the same name as the file that it is
pointing to if that is not purely lower case or upper case. If the argument is
`\texttt{remove}', then all the soft links are removed. The script checks to see
that it is only removing soft links and not files of the same name.

The third argument is normally absent. If it is present, then the script does
not execute any commands to create soft links, but writes those command to the
file named by the third argument. For example, the command

\begin{terminalv}
% softlink /star/include/par_par all fred
\end{terminalv}

will create a file called \texttt{fred} containing the lines:

\begin{terminalv}
ln -s /star/include/par_par par_par
ln -s /star/include/par_par PAR_PAR
\end{terminalv}

If the output file already exists, then the commands are appended to the file
so that a set of link commands can be built up. If the name of the output file
is \texttt{mfile}, then the commands are output in a form suitable for inclusion
in a make file. In particular, the first character of each line is a tab
character.

\subsection{Defining Standard Soft Links}

When writing programs that call Starlink libraries, you will find yourself
setting up the same soft links over and over again. To ease this process, shell
scripts are provided to define the soft links for all of the Starlink
subroutine libraries. The scripts are called \textit{lib}{\tt\_dev}, e.g.\ to set
up the soft links needed when writing a program that calls FIO, just type:

\begin{terminalv}
% fio_dev
\end{terminalv}

The soft links can be removed by typing:

\begin{terminalv}
% fio_dev remove
\end{terminalv}

The scripts provide fewer functions than the softlink script. All links are
created in upper case and the only option is \texttt{remove}, which does not check
to see that the names referred to are soft link rather than files. To offset
these restrictions, the \textit{lib}{\tt\_dev} scripts run faster than they would
if they provided all the features of softlink. The list of scripts is as
follows:

\begin{tabular}{ll}
\underline{Package} & \underline{Script} \\
& \\
AGI & agi\_dev \\
ARY & ary\_dev \\
CHR & chr\_dev \\
EMS & ems\_dev \\
ERR & err\_dev \\
FIO & fio\_dev \\
GKS & gks\_dev \\
GNS & gns\_dev \\
GRP & grp\_dev \\
GWM & gwm\_dev \\
HDS & hds\_dev \\
IDI & idi\_dev \\
NDF & ndf\_dev \\
PAR & par\_dev \\
PRIMDAT & prm\_dev \\
PSX & psx\_dev \\
REF & ref\_dev \\
SGS & sgs\_dev \\
TRANSFORM & trn\_dev
\end{tabular}

There are also scripts \texttt{star\_dev}, which creates the soft links to \texttt{SAE\_PAR}, and \texttt{adam\_dev}, which creates the soft links needed when
writing programs that call the internals of ADAM.

\end{document}
