\documentclass[twoside,11pt,nolof,noabs]{starlink}

% ? Specify used packages
% ? End of specify used packages

% -----------------------------------------------------------------------------
% ? Document identification
% Fixed part
\stardoccategory    {Starlink User Note}
\stardocinitials    {SUN}
\stardocsource      {sun\stardocnumber}

% Variable part - replace [xxx] as appropriate.
\stardocnumber      {129.5}
\stardocauthors   {M.\,J.\,Bly, P.\,M.\, Allan, \\
                                B.\,K.\,McIlwrath}
\stardocdate        {3 February 2000}
\stardoctitle       {XDISPLAY \\[1.5ex] Setting remote X windows}
\stardocversion     {Version 2.1}
% ? End of document identification
% -----------------------------------------------------------------------------

% -----------------------------------------------------------------------------
% ? Document specific \providecommand or \newenvironment commands.
% ? End of document specific commands
% -----------------------------------------------------------------------------
%  Title Page.
%  ===========
\begin{document}
\scfrontmatter

% ? Main text

\section{\label{intro}\xlabel{intro}Introduction}

This note describes how to use the XDISPLAY utility on Starlink systems
that are running X~windows.  It makes no attempt to describe the
X~windows system.  It assumes that you have some familiarity with using
X~windows, although not to any great depth.

\section{\label{problem}\xlabel{problem}The problem}

If you are using X~windows purely on a workstation, then there is no
problem.  Graphical output created by X will appear automatically on
your screen.  The problem arises if you are logged on to an X~client
other than the machine that you are sat in front of (the X~server).

Suppose that you are sat in front of a SPARCstation called \texttt{rlssp0}
, but that you are remotely logged into \texttt{rlsaxps}, a Compaq
Alpha.  To get X output from a program run on \texttt{rlsaxps} to be
displayed on the screen of \texttt{rlssp0}, you need to type

\begin{terminalv}
% setenv DISPLAY rlssp0.bnsc.rl.ac.uk:0
\end{terminalv}

This is easy to forget and too much to type every time you do a remote
login.

\section{\label{ssh}\xlabel{ssh} The preferred solution}

Starlink now recommends that users use the \textit{Secure Shell} login
command (\texttt{ssh} or its alias \texttt{slogin}) between machines which
support it (all the Starlink Unix computers and even Microsoft Windows
systems using additional software -- see your Site Manager). This uses
sophisticated means to authenticate that the remote machine is the
correct one and encrypts all data traffic over the network.

An additional benefit is that \texttt{ssh} on Unix automatically creates
an encrypted-channel X11 device on the remote computer, points the
\texttt{DISPLAY} environment variable towards this and sets up local X11
authentication to allow remote access from this machine. The setting of
\texttt{DISPLAY} must be left under the control of \texttt{ssh} to use this
facility and it is important to check login files (such as \texttt{.login})
to ensure that these do not alter this environment variable.

\section{\label{solution}\xlabel{solution}The Xdisplay alternative}

If \texttt{ssh} is \textbf{NOT} being used the \texttt{xdisplay} procedure
is available on all Starlink machine to wrap up what you need to type
to get X output from the machine that you are logged into back to your
X~server.

To get X output drawn on your X~server from a remote client, just type
\texttt{ xdisplay} before running the program.  If you type \texttt{ xdisplay}
when you are not logged into a remote system, then nothing will happen
and the output will still appear on your screen.

What XDISPLAY does is to see where you logged in from and executes an
appropriate command to point the X output back to you.

\section{\label{wrong}\xlabel{wrong}What can go wrong?}

There are a few things that might stop XDISPLAY from working as you intended.
These are problems with security, multi-hop logins and inappropriate use.

\subsection{\label{security}\xlabel{security}Authorisation}

If you are logged into the Compaq Alpha \texttt{ rlsaxps} from the
SUN Sparc \texttt{ rlssp0}, and you type \texttt{ xdisplay}, then it is possible
that you will get a message such as:

\begin{terminalv}
Xlib:  Client is not authorized to access server
\end{terminalv}

To fix this, you need to make sure that the client is authorized to send X
output to the server.  If you do not know how to do this, consult your site
manager.

\subsection{\label{multihop}\xlabel{multihop}Multiple Hops}

A problem that is not so easily overcome, but is probably less likely
to occur, is that of multi-hop logins.

If you are using X-terminal \texttt{ xacc} with an X session on Compaq Alpha
\texttt{ rlsaxps},  and you log into Compaq Alpha \texttt{ rlsaxp2}, and then type
\texttt{ xdisplay} on \texttt{ rlsaxp2}, XDISPLAY thinks you are on \texttt{ rlsaxps}
and will point the display back to \texttt{ rlsaxps}.  This will fail because
\texttt{ rlsaxps} is a server machine and does not have an X console.

A somewhat more embarrassing possibility is that you have logged in from
 one workstation (A) to a second (B), and thence to a third machine
(C). If you then type \texttt{ xdisplay} on C it will successfully point
the X output back to B if the security setting allows it. The trouble
is that you are sat in front of A wondering where the window has gone
and the user sat in front of B is wondering why a random window has
suddenly appeared, and worse, possibly trashed what was displayed on
a pre-existing graphics window!

Making XDISPLAY automatically cope with multi-hop logins verges on the
impossible, but there is a simple manual override available. If you
type

\begin{terminalv}
xdisplay <nodename>
\end{terminalv}

where \texttt{ <nodename>} is the name of the computer or X-terminal that
you want the display to appear on, then the display will appear on that
X-server, regardless of the route that you used to log in.

\subsection{\label{inappropriate}\xlabel{inappropriate}Inappropriate Use}

Note that you can only use the \texttt{ xdisplay} command interactively.  You
cannot put it in a shell script as \texttt{ xdisplay} is actually an alias
that uses the shell's command line recall features.

If you need to use XDISPLAY in a file there are two possibilities.  If
you want to have XDISPLAY work out where you logged in from, put:

\begin{terminalv}
% source /star/etc/xdisplay
\end{terminalv}

in your file. This will generally be your \texttt{ .login} file. If you want to
explicitly set the node that output will appear on, then simply set the
environment variable DISPLAY, \emph{e.g.}:

\begin{terminalv}
% setenv DISPLAY adam4.bnsc.rl.ac.uk:0
\end{terminalv}

which is all that XDISPLAY does anyway.

If you give XDISPLAY the explicit node name, it is also possible to
specify the transport mechanism to use.  On Unix systems, you can
select the transport to be used by the case of the node name.  If it is
purely in upper case, then DECnet will be used.  Anything else will
cause TCP/IP to be used, which is normally what is required.

Finally, if you try to use XDISPLAY in an inappropriate way then you
probably will not get what you expected.  For example, if you log into a
computer from a VT type terminal connected to a TCP/IP terminal server,
then XDISPLAY will quite happily point your X output to the terminal
server. It is only when you come to run an X application that you will
see the problem.
% ? End of main text
\end{document}
