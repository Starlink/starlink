\documentclass[twoside,11pt]{starlink}
%
% CATREMOTE - a Tool for Querying Remote Catalogues.
%
% Copyright 2001  Starlink, CCLRC.
%
% A.C. Davenhall (Edinburgh), 3/10/00.
%

% ? Specify used packages
% ? End of specify used packages

%------------------------------------------------------------------------------

% ? Document identification
% Fixed part
\stardoccategory    {Starlink System Note}
\stardocinitials    {SSN}
\stardocsource      {ssn\stardocnumber}
\stardoccopyright
{Copyright \copyright\ 2001 Council for the Central Laboratory of the Research Councils}

% Variable part - replace [xxx] as appropriate.
\stardocnumber      {76.2}
\stardocauthors     {A.C.~Davenhall}
\stardocdate        {24 May 2001}
\stardoctitle     {CATREMOTE --- a Tool for Querying Remote Catalogues}
\stardocabstract
{\texttt{catremote} is a tool for querying remote astronomical catalogues,
databases and archives via the Internet.  It allows remote catalogues
to be queried and the resulting table to be saved as a local file.  It also
provides a number of related auxiliary functions.  \texttt{catremote} can
be used interactively.  However, it is more likely to be incorporated in
a script or GUI.  This document describes the use of \texttt{catremote} in
sufficient detail to allow a programmer to use it in a script or GUI that
he is developing.

\begin{latexonly}
\vspace{5mm}
\end{latexonly}

\begin{center}
\textbf{Who Should Read this Document?}
\end{center}

This document is aimed primarily at programmers who wish to incorporate
\texttt{catremote} in a script or GUI that they are developing.  It may also
be useful to users who wish to simply run \texttt{catremote} interactively,
particularly if they wish to make use of all its facilities.  A simple
introduction to \texttt{catremote}, which is more suitable for new users,
is included in \xref{SUN/190}{sun190}{}.}
% ? End of document identification
% -----------------------------------------------------------------------------
% ? Document specific \providecommand or \newenvironment commands.

% Define commands for displaying angles as sexagesimal hours and minutes
% or degrees and minutes.

\providecommand{\tmin}   {\mbox{$^{\rm m}\!\!.$}}
\providecommand{\hm}[3] {$#1^{\rm h}\,#2\tmin#3$}
\providecommand{\dm}[2] {$#1^{\circ}\,#2\raisebox{-0.5ex}{$^{'}$}$}
\providecommand{\arcmin} {\raisebox{-0.5ex}{$^{'}$} }

\providecommand{\arcsec} {$\hspace{-0.05em}\raisebox{-0.5ex}
                     {$^{'\hspace{-0.1em}'}$}
                     \hspace{-0.7em}.\hspace{-0.05em}$}
\providecommand{\tsec}   {\mbox{$^{\rm s}\!\!.$}}
\providecommand{\hms}[4] {$#1^{\rm h}\,#2^{\rm m}\,#3\tsec#4$}
\providecommand{\dms}[4] {$#1^{\circ}\,#2\raisebox{-0.5ex}{$^{'}$}\,#3\arcsec#4$}

\providecommand{\catremote}{\htmlref{\texttt{catremote}}{CATREMOTE}}
\providecommand{\geturl}{\htmlref{\texttt{geturl}}{GETURL}}

% ? End of document specific commands

% -----------------------------------------------------------------------------
%  Title Page.
%  ===========
\begin{document}
\scfrontmatter

\section*{Modes of operation}

\begin{center}
\begin{tabular}{rl}
Mode          &  Description \\ \hline
\texttt{list}    & list the databases currently available \\
\texttt{details} & show details of a named database \\
\texttt{query}   & submit a query to a remote database and retrieve the results \\
\texttt{name}    & resolve an object name into coordinates \\
\texttt{help}    & list the modes available \\
\end{tabular}
\end{center}

\section*{Command-line arguments for the various modes}

\begin{center}
\begin{tabular}{ll}
\texttt{catremote list}    & \textit{server-type} \\
\texttt{catremote details} & \textit{db-name} \\
\texttt{catremote query}   & \textit{db-name $\alpha$ $\delta$ radius
    additional-condition} \\
\texttt{catremote name}    & \textit{db-name object-name} \\
\texttt{catremote help}    & \\
\end{tabular}
\end{center}

The individual arguments are as follows.

\begin{description}

  \item[\textit{server-type}\/] Type of server to be listed; one of:
   \texttt{all}, \texttt{catalog}, \texttt{archive}, \texttt{namesvr}, \texttt{imagesvr}
   or \texttt{local} (optional).

  \item[\textit{db-name}\/] Name of the database to be queried.

  \item[$\alpha$\/] Central Right Ascension of the query.  The value
   should be for equinox J2000 and given in sexagesimal hours with
   a colon (`\texttt{:}') as the separator.

  \item[$\delta$\/] Central Declination of the query.  The value
   should be for equinox J2000 and given in sexagesimal degrees with
   a colon (`\texttt{:}') as the separator.  Southern Declinations are
   negative.

  \item[\textit{radius}\/] Radius of the query in minutes of arc.

  \item[\textit{additional-condition}\/] Any additional condition applied
   to the query (optional).

  \item[\textit{object-name}\/] The name of an astronmical objects whose
   coordinates are to be found.  Give the name without embedded spaces
   and in either case (upper or lower).

\end{description}

\section*{Environment variables}

\begin{center}
\begin{tabular}{rcl}
Variable                & Default   &  Description \\ \hline
\texttt{CATREM\_URLREADER} &      & Program to submit query \\
\texttt{CATREM\_CONFIG}    &      & URL of configuration file \\
\texttt{CATREM\_MAXOBJ}    & 1000 & Maximum number of objects in results
table \\
\texttt{CATREM\_ECHOURL}   & no   & Echo URL sent to remote server? \\
\end{tabular}
\end{center}


\newpage
\section*{Accessing this document}

A hypertext version of this document is available.  To access it on
Starlink systems type:

\begin{terminalv}
% showme  ssn76
\end{terminalv}

On non-Starlink systems access URL:

\begin{quote}
\url{http://www.starlink.ac.uk/docs/ssn76.htx/ssn76.html}
\end{quote}

\section*{Obtaining assistance}

\catremote\ is part of the CURSA package for manipulating astronomical
catalogues and tables.  Reports of bugs in CURSA, including \catremote, should always be sent to username \texttt{starlink@jiscmail.ac.uk}.
However, you are welcome to contact me for advice and assistance; details
of how to do so are given below.

Postal address: \\
\begin{tabular}{l}
A.C.~Davenhall.  Institute for Astronomy, Royal Observatory, Blackford Hill, \\
Edinburgh, EH9 3HJ, United Kingdom.  \\
\end{tabular}

% \vspace{4mm}

Electronic mail: \texttt{acd@roe.ac.uk}

% \vspace{4mm}

Fax: \\
\begin{tabular}{lr}
from within the United Kingdom: &    0131-668-8416 \\
from overseas:                  & +44-131-668-8416 \\
\end{tabular}



\newpage
\section*{Revision history}

\begin{enumerate}

  \item 24th May 2001: Version 1. Original version (ACD).

\end{enumerate}

\cleardoublepage


\section{\xlabel{INTRO}\label{INTRO}Introduction}

\catremote\ is a tool for querying remote astronomical catalogues,
databases and archives via the Internet.  It allows remote catalogues
to be queried and the resulting table to be saved as a local file.  It
also provides a number of related auxiliary functions.

\catremote\ can be used interactively.  However, it is more likely
to be incorporated in a script or GUI.  This document describes the use
of \catremote\ in sufficient detail to allow a programmer to use it
in a script or GUI which he is developing.  Additionally the document may
be useful to users who simply wish to run \catremote\ interactively,
particularly if they wish to make use of all its facilities.  \textit{There
is a simple introduction to the use of \catremote\ in
\xref{SUN/190}{sun190}{}\cite{SUN190} which may be more suitable for new
users than the present document.}

\catremote\ is part of the astronomical catalogue and table
manipulation package CURSA (see \xref{SUN/190}{sun190}{}\cite{SUN190}).
It is written in Perl and uses the `ACL format' (discussed briefly below)
to access remote catalogues and databases.   A variety of remote
catalogues and databases can be accessed using this format; in the first
instance see SUN/190 for details of what is available.


\section{\xlabel{ACL}\label{ACL}The ACL Format}

\catremote\ uses the ACL format developed by Allan~Brighton and
colleagues at ESO to access remote catalogues and databases.  The ACL
format is fully documented in \xref{SSN/75}{ssn75}{}\cite{SSN75}.  The
following brief description merely gives sufficient details to allow the
operation of \catremote\ to be understood.

The ACL format is implemented using the Hyper-Text Transfer Protocol (HTTP)
developed as part of the World Wide Web.  \catremote\ querying a
remote database is an example of a `client-server architecture', with
\catremote\ acting as the client and the remote database, or more
exactly the program which accesses it, as the server.  In the ACL format
a so-called `configuration file' mediates the interaction between client
and server.  This configuration file comprises a list of one or more
databases, giving details for each.  Usually each `database' will be a
simple astronomical catalogue.  However, other alternatives are possible:
archives, name servers, \emph{etc}\/ (Table~\ref{SERVTYPES}, below, lists
all the possibilities).  Consequently, in this document the
generic term `database' is used to denote each entry.

\catremote\ accesses a given configuration file and the databases
which this file contains are the ones that \catremote\ currently
knows about.

The configuration file lists various details for each database, such as:
the URL to access it, the type of database it is (its so-called
`server type': catalogue, image server, name server \emph{etc}\/), the
type of queries supported, \emph{etc}.  Most of these details are not
germane here.  However, one item which is important is the so-called
`\texttt{short\_name}' or simply `name' of the database.  This quantity
is used to identify the database, for example you would supply it
in response to a prompt from \catremote.  It is a short character
string (without embedded spaces) and conventionally it has the form:

\begin{quote}
\textit{database}\texttt{@}\textit{institution}
\end{quote}

where \textit{database}\, is an abbreviation for the database and \textit{institution}\, a standardised abbreviation for the institution where the
on-line version is located.  By convention \textit{institution}\/ has three
or four characters.  For example, the usual name of the version of the
\htmladdnormallinkfoot{USNO}{http://www.nofs.navy.mil/}
PMM astrometric catalogue maintained by ESO is \texttt{usno@eso}.


\section{\xlabel{SYSREQ}\label{SYSREQ}System Requirements and Getting
Started}

\catremote\ is available for all the versions of Unix supported by
Starlink: Compaq Alpha/Tru64, Sun/Solaris and PC/Linux.  It may well work
on other versions of Unix, but is not supported on them.

\catremote\ is a simple Perl script (see, for example,
\xref{SUN/193}{sun193}{}\cite{SUN193}) and hence Perl must be
available on your system before you can run it.  It was developed using
Perl~5.  On Starlink systems \catremote\ is located in file:

\begin{terminalv}
/star/bin/cursa/catremote
\end{terminalv}

On non-Starlink systems a release of \catremote\ includes a copy
of the script and this document.  On such systems you may need to edit
the first line of \catremote\ to pick up Perl from wherever on your
system it is located.

The purpose of \catremote\ is to query remote catalogues via the
Internet.  The ACL format that it uses for this purpose itself uses the
World Wide Web Hyper-Text Transfer Protocol (HTTP).  Consequently, \catremote\ will only run on computers connected to the Internet and
configured to run Web browsers.

Though \catremote\ assembles the queries and writes the resulting
tables of selected objects, it runs a separate utility program to submit
the query to the server and retrieve the results.  A number of alternative
utility programs can be used for this purpose, including the C program
\geturl\ which is supplied with \catremote\ and is the default on
Starlink systems.  The various alternatives are described in
Section~\ref{WEBPROG}.  The Unix environment variable \texttt{CATREM\_URLREADER}
(below) specifies which utility program is to be used and must be set
before \catremote\ is invoked.


\section{\xlabel{ENVIR}\label{ENVIR}Environment Variables}

\catremote\ takes some input from Unix shell environment variables
and these variables can be used to control its behaviour.  Some of the
variables are optional, but others are mandatory and must be set before
\catremote\ is invoked.  The variables used are listed in
Table~\ref{ENVARS} and described briefly below.

\begin{table}[htbp]

\begin{center}
\begin{tabular}{rcl}
Variable                & Default   &  Description \\ \hline
\texttt{CATREM\_URLREADER} &      & Program to submit query \\
\texttt{CATREM\_CONFIG}    &      & URL of configuration file \\
\texttt{CATREM\_MAXOBJ}    & 1000 & Maximum number of objects in results table \\
\texttt{CATREM\_ECHOURL}   & no   & Echo URL sent to remote server? \\
\end{tabular}
\end{center}

\begin{quote}
\caption[Environment variables used by catremote]{Environment
variables used by \texttt{catremote}.  The variables for which no default
is given are mandatory
\label{ENVARS} }
\end{quote}

\end{table}

\begin{description}

  \item[\texttt{CATREM\_URLREADER}] \catremote\ uses a separate program
   to submit the URL constituting a query to the server and return the
   table of results.  The options available are described in
   Section~\ref{WEBPROG}.

  \item[\texttt{CATREM\_CONFIG}] specifies the configuration file to be used.
   It should be set to either the URL (for a remote file) or the local
   file name, including a directory specification (for a local file).
   Configuration files are described in \xref{SSN/75}{ssn75}{}\cite{SSN75}.
   The default configuration file used by CURSA is at URL:

  \begin{quote}
\url{http://dev.starlink.ac.uk/\~pwd/catremote/cursa.cfg}
  \end{quote}

  \item[\texttt{CATREM\_MAXOBJ}] is the maximum number of objects which the
   returned table is allowed to contain.

  \item[\texttt{CATREM\_ECHOURL}] Controls whether the URL representing the
   query submitted to the remote server is also displayed to the user.
   The default is `\texttt{no}'; to see the URL set \texttt{CATREM\_ECHOURL} to
   `\texttt{yes}'.  Seeing the URL is potentially useful when debugging
   configuration files and servers but is not usually required for
   normal operation.

\end{description}

These environment variables are set up automatically when CURSA is started.
However, if \catremote\ is being used outside CURSA they need to be
set up manually.
Figure~\ref{SETENVAR} shows an example script for this purpose.  On
Starlink systems it is available as file:

\begin{terminalv}
/star/share/cursa/cursacatremote-setup.csh
\end{terminalv}

\begin{figure}[htbp]
\latex{\begin{small}}
\begin{terminalv}
#! /bin/csh -f
#+
# catremote-setup.csh
#
# Example shell script to setup the environment variables used by
# catremote.  Note that this file should be sourced.
#
# Author:
#   A C Davenhall (Edinburgh)
# History:
#   23/5/01 (ACD): Original version.
#-

#  Define the utility program to be used to submit the query.  The options
#  are:
#    "/star/bin/cursa/geturl" -- geturl utility,
#    "lynx -source"           -- lynx command line browser,
#    "java  UrlReader"        -- Java utility.
#    "wget -q -O -"           -- wget utility (probably RedHat Linux only).

setenv  CATREM_URLREADER "/home/acd/starbase/cursa/catremote/geturl/geturl"

#  Specify the configuration file to be used.  The URL given here is the
#  the CURSA default.

setenv  CATREM_CONFIG  http://dev.starlink.ac.uk/~pwd/catremote/cursa.cfg

#  Define the maximum number of objects which may be included in the
#  returned table.

setenv  CATREM_MAXOBJ  200

#  Specify whether the URL constituting the query is echoed to the
#  command line.  The options re:
#    no   -  do not echo the URL (default),
#    yes  -  echo the URL.

setenv  CATREM_ECHOURL no

#  Set the Java CLASSPATH environment variable to pick up the URLreader
#  (note that CLASSPATH has to be set rather than setting the corresponding
#  command line option because the latter does not work on alphas).

setenv CLASSPATH /home/acd/starbase/cursa/catremote:/usr/lib/netscape/java/classes
\end{terminalv}
\latex{\end{small}}

\begin{quote}
\caption{Example shell script to set up \texttt{catremote} environment
variables  \label{SETENVAR} }
\end{quote}

\end{figure}


\section{\xlabel{FUNC}\label{FUNC}The Functionality of catremote}

The purpose of \catremote\ is to query remote astronomical databases.
It has a number of separate functions to realise this purpose, and each
function corresponds to a mode of the program.  The mode to be used is
specified when \catremote\ is run.  The various modes are listed in
Table~\ref{MODES} and described briefly below.  Running \catremote\ to
invoke the various modes is described in detail in Section~\ref{RUN},
below.

\begin{description}

  \item[\texttt{list}] list all the databases which are currently available.

  \item[\texttt{details}] show details of a named database.

  \item[\texttt{query}] submit a query to a named database and retrieve the
   results.  The basic type of query supported is the `cone search' or
   `circular area search' which returns all the objects found in a given
   circular area of sky.  This area is specified by its central Right
   Ascension and Declination and angular radius.  The objects returned are
   formatted as a catalogue and written to a local file.

  \item[\texttt{name}] submit a name of an astronomical object to a remote
   name-resolver database.  If the name-resolver finds this name in its
   database then the Right Ascension and Declination of the object are
   returned and displayed.

  \item[\texttt{help}] list the modes available.

\end{description}

\begin{table}[htbp]

\begin{center}
\begin{tabular}{rl}
Mode          &  Description \\ \hline
\texttt{list}    & list the databases currently available \\
\texttt{details} & show details of a named database \\
\texttt{query}   & submit a query to a remote database and retrieve the results \\
\texttt{name}    & resolve an object name into coordinates \\
\texttt{help}    & list the modes available \\
\end{tabular}
\end{center}

\caption{The modes of \texttt{catremote}
\label{MODES} }

\end{table}


\section{\xlabel{RUN}\label{RUN}Running catremote}

\catremote\ is invoked by simply typing:

\begin{terminalv}
% catremote
\end{terminalv}

Arguments may be supplied on the command line or prompted for.  Obviously,
if \catremote\ is being invoked from a script the arguments will
usually be supplied on the command line.

The first command-line argument is the mode of operation.  The permitted
values are listed in Table~\ref{MODES}.  The mode can only be specified on
the command line.  If it is omitted then `\texttt{help}' mode is assumed.

The subsequent arguments required depend on the mode chosen and are
summarised in Table~\ref{ARGS}.  Command line arguments are identified
by position.  They may optionally be omitted, starting at the right.
Omitted arguments will usually be prompted for.  Exceptions are the
\textit{server-type}\/ in \texttt{list} mode, any \textit{additional-condition}\/
in \texttt{query} mode and the mode itself (see below).  For each mode the
input required and output produced is described below.

\begin{table}[htbp]

\begin{center}
\begin{tabular}{ll}
\texttt{catremote list}    & \textit{server-type} \\
\texttt{catremote details} & \textit{db-name} \\
\texttt{catremote query}   & \textit{db-name $\alpha$ $\delta$ radius
    additional-condition} \\
\texttt{catremote name}    & \textit{db-name object-name} \\
\texttt{catremote help}    & \\
\end{tabular}
\end{center}

\caption{Arguments for the various modes of \texttt{catremote}
\label{ARGS} }

\end{table}

\subsection{\texttt{list} mode}

\begin{description}

  \item[Input] \texttt{list} mode has a single optional argument, the
   \textit{server-type}.  The values permitted are listed in
   Table~\ref{SERVTYPES} (and \xref{SSN/75}{ssn75}{}\cite{SSN75} gives
   more details of the individual types).  \catremote\ lists all the
   databases in the current configuration file which match the given
   \textit{server-type}.  If the argument is omitted then `\texttt{all}' is
   assumed (that is, all the databases in the configuration file are
   listed).

\begin{table}[htbp]

\begin{center}
\begin{tabular}{rl}
Server type     & Description \\ \hline
\texttt{all}       & all types of server \\
\texttt{catalog}   & simple catalogue \\
\texttt{archive}   & archive \\
\texttt{namesvr}   & name server \\
\texttt{imagesvr}  & image server \\
\texttt{local}     & local file \\
\texttt{directory} & link to another configuration file \\
\end{tabular}
\end{center}

\caption{Types of server \label{SERVTYPES} }

\end{table}

  \item[Output] All the databases in the configuration file which match
   the specified server type are listed, one per line.  Typical output
   might look something like:

  \begin{terminalv}
usno@eso  catalog  USNO at ESO
gsc@lei  catalog  Guide Star Catalog at LEDAS
simbad_ns@eso  namesvr  SIMBAD Names
  \end{terminalv}

   For each database, the first item is its name, the second its server
   type and the rest of the line gives a short description.

\end{description}

\subsection{\texttt{details} mode}

\begin{description}

  \item[Input] \texttt{details} mode has a single input argument, \textit{db-name}\/; the name of the database for which the details are to be
   shown (see Section~\ref{ACL}).  This database must be included in the
   current configuration file.

  \item[Output] The following details are shown for the specified
   database, listed one per line:

  \begin{tabular}{ll}
   \texttt{short\_name:}  & name of the database \\
   \texttt{serv\_type:}   & server type of the database (see
                         Table~\ref{SERVTYPES}) \\
   \texttt{long\_name:}   & a short description of the database \\
   \texttt{url:}          & URL of the database server \\
   \texttt{search\_cols:} & columns on which range searches are supported \\
   \texttt{help:}         & URL of help page for the database \\
  \end{tabular}

   See \xref{SSN/75}{ssn75}{}\cite{SSN75} for a description of each item.

\end{description}

\subsection{\texttt{query} mode}

\begin{description}

  \item[Input] \texttt{query} mode has the arguments described below.  Any
   arguments which are omitted will be prompted for, with the exception of
   \textit{additional-condition}.  Thus, if an \textit{additional-condition}\/
   is specified then all the arguments must be included on the command line.

  \begin{description}

    \item[\textit{db-name}\/] Name of the database to be queried (see
     Section~\ref{ACL}).

    \item[$\alpha$\/] Central Right Ascension of the query.  The value
     should be for equinox J2000 and given in sexagesimal hours with
     a colon (`\texttt{:}') as the separator.

    \item[$\delta$\/] Central Declination of the query.  The value
     should be for equinox J2000 and given in sexagesimal degrees with
     a colon (`\texttt{:}') as the separator.  Southern Declinations are
     negative.

    \item[\textit{radius}\/] Radius of the query in minutes of arc.

    \item[\textit{additional-condition}\/] Any additional condition applied
     to the query.  Databases vary in which, if any, additional queries
     they support.  Three forms of \textit{additional-condition}\/ are
     accepted.  The first is:

    \begin{quote}
     \textit{column-name=minimum-value,maximum-value}
    \end{quote}

     and objects will only be selected if their value for column \textit{column-name}\/ lies between \textit{minimum-value}\/ and \textit{maximum-value}.  The second is:

    \begin{quote}
     \textit{minimum-magnitude,maximum-magnitude}
    \end{quote}

     here the range is assumed to be a magnitude and no column name is
     specified (remember that magnitudes increase the `wrong way round'
     so that \textit{minimum-magnitude}\/ corresponds to the brightest
     object).  The third is:

    \begin{quote}
     \textit{maximum-magnitude}
    \end{quote}

     which is again assumed to be a magnitude and only objects brighter
     than \textit{maximum-magnitude}\/ are selected.  For completeness, the
     relationship between the forms of \textit{additional-condition}\/ and
     the `query tokens' for the database specified in the configuration
     file (as described in \xref{SSN/75}{ssn75}{}\cite{SSN75}) is that
     \textit{additional-condition}\/ replaces the tokens as follows:

    \begin{center}
    \begin{tabular}{rll}
             & Form                                & replaces token \\ \hline
     first:  & \textit{column-name=minimum-value,maximum-value} & \texttt{\%cond} \\
     second: & \textit{minimum-magnitude,maximum-magnitude}     & \texttt{\%m1},
                                                               \texttt{\%m2} \\
     third:  & \textit{maximum-magnitude}                       & \texttt{\%m}  \\
    \end{tabular}
    \end{center}

  \end{description}

  \item[Output] The objects selected are written as a catalogue in the
   Tab-Separated Table (TST) format in the current directory.  The TST
   format is described in \xref{SSN/75}{ssn75}{}\cite{SSN75}.  The name
   of the catalogue file is derived automatically from the name of the
   database and the central Right Ascension and Declination.  \catremote\ displays a line showing the name of the file created,
   for example:

  \begin{terminalv}
!(Info.) Catalogue usno_eso_101500_303000.tab written successfully.
  \end{terminalv}

   If the catalogue contained no objects which satisfied the query
   \catremote\ will report:

  \begin{terminalv}
! Failure: no objects found in the region specified.
  \end{terminalv}

\end{description}

\subsection{\texttt{name} mode}

\begin{description}

  \item[Input] \texttt{name} mode has the following two arguments.  If they
   are omitted then they will be prompted for.

  \begin{description}

    \item[\textit{db-name}\/] The name of the name resolver database which
     is to be queried (see Section~\ref{ACL}).  The usual choice is \texttt{simbad\_ns@eso}, the SIMBAD name resolver provided by ESO using the
     \htmladdnormallinkfoot{SIMBAD}{http://simbad.u-strasbg.fr/Simbad}
     integrated database maintained by the
     \htmladdnormallink{Centre de Donn\'{e}es astronomiques de Strasbourg}
     {http://cdsweb.u-strasbg.fr/CDS.html} (CDS).

    \item[\textit{object-name}] The name of an astronomical object which is
     to be resolved.  It should be entered without embedded spaces.  The
     case of letters (upper or lower) is not usually significant.  That
     is, case is not significant for \texttt{simbad\_ns@eso} and probably
     will not be significant for other name resolvers.

  \end{description}

  \item[Output] If the name resolver resolves the given name, that is
   successfully looks it up and finds coordinates for it, then they are
   displayed, for example:

  \begin{terminalv}
Right Ascension: +10:47:50
Declination: +12:34:57
  \end{terminalv}

   The Right Ascension is in sexagesimal hours, the Declination in
   sexagesimal degrees and both are for equinox J2000.  If the name
   could not be resolved then \catremote\ reports:

  \begin{terminalv}
! Failure: unable to resolve object name.
  \end{terminalv}

\end{description}

\subsection{\texttt{help} mode}

\begin{description}

  \item[Input] No arguments are required.

  \item[Output] All the various modes are listed, with a one-line
   summary of each.

\end{description}


\section{\xlabel{I_EXAMPLES}\label{I_EXAMPLES}Interactive Examples}

This section gives some examples of using \catremote\ interactively.
All the arguments are given on the command line.  However, if they are
omitted then arguments other than the first (the mode) will usually be
prompted for.

\begin{enumerate}

  \item List the various modes in which \catremote\ may be used.
   Type either of:

  \begin{terminalv}
% catremote
% catremote help
  \end{terminalv}

  \item List all the databases in the current configuration file:

  \begin{terminalv}
% catremote list
  \end{terminalv}

  \item List all the name servers (that is, databases of server type
   `\texttt{namesvr}') in the current configuration file:

  \begin{terminalv}
% catremote list namesvr
  \end{terminalv}

  \item Show details of the USNO PMM astrometric catalogue:

  \begin{terminalv}
% catremote details usno@eso
  \end{terminalv}

  \item Find all the objects in the USNO PMM which lie within ten
   minutes of arc of Right Ascension \hms{12}{15}{00}{0} and
   Declination \dms{30}{30}{00}{0} (J2000):

  \begin{terminalv}
% catremote query usno@eso 12:15:00 30:30:00 10
  \end{terminalv}

   The objects selected will be saved as a catalogue called \texttt{usno\_eso\_121500\_303000.tab} created in your current directory.
   This catalogue will be written in the Tab-Separated Table (TST)
   format.

  \item Find all the objects in the USNO PMM which lie within ten
   minutes of arc of Right Ascension \hms{12}{15}{00}{0} and
   Declination \dms{30}{30}{00}{0} (J2000) which also lie in the
   magnitude range 14 to 16:

  \begin{terminalv}
% catremote query usno@eso 12:15:00 30:30:00 10 14,16
  \end{terminalv}

  \item Find the equatorial coordinates of the galaxy NGC 3379:

  \begin{terminalv}
% catremote name simbad_ns@eso ngc3379
  \end{terminalv}

   The coordinates returned are for equinox J2000.

\end{enumerate}


\section{\xlabel{S_EXAMPLES}\label{S_EXAMPLES}Scripting Examples}

Some complete examples which illustrate the use of \catremote\ in
scripts are available.  They are written in the tcl scripting
language\footnote{Tcl is described by its author, John~Ousterhout, in his
\textit{Tcl and the Tk Toolkit}\/\cite{OUSTERHOUT94}; see also
\xref{SUN/200}{sun200}{}\cite{SUN200}.} and on Starlink systems are in
directory:

\begin{terminalv}
/star/share/cursa
\end{terminalv}

The scripts provided are:

\begin{description}

  \item[\texttt{listavaildb.tcl}] list the databases currently available,

  \item[\texttt{querycat.tcl}] query a remote catalogue,

  \item[\texttt{resolvename.tcl}] find the coordinates of a named object.

\end{description}

On Starlink systems the scripts should simply run without requiring any
modifications.  On non-Starlink systems you may need to edit them to
change the specified location of both the tcl interpreter and the \catremote\ utility.  In all cases the former is the first line of the
script and the latter the first executable line after the introductory
comments.  Sufficient comments should be included to document the use of
\catremote\ in each script.

The utility \texttt{findcoords} (see \xref{SUN/240}{sun240}{}\cite{SUN240})
is a simple wrap-around for the name-resolver function of \catremote.
It can serve as an example of invoking \catremote\ from a Perl script.
On Starlink systems the source is available in file:

\begin{terminalv}
/star/bin/findcoords
\end{terminalv}


\section{\xlabel{WEBPROG}\label{WEBPROG}Remote Access Utility}

\catremote\ runs a separate utility program to submit the URL
representing the query of a remote catalogue and retrieve the results.
The utility to be used is not hard-wired into \catremote, but rather
is specified using environment variable \texttt{CATREM\_URLREADER} (see
Section~\ref{ENVIR}).  Several options are available:

\begin{itemize}

  \item \geturl, a C program supplied with \catremote,

  \item \texttt{UrlReader}, a Java program also supplied with \catremote,

  \item the \htmladdnormallinkfoot{\texttt{lynx}}{http://lynx.browser.org/}
   command-line browser,

  \item the \texttt{wget} URL-retrieval tool.

\end{itemize}

\geturl\ is the default on Starlink systems.  It is reasonably fast, as
are \texttt{lynx} and and \texttt{wget}, though \texttt{UrlReader} seems somewhat
slower.  Brief details of using the various options follow.

\begin{description}

  \item[\geturl] specify:

  \begin{terminalv}
% setenv CATREM_URLREADER "/star/bin/cursa/geturl"
  \end{terminalv}

  \item[\texttt{UrlReader}] specify:

  \begin{terminalv}
% setenv CATREM_URLREADER  "java  UrlReader"
  \end{terminalv}

   It is also necessary to set the java environment variable \texttt{CLASSPATH}
   so that \texttt{UrlReader} is picked up in addition to the standard Java
   classes.  For example, I might set:

  \begin{terminalv}
% setenv CLASSPATH /star/bin/cursa:/usr/lib/netscape/java/classes
  \end{terminalv}

   Note that it is necessary to specify the location of \texttt{UrlReader}
   using \texttt{CLASSPATH} rather than the corresponding Java command-line
   option    because the latter appears not to work on Compaq Alpha/Tru64.

  \item[\texttt{lynx}] specify:

  \begin{terminalv}
% setenv  CATREM_URLREADER  "lynx -source"
  \end{terminalv}

  \item[\texttt{wget}] specify:

  \begin{terminalv}
% setenv  CATREM_URLREADER  "wget -q -O -"
  \end{terminalv}

   \texttt{wget} is distributed as part of RedHat Linux systems and will
   probably only be available as part of such systems.  On these systems
   an on-line manual can be accessed by typing:

  \begin{terminalv}
% info  wget
  \end{terminalv}

\end{description}






% \newpage
\addcontentsline{toc}{section}{Acknowledgements}
\section*{Acknowledgements}

I am grateful to Allan~Brighton, Martin~Bly, Peter~Draper, Horst~Meyerdierks
and Mike~Read for either advice or comments on the draft version of the
document. Any mistakes, of course, are my own.


% References ----------------------------------------------------------

% \section{References}

% \input{refs.tex}
% \newpage
\addcontentsline{toc}{section}{References}
\begin{thebibliography}{99}

  \bibitem{SUN193} M.J.~Bly, 16 June 1999,
   \xref{SUN/193.4}{sun193}{}: \textit{PERL ---  Practical Extraction and
   Report Language}, Starlink.

  \bibitem{SUN200} M.J.~Bly, 13 October 2000,
   \xref{SUN/200.3}{sun200}{}: \textit{TCLSYS --- TCL, TK and EXPECT
   utilities}, Starlink.

  \bibitem{SSN75} A.C.~Davenhall, 26 July 2000,
   \xref{SSN/75.1}{ssn75}{}: \textit{Writing Catalogue and Image Servers for
   GAIA and CURSA}, Starlink.

  \bibitem{SUN190} A.C.~Davenhall, 14 May 2001,
   \xref{SUN/190.9}{sun190}{}: \textit{CURSA --- Catalogue and Table
   Manipulation Applications}, Starlink.

  \bibitem{SUN240} A.C.~Davenhall, 29 May 2001,
   \xref{SUN/240.1}{sun240}{}: \textit{FINDCOORDS --- Finding the Coordinates
   of a Named Object}, Starlink.

  \bibitem{OUSTERHOUT94} J.K.Ousterhout, 1994, \textit{Tcl and the Tk
   Toolkit}\/ (Addison-Wesley: Reading, Massachusetts).

\end{thebibliography}

% ---------------------------------------------------------------------
\appendix
\section{Detailed Description of Applications}

\newpage
\sstroutine{
   CATREMOTE
}{
   A simple script to query remote catalogues
}{
   \sstdescription{
      catremote is a tool for querying remote astronomical catalogues,
      databases and archives via the Internet.  It allows remote
      catalogues to be queried and the resulting table saved as a local
      file written in the Tab-Separated Table (TST) format.  It also
      provides a number of related auxiliary functions.

      catremote has several different modes of usage, each providing a
      different function.  The modes are:

      list    - list the catalogues currently available,

      details - show details of a named catalogue,

      query   - submit a query to a remote catalogue and retrieve the results,

      name    - resolve an object name into coordinates,

      help    - list the modes available.

      There is an introduction to using catremote in SUN/190 and it is
      comprehensively documented in SSN/76.
   }
   \sstusage{
      Arguments for catremote can be specified on the command line.
      If arguments other than the first are omitted then they will usually
      be prompted for.  The first argument is the mode of operation and
      its value determines the other arguments which are required.  The
      arguments for the various modes are:

       catremote list    server-type

       catremote details db-name

       catremote query   db-name alpha delta radius additional-condition

       catremote name    db-name object-name

       catremote help

      The individual arguments are described in the `Arguments\texttt{'} section.
      If the mode is omitted then  \texttt{'}help\texttt{'} mode is assumed.

      In addition to the command-line arguments, catremote takes some
      input from Unix shell environment variables and these variables can
      be used to control its behaviour.
   }
   \sstexamples{
      \sstexamplesubsection{
         catremote
      }{
      }
      \sstexamplesubsection{
         catremote help
      }{
         List the various modes in which catremote may be used.
      }
      \sstexamplesubsection{
         catremote list
      }{
         List all the catalogues and databases in the current configuration
         file.
      }
      \sstexamplesubsection{
         catremote list namesvr
      }{
         List all the name servers (that is, databases of server type
         \texttt{'}namesvr\texttt{'}) in the current configuration file.
      }
      \sstexamplesubsection{
         catremote details usno@eso
      }{
         Show details of the USNO PMM astrometric catalogue (whose name
         is \texttt{'}usno@eso\texttt{'}).
      }
      \sstexamplesubsection{
         catremote query usno@eso 12:15:00 30:30:00 10
      }{
         Find all the objects in the USNO PMM which lie within ten minutes
         of arc of Right Ascension 12:15:00.0 (sexagesimal hours) and
         Declination 30:30:00.0 (sexagesimal degrees, both J2000).  The
         objects selected will be saved as a catalogue called
         usno\_eso\_121500\_303000.tab created in your current directory.
         This catalogue will be written in the Tab-Separated Table (TST)
         format.
      }
      \sstexamplesubsection{
         catremote query usno@eso 12:15:00 30:30:00 10 14,16
      }{
         Find all the objects in the USNO PMM which lie within ten minutes
         of arc of Right Ascension 12:15:00.0 (sexagesimal hours) and
         Declination 30:30:00.0 (sexagesimal degrees, both J2000) which
         also lie in the magnitude range 14 to 16.
      }
     \sstexamplesubsection{
         catremote name simbad\_ns@eso ngc3379
      }{
         Find the equatorial coordinates of the galaxy NGC 3379.  The
         coordinates returned are for equinox J2000.
      }
   }
   \sstdiytopic{
      Environment Variables
   }{
      CATREM\_URLREADER (read)
         catremote uses a separate program to submit the URL constituting
         a query to the server and return the table of results.  This
         environment variable specifies the program to be used.  See
         SSN/76 for further details.  (Mandatory.)

      CATREM\_CONFIG (read)
         This environment variable specifies the configuration file to be
         used.  It should be set to either the URL (for a remote file) or
         the local file name, including a directory specification (for a
         local file).  Configuration files mediate the interaction between
         catremote and the remote catalogue; see SSN/76 for further
         details.  (Mandatory.)

      CATREM\_MAXOBJ (read)
         The maximum number of objects which the returned table is allowed
         to contain.

      CATREM\_ECHOURL (read)
         This environment controls whether the URL representing the query
         submitted to the remote catalogue is also displayed to the user.
         The default is \texttt{'}no\texttt{'}; to see the URL set CATREM\_ECHOURL to \texttt{'}yes\texttt{'}.
         Seeing the URL is potentially useful when debugging configuration
         files and remote catalogue servers but is not usually required
         for normal operation.
   }
}

\newpage
\sstroutine{
   GETURL
}{
   Retrieve a specified URL and write it to standard output
}{
   \sstdescription{
      Retrieve the contents of a specified URL and write them to standard
      output.
   }
   \sstusage{
      geturl  url-to-be-retrieved  [show-HTTP-header]
   }
   \sstparameters{
      \sstsubsection{
         url-to-be-retrieved
      }{
         The URL whose contents are to be written to standard output.
      }
      \sstsubsection{
         show-HTTP-header
      }{
         If any value is given for this optional argument then the
         HTTP header at the start of the requested page (which is usually
         hidden) is echoed to standard output.
      }
   }
   \sstexamples{
      \sstexamplesubsection{
         geturl  http://www.roe.ac.uk/acdwww/cursa/home.html
      }{
         Retrieve the contents of URL
         http://www.roe.ac.uk/acdwww/cursa/home.html.
      }
      \sstexamplesubsection{
         geturl  http://www.roe.ac.uk/acdwww/cursa/home.html  head
      }{
         Retrieve the contents of URL
         http://www.roe.ac.uk/acdwww/cursa/home.html and show the header
         (which is usually hidden) at the start of the page.
      }
   }
}


\typeout{  }
\typeout{*****************************************************}
\typeout{  }
\typeout{Reminder: run this document through Latex three times}
\typeout{to resolve the references.}
\typeout{  }
\typeout{*****************************************************}
\typeout{  }

\end{document}
