\documentstyle[11pt]{article}
\pagestyle{myheadings}

% -----------------------------------------------------------------------------
% ? Document identification
\newcommand{\stardoccategory}  {Starlink User Note}
\newcommand{\stardocinitials}  {SUN}
\newcommand{\stardocsource}    {sun\stardocnumber}
\newcommand{\stardocnumber}    {56.10}
\newcommand{\stardocauthors}   {P T Wallace}
\newcommand{\stardocdate}      {21st June 1995}
\newcommand{\stardoctitle}     {COCO --- Conversion of Celestial Coordinates}
\newcommand{\stardocversion}   {v2.2-3}
\newcommand{\stardocmanual}    {User Guide}
% ? End of document identification
% -----------------------------------------------------------------------------

\newcommand{\stardocname}{\stardocinitials /\stardocnumber}
\markright{\stardocname}
\setlength{\textwidth}{160mm}
\setlength{\textheight}{230mm}
\setlength{\topmargin}{-2mm}
\setlength{\oddsidemargin}{0mm}
\setlength{\evensidemargin}{0mm}
\setlength{\parindent}{0mm}
\setlength{\parskip}{\medskipamount}
\setlength{\unitlength}{1mm}

% -----------------------------------------------------------------------------
%  Hypertext definitions.
%  ======================
%  These are used by the LaTeX2HTML translator in conjunction with star2html.

%  Comment.sty: version 2.0, 19 June 1992
%  Selectively in/exclude pieces of text.
%
%  Author
%    Victor Eijkhout                                      <eijkhout@cs.utk.edu>
%    Department of Computer Science
%    University Tennessee at Knoxville
%    104 Ayres Hall
%    Knoxville, TN 37996
%    USA

%  Do not remove the %\begin{latexonly} and %\end{latexonly} lines (used by 
%  star2html to signify raw TeX that latex2html cannot process).
%begin{latexonly}
\makeatletter
\def\makeinnocent#1{\catcode`#1=12 }
\def\csarg#1#2{\expandafter#1\csname#2\endcsname}

\def\ThrowAwayComment#1{\begingroup
    \def\CurrentComment{#1}%
    \let\do\makeinnocent \dospecials
    \makeinnocent\^^L% and whatever other special cases
    \endlinechar`\^^M \catcode`\^^M=12 \xComment}
{\catcode`\^^M=12 \endlinechar=-1 %
 \gdef\xComment#1^^M{\def\test{#1}
      \csarg\ifx{PlainEnd\CurrentComment Test}\test
          \let\html@next\endgroup
      \else \csarg\ifx{LaLaEnd\CurrentComment Test}\test
            \edef\html@next{\endgroup\noexpand\end{\CurrentComment}}
      \else \let\html@next\xComment
      \fi \fi \html@next}
}
\makeatother

\def\includecomment
 #1{\expandafter\def\csname#1\endcsname{}%
    \expandafter\def\csname end#1\endcsname{}}
\def\excludecomment
 #1{\expandafter\def\csname#1\endcsname{\ThrowAwayComment{#1}}%
    {\escapechar=-1\relax
     \csarg\xdef{PlainEnd#1Test}{\string\\end#1}%
     \csarg\xdef{LaLaEnd#1Test}{\string\\end\string\{#1\string\}}%
    }}

%  Define environments that ignore their contents.
\excludecomment{comment}
\excludecomment{rawhtml}
\excludecomment{htmlonly}

%  Hypertext commands etc. This is a condensed version of the html.sty
%  file supplied with LaTeX2HTML by: Nikos Drakos <nikos@cbl.leeds.ac.uk> &
%  Jelle van Zeijl <jvzeijl@isou17.estec.esa.nl>. The LaTeX2HTML documentation
%  should be consulted about all commands (and the environments defined above)
%  except \xref and \xlabel which are Starlink specific.

\newcommand{\htmladdnormallinkfoot}[2]{#1\footnote{#2}}
\newcommand{\htmladdnormallink}[2]{#1}
\newcommand{\htmladdimg}[1]{}
\newenvironment{latexonly}{}{}
\newcommand{\hyperref}[4]{#2\ref{#4}#3}
\newcommand{\htmlref}[2]{#1}
\newcommand{\htmlimage}[1]{}
\newcommand{\htmladdtonavigation}[1]{}
\newcommand{\latexhtml}[2]{#1}

%  Starlink cross-references and labels.
\newcommand{\xref}[3]{#1}
\newcommand{\xlabel}[1]{}

%  LaTeX2HTML symbol.
\newcommand{\latextohtml}{{\bf LaTeX}{2}{\tt{HTML}}}

%  Define command to re-centre underscore for Latex and leave as normal
%  for HTML (severe problems with \_ in tabbing environments and \_\_
%  generally otherwise).
\newcommand{\latex}[1]{#1}
\newcommand{\setunderscore}{\renewcommand{\_}{{\tt\symbol{95}}}}
\latex{\setunderscore}

%  Redefine the \tableofcontents command. This procrastination is necessary 
%  to stop the automatic creation of a second table of contents page
%  by latex2html.
\newcommand{\latexonlytoc}[0]{\tableofcontents}


% -----------------------------------------------------------------------------
%  Debugging.
%  =========
%  Remove % on the following to debug links in the HTML version using Latex.

% \newcommand{\hotlink}[2]{\fbox{\begin{tabular}[t]{@{}c@{}}#1\\\hline{\footnotesize #2}\end{tabular}}}
% \renewcommand{\htmladdnormallinkfoot}[2]{\hotlink{#1}{#2}}
% \renewcommand{\htmladdnormallink}[2]{\hotlink{#1}{#2}}
% \renewcommand{\hyperref}[4]{\hotlink{#1}{\S\ref{#4}}}
% \renewcommand{\htmlref}[2]{\hotlink{#1}{\S\ref{#2}}}
% \renewcommand{\xref}[3]{\hotlink{#1}{#2 -- #3}}
%end{latexonly}
% -----------------------------------------------------------------------------
% ? Document specific \newcommand or \newenvironment commands.

\newcommand{\radec}  {$[\alpha,\delta\,]$}

\newcommand{\qt}[1]{``{\tt{#1}}''}
\begin{htmlonly}
   \newcommand{\qt}[1]{{\tt{"#1"}}}
\end{htmlonly}

\newcommand{\fstring}[1]{\hbox{\hspace{0.05em}{\qt{#1}}\hspace{0.05em}}}

% ? End of document specific commands
% -----------------------------------------------------------------------------
%  Title Page.
%  ===========
\renewcommand{\thepage}{\roman{page}}
\begin{document}
\thispagestyle{empty}

%  Latex document header.
%  ======================
\begin{latexonly}
   CCLRC / {\sc Rutherford Appleton Laboratory} \hfill {\bf \stardocname}\\
   {\large Particle Physics \& Astronomy Research Council}\\
   {\large Starlink Project\\}
   {\large \stardoccategory\ \stardocnumber}
   \begin{flushright}
   \stardocauthors\\
   \stardocdate
   \end{flushright}
   \vspace{-4mm}
   \rule{\textwidth}{0.5mm}
   \vspace{5mm}
   \begin{center}
   {\Huge\bf  \stardoctitle \\ [2.5ex]}
   {\LARGE\bf \stardocversion \\ [4ex]}
   {\Huge\bf  \stardocmanual}
   \end{center}
   \vspace{5mm}

% ? Heading for abstract if used.
%  \vspace{10mm}
%  \begin{center}
%     {\Large\bf Abstract}
%  \end{center}
% ? End of heading for abstract.
\end{latexonly}

%  HTML documentation header.
%  ==========================
\begin{htmlonly}
   \xlabel{}
   \begin{rawhtml} <H1> \end{rawhtml}
      \stardoctitle\\
      \stardocversion\\
      \stardocmanual
   \begin{rawhtml} </H1> \end{rawhtml}

% ? Add picture here if required.
% ? End of picture

   \begin{rawhtml} <P> <I> \end{rawhtml}
   \stardoccategory \stardocnumber \\
   \stardocauthors \\
   \stardocdate
   \begin{rawhtml} </I> </P> <H3> \end{rawhtml}
      \htmladdnormallink{CCLRC}{http://www.cclrc.ac.uk} /
      \htmladdnormallink{Rutherford Appleton Laboratory}
                        {http://www.cclrc.ac.uk/ral} \\
      \htmladdnormallink{Particle Physics \& Astronomy Research Council}
                        {http://www.pparc.ac.uk} \\
   \begin{rawhtml} </H3> <H2> \end{rawhtml}
      \htmladdnormallink{Starlink Project}{http://www.starlink.ac.uk/}
   \begin{rawhtml} </H2> \end{rawhtml}
   \htmladdnormallink{\htmladdimg{source.gif} Retrieve hardcopy}
      {http://www.starlink.ac.uk/cgi-bin/hcserver?\stardocsource}\\

%  HTML document table of contents. 
%  ================================
%  Add table of contents header and a navigation button to return to this 
%  point in the document (this should always go before the abstract \section). 
  \label{stardoccontents}
  \begin{rawhtml} 
    <HR>
    <H2>Contents</H2>
  \end{rawhtml}
  \newcommand{\latexonlytoc}[0]{}
  \htmladdtonavigation{\htmlref{\htmladdimg{contents_motif.gif}}
        {stardoccontents}}

% ? New section for abstract if used.
% \section{\xlabel{abstract}Abstract}
% ? End of new section for abstract
\end{htmlonly}

% -----------------------------------------------------------------------------
% ? Document Abstract. (if used)
%   ==================
% ? End of document abstract
% -----------------------------------------------------------------------------
% ? Latex document Table of Contents (if used).
%  ===========================================
 \newpage
 \begin{latexonly}
   \setlength{\parskip}{0mm}
   \latexonlytoc
   \setlength{\parskip}{\medskipamount}
   \markright{\stardocname}
 \end{latexonly}
% ? End of Latex document table of contents
% -----------------------------------------------------------------------------
\newpage
\renewcommand{\thepage}{\arabic{page}}
\setcounter{page}{1}

%------------------------------------------------------------------------------

\section{Introduction}

The COCO program converts star coordinates from one system to
another.  Both the improved IAU system, post-1976, and the old pre-1976
system are supported.
COCO can perform accurate transformations between the following
coordinate systems:
\begin{itemize}
\item mean \radec, old system, with E-terms (loosely FK4, usually B1950)
\item mean \radec, old system, no E-terms (some radio positions)
\item mean \radec, new system (loosely FK5, usually J2000)
\item geocentric apparent \radec, new system
\item ecliptic coordinates $[\lambda,\beta\,]$, new system (mean of date)
\item galactic coordinates $[l^{II},b^{II}]$, IAU 1958 system
\end{itemize}
COCO's user-interface is spartan but efficient.  The program
offers control over report resolution, and there has a simple online help
facility.
All input is free-format, and defaults are provided where this
is meaningful.

The input/output arrangements of COCO are flexible, to allow
a variety of operating styles -- interactive, input from a file,
report to a file, batch, {\it etc}.
Also, in addition to the report which is produced, the results of the
conversions are also available in a raw form, to a fixed resolution
and free from extraneous formatting; this file is intended
to be read easily by other programs.

In order to comply with the IAU 1976 recommendations, all
position data published from 1984 on should be given in
the new system, using equinox J2000.0.
However, positions are still frequently given in
the old system, using equinox B1950.0.
Discriminating astronomers and astrophysicists are best
advised to give both B1950 and J2000
positions for sources mentioned in their publications; the
conversion can be done using COCO.

\section{Operating Instructions}
To run COCO interactively, on either VAX/VMS platforms or
the Sun SPARCstation and DECstation Unix-based platforms, simply type:
\begin{verbatim}
      COCO
\end{verbatim}
The program then accepts commands and outputs appropriate replies.

\goodbreak
The primary commands are as follows:
\begin{verbatim}
   Command                     Function

     I p    specify input coordinate system (p defined below)
     O p    specify output coordinate system (       "       )
  <coords>  perform conversion  (or = to repeat last coordinates)
     E      exit
\end{verbatim}
where the parameter {\tt p}, specifying the coordinate system, is
as follows:
\begin{verbatim}
     4 [eq] [ep]     equatorial, FK4 (barycentric)
     B [eq] [ep]     like FK4 but without E-terms (barycentric)
     5 [eq] [ep]     equatorial, FK5 (barycentric)
     A  ep           equatorial, geocentric apparent
     E  ep           ecliptic (barycentric)
     G [ep]          galactic (barycentric)

     eq = equinox, e.g. 1950 (optional B or J prefix)
     ep = epoch, e.g. 1984.53 or 1983 2 26.4
\end{verbatim}
Coordinates type 4 and B default to equinox B1950. Coordinates type 5 default 
to equinox J2000. In all three of these coordinate systems the epoch defaults 
to the equinox. For coordinate type G the epoch defaults to B1950.

The following commands are also available:
\begin{verbatim}
   Command                     Function

     F         specify RA mode: x = H for hours, D for degrees
     S         display current settings
     ?         show format of <coords>
     /<file>   switch to secondary input file <file>
     H         list the commands
     R x       select report resolution:  x = H, M or L
                                        (high,medium,low)
\end{verbatim}

Input is free-format, with spaces separating the fields (comma is
also acceptable as a field separator within coordinates).
Both upper and lower case letters are acceptable.
Blank lines can be input freely, and a comment can be appended to
any line by preceding it with an asterisk.

The $\alpha$ format command, F, selects either hours or degrees as
the units for $\alpha$, affecting both the input formats which are
accepted and the format of the outputs.  The command ``F~D"
selects degrees as the $\alpha$ unit, limits the input formats
to a single number in degrees for both $\alpha$ and $\delta$, and
causes the output formats to be a single number in degrees
for both $\alpha$ and $\delta$.  The command ``F~H" switches to the
hours format, enabling a variety of sexagesimal input
formats as well as plain hours and degrees, and causing
the output formats to be h,m,s,d,$'$,$''$.

On startup, COCO is set to input FK4~B1950 and output
FK5~J2000, with $\alpha$ in hours and medium report resolution.

In a typical COCO run, the first step would be to specify the
input and output coordinate systems by means of the I and O
commands, and then to enter the coordinates to be
transformed.
For example, suppose that we wish to convert
a QSO position measured from a plate taken in mid-1976 and
using reference stars from the B1950.0 SAO catalogue to coordinates
in the new J2000.0 system.
The following commands could be used:
\begin{verbatim}
  I 4 1950 1976.5        * input system is FK4 B1950; epoch is 1976.5
  O 5                    * output system is FK5 J2000
  12 43 25.3 +32 15 29   * measured 1950 position; no proper motion
\end{verbatim}

Some additional operating modes are described in a later section.

Notes:
\begin{enumerate}
\item The results output by COCO are believed to be of more
than adequate accuracy for all practical purposes at
present and are far more precise than any
available star coordinates.
Geocentric apparent place
is the least accurate form,
and is limited (in the worst case) to about 0.3~milliarcsec
by the model for Earth velocity and position that is
used.
The more straightforward conversions are, as
implementations of the accepted algorithms, several
orders of magnitude better than this figure.
It should
be noted, however, that there is a lingering debate
about the precise formulation of the conversion between
FK4 data and the new FK5 system. COCO uses the algorithm
published in the 1985 Astronomical Almanac.
The differences are far too small to pose problems for non-specialists.
\item The three report resolutions provided are referred
to simply as L (low), M (medium),
and H (high).
At resolutions ``L" and ``M", all the
figures output are trustworthy.
Resolution ``H" is provided
mainly to allow comparison with other predictions and to
decrease rounding errors where differences are taken.
\item COCO is for use only with sources well outside the solar
system.
Where appropriate, stellar parallax and aberration
are allowed for, but the corrections for gravitational
deflection assume that the source is distant.
In particular, COCO is not suitable for predicting apparent places
for the Sun.
\item COCO is not intended for the conversion of catalogue data,
and reports positions only; updated proper motions {\it etc.}\
are not reported.
Full conversion of catalogue data is best done by writing
{\it ad hoc}\, programs, using the subprograms in the SLALIB
library (see SUN/67).
\item COCO accepts both commands and data from up to two
sources, called the {\it primary}\, and {\it secondary}\ input files.
To switch to a secondary input file called STARS.DAT, for
example, would require the command /STARS.DAT.
(The switch command itself is available only from the primary input file.) 
When an E command, or end of file, is detected during input from
a secondary file, control reverts to the primary file.
The same secondary input file may be processed several times in
one COCO run, as in the example of producing the apparent places of
a fixed list of stars for a series of dates -- the secondary file
could be the list of stars and the primary file a series of pairs of
commands, each pair specifying a new epoch and then switching again
to the secondary file to process all the stars anew.
\item All of COCO's coordinate systems except for geocentric apparent
are barycentric, {\it i.e.}\ unaffected by parallax.
Mean places which include displacements
due to parallax can be handled by working
via the intermediary of geocentric
apparent place.  For example, the following procedure takes a J2000
barycentric place and adds the (small) effect of parallax for a
given date:
\begin{enumerate}
\item Set the report resolution to high.
\item Set the input system to J2000.
\item Set the output system to geocentric apparent place for
      the required date.
\item Enter the mean place with proper motions and parallax (and radial
      velocity if available).
\item Note the apparent place.
\item Set the input system to geocentric apparent, specifying the
      same date as in step~(c).
\item Set the output system to J2000.
\item Type in the apparent place from step~(e).  The result will be close
      to the position entered in step~(d), but with the effects of parallax
      added.
\end{enumerate}
\end{enumerate}

\section{Input Formats}

The input and output coordinate systems require, variously,
{\it equinoxes}\, and {\it epochs}. The timescale is TDB, which
for most uses of COCO can be regarded as the same as UTC.

Equinoxes are Besselian or Julian epochs.
They can be preceded by a B or J as appropriate; in default,
epochs before 1984.0 are assumed to be Besselian, while epochs
1984.0 and after are assumed to be Julian (the distinction is
usually unimportant in this context). 
Valid examples are B1950.0, J2000, 1975.

Epochs may either be expressed as Besselian or Julian epochs,
or as year,month,day in the Gregorian calendar.
Valid examples are J1984.3296 and 1985~2~13.2439.
Calendar dates have to have valid years and months, but a days
value outside the
conventional range is permissible ({\it e.g.}~1992~12~32)

In the mean \radec\ systems, the equinox defines the coordinate
system while the epoch defines the date of observation.
In the two cases where the reference frame is inertial -- FK5
and galactic -- the epoch is required merely to allow the proper
motion to be calculated.

Each input coordinate system has its own data format, as follows.

\subsection{The three sorts of Mean RA,Dec}

For $\alpha$ in hours the following formats are accepted:
\begin{verbatim}
        RA     Dec        PM       Px    RV
       h m s  d ' "  [s/y  "/y     ["   [km/s]]]
 or    h m s  d '
 or    h m    d '
 or    h      d
\end{verbatim}

The first two, four, five, or six numeric fields, representing
$\alpha$ and $\delta$,
are mandatory. The seventh, eighth, ninth and tenth, representing proper motion,
parallax and radial velocity, are optional, with the proviso
that both proper motions are required if present at all.
The \radec\ fields are permitted to exceed the conventional ranges; this is
to remain consistent with many existing tabulations (for example
$\alpha$~24~00~01.063 would be accepted and correctly interpreted).
The scope for data validation is correspondingly limited.

Note that the proper motions are per year rather than per century.
Note also that east-west proper motion is normally seconds of time
(of $\alpha$) rather than arcseconds (on the sky).  However, COCO will
accept east-west proper motion expressed in arcseconds on the sky
if the $\alpha$ proper motion is followed by the double-quote
character fstring{"}.
Star data from catalogues which express proper motions
in other ways (for example per century or in terms of position-angle)
have to be transformed before they can be used by COCO.

In the case of FK4 coordinates (with or without E-terms), omitting
proper motions is interpreted as meaning that the proper motions
are assumed negligible in an inertial frame.
If a star has zero proper motion in the FK4 system, then zeroes must
be entered explicitly; such a star will have a real proper motion of
up to 0.5~arcsec per century (just as distant sources, galaxies
for example, have a fictitious proper motion in the FK4 system).
In addition, because the FK4 system is rotating relative to an
inertial frame, if the proper motions are omitted it is important to
specify the epoch at which the position was correct.

In the case of FK5 coordinates, omission of proper motions
simply implies zero proper motion (FK5 is presumed not to
be rotating).

The parallax and radial velocity both default to zero.  (Even when the
parallax is not zero, the \radec\ is barycentric -- see Section~2, Note~6.)

For $\alpha$ in degrees (selected by the command ``F~D"), the
following format is accepted:
\begin{verbatim}
        RA     Dec
        d       d
\end{verbatim}

Both numbers are required.  There is provision for neither sexagesimal
notation nor proper motion.

\subsection{Geocentric Apparent}

For $\alpha$ in hours the following formats are accepted:
\begin{verbatim}
        RA     Dec
       h m s  d ' "
 or    h m s  d '
 or    h m    d '
\end{verbatim}

Four, five, or six numbers are required.

For $\alpha$ in degrees (selected by the command ``F~D"), the
following format is accepted:
\begin{verbatim}
        RA     Dec
        d       d
\end{verbatim}

Both numbers are required.

\subsection{Ecliptic}

\begin{verbatim} 
       lambda  beta
          d      d
\end{verbatim}

Both numbers are required.

\subsection{Galactic}

\begin{verbatim}
       L2   B2
        d    d
\end{verbatim}

Both numbers are required.

\section{Further Operating Modes}

The VMS DCL command procedure and Unix C-shell script which
invoke COCO
both have optional parameters which
allow the sources and destinations of the main input and
output streams to be defined:
\begin{verbatim}
      COCO   [input]  [report]  [output]
\end{verbatim}

In the case of interactive operation, both input and report
are assigned to the terminal.
The output file includes only the results, in a fixed format and
without extraneous formatting; it is intended to be
easy to read in other programs.

The combination of parameters presented causes COCO's
input/output streams to be assigned one of several configurations
which are intended to produce sensible results in a variety
of modes of use, both interactive and batch.
More direct control of COCO may be desirable on occasion.

Under VMS, it is permissible to invoke the COCO executable program
by first assigning the logical names {\tt FOR0nn} to the required
files:
\begin{verbatim}
     FOR011      output     raw output for redirection
     FOR012      output     echo of input
     FOR013      output     report (including error warnings)
     FOR015      input      commands and data
     FOR016      output     prompts
     FOR017      output     error warnings
\end{verbatim}

(In addition, I/O unit 14 is used internally to read secondary
files.)

As an example, suppose that we wish to run COCO in batch on a VMS
system, with
the commands included in the batch command procedure itself,
with no output file, and with the results to appear in the
printed log.
This could be accomplished by submitting the following command procedure:
\begin{verbatim}
      $  ASSIGN/USER_MODE NL:         FOR011  ! raw output
      $  ASSIGN/USER_MODE SYS_$OUTPUT FOR012  ! echo of input
      $  ASSIGN/USER_MODE SYS_$OUTPUT FOR013  ! report
      $  ASSIGN/USER_MODE SYS_$INPUT  FOR015  ! commands
      $  ASSIGN/USER_MODE SYS_$OUTPUT FOR016  ! prompts
      $  ASSIGN/USER_MODE NL:         FOR017  ! errors
      $  RUN COCO_DIR:COCO

      I 5      * input J2000 FK5
      O 4      * output B1950 FK4

      *  Source position
      14 39 36.087 -60 50 07.14   -0.49486 +0.6960   0.752 -22.2

      E
\end{verbatim}

On the Unix platforms, the COCO executable may be invoked by means of
the following shell command:

\begin{verbatim}
      coco.x terml output echo report input prompts errors
\end{verbatim}

The seven command-line parameters, all of which must be present,
are, respectively, the string identifying the terminal (perhaps
obtained by using the command {\tt set terml = `tty`}) and the
six filenames.

\section{Acknowledgments}

During COCO's development,
invaluable advice was received from the staff of Her Majesty's
Nautical Almanac Office at the Royal Greenwich
Observatory, in particular Catherine~Hohenkerk,
Bernard~Yallop and Brian~Emerson.
Thanks are also due to No\"{e}l~Argue of Cambridge,
to Andrew~Lyne of Jodrell~Bank,
to Dick~Manchester of CSIRO~Radiophysics in
Sydney for testing a pre-release version of COCO, to
Mark~Calabretta, also of Radiophysics, for supplying changes to
allow COCO to be built for the Convex/ConvexOS platform, to Phil~Hill
(St.~Andrews) for his advice on proper-motion handling, and to
Leslie~Morrison (RGO) for suggestions concerning parallax.
\end{document}
