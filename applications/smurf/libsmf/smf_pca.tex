\documentclass{article}
\usepackage{amsmath}
\renewcommand{\vec}[1]{\underline{#1}}

\begin{document}
\title{Description of the algorithm used by \texttt{smf\_pca}}
\author{David S Berry}
\maketitle

If we have $N_b$ usable bolometers, each with $N_t$ time slices, first form
the correlation matrix, $C$ - a square symmetric $(N_b,N_b)$ matrix in which element
$(i,j)$ is the coefficient of correlation between bolometers $i$ and $j$.
To do this we first \emph{standardise} the bolometer data values so that each
bolometer has mean zero and standard deviation 1.0:

\begin{equation*}
d_{standardised}^{tb} = \frac{( d^{tb} - mean^b )}{\sigma^b}
\end{equation*}

where $d^{tb}$ is the value of bolometer $b$ at time slice $t$, $mean^b$ is
the mean of the data values for bolometer $b$ and $\sigma^b$ is the
standard deviation of the data values for bolometer $b$. Note, when
finding these mean and standard deviation values, samples that fall
within source regions are ignored. Element $i,j$ of the correlation matrix
is then given by:

\begin{equation*}
C_{ij} = \frac{1}{N'_t-1}.\sum_{t=0}^{N_t} d_{standardised}^{ti}.d_{standardised}^{tj}
\end{equation*}

Again, samples that fall within source regions are excluded from the
above sum, and $N'_t$ is the number of time slices that do \emph{not}
fall within source regions.


Singular Value Decomposition (function \texttt{smf\_svd}) is then used to find the
eigenvalues ($L$) and eigenvectors ($V$) of matrix $C$:

\begin{equation*}
C = V.L.V^T
\end{equation*}

The eigenvectors returned by \texttt{smf\_svd} are of unit length and are
sorted so that the corresponding eigenvalues are monotonic decreasing. So
the first eigenvector holds the ratios of the standardised bolometer values
corresponding to the strongest correlation in the data. Each eigenvector
is returned by \texttt{smf\_svd} in a single row (i.e. a contiguous group
of elements) of array ``\texttt{c}''. Thus if there are three bolometers,
$b$ is the bolometer index and $k$ is the eigenvector (i.e. component)
index, then the elements $V_b^k$ of the eigenvector matrix $\vec{V}$ are
returned by \texttt{smf\_svd} in an array in the following order:

\begin{equation*}
[ V_0^0, V_1^0, V_2^0, V_0^1, V_1^1, V_2^1, V_0^2, V_1^2, V_2^2 ]
\end{equation*}

An eigenvector corresponds to a ``principal axis'' in PCA. If you imagine
an $N_b$-dimensional space where each axis corresponds to the standardised
value of a single bolometer then a principal axis is a unit vector within this space,
indicating the relative standardised bolometer values corresponding to a principal
component. A ``principal component'' is the projection of the $N_t$ time slices
(each of which can be thought of as a single position in the $N_b$-space) onto the
principal axis. Thus a principal component is a time-stream containing $N_t$
standardised values for a notional ``bolometer'' corresponding to the principal axis.

The eigenvalue associated with an eigenvector indicates the degree to
which the bolometer time streams are correlated to the corresponding
principal component. In fact, from empirical experiment, it seems that if
$c_i$ is the coefficient of correlation between the time-stream for bolometer
$i$ and a specific principal component, then the eigenvalue $ev$ is:

\begin{equation*}
ev = \sum_{i=0}^{N_b}c_i^2
\end{equation*}

So if $c_{rms}$ is the RMS of the $c_i$ values,
$c_{rms} = \sqrt{\frac{\sum_{i=0}^{N_b}c_i^2}{N_b}}$ then

\begin{equation*}
ev = N_b.c_{rms}^2
\end{equation*}

The $c_{rms}$ value represents a ``typical'' correlation between the
principal component and the $N_b$ bolometer time-streams, in the range
$0.0$ to $1.0$. Thus if the user supplies a $c_{rms}$ value, it can be
converted to an $ev$ value and used to identify the principal components
that are to be removed (all principal components with corresponding
eigenvalues above $ev$ are removed).

To remove a correlated component from all bolometer time streams, we
would do the following:

\begin{verbatim}
- loop round all time slices:
   - Form a vector from the usable bolometer standardised values at
     the current time slice
   - Find the dot product of this vector with the eigenvector
     representing the strongest correlation. This gives you the
     length of the projection of the bolometer value vector in
     the direction of the eigenvector.
   - Multiply the eigenvector by the value of this dot product
   - Loop round all elements of the eigenvector:
     - Multiply the vector element (a standardised bolometer value)
       by the $\sigma_b$ value for the bolometer to un-standardise it
     - Subtract the un-standardised value from the original bolometer
       value
\end{verbatim}

Other components of the data  (i.e. correlations) can be removed in the same way,
using a different vector $V^k$ in place of the first vector $V^0$. The process
can be described by the following equation:

\begin{equation*}
\vec{D_t} = \vec{d_t} - \vec{\sigma}.(\vec{d'_t}.\vec{V_k}).\vec{V_k}
\end{equation*}

where $\vec{d_t}$ is the vector of $N_b$ bolometer values at time slice
$t$ before correction, $\vec{D_t}$ is the equivalent vector of corrected
bolometer values, $k$ is the component index (in the range 0 to
$N_b-1$), and $\vec{\sigma}$ is a diagonal $(N_b,N_b)$ matrix containing
the $\sigma_b$ values on the diagonal. To remove the strongest $N_k$ components,
this becomes:

\begin{equation}
\label{eq:1}
\vec{D_t} = \vec{d_t} - \vec{\sigma}.\sum_{k=0}^{N_k-1}( (\vec{d'_t}.\vec{V_k}).\vec{V_k} )
\end{equation}

Flux from astronomical sources will confuse things because such flux is
not well correlated across bolometers, since different bolometers will
see the source at different times. For strong sources this will make it
difficult the measure the underlying genuine correlations. One way to
reduce this problem is to exclude source samples when evaluating the
correlation matrix, $C$. The resulting gaps in the bolometer time streams
do not need to be filled with artificial interpolated data values, since
the values in the correlation matrix do not depend on the relative
position in time or space of the bolometer samples. However, the gaps
\emph{should} be filled when using the time streams to form the dot
products in the above equation. This is because forming the dot product
requires a full set of $N_b$ bolometer samples (since the $V$ vectors all
contain $N_b$ elements). Using the original unmasked data values would be
inappropriate since the strong source flux would bias the value of the
dot product. A prime is used ($\vec{d'_t}$) above to indicate the
time-streams that are are masked and gap-filled to remove astronomical
flux. Having found the correction, it should be removed from the original,
non-masked, time-stream ($\vec{d_t}$).

\end{document}
