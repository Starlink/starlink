%\scrollmode
%\documentclass[dvips,a4paper,twoside]{article}
\documentclass[dvips,a4paper]{article}
\usepackage{html,htmllist,makeidx,enumerate}
\usepackage[dvips]{graphicx}
\usepackage[dvips]{color}
%\usepackage{frames}
% \usepackage{times}

%
% use the url.sty package, by  Donald Arseneau <asnd@triumf.ca>
% to typeset email-addresses, URLs and directory paths in LaTeX ...
%
%begin{latexonly}
 \usepackage[dvips,rightbars]{changebar}
 \usepackage{l2hman}
 \usepackage{url}
 \usepackage{longtable}
 \def\email{\begingroup \urlstyle{tt}\Url}
 \def\Email#1{\email{<#1>}}
 \urldef\onlinedoc\url{http://www-dsed.llnl.gov/files/programs/unix/latex2html/manual/}
 \urldef\onlinedocRM\url{http://www-math.mpce.mq.edu.au/~ross/latex/manual/manual.html}
 \urldef\EXcolors\url{http://www-math.mpce.mq.edu.au/~ross/latex/crayola/crayola.html}
 \urldef\CVSrepos\url{http://cdc-server.cdc.informatik.th-darmstadt.de/~latex2html/user/}
 \urldef\CVSsite\url{http://cdc-server.cdc.informatik.th-darmstadt.de/latex2html/}
 \urldef\patches\url{http://www-dsed.llnl.gov/files/programs/unix/latex2html/}
 \urldef\sourceA\url{http://www-dsed.llnl.gov/files/programs/unix/latex2html/sources/latex2html-97.1.tar.gz}
 \urldef\sourceB\url{ftp://ftp.mpn.com/pub/nikos/latex2html-97.1.tar.gz}
 \urldef\sourceC\url{ftp://ftp.rzg.mpg.de/outgoing/latex2html-97.1.tar.gz}
 \urldef\CTAN\path{tex-archive/support/latex2html}
 \urldef\curia\url{http://curia.ucc.ie/info/TeX/menu.html}
 \urldef\fernuni\url{http://es-sun2.fernuni-hagen.de/info2html?(latex.info)Top}

%
%  a little hack, needed since a footnote occurs in a figure-caption
%
\def\adjustfootnote{\setcounter{footnote}{6}\footnotetext{http://csep1.phy.ornl.gov/csep.html}}

%
% These are needed for the Glossary and Index.
%
\newenvironment{theglossary}{\begin{list}{}{\setlength{\labelwidth}{20pt}%
 \setlength{\leftmargin}{\labelwidth}\setlength\itemindent{-\labelwidth}%
 \setlength\itemsep{0pt}\setlength\parsep{0pt}\rmfamily}}{\end{list}}
\def\dotfill{\leaders\hbox to.6em{\hss .\hss}\hskip 0pt plus  1fill}%
\def\dotfil{\leaders\hbox to.6em{\hss .\hss}\hfil}%
\def\pfill{\unskip~\dotfill\penalty500\strut\nobreak\dotfil~\ignorespaces}%

\newcommand\Glossary[2]{\glossary{#1@#2}}
\newcommand{\gsl}{\textsl}
\newcommand{\indexentry}[2]{\item #1 #2}

%\newcommand{\latextohtml}{\textup{\LaTeX 2\texttt{HTML}}}%

%
% These macros are built-in to LaTeX2HTML:
%
\newcommand{\Xy}{\leavevmode
 \hbox{\kern-.1em X\kern-.3em\lower.4ex\hbox{Y\kern-.15em}}}
\newcommand{\AmSTeX}{\protect\AmS-\protect\TeX{}}
\newcommand{\AmS}{{\protect\AmSfont A\kern-.1667em\lower.5ex\hbox{M}\kern-.125emS}}
\gdef\AmSfont{\usefont{OMS}{cmsy}{m}{n}}

\newcommand{\sameas}[1]{\ (Same as setting: #1)}
\setcounter{footnote}{0}
%end{latexonly}

%  get rid of changebar.
\renewenvironment{changebar}{}{}


%
\begin{htmlonly}
 \def\makeglossary{}
\end{htmlonly}

\input manhtml.tex


% use %sort -f -u manual.idx > manual.index for a primitive index
%
%  NOTE:  You must use LaTeX2e in order to process this document
%       If you do not have LaTeX2e, a PostScript version
%       (manual.ps) is included with this distribution.
%
%%%%%%%%%%%%%%%%%%% No changes beyond this point %%%%%%%%%%%%%%%%%%%%%%%%%%%%%

\makeindex
\makeglossary
\sloppy
%
\setlength{\textwidth}{5.5in}
%\addtolength{\oddsidemargin}{-1in}
%\addtolength{\evensidemargin}{-1in}
\setlength{\changebarwidth}{1pt}

%
% read own internals for sections/contents before any
% from the segments.
%
%\internal[sections]{}
%\internal[contents]{}

\internal[figure]{O}
\internal[figure]{S}
\internal[figure]{M}
\internal[figure]{H}
\internal[figure]{E}%{F}

\internal[table]{O}
\internal[table]{S}
\internal[table]{M}
\internal[table]{H}
\internal[table]{E}%{F}

\internal[sections]{O}
\internal[sections]{S}
\internal[sections]{M}
\internal[sections]{H}
\internal[sections]{E}%{F}
\internal[sections]{P}
%\internal[sections]{C}

\internal[contents]{O}
\internal[contents]{S}
\internal[contents]{M}
\internal[contents]{H}
\internal[contents]{E}%{F}
\internal[contents]{P}
%\internal[contents]{C}


\internal[internals]{O}
\internal[internals]{S}
\internal[internals]{M}
\internal[internals]{H}
\internal[internals]{E}%{F}
\internal[internals]{P}
%\internal[internals]{C}

%\internal[index]{O}
%\internal[index]{S}
%\internal[index]{M}
%\internal[index]{H}
%\internal[index]{E}%{F}
%\internal[index]{P}
%%\internal[index]{C}


\begin{document}
\sloppy
%
%  TITLE-PAGE
%
\Glossary{latex2html}{\LaTeX2HTML}{}
\title{The \LaTeX2HTML{} Translator}
\author{Nikos Drakos\\Computer Based Learning Unit\\University of Leeds.}
\date{\today}
\index{Computer~Based~Learning~Unit!University of Leeds}%
\maketitle

\htmlrule
%
\begin{center}{
Documentation revised and updated for \textsc{v97.1} and \texttt{HTML} 3.2 by:}
\end{center}
\medskip
\begin{center}
%begin{latexonly}
\begin{large}
%end{latexonly}
Ross Moore\\
Mathematics Department\\
Macquarie University, Sydney.
%begin{latexonly}
\end{large}
%end{latexonly}
\end{center}
\bigskip
\htmlrule

%
%  for printed version only
%
\begin{latexonly}
\begin{center}
{\large
This document accompanies \latextohtml{} version 97.1}%
\footnote{The manuscript was updated to version \textsc{v96.1}\,,
as indicated with thin change-bars, by Herbert W.\ Swan \Email{lanhws@expl.aai.arco.com}
and converted to \LaTeXe{} by \Goossens~\Email{goossens@cern.ch}.
Extensive revisions of the manuscript were made by \RossMoore~\Email{ross@mpce.mq.edu.au}
for \textsc{v96.1} \texttt{rev-f}, incorporating also suggestions from \Goossens.
Another major revision was required to adequately describe the new features
made possible with \texttt{HTML} 3.2\,,
and recent developments in image-generation and macro-handling.
This work was done by \RossMoore.
Wider change-bars indicate where newly added features are described.}%
\Glossary{latex2e}{\LaTeXe}%
\end{center}
\end{latexonly}

%
%  for HTML version only
%
\begin{htmlonly}

This document accompanies \latextohtml{}\Glossary{latex2html}{\LaTeX2HTML}{}, version 97.1.
\Glossary{latex2e}{\LaTeXe}{}

The manuscript was updated to version 96.1, as indicated with change-bar icons,
by Herbert W\,Swan \Email{lanhws@expl.aai.arco.com}
and converted to \LaTeXe{} by \Goossens~\Email{goossens@cern.ch}.

Updates and extensive revisions to the manuscript for version \textsc{v96.1} \texttt{rev-f},
were made by \RossMoore~\Email{ross@mpce.mq.edu.au},
also incorporating suggestions from \Goossens.

\bigskip
Another major revision was required to adequately describe the new features
made possible with \texttt{HTML} 3.2\,,
and recent developments in image-generation and macro-handling.
This work was done by \RossMoore.
Appropriate change-bar icons indicate where newly added features are described.

\htmlrule

%
% customise the URLs below, for the nearest copies of  manual.ps(.gz)
%
\index{PostScript version}
A \htmladdnormallink{PostScript version}%{manual.ps}
{ftp://www.mpce.mq.edu.au/pub/maths/tex/l2hmanual.ps}
is available, also \htmladdnormallink{compressed}%{manual.ps.gz}
{ftp://www.mpce.mq.edu.au/pub/maths/tex/l2hmanual.ps.gz}.\\

\htmlrule

Browse the following links for information concerning:

\begin{htmllist}\htmlitemmark{RedBall}
\item[Contributions from others --- \htmlref{early development}{credits}]
\item[Contributions from others --- \htmlref{recent developments, 1996}{recent96}]
\item[Contributions from others --- \htmlref{recent developments, 1997}{recent97}]
\item[Proposals for \htmlref{future development}{future}]
\item[\htmlref{Licensing}{licence}]
\end{htmllist}

\htmlrule

\textbf{Warning:} The contents of this document are likely to change.\\
It is advisable not to use links to any pages other than the first page (this page).

\end{htmlonly}

\htmlrule
%
%
%  ABSTRACT
%
\Glossary{latex}{\LaTeX}{}%
\Glossary{perl}{\textsl{Perl}}{}%
\glossary{HTML}%
\begin{abstract}%
\latextohtml{} is a conversion tool that allows documents
written in \LaTeX{}  to become part of the World-Wide Web.
In addition, it offers an easy migration path towards
authoring complex hyper-media documents using
familiar word-processing concepts, including the power of a \LaTeX-like
macro language capable of producing correctly structured \texttt{HTML} tags.

\latextohtml{} replicates the basic structure of a \LaTeX{}  document
as a set of interconnected \texttt{HTML} files which can be explored using
automatically generated navigation panels.
The cross-references, citations, footnotes, the table-of-contents and the lists
of figures and tables, are also translated into hypertext links. Formatting
information which has equivalent ``tags'' in \texttt{HTML}
(lists, quotes, paragraph-breaks, type-styles, etc.)
is also converted appropriately.
The remaining heavily formatted items
such as mathematical equations, pictures etc. are converted to images
which are placed automatically at the correct position in the
final \texttt{HTML} document.

\latextohtml{} extends \LaTeX{}  by supporting arbitrary hypertext links
and symbolic cross-references between evolving
remote documents. It also allows the specification
of \emph{conditional text} and the inclusion of raw \texttt{HTML} commands.
These hyper-media extensions to \LaTeX{}  are available as
new commands and environments from within a \LaTeX{}  document.

This document presents the main features of \latextohtml{} and
describes how to obtain and install it, and how to use it effectively.
\end{abstract}


%
%  CREDITS
%
\latex{\pagenumbering{roman}}
\clearpage\input{credits.tex}
\clearpage\section*{General License Agreement and Lack of Warranty\label{licence}}%
This software is distributed in the hope that it will be useful
but \emph{without any warranty}. The author(s) do not accept responsibility
to anyone for the consequences of using it or for whether it serves
any particular purpose or works at all. No warranty is made about
the software or its performance.

Use and copying of this software and the preparation of derivative
works based on this software are permitted, so long as the following
conditions are met:
\begin{itemize}
\item
The copyright notice and this entire notice are included intact
and prominently carried on all copies and supporting documentation.
\item
No fees or compensation are charged for use, copies, or
access to this software. You may charge a nominal
distribution fee for the physical act of transferring a
copy, but you may not charge for the program itself.
\item
If you modify this software, you must cause the modified
file(s) to carry prominent notices (a \texttt{ChangeLog})
describing the changes, who made the changes, and the date
of those changes.
\item
Any work distributed or published that in whole or in part
contains or is a derivative of this software or any part
thereof is subject to the terms of this agreement. The
aggregation of another unrelated program with this software
or its derivative on a volume of storage or distribution
medium does not bring the other program under the scope
of these terms.
\end{itemize}
This software is made available \emph{as is}, and is distributed without
warranty of any kind, either expressed or implied.
In no event will the author(s) or their institutions be liable to you
for damages, including lost profits, lost monies, or other special,
incidental or consequential damages arising out of or in connection
with the use or inability to use (including but not limited to loss of
data or data being rendered inaccurate or losses sustained by third
parties or a failure of the program to operate as documented) the
program, even if you have been advised of the possibility of such
damages, or for any claim by any other party, whether in an action of
contract, negligence, or other tortuous action.

\smallskip
\index{copyright!Leeds}\html{\\}%
The \latextohtml{} translator is written by Nikos Drakos,
Computer Based Learning Unit,  University of Leeds,  Leeds,  LS2 9JT.
Copyright \copyright 1993--1997. All rights reserved.

\smallskip
\index{copyright!Macquarie}\html{\\}%
The \textsc{v97.1} revision of the \latextohtml{} translator and this manual
was prepared by Ross Moore, Mathematics Department,
Macquarie University,  Sydney~2109,  Australia.
Copyright \copyright 1996--1997. All rights reserved.










%
%  CONTENTS,
%  Lists of figures, tables
%
\clearpage
\tableofcontents
\clearpage
\listoffigures
\listoftables
%begin{latexonly}
 \adjustfootnote
%end{latexonly}
\clearpage


%
%  MAIN MANUAL
%

%begin{latexonly}
\cleardoublepage
\pagenumbering{arabic}\setcounter{page}{1}
%end{latexonly}

\relax   %% this is important, else the next segment doesn't get processed

\segment{overview}{section}{Overview
 \protect\label{sec:ovw}\protect\index{overview}}
\latex{\vfil\goodbreak}
\segment{support}{section}{Installation and Further Support
 \protect\label{sec:sup}\protect\index{install}}
\latex{\vfil\goodbreak}
\segment{features}{section}{Environments and Special Features
 \protect\label{sec:fea}\protect\index{environments}\protect\index{special features}}
\latex{\vfil\goodbreak}
\segment{hypextra}{section}{Hypertext Extensions to \LaTeX
 \protect\label{sec:hyp}\protect\index{special}}
\latex{\vfil\goodbreak}
\segment{userman}{section}{Customising the Layout of HTML pages
 \protect\label{sec:man}\protect\index{customised layout}}
\latex{\vfil\goodbreak}
\segment{problems}{section}{Known Problems
 \protect\label{sec:prb}\protect\index{problems|(}}

%\segment{changes}{section}{Changes from Previous Versions
% \protect\label{sec:chg}\protect\index{changes|(}}

%
%  CHANGES
%
%\begin{htmlonly}
%\relax   %% this is important, else the next section doesn't get handled correctly
%\section{Changes from Previous Versions}
%\input{changes.tex}
%\end{htmlonly}


%
%  BIBLIOGRAPHY
%
\bibliographystyle{plain}
\begin{thebibliography}{1}\label{biblio}

\index{LaTeX blue book@\LaTeX{} blue book!Leslie Lamport}%
\htmladdimg[ALIGN=RIGHT]{http://www.aw.com/thumaw.gif}
\bibitem{lamp:latex}
Leslie Lamport,
\newblock \textit{\LaTeX, A Document Preparation System}. User's Guide \& Reference Manual, 2nd edition.
\newblock ISBN 0--201--52983--1, Paperback 256 pages, %
\htmladdnormallink{Addison--Wesley}%
{http://heg-school.aw.com/cseng/authors/lamport/latex/latex.html}, 1994.
\newblock Online information on {\TeX} and {\LaTeX} is available at \curia\\and~\fernuni~.

%begin{latexonly}
\index{Companion|see{\emph{The \LaTeX{} Companion}}}%
\index{LaTeX Companion|see{\emph{The \LaTeX{} Companion}}}%
\index{The LaTeX Companion@\emph{The \LaTeX{} Companion}\label{IIIlatcomp}!Goossens--Mittelbach--Samarin}%
%end{latexonly}
\begin{htmlonly}
\index{Companion|see{\htmlref{The \LaTeX{} Companion}{IIIlatcomp}}}%
\index{LaTeX Companion|see{\htmlref{The \LaTeX{} Companion}{IIIlatcomp}}}%
\index{The LaTeX Companion@\htmlref{The \LaTeX{} Companion}\label{IIIlatcomp}!Goossens--Mittelbach--Samarin}%
\end{htmlonly}
\htmladdimg[ALIGN=RIGHT]{http://www.aw.com/cp/tlc.small.gif}
\bibitem{goossens:latex}
Michel Goossens, Frank Mittelbach, Alexander Samarin,
\newblock \textit{The \LaTeX{} Companion}
\newblock ISBN 0--201--54199--8, Paperback 530 pages, %
\htmladdnormallink{Addison--Wesley}{http://www.aw.com/cp/tlc.html}, 1994.\medskip

%begin{latexonly}
\index{Graphics Companion|see{\hfil\emph{\hfil The \LaTeX{} Graphics Companion}}}%
\index{LaTeX Graphics Companion|see{\hfil\emph{\hfil The \LaTeX{} Graphics Companion}}}%
\index{The LaTeX Graphics Companion@\emph{The \LaTeX{} Graphics Companion}\label{IIIlatgraph}!Goossens--Rahtz--Mittelbach}%
%end{latexonly}
\begin{htmlonly}
\index{Graphics Companion|see{\htmlref{The \LaTeX{} Graphics Companion}{IIIlatgraph}}}%
\index{LaTeX Graphics Companion|see{\htmlref{The \LaTeX{} Graphics Companion}{IIIlatgraph}}}%
\index{The LaTeX Graphics Companion@The \LaTeX{} Graphics Companion\label{IIIlatgraph}!Goossens--Rahtz--Mittelbach}%
\end{htmlonly}
\htmladdimg[ALIGN=RIGHT]{http://www.aw.com/cp/tlgc.small.gif}
\bibitem{goossens:latexGraphics}
Michel Goossens, Sebastian Rahtz and Frank Mittelbach,
\newblock \textit{The \LaTeX{} Graphics Companion}.
\newblock ISBN 0--201--85469--4, Softcover 608 pages, %
\htmladdnormallink{Addison--Wesley}{http://www.aw.com/cp/tlgc.html}, 1997.

\bibitem{drakos:bask}
Nikos Drakos
\newblock Text to Hypertext conversion with \latextohtml.
\newblock \textit{Baskerville}, December 1993, Vol.\,3, No.\,2, pp 12--15.
\newblock May 1994, CERN, Geneva, Switzerland.
\newblock \url{http://cbl.leeds.ac.uk/nikos/doc/www94/www94.html}

\bibitem{drakos:www94}
Nikos Drakos
\newblock From Text to Hypertext: A Post-Hoc Rationalisation of \latextohtml.
\newblock Published in ``Proceedings of the 1st World Wide Web Conference'',
\newblock May 1994, CERN, Geneva, Switzerland.
\newblock \url{http://cbl.leeds.ac.uk/nikos/doc/www94/www94.html}

\end{thebibliography}


%
%  GLOSSARY
%
% Glossary info stored in:  manual.gls ,  which was created using:
%
%       makeindex -o manual.gls -s l2hglo.ist manual.glo
%
\begin{latexonly}
\InputIfFileExists{manual.gls}{\clearpage\typeout{^^Jcreating Glossary...}}%
{\typeout{^^JNo Glossary, since  manual.gls  could not be found.^^J}}
\end{latexonly}

\begin{htmlonly}
\section{Glossary of variables and file-names\label{Glossary}}
\begin{htmllist}\htmlitemmark{OrangeBall}
\input l2hfiles.dat     %%%%  <<<<<<<<<<  don't forget, in final version !!!
\end{htmllist}
\end{htmlonly}

%
%  INDEX
%
\internal[index]{O}
\internal[index]{S}
\internal[index]{E}
\internal[index]{H}
\internal[index]{M}
\internal[index]{P}
%
% Index info stored in:  manual.ind ,  which was created using:
%
%       makeindex -s l2hidx.ist manual.idx
%
\printindex

%
%  Alphabetization and navigation within the index
%  ...these special index entries must come *after* the  \printindex
%  else half of the hyperlinks will point to the preceding page.
%
\begin{htmlonly}
\newcommand{\indexAlpha}[5]{\index{#1@\htmlref{_}{#2}%
 \htmlref{\HTML{SUB}{\LARGE #3}}{AZ}\htmlref{_}{#4}\label{#5}| }}
%
\indexAlpha{\$}{Z}{\$}{dot}{doll}%
\indexAlpha{.}{doll}{~.~}{A}{dot}%
\indexAlpha{A}{dot}{A}{B}{A}%
\indexAlpha{B}{A}{B}{C}{B}%
\indexAlpha{C}{B}{C}{D}{C}%
\indexAlpha{D}{C}{D}{E}{D}%
\indexAlpha{E}{D}{E}{F}{E}%
\indexAlpha{F}{E}{F}{G}{F}%
\indexAlpha{G}{F}{G}{H}{G}%
\indexAlpha{H}{G}{H}{I}{H}%
%\indexAlpha{I}{H}{I}{J}{I}%
\indexAlpha{I}{H}{I}{L}{I}%
\indexAlpha{J}{I}{J, K}{L}{K}%
%\indexAlpha{J}{I}{J}{K}{J}%
%\indexAlpha{K}{J}{K}{L}{K}%
\indexAlpha{L}{K}{L}{M}{L}%
\indexAlpha{M}{L}{M}{N}{M}%
\indexAlpha{N}{M}{N}{O}{N}%
\indexAlpha{O}{N}{O}{P}{O}%
%\indexAlpha{P}{O}{P}{Q}{P}%
\indexAlpha{P}{O}{P}{R}{P}%
\indexAlpha{Q}{P}{Q, R}{S}{R}%
%\indexAlpha{Q}{P}{Q}{R}{Q}%
%\indexAlpha{R}{Q}{R}{S}{R}%
\indexAlpha{S}{R}{S}{T}{S}%
\indexAlpha{T}{S}{T}{U}{T}%
\indexAlpha{U}{T}{U}{V}{U}%
\indexAlpha{V}{U}{V}{W}{V}%
\indexAlpha{W}{V}{W}{X}{W}%
%\indexAlpha{X}{W}{X}{Y}{X}%
\indexAlpha{X}{W}{X}{Z}{X}%
\indexAlpha{Y}{X}{Y, Z}{doll}{Z}%
%\indexAlpha{Y}{X}{Y}{Z}{Y}%
%\indexAlpha{Z}{Y}{Z}{doll}{Z}%
%
%
%% This is an alphabetical navigation panel.
\index{@\label{AZ}\textbf{\LARGE
\htmlref{\$}{doll} \htmlref{.}{dot} \htmlref{ A }{A}
 \htmlref{B}{B} \htmlref{C}{C} \htmlref{D}{D} \htmlref{E}{E} \htmlref{F}{F}
 \htmlref{G}{G} \htmlref{H}{H} \htmlref{I}{I} \htmlref{J}{K} \htmlref{K}{K}
 \htmlref{L}{L} \htmlref{M}{M} \htmlref{N}{N} \htmlref{O}{O} \htmlref{P}{P}
 \htmlref{Q}{R} \htmlref{R}{R} \htmlref{S}{S} \htmlref{T}{T} \htmlref{U}{U}
 \htmlref{V}{V} \htmlref{W}{W} \htmlref{X}{X} \htmlref{Y}{Z} \htmlref{Z}{Z}\\
\htmlrule[all]| }}

\end{htmlonly}

\end{document}
