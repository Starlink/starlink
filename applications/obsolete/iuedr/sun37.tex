\documentstyle[11pt]{article}
\pagestyle{myheadings}

%------------------------------------------------------------------------------
\newcommand{\stardoccategory}  {Starlink User Note}
\newcommand{\stardocinitials}  {SUN}
\newcommand{\stardocsource}    {sun37.11}
\newcommand{\stardocnumber}    {37.11}
\newcommand{\stardocauthors}
                       {Jack Giddings, Paul Rees, Dave Mills \& Martin Clayton}
\newcommand{\stardocdate}      {12 March 1996}
\newcommand{\stardoctitle}
                            {IUEDR---IUE Data Reduction package, version 3.2-0}
%------------------------------------------------------------------------------

\newcommand{\stardocname}{\stardocinitials /\stardocnumber}
\markright{\stardocname}
\setlength{\textwidth}{160mm}
\setlength{\textheight}{230mm}
\setlength{\topmargin}{-2mm}
\setlength{\oddsidemargin}{0mm}
\setlength{\evensidemargin}{0mm}
\setlength{\parindent}{0mm}
\setlength{\parskip}{\medskipamount}
\setlength{\unitlength}{1mm}

% -----------------------------------------------------------------------------
%  Hypertext definitions.
%  ======================
%  These are used by the LaTeX2HTML translator in conjuction with star2html.

%  Comment.sty: version 2.0, 19 June 1992
%  Selectively in/exclude pieces of text.
%
%  Author
%    Victor Eijkhout                                      <eijkhout@cs.utk.edu>
%    Department of Computer Science
%    University Tennessee at Knoxville
%    104 Ayres Hall
%    Knoxville, TN 37996
%    USA

%  Do not remove the %\begin{rawtex} and %\end{rawtex} lines (used by
%  star2html to signify raw TeX that latex2html cannot process).
%\begin{rawtex}
\makeatletter
\def\makeinnocent#1{\catcode`#1=12 }
\def\csarg#1#2{\expandafter#1\csname#2\endcsname}

\def\ThrowAwayComment#1{\begingroup
    \def\CurrentComment{#1}%
    \let\do\makeinnocent \dospecials
    \makeinnocent\^^L% and whatever other special cases
    \endlinechar`\^^M \catcode`\^^M=12 \xComment}
{\catcode`\^^M=12 \endlinechar=-1 %
 \gdef\xComment#1^^M{\def\test{#1}
      \csarg\ifx{PlainEnd\CurrentComment Test}\test
          \let\html@next\endgroup
      \else \csarg\ifx{LaLaEnd\CurrentComment Test}\test
            \edef\html@next{\endgroup\noexpand\end{\CurrentComment}}
      \else \let\html@next\xComment
      \fi \fi \html@next}
}
\makeatother

\def\includecomment
 #1{\expandafter\def\csname#1\endcsname{}%
    \expandafter\def\csname end#1\endcsname{}}
\def\excludecomment
 #1{\expandafter\def\csname#1\endcsname{\ThrowAwayComment{#1}}%
    {\escapechar=-1\relax
     \csarg\xdef{PlainEnd#1Test}{\string\\end#1}%
     \csarg\xdef{LaLaEnd#1Test}{\string\\end\string\{#1\string\}}%
    }}

%  Define environments that ignore their contents.
\excludecomment{comment}
\excludecomment{rawhtml}
\excludecomment{htmlonly}
%\end{rawtex}

%  Hypertext commands etc. This is a condensed version of the html.sty
%  file supplied with LaTeX2HTML by: Nikos Drakos <nikos@cbl.leeds.ac.uk> &
%  Jelle van Zeijl <jvzeijl@isou17.estec.esa.nl>. The LaTeX2HTML documentation
%  should be consulted about all commands (and the environments defined above)
%  except \xref and \xlabel which are Starlink specific.

\newcommand{\htmladdnormallinkfoot}[2]{#1\footnote{#2}}
\newcommand{\htmladdnormallink}[2]{#1}
\newcommand{\htmladdimg}[1]{}
\newenvironment{latexonly}{}{}
\newcommand{\hyperref}[4]{#2\ref{#4}#3}
\newcommand{\htmlref}[2]{#1}
\newcommand{\htmlimage}[1]{}
\newcommand{\htmladdtonavigation}[1]{}

%  Starlink cross-references and labels.
\newcommand{\xref}[3]{#1}
\newcommand{\xlabel}[1]{}

%  LaTeX2HTML symbol.
\newcommand{\latextohtml}{{\bf LaTeX}{2}{\tt{HTML}}}

%  Define command to recentre underscore for Latex and leave as normal
%  for HTML (severe problems with \_ in tabbing environments and \_\_
%  generally otherwise).
\newcommand{\latex}[1]{#1}
\newcommand{\setunderscore}{\renewcommand{\_}{{\tt\symbol{95}}}}
\latex{\setunderscore}

%  Redefine the \tableofcontents command. This procrastination is necessary
%  to stop the automatic creation of a second table of contents page
%  by latex2html.
\newcommand{\latexonlytoc}[0]{\tableofcontents}

% -----------------------------------------------------------------------------
%  Debugging.
%  =========
%  Un-comment the following to debug links in the HTML version using Latex.

% \newcommand{\hotlink}[2]{\fbox{\begin{tabular}[t]{@{}c@{}}#1\\\hline{\footnotesize #2}\end{tabular}}}
% \renewcommand{\htmladdnormallinkfoot}[2]{\hotlink{#1}{#2}}
% \renewcommand{\htmladdnormallink}[2]{\hotlink{#1}{#2}}
% \renewcommand{\hyperref}[4]{\hotlink{#1}{\S\ref{#4}}}
% \renewcommand{\htmlref}[2]{\hotlink{#1}{\S\ref{#2}}}
% \renewcommand{\xref}[3]{\hotlink{#1}{#2 -- #3}}
% -----------------------------------------------------------------------------
%  Add any document specific \newcommand or \newenvironment commands here

% -----------------------------------------------------------------------------
%  Title Page.
%  ===========
\renewcommand{\thepage}{\roman{page}}
\begin{document}
\thispagestyle{empty}

%  Latex document header.
%  ======================
\begin{latexonly}
   CCLRC / {\sc Rutherford Appleton Laboratory} \hfill {\bf \stardocname}\\
   {\large Particle Physics \& Astronomy Research Council}\\
   {\large Starlink Project\\}
   {\large \stardoccategory\ \stardocnumber}
   \begin{flushright}
   \stardocauthors\\
   \stardocdate
   \end{flushright}
   \vspace{-4mm}
   \rule{\textwidth}{0.5mm}
   \vspace{5mm}
   \begin{center}
   {\Large\bf \stardoctitle}
   \end{center}
   \vspace{5mm}

%  Add heading for abstract if used.
%   \vspace{10mm}
%   \begin{center}
%      {\Large\bf Description}
%   \end{center}
\end{latexonly}

%  HTML documentation header.
%  ==========================
\begin{htmlonly}
   \xlabel{}
   \begin{rawhtml} <H1> \end{rawhtml}
      \stardoctitle
   \begin{rawhtml} </H1> \end{rawhtml}

%  Add picture here if required.

   \begin{rawhtml} <P> <I> \end{rawhtml}
   \stardoccategory \stardocnumber \\
   \stardocauthors \\
   \stardocdate
   \begin{rawhtml} </I> </P> <H3> \end{rawhtml}
      \htmladdnormallink{CCLRC}{http://www.cclrc.ac.uk} /
      \htmladdnormallink{Rutherford Appleton Laboratory}
                        {http://www.cclrc.ac.uk/ral} \\
      Particle Physics \& Astronomy Research Council \\
   \begin{rawhtml} </H3> <H2> \end{rawhtml}
      \htmladdnormallink{Starlink Project}{http://www.starlink.ac.uk/}
   \begin{rawhtml} </H2> \end{rawhtml}
   \htmladdnormallink{\htmladdimg{source.gif} Retrieve hardcopy}
      {http://www.starlink.ac.uk/cgi-bin/hcserver?\stardocsource}\\

%  HTML document table of contents.
%  ================================
%  Add table of contents header and a navigation button to return to this
%  point in the document (this should always go before the abstract \section).
  \label{stardoccontents}
  \begin{rawhtml}
    <HR>
    <H2>Contents</H2>
  \end{rawhtml}
  \renewcommand{\latexonlytoc}[0]{}
  \htmladdtonavigation{\htmlref{\htmladdimg{contents_motif.gif}}
        {stardoccontents}}

%  Start new section for abstract if used.
%  \section{\xlabel{abstract}Abstract}

\end{htmlonly}

% -----------------------------------------------------------------------------
%  Document Abstract. (if used)
%  ==================
% -----------------------------------------------------------------------------
%  Latex document Table of Contents (if used).
%  ===========================================
 \begin{latexonly}
   \setlength{\parskip}{0mm}
   \latexonlytoc
   \setlength{\parskip}{\medskipamount}
   \markright{\stardocname}
 \end{latexonly}
% -----------------------------------------------------------------------------

%%%%%%%%%%%%%%%%%%%%%%%%%%%%%%%%%%%%%%%%%%%%%%%%%%%%%%%%%%%%%%%%%%%%%%%%%%%
\newpage
\renewcommand{\thepage}{\arabic{page}}
\setcounter{page}{1}
\section {\xlabel{introduction}\label{se:introduction}Introduction}

IUEDR is a program that provides facilities for the reduction of IUE data.
It addresses the problem of working from the IUE Guest Observer tape or disk
file through to a calibrated spectrum that can be used in scientific analysis.
In this respect, it aims to be a ``complete'' system for IUE data reduction.

%%%%%%%%%%%%%%%%%%%%%%%%%%%%%%%%%%%%%%%%%%%%%%%%%%%%%%%%%%%%%%%%%%%%%%%%%%%
\section {\xlabel{facilities}\label{se:facilities}Summary of Facilities}

Here is a brief summary of what can be done using IUEDR\@.
It should help you to decide whether it can help with your IUE work.

\begin {description}

\item [Tape and file analysis]
The contents of IUE tapes and disk files can be examined interactively to find
what images are present, and so to plan the data reduction.

\item [Reading iue images]
RAW, GPHOT and PHOT images can be read from IUE tapes or disk files in GO
format into IUEDR files stored on disk.
These files are provided with default calibrations.

\item [Spectrum extraction]
This uses techniques that are an enhancement of those pre\-sent in the TRAK
program (Giddings, 1982).
Spectra taken using either resolution mode (HIRES or LORES) can be
extracted from GPHOT or PHOT images (the latter being the newer
style photometric images that retain geometric distortion).
It is possible to correct photometric LORES images obtained with the SWP
camera for defects in the original ITF calibration.

\item [LBLS]
Create a line-by-line-spectrum, corresponding to the IUESIPS product.

\item [Spectrum calibration]
Fully calibrated spectra can be produced.
This includes various forms of wavelength corrections,
absolute flux calibration, and \'{e}chelle ripple correction for HIRES
spectra.
There is also a semi-empirical correction for the HIRES (order-overlap)
background problem.

\item [Graphical display]
Graphical display facilities are provided to aid spectrum extraction and
calibration operations.
Any GKS workstation type can be used (see
\xref{SUN/83}{sun83}{} for details of GKS on Starlink).

\item [Image display]
The image can be displayed on any available GKS workstation which has colour
image display capability.
Various cursor operations can be performed including the marking of bad
pixels, image modification and feature identification.

\item [Reading IUESIPS extracted spectra]
Extracted spectra from IUESIPS \#1 and \#2, both at low and high resolution
(MELO and MEHI) can also be read from tape or disk files in GO format.

\item [Spectrum averaging]
It is possible to combine the spectra from groups of \'{e}chelle orders
(HIRES) or from different apertures (LORES) by mapping and averaging them onto
an evenly spaced wavelength grid.

\item [Output products]
Both individual extracted spectra (orders/apertures) and combined spectra can
be output to files which may be read by \xref{DIPSO (SUN/50)}{sun50}{}.
DIPSO SP0 format files use Starlink NDF facilities.
This allows the use of all standard Starlink packages to analyse IUEDR output
products.
Files can also be created in the same format as the TRAK program.

\end {description}

The user interface takes the form of a ``dialogue'' consisting of
prompts supplied by the program to ask for the information it needs
and  commands typed  at the terminal.

%%%%%%%%%%%%%%%%%%%%%%%%%%%%%%%%%%%%%%%%%%%%%%%%%%%%%%%%%%%%%%%%%%%%%%%%%%%
\section {\xlabel{getting_started}\label{se:getting_started}Getting Started}

\subsection{Running IUEDR}

To use the IUEDR package, first type:

\begin{verbatim}
   % iuedrsetup
\end{verbatim}
at the system shell prompt {\tt \%}.  You will have to have the Starlink
Software Collection login files executed before you type {\tt iuedrsetup},
this should have been set up by your site manager.

{\tt iuedrsetup} sets up all the IUEDR commands and any necessary environment
variables. IUEDR may then by used by typing {\tt iuedr} at the shell prompt.
Note that the command must be in lower case.

\subsection{Moving IUEDR files to UNIX systems} \label{se:iuecnv}

The file formats used by UNIX IUEDR are based on the STARLINK standard
NDF library. This makes the files transportable between all supported
systems. If you have old VMS IUEDR files ({\tt i.e.,}\ {\tt .UEC}, {\tt
.UED}, {\tt .UES}, {\tt .UEM} files) then these will need to be
converted into the new format before they are transferred to a UNIX
system.

There is a VMS executable provided for this purpose, and a command file
to use it. To use the executable you must first copy:

\begin{verbatim}
   /star/bin/iuedr/iuecnv.exe
   /star/bin/iuedr/iuecnv.com
\end{verbatim}

onto your VMS system (use binary transfer for {\tt iuecnv.exe}).

When {\tt iuecnv.exe} is installed, you can then move to a directory
where your IUEDR datasets are stored and type:

\begin{verbatim}
   $ @somedisk:[somedir]iuecnv dataset
\end{verbatim}

where {\tt somedisk:[somedir]} is wherever you copied {\tt iuecnv} to, and
{\tt dataset} is the name of an IUE dataset ({\it{e.g.}} SWP23456).

The program will then locate and convert all the IUEDR datafiles
associated with the named dataset. Note that the {\tt .UEC} file is also
converted (although its name stays the same).

When conversion is complete you may copy the files ({\tt .UEC} and {\tt
*\_UE*.sdf}) to your UNIX system. The {\tt .UEC} files {\bf must} be
transferred in ascii mode, and the {\tt .sdf} files {\bf must} be transferred
in binary mode.

UNIX NDF expects that NDF container files end in the extension .sdf and
does not yet recognise {\tt .sdf} files. Thus you may need to rename
files to have the lowercase {\tt .sdf} extension (depending upon how
you do the transfer).

%%%%%%%%%%%%%%%%%%%%%%%%%%%%%%%%%%%%%%%%%%%%%%%%%%%%%%%%%%%%%%%%%%%%%%%%%%%
\section {\xlabel{documentation}\label{se:documentation}Documentation}

An experimental on-line documentation facility for IUEDR is available
at:

\begin{itemize}

\item \htmladdnormallink{\verb+http://www.star.ucl.ac.uk/~mjc/iuedr/iuedr.html+}
      {http://www.star.ucl.ac.uk/~mjc/iuedr/iuedr.html}

\end{itemize}

These web pages provide full documentation, help and other facilities for the
program and IUE data reduction.

Paper documentation is stored in a mixture of text and \LaTeX\ files. The
\LaTeX\ files are STARLINK documents and your site manager should be
able to provide paper copies.

The text files may be found in the directory {\tt \$IUEDR\_DOC}.

A good introduction to IUEDR is given by Richard Tweedy in the Starlink
document \xref{SG/7 {\sl IUE Analysis---A Tutorial.}}{sg3}{}

The main document that you should read is \xref{MUD/45 {\sl IUEDR User Guide}}
{mud45}{} which provides an introduction to IUEDR, along with much practical
advice.

The User Guide describes other IUEDR documents including:

\begin {description}

\item [\xref{IUEDR Reference Manual (SG/3)}{sg3}{}]
A formal description of each command and its parameters can be found in
Starlink Guide 3 (SG/3).
Your site manager should be able to provide a copy.

\item [IUEDR Problems]
A list of the main problems (deficiencies) associated with existing IUEDR
facilities.
The problems are described in file {\tt \$IUEDR\_DOC/problems.doc}

\item [IUEDR Versions]
A chronological list of changes made to IUEDR and its documentation.
The different versions are described in the following files:

\begin {itemize}
\item {\tt \$IUEDR\_DOC/ver10.doc} --- Changes made between pre-release and 1.0
\item {\tt \$IUEDR\_DOC/ver11.doc} --- Changes made between 1.0 and 1.1
\item {\tt \$IUEDR\_DOC/ver12.doc} --- Changes made between 1.1 and 1.2
\item {\tt \$IUEDR\_DOC/ver13.doc} --- Changes made between 1.2 and 1.3
\item {\tt \$IUEDR\_DOC/ver14.doc} --- Changes made between 1.3 and 1.4
\item {\tt \$IUEDR\_DOC/ver20.doc} --- Changes made between 1.4 and 2.0
\item {\tt \$IUEDR\_DOC/ver20a.doc} --- Changes made between 2.0 and 2.0a
\item {\tt \$IUEDR\_DOC/ver30.doc} --- Changes made between 2.0a and 3.0
\item {\tt \$IUEDR\_DOC/ver31-9.doc} --- Changes made between 3.0 and 3.1-9
\item {\tt \$IUEDR\_DOC/ver32.doc} --- Changes made between 3.1-9 and 3.2-0
\end {itemize}

The files {\tt ver12.doc} and {\tt ver13.doc} contain information about new
features which are NOT described in the User Guide, but the Manual
and HELP are updated.

\end {description}

%%%%%%%%%%%%%%%%%%%%%%%%%%%%%%%%%%%%%%%%%%%%%%%%%%%%%%%%%%%%%%%%%%%%%%%%%%%
\section {\xlabel{v3.2-0}\label{se:v3.2-0}IUEDR Version 3.2-0}

This version of IUEDR contains numerous fixes to bugs in previous releases.
The program is used in the same way as before.
Some new features have been added.
\verb+$IUEDR_DOC/ver32.doc+ explains all changes.
The important points to note are:

\begin {itemize}

\item Command-line edit and recall are available.
\item New dispersion data is provided, giving smaller \verb+GSHIFT+s,
      \verb+WSHIFT+s {\it etc.}
      for recent IUE images.
\item There are three new commands available {\tt AESHIFT}, {\tt AGSHIFT} and
      {\tt CLEAN}, review SG/3 for more details.
\item The program records user input in a text file {\tt session.lis}.
      It should also be noted that the {\tt session.lis} file is overwritten
      each time that IUEDR is run.
\item Command output redirection is now possible.

\end {itemize}


%%%%%%%%%%%%%%%%%%%%%%%%%%%%%%%%%%%%%%%%%%%%%%%%%%%%%%%%%%%%%%%%%%%%%%%%%%%
\section {\xlabel{support}\label{se:support}Support}

IUEDR was written at UCL by Dr.~Jack Giddings. The available
documentation should provide sufficient information for you to use
IUEDR effectively. However, if you find something difficult to
understand,  please ask for help.

If the program appears to produce incorrect results, then it is worth
reading  through the IUEDR documentation carefully before reporting a
``bug''. However, if the program fails (crashes), then this should be
reported immediately. Your report should include the names of any
files that you were using at the time of the error, and the log file
produced during the session ({\tt session.lis}). Keep the  data and
the log file  somewhere safe and give their whereabouts in the error
report. Then they can be inspected remotely over the Network if
necessary. It is very helpful in tracing programming errors to have as
much information to hand as possible, and you are urged to give as
full an account of the error as you can.

The preferred mechanism for correspondence is the EMAIL system.
Telephone enquiries can be very hard to understand, so please send
enquiries and bug reports to {\tt starlink@jiscmail.ac.uk}. Your enquiry
could help to provide improved documentation which would benefit other
Starlink users.

\end {document}
