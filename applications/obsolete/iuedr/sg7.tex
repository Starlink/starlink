\documentstyle[11pt]{article}
\pagestyle{myheadings}

%------------------------------------------------------------------------------
\newcommand{\stardoccategory}  {Starlink Guide}
\newcommand{\stardocinitials}  {SG}
\newcommand{\stardocsource}    {sg7.4}
\newcommand{\stardocnumber}    {7.4}
\newcommand{\stardocauthors}   {R.~W.~Tweedy \& Martin Clayton}
\newcommand{\stardocdate}      {12 March 1996}
\newcommand{\stardoctitle}     {IUE Analysis---A Tutorial}
%------------------------------------------------------------------------------

\newcommand{\stardocname}{\stardocinitials /\stardocnumber}
\newcommand{\numcir}[1]{\mbox{\hspace{3ex}$\bigcirc$\hspace{-1.7ex}{\small #1}}}
\newcommand{\lsk}{\raisebox{-0.4ex}{\rm *}}
%\renewcommand{\_}{{\tt\char'137}}     % re-centres the underscore
\markright{\stardocname}
\setlength{\textwidth}{160mm}
\setlength{\textheight}{230mm}
\setlength{\topmargin}{-2mm}
\setlength{\oddsidemargin}{0mm}
\setlength{\evensidemargin}{0mm}
\setlength{\parindent}{0mm}
\setlength{\parskip}{\medskipamount}
\setlength{\unitlength}{1mm}

% -----------------------------------------------------------------------------
%  Hypertext definitions.
%  ======================
%  These are used by the LaTeX2HTML translator in conjuction with star2html.

%  Comment.sty: version 2.0, 19 June 1992
%  Selectively in/exclude pieces of text.
%
%  Author
%    Victor Eijkhout                                      <eijkhout@cs.utk.edu>
%    Department of Computer Science
%    University Tennessee at Knoxville
%    104 Ayres Hall
%    Knoxville, TN 37996
%    USA

%  Do not remove the %\begin{rawtex} and %\end{rawtex} lines (used by 
%  star2html to signify raw TeX that latex2html cannot process).
%\begin{rawtex}
\makeatletter
\def\makeinnocent#1{\catcode`#1=12 }
\def\csarg#1#2{\expandafter#1\csname#2\endcsname}

\def\ThrowAwayComment#1{\begingroup
    \def\CurrentComment{#1}%
    \let\do\makeinnocent \dospecials
    \makeinnocent\^^L% and whatever other special cases
    \endlinechar`\^^M \catcode`\^^M=12 \xComment}
{\catcode`\^^M=12 \endlinechar=-1 %
 \gdef\xComment#1^^M{\def\test{#1}
      \csarg\ifx{PlainEnd\CurrentComment Test}\test
          \let\html@next\endgroup
      \else \csarg\ifx{LaLaEnd\CurrentComment Test}\test
            \edef\html@next{\endgroup\noexpand\end{\CurrentComment}}
      \else \let\html@next\xComment
      \fi \fi \html@next}
}
\makeatother

\def\includecomment
 #1{\expandafter\def\csname#1\endcsname{}%
    \expandafter\def\csname end#1\endcsname{}}
\def\excludecomment
 #1{\expandafter\def\csname#1\endcsname{\ThrowAwayComment{#1}}%
    {\escapechar=-1\relax
     \csarg\xdef{PlainEnd#1Test}{\string\\end#1}%
     \csarg\xdef{LaLaEnd#1Test}{\string\\end\string\{#1\string\}}%
    }}

%  Define environments that ignore their contents.
\excludecomment{comment}
\excludecomment{rawhtml}
\excludecomment{htmlonly}
%\end{rawtex}

%  Hypertext commands etc. This is a condensed version of the html.sty
%  file supplied with LaTeX2HTML by: Nikos Drakos <nikos@cbl.leeds.ac.uk> &
%  Jelle van Zeijl <jvzeijl@isou17.estec.esa.nl>. The LaTeX2HTML documentation
%  should be consulted about all commands (and the environments defined above)
%  except \xref and \xlabel which are Starlink specific.

\newcommand{\htmladdnormallinkfoot}[2]{#1\footnote{#2}}
\newcommand{\htmladdnormallink}[2]{#1}
\newcommand{\htmladdimg}[1]{}
\newenvironment{latexonly}{}{}
\newcommand{\hyperref}[4]{#2\ref{#4}#3}
\newcommand{\htmlref}[2]{#1}
\newcommand{\htmlimage}[1]{}
\newcommand{\htmladdtonavigation}[1]{}

%  Starlink cross-references and labels.
\newcommand{\xref}[3]{#1}
\newcommand{\xlabel}[1]{}

%  LaTeX2HTML symbol.
\newcommand{\latextohtml}{{\bf LaTeX}{2}{\tt{HTML}}}

%  Define command to recentre underscore for Latex and leave as normal
%  for HTML (severe problems with \_ in tabbing environments and \_\_
%  generally otherwise).
\newcommand{\latex}[1]{#1}
\newcommand{\setunderscore}{\renewcommand{\_}{{\tt\symbol{95}}}}
\latex{\setunderscore}

%  Redefine the \tableofcontents command. This procrastination is necessary 
%  to stop the automatic creation of a second table of contents page
%  by latex2html.
\newcommand{\latexonlytoc}[0]{\tableofcontents}

% -----------------------------------------------------------------------------
%  Debugging.
%  =========
%  Un-comment the following to debug links in the HTML version using Latex.

% \newcommand{\hotlink}[2]{\fbox{\begin{tabular}[t]{@{}c@{}}#1\\\hline{\footnotesize #2}\end{tabular}}}
% \renewcommand{\htmladdnormallinkfoot}[2]{\hotlink{#1}{#2}}
% \renewcommand{\htmladdnormallink}[2]{\hotlink{#1}{#2}}
% \renewcommand{\hyperref}[4]{\hotlink{#1}{\S\ref{#4}}}
% \renewcommand{\htmlref}[2]{\hotlink{#1}{\S\ref{#2}}}
% \renewcommand{\xref}[3]{\hotlink{#1}{#2 -- #3}}
% -----------------------------------------------------------------------------
%  Add any document specific \newcommand or \newenvironment commands here

% -----------------------------------------------------------------------------
%  Title Page.
%  ===========
\begin{document}

\thispagestyle{empty}

%  Latex document header.
%  ======================
\begin{latexonly}
   CCLRC / {\sc Rutherford Appleton Laboratory} \hfill {\bf \stardocname}\\
   {\large Particle Physics \& Astronomy Research Council}\\
   {\large Starlink Project\\}
   {\large \stardoccategory\ \stardocnumber}
   \begin{flushright}
   \stardocauthors\\
   \stardocdate
   \end{flushright}
   \vspace{-4mm}
   \rule{\textwidth}{0.5mm}
   \vspace{5mm}
   \begin{center}
   {\Large\bf \stardoctitle}
   \end{center}
   \vspace{5mm}

%  Add heading for abstract if used.
%   \vspace{10mm}
%   \begin{center}
%      {\Large\bf Description}
%   \end{center}
\end{latexonly}

%  HTML documentation header.
%  ==========================
\begin{htmlonly}
   \xlabel{}
   \begin{rawhtml} <H1> \end{rawhtml}
      \stardoctitle
   \begin{rawhtml} </H1> \end{rawhtml}

%  Add picture here if required.

   \begin{rawhtml} <P> <I> \end{rawhtml}
   \stardoccategory \stardocnumber \\
   \stardocauthors \\
   \stardocdate
   \begin{rawhtml} </I> </P> <H3> \end{rawhtml}
      \htmladdnormallink{CCLRC}{http://www.cclrc.ac.uk} /
      \htmladdnormallink{Rutherford Appleton Laboratory}
                        {http://www.cclrc.ac.uk/ral} \\
      Particle Physics \& Astronomy Research Council \\
   \begin{rawhtml} </H3> <H2> \end{rawhtml}
      \htmladdnormallink{Starlink Project}{http://www.starlink.ac.uk/}
   \begin{rawhtml} </H2> \end{rawhtml}
   \htmladdnormallink{\htmladdimg{source.gif} Retrieve hardcopy}
      {http://www.starlink.ac.uk/cgi-bin/hcserver?\stardocsource}\\

%  HTML document table of contents. 
%  ================================
%  Add table of contents header and a navigation button to return to this 
%  point in the document (this should always go before the abstract \section). 
  \label{stardoccontents}
  \begin{rawhtml} 
    <HR>
    <H2>Contents</H2>
  \end{rawhtml}
  \renewcommand{\latexonlytoc}[0]{}
  \htmladdtonavigation{\htmlref{\htmladdimg{contents_motif.gif}}
        {stardoccontents}}

%  Start new section for abstract if used.
%  \section{\xlabel{abstract}Abstract}

\end{htmlonly}

% -----------------------------------------------------------------------------
%  Document Abstract. (if used)
%  ==================
% -----------------------------------------------------------------------------
%  Latex document Table of Contents (if used).
%  ===========================================
\begin{latexonly}
   \setlength{\parskip}{0mm}
   \latexonlytoc
   \setlength{\parskip}{\medskipamount}
   \markright{\stardocname}
\end{latexonly}
% -----------------------------------------------------------------------------

\newpage

\section{Introduction}

The techniques for analysing IUE data have evolved over the fourteen
years of the satellite's existence, and are consequently well-developed.
However, it remains difficult for someone new to the area to become familiar
with the procedures. This document is designed for someone acquiring their
first IUE tape, and gives a step-by-step guide from mounting the tape to
doing simple analysis of the extracted spectrum.

\section{The IUE satellite}

The {\it International Ultraviolet Explorer}\, satellite (IUE) was launched in
January 1978 as a result of a three-agency collaboration between SERC, ESA,
and NASA\@. It was put into a 24-hour elliptical orbit so that while it is
permanently accessible to the Goddard Space Flight Center, it can also be
operated for eight hours a day by the Vilspa ground-station at Villafranca
del Castillo, near Madrid. This is the ESA shift; the remaining 16 hours are
divided into two shifts, US1 and US2, with the latter spanning the period
characterised by a high particle background.

\begin{htmlonly}
IUE observes in the range 1150-2000{\AA}, covered by the Short Wavelength Prime
camera (SWP), and between 1900 and 3200{\AA}, which involves the Long
Wavelength Prime and Redundant cameras (LWP and LWR). (The SWR has a faulty
readout section, so consequently has not been used since the initial
commissioning period). Two types of spectrograph exist: one uses a single
spherical grating, producing low-dispersion spectra with a resolution of
\begin{rawhtml}~6&Aring;\end{rawhtml}
; the other involves an \'{e}chelle grating as well, providing about
60 spectral orders, and a resolution 
\begin{rawhtml}~0.1&Aring;.\end{rawhtml}
Only the low-dispersion grating provides flux calibration because at the
short-wavelength end the orders from the \'{e}chelle spectrum begin to overlap,
so that the background cannot be properly defined.
\end{htmlonly}

\begin{latexonly}
IUE observes in the range 1150--2000{\AA}, covered by the Short Wavelength Prime
camera (SWP), and between 1900 and 3200{\AA}, which involves the Long
Wavelength Prime and Redundant cameras (LWP and LWR). (The SWR has a faulty
readout section, so consequently has not been used since the initial
commissioning period). Two types of spectrograph exist: one uses a single
spherical grating, producing low-dispersion spectra with a resolution of
$\sim6$\AA; the other involves an \'{e}chelle grating as well, providing about
60 spectral orders, and a resolution $\sim0.1${\AA}. Only the low-dispersion
grating provides flux calibration because at the short-wavelength
end the orders from the \'{e}chelle spectrum begin to overlap, so that the
background cannot be properly defined.
\end{latexonly}

During an observation, the spectrum appears as an image on the target of the
television cameras. There is a limit to the amount of charge that can be
accumulated there without saturation occurring, which means that the optimum
signal-to-noise ratio is about 20:1. This is quite difficult to
achieve, largely because of the varying sensitivity of the cameras at different
wavelengths, and figures of around 10:1 are more typical. Consequently, even
in high-dispersion only the centroid and equivalent width of a
narrow absorption line can be used reliably, unless several datasets are
used and carefully co-added.

Full details of the IUE satellite just after launch are given in
Boggess, A.~et al., 1978, {\it Nature}, {\bf 275}, 372, with the subsequent
series of papers giving an overview of the science that can be achieved with
it. There are frequent symposia reviewing recent research, including
{\em `Exploring the Universe with the IUE satellite'}, ed.\ Y.Kondo,
publ.\ D.~Reidel, 1987, and {\em `Evolution in Astrophysics: IUE Astronomy in
the era of new space missions'}, ed.\ E.~J.~Rolfe, ESA SP-310.

\section{Reading the tape and displaying an image}

A tape from a single observation contains four files. The first contains the
raw data, and is of little practical use, but the second contains the
photometrically calibrated spectrum (the PHOT file), which is used with the
Starlink IUE data-reduction package IUEDR (the {\it Starlink Guide no.3\,}
gives a complete description of the commands available). The third and fourth
are the extracted data using the standard NASA program IUESIPS for both
high- and low-dispersion, irrespective of whether both modes were used in the
observation. Although this represents a quick way to look at the data, it is
scientifically unusable in high-dispersion. In low-dispersion it is usually
preferable to do the extraction in a more controlled way, and thus account
for odd features (such as cosmic ray hits, or broad emission lines).

\subsection{Reading the tape headers}

The tape should be mounted in the normal way. Make sure you know the name
of the tape device and that you are logged on to the right machine.
Various environment variables need to be set up, so this should be done using

\begin{verbatim}
   % iuedrsetup
\end{verbatim}

which then gives some brief information. Entering IUEDR can then be done

\begin{verbatim}
   % iuedr
\end{verbatim}

which, after a ``welcome'' message, gives a `\verb+>+' prompt.
There is a moderately useful HELP facility, if rather slow. Each time IUEDR
is run, a file \verb+session.lis+ is created which records all the input from
the keyboard and output from the terminal. The command for reading the header
is \xref{\verb+LISTIUE+}{sg3}{LISTIUE}, but unless its default parameters are 
changed this will give just the initial ten lines of the first file.
The most useful approach is to read all the headers, which can be done using

\begin{verbatim}
   > LISTIUE PR
\end{verbatim}

The \verb+PR+ is short for \verb+PROMPT+ and may be used with any command in
IUEDR to request prompting for all parameters of a command.  Typical responses
might be

\begin{verbatim}
> LISTIUE PR
DRIVE - Tape Drive or File Name. > /dev/nrmt0h
FILE - File Number. > 1
NLINE - Number of IUE header lines printed. /10/ > -1
SKIPNEXT - Whether skip to next file. /FALSE/ > <CR>
NFILE - Number of Files to be processed. /1/ > -1
\end{verbatim}

Where

\begin{itemize}
\item \verb+/dev/nrmt0h+ is the tape drive to be read from.
\item \verb+FILE=1+ means start from the first file on the tape.
\item \verb+NLINE=-1+ means print all lines in the header for the file.
\item \verb+SKIPNEXT=FALSE+ means don't skip next file.
\item \verb+NFILE=-1+ means list all files on the tape.
\end{itemize}

The \verb+-1+ response is unique to this command.  Much of what is produced is
rubbish, and once \verb+LISTIUE+ has completed it is probably best
to \xref{\verb+QUIT+}{sg3}{QUIT} from IUEDR and delete large sections from 
the \verb+session.lis+
file before sending to a printer.  While the first ten lines are the most
useful, important information is included at the tail of the IUESIPS files,
which is why it is necessary to extract the full header information.  Remember
that each time you run IUEDR the \verb+session.lis+ file is rewritten so you
should make a copy if you want to keep the file.

\subsection{Reading the data}

Re-entering IUEDR, data may be extracted from the tape using the
\xref{\verb+READIUE+}{sg3}{READIUE}
command.  Since this package is about ten years old, there are some prompts
that would not be necessary in a more automated system, and responses are
often not obvious.  An example of running it is shown below.

\begin{verbatim}
> READIUE PR
DRIVE - Tape Drive or File Name. > /dev/nrmt0h
FILE - File Number. > 3
NLINE - Number of IUE header lines printed. /10/ > <CR>
 Tape is positioned at the start of file 3
                          895 895 7681536   1 2 013038776   +101     1  C
   9152*   2*IUESOC  *   *   *20,400*      *   *  * * * * * *     *  2  C
 SWP 38776, LT-5, 340 MIN EXPO, HI DISP, LARGE APERTURE              3  C
                                                                     4  C
                                                                     5  C
                                                                     6  C
 OBSERVER: CHENG    ID: PNMWF    12 MAY 1990 DAY 132                 7  C
                                                                     8  C
                                                                     9  C
 90132142546* 10  * 218 *OPS2PR12*142042 FES CTS 98 1 0 2560      * 10  C
 VICAR Image is 768 records,  each consisting 1536 bytes
 This is either a PHOT or a GPHOT Image
TYPE - Dataset Type (RAW, PHOT, GPHOT). > PHOT
 Assumed to be PHOT Image.
DATASET - Dataset name. > LT5
CAMERA - Camera Name (LWP, LWR, SWP, SWR). > SWP
IMAGE - Image Number. > 38776
RESOLUTION - Spectrograph resolution (HIRES or LORES). > HIRES
APERTURES - Aperture name. > LAP
EXPOSURES - Spectrum exposure time(s) (seconds). > 20400
YEAR - Year number (A.D.). > 1990
MONTH - Month Number (1-12). > 5
DAY - Day Number in Month. > 12
OBJECT - Object Identification Text. > "LT-5: 340 mins"
THDA - IUE Camera Temperature (C). > 7.83
NGEOM - Number of Chebyshev terms used to represent geometry. /5/ > <CR>
 No Absolute Calibration.
 No Spectrum Template Data.
 Reading Image from Tape.
 Calibrating Geometry.
 Writing LT5.UEC (Calibration File).
 Writing LT5_UED (Image & Quality File).
\end{verbatim}

Most of the tapes will include PHOT files, but prior to about 1981, geometric
corrections were applied at the same time as the photometric ones.
Consequently, GPHOT files appear instead, which require a different procedure.
The program will ask for {\tt ITFMAX} which is obtainable by using the
header information, combined with information in the HELP facility: type
``?'' at the {\tt ITFMAX} prompt. It has been found that correcting for
the geometric distortions in the image is best done separately, which
is why the Chebyshev polynomial co-efficients are required. There is rarely
any need to change from the default. The {\tt THDA} parameter needs to be
obtained from the IUESIPS output header information. Two values are given,
one derived from reseau motion, the other from spectrum motion; it is
believed that the former gives the more accurate result where there is a
difference.

For low-resolution data there are two extra prompts: `{\tt ITF}' requires
the answer `2' for SWP spectra, and `1' or `2' for those from the LWP or LWR.
`{\tt BADITF}' should almost always be answered `YES'; this counter-intuitive
reply requests that bad data points will be quality-flagged.

To display the image use \xref{\verb+DRIMAGE+}{sg3}{DRIMAGE}\@.

\begin{verbatim}
> DRIMAGE PR
DATASET - Dataset name. /'LT5'/ > <CR>
COLOUR - Whether to use false colour in display. /FALSE/ > <CR>
DEVICE - GKS/SGS graphics device name /'xw'/ > <CR>
ZONE - Zone to be used for plotting. /0/ > <CR>
FLAG - Whether data quality for faulty pixels is displayed. /TRUE/ > <CR>
XP - X-axis pixel limits, undefined means full extent. /[0,0]/ > <CR>
YP - Y-axis pixel limits, undefined means full extent. /[0,0]/ > <CR>
ZL - Data limits for image display, undefined means full range. /[0,0]/ > <CR>
\end{verbatim}

Changing the limits can be done using \xref{\verb+CULIMITS+}{sg3}{CULIMITS},
and then once the image has been re-displayed \xref{\verb+CURSOR+}{sg3}{CURSOR}
will allow wavelengths and other information to be read directly.

\begin{verbatim}
> CULIMITS
> DRIMAGE
> CURSOR
 S(PIXEL)    L(PIXEL)     Inten.(FN)   R(PIXEL)    Wave(A)      Order
 279.890     264.732      2146.       -3.39370     1528.42       90
 275.230     272.225      1896.       -1.32646     1545.71       89
 200.673     183.810      1922.       -3.15636     1540.92       89
\end{verbatim}

\begin{latexonly}
Within all but those with the shortest exposures, there will be defects
in the image that need to be quality-flagged, and which cannot be done
automatically.  This mostly involves cosmic-ray hits, which are usually
distinguished from, say, emission lines by affecting far fewer pixels ---
as well as often occupying the inter-order region.  A typical emission line
is the geocoronal Lyman~$\alpha$ at 1215{\AA}, visible in orders 113 and 114,
which significantly contaminates all large-aperture spectra.
The other main problem that is found occasionally is the result of a telemetry
error when the spectrum was read down from the satellite, so that a strip of
data may be missing.  Quality-flagging may be done using the command
\xref{\verb+EDIMAGE+}{sg3}{EDIMAGE} and the mouse.
It will be necessary to expand the image to a
size that makes it easy to distinguish individual pixels --- usually
$200\times 200$ is a good compromise between compactness and usability.
Occasionally an image will be strewn with cosmic-ray hits, and unfortunately
there is no substitute for going through the image systematically.  In this
case, rather than using \xref{\verb+DRIMAGE+}{sg3}{DRIMAGE},
\xref{\verb+CULIMITS+}{sg3}{CULIMITS}, and \xref{\verb+EDIMAGE+}{sg3}{EDIMAGE},
it is probably simpler to ignore the \verb+CULIMITS+ stage and use the
\xref{\verb+XL+}{sg3}{XL} and \xref{\verb+YL+}{sg3}{YL} parameters of 
\verb+DRIMAGE+\@. \verb+EDIMAGE+ works by
changing the quality of all pixels within a box defined by the user using
cursor hits at opposite corners of the box required.  The middle button of the
mouse marks the pixels ``bad'', whereas the left marks them ``good'' --- which
can thus be used to rectify mistakes.  The third button exits from
\verb+EDIMAGE+\@.
\end{latexonly}

\begin{htmlonly}
Within all but those with the shortest exposures, there will be defects in the
image that need to be quality-flagged, and which cannot be done automatically. 
This mostly involves cosmic-ray hits, which are usually distinguished from,
say, emission lines by affecting far fewer pixels --- as well as often
occupying the inter-order region.  A typical emission line is the geocoronal
Lyman~$\alpha$ at 1215{\AA}, visible in orders 113 and 114, which significantly
contaminates all large-aperture spectra. The other main problem that is found
occasionally is the result of a telemetry error when the spectrum was read down
from the satellite, so that a strip of data may be missing.  Quality-flagging
may be done using the command
\verb+EDIMAGE+ and the mouse.  It will be necessary to expand the image to a
size that makes it easy to distinguish individual pixels --- usually 200 by 200
is a good compromise between compactness and usability. Occasionally an image
will be strewn with cosmic-ray hits, and unfortunately there is no substitute
for going through the image systematically.  In this case, rather than using
\verb+DRIMAGE+,
\verb+CULIMITS+, and \verb+EDIMAGE+, it is probably simpler to ignore the
\verb+CULIMITS+ stage and use the
\verb+XL+ and \verb+YL+ options of \verb+DRIMAGE+\@. \verb+EDIMAGE+ works by
changing the quality of all pixels within a box defined by the user using
cursor hits at opposite corners of the box required.  The middle button of the
mouse marks the pixels ``bad'', whereas the left marks them ``good'' --- which
can thus be used to rectify mistakes.  The third button exits from
\verb+EDIMAGE+\@.
\end{htmlonly}

Pixels defined ``bad'' in this way will be coloured yellow when the image is
displayed with \xref{\verb+FLAG=TRUE+}{sg3}{FLAG}\@. Other colours may be 
present on the image:
red signifies overexposed pixels, and blue those that fall outside the
intensity range for plotting. There will also be a grid of green pixels, which
identify those that are affected by reseau marks.  These are a set of points
within the camera on IUE which provide a reference frame against which
geometrical distortions in the camera target are measured.  If an archival
image is used from before 1981, and the PHOT option was incorrectly
selected instead of GPHOT, these green pixels will fail to cover the
reseau marks.  It will then be necessary to re-run
\\xref{verb+READIUE+}{sg3}{READIUE}\@.

\section{Extracting a low-resolution spectrum}

\subsection{Point sources}

Extracting a low-resolution spectrum for a point-source is quite simple,
since there is just a single order, and nothing more need be done than just
typing the command \xref{\verb+TRAK+}{sg3}{TRAK}\@.  As well as extracting the 
spectrum, it does a background subtraction from the surrounding region of the 
image.  Ideally,
the output pixel size should be changed from the default, but rather than
going through the full array of parameters it can be done on a single line:

\begin{verbatim}
> TRAK GSAMP=0.707
\end{verbatim}

\begin{latexonly}
The default is \verb+GSAMP=1.414+\@.  There is a fair amount of output to the
terminal, most of which is unnecessary.  The resultant spectrum will be both
flux and wavelength calibrated, with maximum signal-to-noise of $\sim 20$ and
resolution of 6\AA\@.
\end{latexonly}

\begin{htmlonly}
The default is \xref{\verb+GSAMP=1.414+}{sg3}{GSAMP}\@.  There is a fair 
amount of output to the
terminal, most of which is unnecessary.  The resultant spectrum will be both
flux and wavelength calibrated, with maximum signal-to-noise of
\begin{rawhtml}~20\end{rawhtml} and resolution of 6\AA\@.
\end{htmlonly}

Plotting the spectrum may be done with \xref{\verb+PLFLUX+}{sg3}{PLFLUX}, 
which will work unless
the axes have been changed while using \xref{\verb+PLSCAN+}{sg3}{PLSCAN} 
(see below), in which
case \verb+PLFLUX PR+ is essential so that the plotting limits can be changed
to something sensible.  However, a more useful procedure is to output it to
\xref{DIPSO}{sun50}{} using \xref{\verb+OUTSPEC+}{sg3}{OUTSPEC} --- 
hitting the defaults except with
\xref{\verb+OUTFILE+}{sg3}{OUTFILE}, which should probably be something less 
opaque than the original image number.  Essential details of DIPSO will be 
given below.

CAUTION: never hit $<$CTRL$>$-C, or $<$CTRL$>$-D, which will cause the loss of
the data being analysed.
To quit the program, type \xref{\verb+QUIT+}{sg3}{QUIT} or 
\xref{\verb+EXIT+}{sg3}{EXIT}\@.

\subsection{Extended sources}

With extended sources and those without continua, such as high-excitation
planetary nebulae, the extraction is more complicated, but the responses
to the prompts are straightforward. An example of the responses is given below.

\begin{verbatim}
> TRAK PR
DATASET - Dataset name. /'LT5'/ > <CR>
GSAMP - Spectrum grid sampling rate (geometric pixels). /1.414/ > 0.707
CUTWV - Whether wavelength cutoff data used for extraction grid. /TRUE/ > <CR>
CENTM - Whether pre-existing centroid template is used. /FALSE/ > <CR>
CENSH - Whether the spectrum template is just shifted linearly. /FALSE/ > <CR>
CENSV - Whether the spectrum template is saved in the dataset. /FALSE/ > <CR>
CENIT - Number of centroid tracking iterations. /1/ > <CR>
CENAV - Centroid averaging filter FWHM (geometric pixels). /30/ > <CR>
CENSD - Significance level for signal to be used for centroids. /4/ >  <CR>
BKGIT - Number of background smoothing iterations. /1/ > <CR>
BKGAV - Background averaging filter FWHM (geometric pixels). /30/ > <CR>
BKGSD - Discrimination level for background pixels (s.d.). /2/ > <CR>
EXTENDED - Whether the object spectrum is expected to be extended. /FALSE/
> <CR>
AUTOSLIT - Whether GSLIT, BDIST and BSLIT are determined automatically. /TRUE/
> <CR>
\end{verbatim}

\subsection{Unusual extractions}

Occasionally a non-standard extraction is necessary.  For example, two spectra
may have been taken side-by-side in the same aperture and on the same image.
This is done in order to reduce the overheads during the observing, which arise
from the time required to read down an image and prepare for the next one.
Another reason may be that cosmic ray hits may affect a critical part of the
spectrum, but there is enough remaining that is usable.  In both cases the
\xref{\verb+AUTOSLIT+}{sg3}{AUTOSLIT} parameter needs to be set to false, 
since the automatic
extraction slit --- which is defined by the camera, resolution and aperture,
as well as whether the source is extended and has a continuum --- is no longer
appropriate.

If the object is well centred in the aperture, it will be located at the
position expected from the dispersion constants of the spectrograph.  For
a point-source in low-dispersion it is not necessary to check this, but for
unusual extractions it is vital.  This is done by using
\xref{\verb+CGSHIFT+}{sg3}{CGSHIFT} on a cut across the spectrum, provided 
by \xref{\verb+SCAN+}{sg3}{SCAN} and plotted using
\xref{\verb+PLSCAN+}{sg3}{PLSCAN}\@.  
\verb+CGSHIFT+ requires the location of the centroid of the
spectrum by hand, which thus defines the zero-line.

Three parameters follow once \verb+AUTOSLIT+ is set to false, whose values
determine the new extraction.  In each case a pair of numbers is required
unless one is the symmetric pair of the other, in which case only one need be
entered.  \xref{\verb+GSLIT+}{sg3}{GSLIT} defines the limit of the spectrum 
channel, so \verb+GSLIT=5+ is the equivalent of \verb+GSLIT=[-5,5]+ but 
\verb+GSLIT=[0,5]+
would be an asymmetric extraction.  \xref{\verb+BSLIT+}{sg3}{BSLIT} sets the 
half-width of each
background channel, so setting one of them to 0 enables only one to be
used, which is useful for the two spectra within the same aperture.
\xref{\verb+BDIST+}{sg3}{BDIST} specifies the distance of each background 
channel from the centre.

The flux calibration from this new extraction is not necessarily correct and
should be checked, where possible, by comparing with the spectrum obtained
with the default parameters.  Multiplying the new fluxes by a constant may
suffice to provide sensible values.

\section{Analysing the spectrum with DIPSO}

The \xref{Starlink User Note SUN/50}{sun50}{} is an excellent description of 
DIPSO, and also
provides a tutorial for first-time users.  Much of the information is also
contained in an on-line HELP facility. However, brief details are given here
in order that basic measurements may be made.

The package is entered simply by typing

\begin{verbatim}
% dipsosetup
% dipso
\end{verbatim}

which gives a brief ``welcome'' and, like IUEDR, changes the prompt to `$>$'.
Spectra produced by \xref{\verb+OUTMEAN+}{sg3}{OUTMEAN} within IUEDR can be 
read in using
\verb+SP0RD+, where the 0 represents the format selected within
\verb+OUTMEAN+\@.  It is useful to push the data onto the stack so that it can
easily be reclaimed if several have been read in, or if various manipulations,
like flattening the continuum or smoothing, have been performed.  A typical
start to the session might be

\begin{verbatim}
> sp0rd ngc7293_neb96
> push
> sp0rd ngc7293_neb10, push   *** several commands may be used on one line
> sl                          *** ``sl'' = Stack List
      I    N     X1         X2              TITLE
      1   611   1150.      1969.     SWP 6157: 96 arcsec from CS
      2   639   1150.      1969.     SWP 42072: 10 arcsec from CS
\end{verbatim}

The command line \verb+DEV xw, PM 2+ then selects the plotting and plots
up the second stack entry.  Had the number 2 not been added, the data in the
current arrays --- usually the last dataset accessed --- would be plotted
instead.  Measuring X and/or Y values may be done using \verb+XV+, \verb+YV+,
or \verb+XYV+, using the graphics cursor.  Exiting from either of these
routines can be done by hitting the cursor twice at the same point.

Crude measurements of the flux and equivalent widths may be performed using
\verb+FLUX+ and \verb+EW+ respectively.  In each case, a linear continuum is
defined between the two cursor hits delineating the range over which the
measurement is made.  However, a more sophisticated method is to flatten
the continuum, by fitting a polynomial to it and then dividing it out.
This uses \verb+CREGS+, \verb+PF+, and \verb+ADIV+, and if \verb+EW+ is used
immediately afterwards, an error estimate is given.  A more detailed discussion
is beyond the scope of this document.

Saving the stack data can be done with \verb+SAVE <filename>+\@.  It can be
restored in a new session with the command \verb+RESTORE <filename>+\@.

CAUTION: as with IUEDR, $<$CTRL$>$-C and $<$CTRL$>$-D should never be used
since all of the data analysed in the current session will be lost.
Only in desperate circumstances should $<$CTRL$>$-C be hit, which will cause
the stack to be saved in CRASH.STK, but the current arrays will be lost.

\section{Calibrating spectra}

Low-resolution spectra are both flux- and wavelength-calibrated.  However, while
the standard calibration provided by IUEDR is adequate for many purposes, there
are problems.  In all but the shortest exposures, there will be an emission
line at 1215.7\AA\ from geocoronal Lyman~$\alpha$, but in some spectra it may
appear anything up to 8\AA\ either side of this wavelength. Given that the
emission is extended, this cannot be due to a velocity shift artificially
induced by the object not being correctly centred in the aperture.  One
solution is to apply a single wavelength shift to the whole spectrum, although
the correct procedure may not be so simple.  The reason is that the calibration
lamp on IUE has few emission lines visible in low-resolution mode shortward of
1480\AA , and none below 1380\AA\ --- so the 8\AA\ error may be because of a
failure to correctly extrapolate the wavelength scale towards 1200\AA\@.
Applying the constant shift may be done in 
\xref{DIPSO}{sun50}{} by using \verb+XV+ to
ascertain its value, followed by \verb+XADD+ or \verb+XSUB+\@.

\begin{latexonly}
The flux calibration provided by IUEDR takes into account the temperature
dependence (with the \verb+THDA+ parameter) and the gradual loss of sensitivity
over time (using the date given in \verb+READIUE+)\@.  However, it has recently
been found that there are errors $\sim 20$\% compared to model continua of
white dwarfs, so that a further wavelength-dependent correction needs to be
applied (Finley, D.~S., Basri, G., Bowyer, S., 1990, {\it Astrophys. J.,} {\bf
359}, 483). These have been provided for all three fully-operative cameras
(SWP, LWP and LWR), and are available in the directory
\verb+$IUEDR_SG7+ which is defined when \verb+iuedrsetup+ is typed.
To use on a given low-resolution dataset which has already been pushed onto the
stack, place the correction spectrum in the current arrays (either by reading
the file from the directory or using \verb+POP+), and then use
\verb+ADIV+\@.
\end{latexonly}

\begin{htmlonly}
The flux calibration provided by IUEDR takes into account the temperature
dependence (with the \xref{\verb+THDA+}{sg3}{THDA} parameter) and the gradual 
loss of sensitivity
over time (using the date given in \xref{\verb+READIUE+}{sg3}{READIUE})\@.
However, it has recently
been found that there are errors \begin{rawhtml}~20%\end{rawhtml} compared to
model continua of white dwarfs, so that a further wavelength-dependent
correction needs to be applied (Finley, D.~S., Basri, G., Bowyer, S., 1990,
{\it Astrophys. J.,} {\bf 359}, 483). These have been provided for all three
fully-operative cameras (SWP, LWP and LWR), and are available in the directory
\verb+$IUEDR_SG7+ which is defined when \verb+iuedrsetup+ is typed.
To use on a given low-resolution dataset which has already been pushed onto the
stack, place the correction spectrum in the current arrays (either by reading
the file from the directory or using \verb+POP+), and then use
\verb+ADIV+\@.
\end{htmlonly}

\section{Lines found in low-resolution spectra}

\begin{latexonly}
There are only a few lines that are strong enough to be detected in
low-resolution data.  They are summarised in Table~\ref{ta:one}, with
laboratory
wavelengths used.  For high-resolution data, it is best to use a more detailed
line-list, such as Howarth, I.D.\ and Phillips, A.P., 1986, {\it Mon.\ Not.\
R.\ astr.\ Soc.}, {\bf 222}, 809, or Morton, D.C., and Smith, W.H., 1973,
{\it Astrophys.\ J.\ Suppl.}, {\bf 26}, 333, although while both are valuable
sources of information, neither are in any sense complete.  The information on
AGNs is taken from Francis, P.J.~et al., 1991, {\it Astrophys. J.,} {\bf 373},
465.

\begin{table}
\caption[Lines found in low-resolution IUE spectra]
{Lines found in low-resolution IUE spectra.}
\begin{tabular}{lll} \hline
Wavelength & Ion & Comments \\ \hline
1216   & H I & Geocoronal emission + everything else! \\
1241   & N V & Doublet at 1238.8 and 1242.8. Phot, C/S, PC, AGN. \\
1260   & Si II & I/S \\
1300   & Si III & Sextuplet. Phot. \\
1335   & C II  & Interstellar lines at 1334.5 and 1335.7. \\
1371   & O V   & Phot, PC \\
1399   & Si IV & Doublet at 1393.8 and 1402.8. Phot, C/S, AGN. \\
1549   & C IV & Doublet at 1548.2, 1550.7. Phot, C/S, AGN, PN, PC. \\
1640   & He II & Balmer $\alpha$. Phot, AGN, PN, PC.  \\
1663   & O III] & PN, AGN.\\
1718   & N IV & PN, PC. \\
1750   & N III] & PN. \\
1858   & Al III & AGN. \\
1908   & C III] & PN, AGN. \\
2326   & C II]  & PN, AGN. \\
2423  & [Ne IV] & PN, AGN. \\
2798  & Mg II & I/S, PN, AGN. \\
\multicolumn{2}{c}{Fe II blends in AGNs:} & \\
Start & Finish & \\
1610 & 2210 & \\
2210 & 2730 & \\
2960 & 4040 & \\ \hline
\end{tabular}
\label{ta:one}
\end{table}
\end{latexonly}

\begin{htmlonly}
There are only a few lines that are strong enough to be detected in
low-resolution data.  They are summarised in the Table below with laboratory
wavelengths used.  For high-resolution data, it is best to use a more detailed
line-list, such as Howarth, I.D.\ and Phillips, A.P., 1986,
{\it Mon.\ Not.\ R.\ astr.\ Soc.},
{\bf 222}, 809, or Morton, D.C., and Smith, W.H., 1973, {\it
Astrophys.\ J.\ Suppl.}, {\bf 26}, 333, although while both are valuable
sources of information, neither are in any sense complete.  The information on
AGNs is taken from Francis, P.J.~et al., 1991, {\it Astrophys. J.,} {\bf 373},
465.

\begin{rawhtml}
<PRE>
<B>Wavelength  Ion      Comments</B>
   1216     H I      Geocoronal emission + everything else!
   1241     N V      Doublet at 1238.8 and 1242.8. Phot, C/S, PC, AGN.
   1260     Si II    I/S.
   1300     Si III   Sextuplet. Phot.
   1335     C II     Interstellar lines at 1334.5 and 1335.7.
   1371     O V      Phot, PC.
   1399     Si IV    Doublet at 1393.8 and 1402.8. Phot, C/S, AGN.
   1549     C IV     Doublet at 1548.2, 1550.7. Phot, C/S, AGN, PN, PC.
   1640     He II    Balmer alpha. Phot, AGN, PN, PC.
   1663     O III]   PN, AGN.
   1718     N IV     PN, PC.
   1750     N III]   PN.
   1858     Al III   AGN.
   1908     C III]   PN, AGN.
   2326     C II]    PN, AGN.
   2423     [Ne IV]  PN, AGN.
   2798     Mg II    I/S, PN, AGN.

<B>Fe II blends in AGNs:</B>
  <B>Start   Finish</B>
   1610    2210
   2210    2730
   2960    4040
</PRE>
\end{rawhtml}
\end{htmlonly}

{\bf Notes on typical occurrence:}

\begin{itemize}
\item Phot =  Photospheres of hot stars, {\it{e.g.,}}\ white dwarfs.
\item C/S = circumstellar material around hot stars.
\item I/S = interstellar medium.
\item PN = planetary nebulae (emission lines).
\item PC = P Cygni profiles in PNs.
\item AGN = Active Galactic Nuclei (emission lines).
\end{itemize}

\section{Extracting high-resolution data}

\begin{latexonly}
Spectra from the IUE \'{e}chelle are complicated not only because of the number
of orders ($\sim 60$ instead of 1), but because at the short-wavelength end
the wings of these orders overlap. Consequently, there is no absolute flux
calibration, and providing an approximate one is something of a black art.
\end{latexonly}

\begin{htmlonly}
Spectra from the IUE \'{e}chelle are complicated not only because of the number
of orders (\begin{rawhtml}~60\end{rawhtml} instead of 1), but because at the
short-wavelength end the wings of these orders overlap. Consequently, there is
no absolute flux calibration, and providing an approximate one is something of
a black art.
\end{htmlonly}

\xref{\verb+TRAK+}{sg3}{TRAK} is generally able to find the centroid of a 
low-dispersion
spectrum without difficulty, but without guidance in high-dispersion the
performance is unreliable.  Consequently, it is necessary to 
\xref{\verb+SCAN+}{sg3}{SCAN} the
spectrum, and use \xref{\verb+CGSHIFT+}{sg3}{CGSHIFT} to locate the peaks of 
one of the orders.
It is best to do this at the short-wavelength end, and
\xref{\verb+CULIMITS+}{sg3}{CULIMITS} should probably be used with 
\xref{\verb+PLSCAN+}{sg3}{PLSCAN} to expand the plot to a useful scale.
One flaw with \verb+CGSHIFT+ is that the shift used subsequently by \verb+TRAK+
is the last one obtained: it does not take an average obtained from several
orders.  Consequently, the best approach is to click the cursor on several
orders and find the one that is most representative.

\begin{verbatim}
> SCAN PR
DATASET - Dataset name. /'LT5'/ > <CR>
SCANDIST - Distance of HIRES scan from faceplate centre (geometric pixels). /0/
> <CR>
SCANAV - Averaging filter FWHM for image scan (geometric pixels). /5/ > <CR>
ORDERS - This delineates a range of Echelle orders. > 66,125
 Scan grid (-255.7,349.3, 0.5), offset 0.0, half-width 5.0
 Includes echelle orders (66,125).
> PLSCAN             *** all orders from 66 to 125 displayed
> CULIMITS           *** this was used to locate the peaks at the short
> PLSCAN                 wavelength end.
> CGSHIFT
 (nearest_order, R, W) = (108,-1.51,1279.367)
 Relative geometric shift (-1.18, 0.94)      *** these are the relevant
 Absolute geometric shift (-1.18, 0.94)          figures
 (nearest_order, R, W) = (106, -0.99,1303.489)
 Relative geometric shift ( -0.78, 0.62)
 Absolute geometric shift ( -0.78, 0.62)
 (nearest_order, R, W) = (105,-1.83,1315.894)
 Relative geometric shift (-1.42,1.14)
 Absolute geometric shift (-1.42,1.14)
 (nearest_order, R, W) = (108,-1.51,1279.367)
 Relative geometric shift (-1.18, 0.94)   *** this is the most typical
 Absolute geometric shift (-1.18, 0.94)       value. A better exposed
 Last Shift Retained.                         image would have a smaller
                                              range of shifts.
> TRAK PR
DATASET - Dataset name. /'LT5'/ > <CR>
GSAMP - Spectrum grid sampling rate (geometric pixels). /1.414/ > 0.707
CUTWV - Whether wavelength cutoff data used for extraction grid. /TRUE/ > <CR>
CENTM - Whether pre-existing centroid template is used. /FALSE/ > <CR>
CENSH - Whether the spectrum template is just shifted linearly. /FALSE/ > <CR>
CENSV - Whether the spectrum template is saved in the dataset. /FALSE/ > <CR>
CENIT - Number of centroid tracking iterations. /1/ > <CR>
CENAV - Centroid averaging filter FWHM (geometric pixels). /30/ > <CR>
CENSD - Significance level for signal to be used for centroids. /4/ >  <CR>
BKGIT - Number of background smoothing iterations. /1/ > <CR>
BKGAV - Background averaging filter FWHM (geometric pixels). /30/ > <CR>
BKGSD - Discrimination level for background pixels (s.d.). /2/ > <CR>
EXTENDED - Whether the object spectrum is expected to be extended. /FALSE/
> <CR>
AUTOSLIT - Whether GSLIT, BDIST and BSLIT are determined automatically. /TRUE/
> <CR>
ORDER - Echelle order number. > 115
NORDER - Number of Echelle orders to be processed. /0/ > 4
         *** the 4 orders 115 to 112 include Lyman Alpha (113 and 114)
              plus the two adjacent orders. Going from 115 to 112
              provides the correct wavelength sequence.
 Will extract echelle orders (115,112).
 Sample width  0.71 pixels ( 0.50 that of IUESIPS#1).
 Background folding FWHM 30.0 pixels, evaluated 2 times,
 with pixels outside 2.0 s.d. rejected.
 Will base initial templates on dispersion constants.
 Centroid folding FWHM 30.0 pixels, evaluated 2 times,
 with signal above 4.0 s.d. used for tracking.
 Point source Object.
 Slit determined automatically.

 Echelle Order 115
 Channels:  Object (-2.6,2.6),  Backgrounds (-3.6,-2.6),(2.6,3.6).
 Wavelength grid (1193.007,1205.005, 0.022) based on cutoff limits.
                               Background                  Object
                             Left       Right         Net       Shift
 Good pixels used             361         368        2212        2212
 Bad pixels used                0           0          13          13
 Pixels not used               16          16           0           0
 Rejected pixels                9          16           0           0
 Mean value                 697.6       635.4      1554.6       0.260
 RMS variation                0.2         0.2      1442.2       0.065
 Evaluations                    4           4           2           2
 :
 :
 :
 Echelle Order 112
 Channels:  Object (-2.7,2.7),  Backgrounds (-3.8,-2.8),(2.8,3.8).
 Wavelength grid (1224.492,1237.506, 0.022) based on cutoff limits.
                               Background                  Object
                             Left       Right         Net       Shift
 Good pixels used             386         376        2419        2419
 Bad pixels used                0           0          24          24
 Pixels not used               16          20           0           0
 Rejected pixels               16          18           0           0
 Mean value                 710.1       844.0      2659.5       0.079
 RMS variation                0.2         0.2      2572.2       0.081
 Evaluations                    4           4           2           2
 Changing ORDER and NORDER parameters.
 Spectrum Extraction Completed.
\end{verbatim}

The extracted orders, if saved (either be typing \xref{\verb+SAVE+}{sg3}{SAVE} 
or by quitting
the program) will appear on a \verb+_UES.sdf+ file.  It will appear
order-by-order, and it is generally more useful to amalgamate them into a
single spectrum.  First of all, however, it is necessary to correct the data
for a ripple that originates in the \'{e}chelle grating itself.  This is done
by using the eponymous routine \xref{\verb+BARKER+}{sg3}{BARKER}, 
which is fully automatic (see
Barker, P.K., 1984, {\it Astron.~J.}, {\bf 89}, 899).  The only parameter is
\xref{\verb+ORDERS+}{sg3}{ORDERS}, 
which should be set to include all orders to be corrected.

Converting the individual orders on a single scale involves the routine
\xref{\verb+MAP+}{sg3}{MAP}, 
which takes into account the common-wavelength region of adjacent
orders (for example, the geocoronal Lyman~$\alpha$ feature appears in both
order 113 and 114, which can be seen on any image).

\begin{latexonly}
To account for the order overlap, a first-order correction needs to be applied.
The usual assumption is that the background region is contaminated by
contributions from the two adjacent orders. At $\sim 1400$\AA\ the value of
this halation correction, \xref{\verb+HALC+}{sg3}{HALC}, is zero, which is 
included within the
default parameters of IUEDR --- it is not necessary to change this.
It rises linearly toward shorter wavelengths, and so needs to be defined at
one other point.  This is possible whenever the Lyman~$\alpha$ trough at
1215\AA\ is fully saturated, so that at the centre the flux is zero.
Therefore, the correct value of \verb+HALC+ at, say, 1200\AA\ (the default),
is the one which lifts this feature to the zero flux level. It varies
according to the individual image from about 0.1 to 0.3, but for B stars the
average is around 0.12, for O stars it is about 0.15, and for white dwarfs
about 0.20: these are useful as initial values, but it will be necessary to
iterate manually to find the optimum.
\end{latexonly}

\begin{htmlonly}
To account for the order overlap, a first-order correction needs to be applied.
The usual assumption is that the background region is contaminated by
contributions from the two adjacent orders. At
\begin{rawhtml}~1400\end{rawhtml}\AA\ the value of this halation correction,
\xref{\verb+HALC+}{sg3}{HALC}, 
is zero, which is included within the default parameters of IUEDR
--- it is not necessary to change this. It rises linearly toward shorter
wavelengths, and so needs to be defined at one other point.  This is possible
whenever the Lyman~$\alpha$ trough at 1215\AA\ is fully saturated, so that at
the centre the flux is zero. Therefore, the correct value of \verb+HALC+ at,
say, 1200\AA\ (the default), is the one which lifts this feature to the zero
flux level. It varies according to the individual image from about 0.1 to 0.3,
but for B stars the average is around 0.12, for O stars it is about 0.15, and
for white dwarfs about 0.20: these are useful as initial values, but it will be
necessary to iterate manually to find the optimum.
\end{htmlonly}

Consequently, \xref{\verb+BARKER+}{sg3}{BARKER} and \xref{\verb+MAP+}{sg3}{MAP} 
should initially be used only on
the orders containing the Lyman~$\alpha$ feature, combined with the two
adjacent ones, until a satisfactory value of \verb+HALC+ is obtained.  It is
essential that this is done, otherwise the equivalent width measurements out
to 1400\AA\ will be worthless.  A typical start to this cycle is given below.

\begin{verbatim}
> SETD HALC=0.05
> BARKER
 Orders in the range 110:116 used.
 4 orders optimised for K.
> MAP PR
DATASET - Dataset name. /'SWP14931'/ > <CR>
RM - Whether mean spectrum is reset before averaging. /TRUE/ > <CR>
 New Spectrum.
ORDERS - This delineates a range of Echelle orders. /[116,110]/ > <CR>
 Orders in the range (110,116) can be used.
ML - Wavelength grid limits for mean spectrum. > 1193,1237
MSAMP - Wavelength sampling rate for mean spectrum grid. /0.1/ > 0.04
 Wavelength grid is (1193.000, 1237.000, 0.040).
FILLGAP - Whether gaps can be filled within order. /FALSE/ > <CR>
COVERGAP - Whether gaps can be filled by covering orders. /FALSE/ > <CR>
 Gaps are not filled.
 Using order 110
 Using order 111
 Using order 112
 Using order 113
 Using order 114
 Using order 115
 Using order 116
\end{verbatim}

The output spectrum can then be examined using
\xref{\verb+PLMEAN+}{sg3}{PLMEAN}\@.  If the zero
line passes through the base of the Lyman~$\alpha$ trough, then the value of
\xref{\verb+HALC+}{sg3}{HALC} is acceptable; otherwise it will have to be 
increased if the
spectrum needs to be lifted, or lowered otherwise.  If \verb+HALC+ does need
to be altered, then the \xref{\verb+SETD+}{sg3}{SETD}, 
\xref{\verb+BARKER+}{sg3}{BARKER}, and \xref{\verb+MAP+}{sg3}{MAP}
commands will have to be repeated until a satisfactory value is found.
Occasionally, the Lyman~$\alpha$ is not saturated, which is obvious since the
profile is not flat at the base; a default value will have to be assumed.  This
is not satisfactory, but is the best that can be done.

Once \verb+HALC+ has been found, \verb+MAP+ can be run on the whole spectrum.
Thus, \\xref{verb+ORDERS+}{sg3}{ORDERS} should be set to [125,66]\@.  
Although this full spectrum
can be examined in IUEDR, it is much more convenient to output it to 
\xref{DIPSO}{sun50}{}
using \xref{\verb+OUTMEAN+}{sg3}{OUTMEAN}\@.

\begin{verbatim}
> OUTMEAN
DATASET - Dataset name. /'LT5'/ > <CR>
SPECTYPE - "SPECTRUM" file type (0=NDF, 1 or 2). /0/ > <CR>
OUTFILE - Name of output file. /'LT5M.sdf'/ > <CR>
 Writing NDF SPECTRUM File (LT5M.sdf).
\end{verbatim}

\section{Examining a full IUE spectrum with DIPSO}

There is no single command within \xref{DIPSO}{sun50}{} for displaying a 
full IUE
high-resolution spectrum on one page of output, but a procedure has been
written to do this.  However, it is necessary to normalise the spectrum,
since the scale plotted is from 0 to 1 only. A typical command sequence
might be:

\begin{verbatim}
> sp0rd lt5         *** DO NOT ``push'' data onto stack!
> dev mg100
> xr 1200 1400      *** for a hot star, this interval encompasses the peak
                        of the spectrum.
> pm                *** plots data located in current arrays, i.e. what has
                        just been read in. Suppose the maximum is 6.5...
> ydiv 7            *** resets scale so maximum is less than 1
> push
> @$IUEDR_SG7/page  *** plots out whole spectrum as 8 100A strips on a
                        single sheet of A4. The graphics device has been
                        set to ps_l within the procedure itself.
> q                 *** quit session
% lpr -Psys_laser gks74.ps
                    *** the plot file is not automatically queued.
\end{verbatim}

Once in DIPSO, line centroids and equivalent widths may be computed
easily, but the details are beyond the scope of this document.

\section{The IUE log and de-archiving images}

There is a log of all the IUE observations that is produced periodically, and
an online version is kept on the STADAT (VMS) node.  Examining this is done by
logging onto the captive account on this machine, username IUE, and using the
{\tt IUELOG option}, following the instructions given.  In addition to
the usual information (like co-ordinates, object name, exposure time etc.)
each spectrum is assigned a class between 01 and 99, so it is possible to
obtain a subset of the IUE log for one particular class of objects. The
option {\tt CLASSKEY} will provide the code used.

Two methods exist for examining spectra in the archive.  Low-resolution data
are available on-line, also on STADAT, but it is necessary to use the LEI
account.  The command needed is QUEST, which is described fully in
\xref{SUN/20}{sun20}{}.  It is often necessary to analyse the original 
images, though,
and this is unavoidable for high-resolution data.  For this, {\tt IUEDEARCH}
should be used, described in \xref{SUN/58}{sun58}{}.

\section{Future developments}

\begin{latexonly}
The method of obtaining spectra from images is currently being overhauled, so
that the whole archive can be re-processed.  The new developments have recently
been reviewed in the {\em `Evolution in Astrophysics'} proceedings (see above),
and are likely to result in an increase in the signal-to-noise ratio of $\sim
50$\%.  However, it is unlikely that the results of this will be generally
available for a couple of years.
\end{latexonly}

\begin{htmlonly}
The method of obtaining spectra from images is currently being overhauled, so
that the whole archive can be re-processed.  The new developments have recently
been reviewed in the {\em `Evolution in Astrophysics'} proceedings (see above),
and are likely to result in an increase in the signal-to-noise ratio of
\begin{rawhtml}~50\end{rawhtml}\%.  However, it is unlikely that the results of
this will be generally available for a couple of years.
\end{htmlonly}

\section*{Acknowledgements.}

We would like to thank Dave Finley, of the University of California at
Berkeley, for making the low-resolution correction spectra generally
available on Starlink.

\typeout{  }
\typeout{*****************************************************}
\typeout{  }
\typeout{Reminder: run this document through Latex twice}
\typeout{to resolve cross references.}
\typeout{  }
\typeout{*****************************************************}
\typeout{  }

\end{document}
