\documentstyle[11pt]{book}     % 10% larger letters, equns to left
\pagestyle{myheadings}
%------------------------------------------------------------------------------
\newcommand{\astdoccategory}  {ASTERIX Programmer Note}
\newcommand{\astdocinitials}  {PRG}
\newcommand{\astdocnumber}    {001}
\newcommand{\astdocauthors}   {David J Allan}
\newcommand{\astdocdate}      {27 May 1996}
\newcommand{\astdoctitle}     {ASTERIX Programming}
\newcommand{\astdocname}      {\astdocinitials /\astdocnumber}
\renewcommand{\_}             {{\tt\char'137}}

%------------------------------------------------------------------------------

\setlength{\textwidth}{160mm}           % Text width 16 cm
\setlength{\textheight}{240mm}          % Text height 24 cm
\setlength{\oddsidemargin}{0pt}         % LH margin width, -1 inch
\setlength{\evensidemargin}{0pt}        % LH margin width, -1 inch
\setlength{\topmargin}{-5mm}            %
\setlength{\headsep}{8mm}               % 
\setlength{\parindent}{0mm}

\begin{document}                                % Start document
\thispagestyle{empty}
ASTROPHYSICS and SPACE RESEARCH GROUP \hfill \astdocname \\
DEPARTMENT OF PHYSICS AND SPACE RESEARCH\\
UNIVERSITY OF BIRMINGHAM\\
{\large\bf Asterix System Note} \hfill \astdocauthors\\
{\large\bf \astdoccategory\ \astdocnumber} \hfill \astdocdate\\
\vspace{-4mm}
\rule{\textwidth}{0.5mm}
\vspace{5mm}
\begin{center}
{\huge\bf \astdoctitle}
\end{center}
\vspace{5mm}

\parskip=4.0truemm plus 0.5truemm       % Paragraph spacing

\markright{\astdocname}
\tableofcontents
\newpage

\chapter{Introduction}

\chapter{BDI -- The Binned Data Interface}

\begin{itemize}
\item  BDI knows what shape every component in an NDF should be given the
     dimensions of the data array. Checks dimensions of all file components
     on read/update. No create component routines needed.

\item  Supports access to structures of type ARRAY as if they are simple
     arrays of numbers. Lets you do 

\begin{verbatim}
        add file.axis(1).data_array 2.0 over
\end{verbatim}

     without worrying about representation of the array

\item  Fewer basic routines. Switching by object is performed using the item
     argument. Means applications can loop over file components.

\item  Consistent argument ordering

       fileid, {[<iax>], <item>}, {<mode>, [<type>]}, status

\item  Access multiple bits of data simultaneously, means fewer calls
     required.

\item  Asymmetric widths supported on all axes

\item  Can refer to axes using physical quantity codes, eg. X,Y,E,T,P for X
     position, Y position, energy, time and phase, eg.
 
\begin{verbatim}
       bdi_mapr( id, 'E_Axis_Data', 'READ', eptr, status )
\end{verbatim}
    
\item  Invented data supported for read, update and write where appropriate.

\item No create routines needed. Objects are created on demand so its harder
     to create objects which are subsequently unused

\item No common block, no limit on number of active datasets
\end{itemize}

\chapter{EDI -- The Event Data Interface}

\chapter{PSF -- Point Spread Function Access}

\section{Introduction}

The psf system provides an instrument independent interface to psf
information for ASTERIX applications. A client application
associates a psf with a dataset (\verb+psf_associ+), which can be 
a binned dataset or an event dataset. The psf system identifies the
spatial axes within the dataset (using BDI and EDI), and the energy
axis if present. The psf system returns an ADI identifier to the application
which contains the psf state variables for that dataset. Subsequent
calls to the PSF system need only supply this identifier. 

The psf system is in a fairly good state -- the only unfinished work
was to split up the \verb+psfrtn.f+ module to document the routines
properly and make debugging easier.

\section{How Does it Work?}

The psf system chooses a routine to initialise a particular psf by
looking at the name supplied at the PSF prompt by the user. If this
name is a valid abbreviation of any known psf then the initialisation
routine is called. The mapping between the full name and the routine
is defined using the internal routine \verb+PSF0_DEFPSF+ in the 
\verb+PSF_INIT+ module. The initialisation routrine has the standard
signature,
\begin{verbatim}
   SUBROUTINE psf_init_rtn( INTEGER psid, INTEGER status )
\end{verbatim}
where $psid$ is the psf identifier. The dataset with which the psf
is associated can be extracted from the \verb+FileID+ member of the
structure referenced by $psid$ (see the file \verb+$AST_ETC/psf.adi+
for the structure definition).

The psf initialisation routine's job is to ask for or locate any
additional information required to specify the psf when actual data is
requested. This extra information is known as psf {\em instance data}
as it is peculiar to the particular psf/dataset combination for which
it is computed. This data can be stored in $psid$ using the 
\verb+psf0_setid0+$t$ routines for subsequent retrieval by the routines
evaluating the psf data.

The most important job in initialisation is to associate other routines
to perform other functions in the psf system -- only one of these is
mandatory, the data routine. 

\section{The Data Routine}

The data routines are the most important part of the psf system. Given
a psf identifier, a detector position in radians, bin sizes and dimensions
a data routine returns a 2-D probability array. An offset may also be
applied to shift the psf centre from the centre of the 2-D array. It is
important to realise that when writing the data routine it could be called
with any bin size and not just some default detector pixel size, or
multiply thereof. After writing a new data routine evaluate the psf using
the CREPSF application using different bin sizes and  image extents to
check whether the normalisation is being computed correctly. Extra
information beyond that supplied as input arguments should be passed
using the psf instance data described in the previous section.

The data routine signature is,
\begin{verbatim}
   SUBROUTINE psf_data_rtn( INTEGER psid, REAL x0, REAL y0, REAL qx, REAL qy, 
                            REAL dx, REAL dy, LOGICAL integ, 
                            INTEGER nx, INTEGER ny, REAL array[],
                            INTEGER status )
\end{verbatim}
where $x0,y0$ is the detector position in radians, $qx,qy$ is the offset
from the centre of the output probability $array$ to the psf centre,
$dx,dy$ are the bin sizes in radians and $nx,ny$ the dimensions of
the output array. The $integ$ flag is historical and is not currently
used.

\section{The Psf Hint Routine}

While applications should be written in an instrument independent fashion,
it is still useful to obtain information about the properties of a
particular psf, often resulting in substantial efficiency improvements.
If a psf initialiser has specified a {\em hint routine} then the psf system
will call it in attempt to get the value of a named hint. Standard hint
names are defined in PSF\_PAR. Not all hints need be supported -- the psf
system returns both a "hint available" flag and the hint data to the
client.

The hint routine signature is,
\begin{verbatim}
   SUBROUTINE psf_hint_rtn( INTEGER psid, CHAR hintname, BYTE data[], 
                            INTEGER status )
\end{verbatim}
Each hint has a natural data type. Most are logical but a few are single
precision real numbers. Where the real number refers to a spatial scale
the units are radians. A type hint might be whether or not the psf exhibits
a particular symmetry on the detector, such as azimuthal symmetry.

\section{Energy Profiling}

The psf system provides a facility to evaluate the enclosed energy function
for a pariticular psf at any point on the detector. The default way to do
this is to evaluate the psf using the data routine in thin strips starting
at the psf position on the detector. This technique is fine as long as the
normalisation of the psf is rigorously computed in the data routine. If
this is not the case the profile routine can return garbage or even loop
indefinitely. If this happens with a new psf then either look at your
normalisation in the data routine carefully, or explicitly supply a profile
routine of your own. The latter option is often preferrable, as most
instrument calibrations produce a parameterised energy profile early on
in the calibration programme.

\section{The Closure Routine}

If the initialisation routine defines a closure routine then it will be
invoked when the application releases the psf using the
\verb+psf_release+ call. This allows any workspace allocated or files
opened to be freed or closed before the application exits. The closure
routine signature is the same as that of the initialisation routine.


\section{Psf Models}

Psf models are high level manipulations of the low level instrument psfs
supported by the psf system. There are spatial models and energy models,
and the two can be used together. The purpose of spatial modelling is
to impose granularity of psf access to speed things up. An application
like PSS which having determined using the hints system that a psf varies
across the field, is most easily written to simply evaluate the psf at
every position of interest. For some psfs this would be extremely slow,
so spatial modelling by the psf system allows grids of psfs to be set up,
the number of evaluations being reduced to the number of psfs in the grid.

Energy modelling serves a slightly different purpose. When processing
image data from an instrument with an energy dependent psf it is not
strictly correct to use a mean photon energy, even if the source spectrum
is monochromatic. The psf system CSPEC psf option allows a channel spectrum
of a source to be supplied which the high level data routine
\verb+psf_2d_data+ uses to construct a composite psf by performing a
weighted average of psfs at different energies. This is the correct way
to get the best source flux estimate for a source in 2-D image.


\section{The Energy Dimension}

If the user supplies a spectral image to the psf system (this is only done
in the cluster fitting at present) then when the data routine is called 
energy information must propogate from the client down to that point. This
is achieved using either \verb+psf_def+ or \verb+psf_defb+. In the first
case the user supplies energy and time bands (the latter are not currently
used anywhere) for psf evaluation. The alternative \verb+psf_defb+ supplies
bin numbers in the energy and time dimensions for psf evaluation. The psf
data routine for an energy dependent psf should check for the energy band psf
instance values before computing the psf. See \verb+psf_xrt_pspc+ in
\verb+psfrtn.f+ for an example. If the psf is energy dependent bu the
user supplies only an image, then the initialisation routine should
prompt for a mean photon energy.

\section{Utility Routines}

Two additional routines are supplied which don't require a psf identifier
but which might be of more general use -- they are \verb+psf_convolve+ and
\verb+psf_resample+. The first convolves a psf with a 2-D kernel. The
second performs sub-pixel resamling of 2-D psfs by X and Y offsets. This
latter routine is particularly useful in applications which must allow
sources to appear at positions other than pixel centres.

\end{document}
