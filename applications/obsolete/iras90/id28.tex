\documentstyle[11pt]{article}
\pagestyle{myheadings}

%------------------------------------------------------------------------------
\newcommand{\stardoccategory}  {IRAS90 Document}
\newcommand{\stardocinitials}  {ID}
\newcommand{\stardocnumber}    {28.0}
\newcommand{\stardocauthors}   {David S. Berry}
\newcommand{\stardocdate}      {28th June 1993}
\newcommand{\stardoctitle}     {IRAS90 -- Maintenance Guide}
%------------------------------------------------------------------------------

\newcommand{\stardocname}{\stardocinitials /\stardocnumber}
\renewcommand{\_}{{\tt\char'137}}     % re-centres the underscore
\markright{\stardocname}
\setlength{\textwidth}{160mm}
\setlength{\textheight}{230mm}
\setlength{\topmargin}{-2mm}
\setlength{\oddsidemargin}{0mm}
\setlength{\evensidemargin}{0mm}
\setlength{\parindent}{0mm}
\setlength{\parskip}{\medskipamount}
\setlength{\unitlength}{1mm}
\setlength{\unitlength}{1mm}

%------------------------------------------------------------------------------
% Add any \newcommand or \newenvironment commands here

% degrees symbol
\newcommand{\dgs}{\hbox{$^\circ$}}
% centre an asterisk
\newcommand{\lsk}{\raisebox{-0.4ex}{\rm *}}
% A kind of list item, like description, but with an easily adjustable
% item separation.
\newcommand{\menuitem}[2]
  {{\bf #1}: \addtolength{\baselineskip}{-0.4ex}
  \parbox[t]{128mm}{#2} \addtolength{\baselineskip}{0.4ex} \\ \hspace{-5pt}}
% an environment for references
\newenvironment{refs}{\goodbreak
                      \vspace{3ex}
                      \begin{list}{}{\setlength{\topsep}{0mm}
                                     \setlength{\partopsep}{0mm}
                                     \setlength{\itemsep}{0mm}
                                     \setlength{\parsep}{0mm}
                                     \setlength{\leftmargin}{1.5em}
                                     \setlength{\itemindent}{-\leftmargin}
                                     \setlength{\labelsep}{0mm}
                                     \setlength{\labelwidth}{0mm}}
                    }{\end{list}}

\begin{document}
\thispagestyle{empty}
SCIENCE \& ENGINEERING RESEARCH COUNCIL \hfill \stardocname\\
RUTHERFORD APPLETON LABORATORY\\
{\large\bf IRAS90\\}
{\large\bf \stardoccategory\ \stardocnumber}
\begin{flushright}
\stardocauthors\\
\stardocdate
\end{flushright}
\vspace{-4mm}
\rule{\textwidth}{0.5mm}
\vspace{5mm}
\begin{center}
{\Large\bf \stardoctitle}
\end{center}
\vspace{5mm}

\setlength{\parskip}{0mm} \tableofcontents
\setlength{\parskip}{\medskipamount} \markright{\stardocname}

\section{Introduction}
This document contains a brief description of the {\small IRAS90} directory
trees on {\small VMS} and {\small UNIX}, and descriptions of how to perform some
common maintenance tasks.

\section{Working with Two Versions of IRAS90}
An {\small IRAS90} user at a typical Starlink site will only see one version of
{\small IRAS90}, that is the version released as part of the Starlink Software
Collection. However, the maintainer of {\small IRAS90} must have at least one
other version (here called the ``master version'') from which the released
version is derived. The master version should be subject to strict quality
control, with protection measures taken to ensure that no-one (not even the
maintainer) can accidentally or intentionally add un-tested items into it. Only
thoroughly vetted modifications should be allowed into the master version. When
a significant number of such modifications have been made, a copy of the master
version should be submitted for release over Starlink so that the released
version catches up with the master version.

It is an extremely good idea to have procedures for automatically generating
both {\small UNIX} and {\small VMS} master systems from a {\em single} set of
source files held in a {\small CMS} library (or several {\small CMS} libraries).
\footnote{Since {\small CMS} is only available for {\small VMS} it may be worth
looking into equivalent {\small UNIX} tools as the days of the {\small VAX} are
numbered.}

Care must be taken by the maintainer of {\small IRAS90} to ensure that
maintenance operations are performed on the correct version (the released
version should {\em never} be modified by the maintainer). Exactly how to do
this depends on the operating system being used:

\subsection{UNIX}
On {\small UNIX} machines, files associated with the released version
are usually stored in the directory {\bf /star/iras90}. However, it is not easy
for the user to access them from this directory. To get round this, all
essential files are copied (or ``installed'') into {\bf /star/bin/iras90}. In
this context, ``essential files'' are all executable files and any other files
needed for the executable system to function correctly (i.e. data files, help
files, etc). Source files, subroutine libraries, etc, are not essential to the
executable system and are therefore left in {\bf /star/iras90}. The idea is
that, after installation, the contents of {\bf /star/iras90} could be deleted
(to save disk space for instance) without effecting the behaviour of the {\small
IRAS90} package.

There are two environment variables related to the ``source'' and the
``executable'' directories; {\small IRAS90\_DIR} is used to refer to the
directory containing the executable system, and {\small IRAS90\_SRC} is used to
refer to the source directory. If these environment variables are not defined,
then default values appropriate for the installed released version are used
(i.e. {\bf /star/iras90} for {\small IRAS90\_SRC}, and {\bf /star/bin/iras90}
for {\small IRAS90\_DIR}). A different version of {\small IRAS90} can be used by
assigning appropriate values to these environment variables. The master version
will usually contain both executable and source files in the same directory
and so {\small IRAS90\_DIR} and {\small IRAS90\_SRC} should be given the same
value. So for instance, if the master version (source {\em and} executables) is
stored in {\bf /home/dsb/iras90}, the commands:

\small
\begin{verbatim}
   % setenv IRAS90_DIR /home/dsb/iras90
   % setenv IRAS90_SRC /home/dsb/iras90
\end{verbatim}
\normalsize

should be executed before starting any maintenance. Once these commands have
been executed the master version (rather than the released version) is invoked
by the command:

\small
\begin{verbatim}
   % iras90
\end{verbatim}
\normalsize

and the command:

\small
\begin{verbatim}
   % iras90_dev
\end{verbatim}
\normalsize

then gives access to the utility libraries stored in the master version.

\subsection{VMS}
All files (source {\em and} executables) are stored in the directory pointed to
by logical name {\small IRAS90\_DIR}. This logical name usually points to the
released version, and should be re-defined to point to the master version before
performing any maintenance tasks.

\section{IRAS90 Directory Trees}
This section describes the directory trees in which the {\small IRAS90} system
is stored.

\subsection{UNIX}
The contents of the source directory depends on whether
or not the system has been ``cleaned'' (i.e. intermediate files created during
the build process have been deleted, see section \ref{SEC:BUILD}). After
cleaning, the source directory contains:

\begin{itemize}

\item The master archive {\bf iras90.a} containing all the files needed to build
the system from source.

\item The {\small IRAS90} help library {\bf iras90\_help.shl}.

\item The top-level {\small IRAS90} {\bf makefile} and the {\bf mk} script which
activates the {\bf make} utility (see section \ref{SEC:BUILD}).

\item The data files needed by various applications (template profiles, detector
{\small PSF}s, spectral response files, etc).

\item The executable files associated with {\small IRAS90},
together with the individual application compiled interface files.

\item Various sub-directories as listed below.

\end{itemize}

After cleaning, the following sub-directories are left in the source directory:
\begin{description}

\item [bin] - Contains link scripts for the {\small IRAS90} utility
libraries.

\item [doc] - Contains the documentation for the {\small IRAS90} utility
libraries.

\item [include] - Contains the public {\small FORTRAN} include files associated
with the {\small IRAS90} utility libraries.

\item [lib] - Contains the object libraries holding the
compiled {\small IRAS90} utility routines.
\end{description}

Before cleaning, the source directory also contains the following
subdirectories:
\begin{description}
\item [par] - Contains all the files associated with the {\small PAR\_} library
written by Malcolm Currie. This library will shortly be released as an
independent package on {\small UNIX} at which time it can be removed from the
{\small IRAS90} system.
\item [ndg] - Contains all the files associated with the {\small NDG\_} library.
This library will eventually be released as an independent package on {\small
UNIX} at which time it can be removed from the {\small IRAS90} system.
\item [i90] - Contains all the files associated with the {\small I90\_DAT} file
which defines various {\small IRAS90} constants.
\item [ira] - Contains all files associated with the {\small IRA\_} astrometry
library.
\item [irc] - Contains all files associated with the {\small IRC\_ CRDD} handling
library.
\item [iri] - Contains all files associated with the {\small IRI\_} image handling
library.
\item [irm] - Contains all files associated with the {\small IRM\_} miscellaneous
routines library.
\item [irq] - Contains all files associated with the {\small IRQ\_} quality
library.
\end{description}

Each of these ``sub-system'' directories contains an archive file containing all the source
files, etc, for the relevant sub-system. They also contain a {\bf makefile} and
{\bf mk} script for making the sub-system. The main {\small IRAS90} makefile
(contained in the source directory)
``calls'' these sub-system specific makefiles to make the sub-system. The
necessary sub-system files are then moved to their permanent resting places
(in the other sub-directories; {\bf bin}, {\bf lib}, etc), so that they are not
deleted when the system is cleaned.

When the released system is ``installed'' all the executable files, interface
files, data files, help files and scripts are moved into other sub-directories
of {\bf /star}. Soft links to the moved files are left in the source directory,
so it {\em looks} as if the files are still there.


\subsection{VMS}
The main {\small IRAS90} directory contains:

\begin{itemize}
\item All the individual application executable files and compiled interface
files.
\item Command procedures for building the system.
\item The data files needed by various applications.
\item The monolith executable image and compiled interface files.
\item A file holding the {\small LSE} templates for all the utility routines.
\item The help library (in Starlink help library format).
\item Options files for linking with the {\small IRAS90} libraries.
\item Text libraries holding all the source files.
\item An object libraries holding the compiled application-specific routines.
\item A sub-directory for each of the utility libraries {\small I90, NDG, IRA,
IRC, IRI, IRQ} and {\small IRM}.
\end{itemize}

\section{Building the System}
\label{SEC:BUILD}
This section describes how to build the {\small IRAS90} system. Building the
system consists of creating an executable system from the source code.

\subsection{UNIX}
The {\bf make} utility is used to build the {\small IRAS90} system on {\small
UNIX} systems. Before making {\small IRAS90}, the environment variable {\small
SYSTEM} must be defined holding a string which identifies the target machine.
At the moment, {\small SYSTEM} can take either of the values ``sun4'', or
``mips''. Thus if working on a DECstation, the command

\small
\begin{verbatim}
   % setenv SYSTEM mips
\end{verbatim}
\normalsize

must be performed before using {\bf make}. The {\bf make} utility is not
activated directly by the user, but by a script called {\bf mk}. This script
checks the value of environment variable {\small SYSTEM} and sets up several
other environment variables to appropriate values, before activating {\bf make}.
What exactly {\bf make} does is determined by the first argument supplied on the
{\bf mk} command line. It can take various values; the most important ones are:

\begin{description}
\item [build] - This causes the executable files (etc) to be created from the
source files. The created files are left in the source directory. Supplying no
argument to the {\bf mk} command has the same effect.

\item [install] - This causes executable files and all related files to be moved
to the installation directory. The released version is installed into {\bf
/star/bin/iras90} but some other installation direcetory can be used if
necessary by defining an environment variable {\small INSTALL}. The system is
installed into a subdirectory called {\bf bin/iras90} within the directory
specified by {\small INSTALL}. Soft links to the moved files are left in the
source directory.

\item [clean] - This removes all intermediate files from the source directory
which were created by the build process.

\item [deinstall] - This performs the reverse operation to ``install''; the soft
links to the executable files are removed from the source directory and the
actual files are moved back from the installation directory.

\end{description}

A typical build sequence on a DECstation would be (assuming the current default
directory is the source directory):

\small
\begin{verbatim}
   % setenv SYSTEM mips
   % mk build
   % mk clean
\end{verbatim}
\normalsize

The {\bf install} and {\bf deinstall} targets are not usually needed for
maintenance purposes as they only related to the released system.


\subsection{VMS}
The commands:

\small
\begin{verbatim}
   $ SET DEF IRAS90_DIR
   $ @BUILD
\end{verbatim}
\normalsize

will compile and link all {\small IRAS90} sub-systems and applications using the
source code contained in the various text libraries included in the system.
Other command procedures exist in the {\small IRAS90\_DIR} directory for
building individual sub-systems, individual applications, etc. The names of
these command procedures all start with the string ``{\small BUILD\_}''

\section{Modifying an Existing Application}
The {\small IRAS90} programmers guide (SUN/165) contains a description of the
location of source code within the {\small IRAS90} system, which should be read
before this section.

\subsection{UNIX}
\begin{enumerate}
\item Ensure that the environment variable {\small IRAS90\_DIR} points to the
source directory for the {\em master} version of the system (i.e. {\em not} the
released copy in {\bf /star/iras90}).
\item Extract the source code modules to be modified. For instance, if a
modification is to be made to {\small TRACECRDD}, first extract the A-task
routine:

\small
\begin{verbatim}
   % ar x $IRAS90_DIR/iras90.a tracecrdd.f
\end{verbatim}
\normalsize


Examining {\bf tracecrdd.f} will show which other {\small TRACECRDD}-specific
routines need to be modified. These can be extracted from the archive file {\bf
iras90.a} using a similar {\bf ar} command. All routines associated with each
application have the same first four letters in their names (assuming the
programming conventions described in SUN/165 have been adhered to). Thus, to
extract {\em all} files associated with {\small TRACECRDD} (which start
with the string ``{\bf trac}'') the following command could be used:

\small
\begin{verbatim}
   % ar x $IRAS90_DIR/iras90.a `ar t $IRAS90_DIR/iras90.a | grep trac`
\end{verbatim}
\normalsize

Note the use of left quotes (\verb+`+) rather than the more common right quotes
(\verb+'+).

\item Edit the source files to include the required modifications.

\item Compile and link the modified application following the instructions
in SUN/165.

\item Test the modified application. Remedy any problems which are found.

\item When you are happy that the modified files are correct, put them back into
the archive file {\bf iras90.a}:

\small
\begin{verbatim}
   % ar r iras90.a traca0.f
   % ar r iras90.a tracb3.f
     ...
     (etc, for all the modified files)
\end{verbatim}
\normalsize

\item Build the entire system using the procedure described in section
\ref{SEC:BUILD}.

\item Record the changes made in the ``system maintenance log file'' so that the
current state of the system (compared with the released system) can always be
determined.

\end{enumerate}

\subsection{VMS}
\begin{enumerate}

\item Define a job logical name {\small IRAS90\_DIR} pointing to the
{\em master} version of the system (i.e. {\em not} the
released copy in the starlink tree).

\item Extract the source code modules to be modified. For instance, if a
modification is to be made to {\small TRACECRDD}, first extract the A-task
routine:

\small
\begin{verbatim}
   $ LIB/EXT=TRACECRDD/OUT=TRACECRDD.FOR IRAS90_DIR:IRAS90.TLB
\end{verbatim}
\normalsize

Examining {\small TRACECRDD.FOR} will show which other {\small
TRACECRDD}-specific routines need to be modified. These can be extracted from
the text library {\small IRAS90\_DIR:IRAS90\_SUBS.TLB} using a similar {\small
LIB} command. All routines associated with each application have the same first
four letters in their names (assuming the programming conventions described in
SUN/165 have been adhered to). Thus, to extract {\em all} files associated with
{\small TRACECRDD} (which start with the string ``TRAC'') the following command
procedure could be used (where the first parameter, P1, is given the value
``{\small TRAC}''):

\small
\begin{verbatim}
   $  IF P1 .EQS. "" THEN INQUIRE P1 "Subroutine prefix"
   $  IF P1 .EQS. "" THEN EXIT
   $  LIB/LIST=LIB.LIS IRAS90_DIR:IRAS90_SUB.TLB
   $  SEA/OUT=ROUTINES.LIS LIB.LIS 'P1'
   $  DELETE LIB.LIS;0
   $  CLOSE/NOLOG FILE
   $  OPEN FILE ROUTINES.LIS
   $LOOP:
   $  READ/END=NO_MORE FILE NAME
   $  WRITE SYS$OUTPUT " Extracting ",NAME,".FOR"
   $  LIB/EXT='NAME'/OUT='NAME'.FOR IRAS90_DIR:IRAS90_SUB.TLB
   $  GOTO LOOP
   $NO_MORE:
   $  CLOSE FILE
   $  DELETE ROUTINES.LIS;0
   $  EXIT
\end{verbatim}
\normalsize

\item Edit the source files to include the required modifications.

\item Compile and link the modified application following the instructions in SUN/165.

\item Test the modified application. Remedy any problems which are found.

\item When you are happy that the modified files are correct, put them back into
the text libraries {\small IRAS90\_DIR:IRAS90\_SUB.TLB} and {\small
IRAS90\_DIR:IRAS90.TLB} (and {\small IRAS90\_DIR:IRAS90\_IFL.TLB} if the interface
files has been modified):

\small
\begin{verbatim}
   $ LIB IRAS90_DIR:IRAS90.TLB     TRACECRDD.FOR
   $ LIB IRAS90_DIR:IRAS90_SUB.TLB TRACA0.FOR
   $ LIB IRAS90_DIR:IRAS90_SUB.TLB TRACB3.FOR
     ...
     (etc, for all the modified files)
     ...
\end{verbatim}
\normalsize

\item Build the modified system. The command procedure {\small
IRAS90\_DIR:BUILD\_USER\_SYS} can be used to do this. This builds the system
{\em excluding} the utility routine sub-systems (which are not modified by the
above steps and so don't need to be re-built).

\item Record the changes made in the ``system maintenance log file'' so that the
current state of the system (compared with the released system) can
always be determined.

\end{enumerate}

\section{Adding a New Application}
New applications can be written following the guide lines in SUN/165. They
should be checked for conformity to the {\small IRAS90} programming conventions
listed in that document, and should be thoroughly tested before including them
in the {\small IRAS90} system as described below.

\subsection{UNIX}
\begin{enumerate}
\item Ensure that the environment variable {\small IRAS90\_DIR} points to the
source directory for the {\em master} version of the system (i.e. {\em not} the
released version).

\item Put the interface file (in stream\_LF format) the A-task routine, and all
the application-specific routines into the archive file {\bf iras90.a}. Thus, if
the new applications is called {\bf newapp}:

\small
\begin{verbatim}
   % ar r iras90.a newapp.f
   % ar r iras90.a newapp.ifl
   % ar r iras90.a newaa0.f
   % ar r iras90.a newaa1.f
      ...
     (etc, for all the applications source files)
      ...
\end{verbatim}
\normalsize

Note, application-specific include files should be included in {\bf iras90.a} but
should have no ``file type'', i.e. if the source code includes the line

\small
\begin{verbatim}
         INCLUDE 'NEW_COM'
\end{verbatim}
\normalsize

then the include file should be called {\bf new\_com} (lower case) and should be
stored in {\bf iras90.a} with the other application-specific files.


\item Extract four files from the master archive as follows:

\small
\begin{verbatim}
   % ar x $IRAS90_DIR/iras90.a iras90_help.hlp
   % ar x $IRAS90_DIR/iras90.a iras90_pm.f
   % ar x $IRAS90_DIR/iras90.a makefile
   % ar x $IRAS90_DIR/iras90.a iras90
\end{verbatim}
\normalsize

\item Process the application using {\small PROHLP} (see SUN/110) to obtain
a help module ({\bf newapp.hlp}) for the application.

\item Edit the help file {\bf iras90\_help.hlp} by including the help module
from the new application at the correct alphabetical position among the other
application help modules.

\item Edit {\bf iras90\_pm.f} (the pseudo-monolith source file) to include an
IF-clause for the new application in the form:

\small
\begin{verbatim}
         ELSE IF ( NAME .EQ. 'NEWAPP' ) THEN
            CALL NEWAPP( STATUS )
\end{verbatim}
\normalsize

\item Edit {\bf makefile}, making the following changes:
\begin{itemize}
\item Add the new application name to the list of applications
assigned to the macro ``{\small APPLICATIONS}''. The name should be terminated
with a period.
\item If the application has any private include files associated with it, add
them to the end of the list assigned to the macro ``{\small PRIVATE\_INCLUDES}''
(in lower case). Also add them to the end of the list assigned to the macro
``{\small INCLUDE\_LINKS}'' (in upper case). Add a new target for each include
file, modelling it on an existing one such as the target for include file
{\small TRA\_COM}:

\small
\begin{verbatim}
   #
   #  IRAS90 application-specific include files.
   TRA_COM:                      tra_com;   $(SYM_LINK)
\end{verbatim}
\normalsize

\item Add the names of the A-task routine and all the applications specific
routines to the end of the macro ``{\small ROUTINES}''. The names should be in
lower case and should be terminated with a period.

\item Add targets describing the dependencies of each routine on any include
files. These targets should be appended to the (long) list at the end of the
makefile.

\small
\begin{verbatim}
   newapp.o:  SAE_PAR
   newapp.o:  MSG_PAR
   newapp.o:  PAR_ERR
   newaa0.o:  SAE_PAR
   newaa0.o:  DAT_PAR
   newaa1.o:  SAE_PAR
   newaa1.o:  IRA_PAR
    ...
    (etc, for all the applications source files)
    ...
\end{verbatim}

\item A check should be made to ensure that all the include files referenced in
the list of dependencies at the end of the makefile have corresponding targets
of the form:

\small
\begin{verbatim}
   #
   #  Starlink include files.
   DAT_PAR:       $(STAR_INC)/dat_par;      $(SYM_LINK)
   MSG_PAR:       $(STAR_INC)/msg_par;      $(SYM_LINK)
   NDF_PAR:       $(STAR_INC)/ndf_par;      $(SYM_LINK)
   ...
   (etc, for all include files)
   ...
\end{verbatim}
\normalsize

\end{itemize}

\item Edit the {\bf iras90} script to add two alias definitions for the new
application:

\small
\begin{verbatim}
   alias newapps  $IRAS90_DIR/newapps
   ...
   alias iras90_newapps  $IRAS90_DIR/newapps
\end{verbatim}
\normalsize

\item Replace all modified items back in the master archive file:

\small
\begin{verbatim}
   % ar r $IRAS90_DIR/iras90.a iras90_help.hlp
   % ar r $IRAS90_DIR/iras90.a iras90_pm.f
   % ar r $IRAS90_DIR/iras90.a makefile
   % ar r $IRAS90_DIR/iras90.a iras90
\end{verbatim}
\normalsize

\item Build the entire system using the procedure described in section
\ref{SEC:BUILD}.

\item Edit {\small SUN163.TEX} to include the new application in the following
sections:
\begin{itemize}
\item ``An Alphabetical Summary of {\small IRAS90} Commands''
\item ``Classified {\small IRAS90} Commands''
\item ``Specifications of {\small IRAS90} Applications'' - you will need to
generate {\small LATEX} source using {\small PROLAT} (see SUN/110).
\end{itemize}

\item Record the changes made in the ``system maintenance log file'' so that the
current state of the system (compared with the released system) can
always be determined.

\end{enumerate}

\subsection{VMS}
\begin{enumerate}
\item Define a job logical name {\small IRAS90\_DIR} pointing to the
{\em master} version of the system (i.e. {\em not} the
released copy in the starlink tree).

\item Put the interface file into text library {\small
IRAS90\_DIR:IRAS90\_IFL.TLB}.

\item Put the A-task routine into text library {\small IRAS90\_DIR:IRAS90.TLB}.

\item Put all the application-specific routines into text library
{\small IRAS90\_DIR:IRAS90\_SUB.TLB}.

\item Any application-specific include files should be left ``loose'' (i.e. not
stored in a text library) in {\small IRAS90\_DIR}.

\item Edit the file {\small IRAS90\_DIR:BUILD\_ATASKS.COM}, adding a line of the
form:

\small
\begin{verbatim}
   $@BUILD_APPLICATION NEWAPP
\end{verbatim}
\normalsize

(assuming the new application is called {\small NEWAPP}).

\item Process the application using {\small PROHLP} (see SUN/110) to obtain
a help module ({\small NEWAPP.HLP}) for the application .

\item Edit the {\small IRAS90} help file {\small IRAS90\_DIR:IRAS90\_HELP.HLP}
by including the help module from the new application
at the correct alphabetical position among the other application help modules.

\item Edit {\small IRAS90\_DIR:IRAS90.FOR} (the monolith source file) to
include an IF-clause for the new application in the form:

\small
\begin{verbatim}
         ELSE IF ( NAME .EQ. 'NEWAPP' ) THEN
            CALL NEWAPP( STATUS )
\end{verbatim}
\normalsize

\item Edit the file {\small IRAS90\_DIR:IRAS90.COM} to add two symbol
definitions for the new application:

\small
\begin{verbatim}
   $      NEWAPP   :== $IRAS90_DIR:NEWAPP
   ...
   $      IRAS90_NEWAPP  :== $IRAS90_DIR:NEWAPP
\end{verbatim}
\normalsize

\item Edit the file {\small IRAS90\_DIR:IRAS90.ICL} to add {\small ICL}
command and help key definitions for the new application:

\small
\begin{verbatim}
   define newapp   iras90_dir:iras90
   ...
   define i90_newapp   iras90_dir:iras90
   ...
   defhelp newapp   iras90_dir:iras90_help
   ...
   defhelp i90_newapp   iras90_dir:iras90_help newapp
\end{verbatim}
\normalsize

\item Produce an edited version of file {\small SSC:ADAM\_PACKAGES.RNH} which
includes a summary of the new {\small IRAS90} application in the {\small IRAS90}
section (you will need to include a copy of this file in any Starlink-wide
release).

\item Build the modified system. The command procedure {\small
IRAS90\_DIR:BUILD\_USER\_SYS} can be used to do this. This builds the system
{\em excluding} the utility routine sub-systems (which are not modified by the
above steps and so don't need to be re-built).

\item Edit {\small SUN163.TEX} to include the new application in the following
sections:
\begin{itemize}
\item ``An Alphabetical Summary of {\small IRAS90} Commands''
\item ``Classified {\small IRAS90} Commands''
\item ``Specifications of {\small IRAS90} Applications'' - you will need to
generate {\small LATEX} source using {\small PROLAT} (see SUN/110).
\end{itemize}

\item Record the changes made in the ``system maintenance log file'' so that the
current state of the system (compared with the released system) can
always be determined.

\end{enumerate}

\section{Modifying a Utility Routine}
\label{SEC:UT}
The first thing to note about all utility libraries is that they contain two
types of routines:
\begin{itemize}
\item User-callable routines. These are the routines which can be called from
application-specific code. They have names which are prefixed by the name of the
library (eg {\small IRA\_DRGRD}).
\item Internal routines. These are routines which are used internally within
the library and which should not (indeed on {\small VMS}, {\em cannot}) be
called from application-specific routines. They have names which are prefixed by the name of the
library followed by the digit ``1'' (eg {\small IRA1\_IPROJ}).
\end{itemize}

Care should be taken to keep clean interfaces between the various libraries.
There is a definite structure within this set of libraries. For instance,
{\small IRA} routines should not call {\small IRM} routines because {\small IRA}
is a ``lower level'' package than {\small IRM}. On the other hand, this means
that {\small IRM} {\em can} call {\small IRA}. The reason for this is obvious.
If there {\em were} connections in both directions then the system couldn't
easily be built. {\small IRA} would need to be built before {\small IRM} was
built so that the {\small IRA} routines called by {\small IRM} would be
available. But on the other hand, {\small IRM} would need to be built before
{\small IRA} was built so that the {\small IRM} routines called by {\small IRA}
would be available. This wouldn't be a problem but for the use of sharable
images (on {\small VMS}) and sharable libraries (on {\small UNIX}). Because of
this, the following restrictions are imposed upon inter-connections between the
various sub-systems:

\begin{itemize}
\item {\small NDG} cannot call routines from any other sub-systems.
\item {\small IRA} cannot call routines from any other sub-systems.
\item {\small IRQ} cannot call routines from any other sub-systems.
\item {\small IRC} can call routines from any other sub-system except {\small
IRM} and {\small IRI}.
\item {\small IRI} can call routines from any other sub-system except {\small
IRM} and {\small IRC}.
\item {\small IRM} can call routines from any other sub-system.
\end{itemize}

The {\small I90\_DAT} module can be used within any utility routines.

\subsection{UNIX}
The source files for each utility library are stored in an archive file with the
same name as the library (with an extension of {\bf .a}). Thus the {\small IRA}
source files are in {\bf ira.a} which is itself stored within the master archive
file {\bf iras90.a}. So for instance, if a modification was to be made to the
routine {\small IRA\_DRGRD}, the following commands would be used to get the
top level fortran source code:

\small
\begin{verbatim}
   % ar x iras90.a ira.a
   % ar x ira.a ira_drgrd.f
\end{verbatim}
\normalsize

The source file is then inspected and any necessary changes made. If changes are
required within any internal {\small IRA} routines ({\small IRA1\_BND} for
instance) you can get them out of {\bf ira.a} using commands like:

\small
\begin{verbatim}
   % ar x ira.a ira1_bnd.f
\end{verbatim}
\normalsize

The modified source files are then put back into {\bf ira.a} using the {\bf r}
option instead of the {\bf x} option on the {\bf ar} command, and finally {\bf
ira.a} is put back into {\bf iras90.a}.

Each sub-system has its own {\bf makefile} and {\bf mk} script with targets
similar to those described for the master makefile in section \ref{SEC:BUILD}.
Thus to build the modified system the following commands should be used:

\small
\begin{verbatim}
   % cd $IRAS90_DIR/ira
   % setenv SYSTEM mips
   % setenv INSTALL $IRAS90_DEV
   % mk deinstall
   % mk build
   % mk install
\end{verbatim}
\normalsize

The environment variable {\small INSTALL} is used to specify that the subroutine
library is to be installed into the {\small IRAS90} system rather than {\bf
/star/lib}.

Note, these commands only re-builds the utility libraries. The entire {\small
IRAS90} system should be re-built to ensure that the modified utility routines
are used by the applications rather than the original versions (see section
\ref{SEC:BUILD})

Changes to utility routines should be recorded in the ``system maintenance log
file'' so that the current state of the system (compared with the released
system) can always be determined. They should also be recorded in the final
section of the appendix of the ``{\small ID}'' document which describes the
utility library.

\subsection{VMS}
The source files for each utility library are stored in a text library with the
same name as the library. Thus the {\small IRA} source files are in {\small
IRA.TLB} which is stored within a sub-directory {\small [.IRA]} of the {\small
IRAS90} source directory. This directory is pointed to be logical name {\small
IRA\_DIR} (set up by the {\small IRAS90} initialisation command). So for
instance, if a modification was to be made to the routine {\small IRA\_DRGRD},
the following commands would be used to get the top level fortran source code:

\small
\begin{verbatim}
   $ IRAS90
   $ IRAS90_DEV
   $ LIB/EXT=IRA_DRGRD/OUT=IRA_DRGRD.FOR IRA_DIR:IRA.TLB
\end{verbatim}
\normalsize

The source file is then inspected and any necessary changes made. If changes are
required within any internal {\small IRA} routines ({\small IRA1\_BND} for
instance) you can get them out of {\small IRA.TLB} using commands like:

\small
\begin{verbatim}
   $ LIB/EXT=IRA1_BND/OUT=IRA1_BND.FOR IRA_DIR:IRA.TLB
\end{verbatim}
\normalsize

The modified source files are then thoroughly checked, and then put back into
{\small IRA.TLB}:

\small
\begin{verbatim}
   $ LIB IRA_DIR:IRA.TLB IRA_DRGRD.FOR
   $ LIB IRA_DIR:IRA.TLB IRA1_BND.FOR
\end{verbatim}
\normalsize

To build the new system, use the procedure {\small BUILD.COM} within the
sub-system directory:

\small
\begin{verbatim}
   $ SET DEF IRA_DIR
   $ @BUILD
\end{verbatim}
\normalsize

The {\small VMS} system is based on the use of sharable images (following
the guide lines in SSN/8), and so the applications do not need to be re-linked
in order for the modified utility routines to take effect.

\section{Adding a new Utility Routine}
\begin{enumerate}
\item The new routines should be thoroughly tested and checked for conformity
with the conventions described in SUN/165 before adding them to the
utility libraries stored in the master version of {\small IRAS90}.

\item {\small PROLAT} (SUN/110) should be used to generate {\small LATEX}
documentation describing the user-callable routines (i.e. not any internal
routines). The documentation should be put through a spelling checker, and any
spelling mistakes should be corrected in the fortran source file {\em as well}
as the {\small .TEX} file. The corrected {\small .TEX} file should then be
included in the ``{\small ID}'' document describing the utility library. Other
sections in the document should be updated to take account of the new routines
(eg alphabetical and classified lists of subroutines, etc). The changes made to
the utility library should be recorded in the final section of the appendix of
the {\small ID} document.

\item Store the new source files in the master versions of the package's {\small
VMS} text library and {\small UNIX} archive file.

\item On {\small VMS} systems only, use {\small PROHLP} (SUN/110) to generate
help modules from the new user-callable routines, and insert these help modules
into the utility package help library:

\small
\begin{verbatim}
   $ PROHLP IRA_NEWR1.FOR IRA_NEWR1.HLP ATASK=F
   $ PROHLP IRA_NEWR2.FOR IRA_NEWR2.HLP ATASK=F
     ...
     (etc for all new user-callable routines)
     ...
   $ LIB IRA.HLB IRA_NEWR1.HLP
   $ LIB IRA.HLB IRA_NEWR2.HLP
     ...
     (etc for all new user-callable routines)
     ...
\end{verbatim}
\normalsize

\item On {\small VMS} systems only, use {\small PROPAK} (SUN/110) to generate a
new package definition file for use with {\small STARLSE}. Extract all the
{\small .FOR} files containing the user callable routines (including the new
ones) into an otherwise empty directory, and then do the following (using
{\small IRA} as an example):

\small
\begin{verbatim}
   $ PROPAK IRA_*.FOR IRA.LSE IRA IRA_DIR:IRA.HLB
\end{verbatim}
\normalsize

The last parameter is the name of the help library describing the utility
package. This help library should already exist and should contain help on the
new user-callable routines.

\item Modify the {\small VMS} file {\small BUILD\_LIBRARY.COM} stored in the
utility package sub-directory to include the new routines (user-callable {\em
and} internal).

\item Modify the {\small VMS} file {\small BUILD\_IMAGE.MAR} stored in the
utility package sub-directory to include any new user-callable routines which do
not make any internal calls to any {\small ADAM} parameter system routines.
Note, new routines should be added to the {\em end} of the list to avoid
altering the transfer vectors for any existing routines.

\item Modify the {\small VMS} file {\small BUILD\_IMAGE\_ADAM.MAR} stored in the
utility package sub-directory to include {\em all} new user-callable routines
(including those which make internal calls to {\small ADAM} parameter system
routines). Note, new routines should be added to the {\em end} of the list to
avoid altering the transfer vectors for any existing routines.

\item Modify the {\small UNIX} file {\bf makefile} stored in the
utility package archive file:

\begin{itemize}
\item Add all new routine names to the list assigned to the macro {\small ROUTINES}.
The routine names should be terminated with a period.

\item Add descriptions of all include file dependencies in the new files to the
long list at the end of the makefile.

\item Ensure that all include files referenced by the new routines have
corresponding targets of the form:

\small
\begin{verbatim}
   MSG_PAR:       $(STAR_INC)/msg_par;      $(SYM_LINK)
\end{verbatim}
\normalsize

\item Ensure that all include files are contained within the specification of
macro {\small INCLUDE\_LINKS}.

\end{itemize}

\item Check to see if the new routines need to be linked against any libraries
not already referenced by other routines. The following files should be checked
and any new libraries included in them ({\small IRA} is used as an example):

\begin{itemize}
\item {\small IRA\_DIR:BUILD\_IMAGE.COM ( VMS )}
\item {\small IRA\_DIR:BUILD\_IMAGE\_ADAM.COM ( VMS )}
\item {\bf ira\_link} ({\small UNIX})
\item {\bf ira\_link\_adam} ({\small UNIX})
\end{itemize}

The {\small UNIX} files should be extracted from the relevant archive files (eg
first get {\bf ira.a} out of {\bf iras90.a} and then get {\bf ira\_link} out of
{\bf ira.a}) and should be replaced when modified (remembering to put {\bf
ira.a} back into {\bf iras90.a}).

Note, if new routines are being added to the {\small IRM} library then the first
two {\small VMS} files should be replaced by {\small IRM\_LINK.OPT} and {\small
IRM\_LINK\_ADAM.OPT}. This is because {\small IRM} is stored as an object
library, whereas all other utility packages are stored as sharable images.

\item Re-build the package following the instructions in section \ref{SEC:UT}.

\item Changes to utility packages should be recorded in the ``system maintenance log
file'' so that the current state of the system (compared with the released
system) can always be determined.
\end{enumerate}

\section{Preparing a Release}
User opinion seems to be that Starlink-wide releases should not be made too
frequently, as it makes the users feel that ``the ground is constantly moving
under their feet''. It has been suggested that releases be made about once every
six months, although obviously it is sometimes necessary to do so more
frequently than this.

When a release is ready, a mail message should be sent to RLVAD::STAR giving the
following information:

\begin{itemize}
\item Why is the release being made? The ``system maintenance log file'' should
contain a list of all the changes that have been made to the system since the
last release, and so can be used to answer this question.
\item The location of the modified Latex files containing the up-to-date
Starlink User Notes. Make sure that the changes introduced in the new release
have been incorporated in the documentation.
\item The location of a compressed {\bf tar} file containing the source files
for the {\small UNIX} system. This can be generated by changing directory to the
source directory and then issuing the following commands:

\small
\begin{verbatim}
   % setenv SYSTEM mips
   % setenv EXPORT $HOME
   % mk export_source
\end{verbatim}
\normalsize

This produces a compressed {\bf tar} file in your login directory called
{\bf iras90.tar.Z} (this assumes you are working on a DECstation. If you are
working on a {\small SUN} then ``mips'' should be replaced by ``sun4'').
\item The location of a backup save-set containing the built {\small VMS}
system. This can be generated by setting the default directory to the master
source directory, and then issuing the commands:

\small
\begin{verbatim}
   $ BACKUP/LOG [...] SYS$SCRATCH:IRAS90.BCK/SAVE
   $ LZCMP -v SYS$SCRATCH:IRAS90.BCK SYS$SCRATCH:IRAS90_LZCMP.BCK
   $ DEL IRAS90.BCK;*
\end{verbatim}

This creates the compressed save-set {\small SYS\$SCRATCH:IRAS90\_LZCMP.BCK},
which can be uncompressed using the {\small LZDCM} command.
\item A new version of the file {\small SSC:ADAM\_PACKAGES.RNH} which includes a
summary of each {\small IRAS90} application for use in {\small ICL}.
\end{itemize}


\end{document}
