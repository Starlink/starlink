\documentstyle[11pt]{article} 
\pagestyle{myheadings}

%------------------------------------------------------------------------------
\newcommand{\stardoccategory}  {Starlink User Note}
\newcommand{\stardocinitials}  {SUN}
\newcommand{\stardocnumber}    {41.3}
\newcommand{\stardocauthors}   {C. Aspin (JAC), J.P. Emerson (QMW) \&
                               Malcolm J. Currie}
\newcommand{\stardocdate}      {30 September 1991}
\newcommand{\stardoctitle}     {IRCAM --- Infra-Red Data Reduction Package}
%------------------------------------------------------------------------------

\newcommand{\stardocname}{\stardocinitials /\stardocnumber}
\markright{\stardocname}
\setlength{\textwidth}{160mm}
\setlength{\textheight}{230mm}
\setlength{\topmargin}{-2mm}
\setlength{\oddsidemargin}{0mm}
\setlength{\evensidemargin}{0mm}
\setlength{\parindent}{0mm}
\setlength{\parskip}{\medskipamount}
\setlength{\unitlength}{1mm}

\begin{document}
\thispagestyle{empty}
SCIENCE \& ENGINEERING RESEARCH COUNCIL \hfill \stardocname\\
RUTHERFORD APPLETON LABORATORY\\
{\large\bf Starlink Project\\}
{\large\bf \stardoccategory\ \stardocnumber}
\begin{flushright}
\stardocauthors\\
\stardocdate
\end{flushright}
\vspace{-4mm}
\rule{\textwidth}{0.5mm}
\vspace{5mm}
\begin{center}
{\Large\bf \stardoctitle}
\end{center}
\vspace{5mm}

\section {Introduction}

A package of software designed to reduce,display and analyse two-
dimensional images and imaging polarimetry from the UKIRT infrared
camera, IRCAM is available. It has been written under the ADAM
environment and uses GKS~7.2 for image display and line graphical
output. This note shows how to startup this package, and describes
briefly what it does. References are given to sources of further
information. 

Since the last release the structure of the IRCAM software at JAC was
rationalised so that the Starlink release is a self-contained unit and
is the default for operation at JAC. This aids maintenance of the
software. The new version has various improvements and allows the use of
a Vaxstation running DECwindows as the image-display device (XWINDOWS),
and extra documentation on some procedures. 

\section {The IRCAM Data Reduction Package}

The IRCAM software can be installed on any Starlink node that has an ARGS,
or Ikon, or VWS/DEC-windows image-display system. If this software has been
installed at your local Starlink site, access to it can be obtained by
typing the command: 
\begin{verbatim}
      $ IRCAM_SETUP
\end{verbatim}
This defines all the symbols and logical names required to run the software.
The user interface available is
the {\em command line interface} in which you type commands when prompted
and run various options. This interface is activated by the command:
\begin{verbatim}
      $ IRCAM_CLRED
\end{verbatim}
The {\em SMS menu interface} in which you selected options from menus
generated by SMS is no longer supported in this release.

Once this commands has been typed, you are prompted for a possible
change to the logical name IMAGEDIR. This logical name must have been defined
with the /JOB qualifier. This logical name should point to the directory
containing your data files and  is where the result of all data processing will
be written.

Note any definition of the logical name IMAGEDIR which defines where the
software will search for the image data MUST use the /JOB qualifier with DEFINE
{\it e.g.}
\begin{verbatim}
      $ DEFINE/JOB IMAGEDIR DISK$USER:[ME.MYIRCAMDATA]
\end{verbatim}
When you are asked if you wish to change IMAGEDIR, you are {\em NOT\/}
warned if the default directory given in the prompt was not defined with
/JOB. Failure to define with /JOB will cause failure to access the
graphics device (a couple of minutes) after the number for the graphics
device is entered. If IMAGEDIR is undefined on setting up IRCAM there is
no problem, though it must be defined in response to the prompt. 

Once you have selected the IMAGEDIR to be used, you are prompted for the
graphics device and afterwards the hardcopy device. The system enters
the ADAMCL command language program and automatically loads the ADAM
tasks necessary to reduce/display data images. These are called RAPI2D,
OBSRAP, POLRAP and PLT2D. The first three contain all the data
reduction/analysis routines, while the fourth controls plotting on the
image display device. 

The {\em data reduction tasks} contain things like:
\begin{itemize}
\item mathematics on images,
\item smoothing images,
\item median filtering flat-field images,
\item interpolating images.
\end{itemize}
The {\em plotting task} allows you to:
\begin{itemize}
\item plot images in several different ways,
\item display line graphics such as 1D cuts/slices through images and contour
maps,
\item annotate your images.
\end{itemize}
Hardcopy can be obtained on the Canon Laser printer or other devices. 

There are also programs for converting between other file formats to the
IMAGE (.SDF) format used in IRCAM. These are: BDF2SDF and SDF2BDF (from
and to ASPIC), DST2SDF and SDF2DST (from and to Figaro) and SPECX2SDF
(from SPECX). Note that although these programs have the same names as
programs to perform similar functions elsewhere in the Starlink software
collection ({\it e.g.}\ CONVERT) the programs elsewhere will (probably)
not be capable of importing all the appropriate header information the
IRCAM software expects. 

\section {Further Information}

Documents describing the IRCAM system startup, operation, and data
formats are included with the software release. A summary of the
commands available is given in IUN2.TEX. A description of using the
infrared  camera at UKIRT, is also among the documents available. The
list (see Appendix A) can be inspected by typing from DCL: 
\begin{verbatim}
      $ TYPE LIRCAMDIR:0CONTENTS.LIST
\end{verbatim}
Copies of these documents should also be available at your Starlink site.

\section {Procedures}

Much of the IRCAM package consists of procedures (.PRC) to acquire and
pass appropriate parameters to the main programs.  You can consult
these to determine exactly how an action is carried out, and you could
write your own procedures if you are unsatisfied with the procedures
provided. In principle KAPPA (which has, in part, a common ancestry to
IRCAM) could be used to perform similar functions, but procedures to do
this in a convenient way for IRCAM data do not exist. 

\section {Using X-windows}

If you wish to display data on an X-windows device, for example a
VAXstation running DECwindows, there are a couple of extra operations you
must perform because of the inclusion of the graphics window manager,
GWM ({\it c.f.}\ SUN/130), into the GKS X-windows driver. This results in a
window with incorrect attributes for IRCAM display purposes being created as
the default.

\subsection{Creating an Xwindow}

To create an X-window with the correct attributes a command procedure has been
provided, which must be invoked {\em BEFORE\/} starting the command interface.
Run the command procedure by typing
\begin{verbatim}
      $ IRCAM_XCREATE
\end{verbatim}
The IRCAM\_STARTUP procedure reminds you to do this if necessary.

Alternatively, you can make IRCAM (and all other applications) use a window 
with the necessary attributes by either: 
\begin{verbatim}
      $ COPY LIRCAMDIR:DECW$XDEFAULTS.DAT SYS$LOGIN:*
\end{verbatim}
if {\tt SYS\$LOGIN:DECW\$XDEFAULTS.DAT} does not exist, or editing it to 
include the contents of {\tt LIRCAMDIR:DECW\$XDEFAULTS.DAT}.
This operation need only be done once, and will take effect at the next login.
However, these settings will apply to all subsequent use of the GKS
X-windows driver unless overridden by GWM command {\tt xmake}. Issuing the
{\tt IRCAM\_XCREATE} command instead prevents this side effect.

{\tt LIRCAMDIR:DECW\$XDEFAULTS.DAT}
specifies the size of the window in pixels, the number of colours
associated with the window, and the foreground and background colours.

\subsection{Destroying and Xwindow}

The second operation needs to be performed before exiting IRCAM
via the {\tt EXIT} or {\tt QUIT} command. Either
\begin{verbatim}
    Ircam-CLRED : > $ xdestroy "GKS_3800"
\end{verbatim}
or
\begin{verbatim}
    Ircam-CLRED : > $ ircam_xdestroy
\end{verbatim}
will delete the window.  If it is omitted a `phantom' window remains
on the X server, and trying to create another window of the same name
will fail, unless you logout.

\section {Relationship to KAPPA}
KAPPA (SUN/95) is the officially supported Starlink Kernel of Picture
Processing Applications, and users sometimes ask about the relationship
between IRCAM and KAPPA. The full IRCAM HDS files (the ones with a
night's observations) cannot be read by KAPPA. At present individual
IRCAM frames extracted from a night's observations (by default in IMAGE
format) are readable by KAPPA, but any arbitrary extensions ({\it i.e.}\
not in MORE) or miscellaneous items at the top level of the HDS
structure will get lost in the KAPPA processing.  At present the IMAGE
format .SDFs used in KAPPA should be able to be read into IRCAM, though
QUALITY and VARIANCE may be lost in processing. However, KAPPA is being
converted to use the new NDF format rather than IMAGE format. At present
some NDF functionality is switched off in KAPPA to allow the converted
(NDF format) and unconverted (IMAGE format) applications to work in
harmony. In the future bad-pixel flagging, origin information and
specifying the bounds after the NDF's name will be switched on in KAPPA.
Then the NDF files created in KAPPA will not work within IRCAM in its
current form.

\section {Known Deficiencies}
On the command HARDCOPY the hardcopy is automatically submitted to a
queue for printing. The queues defined in {\tt LIRCAMDIR:SETUP\_USER.COM}
do not correspond to Starlink queue names. At the moment the file is
just kept for later printing, and an error message occurs. 

Some procedures ({\it e.g.}\ DARKSUB) described in the documentation are
not in the release or have been renamed, and so the documentation may
sometimes be misleading. Some procedures which are present are for IRCAM
observing and instrument control only, and are not strictly necessary
for the Starlink release. 
\newpage
\vspace{-20mm}
\begin{center}
\large{\bf Appendix A - IRCAM Documentation Summary}
\end{center}
\small
\begin{verbatim}
IRCAM documentation in current release : directory LIRCAMDIR : 13-Jun-1991
**************************************************************************

IRCAM OPERATIONS MANUAL
-----------------------
IRCAM_MANUAL.DOC - Description of IRCAM data acquisition/simple 
                   reduction.

IRCAM REDUCTION SUMMARY
-----------------------
IRCAM_REDUCTION.NOTES - Describes the procedure for reduction of
                        'simple' IRCAM data.

IRCAM Information (file is LaTeX Files with .TEX extension)
------------------------------------------------------------
UKIRT_NEWSLETTER_FEB88 - IRCAM description from Feb 88 UKIRT Newsletter 

IRCAM USER NOTES (files are LaTeX Files with .TEX extension)
------------------------------------------------------------
IUN1 - 
IUN2.1 - Summarizes IRCAM command language data reduction commands
IUN3.1 - Summarizes IRCAM SMS menu system reduction commands 
IUN4.1 - Summarizes IRCAM observing commands
IUN5.1 - Summarizes GENPLT menu commands
IUN6.1 - Describes IRCAM end of night procedure
IUN7.1 - Describes writing IRCAM images to tape in FITS format
IUN8.1 - Describes data compression/transfer
IUN9.1 - IRCAM observers guide to operation and commands
IUN10.1 - Describes read data from archive
IUN11.0 - IRCAM observation file description
IUN12 -
IUN13.0 - IRCAM imaging polarimetry reduction procedure

IRCAM Data Reduction Program Documentation
------------------------------------------
Each file describes in detail what the program does.

ADD.DOC
AGAIN.DOC
CNSIGMA.DOC
CPLOT.DOC
DARKLOT.DOC
DEGLOT.DOC
DISP.DOC
DIV.DOC
FLATLOT.DOC
GLITCHMARK.DOC
MED3D.DOC
MEDLOT.DOC
MULT.DOC
NSIGMA.DOC
SUB.DOC
\end{verbatim}
\normalsize
\end{document}
