\documentstyle[11pt]{article}
\pagestyle{myheadings}

%------------------------------------------------------------------------------
\newcommand{\stardoccategory}  {Starlink User Note}
\newcommand{\stardocinitials}  {SUN}
\newcommand{\stardocnumber}    {22.6}
\newcommand{\stardocauthors}   {Alan J Penny \\ Bill L Martin}
\newcommand{\stardocdate}      {23 January 1996}
\newcommand{\stardoctitle}     {FORMLOAD --- \LaTeX\ Form Generator}
%------------------------------------------------------------------------------

\newcommand{\stardocname}{\stardocinitials /\stardocnumber}
\renewcommand{\_}{{\tt\char'137}}     % re-centres the underscore
\markright{\stardocname}
\setlength{\textwidth}{160mm}
\setlength{\textheight}{230mm}
\setlength{\topmargin}{-2mm}
\setlength{\oddsidemargin}{0mm}
\setlength{\evensidemargin}{0mm}
\setlength{\parindent}{0mm}
\setlength{\parskip}{\medskipamount}
\setlength{\unitlength}{1mm}

%------------------------------------------------------------------------------
% Add any \newcommand or \newenvironment commands here
%------------------------------------------------------------------------------

\begin{document}
\thispagestyle{empty}
CCLRC / {\sc Rutherford Appleton Laboratory} \hfill {\bf \stardocname}\\
{\large Particle Physics \& Astronomy Research Council}\\
{\large Starlink Project\\}
{\large \stardoccategory\ \stardocnumber}
\begin{flushright}
\stardocauthors\\
\stardocdate
\end{flushright}
\vspace{-4mm}
\rule{\textwidth}{0.5mm}
\vspace{5mm}
\begin{center}
{\Large\bf \stardoctitle}
\end{center}
\vspace{5mm}

%------------------------------------------------------------------------------
%  Add this part if you want a table of contents
\setlength{\parskip}{0mm}
\tableofcontents
\setlength{\parskip}{\medskipamount}
\markright{\stardocname}
%------------------------------------------------------------------------------

\newpage
\section{Introduction}

FORMLOAD is a programme which embeds user supplied text at positions
within suitably fashioned \LaTeX\ documents. It is thus useful for `filling
in' responses in forms and documents.

Whilst the programme is a general purpose one for use with any type of
document, it is at present supplied with the templates for telescope time
application documents.

The user may make other documents, by making her own template and input
forms, following the directions given in section~\ref{se:make}.

To simplify this description of the programme, this document will now
assume that the user intends to use it for making telescope time
application forms.

\section{Telescope Application Forms}

As of January 1996, the templates available are for three types of telescope
application forms. These are for the following:

\begin{itemize}

\item The UK Optical and Infrared telescopes (PATT2).
\item The UK Millimetre and Submillimetre telescopes (PATT3).
\item The AAT telescopes scheduling.

\end{itemize}

\section{How to use FORMLOAD }
\label{se:use}

The user places her responses into an ordinary text file. She then uses
FORMLOAD to turn that file into a `.tex' file which contains both the
structure of the desired form and her entries. She then uses \LaTeX\ in the
normal way to make a hardcopy of the form.

An additional advantage in this is that the source for the user's proposal
resides in that single text file, making modification, storage,
transmission, and re-use easy.

By using the `NULL' example file (see section~\ref{se:use}), the user can
get an empty form, which can be filled in by hand, if so desired.

Before using the programme, the user thinks of the form as a series of
questions {\em e.g.}, taking the PATT2 form, the first question on the form
is {\it `Specify one of AAT,WHT,INT,JKT,UKIRT'\/}.  Now when the user
fills in the form by hand, she answers this by putting WHT (say) in the
little box if it is WHT and so on down the form, either putting in
information or an "x" or a "yes" or "no" as prompted.

With this concept of the form, it can be seen that the answers to the
various questions can be loaded into a text file, and then a program run to
take the answers from this text file, insert them into a \LaTeX\ template,
and generate a complete form in the {\tt .tex} \LaTeX\ file
format.

Such an approach has been taken by FORMLOAD. The user has to prepare an
input text file, which has to be in a standard format (so the program can
recognise the various answers). Rather than describing this format, two
sample files have been made for each form:

\begin{center}
\begin{tabular}{|l|l|}\hline
{\it Name}        & {\it Contents}    \\ \hline \hline
\$FORMLOAD\_DIR/xxx\_sample.dat & A sample series of answers for `XXX' \\ \hline
\$FORMLOAD\_DIR/xxx\_null.dat   & A series of `null' answers for `XXX' \\ \hline
\end{tabular}
\end{center}

[The `XXX' stands for one of the three forms (PATT2, PATT3, and AAT)].

The user can take these and modify them by typing in the answers she wants
to put in, in a simple way. The `SAMPLE' file shows an example of how to
make an input file. The `NULL' file can also be used to make an empty form.
(It should be clear from the contents of the `SAMPLE' file how the
modifications are done, but section~\ref{se:details} describes this in more
detail)

FORMLOAD then turns this modified file into the desired \LaTeX\ file,
which is \LaTeX ed in the normal way.

\section{UNIX and VMS}

FORMLOAD can be run under either UNIX or VMS. The text in this document
assumes that UNIX is being used as the operating system. If VMS is being
used then the changes that should be made throughout are:

\begin{itemize}
\item {\tt \$FORMLOAD\_DIR/}should be replaced by {\tt FORMLOAD\_DIR:}
\item The names of the various files need not be in lower case.
\end{itemize}

\section{Performing a Demonstration Run}
\label{s:demo}

To demonstrate FORMLOAD, the following procedure may be used ({\tt
\%} is the Unix shell prompt):

\begin{enumerate}

\item Type:

\begin{quote}\tt
\% formload
\end{quote}

The programme then lists the types of forms available.  The program
then asks for the type of form to make. Suppose we want the  PATT2, we
then reply

\begin{quote}\tt
patt2
\end{quote}

The program then asks for the input file name. Suppose we want to use
the `sample'. We can either reply with a `carriage return', or reply
with:

\begin{quote}\tt
\$FORMLOAD\_DIR/patt2\_sample
\end{quote}

The program then asks what pages to output. Reply with a `carriage return'.

\item The program then makes a new file, in the default directory, called
{\tt patt2\_sample.tex}.

\item \LaTeX\ this new {\tt patt2\_sample.tex} file. How this is done
will depend on the site, so consult the Starlink documentation. An
example of how to do it at certain sites would be:

\begin{quote}\tt
\% latex patt2\_sample \\
\% dvips patt2\_sample \\
\% lp patt2\_sample.ps \\
(for VMS: \$ print/que=sys\_post/passall patt2\_sample.ps)
\end{quote}

\item Most sites have a means of previewing the form on the terminal
before the print out. This will use the (say) {\tt patt2\_sample.dvi}
file produced with {\tt latex patt2\_sample}. Which method of previewing
that the user selects will depend on your site, so the user should
consult the local Starlink documentation. (A useful tip with the
{\tt dvitovdu} previewer is that the output can seem too large, and then
{\tt dvitovdu/xsize=210mm/ysize=297mm} will get round that).

\end{enumerate}

\section{Input and Output for FORMLOAD }

There are thus three responses the user has to make: the type of form to
produce; the name of the input file; and the number of the pages of the
form wanted. The program then makes the output file.

\subsection{Selection of Form Type}

The programme puts out one line descriptions of the available forms, together
with a short response for choosing the type of form. There will be a
default response offered to you, which can be selected by entering the
'return', 'enter', 'carriage return' response.

(Note the possibility of changing the default or adding or removing
different types of forms from the list. This is described in
section~\ref{se:make}.)

\subsection{The Input file Name}

The user inputs the name of the input file (a default is suggested). If no
file type (the {\tt .dat} for example) is given, then the input file is
assumed to be {\tt .dat}. It is advisable that the input file does not have
the type {\tt .tex}.

\subsection{The Pages Desired}

FORMLOAD asks the user for what pages she wants it to deal with.
Normally, this is all of them, so the user replies with a carriage return or
an {\tt all}, to select the default suggested. However, the user may be just
modifying one page (say that with the scientific justification) or just two
pages, in which case she will not want FORMLOAD to make a \LaTeX\
file of the entire form. In this case, the user can specify which pages she
wants. This is done by inputting the page desired numbers (for example, for
PATT2 :- 1,2,3,4,4a,4b, and/or 5) with spaces or commas between them.

[For PATT2, the pages 4a and 4b are for the optional extra pages
of technical information and of references and figures for the scientific
justification. PATT3 "justification and technical info" are added by the
author of the application, separately].

\subsection{The Output File Name}

The program gives the output file the same name as the input file,
but places it in the current directory, with the type {\tt .tex} .

\section{Location on Output Paper}
\label{se:locate}

\subsection{Different Printers}

On some printers, the location of the printed output can vary. It is thus
possible for the output to run off the left or right sides of a page, or
off the top or bottom. To allow for this, the system manager has the option
of changing one of the files in `\$FORMLOAD\_DIR'. The user is then, whilst
running the programme, presented with a menu from which the printer the
output is intended for can be selected. The output will then have a
suitable offset for that printer.


\subsection{Cropping}

On some laser printers, particularly Postscript printer, the extreme tops
and bottoms of a page are not printed out. This has no effect, except for
the PATT3 form, where the page numbers at the bottom of the pages can be
omitted for page 3.

\section{Detailed Help on Filling in an Input File}
\label{se:details}

\subsection{A Description of the Lines}

We take the PATT2 form as an example. The other types of form are very
similar.

Looking at {\tt \$FORMLOAD\_DIR/patt2\_sample.dat}, it will first be noticed
that each question is preceded by a number. This denotes the PATT2 section,
sub-section, and question number of each question. This is followed by
`\#?\$\&', which is used, together with number, as the flag that this line
is a question start. There is one exception to this numbering scheme, the
last one `XE\_0\_00'. This refers to something that is not a section of the
PATT2 form, and thus has its own number. Its use is explained below.

Then there is a short summary of the question (compare these with those on
the form itself).

Then there is a symbol and an exclamation mark. This denotes the `type' of
answer expected and the start of the answer field.

Lastly there are the somewhat light-hearted sample answers, which are
designed to show the ranges of responses possible.

\subsection{How to Put in the Responses}

The user will make her own file ({\it e.g.}) {\tt patt2\_mine.dat} by
replacing these answers in \newline ({\it e.g.}) {\tt
\$FORMLOAD\_DIR/patt2\_sample.dat} with the ones she wants.

There are two types of questions which are shown on each line of the file
by the symbol before the !. The types are `p', and `line'. The user can
fill in something after the ! on each line. If no response is made, then
the matching question on the form will have no answer.

{\small
\begin{tabular}{|l|p{0.6in}|p{1.2in}|p{3.5in}|} \hline
{\it Type} & {\it symbol in {\tt .dat} file}   & {\it Reply normally seen on
form} & {\it Response to put into the {\tt .dat} file} \\ \hline
line  &     !  &   a line of text  &  A line of text. If the user wants to,
                                      she can enter this line after the !,
                                      and/or as new extra lines inserted
                                      after the question line.
                                      These lines must NOT begin with the
                                      7 characters of a question number
                                      (say 07\_2\_01) and the 4 characters
                                      ``\#?\$\&'' (e.g. 07\_2\_01\#?\$\&). Such a
                                      start to a line tells the FORMLOAD
                                      program that a new question is
                                      being started. It is hoped that the user will
                                      never by mistake insert such a
                                      sequence --- it seems very unlikely. \\  \hline
p     &    p!  & lines of text    &   Lines of text. If the user wants to,
                                      she can enter this line after the !,
                                      and/or as new extra lines inserted
                                      after the question line.
                                      The same prohibition on the line
                                      starting applies. \\  \hline
\end{tabular}}


An advantage of this procedure is that the user only has to insert answers
to the questions where she wants something to appear, so many of the lines
will be blank.

A further note on the `line' and `p' entries --- the user may, and in many
cases will, want to put \LaTeX\ instructions in these. The default starting
position of the text is at the top left corner of where text can go in the
form. Some of the `p' entries may seem to be more suitable to `line' mode,
but in these cases the  user will have been given the freedom to completely
fill the space available.

\subsection {Special Responses in the UK PATT forms}

In the UK PATT application forms, there are three other types, `x', `y/n',
and `n/w'.

In the `x' mode, it is recommended that only `x' be used, but any smallish
single character ({\it e.g.}~{\small Y}) will be ok.

In the `y/n' mode, no \LaTeX\ instructions are allowed, only entries of the
form `yes' or `no' are allowed. Any other entries will be searched for `y'
and `n' characters. If they are found, then `yes' or `no' will be assumed.

In the `n/w' mode, no \LaTeX\ instructions are allowed, only entries of the
form `weeks' or `nights' are allowed. Any other entries will be searched
for `n' and `w' characters. If they are found, then `nights' or `weeks' will
be assumed.


{\small
\begin{tabular}{|l|p{0.6in}|p{1.7in}|p{3.0in}|} \hline
{\it Type} & {\it symbol in {\tt .dat} file}   & {\it Reply normally seen on
form} & {\it Response to put into the {\tt .dat} file} \\ \hline
x     &    x!  &   x or nothing    &  x or nothing --- i.e. blank characters \\ \hline
y/n   &  y/n!  &   Y/N or yes/no   &  y or n \\  \hline
n/w   &  n/w!  &   N/W or nights/weeks &  n  or w \\  \hline
\end{tabular}}


\subsection{Warnings}

There are several points of which the user of FORMLOAD should be aware:

\begin{itemize}

\item The `{\tt verbatim}' environment cannot be used in the entries.

\item If the user puts too much text in anywhere, the program cannot pick
this up. The user may simply get a \LaTeX\ or \TeX\ warning, the page will
stretch itself automatically to fit the extra text, or the text will
overrun onto the areas below its proper area. Thus the user
should check that the the entries do not overflow their allotted areas, and
that the pages end up the correct length.

\item No line in the input file to FORMLOAD should be longer than 132
characters (including the question numbering, if any).

\item Remember the limits imposed by the people who read the forms on the
smallness of the fonts used.

\item \LaTeX\ assigns a special meaning to many characters. The non-user of
\LaTeX\  should be aware that use of some characters in the responses can
cause \LaTeX\ to fail to make the correct file. Some examples of this are:

\begin{center}
\begin{tabular}{|l|l|} \hline
  {\it Sign}     &  {\it Meaning assumed by \LaTeX\ } \\ \hline \hline
 `\%'            &  ignore rest of line \\ \hline
 `\$'            &  begin and end `math environment' \\ \hline
 `\{' and `\}'   &  begin and end a section of \LaTeX\ \\ \hline
 `$\backslash$'  &  \LaTeX\ command follows \\ \hline
\end{tabular}
\end{center}
\end{itemize}

So be careful about putting in these characters, and be prepared if
necessary to learn a bit about how to use \LaTeX .

\subsection{A Word of Explanation on the Question Numbering}

The numbers at the start of each question line signify which section of the
form they come from, and which sub-section and which question in that
sub-section they are. Sometimes the form has a `i)' showing the first
sub-section, sometimes nothing as the first, and sometimes nothing for the
first, then i) for the second. I have followed the forms in general, and
put 0 where the forms have put nothing. Thus the very first question in
PATT2 is 01\_1\_01. [There are four sections which PATT2 does not number,
so I have called these X1, X2, X3, and X4.]

\subsection{The Last `Question'}

At the end there is an extra question, `XE\_0\_00'. This does not
correspond to any on the forms, but can be used to add some definitions
into the \LaTeX\ file made from these questions. The answer to this last
question is inserted in the output \LaTeX\ files before the
`\verb+\begin{document}+' statement and so can be used to set up the user's
personal definitions (say a font declaration for example).

\subsection{A Possible Simplification}

It sometimes happens that the user's entry is too long to fit on the
question line. Normally then the user adds a line after the question.
However, it is sometimes clearer for editing or sharing the input file to
get it on the same line. A possible way of achieving this is to trim down
the characters of the question between the question number (say
07\_2\_01\#?\$\&) and the ! sign. These characters (which are the brief
question and the answer type) are not accessed by FORMLOAD.



\section{Changing the Possible Locations on Output Paper}
\label{se:offset}

On some printers, the location of the printed output can vary. As described
above in section~\ref{se:locate}, the user can automatically be supplied
with the correct shifts, or be provided with a choice of offsets.

To bring this about, the system manager has to change the contents of the
`offsets.dat' file. This allows the system manager to select suitable page
offsets for the printers at her site, or, if necessary, for the user to
select suitable offsets depending on which printer is being used.

The system manager should change the `offsets.dat' file as follows:-

\subsection{Single Offsets}

If all the site printers have the same problem, the output location can be
changed by just changing the three numbers on the second `single option'
line in the file. The first number is the vertical offset in centimetres
(positive moving the output down). The second number is the horizontal
offset in centimetres (positive moving the output to the right) for odd
numbered output pages. The third is the horizontal shift for even numbered
pages. The numbers should be in the same format, three digits (within which
the first may be a `negative sign') before a decimal point and two after.

\subsection{Differing Offsets}

If there are different printers with different problems, then the user has
to be presented with a choice. This is done by changing the third line in
the file from:

\begin{quote}\tt
no more     (more)
\end{quote}
to:
\begin{quote}\tt
more     (no more)
\end{quote}

Then subsequent lines should contain the various choices and the default.
Again, these must be in the same format as provided originally. The default
is in the last line.

The `choice' lines have the format:-

Text to give the user to describe the choice. This space on the line has to
fall within the `a' and `b' space of the first line. [However if desired
this space may be increased or decreased by changing the number of `b's.]

Text to give the user the code she must type to select a choice. This space
on the line has to fall within the `c' space of the first line. [However if
desired this space may be increased or decreased by changing the number of
`c's.]

The last, `default', line must contain a `y' in the first column, and then
the code for the default choice. This must be one of those defined in the
`choice' lines above. This space on the line has to fall within the `a'
space of the first line, and must be the same length as the `c' space.


\section{Use at Non-Starlink Sites}

At Starlink sites, the Site Manager will have installed the Starlink
version of FORMLOAD.  At other sites the you will need to install
FORMLOAD for yourself.

Starlink will supply FORMLOAD in a compressed tar archive which you can
ftp from the Starlink FTP server: {\tt starlink-ftp.rl.ac.uk}. For
details of the location of the archive file, please email {\tt
starlink@jiscmail.ac.uk}, since it may vary.

The file you need is \verb|formload.tar.Z| which should be binary
\verb|ftp'd| to your machine.  Once you have the archive file, proceed as
follows:

\begin{enumerate}

\item  Unpack the archive file to an empty directory:
\begin{quote}\tt
\% cd empty-directory\\
\% zcat /wherever/formload.tar.Z | tar xvf -
\end{quote}

\item  Set some environment variables appropriate to your Fortran and C
compilers, and for tar, for example:

\begin{quote}\tt
\% setenv FC f77 \\
\% setenv CC cc \\
\% setenv TAR\_OUT 'tar -xf'
\end{quote}

\item  Build the executable using the makefile:
\begin{quote}\tt
\% make -e build
\end{quote}
This will also extract the various examples and templates from the source
archive.

\item  Define the {\tt FORMLOAD\_DIR} environment variable to point to the
location of the sample and template files:
\begin{quote}\tt
\% setenv FORMLOAD\_DIR wherever
\end{quote}

\item  Either define an alias \verb|formload| to point at the executable
program, or copy the executable to a location on your PATH:
\begin{quote}\tt
\% alias formload /wherever/formload \\
{\rm or} \\
\% cp formload \$HOME/bin
\end{quote}

\end{enumerate}

You can now run FORMLOAD as detailed in Section \ref{s:demo}.

\section{How To Make A New Type of Form}
\label{se:make}

If you wish to make your own form, then you must make a suitable template
and a suitable sample file.

\subsection{The Template}

The best way to make a template is to look at one of the existing ones, and
modify it to your needs. This must be a `.tex' file.

The component parts of the template are:-


\vspace*{0.2cm}
{\bf \large 1. The `pages' section -- Mandatory}
\vspace*{0.2cm}

This section tells the programme the two character codes for the pages of
the form. Each line of the section has to have five of these codes, with
the last line padded out with blanks. Thus in the example below the page
codes are `1 ', `2 ', `3 ', `4a', `4b', and `5 '. The format must be
exactly as given below.

\begin{verbatim}
%?#$&pages
% a xx xx xx xx xx a
% a 1  2  3  4  4a a
% a 5              a
% x xx xx xx xx xx x
\end{verbatim}

\vspace*{0.2cm}
{\bf \large 2. The `questions' section -- Mandatory}
\vspace*{0.2cm}

This section tells the programme the code names of the questions.

These must be seven characters long, and in the format given below. There
must be a last question code of `XE\_0\_00' (see in the `commands' section,
below).

\begin{verbatim}
%?#$&questions
% a xxxxxxx xxxxxxx xxxxxxx xxxxxxx xxxxxxx a
% a 01_1_01 02_1_01 02_1_02 02_1_03 02_1_04 a
% a 19_0_04 19_0_05 20_0_01 XE_0_00         a
% x xxxxxxx xxxxxxx xxxxxxx xxxxxxx xxxxxxx a
\end{verbatim}

\vspace*{0.2cm}
{\bf \large 3. The `special' section -- Optional}
\vspace*{0.2cm}

This section tells the computer which questions have special responses if
the user puts an answer containing specified character strings. If no such
questions exist, this section may be omitted, or the entries themselves
omitted.

For each question marked, the specified character strings are given, and
the character string the programme should insert in the document. This
different string will be the entire input for that question. Thus any other
character strings in the input document for the question will be ignored.

The lines of this section comprise:- the question code (this must be
exactly seven characters); then sets of  pairs of character strings each
delineated by the double quotation mark `"'. Each character string can be
up to twenty characters long. Each question can occur in a number of lines,
but the total number of pairs for question is twenty.

An example of this section is:-

\begin{verbatim}
%?#$&special
% a xxxxxxx xxxxxxxxx1xxxxxxxxx2xxxxxxxxx3xxxxxxxxx4xxxxxxxxx5xxxxxxxxx6
% a 02_1_04  "y"  "\ajby"  "Y"  "\ajby"  "n"  "\ajbn"  "N"  "\ajbn"
% a 12_1_01  "y"  "\ajbyes" "Y" "\ajbyes" "n"  "\ajbno"
% a          "N"  "\ajbno"
% x xxxxxxx xxxxxxxxx1xxxxxxxxx2xxxxxxxxx3xxxxxxxxx4xxxxxxxxx5xxxxxxxxx6
\end{verbatim}

\vspace*{0.2cm}
{\bf \large 4. The `message' section -- Optional}
\vspace*{0.2cm}

This section contains a message which is put out on the user's terminal
when the template is selected. If no such output is desired, this section
may be omitted, or the entries themselves omitted.

An example is given below:-

\begin{verbatim}
%?#$&message
% a xxxxxxxxx1xxxxxxxxx2xxxxxxxxx3xxxxxxxxx4xxxxxxxxx5xxxxxxxxx6xxxxx a
% a  This PATT3 template version correct as of September 1992.        a
% x xxxxxxxxx1xxxxxxxxx2xxxxxxxxx3xxxxxxxxx4xxxxxxxxx5xxxxxxxxx6xxxxx x
\end{verbatim}

\vspace*{0.2cm}
{\bf \large 5. The `scale' section -- Mandatory}
\vspace*{0.2cm}

This section gives the scale to be applied to the numbers (in the `input'
section below) which define the locations of the answers in the output.

The numbers are, in order:- the X scale; the Y scale; the X offset; the Y
offset.

The format for these numbers is given below.

\begin{verbatim}
%?#$&scale(001.000)(001.000)(000.000)(000.000)
\end{verbatim}

The leading and trailing numerals are important. Numbers like `(1.000)'
or `(001.0)' are not permitted.

\vspace*{0.2cm}
{\bf \large 6. The `startbeg' section -- Mandatory}
\vspace*{0.2cm}

This section precedes the \LaTeX\ `documentstyle' entries. The format is:-

\begin{verbatim}
%?#$&startbeg
\end{verbatim}

\vspace*{0.2cm}
{\bf \large 7. The `offset' section -- Mandatory}
\vspace*{0.2cm}

This section gives the offsets (vertical and horizontal) that are applied
to the output. This must occur in the `documentstyle' section and must have
the format given below. The reason for the specification is that the
offsets can be changed by the use of the `offsets.dat' file (see
section~\ref{se:offset}).

\begin{verbatim}
%?#$&offset
\topmargin      -1.25in
\oddsidemargin  -0.55in
\evensidemargin -0.55in
%?#$&endoffset
\end{verbatim}

\vspace*{0.2cm}
{\bf \large 8. The `commands' section -- Mandatory}
\vspace*{0.2cm}

This section supplies the special \LaTeX\ commands that the programme
needs. This must occur in the `documentstyle' section and must have the
format given below. The font determines the default font of the inserted
text, and thus its value may have to be changed as required, but its name
must stay as `ajinpfont'.

\begin{verbatim}
\font\ajinpfont=cmr10

\newcommand{\ajnull}{\makebox[0.05in]{\phantom{a}}}

%?#$&startmid

%?#$&input XE_0_00
\end{verbatim}

\vspace*{0.2cm}
{\bf \large 9. The `startend' section -- Mandatory}

\vspace*{0.2cm}

This is the marker that tells the programme that the `documentstyle'
section has ended. All the lines above this will form the start of the
output file. It must thus appear after the `$\backslash$begin\{document\}'
statement. It must have the format of:-

\begin{verbatim}
%?#$&startend
\end{verbatim}

\vspace*{0.2cm}
{\bf \large 10. The `page' section -- Mandatory}
\vspace*{0.2cm}

This section tells the programme that what comes next is the a page of the
document. It will be repeated for each page.

It consists of the header line, where the last two characters are the
`page' code. [In this example the page code is `1 '.)

\begin{verbatim}
%?#$&page1
\end{verbatim}

After this header comes the \LaTeX\ lines you use to specify the template
of the page. They must end with a  line setting the position back at the
top of the page, and an open `\{$\backslash$ajinpfont' line. This is mandatory.

A sensible way of specify the page is to use the `picture' environment. An
example of this is:-

\begin{verbatim}
\vspace*{1.3cm}\hspace*{-0.4cm}{\setlength{\unitlength}{1.0cm}
\begin{picture}(19.00,25.80)(00.00,01.30)

\put(17.60,28.70){{\ajbf PATT 3 }}
\put(17.20,28.30){{(\bf \footnotesize \sf Millimetre \& }}
\put(17.20,28.05){{\bf \footnotesize \sf Submillimetre) }}
\put(00.20,01.30){{\footnotesize \sf May 1989 revision }}
\put(09.30,01.30){{\small \sf 1 }}

\end{picture}}

\vspace*{-26.75cm}

{\ajinpfont
\end{verbatim}

\vspace*{0.2cm}
{\bf \large 11. The `input' section -- Mandatory}
\vspace*{0.2cm}

This contains the specifications for the locations of the questions. These
must be in the format specified. It must be repeated at the end of each
page.

The entries give four numbers, specifying in centimetres:- the horizontal
location of the top left hand corner of the box containing the answer (from
the left hand side); its vertical location (from the top); its maximum
horizontal extent; its maximum vertical extent.  The entry also contains
the question code. It may also include other text, as seen in the second
example, which will be placed before or after the input text. This is
particularly useful in setting a default font size for a particular
question.

Example are:-

\begin{verbatim}
{%?#$&\bjpos(02.20)(01.75)(02.00)(03.00)
\ajnull%?#$&input 01_1_01
}

{%?#$&\bjpos(01.40)(05.65)(11.00)(03.00)\ajfrn
\ajfrL
\ajnull%?#$&input 02_2_01
}
\end{verbatim}

The leading and trailing numerals are important. Numbers like `(1.00)'
or `(01.0)' are not permitted.

This section must end with a final line:-

\begin{verbatim}
\eject
\end{verbatim}

\vspace*{0.2cm}
{\bf \large 12. The `end' section -- Mandatory}
\vspace*{0.2cm}

This contains code telling the programme the document has ended. It must be
in the format specified.

\begin{verbatim}
%?#$&end

\end{document}
\end{verbatim}

\subsection{The Input File}

The best way to make an input file is to look at one of the existing sample
files, and modify it to you needs. This must be a `.dat' file

The component parts of a sample file are the questions. These must be in
the format:-

\begin{verbatim}
01_1_01#?$& Semester                   !
\end{verbatim}

Thus the first seven characters are the question code. The next four are
check characters to designate this line as the start of a new question.
Then comes a character string (of any length), which is used to give the
user an indication of the question. This is ended with an exclamation mark
(and thus there must not be an exclamation mark in the string). Before the
exclamation mark there may be an optional character to indicate to the user
the type of response suitable for the question. The user's reply is
contained after the exclamation mark and in any succeeding lines.

The last question must be the `XE\_0\_00' one as specified above.

\begin{verbatim}
XE_0_00#?$& LaTeX setup commands      p!
\end{verbatim}

Lines at the beginning of the file, before the first question, are ignored.


\subsection{The Options File}

If you wish to have your new form made available to the user of FORMLOAD
automatically, then it should be included in the `options.dat' file. Also
the template and input file names must have a specific structure.

The format of the `options.dat' file should be clear from inspection, but a
description is given here. The first line contains the spacings. The number
of `a's, `b's, and `c's may be changed, but there may be no more than
twenty `a's or `c's. The `a's delineate the place where the last line
(which starts with a `y') will store the `default' choice. The `a's and
`b's delineate where the intermediate lines store the message where the
user is given information about the choice. The `c's delineate where the
selection code is. The intermediate lines must leave blanks in the first

An example is:-

\begin{verbatim}
x!aaaaa!bbbbbbbbbbbbbbbbbbbbbbbbbbb!ccccccccc
  U.K. PATT Optical/IR              PATT2
y PATT2
\end{verbatim}

The names of the template and input sample files are constructed from the
selection code, which is added as a prefix to the two names
`\_template.tex' and `\_sample.dat'. Thus if `PATT2' is selected, the
programme looks for the files `patt2\_template.tex' and `patt2\_sample.dat'
(in the directory \$FORMLOAD\_DIR).

\end{document}
