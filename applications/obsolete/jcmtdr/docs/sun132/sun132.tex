\documentclass[twoside,11pt,nolof]{starlink}

% +
%  Name:
%     sun132.tex
%
%  Purpose:
%     SUN documentation for JCMTDR (SUN/132)
%
%  Authors:
%     John F. Lightfoot (ATC)
%     Paul A. Harrison (Starlink)
%     Horst Meyerdierks (ATC/IfA/Starlink)
%     Tim Jenness (JAC, Hilo)

%  Copyright:
%     Copyright (C) 1989-1995,2002,2003 Particle Physics and Astronomy
%     Research Council. All Rights Reserved.
%
%  History:
%     1989-1995 Unknown
%     20 Mar 2003 (TIMJ):
%       Tweak for linux release. Add version history from HTML help.
%       Latex2e compliant
% -

% -----------------------------------------------------------------------------
% ? Document identification
\stardoccategory    {Starlink User Note}
\stardocinitials    {SUN}
\stardocsource      {sun\stardocnumber}
\stardocnumber      {132.3}
\stardocauthors
   {J F Lightfoot, P A Harrison, H Meyerdierks, T Jenness}
\stardocdate        {20 March 2003}
\stardoctitle       {JCMTDR \\[2ex] Applications for Reducing JCMT GSD Data}
\stardoccopyright   {Copyright \copyright\ 1989-2003 Particle Physics and Astronomy Research Council}
\stardocversion     {1.2-2}
\stardocmanual      {User's manual}
\stardocabstract  {
  JCMTDR is a Starlink applications that is intended to replace NOD2 for the
  reduction of continuum on-the-fly mapping data obtained with UKT14 or the
  heterodyne instruments using the IFD on the James Clerk Maxwell Telescope.
  This programme is intended for reduction of archive data and of heterodyne
  beam maps.  }
% ? End of document identification
% -----------------------------------------------------------------------------
% ? Document specific \providecommand or \newenvironment commands.

% New commands
\providecommand{\oracdr}{\xref{\textsc{orac-dr}}{sun230}{}}
\providecommand{\oracsun}{\xref{SUN/230}{sun230}{}}

\providecommand{\task}[1]{{\textsf{#1}}}
\providecommand{\recipe}[1]{{\small\textsf{#1}}}
\providecommand{\primitive}[1]{{\small\texttt{#1}}}

\providecommand{\Kappa}{\xref{{\sc{Kappa}}}{sun95}{}}
\providecommand{\Fluxes}{\xref{{\sc{Fluxes}}}{sun213}{}}
\providecommand{\SURF}{\xref{{\sc{Surf}}}{sun216}{}}
\providecommand{\gaia}{\xref{{\sc{Gaia}}}{sun214}{}}
\providecommand{\cgsdr}{\xref{{\sc{cgs4dr}}}{sun27}{}}


\providecommand{\fitsmod}{\xref{\task{fitsmod}}{sun95}{FITSMOD}}
\providecommand{\qdraw}{\xref{\task{qdraw}}{sun216}{QDRAW}}
\providecommand{\extinction}{\xref{\task{extinction}}{sun216}{EXTINCTION}}
\providecommand{\scusetenv}{\xref{\task{scusetenv}}{sun216}{SCUSETENV}}


% Environment for indenting.
\newenvironment{myquote}{\begin{quote}\begin{small}}{\end{small}\end{quote}}


% ? End of document specific commands
% -----------------------------------------------------------------------------
%  Title Page.
%  ===========
\begin{document}
\scfrontmatter

\section{Introduction\xlabel{introduction}\xlabel{JCMT_HELP}\xlabel{JCMT_XHELP}}

JCMTDR is a Starlink applications that is intended to replace NOD2 \cite{nod2}
for the reduction of continuum on-the-fly mapping data obtained with UKT14
\cite{ukt14} or the heterodyne instruments using the IFD on the James Clerk
Maxwell Telescope. This programme is intended for reduction of archive data
and of heterodyne beam maps.

For processing of SCUBA data use SURF (see \xref{SUN/216}{sun216}{}).

A cookbook is available to provide a more complete overview of data processing
with JCMTDR (see \xref{SC/1}{sc1}{}).

The Unix version should be run from the shell, although it should also
run from ICL. Make the commands available by typing:

\begin{terminalv}
% jcmtdr
\end{terminalv}

The applications have lower-case names, a fact ignored in the rest of
this document.

Help information can be obtained by typing:

\begin{terminalv}
% jcmt_help
\end{terminalv}

for command line help or

\begin{terminalv}
% showme sun132
% showme sc1
\end{terminalv}

for online web pages (either this document or the cookbook respectively).

\goodbreak

\section{The Applications\xlabel{the_applications}}

\subsection{\xlabel{AE2RD1}AE2RD1}

This application will transform JCMT map data into a tangent  plane
image centred on a specified RA, Dec. The rebinning is performed by
convolving the input data with a truncated Bessel function. It is
similar but not identical to the NOD2 CONVERT function.

The width of the Bessel function used is such that it would just be
fully  sampled by data points at the spacing of the pixels of the
output  map. The pixel spacing of the output map is set to be the
minimum  of the nominal pixel spacings of the input maps. To minimise
edge  effects the Bessel function is truncated at a radius of 7 pixels
from its centre, and apodised over its outer third by a cosine
function.

In Fourier terms the method approximates to Fourier transforming the
input dataset(s), multiplying the transform by a cylindrical top hat
(the F.T. of the Bessel function), then transforming back into image
space.

The application can read in up to 10 separate input datasets. The
output map will be large enough to cover all the input data  points
and the map centre can be specified by the user (the  default is the
map centre of the first input dataset). Each input  dataset can be
assigned a weight relative to that of the first.  Output pixels
further than one pixel spacing from any input data  point will be
flagged as a bad value. Input pixels flagged as bad will be ignored.
Input error arrays are not propagated.

The convolution assumes that all pixels outside the measured  area are
zero. This will lower the brightness of objects that are  less than
the radius of the convolution function (7 output pixels)  from the map
edge, though the effect is only a few percent unless  the object is
very close (2 or 3 pixels) to the edge.

\goodbreak

The parameters used are:

\begin{small}
\begin{terminalv}
PARAMETER  INFILE
  TYPE    CHAR
  NAME    IN(file)
  PROMPT  "(INfile) The name of a file containing JCMT map data"
  TEXT   ---------------------------------------------------
         The name of a file containing the map data. Giving
         END as the name of the file will finish the input
         sequence.
         ---------------------------------------------------

PARAMETER  OUTPUT
  TYPE    FILE
  NAME    OUT(PUT)
  OPTIONS OUTPUT
  PROMPT  "(OUTput) The name of the file to contain the output map"

PARAMETER B1950
  TYPE    KEY
  NAME    B1(950)
  PROMPT "(B1950) make output coordinates B1950 equinox"
  TEXT   ---------------------------------------------------
         By default the application will transform the data
         into a map whose centre and tangent plane offsets
         are in the FK5 system with Julian 2000 equinox.
         Setting B1950 to YES will give results in the FK4
         system with Besselian 1950 equinox.
         ---------------------------------------------------

PARAMETER WEIGHT
  TYPE   FLOAT
  PROMPT "(WEIGHT) weight of this dataset relative to the first"
  TEXT   ---------------------------------------------------
         The first dataset will be coadded into the output
         data array with unit weight. Subsequent datasets
         can be coadded in with specified weights relative
         to the first.
         ---------------------------------------------------

PARAMETER RA_CENTRE
  TYPE   CHAR
  PROMPT "(RA_centre) The RA of the centre of the output map"
  TEXT   ---------------------------------------------------
         The positions of the output map pixels will be
         given in terms of a tangent plane offset in arcsec
         from a map centre specified by the user. This parameter
         specifies the RA of the centre, it must be input in
         hh mm ss.s format.
         ---------------------------------------------------

PARAMETER DEC_CENTRE
  TYPE   CHAR
  PROMPT "(DEC_centre) The Dec. of the centre of the output map"
  TEXT   ---------------------------------------------------
         The positions of the output map pixels will be
         given in terms of a tangent plane offset in arcsec
         from a map centre specified by the user. This parameter
         specifies the declination of the centre, it must be
         input in (sign)dd mm ss.s format.
         ---------------------------------------------------
\end{terminalv}
\end{small}

\goodbreak

There is a problem with this method if the mapped area, or part of it,
was not sampled closely enough to fully sample the telescope beam.
Users should treat the resampled output for those  areas with
suspicion. In such cases it would be better to use AE2RD2, which uses
a NAG interpolation routine to perform the rebinning. \textit{AE2RD2 is not
  available in the current version due to the removal of NAG libraries.}


\goodbreak

\subsubsection {Pointing Corrections}

If required, AE2RD1 will correct the nominal offsets of the  input map
pixels for pointing errors. To make this happen you should develop
the .PCORR structure in each input map file to describe the pointing
correction required. AE2RD1 will look for a structure of the following form:

\begin{small}
\begin{terminalv}
.PCORR
   .LST   (ncorr) lst1, lst2, .... lstn
   .D_AZ  (ncorr) d_az1, d_az2, .... d_azn
   .D_ALT (ncorr) d_alt1, d_alt2, .... d_altn
\end{terminalv}
\end{small}

where \verb+ncorr+ is the number of points at which the correction has
been specified. The three arrays specify the magnitude of the correction
in azimuth and altitude (elevation) as a function of LST. The LST must be
specified in radians, the pointing corrections in arcsec. Following standard
JCMT usage, the azimuth increases from north through east. The
correction for each map point is calculated by linear interpolation
between the LSTs that bracket the LST of the measurement. If the
measurement lies outside the described  range, the correction will be
that with the nearest LST. The corrections are \emph{added} to the azimuth
and elevation of the pixels.

On a VAX the required elements in the .PCORR structure can be created and
set using the Figaro LET command (note that using LET to create an array
requires the last element of the array to be created first). On a Unix machine
you will have to use the Figaro command CREOBJ to create the HDS objects and
SETOBJ to give them values.


\goodbreak

\subsection{\xlabel{AE2RD2}AE2RD2}

\textit{Not available at this time due to removal of the NAG libraries.}

This application performs the same function as AE2RD1 but rebins each
input dataset individually using NAG interpolation routines (E01SAF
and E01SBF) before coadding them into the result. Errors associated
with the input datasets are not propagated.


\goodbreak

\subsection{\xlabel{IRAS_TAG}IRAS\_TAG}

This application will append an IRAS astrometry structure to a
rebinned map so that the IRAS90 package
(\xref{SUN/161}{sun161}{}) can be used on it,
for example to overlay coordinate grids, object positions, \emph{etc}.
This information is also readable by AST-aware \Kappa\ routines (\xref{SUN/95}{sun95}{}).

\goodbreak

\subsection{\xlabel{JCMTEXTC}JCMTEXTC}

This application corrects JCMT data for the effect of  atmospheric
extinction. The airmass at which each map pixel was  observed is
calculated, and multiplied by the estimated zenith  atmospheric
extinction at the time of observation to give the  extinction optical
depth along the line of sight. Multiplying the data point by the
exponential of the optical depth gives  the value that would have been
measured above the atmosphere.

The zenith optical depth is assumed to vary linearly with  time
between the values given by the parameters TAU and ENDTAU.

\goodbreak

The parameters used are:

\begin{small}
\begin{terminalv}
PARAMETER  INPUT
  TYPE    FILE
  NAME    IN(PUT)
  OPTIONS INPUT
  PROMPT  "(INput) file containing map data to be corrected"

PARAMETER  OUTPUT
  TYPE    FILE
  NAME    OUT(PUT)
  OPTIONS OUTPUT
  PROMPT  "(OUTput) file to contain the corrected data"

PARAMETER  TAU
  TYPE    FLOAT
  NAME    TAU
  OPTIONS REQUIRED
  PROMPT  "(TAU) initial value of the atmospheric extinction"
  TEXT    --------------------------------------------------
          The value of the zenith atmospheric extinction at
          the start of the map observation.
          --------------------------------------------------

PARAMETER  ENDTAU
  TYPE    FLOAT
  NAME    ENDTAU
  OPTIONS REQUIRED
  PROMPT  "(ENDTAU) final value of the atmospheric extinction"
  TEXT    --------------------------------------------------
          The value of the zenith atmospheric extinction at
          the end of the map observation. The zenith extinction
          for each pixel will be calculated assuming a linear
          gradient with time between the start and end values.
          --------------------------------------------------
\end{terminalv}
\end{small}


\goodbreak

\subsection{\xlabel{MAKEMAP}MAKEMAP}

This takes a JCMT data file in GSD format and converts it to a Figaro
file. It is recommended that the JOB logical name (on Unix an
environment variable) FIGARO\_FORMATS be
set  to `NDF' before running the application so that the output file
will be a Starlink NDF, readable by both \Kappa\ and Figaro. The command
has the following parameters:

\goodbreak

\begin{small}
\begin{terminalv}
PARAMETER  GSDFILE
  TYPE    CHAR
  NAME    GSD(FILE)
  PROMPT  "(GSDfile) GSD file containing JCMT data"
  TEXT    --------------------------------------------------
          On VMS, the file name has the default extension
          .DAT which should not be specified. On Unix the
          extension must be specified.
          --------------------------------------------------

PARAMETER  OUTPUT
  TYPE    FILE
  NAME    OUT(PUT)
  OPTIONS OUTPUT
  PROMPT  "(OUTput) name of resulting Figaro data file"
  TEXT    --------------------------------------------------
          This file will have the extension .DST unless the
          logical name FIGARO_FORMATS is set to NDF, in which
          case it will be a Starlink NDF with file extension
          .SDF.
          --------------------------------------------------

PARAMETER  CHANNEL
  TYPE    INT
  NAME    CHAN(NEL)
  OPTIONS NOPROMPT
  PROMPT  "(CHANnel) backend channel to be used"
  TEXT    --------------------------------------------------
          In a multi backend channel system the channel
          number that should be used to make the map has to
          be specified. The default is channel 1.
          --------------------------------------------------

PARAMETER  TEL_BEAM
  TYPE    CHAR
  NAME    TEL(_BEAM)
  PROMPT  "(TEL_beam) beam that telescope was in during map"
  TEXT    --------------------------------------------------
          The beam towards which the telescope was pointing
          while the map was made. The choice is L, M or R.
          The majority of maps are made with the telescope
          in the middle (M) beam, in fact this version of the
          JCMTDR package won't work unless this is the case.
          --------------------------------------------------

PARAMETER  POS_BEAM
  TYPE    CHAR
  NAME    POS(_BEAM)
  PROMPT  "(POS_beam) positive beam of chopper"
  TEXT    --------------------------------------------------
          The positive beam of the secondary chopper while
          the map was made. This should be L or R.
          --------------------------------------------------
\end{terminalv}
\end{small}

\goodbreak

The output from MAKEMAP is a Figaro file whose main data array is a
2-d map made up of from the measured data arranged according to its
cell offset from the observation map centre. The x and y axes of the
data are the pixel offsets along the cell axes from the map centre. If
the data were obtained pixel by pixel rather than by scanning then it
is possible that not all the pixels in the mesh were observed. Such
pixels will be set to a `bad' value.

The application creates a JCMT-specific section in the file in which
are stored parameters relating to the telescope, the environment and
the way in which the map was made.

If MAKEMAP fails to find important items in the GSD file, they will be
prompted for. If all else fails you can create, delete, rename or set
individual items in the datafile using Figaro commands.
On VMS the relevant commands are CROBJ, DELOBJ, RENOBJ and LET, on Unix
they are CREOBJ, DELOBJ, RENOBJ and SETOBJ.
(See section~\ref{sec-data} for a description
of the data structures used.)


\goodbreak

\subsection {\xlabel{MAP2MEM}MAP2MEM}

This application takes a JCMT map file and converts the data into a format
suitable for further processing (deconvolution and resampling) by the DBMEM
package written by J.S.\,Richer (see \cite{dbmem}). The parameters required
are:

\goodbreak

\begin{small}
\begin{terminalv}
PARAMETER  INPUT
  TYPE    FILE
  NAME    IN(PUT)
  OPTIONS INPUT
  PROMPT  "(INput) The name of the file containing JCMT map data"

PARAMETER  OUTPUT
  TYPE    CHAR
  NAME    OUT(PUT)
  PROMPT  "(OUTput) The rootname of the DBMEM file to contain the map"
  TEXT   ---------------------------------------------------
         The map will be output to 2 files; the header file
         <rootname>.DBH, and the data file which will
         either be <rootname>.MEM or .BIN depending on the
         value of parameter BINARY.
         ---------------------------------------------------

PARAMETER B1950
  TYPE    KEY
  NAME    B1(950)
  PROMPT "(B1950) make output coordinates B1950 equinox"
  TEXT   ---------------------------------------------------
         By default the application will transform the data
         into a map whose centre and tangent plane offsets
         are in the FK5 system with Julian 2000 equinox.
         Setting B1950 to YES will give results in the FK4
         system with Besselian 1950 equinox.
         ---------------------------------------------------

PARAMETER BINARY
  TYPE   KEY
  NAME   BINARY
  PROMPT "(BINARY) output data to unformatted file"
  TEXT   ---------------------------------------------------
         YES will result in the output data being written to
         file <rootname>.BIN, NO will give ASCII numbers in
         <rootname>.MEM.
         ---------------------------------------------------

PARAMETER RA_CENTRE
  TYPE   CHAR
  PROMPT "(RA_centre) The RA of the centre of the output map"
  TEXT   ---------------------------------------------------
         The positions of the output map pixels will be
         given in terms of a tangent plane offset in arcsec
         from a map centre specified by the user. This parameter
         specifies the RA of the centre, it must be input in
         hh mm ss.s format.
         ---------------------------------------------------

PARAMETER DEC_CENTRE
  TYPE   CHAR
  PROMPT "(DEC_centre) The Dec. of the centre of the output map"
  TEXT   ---------------------------------------------------
         The positions of the output map pixels will be
         given in terms of a tangent plane offset in arcsec
         from a map centre specified by the user. This parameter
         specifies the declination of the centre, it must be
         input in (sign)dd mm ss.s format.
         ---------------------------------------------------

PARAMETER NOISE
  TYPE   FLOAT
  PROMPT "(NOISE) Estimate of noise on each map point"
  TEXT   ---------------------------------------------------
         An estimate of the noise on each measured data point.
         If the input data already has an associated error
         array then those numbers will be used.
         ---------------------------------------------------
\end{terminalv}
\end{small}


\goodbreak

\subsection{\xlabel{MAP2TS}MAP2TS}

This application converts an image into a `time spectrum' with the map
pixels sorted in order of increasing LST of observation. The LST is
stored as  the X axis. It is useful to have the data in this form if
you want to spot and correct variations in the data that are functions
of time. The data will have to be transformed back into map format
using TS2MAP before being input to other JCMTDR applications.

\goodbreak

The application has the following parameters:

\begin{small}
\begin{terminalv}
PARAMETER  INPUT
  TYPE    FILE
  NAME    IN(PUT)
  OPTIONS INPUT
  PROMPT  "(INput) file containing map data"

PARAMETER  OUTPUT
  TYPE    FILE
  NAME    OUT(PUT)
  OPTIONS OUTPUT
  PROMPT  "(OUTput) "file to contain time-sorted data"
\end{terminalv}
\end{small}


\goodbreak

\subsection{\xlabel{RESTORE}RESTORE}

This application deconvolves the chopped-beam from a JCMT dual  beam
map of a source in a manner similar to that described by  Emerson,
Klein \& Haslam (Astron. Astrophys., 76, 92). The algorithm  is exactly
the same as that used in NOD2; first the spatial  frequencies to which
the chop is insensitive are set to zero,  then the chop is
deconvolved. The map is deconvolved scan by scan.

The source will come out negative if the positive beam of the chopper
was specified wrongly in MAKEMAP. This can be fixed by using LET (on
VMS) or SETOBJ (on Unix) to
set \emph{file}.MORE.JCMT.\-MAP.POS\_BEAM to the correct value and
re-running RESTORE. Similarly, the chop throw  is read from \emph{file}.MORE.\-JCMT.\-MAP.CHOP\_THRW and can be changed if you  believe that
the nominal value was inaccurate.

If there are any bad pixels in the input data they will be ignored,
effectively treated as zero in the restoration process. If the pixel
is in the source then this is \emph{not what you want to happen}. The
pixel should be set to something nearer to its real value by using one
of the Figaro data editing applications. Similarly  the process
carries the implicit assumption that all data points outside the raw
map would have been zero had they been measured. Bad results will be
obtained if the raw map does not extend right off the source in the
scan direction. Lastly, the zero level in the raw map must be set to 0
before RESTORE is run, often the raw data has a DC offset.

\goodbreak

The parameters used are:

\begin{small}
\begin{terminalv}
PARAMETER  INPUT
  TYPE    FILE
  NAME    IN(PUT)
  OPTIONS INPUT
  PROMPT  "(INput) file containing map data"

PARAMETER  OUTPUT
  TYPE    FILE
  NAME    OUT(PUT)
  OPTIONS OUTPUT
  PROMPT  "(OUTput) file to contain restored map"

PARAMETER  UNBAL
  TYPE    FLOAT
  NAME    UNBAL
  OPTIONS NOPROMPT
  PROMPT  "(UNBAL) amplitude (L beam) / amplitude (R beam)"
  TEXT    ------------------------------------------------------
          The ratio of the amplitude of the left hand chop beam
          and that of the right hand chop beam. For the JCMT this
          will normally be 1.
          ------------------------------------------------------
\end{terminalv}
\end{small}


\goodbreak

\subsection{\xlabel{TS2MAP}TS2MAP}

Performs the inverse operation of MAP2TS. The parameters required are:

\begin{small}
\begin{terminalv}
PARAMETER  INPUT
  TYPE    FILE
  NAME    IN(PUT)
  OPTIONS INPUT
  PROMPT  "(INput) Name of file containing time-sorted data"

PARAMETER  OUTPUT
  TYPE    FILE
  NAME    OUT(PUT)
  OPTIONS OUTPUT
  PROMPT  "(OUTput) Name of file to contain map data"
\end{terminalv}
\end{small}


\goodbreak

\section{GSD Utilities\xlabel{gsd_utilities}}

\textit{The GSD utilities are now distributed separately from JCMTDR
as part of the GSD distribution, see \xref{SUN/229}{sun229}{}}

On VMS two utilities are provided to allow examination and  modification of
the raw GSD data files. On Unix only an examination utility is
available.
The utilities are available as DCL symbols (shell commands) and should
be prefaced by a \$ when run from VAX/ICL.

\subsection{\xlabel{GSD_PRINT}GSD\_PRINT}

This reads the contents of a GSD file and writes them to an ASCII file
suitable for printing or examining with an editor.

\subsection{\xlabel{GSD_FORMAT}GSD\_FORMAT}

This is used to display individual, named items in the GSD file and
can modify their values. The program operates in a loop which can be
exited by responding to a prompt with CTRL-C. This utility is not
available in the Unix version.

\goodbreak

\section{Datafile Formats
\xlabel{datafile_formats}\label{sec-data}}

The applications can access data files either in `old Figaro' or
Starlink  NDF formats. It is recommended that the NDF format be used
as this will allow most other Starlink packages (\emph{e.g.} \xref{KAPPA}{sun95}{} and
\xref{PHOTOM}{sun45}{}) to access the data as well. The IRCAM data reduction package
should also be able to access the main data arrays, though other parts
of the data structure will be inaccessible.

To make these applications (and the rest of Figaro) access Starlink
\xref{NDF}{sun33}{} files the user should define the logical name (on VMS) or set the
environment variable (in a C- or Tc-shell) FIGARO\_FORMATS to
be \texttt{"NDF"}. If the applications are being run as an ADAM monolith from
\xref{ICL}{sg5}{} then the logical name should be defined /JOB. NDF files have the
extension \texttt{.SDF} while `old Figaro' files have the extension \texttt{.DST}.

A list of all the structures that might be found in a JCMT datafile
before the rebinning stage follows (the example is in Starlink NDF
format).  The actual items and  structures present in a given file and
the order in which they appear will depend on the stage that has been
reached in the reduction process. The temporary structures used by the
MAP2TS and TS2MAP applications  have been left out.

\goodbreak

The HDSTRACE of a JCMT datafile structure before rebinning should
resemble this:
\begin{small}
\begin{terminalv}
STRUCT  < >

   DATA_ARRAY(71,57)  <_REAL>     0.829641,-0.907788,1.163677,-1.149052,
                                  ... 0.841277,-0.532137,0.25853,-0.551927

! the x-axis structures. Before it has been re-binned to an RA, dec grid
! the x-axis values will be offsets along the x cell axis. After
! rebinning the offsets will be true RA offsets from the map centre.

   AXIS(2)        <AXIS>          {array of structures}

   Contents of AXIS(1)
      DATA_ARRAY(71)  <_REAL>        -140,-136,-132,-128,-124,-120,-116,
                                     ... 108,112,116,120,124,128,132,136,140
      UNITS          <_CHAR*32>      'ARCSEC'
      LABEL          <_CHAR*32>      'X offset'

   MORE           <EXT>           {structure}

! the extension structure that will hold the JCMT-specific information
! used by the JCMTDR package.

      JCMT           <EXT_JCMT>      {structure}

! the array holding the local sidereal time at which each raw map measurement
! was made (radians).

         LST            <NDF>           {structure}
            DATA_ARRAY(71,57)  <_DOUBLE>   1.70557566383623,
                                           ... 2.04654424975883

         OBS            <STRUC>         {structure}
            OBJECT         <_CHAR*64>      'OMC1'
            OBS_1          <_CHAR*32>      'gs'
            OBS_2          <_CHAR*32>      'el'
            PROJECT        <_CHAR*32>      'ms55'
            NOBS           <_INTEGER>      146

         TEL            <STRUC>         {structure}
            NAME           <_CHAR*32>      'JCMT'
            FRONTEND       <_CHAR*32>      'RXB2'
            BACKEND        <_CHAR*32>      'RXA'
            LONG           <_DOUBLE>       3.56955225443728
            LAT            <_DOUBLE>       0.34602605175162
            HEIGHT         <_DOUBLE>       4092.093

         MAP            <STRUC>         {structure}
            CENTRE_CRD     <_CHAR*32>      'RB'
            EPOCH          <_DOUBLE>       1950
            RACEN          <_DOUBLE>       1.4520339434019
            DECCEN         <_DOUBLE>       -0.09437286153742
            LOCAL_CRD      <_CHAR*32>      'AZ'
            V2Y            <_REAL>         0
            X2Y            <_REAL>         1.570796
            CELL_X         <_REAL>         4
            CELL_X_UNIT    <_CHAR*10>      'ARCSEC'
            CELL_Y         <_REAL>         4
            CELL_Y_UNIT    <_CHAR*10>      'ARCSEC'
            ON_FLY         <_INTEGER>      255
            INT_TIME       <_REAL>         1
            SCAN_DIR       <_CHAR*16>      'HORIZONTAL'
            MJD_START      <_DOUBLE>       48236.4812936717
            CHOP_FUN       <_CHAR*32>      'SQUARE'
            CHOP_THRW      <_REAL>         38.5
            CHOP_PA        <_REAL>         1.570796
            CHOP_CRD       <_CHAR*32>      'AZ  RE'
            TEL_BEAM       <_CHAR*1>       'M'
            POS_BEAM       <_CHAR*1>       'R'
            FREQ           <_REAL>         230.5252

         ENVRNMNT       <STRUC>         {structure}
            AMB_TEMP       <_REAL>         272.55
            PRESSURE       <_REAL>         629
            REL_HUMID      <_REAL>         0.25

         PCORR          <STRUC>         {structure}
            {structure is empty}

         SOFTWARE       <STRUC>         {structure}
            MAKEMAP        <_REAL>         1.2

! the structure generated by the JCMTEXTC application to store the airmass
! at which each pixel was measured.

         AIRMASS        <STRUC>         {structure}
            DATA_ARRAY(71,57)  <_REAL>     1.140773,1.140794,1.140815,
                                           ... 1.334956,1.335022,1.335088

! the structure generated by the JCMTEXTC application to store the zenith
! opacity used to extinction correct each pixel.

         TAU            <STRUC>         {structure}
            DATA_ARRAY(71,57)  <_REAL>     0.254,0.254001,0.254002,
                                           ... 0.257998,0.257999,0.258


   TITLE          <_CHAR*32>      'OMC1'
\end{terminalv}
\end{small}

\goodbreak

A fully reduced and rebinned data file should look something like the
following:

\begin{small}
\begin{terminalv}
STRUCT  < >

   DATA_ARRAY(98,96)  <_REAL>     *,*,*,*,*,*,*,*,*,*,*,*,*,*,*,*,*,*,*,*,
                                  *,*,*,*,*,*,*,*,*,*,*,*,*,*,*,*,*,*,*,*
   AXIS(2)        <AXIS>          {array of structures}

   Contents of AXIS(1)
      DATA_ARRAY(98)  <_REAL>        192,188,184,180,176,172,168,164,160,
                                     ... -172,-176,-180,-184,-188,-192,-196
      UNITS          <_CHAR*32>      'ARCSEC'
      LABEL          <_CHAR*32>      'RA'

   TITLE          <_CHAR*32>      'OMC1'
   MORE           <EXT>           {structure}

! the coords and size of the map ready to for input to the IRAS_TAG
! application

      JCMT_COORDS    <EXT_JCMT>      {structure}
         RACEN          <_DOUBLE>       1.46275826443488
         DECCEN         <_DOUBLE>       -0.09382259800936
         SYSTEM         <_CHAR*64>      'FK5 J2000.0'
         ICEN           <_INTEGER>      49
         JCEN           <_INTEGER>      46
         PIXSIZE        <_DOUBLE>       0.00001939254724

! an array of FITS items that hold the names and weights of the files input
! to the rebinning application, and the method of rebinning used.

      FITS(9)        <_CHAR*80>      'FILE_1  = 'RXA_146ERJ'         / Na...'
                                     ... 'METHOD  = 'Bessel convolut...','END'


      FIGARO         <FIGARO_EXT>    {structure}
         RANGE          <RANGE_STRUCT>   {structure}
            VALID          <_INTEGER>      1
            MAX            <_REAL>         90.53838
            MIN            <_REAL>         -149.2703
\end{terminalv}
\end{small}

\section*{Release Notes}

\subsection*{V1.2-2}

\begin{itemize}
\item Port to Linux. This involved the replacement of all NAG routines with
  routines from the PDA library (see \xref{SUN/194}{sun194}{} \cite{pda}). \texttt{ae2rd2} has been removed
  as no replacement PDA routine is available.
\item Include \xref{SC/1}{sc1}{} and SUN/132 in the distribution. Update SUN/132 to current
  standard although references to VMS and old Figaro support are still present
and should be removed.
\item GSD inspection routines have been removed. They are now distributed as
  part of the separate GSD package (see \xref{SUN/229}{sun229}{}).
\item The \texttt{jcmt\_xhelp} command has been replaced with the standard
  Starlink HTML access system ('findme sun132').

\end{itemize}

\subsection*{V1.2}

Since the previous version (1.1), GSD support has been added.
\begin{itemize}
\item GSD files can be converted with ``makemap'' into NDF format for (i)
    further use in JCMTDR, (ii) use in Figaro, (iii) use in \Kappa\ and
    associated packages.

\item GSD files can be inspected with ``gsd\_print''. The name of the GSD
    file must be given as first argument. Output is on standard output,
    it can be re-directed to a file or piped into other commands like
    ``more''.

    The exact format of gsd\_print is changed compared to the VAX
    version.
\end{itemize}

 GSD files must be specified by full name, including the file name
 extension (usually .DAT or .dat).

 GSD files can be taken across from the VAX/VMS system to the Unix
 system either by ftp transfer in binary mode, or with the Unix ``cp''
 command if the VAX file system is mounted by the Unix machine.

 A further change is in ``ae2rd1'' and ``ae2rd2''. The parameter INFILE is
 now prompted with blank default for the first file and with default
 `END' for any further files.

 Starlink User Note 132 is available on-line as Web pages. The browser
 for these pages can be started with ``jcmt\_xhelp'' from the Unix shell.


\subsection*{V1.1}

This is the Port of JCMTDR 1.0 to Unix. Note the following differences
in the Unix version:

\begin{itemize}
\item So far it is common to run the commands from the Unix shell. To start
   up JCMTDR from the shell you need to:

\begin{terminalv}
source /star/etc/cshrc in your .chsrc
source /star/etc/login in your .login
\end{terminalv}
   Then the command:
\begin{terminalv}
% jcmtdr
\end{terminalv}
   will define all the commands of the JCMTDR package.

\item On-line help is available, from the Unix shell use ``jcmt\_help''.

\item GSD format is currently not supported. Thus MAKEMAP, GSD\_PRINT and
   GSD\_FORMAT are not available. You will have to process GSD data on a
   VAX with JCMTDR 1.0.

\item JCMTDR data are in Figaro/NDF format and can be transferred as binary
   files between VMS and various Unix flavours. It is not necessary to
   process the files. However, you might run the ``native'' command from
   the \Kappa\ package on the files on the destination machine. This may
   save some time in further access to the file.
\end{itemize}


\begin{thebibliography}{}
\addcontentsline{toc}{section}{References}

\bibitem{nod2}
Haslam~C.~G.~T., 1974, \textit{A\&AS}, \textbf{15}, 333

\bibitem{ukt14}
Duncan~W.~D., Sandell~G., Robson~E.~I., Ade~P.~A.~R., Griffin~M.~J.,
1990, \textit{MNRAS}, \textbf{243}, 126

\bibitem{dbmem}
Richer~J.~S., 1992, \textit{MNRAS}, \textbf{254}, 165

\bibitem{pda}
Meyerdierks~H., Berry~D., Draper~P., Privett~G., Currie~M,
1997, \textit{ PDA -- Public Domain Algorithms},
\xref{Starlink User Note 194}{sun194}{}

\end{thebibliography}

\end{document}

