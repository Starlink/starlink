\documentclass[twoside,11pt]{starlink}

% -----------------------------------------------------------------------------
\stardoccategory  {Starlink User Note}
\stardocinitials  {SUN}
\stardocsource    {sun95.46}
\stardocnumber    {95.46}
\stardocauthors   {Malcolm J. Currie \& David S. Berry}
\stardocdate      {2025 July 23}
\stardoctitle     {KAPPA --- Kernel Application Package}
\stardocversion   {2.6-13}
\stardocmanual    {User's Guide}
% -----------------------------------------------------------------------------
\stardocabstract{
\KAPPA\ is an applications package comprising about 180
general-purpose commands for image processing, data visualisation, and
manipulation of the standard Starlink data format---the NDF. It is
intended to work in conjunction with starlink's various specialised
packages.

In addition to the NDF, \KAPPA\ can also process data in
other formats by using the `on-the-fly' conversion scheme.  Many
commands can process data arrays of arbitrary dimension, and others
work on both spectra and images.  \KAPPA\ operates from
both the UNIX C-shell and the \ICL\ command language.

This document describes how to use \KAPPA\ and its
features.  There is some description of techniques too, including a
section on writing scripts.  This document includes several tutorials
and is illustrated with numerous examples.  The bulk of this document
comprises detailed descriptions of each command as well as classified
and alphabetical summaries.}
% -----------------------------------------------------------------------------

\stardocname{\stardocinitials /\stardocnumber}

% centre an asterisk
\providecommand{\lsk}{\raisebox{-0.4ex}{\textrm{*}}}

\hyphenation{which-ever}

% Lines for breaking up Appendices A and B.
\providecommand{\jrule}{\noindent\rule{\textwidth}{0.45mm}}
\providecommand{\krule}{\vspace*{-1.5ex}
                    \item [\textrm{ \rule{\textwidth}{0.15mm}}]}

% Frame attributes fount.
\providecommand{\att}[1]{\textsf{#1}}
\providecommand{\htmlattref}[2]{\htmlref{\att{#1}}{#2}}

% Shorthands for hypertext links.
% -------------------------------
\providecommand{\AGIref}{\xref{AGI}{sun48}{}}
\providecommand{\ARDref}{\xref{ARD}{sun183}{}}
\providecommand{\ASTERIXref}{\xref{{\footnotesize ASTERIX}}{sun98}{}}
\providecommand{\CCDPACK}{{\footnotesize CCDPACK}\normalsize}
\providecommand{\CCDPACKref}{\xref{\CCDPACK}{sun139}{}}
\providecommand{\CGSDRref}{\xref{{\footnotesize CGS4DR}}{sun27}{}}
\providecommand{\CONVERT}{{\footnotesize CONVERT}\normalsize}
\providecommand{\CONVERTref}{\xref{{\footnotesize CONVERT}}{sun55}{}}
\providecommand{\ECHOMOPref}{\xref{{\footnotesize ECHOMOP}}{sun152}{}}
\providecommand{\ESPref}{\xref{{\footnotesize ESP}}{sun180}{}}
\providecommand{\EVINCE}{{\footnotesize EVINCE}\normalsize}
\providecommand{\EVINCEref}{\htmladdnormallink{\EVINCE}{http://wiki.gnome.org/Apps/Evince}}
\providecommand{\EXTRACTORref}{\xref{{\footnotesize EXTRACTOR}}{sun226}{}}
\providecommand{\FIGARO}{{\footnotesize FIGARO}\normalsize}
\providecommand{\Figaroref}{\xref{\FIGARO}{sun86}{}}
\providecommand{\FITSref}{\htmladdnormallink{FITS}{http://fits.gsfc.nasa.gov/}}
\providecommand{\GAIA}{{\footnotesize GAIA}\normalsize}
\providecommand{\GAIAref}{\xref{\GAIA}{sun214}{}}
\providecommand{\GWM}{{\footnotesize GWM}\normalsize}
\providecommand{\GWMref}{\xref{GWM}{sun130}{}}
\providecommand{\HDSref}{\xref{HDS}{sun92}{}~}
\providecommand{\HDSTRACEref}{\xref{{\footnotesize HDSTRACE}}{sun102}{}}
\providecommand{\ICL}{{\footnotesize ICL}\normalsize}
\providecommand{\ICLref}{\xref{{\footnotesize ICL}}{sg5}{}}
\providecommand{\IRCAMPACKref}{\xref{{\footnotesize IRCAMPACK}}{sun177}{}}
\providecommand{\IRAF}{\footnotesize{IRAF}\normalsize}
\providecommand{\IRAFref}{\htmladdnormallink{\IRAF}{http://iraf.noao.edu/iraf-homepage.html}}
\providecommand{\IRASref}{\xref{{\footnotesize IRAS90}}{sun163}{}}
\providecommand{\JCMTDRref}{\xref{{\footnotesize JCMTDR}}{sun132}{}}
\providecommand{\KAPLIBS}{{\footnotesize KAPLIBS}\normalsize}
\providecommand{\KAPPA}{{\footnotesize KAPPA}\normalsize}
\providecommand{\KAPRHref}{\xref{{\footnotesize KAPRH}}{sun239}{}}
\renewcommand{\NDFref}[1]{\htmlref{#1}{ap:NDFformat}~}
\providecommand{\NDFextref}[1]{\xref{#1}{sun33}{}}
\providecommand{\NDFPACK}{{\footnotesize NDFPACK}\normalsize}
\providecommand{\OKULAR}{{\footnotesize OKULAR}\normalsize}
\providecommand{\OKULARref}{\htmladdnormallink{\OKULAR}{http://okular.kde.org/}}
\providecommand{\PDAref}{\xref{PDA}{sun194}{}}
\providecommand{\PGPLOT}{{\footnotesize PGPLOT}\normalsize}
\providecommand{\PGPLOTref}{\htmladdnormallink{\PGPLOT}{http://www.astro.caltech.edu/~tjp/pgplot/}}
\providecommand{\PHOTOMref}{\xref{{\footnotesize PHOTOM}}{sun45}{}}
\providecommand{\PISAref}{\xref{{\footnotesize PISA}}{sun109}{}}
\providecommand{\PONGOref}{\xref{{\footnotesize PONGO}}{sun137}{}}
\providecommand{\PSMERGEref}{\xref{{\footnotesize PSMERGE}}{sun164}{}}
\providecommand{\TSPref}{\xref{{\footnotesize TSP}}{sun66}{}}
\providecommand{\POLPACKref}{\xref{{\footnotesize POLPACK}}{sun223}{}}
\providecommand{\STLref}{\xref{{\footnotesize STL}}{sun190}{STLREF}}
\providecommand{\TWODSPECref}{\xref{{\footnotesize TWODSPEC}}{sun16}{}}

%------------------------------------------------------------------------------

\begin{document}
\scfrontmatter
% -----------------------------------------------------------------------------
\section{\xlabel{se_kappaintro}Introduction\label{se:kappaintro}}

\subsection{Background}

It is Starlink's aim to provide {\em maintainable\/}, {\em
portable\/}, and {\em extensible\/} applications packages that work in
harmony by sharing a common infrastructure toolkit, standards,
conventions and above all, a standard data format.  Individual
packages are no longer required to perform all functions, thus carry
less inertia, and are more adaptable to outside developments.
Additional functionality can be added piecemeal to the relevant
package.  New user interfaces, such as graphical, could be layered
within the toolkit for obtaining parameters and so make the
enhancement available to all applications that make use of those
tools.  An example of this approach has allowed us to access
`foreign data formats' throughout Starlink packages, because the
packages use a common infrastructure library.

An important part of the rationalisation is that applications are
unified by sharing the same basic data structure---the
\NDFref{NDF} (Extensible \textit{n}-dimensional Data Format).  This contains
an \textit{n}-dimensional data array that can store most astronomical data such
as spectra, images and spectral-line data cubes.  The NDF may also
contain information like a title, axis labels and units, error and
quality arrays, and World Co-ordinate System information.
There are also places in the NDF, called {\em extensions}, to store
any ancillary data associated with the data array, even other NDFs.

\subsection{R\^{o}le of KAPPA}

The backbone of the applications packages is \KAPPA\ ({\bf
K}ernel {\bf AP}plication {\bf PA}ckage).  It provides general-purpose
applications that have wide applicability, concentrating on image
processing, data visualisation, and manipulating
\xref{NDF}{sun33}{overview_of_an_ndf} components.  \KAPPA\ provides
facilities that integrate with specialised Starlink packages such as
those for CCD reduction (\CCDPACKref ), stellar and galaxy photometry
(\PHOTOMref , \EXTRACTORref , \PISAref , \ESPref ), spectroscopy
(\ECHOMOPref , \Figaroref ), polarimetry (\POLPACKref, \TSPref ),
format conversion (\CONVERTref ), \emph{etc.} {\em Thus the
functionality of \KAPPA\ should not be regarded in
isolation.}

In a wider context, \KAPPA\ offers facilities not in \IRAFref ,
for instance handling of data errors, quality masking, a graphics
database, availability from the shell, as well as more \textit{n}-dimensional
applications, widespread use of data axes, and a different style.  It
integrates with instrument packages developed at UK observatories.
With the automatic data conversion and the availability of \KAPPA\ and
other Starlink packages from within the
\IRAF\ command language, you should be able to pick the
best of the relevant tools from both systems to get the job done.

\subsection{Functionality of KAPPA}

\subsubsection{Applications}

Currently, \KAPPA\ has over 200 commands that are available
both from the UNIX C-shell and from the \ICLref\ command language.
They provide the following facilities for data processing:

\begin{itemize}
\item \htmlref{FITS readers}{se:fitsreaders} that generate NDFs and
text tables, and the import and export of ancillary data through the
NDF \htmlref{FITS extension}{se:fitsairlock};
\item \htmlref{generation of test data}{cl:datagen}, and
\htmlref{NDF creation from text files}{TRANDAT};
\item setting and examining \htmlref{NDF components}{ap:NDFformat};
\item \htmlref{world co-ordinate systems}{cl:wcs}, and
\htmlref{calculation of a sky co-ordinate system}{SETSKY};
\item arithmetic including a \htmlref{powerful application}{MATHS}
that handles expressions;
\item \htmlref{pixel and region editing}{cl:pixedit}, including polygons and
circles; re-flagging of bad pixels by value or by \htmlref{median
filtering}{GLITCH}; and \htmlref{pasting arrays over others}{PASTE};
\item \htmlref{masking}{se:masking} of regions, and of \htmlref{pixels
whose variances are too large}{ERRCLIP};
\item configuration change: \htmlref{flip}{FLIP},
\htmlref{rotate}{ROTATE}, \htmlref{shift}{SLIDE},
\htmlref{reshape}{RESHAPE}, subset, \htmlref{permute axes}{PERMAXES}
change dimensionality;
\item \htmlref{normalisation}{NORMALIZE} of NDF pairs;
\item \htmlref{compression and expansion of images}{cl:compexp};
\item \htmlref{generalised resampling}{cl:regrid} of NDFs using
arbitrary transformations;
\item \htmlref{mosaic creation}{WCSMOSAIC};
\item filtering: \htmlref{box}{BLOCK}, \htmlref{Gaussian}{GAUSMOOTH},
and \htmlref{median smoothing}{MEDIAN}; very efficient
\htmlref{Fourier transform}{FOURIER}, \htmlref{convolution}{CONVOLVE};
\item deconvolution: \htmlref{maximum-entropy}{MEM2D},
\htmlref{Lucy-Richardson}{LUCY}, \htmlref{Wiener filter}{WIENER};
\item \htmlref{surface and trend fitting}{cl:surfit};
\item statistics including \htmlref{ordered statistics}{HISTAT},
\htmlref{histogram}{HISTOGRAM}; \htmlref{pixel-by-pixel
statistics over a sequence of images}{MSTATS};
\item \htmlref{inspection}{LOOK} of image values;
\item \htmlref{centroids of features}{CENTROID}, particularly stars;
\htmlref{stellar PSF fitting}{PSF};
\item detail enhancement using \htmlref{histogram equalisation}{HISTEQ}
and \htmlref{Laplacian}{LAPLACE}, convolution, edge enhancement via a
\htmlref{shadow effect}{SHADOW}, \htmlref{thresholding}{THRESH};
\item \htmlref{calculation of polarimetry images}{CALPOL};
\item \htmlref{creation of one-dimensional profiles through \textit{n}-dimensional
data sets}{PROFILE}; and
\item \htmlref{conversion between various forms of complex data}{COMPLEX}.
\end{itemize}

There are also many applications for data visualisation:
\begin{itemize}
\item use of the \htmlref{graphics database}{se:agitate}, AGI, to pass
information about pictures between tasks; tools for the creation, labelling,
selection of pictures, and obtaining co-ordinate information from them;
\item \htmlref{image}{DISPLAY} plots with
a selection of scaling modes and many options such as axes;
\item creation, selection, saving and manipulation of \htmlref{colour
tables and palettes}{se:coltab} (for axes, annotation, coloured markers
and borders);
\item line graphics: \htmlref{contouring}{CONTOUR}; histogram; \htmlref{line
plot}{LINPLOT} of one-dimensional arrays, \htmlref{multiple-line
plot}{MLINPLOT} of images, and a \htmlref{grid of line plots}{CLINPLOT}
for cubes; \htmlref{pie sections}{ELPROF}, and slices
through an image; \htmlref{vector plot}{VECPLOT} of an image.
\item many aspects of the appearance of line graphics can be tailored
to individual needs and stored within \htmlref{`style files'}{se:style}.
\end{itemize}

\subsubsection{General}

\KAPPA\ handles \htmlref{bad pixels, and
processes quality}{se:masking}, variance, \htmlref{World Co-ordinate
System}{se:wcsuse} (WCS), and other information stored within NDFs
\latex{(SUN/33 and Section~\ref{se:datastr})}.  In order to achieve
generality \KAPPA\ does not process non-standard extensions; however,
it does not lose non-standard ancillary data since it copies
extensions to any NDFs that it creates.  The standard extensions that
\KAPPA\ recognises are the \htmlref{FITS airlock}{se:fitsairlock},
that holds metadata in the form of FITS headers; and
\htmlref{PROVENANCE}{apndf:provenance} that records the lineage of an
NDF, much like a family tree.

\KAPPA\ can also process data in other formats, such as FITS
and \IRAF, using an automatic-data conversion facility
(\CONVERTref\normalsize , \xref{SSN/20}{ssn20}{}).

Although oriented to image processing, many commands will work on NDFs
of arbitrary dimension and others operate on both spectra and images,
and cubes.  Many applications handle all non-complex data types
directly, for efficient memory and disc usage.  Those that do not will
usually undergo automatic data conversion to produce the desired
result.

\KAPPA's graphics are produced using the widely used
\PGPLOTref\ \normalsize package, and are thus device independent.

Most commands can be automatically re-invoked to process multiple NDFs by
supplying a group of NDFs as input.  \htmlref{Groups of NDFs}{se:groups}
can be specified using wild-cards, or by listing them explicitly either
in response to a prompt or within a text file.


\subsection{This document}

This document is arranged as follows.  First are two annotated \KAPPA\ tutorials
to give you a quick summary of basic usage.  The main
text follows, which amplifies the points sketched in the demonstrations,
and describes other functionality and modes of use illustrated with
further examples.  Finally, there are extensive appendices, including a
classified list of commands and detailed descriptions of each command,
which are also available on a quick-reference card.


\newpage
\section{\xlabel{se_demo}Tutorials\label{se:demo}}

So the facilities summarised in the introduction sound appealing.  Now
you want to know how to access them, but the thick manual looks
daunting.  Actually, most of this manual comprises descriptions of each
application.  The best way to learn the basics is to try some example
sessions.

Login to a colour workstation or X-terminal.  Then enter the commands
following the prompts shown below.  The \texttt{\%} is the shell prompt
string, which you don't type.  As we go along there will be
commentary explaining what is happening and why.  Let's begin.

\subsection{From the C-shell}

\begin{terminalv}
     % kappa
\end{terminalv}

This defines C-shell aliases for each \KAPPA\ \normalsize
command, includes the help information, and shows the version number.  It need
only be issued once per login session.  Thus you will see

\begin{terminalv}
      KAPPA commands are now available -- (Version 1.0)

      Type kaphelp for help on KAPPA commands
\end{terminalv}

Let's run a \KAPPA\ \normalsize application.  \htmlref{CADD}{CADD} adds a
scalar constant to an \NDFref{NDF} file---the Starlink standard data
format---to make a new NDF file (usually called an {\em NDF\/} for
short).  In this case ten is added to the pixels in \file{\$KAPPA\_DIR/comwest.sdf}
to create \file{test.sdf} in the current directory.

\begin{terminalv}
     % cadd $KAPPA_DIR/comwest 10 test
\end{terminalv}

There are three \htmlref{{\em parameters\/}}{se:param} qualifying
the CADD command: the names of the input and output NDFs and the
constant.  Notice that these parameters are separated by spaces.  Most
applications have a few of these positional parameters, usually the
most commonly used.  Parameters given on the command line are not
subsequently prompted for by the application.  Also you see that the
NDF file extensions are not given.  The \file{.sdf} extension indicates
that it was created by the \xref{Hierarchical Data System}{sun92}{}
\latex{ (SUN/92)}, or HDS for short.  Note that an arbitrary
\file{.sdf} file is not necessarily an NDF.

Next we run the statistics task.  Here we have not given any parameters.
In this case the application will ask for the values of any parameters
it needs.

\begin{terminalv}
     % stats
\end{terminalv}

The only parameter required is called NDF, and \htmlref{STATS}{STATS}
prompts us for it.

\begin{terminalv}
     NDF - Data structure to analyse /@test/ >
\end{terminalv}

In this example, STATS wants to know for which NDF we require
statistical data.  The text between the \texttt{//} delimiters is the
suggested default for the parameter.  By pressing the carriage return we
accept this default as the parameter's value.  Here the suggested
default is the name of the NDF created by CADD.  (Ignore the \texttt{@} for
the moment---it just tells the application that it is a file.)
\KAPPA\ remembers the last NDF used or created, and uses it for the
suggested default to save typing.  Since \file{test} is the NDF whose
statistics we want we just hit the return key.  Again we exclude the
\file{.sdf} extension.  Here is the output from STATS.

\begin{terminalv}
        Pixel statistics for the NDF structure /home/scratch/mjc/test

           Title                     : Comet West, low resolution
           NDF array analysed        : DATA

              Pixel sum              : 11851773
              Pixel mean             : 180.8437
              Standard deviation     : 63.47324
              Minimum pixel value    : 13.89063
                 At pixel            : (59, 83)
                 Co-ordinate         : (58.5, 82.5)
              Maximum pixel value    : 255.9375
                 At pixel            : (248, 45)
                 Co-ordinate         : (247.5, 44.5)
              Total number of pixels : 65536
              Number of pixels used  : 65536 (100.0%)

\end{terminalv}

Of course, in your case the current directory will not be \file{
/home/scratch/mjc}.  The NDF title is the unchanged from the
\file{\$KAPPA\_DIR/comwest} NDF.  This is the normal behaviour for tasks
that create a new NDF from an old one; they do, however, have a
parameter for changing this default.  To alter a defaulted parameter you
supply its new value on the command line.  Defaulted parameters exist to
prevent a long series of prompts where reasonable values can be defined,
and hence save time.  (However, there is a way of being prompted for all
parameters of a command should you wish.)

NDFs may contain three standard arrays---the data array, the data
variance and quality.  STATS can calculate statistics for any of
these.  By default, STATS uses the data array, as indicated here.

Next we wish to smooth our data.  \htmlref{GAUSMOOTH}{GAUSMOOTH}
performs a Gaussian smooth of neighbouring pixels.

\begin{terminalv}
     % gausmooth
\end{terminalv}

Again we are prompted with the same suggested default, since we
have not created any new NDFs within STATS.  Say we don't want to
smooth that NDF, but the original one.  We just enter the name of the
NDF at the prompt.  Notice that we don't need the \texttt{@}
prefix, since Parameter IN expects a file.  (One occasion where you
would need it is when the filename is a number, {\emph{e.g.}}\ if your
NDF was called 234 you must enter \texttt{@234}, otherwise the parameter
system will think you are giving the integer 234.  Yes \ldots I
know \ldots the parameter system is trying to be too clever.)

\begin{terminalv}
     IN - Input NDF /@test/ > $KAPPA_DIR/comwest
\end{terminalv}

The description of Parameter FWHM is too brief for us to select a value.
So we obtain some help on this parameter, and then GAUSMOOTH reprompts
for a value.  The smoothed NDF is written to the NDF called \file{testsm}
in the current directory.

\begin{terminalv}
     FWHM - Gaussian PSF full-width at half-maximum /5/ > ?

     GAUSMOOTH

       Parameters

         FWHM

           FWHM() = _REAL (Read)
              This specifies whether a circular or elliptical Gaussian
              point-spread function is used in smoothing a two-dimensional
              image.  If one value is given it is the full-width at
              half-maximum of a one-dimensional or circular Gaussian PSF.
              (Indeed only one value is permitted for a one-dimensional
              array.)  If two values are supplied, this parameter becomes the
              full-width at half-maximum of the major and minor axes of an
              elliptical Gaussian PSF.  Values between 0.1 and 10000.0 pixels
              should be given.  Note that unless a non-default value is
              specified for the BOX parameter, the time taken to perform the
              smoothing will increase in approximate proportion to the
              value(s) of FWHM.  The suggested default is the current value.
              will increase in approximate proportion to the value of FWHM.

     FWHM - Gaussian PSF full-width at half-maximum /5/ >
     OUT - Output NDF > testsm
\end{terminalv}

Next we want to look at the result of our image processing.  The first
thing to do is to select an graphics device.  The \texttt{xwindows} device
becomes the current graphics device and remains so until the next
\htmlref{GDSET}{GDSET} command.  (You may need to enter the
\texttt{xdisplay} command (\xref{SUN/129}{sun129}{}) to redirect
the output from the host computer to the screen in front you.)

\begin{terminalv}
     % gdset xwindows
\end{terminalv}

Now we actually display it on the screen.  Some applications have
many parameters, and it would be impractical to have to specify all
those preceding the ones we wanted to alter.  The solution is to specify
the parameter by keyword.  Here we have requested that the scaling of
the data values to colour indices within the graphics device uses the
current percentile range.  Note that you may abbreviate the options
of a menu, such as offered by Parameter MODE, to any unambiguous
string.

\begin{terminalv}
     % display mode=pe accept
     Data will be scaled from 78.38278 to 235.3536.
\end{terminalv}

If you have just created the window, the image will not look much like
the comet, because the existing colour table is poor.  If we replace
the table with a grey-scale ramp from white to black,

\begin{terminalv}
     % lutneg
\end{terminalv}

what happens depends on your workstation hardware and settings.  If your
graphics system is set to 256 colours (technically, an \emph{8-bit
pseudo-colour visual}), then the effects of the above
\htmlref{LUTNEG}{LUTNEG} command will
be immediately visible, and you will see a blurred image of the
ubiquitous Comet West on the screen.  If, on the other hand, your graphics
system is set to 16-bit or 24-bit graphics, then the effects of the
LUTNEG command will only become visible when you next display an image.
In this case, re-invoking the above DISPLAY command will make the image
appear correctly with the requested grey-scale colour table.


The \htmlref{ACCEPT keyword}{se:parspeckey} is a very useful feature.
It tells an application to accept all the suggested defaults.  In this
case \htmlref{DISPLAY}{DISPLAY} uses the current NDF and scales
between the current percentile limits---10 and 90.  The keyword can be
abbreviated to double backslash from the shell.  Aside: the parameter
system actually requires a single backslash.  From the shell, however,
backslash is a {\em metacharacter\/}, and so must be `escaped' to
treat the character literally.  One way is to place a \texttt{\textbackslash}
before each metacharacter.  You can escape a series of
characters by placing them inside single quotes \texttt{' '}.  Other
metacharacters to watch out for when using \KAPPA\ include
\texttt{[]()\textbackslash"*?\$}.

Next we want to make the image colourful.  There are a number of
predefined lookup tables, or you may create and modify your own.  Here
we've given the X-window a `warm' brown-yellow colour
table\footnote{Again, you will need to re-display the image to see the
effects of this command unless your graphics system is set to 256
colours.}:

\begin{terminalv}
     % lutwarm
\end{terminalv}

If you are not happy with this colour table, you may want to explore a
wider range of colour tables using the \htmlref{LUTEDIT}{LUTEDIT}
command which provides a complete graphical user interface for
manipulating and viewing colour tables.

\begin{terminalv}
     % lutedit image=$KAPPA_DIR/comwest
\end{terminalv}

\subsection{From ICL}

\ICLref\ is a command language designed for use with Starlink applications,
such as \KAPPA.  It is now of some antiquity but is still in use.  The
main advantages for the \KAPPA\ user are that shell metacharacters
like \texttt{[]()$\backslash$"} need not be escaped; command names may be
abbreviated; far fewer executables need be loaded, and therefore it is
slightly faster than using the shell when you want to invoke more than
a few commands on a busy system; there is a wide selection of
intrinsic functions and floating-point arithmetic; and results may be
passed between applications via \ICL\ variables.  However, in these
two demonstrations the command languages are interchangeable apart
from the accept backslash.

Let's start the second example.

\begin{terminalv}
     % icl
\end{terminalv}

This starts \ICL.  System, local and user-defined \ICL
login files are invoked.  Here there is only a system login procedure
which sets up help on Starlink packages, and commands for setting
up definitions for those packages.  One of those commands is \texttt{
kappa}; it is analogous to the \texttt{kappa} command from the shell.
We enter it after receiving the \ICL\ prompt.

You should see something like the following.

\begin{terminalv}
    ICL (UNIX) Version 3.1-9 14/02/2000

    Loading installed package definitions...

       - Type HELP package_name for help on specific Starlink packages
       -   or HELP PACKAGES for a list of all Starlink packages
       - Type HELP [command] for help on ICL and its commands

     ICL> kappa

        KAPPA commands are now available (Version 1.0).

        Type `help kappa' or `kaphelp' for help on KAPPA commands.

\end{terminalv}

Now we run an application, \htmlref{ADD}{ADD}, that adds the pixels in
\file{\$KAPPA\_DIR/comwest.sdf} to those in \file{\$KAPPA\_DIR/ccdframec.sdf}.
Although these images have different
dimensions, the intersection is made.

\begin{terminalv}
     ICL> add $KAPPA_DIR/comwest $KAPPA_DIR/ccdframec
\end{terminalv}

After the first \KAPPA\ command is issued you'll see an arcane
message like this.

\begin{terminalv}
     Loading /star/bin/kappa/kappa_mon into kappa_mon11601 (attached)
\end{terminalv}

It just tells you that the \KAPPA\ monolith is being loaded.
You'll see similar messages for each of the three monoliths when they
are first wanted.

Since we did not give the name of the destination
\xref{NDF}{sun33}{overview_of_an_ndf} that will hold the co-added NDFs,
ADD prompts for it.  Notice that literal parameters are case insensitive.

\begin{terminalv}
     OUT - Output NDF / / > demo1
\end{terminalv}

\begin{terminalv}
     ICL> ndftrace \

        NDF structure /home/soft2/mjc/alpha_OSF1/kappa/package/demo1:
           Title:  Comet West, low resolution

        Shape:
            No. of dimensions:  2
            Dimension size(s):  256 x 256
            Pixel bounds     :  1:256, 1:256
            Total pixels     :  65536

        Data Component:
            Type        :  _REAL
            Storage form:  PRIMITIVE
            Bad pixels may be present
\end{terminalv}

This shows that the \file{demo1} NDF has the same dimensions as the
smaller of the two NDFs.

We were going to display the image on the current graphics device, but
then changed our minds.  A \texttt{!!} in response to a prompt
\htmlref{aborts}{se:abortnull} a task.

\begin{terminalv}
     ICL> display demo1
     MODE - Method to define the scaling limits /'PERCENTILES'/ > !!
     !! SUBPAR: Abort (!!) response to prompt for Parameter MODE
     OBEYW unexpected status returned from task "kapview_mon11601", action
     - "DISPLAY"
     ADAMERR  %PAR, Parameter request aborted
\end{terminalv}

\KAPPA\ uses the graphics database, which records the positions
and extents of graphs and images, collectively called pictures.

\begin{terminalv}
     ICL> picgrid 2 1
\end{terminalv}

This instruction divides the display surface into two equally sized
pictures, side by side.  They are labelled 1 and 2 in the database.
Picture 1 is the current picture, in which future pictures are drawn,
unless we select a new current picture.

Thus in Picture~1 we display an image of Comet West around which we
draw annotated axes.  The backslash causes the current scaling method
to be used.

\begin{terminalv}
     ICL> display comwest axes \
     !! Error accessing file '/home/scratch/mjc/comwest.sdf' -
     !     No such file or directory
     !  HDS_OPEN: Error opening an HDS container file.
     !  NDF_ASSOC: Unable to associate an NDF structure with the '%IN' parameter.
\end{terminalv}

\htmlref{DISPLAY}{DISPLAY} could not find the a \file{comwest.sdf} in
the current directory.  So there is an error message and we are
prompted.  This time we remember to add the environment variable.

\begin{terminalv}
     IN - NDF to be displayed /@comwest_bas/ > $KAPPA_DIR/comwest
     Data will be scaled from 67.46276 to 226.5568.
\end{terminalv}
An image of Comet West should be visible to the left of the screen.

\htmlref{SHADOW}{SHADOW} creates an image that appears like a bas-relief.  We've called
the resulting NDF comwest\_bas.  The backslash causes the current NDF
to be the input NDF for SHADOW.

\begin{terminalv}
     ICL> shadow out=comwest_bas \
\end{terminalv}

We select the right-hand picture created earlier.

\begin{terminalv}
     ICL> picsel 2
\end{terminalv}

As above we display the current NDF, the bas-relief image, with annotated
axes on the right of the raw comet image (we do not need to request the
axes explicitly this time, since the \param{axes} parameter retains the
value used in the previous invocation until changed).

\begin{terminalv}
     ICL> display border \
     Data will be scaled from -4.721756 to 5.697861.
\end{terminalv}

The relief looks best with a grey-scale colour table.  Note that this
does not affect the colour of the border.  \htmlref{LUTGREY}{LUTGREY}
is a procedure which calls a more-general application.  Since it is
the first procedure we've invoked there is a short pause while all the
\KAPPA\ procedures are compiled and loaded.

\begin{terminalv}
     ICL> lutgrey
     Loading procedure file $KAPPA_DIR/kappa_proc.icl
\end{terminalv}

Next we decide to make a hard copy of the bas-relief image.
\htmlref{DISPLAY}{DISPLAY} does this and can add a key of grey levels
and their corresponding values.  The chosen device is \texttt{ps\_l};
this overrides the \texttt{xwindows} device for the duration of DISPLAY.
If this name isn't recognised at your site, issue the
\htmlref{GDNAMES}{GDNAMES} command for a list of your local device
names.  Select the landscape PostScript device.  We scale between
wider limits to reduce the glare.

\begin{terminalv}
     ICL> display key=yes device="/PS"
     IN - NDF to be displayed /@comwest_bas/ >
     MODE - Method to define the scaling limits /'SCALE'/ >
     LOW - Low value for display /187.5625/ > 11
     HIGH - High value for display /-183.453125/ > -8.33
\end{terminalv}

DISPLAY does not send your plot to the printer, since this is hardware
and node dependent.  Therefore, you must issue a shell command
from \ICL\ to perform this action.  That's not difficult---just
insert a \texttt{{!}} before the UNIX command, and in most cases just issue
the command as if you were in the shell, like we do below.

\begin{terminalv}
     ICL> !lpr -P1 pgplot.ps
\end{terminalv}

Shell aliases may also be used, so if \texttt{ri} equates to \texttt{rm -i},
we could remove any unwanted \HDSref\ files.  If you don't
have this symbol, as is likely, then you will receive the appropriate
error message from \ICL.

\begin{terminalv}
     ICL> ri *.sdf
\end{terminalv}

That's the end of the second demonstration.  Of course, these
introductions have only scratched the surface of what \KAPPA\ can
do for you.  You should look at
\latexhtml{Appendix~\ref{ap:classified}}{the
\htmlref{classified lists}{ap:classified}} to search for the desired
function, and then find more details in
\latexhtml{Appendix~\ref{ap:full}.}{the \htmlref{specifications}{ap:full}.}

If you get stuck or something untoward happens, there is a \texttt{Hints}
help topic.

\newpage
\section{Getting started}

\subsection{Running KAPPA}

\KAPPA\ runs from the C-shell and variants, and also from the
interactive command language---\ICLref\@.  Both run
monolithic programmes for efficiency.  Both have their advantages and
disadvantages.  Of the latter, the shell forces you to escape certain
characters, and \ICL\ does not have a \texttt{foreach} to loop
through a wildcarded list of \xref{NDFs}{sun33}{overview_of_an_ndf}.
You may simply prefer the familiar
shell to \ICL, though UNIX commands, including editing, are
accessible from \ICL\ via a \texttt{{!}} prefix.  This is not the
soapbox to expound the intrinsic merits of the two command languages,
but where there are differences affecting \KAPPA, they'll be
indicated.  The choice is yours.

To run \KAPPA\ from the shell just enter the following command.

\begin{terminalv}
     % kappa
\end{terminalv}
This executes a procedure setting up aliases for \KAPPA's
command names and to make help information available.  Then you'll be
able to mix \KAPPA\ commands with the familiar shell ones.

If the \texttt{kappa} command is not recognised, you probably haven't
enabled the Starlink software.  In your \file{.cshrc} or \file{.tcshrc}
file, you insert the line

\begin{terminalv}
     source /star/etc/cshrc
\end{terminalv}
and in \file{.login} you include the equivalent line

\begin{terminalv}
     source /star/etc/login
\end{terminalv}
At non-Starlink sites the \file{/star} path may be different.

To run \KAPPA\ from \ICL\ you have to start up the
command language if you are not already using it.  This requires just
one extra command, namely

\begin{terminalv}
     % icl
\end{terminalv}
You will see any messages produced by system and user procedures, followed
by the \texttt{ICL>} prompt.  Again there is a procedure for making the
commands known to the command language, and not unexpectedly, it too is

\begin{terminalv}
     ICL> kappa
\end{terminalv}
Then you are ready to go.  Not too painful, was it?  In either case
you'll see message from \KAPPA\ telling you which version is
ready for use.

So what do you get for your trouble?
\latexhtml{Appendix~\ref{ap:summary}}{\htmlref{The application
specifications}{ap:summary}} lists
in alphabetical order all the commands and their functions, and
\latexhtml{Appendix~\ref{ap:classified}}{there}
is a \htmlref{classified list}{ap:classified} of the same commands.
Many examples are given in subsequent sections.

\subsection{Issuing Commands}

To run an application you then can just give its name---you will be
prompted for any required parameters.  Alternatively, you may enter
parameter values on the command line specified by position or by
keyword.  More on this in
\latexhtml{Section~\ref{se:param}.}{\htmlref{Parameters}{se:param}}

Commands are interpreted in a case-independent way from \ICLref , but
from the shell they must be given in lowercase.  In \ICL,
commands may also be abbreviated provided they are unambiguous strings
with at least four characters.  Commands shorter than five characters,
therefore, cannot be shortened.  So

\begin{terminalv}
     ICL> CREF
     ICL> crefr
     ICL> CreFra
     ICL> CREFRAM
\end{terminalv}
would all run CREFRAME.  Whereas

\begin{terminalv}
     ICL> FITS
     ICL> FITSI
\end{terminalv}
would be ambiguous, since there are several commands beginning
\texttt{FITS}, and two starting \texttt{FITSI}, namely FITSIN and FITSIMP.

Note if other packages are active there is the small possibility of a
command-name clash.  Issuing such a
command will run that command in the package last activated.  You can
ensure running the \KAPPA\ command by inserting a \texttt{kap\_}
prefix before the command name.  For example,

\begin{terminalv}
      % kap_rotate
\end{terminalv}
will execute \KAPPA's \htmlref{ROTATE}{ROTATE} application.  There may also be a
clash with UNIX commands and shell built-in functions, though there are
now far fewer conflicts than in earlier versions of \KAPPA, with
only {\bf look} being ambiguous.  There is also a {\bf glob} in the
C-shell which might confuse you should you forget that
\htmlref{GLOBALS}{GLOBALS} cannot be abbreviated from the shell.

\medskip

{\large {\bf Since the {\normalsize{\bf KAPPA}} commands are the same
in both the shell and {\normalsize{\bf ICL}}, the \texttt{\%} and \texttt{
ICL>} prompts in the examples and description
\latexhtml{below}{in subsequent pages} are
interchangeable unless noted otherwise.}}

\subsection{\xlabel{se_kaphelp}Obtaining Help\label{se:kaphelp}}

\subsubsection{\xlabel{se_hypertext}Hypertext Help\label{se:hypertext}}

A modified version of this document exists in hypertext form.  One way
to access it is to use the
\xref{\texttt{showme}}{sun188}{displaying_parts_of_documents} command

\begin{terminalv}
     % showme sun95
\end{terminalv}
and a Web browser will appear, presenting the index to the hypertext
form of this document.  The hypertext permits easy location of
referenced documents and applications.  It also includes colour
illustrations.

The \xref{\texttt{findme}}{sun188}{performing_keyword_searches}
command lets you search the Starlink documents by
keywords.  For instance,

\begin{terminalv}
     % findme masking
\end{terminalv}
searches the document looking for the word `masking' in
them.  The level of searching depends on whether a match is found.
The search starts with the document title, the page (section)
titles, and finally the document text.  The deeper the search,
the longer it will take.  There are switches provided to limit the
level of the search.  The search string may include {\bf sed} or
{\bf grep} regular expressions.  See \xref{SUN/188}{sun188}{}
or enter

\begin{terminalv}
     % findme findme
     % findme showme
\end{terminalv}
to learn more about the \texttt{findme} and \texttt{showme} commands.

\subsubsection{Entering the Help System}

To access the \KAPPA\ help use \htmlref{KAPHELP}{KAPHELP}.

\begin{terminalv}
     ICL> kaphelp
\end{terminalv}
The system responds by introducing \KAPPA's help library,
followed by a long list of topics for which help is available,
followed by the prompt \texttt{Topic?}.  These topics are mostly the
commands for running applications, but they also include global
information on matters such as parameters, data structures and
selecting a graphics device.

From \ICLref\ \normalsize you can issue other commands for
obtaining help about \KAPPA.

\begin{terminalv}
     ICL> help kappa
     ICL> help packages
\end{terminalv}
The former is nearly equivalent to entering \texttt{kaphelp}.  However,
it is less easy to use as it lacks many of the navigational aids of
KAPHELP.  The latter gives a summary of Starlink packages available
from \ICL.  If you select the \texttt{KAPPA} subtopic, you'll get
a precis of the package's facilities.  (This is part of an index of
Starlink packages.)

If you have commenced running an application you can still access the
help library whenever you are prompted for a parameter.  See
\latexhtml{Section~\ref{se:parhelp}}{\htmlref{Entering the Help
System}{se:parhelp}} for details.

\subsubsection{Navigating Help Hierarchies}

The help information is arranged hierarchically.  The help system
enables you to navigate the library by prompting when it has either
presented a screen's worth of text or has completed displaying the
previously requested help.  The information displayed by the help
system on a particular topic includes a description of the topic and
a list of subtopics that further describe the topic.

You can select a new topic by entering its name or an unambiguous
abbreviation.  If you press the carriage-return key (\texttt) you
will either continue reading a topic where there is further text to
show, or move up one level in the hierarchy.  Entering a \texttt{CTRL/D}
(pressing D whilst holding the CTRL key) terminates the help session.
See the description of \htmlref{KAPHELP}{KAPHELP} for a full
list of the options available at prompts inside the help system, and
the rules for wildcarding and abbreviating topics.

\subsubsection{Help on KAPPA commands}

Help on an individual \KAPPA\ application is simply achieved by
entering \texttt{kaphelp} followed by the command name, for example

\begin{terminalv}
     % kaphelp centroid
\end{terminalv}
will give the description and usage of the CENTROID command.  There
are subtopics which contain details of the parameters, including
defaults, and valid ranges; examples; notes expanding on the
description; implementation status; and occasionally timing. For
example,

\begin{terminalv}
     ICL> kaphelp hist param ndf
\end{terminalv}
gives details of Parameter \texttt{NDF} in all applications prefixed by
\texttt{HIST}.

(From \ICL\ you can also invoke its help system, thus
\begin{terminalv}
     ICL> help centroid
\end{terminalv}
is similar to \texttt{kaphelp centroid}, though the \ICL\ system has
drawbacks, and you are recommended to run \htmlref{KAPHELP}{KAPHELP}.)

The instruction

\begin{terminalv}
     ICL> kaphelp classified
\end{terminalv}
displays a list of subject areas as subtopics.  Each subtopic lists and
gives the function of each \KAPPA\ application in that
classification.  There is also an alphabetic list which can be obtained
directly via the command

\begin{terminalv}
     ICL> kaphelp summary
\end{terminalv}

\subsection{Changing the Current Directory in ICL}

You should change default directories in \ICLref\ using its
DEFAULT command, and not {\bf cd}.  Thus

\begin{terminalv}
     ICL> default /home/scratch/dro
\end{terminalv}
makes \file{/home/scratch/dro} the default directory for the
\ICL\ \normalsize session, and for existing and future subprocesses,
including application packages.

\subsection{\xlabel{se_exiting}Exiting an Application\label{se:exiting}}

In normal circumstances when you've finished using \KAPPA
nothing need be done from the shell, but to end an \ICLref\ session,
enter the \texttt{EXIT} command to return to the shell.

What if you've done something wrong, like entering the wrong value for a
parameter?  If there are further prompts you can enter the abort code
\texttt{!!} to exit the application.  This is recommended even from the
shell because certain files like your
\xref{NDFs}{sun33}{overview_of_an_ndf} may
become corrupted if you use a crude \texttt{CTRL/C}.  If, however,
processing of the data has begun in the application, it is probably best
to let the task complete, unless it is a long job like image
deconvolution.  If you really must abort, \texttt{CTRL/C} should be hit.
From \ICLref\ this ought to return you to a prompt, but the
processing will continue.  Then you can stop the running process by
`killing' it.  First find the task name

\begin{terminalv}
     ICL> tasks
                           TASKNAME  Process Id

                   ndfpack_mon16528  15186
\end{terminalv}

and then kill it.

\begin{terminalv}
     ICL> kill ndfpack_mon16528
\end{terminalv}
This removes a the \NDFPACK\ monolith.  \NDFPACK\ will be
loaded again once you enter one its commands.  If pressing \texttt{CTRL/C}
several times fails to return you to an \ICL\ prompt then
it's time for the heavy artillery---you may have to kill your window.
Once back to the shell enter \texttt{icl} to return to \ICL, and
then kill the process as described above.

If you have interrupted a task, it may be necessary to delete the
\htmlref{parameter file}{se:defaults} \latex{(Section~\ref{se:defaults})} and
the \htmlref{graphics database}{se:agitidy}
\latex{ (Section~\ref{se:agitidy})}.

\newpage
\section{\xlabel{se_param}Parameters\label{se:param}}

\KAPPA\ is a command-driven package.  Commands have
{\em parameters\/} by which you can qualify their behaviour.
Parameters are obtained in response to prompts or supplied on a
command line.

For convenience, the main aspects of the parameter system
as seen by a user of \KAPPA\ are described below, but note
that most of what follows is applicable to any Starlink
application.

\subsection{\xlabel{se_parsummary}Summary\label{se:parsummary}}

For your convenience, here is a summary of how to use parameters.
If you want more information, go to the appropriate section.

\begin{description}[style=nextline]

\item[Command-line values]% \newline
 On the command line you can supply values by keyword or by position.
 See \latexhtml{Section~\ref{se:cmdlindef}}{\htmlref{Specifying Parameter
 Values on Command Lines}{se:cmdlindef}} for more details including abbreviated
 keywords.

\item [\texttt{ACCEPT, PROMPT, RESET} command-line special keywords] %\newline
 \texttt{ACCEPT} accepts all the suggested defaults that would otherwise
 be prompted.  \texttt{PROMPT} prompts for all the parameters not given on
 the command line, and \texttt{RESET} resets all the suggested defaults to
 their initial values.  You can find more details and examples in
 \latexhtml{Section~\ref{se:parspeckey}.}{\htmlref{Special Keywords: ACCEPT,
 PROMPT, RESET}{se:parspeckey}.}

\item [\texttt{NAME - Prompt string /Suggested default/ $>$} ]%\newline
 This is a schematic of a prompt.  NAME is the parameter's name.
 You normally respond with the value for the parameter, but there are
 special responses available (see below).
 If you just hit the return key, the suggested default becomes the
 parameter value.  Many parameters are defaulted without prompting.  See
 \latexhtml{Section~\ref{se:defaults} and
 Section~\ref{se:parglobals}}{\htmlref{Defaults}{se:defaults} and
 \htmlref{Globals}{se:parglobals}} for more details.
\end{description}

Here is a list of some example parameter values to illustrate the
possible ways you can respond to a prompt.  Where there are command-line
differences, they are noted.

\begin{description}

\item[\texttt{5409.12}]% \newline
 This is a scalar.  Numerical values can be integer, real, or double precision.

\item[\texttt{12,34,56,78}]% \newline
 This is a vector.  They must be enclosed in \texttt{[~]} if the array
 is supplied on the command line, or for character arrays.

\item [\texttt{[[11,21,31],[12,22,32]]}]% \newline
 This is a 3$\times$2 array.  Arrays of dimension $>$ 2 should
 appear in nested brackets.  See
 \latexhtml{Section~\ref{se:pararrays}}{\htmlref{Arrays}{se:pararrays}}
 for more about array values.

\item [\texttt{T}]% \newline
%\vspace{-5ex}
\item [\texttt{no}]% \newline
 This is a \texttt{TRUE} value followed by a \texttt{FALSE} values for
 logical parameters.  Acceptable values are \texttt{TRUE}, \texttt{FALSE},
 \texttt{YES}, \texttt{NO}, \texttt{T}, \texttt{F},  \texttt{Y}, \texttt{N} and their
 lowercase equivalents.  On the command line,
 the parameter name as a keyword means \texttt{TRUE}.  If the name is
 prefixed with \texttt{NO}, the value is set to \texttt{FALSE}.

\item [\texttt{a string}]% \newline
%\vspace{-5ex}
\item [\texttt{"a string"}]% \newline
 This is a string.  Strings need not be quoted at prompts.  Quotes are
 required on the command line if the string includes spaces or
 wildcards, or is a comma-separated array of strings.  There is more in
 \latexhtml{Section~\ref{se:parstrings}.}{\htmlref{Strings}{se:parstrings}.}
 Some parameters offer a selection from a menu to which you give
 an unambiguous abbreviation to select an option.  Other parameters
 can be numerical or a string.  (See
 \latexhtml{Section~\ref{se:parmenu}}{\htmlref{Menus}{se:parmenu}}
 for more information.)

\item [\texttt{filename}]% \newline
%\vspace{-5ex}
\item [\texttt{@123}]% \newline
 This enters a filename (or tape drive).  You give a text filename
 verbatim, and \xref{NDFs}{sun33}{overview_of_an_ndf} without the file
 extension.  Foreign formats will
 usually have the file extension.  Should the filename be a numerical
 value, it must be preceded by an \texttt{@}.  There is more in
 \latexhtml{Section~\ref{se:parstrings}.}{\htmlref{Strings}{se:parstrings}.}

\item [\texttt{min}]% \newline
%\vspace{-5ex}
\item [\texttt{max}]% \newline
 This selects the minimum- or maximum-allowed value, but not all
 parameters have a defined range of permitted values.  See
 \latexhtml{Section~\ref{se:parminmax}.}{\htmlref{MIN and MAX
 parameter values}{se:parminmax}.}

\item [\texttt{{!}}]% \newline
 Enters the null value.  This has a variety of special meanings; which
 one will depend on the particular parameter.  For example, null might
 indicate that an output file is not to be created, or a loop is to be
 ended.  There are more examples in
 \latexhtml{Section~\ref{se:abortnull}.}{\htmlref{Abort and
 Null}{se:abortnull}.}

\item [\texttt{!!}]% \newline
 This aborts the application cleanly.

\item [\texttt{?}]% \newline
%\vspace{-5ex}
\item [\texttt{??}]% \newline
 A single question mark presents the online help for the parameter,
 and then reprompts.  A double question mark leaves you in the help
 system to explore other help information.  See
 \latexhtml{Section~\ref{se:parhelp}}{\htmlref{Help}{se:parhelp}}
 for examples.  These special values are not supported from the
 command line.

\item [\texttt{\textbackslash} ]%\newline
 This accepts the suggested default for the prompted parameter and the
 respective suggested defaults for all subsequent parameters for which
 prompting would otherwise occur.  On the command line \verb+\+
 is an abbreviation of the \texttt{ACCEPT} keyword, and it
 applies to all parameters that would otherwise be prompted.  Note that
 from the shell you write \verb+\\+, as \verb+\+ is a
 shell metacharacter.


\end{description}

\subsection{\xlabel{se_defaults}Defaults\label{se:defaults}}

Command-line values are used mostly for those parameters that are
normally defaulted by the application.  Defaulted parameters enable
applications to have many options, say for controlling the appearance of
some graphical output, without making routine operations tedious because
of a large number of prompts.  The values of normally defaulted
parameters are given in
\latexhtml{Appendix~\ref{ap:full}.}{the \htmlref{application
specifications}{ap:full}.}
You can also find them by obtaining online help on a specific
parameter.  They are enclosed in square brackets at the end of the
parameter description.

\begin{terminalv}
     ICL> kaphelp median param *
\end{terminalv}
gives details of all parameters in application
\htmlref{MEDIAN}{MEDIAN}.  Other packages have similar help
commands.  If you want to override one of these defaults,
then you must specify the parameter's value on the command line.

When you are prompted you will usually be given a suggested default
value in \texttt{/ /} delimiters.  You can choose to accept the default by
pressing carriage return.  For example, \texttt{64} is the suggested value
below.

\begin{terminalv}
     XDIM - x dimension of output array /64/ >
\end{terminalv}
Alternatively, enter a different value

\begin{terminalv}
     XDIM - x dimension of output array /64/ > 109
\end{terminalv}
to override the default.
Some defaults begin with an \texttt{@}.

\begin{terminalv}
     IN - Input image /@starfield/ >
\end{terminalv}
These are associated with files (text and \HDSref ) and devices (graphics
and tape).  If you want to override the default given, you do not have
to prefix your value with an \texttt{@}, {\emph{e.g.}}

\begin{terminalv}
     DEVICE - Name of display device /@xwindows/ > x2w
\end{terminalv}
There are rare cases when the syntax is ambiguous, and it is then that you
need to include the \texttt{@}.
\latexhtml{Section~\ref{se:parstrings}}{\htmlref{Strings}{se:parstrings}}
describes when the \texttt{@} is needed.

From both \ICLref\ \normalsize and the shell the default value can be edited to save
typing by first pressing the \texttt{<TAB>} key.  The editor behaves like the
shell command-line editor.

Defaults may change as data are processed.  Often the current (last)
value of the parameter will be substituted, or a dynamic value is
suggested depending on the values of other parameters.  Here is an
example comprising a loop within an application.

\begin{terminalv}

     NDF - NDF to examine /@horsehead >
     CENTRE - Position at the centre of the listing /'64 64'/ > 100 120
        .              .               .              .
        .              .               .              .
        .              .               .              .
     CENTRE - Position at the centre of the listing /'100 120'/ >
     SIZE - The dimensions (in pixels) of the area to be listed /7/ >
        .              .               .              .
        .              .               .              .
        .              .               .              .

\end{terminalv}
and so on.  Notice that the current values of the centres are the
suggested values at the second prompt.

Current values of parameters are stored in a {\em parameter file}, and
so they persist between sessions.  For tasks run from the shell, this
is an \HDSref\ file \file{\$ADAM\_USER/}{\it $<$application$>$}\file{.sdf},
where {\it $<$application$>$} is the name of the application. (If the
environment variable \texttt{ADAM\_USER} is not defined the parameter
file is situated in \file{\$HOME/adam}).

Unfortunately, tasks invoked from \ICLref\ \normalsize use a monolith parameter
file, which contains the individual application parameter files for
the members.  So for example, \KAPPA\ has \file{kappa\_mon.sdf},
\file{kapview\_mon.sdf}, and \file{ndfpack\_mon.sdf} stored in the same
directory as the individual files.  This duality means that there are
two independent sets of current values for each task depending on
where you ran it from.

The parameter file should not be deleted unless the parameters
values are to be completely reset, or the file has been corrupted in
some way.  If you suspect the latter case, say after
\htmlref{interrupting a task}{se:exiting}\latex{(Section~\ref{se:exiting})},
run {\HDSTRACEref}~\latex{(SUN/102)}
on the file to check its integrity.

\subsection{\xlabel{se_parglobals}Globals\label{se:parglobals}}

\KAPPA\ stores a number of global parameters that are used
as defaults to reduce typing in response to prompts.  Global means that
they are shared between applications.  The most common is the last
dataset (usually \xref{NDF}{sun33}{overview_of_an_ndf}) written or
accessed.  In the example above, this
was \file{horsehead.sdf}.  If you just press \texttt{<CR>} in response to the
prompt, the global value is unchanged.  Only when you modify the
parameter and the application completes without error is the global value
updated.

All global parameters are stored in the HDS file \file{
\$ADAM\_USER/GLOBAL.sdf}, or \file{\$HOME/adam/GLOBAL.sdf} if the \texttt{
ADAM\_USER} environment variable is not defined.  The full list is
given below.
\newline
\newline
\begin{tabular}{@{\hspace{5ex}}l@{~~ --- ~~}l}
\texttt{GLOBAL.DATA\_ARRAY}      & Last NDF or foreign data file accessed or written.\\
\texttt{GLOBAL.GRAPHICS\_DEVICE} & Current graphics workstation.\\
\texttt{GLOBAL.INTERACTIONMODE}  & Current interaction mode. \\
\texttt{GLOBAL.LUT} & Last lookup table file accessed or written. \\
\texttt{GLOBAL.TRANSFORM} & Current transformation structure. \\
\end{tabular}

\KAPPA\ uses the last DATA\_ARRAY written or accessed as
the suggested default value for the next prompt for an NDF structure
or foreign data format.  The same applies to the current lookup table
and transformation structure.  However, the remaining, including the
graphics global parameter are defaulted---you will not be
prompted.  Details of how to control these parameters are given in the
relevant sections.

The values of all global parameters may be inspected with the
\htmlref{GLOBALS}{GLOBALS} task.  You can make them undefined using
\htmlref{NOGLOBALS}{NOGLOBALS}.

\begin{terminalv}
     ICL> globals
     The current data file                : @/home/dro/jkt/ccdpic
     The current graphics device is       : @ps_l
     The current lookup table file is     : @$KAPPA_DIR/spectrum_lut
     The current transformation is        : @/home/dro/deform/warpit
     The current interaction mode is      : <undefined>
     The current message-report level is  : NORMAL
\end{terminalv}
In the above example no interaction mode is defined.  The next time
you call an application that uses the interaction mode you
would be prompted for a value.  (Under normal circumstances you will not
have to enter the \texttt{@} prefix yourself.)

Note that the message-reporting level is not a global parameter, but
is defined by the \texttt{MSG\_FILTER} environment variable.  It controls
the verbosity of informational messages.  Since it applies to all
tasks that report such messages, it is convenient to allow GLOBALS to
show its value.  Further details are in \slhyperref{Verbosity of
Messages}{Section~}{}{se:msglev} for details.  Since it is not a
parameter as such, the message-reporting level is not unset by
\htmlref{NOGLOBALS}{NOGLOBALS}.


\subsection{\xlabel{se_parstrings}Strings\label{se:parstrings}}

Notice that the single or double quotes around strings given in
response to prompts for a character parameter can be omitted.  However,
on the command line these delimiters are needed if the string
contains spaces, otherwise the second and subsequent words could be
treated as separate positional parameters.

From the shell the quotes must be escaped.  For example,

\begin{terminalv}
     % settitle myndf \"A new title\"
\end{terminalv}
would assign the title \texttt{"A new title"} to the NDF called myndf.

To indicate that the parameter should come from a data-structure object,
prefix the name with an \texttt{@} to tell the parameter system that it is
a file name, and not a literal value.

\begin{terminalv}
     TITLE - New NDF title /' '/ @$ADAM_USER/galaxy.mytitle
\end{terminalv}
In this example TITLE has the value of object MYTITLE in \file{
galaxy.sdf}.  If the \texttt{@} were omitted TITLE would be \file{
"\$ADAM\_USER/galaxy.mytitle"}.  You will need the \texttt{@} prefix if your
file name is a number.  Note that the file extension should not be
included when giving the name of an \HDSref\ data file, including
\NDFref{NDFs}.  Otherwise HDS will look for an object called SDF
nested within the file.

Responses to prompts are case insensitive for comparison purposes.
Strings for character components in NDFs, plot captions and labels
are treated literally.

\subsection{\xlabel{se_pararrays}Arrays\label{se:pararrays}}

If a parameter requires an array of values, the series
should be in brackets separated by commas or spaces.  For example,

\begin{terminalv}
     PERCENTILES - List of percentiles > [25,95.5,75]
\end{terminalv}
would input three values: 25, 95.5, and 75 into the real parameter
PERCENTILES.  If the application is expecting an exact number of values
you will be reprompted, either for all the values if you give too many,
or the remaining values if you supply too few.  There is one exception
where you can omit the brackets---a fairly common one---and that is in
response to a prompt for a one-dimensional numeric array as above.

From the shell you must escape the brackets.

\begin{terminalv}
     % display key=yes mode=pe percentiles=\[95,5\]
     % display key=yes mode=pe percentiles='[95,5]'
     % display key=yes mode=pe percentiles="[95,5]"
\end{terminalv}
All the above do this.  Single quotes have the advantage
that you can protect all the metacharacters that lie between
the quotes, so you don't need to escape each metacharacter.

Arrays of parameter values should appear in nested brackets.  For
example,

\begin{terminalv}
     CVALUE - Component values > [[2,3],[5,4],[7,1]]
\end{terminalv}
supplies the values for a 2$\times$3-element parameter array, where
element (1,3) has value 7.

\subsection{\xlabel{se_abortnull}Abort and Null\label{se:abortnull}}

Responding to a prompt with a null value \texttt{{!}} will not necessarily
cause the application to abort, but it can force a suitable default to
be used, where this is the most-sensible action.  A further use is
when an optional file may be created, such as a lookup table; a \texttt{
!} entered in response to the prompt for the filename means that no
file is to be output.  Many tasks use null as a default for optional
files. In some applications, a null ends an interactive loop.

Responding to a prompt with \texttt{!!} will abort the application.  This
process includes the various tidying operations such as the unmapping
and closing of files.  Any other method of stopping an application
prematurely can leave files mapped or corrupted.

\subsection{\xlabel{se_parhelp}Help\label{se:parhelp}}

To get help about a parameter enter \texttt{?}.  Usually, this will give
access to the help-library information for that parameter, for example,

\begin{terminalv}
     BOX - Smoothing box size /3,3/ > ?

     BLOCK

       Parameters

         BOX() = _INTEGER (Read)
            The sizes (in pixels) of the rectangular box to be applied to
            smooth the data.  These should be given in axis order.  A value
            set to 1 indicates no smoothing along that axis.  Thus, for
            example, BOX=[3,3,1] for a three-dimensional NDF would apply a
            3x3-pixel filter to all its planes independently.

            If fewer values are supplied than the number of dimensions of
            the NDF, then the final value will be duplicated for the
            missing dimensions.

            The values given will be rounded up to positive odd integers, if
            necessary, to retain symmetry.

     BOX - Smoothing box size /3,3/ >
\end{terminalv}
and then immediately reprompt you for the parameter.  There are
occasions when information about the parameter is insufficient; you
may require to examine the examples or the description of related
parameters.  This can be achieved by entering \texttt{??} to the prompt.
You can then delve anywhere in the help information.  When you exit the
help system you're reprompted for the parameter.

\subsection{\xlabel{se_parmenu}Menus\label{se:parmenu}}

Some parameters offer menus from which you select an option.
You do not have to enter the full option string, but merely a string
that selects a choice unambiguously.  In many cases this can be as
little as a single character.  Here is an example,

\begin{terminalv}
     MODE - Method for selecting contour heights /'Free'/ > ?
         The method used to select the contour levels.  The options are
         described below.

           "Area"      - The contours enclose areas of the array for
                         which the equivalent radius increases by equal
                         increments.  You specify the number of levels.
           "Automatic" - The contour levels are equally spaced between
                         the maximum and minimum pixel values in the
                         array.  You supply the number of contour
                         levels.
           "Free"      - You define a series of contour values
                         explicitly.
           "Linear"    - You define the number of contours, the start
                         contour level and linear step between contours.
           "Magnitude" - You define the number of contours, the start
                         contour level and step between contours.  The
                         step size is in magnitudes so the nth contour
                         is dex(-0.4*(n-1)*step) times the start contour
                         level.

\end{terminalv}
where an \texttt{F} would be sufficient to select the \texttt{"Free"} option,
but at least two characters would be needed if you wanted \texttt{"Area"}
or \texttt{"Automatic"}.

Some parameters permit a mixture---a choice from a menu, or a
numerical value within a range.  The options are described in full in
\latexhtml{the help system and
Appendix~\ref{ap:full}.}{\htmlref{application specifications}{ap:full}.}

\subsection{\xlabel{se_envvar}Environment Variables\label{se:envvar}}

Environment variables operate both on the command line and prompts,
and both from the shell and \ICLref\@.  Thus if \texttt{IMAGEDIR} is
an environment variable pointing to a directory containing the
NDF called ngc1365, you could access it at a prompt as shown
below.

\begin{terminalv}
     IN - Input image /@starfield/ > $IMAGEDIR/ngc1365
\end{terminalv}

\subsection{\xlabel{se_cmdlindef}Specifying Parameter Values on Command Lines
\label{se:cmdlindef}}

Parameters may be assigned values on the command line.  This is useful
for running tasks in batch mode and in procedures, and for specifying
the values of parameters that would otherwise be defaulted.  A
command-line parameter will prevent prompting for that parameter unless
there is an error with the given value, say giving an alphabetic
character string where a floating-point value is demanded.

There are two ways in which parameter values may be given on the
command line: by keyword and by position.  The two forms may be
mixed with care.  The parser looks for positional parameters then
keywords, so you can have some positional values followed by keyword
values.  See some of the examples presented in
\latexhtml{Appendix~\ref{ap:full}.}{the \htmlref{application
specifications}{ap:full}.}

\subsubsection{\xlabel{se_parkeyword}Keyword\label{se:parkeyword}}

Keywords may appear in any order.
Here is an example of command-line defaults using keywords.

\begin{terminalv}
     ICL> picdef current fraction=0.4
\end{terminalv}
FRACTION is a real parameter.  CURRENT is a logical parameter; by giving
just its name it is assigned the value \texttt{TRUE}.  \texttt{CURRENT=T} would
have the same effect.  To obtain a \texttt{FALSE} value for a logical parameter
you add a \texttt{NO} prefix to keyword, for example,

\begin{terminalv}
     icl> picdef nocurrent
\end{terminalv}
would be equivalent to the following.

\begin{terminalv}
     icl> picdef current=false
\end{terminalv}

\subsubsection{\xlabel{se_parabbrev}Abbreviations\label{se:parabbrev}}

There is an experimental system that allows you to abbreviate
parameter keywords to the minimum unambiguous length.  To use it, you
must first create an environment variable called \texttt{ADAM\_ABBRV}
with an arbitrary value.

So for example
\begin{terminalv}
     % setenv ADAM_ABBRV true
     % display mo=pe pe=\[5,95\] ba=blue
\end{terminalv}
would display an NDF between the 5 and 95 percentiles, and marking bad
pixels in blue.

If you give an ambiguous keyword, the parameter system will present
the list of possible keywords and ask you to select the one you
intended.

\subsubsection{\xlabel{se_parposition}Position\label{se:parposition}}

Alternatively, you can specify command-line values by position.
Here is an example.

\begin{terminalv}
     % thresh raw clipped 0 255
\end{terminalv}
This applies thresholds to the NDF called raw to form a new NDF
called clipped.  The values between 0 and 255 are unchanged.  Note that
trailing parameters may be omitted---NEWLO and NEWHI in the above
example---but intermediate ones may not.  The position of a parameter
can be found in the \texttt{Usage} heading in
\latexhtml{Appendix~\ref{ap:full}}{the \htmlref{application
specifications}{ap:full}} or the help for the application.

\subsubsection{\xlabel{se_keyvspar}Keyword versus Positional Parameters
\label{se:keyvspar}}

For tasks with a few parameters, using position is quick and
convenient.  However, in complex applications with many parameters it
would be tedious not only to enter all the intermediate values between
the ones you want to define, but also to remember them all.  Another
consideration is that some parameters do not have defined positions
because they are normally defaulted.  Keywords may also be
\htmlref{abbreviated}{se:parabbrev}\latex{(Section~\ref{se:parabbrev})}.
Thus the keyword technique is recommended for most parameters,
especially in scripts and procedures.  Unabbreviated keywords insulate
scripts against new keywords and positional changes that are sometimes
needed.

See
\latexhtml{Section~\ref{se:custom}}{\htmlref{custom \KAPPA\
commands}{se:customcom}}
if you want to learn how further to abbreviate command strings to
reduce typing for manual operation.

\subsubsection{\xlabel{se_parspecial}Special Behaviour\label{se:parspecial}}

Sometimes specifying a parameter on the command line induces different
behaviour, usually to inhibit a loop for procedures, or to eliminate
unnecessary processing.
For instance,

\begin{terminalv}
     ICL> centroid blob init="51,42" mode=i
\end{terminalv}
will determine the centroid near the point (51,42) in the NDF called
blob, and then it exits, whereas without the INIT value you would be
reprompted for a further initial position; and

\begin{terminalv}
     % display galaxy mode=sc high=3000 low=1000
\end{terminalv}
prevents the calculation of the extreme values of the NDF called
galaxy that are normally given as suggested defaults for parameters
HIGH and LOW, because HIGH and LOW are already known.

\subsection{\xlabel{se_parspeckey}Special Keywords: ACCEPT, PROMPT, RESET
\label{se:parspeckey}}

Another way in which prompts and default values can be controlled is
by use of the keywords ACCEPT, PROMPT and RESET.

The RESET keyword causes the {\em suggested\/} default value of all
parameters (apart from those already specified before it on the
command line) to be set to the original values specified by the
application or its interface file.  In other words global and current
values are ignored.

The PROMPT keyword forces a prompt to appear for every application
parameter.  This can be useful if you cannot remember the name of a
defaulted parameter or there would be too much to type on the command
line.  However, it may prove tedious for certain applications that have
tens of parameters, most of which you normally never see.  You can abort
if it becomes too boring.

The ACCEPT keyword forces the parameter system to accept the suggested
default values either for every application parameter if the keyword
appears on the command line, or all subsequent parameters if it is
supplied to a prompt.  In other words, those parameters that would
normally be prompted with a value between \texttt{/ /} delimiters take
the value between those delimiters, {\emph{e.g.}}\ XDIM we saw in
\latexhtml{Section~\ref{se:defaults}}{\htmlref{parameter
defaults}{se:defaults}} would take the value \texttt{64}.  Parameters
that are normally defaulted behave as usual.  The ACCEPT keyword needs
to be used with care in scripts because not every parameter has a
default, and therefore must be given on the command line for the
application to work properly.  For example,
\htmlref{CREFRAME}{CREFRAME} must have a value specified for parameter
OUT, the name of the output NDF.  If we run the application
like this:

\begin{terminalv}
     ICL> creframe accept
\end{terminalv}
it would fail in the sense that it would still have to prompt for a
value---it does not know where to write the output NDF.
However, if we run CREFRAME like this:

\begin{terminalv}
     ICL> creframe out=stars accept
\end{terminalv}
it would generate an output image using default values for all the
parameters except OUT, and write the output to file \file{stars.sdf}.
Another point to be wary of is that some applications have loops,
{\emph{e.g.}}\ \htmlref{LOOK}{LOOK},
\htmlref{LUTABLE}{LUTABLE}, and if you use the ACCEPT keyword
it will only operate the first time the application gets a parameter
value.

Sometimes the keyword ACCEPT can be used without any parameter value
specifications on the command line.
For example, we could follow the above command by the command:

\begin{terminalv}
     ICL> look accept
\end{terminalv}
and the central 7$\times$7 array of the image created by CREFRAME would
be displayed on your terminal without any parameter values being
prompted.  The symbol \verb+\+ has the same effect as ACCEPT
when used on the command line or at prompts, thus:

\begin{terminalv}
     ICL> look \
\end{terminalv}
would have the same effect as the previous example---and is quicker
to type.  In command lines from the shell, the backslash is a
metacharacter and has to be escaped.  The easiest way to do that is
to double the backslash.

\begin{terminalv}
     % look \\
\end{terminalv}

How do you find out which parameters have suggested defaults, as
opposed to those that are normally defaulted?  Well, a good rule of
thumb is that parameters for output files (images, lookup tables and
text) will not have a default, but the remainder will.  There are some
exceptions, such as where null is the default for optional files.
Consulting the description of the parameters should give the suggested
defaults, where appropriate.  If a parameter is given a suggested
default it will have a line beginning \texttt{ppath} or a \texttt{default}
line.  If you want to use ACCEPT for an automatic procedure or batch
job you could do some tests to find which parameters get prompted and
then put them on the command line in your procedure.

The RESET and ACCEPT keywords will work in tandem.  So for instance,

\begin{terminalv}
     ICL> look reset accept
\end{terminalv}
will reset the suggested defaults of LOOK to their
original, preset values, and accept these as the parameter values.

These special keywords may be abbreviated to no fewer than
two characters, if you have enabled
\htmlref{keyword-abbreviation}{se:parabbrev}\latex{ (Section~\ref{se:parabbrev})}.

\subsection{\xlabel{se_parminmax}MIN and MAX parameter values
\label{se:parminmax}}

Many parameters have well-defined ranges of allowed values.  In some
cases it is useful to assign the maximum or minimum value to the
parameter.  Rather than give some numerical value, you can instead
supply \texttt{MIN} to select the minimum-allowed value, and \texttt{MAX}
to select the maximum.  This applies both on the command line and at
prompts.  In the example,

\begin{terminalv}
     % block wlim=max
\end{terminalv}
Parameter WLIM takes its maximum (1.0) meaning that if any of the
input pixels in the smoothing box is bad, the corresponding output
pixel is set bad.

Consult the reference section or the online help to see if a given
parameter has such a range.  If you attempt to use \texttt{MIN} and
\texttt{MAX} where there is no range defined, you'll see an error
message like

\begin{terminalv}
    !! SUBPAR_MNMX: Parameter FONT - no upper limit set
\end{terminalv}
and you'll be invited to give another value.

\subsection{\xlabel{se_groups}Specifying Groups of Objects\label{se:groups}}

Some parameters are describing in the reference documentation as being
associated with a \emph{group} of objects.  For instance, some parameters may
require a group of strings, others may require a group of numerical
values, or data files.  No matter what the nature of the object, groups
are specified using a syntax called a \emph{group expression}.  The GRP
library is used to interpret these group expressions, and the \xref{GRP
documentation}{sun150}{} \latex{ (SUN/150)}, should be consulted for
full details.  A summary is given here.

A group expression can identify the members of the group in any of the
following ways:

\begin{itemize}
\item As a comma-separated list ( \emph{e.g.} \texttt{"12.1, 23.2, 1.3"}
     or \texttt{"HH1\_B1S1,HH2\_B1S2"} ).

\item By reading them from a text file (see
     ``\htmlref{Indirection}{SEC:IND}'').

\item By modifying an existing group of objects using editing
     specified within the group expression (see
     ``\htmlref{Modification}{SEC:MOD}'').
\end{itemize}

A typical group expression will often include characters that are of
significance to the shell.  If the group expression is supplied on the
command line it may be necessary to place quotes around the string to
prevent the shell from removing these characters.  Two lots of quotes are
usually required, as in the following example where a group expression
(in this case, a comma-separated list) is used to specify a plotting
style:

\begin{terminalv}
   % display style='"grid=1,colour(grid)=red,title=My new image"'
\end{terminalv}

These quotes are not required if the group expression is given in
response to a prompt.

If the supplied group expression is terminated with a hyphen, the
user is re-prompted for another group expression (using the same
parameter).  The objects specified by this second group expression are
added to those specified by the first.  This re-prompting continues until
a group expression is supplied that does not end with a hyphen.

Certain classes of objects have additional features, for instance if the
objects are the names of data files, then \htmlref{wild-card
characters}{SEC:NDF} are allowed in the supplied values.

\subsubsection{\label{SEC:IND}Indirection}

It is sometimes convenient to store the strings specifying the objects to
be used within a text file.  The name of the text file can then be given
in response to a prompt for a group expression, rather than giving a long
list of explicit values.  This is done by preceding the name of the text
file with an up-arrow (\verb+"^"+) character.  For instance, the group
expression \texttt{{"}}\^{}\verb+style.dat+\texttt{{"}} would result in the file
\verb+style.dat+ being opened and the strings read from the file.  Each
line within the file is considered to be a group expression, and is
processed in the same way as a group expression supplied directly.  In
particular, a text file may contain references to other text files.  If
the file \verb+style.dat+ contained the following two lines:

\begin{terminalv}
   grid=1,colour(grid)=red,border=1
   colour(border)=red,^labels.dat
\end{terminalv}

then the strings \verb+grid=1+, \verb+colour(grid)=red+,
\verb+border=1+ and \verb+colour(border)=red+ would be returned to the
application, and in addition the file \verb+labels.dat+ would be
searched for further strings.  This nesting of text files can go down
to seven levels.  Text files may also contain comments.  Anything
occurring after a \texttt{{"}}\verb+#+\texttt{{"}} character is ignored.  To
ignore an entire line the \verb+#+ character must be in column 1 (any
blanks in front of the \verb+#+ character are considered to be
significant).

\subsubsection{\label{SEC:GRPEDIT}Editing}

A group expression can contain a request to edit the supplied strings
before passing them to the application.  The editing facilities provided
are fairly simple.  You can:

\begin{itemize}
\item Add a specified prefix to the start of each string.
\item Add a specified suffix to the end of each string.
\item Replace all occurences of a given sub-string in each string.
\item Any combination of the above.
\end{itemize}

To perform this editing, you:
\begin{enumerate}

\item Enclose the group expression specifying the strings to be edited
within curly braces (\verb+"{"+ and \verb+"}"+).  Note, if no prefix
or suffix is supplied, and the group expression is not a comma-separated
list, then the curly braces can be omitted.

\item Precede the opening curly brace with the prefix (if any) to be added
to the start of each string.

\item Follow the closing curly brace with the suffix (if any) to be added
to the end of each string.

\item Append a string specifying the substitution to be performed (if any)
to the end of the whole thing.  This string should be of the form
\verb+|<old>|<new>|+ where \verb+<old>+ is the text to be replaced and
\verb+<new>+ is the text with which to replace it.  Note, the substitutions
occur \emph{before} any specified prefix or suffix is added to the
strings.

\end{enumerate}

For instance;

\begin{terminalv}
   A{^file}B|my|your|
\end{terminalv}

This will read strings from the text file \verb+file+.  Each occurence of
the string \texttt{"my"} will then be replaced by \texttt{"your"}.  The resulting
strings will then have \texttt{"A"} added at the start, and \texttt{"B"} added at the
end.

\subsubsection{\label{SEC:MOD}Modification}

A group of objects can be given by specifying some editing to
apply to another already existing group of objects.  For instance,
if the string \verb+new_*b|_ds|_im|+ was given in response to a request
for a group expression, then the following steps occur:

\begin{itemize}
\item   Each element in some existing group of objects (identified in
     the description of the parameter concerned) is substituted
     in turn for the \verb+"*"+ character.
\item  Any occurrences of the string \texttt{"\_ds"} is replaced by the string
     \texttt{"\_im"}.
\item  The string \texttt{"b"} is added to the end of the string.
\item  The string \texttt{"new\_"} is added to the start of the string.
\end{itemize}

Thus if the existing group contained the strings \verb+file1_ds+ and
\verb+file2_ds+, the resulting group would be \verb+new_file1_imb+
and \verb+new_file2_imb+.  Note, this facility is only available if
the parameter description identifies an existing group which will be used
as the basis for the modified strings.

\subsubsection{Ignoring Syntax Characters}

You can see from the above that several characters have special meanings
within group expressions.  Examples are \verb+^+ \verb+|+ \verb+-+.
This may sometimes cause a problem if you want to include
these characters within the strings being passed to the application.  For
instance, if you want to specify a group of data files using a shell
pipe-line, you may want to do something like:

\begin{terminalv}
   % display
     IN - NDF to be displayed > `find . -newer a.fit | grep good `
\end{terminalv}

Within a group expression, the \verb+|+ character indicates a request for
a string substitution (as described above).  In this case, the GRP library
considers the request to be incomplete because there is only one \verb+|+
character, and issues an error report.  Of course, the \verb+|+ character
was actually intended to indicate that the output from the \verb+find+
command should become the input to the \verb+grep+ command.  This can be
accomplished by \emph{escaping} the \verb+|+ character so that its
special meaning within the context of a group expression is ignored.

To escape a group expression syntax character, it should be preceded
with a backslash (\verb+"\"+).  So the above command should be changed to:

\begin{terminalv}
   % display
     IN - NDF to be displayed > `find . -newer a.fit \| grep good `
\end{terminalv}

Any other special character can be escaped in the same way.  For instance,
you can escape commas within text strings using this method.

\subsubsection{Groups of Data Files}
\label{SEC:NDF}

If a group expression is used to specify a list of input data files
(\xref{NDFs}{sun33}{overview_of_an_ndf} or positions lists), then file
names may be specified that contain wild-card characters
(\verb+"*"+ and \texttt{"?"}---character classes can also be matched using
strings such as \texttt{"[0-9]"}, \texttt{"[abcd]"}).  These will be
expanded into a list of explicit file names before returning the group
to the application.

If a group of output data files are specified by modification of a
previously supplied group of input data files, the asterisk in the output
group expression refers just to the file base-name (\emph{i.e.} without
directory path or file type).  So, for instance, the group expression
\verb+B_*+ would cause each output file name to be equal to the
corresponding input file name, but with \texttt{"B\_"} added to the start of the
file base name.  Thus an input file \verb+/home/dsb/data.fit+ would result
in an output file \verb+/home/dsb/B_data.fit+.  If no directory is given
in the output group expression, the directory associated with the input
file is then added to the start of the file name.  Likewise, any HDS path
or file type is inherited from the input file if none are given in the
output group expression.

If the final character in a group expression is a colon (:), then a list
of the data files represented by the group expression (minus the colon)
is displayed, but no data files are actually added to the group of files
to be processed.  The user is then re-prompted for another group
expression, using the same parameter.

If an HDS container file\footnote{HDS container files can usually be
identified by the fact they have a file type of \file{.sdf}.  They can be
used to store one or more standard Starlink NDF structures.} is supplied
that contains two or more NDF structures, then each NDF within the
container file is processed as a separate image.  NDFs that are contained
within an extension of another NDF are not included.

If a group of native output NDFs are created by modification of a group
of native input NDFs (\emph{i.e.} if the supplied string includes an
asterisk), then the structure of each output container file will be
copied from the corresponding input container file.  For instance, if the
container file \verb+o66_int.sdf+ contains 16 NDFs in components
\verb+I1+ to \verb+I16+, then specifying \verb+"o66_int"+ when asked
for a group of input images will result in all 16 NDFs being used.  If the
corresponding output images are specified using the string \verb+"*_A"+
then a single output file named \verb+o66_int_A.sdf+ will be created.  The
structure of this file will be copied from the input file, and will
therefore contain the 16 output NDFs in components \verb+I1+ to
\verb+I16+.

\subsubsection{Examples}

\begin{itemize}
\item If an application asks for a group of input data files, the following
are all possible responses:
\medskip

\begin{terminalv}
   b1,b2,b3,b4
\end{terminalv}
\latex{\vspace{-4mm}}

This means ``Use the NDFs stored in files \verb+b1.sdf+, \verb+b2.sdf+,
\verb+b3.sdf+ and \verb+b4.sdf+''.  If \htmlref{`on-the-fly format
conversion'}{se:autoconvert} (see \latex{Section~\ref{se:autoconvert}
and} \xref{SUN/55}{sun55}{}) is being used, then this example would pick
up data files with the highest priority data format (\emph{i.e.} the
format nearest to the start of the list of formats supplied in
environment variable \texttt{NDF\_FORMATS\_IN}).  So for instance, if the current
directory contained both \verb+b1.sdf+ and \verb+b1.fit+, then only one
file would be used, depending on the relative positions of the \file{.sdf}
and \file{.fit} formats within \texttt{NDF\_FORMATS\_IN}.  If you want to restrict
things explicitly to a particular data format, then you should include
the corresponding file type in the group expression.  An example such as:

\begin{terminalv}
   b1.fit,b2.fit,b3.fit,b4.fit
\end{terminalv}
\latex{\vspace{-4mm}}
would read just the specified FITS files.
\medskip

\begin{terminalv}
   cena_b1-
\end{terminalv}
\latex{\vspace{-4mm}}
This means ``Use \verb+cena_b1.sdf+ and then (because of the hyphen
at the end) ask the user for more data files''.
\medskip

\begin{terminalv}
   *
\end{terminalv}
\latex{\vspace{-4mm}}
This means ``Use all accessible data files in the current directory''.
\medskip

\begin{terminalv}
   hh1_b1s*_ds
\end{terminalv}
\latex{\vspace{-4mm}}
This means ``Use \verb+hh1_b1s1_ds.sdf+, \verb+hh1_b1s2_ds.sdf+, \emph{etc.}''.
\medskip

\begin{terminalv}
   hh1_b[12]s*_ds
\end{terminalv}
\latex{\vspace{-4mm}}
This means ``Use \verb+hh1_b1s1_ds.sdf+, \verb+hh1_b1s2_ds.sdf+,
\emph{etc.}, and also \verb+hh1_b2s1_ds.sdf+, \verb+hh1_b2s2_ds.sdf+,
\emph{etc.}''.  The string \texttt{"[12]"} matches either a single
character \texttt{"1"}
or a single character \texttt{"2"}.  The string \texttt{"[0-9]"} would match any single
digit character.  The string \texttt{"[a-z]"} would match any single lowercase
alphabetical character.
\medskip

\begin{terminalv}
   data.fit[12]
\end{terminalv}
\latex{\vspace{-4mm}}
This means ``Use files \verb+data.fit1+ and \verb+data.fit2+ if they
exist.  If neither of these files exists, use the twelth image extension
in the multi-extension FITS file \verb+data.fit+''.
\medskip

\begin{terminalv}
   ^files.lis
\end{terminalv}
\latex{\vspace{-4mm}}
This means ``Read the names of data files from the text file
\verb+files.lis+.''
\medskip

\begin{terminalv}
   ../data/*
\end{terminalv}
\latex{\vspace{-4mm}}
This means ``Use all accessible data files contained in the directory
\verb+../data+''.
\medskip

\begin{terminalv}
   data_{new,old,back}
\end{terminalv}
\latex{\vspace{-4mm}}
This means ``Use files \verb+data_new.sdf+, \verb+data_old.sdf+ and
\verb+data_back.sdf+''.
\medskip

\begin{terminalv}
   {^files}_A
\end{terminalv}
\latex{\vspace{-4mm}}
This means ``Read names of data files from text file \verb+files+ and
append \verb+_A+ to the end of each one''.
\medskip

\begin{terminalv}
   `grep -l "OBJECT  = 'm57'" *.fit`
\end{terminalv}
\latex{\vspace{-4mm}}

The string is enclosed in back quotes (\verb+`+) which causes the string
to be executed as a shell command, and the resulting output to be used as
the group expression.  Thus this example means ``Use all FITS files that
contain an OBJECT keyword equal to \texttt{'m57'}''.


\item If an application asks for a group of output data files, the following
are possible responses:

\begin{terminalv}
   file1,file2,file3
\end{terminalv}
\latex{\vspace{-4mm}}
This means ``Create \verb+file1.sdf+, \verb+file2.sdf+ and \verb+file3.sdf+''.
\medskip

\begin{terminalv}
   ^out.dat
\end{terminalv}
\latex{\vspace{-4mm}}
This means ``Read the names of the output data files from text file \verb+out.dat+''.
\medskip

\begin{terminalv}
   *_ds
\end{terminalv}
\latex{\vspace{-4mm}}
This means ``Append the string \texttt{"\_ds"} to the end of all
                      the input data file names.''
\medskip

\begin{terminalv}
   ../bk/*|_ds|_bk|
\end{terminalv}
\latex{\vspace{-4mm}}
This means ``Substitute the string \texttt{"\_bk"} for all  occurrences of the string
\texttt{"\_ds"} in the  input data file names, and put the files in
directory \verb+../bk+''.
\end{itemize}

\subsection{\xlabel{se_parout}Output Parameters\label{se:parout}}

Not only can programmes have parameters to which you supply values,
but they can also write out the results of their calculations to {\em
output\/} or {\em results parameters}.  This makes the results
accessible to subsequent tasks and to shell and \ICLref\ variables.
Example results are statistics like the standard deviation or the FWHM
of the point-spread function, the co-ordinates of points selected by a
cursor, or the attributes of an NDF.  They are not data files created
by the application.  In
\latexhtml{Appendix~\ref{ap:full}}{the \htmlref{application
specifications}{ap:full}} they are listed separately from
other parameters.

From the shell you can access these output parameters using the
\KAPPA\ tool \htmlref{PARGET}{PARGET}.  Suppose that you
want to subtract the mean of an NDF called myndf to make an
a new NDF called outndf.

\begin{terminalv}
    % stats myndf > /dev/null
    % set mean = `parget mean stats`
    % csub myndf $mean outndf
\end{terminalv}

\htmlref{STATS}{STATS} calculates the statistics of myndf,
the displayed output being discarded.  PARGET reports the mean value
which is assigned to shell variable \texttt{mean}.  Thereafter the mean
value is accessible as \texttt{\$mean} in that process.  Thus the final
command subtracts the mean from myndf.

You can obtain vector values too.

\begin{terminalv}
    % ndftrace myndf > /dev/null
    % set axlab = `parget alabel ndftrace`
    % display otherndf style="'label(1)=$axlab[1],label(2)=$axlab[2]"' axes
\end{terminalv}

This displays the image otherndf surrounded by axes, but the plot's
axis labels originate from another dataset called myndf
\footnote{ The \att{style} parameter specifies the appearance of the annotated axes,
and is given as a comma-separated group of \emph{attribute settings}, each
of which is a \texttt{name=value} string specifying the attribute name and
value.  In this case, we assign values to the two attributes \att{label(1)}
and \att{label(2)}.  The whole group must be enclosed in single quotes to
prevent the parameter system splitting the string at the commas.  The
string must then also be enclosed in double quotes to prevent the UNIX shell
from interpreting the parantheses and equals signs.  A simpler way of
specifying a plotting style is to put the attribute settings in a
\htmlref{text file}{se:style} \latex{(see Section~\ref{se:style})}.}.
There are more examples in the \xref{{\sl C-Shell Cookbook}}{sc4}{}.

At the time of writing, \ICLref\@ only permits scalar variables.  To do the
first example above from \ICL, you would enter something like this.

\begin{terminalv}
    ICL> stats myndf mean=(vmean)
    ICL> csub myndf (vmean) outndf
\end{terminalv}
\texttt{vmean} is an ICL variable.  The parentheses have the same effect
as the \verb+$+ in the C-shell example, meaning ``the value of'' the
variable.  Note that you can't redirect the output to \file{/dev/null}.

If you use the \htmlref{PROMPT keyword}{se:parspeckey}
\latex{(see Section~\ref{se:parspeckey})} for an application with output
parameters, the programme will bizarrely prompt you for these.  It is
not asking for a value, but a {\em location\/} where to store the
value.  It is strongly recommended that you just accept the default
(normally zero) so that the values are written to their parameter
file, and hence permits \htmlref{PARGET}{PARGET} to work.

\section{\xlabel{se_msglev}Verbosity of Messages\label{se:msglev}}

Informational messages (as opposed to error messages) are tagged with a
reporting level that permits some control in the detail and quantity of
messages you see.  Three levels are supported in \KAPPA.
\begin{description}
\item [NORMAL] the normal reporting level, that aims to give a balanced,
`Goldilocks' level of reporting.  However, for historical reasons
most messages are set to this level, although that is gradually changing.
It excludes messages tagged as verbose.

\item [VERBOSE] reports additional information, such as instructions
or warnings for new users to a task, progress reports in a
long-running command, and detailed analysis.

\item [QUIET] reports only the most important messages that you would
not want to miss, but eliminates the messages at normal and verbose
levels.
\end{description}

The chosen level is set through environment variable \texttt{MSG\_FILTER}.
Abbreviations may be used such as \texttt{"N"}, \texttt{"v"}, \texttt{"Qu"} for
normal, verbose, or quit reporting respectively.  The normal mode is the
default if you do not define \texttt{MSG\_FILTER}.

\newpage
\section{\xlabel{se_graphdev}Graphics Devices and Files\label{se:graphdev}}

\subsection{\xlabel{se_selgradev}Selecting a Graphics Device
\label{se:selgradev}}

You can find the list of available devices and their names with task
\htmlref{GDNAMES}{GDNAMES}.  Names can be abbreviated provided they
remain unambiguous.  Two alternative naming schemes are supported, and
the list produced by GDNAMES will include both.

\begin{description}
\item [PGPLOT] A general graphics device specification is of the form
\verb+<file>/<type>+, where \verb+<type>+ indicates the type of the
graphics device (\emph{e.g.} PostScript printer, X-window, \emph{etc.})
and \verb+<file>+ is an optional string which indicates either a file in
which the graphical output should be stored or a specification for a
particular device of the specified type.  For instance, \verb+m31.ps/VPS+
produces a file called \verb+m31.ps+ containing output suitable for
sending to a PostScript printer in portrait mode, and
\verb+xwindows2/GWM+ sends graphical output to the GWM X-window with name
\verb+xwindows2+.  If the \verb+<file>+ string is omitted, a default
device-dependent value is used (for instance, \verb+pgplot.ps+ for
postscipt files and \verb+xwindows+ for X-windows).

\item [GNS] For compatibility with previous versions of \KAPPA\
graphics devices may also be specified using a scheme that approximates
closely to that of the Starlink Graphics workstation Name Service (GNS)
library (see \xref{SUN/57}{sun57}{GKSWorkstationNames}).  In this scheme a complete
workstation specification is of the form \verb+<type>;<file>+.  This is
very similar to the \PGPLOT\  scheme described above, but the device type
and file are swapped round, and the separator is a semicolon instead of
a solidus.  As for \PGPLOT, the device file can be omitted, in which case a
default is used, but note that \emph{in this case the semicolon should
also be omitted}.  The only supported devices are those for which \PGPLOT\
drivers are available.  If a device was also available in the GKS-based
graphics systems previously used by \KAPPA, then you can
refer to it either using its \PGPLOT\  device type or its GNS device type.

\end{description}

In either scheme, either the device type or the file name or the entire
device specification may be replaced by a logical name, in which case the
value of the logical name will be used instead.


\subsubsection{\xlabel{se_devglobal}Global Parameters\label{se:devglobal}}

There is a global parameter for the graphics device.
The purpose of this global parameter is ostensibly to prevent
unnecessary prompting.  However, there is an ulterior motive as well.
The selection of devices outside of the graphics applications enables
us to perform other necessary actions just once.

There is a command for selecting the current graphics device:
\htmlref{GDSET}{GDSET}.  For example,

\begin{terminalv}
     ICL> gdset xwindows
\end{terminalv}

A selection remains in force until you change it using GDSET again, use
\htmlref{NOGLOBALS}{NOGLOBALS}, or delete the globals file.  The current
choice can be inspected via the \htmlref{GLOBALS}{GLOBALS} command.  If the
global parameter is undefined you will be prompted for the device if an
application requires it.

You can override the global parameter for the duration of a single
application by specifying it by keyword (normally \texttt{DEVICE=}), or in
some applications, by position.  Here is an example.

\begin{terminalv}
     ICL> contour device=ps_p
\end{terminalv}

\subsubsection{\xlabel{se_x-windows}X-windows\label{se:x-windows}}

The most commonly used devices are X-windows.  These can require a little
preparation before you select a device.  Starlink graphics use
\GWMref\ to manage windows.  It enables a window to persist between
separate applications; or to be shared by different applications,
potentially even running on different machines.  See
\xref{SUN/130}{sun130}{} for details of \GWM\ and how to
change your X-defaults file (\file{{\$HOME/.Xdefaults}}), but the salient
points are given below.

If the window appears on a terminal or workstation other than the one
running the \KAPPA\ executables you will need to redirect
output to your screen, if you have not already done so for some other
software.  You either use the \xref{\texttt{xdisplay}}{sun129}{} command

\begin{terminalv}
    % xdisplay myterm.mysite.mydomain.mycountry
\end{terminalv}
or set the \texttt{DISPLAY} environment variable to point to the address of
your screen.

\begin{terminalv}
    % setenv DISPLAY myterm.mysite.mydomain.mycountry:0
\end{terminalv}
You substitute your machine's address or IP number.  (Ask your
computer manager.)

If you do not create the window before running \KAPPA, the
first graphics application to open an X-windows device will create the
window, using certain defaults.  The defaults control amongst others
the foreground and background colours, the number of colours
allocated, the size and location of the window.  These defaults may be
altered with an X-defaults file, or a window created with the
\GWM\ {\bf xmake} command.

\begin{terminalv}
    % xmake xwindows -geom 600x450 -fg yellow -bg black
\end{terminalv}
This example makes a window of dimension 600-by-450 pixels, the
background colour is black and colour for the line graphics is
yellow.

The following set up to place in your X-defaults file is a reasonable
compromise, as it maximises the number of colour indices for the
graphics window (xwindows), and has a second graphics window (x2windows).
In the defaults file there are the following lines

\begin{terminalv}
     gwm*xwindows*colours:         80
     gwm*xwindows2*colours:        20
\end{terminalv}
and you can also set the sizes of the windows too.  Notice that the
second device name is x2windows, but the window name is xwindows2.
Don't ask why.  This
confusing name rule applies also to all but the first window of the
maximum of four windows allowed.

The device names can be abbreviated, to give unambiguous names.  Thus you
can enter \texttt{xw} for the xwindows device; \texttt{x2w} for the x2windows
device; and so on.  This is the reason for having device names as they
are.

The following tells \KAPPA\ that this is the current device.
This remains as a global parameter, so you probably will not need to
issue this command that often.

\begin{terminalv}
     % gdset x2windows
\end{terminalv}
\bigskip

\subsection{Composite Hardcopy Plots}

\KAPPA\ applications are modular so that you can build up
more-complicated plots, and possibly add annotations with other packages
especially \PONGOref\@.  This is fine if the device is some sort of
screen.  However, care needs to be taken when using other types of
device. For instance, some \PGPLOTref\ PostScript devices put the
output from each command into a separate disk file.  So how do you get the
composite plot out on paper?  There are two solutions:

\begin{enumerate}
\item the easy way---use an ``accumulating'' PostScript device instead.
If you look at the list of device names produced by
\htmlref{GDNAMES}{GDNAMES} you will see that some of the encapsulated
PostScript (EPS) devices are described as ``accumulating'' such as the
following.

\begin{terminalv}
    % gdnames
    ...
    aps_p     (/AVPS)          Accumulating EPS, monochrome, portrait
    aps_l     (/APS)           Accumulating EPS, monochrome, landscape
    apscol_p  (/AVCPS)         Accumulating EPS, color, portrait
    apscol_l  (/ACPS)          Accumulating EPS, color, landscape
    ...
\end{terminalv}

If you use one of these devices, each subsequent graphics application
will merge its PostScript output automatically into any existing
\file{pgplot.ps} file. If you display \file{pgplot.ps} using a
modern PDF/PostScript viewer such \OKULARref\ or \EVINCEref, the
display will update automatically as each subsequent graphics application modifies
the file. Once the final graphics application has been run, you can rename
the \file{pgplot.ps} file to something more covenenient\footnote{You
could alternatively include the required final file name in the device
name before running the graphics applications.}.

To give an example, suppose we wanted to overlay a contour plot
on an image.

\begin{terminalv}
    % gdset apscol_l
    % display $KAPPA_DIR/comwest lut=$KAPPA_DIR/spectrum_lut mode=ra axes
    % contour noclear mode=au noaxes \\
    % mv pgplot.ps myplot.ps

\end{terminalv}

\item The hard way---use \PSMERGEref\ to merge separate PostScript files.
If you use one of the non-accumulating PostScript devices, the output
from each graphics application goes into a separate file which must then
be merged.  Here is what you must do.

\begin{itemize}
\item  Select one of the non-acculumlating encapsulated PostScript device, be
it colour or monochrome.
\item  Produce your graphics, being careful to rename the output file
(usually \verb+pgplot.ps+) created by each command to avoid it being
over-written by the output from the next command.  Alternatively, you can
specified the device to use separately when running each command, including an
explicit unique file name in each device specification (see above).
\item Use \PSMERGEref\ (the Starlink version, usually in
\file{/star/bin}) to combine the plots.
\end{itemize}

To use the same example as above:

\begin{terminalv}
    % gdset epsfcol_l
    % display $KAPPA_DIR/comwest lut=$KAPPA_DIR/spectrum_lut mode=ra axes
    % mv pgplot.ps display.ps
    % contour noclear mode=au noaxes \\
    % mv pgplot.ps contour.ps
    % psmerge display.ps contour.ps  > myplot.ps
    % rm display.ps contour.ps
\end{terminalv}

or alternatively:

\begin{terminalv}
    % display $KAPPA_DIR/comwest device="epsfcol_l;display.ps" \
              lut=$KAPPA_DIR/spectrum_lut mode=ra axes
    % contour noclear device="epsfcol_l;contour.ps" mode=au noaxes \\
    % psmerge display.ps contour.ps  > myplot.ps
    % rm display.ps contour.ps
\end{terminalv}

to form the merged graphic.  You then print \file{myplot.ps} to the
colour PostScript printer. {\footnotesize PSMERGE} \normalsize also
has options for scaling and rotating plots. \end{enumerate}

\newpage
\section{\xlabel{se_style}Plotting Styles and Attributes\label{se:style}}

\subsection{\xlabel{se_styles}Plotting Styles and Attributes\label{se:styles}}

Many different aspects of the appearance of line graphics produced by
\KAPPA\ applications can be controlled by specifying a
{\em plotting style} when running the application.  This includes things
like the colour, line width, line style, character size, and fount, each
of which can be specified individually for different components of a plot
(\emph{e.g.} the tick marks, numerical labels, border, textual labels,
\emph{etc.}).

Each aspect of a plot is controlled by a \emph{plotting attribute}.
A plotting attribute has a \emph{name} and a \emph{value}.

Strings that assign new values to particular plotting attributes are
called attribute setting strings.  For instance:

\begin{terminalv}
   Border=1
   Title=This is my plot title
\end{terminalv}

are two attribute setting strings.  The first assigns the value \texttt{1} to
the attribute called \att{Border}, and the second assigns the string
\texttt{"This is my plot title"} to the attribute with name \att{Title}
(attribute names are case insensitive but cannot be abbreviated).

Here is a list of the available attributes, with a brief description
of each.  Full descriptions are included \slhyperref{later}{in
  Appendix~}{}{ap:plotting_attr}.


\begin{aligndesc}
\item[\att{Border}] Draw a border around valid regions of a plot?
\item[\att{Colour(element)}] Colour for a plot element
\item[\att{DrawAxes}] Draw axes?
\item[\att{DrawDSB}] Annotate both sidebands in a dual sideband spectrum?
\item[\att{DrawTitle}] Draw a title?
\item[\att{Edge(axis)}] Which edges to label
\item[\att{FileInTitle}] Include the NDF name as a second line in the title?
\item[\att{Font(element)}] Character fount for a plot element
\item[\att{Gap(axis)}] Interval between major axis values
\item[\att{Grid}] Draw grid lines?
\item[\att{LabelAt(axis)}] Where to place numerical labels
\item[\att{LabelUnits(axis)}] Include unit descriptions in axis labels?
\item[\att{LabelUp(axis)}] Draw numerical axis labels upright?
\item[\att{Labelling}] Label and tick placement option
\item[\att{MajTickLen}] Length of major tick marks
\item[\att{MinTickLen}] Length of minor tick marks
\item[\att{MinTick(axis)}] Density of minor tick marks
\item[\att{NumLab(axis)}] Draw numerical axis labels?
\item[\att{NumLabGap(axis)}] Spacing of numerical axis labels
\item[\att{Size(element)}] Character size for a plot element
\item[\att{Style(element)}] Line style for a plot element
\item[\att{TextBackColour}] Background colour to use when drawing text
\item[\att{TextLab(axis)}] Draw descriptive axis labels?
\item[\att{TextLabGap(axis)}] Spacing of descriptive axis labels
\item[\att{TextMargin}] Width of margin to clear when drawing text
\item[\att{TickAll}] Draw tick marks on all edges?
\item[\att{TitleGap}] Vertical spacing for the title
\item[\att{Tol}] Plotting tolerance
\item[\att{Width(element)}] Line width for a plot element
\end{aligndesc}

Some attribute names can be qualified so that they refer to a particular
component of the plot.  This is done by putting the qualifier in
parentheses after the attribute name.  For instance:

\begin{terminalv}
   Colour(border)=2
   Edge(2)=left
\end{terminalv}

assigns the value \texttt{2} to the \att{Colour} attribute for the plot border,
and assigns the value \texttt{left} to the \att{Edge} attribute for the second
axis.  The full description of each attribute describes what happens if
you omit the qualifier.  Attribute names that end in \texttt{"(axis)"} in
the above list can be qualified by an integer axis index to refer to a
particular plot axis.  Attribute names that end in \texttt{"(element)"} in
the above list can be qualified by any of the following strings to refer to
a particular component of the plot (the qualifiers are case-insensitive and
unambiguous abbreviations may be used):

\begin{aligndesc}
\item[\att{Axes}] Axis lines drawn through tick marks within the
  plotting area, drawn if attribute \att{DrawAxes} is given a
  non-zero value.  Values set using \att{Axes} are only only used if
  axis-specific values have not been set using \att{Axis1} or {\att
    Axis2}.  Thus, \att{Axes} provides default values for {\att
    Axis1} and \att{Axis2}.

\item[\att{Axis1}] An axis line drawn through tick marks on Axis~1
  within the plotting area, drawn if attribute \att{DrawAxes} is
  given a non-zero value.  Values specified using \att{Axis1}
  override any supplied
  using \att{Axes}.
\item[\att{Axis2}] An axis line drawn through tick marks on Axis~2
  within the plotting area, drawn if attribute \att{DrawAxes} is
  given a non-zero value.  Values specified using \att{Axis2}
  override any supplied using \att{Axes}.

\item[\att{Border}] The plot border, drawn if attribute {\att
    Border} is given a non-zero value.

\item[\att{Curves}] Curves drawn over the top of a plot (\emph{e.g.}
  contours, data curves).

\item[\att{Grid}] The grid lines, drawn if attribute \att{Grid} is
  given a non-zero value.  Values set using \att{Grid} are only only
  used if axis-specific values have not been set using \att{Grid1} or
  \att{Grid2}.  Thus, \att{Grid} provides default values
  for \att{Grid1} and \att{Grid2}.

\item[\att{Grid1}] Grid lines that cross Axis~1, drawn if attribute
  \att{Grid} is given a non-zero value.  Values specified using {\att
    Grid1} override any supplied using \att{Grid}.

\item[\att{Grid2}] Grid lines that cross Axis~2, drawn if attribute
  \att{Grid} is given a non-zero value.  Values specified using {\att
    Grid2} override any supplied using \att{Grid}.

\item[\att{Markers}] Graphical markers (symbols) drawn over a plot.
\item[\att{NumLab}] Numerical axis labels drawn around annotated
  axes.  Values set using \att{NumLab} are only only used if
  axis-specific values have not been set using \att{NumLab1} or {\att
    NumLab2}.  Thus, \att{NumLab} provides default values for {\att
    NumLab1} and \att{NumLab2}.

\item[\att{NumLab1}] Numerical labels drawn along Axis~1.  Values
  specified using \att{NumLab1} override any supplied using {\att
    NumLab}.

\item[\att{NumLab2}] Numerical labels drawn along Axis~2.  Values
  specified using\att{NumLab2} override any supplied using {\att
    NumLab}.

\item[\att{Strings}] Text strings drawn over a plot (except for axis
  labels and plot title).

\item[\att{TextLab}] Descriptive axis labels drawn around annotated
  axes.  Values set using \att{TextLab} are only only used if
  axis-specific values have not been set using \att{TextLab1} or
  \att{TextLab2}.  Thus, \att{TextLab} provides default values for
  \att{TextLab1} and \att{TextLab2}.

\item[\att{TextLab1}] Descriptive label for Axis~1.  Values
  specified using \att{TextLab1} override any supplied using {\att
    TextLab}.

\item[\att{TextLab2}] Descriptive label for Axis~2.  Values
  specified using \att{TextLab2} override any supplied using {\att
    TextLab}.

\item[\att{Ticks}] Tick marks (both major and minor) drawn along
  annotated axes.  Values set using \att{Ticks} are only only used if
  axis-specific values have not been set using \att{Ticks1} or {\att
    Ticks2}.  Thus, \att{Ticks} provides default values for {\att
    Ticks1} and \att{Ticks2}.

\item[\att{Ticks1}] Tick marks (both major and minor) drawn along
  Axis~1.  Values specified using \att{Ticks1} override any supplied
  using \att{Ticks}.

\item[\att{Ticks2}] Tick marks (both major and minor) drawn along
  Axis~2.  Values specified using \att{Ticks2} override any supplied
  using \att{Ticks}.

\label{plotel:Title}
\item[\att{Title}] The title drawn at the top of a plot.
\end{aligndesc}

A collection of attribute settings is called a \emph{plotting style}.
Attributes that are not specified in a plotting style take on default
values as described later.  An example of a plotting style is contained in
the file \file{\$KAPPA\_DIR/kappa\_style.def}.  This file defines the default
style used by \KAPPA.

In general each graphical application will determine the plotting style
to be used when drawing line graphics by accessing a parameter called
STYLE.  Some applications have more than one style parameter.  For instance,
applications such as \htmlref{CONTOUR}{CONTOUR} and
\htmlref{LINPLOT}{LINPLOT} that produce keys to the content of the plot
have a parameter called KEYSTYLE, in addition to the normal STYLE
parameter.  KEYSTYLE is used to specify the plotting style for the key,
whereas STYLE is used to specify the plotting style for the main plot.

\subsection{Specifying a Plotting Style}

\subsubsection{Group Expressions}

The application parameters that are used to access plotting styles
(STYLE, KEYSTYLE, \emph{etc.}) expect a group
of attribute setting strings to be supplied for the parameter.  These
should be supplied in the form of a \emph{group expression}.\footnote{The
complete group expression syntax is described in \xref{SUN/150}{}{sun150}.
This is the documentation for the GRP subroutine library, which provides
a programming interface for obtaining groups of strings.}

A group expression is a character string containing a list of one or more
sub-strings separated by some specified delimiter.  The usual delimiter
used when accessing plotting style parameters is a comma (although other
parameters that require group expressions for other purposes may use a
different delimiter).  Each sub-string is known as a group \emph{element}.
Each group element must be either:

\begin{itemize}
\item an attribute setting string (\emph{e.g.} \texttt{"Border=1"}), or
\item the name of a text file, preceded by an up-arrow character (\verb+"^"+'').
\end{itemize}

If a group element starts with an up-arrow character, then the rest of
the element is interpreted as a file name, and an attempt is made to read
further group elements from the specified file.  Each
line in the file is interpreted as a group expression in its own right,
using exactly the same rules as described above.  In particular,
references to text files can be nested (\emph{i.e.} a text file can
include a group element that refers to another text file).  Blank lines
and lines in which the first non-blank character is a hash (\verb+"#"+)
are ignored and can be used to add textual comments to a text file.

Attribute settings are used in the order in which they occur in the group
expression.  If a group expression includes more than one setting string for
a given attribute, then the value that occurs nearest to the end of the
group expression will overwrite any earlier values.

All this means that there are several ways in which plotting styles can be
supplied.  The simplest method is probably to store all your attribute
setting strings in a text file, one per line, and then just give the name
of this text file (preceded by an up-arrow) as the value for the STYLE
parameter.  For instance, the file \file{style.dat} may contain:

\begin{terminalv}
   # A test plotting style

   width=2
   edge(2)=right
   title=This is my title
\end{terminalv}

The first line and the following blank line are ignored.  The remaining three
lines specify three attribute values to use.  When running an application
such as \htmlref{DISPLAY}{DISPLAY}, this style file would be specified on
the command line as follows:

\begin{terminalv}
   % display style=^style.dat
\end{terminalv}

Alternatively, you can specify the setting strings explicitly on the
command line.  This can get a bit messy because you need to protect
special characters (commas, parentheses, spaces, equals signs, and so on)
both from the UNIX shell and from the parameter system.  One safe way
to do this is to enclose the whole group expression in single quotes, and
then enclose the whole thing again in double quotes.\footnote{The up-arrow
character is not one of these special characters, and so a simple
reference to a single text file does not need to be enclosed in quotes.}
So the above style could be given on the command line as follows:

\begin{terminalv}
   % display style="'width=2,edge(2)=right,title=This is my title'"
\end{terminalv}

A bit messy as I said! However, it can be useful to combine this method
with the previous method.  If you have a long, complicated style file, and
you want to change one or two attribute settings, one method would be to
take a copy of the style file and edit it.  This would probably be the
best thing to do if you intend to re-use the edited style file several
times.  But if you just want to try a quick experiment, just to see what
the results look like, you can avoid the trouble of editing the style file
by giving both the original style file \emph{and} the new attribute settings
on the command line.  For instance:

\begin{terminalv}
   % display style="'^style.dat,width=3'"
\end{terminalv}

This reads the contents of our test style file \file{style.dat}, which
includes the attribute setting \texttt{width=2}.  It then also applies the
attribute setting \texttt{width=3}, over-writing the effect of the \texttt{Width}
value included in the style file.  If you wanted to try temporarily changing
the value of several attributes at once, you could put the new attribute
settings into a second file, say \file{style2.dat}, and then run the
application as follows:

\begin{terminalv}
   % display style="'^style.dat,^style2.dat'"
\end{terminalv}

Again, the contents of \file{style.dat} would be read, followed by the
contents of \file{style2.dat}, over-writing any settings for the same
attributes included in \file{style.dat}.

\subsubsection{\xlabel{se_temp_attr}Temporary Attributes\label{se:temp_attr}}

Sometimes you need to effect a change of style that only lasts for a
single invocation of a task.  For example, plotting data from
different NDFs or NDF sections in different colours.  This is available
at the cost of a little further syntax to learn.  The temporary
attributes should be preceded by a plus sign.

In the following example, three histograms are plotted on a single
graphic.

\begin{terminalv}
   % histogram ndf1 \\
   % histogram ndf2 style="+colour(curve)=yellow" noclear noaxes \\
   % histogram ndf3 style="+width(curve)=4" noclear noaxes \\
\end{terminalv}

The first uses the current attributes including line colour to plot
the histogram for NDF ndf1.  (Recall \verb+\\+ is a synonym for the
accept keyword, so other defaults are used for the likes of the data
range and number of bins omitted for clarity.)  The second NDF's
histogram is plotted in yellow.  For the third NDF the locus is again
in the colour of the first plot (since the yellow was only temporary)
but has thickness four times normal.  If you ran
\htmlref{HISTOGRAM}{HISTOGRAM} again, the line thickness would return
to its original value used in the first graph.

You are not limited to just one temporary attribute.  You can supply
a list that can include indirection to text files.

\begin{terminalv}
   % linplot spectrum style="'+style(curve)=2,grid,^temp.sty,colour(textlab)=green'"
\end{terminalv}

Here \htmlref{LINPLOT}{LINPLOT} uses three temporary attributes plus
whatever is defined in the text file file \file{temp.sty}.  Note for a
list the string requires an extra set of enclosing quotes to protect
these from being misinterpreted by the UNIX shell.  The experimental
method that used a second parameter called TEMPSTYLE will be
withdrawn.

It is also possible to combine persistent and temporary attributes.
Persistent style attributes must be supplied first, then after a plus
sign comes the list of temporary attributes.

\begin{terminalv}
   % linplot spectrum style="'colour(line)=red,width=3+style(curve)=2'"
\end{terminalv}

Literal plus signs should be avoided if using both temporary and
persistent attributes in a group expression.

\subsubsection{Synonyms for Attribute Names}

The available plotting attributes and their names are defined by the
AST subroutine library (see \xref{SUN/210}{}{sun210}) upon which
\KAPPA\ graphics are based (together with \PGPLOT).
However, \KAPPA\ provides synonyms for certain
plotting attributes where this would provide a clearer indication of the
purpose of the attribute within the context of a particular application.
These synonyms are listed in the description of the STYLE parameter for
the particular applications concerned.  For instance, the
\htmlref{CONTOUR}{CONTOUR} applications draws contours as `curves',
that is, it uses the attributes \att{Colour(Curves)}, \att{Width(Curves)}
and \att{Style(Curves)} to determine the appearance of the contours.
However, the synonym \att{Contours} (minimum abbreviation \att{Cont}) may be
used in place of \att{Curves}, so the appearance of the contours can also
be specified using the `pseudo-' attributes \att{Colour(Cont)}, {\att
Width(Cont)} and \att{Style(Cont)}.

You should remember that a synonym is simply an alternative way of
specifying a particular attribute.  So if you are running CONTOUR and you
give the two setting strings:

\begin{terminalv}
   Colour(cont)=red
   Colour(curve)=blue
\end{terminalv}

the contours will appear blue, not red, because the second attribute
setting overrides the first one.

Any particular synonym will only be recognised by certain applications.
Thus, \att{Contours} is only recognised as a synonym for \att{Curves} when
running CONTOUR.  Applications ignore attributes (or synonyms) that they
do not recognise without reporting an error.  Thus if the file containing
your default plotting style (used by all applications---see later)
contains the two lines:

\begin{terminalv}
   Colour(curve)=blue
   Colour(cont)=red
\end{terminalv}

then CONTOUR will draw contours in red, but other applications will draw
their curves in blue since they ignore the \texttt{Colour(cont)=red} line.
Note, if these two lines were the other way round:

\begin{terminalv}
   Colour(cont)=red
   Colour(curve)=blue
\end{terminalv}

then all curves, \emph{including} contours drawn by CONTOUR, would be
blue.  This is because CONTOUR will process both lines in the order
supplied, ending up with blue contours.

\subsubsection{\xlabel{se_colour_attributes}Colour Attributes\label{se:colattr}}

The \att{Colour} attribute can be used to specify the colours of various
components of a plot.  The value assigned to the attribute can be one
of the following options.
\begin{itemize}

\item An integer colour index.  Colour indices can be thought of as `pen
numbers'.  For instance, the string \texttt{"Colour(border)=3"} says ``Draw the
border using pen number 3''.  The resulting colour depends on the colour of
Pen~3, which can be set using \htmlref{PALENTRY}{PALENTRY}.

\item A standard X colour name, \emph{e.g.} \texttt{"Colour(border)=red"}.

\item A triple of floating-point red, green and blue intensity values in
the range zero to one, separated by spaces, \emph{e.g.}
\texttt{Colour(border)=1.0 0.0 0.0}.

\label{htmlcolour}
\item An HTML colour code.  This is a hash followed by three pairs of
hexadecimal digits giving red, green, and blue intensity in the range
0 to 255, \emph{e.g.} \texttt{"Colour(border)=\#ffa700"}.  Within style
files, the \texttt{"\#"} character is used to introduce a comment, and so the
colour code would be ignored.  To avoid this, the \texttt{"@"} character can
be used in place of \texttt{"\#"}.

\end{itemize}

If no pen is currently available in the palette with the requested
colour, then the `nearest' colour will be used instead---sometimes
this may not be very near at all!  If you specifically want the requested
colour, then you should use PALENTRY first to set one of the available
pens to the required colour.

Note, if you produce a plot on an X-window and then change the representation
of pens using PALENTRY, then any components of the existing plot that were
drawn with the modified pens will change colour immediately \emph{if and
only if} your X window is set to 256 colours (\emph{i.e.} if you have an 8 bit
pseudo-colour visual).  If you are in the more usual situation of having a 16-
or 24-bit display, then changes to pen colours will only affect subsequently
drawn graphics.

\subsection{Establishing Defaults for Plotting Attributes}

When an application needs a plotting style, it uses a style parameter
to get a group of attribute settings.  But what values are used for
attributes that are not included in this group?

Obviously, if an attribute has been assigned an explicit value using the
parameter then that value is used, but you should note that most styles
are `sticky'.\footnote{The exceptions are the TEMPSTYLE parameters. These
parameters are used to make temporary style changes and always forget any
previous values assigned to them.} That is, once a group of attribute
settings has been specified for a style parameter, that group continues
to be used by subsequent invocations of the application until a new group is
supplied for the parameter. If the group is supplied within a text file, then
the `current value' stored for the parameter is the list of attribute
settings read from the file, \emph{not} the name of the file.  Thus,
changing the contents of the file at a later time will have no effect on
the value used for the parameter unless you re-specify the parameter on
the command line.

If an attribute is not specified in the supplied group, then a default
value is used for the attribute, determined as follows:

\begin{enumerate}
\item If the plot is being overlayed on another existing plot, then the
      value that was used for the attribute when the existing plot was
      created is used (but only if it was set to an explicit value).

\item Otherwise, the attribute value specified in a \emph{defaults file} is
      used.  The defaults file is found as follows:

\begin{itemize}
\item If the environment variable \texttt{KAPPA\_<APP>\_<PARAM>} is defined (where
      \texttt{<APP>} is the name of the application \emph{e.g.} DISPLAY, and
      \texttt{<PARAM>} is the name of the parameter, \emph{e.g.} STYLE, both
      in upper-case), its value is taken to be the full path to
      the defaults file.

\item If \texttt{KAPPA\_<APP>\_<PARAM>} is not defined, the file \texttt{
      \$HOME/kappa\_<app>\_<param>.def} is used (where \texttt{<app>} and
      \texttt{<param>} are in lower case this time).

\item If the file \file{\$HOME/kappa\_<app>\_<param>.def} cannot be accessed,
      the file \linebreak \file{\$KAPPA\_DIR/kappa\_<app>\_<param>.def} is used.

\item If the file \file{\$KAPPA\_DIR/kappa\_<app>\_<param>.def} cannot be
      accessed, then the value of the environment variable
      \texttt{KAPPA\_<PARAM>} is used as the full path to the defaults file.

\item If \texttt{KAPPA\_<PARAM>} is not defined, the file \texttt{
      \$HOME/kappa\_<param>.def} is used.

\item If the file \file{\$HOME/kappa\_<param>.def} cannot be accessed, the file
      \linebreak \file{\$KAPPA\_DIR/kappa\_<param>.def} is used.
\end{itemize}

\item If the above process failed to produce a value for the attribute
(either because no file was found, or the file did not contain a setting
for the attribute), then the default value supplied by the AST library is
used.  These defaults are included in the full description of the relevant
attributes \slhyperref{later}{in Appendix~}{}{ap:plotting_attr}.
\end{enumerate}

From the above, you can see that defaults can be specified either for
individual applications, or for all applications (any application-specific
defaults file will be used in preference to the general defaults file).

It should be remembered that settings for unknown attributes are ignored,
and do not cause the application to abort.\footnote{This is because they
may be synonyms recognised by other other applications.} So if you set a
value for an attribute and it seems to have no effect, it may be worth
checking that you have used the correct spelling for the attribute name.

\subsection{\xlabel{se_escseq}Graphical Escape Sequences\label{se:escseq}}

Strings used for axis labels, plot titles, \emph{etc.}, can include special
\emph{escape sequences} which control the appearance of subsequent text when
the string is drawn as part of a plot.\footnote{When displayed in a
non-graphical environment (for instance, on an alpha-numeric terminal)
the characters forming an escape sequence are stripped out of the string
before the string is displayed.} Escape sequences are introduced using a
percent character.  For instance, if the string \verb="10%^50+%^s50+Z"=
was used as an axis label in a plot, it would produce a string similar to
\texttt{"$10^{\textrm{Z}}$"}---that is, the character \texttt{"Z"} would be displayed as a small
superscript character.  Any unrecognised, illegal or incomplete escape
sequences are printed literally.  The following escape sequences are
currently recognised (\texttt{"..."} represents a string of one or more decimal
digits):

\begin{itemize}
\item \verb=%%= :  Print a literal \texttt{"\%"} character.

\item \verb=%^...+= :   Draw subsequent characters as superscripts.  The digits
           \texttt{"..."} give the distance from the base-line of `normal'
           text to the base-line of the superscript text, scaled
           so that a value of \texttt{"100"} corresponds to the height of
           `normal' text.

\item \verb=%^+= :      Draw subsequent characters with the normal base-line.

\item \verb=%v...+= :   Draw subsequent characters as subscripts.  The digits
           \texttt{"..."} give the distance from the base-line of `normal'
           text to the base-line of the subscript text, scaled
           so that a value of \texttt{"100"} corresponds to the height of
           `normal' text.

\item \verb=%v+= :   Draw subsequent characters with the normal base-line
           (equivalent to \verb=%^+=).

\item \verb=%>...+= :   Leave a gap before drawing subsequent characters.
           The digits \texttt{"..."} give the size of the gap, scaled
           so that a value of \texttt{"100"} corresponds to the height of
           `normal' text.

\item \verb=%<...+= :   Move backwards before drawing subsequent characters.
           The digits \texttt{"..."} give the size of the movement, scaled
           so that a value of \texttt{"100"} corresponds to the height of
           `normal' text.

\item \verb=%s...+= :   Change the \att{Size} attribute for subsequent characters.  The
           digits \texttt{"..."} give the new Size as a fraction of the `normal'
           Size, scaled so that a value of \texttt{"100"} corresponds
           to 1.0;

\item \verb=%s+= :   Reset the \att{Size} attribute to its `normal' value.

\item \verb=%w...+= :   Change the \att{Width} attribute for subsequent characters.  The
           digits \texttt{"..."} give the new width as a fraction of the
           `normal' Width, scaled so that a value of \texttt{"100"} corresponds
           to 1.0;

\item \verb=%w+= :      Reset the \att{Width} attribute to its `normal' value.

\item \verb=%f...+= :   Change the \att{Font} attribute for subsequent characters.  The
           digits \texttt{"..."} give the new Font value.

\item \verb=%f+= :  Reset the \att{Font} attribute to its `normal' value.

\item \verb=%c...+= :   Change the \att{Colour} attribute for subsequent characters.  The
           digits \texttt{"..."} give the new Colour value.

\item \verb=%c+= :   Reset the \att{Colour} attribute to its `normal' value.

\item \verb=%t...+= :   Change the \att{Style} attribute for subsequent characters.  The
           digits \texttt{"..."} give the new Style value.

\item \verb=%t+= :   Reset the \att{Style} attribute to its `normal' value.

\item \verb=%-= :   Push the current graphics attribute values on to the top of
           the stack (see \verb=%+=).

\item \verb=%+= :   Pop attributes values of the top the stack (see \verb=%-=).  If
           the stack is empty, `normal' attribute values are restored.
\end{itemize}

\PGPLOT\  escape sequences may also be included in strings that are to be drawn.


\newpage
\section{\xlabel{se_datastr}Data structures\label{se:datastr}}

In an ideal world you would not need to know how your data are stored.  It
would be transparent.  Starlink applications attempt to achieve this
through standard, but extensible, data structures, and the ability to
apparently operate on other formats through the so-called
\htmlref{`on-the-fly conversion'}{se:autoconvert} (see
\latex{Section~\ref{se:autoconvert} and} \xref{SUN/55}{sun55}{}).

The official standard data format used by Starlink applications is the
\NDFextref{NDF}\ (Extensible \textit{n}-dimensional Data Format, SUN/33).  The
data in an NDF is stored using \HDSref\ which has numerous advantages,
not least that {\em HDS files are portable between operating systems};
both have file extension \file{.sdf}.

The NDF has been carefully designed to facilitate processing by both
general applications like \KAPPA\ and specialist packages.
It contains an \textit{n}-dimensional data array that can store most
astronomical data such as spectra, images and spectral-line data cubes.
The NDF may also contain other items such as a title, axis labels and
units, error and quality arrays, and World Co-ordinate System (WCS)
information.  There are also places in the NDF, called {\em extensions},
to store any ancillary data associated with the data array, even other
NDFs.

The NDF format and its components are described more fully in
\latexhtml{Appendices~\ref{ap:NDFformat},}{\htmlref{NDF standard components}
{ap:NDFformat},}  which includes commands for manipulating the components.

The NDF format permits arrays to have seven dimensions, but some
applications only handle one-dimensional and/or two-dimen\-sional data
arrays.  The data and variance arrays are not constrained to a single
data type.  Valid types are the \htmlref{HDS numeric primitive
types}{ap:HDStypes}\latex{, see \slhyperref{}{Appendix~}{}{ap:HDStypes}}.

Many applications are generic, that is they can work on all or some of
these data types directly.  This makes these applications faster, since
there is no need to make a copy of the data converted to the type
supported by the application.  If an application is not generic it only
processes {\_REAL} data.  Look in the \texttt{Implementation Status} in
the help or the reference manual.  If none is given you can assume that
processing will occur in \_REAL.

In \KAPPA\ the elements of the data array are often called
{\em pixels}, even if the NDF is not two-dimensional.

\subsection{Restrictions on the Usage of Data Structures}

By default, \KAPPA\ plays safe and will not allow you to
use the same data structure as both input and output for a command.  This
is to minimise the risks of accidentally over-writing valuable data.  So,
for instance, if you try the following command:

\begin{terminalv}
   % cadd in=m31 scalar=10 out=m31
\end{terminalv}

you will find that the value of \texttt{m31} for Parameter OUT is rejected with a
message indicating that the data structure is already in use, and you
will be prompted for an alternative value.

However, \KAPPA\ does allow you to `live on the edge' if
you prefer---if you define the environment variable \texttt{KAPPA\_REPLACE} before
running a command, then \KAPPA\ will happily overwrite the
input data structure if requested to do so.  You can assign any value you
like to this environment variable, since its mere existence is the trigger
for this optional behaviour.  Note, this facility is only available in
those commands that access the input data structures \emph{before} the
output data structures (the vast majority).

\subsection{Looking at the Data Structures}

You can look at a summary of an \NDFref{NDF} structure using the task
\htmlref{NDFTRACE}{NDFTRACE}, and obtain the values of NDF extension
components with the \texttt{\htmlref{setext}{SETEXT} option=get} command.
\HDSTRACEref\ \latex{(SUN/102)} can be used to look at array values and
extensions.

\subsection{Editing the Data Structures}

There are facilities for editing \HDSref\ components, though these should be
used with care, lest you invalidate the file.  For instance, if you
were to erase the DATA\_ARRAY component of an \NDFref{NDF}, the file would no
longer be regarded as an NDF by applications software.

In \KAPPA, \htmlref{ERASE}{ERASE} will let you remove any
component from within an HDS container file, but you have to know the
full path to the component.  \htmlref{SETEXT}{SETEXT} has options to
erase extensions and their contents, without needing to know how these
are stored within the HDS file.  It also permits you to create and
rename extension components, and assign new values to existing
components.  There are a number of commands for manipulating
FITS-header information stored in the NDF's FITS extension.  These are
described in
\latexhtml{Section~\ref{se:fitsairlock}.}{\htmlref{the FITS
airlock}{se:fitsairlock}.}

\FIGAROref\ offers some additional tasks (\xref{CREOBJ}{sun86}{CREOBJ},
\xref{DELOBJ}{sun86}{DELOBJ}, and \xref{RENOBJ}{sun86}{RENOBJ}) for
editing HDS components.

\subsection{\xlabel{se_native}Native Format\label{se:native}}

Although \HDSref\ files are portable you are recommended to copy them to
the host machine, and run application \htmlref{NATIVE}{NATIVE} on them
for efficiency gains.  NATIVE converts the data to the native format of
the machine on which you issue the command.  If you don't do this, every
time you access the data in your \NDFref{NDF}, this conversion
process occurs.  NATIVE also replaces any IEEE
floating-point NaN or Inf values with the appropriate Starlink bad value.
The following converts all the HDS files in the current directory.

\begin{terminalv}
% native "*"
\end{terminalv}

\newpage
\section{\xlabel{se_ndfsect}NDF Sections\label{se:ndfsect}}

You will frequently want to examine or process only a portion of your
dataset, be it to focus on a given object in an image, or a single
spectrum between nominated wavelengths, or a plane of a cube.  You
could use \htmlref{NDFCOPY}{NDFCOPY} or \htmlref{MANIC}{MANIC} in some
circumstances to make a new NDF containing the required data, but this
would be inconvenient as you would need more disc space, and to invent
and remember a new filename.
You will be pleased to learn that there is a succinct and
powerful alternative that obviates the need to create a new file---the
NDF section.  The application just processes a `rectangular' subset,
or section, of the NDF that you nominate.  Certainly, it requires you
to learn a little syntax, but after you use it a few times it will
seem cheap at the price for the advantages it offers.

An NDF section is defined by specifying the bounds of the portion of
the NDF to be processed immediately following the name of the NDF.
You can do this in any place where an NDF name alone would suffice,
for example, on the command line or in response to a prompt or as a
default in an interface file.  The syntax is a series of subscripts
within parentheses and may be given in several ways.  Here is a simple
example.

\begin{terminalv}
ICL> stats cluster(101:200,51:150)
\end{terminalv}

This would derive statistics of a 100$\times$100-pixel region
starting at pixel indices (101,~51) in the NDF called cluster.
Alternatively, ranges of axis co-ordinates may be given instead of
pixel indices.  Besides giving lower and upper bounds as above, you may
specify a centre and extent.  Sections are not limited to
subsets---supersets are allowed.  See the paragraphs below for more
details of these features.

If you {\em do\/} want to make a new NDF from a portion of an existing one,
you should use the command NDFCOPY.  An NDF's shape may be changed
{\it in situ\/} by \htmlref{SETBOUND}{SETBOUND}.

{\em Note if you supply an NDF section on a C-shell command line,
you must escape the parentheses.}  For example, the following
are both equivalent to the earlier example.

\begin{terminalv}
% stats cluster"(101:200,51:150)"
% stats cluster\(101:200,51:150\)
\end{terminalv}



\subsection{\xlabel{se_ndfsec-bounds}Specifying Lower and Upper Bounds
\label{se:ndfsec-bounds}}

The subscript expression appended to an NDF name to specify a section
may be given in several ways.  One possible method (corresponding with
the example above) is to give the lower and upper bounds in each
dimension, as follows

\begin{terminalv}
NAME( a:b, c:d, ... )
\end{terminalv}

where `\texttt{a:b}', `\texttt{c:d}', (\emph{etc.}) specify the lower and upper
bounds.  The bounds specified need not necessarily lie within the actual
bounds of the NDF, because \htmlref{{\em bad\/} pixels}{se:masking}
\latex{(see Section~\ref{se:masking})} will be supplied in the
usual way, if required, to pad out the NDF's array components whenever
they are accessed.  However, none of the lower bounds should exceed the
corresponding upper bound.

Omitting any of the bounds from the subscript expression will cause the
appropriate (lower or upper) bound of the NDF to be used instead.  If you
also omit the separating `:', then the lower and upper bounds of the
section will both be set to the same value, so that a single pixel will
be selected for that dimension.  Omitting the bounds entirely for a
dimension (but still retaining the comma) will cause the entire extent
of that dimension to be used.  Thus,

\begin{terminalv}
image(,64)
\end{terminalv}

could be used to specify row 64 of a two-dimensional image, while

\begin{terminalv}
cube( 1, 257:, 100 )
\end{terminalv}

would specify column 1, pixels 257 onwards, selected from plane number
100 of a three-dimensional `data cube', forming a one-dimensional section.

\subsection{\xlabel{se_ndfsec-cenext}Specifying Centre and Extent
\label{se:ndfsec-cenext}}

An alternative form for the subscript expression involves specifying the
centre and extent of the region required along each dimension, as
follows

\begin{terminalv}
name( p~q, r~s ... )
\end{terminalv}

where `\texttt{p}$\sim$\texttt{q}', `\texttt{r}$\sim$\texttt{s}', (\emph{etc.})
specify the centre and extent respectively.  The extent must be positive.
Thus,

\begin{terminalv}
name(100~11,200~5)
\end{terminalv}

would refer to an 11$\times$5-pixel region of an image centred on pixel
(100,~200).

If the value before the delimiting `$\sim$' is omitted, it will default
to the index of the central pixel in that dimension (rounded downwards
if there are an even number of pixels).  If the value following the
`$\sim$' is omitted, it will default to the number of pixels in that
dimension.  Thus,

\begin{terminalv}
image( ~100, ~100)
\end{terminalv}

could be used to refer to a 100$\times$100-pixel region located
centrally within an image, while

\begin{terminalv}
image( 10~, 20~ )
\end{terminalv}

would specify a section that is the same size as the original image, but
displaced so that it is centred on pixel (10,~20).

\subsection{\xlabel{se_ndfsec-wcs}Using World or Axis Co-ordinates to Specify
Sections\label{se:ndfsec-wcs}}

Not only can you specify sections in terms of pixel indices but also
in terms of more-tangible co-ordinates such as right ascension and
declination, wavelength, frequency, and time.  Section locations in
such \slhyperref{world co-ordinate systems (WCS)}{world co-ordinate
systems (WCS) (see Section~}{)}{se:wcsuse} are indicated by
non-integer values; whereas integer values in an NDF section are
intepreted as pixel indices.  Here an integer value is defined as one
neither containing a decimal point nor an exponent.  The WCS bounds of
each section are converted to the nearest pixel indices in order to
specify the included data elements.

Since there may be many co-ordinate systems that could represent your
desired section, some rules are necessary to decide how to interpret
the section limits.  First, to retain backwards compatibility (with
pre-V1.8 \KAPPA), if your NDF has \htmlref{AXIS components}{apndf:axis}
then non-integer values will refer to {\em axis\/} co-ordinates
\latexhtml{(for a description of {\em AXIS\/} co-ordinates, see
Section~\ref{se:domains}).}{(see \htmlref{``Co-ordinate Frames, Axes
and Domains''}{se:domains} for a description).} Otherwise the values
are specified within the co-ordinate system represented by the
\htmlref{current Frame in the NDF's FrameSet}{se:curframe}
\latexhtml{(see Section~\ref{se:curframe}).}{.} Command
\htmlref{NDFTRACE}{NDFTRACE} will show whether or not an NDF has AXIS
components, and if so, it reports their extents, labels, and units.
NDFTRACE also summarises the properties of the current WCS Frame (or
optionally all the WCS Frames present), if the NDF has a \htmlref{WCS
component}{apndf:wcs}.  You can change the current Frame
with \htmlref{WCSFRAME}{WCSFRAME}.

\subsubsection{World co-ordinates:}
The standard formats used to specify WCS co-ordinates apply.\latex{See
SUN/210 AST\_UNFORMAT}\html{See \xref{AST\_UNFORMAT}{sun210}{AST_UNFORMAT}}
sections ``Frame Input Format'' and``SkyFrame Input Format''
for details and examples.

Here are some examples of WCS co-ordinates defining NDF sections.

\begin{terminalv}
spectrum(9000:2E4)
\end{terminalv}

This could specify a region of a spectrum between 9000 and 20000
{\AA}ngstrom.\smallskip

\begin{terminalv}
image(12h59m49s~4.5m,27.5d:28d29.8m)
\end{terminalv}

This could refer to a region approximately one degree on each side
centred the Coma Cluster.  The section is 4.5 minutes of time along
the right-ascension axis centred at \ra{12;59;49}, and
extends from \dec{+27;30;} to \dec{+28;29;48}
in declination.\smallskip

\begin{terminalv}
ifu(1h34m10.1s:1h34m12.4s,-2d35m:-2d35.5m,6563)
\end{terminalv}

This might specify a region of sky 2.3 seconds of time by 0.5 arcminutes
in $H_\alpha$ light from an integral-field unit.\smallskip

It is possible to use a colon to separate the fields in sexagesimal
celestial and time co-ordinates, as in the following example that
defines the equivalent section as the preceding Coma Cluster example.

\begin{terminalv}
image(12:57:34;13:02:04,27:30:00;28:29:48)
\end{terminalv}

There is a cost; you must use a semicolon to demarcate lower and upper
bounds.  This is to enable differentiation of the two uses of the
colon.

Using the colon as co-ordinate field separator can leave some degree
of ambiguity. For instance, a subscript expression of ``12:34'' could
mean ``use pixel indices 12 to 34'', or could mean ``use the single
declination value 12:34''.  It is also harder to read, is an extra
rule to remember, and could be error prone.  For these reasons, the
strong recommendation is to use the {\em dms} and {\em hms} notation
for celestial and time co-ordinates.

Even given no sky co-ordinate range as in this example,
\begin{terminalv}
cube(12:34:56.7,-41:52:09,-100.0;250.0)
\end{terminalv}

do not mix colons for the two uses in the same expression.  Hence
there is a semicolon to define the range in the third co-ordinate.
The semicolon expression delimiter works with all classes of
co-ordinate system.  This example might extract a spectrum from a cube
at the right ascension \ra{12;34;56.7} declination
\dec{-41;52;09} between $-100$ and 250\,km\,s$^{-1}$
velocity.

You should also be aware that the non-integer co-ordinates within an
NDF section apply to \emph{WCS} axes, in contrast to integer bounds
that define pixel indices along \emph{pixel} axes.  These are not
necessarily the same.  For example, the WCS axes may be rotated or
permuted.  If the WCS axes are rotated, the NDF section actually used
will be a box just large enough to hold the requested range of
WCS-axis values.   Be careful.

\subsubsection{Axis co-ordinates:}
For axis co-ordinates double-precision arithmetic is used to process the
section values, but either double- or single-precision notation may be used
when supplying them.  Both linear and non-linear {\em axis\/}
co-ordinates are supported, the values supplied being automatically
converted into the corresponding pixel indices before use.  For
instance,

\begin{terminalv}
spectrum(6500.0:7250.0)
\end{terminalv}

could be used to select the appropriate region of a spectrum calibrated
in {\AA}ngstroms, while

\begin{terminalv}
spectrum(6000.0~500.0)
\end{terminalv}

would select a region of the spectrum approximately from 5750 to
6250.0~{\AA}ngstroms (the exact extent depending the values of the
axis co-ordinates), and

\begin{terminalv}
spectrum(5500.0~21)
\end{terminalv}

would select a 21-pixel-wide region of the spectrum centred on
5500~{\AA}ngstroms.


\subsection{\xlabel{se_ndfsec-fraext}Specifying Fractional Extents
\label{se:ndfsec-fraext}}

A further form for the subscript expression involves specifying
a fractional position along each dimension as a percentage, as follows

\begin{terminalv}
name( p%~q%, r%:s%, ... )
\end{terminalv}

where `\texttt{p\%}$\sim$\texttt{q\%}' specifies the centre and extent as
percentages, and `\texttt{r\%:s\%}' specifies a percentage range. Thus,

\begin{terminalv}
image(25%:75%,50%:~50%)
\end{terminalv}

would refer to the central quarter of the image.  Both sections
are equivalent.  The percentages are converted to the nearest pixel
centre to decide the centre and extents of the sections.

The \htmlref{rules}{se:ndfsec-cenext} concerning omitted values before
or after the delimiting $\sim$ apply. Thus,

\begin{terminalv}
image(40%~, ~50%)
\end{terminalv}

could be used to refer to a full-width, half-height section centred at
the (40\%, 50\%) fractional position.

A percentage value is not limited to the range 0--100\%.  In such
circumstances the areas beyond the bounds of the NDF are set to bad.
For example,

\begin{terminalv}
image(~110%, -5%:105%)
\end{terminalv}

would give an enclosing border of bad pixels, extending 5\% of the original
dimensions.  Both sections are equivalent.

\subsection{\xlabel{se_ndfsec-dim}Changing Dimensionality
\label{se:ndfsec-dim}}

The number of dimensions given when specifying an NDF section need not
necessarily correspond with the actual number of NDF dimensions,
although usually it will do so.

If you specify fewer dimensions than there are NDF dimensions, then
any unspecified bounds will be set to (1:1) for the purposes of
identifying the pixels to which the section should refer.  Conversely, if
extra dimensions are given, then the shape of the NDF will be padded
with extra bounds set to (1:1) in order to match the number of
dimensions.  In all cases, the resulting section will have the number of
dimensions you have actually specified, the padding serving only to
identify the pixels to which the section should refer.

In \KAPPA\ there are a number of applications that can only
handle a fixed number of dimensions ({\it{e.g.}}\
\htmlref{DISPLAY}{DISPLAY}, \htmlref{LINPLOT}{LINPLOT},
\htmlref{MEDIAN}{MEDIAN}).  NDF sections permit such applications to
have wider applicability, since the applications can operate on full
NDFs of arbitrary dimensionality.  So for instance, DISPLAY could show
planes of a datacube.

\subsection{\xlabel{se_ndfsec-mixing}Mixing Bounds Expressions
\label{se:ndfsec-mixing}}

In the \htmlref{last example}{se:ndfsec-wcs}
\latex{ (in Section~\ref{se:ndfsec-wcs})} both {\em axis\/}
co-ordinates and pixel indices were mixed in the same subscript
expression.  In fact, any of the features described earlier may be
combined when specifying an NDF section, the only restrictions are as
follows.

\begin{enumerate}
\item
When the shape of the resulting section is expressed in pixel
indices, the lower bound must not exceed the upper bound in any dimension.

\item
If the bounds for an axis are specified by centre and width values
(rather than as lower and upper bounds), then a WCS value should not be
used with a pixel index.  That is, the centre and width values must both
refer to the same co-ordinate system.
\end{enumerate}

Thus, all the following might be used as valid specifications for NDF
sections

\begin{terminalv}
ndf(3.7)
ndf(,5:)
ndf(-77:01h23m45s,,4)
ndf(66~9,4:17)
ndf(~5,6~)
ndf(~,:)
ndf(5500.0~150,)
ndf(2.137~10%)
ndf(3.0~1.5,-78.06D-3:13.0545,,,,)
\end{terminalv}


Many other combinations are obviously possible. In cases where some
bounds are given in pixel indices and some in WCS co-ordinates, two
boxes will be formed; one representing the pixel-index bounds and one
representing the WCS bounds.  The actual NDF section used will be the
overlap of the two boxes.  The pixel box will inherit any pixel index
limits supplied in the bounds expression, and will use default values
for any missing limits.  These default pixel-index bounds are just the
bounds of the full NDF.  Likewise, the WCS box will inherit any WCS
limits supplied in the bounds expression, and will use default values
for any missing limits.  The default WCS limits are the bounds of a
box that just includes the whole pixel box.

\newpage
\section{\xlabel{se_ndfhistory}NDF History\label{se:ndfhistory}}

During a spring clean of directories to free some space (what d'
y'mean you don't?), most of us will have encountered data files whose
purpose and worth are long forgotten.  We're reluctant to remove them
in case they contain irreplaceable data.  Some people are very good and
make copious notes\ldots\ Even then the result of a casual experiment
might not be recorded.  For those who are lazy, such files can be a
frequent dilemma.  Even a quick look at a plot of the data is often
of little assistance.  As you've probably surmised, the
\xref{NDF}{sun33}{the_history_component}\ offers a solution.

Within an NDF you may record {\em history\/} information.  This is
usually a chronicle of the processing stages used to form the NDF,
including the parameter values of the applications invoked; but it may
also include commentary you provide, for example, the rationale for
doing certain operations.

History is associated with individual NDFs; it is not some global
attribute of a data-processing session.  An NDF has a {\em history
update mode}, which remains with the NDF and any descendant NDF, until
the update mode is altered or the history erased.  By default, the update
mode is \texttt{"Disabled"}, meaning that no history recording occurs.  To
permit history recording you must first switch it on, selecting from
three update modes---\texttt{"Quiet"}, \texttt{"Normal"}, and \texttt{
"Verbose"}---which give increasingly more detailed information.
% There are some examples below, for you to gauge which is appropriate.

\subsection{\xlabel{se_conhistory}Control and Content of History Recording
\label{se:conhistory}}

Task \htmlref{HISSET}{HISSET} lets you set the history update mode.
The default is \texttt{"Normal"}, thus here the command

\begin{terminalv}
     % hisset hr1068
\end{terminalv}
switches normal history recording on for NDF hr1068.  Thereafter
whenever you alter this NDF, or create another NDF from it,
the task automatically records the name of the application that was
run, the date and time, a reference name that identifies the NDF, your
username, and some text comprising the command-line parameters and the
full path of the application.  In \KAPPA\ the package
name and version is appended to the application name.  Other packages
may provide task-dependent additional text.

If disc space is not a concern, you might prefer the verbose level.

\begin{terminalv}
     % hisset hr1068 verbose
\end{terminalv}
the supplementary information being the machine type, and its operating
system name and version.

For small datasets, such as spectra, the history can amount to a
significant part of the NDF's size, so for these you might prefer
the quiet level.  This does not record the command line.

HISSET lets you switch off history recording, if you want to do
something `off the record', or erase the history altogether.

\begin{terminalv}
     % hisset hr1068 disabled
     % hisset hr1068 erase
\end{terminalv}


Some applications create new NDFs from scratch, not inheriting the
history records and update mode from an input NDF.  Some examples are
\htmlref{CREFRAME}{CREFRAME}, \htmlref{TRANDAT}{TRANDAT}, and
\htmlref{PSF}{PSF}.  Should you wish these to have history
recording enabled as you create such NDFs, there is an environment
variable \texttt{NDF\_AUTO\_HISTORY} that should be set to a non-zero
integer value, or immediately run \htmlref{HISSET}{HISSET} with the new NDF.
Note that some applications may choose to disregard the value of \texttt{
NDF\_AUTO\_HISTORY} for good reason, such as for ancillary NDFs
created with an NDF extension.  Whenever this option is exercised, the
reference documentation for the task should indicate this behaviour in
its Implementation Notes.  There are currently no such applications in
\KAPPA.

\subsection{\xlabel{se_commenthistory}Adding Commentary to History Recording
\label{se:commenthistory}}

Once history recording is enabled, you can add commentary to an
NDF using \htmlref{HISCOM}{HISCOM}.

\begin{terminalv}
     % hiscom hr1068 i "There may have been cloud during the integration."
\end{terminalv}
You aren't limited to single lines if you respond to the prompt for
the comment.  You can give a series of lines, terminated by supplying
\texttt{{!}}.

\begin{terminalv}
     % hiscom hr1068
     COMMENT - Comment line > The dome may have been obstructing the telescope
     COMMENT - Comment line > during the integration.  We are not sure that the
     COMMENT - Comment line > filter is correct either.
     COMMENT - Comment line > !
\end{terminalv}

If you prefer, you may edit some text into a file and append its
contents to the history records.  Thus

\begin{terminalv}
     % hiscom hr1068 f file=comments.lis
\end{terminalv}
appends the text contained in \file{comments.lis} to the history
records of NDF hr1068.

\subsection{\xlabel{se_listhistory}Listing History Records
\label{se:listhistory}}

At some point you will want to refer back to the history records.
The \htmlref{HISLIST}{HISLIST} task does this.

\begin{terminalv}
     % hislist hr1068

        History listing for NDF structure /home/scratch/dro/hr1068:

        History structure created 1995 Sep 24 11:16:15.000

     1: 1995 Sep 24 11:16:15.000 - HISSET          (NDFPACK V1.0)

        Parameters: MODE='Normal' NDF=@hr1068
        Software: /star/bin/kappa/hisset
\end{terminalv}
Before you ask\ldots at present there are no parameters for selecting a
time interval and there is no output of the machine and username, but
they're not forgotten.

Here is another example showing a series of history records.

\begin{terminalv}
     % hislist hr1068 \\

        History listing for NDF structure /home/scratch/dro/hr1068sm2:

        History structure created 1995 Nov 24 11:16:15.000

     1: 1995 Sep 24 11:16:15.000 - HISSET          (NDFPACK V1.0)

        Parameters: MODE='Normal' NDF=@hr1068
        Software: /star/bin/kappa/hisset

     2: 1995 Sep 24 11:19:53.000 - GAUSMOOTH       (KAPPA V0.9)

        Parameters: BOX=13 FWHM=5 IN=@hr1068 OUT=@hr1068sm TITLE=! WLIM=!
        Software: /star/bin/kappa/gausmooth

     3: 1995 Sep 24 11:20:15.000 - HISSET          (NDFPACK V1.0)

        History update mode changed from NORMAL to VERBOSE.
        Parameters: MODE='Normal' NDF=@hr1068sm
        Software: /star/bin/kappa/hisset

     4: 1995 Sep 24 11:20:49.000 - GAUSMOOTH       (KAPPA V0.9)

        Parameters: BOX=9 FWHM=3 IN=@hr1068sm OUT=@hr1068sm2 TITLE=! WLIM=!
        Software: /star/bin/kappa/gausmooth
        Machine: alpha, System: OSF1 214 (release V3.2)

     5: 1995 Sep 24 11:22:32.000 - HISCOM          (NDFPACK V1.0)

        Parameters: MODE='Interface' NDF=@hr1068sm2 WRAP=TRUE
        Software: /star/bin/kappa/hiscom
        The dome may have been obstructing the telescope during the integration.
        We are not sure that the filter is correct either.
\end{terminalv}

The first history item shows HISSET enabling history.  This was followed by
a smooth of the data with \htmlref{GAUSMOOTH}{GAUSMOOTH}.  Then the
recording level was set to verbose.  The fourth record recalls another
smooth, and this time you can see the machine details.  Finally, some
commentary is added with HISCOM.

\newpage
\section{\xlabel{se_agitate}The Graphics Database\label{se:agitate}}

Have you ever been in a situation where you would like an application to
know about graphics drawn by some other programme?  For instance, you
display an image of the sky, then later you want to obtain the
co-ordinates of the stars within the image via the cursor.  There are
two main approaches to achieving this functionality.  The first is to
duplicate the display code in the \htmlref{CURSOR}{CURSOR} application.  This is
wasteful and inflexible.  The second is to store information about
`pictures' in a database that can be accessed by subsequent graphics
programmes.  This is the technique used by \KAPPA.

Each graphics application that creates a display on a graphics device,
also stores information describing the display in the \emph{graphics
database}.  This is a file, which usually resides in the user's home
directory, and is often referred to as the \emph{AGI
database.}\footnote{``AGI'' is the name of the subroutine library that
provides access to the graphics database.  AGI stands for ``Applications Graphics
Interface'', and is documented in \xref{SUN/48}{sun48}{}.}  Displays are
described in terms of \emph{pictures}.  A picture is basically a
rectangular area on the graphics device within which an application
produces graphical output.  Each time an application creates a picture,
the dimensions and position of the picture (together with other ancillary
information) are stored in the graphics database.  Subsequent applications
can then read this information back from the database, and use it (for
instance) to align new graphics with previously displayed graphics.

The best way to demonstrate the use of the graphics database is to give
some illustrated examples.

\subsection{\xlabel{se_agiaction}The Graphics Database in Action\label{se:agiaction}}

The following examples assume that \KAPPA\ is loaded and
the graphics device---an X-window containing a plotting area of 850 by 500
pixels---is available.  To create such a window use the \texttt{xmake}
(\xref{SUN/130}{sun130}{}) command:

\begin{terminalv}
     % xmake xwindows -height 500 -width 850 -colours 64
\end{terminalv}

This command limits the number of colours used by the X-window to
64.  It is usually a good idea to be sparing with X-window colours.
Creating X-windows with too many colours can restrict the availability of
colours for other X applications.

First of all we make the X-window the current graphics device
(as described in \latexhtml{Section~\ref{se:devglobal}).}{in
\htmlref{global device names}{se:devglobal}.)}  This selection will remain in force
until changed.  The following command would not be necessary if the
global parameter already had this value:

\begin{terminalv}
     % gdset xwindows
\end{terminalv}
Next we shall clear the X-window, and empty the graphics database
of xwindows pictures (pictures relating to other graphics
devices are retained):

\begin{terminalv}
     % gdclear
\end{terminalv}

The display is now clear, but one picture remains in the graphics
database; this is the BASE picture and corresponds to the entire
plotting area.  To understand why this BASE picture is required we need to
introduce the idea of the \emph{current picture}.

At any time, one of the pictures stored within the graphics database is
nominated as the current picture.  All graphical applications are
written so that the graphical output that they create is scaled to fit
within the current picture.  Thus, the plotting area used by a graphics
application can be controlled by selecting a suitable current picture
before running the application.  Since we have just cleared the database,
the only picture remaining is the BASE picture, which consequently
becomes the current picture.  This allows subsequent applications to draw
in any part of the plotting area.

We now load a grey-scale colour table into the X-windows and display an
IRAS 12~$\mu$m image of M31.  The pixel values are scaled so that 10\% of
the pixels appear as pure black, and 1\% appear as pure white.  We ensure
that no annotated axes are produced (this will result in the image being
a little larger since no margins need to be left for the axes):

\begin{terminalv}
     % lutgrey
     % display $KAPPA_DIR/iras noaxes
     MODE - Method to define the scaling limits /'FLASH'/ > perc
     PERCENTILES - Percentiles for scaling /[10,90]/ > 10,99
     Data will be scaled from 0.05178363 to 1.051904.
\end{terminalv}

   \begin{figure}[hbt]
   \begin{center}
   \includegraphics[clip,height=110mm]{sun95_gd1}
   \caption{An IRAS 12 $\mu$m image of M31 displayed in the middle of the BASE picture.}
   \label{fi:agi1}
   \end{center}
   \end{figure}

The X-window should now look like Figure~
\latexhtml{\ref{fi:agi1}}{1}.  The application has displayed the
image in the middle of the current picture (the BASE picture), and has made
it as large as possible, subject to the constraints that it must lie entirely
within the current picture.  Note, as with many IRAS images, equatorial
north is downwards in this image

Let's now use the \htmlref{PICLIST}{PICLIST} command to look at the contents
of the graphics database (press return in response for the prompt for
Parameter PICNUM):

\begin{terminalv}
     % piclist

       No. Name             Comment                   Label            Ref
       -------------------------------------------------------------------
     C   1 BASE             Base picture
         2 DATA             KAPPA_DISPLAY                              Ref
     PICNUM - Number of new current picture /!/ >
\end{terminalv}

This shows us that there are now two pictures in the database, listed in
the order in which they were created.  The current picture is still the
base picture, as indicated by the letter \texttt{C} at the left of the line
describing picture number 1.  The second picture was created by the
\KAPPA\ application DISPLAY, and has the name DATA,
indicating that it contains a representation of a set of data values.
DATA is one of four standard picture \emph{names}.  BASE is another of
these standard names.  We shall come across the other two shortly.  The
PICLIST application allows you to select a new current picture by
supplying a picture number in response to the prompt for parameter
PICNUM.  Accepting the null default by pressing return causes the current
picture on entry to be retained.

The above use of \htmlref{DISPLAY}{DISPLAY} illustrates an important
rule regarding the behaviour of most graphical applications; {\em the
current picture is not changed by applications that produce graphical
output}.\footnote{This rule does not apply to applications that
manage the database itself rather than producing graphical output.  Thus,
for instance, it is legal for PICLIST to change the current picture if
you request such a change.  Also, an uncontrolled exit from an application,
{\it{e.g.}}\ \texttt{CTRL/C} may leave the database in an abnormal state.}
If it were not for this rule, pictures would become progressively
smaller, vanishing into the distance, since new pictures cannot be
drawn outside the current picture.

We will now display an optical image of M31.  This time we will arrange
for it to be placed towards the left hand side of the X-window.  To do this,
we first clear the whole X-window and graphics database using
\htmlref{GDCLEAR}{GDCLEAR}:

\begin{terminalv}
     % gdclear
\end{terminalv}

We now create a new picture using the \htmlref{PICDEF}{PICDEF} command
(note, the \verb+"\"+ characters are needed to prevent the UNIX shell
interpreting the square bracket characters):

\begin{terminalv}
     % picdef mode=cl fraction=\[0.6,1.0\] outline=no
\end{terminalv}

The MODE parameter specifies the position of the new picture within the
BASE picture; in this case \texttt{"cl"} indicates that the new picture is to be
centred (\texttt{"c"}) vertically within the BASE picture and placed at the left
(\texttt{"l"}) hand edge.  The FRACTION parameter specifies the dimensions of the
picture; the first value (\texttt{0.6}) gives the horizontal size of the picture
as a fraction of the horizontal size of the BASE picture, and the second
value (\texttt{1.0}) gives the vertical size of the picture as a fraction of the
vertical size of the BASE picture.  Thus, the new picture is just over
half the width of the BASE picture, and is the full height of the BASE
picture.  The OUTLINE parameter specifies whether a box should be drawn on
the screen showing the outline of the new picture.  In this case we switch
this option off.

If we run PICLIST again, we get:

\begin{terminalv}
     % piclist

       No. Name             Comment                   Label            Ref
       -------------------------------------------------------------------
         1 BASE             Base picture
     C   2 FRAME            KAPPA_PICDEF
     PICNUM - Number of new current picture /!/ >
\end{terminalv}

The picture created earlier by DISPLAY was deleted when we ran GDCLEAR.
The BASE picture is still there (of course), and we also have the picture
created by PICDEF.  This is a FRAME picture, another of the four standard
picture names.  A FRAME picture acts as a `frame' for other pictures.  A
FRAME picture can itself contain other nested FRAME pictures, together
with DATA and KEY pictures.  The picture created by PICDEF is the current
picture (indicated by the letter \texttt{C} again), and so subsequent
graphics applications will arrange for any pictures they create to fall
entirely within this FRAME picture.

Now display the image, this time including annotated axes around the
edges.\footnote{The current co-ordinate Frame in the image being used is
RA/DEC and so the axis will be annotated in RA and DEC.} DISPLAY will
ensure that all its output (both the image and the axis annotation) fall
within the current picture (\emph{i.e.} the picture created by PICDEF above).  We
scale the pixel values to produce a negative image in which 1\% of
the pixels appear black and 30\% appear white:

\begin{terminalv}
     % display axes
     IN - NDF to be displayed /@$KAPPA_DIR/iras/ > $KAPPA_DIR/m31
     MODE - Method to define scaling limits /'perc'/ >
     PERCENTILES - Percentiles for scaling /[10,99]/ > 99,30
     Data will be scaled from 10854.78 to 3889.016.
\end{terminalv}

   \begin{figure}[hbt]
   \begin{center}
   \includegraphics[clip,height=110mm]{sun95_gd2}
   \caption{Optical M31 image with axes displayed toward the left of the BASE picture.}
   \label{fi:agi2}
   \end{center}
   \end{figure}

The X-window should now look like Figure~
\latexhtml{\ref{fi:agi2}}{2}.  We can use \htmlref{PICLIST}{PICLIST}
again to list the pictures stored in the database:

\begin{terminalv}
     % piclist
       No. Name             Comment                   Label            Ref
       -------------------------------------------------------------------
         1 BASE             Base picture
     C   2 FRAME            KAPPA_PICDEF
         3 FRAME            KAPPA_DISPLAY
         4 DATA             KAPPA_DISPLAY                              Ref
     PICNUM - Number of new current picture /!/ >
\end{terminalv}

There are four pictures this time; the BASE picture, the picture created
by \htmlref{PICDEF}{PICDEF} (number 2), and two pictures created by
\htmlref{DISPLAY}{DISPLAY} (numbers 3 and 4).  Picture number 2 was made the
current picture by PICDEF and as explained above,
DISPLAY did not change this.  DISPLAY creates two
pictures, one (the DATA picture) containing just the image area itself,
and another (the FRAME picture) to act as a frame for the DATA picture
and the annotated axes.

Let's say you wanted to display an enlarged sub-section of the image in
the top-right corner of the X-window.  First, you need to decide on the
bounds of the sub-section to be displayed.  Here, we use a graphics cursor to
indicate the bottom left and top-right corners of a box enclosing the
required area.  Click the left mouse button at the bottom left and top
right corners of a box enclosing the required sub-section of the image:

\begin{terminalv}
     % cursor showpixel plot=box maxpos=2

       Use the cursor to select up to 2 positions to be reported.
         To select a position press the space bar or left mouse button
         To forget the previous position press "d" or the middle mouse button
         To quit press "." or the right mouse button

      Picture comment: KAPPA_DISPLAY, name: DATA, reporting: SKY co-ordinates

      RA = 0:41:33.5 (hh:mm:ss.s)   Dec = 40:33:01 (ddd:mm:ss)
      (172.4       78.2)

      RA = 0:39:25.0    Dec = 40:54:17
      (212.8       113.9)

\end{terminalv}

   \begin{figure}[hbt]
   \begin{center}
   \includegraphics[clip,height=110mm]{sun95_gd3}
   \caption{A box is drawn using the CURSOR application.}
   \label{fi:agi3}
   \end{center}
   \end{figure}

The box I selected is indicated in Figure~ \latexhtml{\ref{fi:agi3}}{3}.

When an application such as DISPLAY produces a DATA picture containing
a representation of an NDF, it saves a copy of the NDF's \htmlref{WCS
component}{apndf:wcs}~ (which contains co-ordinate
\xref{Frame}{sun210}{Frame}~ information) with the DATA picture in the
graphics database.  When \htmlref{CURSOR}{CURSOR} subsequently reports
a position, it uses this saved WCS information to convert the graphics
co-ordinates at the cursor into any of the available co-ordinate
Frames.  By default, CURSOR reports positions in the co-ordinate Frame
that was current when the NDF was displayed (RA/DEC in this case),
but other co-ordinate Frames may be requested using the FRAME
parameter (\emph{e.g.} \texttt{FRAME=PIXEL} displays pixel co-ordinates).
In addition to the co-ordinate Frames inherited from the NDF, there
are also four extra Frames available.

\begin{description}

\item [GRAPHICS] -- specifies positions in terms of millimetres
from the bottom left corner of the graphics device (\emph{e.g.} X-window or
paper).

\item [BASEPIC] -- similar to GRAPHICS but specifies positions in a
normalised co-ordinate system in which the shorter dimension of the
screen or paper has a length of 1.0 (the scales on both axes are equal).

\item [NDC] -- similar to BASEPIC but specifies positions in a
normalised co-ordinate system in which the bottom-left corner has
co-ordinates $(0,0)$ and the top-right corner has co-ordinates (1,1).
Thus, for a non-square device the scales on the two axes will be different.

\item [CURPIC] -- similar to BASEPIC except that it covers only the
specified picture.  Thus, the bottom-left corner of each picture is
at $(0,0)$ and the shorter dimension of each picture has length 1.0.

\end{description}

Going through the parameters supplied to the above CURSOR command, the
first one (SHOWPIXEL) causes the pixel co-ordinates at each selected
position to be displayed, in addition to the co-ordinates selected using
Parameter FRAME (which defaults in this case to RA/DEC since FRAME was
not specified).  For instance, the first position is at a right
ascension of \ra{0;41;33.5}
and a declination of
\dec{+40;33;01}, and has pixel
co-ordinates $(172.4,78.2)$.\footnote{Pixel \emph{co-ordinates} are
fractional, where-as pixel \emph{indices} are integer.  The pixel with
indices $(1,1)$ covers a range of pixel co-ordinates between $0.0$ and
$1.0$ on each axis, with its centre at pixel co-ordinates $(0.5,0.5)$.}
The second parameter (\texttt{PLOT=BOX}) causes a box to be drawn on the
screen between each pair of positions.  There are several other allowed
values for the PLOT parameter, which mark the positions in different ways
(markers, poly-lines, chains, text, \emph{etc.}).  The last parameter
(\texttt{MAXPOS=2})
is purely a convenience, and causes the application to
terminate when two positions have been supplied.  Without this, you would
need to press the right mouse button once the two positions had been
given to indicate that you do not want to supply any more positions.

We now need to create a new FRAME picture to contain the magnified
image section.  We have seen how \htmlref{PICDEF}{PICDEF} can be used
to create a FRAME picture of a given size at a given position within
the BASE picture, but there are also several other ways in which
PICDEF can be used.  Here, we use PICDEF in `cursor' mode; the bottom
left and top-right corners of the new picture are specified by
pointing and clicking with the mouse.\footnote{Note, by default, the
FRAME picture created by PICDEF is \emph{not} constrained to be within
the current picture, it can be anywhere within the BASE picture.} We
use this mode to create a new FRAME picture occupying the area to the
top right of the existing image:

\begin{terminalv}
     % picdef mode=cursor nooutline

       Use the cursor to select 2 distinct points.
         To select a point press the space bar or left mouse button
         To quit press "." or the right mouse button

     Co-ordinates are ( 0.9155139, 0.5470942 ) and ( 1.683106, 0.9799599 )

\end{terminalv}

The co-ordinates displayed by PICDEF are BASEPIC co-ordinates.

We now display the required image section in this new FRAME picture.  The
screen output from CURSOR above shows that the selected image section
runs from pixel 173 to 213 on the first (X) axis, and from pixel 79 to
114 on the second (Y) axis.  To display just this section we include the
pixel bounds in the specification of the input NDF when we run DISPLAY:

\begin{terminalv}
     % display border noaxes mode=scale high=3889.016 low=10854.78
     IN - NDF to be displayed /@m31/ > m31(173:213,79:114)
     Data will be scaled from 10854.78 to 3889.016.
\end{terminalv}

   \begin{figure}[hbt]
   \begin{center}
   \includegraphics[clip,height=110mm]{sun95_gd4}
   \caption{The selected section of the NDF is re-displayed.}
   \label{fi:agi4}
   \end{center}
   \end{figure}

The X-window should now look like Figure~ \latexhtml{\ref{fi:agi4}}{4}.

The string \verb+m31(173:213,79:114)+ is called an \emph{NDF section
specifier}.  These are described more fully \slhyperref{here}{in Section~}
{}{se:ndfsect}.  The image is surrounded by a thin border instead of fully
annotated axes in order to make the image larger.  This is achieved using
the BORDER and NOAXES keywords (equivalent to setting BORDER=YES and
AXES=NO).  The pixel scaling was specified explicitly using parameters HIGH
and LOW in order to ensure that the image was displayed with the same
grey-scale as the main image (the high and low data values are the data
values corresponding to black and white---or white and black if reversed
-- and were reported when the main image was displayed earlier).

Remember the IRAS image of M31 we started with? We'll now overlay
contours of the IRAS image on top of the magnified section of the visual
image we have just displayed.  Don't forget, these two images have not
been aligned---for instance, north is up in the visual image, but down
in the IRAS image.  However, both images have information within the WCS
component describing the relationship between pixel co-ordinates and
RA/DEC.  This WCS information was stored with the DATA picture created by
DISPLAY above, and so can be used by the \htmlref{CONTOUR}{CONTOUR}
application in order to draw the IRAS contours in alignment with the
displayed visual image.\footnote{The accuracy of this alignment will
depend on the accuracy of the astrometry information supplied with the
image.} In addition, we also use CURSOR to mark a position with a text
string giving a title for the contour plot:

\begin{terminalv}
     % contour noclear noaxes nokey iras mode=perc percentiles=\[55,75,95\]
     Alignment has occurred within the SKY Domain.

     Contour heights used:
     0.5499523,   0.7577766,   1.012756.

     % cursor plot=text maxpos=1 minpos=1 strings='"IRAS 12 \gmm contours"'

     Use the cursor to select 1 position to be reported.
       To select a position press the space bar or left mouse button
       To quit press "." or the right mouse button

       Picture comment: Base picture, name: BASE, reporting: BASEPIC
       co-ordinates
       X = 1.308796      Y = 0.9839679

\end{terminalv}

   \begin{figure}[hbt]
   \begin{center}
   \includegraphics[clip,height=110mm]{sun95_gd5}
   \caption{IRAS contours overlayed on the visual image.}
   \label{fi:agi5}
   \end{center}
   \end{figure}

The X-window should now look like Figure~ \latexhtml{\ref{fi:agi5}}{5}.

The most important parameter in the above invocation of CONTOUR is the
NOCLEAR keyword (equivalent to CLEAR=NO).  This tells CONTOUR not to clear
the graphics device before drawing the contours.  Instead, CONTOUR
looks for an existing DATA picture contained within the current picture.
If one is found, then CONTOUR attempts to align the contours using the
WCS information stored with the existing DATA picture.  In our case, the
current picture is still the top-right FRAME picture created by PICDEF,
and so the contours are aligned using the WCS information stored with the
DATA picture contained within this FRAME picture (\emph{i.e.} the magnified image
section).  CONTOUR tells us that this alignment occurred within the `SKY
Domain'---if either of the two NDFs had not contained a calibration in a
suitable celestial co-ordinate system, then alignment on the sky could
not have been performed.  In this case, CONTOUR would have aligned the
contours in the `PIXEL Domain' (\emph{i.e.} in pixel co-ordinates).

Since we produced no annotated axes, a title string was added using the text
drawing abilities of CURSOR.  The pointer was positioned above the contour
plot, and the left button clicked.  The specified text string was drawn
centred at this cursor position.  Note, the string \verb+\gm+ is a
`\PGPLOT\  escape sequence' that represents the Greek letter mu
(\texttt{"$\mu$"}).  See the \PGPLOT\  manual for a detailed description of the
available escape sequences.

Let's say we wanted to compare the data values in the visual and IRAS
images along a given line through the magnified sub-section.  To do this,
we first use the application \htmlref{PROFILE}{PROFILE} to create a pair
of one-dimensional NDFs containing the data values along the line in each
of the two images.  We then display these one-dimensional NDFs using
application \htmlref{LINPLOT}{LINPLOT}.  First we choose the line along
which the profiles are to be taken, using CURSOR (again!):

\begin{terminalv}
     % cursor plot=chain marker=3 style='colour=white' outcat=zz maxpos=2

     Use the cursor to select up to 2 positions to be reported.
       To select a position press the space bar or left mouse button
       To forget the previous position press "d" or the middle mouse button
       To quit press "." or the right mouse button


     Picture comment: KAPPA_CONTOUR, name: DATA, reporting: SKY co-ordinates
       RA = 0:41:22.7 (hh:mm:ss.s)   Dec = 40:35:12 (ddd:mm:ss)
       (175.8467    81.82527)

       RA = 0:39:56.5    Dec = 40:50:27
       (202.917     107.3986)

\end{terminalv}

The pointer is positioned at the two ends of the required profile, and the
left button clicked at each position.  The supplied positions are marked by
two markers of \PGPLOT\  type 3 (small crosses), and a white line is drawn between
them (forming a `chain').  The selected positions are written to an
output catalogue stored in file \file{zz.FIT}.  We now sample the data in the
two images along this profile to create a pair of  one-dimensional NDFs
(\verb+m31_prof+ and \verb+iras_prof+):

\begin{terminalv}
     % profile m31 incat=zz out=m31_prof
       Alignment has occurred within the SKY Domain.
       Profile contains 38 samples.

     % profile iras incat=zz out=iras_prof
       Alignment has occurred within the SKY Domain.
       Profile contains 38 samples.

\end{terminalv}

We want to draw the two line profiles in a new plot in the clear area at
the bottom right of the X-window.  We therefore need to create a new FRAME
picture.  This time we use PICDEF in `XY mode', in which the bounds of
the new FRAME picture are given explicitly using parameters UBOUND and
LBOUND (we could have used cursor mode again; we use XY mode just to
explore the different possibilities):

\begin{terminalv}
     % picdef mode=xy lbound=\[0.93,0.03\] ubound=\[1.67,0.54\] nooutline
       Bounds are ( 0, 0 ) and ( 1.701635, 1 )
\end{terminalv}

The bounds are supplied in the BASEPIC Frame (the bounds reported by the
application are the bounds of the entire BASE picture).  The new FRAME
picture is, as usual, made the current picture by PICDEF and so any
subsequent graphics applications will draw in the new FRAME picture.

We now draw the first of the two line profiles:

\begin{terminalv}
     % linplot m31_prof style='"tickall=0,colour(curve)=black, \
               drawtitle=0,label(2)=DSS data value (black)"'
\end{terminalv}

   \begin{figure}[hbt]
   \begin{center}
   \includegraphics[clip,height=110mm]{sun95_gd6}
   \caption{Data trace through the visual image.}
   \label{fi:agi6}
   \end{center}
   \end{figure}

The X-window should now look like Figure~ \latexhtml{\ref{fi:agi6}}{6}.
The plotting attributes specified by the STYLE parameter do the following;
\verb+tickall=0+ stops tick marks from being drawn on the un-labelled
edges (\emph{i.e.} the top and right edges); \verb+colour(curve)=black+
specifies that the data curve is to be drawn in black; \verb+drawtitle=0+
prevents a title being drawn at the top of the plot;
\verb+label(2)=DSS data value (black)+ specifies the textual label for
the left hand edge.

We now draw the second line plot over the top of the first line plot:

\begin{terminalv}
     % linplot iras_prof noclear noalign style='"edge(2)=r,tickall=0, \
               colour(curve)=white,drawtitle=0, \
               label(2)=IRAS data value (white)"'

\end{terminalv}

Again, the NOCLEAR keyword prevents \htmlref{LINPLOT}{LINPLOT} from
clearing the graphics device before drawing the new line plot.
Instead, the new line plot is drawn within the same axes as any
existing line plot within the current picture.  By default, the new
line plot would adopt the bounds of the two existing axes.  This would
be inappropriate in this case since the two images have very different
data scales.  To prevent this, we specify the NOALIGN keyword, which
causes the default bounds for the new axes to be determined from the
supplied data instead of the existing line plot.  The STYLE parameter
is similar to the previous line plot except that the data values are
annotated on the right hand edge (\verb+edge(2)=r+), the data curve is
drawn in white, and data label is different.

   \begin{figure}[hbt]
   \begin{center}
   \includegraphics[clip,height=110mm]{sun95_gd7}
   \caption{Data traces through both images.}
   \label{fi:agi7}
   \end{center}
   \end{figure}

The X-window should now look like Figure~ \latexhtml{\ref{fi:agi7}}{7}.

As a final touch, we will add two `warp' lines to the display joining
the corners of the box marking the enlarged image area to the
corresponding corners on the enlarged image.  \htmlref{CURSOR}{CURSOR}
can be used to draw these lines, but we need to take a little care.
Normally, graphics produced by CURSOR will be clipped at the edge of
the picture in which the supplied positions fall.  In our case, the
required lines start in one picture (the main image display DATA
picture), but end in another (the magnified image DATA picture).  To
avoid the lines being clipped when they leave the main image DATA
picture, we first ensure that the current picture is the BASE picture
(\emph{i.e.} the whole screen), and we tell CURSOR to report positions
within the current picture.  Normally, CURSOR reports each position
within the most recent picture under the cursor, but setting parameter
MODE=CURRENT when running CURSOR means that the reported positions
always refer to the co-ordinate system of the current picture.  So,
first make the BASE picture the current picture.  This is done using
PICLIST, remembering that the BASE picture is always picture number 1:

\begin{terminalv}
     % piclist picnum=1
\end{terminalv}

We now run CURSOR to draw the first warp line:

\begin{terminalv}
     % cursor plot=poly mode=current maxpos=2
\end{terminalv}

Position the pointer over the top left corner of the black box marking
the selected image section in the main image display, and click the left
button.  Then position the pointer over the top left corner of the border
surrounding the enlarged image display, and click again.  A line is drawn
between these two points.

We now run CURSOR again to draw the second warp line:

\begin{terminalv}
     % cursor plot=poly mode=current maxpos=2
\end{terminalv}

Point and click over the bottom right corners of the two boxes this time.
The final X-window should look now look like Figure~
\latexhtml{\ref{fi:agi8}}{8}.
   \begin{figure}[hbt]
   \begin{center}
   \includegraphics[clip,height=110mm]{sun95_gd8}
   \caption{The complete display with warps.}
   \label{fi:agi8}
   \end{center}
   \end{figure}

\subsection{\xlabel{se_agiother}Other Graphics Database Facilities \label{se:agiother}}

Various other facilities related to the graphics database exist as well
as those described in the previous section.  This section gives a brief
description of a few more:

\begin{itemize}

\item A `label' can be associated with a picture, using application
\htmlref{PICLABEL}{PICLABEL}.  This provides a convenient `handle' by which
pictures can be referred to.  For instance:

\begin{terminalv}
     % piclabel fred
\end{terminalv}

will give the label FRED to the current picture.  This picture could be
made current again at a later time by re-selecting it using
\htmlref{PICSEL}{PICSEL}:

\begin{terminalv}
     % picsel fred
\end{terminalv}

Any label associated with a picture is displayed in the list produced by
\htmlref{PICLIST}{PICLIST}.

\item PICDEF has one further mode---Array.  This enables you to create an
$n\times m$ grid of new FRAME pictures.  It also has a mechanism for
labelling all the pictures, so you can easily switch between the
elements of the picture array.  You might use the following command in
a shell script to display a series of up to twelve spectra:

\begin{terminalv}
     % picdef mode=a prefix=spec xpic=3 ypic=4
\end{terminalv}

The bottom-left picture would be labelled SPEC1 and the rest are
numbered in sequence from left to right to SPEC12---the top-right
picture.  You'd call \htmlref{PICSEL}{PICSEL} to select each picture
in turn via a \texttt{while} loop in a C-shell
script\latex{ (see Section~\ref{se:cshscript})}.  Since this is a
common operation a shorthand command, \htmlref{PICGRID}{PICGRID}, is
available.  For instance:

\begin{terminalv}
     % picgrid 3 4
\end{terminalv}

is equivalent to the previous example, except that the pictures are
labelled 1 to 12.

\item You can see that montages of pictures can rapidly be built.
Occasionally, you will want some earlier picture to become the current
picture.  As we've seen, a labelled picture can be recalled via PICSEL,
but not all pictures will be labelled, especially ones with name \texttt{
DATA}, because of the rule that applications must not change the
current picture.  Another way to select a new current picture is via
the command \htmlref{PICCUR}{PICCUR}.  It displays a cursor.  Move the
cursor to lie on top of the picture you require and select a point
following the instructions (usually by pressing the left-button of the
mouse), then exit (normally by hitting the right-hand mouse button).
Generally, this will be fine, but you can have cases where one plot is
still visible through a transparent plot drawn subsequently.  If the
later picture extends entirely over the image you require, PICCUR will
not let you access it.  The moral is ``be careful when arranging your
pictures''.  A picture may only be partially obscured, so by moving
the cursor around and hitting the left-hand button you can often find
a portion that is topmost.  PICCUR reports the name, comment and the
label (if it there is one) of the picture in which the cursor is
located to assist you.  It is usually quite obvious where pictures
begin and end, so in practice it is easier than described here.

\item If you do get lost or forget what and where the current picture is,
the \htmlref{GDSTATE}{GDSTATE} command will come to your rescue.  You can
even plot an outline with the OUTLINE keyword if you can't visualise the
device co-ordinates.  In the following example, the current picture does
not have a label.  If it did this too would be listed by GDSTATE:

\begin{terminalv}
     % gdstate

     Status of the xwindows window graphics device...

        The current picture is a FRAME picture.
        Comment: KAPPA_PICDEF
        Current co-ordinate Frame: BASEPIC

     Picture bounds in the BASEPIC Frame:
        Axis 1 (X) : 0.516 to 0.891
        Axis 2 (Y) : 0.010 to 0.605

\end{terminalv}

\end{itemize}

\subsection{\xlabel{se_agiframes}The Co-ordinate Frames Associated with a
Picture\label{se:agiframes}}

Each picture in the graphics database has associated with it several
co-ordinate Frames.  Some of these describe positions within the displayed
data array, and others describe the corresponding positions on the
graphics device.  Each Frame is referred to by a \emph{Domain name} (see
\slhyperref{here}{Section~}{}{se:domains} for more information about
co-ordinate Frames and Domains).

The following Domains may be used to specify positions within any
picture:

\begin{description}

\item [GRAPHICS] --- This gives positions in millimetres from the bottom
left corner of the graphics device.

\item [BASEPIC] --- This gives positions within a normalised co-ordinate
system spanning the BASE picture (\emph{i.e.} the entire graphics device).  The
units on both axes are set so that either the width or the
height of the graphics device, whichever is smaller, is set to 1.0
(put another way, a unit square would fit in the picture and span the
shorter axis).  The bottom-left corner of the graphics device is (0,~0).

\item [NDC] --- Normalised Device Co-ordinates, which are similar to BASEPIC
but specifies positions in a normalised co-ordinate system in which
the bottom-left corner has co-ordinates (0,~0) and the top-right corner
has co-ordinates (1,~1).  Thus, for a non-square device the scales on
the two axes will be different.

\item [CURPIC] --- similar to BASEPIC except that it covers only the
specified picture.  Thus, the bottom-left corner of each picture is
at (0,~0) and the shorter dimension of each picture has length 1.0.

\end{description}

DATA pictures have additional co-ordinate
\xref{Frames}{sun210}{Frame}~ inherited from the
\slhyperref{WCS information}{WCS information (see Section~}{)}{se:wcsuse}
associated with the displayed data.  The details of the available Frames
will depend on the application that created the picture and the nature
of the data.  For instance, if an image is displayed in which the current
Frame is a SKY Frame (calibrated in RA/DEC for instance), then the
Domains GRID, PIXEL, AXIS and SKY will also be available.

Pictures created by older applications that do not yet support WCS
information will not have all of these Frames.  They will still have
GRAPHICS, BASEPIC, NDC and CURPIC Frames, but the only additional Frames
will be:

\begin{description}
\item [AGI\_WORLD] - This corresponds to AGI `world' co-ordinates.
\item [AGI\_DATA] - This corresponds to AGI `data' co-ordinates.
\end{description}

The interpretation of both these Frames will depend on the application
that created the picture, and should be described in the associated
documentation.  Note, the AGI\_DATA Frame may not always be present.  This
again depends on the application.

\subsection{\xlabel{se_agitidy}The Graphics Database File\label{se:agitidy}}

The location of the graphics database file can be controlled by setting the
environment variable \texttt{AGI\_USER} to the desired location.  If \texttt{AGI\_USER}
is undefined, the database file is placed in your home directory.
Thus by default the file is \file{\$HOME/agi\_}{\it $<$node$>$}\file{{.sdf}},
where you substitute your computer's node name for {\it
$<$node$>$}, {\it{e.g.}}\ \file{/home/dro/agi\_rlsaxp.sdf}.  A new
database is created when you run a graphics application if none exists.
AGI will purge the database for a device if the graph window has
changed size, or if you switch between portrait and landscape formats
for a printer device.

The contents of the graphics database are ephemeral.  Therefore you
should regularly purge the database entries with
\htmlref{GDCLEAR}{GDCLEAR} or delete the database file.

If a graphics application is aborted abnormally (for instance by pressing
\texttt{CTRL/C}), the graphics database may be corrupted, potentially
resulting in various forms of peculiar behaviour when subsequent graphics
applications are run.  If you get `strange' behaviour when running a
graphics application, deleting the graphics database file and starting
again will often cure the problem.


\subsection{\xlabel{se_psfiles}Working With PostScript Files\label{se:psfiles}}

Having produced a pretty picture in an X-window, how do we get it on to a
piece of paper? There are basically two ways.  The simplest way is to do
a screen-shot (\emph{i.e.} copy the pixel values from the X-window into a
PNG file, or some other suitable graphics format), and then print the file
on a suitable printer.  There are many ways to do this, depending on your
operating system (\texttt{GIMP}, \texttt{ImageMagick}, \texttt{ksnapshot},
\texttt{print screen}, \emph{etc}). The figures in this chapter were created
using this method.  It's easy---but the results only have the resolution of
your X screen (typically 72 dots per inch) which is usually not good enough
for general publication.

The second method is more involved, but retains the full resolution of
your printer (typically 600 dots per inch or more), producing far
superior results.  It involves running all the \KAPPA applications again,
but using a suitable encapsulated PostScript graphics device instead of
the X-windows device used earlier in this chapter. Each application will
then write its graphical output to one or more disk files.  If you elect
to use one of the ``accumulating'' PostScript devices (as indicated in
the device description displayed by \htmlref{GDNAMES}{GDNAMES}), then the
PostScript output from each application will be merged automatically into
a single file. If you use one of the other PostScript devices, each
application will produce a separate output file, all of which must be
combined using \PSMERGEref\ to create a single PostScript file containing
the entire plot, which can be printed or included in another document.

We will look at these method in more detail in the rest of this section.

\subsubsection{The Choice of Graphics Device}

If you want to combine the results of several applications into a single
plot, you need to specify an \emph{encapsulated} PostScript device.
These differ from normal PostScript files in that they contain information
which allows them to be included within other PostScript files.  The
\htmlref{GDNAMES}{GDNAMES} command displays a list of all the available
graphics devices, together with a brief description of each.  This should
enable you to select an appropriate device name.\footnote{Most
encapsulated PostScript devices start with the string \texttt{"epsf"} or
\texttt{"aps"}.} There may be various encapsulated PostScript devices
available.  For instance, you can choose between colour or monochrome
devices, and between portrait or landscape devices.  The choice of colour
or monochrome obviously depends on the printer you will use.  The choice
of landscape or portrait depends on the shape of the total plot you are
producing---for tall narrow plots choose portrait, for short wide plots
choose landscape.\footnote{The choice or portrait or landscape may also
affect the orientation of the plot when the resulting PostScript file is
included in a document.}

Creating composite graphical output will be easier if you choose one the
devices that are described as ``accumulating'', as this will avoid the
need for you to merge separate PostScript files yourself.

For instance, if you want all applications to write graphical output to
a single colour landscape encapsulated PostScript file set the graphics
device as follows:

\begin{terminalv}
   % gdset apscol_l
\end{terminalv}

\subsubsection{The PostScript Files}

Having selected a suitable encapsulated PostScript graphics device, each
\KAPPA\ application will either modify an existing graphics file, or
produce an new output file in your current directory containing PostScript
commands, depending on whether you choose an accumulating device or not.  These
files can be examined without needing to be printed by using \OKULARref,
\EVINCEref, \texttt{ghostview}, \emph{etc}. By default, the output file
will be called \file{pgplot.ps}.

\begin{quote}
\emph{ Note, if you choose not to use an accumulating device, subsequent graphics
commands will overwrite any file created by earlier graphics commands.  For this
reason you should rename the output file after running each graphics command.  An
alternative approach is to assign a unique value to the DEVICE parameter for each
graphics command (rather than just allowing it to default to the value of the global
parameter set using GDSET).  For instance, \emph{\texttt{DEVICE="epsf\_l;contour.ps"}}
would cause the command to write its output to file \file{contour.ps}.}
\end{quote}

\subsubsection{Combining the Files into a Single File}

\emph{This section is only relevant if you are using a non-accumulating device.}

Once all applications have been run, you will have a potentially long
list of PostScript files in your current directory (unless you use an
accumulating device, in which case you will have only one).  These need to be
stacked together to produce a single PostScript file which can either be
printed or included in a document.  To do this, we use \PSMERGEref\  as
follows\footnote{You can include all the PostScript files, including any
that do not contain any actual drawing commands.}:

\begin{terminalv}
   % psmerge display.ps contour.ps stars.ps > total.ps
\end{terminalv}

This stacks the three specified PostScript files into a single file called
\file{total.ps}.  This will be a normal (\emph{i.e.} not encapsulated) PostScript
file, so you can print it, but it will be difficult to include it in a
document.  If you include the \texttt{-e} option when running
PSMERGE, then an encapsulated PostScript file is created instead:

\begin{terminalv}
   % psmerge -e display.ps contour.ps stars.ps > total.ps
\end{terminalv}

The output file can be included in other documents but often causes problem
when being printed.

Note, the order in which you supply the input files is important because
later files are pasted `on top of' earlier files, and so can potentially
obscure them.  In general, you need to list the input files in the order
in which they were created.

PSMERGE has other facilities which allow you to scale,
shift and rotate individual input files before including them in the
output.  See \xref{SUN/164}{sun164}{} for details.

\subsubsection{Running the Applications}

The main difference between producing X-window output and PostScript
output is that there is no cursor available when using PostScript.  This
means, for instance, that you cannot use \htmlref{PICDEF}{PICDEF} in
cursor mode, and you cannot use \htmlref{CURSOR}{CURSOR} at all! We made
use of both these facilities when we produced
Figure~ \latexhtml{\ref{fi:agi8}}{8}.
Pictures must be defined using PICDEF in \texttt{XY} mode, or one
of the other positional mode (\texttt{CC}, \texttt{BL}, \emph{etc.}).  Annotation
such as produced by CURSOR can be created using
\htmlref{LISTMAKE}{LISTMAKE} and \htmlref{LISTSHOW}{LISTSHOW}.  You specify
the reference positions for the annotation in any suitable co-ordinate
Frame using LISTMAKE.  This creates a \emph{positions list} file
containing these positions, together with associated WCS information.
You then supply this file to LISTSHOW which marks the positions on the
graphics device in any of the ways provided by CURSOR.

\subsubsection{Using X-windows to Produce a Prototype}

It can often be advantageous to design your final plot using an X-windows
graphics device.  This provides more versatility in the form of a graphics
cursor and dynamic lookup table, and also allows you to see the plot more
easily.  Once the plot looks right in the X-window, you can then run the
drawing commands again specifying a suitable PostScript device instead.

If you choose to do this, it can be a big help to ensure that the X-window
you are using has the same shape as your selected PostScript device.  This
prevents you accidentally drawing things within regions of the X-window
that are not available in PostScript.  To find the require aspect ratio,
do:

\begin{terminalv}
   % gdclear device=apscol_l
   % gdstate device=apscol_l
\end{terminalv}

This clears the PostScript device, and then displays the bounds of the
BASE picture (all the other pictures were erased by
\htmlref{GDCLEAR}{GDCLEAR}).  The results
will look something like this:

\begin{terminalv}

Status of the apscol_l graphics device...

   The current picture is a BASE picture.
   Comment: Base picture
   Current co-ordinate Frame: BASEPIC

   Picture bounds in the BASEPIC Frame:
      Axis 1 (X) : 0.000 to 1.455
      Axis 2 (Y) : 0.000 to 1.000

\end{terminalv}

This tells you that the aspect ratio (the ratio of the width to the height)
of the PostScript device is 1.455.  You should now make an X-window with
this aspect ratio:

\begin{terminalv}
   % xdestroy xwindows
   % xmake xwindows -height 500 -width 730
\end{terminalv}

The \texttt{xdestroy} command deletes any existing X-window, and the \texttt{
xmake} command creates a new one with a height of 500 pixels and a
width of 730 pixels, giving the required aspect ratio.  Of course, you
could produce a larger or smaller X-window so long as the ratio of width to
height is close to 1.455.

Another tip to ease prototyping on an X-window---when you run \htmlref{CURSOR}{CURSOR},
always store the selected positions in an output positions list, using
Parameter OUTCAT.  When you come to produce the PostScript files, you can
then supply these positions lists as inputs to
\htmlref{LISTSHOW}{LISTSHOW}, in order to mimic
the annotation produced by CURSOR when you were prototyping.


\subsubsection{An Example}

As an illustrated example, this section describes how to produce a
PostScript file containing a plot similar to that in Figure~
\latexhtml{\ref{fi:agi8}}{8}.  For clarity (at the time), the earlier
sections of this chapter did not adhere to either of the tips given above
when prototyping in an X-window; that is, we did not choose the shape of
the X-window to match our PostScript device, and we did not save the
output from each run of \htmlref{CURSOR}{CURSOR} in a positions list file.
For this reason, the steps taken here are a little more complicated than
they need to be, and do not mimic exactly the steps taken while prototyping.

First, select and clear the graphics device.  We choose a accumulating, monochrome,
portrait, encapsulated PostScript device:

\begin{terminalv}
   % gdset aps_p
   % gdclear
   % okular pgplot.ps &
\end{terminalv}

Since we are using an accumulating PostScript device, we have chosen to use
\OKULARref\ above to display the contents of \file{pgplot.ps}.  We have
just run \htmlref{GDCLEAR}{GDCLEAR}, and so this will produce a blank display.
Note that we have put \OKULAR\ into the background by appending \texttt{\&} to
the end of the command line.  This means that \OKULAR\ will continue to run
throughtout the following example.  As each subsequent graphics application
modifies the contents of \file{pgplot.ps}, the resulting composite plot will
be displayed immediately within the \OKULAR\ window, allowing you to review
progress.  Other modern document viewers such as \EVINCEref\ work in the same
way.

In order to produce higher quality axis annotations, we now tell \PGPLOT\ to use
embedded PostScript fonts rather than it's own internal lower-quality
fonts. Specifically, we use a Times New Roman font:

\begin{terminalv}
   % setenv PGPLOT_PS_FONT Times
\end{terminalv}

We could instead have set \texttt{PGPLOT\_PS\_FONT} to ``Helvetica'',
``Courier'', ``NewCentury'' or ``Zapf'', to use other fonts.

We will sometimes need to specify positions within the BASE picture.
To do this we can either use the GRAPHICS, the BASEPIC or the NDC Frame,
all of which span the entire graphics device.  GRAPHICS is an absolute
co-ordinate system giving millimetres from the bottom left corner of the
graphics device, and BASEPIC is a normalised co-ordinate system in which
both axes have the same scale and the shorter dimension of the graphics
device has length 1.0\footnote{NDC is like BASEPIC but the top-right
corner of the device has NDC co-ordinates (1,~1).}.  The BASEPIC Frame is
usually easier to work with
since it does not depend on the size of the paper.  To determine the
bounds of the available plotting space in the BASEPIC Frame, use
\htmlref{GDSTATE}{GDSTATE}, having first ensured that the BASE picture is
the current picture (this will be the case at the moment since we have just
cleared the database using \htmlref{GDCLEAR}{GDCLEAR}):

\begin{terminalv}
   % gdstate

   Status of the epsf_p graphics device...

      The current picture is a BASE picture.
      Comment: Base picture
      Current co-ordinate Frame: BASEPIC

      Picture bounds in the BASEPIC Frame:
         Axis 1 (X) : 0.000 to 1.000
         Axis 2 (Y) : 0.000 to 1.455

\end{terminalv}

The choice of a portrait PostScript device may have surprised you, given
that Figure~ \latexhtml{\ref{fi:agi8}}{8} seems more naturally to fit
into a landscape page.  Portrait mode was chosen so that the final plot
appears up-right when included in a document.  If landscape mode had been
chosen, the final plot would have been rotated by 90\dgs\ when included
in a latex document such as this one, requiring the page to be viewed on
its side.  However, if we use the entire portrait page, we will have too
much vertical space between the components of the plot.  To avoid this we
only use the bottom third (roughly) of the page---that is, we pretend
that the top of our `page' is at 0.6 instead of 1.454659.

We first create a FRAME picture at the left hand side of the page covering
the full height of our reduced `page' and 0.6 of the width.  We specify
the upper and lower bounds of the picture within the BASEPIC Frame, using
the limits found above, giving the FRAME picture the label `main' so
that we can easily re-select the picture later.  We then display the
optical image within it, including annotated axes:

\begin{terminalv}
   % picdef mode=xy lbound=\[0.0,0.0\] ubound=\[0.6,0.6\] outline=no
   % piclabel main
   % display m31 axes mode=perc percentiles=\[30,99\] style="'colour=black,size=0.6'" \
             margin=\[0.2,0.02\]
\end{terminalv}

Note, we are using a monochrome PostScript device, and so the colour of
all annotation produced by \htmlref{DISPLAY}{DISPLAY} was set to black using the STYLE
parameter.  This is done for all applications that produce graphical
output.  We also request text 0.6 of the default size.  We specified
the MARGIN parameter when running DISPLAY in order to reduce the margin
on the right hand side of the image.  By default, DISPLAY leaves a
margin equal to 0.2 of the width of the DATA picture.  We reduced this to
0.02 to make more room for the other components of the plot (the first
value supplied for MARGIN is the bottom margin and is left at 0.2).

We now draw a box over this image to mark the `selected region'
indicated in Figure~ \latexhtml{\ref{fi:agi3}}{3}.  In this
particular case, we use \htmlref{LISTMAKE}{LISTMAKE} to create a positions list holding the
RA and DEC at the two opposite corners of the box.  Alternatively, we
could have saved the output from CURSOR while prototyping, using the
OUTCAT parameter.  In either case, we use \htmlref{LISTSHOW}{LISTSHOW} to draw the box:

\begin{terminalv}
   % listmake ndf=m31 outcat=boxpos
     POSITION - A position for the output list /!/ > 0:41:33.5 40:33:01
     POSITION - A position for the output list /!/ > 0:39:25.0 40:54:17
     POSITION - A position for the output list /!/ >
   % listshow boxpos plot=box style='colour=black'
\end{terminalv}

Note, the PostScript output generated by \htmlref{LISTSHOW}{LISTSHOW} is
merged automatically into the \file{pgplot.ps} file created by the
previous graphics applications. If we had chosen not to use an
accumulating device, we would need to rename the new \file{pgplot.ps}
file created by each graphics application, and then merge them all together at
the end using PSMERGE.

Specifying \texttt{ndf=m31} resulted in LISTMAKE interpreting the supplied
positions as positions within the current co-ordinate Frame of the NDF
\file{m31}.  It also results in all the \htmlref{WCS}{apndf:wcs}~ information being copied from the
NDF into the output positions list.

We now create a Frame in the top-right corner of the reduced `page' in
which a magnified image of the selected region will be displayed.  The
picture occupies the remainder of the width left over by the main image
(0.4 of the total width) and half the height.:

\begin{terminalv}
   % picdef mode=xy lbound=\[0.6,0.3\] ubound=\[1.0,0.6\] outline=no
\end{terminalv}

We now display the selected image section within the FRAME picture just
created, and overlay contours from \file{iras.sdf}.  We draw the border
with twice the default width:

\begin{terminalv}
   % display 'm31(173:213,79:114)' border noaxes mode=scale low=3889 high=10854.78 \
              borstyle='width=2'
   % picdata
   % piclabel mag
   % contour noclear noaxes nokey iras mode=perc percentiles=\[55,75,95\]
\end{terminalv}

What are those two extra commands in the middle? Well, we will need to be
able to re-select the DATA picture holding the image when we come to draw
the warp-lines.  To do this we need to label the picture now.  For this
reason, we use \htmlref{PICDATA}{PICDATA} to select the DATA picture
created by DISPLAY as the current picture, and then use
\htmlref{PICLABEL}{PICLABEL} to give the label \texttt{"mag"} to the picture.

We now display a title above this DATA picture.  The string is centred
horizontally within the enclosing FRAME picture ({\it{i.e.}} half way
between 0.6 and 1.0), and drawn so that its bottom edge is placed at the
top of the Frame:

\begin{terminalv}
   % listmake frame=basepic outcat=ttlpos
     POSITION - A position for the output list /!/ > 0.8 0.6
     POSITION - A position for the output list /!/ >
   % listshow ttlpos plot=text strings='"IRAS 12 \gmm contours"' just=bc
\end{terminalv}

LISTMAKE is used to store the position in a positions list, and then
LISTSHOW is used to draw the title.  Again, we could have saved the output
from CURSOR in a positions list while prototyping, instead of using
LISTMAKE to create the positions list.  The JUST parameter is set to \texttt{
bc} so that the title string is displayed with its bottom centre at the
supplied position.

We now draw the line marking the path of the profiles that will be
displayed later.  We draw them in white so that they show up against
the predominantly dark background of the image:

\begin{terminalv}
   % listmake ndf=m31 outcat=prfpos
     POSITION - A position for the output list /!/ > 0:41:22.7 40:35:12
     POSITION - A position for the output list /!/ > 0:39:56.5 40:50:27
     POSITION - A position for the output list /!/ >
   % listshow prfpos plot=chain marker=3 style="'colour=white,width=3'"
\end{terminalv}

The line is drawn with three times the default width.  We now create the
profiles as before:

\begin{terminalv}
   % profile m31 incat=prfpos out=m31_prof
   % profile iras incat=prfpos out=iras_prof
\end{terminalv}

We now create a FRAME picture occupying the remaining area at the bottom
right corner of the page, and display the two profiles in it.  We use
line style to differentiate the two curves, instead of colour as we
did when prototyping:

\begin{terminalv}
   % picdef mode=xy lbound=\[0.6,0.0\] ubound=\[1.0,0.3\] outline=no
   % linplot m31_prof style=^style1 keystyle='size=0.7'
   % linplot iras_prof noclear noalign style=^style2
\end{terminalv}

The appearances of the two plots are controlled by the two plotting
styles contained in the text files \file{style1} and \file{style2}.
\file{style1} contains the following attribute settings:

\begin{terminalv}
   tickall=0
   colour=black
   gap(2)=1000
   drawtitle=0
   label(2)=DSS data value (solid)
   textlabgap(1)=0.02
   labelunits=0
\end{terminalv}

The file \file{style2} contains the following attribute settings:

\begin{terminalv}
   edge(2)=r
   gap(2)=0.2
   tickall=0
   colour=black
   style(curve)=2
   drawtitle=0
   label(2)=IRAS data value (dashed)
   textlabgap(1)=0.02
   labelunits=0
\end{terminalv}

Now comes the complicated bit---the warp lines! The problem is that we
used the cursor to define the start and end of these lines when
prototyping.  This was good enough for an X-window that has relatively
low resolution, but will not do for PostScript.  We cannot place the
cursor accurately enough to ensure that there is no visible gap between
the end of the line and the corresponding box corner when using
PostScript, so we cannot use the positions found while prototyping.  The
only way to ensure that the lines join up properly with the box corners
is to use the co-ordinates of the box corners to define the lines.  First,
we need to transform the box corners into the BASEPIC Frame since both
ends of each line must be given within the same co-ordinate Frame.

We start with the `selection box' drawn over the main image.  The
corners of this box are currently stored in the \texttt{boxpos} positions
list.  This file contains the RA and DEC at the corners of the box,
together with WCS information copied from the NDF \file{m31}.  But this
does \emph{not} include the BASEPIC, NDC or CURPIC Frame, which are only
available within
the WCS information stored with graphics database pictures.  In order to
find the BASEPIC co-ordinates at the corners of the box, we need to
align the RA/DEC values in the positions list with the WCS information
stored with the first DATA picture we created (the main image), and then
display the corresponding BASEPIC co-ordinates.  LISTSHOW can do this, but
we need to change the current picture first.  At the moment,
the current picture is the FRAME containing the line plots.  If we did not
change the current picture LISTSHOW would assume that the supplied RA/DEC
values refer to the line plots (not to the main image), and would
consequently give the wrong BASEPIC co-ordinates.  We use
\htmlref{PICSEL}{PICSEL} to
select the FRAME picture containing the main image (so that it becomes
the current picture), and then run LISTSHOW:

\begin{terminalv}
   % picsel main
   % listshow boxpos plot=blank frame=BASEPIC

     Title: M31 (Digitised Sky Survey)

      Position        X           Y
     identifier
     -----------------------------------

       #1         0.3657834   0.1911708
       #2         0.4269177   0.245274

\end{terminalv}

The parameter assignment \texttt{PLOT=BLANK} tells LISTSHOW to search the graphics
database for a picture with which the positions in \texttt{boxpos} can be
aligned, but without marking the positions in any way on the screen.
Doing this means that the co-ordinate Frames stored with the picture are
available when specifying a value for Parameter FRAME.  This is necessary
because the BASEPIC Frame is only available within graphics data base
pictures since it describes positions on a graphics device.

We now find the BASEPIC co-ordinates at the corners of the magnified image.
The most simple way of doing this is to use the same RA/DEC positions.  The
problem with this approach is that the bounds of the magnified image were
rounded to a whole number of pixels and so may not correspond exactly to
the RA an DEC at the corners of the selection box.  A more accurate method
is to use the pixel indices at the corners of the image section to
define the box corners.  We first need to create a positions list holding
the pixel co-ordinate bounds of the image section, remembering to reduce
the lower pixel index bounds by 1.0 in order to convert the integer
\emph{pixel indices} into floatingpoint \emph{pixel co-ordinates}
(see \slhyperref{here}{Section~}{}{se:pixgrd} for more information about
pixel co-ordinates and indices):

\begin{terminalv}
   % listmake frame=pixel dim=2 outcat=magpos
     POSITION - A position for the output list /!/ > 172.0 78.0
     POSITION - A position for the output list /!/ > 213.0 114.0
     POSITION - A position for the output list /!/ >
\end{terminalv}

We now need to use LISTSHOW to get the corresponding BASEPIC co-ordinates,
aligning the above PIXEL co-ordinates with the DATA picture holding the
magnified image section.  Another slight complication arises here since the
required DATA picture is obscured by the DATA picture holding the IRAS
contours.  Pixel co-ordinates in the IRAS image are totally different to
those in the grey-scale image, and so we need to make sure that LISTSHOW
is using the correct DATA picture.  However, we labelled the required DATA
picture when it was created and so we can just use PICSEL to re-select it:

\begin{terminalv}
   % picsel mag
   % listshow magpos plot=blank frame=BASEPIC
     Alignment has occurred within the PIXEL Domain.


     Title: Output from LISTMAKE

      Position        X           Y
     identifier
     -----------------------------------

       #1         0.6473798   0.3159827
       #2         0.9524395   0.5838399

\end{terminalv}

We now create a positions list containing the BASEPIC co-ordinates at
the ends of the upper warp line, and draw it.  By default, LISTSHOW draws
within the most recent DATA picture contained within the current picture.
Since the line is not contained within a single DATA picture, we need to
tell LISTSHOW to draw within the BASE picture.  This is done by setting
Parameter NAME to BASE (we also revert to drawing in black):

\begin{terminalv}
   % listmake frame=basepic outcat=war1pos
     POSITION - A position for the output list /!/ > 0.3657834 0.245274
     POSITION - A position for the output list /!/ > 0.6473798 0.5838399
     POSITION - A position for the output list /!/ >
   % listshow war1pos plot=poly style='colour=black' name=base
\end{terminalv}

   \begin{figure}[hbt]
   \begin{center}
   \includegraphics[clip,height=110mm]{sun95_gd9}
   \caption{The equivalent plot produced directly in PostScript.}
   \label{fi:agi9}
   \end{center}
   \end{figure}

The last drawing we need to do is to create the second warp line in the
same way:

\begin{terminalv}
   % listmake frame=basepic outcat=war2pos
     POSITION - A position for the output list /!/ > 0.4269177 0.1911708
     POSITION - A position for the output list /!/ > 0.9524395 0.3159827
     POSITION - A position for the output list /!/ >
   % listshow war2pos plot=poly name=base
\end{terminalv}

That's it.  The final output is left in file \file{pgplot.ps} and
should be visible in the \OKULAR\ window.  It should look like
Figure~ \latexhtml{\ref{fi:agi9}}{9}.  However, if we had chosen not to use
an accumulating PostScript device, we would now be left with lots of PostScript
files---one from each graphics application---that now need to be stacked together
to produce the final encapsulated PostScript file:

\begin{terminalv}
   % psmerge -e display1.ps box.ps display2.ps contour.ps title.ps \
                line.ps m31_prof.ps iras_prof.ps war1.ps war2.ps > total.eps
\end{terminalv}

Note, here we assume that each \file{pgplot.ps} file has been renamed
to one of the above names immediately after it has been created, in
order to avoid it being over-written by the next graphical application.


\newpage
\section{\xlabel{se_wcsuse}Using World Co-ordinate Systems\label{se:wcsuse}}

\subsection{\xlabel{se_pixgrd}Pixel Indices, Pixel Co-ordinates, and Grid
Co-ordinates\label{se:pixgrd}}

In this sub-section we will look at the definition of the four basic
co-ordinate systems available in all NDFs---pixel indices, pixel
co-ordinates, grid co-ordinates, and normalised co-ordinates.

\emph{Pixel indices} are integer values that are used to count the
pixels along each axis of an NDF.  The first pixel can be given any
arbitrary pixel index, and this value is known as the \emph{pixel origin}.
When a section is extracted from an NDF, the pixel origin in the
extracted section is set so that each pixel retains its original pixel
indices (see Figure~\latexhtml{\ref{fi:pixind}}{10}).

   \begin{figure}[hbt]
   \begin{center}
   \includegraphics[clip,width=0.65\textwidth]{sun95_pixind}
   \caption{Pixel indices.}
   \label{fi:pixind}
   \end{center}
   \end{figure}

\emph{Pixel co-ordinates} are floating-point values that allow positions
to be specified with sub-pixel accuracy.  They are related to pixel indices
as indicated in Figure~\latexhtml{\ref{fi:pixco}}{11}).

   \begin{figure}[hbt]
   \begin{center}
   \includegraphics[clip,width=0.75\textwidth]{sun95_pixco}
   \caption{Pixel co-ordinates.}
   \label{fi:pixco}
   \end{center}
   \end{figure}

\emph{Grid co-ordinates} are floating-point values that are similar to
pixel co-ordinates except that the origin is fixed so that the first
pixel on an axis is centred at a grid co-ordinate value of 1.0, no matter
what the pixel origin is.  This corresponds to the FITS idea of `pixel
co-ordinates' (FITS makes no provision for an arbitrary pixel
origin).  When a section is extracted from an array, the grid co-ordinates
of the extracted section include no knowledge of where the section
was located in the original array.  See
Figure~\latexhtml{\ref{fi:gridco}}{12}).

   \begin{figure}[hbt]
   \begin{center}
   \includegraphics[clip,width=0.75\textwidth]{sun95_gridco}
   \caption{Grid co-ordinates.}
   \label{fi:gridco}
   \end{center}
   \end{figure}

\emph{Fraction co-ordinates} are floating-point values that are normalised
pixel or grid co-ordinates such that each axis extends from zero to one.
Thus in Figure~\latexhtml{\ref{fi:fraco}}{13} pixel co-ordinate 2.0 or
grid co-ordinate 0.5 becomes 0.0 in fraction co-ordinates, and pixel
co-ordinate 7.0 or grid co-ordinate 5.5 becomes 1.0 in fraction
co-ordinates.

   \begin{figure}[hbt]
   \begin{center}
   \includegraphics[clip,width=0.7\textwidth]{sun95_fraco}
   \caption{Fraction co-ordinates.}
   \label{fi:fraco}
   \end{center}
   \end{figure}

\subsection{\xlabel{se_domains}Co-ordinate Frames, Axes and Domains\label{se:domains}}

A \emph{co-ordinate Frame} is a system of co-ordinate axes which can be
used to specify a position within an NDF data array.  Within an NDF such
co-ordinate Frames also have associated descriptive information such as
axis labels, axis units, a Frame title, \emph{etc.} These are called the
\emph{attributes} of the Frame, and the most commonly used are listed
briefly below (full descriptions of these attributes are included
\slhyperref{later}{in Appendix~}{}{ap:frmatt}.

\begin{aligndesc}
\item[\att{Digits}]: Number of digits of precision
\item[\att{Domain}]: Physical domain described by the co-ordinate system
\item[\att{Epoch}]: A date \& time that defines the co-ordinate system
\item[\att{Format(axis)}]: Format specification for axis values
\item[\att{Label(axis)}]: Axis label
\item[\att{Naxes}]: Number of Frame axes
\item[\att{Symbol(axis)}]: Axis symbol
\item[\att{System}]: Specific co-ordinate system used to describe the domain
\item[\att{Title}]: Frame title
\item[\att{Unit(axis)}]: Axis physical units
\end{aligndesc}

The single most important attribute is the
\xref{Frame}{sun210}{Frame}~ \att{Domain}, which is an uppercase
character string indicating the physical domain in which the
co-ordinate system is defined.  Frames with the same \att{Domain} can,
in general, be aligned with each other.

Frames with the following \att{Domain} values are defined within every
NDF, no matter how the NDF is created.

\begin{description}

\item[GRID] --- Corresponds to grid co-ordinates.

\item[PIXEL] --- Corresponds to pixel co-ordinates.

\item[FRACTION] --- Corresponds to normalised pixel or grid
co-ordinates where each scaled pixel axis spans the range zero to one.

\item[AXIS] --- Corresponds to the co-ordinate system defined by the
\htmlref{AXIS}{apndf:axis}~ structures within the NDF.

An AXIS structure is a one-dimensional array that maps pixel index along a
given axis on to another related co-ordinate axis (\emph{e.g.} wavelength)
-- the values on the other axes are assumed to be constant.  Each
dimension in the NDF should have an associated AXIS structure.  Such
structures can be created (for instance) using \htmlref{SETAXIS}{SETAXIS}.
If the NDF does \emph{not} have defined AXIS structures, then a default AXIS
Frame will be used in which the values along each axis correspond to
pixel co-ordinates.

AXIS structures can only be used to represent independent axes.  For
instance, celestial co-ordinates cannot (in general) be described using
AXIS structures because celestial longitude and latitude may both vary
along any given row or column of an NDF.\footnote{For this and other
reasons, new applications will avoid using AXIS structures, and make use
of the more versatile \htmlref{WCS component}{apndf:wcs}~ instead.}  If
an application is used that renders the AXIS structures in an NDF
invalid, then the AXIS structures will not be copied to the output
NDF.  For instance, if \htmlref{ROTATE}{ROTATE} is used to rotate an
NDF by an angle that is not a multiple of 90\dgs, then the output NDF
will not have any AXIS structures, and so the AXIS Frame will default
to pixel co-ordinates.

\end{description}

The dimensionality of each of these Frames is equal to that of the NDF
(\emph{e.g.} a two-dimensional NDF will have two-dimensional PIXEL,
FRACTION, GRID, and AXIS Frames).  Axes within a Frame are identified
by an integer index in the range 1 to the number of axes in the Frame.
The above Frames are not stored permanently in the NDF, but are
generated automatically by the NDF access library each time a
reference is made to them.

In addition to the PIXEL, FRACTION, AXIS, and GRID Frames, an NDF may
also contain any number of additional co-ordinates Frames.  Descriptions
of these extra Frames, together with recipes for converting positions between
them, are stored permanently in the NDFs WCS component which may be
examined using the \htmlref{NDFTRACE}{NDFTRACE} command.  Application
\htmlref{WCSADD}{WCSADD} allows new Frames to be added to the WCS
component (positions in the new Frame must be related to the
corresponding positions in an existing Frame by one of several different
supported types of \xref{Mapping}{sun210}{Mapping}).
\htmlref{WCSREMOVE}{WCSREMOVE} allows Frames to be removed from the WCS
component.

Two common additional Frame \att{Domain} options are SKY and SPECTRUM.  SKY
is reserved to refer to two-dimensional Frames that describe
celestial longitude and latitude (known as
\xref{`SkyFrames'}{sun210}{SkyFrame}). SPECTRUM is reserved to refer
to one-dimensional Frames that describe position within a spectrum
(known as \xref{`SpecFrames'}{sun210}{SpecFrame}).  These sub-classes of
Frame have additional attributes indicating the particular celestial
or spectral co-ordinate system in use (\emph{e.g.} equatorial,
galactic, ecliptic, wavelength, frequency, radio velocity, optical
velocity, \emph{etc.}), and any required qualifying parameters
(equinox, rest frequency, standard of rest, \emph{etc.}).  Instances of
these Frames can be added to an NDF by importing suitable FITS headers
using \htmlref{FITSDIN}{FITSDIN},
\htmlref{FITSIN}{FITSIN} or the CONVERT package (see \xref{SUN/55}{sun55}{}).
Alternatively, a SkyFrame can be created directly by specifying the pixel
co-ordinates of stars with known celestial co-ordinates (see
\htmlref{SETSKY}{SETSKY} and \GAIAref\ ).

\subsection{\xlabel{se_curframe}FrameSets, and the Current Frame\label{se:curframe}}

Any co-ordinate Frames described in the \htmlref{WCS component}{apndf:wcs},
together with the standard GRID, FRACTION, PIXEL, and AXIS
\xref{Frames}{sun210}{Frame}, form a collection of inter-related Frames called a
\emph{\xref{FrameSet}{sun210}{FrameSet}}.  A FrameSet contains
descriptions of each Frame, plus recipes (called
\emph{\xref{Mappings}{sun210}{Mapping}}) describing how to convert
positions from one Frame to another.\footnote{A Mapping need not
necessarily be defined in both directions.  For instance, a Mapping
that goes from a three-dimensional Frame to a two-dimensional Frame
(maybe by simply throwing away one of the axis values) may not be
defined in the other direction.  KAPPA applications will report an
error if a Mapping is not defined in the required direction.  Another
common example of a uni-directional Mapping is that between the GRID
and AXIS Frames in cases where the AXIS structures are non-monotonic.}

Frames within a FrameSet are distinguished by their attribute values,
primarily their \att{Domain} value.  In addition, each Frame has an integer
\emph{Frame index}.  These indices are (in general) allocated sequentially
in chronological order as Frames are added into the FrameSet.
Applications that require the user to specify a Frame will usually allow
the Frame to be specified either by \att{Domain} name, or by index.  Frame indices
can be examined using \htmlref{NDFTRACE}{NDFTRACE} (see below).

One Frame within the FrameSet of an NDF is nominated as the \emph{current
co-ordinate Frame}.  When-ever an application reports positions within
an NDF (for instance, when annotating plot axes), the positions reported
will refer to the current co-ordinate Frame.  Likewise, when-ever an
application requests a position from the user (for instance, the position
to be placed at the centred of a displayed image), it will expect the
position to be given within the current co-ordinate Frame of the NDF,

The contents of the WCS component of an an NDF can be examined using
NDFTRACE.  By default, this just shows the number of
co-ordinates Frames defined within the NDF (including the four standard
ones), and the \att{Title} and \att{Domain} of the current co-ordinate Frame.  More
detail can be obtained using the keywords FULLWCS and FULLFRAME.  The
first causes the \att{Title}, index and \att{Domain} of all Frames to be displayed,
together with the index of the current Frame.  The second causes a more
complete description of each Frame to be displayed, including axis
labels, units, celestial co-ordinate system (for
\xref{SkyFrames}{sun210}{SkyFrame}), \emph{etc.}

The current co-ordinate Frame can also be examined using
\htmlref{WCSFRAME}{WCSFRAME}.  This is an important application because
it also allows you to change the current co-ordinate Frame in the NDF.
For instance:

\begin{terminalv}
     % wcsframe m31 pixel
\end{terminalv}

will make the PIXEL Frame the current co-ordinate Frame in the NDF
\verb+m31+.  After this, all references to positions within the NDF
\verb+m31+ will refer to pixel co-ordinates.  You can also change the
current co-ordinate system using \htmlref{WCSATTRIB}{WCSATTRIB}:


\begin{terminalv}
     % wcsattrib myspectrum set system Frequency
     % wcsattrib myspectrum set unit GHz
     % wcsattrib myspectrum set sor Helio
\end{terminalv}

The above changes the values of the `System', `Unit' and `SOR'
attributes of the current Frame in the NDF \verb+myspectrum+ so that
it represents heliocentric frequency in units of \verb+GHz+ (this assumes
that the current Frame in \verb+myspectrum+ is a
\xref{SpecFrame}{sun210}{SpecFrame}~ since only a
SpecFrame supports these particular attribute values).

\subsection{\xlabel{se_resdoms}Reserved Domain Names\label{se:resdoms}}

New co-ordinates Frames can be created and added into an NDF using
\htmlref{WCSADD}{WCSADD}.  When you create a new Frame you should give it
a meaningful \att{Domain} name which indicates the sort of
co-ordinates that it
represents.  You are free to choose any name you like (white space is removed,
and lower case characters are converted to upper case before using the
supplied \att{Domain} string), but you should usually avoid the following reserved
\att{Domain} names:

\begin{description}
\item [GRID] --- Reserved to represent grid co-ordinates (in units of
pixels).
\item [PIXEL] --- Reserved to represent pixel co-ordinates (in units of
pixels).
\item [AXIS] --- Reserved to represent AXIS co-ordinates (in arbitrary units).
\item [FRACTION] --- Reserved to represent normalised pixel/grid
co-ordinates from zero to one (unitless).
\item [SKY] --- Reserved to represent celestial longitude and latitude (in
any suitable system, but always stored internally in units of radians).
\item [SPECTRUM] --- Reserved to represent position within an
electro-magnetic spectrum (various spectral systems and units are
supported).
\item [TIME] --- Reserved to represent moments in time (various
time-scales and units are supported).
\item [DSBSPECTRUM] --- Reserved to represent position within an
dual-sideband electro-magnetic spectrum.
\item [GRAPHICS] --- Reserved to represent positions on a graphics device
in units of millimetres, measured from the bottom-left corner
of the device.
\item [BASEPIC] --- Reserved to represent positions on a graphics device
in normalised units, in which the shorter axis of the graphics device has
length 1.0.
\item [NDC] --- Reserved to represent positions on a graphics device
in normalised units, in which the both axes of the graphics device have
length 1.0.
\item [CURPIC] --- Reserved to represent positions in normalised units within
each individual graphics database picture.  The shorter axis of each
picture has length 1.0 in this \xref{Frame}{sun210}{Frame}.
\end{description}

When two Frames are joined together to form a compound Frame describing a
co-ordinate space of higher dimensionality, the default Domain name for
the compound Frame is formed from the individual Domain names, separated
by a hyphen. For instance, the WCS FrameSet associated with a spectral
cube will often contain a (compound) Frame with the Domain name
``SKY-SPECTRUM''. Thus you should also avoid Domain names that contain
a hyphen, particularly if they also contain any of the reserved Domain
names listed above.


\subsection{\xlabel{se_scs}Specifying a Co-ordinate Frame\label{se:scs}}

Several applications (\htmlref{WCSFRAME}{WCSFRAME},
\htmlref{CURSOR}{CURSOR}, \htmlref{LISTMAKE}{LISTMAKE}, \emph{etc.}) have
parameters that are used to select a co-ordinate \xref{Frame}{sun210}{Frame}.
These parameters are usually called FRAME.  When selecting a Frame from an
existing \xref{FrameSet}{sun210}{FrameSet}~ (\emph{e.g.} read from the \htmlref{WCS
component}{apndf:wcs}~ of an NDF),
the Frame may be specified in one of the following ways:

\begin{itemize}

\item As an integer Frame index within the specified FrameSet starting at
1 for the first Frame.

\item As a \htmlref{`Domain'}{se:resdoms} name (\emph{e.g.} PIXEL, AXIS, SKY),
selected from those available in the FrameSet.

\item As a \emph{Sky Co-ordinate System} (SCS) specification.  This notation
has been inherited from the IRAS90 package (see \xref{SUN/163}{sun163{}}),
and can be used to indicate a specific celestial co-ordinate system.
Simply specifying the \att{Domain} name SKY will select any
\xref{SkyFrame}{sun210}{SkyFrame}~ that is present, irrespective of the
particular celestial co-ordinate system that
the SkyFrame represents.  If you want to request a particular celestial
co-ordinate system (\emph{e.g.} galactic, equatorial, ecliptic, \emph{etc.}), then use an
SCS specification.  If the requested system is not present, but a related
SkyFrame can be found for a different system, then the SkyFrame will be
modified so that it represents the requested system (the Mappings between
the SkyFrame and the other Frames in the FrameSet will also be modified
appropriately).

An SCS specification is made up of two parts; a co-ordinate system name,
followed by an optional epoch giving the reference equinox.  Any of the
three system names {\latex{\small} EQUATORIAL, ECLIPTIC} and {\small GALACTIC}
can be used.  Case is insignificant, and abbreviations may be given.

Ecliptic and equatorial co-ordinates are referred to the mean equinox of a
given epoch.  This epoch is specified by appending it to the end of the
name of the sky co-ordinate system, in parentheses; for instance {\small
EQUATORIAL(1983.5)} (only the four most significant decimal places are
used).  The epoch may be preceded by a single character, B or J,
indicating if the epoch is a Besselian epoch (B) or a Julian epoch (J).
If this character is missing (as in the above example), then the epoch is
assumed to be Besselian if it less than 1984.0, and Julian otherwise.  If no
equinox is specified in this way, then a default of B1950.0 is used.

If a Julian epoch is used to specify the reference equinox for
an equatorial co-ordinate system, then the equatorial co-ordinates
are assumed to be in the IAU 1976, FK5, Fricke system.  If the
equinox is specified using a Besselian epoch, then the
co-ordinates are assumed to be in the FK4, Bessel-Newcomb system.

When a Frame is specified using an SCS specification, it will usually
also be necessary to specify the epoch at which the positions were
determined.  This will be done using the separate Parameter EPOCH.  This
epoch is required because some celestial co-ordinate systems are
non-inertial and rotate slowly with respect to other celestial
co-ordinate systems, introducing fictitious proper motions.  Knowing the
date at which the positions were determined allows the effect of this
fictitious proper motion to be eliminated when converting between
different systems.

\end{itemize}

When specifying a new Frame (rather than selecting an existing Frame
from a FrameSet), you can either give a \att{Domain} name, or an SCS
specification, but you cannot give a Frame index.  Any string may be used
as a \att{Domain} name, and you will usually be required to specify
the number of axes in the Frame.  The exception to this is if you specify
one of the \att{Domain} names SKY, GRAPHICS, NDC, CURPIC or BASEPIC, in which
case a two-dimensional Frame is always created.

The nature of the current Frame can also be changed using
\htmlref{WCSATTRIB}{WCSATTRIB}, which allows new values to be assigned
to specified \htmlref{Frame attributes}{ap:frmatt}.  For instance,
assigning a new value to the \att{System} attribute will change the
co-ordinate system used to describe positions within the domain
covered by the Frame.  The main attributes relevant to Frames are
described \slhyperref{here}{in Appendix~}{}{ap:frmatt}.

\subsection{Propagation of WCS Information}

\KAPPA\ applications that create an output NDF on the
basis of a given input NDF usually copy the contents of the
\htmlref{WCS component}{apndf:wcs}~
from the input to the output, modifying it as appropriate to take account
of any linear mapping of pixel co-ordinates introducing by the
application.  The following are exceptions to this rule:

\begin{itemize}

\item Applications in which positions in the output NDF are
not straight-forwardly related to corresponding positions in the input NDF
(\emph{e.g.} \htmlref{ELPROF}{ELPROF},
\htmlref{CHAIN}{CHAIN}, \htmlref{RESHAPE}{RESHAPE}).

\item Applications in which the output NDF is not a direct representation
of the input NDF (\emph{e.g.} \htmlref{FOURIER}{FOURIER}).

\end{itemize}

The output NDFs produced by such applications will contain no
\htmlref{WCS component}{apndf:wcs}~
(but they will still have the four \htmlref{standard Frames}{se:resdoms}---GRID,
FRACTION, PIXEL and AXIS---albeit the AXIS Frame will probably just describe
\htmlref{pixel co-ordinates}{se:pixgrd} since these applications will not
usually copy the AXIS structures either).

The application \htmlref{WCSCOPY}{WCSCOPY} can be used to copy the WCS
component from one NDF to another, optionally introducing a linear
transformation of pixel co-ordinates in the process.  This can be used to
add WCS information back into an NDF that has been stripped of WCS
information by one of the above applications.

\subsection{Reading WCS Information Stored in Other Forms}

When a \KAPPA\ application requires WCS information, it looks
first in the WCS component of the NDF.  If no WCS component is defined
within the NDF, then it will attempt to obtain WCS information from two
other locations, in the following order:

\begin{enumerate}

\item If an IRAS90 astrometry structure (see \xref{SUN/163}{sun163}{}) is
present within the NDF, then WCS information will be read from it, and
added to the \xref{FrameSet}{sun210}{FrameSet}~ holding the GRID, FRACTION,
PIXEL, and AXIS Frames.  IRAS90 astrometry structures have been used by
several applications packages in the past for storing astrometry
information.  \KAPPA\ contains the \htmlref{SETSKY}{SETSKY}
application which can be used to create such a structure, either by
supplying the required numerical parameters (pixel size, image
orientation, \emph{etc.}), or by doing a least-squares fit to a set of
pixel co-ordinates with corresponding celestial co-ordinates.

\item If no IRAS90 astrometry structure can be found, an attempt is made
to read WCS information from the FITS header cards in the FITS extension.

\end{enumerate}

Note, \KAPPA\ applications always \emph{write} WCS
information in the form of a standard \htmlref{WCS component}{apndf:wcs}.  You should remember
that astrometry information stored within an IRAS90 or FITS extension will
not be corrected to take account of geometric manipulation produced by
applications such as \htmlref{ROTATE}{ROTATE}, \htmlref{COMPAVE}{COMPAVE},
\emph{etc.} Use of such applications will render IRAS90 and FITS astrometry
information incorrect.  For this reason, applications always warn the user
if astrometry information is being read from an IRAS90 or FITS extension.
These extensions can be deleted if necessary, using the \htmlref{ERASE}{ERASE}
command.  For instance:

\begin{terminalv}
   % erase m31.more.iras90
   % erase m31.more.fits
\end{terminalv}

will erase the IRAS90 and FITS extensions from the NDF \verb+m31+.

\subsection{Using SETSKY to Add a Celestial Co-ordinate Frame to an NDF}

As mentioned in the previous section, the \htmlref{SETSKY}{SETSKY}
application stores astrometry information within an NDF in the form of
either a WCS component or an IRAS90 astrometry structure.

To use SETSKY, you need to know the celestial co-ordinates at a set of
points within the image.  You may be able to find these by comparing
your image with other images, such as those available from the
Digitised Sky Survey, which already have astrometry information
associated with them.  You create a text file holding the pixel and
celestial co-ordinates at a single position on each line.  For instance,
if you have five known RA/DEC (B1950) positions in your image, the file
may look like:

\begin{terminalv}

 0 49 05.9,   42 25 30,    32,    266
 0 48 31.7,   40 03 36,    39,    29
 0 37 03.0,   40 04 48,    258,   31
 0 36 54.6,   42 26 47,    257,   268
 0 45 47.7,   41 54 03,    93,    213

\end{terminalv}

The first column gives the RA values (hours, minutes and seconds), the
second gives the DEC values (degrees, arcminutes and arcseconds), the
third gives the pixel X co-ordinates, and the fourth gives the pixel Y
co-ordinates.

If this file is called \verb+pos.dat+, then the following command can be
used to create a WCS component:

\begin{terminalv}
   % setsky m31 ^pos.dat coords='equ(b1950)' epoch=1998.0 projtype=gno

     Trying GNOMONIC projection...

     These parameter values give an RMS positional error of 0.3647723 pixels ...
       Projection type                      : GNOMONIC
       Sky co-ordinates of reference point  : 0h 40m 31.29s, 42d 37m 53.15s
       Image co-ordinates of reference point: (190.3287,285.795)
       Pixel dimensions                     : (36.04451,36.00686) arcsecs
       Position angle of image Y axis       : 359d 38m 35.77s
       Tilt of celestial sphere             : 0d 0m 0.00s

\end{terminalv}

Note the up-arrow character (\verb+"^"+) before the file name
(\verb+pos.dat+).  This tells SETSKY that the string is a file name.  A
gnomonic (or \emph{tangent plane}) projection was requested using the
PROJTYPE parameter.  If no projection type is specified then SETSKY will
try four different projections (gnomonic, Aitoff, Lambert equivalent
cylindrical, and orthographic) in turn, and choose the one that gives
the smallest RMS position error.

An alternative way to add a celestial co-ordinate
\xref{Frame}{sun210}{Frame}~ to an NDF is to
use the facilities of GAIA (see \xref{SUN/214}{sun214}{}).  This provides
much more sophisticated facilities.

\subsection{Converting an AXIS structure to a SpecFrame}

For many years, the calibration of spectral axes has been recorded in the
form of AXIS structures within the NDF.  As mentioned above, each pixel
axis in an NDF has an associated AXIS structure, which is a one-dimensional
array containing an element for each pixel on the associated axis.
The value of each element gives the `axis' value for the pixel.

Spectral axis calibration can now also be recorded in the form of a
\xref{`SpecFrame'}{sun210}{SpecFrame}~ within the WCS component.  A
SpecFrame is a \xref{Frame}{sun210}{Frame}~ that can
describe spectral axes in many different forms (wavelength, frequency,
various forms of velocity, \emph{etc.}), with many different units, and
measured in various rest frames.  A SpecFrame `knows' how to convert
between all these different forms.  Let's say you have two spectra---in one
the current co-ordinate Frame is a SpecFrame representing frequency in
GHz as measured in the rest frame of the telescope (\emph{i.e.} `topocentric
frequency'), and in the other the current co-ordinate Frame is a SpecFrame
representing radio velocity in km/s measured in the kinematic Local
Standard of Rest.  You want to overlay plots of these two spectra for
comparison purposes, so you display the first using
\htmlref{LINPLOT}{LINPLOT}.  This produces a plot in which the \textit{x} axis is
frequency measured in GHz.  You then display the second spectrum, again
using LINPLOT but this time specifying \verb+clear=no+ on the command
line in order to prevent the previous plot from being erased.  The
SpecFrame representing radio velocity in the second spectrum
automatically adjusts itself to represent the same system as the
currently displayed plot (topocentric frequency in this example), so the
plots can be compared directly.  The conversion includes the effects of
the Doppler shift caused by the differing standards of rest used by the
two spectra.

So the question arises; ``How can I add a SpecFrame to my existing NDFs
that only have an AXIS structure?''.  This is simple to do using the
\htmlref{WCSADD}{WCSADD} command.  Let's assume you have a one-dimensional
NDF called \file{fred} containing an AXIS structure holding the frequency
at the centre of each pixel in units of GHz.  The following command will add
an appropriate SpecFrame to the WCS component of the NDF (and will also
make it the \emph{current} Frame):

\begin{terminalv}
    % wcsadd fred frame=axis maptype=unit frmtype=spec domain=SPECTRUM \
             attrs="'System=freq,Unit=GHz'"
\end{terminalv}

The FRMTYPE parameter indicates that a SpecFrame should be created.  The
ATTRS parameter gives the attribute values to be assigned to the
SpecFrame (note the quotes to protect the string from interpretation by
the UNIX shell).  The DOMAIN parameter is given the default domain for a
SpecFrame (and should usually not be changed).  The MAPTYPE and FRAME
parameters indicate that this new SpecFrame should be connected to the
existing AXIS Frame using a UnitMap---\emph{i.e.} the frequency values held
within the AXIS structure are identical to the frequency values described
by the SpecFrame.

The above setting for the ATTRS parameter gives the bare minimum of
information---there are several other items of information that
\emph{could} have been given.  For instance, the above command does not
indicate the standard of rest to which the frequency values refer.
Neither does it indicate the source position, date of observation,
or the observers geographical position (all of which may be needed to
enable conversion between different standards of rest).  In general you
should specify as much information as you can.  To do this, you can either
include the extra information in the ATTRS parameter value above, or you
can add the information later using \htmlref{WCSATTRIB}{WCSATTRIB}.  For
example:

\begin{terminalv}
    % wcsattrib fred remap=no set Stdofrest topo
    % wcsattrib fred remap=no set RefRA '"10:12:24.2"'
    % wcsattrib fred remap=no set RefDec '"-32:10:14"'
    % wcsattrib fred remap=no set Epoch '"2003-10-2 12:13:00"'
    % wcsattrib fred remap=no set ObsLat '"N19:49:33"'
    % wcsattrib fred remap=no set ObsLon '"W155:28:47"'
\end{terminalv}

These commands set new values for named attributes within the current
Frame of NDF \file{fred}.  These attributes are described
\slhyperref{here}{in Appendix~}{}{ap:frmatt}.  The inclusion of \texttt{"remap=no"}
is important: it tells the command to change the Frame attribute
\emph{ without} changing the \xref{Mappings}{sun210}{Mapping}~ between
Frames accordingly.  If the default value of \texttt{"remap=yes"} were
used, the Mappings that connect the
\xref{SpecFrame}{sun210}{SpecFrame}~
to the other Frames within the WCS \xref{FrameSet}{sun210}{FrameSet}~
would be modified in order to ensure that the position of each pixel
is unchanged.  The default mode (\texttt{"remap=yes"}) should be used if
you already have a fully and correctly specified SpecFrame within the
WCS component which you want to change to describe a different system.
For instance, if you have a SpecFrame describing frequency at each
pixel, and you want to change it so that it describes the
corresponding wavelength at each pixel, then doing:

\begin{terminalv}
    % wcsattrib fred set System wave
\end{terminalv}

will modify the Mapping between the pixel Frame and the SpecFrame using
the relationship ``wavelength = speed of light/frequency''.

However, if you have a partially or incorrectly specified SpecFrame you
should usually use \texttt{"remap=no"}.  For instance, of the above
SpecFrame, which gives the frequency at each pixel, was discovered to be
incorrect in that the AXIS values were actually wavelength values and
not frequency at all, then you would want to \emph{correct} the WCS
component by changing the System attribute of the SpecFrame from
\texttt{"freq"} to \texttt{"wave"}.  In this situation you want to leave the Mapping
from the pixel Frame to the spectral Frame unchanged since the Mapping
already gives the correct \emph{wavelength} value (previously, but
erroneously, thought to be a frequency value).  So here you would do:

\begin{terminalv}
    % wcsattrib fred remap=no set System wave
\end{terminalv}


\subsection{Specifying Attributes for sub-Frames within Compound Frames}

Let's say you have a spectral cube in which the WCS axes are wavelength,
RA and Dec.  In this case, the current co-ordinate Frame will actually be
a `compound' Frame containing two `component' Frames; a one-dimensional
spectral Frame and a two-dimensional sky Frame.  If we consider an attribute
such as \att{System} (which all classes of Frame have), we now seem to have
three different Systems to consider; the \att{System} value for the component
spectral Frame, the \att{System} value for the component sky Frame and the
\att{System} value for the compound Frame as a whole.  So if we want to set the
spectral system to \texttt{"optical velocity"}, and celestial system to
\texttt{"Galactic"} what do we do?  The answer is to include the index of an axis
within the attribute name.  If WCS axis 1 is the spectral axis, 2 is the
RA axis and 3 is the Dec. axis, then we would do the following:


\begin{terminalv}
    % wcsattrib fred set "System(1)" vopt
    % wcsattrib fred set "System(2)" galactic
\end{terminalv}

In fact, we could have used \texttt{"System(3)"} in place of \texttt{"System(2)"} since
Axes 2 and 3 are both contained within the same sky Frame and so setting
the \att{System} attribute for either one will have the same effect.  We could
now examine the attributes as follows:

\begin{terminalv}
    % wcsattrib fred get "System(1)"
    % wcsattrib fred get "System(2)"
\end{terminalv}

These commands will display \texttt{vopt} and \texttt{Galactic}, as expected.  Note,
the following command:

\begin{terminalv}
    % wcsattrib fred get "System"
\end{terminalv}

will display \texttt{Compound} since it is displaying the \att{System} value of the
compound Frame \emph{as a whole} (because no axis index was included).

To continue this example, the \xref{SpecFrame}{sun210}{SpecFrame}~
class (which represents spectral axes) has an attribute called
\htmlattref{StdOfRest}{StdOfRest}~ (standard of rest).  This attribute is
specific to the SpecFrame class and is not recognised by other classes
of Frames.  If we do:

\begin{terminalv}
    % wcsattrib fred set "StdOfRest" LSRK
\end{terminalv}

we will get an error saying the attribute name is unknown.  This is because
compound Frames do not have a \att{StdOfRest} attribute.  If we want to set the
standard of rest, we must indicate the index of the spectral axis as follows:

\begin{terminalv}
    % wcsattrib fred set "StdOfRest(1)" LSRK
\end{terminalv}

Likewise, if we wanted to set the \htmlattref{Equinox}{Equinox}~
attribute of the sky Frame, we could say:

\begin{terminalv}
    % wcsattrib fred set "Equinox(2)" "J2003.5"
\end{terminalv}


\newpage
\section{\xlabel{se_interaction}Interaction Mode\label{se:interaction}}

We have seen the different co-ordinate systems \KAPPA\ uses.
Now we address how the applications obtain co-ordinate information
itself.  Applications often permit a variety of mechanisms for obtaining those
co-ordinates.  Typical possibilities are as follows.
\begin{description}
\item [Catalogue] --- In this mode the application reads a file
containing a positions list.  This can either be a FITS binary
table, or a text file in \STLref format, and can contain WCS information
allowing positions within the catalogue to be aligned with other data.
Positions lists can be created by several applications such as
\htmlref{CURSOR}{CURSOR}, \htmlref{LISTMAKE}{LISTMAKE},
\htmlref{CENTROID}{CENTROID}, \emph{etc.}
\item [Cursor] --- This mode utilises the cursor of the current graphics
device.  For this to work the array must already be displayed as an
image, or a contour plot, or line plot (provided the application handles
one-dimensional data), and the picture is stored in the graphics database.
\item [Interface] --- This mode obtains co-ordinates from
the parameter system, usually in response to prompting.
\item [File] --- In this mode the application reads a text file
containing a list of co-ordinates in free format, one object per record.
There may be commentary lines in the file beginning with \texttt{\#} or
\texttt{{!}}.  The format and syntax of the files are \emph{ad hoc}, and are
described in the application documentation.
\end{description}

Applications that permit these options have a parameter, called
MODE, by which you can control how positional data are to be acquired.
It would be tedious to have to specify a mode for each application,
therefore \KAPPA\ has a
\htmlref{global parameter}{se:parglobals}---the interaction
mode---to which each application's interaction-mode parameter is
defaulted.  The global value remains in force until you change it by
assigning an application's interaction mode on the command line.  The
following examples shows the effect of the global parameter.  For
compactness \htmlref{GLOBALS}{GLOBALS} will merely show the
interaction mode.

First we display an image on the \texttt{xw} windows device.

\begin{terminalv}
     ICL> gdset xw
     ICL> display $KAPPA_DIR/ccdframec mode=pe \
     Data will be scaled from 2366.001 to 2614.864.
     ICL> globals
     The current interaction mode is      : <undefined>
\end{terminalv}
Now we obtain the centroids of a couple of stellar/galaxian images via
each of the interaction modes.  First in cursor mode.  Note that
\htmlref{CENTROID}{CENTROID} obtains the name of the input NDF from the
graphics database in this mode.  If you need to preview which NDF is
going to be selected use the \htmlref{PICIN}{PICIN} command.

\begin{terminalv}
     ICL> centroid mode=c
     Current picture has name: DATA, comment: KAPPA_DISPLAY.
     Using /star/bin/kappa/ccdframec as the input NDF.

     To select a point press the left button on the mouse or trackerball.
     To exit press the right button.
     Use the cursor to select one point.

     Input guess position was     86.23534, 295.0848
     Output centroid position is  86.41057, 295.1141

     Use the cursor to select one point.

     Input guess position was     73.32529, 318.9757
     Output centroid position is  72.76437, 318.9484

     Use the cursor to select one point.
\end{terminalv}
If we look at the global parameters again, indeed we see that it has
become cursor mode.

Now we'll see the effect of changing the mode parameter.  Note that
unless it is undefined or the application does not support the current
mode, you must change the mode on the command line.  First we shall
prompt for the co-ordinates.  A null ends the loop.

\begin{terminalv}
     ICL> centroid mode=i
     NDF - Array to be analysed /@/star/bin/kappa/ccdframec/ >
     INIT - Guess at co-ordinates of star-like feature /108.8,403.5/ > 86,295

     Input guess position was     86, 295
     Output centroid position is  86.41057, 295.1141

     INIT - Guess at co-ordinates of star-like feature /86,295/ > 73.3,319

     Input guess position was     73.3, 319
     Output centroid position is  72.76437, 318.9484

     INIT - Guess at co-ordinates of star-like feature /73.3,319/ > !
\end{terminalv}
Finally, we can create a text file called \file{starlist.dat} and run
CENTROID in file mode.

\begin{terminalv}
     ICL> cat > starlist.dat
     86 295
     73 320
     CTRL/D
     ICL> centroid mode=f
     COIN - File of initial positions /@centroid.lis/ > starlist.dat
     NDF - Array to be analysed /@$KAPPA_DIR/ccdframec/ >

     Input guess position was     86, 295
     Output centroid position is  86.41057, 295.1141

     Input guess position was     73, 320
     Output centroid position is  72.76437, 318.9484
\end{terminalv}
Such co-ordinate files can also be created interactively with images by
\htmlref{CURSOR}{CURSOR}.

\newpage
\section{\xlabel{se_coltab}Graphics Device Colour Table and Palette
\label{se:coltab}}

The \PGPLOTref\ graphics package, which is used by \KAPPA,
draws images and line graphics using a set of `pens'.  The number of pens
available is limited to 256, even on modern 16- or 24-bit graphics devices which
nominally have `millions' of colours.  On older 8-bit graphics devices, the
number of available pens may be fewer than 256 if any other applications
have `grabbed' colours for their own use.

Each \PGPLOT\  pen draws in a single colour, but you can choose what that
colour will be for each pen.  This allocation of colours to pens is called
the \PGPLOT\  `colour table'.  Each pen has a corresponding integer
\emph{index} within the table.  On 8-bit graphics devices you can allocate
any arbitrary combination of red, green and blue to a pen (each colour is
specified as an `intensity' in the range zero to one).  On 16- and 24-bit
devices you can only allocate one of the `millions' of colours known to
the graphics device.  For instance, on 16 bit devices it is common to
allocate 5 bits each to the red and blue intensity and the remaining 6
bits to the green intensity.  This means that red and blue can only be set
accurate to 1 part in 32 on these devices, which may result in colours
not being exactly as you want them.

On an 8-bit device, any changes thatyou make to the colour table are
immediately reflected in the visual appearance of the display.  For
instance, if you set Pen~1 to red, then draw something using Pen~1, it
will appear red on the screen.  If you then change Pen~1 to blue, the
previously drawn graphics will immediately change colour without you
needing to re-draw them.  This is \emph{not} usually true for 16- and 24-bit
devices.  That is, changing pen colours will usually have no effect on
previously drawn graphics.  In order to see the effects of the changed pen
colour, you will need to re-display the graphics.

In many image procesing and visualisation systems the full colour table
is used to draw images.  This has the disadvantage that if you want to
annotate images with captions or axes, plot coloured borders about
images, plot graphs \emph{etc}, yet simultaneously display images with
certain colour tables, there may be conflict of interests.  For instance,
a linear grey-scale colour table's first few pens will be almost
black.  By default, these same pens, particularly Pen~1, are used by
the graphics system for line graphics, thus any plots will be invisible.
If you reset colour Pen~1 to white, the appearance of your image
alters.  Whenever you alter the colour table to enhance the look of your
image, it will affect the line graphics.

To circumvent this dilemma, \KAPPA\ reserves a portion of the
colour table, called the {\em palette}, that is unaffected by changes
to the rest of the colour table.  It is shown schematically below.  The
palette currently contains a fixed 16 pens.   \textit{n} is the total
number of pens.  In \KAPPA\ the remainder of the pens
is called the {\em colour table}.  It is easy to confuse this use of the
term `colour table', with the \PGPLOT\  colour table described above.  To
sumarize again, in \KAPPA\ the `colour table' is that
part of the \PGPLOT\  colour table that has not been reserved for annotation
(\emph{i.e.} the whole colour table minus the first 16 pens which form the
annotation \emph{palette}).  The context should usually make it obvious
which understanding of the phrase `colour table' is being used.

% Save current unit length for pictures.
\newlength{\oldunit}
\setlength{\oldunit}{\unitlength}
\setlength{\unitlength}{3pt}

\begin{center}
\begin{picture}(136,28)
% Thick outline.
\thicklines

% Draw the palette outline.  0.3 fudge to make lines match!
\put(2,15){\framebox(32,7.7){}}

% Draw the colour-table outline broken with dots to indicate an
% arbitrary length and so three frameboxes cannot be drawn.
\put(34,15){\line(1,0){50}}
\put(34,23){\line(1,0){50}}
\multiput(84.8,15)(2,0){4}{{\huge .}}
\multiput(84.8,22.4)(2,0){4}{{\huge .}}
\put(94,15){\line(1,0){40}}
\put(94,23){\line(1,0){40}}
\put(134,15){\line(0,1){8}}

% Switch to thin lines to mark the individual colour indices.
\thinlines

% Mark the colour indices as vertical lines.
\multiput(4,15)(2,0){15}{\line(0,1){8}}
\multiput(36,15)(2,0){25}{\line(0,1){8}}
\multiput(94,15)(2,0){20}{\line(0,1){8}}

% Label the colour indices.
\put(2,24){0}
\put(30,24){15}
\put(34,24){16}
\put(131,24){$N\!\!-\!\!1$}

% Make braces to indicate the two parts.
\put(18,10){\makebox(0,0)[c]{$\underbrace{\rule{31mm}{0mm}}$}}
\put(84,10){\makebox(0,0)[c]{$\underbrace{\rule{99mm}{0mm}}$}}

% Write the captions
\put(2,4){\makebox[32mm][c]{{\large Palette}}}
\put(34,4){\makebox[100mm][c]{{\large Colour Table}}}
\end{picture}
\end{center}

% Restore default unit length.
\setlength{\unitlength}{\oldunit}

\subsection{\xlabel{se_lookuptables}Lookup Tables\label{se:lookuptables}}

A list of colours to be allocated to each pen in the colour table is
called a \emph{lookup table}.  Lookup tables comprise a series of red,
green and blue (RGB) intensities, each normalised to 1.0; they may be
stored in NDFs---indeed some are provided with \KAPPA---or
be coded within applications.

A lookup table may be transferred into the display's colour table.
However, the number of pens in the colour table is usually not the
same as the number of colours in the lookup table and so a simple
substitution is not possible.  Therefore, \KAPPA\ squeezes
or stretches the lookup table to make it fit in the available number of
colour-table pens.  Normally, linear interpolation between adjacent
lookup-table entries defines the resultant colour, though you can select
a nearest-neighbour algorithm.  The latter is suited to lookup tables
with sharp boundaries between contrasting colours, {\it{e.g.}}\ a series of
coloured blocks, and the former to smoothly varying lookup tables where
there are no obvious discontinuities, {\it{e.g.}}\ spectrum-like.

Let's have a few examples.

\begin{terminalv}
     % lutheat
     % lutramps
     % lutread pastel
     % lutable li ex sawtooth nn
     % lutsave pirated
\end{terminalv}
\htmlref{LUTHEAT}{LUTHEAT} loads the standard `heat' lookup table into
the colour table using linear interpolation, whilst
\htmlref{LUTRAMPS}{LUTRAMPS} loads the standard coloured ramps using
the nearest neighbours in the lookup table.
\htmlref{LUTREAD}{LUTREAD} reads the lookup table stored in the
DATA\_ARRAY of the NDF called pastel and maps it on to the colour table
via linear interpolation.  In the fourth example the lookup table in
NDF sawtooth is mapped on to the colour table via a linear
nearest-neighbour method.  \texttt{ex} tells \htmlref{LUTABLE}{LUTABLE} to
read an external file.  In the final example
\htmlref{LUTSAVE}{LUTSAVE} saves the current colour table into a
lookup-table NDF called pirated.  LUTSAVE is quite useful as you can
steal other people's attractive colour tables that they've carelessly
left in the display's memory!  It does not matter should the display
not have a palette, since

\begin{terminalv}
     ICL> lutsave pirated full
\end{terminalv}
will save the full set of pens (including the first 16) to the NDF.

\subsection{\xlabel{se_mancoltab}Manipulating Colour Tables
\label{se:mancoltab}}
\htmlref{LUTEDIT}{LUTEDIT} provides a complete graphical-user-interface which
allows colour tables to be created or modified in many different ways.

\subsection{\xlabel{se_creluts}Creating Lookup Tables\label{se:creluts}}

\subsubsection{From a Text File}
You can make a text file of the RGB intensities and use
\htmlref{TRANDAT}{TRANDAT} to create the \NDFref{NDF}, or manipulate
the colour table and then save it in a lookup-table NDF.  If you
choose the second option remember that all RGB intensities must lie in
the range 0.0--1.0, where 1.0 is the maximum intensity; and that equal
red, green, and blue intensities yields a shade of grey.  So for
example if you want a six equal blocks of red, blue, yellow, pink,
sienna and turquoise you could create the text file \file{col6.dat}
with contents

\begin{terminalv}
     # Red, blue, yellow, pink, sienna, and turquoise LUT
     1.0 0.0 0.0
     0.0 0.0 1.0
     1.0 1.0 0.0
     0.9 0.56 0.56
     0.56 0.42 0.14
     0.68 0.92 0.92
\end{terminalv}
and then run TRANDAT to make the NDF called collut6.

\begin{terminalv}
     % trandat col6 collut6 shape='[3,6]' auto
\end{terminalv}

\subsubsection{\xlabel{se_runlutedit}Running LUTEDIT\label{se:runlutedit}}

There is an interactive task called \htmlref{LUTEDIT}{LUTEDIT} for
creating and editing lookup tables.  The LUTEDIT command fires up a
complete graphical-user-interface.  This includes its own help system
via a "short help" window at the bottom of the interface which describes
the control currently under the pointer, and also via the usual "Help"
button at the right-hand end of the menu bar.


\subsection{\xlabel{se_palette}Palette\label{se:palette}}

There are four commands for controlling the
\htmlref{palette}{se:coltab}\latex{ (Section~\ref{se:coltab})}, all
beginning PAL (in addition, the colours in the current palette can be
listed using \htmlref{GDSTATE}{GDSTATE}). If you inherit the graphics
device after a non-\KAPPA\ user or after a device reset, you will probably
have to reset the palette.  You can do this either by loading the default
palette---black, white, the primary then secondary colours, and eight
equally spaced grey levels---with the command \htmlref{PALDEF}{PALDEF};
or load a palette you've created yourself via \htmlref{PALREAD}{PALREAD}.
You modify the palette by changing individual colours within it using
\htmlref{PALENTRY}{PALENTRY}.  The colour
specification can be a \htmlref{named colour}{ap:colset}
\latex{(see Appendix~\ref{ap:colset} for a list)}, or RGB intensities.
For example,

\begin{terminalv}
     % palentry 1 Skyblue
     % palentry 14 [1.0,1.0,0.3]
\end{terminalv}
would make palette index 1 sky blue and index 14 a pale yellow.  Once
you have a palette you like, save it in an NDF with
\htmlref{PALSAVE}{PALSAVE}.

Palette entry 0 is the background colour.  By choosing a palette colour
equal to the background colour, features may be `erased'.

\subsection{Persistence of Palettes and Colour Tables}
\PGPLOT\  re-initializes the palette and colour table each time it is
started up, wiping out the colours you had previously selected with such
care! For this reason, \KAPPA\ keeps a copy of the `current' palette and
colour table in a special file.  Each time a graphics application is run,
the current colour table and palette are read back from this file, and
used to reset the pen colours in the \PGPLOT\  colour table before any
drawing is performed.  Some application change the colour table or
palette; \htmlref{PALENTRY}{PALENTRY}, \htmlref{LUTABLE}{LUTABLE},
\emph{etc.}.  When such applications terminates, they write the modified
colour table or palette back to the file so that it will be used by
subsequent graphics applications.

Separate palettes and colour tables are maintained for each known
graphics device, so running \htmlref{LUTGREY}{LUTGREY} on an xwindows
device will have no effect on the colour table used for PostScript
devices (for instance).  The palettes for all known devices are stored in
a file called \file{kappa\_palette.sdf} located within your ADAM directory
(usually \file{\$HOME/adam}).  The colour tables for all known devices are stored
in a file called \file{kappa\_lut.sdf} located within the same directory.

\newpage
\section{\xlabel{se_masking}Masking, Bad Values, and Quality\label{se:masking}}

{\em Masking\/} is the process by which you can exclude portions of
your data from data processing or analysis.  Suppose that you are
doing surface photometry of a bright galaxy, part of the data reduction
is to measure the background contribution around the galaxy and to
subtract it.  You usually want to avoid inclusion of light from the galaxy
in your estimation of the background.  A convenient method for doing
this is to mask the galaxy during the background fitting.

There are two techniques used for masking.  One employs special {\em
bad\/} values (also known as {\em magic\/} or {\em invalid\/} values).
These appear within the data or variance arrays in place of the actual
values, and indicate that the pixel is to be ignored or is undefined.
They are destructive\footnote{That is, the special bad value replaces the
original data values, and so the original data values are lost.} and so
some people don't like them, but you can always mask your data into a
new, temporary NDF.  With a little care, bad values are quite effective
and they are used throughout \KAPPA.  By its nature, a bad
value can only indicate a logical, two-state condition about a data
element---it is either good or bad---and so this technique is sometimes
called {\em flagging}.

In contrast, the second technique, uses a quality array.  This permits
many more attributes or qualities of the data to be associated with each
pixel.  In the current implementation there may be up to 255 integer
values, or 8 single-bit logical flags.  Thus quality can be regarded as
offering 8 logical masks extending over the data or variance arrays, and
can signify the presence or absence of a particular property if the bit
has value 1 or 0 respectively.  An application of quality to satellite
data might include the detector used to measure the value, some indicator
of the time each pixel was observed, was the observation made within the
Earth's radiation belts, and whether or not the pixel contains a reseau
mark.  By selecting only those data with the appropriate quality values,
you process only the data with the desired properties.  This can be very
powerful.  However, it does have the drawback of having to store at least
an extra byte per pixel in your NDF.

The two methods are {\em not\/} mutually exclusive; the NDF permits
their simultaneous use in a dataset.

Now we'll look at both of these techniques in detail and demonstrating
the relevant \KAPPA\ tasks.

\subsection{\xlabel{se_badmasking}Bad-pixel Masking\label{se:badmasking}}

Bad pixels are flagged with the Starlink standard values (see
Section~5 of \xref{SUN/39}{sun39}{}), which for
\htmlref{\_REAL}{ap:HDStypes} is the most-negative value possible.

In addition to tasks that routinely create bad values in the output
value is undefined, \KAPPA\ offers many applications for
flagging pixels with certain properties or locations.

\subsubsection{\xlabel{se_ardwork}Doing it the ARD Way\label{se:ardwork}}

To mask a region or a series of regions within an NDF, you can create
an \xref{ASCII Region Definition}{sun183}{} (ARD) text file.  ARD has
a powerful syntax for combining regions and supplying WCS information,
described fully in \xref{SUN/183}{sun183}{}.  An ARD file comprises
\xref{keywords}{sun183}{ARDKeywords} that define a region, such as
\texttt{RECT} to specify a rectangular box;
\xref{operators}{sun183}{ARDOperators} that enable regions to be
combined, for instance \texttt{.AND.} that will form the intersection of
two regions; and \xref{statements}{sun183}{ARDStatements} to define
the \htmlref{world co-ordinate system}{se:wcsuse} and dimensionality.
\latex{For further details see the three sections called
\emph{Regions}, \emph{Operators}, and \emph{Statements} in SUN/183.}

Here is an example of the creation of an ARD file.

\begin{terminalv}
     % cat myard.ard
     COFRAME(PIXEL)
     PIXEL( 23.5, -17.2 )
     ELLIPSE( 75.2, 296.6, 33, 16, 78 )
     POLYGON( 109.5, 114.5, 122.5, 131.5, 199.5, 124.5 )
     CIRCLE( 10, 10, 40 ) .AND. .NOT. CIRCLE( 10, 10, 30 )
     COFRAME(SKY,SYSTEM=FK5,EQUINOX=2000)
     CIRCLE( 10:09:12.2, -45:12:13, ::40 )
     CTRL/D
\end{terminalv}

The COFRAME statements indicate the co-ordinate system in which
subsequent positions are supplied.  Its first argument is the
\htmlref{domain}{se:resdoms}.  Here the first \texttt{COFRAME(PIXEL)}
refers to \htmlref{pixel co-ordinates}{fi:pixco}.  Note that these are
not the same as \htmlref{pixel indices}{fi:pixind}, as they are
displaced by $-$0.5 with respect to pixel indices.  The second \texttt{
COFRAME} selects a SKY domain using the FK5 system, so that regular
equatorial co-ordinates may be supplied as arguments to subsequent
keywords.  Other possible values for \htmlref{System}{System}~ include
\texttt{ECLIPTIC} and \texttt{GALACTIC}.  If no COFRAME or WCS statement is
present, the default co-ordinate system is pixel co-ordinates
transformed by any COEFFS, OFFSET, TWIST, STRETCH, SCALE statements.
Note that the \htmlref{ARDMASK}{ARDMASK} application, used to mask
data with an ARD file, has a DEFPIX parameter where you can choose
whether the default co-ordinates are pixel or those of the
\htmlref{current WCS Frame}{se:curframe}. if there is no COFRAME or
WCS statement in your ARD file.  Still you are recommended to supply a
COFRAME or WCS statement in your ARD files to avoid accidentally
selecting the wrong regions.

In this example, the regions are: the single pixel at co-ordinates
(23.5,~-17.2); an ellipse centred at (75.2,~296.6) with semi-major
axis of 33 pixels and semi-minor axis of 16 pixels, at orientation
78\dgs\ clockwise from the \textit{x} axis; a triangle with vertices
at pixel indices (110,~115), (123,~132), (200,~125); an annulus
centred on pixel co-ordinates (10.0,~10.0) between radius 30 and 40
pixels; and a circle centred on RA 10:09:12.2, and DEC -45:12:13 of
radius 40 arcseconds.

Operators combine regions using a Fortran-like logical expression,
where each keyword acts like a logical operand acted upon by the
adjoinning operators.  Statements are ignored in such logical
expressions.  There is an implicit \texttt{.OR.} operator for every
keyword on a new line.  Thus pixels that lie in any of the above
regions (the union) are selected.

Where a keyword (such as CIRCLE, RECT, POLYGON) defines an area or
volume a pixel is deemed to be part of that region if its
\emph{centre} lies on or within the boundary of the region.  For
regions of zero volume (such as keywords PIXEL, LINE, COLUMN), the
pixel is regarded as part of the region when the locus of the region
passes through that pixel.  So for example, a PIXEL region will be the
pixel emcompassing the supplied co-ordinates; and for a LINE, the
selected pixels are all that intersect with the line's locus.


Here are some more examples of ARD files.

\begin{terminalv}
     COFRAME(GRID)
     ROTBOX( 12, 15, 20, 10, 36.3 ) .AND. .NOT. ( COLUMN( 13 ) .OR. ROW( 8 ) )
\end{terminalv}

Now the co-ordinates are \htmlref{Grid co-ordinates}{fi:gridco}.  This
selects all the pixels with a rotated box except those in thirteenth
column or eighth row.  Note the use of parentheses to adjust or
clarify the precedence.  The box is centred on grid pixel (12,~15) has
sides of length 20 and 10 pixels.  The first side of the box---the one
with length 20---is at an angle of  \ang{36.3}  measured anticlockwise
from the X axis.
\medskip

\begin{terminalv}
     DIMENSION(3)
     CIRCLE( 10.3, 21.6, 32.9, 10.4 )
     LINE( 1.1, 2.2, 3.3, 4.4, 5.5, 6.6 )
\end{terminalv}

This defines a sphere centred at pixel co-ordinates (10.3,~21.6,~32.9)
with radius 10.4 pixels, and a line from (1.1,~2.2,~3.3) to
(4.4,~5.5,~6.6).
\medskip

\begin{terminalv}
     CIRCLE( 10.3, 21.6, 32.9, 10.4 )
\end{terminalv}

This defines a sphere centred at pixel co-ordinates (10.3,~21.6,~32.9)
with radius 10.4 pixels.
\medskip

\begin{terminalv}
     .NOT. ELLIPSE( 75.2, 296.6, 33, 16, 78 )
\end{terminalv}

This selects the whole array except for the ellipse defined as before.
Something like this might be useful for excluding a galaxy image before
fitting to the background around the galaxy.
\medskip


There are more details and further ARD facilities described in
\xref{SUN/183}{sun183}{}.  If you do not wish to read SUN/183, you'll be
relieved to learn that there are shortcuts for two-dimensional
data\ldots

The first is is provided by \xref{GAIA}{sun214}{} by its \texttt{Image
Analysis} $\rightarrow$ \xref{\texttt{Image Regions\ldots}
tool}{gaia}{regions}.  Here you can select region types and
interactively adjust the region locations and shapes, and then record
the selected regions in an ARD file.  However, it does not provide the
boolean operators other than .OR. to combine a series of regions, or
use co-ordinates other than pixel.

\KAPPA\ offers its own interactive graphical tool for generating ARD files.
To use \htmlref{ARDGEN}{ARDGEN} you must first display your data on a device
with a cursor, such as an X-terminal.  \htmlref{DISPLAY}{DISPLAY} with
a grey-scale lookup table is probably best for doing that.  The grey
lets you see the coloured overlays clearly.  The following example
assumes that the current \htmlref{co-ordinate Frame}{se:domains}~ in
the NDF is PIXEL (\emph{i.e.} pixel co-ordinates).  Consequently all
positions are shown below in pixel co-ordinates.  If the current
co-ordinate Frame in the NDF was not PIXEL but (say) SKY, then ARDGEN
would produce positions in SKY co-ordinates.  The ARD file generated
by ARDGEN always contains a description of the co-ordinate system in
which positions are specified, allowing later applications to
interpret them correctly, and convert them (if necessary) into other
co-ordinate systems.

\begin{terminalv}
     % ardgen demo.ard
     Current picture has name: DATA, comment: KAPPA_DISPLAY.
     SHAPE - Region shape /'CIRCLE'/ >
\end{terminalv}
At this point you can select a shape.  Enter \texttt{?} to get the
list.  Once you've selected a shape you'll receive instructions.

\begin{terminalv}
     SHAPE - Region shape /'COLUMN'/ > ellipse

 Region type is "ELLIPSE".  Identify the centre, then one end of the semi-major
 axis, and finally one other point on the ellipse.

    To select a position press the space bar or left mouse button
    To exit press "." or the right mouse button
\end{terminalv}
Once you have defined one ellipse, you can define another or exit to
the OPTION prompt.  In addition to keyboard 1, pressing the right-hand
mouse button has the same effect.  Thus in the example, the new shape is
a rotated box.

\begin{terminalv}
     Region completed. Identify another 'ELLIPSE' region...
     OPTION - Next operation to perform /'SHAPE'/ > shape
     SHAPE - Region shape /'ELLIPSE'/ > rotbox


     Region type is "ROTBOX". Identify the two end points of any edge and then give
     a point on the opposite edge.
     Region completed. Identify another 'ROTBOX' region...
\end{terminalv}

If you make a mistake, use the `Undo' option.  Alternatively, enter \texttt{
List} at the OPTION prompt to see a list of the regions.  Note the
`Region Index' of the region(s) you wish to remove, and select the \texttt{
Delete} option.  At the REGION prompt, give a list of the regions you
want to remove.  If you change your mind, enter \texttt{{!}} at the prompt for
Parameter REGIONS, and no regions are deleted.

Now suppose you want to combine or invert regions in some way, you
supply \texttt{Combine} at the OPTION prompt.  So suppose we have
created the following regions in \file{\$KAPPA\_DIR/ccdframe}.

\begin{terminalv}
       Region          Region Description
       Index

         1   -  ELLIPSE( 174.1, 234.4, 82.2, -43.5, 65.64783 )
         2   -  ELLIPSE( 168.1, 209.1, 29.4, -19.7, 9.441798 )
         3   -  ELLIPSE( 42.2, 244.1, 13, -10.3, 111.8452 )
         4   -  ROTBOX( 40.5, 219.2, 63.8, 38.3, 37.24281 )
         5   -  RECT( 141.5, 1.4, 143.9, 358.8 )
         6   -  POLYGON( 229.8, 247.7,
                         233.4, 247.7,
                         233.4, 258.6,
                         231, 267,
                         229.8, 265.8,
                         228.6, 256.2 )
\end{terminalv}

We want to form the region inside the first ellipse but not inside
the second.  This done in two stages.  First we invert the second
ellipse, meaning that pixels are included if they are not inside
this ellipse, by combining with the \texttt{NOT} operator.

\begin{terminalv}
     OPTION - Next operation to perform /'SHAPE'/ > comb
     OPERATOR - How to combine the regions /'AND'/ > not
     OPERANDS - Indices of regions to combine or invert /6/ > 2
\end{terminalv}

This removes the original Region~2, decrements the region numbers of
the other regions following 2 by one, so that Region~3 becomes 2, 4
becomes 3, and so on.  A new Region~7 is the inverted ellipse.  The
renumbering makes it worth listing the regions before combining
regions.  The second stage is to combine it with Region~1, using the
\texttt{AND} operator.  This includes pixels if they are in both regions.
In this example, that means all the pixels outside the second ellipse
but which lie within the first.

\begin{terminalv}
    OPTION - Next operation to perform /'SHAPE'/ > com
    OPERATOR - How to combine the regions /'AND'/ >
    OPERANDS - Indices of regions to combine or invert /[6,7]/ > 1,6
\end{terminalv}

Here is another example of combination.  This creates a region for
pixels are included provided they are in one of two regions, but not
in both.  Here we apply the \texttt{.XOR.} operator to the small ellipse
and the first rotated box.

\begin{terminalv}
     OPTION - Next operation to perform /'SHAPE'/ > comb
     OPERATOR - How to combine the regions /'AND'/ > xor
     OPERANDS - Indices of regions to combine or invert /[4,5]/ > 1,2
\end{terminalv}

Here is the final set of regions.

\begin{terminalv}
     OPTION - Next operation to perform /'SHAPE'/ > list


       Region          Region Description
       Index

         1   -  RECT( 141.5, 1.4, 143.9, 358.8 )
         2   -  POLYGON( 229.8, 247.7,
                         233.4, 247.7,
                         233.4, 258.6,
                         231, 267,
                         229.8, 265.8,
                         228.6, 256.2 )

         3   -  ( ELLIPSE( 174.1, 234.4, 82.2, -43.5, 65.64783 )
                  .AND.
                  ( .NOT. ELLIPSE( 168.1, 209.1, 29.4, -19.7, 9.441798 ) ) )

         4   -  ( ELLIPSE( 42.2, 244.1, 13, -10.3, 111.8452 )
                  .XOR.
                  ROTBOX( 40.5, 219.2, 63.8, 38.3, 37.24281 ) )
\end{terminalv}

Once you are done, enter \texttt{"Exit"} at the OPTION prompt, and the
ARD file is created.  \texttt{"Quit"} also leaves the programme, but
the ARD file is not made.

Having created the ARD file it is straightforward to generate a
masked image with \htmlref{ARDMASK}{ARDMASK}\footnote{You can also plot the
outline of the selected regions on top of a display image using
\htmlref{ARDPLOT}{ARDPLOT}.}:

\begin{terminalv}
     % ardmask $KAPPA_DIR/ccdframec demo.ard ardccdmask
\end{terminalv}
\medskip

   \begin{figure}[hbt]
   \begin{center}
   \includegraphics[clip,height=118mm]{sun95_ardwork}
   \caption{Masking of \file{\$KAPPA\_DIR/ccdframec}.  To the left shows
   the original ARDMASK regions, and to the right shows the final
   masked regions after some have been combined.}
   \label{fi:ardwork}
   \end{center}
   \end{figure}

Figure~\latexhtml{\ref{fi:ardwork}}{13} shows the image with the
original regions outlined to the left.  Note only the section
(:270,~:360) is displayed.  To see where you have masked, use
\htmlref{DISPLAY}{DISPLAY}, which lets you define a colour for bad
pixels using the BADCOL parameter.

\begin{terminalv}
     % display ardccdmask badcol=red \\
\end{terminalv}
To the right of Figure~
\latexhtml{\ref{fi:ardwork}}{13} is the final masked image.

\subsubsection{\xlabel{se_badsegment}SEGMENT and
ZAPLIN\label{se:badsegment}}

\htmlref{SEGMENT}{SEGMENT} is ostensibly for copying polygonal regions
from one NDF to another.  You may also use SEGMENT to copy bad pixels
into the polygonal regions by giving the null value for one of the two
input NDFs.  For instance,

\begin{terminalv}
     % segment in1=! in2=$KAPPA_DIR/ccdframec out=ccdmask
\end{terminalv}
NDF ccdmask will have bad values inside the polygons, whereas

\begin{terminalv}
     % segment in2=! in1=$KAPPA_DIR/ccdframec out=ccdmask
\end{terminalv}
the pixels exterior to the polygons are flagged.  SEGMENT lets you
define the polygon vertices interactively, like in
\htmlref{ARDGEN}{ARDGEN}, but you can also use text files, or respond
to prompting.

\htmlref{ZAPLIN}{ZAPLIN} also has an option to fill in rectangular
areas when Parameter ZAPTYPE has value \texttt{Bad}.

\subsubsection{\xlabel{se_badspecial}Special Filters for Inserting Bad Values
\label{se:badspecial}}

There are applications that mask pixels if their values meet certain
criteria.

\htmlref{SETMAGIC}{SETMAGIC} flags those pixels with a nominated
value.  It is most useful during conversion of imported data whose
data system uses \htmlref{bad-pixel}{se:masking} values different from
Starlink's.

\htmlref{FFCLEAN}{FFCLEAN} removes defects smaller than a nominated
size from an image or vector \NDFref{NDF}.  It flags those pixels that
deviate from a smoothed version of the NDF by more than some number of
standard deviations from the local mean.

\htmlref{ERRCLIP}{ERRCLIP} flags pixels that have errors larger than
some supplied limit or signal-to-noise ratios below a threshold.  The
errors come from the \htmlref{VARIANCE}{apndf:variance}~ component of the NDF.
Thus you can exclude unreliable data from analysis.

\htmlref{THRESH}{THRESH} flags pixels that have data values within or
outside some specified range.

% The filtering applications have a WLIM parameter

\subsection{\xlabel{se_qualitymask}Quality Masking\label{se:qualitymask}}

All the \NDFref{NDF} tasks in \KAPPA\ use quality yet there is
no obvious sign in individual applications how particular values of
\htmlref{quality}{se:masking} are selected.  What gives?  The meanings
attached to the quality bits will inevitably be quite specific for
specialist software packages, but \KAPPA\ tasks aim to be
general purpose.  To circumvent this conflict there is an NDF
component called the {\em bad-bits mask} that forms part of the
quality information.  Like a QUALITY value, the bad-bits mask is an
unsigned byte.  Its purpose is to convert the eight quality flags into
a single logical value for each pixel, which can then be processed
just like a bad pixel.

When data are read from the NDF by mapping into memory, the quality of
each pixel is combined with the bad-bits mask; if a result of this
quality masking is \texttt{FALSE}, that pixel is assigned the bad value for
processing.  This does not change the original values stored in the NDF;
it only affects the mapped data.

So how do the quality and bad-bits mask combine to form a logical
value?  They form the bit-wise `AND' and test it for equality for 0.
None the wiser?  Regard each bit in the bad-bits mask as a switch
to activate detection of the corresponding bit in a pixel's quality.
The switch is on if it has value \texttt{1}, and is off if it has value
\texttt{0}. Thus if the pixel is flagged only if one or more of the
eight bits has both quality and the corresponding bad-bit set to 1.  Here
are some examples:

\begin{center}
\begin{tabular}{lrr}
QUALITY:  & 10000001 & 10000001 \\
Bad-bits: & 01000100 & 01000101 \\
Bits on:  &          &       \^{} \\
Result:   & TRUE     & FALSE \\
\end{tabular}
\end{center}

The application \htmlref{SETBB}{SETBB} allows you to modify the
bad-bits mask in an NDF.  It allows you to specify the bit pattern in
a number of ways including decimal and binary as illustrated below.

\begin{terminalv}
     % setbb RO950124 5
     % setbb RO950124 b101
\end{terminalv}
These both set the bad-bits mask to 00000101 for the NDF RO950124.
SETBB also allows you to combine an existing NDF bad-bits mask with
another mask using the operators AND and OR.  OR lets you switch on
additional bits without affecting those already on; AND lets you turn
off selected bits leaving the rest unchanged.

\begin{terminalv}
     % setbb RO950124 b00010001 or
     % setbb RO950124 b11101110 and
\end{terminalv}
The first example sets bits 1 and 5 but leaves the other bits of
the mask unaltered, whereas the second switches off the same bits.

Now remembering which bit corresponds to which could be a strain on
the memory.  It would be better if some meaning was attached to each
bit through a name.  There are four general tasks that address this.
\htmlref{SETQUAL}{SETQUAL} sets quality values and names;
\htmlref{SHOWQUAL}{SHOWQUAL} lists the named qualities;
\htmlref{REMQUAL}{REMQUAL} removes named qualities; and
\htmlref{QUALTOBAD}{QUALTOBAD} uses a logical expression
containing the named quality properties to create a copy of your NDF
in which pixels satisfying the quality expression are set bad.
See \slhyperref{``Using Quality Names''}{Section~}{}{se:qnames} for more
information about using these tasks.  Once you have defined quality
names, you can set the bad-bits mask with SETBB to mask pixels
with those named quality attributes.

\begin{terminalv}
     % setbb RO950124 spike
     % setbb RO950124 '"spike,back"'
\end{terminalv}
The first example might set the bad-bits mask to exclude spike
artefacts.  The second could mask both spikes and background pixels.
Thus it might be used to select the spectral lines not affected by
noise spikes in a spectral cube. Other logical combinations are
possible using the AND and OR operators.


\subsection{\xlabel{se_removebad}Removing bad pixels\label{se:removebad}}

Sometimes having bad pixels present in your data is a nuisance, say
because some application outside of \KAPPA\ does not
recognise them, or you want to integrate the flux of a source.
\KAPPA\ offers a number of options for removing bad
values.  Which of these is appropriate depends on the reason why you
want to remove the bad pixels.

First you could replace the bad values with some other reasonable value,
such as zero.

\begin{terminalv}
     % nomagic old new 0 comp=all
\end{terminalv}
Here dataset new is the same as dataset old except that any bad value
in the data or variance array has now become zero.

If you wanted some representative value used based upon neighbouring
pixels, use the \htmlref{GLITCH}{GLITCH} command.

\begin{terminalv}
     % glitch old new mode=bad
\end{terminalv}
This replaces the bad values in the data and variance arrays with the median
of the eight neighbouring pixels.  This works
fine for isolated bad pixels but not for large blocks.  If your data are
generally flat, large areas can be replaced using the
\htmlref{FILLBAD}{FILLBAD} task.

\begin{terminalv}
     % fillbad old new size=4
\end{terminalv}
The value of Parameter SIZE should be about half the diameter of the
largest region of bad pixels.  Both the data array and variance arrays
are filled.

You may replace individual pixels or rectangular sections using
\htmlref{CHPIX}{CHPIX}.

\begin{terminalv}
     % chpix old new
     SECTION - Section to be set to a constant /'55,123'/ >
     NEWVAL - New value for the section /'60'/ >
     SECTION - Section to be set to a constant /'1:30,-10:24'/ >
     NEWVAL - New value for the section /'-1'/ >
     SECTION - Section to be set to a constant /'1:30,-10:24'/ > !
\end{terminalv}
This replaces pixel (55,~123) with value 60, and the region from
(1,~$-$10) to (30,~24) with $-$1.  The final \texttt{{!}} ends the loop of
replacements.  If you supply NEWVAL on the command line, only one
replacement occurs.

It is also possible to paste other datasets where your bad values lie
with the \htmlref{PASTE}{PASTE} and \htmlref{SEGMENT}{SEGMENT} tasks.

\begin{terminalv}
     % paste old fudge"(10:20,29:30)" out=new
\end{terminalv}
The dataset old is a copy of dataset new, except in the 22-pixel region
(10,~29) to (20,~30), where the values originate from the fudge dataset.

\newpage
\section{\xlabel{se_qnames}Using Quality Names\label{se:qnames}}

\subsection{Introduction}

As described in \slhyperref{``Masking, Bad Values, and Quality''}{Section~}{}
{se:masking}, an NDF may optionally contain a component called
\htmlref{QUALITY}{apndf:quality}.  If this component exists, it will be an
array with the same bounds as the main DATA array.  Each element in the
QUALITY array can be used to store several flags that are associated
with the corresponding element in the DATA array.  These flags may be used
to indicate that the DATA value holds some specified property.  For
instance, one of the flags may be used to indicate if the corresponding
DATA values are saturated, another may be used to indicate if the DATA
value lies within a background area, and so on.

You are free to use the flags in whatever way seems most suited to the
particular process being performed.  You can set (or reset) any of the
flags within any sub-region of the NDF using application
\htmlref{SETQUAL}{SETQUAL}.  Each of the flags is referred to by a
{\em Quality Name} specified by the user.  Names that reflect the nature of
the quality should be used, \emph{e.g.} the quality name \texttt{SATURATED} could be
used to flag saturated data values.  These quality names get stored within
the NDF and can be used to refer to the quality flag when running later
applications.  The terminology adopted here is that an element of the DATA
array `holds' the quality \texttt{SATURATED} (for instance) if the flag
that is associated with the quality name \texttt{SATURATED} is set for the
corresponding element within the QUALITY array.

The number of quality names that can be stored within an NDF is limited,
and therefore it may become necessary to remove quality names that are
no longer needed to make room for new ones.  The applications
\htmlref{SHOWQUAL}{SHOWQUAL} and \htmlref{REMQUAL}{REMQUAL} allow you
to do this. SHOWQUAL displays a list of the quality names currently
defined within an NDF, and REMQUAL removes specified quality
names from an NDF.

Some applications have a Parameter QEXP, which may be used to
specify that the application is only to use data values that hold a
specified selection of qualities.  As an example, when running
\htmlref{QUALTOBAD}{QUALTOBAD} you could (for instance) specify a value of
\texttt{BACKGROUND} for the QEXP parameter.  This means that only
those data values for which the flag associated with the quality name
\texttt{BACKGROUND} is set, are to be set bad.  The quality name \texttt{BACKGROUND}
must previously have been defined and assigned to the appropriate data
values using application SETQUAL.

The specification of the data values to be used by an application
can be more complex than this, and can depend on several
qualities combined together using `Boolean' operators. For
instance, assigning the value \texttt{.NOT. (SOURCE\_A .OR. SOURCE\_B) }
would cause the application to use only those data values that
hold neither of the qualities \texttt{SOURCE\_A} and/or \texttt{SOURCE\_B}.  These
sort of strings are known as {\em quality expressions}.

\subsection{Quality Names}

Quality names are names by which the user refers to particular flags
stored  in the \htmlref{QUALITY}{apndf:quality}~ component of the NDF.  They
must not be longer than 15 characters.  Leading blanks are ignored, and
they are always stored in upper case, even if they are supplied by the
user in lower case.  Embedded blanks are considered to  be significant.
Quality names must not contain any fullstops (\texttt{{"."}}), and there
are three reserved names that cannot be used; these are \texttt{ANY},
\texttt{IRQ\_BAD\_SLOT} and \texttt{IRQ\_FREE\_SLOT}.

\subsection{Quality Expressions}

Quality names may be combined together using Boolean operators
into complex {\em quality expressions}.  The quality expression is
evaluated at each element within the NDF by substituting a value
of true or false for each quality name used in the expression,
depending on whether or not that element holds the specified
quality. Elements are used if the quality expression evaluates
to a true value.  Boolean operators are delimited on each side by
a period (\emph{e.g.} \texttt{.AND.} ).  The operands on which these operators
act must be either a quality name (which must be defined within
the NDF), or one of the Boolean constants \texttt{.TRUE.} and \texttt{.FALSE.}
Parentheses can be used to nest expressions.

Quality expressions can be up to 254 characters long, and must
not contain more than 40 symbols (Boolean operators, constants,
or quality names). Some attempts are made to simplify a quality
expression to reduce the run time needed to evaluate the
expression for every data value.

The precedence of the Boolean  operators decreases in the following
order; \texttt{.NOT.}, \texttt{.AND.}, \texttt{.OR.}, \texttt{.XOR.}, \texttt{.EQV.}
 (the final two have equal
precedence). In an expression such as \texttt{( A .XOR. B .EQV. C .XOR. D )} in
which all operators have equal precedence, the evaluation proceeds from
left to right, \emph{i.e.} the expression is evaluated as \texttt{( ( ( A .XOR. B )
.EQV. C ) .XOR. D )}. If there is any doubt about the order in which an
expression will be evaluated, parentheses should be used to ensure the
required order of evaluation.

\begin{description}
\item [.NOT.] -
The expression \texttt{(.NOT.A)} is true only if A is false.

\item [.AND.] -
The expression \texttt{(A.AND.B)} is true only if A and B are both true.

\item [.OR.] -
The expression \texttt{(A.OR.B)} is true if either (or both) of A or B are
true.

\item [.XOR.] -
The expression \texttt{(A.XOR.B)} is true if either A is true and B is
false, or A is false and B is true.

\item [.EQV.] -
The expression \texttt{(A.EQV.B)} is true if either A is true and B is
true, or A is false and B is false.
\end{description}

\newpage
\section{\xlabel{se_multinvoc}Processing Groups of Data Files\label{se:multinvoc}}

When a \KAPPA\ application requests an input or output data
file (either an NDF or a positions list), you may optionally give a
group of several data files rather than just one. In this case, the
application is automatically re-run until all the supplied data files have
been processed. For instance, in the following command:

\begin{terminalv}
    % display in="../a*" mode=perc accept
\end{terminalv}

a group of NDFs (all those beginning with ``a'' in the parent directory)
are assigned to the IN parameter, and the \htmlref{DISPLAY}{DISPLAY} command
is automatically re-run to display each NDF in the fashion of a (rather
slow!) movie.

Another example:

\begin{terminalv}
    % stats ndf=^files
\end{terminalv}

will display the pixel statistics of all NDFs listed within the text file
\verb+files+.  The ``\verb+^+'' character indicates that the following
string (\verb+files+) is not the name of an NDF, but the name of a text
file from which NDF names should be read.

\begin{terminalv}
    % wcsframe '"image_a,image_b,image_c"' sky
\end{terminalv}

This will set the current co-ordinate Frame for the three NDFs
\verb+image_a+, \verb+image_b+ and \verb+image_c+ so that celestial sky
co-ordinates are used to refer to positions within the NDFs, if possible.

\begin{terminalv}
    % cursor outcat="'first,^list_names'"
\end{terminalv}

This will run \htmlref{CURSOR}{CURSOR} several times, allowing you
to select display positions using a cursor.  On the first invocation,
the selected positions are written to a positions list stored in file
\verb+first.FIT+.  Positions selected on subsequent invocations are
written to positions lists with names read from the text file
\verb+list_names+. Finally, if this CURSOR command is followed by:

\begin{terminalv}
    % listshow accept
\end{terminalv}

\htmlref{LISTSHOW}{LISTSHOW} will display the contents of all the
catalogues created previously by CURSOR.

If an application has more than one NDF or positions list parameter, each
parameter should be given the same number of values (\emph{i.e.} data
files).  A warning is issued if any parameter is given too many values, but
processing continues normally until the smallest group is exhausted. For
instance, the following example adds NDF \verb+a1+ to \verb+a2+, and
\verb+b1+ to \verb+b2+, putting the results in \verb+a3+ and \verb+b3+:

\begin{terminalv}
    % add in1="'a1,b1'" in2="'a2,b2'" out="'a3,b3'"
\end{terminalv}

If (say) an extra NDF had been specified for Parameter IN2, the
application would have been invoked twice to process the first two
pairs, and then a warning message would have been displayed saying that
too many NDFs were specified for Parameter IN2.

There is a special case in which this rule does not apply. If only a
\emph{single} value is given for an data file parameter, the same value
is used repeatedly on all invocations of the application. So, for
instance, if a single NDF had been given for Parameter IN2 in the above
ADD example, the application would have again been run twice, using the
same NDF for Parameter IN2 on each invocation.

The OUT parameter in the above example could alternatively have been
specified as \verb+out="'*|1|3|'"+.  Here, the asterisk (*) represents the
base-names, \verb+a1+ and \verb+b1+, of the NDFs supplied for the first
NDF parameter to be accessed (IN1).  The following string, \verb+|1|3|+,
means {\em replace all occurrences of 1 with 3}, thus giving the final NDF
names \verb+a3+ and \verb+b3+.

\subsection{Applications that Process Groups of NDFs}

The majority of applications with NDF parameters use each NDF parameter
to access a \emph{single} NDF, and supplying more than one NDF will
result in the application being re-run as described above.  The
application then accesses a single NDF on each invocation. Some
applications, however, have NDF parameters that are explicitly described
in the reference section of this document as being associated with a
\emph{group} of NDFs.  An example is WCSALIGN that has a Parameter IN
to read a \emph{group} of NDFs that are to be aligned with each
other.  Such applications process all the specified NDFs in a single
invocation.  For the purposes of the multiple invocation scheme described
above, such parameters are not considered to be `NDF' parameters, and
will not cause the application to be re-run.

\subsection{What about the other Parameters?}

When an application is re-run to process multiple data files, all the
parameters not associated with NDFs or positions lists retain their
values from one invocation to the next.  So, for instance, the assignment
for the MODE parameter in the earlier \htmlref{DISPLAY}{DISPLAY} example is
retained and used for all subsequent invocations of the application, you
are not prompted for a new value each time the application runs.

This is usually what you want, but beware that there \emph{are} times
when this behaviour may trip you up.  Sometimes an application may prompt
for a new parameter value while in the middle of processing a group of
NDFs.  This can occur for instance, if the initial value you supplied on
the first invocation is inappropriate for the NDF currently being
processed.  For instance, supposing you use \htmlref{WCSFRAME}{WCSFRAME}
to set the current co-ordinate Frame to SKY for a group of NDFs.  To do
this, you would set the FRAME parameter to SKY either on the command line
or when prompted during the first invocation.  This value would be
retained for subsequent invocations, but what happens if one of the NDFs
does not have a SKY Frame defined in its \htmlref{WCS component}{apndf:wcs}?
Not surprisingly,
you get an error message identifying the NDF, and you are asked to supply a
new value for FRAME.  You could, for instance supply \texttt{PIXEL} as the new
value.  This changes the current value of the FRAME parameter to PIXEL,
and this value will consequently be used for any remaining NDFs.

If you do not specify a value for a parameter, the default value used by
the first invocation will be re-used for all subsequent invocations. Note
that the default value for some parameters (for instance the CENTRE
parameter of the DISPLAY command) is the null value \texttt{{!}}.  This is
usually interpreted as a request for the application to find an
appropriate value itself for the parameter.  In these cases, the parameter
\emph{value} is \texttt{{!}} and is re-used on all invocations, resulting in the
application finding and using a potentially different value on each
invocation. So, for instance, the above DISPLAY example will find and use
an appropriate CENTRE value for each displayed image.  If you want to use
the same CENTRE value for all images you should specify it explicitly on
the command line, for instance:

\begin{terminalv}
    % display in="../a*" mode=perc centre="'12:00:00 -32:00:00'" accept
\end{terminalv}

\subsection{Output Parameters}

If you tried out the examples at the start of this section, you may be
wondering what happened about the output parameters for STATS.  The
\htmlref{STATS}{STATS} application writes the various statistics it
calculates to lots of output parameters, which can be used by subsequent
applications.  If an application is re-run several times to process
different data files then the values left in the application's output
parameters will be the values created on the \emph{last} invocation of the
application.

\subsection{What Happens if an Error Occurs?}

If an application fails to execute successfully, the error report will be
displayed, and then cancelled.\footnote{This is called `flushing' the error.}
This means that any remaining NDFs will continued to be processed normally.

The exception to this is that if the error is an `abort request'
(caused by supplying two exclamation marks for a parameter), then
the loop exits immediately.  That is, no remaining NDFs are processed.

\subsection{What about Applications that Re-use Parameters?}

Some applications use a single parameter to obtain a series of values from
the user. Examples are the INIT parameter of \htmlref{CENTROID}{CENTROID}
and the OPTION parameter of \htmlref{SETEXT}{SETEXT}. Remembering that
parameters that are not associated with either an NDF or a positions
list retain their values between invocations, it is not surprising that
care is needed when using such application to process groups of NDFs.
For instance, when using CENTROID you supply a null parameter value
(\emph{i.e.} a single exclamation mark \texttt{{!}}) as the final value for
the INIT parameter to indicate that you do not wish to find any more
centroids.  Since parameter values are retained between invocations when
processing groups of NDFs, this null value becomes the first value to be
used by any subsequent invocation.  The next invocation of CENTROID finds
INIT set to a null value, assumes that no more centroids are to be found,
and exits immediately!  The same goes for all subsequent invocations until
the group of NDFs has been exhausted.

The only way (currently) to avoid this behaviour is to specify the INIT
parameter value on the command line.  CENTROID takes this as an indication
that you only want to find a single centroid, and so does not attempt to
get a new value for INIT, thus leaving the supplied value for the next
invocation.  The same value for INIT is thus used by all invocations.  Of
course, this means you can only find a single centroid in each
NDF.\footnote{If this is a problem, you can always put the INIT values
into a file or positions list, using a different value for the MODE
parameter.}

Most applications that re-use one or more parameters during a single
invocation have some similar means of indicating that you do not want to
be prompted for a new value.  For some (like CENTROID), putting the
parameter value on the command line accomplishes this.  Some others (such
as SETEXT) have a LOOP parameter that can be set \texttt{FALSE} to indicate that
parameters should not be accessed more than once.  The reference
documentation for each command should be consulted for details.

\subsection{Introducing a Pause Between Invocations}

Sometimes you may want to slow down the speed at which data files are
processed. For instance, if you display several small images using a
single DISPLAY command, you may want time to examine each image before
moving on to the next.  You can introduce a delay between invocations
by setting the shell environment variable \texttt{KAPPA\_LOOP\_DELAY} to the
required delay time (in units of seconds). For instance, in the
C-shell:

\begin{terminalv}
    % setenv KAPPA_LOOP_DELAY 2.5
\end{terminalv}

causes a delay of 2.5 seconds between invocations of any \KAPPA\
command.  To remove the delay, you should undefined \texttt{KAPPA\_LOOP\_DELAY}.
In C-shell:

\begin{terminalv}
    % unsetenv KAPPA_LOOP_DELAY
\end{terminalv}

\subsection{Reporting the Data Files being Processed}

When processing a single data file, some applications report the name of
the file and some do not. Normally, no extra information is given
when processing groups of data files.  This means that sometimes you get
to see the names of the files as they are processed, and some times
you do not. It just depends on the application.

However, it is often very useful to see the names of the files as they are
processed. For instance, if an error occurs processing one of the files,
it is useful to know which file failed.  If the application doesn't display
this information, then you can force it to by setting the shell
environment variable \texttt{KAPPA\_REPORT\_NAMES} to an arbitrary
value.\footnote{The actual value does not matter.} For instance, in the
C-shell:

\begin{terminalv}
    % setenv KAPPA_REPORT_NAMES 1
\end{terminalv}

This causes the value used for each data file parameter to be displayed in the
form \verb+"parameter = value"+ on each invocation.  To go back to the
normal, quiet reporting scheme, you should undefined \texttt{KAPPA\_REPORT\_NAMES}.
In C-shell:

\begin{terminalv}
    % unsetenv KAPPA_REPORT_NAMES
\end{terminalv}

\subsection{The Syntax for Specifying Groups of Data Files}

The group of NDFs or positions lists to be used for a given parameter is
specified by a \emph{group expression}.  This is also the syntax used to
give groups of plotting attributes when specifying graphics STYLE
parameters.  The group expression syntax is described
\slhyperref{here}{in Section~}{}{se:groups}.

A group of output data files may be specified by modifying the names of a
corresponding set of input data files.  This is easy enough when the
application only has one parameter for input data files, but what happens
if more than one parameter is associated with a group of input data
files?  Which parameter is used to define the group of input data files on
which the names of the output data files are based?  The answer is ``the
first one to be accessed''.  For instance, \htmlref{ADD}{ADD} takes two
input NDFs, adds them together and produces a single output NDF.  When
running ADD, you are prompted first for Parameter IN1, and then for
Parameter IN2, and finally for Parameter OUT.  Thus, if you give the
string \verb+"a_"*+ for OUT, the names of the output NDFs will be
derived from the NDFs supplied for Parameter IN1, because IN1 is accessed
first (\emph{i.e.} prompted for \emph{before} IN2).

Note, when choosing the input parameter on which output data files are
based, no significance is attached to whether the input and output file
types match.  That is, the first input parameter to be accessed if used,
irrespective of whether it is associated with an NDF or a positions list.

A feature that may sometimes be useful is the facility for providing a
shell command in response to a prompt for a group of data files.  To do
this, enclose the command within the usual backward quotes (\verb+`+), as
you would when substituting the output from a command into another shell
command.  The command should generate a set of explicit file names, with
file types.  Note, you will need to escape any characters that are
normally interpreted as part of the syntax of a group expression, such as
\verb+"|"+ or \verb+","+, by preceding them with a backslash
\verb+"\"+.

\subsection{Using non-NDF Data Formats}

In addition to processing Starlink NDF structures, \KAPPA
can also process many non-NDF (`foreign') data files.  This is achieved
through \htmlref{`on-the-fly conversion'}{se:autoconvert} (see
\latex{Section~\ref{se:autoconvert} and} \xref{SUN/55}{sun55}{}).

When this scheme is in use, you need not include explicit file types for
all input file names. If no file type is given, the file with the highest
priority file type amongst all files with the specified base name will be
used.  The priority of a file type is determined by its position within the
list of file types given by \htmlref{\texttt{{NDF\_FORMATS\_IN}}}{se_autoconvert}
environment variable \latex{(see Section~\ref{se:autoconvert})}. File
types near the start of the list have higher priority than those that
follow. Note, native NDF files always have the highest priority and will
be used (if they exist) in preference to all other files types.

\subsection{Disabling Multiple Invocations of Applications}

In certain circumstances, you may possibly want to disable the automatic
re-invocation of \KAPPA\ applications to process groups of
data files.  This can be done by setting the environment variable
\texttt{KAPPA\_LOOP\_DISABLE} to an arbitrary value.\footnote{The actual value does
not matter.} For instance, in the C-shell:

\begin{terminalv}
    % setenv KAPPA_LOOP_DISABLE 1
\end{terminalv}

will cause all NDF and positions list parameters to accept only a single
data file, and each application will be run only once. Note, the extra
facilities for specifying data files provided by the group expression
syntax will not then be available.  To re-enable looping, you should
undefine \texttt{KAPPA\_LOOP\_DISABLE}. In C-shell:

\begin{terminalv}
    % unsetenv KAPPA_LOOP_DISABLE
\end{terminalv}

\newpage
\section{\xlabel{se_datainput}Getting Data into KAPPA\label{se:datainput}}

\KAPPA\ utilises general data structures within an
\HDSref\ container file, with file extension \file{.sdf}.  Most of
the examples in this documentation processing is performed on data in
this \NDFref{NDF} format generated from within \KAPPA.
Generally, you will already have data in `foreign' formats, that is
formats other than the Starlink standard, particularly in the
\FITSref\ (Flexible Image Transport System),
\IRAFref , and \FIGAROref\ DST formats.

\subsection{\xlabel{se_autoconvert}Automatic Conversion
\label{se:autoconvert}}

Although \KAPPA\ tasks do not work directly with `foreign'
formats, they can made to appear that they do.  What happens is that
the format is converted `on-the-fly' to a scratch NDF, which is then
processed by \KAPPA.  If the processing creates an output NDF
or modifies the scratch NDF, this may be back-converted `on-the-fly'
too, and not necessarily to the original data format.  At the end, the
scratch NDF is deleted.  So for example you could have an IRAF image
file, use BLOCK to filter the array, and output the resultant array as
a FITS file.

We must first define the names of the recognised formats and a file
extension associated with each format.  In practice you'll most likely
do this with the \xref{\texttt{convert}}{sun55}{} command, which
creates these definitions for many popular formats.  The
file extension determines in which format a file is written.  There is
an environment variable called \texttt{NDF\_FORMATS\_IN} which defines the
allowed formats in a comma-separated list with the file extensions in
parentheses.  Here is an example.

\begin{terminalv}
     % setenv NDF_FORMATS_IN 'FITS(.fit),IRAF(.imh),FIGARO(.dst)'
\end{terminalv}
Once defined it lets you run \KAPPA\ tasks on FITS,
\IRAF, or \FIGARO\ files, like
\begin{terminalv}
     % stats m51.fit
     % stats m51.dst
\end{terminalv}
would compute the statistics of a FITS file \file{m51.fit}, and then
a \FIGARO\ file \file{m51.dst}.

The environment variable also defines a search order.  Had you entered

\begin{terminalv}
     % stats m51
\end{terminalv}
\htmlref{STATS}{STATS} would first look for an NDF called m51 (stored in file \file{
m51.sdf}). If it could not locate that NDF, STATS would then look for
a file called \file{m51.fit}, and then \file{m51.imh}, and finally \file{
m51.dst}, stopping once a file was found and associating the
appropriate format with it. If none of the files exist, you'll receive
a ``file not found'' error message.

You can still define an
\latexhtml{NDF section (see Section~\ref{se:ndfsect})}{\htmlref{NDF
section}{se:ndfsect}} when you access an existing data file in a foreign
format.  Thus

\begin{terminalv}
     % stats m51.imh"(100:200,200~81)"
\end{terminalv}
would derive the statistics for \textit{x} pixels between 100 and 200, and
\textit{y} pixels 160 to 240 in the \IRAF\ file \file{m51.imh}.

The conversion tasks may be your own for some private format, but
normally they will come from the \CONVERTref\ \normalsize package
\latex{ (SUN/55)}.  If you want
to learn how to add conversions to the standard ones, you should
consult \xref{SSN/20}{ssn20}{}.

There is an environment variable that defines the format of new data
files.  This could be assigned the same value as \texttt{NDF\_FORMATS\_OUT},
though they don't have to be.

\begin{terminalv}
     % setenv NDF_FORMATS_OUT 'FITS(.fit),IRAF(.imh),FIGARO(.dst)'
\end{terminalv}
If you supply the file extension when a \KAPPA\ task creates a
new dataset, and it appears in \texttt{NDF\_FORMATS\_OUT}, you'll get a file in
that format.  So for instance,

\begin{terminalv}
     % ffclean in=m51.dst out=m51_cleaned.dst \\
\end{terminalv}

cleans \file{m51.dst} and stores the result in \file{m51\_cleaned.dst}.
On the other hand, if you only give the dataset name
\begin{terminalv}
     % ffclean in=m51.dst out=m51_cleaned \\
\end{terminalv}
the output dataset would be the first in the \texttt{NDF\_FORMATS\_OUT} list.
Thus if you want to work predominantly in a foreign format, place it
first in the \texttt{NDF\_FORMATS\_IN} and \texttt{NDF\_FORMATS\_OUT} lists.

If you want to create an output NDF, you must insert a full stop at the
head of the list.

\begin{terminalv}
     % setenv NDF_FORMATS_OUT '.,FITS(.fit),IRAF(.imh),FIGARO(.dst)'
\end{terminalv}
This is the recommended behaviour.  If you just want to propagate the
input data format, insert an asterisk at the start of the output-format
list.

\begin{terminalv}
     % setenv NDF_FORMATS_OUT '*,.,FITS(.fit),IRAF(.imh),FIGARO(.dst)'
\end{terminalv}
This only affects applications that create a dataset using information
propagated from an existing dataset.  For instance, if the above
\texttt{NDF\_FORMATS\_OUT} were defined,

\begin{terminalv}
     % ffclean in=m51.dst out=m51_cleaned \\
\end{terminalv}
would now create \file{m51\_cleaned.dst}.  If there is no propagation
in the given application, the asterisk is ignored.

You can retain the scratch NDF by setting the environment variable
\texttt{NDF\_KEEP} to 1.  This is useful if you intend to work mostly with NDFs
and will save the conversion each time you access the dataset.

The \texttt{convert} command, which sets up definitions for the \CONVERT\
package, defines the lists of input and output formats as follows.

\begin{terminalv}
     % setenv NDF_FORMATS_IN \
     'FITS(.fit),FIGARO(.dst),IRAF(.imh),STREAM(.das),UNFORMATTED(.unf),UNF0(.dat),
     ASCII(.asc),TEXT(.txt),GIF(.gif),TIFF(.tif),GASP(.hdr),COMPRESSED(.sdf.Z),
     GZIP(.sdf.gz),FITS(.fits),FITS(.fts),FITS(.FTS),FITS(.FITS),FITS(.FIT),
     FITS(.lilo),FITS(.lihi),FITS(.silo),FITS(.sihi),FITS(.mxlo),FITS(.rilo),
     FITS(.rihi),FITS(.vdlo),FITS(.vdhi),STREAM(.str)'

     % setenv NDF_FORMATS_OUT \
     '.,FITS(.fit),FIGARO(.dst),IRAF(.imh),STREAM(.das),UNFORMATTED(.unf),
     UNF0(.dat),ASCII(.asc),TEXT(.txt),GIF(.gif),TIFF(.tif),GASP(.hdr),
     COMPRESSED(.sdf.Z),GZIP(.sdf.gz)'
\end{terminalv}

See the \CONVERT\ documentation for more details of these conversions.

\subsection{Other Routes for Data Import}

You can run \CONVERT\ ({\it{cf.}}~SUN/55) directly to perform
conversions.  There is also \htmlref{TRANDAT}{TRANDAT}, which will read a text file of
data values, or co-ordinates and data values into an NDF, and
\xref{ASCIN}{sun86}{ASCIN} in the \FIGAROref\ package\latex{ (SUN/86)}.

\subsection{\xlabel{se_fitsreaders}FITS readers\label{se:fitsreaders}}

The automatic conversion does not allow you the full control of the
conversion that direct use of a \FITSref\ reader offers and it does
not deal with the special properties of tape. For full control of the
conversion process, you should use the FITS2NDF and MTFITS2NDF commands
form the \CONVERTref package. \xref{FITS2NDF}{sun55}{FITS2NDF} reads disk FITS
files, and \xref{MTFITS2NDF}{sun55}{MTFITS2NDF} reads FITS files from
magnetic tape.

For historical reasons, \KAPPA\ contains its own additional
FITS readers; \htmlref{FITSIN}{FITSIN} for reading data from tape, and
\htmlref{FITSDIN}{FITSDIN} for reading data from disk.  These do not
currently have all the features of the corresponding \CONVERTref commands
(for instance, they do not allow an NDF to be created from a specified
FITS extension). For this reason, you should normally use the CONVERT
commands described in \xref{SUN/55}{sun55}{}.

\latex{Let's see the \KAPPA\ FITS readers in action.}

\subsubsection{\xlabel{se_readfitstape}Reading FITS Tapes\label{se:readfitstape}}

\htmlref{FITSIN}{FITSIN} reads FITS files stored on tape.  For
efficiency, you should select the `no-rewind' device for the
particular tape drive, for example \file{/dev/nrmt0h} on OSF/1 and
\file{/dev/rmt/1n} on Solaris.

We ask for the second file on the tape, and the headers are displayed
so we can decide whether this is the file we want.  It is so we supply
a name of an NDF to receive the FITS file.  If it wasn't we would
enter \texttt{{!}} to the OUT prompt.  The FMTCNV parameter asks whether
the data are to be converted to \htmlref{\_REAL}{ap:HDStypes}, using
the FITS keywords BSCALE and BZERO, if present.  If you are wondering
why there is \texttt{(1)} after the file number, that's present because
FITS files can have sub-files, stored as FITS extensions.

\begin{terminalv}
     % fitsin
     MT - Tape deck /@/dev/nrmt0h/ >
     The tape is currently positioned at file 1.
     FILES - Give a list of the numbers of the files to be processed > 2
     File # 2(1)  Descriptors follow:
     SIMPLE  =                    T
     BITPIX  =                   16
     NAXIS   =                    2
     NAXIS1  =                  400
     NAXIS2  =                  590
     DATE    = '03/07/88'                    /Date tape file created
     ORIGIN  = 'ING     '                    /Tape writing institution
     OBSERVER= 'CL      '                    /Name of the Observer
     TELESCOP= 'JKT     '                    /Name of the Telescope
     INSTRUME= 'AGBX    '                    /Instrument configuration
     OBJECT  = 'SYS:ARCCL.002'               /Name of the Object
     BSCALE  =                  1.0          /Multiplier for pixel values
     BZERO   =                  0.0          /Offset for pixel values
     BUNIT   = 'ADU     '                    /Physical units of data array
     BLANK   =                    0          /Value indicating undefined pixel
                 :                :                :
                 :                :                :
                 :                :                :
     END
     FMTCNV - Convert data? /NO/ >
     OUT - Output image > ff1
     Completed processing of tape file 2 to ff1.
     MORE - Any more files? /NO/ >
\end{terminalv}
We can trace the structure to reveal the 2-byte integer CCD image.  Notice
that the FITS headers are stored verbatim in a component .MORE.FITS.
This is the FITS extension.  The extension contents can be listed with
\htmlref{FITSLIST}{FITSLIST}.  There is more on this NDF extension and
its purpose in
\latexhtml{Section~\ref{se:fitsairlock}.}{the \htmlref{FITS
Airlock}{se:fitsairlock}.}

\begin{terminalv}
     % hdstrace ff1
     FF1  <NDF>

        DATA_ARRAY(400,590)  <_WORD>   216,204,220,221,202,222,220,206,218,221,
                                       ... 216,218,218,204,221,218,219,222,221,218
        TITLE          <_CHAR*13>      'SYS:ARCCL.002'
        UNITS          <_CHAR*3>       'ADU'
        MORE           <EXT>           {structure}
           FITS(84)       <_CHAR*80>      'SIMPLE  =                    T','BI...'
                                          ... '   ...','         ING PACKEND','END'

     End of Trace.
\end{terminalv}
If you have many FITS files to read there is a quick method for
extracting all files or a selection.  In automatic mode the output
files are generated without manual intervention and the headers aren't
reported for efficiency.  Should you want to see the headers, write
them to a text file via the LOGFILE parameter.  The cost of automation
is a restriction on the names of the output files, but if you have
over a hundred files on a tape are you really going to name them
individually?

The following example extracts the fourth to sixth, and eighth files.
Note that the \texttt{[~]} are needed because the value for Parameter FILES is
a character array.

\begin{terminalv}
     % fitsin auto
     MT - Tape deck /@/dev/nrmt0h/ >
     FMTCNV - Convert data? /NO/ > y
     PREFIX - Prefix for the NDF file names? /'FITS'/ > JKT
     FILES - Give a list of the numbers of the files to be processed > [4-6,8]
     Completed processing of tape file 4 to JKT4.
     Completed processing of tape file 5 to JKT5.
     Completed processing of tape file 6 to JKT6.
     Completed processing of tape file 8 to JKT8.
     MORE - Any more files? /NO/ >
\end{terminalv}

You can list selected FITS headers from a FITS tape without attempting
to read in the data into NDFs by using \htmlref{FITSHEAD}{FITSHEAD}.
You can redirect its output to a file to browse at your leisure, and
identify the files you want to convert.  So for instance,

\begin{terminalv}
     % fitshead /dev/nrmt1h > headers.lis
\end{terminalv}
lists all the FITS headers from a FITS tape on device \file{/dev/nrmt1h}
to file \file{headers.lis}.

After running FITSIN you may notice a file \file{USRDEVDATASET.sdf} in
the current directory.  This \HDSref\ file records the current
position of the tape, so you can use FITSIN to read a few files, and
then run it again a little later, and FITSIN can carry on from where
you left off.  In other words FITSIN does not have to rewind to the
beginning of the tape to count files.  When you're finished you should
delete this file.

\subsubsection{Reading FITS Files}

For many years there was officially no such thing as
disc FITS.  However, \emph{ad hoc} implementations have
existed for a long time.  Of these, \htmlref{FITSDIN}{FITSDIN} will handle files
adhering to the FITS rules for blocking (and more), but it doesn't
process byte-swapped `FITS' files.  Thus it can process files with
fixed-length records of semi-arbitrary length; so, for example, files
mangled during network transfer, which have 512-byte records rather
than the customary 2880, may be read.  However, it will not handle,
VAX FITS files as may be produced with \xref{\FIGARO's WDFITS}{sun86}{WDFITS}.
FITSDIN will accept a list of files with
wildcards.  However, a comma-separated list must be enclosed in
quotation marks.  Also wildcards must be protected.  Here are some
examples so you get the idea.

\begin{terminalv}
     % fitsdin '*.fit'
     % fitsdin \*.fit
     ICL> fitsdin *.fit
     % fitsdin '"i*.fit,abc123.fts"'
     ICL> fitsdin "i*.fit,abc123.fts"
\end{terminalv}

In the following example a floating-point file is read (BITPIX=$-$32)
and so FMTCNV is not required.
\begin{terminalv}
     % fitsdin '*.fits'

        2 files to be processed...

     Processing file number 1: /home/scratch/dro/gr.fits.
     File /scratch/dro/gr.fits(1)  Descriptors follow:
     SIMPLE  =                    T / Standard FITS format
     BITPIX  =                  -32 / No. of bits per pixel
     NAXIS   =                    2 / No. of axes in image
     NAXIS1  =                  512 / No. of pixels
     NAXIS2  =                  256 / No. of pixels
     EXTEND  =                    T / FITS extension may be present
     BLOCKED =                    T / FITS file may be blocked

     BUNIT   = 'none given      '   / Units of data values

     CRPIX1  =   1.000000000000E+00 / Reference pixel
     CRVAL1  =   0.000000000000E+00 / Coordinate at reference pixel
     CDELT1  =   1.000000000000E+00 / Coordinate increment per pixel
     CTYPE1  = '                '   / Units of coordinate
     CRPIX2  =   1.000000000000E+00 / Reference pixel
     CRVAL2  =   0.000000000000E+00 / Coordinate at reference pixel
     CDELT2  =   1.000000000000E+00 / Coordinate increment per pixel
     CTYPE2  = '                '   / Units of coordinate

     ORIGIN  = 'ESO-MIDAS'          / Written by MIDAS
     OBJECT  = 'artificial image'   / MIDAS desc.: IDENT(1)
             :                :                :
             :                :                :
             :                :                :
     HISTORY  ESO-DESCRIPTORS END     ................

     END
     OUT - Output image > gr
     Completed processing of disc file /home/scratch/dro/gr.fits to gr.
     File has illegal-length blocks (512).  Blocks should be a multiple (1--10) of the
     FITS record length of 2880 bytes.
     Processing file number 2: /home/scratch/dro/indef.fits.
     File /home/scratch/dro/indef.fits(1)  Descriptors follow:
     SIMPLE  =                    T  /  FITS STANDARD
     BITPIX  =                   32  /  FITS BITS/PIXEL
     NAXIS   =                    2  /  NUMBER OF AXES
     NAXIS1  =                  256  /
     NAXIS2  =                   20  /
     BSCALE  =      3.7252940008E28  /  REAL = TAPE*BSCALE + BZERO
     BZERO   =      7.9999999471E37  /
     OBJECT  = 'JUNK[1/1]'  /
     ORIGIN  = 'KPNO-IRAF'  /
             :                :                :
             :                :                :
             :                :                :
     END
     OUT - Output image > iraf
     Completed processing of disc file /home/scratch/dro/indef.fits to iraf.
\end{terminalv}
\htmlref{NDFTRACE}{NDFTRACE} shows that the object name is written to
the NDF's title, that axes derived from the FITS headers are present,
and that gr is a \_REAL NDF.

\begin{terminalv}
     % ndftrace gr

        NDF structure /home/scratch/dro/iraf:
           Title:  artificial image
           Units:  none given

        Shape:
           No. of dimensions:  2
           Dimension size(s):  512 x 256
           Pixel bounds     :  1:512, 1:256
           Total pixels     :  131072

        Axes:
           Axis 1:
              Label : Axis 1
              Units : pixel
              Extent: -0.5 to 511.5

           Axis 2:
              Label : Axis 2
              Units : pixel
              Extent: -0.5 to 255.5

        Data Component:
           Type        :  _REAL
           Storage form:  PRIMITIVE
           Bad pixels may be present

        Extensions:
              FITS             <_CHAR*80>

\end{terminalv}
Both FITSIN and FITSDIN write the FITS headers into an NDF extension
called FITS within your NDF.  The extension is a literal copy of all
the 80-character `card images' in order.  These can be inspected or
written to a file via the command FITSLIST.  There is more on this NDF
extension and its purpose in
\latexhtml{Section~\ref{se:fitsairlock}.}{the \htmlref{FITS
Airlock}{se:fitsairlock}.}

\subsection{\xlabel{se_fitsairlock}The FITS Airlock\label{se:fitsairlock}}

\subsubsection{\xlabel{se_ndfext}NDF Extensions\label{se:ndfext}}

An important feature of the \NDFref{NDF} is that it is designed to be
extensible.  The NDF has components whose meanings are well defined
and universal, and so they can be accessed by general-purpose
software, such as \KAPPA\ and \CONVERTref\ provide; but the
NDF also allows independent {\em extensions\/} to be defined and
added, which can store auxiliary information to suit the needs of a
specialised software package.  (Note that the term extension here
refers to a structure within the NDF for storing additional data, and
is neither the file extension \file{.sdf} nor extensions like BINTABLE
within the FITS file.) An extension is only processed by software that
understands the meanings obeys the processing rules of the various
components of the extension.  Other programmes propagate the extension
information unaltered.

The existence of extensions makes it straightforward to write general
utilities for converting an arbitrary format into an NDF.  The idea
being that every specialist package should not have to have its own
conversion tools such as a FITS reader.  However, this still leaves the
additional data that requires specialist knowledge to move it into the
appropriate extension components.  The aim is to make the conversions
themselves extensible, with add-on operations to move the specialist
information to and from the extensions.  This is where the FITS
`airlock' comes in.

The \FITSref\ data format comprises a header followed by the data
array or table.  The header contains a series of 80-character lines
each of which contains the keyword name, a value and an optional
comment.  There are also some special keywords for commentary.  The
meanings of most keywords are undefined, and so can be used to
transport arbitrary ancillary information, subject to FITS syntax
limitations.  There is a special NDF extension called FITS, which
mirrors this functionality, and may be added to an NDF.  It therefore
can act as an airlock between the general-purpose conversion tools and
specialist packages.

\subsubsection{\xlabel{se_fitsimpexp}Importing and Exporting from and
to the FITS Extension\label{se:fitsimpexp}}

The FITS extension comprises a one-dimensional array of 80-character
strings that follow FITS-header formatting rules.  In the case of
\htmlref{FITSIN}{FITSIN} and \htmlref{FITSDIN}{FITSDIN}, each FITS
extension is a verbatim copy of the FITS header of the input file.
Other conversion tools like \xref{IRAF2NDF}{sun55}{IRAF2NDF} and
\xref{UNF2NDF}{sun55}{UNF2NDF} of \CONVERTref\ can also create a FITS
extension in the same fashion.  On export, standard conversion tools
propagate the FITS extension to any FITS headers or equivalent in the
foreign format.  However, information which is derivable from the
standard NDF components, such as the array dimensions, data units, and
linear axes, replaces any equivalent headers from FITS extension.

You use your knowledge, or the writer of the specialist package
provides import tools, to recognise certain FITS keywords and to
attribute meaning to them, and then to move or process their values
to make the specialist extensions.  One such is the PREPARE task in
\IRASref\@.  Similarly, the reverse operation---exporting the
extension information---can occur too, prior to converting the NDF
into another data format.

\KAPPA\ offers two simple tools for the importing and exporting
of extension information: \htmlref{FITSIMP}{FITSIMP} and
\htmlref{FITSEXP}{FITSEXP}.  They both use a text file, which acts as a
translation table between the FITS keyword and extension components.
Starting with FITSIMP, its translation table might look like this.

\begin{terminalv}
     ORDER_NUMBER _INTEGER  ORDNUM
     PLATE_SCALE  _REAL SCALE         ! The plate scale in arcsec/mm
     SMOOTHED  _LOGICAL FILTERED
\end{terminalv}

It consists of three fields: the first is the name of the component in
the chosen extension, the second is the \htmlref{HDS data type}{ap:HDStypes}
of that component, and the third is the FITS keyword.  Optional
comments can appear following an exclamation mark. So if we placed
these lines in file \texttt{imptable}, we could create an extension
called MYEXT of data type MJC\_EXT (if it did not already exist)
containing components ORDER\_NUMBER, PLATE\_SCALE, and SMOOTHED.

\begin{terminalv}
     % fitsimp mydata imptable myext mjc_ext
\end{terminalv}
Should any of the keywords not exist in the FITS extension, you'll be
warned.  If the extension already exists, you don't need to specify
the extension data type.  FITSIMP will even handle hierarchical
keywords and those much-loved ING packets from La Palma.

Going in the opposite direction, the text translation file could look
like this
\begin{terminalv}
     MYEXT.ORDER_NUMBER  ORDNUM(LAST) The spectral order number
     MYEXT.PLATE_SCALE   SCALE   The plate scale in arcsec/mm
     MYEXT.SMOOTHED  FILTERED
\end{terminalv}
where the first column is the `name' of the extension component to be
copied to the FITS extension.  The `name' includes the extension name
and substructures.  The second column gives the FITS keyword to which
to write the value.  A further keyword in parentheses instructs FITSEXP
to place the new FITS header immediately before the header with that
keyword.  If the second keyword is absent from the translation-table
record or the FITS extension, the new header appears immediately before
the END header line in the FITS extension.  Thus the value of ORDER\_NUMBER
in extension MYEXT, creates a new keyword in the FITS extension called
ORDNUM, and it is located immediately prior the keyword LAST.

\subsubsection{\xlabel{se_list-fitsext}Listing the FITS Extension and
keywords\label{se:list-fitsext}}

If you don't want to be bothered with \NDFref{NDF} extensions, you
might just want to know the value of some FITS keyword, say the
exposure time, as part of your data processing.
\htmlref{FITSLIST}{FITSLIST} lists the contents of the FITS extension
of an NDF or file.  You can even search for keywords with {\bf grep}.

\begin{terminalv}
     % fitslist myndf | grep "ELAPSED ="
\end{terminalv}
This would find the keyword ELAPSED in the FITS extension of NDF
myndf.  (Keywords are 8 characters long and those with values are
immediately followed by an equals sign.)  However, the recommended
way is to use the \htmlref{FITSVAL}{FITSVAL} command.  Since this
command only reports the value, it is particularly useful in scripts
that need ancillary-data values during processing.  The following
obtains the value of keyword ELAPSED.

\begin{terminalv}
     % fitsval myndf ELAPSED
\end{terminalv}

In a script you may need to know whether the keyword exists and take
appropriate action.

\begin{terminalv}
     filterpre = `fitsexist myndf filter`
     if ( $filterpre == "TRUE" ) then
        filter = `fitsval myndf filter`
     else
        prompt -n "Filter > "
        set filter = $<
     endif
\end{terminalv}
Shell variable \texttt{filterpres} would be assigned \texttt{"TRUE"} when
the FILTER card is present, and \texttt{"FALSE"} otherwise.  (The
\texttt{`~`} quotes cause the enclosed command to be executed.)  So the
user of the script would be prompted for a filter name whenever the
NDF did not contain that information.

\subsubsection{\xlabel{se_manip-fitsext}Creating and Editing
the FITS Extension\label{se:manip-fitsext}}

Besides the conversion utilities, you can import your own FITS
extension using \htmlref{FITSTEXT}{FITSTEXT}.  You first prepare a
FITS-like header in a text file.  For example,

\begin{terminalv}
     % fitstext myndf myfile
\end{terminalv}

\latex{\normalsize}
places the contents of \file{myfile} in the NDF called myndf.  This is
not advised unless you are familiar with the rules for writing FITS
headers.  See the NOST
\htmladdnormallink{{\em A User's Guide to FITS}}
{http://archive.stsci.edu/fits/users_guide/}
\latex{ (URL \texttt{http://archive.stsci.edu/fits/users\_guide/})}.
Other useful FITS documents, test files, and software are available at
the \htmladdnormallink{FITS Support Office Home Page}
{http://fits.gsfc.nasa.gov/}
\latex{ (URL \texttt{http://fits.gsfc.nasa.gov/})}.


FITSTEXT does perform some limited validation of the FITS headers, and
informs you of any problems it detects.  See the
\htmlref{FITSHEAD}{FITSHEAD} Notes in
\latexhtml{Appendix~\ref{ap:full}}{\htmlref{application
specifications}{ap:full}} for details.

A safer bet for a hand-crafted FITS extension is to edit an existing
FITS extension to change a value, or use existing lines as templates
for any new keywords you wish to add.  \htmlref{FITSEDIT}{FITSEDIT}
lets you do this with your favourite text editor.  Define the
environment variable \texttt{EDITOR} to your editor, say

\begin{terminalv}
     % setenv EDITOR jed
\end{terminalv}
to choose {\bf jed}.  If you don't do this, and EDITOR is unassigned,
FITSEDIT selects the {\bf vi} editor.  Then to edit the NDF extension
is simple.

\begin{terminalv}
     % fitsedit myndf
\end{terminalv}
This edits the FITS extension of the NDF called myndf.
FITSEDIT extracts the file into a temporary file (\file{{zzfitsedit.tmp}})
which you edit, and then uses FITSTEXT to restore the FITS extension.
It therefore has the same parsing of the edited FITS headers as FITSTEXT
provides.

\subsubsection{\xlabel{se_emanip-fitsext}Easy way to create and edit
the FITS Extension\label{se:emanip-fitsext}}

Should you wish to write a new value without knowing about FITS, or
in a script where manual editing is undesirable, the
\htmlref{FITSWRITE}{FITSWRITE} command does the job.  So for example,

\begin{terminalv}
     % fitswrite myndf filter value=K
\end{terminalv}
will create a keyword FILTER with value \texttt{K} in the FITS extension
of the NDF called myndf.  If the extension does not exist, this
command will first create it.

The \htmlref{FITSMOD}{FITSMOD} command has several editing options
including the ability to delete a keyword:

\begin{terminalv}
     % fitsmod myndf airmass edit=delete
\end{terminalv}
here it removes the AIRMASS header; or rename a keyword:

\begin{terminalv}
     % fitsmod myndf band rename newkey=filter
\end{terminalv}
as in this example, where keyword BAND becomes keyword FILTER; or
update an existing keyword:

\begin{terminalv}
     % fitsmod myndf filter edit=u value=\$V comment='"Standard filter name"'
\end{terminalv}
this example modifies the comment string associated with the FILTER
keyword, leaving the value unchanged.

For routine operations requiring many operations on a dataset, FITSMOD
lets you specify the editing instructions in a text file.

\newpage
\section{\xlabel{se_procedures}Procedures\label{se:procedures}}

Applications from \KAPPA\ and other packages can be
combined in procedures and scripts to customise and automate data
processing. In addition to giving literal values to application
parameters, you can include \ICLref\ or C-shell variables on the
command line, whose values are substituted at run time.  It is also
possible to write parameter data into variables, and hence pass them
to another application, or use the variables to control subsequent
processing.

\subsection{\xlabel{se_cshscript}C-shell scripts\label{se:cshscript}}

The \xref{{\sl C-shell Cookbook}}{sc4}{} contains many ingredients and
recipes, and features many \KAPPA\ commands.  So there is little
point repeating them here other than to direct you to a documented
script in \file{\$KAPPA\_DIR/multiplot.csh}.

\subsection{\xlabel{se_ICLproc}ICL Procedures\label{se:ICLproc}}

You should consult the \xref{{\sl ICL Users' Guide}}{sg5}{} for details
about writing \ICL\ syntax, procedures, and functions, but you're a busy
researcher\ldots  For a quick overview the {\em two-page\/} summary on
``Writing ICL command files and procedures'' in \xref{SUN/101}{sun101}{} is
recommended reading, even though much of the document is dated and still refers
to VMS.  Here we'll just show some example procedures that can be
adapted and cover points not mentioned in SUN/101.

Let's start with something simple.  You want to `flash' a series of
images, each with a yellow border.  First you write the following
procedure called FLASH.  It has one argument INPIC, that passes the name of
the NDF you want to display.  When you substitute an \ICL
variable for a parameter value you enclose it in parentheses.  The lines
beginning with \texttt{\{ } are comments.

\begin{terminalv}
     PROC FLASH INPIC
     {
     { Procedure for displaying an image without scaling.
     {
        DISPLAY IN=(INPIC) MODE=FL
     END PROC
\end{terminalv}
To make \ICL\ recognise your procedure you must `load' it.  The
command

\begin{terminalv}
     ICL> LOAD FLASH
\end{terminalv}
will load the file \file{FLASH.ICL}.
Thereafter in the \ICL\ session you can invoke FLASH for many
NDFs.  The following will display the NDFs called GORDON and FLOOD
side-by-side.

\begin{terminalv}
     ICL> PICGRID 2 1
     ICL> FLASH GORDON
     ICL> PICSEL 2
     ICL> FLASH FLOOD
\end{terminalv}
It would be tedious to have to load lots of individual procedures, but
you don't. If you have related procedures that you regularly require
they can be concatenated into a single file which you load.  Better
still is to add definitions for each of the procedures in your \ICL\
login file.  This is defined as the value of the \texttt{ICL\_LOGIN}
environment variable.  A reasonable place is in your home directory
and you'd define it like this.

\begin{terminalv}
     % setenv ICL_LOGIN $HOME/login.icl
\end{terminalv}
However, the file doesn't have to be in your home directory, or called
\file{login.icl}, but it's convenient to do so.  Suppose you have three
procedures: FLASH, PICGREY in file \file{\$MY\_DIR/display\_proc.icl},
and FILTER in \file{/home/user1/dro/improc.icl}.  In your \file{
\$HOME/login.icl} you could add the following

\begin{terminalv}
     defproc  flash     $MY_DIR/display_proc.icl
     defproc  sfilt     $HOME/user1/dro/improc.icl filter
     defproc  picgr(ey) $MY_DIR/display_proc.icl
\end{terminalv}
which defines three commands that will be available each time you
use \ICL: FLASH which will run your FLASH
procedure, PICGREY to execute the PICGREY procedure, and SFILT which
runs the FILTER procedure.  In addition PICGREY can be abbreviated
to PICGR or PICGRE.  So now you can load and run your procedure.
Let's have some more example procedures.

Suppose you have a series of commands to run on a number of files.
You could create a procedure to perform all the stages of the
processing, deleting the intermediate files that it creates.

\begin{terminalv}
     PROC UNSHARPMASK NDFIN CLIP NDFOUT

     { Insert ampersands to tell the command-line interpreter than these
     { strings are file names.
        IF SUBSTR( NDFIN, 1, 1 ) <> '@'
           NDFIN = '@' & (NDFIN)
        END IF
        IF SUBSTR( JUNK, 1, 1 ) <> '@'
           NDFOUT = '@' & (NDFOUT)
        END IF

     { Clip the image to remove the cores of stars and galaxies above
     { a nominated threshold.
        THRESH (NDFIN) TMP1 THRHI=(CLIP) NEWHI=(CLIP) \

     { Apply a couple of block smoothings with boxsizes of 5 and 13
     { pixels.  Delete the temporary files as we go along.
        BLOCK tmp1 tmp2 BOX=5
        ! rm tmp1.sdf
        BLOCK tmp2 tmp3 BOX=13
        ! rm tmp2.sdf

     { Multiply the smoothed image by a scalar.
        CMULT tmp3 0.8 tmp4
        ! rm tmp3.sdf

     { Subtract the smoothed and renormalised image from the input image.
     { The effect is to highlight the fine detail, but still retain some
     { of the low-frequency features.
        SUB (NDFIN) tmp4 (NDFOUT)
        ! rm tmp4.sdf
     END PROC
\end{terminalv}
There is a piece of syntax to note which often catches people out.
Filenames, data objects, and devices passed via \ICL\ variables
to applications, such as NDFIN and NDFOUT in the above example, must
be preceded by an \texttt{@}.

A common use of procedures is likely to be to duplicate processing
for several files.  Here is an example procedure that does that.  It uses
some intrinsic functions which look just like Fortran.

\begin{terminalv}
     PROC MULTISTAT

     { Prompt for the number of NDFs to analyse.  Ensure that it is positive.
        INPUTI Number of frames:  (NUM)
        NUM = MAX( 1, NUM )

     { Find the number of characters required to format the number as
     { a string using a couple of ICL functions.
        NC = INT( LOG10( I ) ) + 1

     { Loop NUM times.
        LOOP FOR I=1 TO (NUM)

     { Generate the name of the NDF to be analysed via the ICL function
     { SNAME.
          FILE = '@' & SNAME('REDX',I,NC)

     { Form the statistics of the image.
          STATS NDF=(FILE)
        END LOOP
     END PROC
\end{terminalv}
If NUM is set to 10, the above procedure obtains the statistics of the
images named REDX1, REDX2, \dots REDX10.  The \ICL\ variable FILE is in
parentheses because its value is to be substituted into Parameter NDF.

Here is another example, which could be used to flat field a series of
CCD frames.  Instead of executing a specific number of files, you
can enter an arbitrary sequence of NDFs.  When processing is
completed a !! is entered rather than an NDF name, and that exits the
loop.  Note the \texttt{\~{}} continuation character (it's not
required but it's included for pedagogical reasons).
\pagebreak[3]

\begin{terminalv}
     PROC FLATFIELD

     { Obtain the name of the flat-field NDF.  If it does not have a
     { leading @ insert one.
        INPUT "Which flat field frame?: " (FF)
        IF SUBSTR( FF, 1, 1 ) <> '@'
           FF = '@' & (FF)
        END IF

     { Loop until there are no further NDFs to flat field.
        MOREDATA = TRUE
        LOOP WHILE MOREDATA

     { Obtain the frame to flat field.  Assume that it will not have
     { an @ prefix.  Generate a title for the flattened frame.
           INPUT "Enter frame to flat field (!! to exit): " (IMAGE)
           MOREDATA = IMAGE <> '!!'
           IF MOREDATA
              TITLE = 'Flat field of ' & (IMAGE)
              IMAGE = '@' & (IMAGE)

     { Generate the name of the flattened NDF.
              IMAGEOUT = (IMAGE) & 'F'
              PRINT Writing to (IMAGEOUT)

     { Divide the image by the flat field.
              DIV IN1=(IMAGE) IN2=(FF) OUT=(IMAGEOUT) ~
                TITLE=(TITLE)
           END IF
        END LOOP
     END PROC
\end{terminalv}
Some \KAPPA\ applications, particularly the statistical ones,
produce output parameters, which can be passed between applications via
\ICL\ variables.  Here is an example to draw a
contour plot centred about a star in a nominated data array
from only the star's approximate position.  The region about the star is
stored in an output NDF file.  Note the syntax required to define the
value of Parameter INIT; the space between the left bracket and
parenthesis is essential.

\begin{terminalv}
     PROC COLSTAR FILE,X,Y,SIZE,OUTFILE

     {+
     {  Arguments:
     {     FILE = FILENAME (Given)
     {        Input NDF containing one or more star images.
     {     X = REAL (Given)
     {        The approximate x position of the star.
     {     Y = REAL (Given)
     {        The approximate y position of the star.
     {     SIZE = REAL (Given)
     {        The half-width of the region about the star's centroid to be
     {        plotted and saved in the output file.
     {     OUTFILE = FILENAME (Given)
     {        Output primitive NDF of 2*%SIZE+1 pixels square (unless
     {        constrained by the size of the data array or because the location
     {        of the star is near an edge of the data array.
     {-

     { Ensure that the filenames have the @ prefix.
        IF SUBSTR( FILE, 1, 1 ) <> '@'
           NDF = '@' & (FILE)
        ELSE
           NDF = (FILE)
        END IF
        IF SUBSTR( OUTFILE, 1, 1 ) <> '@'
           NDFOUT = '@' & (OUTFILE)
        ELSE
           NDFOUT = (OUTFILE)
        END IF

     { Search for the star in a 21x21 pixel box.  The centroid of the
     { star is stored in the ICL variables XC and YC.
        CENTROID NDF=(NDF) INIT=[ (X&','&Y)] XCEN=(XC) YCEN=(YC) ~
          MODE=INTERFACE SEARCH=21 MAXSHIFT=14

     { Convert the co-ordinates to pixel indices.
        IX = NINT( XC + 0.5 )
        IY = NINT( YC + 0.5 )

     { Find the upper and lower bounds of the data array to plot.  Note
     { this assumes no origin information in stored in the data file.
        XL = MAX( 1, IX - SIZE )
        YL = MAX( 1, IY - SIZE )
        XU = MAX( 1, IX + SIZE )
        YU = MAX( 1, IY + SIZE )

     { Create a new NDF centred on the star.
        NDFCOPY IN=(NDF)((XL):(XU),(YL):(YU)) OUT=(NDFOUT)

     { Draw a contour plot around the star on the current graphics device
     { at the given percentiles.
        CONTOUR NDF=(NDFOUT) MODE=PE PERCENTILES=[80,90,95,99]

     { Exit if an error occurred, such as not being able to find a star
     { near the supplied position, or being unable to make the plot.
        EXCEPTION ADAMERR
           PRINT Unable to find or plot the star.
        END EXCEPTION
     END PROC
\end{terminalv}

\newpage
\section{\xlabel{se_probpage}Problems Problems\label{se:probpage}}
\subsection{Errors}

A detailed list of error codes and their meanings is not available.
\KAPPA\ produces descriptive contextual error messages, which
are usually straightforward to comprehend.  Some of these originate in
the underlying infrastructure software.  Error messages from \KAPPA\
begin with the name of the application reporting the error.  The
routine may have detected the error, or it has something to say about
the context of the error.

The remainder of the section describes some difficulties you may encounter
and how to overcome them.  Please suggest additions to this compilation.

\subsection{No Match}

When running \KAPPA\ from the UNIX shell, your command fails with a
``No Match'' error message.   This means you have forgotten to protect
a wildcard character, such as \verb+*+, \verb+?+, so they they are
passed to the \KAPPA\ command and not interpreted by the UNIX shell.
You can precede the wildcard character with \texttt{$\backslash$}, or
surround the wildcard characters in \texttt{" "} quotes.  Here are some
examples.

\begin{terminalv}
     % ndftrace ccd\?_ff
     % stats \*118_"[b-d]?"
\end{terminalv}

\subsection{Unable to Obtain Work Space}

Error messages like ``Unable to create a work array'' may puzzle you.
They are accompanied by additional error messages that usually
pinpoint the reason for the failure of the application to complete.
Many applications require temporary or work space to perform their
calculations.  This space is stored in an \HDSref\ file within directory
\texttt{\$HDS\_SCRATCH} and most likely is charged to your disc quota. (If you
have not redefined this environment variable, it will point to your
current directory.) So one cause for the message is insufficient disc
quota available to store the work space container file or to extend
it.  A second reason for the message is that your computer cannot
provide sufficient virtual memory to map the workspace.  In this case
you can try increasing your process limits using the C-shell built-in
function \texttt{limit}.  You can find your current limits by entering
\texttt{limit}.  You should see a list something like this.

\begin{terminalv}
     cputime         unlimited
     filesize        unlimited
     datasize        131072 kbytes
     stacksize       2048 kbytes
     coredumpsize    unlimited
     memoryuse       89232 kbytes
     vmemoryuse      1048576 kbytes
     descriptors     4096
\end{terminalv}
The relevant keywords are \texttt{datasize} and the \texttt{vmemoryuse}.  In
effect \texttt{datasize} specifies the maximum total size of data files
you can map at one time in a single programme.  The default should be
adequate for most purposes and only need be modified for those working
with large images or cubes.  The \texttt{vmemoryuse} specifies the maximum
virtual memory you can use.

\begin{terminalv}
    % limit datasize 32768
\end{terminalv}
sets the maximum size of mapped data to 32 megabytes.  Values cannot
exceed the system limits.  You can list these with the \texttt{-h}
qualifier.

\begin{terminalv}
     % limit -h
     cputime         unlimited
     filesize        unlimited
     datasize        1048576 kbytes
     stacksize       32768 kbytes
     coredumpsize    unlimited
     memoryuse       89232 kbytes
     vmemoryuse      1048576 kbytes
     descriptors     4096
\end{terminalv}
Although you can set your limits to the system maxima, it doesn't mean
that you should just increase your quotas to the limits. You might
become unpopular with some of your colleagues, especially if you
accidentally try to access a huge amount of memory.  If you cannot
accommodate your large datasets this way, you should fragment your
data array, and process the pieces separately.

After receiving this error message in an \ICLref\normalsize session you may
need to delete the scratch file by hand.  The file is called \file{
txxx.sdf}, where \texttt{xxxx} is a process identifier.  A normal exit
from \ICL\normalsize will delete the work-space container file.

\subsection{\xlabel{se_probwrongNDF}Application Automatically Picks up the
Wrong NDF\label{se:probwrongNDF}}

Some applications read the name of the \NDFref{NDF} used to create a plot or
image from the graphics database in order to save typing.  Once in a
while you'll say ``that's not the one I wanted''.  This is because
\AGIref\ finds the last \texttt{DATA} picture situated within the current picture.
Abort the application via \texttt{!!}, then use \htmlref{PICCUR}{PICCUR}
or \htmlref{PICLIST}{PICLIST} to select the required \texttt{FRAME}
picture enclosing the \texttt{DATA} picture, or even select the latter
directly.  You can override the AGI NDF also by specifying the
required NDF on the command line, provided it has pixels whose indices
lies within the world co-ordinates of the \texttt{DATA} picture.  Thus

\begin{terminalv}
     % inspect myndf
\end{terminalv}
will inspect the NDF called myndf.  The command \htmlref{PICIN}{PICIN} will
show the last DATA picture and its associated NDF.

\subsection{Unable to Store a Picture in the Graphics Database}

You may receive an error message, which says failed to store
such-and-such picture in the \htmlref{graphics database}{se:agitate}.
For some reason the database was corrupted due to reasons external to
\KAPPA.  Don't worry, usually your plot will have appeared, and
to fix the problem run \htmlref{GDCLEAR}{GDCLEAR}
or delete the database file
(\file{{\$AGI\_USER/agi\_}}{\it $<$node$>$}\file{.sdf}, where you substitute your
system's node name for {\it $<$node$>$}).  You will need to redraw the
last plot if you still require it, say for interaction.

\subsection{Line Graphics are Invisible on an Graphics Device}

The reason for invisible line graphics on your graphics device is
because it is drawn in black or a dark grey.  Most likely is that some
person has been using other software on your graphics device or that is
has been reset.  \htmlref{PALDEF}{PALDEF} will set up the default
colours for the palette, and so most line graphics will then appear in
white.  Alternatively,

\begin{terminalv}
     % palentry 1 white
\end{terminalv}
will normally suffice.

\subsection{Error Obtaining a Locator to a Slice of an HDS array}

If the above error appears from DAT\_SLICE and you are (re)prompted
for an \NDFref{NDF}, the most likely cause is that you have asked an
\htmlref{IMAGE}{ap:IMAGE-format} application to process an
\htmlref{NDF section}{se:ndfsect}.  Use \htmlref{NDFCOPY}{NDFCOPY} to
make a subset before running the application in question, or process
the whole NDF.

\subsection{Badly placed ()'s}

This means that you have forgotten to `escape' parentheses, probably when
defining an \htmlref{NDF section}{se:ndfsect} in the UNIX shell.
Try inserting a backslash before each parenthesis or enclosing all the
special characters inside \texttt{" "} quotes.

\begin{terminalv}
     % stats myndf\(100:200,\)
     % linplot spectrum"(5087.0~30)"
\end{terminalv}


\subsection{Attempt to use 'positional' parameter value (\texttt{{x}}) in an
unallocated position}

Check the usage of the application you are running.  One way of
adding positional parameters unintentionally, is to forget to escape
the \texttt{{"}} from the shell when supplying a string with spaces or
wildcards.  For example, this error would arise if we entered

\begin{terminalv}
     % settitle myndf "A title"
\end{terminalv}
instead of say

\begin{terminalv}
     % settitle myndf '"A title"'
\end{terminalv}
which protects all special characters between the single quotes.

\subsection{The choice \texttt{x} is not in the menu.  The options
are\ldots}

You have either made an incorrect selection, or you have forgotten
to escape a metacharacter.  For the former, you can select a new
value from the list of valid values presented in the error message.
For the latter, part of another value is being
interpreted as a positional value for the parameter the task is
complaining about.

\begin{terminalv}
     % linplot $KAPPA_DIR/spectrum style="Title=Red-giant plot"
     !! The choice plot is not in the menu.  The options are
     !     Data,Quality,Error,Variance.
     !  Invalid selection for Parameter COMP.
\end{terminalv}
Here it thinks that \texttt{plot} is a positional value.  Escape
the \texttt{{"}} to cure the problem.

\begin{terminalv}
     % linplot $KAPPA_DIR/spectrum style='"Title=Red-giant plot"'
\end{terminalv}

\subsection{Annotated axes show the wrong co-ordinate system}

Each NDF has an associated {\em current co-ordinate system} which is used
when reporting positions within the NDF, or when obtaining positions from
the user. If you want to either see, or give, positions in a {\em
different} co-ordinate system, you need to change the current co-ordinate
system (more often called the current co-ordinate {\em frame}) of the NDF
by using command \htmlref{WCSFRAME}{WCSFRAME}. For instance,

\begin{terminalv}
     % wcsframe m57 pixel
\end{terminalv}

will cause all subsequent commands to use pixel co-ordinates when
reporting positions, or obtaining positions.

\subsection{\xlabel{se_fitsunixtape}``I've Got This FITS Tape''
\label{se:fitsunixtape}}

Certain combinations of magnetic tape produced on one model of tape
drive but read on a different model seem to generate parity errors
that are detected by the MAG\_ library that \htmlref{FITSIN}{FITSIN}
uses.  However, this doesn't mean that you won't be able to read your
FITS tape.  The UNIX tape-reading commands seem less sensitive to
these parity errors.

Thus you should first attempt to convert the inaccessible FITS files
on tape to disc files using the UNIX {\bf dd} command, and then use
the \htmlref{FITSDIN}{FITSDIN} application to generate the output NDF
or foreign format.  For example to convert a FITS file from device
\file{/dev/nrst0} to an NDF called ndfname, you might enter

\begin{terminalv}
     % dd if=/dev/nrst0 ibs=2880 of=file.fits
     % fitsdin files=file.fits out=ndfname
     % rm file.fits
\end{terminalv}
where \file{file.fits} is the temporary disc-FITS file.  The 2880 is
the length of a FITS record in bytes.  Repeated {\bf dd} commands to
a no-rewind tape device (those with the \texttt{n} prefix on OSF/1 and the
\texttt{n} suffix on Solaris) will copy successive files.  To skip over
files or rewind the tape, use the {\bf mt} command.  For example,

\begin{terminalv}
     % mt -f /dev/rmt/1n fsf 3
           :       :       :
     % mt -f /dev/rmt/1n asf 4
\end{terminalv}
moves the tape on device \file{/dev/rmt/1n} forward three files,
then moves to the fourth file,

\begin{terminalv}
     % mt bsf 2
\end{terminalv}
moves back two files on the default tape drive (defined by the
environment variable \texttt{TAPE}), and

\begin{terminalv}
     % mt -f /dev/nrmt0h rewind
\end{terminalv}
rewinds to the start of the tape on device \file{/dev/nrmt0h}.
Thus it is possible to write a script for extracting and converting a
series of files including ranges, just like FITSIN does.

If the above approach fails, try another tape drive.

\subsection{\xlabel{se_probfitsin}FITSIN does not Recognise my FITS Tape
\xlabel{se_probfitsin}}

If you attempt to read a FITS magnetic tape with
\htmlref{FITSIN}{FITSIN}, you might receive an error like this

\vspace*{-\smallskipamount}
\small
\begin{terminalv}
     % fitsin
     % MT - Tape deck /@/dev/nrmt1h/ > /dev/nrmt3l
     !! Object '/DEV/NRMT3L' not found.
     !  DAT_FIND: Error finding a named component in an HDS structure.
     !  /dev/nrmt3l: MAG__UNKDV, Unable to locate device in DEVDATASET
\end{terminalv}
\vspace*{-\medskipamount}
\latex{\normalsize}
when you enter the device name.  The magnetic-tape system uses an
\HDSref\ file called the device dataset (DEVDATASET) to store the
position of the tape between invocations of Starlink applications.
\goodbreak

When FITSIN is given a name, the magnetic-tape system validates the
name to check that it is a known device.  There should be a
\file{devdataset.sdf} file (within \file{/star/etc} at Starlink sites)
containing a list of at least all the available drives at your site.
What FITSIN is complaining about above, is that the device you have
given is not included in the DEVDATASET file.  Now this might be
because you mistyped the device name, or that the particular device is
not accessible on the particular machine, or because your computer manager
has not maintained the DEVDATASET when a drive was added.  You can look
at the contents of the DEVDATASET with this command.

\begin{terminalv}
     % hdstrace /star/etc/devdataset
\end{terminalv}


Oh and one other point: make sure the tape is loaded in the drive.
Yes this mistake has happened (not mentioning any names) and it is
very hard to diagnose remotely.

\subsection{\xlabel{se_probweird}It Used to Work\ldots and Weird
Errors\label{se:probweird}}

There is a class of error that arises when an \HDSref\ file is corrupted.
The specific message will depend on the file concerned and where
in the file the corruption occurred.  The most likely reason for file
corruption is breaking into a task at the wrong moment, or trying
to write to a file at the same time.

If you want to process simultaneously from different sessions---say
one interactive and another in batch---it is wise to redefine the
environment variables \texttt{ADAM\_USER}, and \texttt{AGI\_USER} if you want
graphics on the same machine.  The environment variables should point
to a separate existing directory for each additional session.  This
will keep the \htmlref{global}{se:parglobals} and
\htmlref{application parameters}{se:defaults}, and the
\htmlref{graphics database}{se:agitate} separate for each session.

The way to look for corrupted HDS files is trace them.
Assuming that \$ADAM\_USER and \$AGI\_USER are defined,

\begin{terminalv}
     % hdstrace $ADAM_USER/GLOBALS full
     % hdstrace $ADAM_USER/ardmask full
     % hdstrace $AGI_USER/agi_cacvad full
\end{terminalv}
traces the \file{GLOBALS} file, the application you were running when the
weird error occurred (here \htmlref{ARDMASK}{ARDMASK}), and the
graphics database for machine \texttt{cacvad}.  Once you have identified
the problem file, delete it.  If that proves to be the globals file,
you might want to retain the output from \HDSTRACEref, so that you can
restore their former values.  Deleting the graphics database is
something you should do regularly, so that's not a problem.

If you have been running \KAPPA\ from \ICLref\normalsize, you will need
to check of the integrity of the monolith parameter file, instead
the individual parameter file.  It will be one of these depending
on the type of task that failed: graphics, NDF components, or
the rest (mostly image processing) corresponding to these three
monolith interface files.

\begin{terminalv}
     % hdstrace $ADAM_USER/kapview_mon full
     % hdstrace $ADAM_USER/ndfpack_mon full
     % hdstrace $ADAM_USER/kappa_mon full
\end{terminalv}


If that doesn't cure the problem, send a log of
the session demonstrating the problem to the Starlink Software support
mailing list (\texttt{starlink@jiscmail.ac.uk}), and we shall endeavour to
interpret it for you, and find out what's going wrong.

\newpage
\section{\xlabel{se_custom}Custom KAPPA\label{se:custom}}

\subsection{\xlabel{se_custom}Tasks\label{se:customtasks}}

\KAPPA\ applications can be modified to suit your particular
requirements.  Since this document is not a programmer's guide,
instructions are not given here.  Programmers should contact the
author for details until a new Programmer's Guide appears to replace
the old \xref{SUN/101}{sun101}{}, which {\em was\/} a good summary of
Starlink infrastructure libraries and programming.

All the source files can be found in \file{/star/kappa/*.tar} on
Starlink machines.  The \file{/star} path may be different outside of
Starlink, so check with your computer manager.  There is a separate
tar file for each \KAPPA\ subroutine library (with a \texttt{
\_sub} suffix) and the interface files, with obvious names.  The
remaining files: the monolith routines, link scripts, include files,
the help source, shell scripts, \ICL\ procedures, and test data
are in \file{kappa\_source.tar}.  There is also a Starlink standard
\texttt{makefile} and \texttt{mk} script.

Many of the general-purpose subroutines which previously formed part of
\KAPPA\ have now been moved into a separate software item
called \KAPLIBS\ (see \xref{SUN/238}{sun238}{}).
\KAPPA\ itself now links against the libraries
in \KAPLIBS\ .

Here is a worked example.  Suppose that you have
\htmlref{\_REAL}{ap:HDStypes}-type datasets for which you want to
compute statistics including the skewness and kurtosis.  One way is to
modify STATS.  First to save typing define environment variables, say
\texttt{STAR} and \texttt{KAPPA} and \texttt{KAPLIBS} to point to where the
Starlink software, \KAPPA\ and \KAPLIBS\  source is stored.  Next
we extract the source files to change.

\begin{terminalv}
     % setenv STAR /star
     % setenv KAPPA /star/sources/kappa
     % setenv KAPLIBS /star/sources/kaplibs
     % tar xf $KAPPA/kappa_sub.tar stats.
     % tar xf $KAPPA/kappa_ifls.tar stats.ifl
     % tar xf $KAPLIBS/kapgen_sub.tar kpg1_statr.f
     % tar xf $KAPLIBS/kapgen_sub.tar kpg1_stdsr.f
     % tar xf $KAPPA/kappa_source.tar kappa_link_adam
\end{terminalv}

We modify \file{kpg1\_statr.f} to compute the additional statistics;
\file{kpg1\_stdsr.f} to list the statistics; \file{stats.f} to update the
documentation, to use the revised argument lists of the subroutines,
and to output the new statistics to parameters; and \file{stats.ifl}
to add the output parameters.  \file{kappa\_link\_adam} need not be
modified, but it is needed during linking.

Next some soft links to include files need to be made.

\begin{terminalv}
     % star_dev
     % ndf_dev
     % prm_dev
     % par_dev
     % kaplibs_dev
\end{terminalv}
For some other application and subroutines, you can find what is
needed by trying to compile them and see which include files the
compiler cannot locate.  You then invoke the appropriate package
definitions: {\em pkg}\texttt{\_dev}, where {\em pkg\/} is the three-letter
package abbreviation.  Now compile the modified code.  This is for
OSF/1:

\begin{terminalv}
     % f77 -O -c -nowarn stats.f kpg1_statr.f kpg1_stdsr.f
\end{terminalv}
and this is for Solaris:

\begin{terminalv}
     % f77 -O -PIC -c -w stats.f kpg1_statr.f kpg1_stdsr.f
\end{terminalv}
The \texttt{-nowarn} and \texttt{-w} prevent warning messages appearing.

And this is for Linux:

\begin{terminalv}
     % g77 -fno-second-underscore -O -c stats.f kpg1_statr.f kpg1_stdsr.f
\end{terminalv}

Now link the task to produce a new \texttt{stats} executable.

\begin{terminalv}
     % alink stats.o -o stats kpg1_statr.o kpg1_stdsr.o \
     -L$STAR/lib `./kappa_link_adam`
\end{terminalv}

If you want to use \KAPPA\ subroutines for your own application
here are words of warning: {\em the code may undergo alterations of
subroutine name or argument lists, and those without a {\em pkg\_\/}
prefix will either be replaced or renamed.} Therefore, you should copy
the modules you need.

\subsection{\xlabel{se_custom}Parameters\label{se:custompar}}

If you don't like \KAPPA's parameter defaults, or its choice
of which parameters get prompted for and which get defaulted, then you
can change them.  Extract the interface file from \linebreak
\file{/star/kappa/kappa\_ifls.tar} to your work directory and make the
required modifications, and then recompile it.  See
\xref{SUN/115}{sun115}{} on the meanings and possible values of the
fieldnames, and how to recompile the interface file.  If you use
\ICLref , you'll need to modify a monolith interface file: \file{
\$KAPPA\_DIR/kappa\_mon.ifl}, \file{kapview\_mon.ifl} or \file{
ndfpack\_mon.ifl}.  Finally, you will need to specify a search path
that includes the directory containing your modified interface file.

\begin{terminalv}
     % setenv ADAM_IFL /home/scratch/dro/ifls:/usr/local/kappa
\end{terminalv}
This asks Starlink programmes to look in \file{/home/scratch/dro/ifls}
to find the interface file, and if there isn't one to look in \file{
/usr/local/kappa}.  If the interface file search is unsuccessful, the
directory containing the executable is assumed.  Thus if you've not
created your own interface file for a task, you'll get the released
version.  Of course, once you have done this, the documentation in
\latexhtml{Appendix~\ref{ap:full}}{the \htmlref{application
specifications}{ap:full}} will no longer be correct.

\subsection{\xlabel{se_custom}Commands\label{se:customcom}}

There is an easier method of tailoring \KAPPA\ to your
requirements.  If you frequently use certain commands, especially
those with a long list of keywords and fixed values, you can define
some C-shell aliases or \ICLref\ symbols for the commands.  Like
the shell's \file{\$HOME/.login}, \ICL\ has a {\em login file}.
(See \xref{``ICL for Unix'' Appendix in SUN/144}{sun144}{icl_for_unix}, or
\xref{SSN/64}{ssn64}{} for details.)  If you add symbols to this file,
each time you activate \ICL\ these abbreviations will be
available to you without further typing.  What you should do is to
create a \file{login.icl} in a convenient directory, and assign the
environment variable \texttt{ICL\_LOGIN} to that directory in your \file{
\$HOME/.login} file.

It is possible to have several \ICL\ login files, each for
different work in different directories.  Now to abbreviate a command
you put a DEFSTRING entry into the \ICL\ login file.  For
example,

\begin{terminalv}
     DEFSTRING MYC{ON} CONTOUR CLEAR=F PENROT MODE=AU NCONT=7
\end{terminalv}

defines \texttt{MYC} or \texttt{MYCO} or \texttt{MYCON} to run
\htmlref{CONTOUR}{CONTOUR} without clearing the screen, with pen
rotation and seven equally spaced contour heights.  The symbols are not
limited to \KAPPA.  Indeed they can include shorthands for
shell commands.  For example,

\begin{terminalv}
     DEFSTRING DA ls -al
\end{terminalv}
would make DA produce a directory listing of all files with sizes and
modification dates.

You can check what the current login files are as follows.

\begin{terminalv}
     % printenv | grep ICL_LOGIN
\end{terminalv}

For shell usage similar definitions can be made with aliases. For
example,

\begin{terminalv}
     % alias mycon contour clear=f penrot mode=au ncont=7
\end{terminalv}

is the equivalent of the DEFSTRING above, except that in keeping with
UNIX tradition the command is in lowercase, and the alias cannot be
abbreviated.

\section{Acknowledgments}

Several people have contributed complete \KAPPA
programmes, or have upgraded earlier versions, or have written original
code (which eventually became included in \KAPPA\ after
reworking).  Mark Taylor and Rodney Warren-Smith both re-wrote some of the old
IMAGE applications to use the NDF\_ library. Rodney also supplied several other
programmes, especially ones that now form the basis of \NDFPACK.  Other
contributions have come from Alasdair Allan, Steven Beard,
Wei Gong, Rhys Morris, Jo Murray, Grant Privett, and Richard Saxton.  The
original \KAPPA\ was derived from Mark McCaughrean's
{\footnotesize RAPI2D} and Ken Hartley's {\footnotesize ASPIC Kernel},
though little remains.  Thanks also to Rodney Warren-Smith for permitting
this document to include a few modified pages of his SUN/33 on NDF
sections and co-ordinate systems; and to many useful suggestions from
users and programmers over the years including Chris Clayton, Peter
Draper, Jim Emerson, Horst Meyerdierks, Andy Scott, and Martin Shaw.
Mike Lawden helped produce the quick-reference card.

\section{Acknowledging this Software}
Please acknowledge the use of this software in any publications arising
from research in which it has played a significant r\^{o}le.  Please also
acknowledge the use of any other Starlink software in such publications.
The following is suggested as a suitable form of words:

\begin{center}
\begin{quote}
\emph{The authors acknowledge the use of the following software provided by
the UK Starlink Project: KAPPA,... Starlink is run by CCLRC on behalf of PPARC. }
\end{quote}
\end{center}

\newpage
\appendix


\section{\xlabel{ap_classified}Classified KAPPA commands
\label{ap:classified}}

\KAPPA\ applications may be classified in terms of their
functions as follows.


\subsection{DATA IMPORT \& EXPORT}
\label{cl:datagen}


\subsubsection{Image generation and input}

\begin{aligndesc}
\classitem{CREFRAME}
 Generates a test two-dimensional NDF from a selection of several types.
\classitem{FITSDIN}
 Reads a {\FITSref}~ disc file composed of simple, group or table objects.
\classitem{FITSHEAD}
 Lists the headers of FITS files.
\classitem{FITSIMP}
 Imports FITS information into an \htmlref{NDF extension}{apndf:extensions}.
\classitem{FITSIN}
 Reads a FITS tape composed of simple, group or table files.
\classitem{MATHS}
 Evaluates mathematical expressions applied to NDF data structures.
\classitem{TRANDAT}
 Converts free-format data into an NDF.
\end {aligndesc}


\subsubsection{Preparation for output}

\begin{aligndesc}
\classitem{FITSEDIT}
 Edits the FITS extension of an NDF.
\classitem{FITSEXP}
 Exports NDF-extension information into an NDF FITS extension.
\classitem{FITSMOD}
 Edits an NDF FITS extension via a text file or parameters.
\classitem{FITSTEXT}
 Creates an NDF FITS extension from a text file.
\classitem{FITSWRITE}
 Writes a new keyword to the FITS extension.
\end {aligndesc}




\subsection{\xlabel{cl_datadisplay}DATA DISPLAY}
\label{cl:datadisplay}


\subsubsection{Detail enhancement}


\begin{aligndesc}
\classitem{CARPET}
 Creates a cube representing a carpet plot of an image.
\classitem{COLCOMP}
 Produces a colour composite image from 1, 2 or 3 individual NDFs.
\classitem{HISTEQ}
 Performs an histogram equalisation on an NDF.
\classitem{LAPLACE}
 Performs a Laplacian convolution as an edge detector in a two-dimensional NDF.
\classitem{SHADOW}
 Enhances edges in a two-dimensional NDF using a shadow effect.
\classitem{THRESH}
 Edits an NDF such that array values below and above two thresholds take
 constant values.
\end{aligndesc}


\subsubsection{Device selection}

\begin{aligndesc}
\classitem{GDNAMES}
 Shows which graphics devices are available.
\classitem{GDSET}
 Selects a \htmlref{current graphics device}{se:devglobal}.
\end{aligndesc}

\subsubsection{Display control}

\begin{aligndesc}
\classitem{CURSOR}
 Reports the co-ordinates of points selected using the cursor.
\classitem{GDCLEAR}
 Clears a graphics device and purges its database entries.
\classitem{GDSTATE}
 Shows the current status of a graphics device.
\end{aligndesc}

\subsubsection{Graphics Database}

\begin{aligndesc}
\classitem{PICBASE}
 Selects the BASE picture from the \htmlref{graphics database}{se:agitate}.
\classitem{PICCUR}
 Uses a cursor to select the current picture.
\classitem{PICDATA}
 Selects the last DATA picture from the graphics database.
\classitem{PICDEF}
 Defines a new graphics-database FRAME picture or an array of FRAME pictures.
\classitem{PICEMPTY}
 Finds the first empty FRAME picture in the graphics database.
\classitem{PICENTIRE}
 Finds the first unobscured and unobscuring FRAME picture in the graphics database.
\classitem{PICFRAME}
 Selects the last FRAME picture from the graphics database.
\classitem{PICGRID}
 Creates an array of FRAME pictures.
\classitem{PICIN}
 Finds the attributes of a picture interior to the current picture.
\classitem{PICLABEL}
 Labels the current graphics-database picture.
\classitem{PICLAST}
 Selects the last picture from the graphics database.
\classitem{PICLIST}
 Lists the pictures in the graphics database for a device.
\classitem{PICSEL}
 Selects a graphics-database picture by its label.
\classitem{PICTRANS}
 Transforms co-ordinates between the current and BASE pictures.
\classitem{PICVIS}
 Finds the first unobscured FRAME picture in the graphics database.
\classitem{PICXY}
 Creates a new picture defined by co-ordinate bounds.
\end{aligndesc}

\subsubsection{Lookup/Colour tables}

\begin{aligndesc}
\classitem{LUTABLE}
 Manipulates an graphics device \htmlref{colour table}{se:coltab}.
\classitem{LUTBGYRW}
 Loads the {\it BGYRW\/} \htmlref{lookup table}{se:lookuptables}.
\classitem{LUTCOL}
 Loads the standard colour lookup table.
\classitem{LUTCOLD}
 Loads the {\it cold\/} lookup table.
\classitem{LUTCONT}
 Loads a lookup table to give the display the appearance of a contour plot.
\classitem{LUTEDIT}
 Creates or edits an graphics device colour table.
\classitem{LUTFC}
 Loads the standard false-colour lookup table.
\classitem{LUTGREY}
 Loads the standard grey-scale lookup table.
\classitem{LUTHEAT}
 Loads the {\it heat\/} lookup table.
\classitem{LUTIKON}
 Loads the default {\it Ikon\/} lookup table.
\classitem{LUTNEG}
 Loads the standard negative grey-scale lookup table.
\classitem{LUTRAMPS}
 Loads the coloured-ramps lookup table.
\classitem{LUTREAD}
 Loads an graphics device lookup table from an NDF.
\classitem{LUTSAVE}
 Saves the current colour table of an graphics device in an NDF.
\classitem{LUTSPEC}
 Loads a spectrum-like lookup table.
\classitem{LUTVIEW}
 Draws a colour-table key.
\classitem{LUTWARM}
 Loads the {\it warm\/} lookup table.
\classitem{LUTZEBRA}
 Loads a pseudo-contour lookup table.
\end{aligndesc}


\subsubsection{Output}

\begin{aligndesc}
\classitem{ARDPLOT}
 Plots the boundaries of regions described in an \htmlref{ARD file}{se:ardwork}~ over an existing picture.
\classitem{CLINPLOT}
 Draws a spatial grid of line plots for an axis of a cube NDF.
\classitem{CONTOUR}
 Contours a two-dimensional NDF.
\classitem{DISPLAY}
 Displays a one- or two-dimensional NDF.
\classitem{DRAWNORTH}
 Draws arrows parallel to the axes.
\classitem{DRAWSIG}
 Draws ${\pm}n$ standard-deviation lines on a line plot.
\classitem{ELPROF}
 Creates a radial or azimuthal profile of a two-dimensional image.
\classitem{LINPLOT}
 Draws a line plot of the data values in a one-dimensional NDF.
\classitem{LISTSHOW}
 Displays the positions stored in a positions list.
\classitem{LOOK}
 Outputs the values of specified NDF pixels to the screen or a text file.
\classitem{MLINPLOT}
 Draws a multi-line plot of the data values in a two-dimensional NDF.
\classitem{OUTLINE}
 Draws the outline of a two-dimensional NDF.
\classitem{SCATTER}
 Displays a scatter plot between data in two NDFs.
\classitem{VECPLOT}
 Plots a two-dimensional vector map.
\end{aligndesc}


\subsubsection{Palette}

\begin{aligndesc}
\classitem{PALDEF}
 Loads the default \htmlref{palette}{se:palette}~ to a colour table.
\classitem{PALENTRY}
 Enters a colour into an graphics device's palette.
\classitem{PALREAD}
 Fills the palette of a colour table from an NDF.
\classitem{PALSAVE}
 Saves the current palette of a colour table to an NDF.
\end{aligndesc}

\subsection{DATA MANIPULATION}



\subsubsection{Arithmetic}

\begin{aligndesc}
\classitem{ADD}
 Adds two NDF data structures.
\classitem{CADD}
 Adds a scalar to an NDF data structure.
\classitem{CDIV}
 Divides an NDF by a scalar.
\classitem{CMULT}
 Multiplies an NDF by a scalar.
\classitem{CSUB}
 Subtracts a scalar from an NDF data structure.
\classitem{CUMULVEC}
 Sums the values cumulatively in a one-dimensional NDF.
\classitem{DIV}
 Divides one NDF data structure by another.
\classitem{EXP10}
 Takes the base-10 exponential of each pixel of an NDF.
\classitem{EXPE}
 Takes the exponential of each pixel of an NDF (base $e$).
\classitem{EXPON}
 Takes the exponential (specified base) of each pixel of am NDF.
\classitem{LOG10}
 Takes the base-10 logarithm of each pixel of an NDF.
\classitem{LOGAR}
 Takes the logarithm of each pixel of an NDF (specified base).
\classitem{LOGE}
 Takes the natural logarithm of each pixel of an NDF.
\classitem{MAKESNR}
 Creates a signal-to-noise array from an NDF with Variance.
\classitem{MATHS}
 Evaluates mathematical expressions applied to NDF data structures.
\classitem{MULT}
 Multiplies two NDF data structures.
\classitem{POW}
 Takes the specified power of each pixel of a data array.
\classitem{SUB}
 Subtracts one NDF data structure from another.
\classitem{TRIG}
 Performs a trigonometric transformation on an NDF.
\end {aligndesc}


\subsubsection{Combination}

\begin{aligndesc}
\classitem{CALPOL}
 Calculates polarisation parameters.
\classitem{COLCOMP}
 Produces a colour composite image from 1, 2 or 3 individual NDFs.
\classitem{COMPLEX}
 Converts between representations of complex data.
\classitem{INTERLEAVE}
 Forms a higher-resolution NDF by interleaving a set of NDFs.
\classitem{KSTEST}
 Compares data sets using the Kolmogorov-Smirnov test.
\classitem{NORMALIZE}
 Normalises one NDF to a similar NDF by calculating a scale factor and zero
 difference.
\classitem{WCSMOSAIC}
 Tiles a group of NDFs using World Co-ordinate System information.
\end{aligndesc}


\subsubsection{Compression and expansion}
\label{cl:compexp}

\begin{aligndesc}
\classitem{CARPET}
 Creates a cube representing a carpet plot of an image.
\classitem{CHANMAP}
 Creates a channel map from a cube NDF by compressing slices along a nominated axis
\classitem{COLLAPSE}
 Reduces the number of axes in an NDF by collapsing it along a nominated axis.
\classitem{COMPADD}
 Reduces the size of an NDF by adding values in rectangular boxes.
\classitem{COMPAVE}
 Reduces the size of an NDF by averaging values in rectangular boxes.
\classitem{COMPICK}
 Reduces the size of an NDF by picking equally spaced pixels.
\classitem{INTERLEAVE}
 Forms a higher-resolution NDF by interleaving a set of NDFs.
\classitem{NDFCOMPRESS}
 Compresses an NDF so that it occupies less disk space.
\classitem{PIXDUPE}
 Expands an NDF by pixel duplication.
\classitem{PLUCK}
 Plucks slices from an NDF at arbitrary positions.
\classitem{REGRID}
 Uses an arbitrary mapping to regrid an NDF.
\classitem{SQORST}
 Squashes or stretches an NDF.
\classitem{WCSALIGN}
 Aligns a group of NDFs using \htmlref{WCS}{apndf:wcs}~ information.
\end{aligndesc}


\subsubsection{Configuration change}

\begin{aligndesc}
\classitem{CHAIN}
 Concatenates a series of vectorized NDFs.
\classitem{FLIP}
 Reverses an NDF's pixels along a specified dimension.
\classitem{MANIC}
 Converts all or part of an NDF from one dimensionality to another.
\classitem{NDFCOPY}
 Copies an NDF (or \htmlref{NDF section}{se:ndfsect}) to a new location.
\classitem{PERMAXES}
 Permutes the axes of an NDF.
\classitem{PIXBIN}
 Places each pixel value in an input NDF into an output bin.
\classitem{PLUCK}
 Plucks slices from an NDF at arbitrary positions.
\classitem{PROFILE}
 Creates a one-dimensional profile through an \textit{n}-dimensional NDF.
\classitem{REGRID}
 Uses an arbitrary mapping to regrid an NDF.
\classitem{RESHAPE}
 Reshapes an NDF, treating its arrays as vectors.
\classitem{ROTATE}
 Rotates a two-dimensional NDF about its centre through any angle.
\classitem{SETBOUND}
 Sets new bounds for an NDF.
\classitem{SLIDE}
 Shifts pixels in an NDF by a given amount along each axis.
\classitem{WCSSLIDE}
 Applies a translational correction to the WCS in an NDF.
\end{aligndesc}


\subsubsection{Filtering}

\begin{aligndesc}
\classitem{BLOCK}
 Smooths an NDF using an n-dimensional rectangular box filter.
\classitem{CONVOLVE}
 Convolves a pair of one- or two-dimensional NDFs together.
\classitem{FFCLEAN}
 Removes defects from a substantially flat one- or two-dimensional NDF.
\classitem{FOURIER}
 Performs forward and inverse Fourier transforms of one- or two-dimensional NDFs.
\classitem{GAUSMOOTH}
 Smooths a one- or two-dimensional image using a Gaussian filter.
\classitem{LUCY}
 Performs a Richardson-Lucy deconvolution of a one- or two-dimensional array.
\classitem{MEDIAN}
 Smooths a two-dimensional data array using a weighted median filter.
\classitem{MEM2D}
 Performs a Maximum-Entropy deconvolution of a two-dimensional NDF.
\classitem{ODDEVEN}
 Removes odd-even defects from a one-dimensional NDF.
\classitem{WIENER}
 Applies a Wiener filter to a one- or two-dimensional array.
\end{aligndesc}

\subsubsection{HDS components}

\begin{aligndesc}
\classitem{ERASE}
 Erases an \HDSref\ object.
\classitem{NATIVE}
 Converts an HDS object to native machine data representation.
\end{aligndesc}


\subsubsection{NDF array components}
\begin{aligndesc}
\classitem{NDFCOMPRESS}
 Compresses an NDF so that it occupies less disk space.
\classitem{NDFCOPY}
 Copies an NDF (or NDF section) to a new location.
\classitem{PERMAXES}
 Permutes the axes of an NDF.
\classitem{QUALTOBAD}
 Assigns \htmlref{bad values}{se:masking}~ to pixels with given qualities.
\classitem{REMQUAL}
 Removes \htmlref{named qualities}{se:qnames}~ stored in an NDF QUALITY component.
\classitem{SETBAD}
 Sets new \htmlref{bad-pixel flag}{setbad:badpixelflag}~ values for an NDF.
\classitem{SETBB}
 Sets a new value for the quality \htmlref{bad-bits mask}{se:qualitymask}~ of an NDF.
\classitem{SETBOUND}
 Sets new bounds for an NDF.
\classitem{SETORIGIN}
 Sets a new \htmlref{pixel origin}{apndf:origin}~ for an NDF.
\classitem{SETQUAL}
 Assigns a specified quality to selected pixels within an NDF.
\classitem{SETTYPE}
 Sets a new numeric type for the DATA and VARIANCE components of an NDF.
\classitem{SETVAR}
 Sets new values for the \htmlref{VARIANCE component}{apndf:variance}~ of an NDF data structure.
\classitem{SHOWQUAL}
 Displays the named qualities stored in an NDF \htmlref{QUALITY component}{apndf:quality}.
\end{aligndesc}

\subsubsection{NDF axis components}
\begin{aligndesc}
\classitem{AXCONV}
 Expands spaced axes in an NDF into the primitive form.
\classitem{AXLABEL}
 Sets a new label value for an axis within an NDF data structure.
\classitem{AXUNITS}
 Sets a new units value for an axis within an NDF data structure.
\classitem{PERMAXES}
 Permutes the axes of an NDF.
\classitem{SETAXIS}
 Sets values for an \htmlref{axis array component}{apndf:axis} within an NDF data structure.
\classitem{SETNORM}
 Sets a new value for one or all of an NDF's axis-normalisation flags.
\end{aligndesc}

\subsubsection{NDF character components}
\begin{aligndesc}
\classitem{SETLABEL}
 Sets a new \htmlref{label}{apndf:label}~ for an NDF data structure.
\classitem{SETTITLE}
 Sets a new \htmlref{title}{apndf:title}~ for an NDF data structure.
\classitem{SETUNITS}
 Sets a new \htmlref{units}{apndf:units}~ value for an NDF data structure.
\end{aligndesc}

\subsubsection{NDF extensions}
\begin{aligndesc}
\classitem{FITSEDIT}
 Edits the \htmlref{FITS extension}{se:fitsairlock}~ of an NDF.
\classitem{FITSEXIST}
 Inquires whether or not a keyword exists in a FITS extension.
\classitem{FITSEXP}
 Exports NDF-extension information into an NDF FITS extension.
\classitem{FITSLIST}
 Lists the FITS extension of an NDF.
\classitem{FITSMOD}
 Edits an NDF FITS extension via a text file or parameters.
\classitem{FITSTEXT}
 Creates an NDF FITS extension from a text file.
\classitem{FITSVAL}
 Reports the value of a keyword in the FITS extension.
\classitem{FITSWRITE}
 Writes a new keyword to the FITS extension.
\classitem{SETEXT}
 Manipulates the contents of a specified NDF \htmlref{extension}{apndf:extensions}.
\classitem{SETSKY}
 Stores \htmlref{WCS}{apndf:wcs}~ Information in an NDF.
\end{aligndesc}

\subsubsection{NDF History}
\begin{aligndesc}
\classitem{HISCOM}
 Adds commentary to the \htmlref{history}{se:ndfhistory} of an NDF.
\classitem{HISLIST}
 Lists NDF \htmlref{history records}{apndf:history}.
\classitem{HISSET}
 Sets the NDF history update mode.
\end{aligndesc}




\subsubsection{NDF Provenance}
\label{cl:prov}

\begin{aligndesc}
\classitem{PROVADD}
 Stores \htmlref{provenance information}{apndf:provenance} in an NDF.
\classitem{PROVMOD}
 Modifies provenance information for an NDF.
\classitem{PROVREM}
 Removes selected provenance information from an NDF.
\classitem{PROVSHOW}
 Displays provenance information for an NDF.
\end{aligndesc}


\subsubsection{NDF World Co-ordinate Systems}
\begin{aligndesc}
\classitem{PERMAXES}
 Permutes the axes of an NDF.
\classitem{WCSADD}
Creates a Mapping and optionally adds a new \htmlref{co-ordinate Frame}{se:domains}~ into the
WCS component of an NDF.
\classitem{WCSALIGN}
 Aligns a group of NDFs using \htmlref{WCS information}{se:wcsuse}.
\classitem{WCSATTRIB}
 Manages attribute values associated with the \htmlref{WCS component}{apndf:wcs}~ of an NDF.
\classitem{WCSCOPY}
 Copies WCS information from one NDF to another.
\classitem{WCSFRAME}
 Changes the current co-ordinate Frame in the WCS component of an NDF.
\classitem{WCSMOSAIC}
 Tiles a group of NDFs using World Co-ordinate System information.
\classitem{WCSREMOVE}
 Removes co-ordinate Frames from the WCS component of an NDF.
\classitem{WCSSHOW}
 Examines the internal structure of a WCS description.
\classitem{WCSSLIDE}
 Applies a translational correction to the WCS in an NDF.
\classitem{WCSTRAN}
 Transforms a position from one NDF co-ordinate Frame to another.
\end{aligndesc}



\subsubsection{Pixel editing and masking}
\label{cl:pixedit}
\begin{aligndesc}
\classitem{ARDGEN}
 Creates a text file describing selected regions of an image.
\classitem{ARDMASK}
 Uses an \htmlref{ARD file}{se:ardwork}~ to set some pixels of an NDF to be bad.
\classitem{ARDPLOT}
 Plots the boundaries of regions described in an ARD file over an existing picture.
\classitem{CHPIX}
 Replaces the values of selected pixels in an NDF.
\classitem{COPYBAD}
 Copies the \htmlref{bad-pixel}{se:masking}~ mask from one NDF to another.
\classitem{ERRCLIP}
 Removes pixels with large errors from an NDF.
\classitem{EXCLUDEBAD}
 Copies a two-dimensional NDF excluding any bad rows or columns.
\classitem{FFCLEAN}
 Removes defects from a substantially flat one- or two-dimensional NDF.
\classitem{FILLBAD}
 Removes regions of bad values from an NDF.
\classitem{GLITCH}
 Replaces bad pixels in a two-dimensional image with the local median.
\classitem{MFITTREND}
 Fits independent trends to data lines that are parallel to an axis.
\classitem{MOCGEN}
 Creates a Multi-Order Coverage (MOC) map describing regions of an image.
\classitem{NOMAGIC}
 Replaces all occurrences of \htmlref{magic-value}{se:masking}~ pixels in an NDF array with
 a new value.
\classitem{OUTSET}
 Sets pixels outside a specified circle in a two-dimensional NDF to a specified
 value.
\classitem{PASTE}
 Pastes a series of NDFs upon each other.
\classitem{REGIONMASK}
 Applies a mask to a region of an NDF.
\classitem{RIFT}
 Adds a scalar to a section of an NDF data structure to correct rift-valley defects.
\classitem{SEGMENT}
 Copies polygonal segments from one NDF to another.
\classitem{SETMAGIC}
 Replaces all occurrences of a given value in an NDF array
 with the bad value.
\classitem{SUBSTITUTE}
 Replaces all occurrences of a given value in an NDF array with another value.
\classitem{THRESH}
 Edits an NDF such that array values below and above two thresholds take
\classitem{ZAPLIN}
 Replaces regions in a two-dimensional NDF by bad values or by linear
 interpolation.
\end{aligndesc}


\subsubsection{Polarimetry}
\begin{aligndesc}
\classitem{CALPOL}
 Calculates polarisation parameters.
\end{aligndesc}


\subsubsection{Resampling and transformations}
\label{cl:regrid}

\begin{aligndesc}
\classitem{ALIGN2D}
 Aligns a pair of two-dimensional NDFs by minimising the residuals between them.
\classitem{PIXBIN}
 Places each pixel value in an input NDF into an output bin.
\classitem{PLUCK}
 Plucks slices from an NDF at arbitrary positions.
\classitem{REGRID}
 Uses an arbitrary mapping to regrid an NDF.
\classitem{WCSALIGN}
 Aligns a group of NDFs using \htmlref{WCS}{apndf:wcs}~ information.
\classitem{WCSMOSAIC}
 Tiles a group of NDFs using World Co-ordinate System information.
\end{aligndesc}



\subsubsection{Surface and vector fitting}
\label{cl:surfit}

\begin{aligndesc}
\classitem{FITSURFACE}
 Fits a polynomial surface to two-dimensional data array.
\classitem{MAKESURFACE}
 Creates a two-dimensional NDF from the coefficients of a polynomial surface.
\classitem{MFITTREND}
 Fits independent trends to data lines that are parallel to an axis.
\classitem{SURFIT}
 Fits a polynomial or spline surface to a two-dimensional data array using blocking.
\end{aligndesc}



\subsection{DATA ANALYSIS}


\subsubsection{Statistics}

\begin{aligndesc}
\classitem{APERADD}
 Derives statistics of pixels within a specified aperture of an NDF.
\classitem{HISTAT}
 Computes ordered statistics for an NDF's pixels using an histogram.
\classitem{HISTOGRAM}
 Computes an histogram of an NDF's values.
\classitem{MSTATS}
  Does cumulative statistics over a sequence of NDFs.
\classitem{NUMB}
 Counts the number of elements of an NDF with values or absolute values above
 or below a threshold.
\classitem{STATS}
 Computes simple statistics for an NDF's pixels.
\end{aligndesc}


\subsubsection{Other}

\begin{aligndesc}
\classitem{BEAMFIT}
 Fits beam features in a two-dimensional NDF.
\classitem{CENTROID}
 Finds the centroids of star-like features in an NDF.
\classitem{NORMALIZE}
 Normalises one NDF to a similar NDF by calculating a scale factor
 and zero-point difference.
\classitem{PSF}
 Determines the parameters of a model star profile by fitting star images
 in a two-dimensional NDF.
\classitem{SURFIT}
 Fits a polynomial or spline surface to a two-dimensional data array.
\end{aligndesc}




\subsection{SCRIPTING TOOLS}

\begin{aligndesc}
\classitem{CALC}
 Evaluates a mathematical expression.
\classitem{CONFIGECHO}
 Displays a named parameter from a group of configuration parameters.
\classitem{NDFECHO}
 Expands a group expression into a list of explicit NDF names.
\classitem{PARGET}
 Obtains the value or values of an \htmlref{application parameter}{se:parout}.
\end{aligndesc}


\subsection{INQUIRIES \& STATUS}

\begin{aligndesc}
\classitem{GLOBALS}
 Displays the values of the \KAPPA\ \htmlref{global parameters}{se:parglobals}.
\classitem{FITSEXIST}
 Inquires whether or not a keyword exists in a FITS extension.
\classitem{FITSLIST}
 Lists the \htmlref{FITS extension}{se:fitsairlock}~ of an NDF.
\classitem{FITSVAL}
 Reports the value of a keyword in the FITS extension.
\classitem{NDFCOMPARE}
Compares a pair of NDFs for equivalence.
\classitem{NDFTRACE}
 Displays the attributes of an \htmlref{NDF data structure}{ap:NDFformat}.
\classitem{NOGLOBALS}
 Resets the \KAPPA\ global parameters.
\end{aligndesc}


\subsection{MISCELLANEOUS}

\begin{aligndesc}
\classitem{COMPLEX}
 Converts between representations of complex data.
\classitem{KAPHELP}
 Gives help about \KAPPA.
\classitem{LISTMAKE}
 Creates a catalogue holding a positions list.
\classitem{KAPVERSION}
 Checks the version number of the installed package.
\end{aligndesc}


\section{\xlabel{ap_quotas}Quotas to run KAPPA\label{ap:quotas}}

No special quotas are needed to run \KAPPA.  If you have large
datasets you might need to increase the datasize limit in the C-shell.

\begin{terminalv}
    % limit datasize 65336
\end{terminalv}
sets the maximum size of a data file to 64 megabytes.  To list the
current values use the {\bf limit} command without any arguments.

\newpage
\section{\xlabel{ap_full}Specifications of KAPPA applications\label{ap:full}}
\subsection{Explanatory Notes}

The specification of parameters has the following format.

\begin{terminalv}
     name  =  type (access)
        description
\end{terminalv}
This format also includes a {\em Usage\/} entry.  \label{ap:usage}
This shows how the application is invoked from the command line.  It
lists the positional parameters in order followed by any prompted
keyword parameters using a \mbox{``KEYWORD=?''} syntax.  Defaulted
keyword parameters do not appear.  Positional parameters that are
normally defaulted are indicated by being enclosed in square brackets.
Keyword (\emph{i.e.} not positional) parameters are needed where the
number of parameters are large, and usually occur because they depend
on the value of another parameter.  These are denoted by a curly
brace; the parameters on each line are related, and each line is
mutually exclusive.  An example should clarify.
%\bigskip

     \begin{tabular}{c l}
      \texttt{contour ndf [comp] mode ncont [key] [device]} &
      {  $\begin{cases}
          \mathtt{low=?\ high=?} \\
          \mathtt{percentiles=?} \\
          \mathtt{sigmas=?}
        \end{cases} $}\\
      & \texttt{mode}\\
    \end{tabular}


NDF, COMP, MODE, NCONT, KEY, DEVICE, and SMOOTHING are all positional
parameters.  Only NDF, MODE, and NCONT would be prompted if not given
on the command line.  The remaining parameters depend on the value of
MODE.  If the mode is to nominate a list of contour heights, HEIGHTS
will be needed (MODE~=~\texttt{"Free"}); alternatively, if the mode requires a
start height and spacing between contours FIRSTCNT and STEPCNT should be
specified (MODE~=~\texttt{"Linear"} or \texttt{"Magnitude"}).  Note that there
are other modes that do not require additional information, and hence no
more parameters.

There is also an {\em Examples\/} section. \label{ap:example} This
shows how to run the application from the command line.  More often
you'll enter the command name and just some of the parameters, and be
prompted for the rest.  {\em Note that the examples are the strings
expected by the tasks.} They are operating-system neutral as
\KAPPA\ has run on several different operating systems.  {\em UNIX shells
or operating-system command languages will often interpret as special
characters some or all of \verb+[]()\^~"'$*?+ that may form part of
the \KAPPA\ command-line syntax.  So in practice you should escape any
such special characters that appear in these examples, as appropriate
to your command language or shell.} For instance, from the C-shell the
fourth example of COMPAVE could be written like the following.

\begin{terminalv}
     compave cosmos galaxy '[4,3]' weight title='"COSMOS compressed"'
     compave cosmos galaxy \[4,3\] weight title=\"COSMOS compressed\"
\end{terminalv}

Backslash escapes individual special characters, whereas quotes placed
around text escape all occurrences of special characters within the
quotes.

\medskip

Some parameters will only be used when another parameter has a certain
value or mode.  These are indicated by the name of the mode in
parentheses at the end of the parameter description, but before any
default, {\it{e.g.}}\ Parameter DEVICE in \htmlref{CENTROID}{CENTROID}
is only relevant when Parameter MODE is \texttt{"Cursor"}.

\texttt{\%name} means the value of parameter {\it name}.

The description entry has a notation scheme to indicate
normally defaulted parameters, \emph{i.e.} those for which there will
be no prompt.
For such parameters a matching pair of square brackets (\file{{[]}})
terminates the description.  The content between the brackets mean
\begin{description}
\item[\texttt{{[]}}]
Empty brackets means that the default is created dynamically
by the application, and may depend on the values of other parameters.
Therefore, the default cannot be given explicitly.
\item[\texttt{[,]}]
As above, but there are two default values that are created dynamically.
\item[\texttt{[}{\textrm{default}}\texttt{{]}}]
Occasionally, a description of the default is given in normal type,
{\it{e.g.}}\ the size of the plotting region in a graphics application,
where the exact default values depend on the device chosen.
\item[\texttt{[default]}]
If the brackets contain a value in the teletype typeface, this is the explicit
default value.
\end{description}

\sstminitoc{List of KAPPA commands}
\newpage
\sstroutine{
   ADD
}{
   Adds two NDF data structures
}{
   \sstdescription{
      The routine adds two \NDFref{NDF} data structures pixel-by-pixel to produce
      a new NDF.
   }
   \sstusage{
      add in1 in2 out
   }
   \sstparameters{
      \sstsubsection{
         IN1 = NDF (Read)
      }{
         First NDF to be added.
      }
      \sstsubsection{
         IN2 = NDF (Read)
      }{
         Second NDF to be added.
      }
      \sstsubsection{
         OUT = NDF (Write)
      }{
         Output NDF to contain the sum of the two input NDFs.
      }
      \sstsubsection{
         TITLE = LITERAL (Read)
      }{
         The title for the output NDF.  A null value will cause
         the title of the NDF supplied for Parameter IN1 to be used
         instead.  \texttt{[!]}
      }
   }
   \sstexamples{
      \sstexamplesubsection{
         add a b c
      }{
         This adds the NDF called b to the NDF called a, to make the
         NDF called c.  NDF c inherits its title from a.
      }
      \sstexamplesubsection{
         add out=c in1=a in2=b title="Co-added image"
      }{
         This adds the NDF called b to the NDF called a, to make the
         NDF called c.  NDF c has the title \texttt{"Co-added image"}.
      }
   }
   \sstnotes{
      If the two input NDFs have different pixel-index bounds, then
      they will be trimmed to match before being added.  An error will
      result if they have no pixels in common.
   }
   \sstdiytopic{
      Related Applications
   }{
KAPPA: \htmlref{CADD}{CADD},
\htmlref{CDIV}{CDIV},
\htmlref{CMULT}{CMULT},
\htmlref{CSUB}{CSUB},
\htmlref{DIV}{DIV},
\htmlref{MATHS}{MATHS},
\htmlref{MULT}{MULT},
\htmlref{SUB}{SUB}.
   }
   \sstimplementationstatus{
      \sstitemlist{

         \sstitem
         This routine correctly processes the \htmlref{AXIS}{apndf:axis}, DATA, \htmlref{QUALITY}{apndf:quality},
         \htmlref{LABEL}{apndf:label}, \htmlref{TITLE}{apndf:title}, \htmlref{HISTORY}{apndf:history}, \htmlref{WCS}{apndf:wcs}, and \htmlref{VARIANCE}{apndf:variance} components of an NDF
         data structure and propagates all \htmlref{extensions}{apndf:extensions}.

         \sstitem
         The \htmlref{UNITS}{apndf:units}~ component is propagated only if it has the same
         value in both input NDFs.

         \sstitem
         Processing of \htmlref{bad pixels}{se:masking} and automatic \htmlref{quality masking}{se:qualitymask} are supported.

         \sstitem
         All \htmlref{non-complex numeric data types}{ap:HDStypes} can be handled.

         \sstitem
         Huge NDFs are supported.
      }
   }
}

\sstroutine{
   ALIGN2D
}{
   Aligns a pair of two-dimensional NDFs by minimising the residuals between them.
}{
   \sstdescription{
      This application attempts to align a two-dimensional input NDF with a
      two-dimensional reference NDF in pixel co-ordinates, using an affine
      transformation of the form:

       {\Large
         \[X_{\textnormal{in}} = C_1 + C_2 X_{\textnormal{ref}} + C_3 Y_{\textnormal{ref}}\]

         \[Y_{\textnormal{in}} = C_4 + C_5 X_{\textnormal{ref}} + C_6 Y_{\textnormal{ref}}\]
       }

      where ($X_{\textnormal{in}}$,$Y_{\textnormal{in}}$) are pixel co-ordinates
      in the input NDF, and ($X_{\textnormal{ref}}$,$Y_{\textnormal{ref}}$)
      are pixel co-ordinates in the reference NDF. The coefficient values
      ($C_1$--$C_6$) are determined by doing a least-squares
      fit that minimises the sum of the squared residuals between the
      reference NDF and the transformed input NDF. If variance
      information is present in either NDF, it is used
      to determine the SNR of each pixel which is used to weight the
      residuals within the fit, so that noisy data values have less
      effect on the fit. The best fit coefficients are displayed on
      the screen and written to an output parameter. Optionally, the
      transformation may be applied to the input NDF to create
      an output NDF (see Parameter OUT). It is possible to restrict the
      transformation in order to prevent shear, rotation, scaling,
      \emph{etc}. (see Parameter FORM).

      It is possible to exclude from the fitting process areas of the
      input NDF that are poorly correlated with the corresponding areas
      in the reference NDF (\emph{e.g.} flat background areas that contain
      only noise). See Parameter CORLIMIT.
   }
   \sstusage{
      align2d in ref out
   }
   \sstparameters{
      \sstsubsection{
         BOX = \_INTEGER (Read)
      }{
         The box size, in pixels, over which to calculate the correlation
         coefficient between the input and reference images. This should
         be set to an estimate of the maximum expected shift between the
         two images, but should not be less than typical size of features
         within the two images. See also Parameter CORLIMIT. \texttt{[5]}
      }
      \sstsubsection{
         CONSERVE = \_LOGICAL (Read)
      }{
         If set \texttt{TRUE}, then the output pixel values will be scaled in
         such a way as to preserve the total data value in a feature on
         the sky.  The scaling factor is the ratio of the output pixel
         size to the input pixel size.  This option can only be used if
         the Mapping is successfully approximated by one or more linear
         transformations.  Thus an error will be reported if it used
         when the TOL parameter is set to zero (which stops the use of
         linear approximations), or if the Mapping is too non-linear to
         be approximated by a piece-wise linear transformation.  The
         ratio of output to input pixel size is evaluated once for each
         panel of the piece-wise linear approximation to the Mapping,
         and is assumed to be constant for all output pixels in the
         panel.  This parameter is ignored if the NORM parameter is set
         \texttt{FALSE}.   \texttt{[TRUE]}
      }
      \sstsubsection{
         CORLIMIT = \_REAL (Read)
      }{
         If CORLIMIT is not null (\texttt{!}), each pixel in the input image
         is checked to see if the pixel values in its locality are well
         correlated with the corresponding locality in the reference image.
         The input pixel is excluded from the fitting process if the
         local correlation is below CORLIMIT. The supplied value should
         be between zero and 1.0. The size of the locality used around
         each input pixel is given by Parameter BOX. \texttt{[!]}

         In addition, if a value is supplied for CORLIMIT, the input and
         reference pixel values that pass the above check are scaled so
         that they have a mean value of zero and a standard deviation of
         unity before  being used in the fitting process. \texttt{[!]}
      }
      \sstsubsection{
         FITVALS = \_LOGICAL (Read)
      }{
         If \texttt{TRUE}, the fitting process will adjust the scale and offset
         of the input data values, in addition to the geometric position of
         the the input values, in order to minimise the sum of the squared
         residuals.  \texttt{[FALSE]}
      }
      \sstsubsection{
         FORM = \_INTEGER (Read)
      }{
         The form of the affine transformation to use:

         \sstitemlist{

            \sstitem
            \texttt{0} --- full unrestricted six-coefficient fit;

            \sstitem
            \texttt{1} --- shift, rotation and a common X/Y scale but no shear;

            \sstitem
            \texttt{2} --- shift and rotation but no scale or shear; or

            \sstitem
            \texttt{3} --- shift but not rotation, scale or shear.
         }
         \texttt{[0]}
      }
      \sstsubsection{
         IN = NDF (Read)
      }{
         NDF to be transformed.
      }
      \sstsubsection{
         METHOD = LITERAL (Read)
      }{
         The method to use when sampling the input pixel values (if
         resampling), or dividing an input pixel value between a group
         of neighbouring output pixels (if rebinning). For details of
         these schemes, see the descriptions of routines
         \xref{AST\_RESAMPLEx}{sun210}{AST_RESAMPLE\$<X>\$} and
         \xref{AST\_REBINSEQx}{sun210}{AST_REBINSEQ\$<X>\$} in
         \xref{SUN/210}{sun210}{}.  METHOD can take the following
         values.

         \sstitemlist{

            \sstitem
            \texttt{"Linear"} --- When resampling, the output pixel values are
            calculated by bi-linear interpolation among the four nearest
            pixels values in the input NDF.  When rebinning, the input
            pixel value is divided bi-linearly between the four nearest
            output pixels.  Produces smoother output NDFs than the
            nearest-neighbour scheme, but is marginally slower.

            \sstitem
            \texttt{"Nearest"} --- When resampling, the output pixel values are
            assigned the value of the single nearest input pixel.  When
            rebinning, the input pixel value is assigned completely to the
            single nearest output pixel.

            \sstitem
            \texttt{"Sinc"} --- Uses the ${\textrm{sinc}}({\pi}x)$ kernel, where
            $x$ is the pixel offset from the interpolation point (resampling) or
            transformed input pixel centre (rebinning), and
            ${\textrm{sinc}}(z)=\sin(z)/z$.  Use of this scheme is not recommended.

            \sstitem
            \texttt{"SincSinc"} --- Uses the ${\textrm{sinc}}({\pi}x){\textrm{sinc}}(k{\pi}x)$
            A valuable general-purpose scheme, intermediate in its visual
            effect on NDFs between the bi-linear and nearest-neighbour schemes.

            \sstitem
            \texttt{"SincCos"} --- Uses the ${\textrm{sinc}}({\pi}x)\cos(k{\pi}x)$
            kernel.  Gives similar results to the \texttt{"SincSinc"} scheme.

            \sstitem
            \texttt{"SincGauss"} --- Uses the ${\textrm{sinc}}({\pi}x)e^{-kx^2}$
            kernel.  Good results can be obtained by matching the FWHM of the
            envelope function to the point-spread function of the
            input data (see Parameter PARAMS).

            \sstitem
            \texttt{"Somb"} --- Uses the  ${\textrm{somb}}({\pi}x)$ kernel, where
            $x$ is the pixel offset from the interpolation point (resampling), or
            transformed input pixel centre (rebinning), and
            ${\textrm{somb}}(z)=2*J_{1}(z)/z$. $J_1$ is the first-order Bessel
            function of the first kind.  This scheme is similar to the
            \texttt{"Sinc"} scheme.

            \sstitem
            \texttt{"SombCos"} --- Uses the ${\textrm{somb}}({\pi}x)\cos(k{\pi}x)$
            kernel.  This scheme is similar to the \texttt{"SincCos"} scheme.

            \sstitem
            \texttt{"Gauss"} --- Uses the $e^{-kx^2}$ kernel.  The FWHM of the
            Gaussian is given by Parameter PARAMS(2), and the point at
            which to truncate the Gaussian to zero is given by Parameter
            PARAMS(1).

            \sstitem
            \texttt{"BlockAve"} --- Block averaging over all pixels in the
            surrounding $N$-dimensional cube.  This option is only available
            when resampling (\emph{i.e.} if REBIN is set to \texttt{FALSE}).

         }
         All methods propagate variances from input to output, but the
         variance estimates produced by interpolation schemes other than
         nearest neighbour need to be treated with care since the
         spatial smoothing produced by these methods introduces
         correlations in the variance estimates. Also, the degree of
         smoothing produced varies across the NDF.  This is because a
         sample taken at a pixel centre will have no contributions from
         the neighbouring pixels, whereas a sample taken at the corner
         of a pixel will have equal contributions from all four
         neighbouring pixels, resulting in greater smoothing and lower
         noise.  This effect can produce complex Moir\'{e} patterns in the
         output variance estimates, resulting from the interference of
         the spatial frequencies in the sample positions and in the
         pixel-centre positions.  For these reasons, if you want to use
         the output variances, you are generally safer using
         nearest-neighbour interpolation.  The initial default is
         \texttt{"Nearest"}.  \texttt{[}current value\texttt{{]}}
      }
      \sstsubsection{
         NORM = \_LOGICAL (Read)
      }{
         In general, each output pixel contains contributions from
         multiple input pixel values, and the number of input pixels
         contributing to each output pixel will vary from pixel to
         pixel.  If NORM is set \texttt{TRUE} (the default), then each output
         value is normalised by dividing it by the number of
         contributing input pixels, resulting in each output value being
         the weighted mean of the contributing input values.  However,
         if NORM is set \texttt{FALSE}, this normalisation is not applied.  See
         also Parameter CONSERVE.   \texttt{[TRUE]}
      }
      \sstsubsection{
         OUT = NDF (Write)
      }{
         An optional output NDF to contain a copy of IN aligned with OUT.
         No output is created if null (\texttt{!}) is supplied. If FITVALS
         is \texttt{TRUE}, the output data values will be scaled so that they have
         the same normalisation as the reference values.
      }
      \sstsubsection{
         PARAMS( 2 ) = \_DOUBLE (Read)
      }{
         An optional array which consists of additional parameters
         required by the Sinc, SincSinc, SincCos, SincGauss, Somb,
         SombCos, and Gauss methods.

         PARAMS( 1 ) is required by all the above schemes.
         It is used to specify how many pixels are to contribute to the
         interpolated result on either side of the interpolation or
         binning point in each dimension. Typically, a value of \texttt{2} is
         appropriate and the minimum allowed value is \texttt{1} (i.e. one pixel
         on each side). A value of zero or fewer indicates that a
         suitable number of pixels should be calculated automatically.
         \texttt{[0]}

         PARAMS( 2 ) is required only by the SombCos, Gauss, SincSinc,
         SincCos, and SincGauss schemes.  For the SombCos, SincSinc, and
         SincCos schemes, it specifies the number of pixels at which the
         envelope of the function goes to zero.  The minimum value is
         \texttt{1.0}, and the run-time default value is \texttt{2.0}.  For the
         Gauss and SincGauss scheme, it specifies the full-width at half-maximum
         (FWHM) of the Gaussian envelope measured in output pixels.
         The minimum value is \texttt{0.1}, and the run-time default is
         \texttt{1.0}.  On astronomical images and spectra, good results are
         often obtained by approximately matching the FWHM of the envelope
         function, given by PARAMS( 2 ), to the point-spread function of
         the input data.  []
      }
      \sstsubsection{
         REBIN = \_LOGICAL (Read)
      }{
         Determines the algorithm used to calculate the output pixel
         values.  If a \texttt{TRUE} value is given, a rebinning algorithm is
         used.  Otherwise, a resampling algorithm is used.  See the
         \htmlref{``Choice of Algorithm''}{choice:align} topic below.
         \texttt{[}current value\texttt{{]}}
      }
      \sstsubsection{
         REF = NDF (Read)
      }{
         NDF to be used as a refernece.
      }
      \sstsubsection{
         TOL = \_DOUBLE (Read)
      }{
         The maximum tolerable geometrical distortion that may be
         introduced as a result of approximating non-linear Mappings
         by a set of piece-wise linear transforms.  Both
         algorithms approximate non-linear co-ordinate transformations
         in order to improve performance, and this parameter controls
         how inaccurate the resulting approximation is allowed to be,
         as a displacement in pixels of the input NDF.  A value of
         zero will ensure that no such approximation is done, at the
         expense of increasing execution time.  \texttt{[0.05]}
      }
      \sstsubsection{
         WLIM = \_REAL (Read)
      }{
         This parameter is only used if REBIN is set \texttt{TRUE}. It specifies
         the  minimum number of good pixels which must contribute to an
         output pixel for the output pixel to be valid.  Note,
         fractional values are allowed. A null (\texttt{!}) value causes a very
         small positive value to be used resulting in output pixels
         being set bad only if they receive no significant contribution
         from any input pixel.  \texttt{[!]}
      }
   }
   \sstresparameters{
      \sstsubsection{
         RMS = \_DOUBLE (Write)
      }{
         An output parameter to which is written the RMS residual
         between the aligned data and the reference data.
      }
      \sstsubsection{
         TR( 6 ) = \_DOUBLE (Write)
      }{
         An output parameter to which are written the coefficients of
         the fit.  If FITVALS is \texttt{TRUE}, then this will include
         the scale and offset (written to the seventh and eighth
         entries).
      }
   }
   \sstexamples{
      \sstexamplesubsection{
         align2d my\_data orionA my\_corrected form=2
      }{
         Aligns the two-dimensional NDF called my\_data with the
         two-dimensional NDF called orionA, putting the aligned image in
         a new NDF called my\_corrected.  The transformation is restricted
         to a shift of origin and a rotation.
      }
   }
   \sstdiytopic{
      Related Applications
   }{
      KAPPA: \htmlref{WCSALIGN}{WCSALIGN}.
   }
   \sstimplementationstatus{
      \sstitemlist{

         \sstitem
         This routine correctly processes the DATA, \htmlref{VARIANCE}{apndf:variance},
         \htmlref{WCS}{apndf:wcs}, \htmlref{LABEL}{apndf:label}, \htmlref{TITLE}{apndf:title},
         and \htmlref{UNITS}{apndf:units} components of an NDF data structure.

         \sstitem
         All \htmlref{non-complex numeric data types}{ap:HDStypes} can be handled.
      }
   }
   \label{choice:align}
   \sstdiytopic{
      Choice of Algorithm
   }{
      The algorithm used to produce the output image is determined by
      the REBIN parameter, and is based either on resampling the output
      image or rebinning the input image.

      The resampling algorithm steps through every pixel in the output
      image, sampling the input image at the corresponding position and
      storing the sampled input value in the output pixel.  The method
      used for sampling the input image is determined by the METHOD
      parameter.  The rebinning algorithm steps through every pixel in
      the input image, dividing the input pixel value between a group
      of neighbouring output pixels, incrementing these output pixel
      values by their allocated share of the input pixel value, and
      finally normalising each output value by the total number of
      contributing input values. The way in which the input sample is
      divided between the output pixels is determined by the METHOD
      parameter.

      Both algorithms produce an output in which the each pixel value is
      the weighted mean of the near-by input values, and so do not alter
      the mean pixel values associated with a source, even if the pixel
      size changes. Thus the total data sum in a source will change if
      the input and output pixel sizes differ.  However, if the CONSERVE
      parameter is set \texttt{TRUE}, the output values are scaled by the ratio
      of the output to input pixel size, so that the total data sum in a
      source is preserved.

      A difference between resampling and rebinning is that resampling
      guarantees to fill the output image with good pixel values
      (assuming the input image is filled with good input pixel values),
      whereas holes can be left by the rebinning algorithm if the output
      image has smaller pixels than the input image.  Such holes occur
      at output pixels which receive no contributions from any input
      pixels, and will be filled with the value zero in the output
      image.  If this problem occurs the solution is probably to change
      the width of the pixel spreading function by assigning a larger
      value to PARAMS(1) and/or PARAMS(2) (depending on the specific
      METHOD value being used).

      Both algorithms have the capability to introduce artefacts into the
      output image.  These have various causes described below.

      \sstitemlist{
         \sstitem
         Particularly sharp features in the input can cause rings around
         the corresponding features in the output image. This can be
         minimised by suitable settings for the METHOD and PARAMS
         parameters. In general such rings can be minimised by using a
         wider interpolation kernel (if resampling) or spreading function
         (if rebinning), at the cost of degraded resolution.

         \sstitem
         The approximation of the Mapping using a piece-wise linear
         transformation (controlled by Parameter TOL) can produce artefacts
         at the joints between the panels of the approximation.  These can
         occur when using the rebinning algorithm, or when using the
         resampling algorithm with CONSERVE set to \texttt{TRUE}.  They are caused
         by the discontinuities  between the adjacent panels of the
         approximation, and can be minimised by reducing the value assigned
         to the TOL parameter.
      }
   }
}

\sstroutine{
   APERADD
}{
   Integrates pixel values within an aperture of an NDF
}{
   \sstdescription{
      This routine displays statistics for pixels that lie within a
      specified aperture of an \NDFref{NDF}.  The aperture can either be circular
      (specified by Parameters CENTRE and DIAM), or arbitrary (specified
      by Parameter ARDFILE).  If the aperture is specified using Parameters
      CENTRE and DIAM, then it must be either one- or  two-dimensional.

      The following statistics are displayed:

      \sstitemlist{

         \sstitem
         The total number of pixels within the aperture

         \sstitem
         The number of good pixels within the aperture

         \sstitem
         The total data sum within the aperture

         \sstitem
         The standard deviation on the total data sum (that is, the
           square root of the sum of the individual pixel variances)

         \sstitem
         The mean pixel value within the aperture

         \sstitem
         The standard deviation on the mean pixel value (that is, the
           standard deviation on the total data sum divided by the number of
           values)

         \sstitem
         The standard deviation of the pixel values within the aperture

      }
      If individual pixel variances are not available within the input NDF
      (\emph{i.e.} if it has no \htmlref{VARIANCE}{apndf:variance}~ component), then each pixel is assumed to
      have a constant variance equal to the variance of the pixel values
      within the aperture.  There is an option to weight pixels so that
      pixels with larger variances are given less weight (see Parameter
      WEIGHT).  The statistics are displayed on the screen and written to
      output parameters.  They may also be written to a log file.

      A pixel is included if its centre is within the aperture, and is not
      included otherwise.  This simple approach may not be suitable for
      accurate aperture photometry, especially where the aperture diameter
      is less than about ten times the pixel size.  A specialist photometry
      package should be used if accuracy, rather than speed, is paramount.
   }
   \sstusage{
      aperadd ndf centre diam
   }
   \sstparameters{
      \sstsubsection{
         ARDFILE = FILENAME (Read)
      }{
         The name of an ARD file containing a description of the aperture.
         This allows apertures of almost any shape to be used.  If a null
         (\texttt{{!}}) value is supplied then the aperture is assumed to be circular
         with centre and diameter given by Parameters CENTRE and DIAM.  ARD
         files can be created either `by hand' using an editor, or using a
         specialist application such as \htmlref{ARDGEN}{ARDGEN}.

         The co-ordinate system in which positions within the ARD file are
         given should be indicated by including suitable COFRAME or WCS
         statements within the file (see \xref{SUN/183}{sun183}{}), but will default to
         pixel co-ordinates in the absence of any such statements.  For
         instance, starting the file with a line containing the text
         \texttt{"COFRAME(SKY,System=FK5)"} would indicate that positions are
         specified in RA/DEC (FK5,J2000).  The statement \texttt{"COFRAME(PIXEL)"}
         indicates explicitly that positions are specified in pixel
         co-ordinates.  \texttt{[!]}
      }
      \sstsubsection{
         CENTRE = LITERAL (Read)
      }{
         The co-ordinates of the centre of the circular aperture.  Only
         used if Parameter ARDFILE is set to null.  The position must be
         given in the current \htmlref{co-ordinate Frame}{se:domains}~  of the NDF (supplying
         a colon \texttt{":"} will display details of the current co-ordinate
         Frame).  The position should be supplied as a list of formatted
         axis values separated by spaces or commas.  See also Parameter
         USEAXIS.  The current co-ordinate Frame can be changed using
         application \htmlref{WCSFRAME}{WCSFRAME}.
      }
      \sstsubsection{
         DIAM = LITERAL (Read)
      }{
         The diameter of the circular aperture.  Only used if Parameter
         ARDFILE is set to null.  If the \htmlref{current co-ordinate
         Frame}{se:curframe}~ of the NDF is a SKY Frame (\emph{e.g.} RA and
         DEC), then the value should be supplied as an increment of celestial
         latitude (\emph{e.g.} DEC).  Thus, \texttt{"10.2"} means 10.2 degrees,
         \texttt{"0:30"} would mean 30 arcminutes, and \texttt{"0:0:1"} would mean
         1 arcsecond.  If the current co-ordinate Frame is not a SKY Frame, then
         the diameter should be specified as an increment along Axis~1 of
         the current co-ordinate Frame.  Thus, if the current Frame is
         PIXEL, the value should be given simply as a number of pixels.
      }
      \sstsubsection{
         LOGFILE = FILENAME (Read)
      }{
         Name of the text file to log the results.  If null, there
         will be no logging.  Note this is intended for the human reader
         and is not intended for passing to other applications.  \texttt{[!]}
      }
      \sstsubsection{
         MASK = NDF (Write)
      }{
         An output NDF containing the pixel mask used to evaluate the
         reported statistics. The NDF will contain a positive integer
         value for pixels that are included in the statistics, and bad
         values for all other pixels. The pixel bounds of the NDF will
         be the smallest needed to encompass all used pixels. \texttt{[!]}
      }
      \sstsubsection{
         MEAN = \_DOUBLE (Write)
      }{
         The mean of the pixel values within the aperture.
      }
      \sstsubsection{
         NDF = NDF (Read)
      }{
         The input NDF.
      }
      \sstsubsection{
         NGOOD = \_INTEGER (Write)
      }{
         The number of good pixels within the aperture.
      }
      \sstsubsection{
         NUMPIX = \_INTEGER (Write)
      }{
         The total number of pixels within the aperture.
      }
      \sstsubsection{
         SIGMA = \_DOUBLE (Write)
      }{
         The standard deviation of the pixel values within the
         aperture.
      }
      \sstsubsection{
         SIGMEAN = \_DOUBLE (Write)
      }{
         The standard deviation on the mean pixel value.  If variances are
         available this is the RMS value of the standard deviations
         associated with each included pixel value.  If variances are not
         available, it is the standard deviation of the pixel values
         divided by the square root of the number of good pixels in
         the aperture.
      }
      \sstsubsection{
         SIGTOTAL = \_DOUBLE (Write)
      }{
         The standard deviation on the total data sum.  Only created if
         variances are available this is the RMS value of the standard
         deviations associated with each included pixel value.  If variances
         are not available, it is the standard deviation of the pixel values
         divided by the square root of the number of good pixels in
         the aperture.
      }
      \sstsubsection{
         TOTAL = \_DOUBLE (Write)
      }{
         The total of the pixel values within the aperture.
      }
      \sstsubsection{
         USEAXIS = \htmlref{GROUP}{se:groups} (Read)
      }{
         USEAXIS is only accessed if the current co-ordinate Frame of the
         NDF has too many axes.  A group of strings should be supplied
         specifying the axes which are to be used when specifying the
         aperture using Parameters ARDFILE, CENTRE, and DIAM.  Each axis can
         be specified using one of the following options.

         \ssthitemlist{

            \sstitem
            Its integer index within the current Frame of the
            input  NDF (in the range 1 to the number of axes in the
            current Frame).

            \sstitem
            Its \htmlattref{Symbol}{Symbol(axis)}~ string such as
            \texttt{"RA"} or \texttt{"VRAD"}.

            \sstitem
            A generic option where \texttt{"SPEC"} requests the spectral axis,
            \texttt{"TIME"} selects the time axis, \texttt{"SKYLON"} and
            \texttt{"SKYLAT"} picks the sky longitude and latitude axes
            respectively.  Only those axis domains present are
            available as options.
         }

         A list of acceptable values is displayed if an illegal value is
         supplied.  If a null (\texttt{{!}}) value is supplied, the
         axes with the same indices as the two used pixel axes within the
         NDF are used.  \texttt{[!]}
      }
      \sstsubsection{
         WEIGHT = \_LOGICAL (Read)
      }{
         If a \texttt{TRUE} value is supplied, and the input NDF has a VARIANCE
         component, then pixels with larger variances will be given
         smaller weight in the statistics.  The weight associated with
         each pixel is proportional to the reciprocal of its variance.
         The constant of proportionality is chosen so that the mean weight
         is unity.  The pixel value and pixel variance are multiplied by
         the pixels weight before being used to calculate the statistics.
         The calculation of the statistics remains unchanged in all other
         respects.  \texttt{[FALSE]}
      }
   }
   \sstexamples{
      \sstexamplesubsection{
         aperadd neb1 "13.5,201.3" 20
      }{
         This calculates the statistics of the pixels within a circular
         aperture of NDF neb1.  Assuming the current co-ordinate Frame of
         neb1 is PIXEL, the aperture is centred at pixel co-ordinates
         (13.5, 201.3) and has a diameter of 20 pixels.
      }
      \sstexamplesubsection{
         aperadd neb1 "15:23:43.2 -22:23:34.2" "10:0"
      }{
         This also calculates the statistics of the pixels within a circular
         aperture of NDF neb1.  Assuming the current co-ordinate Frame of
         neb1 is a SKY Frame describing RA and DEC, the aperture is centred
         at RA 15:23:43.2 and DEC -22:23:34.2, and has a diameter of 10
         arcminutes.
      }
      \sstexamplesubsection{
         aperadd ndf=neb1 ardfile=outline.dat logfile=obj1
      }{
         This calculates the statistics of the pixels within an aperture
         of NDF neb1 described within the file \texttt{outline.dat}.  The file
         contains an ARD description of the required aperture.  The results
         are written to the log file \texttt{obj1}.
      }
   }
   \sstnotes{
      \sstitemlist{

         \sstitem
         The statistics are not displayed on the screen when the
         message filter environment variable MSG\_FILTER is set to \texttt{QUIET}.
         The creation of output parameters and the log file is unaffected
         by MSG\_FILTER.

      }
   }
   \sstdiytopic{
      ASCII-region-definition Descriptors
   }{
      The ARD file may be created by ARDGEN or written manually.  In the
      latter case consult \xref{SUN/183}{sun183}{} ~for full details of the ARD
      descriptors and syntax; however, much may be learnt from looking
      at the ARD files created by ARDGEN and the ARDGEN documentation.
      There is also a \slhyperref{summary with examples}{in Section~}{}{se:ardwork}.

   }
   \sstdiytopic{
      Related Applications
   }{
KAPPA: \htmlref{STATS}{STATS},
\htmlref{MSTATS}{MSTATS},
\htmlref{ARDGEN}{ARDGEN},
\htmlref{ARDMASK}{ARDMASK},
\htmlref{ARDPLOT}{ARDPLOT},
\htmlref{WCSFRAME}{WCSFRAME}.
   }
   \sstimplementationstatus{
      \sstitemlist{

         \sstitem
         This routine correctly processes the \htmlref{WCS}{apndf:wcs}, \htmlref{AXIS}{apndf:axis}, DATA, and \htmlref{VARIANCE}{apndf:variance}
         components of an NDF data structure.

         \sstitem
         Processing of \htmlref{bad pixels}{se:masking} and automatic \htmlref{quality masking}{se:qualitymask} are
         supported.

         \sstitem
         \htmlref{Bad pixels}{se:masking} and \htmlref{quality masking}{se:qualitymask} are supported.

         \sstitem
         All \htmlref{non-complex numeric data types}{ap:HDStypes} can be handled.
      }
   }
}
\sstroutine{
   ARDGEN
}{
   Creates a text file describing selected regions of an image
}{
   \sstdescription{
      This is an interactive tool for selecting regions of a displayed
      image using a cursor, and then storing a description of the
      selected regions in a text file in the form of an `ARD
      Description' (see \xref{SUN/183}{sun183}{}).  This text file may subsequently be
      used in conjunction with packages such as \CCDPACKref\ ~or
      \ESPref\normalsize.

      The application initially obtains a value for the SHAPE parameter
      and then allows you to identify either one or many regions of the
      specified shape, dependent on the value of Parameter STARTUP.
      When the required regions have been identified, a value is
      obtained for Parameter OPTION, and that value determines what
      happens next.  Options include obtaining further regions,
      changing the current region shape, listing the currently defined
      regions, leaving the application, \emph{etc}.  Once the selected action
      has been performed, another value is obtained for OPTION, and
      this continues until you choose to leave the application.

      Instructions on the use of the cursor are displayed when the
      application is run.  The points required to define a region of
      the requested shape are described whenever the current region
      shape is changed using Parameter SHAPE.  Once the points required
      to define a region have been given an outline of the entire
      region is drawn on the graphics device using the pen specified by
      Parameter PALNUM.

      In the absence of any other information, subsequent application
      will use the union (\emph{i.e.} the logical OR) of all the defined
      regions.  However, regions can be combined in other ways using the
      COMBINE option (see Parameter OPTION).  For instance, two regions
      originally defined using the cursor could be replaced by their
      region of intersection (logical AND), or a single region could be
      replaced by its own exterior (logical NOT).  Other operators can
      also be used (see Parameter OPERATOR).
   }
   \sstusage{
      ardgen ardout shape option [device] [startup] [palnum] [poicol]
         \newline\hspace*{1.5em}
         $\left\{ {\begin{tabular}{l}
                   operands=? operator=? \\
                   regions=? \\
                   \end{tabular} }
        \right.$
        \newline\hspace*{1.9em}
        \makebox[0mm][c]{\small option}
   }
   \sstparameters{
      \sstsubsection{
         ARDOUT = FILENAME (Write)
      }{
         Name of the text file in which to store the description of the
         selected regions.
      }
      \sstsubsection{
         DEVICE = \htmlref{DEVICE}{se:selgradev} (Read)
      }{
         The graphics device on which the regions are to be selected.
         \texttt{[}Current graphics device\texttt{{]}}
      }
      \sstsubsection{
         OPERANDS() = \_INTEGER (Read)
      }{
         A pair of indices for the regions which are to be combined
         together using the operator specified by Parameter OPERATOR.
         If the operator is \texttt{"NOT"}, then only one region index need be
         supplied.  Region indices are displayed by the \texttt{"List"} option
         (see Parameter OPTION).
      }
      \sstsubsection{
         OPERATOR = \htmlref{LITERAL}{se:parmenu} (Read)
      }{
         The operator to use when combining two regions into a single
         region.  The pixels included in the resulting region depend on
         which of the following operators is selected.

         \ssthitemlist{

            \sstitem
            \texttt{"AND"} --- Pixels are included if they are in both of the regions
            specified by Parameter OPERANDS.

            \sstitem
            \texttt{"EQV"} --- Pixels are included if they are in both or neither of
            the regions specified by Parameter OPERANDS.

            \sstitem
            \texttt{"NOT"} --- Pixels are included if they are not inside the region
            specified by Parameter OPERANDS.

            \sstitem
            \texttt{"OR"} --- Pixels are included if they are in either of the
            regions specified by Parameter OPERANDS.  Note, an OR
            operator is implicitly assumed to exist between each
            pair of adjacent regions unless some other operator is
            specified.

            \sstitem
            \texttt{"XOR"} --- Pixels are included if they are in one, but not both,
            of the regions specified by Parameter OPERANDS.
         }
      }
      \sstsubsection{
         OPTION= LITERAL (Read)
      }{
         A value for this parameter is obtained when you choose to end
         cursor input (by pressing the relevant button as described
         when the application starts up).  It determines what to do
         next.  The following options are available:

         \ssthitemlist{

            \sstitem
            \texttt{"Combine"} --- Combine two previously defined regions
            into a single region using a Boolean operator, or invert a
            previously defined region using a Boolean .NOT. operator.
            See Parameters OPERANDS and OPERATOR.  The original
            regions are deleted and the new combined (or inverted)
            region is added to the end of the list of defined regions.

            \sstitem
            \texttt{"Delete"} --- Delete previously defined regions, see Parameter
            REGIONS.

            \sstitem
            \texttt{"Draw"} --- Draw the outline of the union of one or more previously
            defined regions, see Parameter REGIONS.

            \sstitem
            \texttt{"Exit"} --- Write out the currently defined regions to a text
            file and exit the application.

            \sstitem
            \texttt{"List"} --- List the textual descriptions of the
            currently defined regions on the screen.  Each region is
            described by an index value, a \emph{keyword}
            corresponding to the shape, and various arguments
            describing the extent and position of the shape. These
            arguments are described in the
            \htmlref{``Notes''}{notes:ardgen} section below.

            \sstitem
            \texttt{"Multi"} --- The cursor is displayed and you can then
            identify multiple regions of the current shape, without
            being re-prompted for OPTION after each one.  These
            regions are added to the end of the list of currently
            defined regions.  If the current shape is \texttt{"Polygon"},
            \texttt{"Frame"} or \texttt{"Whole"} (see Parameter SHAPE) then
            multiple regions cannot be defined and the selected option
            automatically reverts to \texttt{"Single"}.

            \sstitem
            \texttt{"Single"} --- The cursor is displayed and you can
            then identify a single region of the current shape.  You
            are re-prompted for Parameter OPTION once you have defined
            the region.  The identified region is added to the end of
            the list of currently defined regions.

            \sstitem
            \texttt{"Shape"} --- Change the shape of the regions created by the
            \texttt{"Single"} and \texttt{"Multi"} options.  This causes a new
            value for Parameter SHAPE to be obtained.

            \sstitem
            \texttt{"Style"} --- Change the drawing style by providing a new value
            for Parameter STYLE.

            \sstitem
            \texttt{"Quit"} --- Quit the application without saving the currently
            defined regions.

            \sstitem
            \texttt{"Undo"} --- Undo the changes made to the list of ARD regions by
            the previous option.  Note, the undo list can contain
            upto 30 entries.  Entries are only stored for options
            which actually produce a change in the list of regions.
         }
      }
      \sstsubsection{
         REGIONS() = LITERAL (Read)
      }{
         The list of regions to be deleted or drawn.  Regions are numbered
         consecutively from 1 and can be listed using the \texttt{"List"} option
         (see Parameter OPTION).  Single regions or a set of adjacent
         regions may be specified, \emph{e.g.} assigning \texttt{[4,6-9,12,14-16]} will
         delete regions 4,6,7,8,9,12,14,15,16.  (Note that the brackets
         are required to distinguish this array of characters from a
         single string including commas.  The brackets are unnecessary
         when there is only one item.)  The numbers need not be in
         ascending order.

         If you wish to delete or draw all the regions enter the
         wildcard \texttt{$*$}.  For instance, \texttt{5-$*$} will delete or draw
         from 5 to the last region.
      }
      \sstsubsection{
         SHAPE = LITERAL (Read)
      }{
         The shape of the regions to be defined using the cursor.
         After selecting a new shape, you are immediately requested to
         identify multiple regions as if \texttt{"Multi"} had been specified for
         Parameter OPTION.  The currently available shapes are listed
         below.

         \ssthitemlist{

            \sstitem
            \texttt{"Box"} --- A rectangular box with sides parallel to the
            co-ordinate axes, defined by its centre and one of its
            corners.

            \sstitem
            \texttt{"Circle"} --- A circle, defined by its centre and radius.

            \sstitem
            \texttt{"Column"} --- A single value on Axis 1, spanning all values on
            Axis 2.

            \sstitem
            \texttt{"Ellipse"} --- An ellipse, defined by its centre, one end of
            the major axis, and one other point which can be
            anywhere on the ellipse.

            \sstitem
            \texttt{"Frame"} --- The whole image excluding a border of constant
            width, defined by a single point on the frame.

            \sstitem
            \texttt{"Point"} --- A single pixel.

            \sstitem
            \texttt{"Polygon"} --- Any general polygonal region, defined by up to
            200 vertices.

            \sstitem
            \texttt{"Rectangle"} --- A rectangular box with sides parallel to the
            co-ordinate axes, defined by a pair of diagonally
            opposite corners.

            \sstitem
            \texttt{"Rotbox"} --- A rotated box, defined by both ends of an edge,
            and one point on the opposite edge.

            \sstitem
            \texttt{"Row"} --- A single value on Axis 2, spanning all values on
            Axis 1.

            \sstitem
            \texttt{"Whole"} --- The whole of the displayed image.
         }
      }
      \sstsubsection{
         STARTUP = LITERAL (Read)
      }{
         Determines if the application starts up in \texttt{"Multi"} or \texttt{"Single"}
         mode (see Parameter OPTION).  \texttt{["Multi"]}
      }
      \sstsubsection{
         UNDO = \_LOGICAL (Read)
      }{
         Used to confirm that it is OK to proceed with an \texttt{"Undo"} option.
         The consequences of proceeding are described before the parameter
         is obtained.
      }
   }
   \sstexamples{
      \sstexamplesubsection{
         ardgen extract.txt circle exit startup=single
      }{
         This example allows you to create a text file (\texttt{extract.txt})
         describing a single circular region of the image displayed on
         the \htmlref{current graphics device}{se:devglobal}.  The application immediately exits
         after the region has been identified.  This example may be
         useful in scripts or command procedures since there is no
         prompting.
      }
   }
   \label{notes:ardgen}
   \sstnotes{
      \sstitemlist{

         \sstitem
         An image must previously have been displayed on the graphics
         device.

         \sstitem
         The arguments for the textual description of each shape are as
         follows :

            \ssthitemlist{

               \sstitem
                  \texttt{"Box"} --- The co-ordinates of the centre, followed by the
                  lengths of the two sides.

               \sstitem
                  \texttt{"Circle"} --- The co-ordinates of the centre, followed by the
                  radius.

               \sstitem
                  \texttt{"Column"} --- The Axis 1 co-ordinate of the column.

               \sstitem
                  \texttt{"Ellipse"} --- The co-ordinates of the centre, followed by the
                  lengths of the semi-major and semi-minor axes,
                  followed by the angle between Axis 1 and the
                  semi-major axis (in radians).

               \sstitem
                  \texttt{"Frame"} --- The width of the border.

               \sstitem
                  \texttt{"Point"} --- The co-ordinates of the pixel.

               \sstitem
                  \texttt{"Polygon"} --- The co-ordinates of each vertex in the order given.

               \sstitem
                  \texttt{"Rectangle"} --- The co-ordinates of two diagonally opposite corners.

               \sstitem
                  \texttt{"Rotbox"} --- The co-ordinates of the box centre, followed by the
                  lengths of the two sides, followed by the angle
                  between the first side and Axis 1 (in radians).

               \sstitem
                  \texttt{"Row"} --- The Axis 2 co-ordinate of the row.

               \sstitem
                  \texttt{"Whole"} --- No arguments.
            }

         \sstitem
          The shapes are defined within the current co-ordinate
          Frame of the displayed NDF.  For instance, if the current
          \htmlref{co-ordinate Frame}{se:domains}~  of the displayed NDF is RA/DEC, then \texttt{"COLUMN"}
          regions will be curves of constant DEC, \texttt{"ROW"} regions will be curves
          of constant RA (assuming Axis 1 is RA and Axis 2 is DEC), straight
          lines will correspond to geodesics, \emph{etc}.  Numerical values will be
          stored in the output text file in the current co-ordinate Frame of
          the NDF.  \htmlref{WCS}{apndf:wcs} information will also be stored in the output text
          file allowing the stored positions to be converted to other systems
          (pixel co-ordinates, for instance).
      }
   }
   \sstdiytopic{
      Related Applications
   }{
KAPPA: \htmlref{ARDPLOT}{ARDPLOT},
\htmlref{ARDMASK}{ARDMASK},
\htmlref{LOOK}{LOOK},
\htmlref{REGIONMASK}{REGIONMASK};
\xref{CCDPACK}{sun139}{};
\xref{ESP}{sun180}{}.
   }
}
\sstroutine{
   ARDMASK
}{
   Uses an ARD file to set some pixels of an NDF to be bad
}{
   \sstdescription{
      This task allows regions of an \NDFref{NDF} to be masked, so
      that they can (for instance) be excluded from subsequent data
      processing.  ARD (ASCII Region Definition) descriptions
      (\xref{SUN/183}{sun183}{}) stored in
      a text file define which pixels of the data array are masked.  An
      output NDF is created which is the same as the input file except
      that all pixels specified by the ARD file have been assigned either
      the bad value or a specified constant value.  This value can be
      assigned to either the inside or the outside of the specified
      ARD region.

      If positions in the ARD description are given using a
      co-ordinate system that has one fewer axes than the input NDF,
      then each line or plane in the NDF will be masked independently
      using the supplied ARD description.  For instance, if a
      two-dimensional ARD description that uses (RA,Dec) to specify
      positions is used to mask a three-dimensional (ra,dec,velocity)
      NDF, then each velocity plane in the NDF will be masked
      independently.
   }
   \sstusage{
      ardmask in ardfile out
   }
   \sstparameters{
      \sstsubsection{
         ARDFILE = FILENAME (Read)
      }{
         The name of the ARD file containing a description of the parts
         of the image to be masked out, \emph{i.e.} set to bad.  The co-ordinate
         system in which positions within this file are given should be
         indicated by including suitable COFRAME or WCS statements within
         the file (see SUN/183), but will default to pixel co-ordinates or
         \htmlref{current WCS Frame}{se:curframe}~ co-ordinates in the absence of
         any such statements (see Parameter DEFPIX).  For instance, starting the
         file with a line containing the text \texttt{"COFRAME(SKY,System=FK5)"} would
         indicate that positions are specified in RA/DEC (FK5,J2000).  The
         statement \texttt{"COFRAME(PIXEL)"} indicates explicitly that positions are
         specified in pixel co-ordinates.
      }
      \sstsubsection{
         COMP = \htmlref{LITERAL}{se:parmenu} (Read)
      }{
         The NDF array component to be masked.  It may be \texttt{"Data"},
         or \texttt{"Variance"}, or \texttt{"Error"},  or \texttt{"All"}, (where
         \texttt{"Error"} is equivalent to \texttt{"Variance"}).  \texttt{["All"]}
      }
      \sstsubsection{
         CONST = LITERAL (Given)
      }{
         The constant numerical value to assign to the region, or the string
         \texttt{"bad"}.  \texttt{["bad"]}
      }
      \sstsubsection{
         DEFPIX = \_LOGICAL (Read)
      }{
         If a \texttt{TRUE} value is supplied for DEFPIX, then co-ordinates in
         the supplied ARD file will be assumed to be pixel co-ordinates.
         Otherwise, they are assumed to be in the current WCS co-ordinate
         system of the supplied NDF.  \texttt{[TRUE]}
      }
      \sstsubsection{
         IN = NDF (Read)
      }{
         The name of the source NDF.
      }
      \sstsubsection{
         INSIDE = \_LOGICAL (Read)
      }{
         If a \texttt{TRUE} value is supplied, the constant value is assigned to the
         inside of the region specified by the ARD file.  Otherwise, it is
         assigned to the outside.  \texttt{[TRUE]}
      }
      \sstsubsection{
         OUT = NDF (Write)
      }{
         The name of the masked NDF.
      }
      \sstsubsection{
         TITLE = LITERAL (Read)
      }{
         \htmlref{Title}{apndf:title} for the output NDF structure.  A null value (\texttt{{!}})
         propagates the title from the input NDF to the output NDF.  \texttt{[!]} }
   }
   \sstexamples{
      \sstexamplesubsection{
         ardmask a1060 galaxies.ard a1060\_sky title="A1060 galaxies masked"
      }{
         This flags pixels defined by the ARD file galaxies.ard within
         the NDF called a1060 to create a new NDF called a1060\_sky.
         a1060\_sky has a title=\texttt{"A1060 galaxies masked"}.  This might be
         to flag the pixels where bright galaxies are located to
         exclude them from sky-background fitting.
      }
      \sstexamplesubsection{
         ardmask in=ic3374 ardfil=ardfile.txt out=ic3374a
      }{
         This example uses as the source image the NDF called ic3374
         and sets the pixels specified by the ARD description contained
         in \texttt{ardfile.txt} to the bad value.  The resultant image is
         output to the NDF called ic3374a.  The title is unchanged.
      }
   }
   \sstdiytopic{
      ASCII-region-definition Descriptors
   }{
      The ARD file may be created by ARDGEN or written manually.  In the
      latter case consult SUN/183 for full details of the ARD
      descriptors and syntax; however, much may be learnt from looking
      at the ARD files created by ARDGEN and the ARDGEN documentation.
      There is also a \slhyperref{summary with examples}{in Section~}{}{se:ardwork}.
   }
   \sstdiytopic{
      Related Applications
   }{
KAPPA: \htmlref{ARDGEN}{ARDGEN},
\htmlref{ARDPLOT}{ARDPLOT},
\htmlref{LOOK}{LOOK},
\htmlref{REGIONMASK}{REGIONMASK}.
   }
   \sstimplementationstatus{
      \sstitemlist{

         \sstitem
         This routine correctly processes the \htmlref{WCS}{apndf:wcs}, \htmlref{AXIS}{apndf:axis}, DATA, \htmlref{QUALITY}{apndf:quality},
         \htmlref{LABEL}{apndf:label}, \htmlref{TITLE}{apndf:title}, \htmlref{UNITS}{apndf:units}, \htmlref{HISTORY}{apndf:history}, and \htmlref{VARIANCE}{apndf:variance}~ components of an NDF
         data structure and propagates all \htmlref{extensions}{apndf:extensions}.

         \sstitem
         Processing of \htmlref{bad pixels}{se:masking} and automatic \htmlref{quality masking}{se:qualitymask} are
         supported.

         \sstitem
         All \htmlref{numeric data types}{ap:HDStypes} can be handled.
      }
   }
}
\sstroutine{
   ARDPLOT
}{
   Plot regions described in an ARD file
}{
   \sstdescription{
      This application draws the outlines of regions described in
      a supplied two-dimensional ARD file (an `ARD Description' (see
      \xref{SUN/183}{sun183}{}).  If there is an existing picture on
      the graphics device, the outlines are drawn over the top of the
      previously displayed picture, aligned (if possible) in the current
      co-ordinate Frame of the previously drawn picture.  If the graphics
      device is empty (or if the CLEAR parameter is set \texttt{TRUE}) the
      outlines are drawn using a default projection---the size of the
      area plotted can be controlled by the SIZE parameter. Note, the
      facility to plot on an empty device is currently only available for
      two-dimensional regions specified using Parameter REGION.
   }
   \sstusage{
      ardplot ardfile [device] [regval]
   }
   \sstparameters{
      \sstsubsection{
         ARDFILE = FILENAME (Read)
      }{
         The name of a file containing an `ARD Description' of the regions
         to be outlined.  The co-ordinate system in which positions within
         this file are given should be indicated by including suitable
         COFRAME or WCS statements within the file (see SUN/183), but will
         default to pixel co-ordinates in the absence of any such
         statements.  For instance, starting the file with a line containing
         the text \texttt{"COFRAME(SKY,System=FK5)"} would indicate that positions
         are specified in RA/DEC (FK5,J2000).  The statement \texttt{"COFRAME(PIXEL)"}
         indicates explicitly that positions are specified in pixel
         co-ordinates.  The ARDFILE parameter is only accessed if
         Parameter REGION is given a null (\texttt{{!}}) value.
      }
      \sstsubsection{
         CLEAR = \_LOGICAL (Read)
      }{
         \texttt{TRUE} if the current picture is to be cleared before the Region
         is display. \texttt{[FALSE]}
      }
      \sstsubsection{
         DEVICE = \htmlref{DEVICE}{se:selgradev} (Read)
      }{
         The plotting device.  \texttt{[}Current graphics device\texttt{{]}}
      }
      \sstsubsection{
         REGION = FILENAME (Read)
      }{
         The name of a file containing an AST Region to be outlined, or
         null (\texttt{{!}}) if the ARD region defined by Parameter ARDFILE is to be
         outlined.  Suitable files can be created using the ATOOLS
         package.  \texttt{[!]}
      }
      \sstsubsection{
         REGVAL = \_INTEGER (Read)
      }{
         Indicates which regions within the ARD description are to be
         outlined.  If zero (the default) is supplied, then the plotted
         boundary encloses all the regions within the ARD file.  If a
         positive value is supplied, then only the region with the
         specified index is outlined (the first region in the ARD file
         has index 2, for historical reasons).  If a negative value is
         supplied, then all regions with indices greater than or equal
         to the absolute value of the supplied index are outlined.  See
         SUN/183 for further information on the numbering of regions
         within an ARD description.  The REGVAL parameter is only accessed
         if Parameter REGION is given a null (\texttt{{!}}) value.  \texttt{[0]}
      }
      \sstsubsection{
         SIZE = \_REAL (Read)
      }{
         The size of the plot to create, given as a multiple of the size
         of the Region being plotted.  This parameter is only accessed if
         no DATA picture can be found on the graphics device, or CLEAR is
         \texttt{TRUE}.  A SIZE value of \texttt{1.0} causes the plot to be the same size as
         the Region being plotted.  A value of \texttt{2.0} causes the plot to be
         twice the size of the Region, \emph{etc}. \texttt{[2.0]}
      }
      \sstsubsection{
         STYLE = \htmlref{GROUP}{se:groups} (Read)
      }{
         A group of attribute settings describing the plotting style to use
         for the curves.

         A comma-separated list of strings should be given in which each
         string is either an attribute setting, or the name of a text
         file preceded by an up-arrow character \texttt{"$\wedge$"}.  Such text files
         should contain further comma-separated lists which will be
         read and interpreted in the same manner.  Attribute settings
         are applied in the order in which they occur within the list,
         with later settings overriding any earlier settings given for
         the same attribute.

         Each individual attribute setting should be of the form:

            $<$name$>$=$<$value$>$


         where $<$name$>$ is the name of a plotting attribute, and $<$value$>$
         is the value to assign to the attribute.  Default values will be
         used for any unspecified attributes.  All attributes will be
         defaulted if a null value (\texttt{{!}})---the initial default---is supplied.
         To apply changes of style to only the current invocation, begin these
         attributes with a plus sign.  A mixture of persistent and temporary
         style changes is achieved by listing all the persistent attributes
         followed by a plus sign then the list of temporary attributes.

         See \slhyperref{Plotting Attributes}{Section~}{}{ap:plotting_attr}
         for a description of the available attributes.  Any unrecognised
         attributes are ignored (no error is reported).

         The appearance of the plotted curves is controlled by the attributes
         \htmlattref{Colour(Curves)}{Colour(element)},
         \htmlattref{Width(Curves)}{Width(element)}, \emph{etc}.
         \texttt{[}current value\texttt{{]}}
      }
   }
   \pagebreak
   \sstexamples{
      \sstexamplesubsection{
         ardplot bulge
      }{
         Draws an outline around all the regions included in the ardfile
         named \texttt{bulge}.  The outline is drawn on the
         \htmlref{current graphics device}{se:devglobal}
         and is drawn in alignment with the previous picture.
      }
   }
   \sstnotes{
      \sstitemlist{

         \sstitem
         A \htmlref{DATA picture}{se:agiaction} must already exist on the selected graphics
         device before running this command.  An error will be reported if no
         DATA picture can be found.

         \sstitem
         The application stores a new DATA picture in the \htmlref{graphics
         database}{se:agitate}.  On exit the current database picture for the chosen
         device reverts to the input picture.
      }
   }
   \sstdiytopic{
      Related Applications
   }{
KAPPA: \htmlref{ARDGEN}{ARDGEN},
\htmlref{ARDMASK}{ARDMASK},
\htmlref{LOOK}{LOOK}.
   }
}
\sstroutine{
   AXCONV
}{
   Expands spaced axes in an NDF into the primitive form
}{
   \sstdescription{
      This application routine converts \emph{in situ} an \NDFref{NDF's} axis centres
      in the `spaced' form into `simple' form.  Applications using the
      NDF\_ library, such as \KAPPA, are not currently capable
      of supporting spaced arrays, but there are packages that produce NDF
      files with this form of axis, notably {\footnotesize ASTERIX}.
      This application provides a temporary method of allowing
      \KAPPA\ ~{\it et al.}\ to handle these NDF datasets.
   }
   \sstusage{
      axconv ndf
   }
   \sstparameters{
      \sstsubsection{
         NDF = NDF (Read and Write)
      }{
         The NDF to be modified.
      }
   }
   \sstexamples{
      \sstexamplesubsection{
         axconv rosat256
      }{
         This converts the spaced axes in the NDF called rosat256 into
         simple form.
      }
   }
   \sstdiytopic{
      Related Applications
   }{
KAPPA: \htmlref{SETAXIS}{SETAXIS}.
   }
   \sstimplementationstatus{
      \sstitemlist{

         \sstitem
         Only axes with a real data type are created.
      }
   }
}
\sstroutine{
   AXLABEL
}{
   Sets a new label value for an axis within an NDF data structure
}{
   \sstdescription{
      This routine sets a new value for a LABEL component of an
      existing \NDFref{NDF} \htmlref{AXIS}{apndf:axis}~ data structure.  The NDF is accessed in update
      mode and any pre-existing LABEL component is over-written with a
      new value.  Alternatively, if a `null' value (\texttt{{!}}) is given for the
      LABEL parameter, then the NDF's axis LABEL component will be
      erased.  If an AXIS structure does not exist, a new one whose
      centres are pixel co-ordinates is created.
   }
   \sstusage{
      axlabel ndf label dim
   }
   \sstparameters{
      \sstsubsection{
         DIM = \_INTEGER (Read)
      }{
         The axis dimension for which the label is to be modified.
         There are separate labels for each NDF dimension.  The value
         must lie between 1 and the number of dimensions of the NDF.
         This defaults to 1 for a one-dimensional NDF.  The suggested
         default is the current value.  \texttt{[]}
      }
      \sstsubsection{
         NDF = NDF (Read and Write)
      }{
         The NDF data structure in which an axis LABEL component is to
         be modified.
      }
      \sstsubsection{
         LABEL = LITERAL (Read)
      }{
         The value to be assigned to the NDF's axis LABEL component
         (\emph{e.g.} \texttt{"Wavelength"} or \texttt{"Fibre index"}).  LABEL describes the
         quantity measured along the axis.  This value may later be
         used by other applications for labelling graphs or as a
         heading for columns in tabulated output.  The suggested
         default is the current value.
      }
   }
   \sstexamples{
      \sstexamplesubsection{
         axlabel ngc253 "Offset from nucleus" 2
      }{
         Sets the LABEL component of the second axis dimension of the
         NDF structure ngc253 to have the value \texttt{"Offset from nucleus"}.
      }
      \sstexamplesubsection{
         axlabel ndf=spect label=Wavelength
      }{
         Sets the axis LABEL component of the one-dimensional NDF
         structure spect to have the value \texttt{"Wavelength"}.
      }
      \sstexamplesubsection{
         axlabel datafile label=! dim=3
      }{
         By specifying a null value (\texttt{{!}}), this example erases any
         previous value of the LABEL component for the third dimension
         in the NDF structure datafile.
      }
   }
   \sstdiytopic{
      Related Applications
   }{
KAPPA: \htmlref{AXUNITS}{AXUNITS},
\htmlref{SETAXIS}{SETAXIS},
\htmlref{SETLABEL}{SETLABEL}.
   }
}
\sstroutine{
   AXUNITS
}{
   Sets a new units value for an axis within an NDF data structure
}{
   \sstdescription{
      This routine sets a new value for a UNITS component of an
      existing \NDFref{NDF} \htmlref{AXIS}{apndf:axis}~ data structure.  The NDF is accessed in update
      mode and any pre-existing UNITS component is over-written with a
      new value.  Alternatively, if a `null' value (\texttt{{!}}) is given for the
      UNITS parameter, then the NDF's axis UNITS component will be
      erased.  If an AXIS structure does not exist, a new one whose
      centres are pixel co-ordinates is created.
   }
   \sstusage{
      axunits ndf units dim
   }
   \sstparameters{
      \sstsubsection{
         DIM = \_INTEGER (Read)
      }{
         The axis dimension for which the units is to be modified.
         There are separate units for each NDF dimension.  The value
         must lie between 1 and the number of dimensions of the NDF.
         This defaults to 1 for a one-dimensional NDF.  The suggested
         default is the current value.  \texttt{[]}
      }
      \sstsubsection{
         NDF = NDF (Read and Write)
      }{
         The NDF data structure in which an axis UNITS component is to
         be modified.
      }
      \sstsubsection{
         UNITS = LITERAL (Read)
      }{
         The value to be assigned to the NDF's axis UNITS component
         (\emph{e.g.} \texttt{"Pixels"} or \texttt{"km/s"}).  UNITS describes the physical units
         of the quantity measured along the axis.  This value may later
         be used by other applications for labelling graphs and other
         forms of display where the NDF's \htmlref{axis co-ordinates}{se:domains}~ are shown.
         The suggested default is the current value.
      }
   }
   \sstexamples{
      \sstexamplesubsection{
         axunits ngc253 "arcsec" 2
      }{
         Sets the UNITS component of the second axis dimension of the
         NDF structure ngc253 to have the value \texttt{"arcsec"}.
      }
      \sstexamplesubsection{
         axunits ndf=spect units=Angstrom
      }{
         Sets the axis UNITS component of the one-dimensional NDF
         structure spect to have the value \texttt{"Angstrom"}.
      }
      \sstexamplesubsection{
         axunits datafile units=! dim=3
      }{
         By specifying a null value (\texttt{{!}}), this example erases any
         previous value of the UNITS component for the third dimension
         in the NDF structure datafile.
      }
   }
   \sstdiytopic{
      Related Applications
   }{
KAPPA: \htmlref{AXLABEL}{AXLABEL},
\htmlref{SETAXIS}{SETAXIS},
\htmlref{SETUNITS}{SETUNITS}.
   }
}

\sstroutine{
   BEAMFIT
}{
   Fits beam features in a two-dimensional NDF
}{
   \sstdescription{
      This fits generalised Gaussians (\emph{cf.} \htmlref{PSF}{PSF}) to beam
      features within the data array of a two-dimensional NDF given approximate
      initial co-ordinates.  It uses an unconstrained least-squares minimisation
      involving the residuals and a modified Levenberg-Marquardt
      algorithm.  The beam feature is a set of connected pixels which
      are either above or below the surrounding background region.  The
      errors in the fitted coefficients are also calculated.

      You may apply various constraints.  These are either fixed, or
      relative.  Fixed values include the FWHM, background level,
      or the shape exponent that defaults to 2 thus fits a normal
      distribution.  Relative constraints define the properties of
      secondary beam features with respect to the primary (first given)
      feature, and can specify amplitude ratios, and beam separations in
      Cartesian or polar co-ordinates.

      Four methods are available for obtaining the initial positions,
      selected using Parameter MODE:

      \ssthitemlist{

         \sstitem
         from the parameter system (see Parameters POS, POS2--POS5);

         \sstitem
         using a graphics cursor to indicate the feature in a previously
         displayed data array (see Parameter DEVICE);

         \sstitem
         from a specified positions list (see Parameter INCAT); or

         \sstitem
         from a simple text file containing a list of co-ordinates (see
         Parameter COIN).

      }
      In the first two modes the application loops, asking for new
      feature co-ordinates until it is told to quit or encounters an
      error or the maximum number of features is reached.  The last is
      five, unless Parameters POS2---POS5 define the location of the
      secondary beams and then only the primary beam's position is
      demanded.

      BEAMFIT both reports and stores in parameters its results.
      These are fit coefficients and their errors, the offsets and
      position angles of the secondary beam features with respect to
      the primary beam, and the offset of the primary beam from a
      reference position.  Also a listing of the fit results may be
      written to a log file geared more towards human readers,
      including details of the input parameters (see Parameter
      LOGFILE).
   }
   \sstusage{
      \mbox{beamfit ndf [mode]}
        $\left\{ {\begin{tabular}{l}
                  incat=? \\
                  \mbox{[beams]} \\
                  coin=? \\
                  \mbox{[beams]} pos pos2-pos5=?
                  \end{tabular} }
        \right.$
        \newline\latexhtml{\hspace*{9.45em}}{~~~~~~~~~~~~~~~~~~}
        \makebox[0mm][c]{\small mode}

   }
   \sstparameters{
      \sstsubsection{
         AMPRATIO( ) = \_REAL (Read)
      }{
         If number of beam positions given by BEAMS is more than one,
         this specifies the ratio of the amplitude of the secondary
         beams to the primary.  Thus you should supply one fewer value
         than the number of beams.  If you give fewer than that the last
         ratio  is copied to the missing values.  The ratios would
         normally be negative, usually $-1$ or $-0.5$.  AMPRATIO is ignored
         when there is only one beam feature to fit.  \texttt{[!]}
      }
      \sstsubsection{
         BEAMS = \_INTEGER (Read)
      }{
         The number of beam positions to fit.  This will normally be
         1, unless a chopped observation is supplied, when there may
         be two or three beam positions.  This parameter is ignored
         for \texttt{"File"} and \texttt{"Catalogue"} modes, where the
         number comes from the number of beam positions read from the
         files; and for "Interface" mode when the beam positions POS,
         POS2, \emph{etc.} are supplied in full on the command line
         without BEAMS.  In all modes there is a maximum of five
         positions, which for \texttt{"File"} or \texttt{"Catalogue"} modes
         will be the first five.  \texttt{[1]}
      }
      \sstsubsection{
         CIRCULAR = \_LOGICAL (Read)
      }{
         If set \texttt{TRUE} only circular beams will be fit.   \texttt{[FALSE]}
      }
      \sstsubsection{
         COIN =  FILENAME (Read)
      }{
         Name of a text file containing the initial guesses at the
         co-ordinates of beams to be fitted.  It is only accessed if
         Parameter MODE is given the value \texttt{"File"}.  Each line should
         contain the \xref{formatted axis values}{sun210}{AST_UNFORMAT} for a single position, in the
         \htmlref{current Frame}{se:curframe}~ of the NDF.  Axis values
         can be separated by spaces, tabs or commas.  The file may
         contain comment lines with the first character \texttt{\#} or \texttt{!}.
      }
      \sstsubsection{
         DESCRIBE = \_LOGICAL (Read)
      }{
         If \texttt{TRUE}, a detailed description of the co-ordinate Frame in
         which the beam positions will be reported is displayed before
         the positions themselves.  \texttt{[}current value\texttt{{]}}
      }
      \sstsubsection{
         DEVICE = DEVICE (Read)
      }{
         The graphics device which is to be used to give the initial
         guesses at the beam positions.  Only accessed if Parameter
         MODE is given the value \texttt{"Cursor"}.
         \texttt{[}Current graphics device\texttt{{]}}
      }
      \sstsubsection{
         FITAREA() = \_INTEGER (Read)
      }{
         Size in pixels of the fitting area to be used.  This should
         fully encompass the beam and also include some background signal.
         If only a single value is given, then it will be duplicated to all
         dimensions so that a square region is fitted.  Each value must
         be at least 9.  A null value requests that the full data
         array is used.  \texttt{[!]}
      }
      \sstsubsection{
         FIXAMP = \_DOUBLE (Read)
      }{
         This specifies the fixed amplitude of the first beam.
         Secondary sources arising from chopped data use FIXAMP
         multiplied by the AMPRATIO.  A null value indicates that
         the amplitude should be fitted.  \texttt{[!]}
      }
      \sstsubsection{
         FIXBACK = \_DOUBLE (Read)
      }{
         If a non-null value is supplied then the model fit will use
         that value as the constant background level otherwise the
         background is a free parameter of the fit.  \texttt{[!]}
      }
      \sstsubsection{
         FIXFWHM = LITERAL (Read)
      }{
         If this is set \texttt{TRUE} then the model fit will use the full-width
         half-maximum values for the beams supplied through Parameter
         FWHM.  \texttt{FALSE} demands that the FWHM values are free parameters
         of the fit.  \texttt{[FALSE]}
      }
      \sstsubsection{
         FIXPOS = \_LOGICAL (Read)
      }{
         If \texttt{TRUE}, the supplied position of each beam is used and
         the centre co-ordinates of the beam features are not fit.
         \texttt{FALSE} causes the initial estimate of the location of each
         beam to come from the source selected by Parameter MODE, and
         all these locations are part of the fitting process (however
         note the exception when FIXSEP=\texttt{TRUE}.  It is advisable not to
         use this option in the inaccurate \texttt{"Cursor"} mode.  \texttt{[FALSE]}
      }
      \sstsubsection{
         FIXSEP = \_LOGICAL (Read)
      }{
         If \texttt{TRUE}, the separations of secondary beams from the primary
         beam are fixed, and this takes precedence over Parameter
         FIXPOS.  If \texttt{FALSE}, the beam separations are free to be fitted
         (although it is actually the centres being fit).
         It is advisable not to use this option in the inaccurate
         \texttt{"Cursor"} mode.  \texttt{[FALSE]}
      }
      \sstsubsection{
         FWHM = LITERAL (Read)
      }{
         The initial full-width half-maximum (FWHM) values for each
         beam.  These become fixed values if FIXFWHM is set \texttt{TRUE}.

         A number of options are available.
         \ssthitemlist{

            \sstitem
            A single value gives the same circular FWHM for all beams.

            \sstitem
            When Parameter CIRCULAR is \texttt{TRUE}, supply a list of values one
            for each of the number of beams.  These should be supplied in
            the same order as the corresponding beam positions.

            \sstitem
            A pair of values sets the major- and minor-axis values for
            all beams, provided Parameter CIRCULAR is \texttt{FALSE}.

            \sstitem
            Major- and minor-axis pairs, whose order should match that
            of the corresponding beams.  Again CIRCULAR should be \texttt{FALSE}.
         }

         Multiple values are separated by commas.  An error is issued
         should none of these options be offered.

         If the current co-ordinate Frame of the NDF is a SKY Frame
         (\emph{e.g.} right ascension and declination), then the value should
         be supplied as an increment of celestial latitude (\emph{e.g.}
         declination).  Thus, \texttt{"5.7"} means 5.7 arcseconds, \texttt{
         "20:0"} would mean 20 arcminutes, and \texttt{"1:0:0"} would
         mean 1 degree.  If the current co-ordinate Frame is not a SKY
         Frame, then the widths should be specified as an increment
         along Axis 1 of the current co-ordinate Frame.  Thus, if the
         Current Frame is PIXEL, the value should be given simply as a
         number of pixels.

         Null requests that BEAMFIT itself estimates the initial FWHM
         values.  \texttt{[!]}
      }
      \sstsubsection{
         GAUSS = \_LOGICAL (Read)
      }{
         If \texttt{TRUE}, the shape exponent is fixed to be 2; in other words
         the beams are modelled as two-dimensional normal distributions.
         If \texttt{FALSE}, the shape exponent is a free parameter in each fit.
         \texttt{[TRUE]}
      }
      \sstsubsection{
         INCAT = FILENAME (Read)
      }{
         A catalogue containing a positions list giving the initial
         guesses at the beam positions, such as produced by applications
         CURSOR, LISTMAKE, \emph{etc.}  It is only accessed if Parameter MODE
         is given the value \texttt{"Catalogue"}.
      }
      \sstsubsection{
         LOGFILE = FILENAME (Read)
      }{
         Name of the text file to log the results.  If null, there
         will be no logging.  Note this is intended for the human reader
         and is not intended for passing to other applications.  \texttt{[!]}
      }
      \sstsubsection{
         MARK = LITERAL (Read)
      }{
          Only accessed if Parameter MODE is given the value \texttt{"Cursor"}.
          It indicates which positions are to be marked on the screen
          using the marker type given by Parameter MARKER.  It can take
          any of the following values.

         \ssthitemlist{

            \sstitem
             \texttt{"Initial"} --- The position of the cursor when the mouse
            button is pressed is marked.

            \sstitem
            \texttt{"Fit"} --- The corresponding fit position is marked.

            \sstitem
             \texttt{"Ellipse"} --- As "Fit" but it also plots an ellipse at the
             HWHM radii and orientation.

            \sstitem
             \texttt{"None"} --- No positions are marked.

         }
          \texttt{[}current value\texttt{{]}}
      }
      \sstsubsection{
         MARKER = INTEGER (Read)
      }{
         This parameter is only accessed if Parameter MARK is set \texttt{TRUE}.
         It specifies the type of marker with which each cursor
         position should be marked, and should be given as an integer
         PGPLOT marker type.  For instance, \texttt{0} gives a box, \texttt{1} gives a
         dot, \texttt{2} gives a cross, \texttt{3} gives an asterisk, \texttt{7} gives a triangle.
         The value must be larger than or equal to $-31$.  \texttt{[}current value\texttt{{]}}
      }
      \sstsubsection{
         MODE = \htmlref{LITERAL}{se:parmenu} (Read)
      }{
         The mode in which the initial co-ordinates are to be obtained.
         The supplied string can be one of the following values.

         \ssthitemlist{

            \sstitem
            \texttt{"Interface"} --- positions are obtained usingparameters POS,
            POS2--POS5.

            \sstitem
            \texttt{"Cursor"} --- positions are obtained using the graphics cursor
            of the device specified by Parameter DEVICE.

            \sstitem
            \texttt{"Catalogue"} --- positions are obtained from a positions list
            using Parameter INCAT.

            \sstitem
            \texttt{"File"} --- positions are obtained from a text file using
            Parameter COIN.
            \texttt{[}current value\texttt{{]}}
         }
      }
      \sstsubsection{
         NDF = NDF (Read)
      }{
         The NDF structure containing the data array to be analysed.  In
         cursor mode (see Parameter MODE), the run-time default is the
         displayed data, as recorded in the graphics database.  In other
         modes, there is no run-time default and the user must supply a
         value.   \texttt{[]}
      }
      \sstsubsection{
         PLOTSTYLE = \htmlref{GROUP}{se:groups} (Read)
      }{
         A group of attribute settings describing the style to use when
         drawing the graphics markers specified by Parameter MARK.

         A comma-separated list of strings should be given in which each
         string is either an attribute setting, or the name of a text
         file preceded by an up-arrow character \texttt{"$\wedge$"}.  Such text files
         should contain further comma-separated lists which will be
         read and interpreted in the same manner.  Attribute settings
         are applied in the order in which they occur within the list,
         with later settings overriding any earlier settings given for
         the same attribute.

         Each individual attribute setting should be of the form:

            $<$name$>$=$<$value$>$


         where $<$name$>$ is the name of a plotting attribute, and $<$value$>$
         is the value to assign to the attribute.  Default values will be
         used for any unspecified attributes.  All attributes will be
         defaulted if a null value (\texttt{{!}})---the initial default---is supplied.
         To apply changes of style to only the current invocation, begin these
         attributes with a plus sign.  A mixture of persistent and temporary
         style changes is achieved by listing all the persistent attributes
         followed by a plus sign then the list of temporary attributes.

         See \slhyperref{Plotting Attributes}{Section~}{}{ap:plotting_attr}
         for a description of the available attributes.  Any unrecognised
         attributes are ignored (no error is reported).  \texttt{[}current value\texttt{{]}}
      }
      \sstsubsection{
         POLAR = \_LOGICAL (Read)
      }{
         If \texttt{TRUE}, the co-ordinates supplied through POS2--POS5 are
         interpreted in polar co-ordinates (offset, position angle)
         about the primary beam.  The radial co-ordinate is a distance
         measured in units of the latitude axis if the
         \htmlref{current WCS Frame}{se:curframe}~
         is a SKY DOMAIN or the first axis for other Frames.  For a
         SKY current WCS Frame, position angle follows the standard
         convention of North through East.  For other Frames the angle
         is measured from the second axis anticlockwise, \emph{e.g.} for a
         PIXEL Frame it would be from \textit{y} through negative \textit{x}, not the
         standard \textit{x} through \textit{y}.

         If \texttt{FALSE}, the co-ordinates are the regular axis co-ordinates
         in the current Frame.

         POLAR is only accessed when there is more than one beam to fit.
         \texttt{[TRUE]}
      }
      \sstsubsection{
         POS = LITERAL (Read)
      }{
         When MODE = \texttt{"Interface"} POS specifies the co-ordinates
         of the primary beam position.  This is either merely an
         initial guess for the fit, or if Parameter FIXPOS is \texttt{
         TRUE}, it defines a fixed location.  It is specified in the
         current co-ordinate Frame of the NDF (supplying a colon \texttt{
         ":"} will display details of the current co-ordinate Frame).
         A position should be supplied as a list of formatted
         WCS axis values separated by spaces or commas, and should
         lie within the bounds of the NDF.

         If the initial co-ordinates are supplied on the command line
         without BEAMS the number of contiguous POS, POS2,\ldots
         parameters specifies the number of beams to be fit.  If the
         initial co-ordinates are supplied on the command line without
         BEAMS specified only one beam will be fit.
      }
      \sstsubsection{
         POS2-POS5 = LITERAL (Read)
      }{
         When MODE = \texttt{"Interface"} these parameters specify the
         co-ordinates of the secondary beam positions.  These should
         lie within the bounds of the NDF.  For each parameter the
         supplied location may be merely an initial guess for the fit,
         or if Parameter FIXPOS is \texttt{TRUE}, it defines a fixed
         location, unless Parameter FIXSEP is \texttt{TRUE}, whereupon it
         defines a fixed separation from the primary beam.

         For POLAR = \texttt{FALSE} each distance should be given as a single
         literal string containing a space- or comma-separated list of
         formatted axis values measured in the current co-ordinate Frame
         of the NDF.  The allowed formats depends on the class of the
         current Frame.  Supplying a single colon \texttt{":"} will display
         details of the current Frame, together with an indication of
         the format required for each axis value, and a new parameter
         value is then obtained.

         If Parameter POLAR is \texttt{TRUE}, POS2--POS5 may be given as an
         offset followed by a position angle.  See Parameter POLAR for
         more details of the sense of the angle and the offset
         co-ordinates.

         The parameter name increments by 1 for each subsequent beam
         feature.  Thus POS2 applies to the first secondary beam
         (second position in all), POS3 is for the second secondary
         beam, and so on.  As the total number of parameters required is
         one fewer than the value of Parameter BEAMS, POS2--POS5 are
         only accessed when BEAMS exceeds 1.
      }
      \sstsubsection{
         REFPOS = LITERAL (Read)
      }{
         The reference position.  This is often the desired position for
         the beam.  The offset of the primary beam with respect to this
         point is reported and stored in Parameter REFOFF.  It is only
         accessed if the current WCS Frame in the NDF is not a SKY
         Domain containing a reference position.

         The co-ordinates are specified in the current WCS Frame of the
         NDF (supplying a colon \texttt{":"} will display details of the current
         co-ordinate Frame).  A position should be supplied either as a
         list of formatted WCS axis values separated by spaces or
         commas.  A null value (\texttt{{!}}) requests that the centre of the
         supplied map is deemed to be the reference position.
      }
      \sstsubsection{
         RESID = NDF (Write)
      }{
         The map of the residuals (data minus model) of the fit.  It inherits
         the properties of the input NDF, except that its data type is
         \_DOUBLE or \_REAL depending on the precision demanded by the
         type of IN, and no variance is propagated.  A null (\texttt{{!}}) value
         requests that no residual map be created.  \texttt{[!]}
      }
      \sstsubsection{
         TITLE = LITERAL (Read)
      }{
         The title for the NDF to contain the residuals of the fit.
         If null (\texttt{{!}}) is entered the NDF will not contain a title.
         \texttt{["KAPPA - BEAMFIT"]}
      }
      \sstsubsection{
         VARIANCE = \_LOGICAL (Read)
      }{
         If \texttt{TRUE}, then any VARIANCE component present within the input
         NDF will be used to weight the fit; the weight used for each
         data value is the reciprocal of the variance.  If set to \texttt{FALSE}
         or there is no VARIANCE present, all points will be given equal
         weight.  \texttt{[FALSE]}
      }
   }
   \sstresparameters{
      \sstsubsection{
         AMP( 2 * BEAMS ) = \_DOUBLE (Write)
      }{
         The amplitude and its error for each beam.
      }
      \sstsubsection{
         BACK( 2 * BEAMS ) = \_DOUBLE (Write)
      }{
         The background level and its error at each beam position.
      }
      \sstsubsection{
         CENTRE( 2 * BEAMS ) = LITERAL (Write)
      }{
         The formatted co-ordinates and their errors of each beam in
         the current co-ordinate Frame of the NDF.
      }
      \sstsubsection{
         GAMMA( 2 * BEAMS ) = \_DOUBLE (Write)
      }{
         The shape exponent and its error for each beam.
      }
      \sstsubsection{
         MAJFWHM( 2 * BEAMS ) = \_DOUBLE (Write)
      }{
         The major-axis FWHM and its error, measured in the current
         co-ordinate Frame of the NDF, for each beam.  Note that the
         unit for sky co-ordinate Frames is radians.
      }
      \sstsubsection{
         MINFWHM( 2 * BEAMS ) = \_DOUBLE (Write)
      }{
         The minor-axis FWHM and its error, measured in the current
         co-ordinate Frame of the NDF, for each beam.  Note that the
         unit for sky co-ordinate Frames is radians.
      }
      \sstsubsection{
         OFFSET( ) = LITERAL (Write)
      }{
         The formatted offset and its error of each secondary beam
         feature with respect to the primary beam.  They are measured in
         the current Frame of the NDF along a latitude axis if that
         Frame is in the SKY Domain, or the first axis otherwise.  The
         number of values stored is twice the number of beams.  The
         array alternates an offset, then its corresponding error,
         appearing in beam order starting with the first secondary beam.
      }
      \sstsubsection{
         ORIENT( 2 * BEAMS ) = \_DOUBLE (Write)
      }{
         The orientation and its error, measured in degrees for each
         beam.  If the \htmlref{current WCS Frame}{se:curframe}~ is a SKY Frame,
         the angle is measured from North through East.  For other Frames the
         angle is from the \textit{x}-axis through \textit{y}.
      }
      \sstsubsection{
         PA() = \_REAL (Write)
      }{
         The position angle and its errors of each secondary beam
         feature with respect to the primary beam.  They are measured in
         the current Frame of the NDF from North through East if that is
         a SKY Domain, or anticlockwise from the \textit{y} axis otherwise.  The
         number of values stored is twice the number of beams.  The
         array alternates a position angle, then its corresponding
         error, appearing in beam order starting with the first
         secondary beam.
      }
      \sstsubsection{
         REFOFF( 2 ) = LITERAL (Write)
      }{
         The formatted offset followed by its error of the primary
         beam's location with respect to the reference position (see
         Parameter REFPOS).  The offset might be used to assess the
         optical alignment of an instrument.  The ofset and its error
         are measured in the current Frame of the NDF along a latitude
         axis if that Frame is in the SKY Domain, or the first axis
         otherwise.  The error is derived entirely from the
         uncertainities in the fitted position of the primary beam,
         \emph{i.e.} the reference position has no error attached to
         it.  By definition the error is zero when FIXPOS is \texttt{TRUE}.
      }
      \sstsubsection{
         RMS = \_REAL (Write)
      }{
         The primary beam position's root mean-squared deviation from
         the fit.
      }
      \sstsubsection{
         SUM = \_DOUBLE (Write)
      }{
         The total data sum of the multi-Gaussian fit above the background.
         The fit is evaluated at the centre of every pixel in the input
         NDF (including bad-valued pixels). The fitted background level
         is then removed from the fit value, and the sum of these is
         written to this output parameter.
      }
   }
   \sstexamples{
      \sstexamplesubsection{
         beamfit mars\_3pos i 1 "5.0,-3.5"
      }{
         This finds the Gaussian coefficients of the primary beam
         feature in the NDF called mars\_3pos, using the supplied
         beam's centre.  The co-ordinates are measured in the NDF's
         current co-ordinate Frame.  In this case they are offsets in
         arcseconds.
      }
      \sstexamplesubsection{
         beamfit ndf=mars\_3pos mode=interface beams=1 init1="5.0,-3.5" fixback=0
      }{
         As above but now the background is fixed to be zero.
      }
      \sstexamplesubsection{
         beamfit mars\_3pos i pos="5.0,-3.5"
      }{
         As the first example.  The presence of POS indicates a single
         is required.
      }
      \sstexamplesubsection{
         beamfit ndf=mars\_3pos mode=interface beams=1 pos="5.0,-3.5" fixfwhm fwhm=16.5 gauss=f
      }{
         As above but now the Gaussian is constrained to have a FWHM of
         16.5 arcseconds and be circular, but the shape exponent is not
         constrained to be 2.
      }
      \sstexamplesubsection{
         beamfit mars\_3pos in beams=1 fwhm=16.5 fitarea=51 pos="5.,-3.5"
      }{
         As above but now the fitted data is restricted to areas
         51$\times$51 pixels about the initial guess positions.  All
         the other examples use the full array.  Also the FWHM value is
         now just an initial guess.
      }
      \sstexamplesubsection{
         beamfit mars\_3pos int 3 "5.0,-3.5" ampratio=-0.5 resid=mars\_res
      }{
         As the first example except this finds the Gaussian
         coefficients of the primary beam feature and two secondary
         features.  The secondary features have fixed amplitudes that
         are half that of the primary feature and of the opposite
         polarity.  The residuals after subtracting the fit are stored
         in NDF mars\_res.  In all the other examples no residual map
         is created.
      }
      \sstexamplesubsection{
         beamfit mars\_3pos int 2 "5.0,-3.5" pos2="60.0,90" fixpos
      }{
         This finds the Gaussian coefficients of the primary beam
         feature and a secondary feature in the NDF called mars\_3pos.
         The supplied co-ordinates (5.0,$-$3.5) define the centre, \emph{i.e.}
         they are not fitted.  Also the secondary beam is fixed at
         60 arcseconds towards the East (position angle 90 degrees).
      }
      \sstexamplesubsection{
         beamfit mars\_3pos int 2 "5.0,-3.5" pos2="60.0,90" fixsep
      }{
         As the previous example, except now the separation of the
         second position is fixed at 60 arcseconds towards the East from
         the primary beam, instead of being an absolute location.
      }
      \sstexamplesubsection{
         beamfit mars\_3pos int 2 "5.0,-3.5" pos2="-60.5,0.6" polar=f fixpos
      }{
         As the last-but-one example, but now location of the secondary
         beam is fixed at ($-$55.5,$-$2.9).
      }
      \sstexamplesubsection{
         beamfit s450 int beams=2 fwhm="7.9,25" ampratio=0.06 circular
                pos='"0:0:0,0:0:0"' nopolar pos2="0:0:0,0:0:0"
      }{
         This fits two superimposed circular Gaussians in the NDF called
         s450, whose current WCS is SKY.  The beam second being fixed
         at 6 percent the strength of the first, with initial
         widths of 7.9 and 25 arcseconds.
      }
      \sstexamplesubsection{
         beamfit mode=cu beams=1
      }{
         This finds the Gaussian coefficients of the primary beam
         feature of an NDF, using the graphics cursor on the current
         graphics device to indicate the approximate centre of the
         feature.  The NDF being analysed comes from the graphics
         database.
      }
      \sstexamplesubsection{
         beamfit uranus cu 2 mark=ce plotstyle='colour=red' marker=3
      }{
         This fits to two beam features in the NDF called uranus
         via the graphics cursor on the current graphics device.  The
         beam positions are marked using a red asterisk.
      }
      \sstexamplesubsection{
         beamfit uranus file 4 coin=features.dat logfile=uranus.log
      }{
         This fits to the beam features in the NDF called uranus.  The
         initial positions are given in the text file \texttt{features.dat} in
         the current co-ordinate Frame.  Only the first four positions
         will be used.  The last three positions are in polar
         co-ordinates with respect to the primary beam.  A log of
         selected input parameter values, and the fitted coefficients
         and errors is written to the text file \texttt{uranus.log}.
      }
      \sstexamplesubsection{
         beamfit uranus mode=cat incat=uranus\_beams polar=f
      }{
         This example reads the initial guess positions from the
         positions list in file \texttt{uranus\_beams.FIT}.  The number of beam
         features fit is the number of positions in the catalogue
         subject to a maximum of five.  The input file may, for
         instance, have been created using the application CURSOR.
      }
   }
   \sstnotes{
      \sstitemlist{

         \sstitem
         All positions are supplied and reported in the current
         co-ordinate Frame of the NDF.  A description of the
         co-ordinate Frame being used is given if Parameter DESCRIBE
         is set to a \texttt{TRUE} value.  Application
         \htmlref{WCSFRAME}{WCSFRAME} can be used to change the
         current co-ordinate Frame of the NDF before running this
         application if required.

         \sstitem
         The uncertainty in the positions are estimated iteratively
         using the curvature matrix derived from the Jacobian, itself
         determined by a forward-difference approximation.

         \sstitem
         The fit parameters are not displayed on the screen when the
         message filter environment variable MSG\_FILTER is set to
         \texttt{QUIET}.

         \sstitem
         If the fitting fails there are specific error codes that can be
         tested and appropriate action taken in scripts: \texttt{PDA\_\_FICMX} when
         it is impossible to derive fit errors, and \texttt{KAP\_\_LMFOJ} when the fitted
         functions from the Levenberg-Marquardt minimisation are orthogonal
         to the Jacobian's columns (usually indicating that FITAREA is too
         small).

      }
   }
   \sstdiytopic{
      Related Applications
   }{
      KAPPA: \htmlref{PSF}{PSF},
      \htmlref{CENTROID}{CENTROID},
      \htmlref{CURSOR}{CURSOR},
      \htmlref{LISTSHOW}{LISTSHOW},
      \htmlref{LISTMAKE}{LISTMAKE};
      \xref{ESP}{sun180}{}: \xref{GAUFIT}{sun180}{GAUFIT};
      \xref{FIGARO}{sun86}{}: \xref{FITGAUSS}{sun86}{FITGAUSS}.
   }
   \sstimplementationstatus{
      \sstitemlist{

         \sstitem
         Processing of \htmlref{bad pixels}{se:masking} and automatic
         \htmlref{quality masking}{se:qualitymask} are supported.

         \sstitem
         All  \htmlref{non-complex numeric data types}{ap:HDStypes} can be handled.  Arithmetic
         is performed using double-precision floating point.
      }
   }
}

\sstroutine{
   BLOCK
}{
   Smooths an NDF using an n-dimensional rectangular box filter.
}{
   \sstdescription{

      This application smooths an n-dimensional NDF using a rectangular
      box filter, whose dimensionality is the same as that of the NDF
      being smoothed.   Each output pixel is either the mean or the
      median of the input pixels within the filter box.  The mean
      estimator provides one of the fastest methods of smoothing an
      image and is often useful as a general-purpose smoothing
      algorithm when the exact form of the smoothing point-spread
      function is not important.

      It is possible to smooth in selected dimensions by setting
      the boxsize to 1 for the dimensions not requiring smoothing.
      For example you can apply two-dimensional smoothing to the planes
      of a three-dimensional NDF (see Parameter BOX).  If it has three
      dimensions, then the filter is applied in turn to each plane in
      the cube and the result written to the corresponding plane in the
      output cube.
   }
   \sstusage{
      block in out box [estimator]
   }
   \sstparameters{
      \sstsubsection{
         BOX() = \_INTEGER (Read)
      }{
         The sizes (in pixels) of the rectangular box to be applied to
         smooth the data.  These should be given in axis order.  A value
         set to \texttt{1} indicates no smoothing along that axis.  Thus, for
         example, BOX=\texttt{[3,3,1]} for a three-dimensional NDF would apply a
         3x3-pixel filter to all its planes independently.

         If fewer values are supplied than the number of dimensions of
         the NDF, then the final value will be duplicated for the
         missing dimensions.

         The values given will be rounded up to positive odd integers, if
         necessary, to retain symmetry.
      }
      \sstsubsection{
         ESTIMATOR = \htmlref{LITERAL}{se:parmenu} (Read)
      }{
         The method to use for estimating the output pixel values.  It can
         be either \texttt{"Mean"} or \texttt{"Median"}.  \texttt{["Mean"]}
      }
      \sstsubsection{
         IN = NDF (Read)
      }{
         The input NDF to which box smoothing is to be applied.
      }
      \sstsubsection{
         OUT = NDF (Write)
      }{
         The output NDF which is to contain the smoothed data.
      }
      \sstsubsection{
         TITLE = LITERAL (Read)
      }{
         The title for the output NDF.  A null value will cause
         the title of the input NDF to be used.  \texttt{[!]}
      }
      \sstsubsection{
         WLIM = \_REAL (Read)
      }{
         If the input image contains bad pixels, then this parameter
         may be used to determine the number of good pixels which must
         be present within the smoothing box before a valid output
         pixel is generated.  It can be used, for example, to prevent
         output pixels from being generated in regions where there are
         relatively few good pixels to contribute to the smoothed
         result.

         By default, a null (\texttt{{!}}) value is used for WLIM, which causes
         the pattern of bad pixels to be propagated from the input
         image to the output image unchanged.  In this case, smoothed
         output values are only calculated for those pixels which are
         not bad in the input image.

         If a numerical value is given for WLIM, then it specifies the
         minimum fraction of good pixels which must be present in the
         smoothing box in order to generate a good output pixel.  If
         this specified minimum fraction of good input pixels is not
         present, then a bad output pixel will result, otherwise a
         smoothed output value will be calculated.  The value of this
         parameter should lie between 0.0 and 1.0 (the actual number
         used will be rounded up if necessary to correspond to at least
         one pixel).  \texttt{[!]}
      }
   }
   \sstexamples{
      \sstexamplesubsection{
         block aa bb 9
      }{
         Smooths the two-dimensional image held in the NDF structure aa,
         writing the result into the structure bb.  The smoothing box is
         9 pixels square.  If any pixels in the input image are bad,
         then the corresponding pixels in the output image will also be
         bad.  Each output pixel is the mean of the corresponding input
         pixels.
      }
      \sstexamplesubsection{
         block spectrum spectrums [5,1] median title="Smoothed spectrum"
      }{
         Smooths the one-dimensional data in the NDF called spectrum
         using a box size of 5 pixels, and stores the result in the NDF
         structure spectrums.  Each output pixel is the median of the
         corresponding input pixels.  If any pixels in the input image
         are bad, then the corresponding pixels in the output image
         will also be bad.  The output NDF has the title \texttt{"Smoothed
         spectrum"}.
      }
      \sstexamplesubsection{
         block ccdin(123,) ccdcol [1,9]
      }{
         Smooths the 123$^{\textrm{rd}}$ column in the two-dimensional NDF called ccdin
         using a box size of 9 pixels, and stores the result in the NDF
         structure ccdcol.  The first value of the smoothing box is
         ignored as the first dimension has only one element.  Each
         output pixel is the mean of the corresponding input pixels.
      }
      \sstexamplesubsection{
         block in=image1 out=image2 box=[5,7] estimator=median
      }{
         Smooths the two-dimensional image held in the NDF structure
         image1 using a rectangular box of size 5$\times$7 pixels.  The
         smoothed image is written to the structure image2.  Each
         output pixel is the median of the corresponding input pixels.
      }
      \sstexamplesubsection{
         block etacar etacars box=[7,1] wlim=0.6
      }{
         Smooths the specified image data using a rectangular box 7$\times$1
         pixels in size.  Smoothed output values are generated only if
         at least 60\% of the pixels in the smoothing box are good,
         otherwise the affected output pixel is bad.
      }
      \sstexamplesubsection{
         block in=cubein out=cubeout box=[3,3,7]
      }{
         Smooths the three-dimensional NDF called cubein using a box
         that has three elements along the first two axes and seven
         along the third.  The smoothed cube is written to NDF cubeout.
      }
      \sstexamplesubsection{
         block in=cubein out=cubeout box=[3,1,7]
      }{
         As the previous example, except that planes comprising the
         first and third axes are smoothed independently for all
         lines.
      }
   }
   \sstdiytopic{
      Timing
   }{
      When using the mean estimator, the execution time is approximately
      proportional to the number of pixels in the image to be smoothed and
      is largely independent of the smoothing box size.  This makes the
      routine particularly suitable for applying heavy smoothing to an image.
      Execution time will be approximately doubled if a variance array is
      present in the input NDF.

      The median estimator is much slower than the mean estimator, and is
      heavily dependent on the smoothing box size.
   }
   \sstdiytopic{
      Related Applications
   }{
KAPPA: \htmlref{CONVOLVE}{CONVOLVE},
\htmlref{FFCLEAN}{FFCLEAN},
\htmlref{GAUSMOOTH}{GAUSMOOTH},
\htmlref{MEDIAN}{MEDIAN};
\xref{FIGARO}{sun86}{}: \xref{ICONV3}{sun86}{ICONV3},
\xref{ISMOOTH}{sun86}{ISMOOTH},
\xref{IXSMOOTH}{sun86}{IXSMOOTH},
\xref{MEDFILT}{sun86}{MEDFILT}.
   }
   \sstimplementationstatus{
      \sstitemlist{

         \sstitem
         This routine correctly processes the \htmlref{AXIS}{apndf:axis}, DATA, \htmlref{QUALITY}{apndf:quality},
         \htmlref{LABEL}{apndf:label}, \htmlref{TITLE}{apndf:title}, \htmlref{UNITS}{apndf:units},
         \htmlref{WCS}{apndf:wcs}, and \htmlref{HISTORY}{apndf:history}~ components of the input NDF
         and propagates all \htmlref{extensions}{apndf:extensions}.  In addition, if the mean
         estimator is used, the \htmlref{VARIANCE}{apndf:variance}~ component is also processed.
         If the median estimator is used, then the output NDF will have
         no \htmlref{VARIANCE}{apndf:variance}~ component, even if there is a
         VARIANCE component in the input NDF.

         \sstitem
         Processing of \htmlref{bad pixels}{se:masking} and automatic
         \htmlref{quality masking}{se:qualitymask} are supported.
         The \htmlref{bad-pixel flag}{setbad:badpixelflag}~ is also written for the data
         and variance arrays.

         \sstitem
         All \htmlref{non-complex numeric data types}{ap:HDStypes} can be handled.  Arithmetic
         is performed using single-precision floating point, or double
         precision if appropriate.

         \sstitem
         Huge NDFs are supported.
      }
   }
}
\sstroutine{
   CADD
}{
   Adds a scalar to an NDF data structure
}{
   \sstdescription{
      The routine adds a scalar (\emph{i.e.} constant) value to each pixel of
      an \NDFref{NDF's} data array to produce a new NDF data structure.
   }
   \sstusage{
      cadd in scalar out
   }
   \sstparameters{
      \sstsubsection{
         IN = NDF (Read)
      }{
         Input NDF data structure, to which the value is to be added.
      }
      \sstsubsection{
         OUT = NDF (Write)
      }{
         Output NDF data structure.
      }
      \sstsubsection{
         SCALAR = \_DOUBLE (Read)
      }{
         The value to be added to the NDF's data array.
      }
      \sstsubsection{
         TITLE = LITERAL (Read)
      }{
         The title for the output NDF.  A null value will cause
         the title of the NDF supplied for Parameter IN to be used
         instead.  \texttt{[!]}
      }
   }
   \sstexamples{
      \sstexamplesubsection{
         cadd a 10 b
      }{
         This adds ten to the NDF called a, to make the NDF called b.
         NDF b inherits its title from a.
      }
      \sstexamplesubsection{
         cadd title="HD123456" out=b in=a scalar=17.3
      }{
         This adds 17.3 to the NDF called a, to make the NDF called b.
         NDF b has the title \texttt{"HD123456"}.
      }
   }
   \sstdiytopic{
      Related Applications
   }{
KAPPA: \htmlref{ADD}{ADD},
\htmlref{CDIV}{CDIV},
\htmlref{CMULT}{CMULT},
\htmlref{CSUB}{CSUB},
\htmlref{DIV}{DIV},
\htmlref{MATHS}{MATHS},
\htmlref{MULT}{MULT},
\htmlref{SUB}{SUB}.
   }
   \sstimplementationstatus{
      \sstitemlist{

         \sstitem
         This routine correctly processes the \htmlref{AXIS}{apndf:axis}, DATA, \htmlref{QUALITY}{apndf:quality},
         \htmlref{LABEL}{apndf:label}, \htmlref{TITLE}{apndf:title}, \htmlref{UNITS}{apndf:units}, \htmlref{HISTORY}{apndf:history}, \htmlref{WCS}{apndf:wcs}, and \htmlref{VARIANCE}{apndf:variance}~ components of an NDF
         data structure and propagates all \htmlref{extensions}{apndf:extensions}.

         \sstitem
         Processing of \htmlref{bad pixels}{se:masking} and automatic \htmlref{quality masking}{se:qualitymask} are supported.

         \sstitem
         All \htmlref{non-complex numeric data types}{ap:HDStypes} can be handled.

         \sstitem
         Huge NDFs are supported.
      }
   }
}
\sstroutine{
   CALC
}{
   Evaluates a mathematical expression
}{
   \sstdescription{
      This task evaluates an arithmetic expression and reports the
      result.  It main r\^{o}le is to perform floating-point arithmetic in
      scripts.  A value \texttt{"Bad"} is reported if there was an error
      during the calculation, such as a divide by zero.
   }
   \sstusage{
      calc exp [prec] fa-fz=? pa-pz=?
   }
   \sstparameters{
      \sstsubsection{
         EXP = LITERAL (Read)
      }{
         The mathematical expression to be evaluated, \emph{e.g.}
         \texttt{"-2.5$*$LOG10(PA)"}.  In this expression constants may either be
         given literally or represented by the variables PA, PB, \ldots
         PZ.  The expression may contain sub-expressions represented by
         the variables FA, FB, \ldots FZ.  Values for those
         sub-expressions and constants which appear in the expression
         will be requested via the application's parameter of the same
         name.

         FORTRAN 77 syntax is used for specifying the expression, which
         may contain the usual intrinsic functions, plus a few extra
         ones.  An appendix in \xref{SUN/61}{sun61}{} gives a full description of the
         syntax used and an up-to-date list of the functions available.
         The arithmetic operators ($+$,-,/,$*$,$*$$*$) follow the normal order
         of precedence.  Using matching (nested) parentheses will
         explicitly define the order of expression evaluation.  The
         expression may be up to 132 characters long.
      }
      \sstsubsection{
         FA-FZ = LITERAL (Read)
      }{
         These parameters supply the values of `sub-expressions' used
         in the expression EXP.  Any of the 26 may appear; there is no
         restriction on order.  These parameters should be used when
         repeated expressions are present in complex expressions, or to
         shorten the value of EXP to fit within the 132-character limit.
         Sub-expressions may contain references to other
         sub-expressions and constants (PA-PZ).  An example of using
         sub-expressions is:
         \begin{description}
         \item EXP $>$ \texttt{PA$*$ASIND(FA/PA)$*$X/FA}
         \item FA $>$ \texttt{SQRT(X$*\textit{x}+\textit{y}*$Y)}
         \item PA $>$ \texttt{10.1}
         \end{description}
         where the parameter name is to the left of $>$ and its value is
         to the right of the $>$.
      }
      \sstsubsection{
         PA-PZ = \_DOUBLE (Read)
      }{
         These parameters supply the values of constants used in the
         expression EXP and sub-expressions FA-FZ.  Any of the 26 may
         appear; there is no restriction on order.  Using parameters
         allows the substitution of repeated constants using one
         reference.  This is especially convenient for constants with
         many significant digits.  It also allows easy modification of
         parameterised expressions provided the application has not
         been used with a different EXP in the interim.  The parameter
         PI has a default value of 3.14159265359D0.  An example of
         using parameters is:
         \begin{description}
         \item EXP $>$ \texttt{SQRT(PX$*$PX$+$PY$*$PY)$*$EXP(PX-PY)}
         \item PX $>$ \texttt{2.345}
         \item PY $>$ \texttt{-0.987}
         \end{description}
         where the parameter name is to the left of $>$ and its value is
         to the right of the $>$.
      }
      \sstsubsection{
         PREC = LITERAL (Read)
      }{
         The arithmetic precision with which the transformation
         functions will be evaluated when used.  This may be either
         \texttt{"\_REAL"} for single precision, \texttt{"\_DOUBLE"} for double precision,
         or \texttt{"\_INTEGER"} for integer precision.  Elastic precisions are
         used, such that a higher precision will be used if the input
         data warrant it.  So for example if PREC=\texttt{"\_REAL"}, but
         double-precision data were to be transformed, double-precision
         arithmetic would actually be used.  The result is reported
         using the chosen precision.  \texttt{["\_REAL"]}
      }
   }
   \sstresparameters{
      \sstsubsection{
         RESULT = LITERAL (Write)
      }{
         The result of the evaluation.
      }
   }
   \sstexamples{
      \sstexamplesubsection{
      {\textrm{Shell usage:}}
      }{
      The syntax in the following examples apply to the shell.
      }
      \sstexamplesubsection{
         calc "27.3$*$1.26"
      }{
         The reports the value of the expression 27.3$*$1.26, \emph{i.e.} 34.398.
      }
      \sstexamplesubsection{
         calc exp="(pa$+$pb$+$pc$+$pd)/4.0" pa=\$med1 pb=\$med2 pc=\$med3 pd=\$med4
      }{
         This reports the average of four values defined by script
         variables med1, med2, med3, and med4.
      }
      \sstexamplesubsection{
         calc "42.6$*$pi/180"
      }{
         This reports the value in radians of 42.6 degrees.
      }
      \sstexamplesubsection{
         calc "(mod(PO,3)$+$1)/2" prec=\_integer po=\$count
      }{
         This reports the value of the expression
         \texttt{"(mod(\$count,3)$+$1)/2)"}
         where \texttt{\$count} is the value of the shell variable count.  The
         calculation is performed in integer arithmetic, thus if
         count equals 2, the result is 1 not 1.5.
      }
      \sstexamplesubsection{
         calc "sind(pa/fa)$*$fa" fa="log(abs(pb$+$pc))" pa=2.0e-4 pb=-1 pc=\$x
      }{
         This evaluates sind(0.0002/log(abs(\$\texttt{{x}}$-$1)))$*$log(abs(\$\texttt{{x}}$-$1)) where
         \texttt{\$x} is the value of the shell variable x.
      }
      \sstexamplesubsection{
      {\textrm{\ICL\ usage:}}
      }{
      For \ICL\ usage only those expressions containing
      parentheses need to be in quotes, though \ICL
      itself provides the arithmetic.  So the above examples would be
      }
      \sstexamplesubsection{
         calc 27.3$*$1.26
      }{
         The reports the value of the expression 27.3$*$1.26, \emph{i.e.} 34.398.
      }
      \sstexamplesubsection{
         calc exp="(pa$+$pb$+$pc$+$pd)/4.0" pa=(med1) pb=(med2) pc=(med3) pd=(med4)
      }{
         This reports the average of four values defined by \ICL
         variables med1, med2, med3, and med4.
      }
      \sstexamplesubsection{
         calc 42.6$*$pi/180
      }{
         This reports the value in radians of 42.6 degrees.
      }
      \sstexamplesubsection{
         calc "(mod(PO,3)$+$1)/2" prec=\_integer po=(count)
      }{
         This reports the value of the expression
         \texttt{"(mod((count),3)$+$1)/2)"}
         where \texttt{(count)} is the value of the \ICL
         variable count.  The calculation is performed in integer
         arithmetic, thus if count equals 2, the result is 1 not 1.5.
      }
      \sstexamplesubsection{
         calc "sind(pa/fa)$*$fa" fa="log(abs(pb$+$pc))" pa=2.0e-4 pb=-1 pc=(x)
      }{
         This evaluates sind(0.0002/log(abs((x)$-$1)))$*$log(abs((x)$-$1)) where
         \texttt{(x)} is the value of the \ICL\ variable x.
      }
   }
   \sstimplementationstatus{
      On OSF/1 systems an error during the calculation results in a
      core dump.  On Solaris, undefined values are set to one.  These
      are due to problems with the TRANSFORM infrastructure.
   }
   \sstdiytopic{
      Related Applications
   }{
KAPPA: \htmlref{MATHS}{MATHS}.
   }
}

\sstroutine{
   CALPOL
}{
   Calculates polarisation parameters
}{
   \sstdescription{
      This routine calculates various parameters describing the
      polarisation described by four intensity arrays analysed at 0\dgs,
      45\dgs, 90\dgs, and 135\dgs\ to a reference direction.  Variance
      values are
      stored in the output \NDFref{NDFs}~ if all the input NDFs have variances and
      you give a \texttt{TRUE}~ value for Parameter VARIANCE.

      By default, three output NDFs are created holding percentage
      polarisation, polarisation angle and total intensity.  However,
      NDFs holding other quantities, such as the Stokes parameters, can
      also be produced by overriding the default null values
      associated with the corresponding parameters.  The creation of any
      output NDF can be suppressed by supplying a null value for the
      corresponding parameter.

      There is an option to correct the calculated values of percentage
      polarisation and polarised intensity to take account of the
      statistical bias introduced by the asymmetric distribution of
      percentage polarisation (see Parameter DEBIAS).  This correction
      subtracts the variance of the percentage polarisation from the
      squared percentage polarisation, and uses the square root of this
      as the corrected percentage polarisation.  The corresponding
      polarised intensity is then found by multiplying the corrected
      percentage polarisation by the total intensity.  Returned variance
      values take no account of this correction.
   }
   \sstusage{
      calpol in1 in2 in3 in4 p theta i
   }
   \sstparameters{
      \sstsubsection{
         DEBIAS = \_LOGICAL (Read)
      }{
         \texttt{TRUE} if a correction for statistical bias is to be made to
         percentage polarisation and polarised intensity.  This
         correction cannot be used if any of the input NDFs do not
         contain variance values, or if you supply a \texttt{FALSE} value
         for Parameter VARIANCE.  \texttt{[FALSE]}
      }
      \sstsubsection{
         I = NDF (Write)
      }{
         An output NDF holding the total intensity derived from all four
         input NDFs.
      }
      \sstsubsection{
         IN1 = NDF (Read)
      }{
         An NDF holding the measured intensity analysed at an angle of
         0\dgs\ to the reference direction.  The primary input NDF.
      }
      \sstsubsection{
         IN2 = NDF (Read)
      }{
         An NDF holding the measured intensity analysed at an angle of
         45\dgs\ to the reference direction.  The suggested default
         is the current value.
      }
      \sstsubsection{
         IN3 = NDF (Read)
      }{
         An NDF holding the measured intensity analysed at an angle of
         90\dgs\ to the reference direction.  The suggested default
         is the current value.
      }
      \sstsubsection{
         IN4 = NDF (Read)
      }{
         An NDF holding the measured intensity analysed at an angle of
         135\dgs\ to the reference direction.  The suggested default
         is the current value.
      }
      \sstsubsection{
         IA = NDF (Write)
      }{
         An output NDF holding the total intensity derived from input
         NDFs IN1 and IN3.  \texttt{[!]}
      }
      \sstsubsection{
         IB = NDF (Write)
      }{
         An output NDF holding the total intensity derived from input
         NDFs IN2 and IN4.  \texttt{[!]}
      }
      \sstsubsection{
         IP = NDF (Write)
      }{
         An output NDF holding the polarised intensity.  \texttt{[!]}
      }
      \sstsubsection{
         P = NDF (Write)
      }{
         An output NDF holding percentage polarisation.
      }
      \sstsubsection{
         Q = NDF (Write)
      }{
         An output NDF holding the normalised Stokes parameter, \textit{Q}.  \texttt{[!]}
      }
      \sstsubsection{
         U = NDF (Write)
      }{
         An output NDF holding the normalised Stokes parameter, \textit{U}.  \texttt{[!]}
      }
      \sstsubsection{
         THETA = NDF (Write)
      }{
         An output NDF holding the polarisation angle in degrees.
      }
      \sstsubsection{
         VARIANCE = \_LOGICAL (Read)
      }{
         \texttt{TRUE} if output variances are to be calculated.  This parameter
         is only accessed if all input NDFs contain variances, otherwise
         no variances are generated.  \texttt{[TRUE]}
      }
   }
   \sstexamples{
      \sstexamplesubsection{
         calpol m51\_0 m51\_45 m51\_90 m51\_135 m51\_p m51\_t m51\_i ip=m51\_ip
      }{
         This example produces NDFs holding percentage polarisation,
         polarisation angle, total intensity and polarised intensity,
         based on the four NDFs M51\_0, m51\_45, m51\_90 and m51\_135.
      }
      \sstexamplesubsection{
         calpol m51\_0 m51\_45 m51\_90 m51\_135 m51\_p m51\_t m51\_i ip=m51\_ip novariance
      }{
         As above except that variance arrays are not computed.
      }
      \sstexamplesubsection{
         calpol m51\_0 m51\_45 m51\_90 m51\_135 m51\_p m51\_t m51\_i ip=m51\_ip
      }{
         As the first example except that there is a correction for
         statistical bias in the percentage polarisation and polarised
         intensity, assuming that all the input NDFs have a VARIANCE
         array.
      }
      \sstexamplesubsection{
         calpol m51\_0 m51\_45 m51\_90 m51\_135 q=m51\_q p=m51\_p
      }{
         This example produces NDFs holding the Stokes \textit{Q} and \textit{U}
         parameters, again based on the four NDFs M51\_0, m51\_45, m51\_90
         and m51\_135.
      }
   }
   \sstnotes{
      \sstitemlist{

         \sstitem
         A \htmlref{bad value}{se:badmasking} will appear in the output data
         and variance arrays when any of the four input data values is bad, or if
         the total intensity in the pixel is not positive.  The output variance
         values are also undefined when any of the four input variances is
         bad or negative, or any computed variance is not positive, or the
         percentage polarisation is not positive.

         \sstitem
         If the four input NDFs have different pixel-index bounds, then
         they will be trimmed to match before being added.  An error will
         result if they have no pixels in common.

         \sstitem
         The output NDFs are deleted if there is an error during the
         formation of the polarisation parameters.

         \sstitem
         The output NDFs obtain their \htmlref{QUALITY}{apndf:quality},
         \htmlref{AXIS}{apndf:axis}~ information, and \htmlref{TITLE}{apndf:title} from
         the IN1 NDF.  The following labels and units are also assigned:

         \begin{tabular}[h]{lll}
            I &  \texttt{"Total Intensity"} &       UNITS of IN1 \\
            IA & \texttt{"Total Intensity"} &       UNITS of IN1 \\
            IB & \texttt{"Total Intensity"} &       UNITS of IN1 \\
            IP & \texttt{"Polarised Intensity"} &   UNITS of IN1 \\
            P &  \texttt{"Percentage Polarisation"} & \texttt{"\%"} \\
            Q &  \texttt{"Stokes Q"} &              --- \\
            U &  \texttt{"Stokes U"} &              --- \\
            THETA & \texttt{"Polarisation Angle"} & \texttt{"Degrees"} \\
         \end{tabular}
      }
   }
   \sstdiytopic{
      Related Applications
   }{
KAPPA: \htmlref{VECPLOT}{VECPLOT};
\xref{POLPACK}{sun223}{}: \xref{POLCAL}{sun223}{POLCAL},
\xref{POLVEC}{sun223}{POLVEC};
\xref{TSP}{sun66}{}.
   }
   \sstimplementationstatus{
      \sstitemlist{

         \sstitem
         This routine correctly processes the \htmlref{AXIS}{apndf:axis}, DATA, \htmlref{QUALITY}{apndf:quality}, \htmlref{VARIANCE}{apndf:variance},
         \htmlref{LABEL}{apndf:label}, \htmlref{TITLE}{apndf:title}, \htmlref{UNITS}{apndf:units}, \htmlref{WCS}{apndf:wcs}, and \htmlref{HISTORY}{apndf:history}~ components of the input NDF
         and propagates all \htmlref{extensions}{apndf:extensions}.

         \sstitem
         Processing of \htmlref{bad pixels}{se:masking} and automatic \htmlref{quality masking}{se:qualitymask} are supported.

         \sstitem
         All \htmlref{non-complex numeric data types}{ap:HDStypes} can be handled.  Arithmetic
         is performed using single-precision floating point.

      }
   }
}
\sstroutine{
   CARPET
}{
   Creates a cube representing a carpet plot of an image
}{
   \sstdescription{
      This application creates a new three-dimensional NDF from an
      existing two-dimensional NDF. The resulting NDF can, for instance,
      be viewed with the three-dimensional iso-surface facilities of the
      \GAIAref\ image viewer, in order to create a display similar to a
      {\em carpet plot} of the image (the iso-surface at value zero represents
      the input image data values).

      The first two pixel axes (\textit{x} and \textit{y}) in the output cube correspond to the
      pixel axes in the input image. The third pixel axis (\textit{z}) in the output
      cube is proportional to data value in the input image. The value of
      a pixel in the output cube measures the difference between the data
      value implied by its \textit{z}-axis position, and the data value of the
      corresponding pixel in the input image. Two schemes are available
      (see Parameter MODE): the output pixel values can be either simply
      the difference between these two data values, or the difference
      divided by the standard deviation at the corresponding pixel in the
      input image (as determined either from the \htmlref{VARIANCE}{apndf:variance}
      component in the input NDF or by Parameter SIGMA).
   }
   \sstusage{
      carpet in out [ndatapix] [range] [mode] [sigma]
   }
   \sstparameters{
      \sstsubsection{
         IN = NDF (Read)
      }{
         The input two-dimensional NDF.
      }
      \sstsubsection{
         MODE = LITERAL (Read)
      }{
         Determines how the pixel values in the output cube are calculated.

         \sstitemlist{

            \sstitem
            \texttt{"Data"} --- the value of each output pixel is equal to the difference
                         between the data value implied by its position along
                         the data value axis, and the value of the
                         corresponding pixel in the input image.

            \sstitem
            \texttt{"Sigma"} --- this is the same as \texttt{"Data"} except that the output
                         pixel values are divided by the standard deviation
                         implied either by the VARIANCE component of the
                         input image, or by the SIGMA parameter.

         }
         [\texttt{"Data"}]
      }
      \sstsubsection{
         NDATAPIX = \_INTEGER (Read)
      }{
         The number of pixels to use for the data value axis in the
         output cube. The pixel origin of this axis will be 1. The
         dynamic default is the square root of the number of pixels in
         the input image. This gives a fairly `cubic' output cube. \texttt{[]}
      }
      \sstsubsection{
         OUT = NDF (Write)
      }{
         The output three-dimensional NDF.
      }
      \sstsubsection{
         RANGE = LITERAL (Read)
      }{
         RANGE specifies the range covered by the data value axis (\emph{i.e.} the
         third pixel axis) in the output cube. The supplied string should
         consist of up to three sub-strings, separated by commas.  For all
         but the option where you give explicit numerical limits, the first
         sub-string must specify the method to use.  If supplied, the other
         two sub-strings should be numerical values as described below
         (default values will be used if these sub-strings are not
         provided).  The following options are available.

         \ssthitemlist{

            \sstitem
            lower,upper --- You can supply explicit lower and upper limiting
            values.  For example, \texttt{"10,200"} would set the lower limit on the
            output data axis to 10 and its upper limit to 200.  No method name
            prefixes the two values.  If only one value is supplied, the
            \texttt{"Range"} method is adopted.  The limits must be within the dynamic
            range for the data type of the input NDF array component.

            \sstitem
            \texttt{"Percentiles"} --- The default values for the output data axis
            range are set to the specified percentiles of the input data.  For
            instance, if the value \texttt{"Per,10,99"} is supplied, then the lowest
            10\% and highest 1\% of the data values are beyond the bounds of
            the output data value axis.  If only one value, p1, is supplied,
            the second value, p2, defaults to (100 - p1).  If no values are
            supplied, the values default to \texttt{"5,95"}.  Values must be in the
            range 0 to 100.

            \texttt{"Range"} --- The minimum and maximum array values are used.  No
            other sub-strings are needed by this option.  Null (\texttt{{!}}) is a
            synonym for the \texttt{"Range"} method.

            \sstitem
            \texttt{"Sigmas"} --- The limits on the output data value axis are set to
            the specified numbers of standard deviations below and above the
            mean of the input data.  For instance, if the supplied value is
            \texttt{"sig,1.5,3.0"}, then the data value axis extends from the mean of
            the input data minus 1.5 standard deviations to the mean plus 3
            standard deviations.  If only one value is supplied, the second
            value defaults to the supplied value.  If no values are supplied,
            both default to \texttt{"3.0"}.

         }
         The limits adopted for the data value axis are reported unless
         Parameter RANGE is specified on the command line.  In this case
         values are only calculated where necessary for the chosen method.

         The method name can be abbreviated to a single character, and
         is case insensitive.  The initial default value is \texttt{"Range"}.  The
         suggested defaults are the current values, or \texttt{{!}} if these do
         not exist.  \texttt{[}current value\texttt{{]}}
      }
      \sstsubsection{
         SIGMA = \_REAL (Read)
      }{
         The standard deviation to use if Parameter MODE is set to \texttt{"Sigma"}.
         If a null (\texttt{{!}}) value is supplied, the standard deviations implied
         by the VARIANCE component in the input image are used (an error
         will be reported if the input image does not have a VARIANCE
         component). If a SIGMA value is supplied, the same value is used
         to scale all output pixels.  \texttt{[!]}
      }
   }
   \sstexamples{
      \sstexamplesubsection{
         carpet m31 m31-cube mode=sigma
      }{
         Asssuming the two-dimensional NDF in file \texttt{m31.sdf} contains a
         VARIANCE component, this will create a three-dimensional NDF called
         m31-cube in which the third pixel axis corresponds to data value in
         NDF m31, and each output pixel value is the number of standard
         deviations of the pixel away from the corresponding input data value.
         If you then use \GAIA\ to view the cube, an iso-surface at value zero
         will be a carpet plot of the data values in m31, an iso-surface at
         value \texttt{-1.0} will be a carpet plot showing data values one standard
         deviation below the m31 data values, and an iso-surface at value \texttt{+1.0}
         will be a carpet plot showing data values one sigma above the
         m31 data values. This can help to visualise the errors in an image.
      }
   }
   \sstimplementationstatus{
      \sstitemlist{

         \sstitem
         Any \htmlref{VARIANCE}{apndf:variance} and \htmlref{QUALITY}{apndf:quality} components
         in the input image are not propagated to the output cube.
      }
   }
}
\sstroutine{
   CDIV
}{
   Divides an NDF by a scalar
}{
   \sstdescription{
      This application divides each pixel of an \NDFref{NDF} by a scalar
      (constant) value to produce a new NDF.
   }
   \sstusage{
      cdiv in scalar out
   }
   \sstparameters{
      \sstsubsection{
         IN = NDF (Read)
      }{
         Input NDF structure whose pixels are to be divided by a
         scalar.
      }
      \sstsubsection{
         OUT = NDF (Write)
      }{
         Output NDF structure.
      }
      \sstsubsection{
         SCALAR = \_DOUBLE (Read)
      }{
         The value by which the NDF's pixels are to be divided.
      }
      \sstsubsection{
         TITLE = LITERAL (Read)
      }{
         A \htmlref{title}{apndf:title} for the output NDF.  A null value will cause the title
         of the NDF supplied for Parameter IN to be used instead.
         \texttt{[!]}
      }
   }
   \sstexamples{
      \sstexamplesubsection{
         cdiv a 100.0 b
      }{
         Divides all the pixels in the NDF called a by the constant
         value 100.0 to produce a new NDF called b.
      }
      \sstexamplesubsection{
         cdiv in=data1 out=data2 scalar=-38
      }{
         Divides all the pixels in the NDF called data1 by $-$38 to give
         data2.
      }
   }
   \sstdiytopic{
      Related Applications
   }{
KAPPA: \htmlref{ADD}{ADD},
\htmlref{CADD}{CADD},
\htmlref{CMULT}{CMULT},
\htmlref{CSUB}{CSUB},
\htmlref{DIV}{DIV},
\htmlref{MATHS}{MATHS},
\htmlref{MULT}{MULT},
\htmlref{SUB}{SUB}.
   }
   \sstimplementationstatus{

      \sstitemlist{

         \sstitem
         This routine correctly processes the \htmlref{AXIS}{apndf:axis}, DATA, \htmlref{QUALITY}{apndf:quality},
         \htmlref{LABEL}{apndf:label}, \htmlref{TITLE}{apndf:title}, \htmlref{UNITS}{apndf:units}, \htmlref{HISTORY}{apndf:history}, \htmlref{WCS}{apndf:wcs}, and \htmlref{VARIANCE}{apndf:variance}~ components of an NDF
         data structure and propagates all \htmlref{extensions}{apndf:extensions}.

         \sstitem
         Processing of \htmlref{bad pixels}{se:masking} and automatic \htmlref{quality masking}{se:qualitymask} are supported.

         \sstitem
         All \htmlref{non-complex numeric data types}{ap:HDStypes} can be handled.
         Arithmetic is carried out using the appropriate floating-point
         type, but the numeric type of the input pixels is preserved in
         the output NDF.

         \sstitem
         Huge NDFs are supported.
      }
   }
}
\sstroutine{
   CENTROID
}{
   Finds the centroids of star-like features in an NDF
}{
   \sstdescription{
      This routine takes an \NDFref{NDF} and returns the co-ordinates of the
      centroids of features in its data array given approximate initial
      co-ordinates.  A feature is a set of connected pixels which are
      above or below the surrounding background region.  For example, a
      feature could be a star or galaxy on the sky, although the
      applications is not restricted to two-dimensional NDFs.

      Four methods are available for obtaining the initial positions,
      selected using Parameter MODE:

      \sstitemlist{

         \sstitem
         from the parameter system (see Parameter INIT);

         \sstitem
         Using a graphics cursor to indicate the feature in a previously
         displayed data array (see Parameter DEVICE);

         \sstitem
         from a specified positions list (see Parameter INCAT); or

         \sstitem
         from a simple text file containing a list of co-ordinates (see
         Parameter COIN).

      }
      In the first two modes the application loops, asking for new
      feature co-ordinates until it is told to quit or encounters an error.

      The results may optionally be written to an output positions list
      which can be used to pass the positions on to another application
      (see Parameter OUTCAT), or to a log file geared more towards human
      readers, including details of the input parameters (see Parameter
      LOGFILE).

      The uncertainty in the centroid positions may be estimated if
      variance values are available within the supplied NDF (see
      Parameter CERROR).
   }
   \sstusage{
      centroid ndf [mode]
        $\left\{ {\begin{tabular}{l}
                  init \\
                  coin=? \\
                  incat=?
                  \end{tabular} }
        \right.$ [search] [maxiter] [maxshift] [toler]
        \newline\latexhtml{\hspace*{10em}}{~~~~~~~~~~~~~~~~~~}
        \makebox[0mm][c]{\small mode}
   }
   \sstparameters{
      \sstsubsection{
         CATFRAME = LITERAL (Read)
      }{
         A string determining the \htmlref{co-ordinate Frame}{se:domains}~  in which positions are
         to be stored in the output catalogue associated with Parameter
         OUTCAT.  The string supplied for CATFRAME can be one of the
         following options.

         \ssthitemlist{

            \sstitem
            A \htmlref{domain name}{se:domains}~ such as \htmlref{SKY, AXIS, PIXEL}{se:resdoms}.

            \sstitem
            An integer value giving the index of the required Frame.

            \sstitem
            An IRAS90 \emph{Sky Co-ordinate System} (SCS) values such as
            \texttt{"EQUAT(J2000)"} (see \xref{SUN/163}{sun163}{}).

         }
         If a null (\texttt{{!}}) value is supplied, the positions will be stored
         in the current Frame. \texttt{[!]}
      }
      \sstsubsection{
         CATEPOCH = \_DOUBLE (Read)
      }{
         The epoch at which the sky positions stored in the output
         catalogue were determined.  It will only be accessed if an epoch
         value is needed to qualify the co-ordinate Frame specified by
         COLFRAME.  If required, it should be given as a decimal years
         value, with or without decimal places (\texttt{"1996.8"} for example).
         Such values are interpreted as a Besselian epoch if less than
         1984.0 and as a Julian epoch otherwise.
      }
      \sstsubsection{
         CERROR = \_LOGICAL (Read)
      }{
         If \texttt{TRUE}, errors in the centroided position will be calculated.
         The input NDF must contain a VARIANCE component in order to
         compute errors.  \texttt{[FALSE]}
      }
      \sstsubsection{
         COIN =  FILENAME (Read)
      }{
         Name of a text file containing the initial guesses at the
         co-ordinates of features to be centroided.  Only accessed if
         Parameter MODE is given the value \texttt{"File"}.  Each line should
         contain the \xref{formatted axis values}{sun210}{AST_UNFORMAT}
         for a single position, in the \htmlref{current Frame}{se:curframe}~ of
         the NDF.  Axis values can be separated by
         spaces, tabs or commas.  The file may contain comment lines
         with the first character \texttt{\#} or \texttt{!}.
      }
      \sstsubsection{
         DESCRIBE = \_LOGICAL (Read)
      }{
         If \texttt{TRUE}, a detailed description of the co-ordinate Frame in which
         the centroided positions will be reported is displayed before the
         positions themselves.  \texttt{[}current value\texttt{{]}}
      }
      \sstsubsection{
         DEVICE = \htmlref{DEVICE}{se:selgradev} (Read)
      }{
         The graphics device which is to be used to give the initial
         guesses at the centroid positions.  Only accessed if Parameter
         MODE is given the value \texttt{"Cursor"}.  \texttt{[}Current graphics device\texttt{{]}}
      }
      \sstsubsection{
         GUESS = \_LOGICAL (Read)
      }{
         If \texttt{TRUE}, then the supplied guesses for the centroid positions
         will be included in the screen and log file output, together
         with the accurate positions.  \texttt{[}current value\texttt{{]}}
      }
      \sstsubsection{
         INCAT = FILENAME (Read)
      }{
         A catalogue containing a positions list giving the initial
         guesses at the centroid positions, such as produced by
         applications CURSOR, LISTMAKE.  Only accessed if Parameter
         MODE is given the value \texttt{"Catalogue"}.
      }
      \sstsubsection{
         INIT = LITERAL (Read)
      }{
         An initial guess at the co-ordinates of the next feature to be
         centroided, in the current co-ordinate Frame of the NDF (supplying
         a colon \texttt{":"} will display details of the current co-ordinate Frame).
         The position should be supplied as a list of formatted axis values
         separated by spaces or commas.  INIT is only accessed if parameter
         MODE is given the value \texttt{"Interface"}.  If the initial co-ordinates
         are supplied on the command line only one centroid will be found;
         otherwise the application will ask for further guesses, which may
         be terminated by supplying the null value (\texttt{{!}}).
      }
      \sstsubsection{
         LOGFILE  =  FILENAME (Read)
      }{
         Name of the text file to log the results.  If null, there
         will be no logging.  Note this is intended for the human reader
         and is not intended for passing to other applications.  \texttt{[!]}
      }
      \sstsubsection{
         MARK = LITERAL (Read)
      }{
         Only accessed if Parameter MODE is given the value \texttt{"Cursor"}.  It
         indicates which positions are to be marked on the screen using the
         marker type given by Parameter MARKER.  It can take any of the
         following values.

         \ssthitemlist{

            \sstitem
            \texttt{"Initial"}:  The position of the cursor when the mouse button is
            pressed is marked.

            \sstitem
            \texttt{"Centroid"}:  The corresponding centroid position is marked.

            \sstitem
            \texttt{"None"}:  No positions are marked.

         }
         \texttt{[}current value\texttt{{]}}
      }
      \sstsubsection{
         MARKER = \_INTEGER (Read)
      }{
         This parameter is only accessed if Parameter MARK is set \texttt{TRUE}.
         It specifies the type of marker with which each cursor position
         should be marked, and should be given as an integer \PGPLOT\  marker
         type.  For instance, \texttt{0} gives a box, \texttt{1} gives a dot,
         \texttt{2} gives a cross, \texttt{3} gives an asterisk, \texttt{7} gives a
         triangle.  The value must be larger than or equal to $-$31.
         \texttt{[}current value\texttt{{]}}
      }
      \sstsubsection{
         MAXITER = \_INTEGER (Read)
      }{
         Maximum number of iterations to be used in the search.  It must
         be in the range 1--9.  The dynamic default is 3.  \texttt{[9]}
      }
      \sstsubsection{
         MAXSHIFT() = \_REAL (Read)
      }{
         Maximum shift in each dimension allowed between the guess and
         output positions in pixels.  Each must lie in the range
         0.0--26.0.  If only a single value is given, then it will be
         duplicated to all dimensions.  The dynamic default is half of
         SEARCH $+$ 1.  \texttt{[9.0]}
      }
      \sstsubsection{
         MODE = \htmlref{LITERAL}{se:parmenu} (Read)
      }{
         The \htmlref{mode}{se:interaction} in which the initial co-ordinates
         are to be obtained.  The supplied string can be one of the
         following values.

         \ssthitemlist{

            \sstitem
            \texttt{"Interface"} --- positions are obtained using Parameter INIT.

            \sstitem
            \texttt{"Cursor"} --- positions are obtained using the graphics cursor of
               the device specified by Parameter DEVICE.

            \sstitem
            \texttt{"Catalogue"} --- positions are obtained from a positions list
               using Parameter INCAT.

            \sstitem
            \texttt{"File"} --- positions are obtained from a text file using Parameter
               COIN.
         }
         \texttt{[}current value\texttt{{]}}
      }
      \sstsubsection{
         NDF = NDF (Read)
      }{
         The NDF structure containing the data array to be analysed.  In
         cursor mode (see Parameter MODE), the run-time default is the
         displayed data, as recorded in the graphics database.  In other
         modes, there is no run-time default and the user must supply a
         value.  \texttt{[]}
      }
      \sstsubsection{
         NSIM =  \_INTEGER (Read)
      }{
         The number of simulations or realisations using the variance
         information in order to estimate the error in the centroid
         position.  The uncertainty in the centroid error decreases
         as one over the square root of NSIM.  The range of acceptable
         values is 3--10000.  \texttt{[100]}
      }
      \sstsubsection{
         OUTCAT = FILENAME (Write)
      }{
         The output catalogue in which to store the centroided positions.
         If a null value (\texttt{{!}}) is supplied, no output catalogue is produced.
         See also Parameter CATFRAME.  \texttt{[!]}
      }
      \sstsubsection{
         PLOTSTYLE = \htmlref{GROUP}{se:groups} (Read)
      }{
         A group of attribute settings describing the style to use when
         drawing the graphics markers specified by Parameter MARK.

         A comma-separated list of strings should be given in which each
         string is either an attribute setting, or the name of a text
         file preceded by an up-arrow character \texttt{"$\wedge$"}.  Such text files
         should contain further comma-separated lists which will be
         read and interpreted in the same manner.  Attribute settings
         are applied in the order in which they occur within the list,
         with later settings overriding any earlier settings given for
         the same attribute.

         Each individual attribute setting should be of the form:

            $<$name$>$=$<$value$>$


         where $<$name$>$ is the name of a plotting attribute, and $<$value$>$
         is the value to assign to the attribute.  Default values will be
         used for any unspecified attributes.  All attributes will be
         defaulted if a null value (\texttt{{!}})---the initial default---is supplied.
         To apply changes of style to only the current invocation, begin these
         attributes with a plus sign.  A mixture of persistent and temporary
         style changes is achieved by listing all the persistent attributes
         followed by a plus sign then the list of temporary attributes.

         See \slhyperref{Plotting Attributes}{Section~}{}{ap:plotting_attr}
         for a description of the available attributes.  Any unrecognised
         attributes are ignored (no error is reported).
         \texttt{[}current value\texttt{{]}}
      }
      \sstsubsection{
         POSITIVE = \_LOGICAL (Read)
      }{
         \texttt{TRUE}, if array features are positive above the
         background.  \texttt{[TRUE]}
      }
      \sstsubsection{
         SEARCH() = \_INTEGER (Read)
      }{
         Size in pixels of the search box to be used.  If only a single
         value is given, then it will be duplicated to all dimensions
         so that a square, cube or hypercube region is searched.
         Each value must be odd and lie in the range 3--51.  \texttt{[9]}
      }
      \sstsubsection{
         TITLE = LITERAL (Read)
      }{
         A title to store with the output catalogue specified by
         Parameter OUTCAT, and to display before the centroid positions
         are listed.  If a null (\texttt{{!}}) value is supplied, the title is taken
         from any input catalogue specified by Parameter INCAT, or is a
         fixed string including the name of the NDF.  \texttt{[!]}
      }
      \sstsubsection{
         TOLER = \_REAL (Read)
      }{
         Accuracy in pixels required in centroiding.  Iterations will
         stop when the shift between successive centroid positions
         is less than the accuracy.  The accuracy must lie in the range
         0.0--2.0.  \texttt{[0.05]}
      }
   }
   \sstresparameters{
      \sstsubsection{
         CENTRE = LITERAL (Write)
      }{
         The formatted co-ordinates of the last centroid position, in
         the current Frame of the NDF.
      }
      \sstsubsection{
         ERROR = LITERAL (Write)
      }{
         The errors associated with the position written to Parameter
         CENTRE.
      }
      \sstsubsection{
         XCEN = LITERAL (Write)
      }{
         The formatted \textit{x} co-ordinate of the last centroid position, in the
         current co-ordinate Frame of the NDF.
      }
      \sstsubsection{
         XERR = LITERAL (Write)
      }{
         The error associated with the value written to Parameter XCEN.
      }
      \sstsubsection{
         YCEN = LITERAL (Write)
      }{
         The formatted \textit{y} co-ordinate of the last centroid position, in the
         current co-ordinate Frame of the NDF.
      }
      \sstsubsection{
         YERR = LITERAL (Write)
      }{
         The error associated with the value written to Parameter YCEN.
      }
   }
   \sstexamples{
      \sstexamplesubsection{
         centroid cluster cu
      }{
         This finds the centroids in the NDF called cluster via the
         graphics cursor on the \htmlref{current graphics device}{se:devglobal}.
      }
      \sstexamplesubsection{
         centroid cluster cu search=21 mark=ce plotstyle='colour=red'
      }{
         This finds the centroids in the NDF called cluster via the
         graphics cursor on the current graphics device.  The search
         box for the centroid is 21 pixels in each dimension.  The centroid
         positions are marked using a red symbol.
      }
      \sstexamplesubsection{
         centroid cluster i "21.7,5007.1"
      }{
         This finds the centroid of the object in the two-dimensional NDF
         called cluster around the current Frame co-ordinate (21.7,5007.1).
      }
      \sstexamplesubsection{
         centroid arp244(6,,) i "40,30" toler=0.01
      }{
         This finds the two-dimensional centroid of the feature near
         pixel (6,40,30) in the three-dimensional NDF called arp244 (assuming
         the current co-ordinate Frame of the NDF is PIXEL).  The centroid
         must be found to 0.01 pixels.
      }
      \sstexamplesubsection{
         centroid cluster cu xcen=(xp) ycen=(yp)
      }{
         This finds the centroid of an object in the two-dimensional NDF
         called cluster using a graphics cursor, and writes the centroid
         co-ordinates to ICL variables XP and YP for use in other applications.
      }
      \sstexamplesubsection{
         centroid cluster mode=file coin=objects.dat logfile=centroids.log
      }{
         This finds the centroids in the NDF called cluster.  The
         initial positions are given in the text file objects.dat in
         the current co-ordinate Frame.  A log of the input parameter
         values, initial and centroid positions is written to the text
         file \texttt{centroids.log}.
      }
      \sstexamplesubsection{
         centroid cluster mode=cat incat=a outcat=b catframe=ecl
      }{
         This example reads the initial guess positions from the
         positions list in file a.FIT, and writes the accurate centroid
         positions to positions list file b.FIT, storing the output
         positions in ecliptic co-ordinates.  The input file may, for
         instance, have been created using the application CURSOR.
      }
   }
   \sstnotes{
      \sstitemlist{

         \sstitem
         All positions are supplied and reported in the current co-ordinate
         Frame of the NDF.  A description of the co-ordinate Frame being used
         is given if Parameter DESCRIBE is set to a \texttt{TRUE} value.  Application
         \htmlref{WCSFRAME}{WCSFRAME} can be used to change the current co-ordinate Frame of the
         NDF before running this application if required.

         \sstitem
         In Cursor or Interface mode, only the first 200 supplied positions
         will be stored in the output catalogue.  Any further positions will be
         displayed on the screen but not stored in the output catalogue.

         \sstitem
         The centroid positions are not displayed on the screen when the
         message filter environment variable MSG\_FILTER is set to \texttt{QUIET}.
         The creation of output parameters and files is unaffected by MSG\_FILTER.

      }
   }
   \sstdiytopic{
      Estimation of Centroid Positions
   }{
      Each centroid position is obtained by projecting the data values
      within a search box centred on the supplied position, on to each
      axis in turn.  This forms a set of profiles for the feature, one for
      each axis.  An estimate of the background at each point in these
      profiles is made and subtracted from the profile.  This flattens the
      profile backgrounds, removing any slope in the data.  Once the
      profiles have been flattened in this way, and estimate of the
      background noise in each is made.  The centroid of the feature is
      then found using only the data above the noise level.

      Successive estimates of the centroid position are made by using the
      previous estimate of the centroid as the initial position for
      another estimation.  This loop is repeated up to a maximum number of
      iterations, though it normally terminates when a desired accuracy
      has been achieved.

      The achieved accuracy is affected by noise, and the presence of
      non-Gaussian or overlapping features, but typically an accuracy
      better than 0.1 pixel is readily attainable for stars.  The error in
      the centroid position may be estimated by a Monte-Carlo method
      using the data variance to generate realisations of the data about
      the feature (see Parameter CERROR).  Each realisation is processed
      identically to the actual data, and statistics are formed to derive
      the standard deviations.
   }
   \sstdiytopic{
      Related Applications
   }{
KAPPA: \htmlref{PSF}{PSF},
\htmlref{CURSOR}{CURSOR},
\htmlref{LISTSHOW}{LISTSHOW},
\htmlref{LISTMAKE}{LISTMAKE}.
   }
   \sstimplementationstatus{
      \sstitemlist{

         \sstitem
         The processing of \htmlref{bad pixels}{se:masking} and all \htmlref{non-complex numeric data types}{ap:HDStypes}
         is supported.
      }
   }
}
\sstroutine{
   CHAIN
}{
   Concatenates a series of vectorized NDFs
}{
   \sstdescription{
      This application concatenates a series of \NDFref{NDFs} in the order
      supplied and treated as vectors, to form a one-dimensional output
      NDF.  The dimensions of the NDFs may be different, and indeed so
      may their dimensionalities.
   }
   \sstusage{
      chain in c1 [c2] [c3] ... [c25] out=?
   }
   \sstparameters{
      \sstsubsection{
         IN = NDF (Read)
      }{
         The base NDF after which the other input NDFs will be
         concatenated.
      }
      \sstsubsection{
         OUT = NDF (Write)
      }{
         The one-dimensional NDF resulting from concatenating the input
         NDFs.
      }
      \sstsubsection{
         C1-C25 = NDF (Read)
      }{
         The NDFs to be concatenated to the base NDF.  The NDFs are
         joined in the order C1, C2, \ldots C25.  There can be no missing
         NDFs, \emph{e.g.} in order for C3 to be processed there must be a C2
         given as well.  A null value (\texttt{{!}}) indicates that there is no
         NDF.  NDFs C2 to C25 are defaulted to \texttt{{!}}.  At least one NDF
         must be pasted, therefore C1 may not be null.
      }
      \sstsubsection{
         TITLE = LITERAL (Read)
      }{
         \htmlref{Title}{apndf:title} for the output NDF structure.  A null value (\texttt{{!}})
         propagates the title from the base NDF to the output NDF.  \texttt{[!]} }
   }
   \sstexamples{
      \sstexamplesubsection{
         chain obs1 obs2 out=stream
      }{
         This concatenates the NDF called obs2 on to the arrays in the
         NDF called obs1 to produce the one-dimensional NDF stream.
      }
      \sstexamplesubsection{
         chain c1=obs2 c2=obs1 in=obs3 out=stream
      }{
         This concatenates the NDF called obs2 on to the arrays in the
         NDF called obs3, and then concatenates the arrays from obs1
         to them to produce the one-dimensional NDF stream.
      }
   }
   \sstdiytopic{
      Related Applications
   }{
KAPPA: \htmlref{PASTE}{PASTE},
\htmlref{RESHAPE}{RESHAPE}.
   }
   \sstimplementationstatus{
      \sstitemlist{

         \sstitem
         This routine correctly processes the DATA, \htmlref{QUALITY}{apndf:quality},
         \htmlref{VARIANCE}{apndf:variance}, \htmlref{LABEL}{apndf:label},
         \htmlref{TITLE}{apndf:title}, \htmlref{UNITS}{apndf:units}, and
         \htmlref{HISTORY}{apndf:history}, components of an NDF data structure and
         propagates all \htmlref{extensions}{apndf:extensions}.  Propagation is from
         the base NDF.  \htmlref{WCS}{apndf:wcs}, and \htmlref{AXIS}{apndf:axis}~ information
         is lost.

         \sstitem
         All \htmlref{non-complex numeric data types}{ap:HDStypes} can be handled.

         \sstitem
         Any number of NDF dimensions is supported.

      }
   }
}

\sstroutine{
   CHANMAP
}{
   Creates a channel map from a cube NDF by compressing slices along
   a nominated axis
}{
   \sstdescription{
      This application creates a two-dimensional channel-map image from
      a three-dimensional NDF.  It collapses along a nominated pixel
      axis in a series of slices.  The collapsed slices are tiled with
      no margins to form the output image.  This grid of channel maps
      is filled from left to right, and bottom to top.  A specified
      range of axis values can be used instead of the whole axis (see
      Parameters LOW and HIGH).  The number of channels and their
      arrangement into an image is controlled through Parameters
      NCHAN and SHAPE.

      For each output pixel, all corresponding input pixel values
      between the channel bounds of the nominated axis to be
      collapsed are combined together using one of a selection of
      estimators, including a mean, mode, or median, to produce the
      output pixel value.
   }
   \sstusage{
      chanmap in out axis nchan shape [low] [high] [estimator] [wlim]
   }
   \sstparameters{
      \sstsubsection{
         AXIS = LITERAL (Read)
      }{
         The axis along which to collapse the NDF.
         This can be specified using one of the following options.

         \ssthitemlist{

            \sstitem
            Its integer index within the \htmlref{current
            Frame}{se:curframe}~ of the input NDF (in the range 1 to
            the number of axes in the current Frame).

            \sstitem
            Its \htmlattref{Symbol}{Symbol(axis)}~ string such as
            \texttt{"RA"} or \texttt{"VRAD"}.

            \sstitem
            A generic option where \texttt{"SPEC"} requests the spectral axis,
            \texttt{"TIME"} selects the time axis, \texttt{"SKYLON"} and
            \texttt{"SKYLAT"} picks the sky longitude and latitude axes
            respectively.  Only those axis domains present are
            available as options.
         }

         A list of acceptable values is displayed if an illegal value
         is supplied.  If the axes of the current Frame are not
         parallel to the NDF pixel axes, then the pixel axis which is
         most nearly parallel to the specified current Frame axis will
         be used.
      }
      \sstsubsection{
         CLIP = \_REAL (Read)
      }{
         The number of standard deviations about the mean at which to
         clip outliers for the \texttt{"Mode"}, \texttt{"Cmean"} and \texttt{"Csigma"}
         statistics (see Parameter ESTIMATOR).  The application first computes
         statistics using all the available pixels.  It then rejects
         all those pixels whose values lie beyond CLIP standard
         deviations from the mean and will then re-evaluate the
         statistics.  For \texttt{"Cmean"} and \texttt{"Csigma"} there is currently
         only one iteration, but up to seven for \texttt{"Mode"}.

         The value must be positive.  \texttt{[3.0]}
      }
      \sstsubsection{
         ESTIMATOR = \htmlref{LITERAL}{se:parmenu} (Read)
      }{
         The method to use for estimating the output pixel values.  It
         can be one of the following options.

         \ssthitemlist{

            \sstitem
            \texttt{"Mean"}  --- Mean value

            \sstitem
            \texttt{"WMean"}  --- Weighted mean in which each data value is weighted
                        by the reciprocal of the associated variance.  (2)

            \sstitem
            \texttt{"Mode"}   --- Modal value  (4)

            \sstitem
            \texttt{"Median"} --- Median value.  Note that this is extremely memory
                        and CPU intensive for large datasets; use with
                        care!  If strange things happen, use \texttt{"Mean"}.  (3)

            \sstitem
            \texttt{"Absdev"} --- Mean absolute deviation from the unweighted mean.  (2)

           \sstitem
            \texttt{"Cmean"}  --- Sigma-clipped mean.  (4)

            \sstitem
            \texttt{"Csigma"} --- Sigma-clipped standard deviation.  (4)

            \sstitem
            \texttt{"Comax"}  --- Co-ordinate of the maximum value.

            \sstitem
            \texttt{"Comin"}  --- Co-ordinate of the minimum value.

            \sstitem
            \texttt{"FBad"}   --- Fraction of bad pixel values.

            \sstitem
            \texttt{"FGood"}  --- Fraction of good pixel values.

            \sstitem
            \texttt{"Integ"}  --- Integrated value, being the sum of the products
                        of the value and pixel width in world co-ordinates.  Note
                        that for sky co-ordinates the width is measured in radians.

            \sstitem
            \texttt{"Iwc"}    --- Intensity-weighted co-ordinate, being the sum of
                        each value times its co-ordinate, all divided by
                        the integrated value (see the \texttt{"Integ"} option).

            \sstitem
            \texttt{"Iwd"}    --- Intensity-weighted dispersion of the
                        co-ordinate, normalised like \texttt{"Iwc"} by the
                        integrated value.  (4)

            \sstitem
            \texttt{"Max"}    --- Maximum value.

            \sstitem
            \texttt{"Min"}    --- Minimum value.

            \sstitem
            \texttt{"NBad"}   --- Count of bad pixel values.

            \sstitem
            \texttt{"NGood"}  --- Count of good pixel values.

            \sstitem
            \texttt{"Rms"}    --- Root-mean-square value.  (4)

            \sstitem
            \texttt{"Sigma"}  --- Standard deviation about the unweighted mean.  (4)

            \sstitem
            \texttt{"Sum"}    --- The total value.
         }
         The selection is restricted if each channel contains three
         or fewer pixels.  For instance, measures of dispersion like
         \texttt{"Sigma"} and \texttt{"Iwd"} are meaningless for single-pixel
         channels. The minimum number of pixels per channel for each estimator
         is given in parentheses in the list above.  Where there is no number,
         there is no restriction.  If you supply an unavailable option, you
         will be informed, and presented with the available options. \texttt{["Mean"]}
      }
      \sstsubsection{
         HIGH = LITERAL (Read)
      }{
         Together with Parameter LOW, this parameter defines the range
         of values for the axis specified by Parameter AXIS to be
         divided into channels.  For example, if AXIS is 3 and the
         current Frame of the input NDF has axes RA/DEC/Wavelength, then
         a wavelength value should be supplied.  If, on the other hand,
         the current Frame in the NDF was the PIXEL Frame, then a pixel
         co-ordinate value would be required for the third axis (note,
         the pixel with index I covers a range of pixel co-ordinates
         from ($I-1$) to $I$).

         Note, HIGH and LOW should not be equal.  If a null value (\texttt{{!}}) is
         supplied for either HIGH or LOW, the entire range of the axis
         fragmented into channels.  \texttt{[!]}
      }
      \sstsubsection{
         IN  = NDF (Read)
      }{
         The input NDF.  This must have three dimensions.
      }
      \sstsubsection{
         LOW = LITERAL (Read)
      }{
         Together with Parameter HIGH this parameter defines the range
         of values for the axis specified by Parameter AXIS to be
         divided into channels.  For example, if AXIS is 3 and the
         current Frame of the input NDF has axes RA/DEC/Frequency, then
         a frequency value should be supplied.  If, on the other hand,
         the current Frame in the NDF was the PIXEL Frame, then a pixel
         co-ordinate value would be required for the third axis (note,
         the pixel with index I covers a range of pixel co-ordinates
         from ($I-1$) to $I$).

         Note, HIGH and LOW should not be equal.  If a null value
         (\texttt{{!}}) is supplied for either HIGH or LOW, the entire range
         of the axis fragmented into channels.  \texttt{[!]}
      }
      \sstsubsection{
         NCHAN = \_INTEGER (Read)
      }{
         The number of channels to appear in the channel map.  It must
         be a positive integer up to the lesser of 100 or
         the number of pixels along the collapsed axis.
      }
      \sstsubsection{
         OUT = NDF (Write)
      }{
         The output NDF.
      }
      \sstsubsection{
         SHAPE = \_INTEGER (Read)
      }{
         The number of channels along the $x$ axis of the output NDF.  The
         number along the $y$ axis will be (NCHAN-1)/SHAPE.  A null value
         (\texttt{{!}}) asks the application to select a shape.  It will
         generate one that gives the most square output NDF possible.
         The value must be positive and no more than the value of
         Parameter NCHAN.
      }
      \sstsubsection{
         TITLE = LITERAL (Read)
      }{
         Title for the output NDF structure.  A null value (\texttt{{!}})
         propagates the title from the input NDF to the output NDF.  \texttt{[!]}
      }
      \sstsubsection{
         USEAXIS = GROUP (Read)
      }{
         USEAXIS is only accessed if the current co-ordinate Frame of
         the input NDF has more than three axes.  A group of three
         strings should be supplied specifying the three axes which are
         to be retained in a collapsed slab.

         Each axis can be specified using one of the following options.

         \ssthitemlist{

            \sstitem
            Its integer index within the current Frame of the
            input  NDF (in the range 1 to the number of axes in the
            current Frame).

            \sstitem
            Its \htmlattref{Symbol}{Symbol(axis)}~ string such as
            \texttt{"RA"} or \texttt{"VRAD"}.

            \sstitem
            A generic option where \texttt{"SPEC"} requests the spectral axis,
            \texttt{"TIME"} selects the time axis, \texttt{"SKYLON"} and
            \texttt{"SKYLAT"} picks the sky longitude and latitude axes
            respectively.  Only those axis domains present are
            available as options.
         }

         A list of acceptable values is displayed if an illegal value
         is supplied.  If a null (\texttt{{!}}) value is supplied, the
         axes with the same indices as the three used pixel axes
         within the NDF are used.  \texttt{[!]}
      }
      \sstsubsection{
         WLIM = \_REAL (Read)
      }{
         If the input NDF contains bad pixels, then this parameter
         may be used to determine the number of good pixels which must
         be present within the range of collapsed input pixels before a
         valid output pixel is generated.  It can be used, for example,
         to prevent output pixels from being generated in regions where
         there are relatively few good pixels to contribute to the
         collapsed result.

         WLIM specifies the minimum fraction of good pixels which must
         be present in order to generate a good output pixel.  If this
         specified minimum fraction of good input pixels is not present,
         then a bad output pixel will result, otherwise a good output
         value will be calculated.  The value of this parameter should
         lie between 0.0 and 1.0 (the actual number used will be rounded
         up if necessary to correspond to at least one pixel).  [0.3]
      }
   }
   \pagebreak
   \sstexamples{
      \sstexamplesubsection{
         chanmap cube chan4 lambda 4 2 4500 4550
      }{
         The current Frame in the input three-dimensional NDF called
         cube has axes with labels \texttt{"RA"}, \texttt{"DEC"} and
         \texttt{"Lambda"}, with the lambda axis being parallel to the
         third pixel axis.  The above command extracts four slabs of
         the input cube between wavelengths 4500 and 4550 {\AA}ngstroms,
         and collapses each slab, into a single two-dimensional array
         with RA and DEC axes forming a channel image.  Each channel
         image is pasted into a 2$\times$2 grid within the output NDF
         called chan4.  Each pixel in the output NDF is the mean of the
         corresponding input pixels with wavelengths in 12.5-{\AA}ngstrom
         bins.
      }
      \sstexamplesubsection{
         chanmap in=cube out=chan4 axis=3 low=4500 high=4550 nchan=4 shape=2
      }{
         The same as above except the axis to collapse along is
         specified by index (3) rather than label (lambda), and it uses
         keywords rather than positional parameters.
      }
      \sstexamplesubsection{
         chanmap cube chan4 3 4 2 9.0 45.0
      }{
         This is the same as the above examples, except that the current
         Frame in the input NDF has been set to the PIXEL Frame (using
         \htmlref{WCSFRAME}{WCSFRAME}), and so the high and low axis values
         are specified in pixel co-ordinates instead of {\AA}ngstroms, and
         each channel covers nine pixels.  Note the difference between
         floating-point pixel co-ordinates, and integer pixel indices (for
         instance the pixel with index 10 extends from pixel co-ordinate
         9.0 to pixel co-ordinate 10.0).
      }
      \sstexamplesubsection{
         chanmap in=zcube out=vel7 axis=1 low=-30 high=40 nchan=7 shape=! estimator=max
      }{
         This command assumes that the zcube NDF has a current
         co-ordinate system where the first axis is radial velocity
         (perhaps selected using WCSFRAME and \htmlref{WCSATTRIB}{WCSATTRIB}),
         and the second and third axes are \texttt{"RA"}, and \texttt{"DEC"}.
         It extracts seven velocity slabs of the input cube between $-30$ and
         $+$40~km/s, and collapses each slab, into a single two-dimensional
         array with RA and DEC axes forming a channel image.  Each channel
         image is pasted into a default grid (likely 4$\times$2) within the
         output NDF called vel7.  Each pixel in the output NDF is the
         maximum of the corresponding input pixels with velocities in
         10-km/s bins.
      }
   }
   \sstnotes{
      \sstitemlist{

         \sstitem
         The collapse is always performed along one of the pixel axes,
         even if the current Frame in the input NDF is not the PIXEL Frame.
         Special care should be taken if the current-Frame axes are not
         parallel to the pixel axes.  The algorithm used to choose the
         pixel axis and the range of values to collapse along this pixel
         axis proceeds as follows.

         The current-Frame co-ordinates of the central pixel in the
         input NDF are determined (or some other point if the
         co-ordinates of the central pixel are undefined).  Two
         current-Frame positions are then generated by substituting in
         turn into this central position each of the HIGH and LOW
         values for the current-Frame axis specified by Parameter
         AXIS.  These two current-Frame positions are transformed into
         pixel co-ordinates, and the projections of the vector joining
         these two pixel positions on to the pixel axes are found.
         The pixel axis with the largest projection is selected as the
         collapse axis, and the two end points of the projection
         define the range of axis values to collapse.

         \sstitem
         The WCS of the output NDF retains the three-dimensional
         co-ordinate system of the input cube for every tile, except that
         each tile has a single representative mean co-ordinate for the
         collapsed axis.

         \sstitem
         The slices may have slightly different pixel depths depending
         where the boundaries of the channels lie in pixel co-ordinates.
         Excise care interpreting estimators like \texttt{"Sum"} or ensure
         equal numbers of pixels in each channel.
      }
   }
   \sstdiytopic{
      Related Applications
   }{
     KAPPA: \htmlref{COLLAPSE}{COLLAPSE},
     \htmlref{CLINPLOT}{CLINPLOT}.
   }
   \sstimplementationstatus{
      \ssthitemlist{

         \sstitem
         This routine correctly processes the DATA, \htmlref{VARIANCE}{apndf:variance},
         \htmlref{LABEL}{apndf:label}, \htmlref{TITLE}{apndf:title}, \htmlref{UNITS}{apndf:units},
         \htmlref{WCS}{apndf:wcs}, and \htmlref{HISTORY}{apndf:history}~ components of the input NDF; and
         propagates all \htmlref{extensions}{apndf:extensions}.  \htmlref{AXIS}{apndf:axis} and
         \htmlref{QUALITY}{apndf:quality} are not propagated.

         \sstitem
         Processing of \htmlref{bad pixels}{se:masking} and automatic
         \htmlref{quality masking}{se:qualitymask} are supported.

         \sstitem
         All \htmlref{non-complex numeric data types}{ap:HDStypes} can be handled.

         \sstitem
         The origin of the output NDF is at (1,1).

         \sstitem
         Huge NDFs are supported.
      }
   }
}

\sstroutine{
   CHPIX
}{
   Replaces the values of selected pixels in an NDF
}{
   \sstdescription{
      This application replaces selected elements of an \NDFref{NDF} array
      component with specified values.

      Two methods are available for obtaining the regions and
      replacement values, selected through Parameter MODE:

      \sstitemlist{
         \sstitem
         from the parameter system (see Parameters SECTION and NEWVAL).
         The task loops until there are no more elements to change,
         indicated by a null value in response to a prompt.  For
         non-interactive processing, supply the value of Parameter NEWVAL
         on the command line.

         \sstitem
         The second approach uses a text file, that is especially
         beneficial where there are too many section and value pairs to
         enter manually.  The file should contain two space-separated
         columns; the first column is the NDF section to replace, and the
         second supplies the value to insert into the section (see
         Parameter FILE).
     }
   }
   \sstusage{
      chpix in out section newval [comp]
   }
   \sstparameters{
      \sstsubsection{
         COMP = \htmlref{LITERAL}{se:parmenu} (Read)
      }{
         The name of the NDF array component to be modified.  The
         options are: \texttt{"Data"}, \texttt{"Error"}, \texttt{"Quality"} or
         \texttt{"Variance"}.  \texttt{"Error"} is the alternative to
         \texttt{"Variance"} and causes the
         square of the supplied replacement value to be stored in the
         output VARIANCE array.  \texttt{["Data"]}
      }
      \sstsubsection{
         FILE = FILENAME (Read)
      }{
         Name of a text file containing the sections and replacement
         values.  It is only accessed if Parameter MODE is given the
         value \texttt{"File"}.  Each line should contain an NDF section of
         a region (see Parameter SECTION), then one or more spaces,
         followed by the replacement value.  The value is either
         numeric or \texttt{"Bad"}, the latter requests that the bad value
         be inserted into the section.  The file may contain comment
         lines with the first character \texttt{\#} or \texttt{!}.
      }
      \sstsubsection{
         IN = NDF (Read)
      }{
         NDF structure containing the array component to be modified.
      }
      \sstsubsection{
         MODE = LITERAL (Read)
      }{
         The mode in which the sections and replacement values are to be
         obtained.  The supplied string can be one of the following values.
         \ssthitemlist{

           \sstitem
           \texttt{"Interface"} --- sections and values are obtained via Parameters
                                    SECTION and NEWVAL respectively.

           \sstitem
           \texttt{"File"} --- sections and values are obtained from a text file
                               specified by Parameter FILE.
        }
        \texttt{["Interface"]}
      \sstsubsection{
         NEWVAL = LITERAL (Read)
      }{
         Value to substitute in the output array element or elements.
         The range of allowed values depends on the data type of the
         array being modified.  NEWVAL=\texttt{"Bad"} instructs that the bad
         value appropriate for the array data type be substituted.
         Placing NEWVAL on the command line permits only one section
         to be replaced.  If there are multiple replacements, a null
         value (\texttt{{!}}) terminates the loop. If the section being
         modified contains only a single pixel, then the original value
         of that pixel is used as the suggested default value.
      }
      \sstsubsection{
         OUT = NDF (Write)
      }{
         Output NDF structure containing the modified version of
         the array component.
      }
      \sstsubsection{
         SECTION = LITERAL (Read)
      }{
         The elements to change.  This is defined as an NDF section, so
         that ranges can be defined along any axis, and be given as
         pixel indices or axis (data) co-ordinates.  So for example
         \texttt{"3,4,5"} would select the pixel at (3,4,5); \texttt{"3:5,"} would
         replace all elements in Columns 3 to 5; \texttt{",4"} replaces Line 4.
         See \slhyperref{NDF Sections}{Section~}{}{se:ndfsect}
         for details.  A null value (\texttt{{!}}) terminates the loop during multiple
         replacements.
      }
      \sstsubsection{
         TITLE = LITERAL (Read)
      }{
         \htmlref{Title}{apndf:title} for the output NDF structure.  A null
         value (\texttt{{!}}) propagates the title from the input NDF to the
         output NDF.  \texttt{[!]}
      }
   }
   \sstresparameters{
      \sstsubsection{
         OLDVAL = LITERAL (Write)
      }{
         If the section being modified contains only a single pixel, then
         the original value of that pixel is written out to this output
         parameter.
      }
   }
   \sstexamples{
      \sstexamplesubsection{
         chpix rawspec spectrum 55 100
      }{
         Assigns the value 100 to the pixel at index 55 within the
         one-dimensional NDF called rawspec, creating the output NDF
         called spectrum.
      }
      \sstexamplesubsection{
         chpix rawspec spectrum 10:19 0 error
      }{
         Assigns the value 0 to the error values at indices 10 to 19
         within the one-dimensional NDF called rawspec, creating the
         output NDF called spectrum.  The rawspec dataset must have a
         variance compoenent.
      }
      \sstexamplesubsection{
         chpix in=rawimage out=galaxy section="$\sim$20,100:109" newval=bad
      }{
         Assigns the bad value to the pixels in the section $\sim$20,100:109
         within the two-dimensional NDF called rawimage, creating the
         output NDF called galaxy.  This section is the central
         20 pixels along the first axis, and pixels 110 to 199 along the
         second.
      }
      \sstexamplesubsection{
         chpix in=zzcha out=zzcha\_c section="45,21," newval=-1
      }{
         Assigns value $-$1 to the pixels at index (45,~21) within all
         planes of the three-dimensional NDF called zzcha, creating
         the output NDF called zzcha\_c.
      }
      \sstexamplesubsection{
         chpix harpcube harpmasked mode=file file=badbaseline.txt
      }
         This reads the text file called \texttt{badbaseline.txt} to obtain the
         editing commands to be applied to the NDF called harpcube to
         form the NDF harpmasked.
      }
   }
   \sstdiytopic{
      Related Applications
   }{
KAPPA: \htmlref{ARDMASK}{ARDMASK},
\htmlref{FILLBAD}{FILLBAD},
\htmlref{GLITCH}{GLITCH},
\htmlref{NOMAGIC}{NOMAGIC},
\htmlref{REGIONMASK}{REGIONMASK},
\htmlref{SEGMENT}{SEGMENT},
\htmlref{SETMAGIC}{SETMAGIC},
\htmlref{SUBSTITUTE}{SUBSTITUTE},
\htmlref{ZAPLIN}{ZAPLIN};
\xref{FIGARO}{sun86}{}: \xref{CSET}{sun86}{CSET},
\xref{ICSET}{sun86}{ICSET},
\xref{NCSET}{sun86}{NCSET},
\xref{TIPPEX}{sun86}{TIPPEX}.
   }
   \sstimplementationstatus{
      \sstitemlist{

         \sstitem
         The routine correctly processes the \htmlref{AXIS}{apndf:axis}, DATA, \htmlref{QUALITY}{apndf:quality},
         \htmlref{VARIANCE}{apndf:variance}, \htmlref{LABEL}{apndf:label}, \htmlref{TITLE}{apndf:title}, \htmlref{UNITS}{apndf:units}, \htmlref{WCS}{apndf:wcs}, and \htmlref{HISTORY}{apndf:history}~ components of an NDF;
         and propagates all \htmlref{extensions}{apndf:extensions}.

         \sstitem
         Processing of \htmlref{bad pixels}{se:masking} and automatic \htmlref{quality masking}{se:qualitymask} are
         supported.

         \sstitem
         All \htmlref{non-complex numeric data types}{ap:HDStypes} can be handled.

      }
   }
}

\sstroutine{
   CLINPLOT
}{
   Draws a spatial grid of line plots for an axis of a cube NDF
}{
   \sstdescription{
      This application displays a three-dimensional \NDFref{NDF} as a
      series of line plots of array value against position, arranged on a
      uniform spatial grid and plotted on the current graphics device.
      The vertical axis of each line plot represents array value, and
      the horizontal axis represents position along a chosen axis (see
      Parameter USEAXIS).  All the line plots have the same axis
      limits.

      This application will typically be used to display a grid of
      spectra taken from a cube in which the
      \htmlref{current WCS Frame}{se:curframe}~
      includes one spectral axis (\emph{e.g.} frequency) and two
      spatial axes (\emph{e.g.} RA and Dec).  For this reason the
      following documentation refers to the `spectral axis' and the
      `spatial axes'.  However, cubes containing other types of axes
      can also be displayed, and references to `spectral' and
      `spatial' axes should be interpreted appropriately.

      A rectangular grid of \textit{NX} by \textit{NY} points (see
      Parameters NX and NY) is defined over the spatial extent of the
      cube, and a spectrum is drawn at each such point.  If \textit{NX}
      and \textit{NY} equal the spatial
      dimensions of the cube (which is the default for spatial axes of
      fewer than 31 pixels), then one spectrum is drawn for every
      spatial pixel in the cube.  For speed, the spectrum will be
      binned up so that the number of elements in the spectrum does
      not exceed the horizontal number of device pixels available for
      the line plot.

      Annotated axes for the spatial co-ordinates may be drawn around
      the grid of line plots (see Parameter AXES).  The appearance of
      these and the space they occupy may be controlled in detail (see
      Parameters STYLE and MARGIN).

      The plot may take several different forms such as a
      \texttt{"join-the-dots"} plot, a \texttt{"staircase"} plot, a
      \texttt{"chain"} plot (see Parameter MODE).  The plotting style
      (colour, founts, text size, \emph{etc.}) may be specified in
      detail using Parameter SPECSTYLE.

      The data value at the top and bottom of each line plot can be
      specified using Parameters YBOT and YTOP.  The defaults can be
      selected in several ways including percentiles (see Parameter
      LMODE).

      The current picture is usually cleared before plotting the new
      picture, but Parameter CLEAR can be used to prevent this,
      allowing the plot (say) to be drawn over the top of a previously
      displayed grey-scale image.

      The range and nature of the vertical and horizontal axes in each
      line plot can be displayed in a key to the right of the main
      plot (see Parameter KEY).  Also, an option exists to add
      numerical labels to the first (\emph{i.e.} bottom-left) line plot, see
      Parameter REFLABEL.  However, due to the nature of the plot, the
      text used may often be too small to read.
   }
   \sstusage{
      clinplot ndf [useaxis] [device] [nx] [ny]
   }
   \sstparameters{
      \sstsubsection{
         ALIGN = \_LOGICAL (Read)
      }{
         Controls whether or not the spectra should be aligned spatially
         with an existing data plot.  If ALIGN is \texttt{TRUE}, each spectrum
         will be drawn in a rectangular cell that is centred on the
         corresponding point on the sky.  This may potentially cause the
         spectra to overlap, depending on their spatial separation.  If
         ALIGN is \texttt{FALSE}, then the spectra are drawn in a regular grid
         of equal-sized cells that cover the entire picture.  This may cause
         them to be drawn at spatial positions that do not correspond to
         their actual spatial positions within the supplied cube.  The
         dynamic default is \texttt{TRUE} if Parameter CLEAR is \texttt{TRUE} and
         there is an existing DATA picture on the graphics device.  \texttt{[]}
      }
      \sstsubsection{
         AXES = \_LOGICAL (Read)
      }{
         \texttt{TRUE} if labelled and annotated axes describing the spatial
         are to be drawn around the outer edges of the plot.  The
         appearance of the axes can be controlled using the STYLE
         parameter.  The dynamic default is to draw axes only if the
         CLEAR parameter indicates that the graphics device is not
         being cleared.  \texttt{[]}
      }
      \sstsubsection{
         BLANKEDGE = \_LOGICAL (Read)
      }{
         If \texttt{TRUE} then no tick marks or labels are placed on the edges
         of line plots that touch the outer spatial axes (other edges
         that do not touch the outer axes will still be annotated).
         This can avoid existing tick marks being over-written when
         drawing a grid of spectra over the top of a picture that
         includes annotated axes.  The dynamic default is \texttt{TRUE} if and
         only if the graphics device is not being cleared (\emph{i.e.}
         Parameter CLEAR is \texttt{FALSE}) and no spatial axes are being drawn
         (\emph{i.e.} Parameter AXES is \texttt{FALSE}).  \texttt{[]}
      }
      \sstsubsection{
         CLEAR = \_LOGICAL (Read)
      }{
         If \texttt{TRUE} the current picture is cleared before the plot is
         drawn.  If \texttt{FALSE}, then the display is left uncleared and an
         attempt is made to align the spatial axes of the new plot
         with any spatial axes of the existing plot.  Thus, for
         instance, a while light image may be displayed using DISPLAY,
         and then spectra drawn over the top of the image using this
         application.  \texttt{[TRUE]}
      }
      \sstsubsection{
         COMP = \htmlref{LITERAL}{se:parmenu} (Read)
      }{
         The NDF array component to be displayed.  It may be \texttt{"Data"},
         \texttt{"Quality"}, \texttt{"Variance"}, or \texttt{"Error"} (where \texttt{"Error"}
         is an alternative to \texttt{"Variance"} and causes the square root of the
         variance values to be displayed).  If \texttt{"Quality"} is specified,
         then the quality values are treated as numerical values (in
         the range 0 to 255).  \texttt{["Data"]}
      }
      \sstsubsection{
         DOWNSAMPLE = \_LOGICAL (Read)
      }{
         If \texttt{TRUE} each spectrum is downsampled prior to plotting so that
         there are no more samples than device pixels.  If MODE is
         "Histogram" or "GapHistogram" there should be four pixels
         per sample.  \texttt{[TRUE]}
      }
      \sstsubsection{
         DEVICE = \htmlref{DEVICE}{se:selgradev} (Read)
      }{
         The name of the graphics device used to display the cube.
         \texttt{[}Current graphics device\texttt{{]}}
      }
      \sstsubsection{
         FILL = \_LOGICAL (Read)
      }{
         If FILL is set to \texttt{TRUE}, then the display will be `stretched'
         to fill the current picture in both directions.  This can be
         useful to elongate the spectra to reveal more detail by using
         more of the display surface at the cost of different spatial
         scales, and when the spatial axes have markedly different
         dimensions.  The dynamic default is \texttt{TRUE} if either of the
         spatial dimensions is one. and \texttt{FALSE} otherwise.  \texttt{[]}
      }
      \sstsubsection{
         KEY = \_LOGICAL (Read)
      }{
         If \texttt{TRUE}, then a `key' will be drawn to the right of the plot.
         The key will include information about the vertical and
         horizontal axes of the line plots, including the maximum and
         minimum value covered by the axis and the quantity
         represented by the axis.  The appearance of this key can be
         controlled using Parameter KEYSTYLE, and its position can be
         controlled using Parameter KEYPOS.  \texttt{[TRUE]}
      }
      \sstsubsection{
         KEYPOS() = \_REAL (Read)
      }{
         Two values giving the position of the key.  The first value
         gives the gap between the right-hand edge of the grid plot
         and the left-hand edge of the key (0.0 for no gap, 1.0 for
         the largest gap).  The second value gives the vertical
         position of the top of the key (1.0 for the highest position,
         0.0 for the lowest).  If the second value is not given, the
         top of the key is placed level with the top of the grid
         plot.  Both values should be in the range 0.0 to 1.0.  If a
         key is produced, then the right-hand margin specified by
         Parameter MARGIN is ignored.  \texttt{[}current value\texttt{{]}}
      }
      \sstsubsection{
         KEYSTYLE = GROUP (Read)
      }{
         A group of attribute settings describing the plotting style
         to use for the key (see Parameter KEY).

         A comma-separated list of strings should be given in which each
         string is either an attribute setting, or the name of a text
         file preceded by an up-arrow character \texttt{"$\wedge$"}.  Such text files
         should contain further comma-separated lists which will be
         read and interpreted in the same manner.  Attribute settings
         are applied in the order in which they occur within the list,
         with later settings overriding any earlier settings given for
         the same attribute.

         Each individual attribute setting should be of the form:

            $<$name$>$=$<$value$>$


         where $<$name$>$ is the name of a plotting attribute, and $<$value$>$
         is the value to assign to the attribute.  Default values will be
         used for any unspecified attributes.  All attributes will be
         defaulted if a null value (\texttt{{!}})---the initial default---is supplied.
         To apply changes of style to only the current invocation, begin these
         attributes with a plus sign.  A mixture of persistent and temporary
         style changes is achieved by listing all the persistent attributes
         followed by a plus sign then the list of temporary attributes.

         See \slhyperref{Plotting Attributes}{Section~}{}{ap:plotting_attr}
         for a description of the available attributes.  Any unrecognised
         attributes are ignored (no error is reported).

         The appearance of the text in the key can be changed by
         setting new values for the attributes
         \htmlattref{Colour(Strings)}{Colour(element)},
         \htmlattref{Font(Strings)}{Font(element)} \emph{etc.}
         \texttt{[}current value\texttt{{]}}
      }
      \sstsubsection{
         LMODE = LITERAL (Read)
      }{
         LMODE specifies how the defaults for Parameters YBOT and YTOP
         (the lower and upper limit of the vertical axis of each line
         plot) should be found.  The supplied string should consist of
         up to three sub-strings, separated by commas.  The first
         sub-string must specify the method to use.  If supplied, the
         other two sub-strings should be numerical values as described
         below (default values will be used if these sub-strings are
         not provided).  The following methods are available.

         \ssthitemlist{

            \sstitem
            \texttt{"Range"} --- The minimum and maximum data values in the
            supplied cube are used as the defaults for YBOT and YTOP.  No
            other sub-strings are needed by this option.

            \sstitem
            \texttt{"Extended"} --- The minimum and maximum data values in the
            cube are extended by percentages of the data range, specified
            by the second and third sub-strings.  For instance, if the
            value \texttt{"Ex,10,5"} is supplied, then the default for YBOT is set
            to the minimum data value minus 10\% of the data range, and
            the default for YTOP is set to the maximum data value plus 5\%
            of the data range.  If only one value is supplied, the second
            value defaults to the supplied value.  If no values are
            supplied, both values default to \texttt{"2.5"}.

            \sstitem
            \texttt{"Percentile"} --- The default values for YBOT and YTOP are
            set to the specified percentiles of the data in the supplied
            cube.  For instance, if the value \texttt{"Per,10,99"} is supplied,
            then the default for YBOT is set so that the lowest 10\% of
            the plotted points are off the bottom of the plot, and the
            default for YTOP is set so that the highest 1\% of the points
            are off the top of the plot.  If only one value, $p1$, is
            supplied, the second value, $p2$, defaults to $(100 - p1)$.  If
            no values are supplied, the values default to \texttt{"5,95"}.

            \sstitem
            \texttt{"Sigma"} --- The default values for YBOT and YTOP are set to
            the specified numbers of standard deviations below and above
            the mean of the data.  For instance, if the value
            \texttt{"sig,1.5,3.0"} is supplied, then the default for YBOT is set
            to the mean of the data minus 1.5 standard deviations, and
            the default for YTOP is set to the mean plus 3 standard
            deviations.  If only one value is supplied, the second value
            defaults to the supplied value.  If no values are provided
            both default to \texttt{"3.0"}.

         }
         The method name can be abbreviated to a single character, and
         is case insensitive.  The initial value is \texttt{"Range"}.
         \texttt{[}current value\texttt{{]}}
      }
      \sstsubsection{
         MARGIN( 4 ) = \_REAL (Read)
      }{
         The widths of the margins to leave around the outer spatial
         axes for axis annotations, given as fractions of the
         corresponding dimension of the current picture.  The actual
         margins used may be increased to preserve the aspect ratio of
         the data.  Four values may be given, in the order: bottom,
         right, top, left.  If fewer than four values are given, extra
         values are used equal to the first supplied value.  If these
         margins are too narrow any axis annotation may be clipped.
         If a null (\texttt{{!}}) value is supplied, the value used is (for all
         edges); \texttt{0.15} if annotated axes are being produced; and \texttt{0.0}
         otherwise.  The initial default is null.  \texttt{[}current value\texttt{{]}}
      }
      \sstsubsection{
         MARKER = \_INTEGER (Read)
      }{
         This parameter is only accessed if Parameter MODE is set to
         \texttt{"Chain"} or \texttt{"Mark"}.  It specifies the symbol with which each
         position should be marked, and should be given as an integer
         \PGPLOT\  marker type.  For instance, \texttt{0} gives a box,
         \texttt{1} gives a dot, \texttt{2} gives a cross, \texttt{3} gives an asterisk,
         \texttt{7} gives a triangle.  The value must be larger than or equal
         to $-$31.   \texttt{[}current value\texttt{{]}}
      }
      \sstsubsection{
         MODE = \htmlref{LITERAL}{se:parmenu} (Read)
      }{
         Specifies the way in which data values are represented.  MODE
         can take the following values.

         \ssthitemlist{

            \sstitem
            \texttt{"Histogram"} --- An histogram of the points is plotted in the
            style of a `staircase' (with vertical lines only joining the
            \textit{y}-axis values and not extending to the base of the plot). The
            vertical lines are placed midway between adjacent \textit{x}-axis
            positions.  Bad values are flanked by vertical lines to the
            lower edge of the plot.

            \sstitem
            \texttt{"GapHistogram"} --- The same as the \texttt{"Histogram"} option except bad
            values are not flanked by vertical lines to the lower edge of
            the plot, leaving a gap.

            \sstitem
            \texttt{"Line"} --- The points are joined by straight lines.

            \sstitem
            \texttt{"Point"} --- A dot is plotted at each point.

            \sstitem
            \texttt{"Mark"} --- Each point is marker with a symbol specified by
            Parameter MARKER.

            \sstitem
            \texttt{"Chain"} --- A combination of \texttt{"Line"} and \texttt{"Mark"}.

         }
         The initial default is \texttt{"Line"}.  \texttt{[}current value\texttt{{]}}
      }
      \sstsubsection{
         NDF = NDF (Read)
      }{
         The input NDF structure containing the data to be displayed.
         It should have three significant axes, \emph{i.e.} whose dimensions
         are greater than 1.
      }
      \sstsubsection{
         NX = \_INTEGER (Read)
      }{
         The number of spectra to draw in each row.  The spectra will
         be equally spaced over the bounds of the \textit{x} pixel axis. The
         dynamic default is the number of pixels along the \textit{x} axis of
         the NDF, so long as this value is no more than 30.  If the \textit{x}
         axis spans more than 30 pixels, then the dynamic default is
         30 (meaning that some spatial pixels will be ignored).  \texttt{[]}
      }
      \sstsubsection{
         NY = \_INTEGER (Read)
      }{
         The number of spectra to draw in each column.  The spectra
         will be equally spaced over the bounds of the \textit{y} pixel axis.
         The dynamic default is the number of pixels along the \textit{y} axis
         of the NDF, so long as this value is no more than 30.  If the
         \textit{y} axis spans more than 30 pixels, then the dynamic default is
         30 (meaning that some spatial pixels will be ignored).  \texttt{[]}
      }
      \sstsubsection{
         REFLABEL = \_LOGICAL (Read)
      }{
         If \texttt{TRUE} then the first line plot (\emph{i.e.} the lower-left
         spectrum) will be annotated with numerical and textual labels
         describing the two axes.  Note, due to the small size of the
         line plot, such text may be too small to read on some
         graphics devices.  \texttt{[}current value\texttt{{]}}
      }
      \sstsubsection{
         SPECAXES = \_LOGICAL (Read)
      }{
         \texttt{TRUE} if axes are to be drawn around each spectrum. The
         appearance of the axes can be controlled using the SPECSTYLE
         parameter.  \texttt{[TRUE]}
      }
      \sstsubsection{
         SPECSTYLE = \htmlref{GROUP}{se:groups} (Read)
      }{
         A group of attribute settings describing the plotting style
         to use when drawing the axes and data values in the spectrum
         line plots.

         A comma-separated list of strings should be given in which each
         string is either an attribute setting, or the name of a text
         file preceded by an up-arrow character \texttt{"$\wedge$"}.  Such text files
         should contain further comma-separated lists which will be
         read and interpreted in the same manner.  Attribute settings
         are applied in the order in which they occur within the list,
         with later settings overriding any earlier settings given for
         the same attribute.

         Each individual attribute setting should be of the form:

            $<$name$>$=$<$value$>$


         where $<$name$>$ is the name of a plotting attribute, and $<$value$>$
         is the value to assign to the attribute.  Default values will be
         used for any unspecified attributes.  All attributes will be
         defaulted if a null value (\texttt{{!}})---the initial default---is supplied.
         To apply changes of style to only the current invocation, begin these
         attributes with a plus sign.  A mixture of persistent and temporary
         style changes is achieved by listing all the persistent attributes
         followed by a plus sign then the list of temporary attributes.

         See \slhyperref{Plotting Attributes}{Section~}{}{ap:plotting_attr}
         for a description of the available attributes.  Any unrecognised
         attributes are ignored (no error is reported).

         By default the axes have interior tick marks, and are without
         labels and a title to avoid overprinting on adjacent plots.

         The appearance of the data values is controlled by the attributes
         \htmlattref{Colour(Curves)}{Colour(element)},
         \htmlattref{Width(Curves)}{Width(element)}, \emph{etc.} (the synonym
         \att{Lines} may be used in place of \att{Curves}).
         \texttt{[}current value\texttt{{]}}
      }
      \sstsubsection{
         STYLE = GROUP (Read)
      }{
         A group of attribute settings describing the plotting style to
         use for the annotated outer spatial axes (see Parameter AXES).

         A comma-separated list of strings should be given in which each
         string is either an attribute setting, or the name of a text
         file preceded by an up-arrow character \texttt{"$\wedge$"}.  Such text files
         should contain further comma-separated lists which will be
         read and interpreted in the same manner.  Attribute settings
         are applied in the order in which they occur within the list,
         with later settings overriding any earlier settings given for
         the same attribute.

         Each individual attribute setting should be of the form:

            $<$name$>$=$<$value$>$


         where $<$name$>$ is the name of a plotting attribute, and $<$value$>$
         is the value to assign to the attribute.  Default values will be
         used for any unspecified attributes.  All attributes will be
         defaulted if a null value (\texttt{{!}})---the initial default---is supplied.
         To apply changes of style to only the current invocation, begin these
         attributes with a plus sign.  A mixture of persistent and temporary
         style changes is achieved by listing all the persistent attributes
         followed by a plus sign then the list of temporary attributes.

         See \slhyperref{Plotting Attributes}{Section~}{}{ap:plotting_attr}
         for a description of the available attributes.  Any unrecognised
         attributes are ignored (no error is reported).
         \texttt{[}current value\texttt{{]}}
      }
      \sstsubsection{
         USEAXIS = LITERAL (Read)
      }{
         The WCS axis that will appear along the horizontal axis of
         each line plot (the other two axes will be used as the spatial
         axes).  The axis can be specified using one of the following
         options.

         \ssthitemlist{

            \sstitem
            Its integer index within the \htmlref{current
            Frame}{se:curframe}~ of the NDF (in the range 1 to 3 in
            the current frame).

            \sstitem
            Its \htmlattref{Symbol}{Symbol(axis)}~ string such as
            \texttt{"RA"} or \texttt{"VRAD"}.

            \sstitem
            A generic option where \texttt{"SPEC"} requests the
            spectral axis, \texttt{"TIME"} selects the time axis,
            \texttt{"SKYLON"} and \texttt{"SKYLAT"} picks the sky longitude
            and latitude axes respectively.  Only those axis domains
            present are available as options.
         }

         A list of acceptable values is displayed if an illegal value is
         supplied.  The dynamic default is the index of any spectral
         axis found in the current Frame of the NDF.  \texttt{[]}
      }
      \sstsubsection{
         YBOT = \_REAL (Read)
      }{
         The data value for the bottom edge of each line plot.  The
         dynamic default is chosen in a manner determined by Parameter
         LMODE.  \texttt{[]}
      }
      \sstsubsection{
         YTOP = \_REAL (Read)
      }{
         The data value for the top edge of each line plot. The dynamic
         default is chosen in a manner determined by Parameter LMODE.  \texttt{[]}
      }
   }
   \sstexamples{
      \sstexamplesubsection{
         clinplot cube useaxis=3
      }{
         Plots a set of line plots of data values versus position
         along the third axis for the three-dimensional NDF called
         cube on the current graphics device.  Axes are drawn around
         the grid of plots indicating the spatial positions in the
         current co-ordinate Frame.  The third axis may not be
         spectral and the other two axes need not be spatial.
      }
      \sstexamplesubsection{
         clinplot cube margin=0.1
      }{
         As above, but if a search locates a spectral axis in the
         world co-ordinate system, this is plotted along the
         horizontal of the line plots, and the other axes are deemed
         to be spatial.  Also the margin for the spatial axes is
         reduced to 0.1 to allow more room for the grid of line plots.
      }
      \sstexamplesubsection{
         clinplot map($\sim$5,$\sim$5,) useaxis=3 noaxes
      }{
         Plots data values versus position for the central 5-by-5
         pixel region of the three-dimensional NDF called map on the
         current graphics device.  No spatial axes are drawn.
      }
      \sstexamplesubsection{
         clinplot map($\sim$5,$\sim$5,) useaxis=3 noaxes device=ps\_l mode=hist
      }{
         As the previous example but now the output goes to a text
         file (\texttt{{pgplot.ps}}) which can be printed on a PostScript
         printer and the data are plotted in histogram form.
      }
      \sstexamplesubsection{
         clinplot nearc v style="title=Ne Arc variance" useaxis=1 reflabel=f
      }{
         Plots variance values versus position along Axis 1, for each
         spatial position in dimensions two and three, for the three
         dimensional NDF called nearc on the current graphics device.
         The plot has a title of \texttt{"Ne Arc variance"}.  No labels are
         drawn around the lower-left line plot.
      }
      \sstexamplesubsection{
         clinplot ndf=speccube noclear specstyle="colour(curves)=blue"
      }{
         Plots data values versus pixel co-ordinate at each spatial
         position for the three-dimensional NDF called speccube on the
         current graphics device.  The plot is drawn over any existing
         plot and inherits the spatial bounds of the previous plot.
         The data are drawn in blue, probably to distinguish it from
         the previous plot drawn in a different colour.
      }
   }
   \sstnotes{
      \sstitemlist{

         \sstitem

         If no \htmlattref{Title}{plotel:Title}~ is specified via the
         STYLE parameter, then the \htmlref{TITLE}{apndf:title}
         component in the NDF is used as the default title for the
         annotated axes.  Should the NDF not have a TITLE component,
         then the default title is instead taken from current
         co-ordinate Frame in the NDF, unless this attribute has not
         been set explicitly, whereupon the name of the NDF is used as
         the default title.

         \sstitem
         If all the data values at a spatial position are bad, no line
         plot is drawn at that location.

         \sstitem
         The application stores a number of pictures in the
         \htmlref{graphics database}{se:agitate}~ in the following order:
         a FRAME picture containing the annotated axes, data plots, and
         optional key; a KEY picture to store the key if present; and a
         DATA picture containing just the data plots.  The world co-ordinates
         in the DATA picture will correspond to the offset along a spectrum
         on the horizontal axis, data value on the vertical axis, and the two
         spatial co-ordinates for that spectrum.  On exit the current
         database picture for the chosen device reverts to the input picture.
      }
   }
   \sstdiytopic{
      Related Applications
   }{
      KAPPA: \htmlref{DISPLAY}{DISPLAY},
      \htmlref{LINPLOT}{LINPLOT},
      \htmlref{MLINPLOT}{MLINPLOT};
      \xref{FIGARO}{sun86}{}: \xref{SPECGRID}{sun86}{SPECGRID};
      \xref{SPLAT}{sun243}{}.
   }
   \sstimplementationstatus{
      \sstitemlist{

         \sstitem
         This routine correctly processes the \htmlref{AXIS}{apndf:axis}, DATA,
         \htmlref{QUALITY}{apndf:quality}, \htmlref{VARIANCE}{apndf:variance},
         \htmlref{LABEL}{apndf:label}, \htmlref{TITLE}{apndf:title},
         \htmlref{WCS}{apndf:wcs}, and \htmlref{UNITS}{apndf:units}~ components
         of the input NDF.

         \sstitem
         Processing of \htmlref{bad pixels}{se:masking} and automatic
         \htmlref{quality masking}{se:qualitymask} are supported.
      }
   }
}

\sstroutine{
   CMULT
}{
   Multiplies an NDF by a scalar
}{
   \sstdescription{
      This application multiplies each pixel of an \NDFref{NDF} by a scalar
      (constant) value to produce a new NDF.
   }
   \sstusage{
      cmult in scalar out
   }
   \sstparameters{
      \sstsubsection{
         IN = NDF (Read)
      }{
         Input NDF structure whose pixels are to be multiplied by a
         scalar.
      }
      \sstsubsection{
         OUT = NDF (Write)
      }{
         Output NDF structure.
      }
      \sstsubsection{
         SCALAR = \_DOUBLE (Read)
      }{
         The value by which the NDF's pixels are to be multiplied.
      }
      \sstsubsection{
         TITLE = LITERAL (Read)
      }{
         A \htmlref{title}{apndf:title} for the output NDF.  A null value will cause the title
         of the NDF supplied for Parameter IN to be used instead.
         \texttt{[!]}
      }
   }
   \sstexamples{
      \sstexamplesubsection{
         cmult a 12.5 b
      }{
         Multiplies all the pixels in the NDF called a by the constant
         value 12.5 to produce a new NDF called b.
      }
      \sstexamplesubsection{
         cmult in=rawdata out=newdata scalar=-19
      }{
         Multiplies all the pixels in the NDF called rawdata by $-$19 to
         give newdata.
      }
   }
   \sstdiytopic{
      Related Applications
   }{
KAPPA: \htmlref{ADD}{ADD},
\htmlref{CADD}{CADD},
\htmlref{CDIV}{CDIV},
\htmlref{CSUB}{CSUB},
\htmlref{DIV}{DIV},
\htmlref{MATHS}{MATHS},
\htmlref{MULT}{MULT},
\htmlref{SUB}{SUB}.
   }
   \sstimplementationstatus{
      \sstitemlist{

         \sstitem
         This routine correctly processes the \htmlref{AXIS}{apndf:axis}, DATA, \htmlref{QUALITY}{apndf:quality},
         \htmlref{LABEL}{apndf:label}, \htmlref{TITLE}{apndf:title}, \htmlref{UNITS}{apndf:units}, \htmlref{HISTORY}{apndf:history}, \htmlref{WCS}{apndf:wcs}, and \htmlref{VARIANCE}{apndf:variance}~ components of an NDF
         data structure and propagates all \htmlref{extensions}{apndf:extensions}.

         \sstitem
         Processing of \htmlref{bad pixels}{se:masking} and automatic \htmlref{quality masking}{se:qualitymask} are supported.

         \sstitem
         All \htmlref{non-complex numeric data types}{ap:HDStypes} can be handled.
         Arithmetic is carried out using the appropriate floating-point
         type, but the numeric type of the input pixels is preserved in
         the output NDF.

         \sstitem
         Huge NDFs are supported.
      }
   }
}
\sstroutine{
   COLCOMP
}{
   Produces a colour composite of up to three two-dimensional NDFs
}{
   \sstdescription{
      This application combines up to three two-dimensional \NDFref{NDFs,} using a
      different primary colour (red, green or blue) to represent each NDF.
      The resulting colour composite image is available in two forms; as an
      NDF with an associated colour table (see Parameters OUT and LUT), and
      as an ASCII PPM image file (see Parameter PPM).  The full pixel
      resolution of the input NDFs is retained.  Note, this application
      does not actually display the image, it just creates various output
      files which must be displayed using other tools (see below).

      The data values in each of the input NDFs which are to be mapped on to
      zero intensity and full intensity can be given manually using
      Parameters RLOW, RHIGH, GLOW, GHIGH, BLOW and BHIGH, but by default
      they are evaluated automatically.  This is done by finding specified
      percentile points within the data histograms of each of the input
      images (see Parameter PERCENTILES).

      The NDF outputs are intended to be displayed with \KAPPA\ application
      DISPLAY, using the command:

      \hspace{5mm} {\latex{\small} \texttt{display $<$out$>$ scale=no lut=$<$lut$>$} }

      where ``\texttt{$<$out$>$}'' and ``\texttt{$<$lut$>$}'' are the names of the
      NDF image and colour table created by this application using
      Parameters OUT and LUT.  The main advantage of this NDF form of
      output over the PPM form is that any \htmlref{WCS}{apndf:wcs} or
      \htmlref{AXIS}{apndf:axis}~ information in the
      input NDFs is still available, and can be used to create axis
      annotations by the DISPLAY command.  The graphics device which will
      be used to display the image must be specified when running this
      application (see Parameter DEVICE).

      The PPM form of output can be displayed using tools such as {\bf xv}, or
      converted into other forms (GIF or JPEG, for instance) using tools
      such as {\bf ppmtogif} and {\bf cjpeg} from the NetPbm or PbmPlus packages.
      These tools provide more sophisticated colour quantisation methods
      than are used by this application when creating the NDF outputs, and
      so may give better visual results.
   }
   \sstusage{
      colcomp inr ing inb out lut [device]
   }
   \sstparameters{
      \sstsubsection{
         BADCOL = \htmlref{LITERAL}{se:parmenu} (Read)
      }{
         The colour with which to mark any bad (\emph{i.e.} missing) pixels in the
         display.  There are a number of options described below.

         \ssthitemlist{

            \sstitem
            \texttt{"MAX"} --- The maximum \htmlref{colour index}{se:coltab} used for the display of the image.

            \sstitem
            \texttt{"MIN"} --- The minimum colour index used for the display of the image.

            \sstitem
            An integer --- The actual colour index.  It is constrained between
            0 and the maximum colour index available on the device.

            \sstitem
            A named colour --- Uses the \htmlref{named colour}{ap:colset} from the \htmlref{palette}{se:palette}, and if it
            is not present, the nearest colour from the palette is selected.

            \sstitem
            An \htmlref{HTML colour code}{htmlcolour} such as \texttt{\#ff002d}.

         }
         If the colour is to remain unaltered as the lookup table is
         manipulated choose an integer between 0 and 15, or a named
         colour.  Note, if only the PPM output is to be created (see
         Parameter PPM), then a named colour must be given for BADCOL.
         \texttt{[}current value\texttt{{]}}
      }
      \sstsubsection{
         BHIGH = \_DOUBLE (Read)
      }{
         The data value corresponding to full blue intensity.  If a null
         (\texttt{{!}}) value is supplied, the value actually used will be determined
         by forming a histogram of the data values in the NDF specified by
         Parameter INB, and finding the data value at the second histogram
         percentile specified by Parameter PERCENTILES.  \texttt{[!]}
      }
      \sstsubsection{
         BLOW = \_DOUBLE (Read)
      }{
         The data value corresponding to zero blue intensity.  If a null
         (\texttt{{!}}) value is supplied, the value actually used will be determined
         by forming a histogram of the data values in the NDF specified by
         Parameter INB, and finding the data value at the first histogram
         percentile specified by Parameter PERCENTILES.  \texttt{[!]}
      }
      \sstsubsection{
         DEVICE = \htmlref{DEVICE}{se:selgradev} (Read)
      }{
         The name of the graphics device which will be used to display
         the NDF output (see Parameter OUT).  This is needed only to
         determine the number of available colours.  No graphics output
         is created by this application.  This parameter is not accessed
         if a null (\texttt{{!}}) value is supplied for Parameter OUT.  The device
         must have at least 24 colour indices or grey-scale intensities.
         \texttt{[}current graphics device\texttt{{]}}
      }
      \sstsubsection{
         GHIGH = \_DOUBLE (Read)
      }{
         The data value corresponding to full green intensity.  If a null
         (\texttt{{!}}) value is supplied, the value actually used will be determined
         by forming a histogram of the data values in the NDF specified by
         Parameter ING, and finding the data value at the second histogram
         percentile specified by Parameter PERCENTILES.  \texttt{[!]}
      }
      \sstsubsection{
         GLOW = \_DOUBLE (Read)
      }{
         The data value corresponding to zero green intensity.  If a null
         (\texttt{{!}}) value is supplied, the value actually used will be determined
         by forming a histogram of the data values in the NDF specified by
         Parameter ING, and finding the data value at the first histogram
         percentile specified by Parameter PERCENTILES.  \texttt{[!]}
      }
      \sstsubsection{
         INB = NDF (Read)
      }{
         The input NDF containing the data to be displayed in blue.  A null
         (\texttt{{!}}) value may be supplied in which case the blue intensity in the
         output will be zero at every pixel.
      }
      \sstsubsection{
         ING = NDF (Read)
      }{
         The input NDF containing the data to be displayed in green.  A null
         (\texttt{{!}}) value may be supplied in which case the green intensity in the
         output will be zero at every pixel.
      }
      \sstsubsection{
         INR = NDF (Read)
      }{
         The input NDF containing the data to be displayed in red.  A null
         (\texttt{{!}}) value may be supplied in which case the red intensity in the
         output will be zero at every pixel.
      }
      \sstsubsection{
         LUT = NDF (Write)
      }{
         Name of the output NDF to contain the colour lookup table which
         should be used when displaying the NDF created using Parameter OUT.
         This colour table can be loaded using LUTREAD, or specified when
         the image is displayed.  This parameter is not accessed if a null
         (\texttt{{!}}) value is given for Parameter OUT.
      }
      \sstsubsection{
         RHIGH = \_DOUBLE (Read)
      }{
         The data value corresponding to full red intensity.  If a null
         (\texttt{{!}}) value is supplied, the value actually used will be determined
         by forming a histogram of the data values in the NDF specified by
         Parameter INR, and finding the data value at the second histogram
         percentile specified by Parameter PERCENTILES.  \texttt{[!]}
      }
      \sstsubsection{
         RLOW = \_DOUBLE (Read)
      }{
         The data value corresponding to zero red intensity.  If a null
         (\texttt{{!}}) value is supplied, the value actually used will be determined
         by forming a histogram of the data values in the NDF specified by
         Parameter INR, and finding the data value at the first histogram
         percentile specified by Parameter PERCENTILES.  \texttt{[!]}
      }
      \sstsubsection{
         OUT = NDF (Write)
      }{
         The output colour composite image in NDF format. Values in
         this output image are integer colour indices into the colour
         table created using Parameter LUT.  The values are shifted to
         account for the indices reserved for the palette (\emph{i.e.} the first
         entry in the colour table is addressed as entry 16, not entry 1).
         The NDF is intended to be used as the input data in conjunction with
         \texttt{display scale=false}.  If a null value (\texttt{{!}}) is supplied, no output NDF
         will be created.
      }
      \sstsubsection{
         PERCENTILES( 2 ) = \_REAL (Read)
      }{
         The percentiles that define the default scaling limits.  For example,
         [25,75] would scale between the quartile values.  \texttt{[5,95]}
      }
      \sstsubsection{
         PPM = FILE (Write)
      }{
         The name of the output text file to contain the PPM form of the
         colour composite image.  The colours specified in this file
         represent the input data values directly.  They are not quantised or
         dithered in any way.  Also note that because this is a text file,
         containing formatted data values, it is portable, but can be
         very large, and slow to read and write.  If a null (\texttt{{!}}) value is
         supplied, no PPM output is created.  \texttt{[!]}
      }
   }
   \sstexamples{
      \sstexamplesubsection{
         colcomp m31\_r m31\_g m31\_b m31\_col m31\_lut
      }{
         Combines the 3 NDFs m31\_r, m31\_g, and m31\_b to create a colour
         composite image stored in NDF m31\_col.  A colour look up table is
         also created and stored in NDF m31\_lut.  It is assumed that the
         output image will be displayed on the \htmlref{current graphics device}{se:devglobal}.
         The created colour composite image should be displayed using the
         command:

         \hspace{5mm} {\latex{\small} \texttt{display m31\_col scale=no lut=m31\_lut} }

      }
      \sstexamplesubsection{
         colcomp m31\_r m31\_g m31\_b out=! ppm=m31.ppm
      }{
         As above, but no NDF outputs are created.  Instead, a file called
         \texttt{m31.ppm} is created which (for instance) can be displayed using
         the command:

         \hspace{5mm} {\latex{\small} \texttt{xv m31.ppm} }

         It can be converted to a GIF (for instance, for inclusion in
         WWW pages) using the command:

         \hspace{5mm} {\latex{\small} \texttt{ppmquant 256 m31.ppm $|$ ppmtogif $>$ m31.gif} }

         These commands assume you have {\bf xv}, {\bf ppmquant} and
         {\bf ppmtogif} installed at your site.  None of them are
         part of \KAPPA.
      }
   }
   \sstnotes{
      \sstitemlist{

         \sstitem
         The output image (PPM or NDF) covers the area of overlap between
         the input NDFs at full resolution.  If the input NDFs are very large
         is is a good idea to compress them first (for instance, using
         COMPAVE) to reduce the size of the output images.  Note, compressing
         the output NDF will normally produce spurious colours in the
         compressed image.

         \sstitem
         The output image is based on the values in the DATA components
         of the input NDFs.  Any \htmlref{VARIANCE}{apndf:variance}~ and \htmlref{QUALITY}{apndf:quality}~ arrays in the input NDFs
         are ignored.
      }
   }
   \sstdiytopic{
      Related Applications
   }{
KAPPA: \htmlref{DISPLAY}{DISPLAY},
\htmlref{LUTREAD}{LUTREAD};
XV; PBMPLUS; NETPBM.
   }
   \sstimplementationstatus{
      \sstitemlist{

         \sstitem
         The \htmlref{HISTORY}{apndf:history}, \htmlref{WCS}{apndf:wcs}, and \htmlref{AXIS}{apndf:axis}~ components, together with any extensions
         are propagated to the output NDF, from the first supplied input NDF.

         \sstitem
         Processing of \htmlref{bad pixels}{se:masking} and automatic \htmlref{quality masking}{se:qualitymask} are
         supported.

         \sstitem
         Only data of type \htmlref{\_REAL}{ap:HDStypes} can be processed directly.  Data of other
         types will be converted to \_REAL before being processed.
      }
   }
}

\sstroutine{
   COLLAPSE
}{
   Reduces the number of axes in an \textit{n}-dimensional NDF by compressing it
   along a nominated axis
}{
   \sstdescription{
      This application collapses a nominated axis of an \textit{n}-dimensional \NDFref{NDF},
      producing an output NDF with one fewer axes than the input NDF.  A
      specified range of axis values can be used instead of the whole axis
      (see Parameters LOW and HIGH).

      For each output pixel, all corresponding input pixel values
      between the specified bounds of the nominated axis to be
      collapsed are combined together using one of a selection of
      estimators, including a mean, mode, or median, to produce the
      output pixel value.

      Possible uses include such things as collapsing a range of
      wavelength planes in a three-dimensional RA/DEC/Wavelength cube to
      produce a single two-dimensional RA/DEC image, or collapsing a
      range of slit positions in a two-dimensional slit
      position/wavelength image to produce a one-dimensional wavelength
      array.
   }
   \sstusage{
      collapse in out axis [low] [high] [estimator] [wlim]
   }
   \sstparameters{
      \sstsubsection{
         AXIS = LITERAL (Read)
      }{
         The axis along which to collapse the NDF.  This can be specified
         using one of the following options.

         \ssthitemlist{

            \sstitem

            Its integer index within the \htmlref{current
            Frame}{se:curframe}~ of the input NDF (in the range 1 to
            the number of axes in the current Frame).

            \sstitem
            Its \htmlattref{Symbol}{Symbol(axis)}~ string such as
            \texttt{"RA"} or \texttt{"VRAD"}.

            \sstitem
            A generic option where \texttt{"SPEC"} requests the spectral axis,
            \texttt{"TIME"} selects the time axis, \texttt{"SKYLON"} and
            \texttt{"SKYLAT"} picks the sky longitude and latitude axes
            respectively.  Only those axis domains present are
            available as options.
         }

         A list of acceptable values is displayed if an illegal value
         is supplied.  If the axes of the current Frame are not
         parallel to the NDF pixel axes, then the pixel axis which is
         most nearly parallel to the specified current Frame axis will
         be used.
      }
      \sstsubsection{
         CLIP = \_REAL (Read)
      }{
         The number of standard deviations about the mean at which to
         clip outliers for the \texttt{"Mode"}, \texttt{"Cmean"} and \texttt{"Csigma"}
         statistics (see Parameter ESTIMATOR).  The application first computes
         statistics using all the available pixels.  It then rejects
         all those pixels whose values lie beyond CLIP standard
         deviations from the mean and will then re-evaluate the
         statistics.  For \texttt{"Cmean"} and \texttt{"Csigma"} there is currently
         only one iteration, but up to seven for \texttt{"Mode"}.

         The value must be positive.  \texttt{[3.0]}
      }

      \sstsubsection{
         COMP = LITERAL (Read)
      }{
         The name of the NDF array component for which statistics are
         required: \texttt{"Data"}, \texttt{"Error"}, \texttt{"Quality"} or
         \texttt{"Variance"} (where \texttt{"Error"} is the alternative to
         \texttt{"Variance"} and causes the square root of the variance
         values to be taken before computing the statistics).  If
         \texttt{"Quality"} is specified, then the quality
         values are treated as numerical values (in the range 0 to
         255).  \texttt{["Data"]}
      }
      \sstsubsection{
         ESTIMATOR = \htmlref{LITERAL}{se:parmenu} (Read)
      }{
         The method to use for estimating the output pixel values.  It
         can be one of the following options.  The first five are
         more for general collapsing, and the remainder are for cube
         analysis.

         \ssthitemlist{

            \sstitem
            \texttt{"Mean"} --- Mean value

            \sstitem
            \texttt{"WMean"}  --- Weighted mean in which each data value is
                        weighted by the reciprocal of the associated variance
                        (not available for COMP=\texttt{"Variance"} or
                        \texttt{"Error"}).

            \sstitem
            \texttt{"Mode"}   --- Modal value

            \sstitem
            \texttt{"Median"} --- Median value.  Note that this is extremely memory
                        and CPU intensive for large datasets; use with
                        care!  If strange things happen, use \texttt{"Mean"}.

            \sstitem
            \texttt{"FastMed"} --- Faster median using Wirth's algorithm for selecting
                        the $k$th value, rather than a full sort.
                        Weighting is not supported, thus this option is
                        unavailable if both Parameter VARIANCE is \texttt{TRUE} and
                        the input NDF contains a VARIANCE component.\\

            \sstitem
            \texttt{"Absdev"} --- Mean absolute deviation from the unweighted mean.

           \sstitem
            \texttt{"Cmean"}  --- Sigma-clipped mean.

            \sstitem
            \texttt{"Csigma"} --- Sigma-clipped standard deviation.

            \sstitem
            \texttt{"Comax"}  --- Co-ordinate of the maximum value.

            \sstitem
            \texttt{"Comin"}  --- Co-ordinate of the minimum value.

            \sstitem
            \texttt{"FBad"}   --- Fraction of bad pixel values.

            \sstitem
            \texttt{"FGood"}  --- Fraction of good pixel values.

            \sstitem
            \texttt{"Integ"}  --- Integrated value, being the sum of the products
                        of the value and pixel width in world co-ordinates.  Note
                        that for sky co-ordinates the width is measured in radians.
                        co-ordinates.

            \sstitem
            \texttt{"Iwc"}    --- Intensity-weighted co-ordinate, being the sum of
                        each value times its co-ordinate, all divided by
                        the integrated value (see the \texttt{"Integ"} option).

            \sstitem
            \texttt{"Iwd"}    --- Intensity-weighted dispersion of the
                        co-ordinate, normalised like \texttt{"Iwc"} by the
                        integrated value.

            \sstitem
            \texttt{"Max"}    --- Maximum value.

            \sstitem
            \texttt{"Min"}    --- Minimum value.

            \sstitem
            \texttt{"NBad"}   --- Count of bad pixel values.

            \sstitem
            \texttt{"NGood"}  --- Count of good pixel values.

            \sstitem
            \texttt{"Rms"}    --- Root-mean-square value.

            \sstitem
            \texttt{"Sigma"}  --- Standard deviation about the unweighted mean.

            \sstitem
            \texttt{"Sum"}    --- The total value.
         }
         \texttt{["Mean"]}
      }
      \sstsubsection{
         HIGH = LITERAL (Read)
      }{
         A formatted value for the axis specified by Parameter AXIS.  For
         example, if AXIS is 3 and the current Frame of the input NDF has
         axes RA/DEC/Wavelength, then a wavelength value should be supplied.
         If, on the other hand, the current Frame in the NDF was the PIXEL
         Frame, then a pixel co-ordinate value would be required for the
         third axis (note, the pixel with index I covers a range of pixel
         co-ordinates from $(I-1)$ to $I$).  Together with Parameter LOW, this
         parameter gives the range of axis values to be compressed.  Note,
         HIGH and LOW should not be equal since.  If a null value (\texttt{{!}}) is
         supplied for either HIGH or LOW, the entire range of the axis
         is collapsed.  \texttt{[!]}
      }
      \sstsubsection{
         IN = NDF (Read)
      }{
         The input NDF.
      }
      \sstsubsection{
         LOW = LITERAL (Read)
      }{
         A formatted value for the axis specified by Parameter AXIS.  For
         example, if AXIS is 3 and the current Frame of the input NDF has
         axes RA/DEC/Wavelength, then a wavelength value should be supplied.
         If, on the other hand, the current Frame in the NDF was the PIXEL
         Frame, then a pixel co-ordinate value would be required for the
         third axis (note, the pixel with index $I$ covers a range of pixel
         co-ordinates from $(I-1)$ to $I$).  Together with Parameter HIGH, this
         parameter gives the range of axis values to be compressed.  Note,
         LOW and HIGH should not be equal since.  If a null value (\texttt{{!}}) is
         supplied for either LOW or HIGH, the entire range of the axis
         is collapsed.  \texttt{[!]}
      }
      \sstsubsection{
         OUT = NDF (Write)
      }{
         The output NDF.
      }
      \sstsubsection{
         TITLE = LITERAL (Read)
      }{
         \htmlref{Title}{apndf:title} for the output NDF structure.  A null value (\texttt{{!}})
         propagates the title from the input NDF to the output NDF.  \texttt{[!]}
      }
      \sstsubsection{
         TRIM = \_LOGICAL (Read)
      }{
         This parameter controls whether the collapsed axis should be
         removed from the co-ordinate systems describing the output NDF.
         If a \texttt{TRUE} value is supplied, the collapsed WCS axis will be
         removed from the WCS FrameSet of the output NDF, and the
         collapsed pixel axis will also be removed from the NDF,
         resulting in the output NDF having one fewer pixel axes than
         the input NDF.  If a \texttt{FALSE} value is supplied, the collapsed WCS
         and pixel axes are retained in the output NDF, resulting in the
         input and output NDFs having the same number of pixel axes.  In
         this case, the pixel-index bounds of the collapse axis will be set
         to (1:1) in the output NDF (that is, the output NDF will span
         only a single pixel on the collapse axis).  Thus, setting TRIM to
         \texttt{FALSE} allows information to be retained about the range of values
         over which the collapse occurred.  \texttt{[TRUE]}
      }
      \sstsubsection{
         VARIANCE = \_LOGICAL (Read)
      }{
         A flag indicating whether a variance array present in the
         NDF is used to weight data values while forming the estimator's
         statistic, and to derive output variance.  If VARIANCE is \texttt{TRUE}
         and the NDF contains a variance array, this array will be
         used to define the weights, otherwise all the weights will be
         set equal.  By definition this parameter is set to \texttt{FALSE} when
         COMP is \texttt{"Variance"} or \texttt{"Error"}.

         The VARIANCE parameter is ignored and set to \texttt{FALSE} when there
         are more than 300 pixels along the collapse axis and
         ESTIMATOR is \texttt{"Median"}, \texttt{"Mode"}, \texttt{"Cmean"}, or
         \texttt{"Csigma"}.  This prevents the covariance matrix from being
         huge.  For \texttt{"Median"} estimates of variance come from
         mean variance instead.  The other affected estimators switch to
         use equal weighting. \texttt{[TRUE]}
      }
      \sstsubsection{
         WCSATTS = GROUP (Read)
      }{
         A group of attribute settings which will be used to make temporary
         changes to the properties of the current co-ordinate Frame in the WCS
         FrameSet before it is used.  Supplying a list of attribute values for
         this parameter is equivalent to invoking WCSATTRIB on the input NDF
         prior to running this command, except that no permanent change
         is made to the NDF (however the changes are propagated through to
         the output NDF).

         A comma-separated list of strings should be given in which each
         string is either an attribute setting, or the name of a text
         file preceded by an up-arrow character \texttt{"$\wedge$"}.  Such
         text files should contain further comma-separated lists which
         will be read and interpreted in the same manner.  Attribute
         settings are applied in the order in which they occur within the
         list, with later settings overriding any earlier settings given
         for the same attribute.

         Each individual attribute setting should be of the form:

            $<$name$>$=$<$value$>$

         where $<$name$>$ is the name of a plotting attribute, and
         $<$value$>$ is the value to assign to the attribute. Any
         unspecified attributes will retain the value they have in the
         supplied NDF.  No attribute values will be changed if a null
         value (\texttt{{!}}) is supplied.  Any unrecognised attributes are
         ignored (no error is reported).  \texttt{[!]}
      }
      \sstsubsection{
         WLIM = \_REAL (Read)
      }{
         If the input NDF contains bad pixels, then this parameter
         may be used to determine the number of good pixels which must
         be present within the range of collapsed input pixels before a
         valid output pixel is generated.  It can be used, for example, to
         prevent output pixels from being generated in regions where there
         are relatively few good pixels to contribute to the collapsed
         result.

         WLIM specifies the minimum fraction of good pixels which must
         be present in order to generate a good output pixel.  If this
         specified minimum fraction of good input pixels is not present,
         then a bad output pixel will result, otherwise a good output
         value will be calculated.  The value of this parameter should lie
         between 0.0 and 1.0 (the actual number used will be rounded up if
         necessary to correspond to at least one pixel).  \texttt{[0.3]}
      }
   }
   \sstexamples{
      \sstexamplesubsection{
         collapse m31 profile axis=RA low="0:36:01" high="0:48:00"
      }{
         Collapses the two-dimensional NDF called m31 along the
         right-ascension axis, from "0:36:01" to "0:48:00", and puts the
         result in an output NDF called profile.
      }
      \sstexamplesubsection{
         collapse cube slab lambda 4500 4550
      }{
         The current Frame in the input three-dimensional NDF called cube has
         axes with labels \texttt{"RA"}, \texttt{"DEC"} and \texttt{"Lambda"}, with the
         lambda axis being parallel to the third pixel axis.  The above command
         extracts a slab of the input cube between wavelengths 4500 and
         4550 {\AA}ngstroms, and collapses this slab into a single
         two-dimensional output NDF called slab with RA and DEC axes.
         Each pixel in the output NDF is the mean of the corresponding
         input pixels with wavelengths between 4500 and 4550 {\AA}ngstroms.
      }
      \sstexamplesubsection{
         collapse cube slab 3 4500 4550
      }{
         The same as the previous example except the axis to collapse along is
         specified by index (3) rather than label (lambda).
      }
      \sstexamplesubsection{
         collapse cube slab 3 101.0 134.0
      }{
         This is the same as the second example, except that the current
         Frame in the input NDF has been set to the PIXEL Frame (using
         \htmlref{WCSFRAME}{WCSFRAME}), and so the high and low axis values are
         specified in pixel co-ordinates instead of {\AA}ngstroms.  Note the
         difference between floating-point pixel co-ordinates, and integer pixel
         indices (for instance the pixel with index 10 extends from pixel
         co-ordinate 9.0 to pixel co-ordinate 10.0).
      }
      \sstexamplesubsection{
         collapse cube slab 3 low=99.0 high=100.0
      }{
         This is the same as the second example, except that a single
         pixel plane in the cube (pixel 100) is used to create the output
         NDF.  Following the usual definition of pixel co-ordinates, pixel
         100 extends from pixel co-ordinate 99.0 to pixel co-ordinate
         100.0.  So the given HIGH and LOW values encompass the single
         pixel plane at pixel 100.
      }
   }
   \sstnotes{
      \sstitemlist{

         \sstitem
         The collapse is always performed along one of the pixel axes,
         even if the current Frame in the input NDF is not the PIXEL
         Frame.  Special care should be taken if the current-Frame axes
         are not parallel to the pixel axes.  The algorithm used to
         choose the pixel axis and the range of values to collapse
         along this pixel axis proceeds as follows.

         The current-Frame co-ordinates of the central pixel in the
         input NDF are determined (or some other point if the
         co-ordinates of the central pixel are undefined).  Two
         current-Frame positions are then generated by substituting in
         turn into this central position each of the HIGH and LOW
         values for the current-Frame axis specified by Parameter
         AXIS.  These two current-Frame positions are transformed into
         pixel co-ordinates, and the projections of the vector joining
         these two pixel positions on to the pixel axes are found.  The
         pixel axis with the largest projection is selected as the
         collapse axis, and the two end points of the projection
         define the range of axis values to collapse.

         \sstitem
         A warning is issued (at the normal reporting level) whenever
         any output values are set bad because there are too few
         contributing data values.  This reports the fraction of flagged
         output data generated by the WLIM parameter's threshold.

         No warning is given when Parameter WLIM=\texttt{0}.  Input data
         containing only bad values are not counted in the flagged
         fraction, since no potential good output value has been lost.
      }
   }
   \sstdiytopic{
      Related Applications
   }{
KAPPA: \htmlref{WCSFRAME}{WCSFRAME},
\htmlref{COMPAVE}{COMPAVE},
\htmlref{COMPICK}{COMPICK},
\htmlref{COMPADD}{COMPADD},
\htmlref{MANIC}{MANIC}.
   }
   \sstimplementationstatus{
      \sstitemlist{

         \sstitem
         This routine correctly processes the \htmlref{AXIS}{apndf:axis}, DATA,
         \htmlref{VARIANCE}{apndf:variance}, \htmlref{LABEL}{apndf:label},
         \htmlref{TITLE}{apndf:title}, \htmlref{UNITS}{apndf:units}, \htmlref{WCS}{apndf:wcs}, and
         \htmlref{HISTORY}{apndf:history}~ components of the input NDF; and
         propagates all \htmlref{extensions}{apndf:extensions}.  \htmlref{QUALITY}{apndf:quality}
         is not propagated.

         \sstitem
         Processing of \htmlref{bad pixels}{se:masking} and automatic \htmlref{quality masking}{se:qualitymask} are
         supported.

         \sstitem
         All \htmlref{non-complex numeric data types}{ap:HDStypes} can be handled.

         \sstitem
         Any number of NDF dimensions is supported.

         \sstitem
         Huge NDFs are supported.
      }
   }
}
\sstroutine{
   COMPADD
}{
   Reduces the size of an NDF by adding values in rectangular boxes
}{
   \sstdescription{
      This application takes an \NDFref{NDF} data structure and reduces it in
      size by integer factors along each dimension.  The compression
      is achieved by adding the values of the input NDF within
      non-overlapping `rectangular' boxes whose dimensions are the
      compression factors.  The additions may be normalised to correct
      for any bad values present in the input NDF.  The exact placement of
      the boxes can be controlled using Parameter ALIGN.
   }
   \sstusage{
      compadd in out compress [wlim]
   }
   \sstparameters{
      \sstsubsection{
         ALIGN = \htmlref{LITERAL}{se:parmenu} (Read)
      }{
         This parameter controls the placement of the compression boxes
         within the input NDF (also see Parameter TRIM).  It can take any
         of the following values:

         \ssthitemlist{

            \sstitem
            \texttt{"ORIGIN"} --- The compression boxes are placed so that the
            origin of the pixel \htmlref{co-ordinate Frame}{se:domains}~
            (\emph{i.e.} pixel co-ordinates (0,0)) in the input NDF
            corresponds to a corner of a compression box.  This results in
            the pixel origin being retain in the output NDF.  For instance,
            if a pair of two-dimensional images which have previously been
            aligned in pixel co-ordinates are compressed, then using this
            option ensures that the compressed images will also be aligned
            in pixel co-ordinates.

            \sstitem
            \texttt{"FIRST"} --- The compression boxes are placed so that the
            first pixel in the input NDF (for instance, the bottom-left
            pixel in a two-dimensional image) corresponds to the first pixel
            in a compression box.  This can result in the pixel origin being
            shifted by up to one compression box in the output image.  Thus,
            images which were previously aligned in pixel co-ordinates may
            not be aligned after compression.  You may want to use this option
            if you are using a very large box to reduce the number of
            dimensions in the data (for instance summing across the entire
            width of an image to produce a one-dimensional array).

            \sstitem
            \texttt{"LAST"} --- The compression boxes are placed so that the
            last pixel in the input NDF (for instance, the top-right
            pixel in a two-dimensional image) corresponds to the last pixel
            in a compression box.  See the \texttt{"FIRST"} option above for further
            comments.
         }
         \texttt{["ORIGIN"]}
      }
      \sstsubsection{
         AXWEIGHT = \_LOGICAL (Read)
      }{
         When there is an AXIS variance array present in the NDF and
         AXWEIGHT=\texttt{TRUE} the application forms weighted averages of the
         axis centres using the variance.  For all other conditions
         the non-bad axis centres are given equal weight during the
         averaging to form the output axis centres.  \texttt{[FALSE]}
      }
      \sstsubsection{
         COMPRESS( ) = \_INTEGER (Read)
      }{
         Linear compression factors to be used to create the output
         NDF.  There should be one for each dimension of the NDF.  If
         fewer are supplied the last value in the list of compression
         factors is given to the remaining dimensions.  Thus if a
         uniform compression is required in all dimensions, just one
         value need be entered.  All values are constrained to be in
         the range one to the size of its corresponding dimension.  The
         suggested default is the current value.
      }
      \sstsubsection{
         IN  = NDF (Read)
      }{
         The NDF structure to be reduced in size.
      }
      \sstsubsection{
         NORMAL = \_LOGICAL (Read)
      }{
         When there are bad pixels present in the summation box these
         are ignored.  Therefore a simple addition of the input-array
         component's values will yield a result discordant with
         neighbouring output pixels that were formed from summation of
         all the pixels in the box.  When NORMAL=\texttt{TRUE} the output values
         are normalised: the addition is multiplied by the ratio of the
         number of pixels in the box to the number of good pixels
         therein to arrive at the output value.  When NORMAL=\texttt{FALSE} the
         output values are always just the sum of the good pixels.
         \texttt{[TRUE]}
      }
      \sstsubsection{
         OUT = NDF (Write)
      }{
         NDF structure to contain compressed version of the input NDF.
      }
      \sstsubsection{
         PRESERVE = \_LOGICAL (Read)
      }{
         If the input data type is to be preserved on output then this
         parameter should be set \texttt{TRUE}.  However, this may result in
         overflows for integer types and hence additional bad values
         written to the output NDF.  If this parameter is set \texttt{FALSE}
         then the output data type will be one of \_REAL or \_DOUBLE,
         depending on the input type.  \texttt{[FALSE]}
      }
      \sstsubsection{
         TITLE = LITERAL (Read)
      }{
         \htmlref{Title}{apndf:title} for the output NDF structure.  A null value (\texttt{{!}})
         propagates the title from the input NDF to the output NDF.  \texttt{[!]}
      }
      \sstsubsection{
         TRIM = \_LOGICAL (Read)
      }{
         If Parameter TRIM is set \texttt{TRUE}, the output NDF only contains data
         for compression boxes which are entirely contained within the
         input NDF.  Any pixels around the edge of the input NDF which are
         not contained within a compression box are ignored.  If TRIM is set
         \texttt{FALSE}, the output NDF contains data for all compression boxes which
         have any overlap with the input NDF.  All pixels outside the
         bounds of the NDF are assumed to be bad.  That is, any boxes which
         extend beyond the bounds of the input NDF are padded with bad
         pixels.  See also Parameter ALIGN.  \texttt{[}current value\texttt{{]}}
      }
      \sstsubsection{
         WLIM = \_REAL (Read)
      }{
         If the input NDF contains bad pixels, then this parameter
         may be used to determine the number of good pixels which must
         be present within the addition box before a valid output
         pixel is generated.  It can be used, for example, to prevent
         output pixels from being generated in regions where there are
         relatively few good pixels to contribute to the smoothed
         result.

         WLIM specifies the minimum fraction of good pixels which must
         be present in the summation box in order to generate a good
         output pixel.  If this specified minimum fraction of good
         input pixels is not present, then a bad output pixel will
         result, otherwise the output value will be the sum of the
         good values.  The value of this parameter should lie between
         0.0 and 1.0 (the actual number used will be rounded up if
         necessary to correspond to at least one pixel).  \texttt{[0.3]}
      }
   }
   \sstexamples{
      \sstexamplesubsection{
         compadd cosmos galaxy 4
      }{
         This compresses the NDF called cosmos summing four times in
         each dimension, and stores the reduced data in the NDF called
         galaxy.  Thus if cosmos is two-dimensional, this command
         would result in a sixteen-fold reduction in the array
         components.
      }
      \sstexamplesubsection{
         compadd cosmos profile [10000,1] wlim=0 align=first trim=no
      }{
         This compresses the two-dimensional NDF called cosmos to produce a
         one-dimensional NDF called profile.  This is done using a
         compression box which is 1 pixel high, but which is wider than
         the whole input image.  Each pixel in the output NDF thus
         corresponds to the sum of the corresponding row in the
         input image.  WLIM is set to zero to ensure that bad pixels
         are ignored.  ALIGN is set to \texttt{"FIRST"} so that each compression box
         is flush with the left edge of the input image.  TRIM is set to
         \texttt{NO} so that compression boxes which extend outside the bounds of
         the input image (which will be all of them if the input image is
         narrower than 10000 pixels) are retained in the output NDF.
      }
      \sstexamplesubsection{
         compadd cosmos galaxy 4 wlim=1.0
      }{
         This compresses the NDF called cosmos adding four times in
         each dimension, and stores the reduced data in the NDF called
         galaxy.  Thus if cosmos is two-dimensional, this command
         would result in a sixteen-fold reduction in the array
         components.  If a summation box contains any bad pixels, the
         output pixel is set to bad.
      }
      \sstexamplesubsection{
         compadd cosmos galaxy 4 0.0 preserve
      }{
         As above except that a summation box need only contains a
         single non-bad pixels for the output pixel to be good, and
         galaxy's array components will have the same as those in
         cosmos.
      }
      \sstexamplesubsection{
         compadd cosmos galaxy [4,3] nonormal title="COSMOS compressed"
      }{
         This compresses the NDF called cosmos adding four times in
         the first dimension and three times in higher dimensions, and
         stores the reduced data in the NDF called galaxy.  Thus if
         cosmos is two-dimensional, this command would result in a
         twelve-fold reduction in the array components.  Also, if there
         are bad pixels there will be no normalisation correction for the
         missing values.  The title of the output NDF is \texttt{"COSMOS
         compressed"}.
      }
      \sstexamplesubsection{
         compadd in=arp244 compress=[1,1,3] out=arp244cs
      }{
         Suppose arp244 is a huge NDF storing a spectral-line data
         cube, with the third dimension being the spectral axis.
         This command compresses arp244 in the spectral dimension,
         adding every three pixels to form the NDF called arp244cs.
      }
   }
   \sstnotes{
      \sstitemlist{

         \sstitem
         The axis centres and variances are averaged, whilst the widths
         are summed and always normalised for bad values. }
   }
   \sstdiytopic{
      Related Applications
   }{
KAPPA: \htmlref{BLOCK}{BLOCK},
\htmlref{COMPAVE}{COMPAVE},
\htmlref{COMPICK}{COMPICK},
\htmlref{PIXDUPE}{PIXDUPE},
\htmlref{SQORST}{SQORST},
\htmlref{REGRID}{REGRID};
\xref{FIGARO}{sun86}{}: \xref{ISTRETCH}{sun86}{ISTRETCH},
\xref{YSTRACT}{sun86}{YSTRACT}.
   }
   \sstimplementationstatus{
      \sstitemlist{

         \sstitem
         This routine correctly processes the \htmlref{AXIS}{apndf:axis}, DATA, \htmlref{VARIANCE}{apndf:variance},
         \htmlref{LABEL}{apndf:label}, \htmlref{TITLE}{apndf:title}, \htmlref{UNITS}{apndf:units}, \htmlref{WCS}{apndf:wcs}, and \htmlref{HISTORY}{apndf:history}~ components of the input NDF and
         propagates all \htmlref{extensions}{apndf:extensions}.  QUALITY is not processed since it is
 a series of flags, not numerical values.

         \sstitem
         Processing of \htmlref{bad pixels}{se:masking} and automatic \htmlref{quality masking}{se:qualitymask} are
         supported.

         \sstitem
         All \htmlref{non-complex numeric data types}{ap:HDStypes} can be handled.

         \sstitem
         Any number of NDF dimensions is supported.
      }
   }
}
\sstroutine{
   COMPAVE
}{
   Reduces the size of an NDF by averaging values in rectangular
   boxes
}{
   \sstdescription{
      This application takes an \NDFref{NDF} data structure and reduces it in
      size by integer factors along each dimension.  The compression
      is achieved by averaging the input NDF within non-overlapping
      `rectangular' boxes whose dimensions are the compression factors.
      The averages may be weighted when there is a variance array
      present.  The exact placement of the boxes can be controlled using
      Parameter ALIGN.
   }
   \sstusage{
      compave in out compress [wlim]
   }
   \sstparameters{
      \sstsubsection{
         ALIGN = \htmlref{LITERAL}{se:parmenu} (Read)
      }{
         This parameter controls the placement of the compression boxes
         within the input NDF (also see Parameter TRIM).  It can take any
         of the following values:

         \ssthitemlist{

            \sstitem
            \texttt{"ORIGIN"} --- The compression boxes are placed so that the
            origin of the pixel \htmlref{co-ordinate Frame}{se:domains}~
            (\emph{i.e.} pixel co-ordinates (0,0)) in the input NDF
            corresponds to a corner of a compression box.  This results in
            the pixel origin being retain in the output NDF.  For instance,
            if a pair of two-dimensional images which have previously been
            aligned in pixel co-ordinates are compressed, then using this
            option ensures that the compressed images will also be aligned
            in pixel co-ordinates.

            \sstitem
            \texttt{"FIRST"} --- The compression boxes are placed so that the
            first pixel in the input NDF (for instance, the bottom-left
            pixel in a two-dimensional image) corresponds to the first pixel
            in a compression box.  This can result in the pixel origin being
            shifted by up to one compression box in the output image.  Thus,
            images which were previously aligned in pixel co-ordinates may
            not be aligned after compression.  You may want to use this option
            if you are using a very large box to reduce the number of
            dimensions in the data (for instance averaging across the entire
            width of an image to produce a one-dimensional array).

            \sstitem
            \texttt{"LAST"} --- The compression boxes are placed so that the
            last pixel in the input NDF (for instance, the top-right
            pixel in a two-dimensional image) corresponds to the last pixel
            in a compression box.  See the \texttt{"FIRST"} option above for further
            comments.
                                                             \texttt{["ORIGIN"]}
         }
      }
      \sstsubsection{
         AXWEIGHT = \_LOGICAL (Read)
      }{
         When there is an AXIS variance array present in the NDF and
         AXWEIGHT=\texttt{TRUE} the application forms weighted averages of the
         axis centres using the variance.  For all other conditions
         the non-bad axis centres are given equal weight during the
         averaging to form the output axis centres.  \texttt{[FALSE]}
      }
      \sstsubsection{
         COMPRESS( ) = \_INTEGER (Read)
      }{
         Linear compression factors to be used to create the output
         NDF.  There should be one for each dimension of the NDF.  If
         fewer are supplied the last value in the list of compression
         factors is given to the remaining dimensions.  Thus if a
         uniform compression is required in all dimensions, just one
         value need be entered.  The suggested default is the current value.
      }
      \sstsubsection{
         IN  = NDF (Read)
      }{
         The NDF structure to be reduced in size.
      }
      \sstsubsection{
         OUT = NDF (Write)
      }{
         NDF structure to contain compressed version of the input NDF.
      }
      \sstsubsection{
         PRESERVE = \_LOGICAL (Read)
      }{
         If the input data type is to be preserved on output then this
         parameter should be set \texttt{TRUE}.  However, this will probably
         result in a loss of precision.  If this parameter is set \texttt{FALSE}
         then the output data type will be one of \_REAL or \_DOUBLE,
         depending on the input type.  \texttt{[FALSE]}
      }
      \sstsubsection{
         TITLE = LITERAL (Read)
      }{
         \htmlref{Title}{apndf:title} for the output NDF structure.  A null value (\texttt{{!}})
         propagates the title from the input NDF to the output NDF.  \texttt{[!]}
      }
      \sstsubsection{
         TRIM = \_LOGICAL (Read)
      }{
         If Parameter TRIM is set \texttt{TRUE}, the output NDF only contains data
         for compression boxes which are entirely contained within the
         input NDF.  Any pixels around the edge of the input NDF which are
         not contained within a compression box are ignored.  If TRIM is set
         \texttt{FALSE}, the output NDF contains data for all compression boxes which
         have any overlap with the input NDF.  All pixels outside the
         bounds of the NDF are assumed to be bad.  That is, any boxes which
         extend beyond the bounds of the input NDF are padded with bad
         pixels.  See also Parameter ALIGN.  \texttt{[}current value\texttt{{]}}
      }
      \sstsubsection{
         WEIGHT = \_LOGICAL (Read)
      }{
         When there is a variance array present in the NDF and
         WEIGHT=\texttt{TRUE} the application forms weighted averages of the
         data array using the variance.  For all other conditions
         the non-bad pixels are given equal weight during averaging.
         \texttt{[FALSE]}
      }
      \sstsubsection{
         WLIM = \_REAL (Read)
      }{
         If the input NDF contains bad pixels, then this parameter
         may be used to determine the number of good pixels which must
         be present within the averaging box before a valid output
         pixel is generated.  It can be used, for example, to prevent
         output pixels from being generated in regions where there are
         relatively few good pixels to contribute to the smoothed
         result.

         WLIM specifies the minimum fraction of good pixels which must
         be present in the averaging box in order to generate a good
         output pixel.  If this specified minimum fraction of good
         input pixels is not present, then a bad output pixel will
         result, otherwise an averaged output value will be calculated.
         The value of this parameter should lie between 0.0 and 1.0
         (the actual number used will be rounded up if necessary to
         correspond to at least one pixel).  \texttt{[0.3]}
      }
   }
   \sstexamples{
      \sstexamplesubsection{
         compave cosmos galaxy 4
      }{
         This compresses the NDF called cosmos averaging four times in
         each dimension, and stores the reduced data in the NDF called
         galaxy.  Thus if cosmos is two-dimensional, this command
         would result in a sixteen-fold reduction in the array
         components.
      }
      \sstexamplesubsection{
         compave cosmos profile [10000,1] wlim=0 align=first trim=no
      }{
         This compresses the two-dimensional NDF called cosmos to produce a
         one-dimensional NDF called profile.  This is done using a
         compression box which is 1 pixel high, but which is wider than
         the whole input image.  Each pixel in the output NDF thus
         corresponds to the average of the corresponding row in the
         input image.  WLIM is set to zero to ensure that bad pixels
         are ignored.  ALIGN is set to \texttt{"FIRST"} so that each compression box
         is flush with the left edge of the input image.  TRIM is set to
         \texttt{NO} so that compression boxes which extend outside the bounds of
         the input image (which will be all of them if the input image is
         narrower than 10000 pixels) are retained in the output NDF.
      }
      \sstexamplesubsection{
         compave cosmos galaxy 4 wlim=1.0
      }{
         This compresses the NDF called cosmos averaging four times in
         each dimension, and stores the reduced data in the NDF called
         galaxy.  Thus if cosmos is two-dimensional, this command
         would result in a sixteen-fold reduction in the array
         components.  If an averaging box contains any bad pixels, the
         output pixel is set to bad.
      }
      \sstexamplesubsection{
         compave cosmos galaxy 4 0.0 preserve
      }{
         As above except that an averaging box need only contains a
         single non-bad pixels for the output pixel to be good, and
         galaxy's array components will have the same as those in
         cosmos.
      }
      \sstexamplesubsection{
         compave cosmos galaxy [4,3] weight title="COSMOS compressed"
      }{
         This compresses the NDF called cosmos averaging four times in
         the first dimension and three times in higher dimensions, and
         stores the reduced data in the NDF called galaxy.  Thus if
         cosmos is two-dimensional, this command would result in a
         twelve-fold reduction in the array components.  Also, if there
         is a variance array present it is used to form weighted means
         of the data array.  The title of the output NDF is \texttt{"COSMOS
         compressed"}.
      }
      \sstexamplesubsection{
         compave in=arp244 compress=[1,1,3] out=arp244cs
      }{
         Suppose arp244 is a huge NDF storing a spectral-line data
         cube, with the third dimension being the spectral axis.
         This command compresses arp244 in the spectral dimension,
         averaging every three pixels to form the NDF called arp244cs.
      }
   }
   \sstnotes{
      \sstitemlist{

         \sstitem
         The axis centres and variances are averaged, whilst the widths
         are summed and always normalised for bad values. }
   }
   \sstdiytopic{
      Related Applications
   }{
KAPPA: \htmlref{BLOCK}{BLOCK},
\htmlref{COMPADD}{COMPADD},
\htmlref{COMPICK}{COMPICK},
\htmlref{PIXDUPE}{PIXDUPE},
\htmlref{SQORST}{SQORST},
\htmlref{REGRID}{REGRID};
\xref{FIGARO}{sun86}{}: \xref{ISTRETCH}{sun86}{ISTRETCH}.
   }
   \sstimplementationstatus{
      \sstitemlist{

         \sstitem
         This routine correctly processes the \htmlref{AXIS}{apndf:axis}, DATA, \htmlref{VARIANCE}{apndf:variance},
         \htmlref{LABEL}{apndf:label}, \htmlref{TITLE}{apndf:title}, \htmlref{UNITS}{apndf:units}, \htmlref{WCS}{apndf:wcs}, and \htmlref{HISTORY}{apndf:history}~ components of the input NDF and
         propagates all \htmlref{extensions}{apndf:extensions}.  QUALITY is not processed since it is
 a series of flags, not numerical values.

         \sstitem
         Processing of \htmlref{bad pixels}{se:masking} and automatic \htmlref{quality masking}{se:qualitymask} are
         supported.

         \sstitem
         All \htmlref{non-complex numeric data types}{ap:HDStypes} can be handled.

         \sstitem
         Any number of NDF dimensions is supported.
      }
   }
}
\sstroutine{
   COMPICK
}{
   Reduces the size of an NDF by picking equally spaced pixels
}{
   \sstdescription{
      This application takes an \NDFref{NDF} data structure and reduces it in
      size by integer factors along each dimension.  The input NDF is
      sampled at these constant compression factors or intervals along
      each dimension, starting from a defined origin, to form an output
      NDF structure.  The compression factors may be different in each
      dimension.
   }
   \sstusage{
      compick in out compress [origin]
   }
   \sstparameters{
      \sstsubsection{
         COMPRESS( ) = \_INTEGER (Read)
      }{
         Linear compression factors to be used to create the output
         NDF.  There should be one for each dimension of the NDF.  If
         fewer are supplied the last value in the list of compression
         factors is given to the remaining dimensions.  Thus if a
         uniform compression is required in all dimensions, just one
         value need be entered.  All values are constrained to be in
         the range one to the size of its corresponding dimension.  The
         suggested default is the current value.
      }
      \sstsubsection{
         IN  = NDF (Read)
      }{
         The NDF structure to be reduced in size.
      }
      \sstsubsection{
         ORIGIN( ) = \_INTEGER (Read)
      }{
         The pixel indices of the first pixel to be selected.
         Thereafter the selected pixels will be spaced equally by
         COMPRESS() pixels.  The origin must lie within the first
         selection intervals, therefore the ith origin must be in the
         range LBND(i) to LBND(i)$+$COMPRESS(i)-1, where LBND(i) is the
         lower bound of the ith dimension.  If a null (\texttt{{!}}) value is
         supplied, the first array element is used.  \texttt{[!]}
      }
      \sstsubsection{
         OUT = NDF (Write)
      }{
         NDF structure to contain compressed version of the input NDF.
      }
      \sstsubsection{
         TITLE = LITERAL (Read)
      }{
         \htmlref{Title}{apndf:title} for the output NDF structure.  A null value (\texttt{{!}})
         propagates the title from the input NDF to the output NDF.  \texttt{[!]} }
   }
   \sstexamples{
      \sstexamplesubsection{
         compick cosmos galaxy 4
      }{
         This compresses the NDF called cosmos selecting every fourth
         array element along each dimension, starting from the first
         element in the NDF, and stores the reduced data in the NDF
         called galaxy.
      }
      \sstexamplesubsection{
         compick cosmos galaxy 4 [3,2]
      }{
         This compresses the two-dimensional NDF called cosmos
         selecting every fourth array element along each dimension,
         starting from the pixel index (3,2), and stores the
         reduced data in the NDF called galaxy.
      }
      \sstexamplesubsection{
         compick in=arp244 compress=[1,1,3] out=arp244cs
      }{
         Suppose arp244 is a huge NDF storing a spectral-line data
         cube, with the third dimension being the spectral axis.
         This command compresses arp244 in the spectral dimension,
         sampling every third pixel, starting from the first wavelength
         at each image position, to form the NDF called arp244cs.
      }
   }
   \sstnotes{
      \sstitemlist{

         \sstitem
         The compression is centred on the origin of the pixel co-ordinate
         Frame.  That is, if a position has a value p(\textit{i}) on
         the \textit{i}'th pixel co-ordinate axis of the input NDF, then
         it will have position p(\textit{i})/COMPRESS(\textit{i}) on the
         corresponding axis of the output NDF.  The
         pixel index bounds of the output NDF are chosen accordingly.
      }
   }
   \sstdiytopic{
      Related Applications
   }{
KAPPA: \htmlref{BLOCK}{BLOCK},
\htmlref{COMPADD}{COMPADD},
\htmlref{COMPAVE}{COMPAVE},
\htmlref{PIXDUPE}{PIXDUPE},
\htmlref{SQORST}{SQORST},
\htmlref{REGRID}{REGRID};
\xref{FIGARO}{sun86}{}: \xref{ISTRETCH}{sun86}{ISTRETCH}.
   }
   \sstimplementationstatus{
      \sstitemlist{

         \sstitem
         This routine correctly processes the \htmlref{AXIS}{apndf:axis}, DATA, \htmlref{QUALITY}{apndf:quality},
         \htmlref{VARIANCE}{apndf:variance}, \htmlref{LABEL}{apndf:label}, \htmlref{TITLE}{apndf:title}, \htmlref{UNITS}{apndf:units}, \htmlref{WCS}{apndf:wcs}, and \htmlref{HISTORY}{apndf:history}~ components of the
         input NDF and propagates all \htmlref{extensions}{apndf:extensions}.

         \sstitem
         Processing of \htmlref{bad pixels}{se:masking} and automatic \htmlref{quality masking}{se:qualitymask} are
         supported.

         \sstitem
         All \htmlref{non-complex numeric data types}{ap:HDStypes} can be handled.

         \sstitem
         Any number of NDF dimensions is supported.
      }
   }
}

\sstroutine{
   COMPLEX
}{
   Converts between representations of complex data
}{
   \sstdescription{
      This application converts between various representations of complex
      data, including complex \NDFref{NDFs}.  The conversion may simply unpack or pack
      real and imaginary parts of a complex NDF, or it may convert between
      polar and Cartesian representation.
   }
   \sstusage{
      complex in1 in2 out1 out2 [intype] [outtype]
   }
   \sstparameters{
      \sstsubsection{
         IN1 = NDF (Read)
      }{
         The first input NDF.  See Parameter INTYPE for its description.
      }
      \sstsubsection{
         IN2 = NDF (Read)
      }{
         The second input NDF.  See Parameter INTYPE for its description.
         IN2 will not be accessed when Parameter INTYPE is set to \texttt{"COMPLEX"}.
         When Parameter INTYPE is set to \texttt{"MOD\_ARG"}, supply a null
         (\texttt{!}) value.
      }
      \sstsubsection{
         INTYPE = LITERAL (Read)
      }{
         The nature of the input NDF(s).  The allowed options are listed
         below.

         \sstitemlist{

            \sstitem
            \texttt{"COMPLEX"}  -- IN1 is a complex NDF containing real
            and imaginary parts.  (IN2 will not be accessed.)

            \sstitem
            \texttt{"REAL\_IMAG"}  -- IN1 contains the real part and IN2
            contains the imaginary parts.

            \sstitem
            \texttt{"MOD\_ARG"}  -- IN1 contains the modulus and IN2 the
            argument in radians.

         }
         The default is\texttt{"COMPLEX"} if IN1 is a complex NDF,
         otherwise it is \texttt{"REAL\_IMAG"}.
         \texttt{[]}
      }
      \sstsubsection{
         OUT1 = NDF (Write)
      }{
         The first output NDF.  Its contents are governed by Parameter OUTTYPE.
      }
      \sstsubsection{
         OUT2 = NDF (Write)
      }{
         The second output NDF.  Its contents are governed by Parameter OUTTYPE.
         OUT2 will not be accessed when Parameter OUTTYPE is set to \texttt{"COMPLEX"}.
         When Parameter OUTTYPE is set to \texttt{"MOD\_ARG"}, supply a null
         (\texttt{!}) value.
      }
      \sstsubsection{
         OUTTYPE = LITERAL (Read)
      }{
         The nature of the output NDF(s).  The same options are available as for
         INTYPE, but relate to NDFs OUT1 and OUT instead of IN1 and IN2.

         The default is \texttt{"REAL\_IMAG"} if IN1 is a complex NDF, otherwise
         it is \texttt{"COMPLEX"}.  \texttt{[]}
      }
      \sstsubsection{
         TITLE1 = LITERAL (Read)
      }{
         The title for the first output NDF.
      }
      \sstsubsection{
         TITLE2 = LITERAL (Read)
      }{
         The title for the second output NDF.
      }
   }
   \sstresparameters{
      \sstsubsection{
         CALCMODE = LITERAL (Write)
      }{
         Set to indicate the type of calculation that was performed:
         \texttt{"To polar"} , \texttt{"From polar"}  or \texttt{"None"}.
      }
  }
   \sstexamples{
      \sstexamplesubsection{
         complex realmap imagmap cmplxmap
      }{
         This example combines real and imaginary parts into a complex NDF.
      }
      \sstexamplesubsection{
         complex cplxmap ! modulusmap ! OUTTYPE=MOD\_ARG
      }{
         This example would compute the modulus from a complex NDF.
      }
   }
   \sstimplementationstatus{
      \sstitemlist{

         \sstitem
         This routine correctly processes the \htmlref{AXIS}{apndf:axis}, DATA,
         \htmlref{QUALITY}{apndf:quality}, \htmlref{LABEL}{apndf:label},
         \htmlref{TITLE}{apndf:title}, \htmlref{UNITS}{apndf:units},
         \htmlref{WCS}{apndf:wcs}, and \htmlref{HISTORY}{apndf:history}~ components
         of the first input NDF and propagates all of that NDF's
         \htmlref{extensions}{apndf:extensions}.
         UNITS is set to \texttt{"radians"} for OUTTYPE = \texttt{"MOD\_ARG"}.

         \sstitem
         The DATA component is processed in double precision.  The output
         NDFs have type \_DOUBLE, except when OUTTYPE ="COMPLEX" where it is
         COMPLEX\_DOUBLE.
      }
   }
}

\sstroutine{
   CONFIGECHO
}{
   Displays one or more configuration parameters
}{
   \sstdescription{
      This application displays the name and value of one or all
      configuration parameters, specified using Parameters CONFIG or
      NDF. If a single parameter is displayed, its value is also
      written to an output parameter. If the parameter value is not
      specified by the CONFIG, NDF or DEFAULTS parameter, then the
      value supplied for DEFVAL is displayed.

      If an input NDF is supplied then configuration parameters
      are read from its history (see Parameters NDF and APPLICATION).

      If values are supplied for both CONFIG and NDF, then the
      differences between the two sets of configuration parameters
      are displayed (see Parameter NDF).
   }
   \sstusage{
      configecho name config [defaults] [select] [defval]
   }
   \sstparameters{
      \sstsubsection{
         APPLICATION = LITERAL (Read)
      }{
         When reading configuration parameters from the history
         of an NDF, this parameter specifies the name of the application
         to find in the history. There must be a history component
         corresponding to the value of this parameter, and it must
         include a CONFIG group.  \tt{[!]}
      }
      \sstsubsection{
         CONFIG = GROUP (Read)
      }{
         Specifies values for the configuration parameters.  If the string
         \texttt{"def"} (case-insensitive) or a null (\texttt{{!}}) value is supplied, the
         configuration parameters are obtained using Parameter NDF. If
         a null value is also supplied for NDF, a set of default
         configuration parameter values will be used, as specified by
         Parameter DEFAULTS.

         The supplied value should be either a comma-separated list of
         strings or the name of a text file preceded by an up-arrow
         character \texttt{"$\wedge$"}, containing one or more comma-separated lists of
         strings.  Each string is either a \texttt{"keyword=value"} setting, or the
         name of a text file preceded by an up-arrow character \texttt{"$\wedge$"}.  Such
         text files should contain further comma-separated lists which
         will be read and interpreted in the same manner (any blank lines
         or lines beginning with \texttt{"\#"} are ignored).  Within a text file,
         newlines can be used as delimiters, as well as commas.  Settings
         are applied in the order in which they occur within the list,
         with later settings overriding any earlier settings given for
         the same keyword.

         Each individual setting should be of the form
         \texttt{"$<$keyword$>$=$<$value$>$"}.
         If a non-null value is supplied for Parameter DEFAULTS, an error
         will be reported if CONFIG includes values for any parameters
         that are not included in DEFAULTS.
      }
      \sstsubsection{
         DEFAULTS = LITERAL (Read)
      }{
         The path to a file containing the default value for every
         allowed configuration parameter.  If null (\texttt{{!}}) is supplied, no
         defaults will be supplied for parameters that are not specified
         by CONFIG, and no tests will be performed on the validity of
         paramter names supplied by CONFIG.  \texttt{[!]}
      }
      \sstsubsection{
         DEFVAL = LITERAL (Read)
      }{
         The value to return if no value can be obtained for the named
         parameter, or if the value is \texttt{"$<$undef$>$"}.  \texttt{[<***>]}
      }
      \sstsubsection{
         LOGFILE = LITERAL (Read)
      }{
         The name of a text file in which to store the displayed
         configuration parameters. \texttt{ [!]}
      }
      \sstsubsection{
         NAME = LITERAL (Read)
      }{
         The name of the configuration parameter to display. If it is
         set to null (\texttt{{!}}), then all parameters defined in the configuration
         are displayed.
      }
      \sstsubsection{
         NDF = NDF (Read)
      }{
         An NDF file containing history entries which include
         configuration parameters.
      }
      \sstsubsection{
         SELECT = GROUP (Read)
      }{
         A group that specifies any alternative prefixes that can be
         included at the start of any parameter name.  For instance, if
         this group contains the two entries \texttt{"450=1"} and \texttt{"850=0"},
         then either CONFIG or DEFAULTS can specify two values for any single
         parameter---one for the parameter prefixed by \texttt{"450."} and another
         for the parameter prefixed by \texttt{"850."}.  Thus, for instance, if
         DEFAULTS defines a parameter called \texttt{"filter"}, it could include
         \texttt{"450.filter=300"} and \texttt{"850.filter=600"}.  The CONFIG parameter
         could then either set the filter parameter for a specific prefix (as
         in \texttt{"450.filter=234"}); or it could leave the prefix unspecified,
         in which case the prefix used is the first one with a
         non-zero value in SELECT (450 in the case of this example---850
         has a value zero in SELECT).  Thus the names of the items in
         SELECT define the set of allowed alternative prefixes, and the
         values indicate which one of these alternatives is to be used
         (the first one with non-zero value).  \texttt{[!]}
      }
      \sstsubsection{
         SORT = \_LOGICAL (Read)
      }{
         If \texttt{TRUE} then sort the listed parameters in to alphabetical order.
         Otherwise, retain the order they have in the supplied
         configuration.  Only used if a null (\texttt{{!}}) value is supplied for
         Parameter NAME.  \texttt{[FALSE]}
      }

   }
   \sstresparameters{
      \sstsubsection{
         VALUE = LITERAL (Write)
      }{
         The value of the configuration parameter, or \texttt{"<***>"} if the
         parameter has no value in CONFIG and DEFAULTS.
      }
   }
   \sstexamples{
      \sstexamplesubsection{
         configecho m81 $\wedge$myconf
      }{
         Report the value of configuration parameter \texttt{"m81"} defined within
         the file \texttt{myconf}.  If the file does not contain a value for
         \texttt{"m81"}, then \texttt{<***>} is displayed.
      }
      \sstexamplesubsection{
         configecho type $\wedge$myconf select="m57=0,m31=1,m103=0"
      }{
         Report the value of configuration parameter \texttt{"type"} defined within
         the file \texttt{myconf}.  If the file does not contain a value for
         \texttt{"type"}, then the value of \texttt{"m31.type"} will be reported
         instead.  If neither is present, then \texttt{<***>} is displayed.
      }
      \sstexamplesubsection{
         configecho flt.filt\_edge\_largescale $\backslash$ \\
                 config=$\wedge$/star/share/smurf/dimmconfig.lis $\backslash$ \\
                 defaults=/star/bin/smurf/smurf\_makemap.def select="450=1,850=0"
      }{
         Report the value of configuration parameter \texttt{flt.filt\_edge\_largescale}
         defined within the file \texttt{/star/share/smurf/dimmconfig.lis}, using
         defaults from the file \texttt{/star/bin/smurf/smurf\_makemap.def}.  If
         \texttt{dimmconfig.lis} does not contain a value for \texttt{flt.filt\_edge\_largescale}
         then it is searched for \texttt{450.flt.filt\_edge\_largescale} instead.  An
         error is reported if \texttt{dimmconfig.lis} contains values for any
         items that are not defined in \texttt{smurf\_makemap.def}.
      }
      \sstexamplesubsection{
         configecho ndf=omc1 config=$\wedge$/star/share/smurf/dimmconfig.lis $\backslash$ \\
                defaults=/star/bin/smurf/smurf\_makemap.def $\backslash$ \\
                application=makemap name=! sort select="450=0,850=1"
      }{
         Show how the configuration used to generate the 850$\mu$m map
         of OMC1 differs from the basic \texttt{dimmconfig.lis} file.
      }
   }
}

\sstroutine{
   CONTOUR
}{
   Contours a two-dimensional NDF
}{
   \sstdescription{
      This application produces a contour map of a two-dimensional \NDFref{NDF} on
      the \htmlref{current graphics device}{se:devglobal}, with single-pixel resolution.  Contour
      levels can be chosen automatically in various ways, or specified
      explicitly (see Parameter MODE).  In addition, this application can
      also draw an outline around either the whole data array, or around
      the good pixels in the data array (set MODE to \texttt{"Bounds"} or \texttt{"Good"}).

      The plot is produced within the \htmlref{current graphics database picture}{se:agitate},
      and may be aligned with an existing DATA picture if the existing
      picture contains suitable \htmlref{co-ordinate Frame}{se:domains}~  information (see
      Parameter CLEAR).

      The appearance of each contour can be controlled in several ways.  The
      pens used can be rotated automatically (see Parameter PENROT).  Contours
      below a given threshold value can be drawn dashed (see Parameter DASHED).
      Alternatively, the appearance of each contour can be set explicitly
      (see Parameter PENS).

      Annotated axes can be produced (see Parameter AXES), and the appearance
      of the axes can be controlled in detail (see Parameter STYLE).  The
      axes show co-ordinates in the current co-ordinate Frame of the supplied
      NDF.

      A list of the contour levels can be displayed to the right of the
      contour map (see Parameter KEY).  The appearance and position of this
      key may be controlled using Parameters KEYSTYLE and KEYPOS.
   }
   \sstusage{
     \begin{tabular}{c l}
      contour ndf [comp] mode ncont [key] [device] &
      {  $\begin{cases}
          \mathtt{low=?\ high=?} \\
          \mathtt{percentiles=?} \\
          \mathtt{sigmas=?}
        \end{cases} $}\\
      & mode\\
    \end{tabular}
   }
   \sstparameters{
      \sstsubsection{
         AXES = \_LOGICAL (Read)
      }{
         \texttt{TRUE} if labelled and annotated axes are to be drawn around the
         contour map, showing the current co-ordinate Frame of the
         supplied NDF.  The appearance of the axes can be controlled using
         the STYLE parameter. If a null (\texttt{{!}}) value is supplied, then axes
         will be drawn unless the CLEAR parameter indicates that the graphics
         device is not being cleared.  \texttt{[!]}
      }
      \sstsubsection{
         CLEAR = \_LOGICAL (Read)
      }{
         \texttt{TRUE} if the graphics device is to be cleared before displaying
         the contour map.  If you want the contour map to be drawn over
         the top of an existing DATA picture, then set CLEAR to \texttt{FALSE}.  The
         contour map will then be drawn in alignment with the displayed
         data.  If possible, alignment occurs within the current co-ordinate
         Frame of the NDF.  If this is not possible, (for instance if
         suitable \htmlref{WCS}{apndf:wcs} information was not stored with the existing DATA
         picture), then alignment is attempted in PIXEL
         co-ordinates.  If this is not possible, then alignment is
         attempted in GRID co-ordinates.  If this is not possible, then
         alignment is attempted in the first suitable Frame found in the NDF
         irrespective of its domain.  A message is displayed indicating the
         domain in which alignment occurred.  If there are no suitable Frames
         in the NDF then an error is reported.  \texttt{[TRUE]}
      }
      \sstsubsection{
         COMP = \htmlref{LITERAL}{se:parmenu} (Read)
      }{
         The NDF component to be contoured.  It may be \texttt{"Data"},
         \texttt{"Quality"}, \texttt{"Variance"}, or \texttt{"Error"} (where \texttt{"Error"} is an
         alternative to \texttt{"Variance"} and causes the square root of the
         variance values to be displayed).  If
         \texttt{"Quality"} is specified, then the quality values are treated as
         numerical values (in the range 0 to 255).  \texttt{["Data"]}
      }
      \sstsubsection{
         DASHED = \_REAL (Read)
      }{
         The height below which the contours will be drawn with dashed
         lines (if possible).  A null value (\texttt{{!}}) results in contours being
         drawn with the styles specified by Parameters PENS, PENROT, and
         STYLE.  \texttt{[!]}
      }
      \sstsubsection{
         DEVICE = \htmlref{DEVICE}{se:selgradev} (Read)
      }{
         The plotting device.  \texttt{[}current graphics device\texttt{{]}}
      }
      \sstsubsection{
         FAST = \_LOGICAL (Read)
      }{
         If \texttt{TRUE}, then a faster, but in certain cases less-accurate,
         method is used to draw the contours.  In fast mode, contours may be
         incorrectly placed on the display if the mapping between graphics
         co-ordinates and the current co-ordinate Frame of the supplied NDF
         has any discontinuities, or is strongly non-linear.  This may be
         the case, for instance, when displaying all-sky maps on top of
         each other.  \texttt{[TRUE]}
      }
      \sstsubsection{
         FILL = \_LOGICAL (Read)
      }{
         The contour plot normally has square pixels, in other words
         a specified length along each axis corresponds to the same number
         of pixels.  However, for images with markedly different
         dimensions this default behaviour may not be suitable or give
         the clearest plot.  When FILL is \texttt{TRUE}, the square-pixel
         constraint is relaxed and the contour plot is the largest
         possible within the current picture.  When FILL is \texttt{FALSE}, the
         pixels are square.  \texttt{[FALSE]}
      }
      \sstsubsection{
         FIRSTCNT = \_REAL (Read)
      }{
         Height of the first contour (Linear and Magnitude modes).
      }
      \sstsubsection{
         HEIGHTS() = \_REAL (Read)
      }{
         The required contour levels (Free mode).
      }
      \sstsubsection{
         KEY = \_LOGICAL (Read)
      }{
         \texttt{TRUE} if a key of the contour level versus pixel value is to be
         produced.  The appearance of this key can be controlled using
         Parameter KEYSTYLE, and its position can be controlled using
         Parameter KEYPOS.  \texttt{[TRUE]}
      }
      \sstsubsection{
         KEYPOS() = \_REAL (Read)
      }{
         Two values giving the position of the key.  The first value gives
         the gap between the right-hand edge of the contour map and the
         left-hand edge of the key (0.0 for no gap, 1.0 for the largest gap).
         The second value gives the vertical position of the top of the key (1.0
         for the highest position, 0.0 for the lowest).  If the second value
         is not given, the top of the key is placed level with the top of the
         contour map.  Both values should be in the range 0.0 to 1.0.  If a
         key is produced, then the right-hand margin specified by Parameter
         MARGIN is ignored.  \texttt{[}current value\texttt{{]}}
      }
      \sstsubsection{
         KEYSTYLE = \htmlref{GROUP}{se:groups} (Read)
      }{
         A group of attribute settings describing the plotting style to use
         for the key (see Parameter KEY).

         A comma-separated list of strings should be given in which each
         string is either an attribute setting, or the name of a text
         file preceded by an up-arrow character \texttt{"$\wedge$"}.  Such text files
         should contain further comma-separated lists which will be
         read and interpreted in the same manner.  Attribute settings
         are applied in the order in which they occur within the list,
         with later settings overriding any earlier settings given for
         the same attribute.

         Each individual attribute setting should be of the form:

            $<$name$>$=$<$value$>$


         where $<$name$>$ is the name of a plotting attribute, and $<$value$>$
         is the value to assign to the attribute.  Default values will be
         used for any unspecified attributes.  All attributes will be
         defaulted if a null value (\texttt{{!}})---the initial default---is supplied.
         To apply changes of style to only the current invocation, begin these
         attributes with a plus sign.  A mixture of persistent and temporary
         style changes is achieved by listing all the persistent attributes
         followed by a plus sign then the list of temporary attributes.

         See \slhyperref{Plotting Attributes}{Section~}{}{ap:plotting_attr}
         for a description of the available attributes.  Any unrecognised
         attributes are ignored (no error is reported).

         The heading in the key can be changed by setting a value for
         the Title attribute (the supplied heading is split into lines
         of no more than 17 characters).  The appearance of the
         heading is controlled by attributes
         \htmlattref{Colour(Title)}{Colour(element)},
         \htmlattref{Font(Title)}{Font(element)}, \emph{etc}.  The
         appearance of the contour indices is controlled by attributes
         \htmlattref{Colour(TextLab)}{Colour(element)},
         \htmlattref{Font(TextLab)}{Font(element)}, \emph{etc.} (the
         synonym \att{Index} can be used in place of
         \htmlattref{TextLab}{TextLab(axis)}). The appearance of the
         contour values is controlled by attributes {\att
         Colour(NumLab)}, \att{Font(NumLab)}, \emph{etc.} (the
         synonym \att{Value} can be used in place of
         \htmlattref{NumLab}{NumLab(axis)}).
         Contour indices are formatted using attributes
         \htmlattref{Format(1)}{Format(axis)},
         \htmlattref{Digits(1)}{Digits/Digits(axis)}, \emph{etc.} (the
         synonym \att{Index} can be used in place of value 1).
         Contour values are formatted using attributes {\att
         Format(2)}, \emph{etc.} (the synonym \att{Value} can be used
         in place of the value 2).  \texttt{[}current value\texttt{{]}}
      }
      \sstsubsection{
         LABPOS = \_REAL() (Read)
      }{
         Only used if Parameter MODE is set to \texttt{"Good"} or \texttt{"Bounds"}.  It
         specifies the position at which to place a label identifying the
         input NDF within the plot.  The label is drawn parallel to the
         first pixel axis.  Two values should be supplied for LABPOS.  The
         first value specifies the distance in millimetres along the first
         pixel axis from the centre of the bottom-left pixel to the left
         edge of the label.  The second value specifies the distance in
         millimetres along the second pixel axis from the centre of the
         bottom-left pixel to the baseline of the label.  If a null (\texttt{{!}})
         value is given, no label is produced.  The appearance of the label
         can be set by using the STYLE parameter
         (for instance \texttt{"Size(strings)=2"}).  \texttt{[}current value\texttt{{]}}
      }
      \sstsubsection{
         LASTCNT = \_REAL (Read)
      }{
         Height of the last contour (Linear and Magnitude modes).
      }
      \sstsubsection{
         MARGIN( 4 ) = \_REAL (Read)
      }{
         The widths of the margins to leave around the contour map for
         axis annotation.  The widths should be given as fractions of the
         corresponding dimension of the current picture.  The actual margins
         used may be increased to preserve the aspect ratio of the DATA
         picture.  Four values may be given, in the order; bottom, right,
         top, left.  If fewer than four values are given, extra values are
         used equal to the first supplied value.  If these margins are too
         narrow any axis annotation may be clipped.  If a null (\texttt{{!}}) value is
         supplied, the value used is \texttt{0.15} (for all edges) if annotated axes
         are being produced, and zero otherwise.  See also Parameter KEYPOS.
         \texttt{[}current value\texttt{{]}}
      }
      \sstsubsection{
         MODE = \htmlref{LITERAL}{se:parmenu} (Read)
      }{
         The method used to select the contour levels.  The options are:

         \ssthitemlist{

            \sstitem
              \texttt{"Area"} --- The contours enclose areas of the array for which
              the equivalent radius increases by equal increments.  You
              specify the number of levels.

            \sstitem
              \texttt{"Automatic"} --- The contour levels are equally spaced between
              the maximum and minimum pixel values in the array.  You supply
              the number of contour levels.

            \sstitem
              \texttt{"Bounds"} --- A single `contour' is drawn representing the
              bounds of the input array.  A label may also be added (see
              Parameter LABPOS).

            \sstitem
              \texttt{"Equalised"} --- You define the number of equally spaced
              percentiles.

            \sstitem
              \texttt{"Free"} --- You specify a series of contour values explicitly.

            \sstitem
              \texttt{"Good"} --- A single `contour' is drawn outlining the good pixel
              values.  A label may also be added (see Parameter LABPOS).

            \sstitem
              \texttt{"Linear"} --- You define the number of contours, the start
              contour level and linear step between contours.

            \sstitem
              \texttt{"Magnitude"} --- You define the number of contours, the start
              contour level and step between contours.  The step size is in
              magnitudes so the \textit{n}th contour is dex(-0.4$*$(n-1)$*$step) times the
              start contour level.

            \sstitem
              \texttt{"Percentiles"} --- You specify a series of percentiles.

            \sstitem
              \texttt{"Scale"} ---  The contour levels are equally spaced between
              two pixel values that you specify.  You also supply the
              number of contour levels, which must be at least two.
         }
         If the contour map is aligned with an existing DATA picture (see
         Parameter CLEAR), then only part of the supplied NDF may be
         displayed.  In this case, the choice of contour levels is based
         on the data within a rectangular section of the input NDF
         enclosing the existing DATA picture.  Data values outside this
         section are ignored.
      }
      \sstsubsection{
         NCONT = \_INTEGER (Read)
      }{
         The number of contours to draw (only required in certain modes).
         It must be between 1 and 50.  If the number is large, the plot
         may be cluttered and take longer to produce.  The initial
         suggested default of \texttt{6} gives reasonable results.
      }
      \sstsubsection{
         NDF = NDF (Read)
      }{
         NDF structure containing the two-dimensional image to be
         contoured.
      }
      \sstsubsection{
         PENROT = \_LOGICAL (Read)
      }{
         If \texttt{TRUE}, the plotting pens are cycled through the contours to
         aid identification of the contour heights.  Only accessed if
         pen definitions are not supplied using Parameter PENS.  \texttt{[FALSE]}
      }
      \sstsubsection{
         PENS = \htmlref{GROUP}{se:groups} (Read)
      }{
         A group of strings, separated by semicolons, each of which specifies
         the appearance of a pen to be used to draw a contour.  The first
         string in the group describes the pen to use for the first contour,
         the second string describes the pen for the second contour, \emph{etc}.  If
         there are fewer strings than contours, then the supplied pens are
         cycled through again, starting at the beginning.  Each string should
         be a comma-separated list of plotting attributes to be used when drawing
         the contour.  For instance, the string \texttt{"width=10.0,colour=red,style=2"}
         produces a thick, red, dashed contour.  Attributes that are
         unspecified in a string default to the values implied by Parameter
         STYLE.  If a null value (\texttt{{!}}) is given for PENS, then the pens
         implied by Parameters PENROT, DASHED and STYLE are used.  \texttt{[!]}
      }
      \sstsubsection{
         PERCENTILES() = \_REAL (Read)
      }{
         Contour levels given as percentiles.  The values must lie
         between 0.0 and 100.0. (Percentiles mode).
      }
      \sstsubsection{
         STATS = \_LOGICAL (Read)
      }{
         If \texttt{TRUE}, the LENGTH and NUMBER statistics are computed.  \texttt{[FALSE]}.
      }
      \sstsubsection{
         STEPCNT = \_REAL (Read)
      }{
         Separation between contour levels, linear for Linear mode
         and in magnitudes for Magnitude mode.
      }
      \sstsubsection{
         STYLE = GROUP (Read)
      }{
         A group of attribute settings describing the plotting style to use
         for the contours and annotated axes.

         A comma-separated list of strings should be given in which each
         string is either an attribute setting, or the name of a text
         file preceded by an up-arrow character \texttt{"$\wedge$"}.  Such text files
         should contain further comma-separated lists which will be
         read and interpreted in the same manner.  Attribute settings
         are applied in the order in which they occur within the list,
         with later settings overriding any earlier settings given for
         the same attribute.

         Each individual attribute setting should be of the form:

            $<$name$>$=$<$value$>$


         where $<$name$>$ is the name of a plotting attribute, and $<$value$>$
         is the value to assign to the attribute.  Default values will be
         used for any unspecified attributes.  All attributes will be
         defaulted if a null value (\texttt{{!}})---the initial default---is supplied.
         To apply changes of style to only the current invocation, begin these
         attributes with a plus sign.  A mixture of persistent and temporary
         style changes is achieved by listing all the persistent attributes
         followed by a plus sign then the list of temporary attributes.

         See \slhyperref{Plotting Attributes}{Section~}{}{ap:plotting_attr}
         for a description of the available attributes.  Any unrecognised
         attributes are ignored (no error is reported).

         The appearance of the contours is controlled by the attributes
         \htmlattref{Colour(Curves)}{Colour(element)},
         \htmlattref{Width(Curves)}{Width(element)}, \emph{etc.} (the synonym
         \att{Contours} may be used in place of \att{Curves}).  The contour
         appearance established in  this way may be modified using Parameters
         PENS, PENROT and DASHED.  \texttt{[}current value\texttt{{]}}
      }
      \sstsubsection{
         USEAXIS = \htmlref{GROUP}{se:groups} (Read)
      }{
         USEAXIS is only accessed if the current co-ordinate Frame of the
         NDF has more than two axes.  A group of two strings should be
         supplied specifying the two axes which are to be used when annotating
         and aligning the contour map.  Each axis can be specified
         using one of the following options.

         \ssthitemlist{

            \sstitem
            Its integer index within the current Frame of the
            input  NDF (in the range 1 to the number of axes in the
            current Frame).

            \sstitem
            Its \htmlattref{Symbol}{Symbol(axis)}~ string such as
            \texttt{"RA"} or \texttt{"VRAD"}.

            \sstitem
            A generic option where \texttt{"SPEC"} requests the spectral axis,
            \texttt{"TIME"} selects the time axis, \texttt{"SKYLON"} and
            \texttt{"SKYLAT"} picks the sky longitude and latitude axes
            respectively.  Only those axis domains present are
            available as options.
         }

         A list of acceptable values is displayed if an illegal value is
         supplied.  If a null (\texttt{{!}}) value is supplied, the axes with
         the same indices as the two significant NDF pixel axes are used.
         \texttt{[!]}
      }
   }
   \sstresparameters{
      \sstsubsection{
         LENGTH() = \_REAL (Write)
      }{
         On exit this holds the total length in pixels of the contours at each
         selected height.  These values are only computed when Parameter STATS
         is \texttt{TRUE}.
      }
      \sstsubsection{
         NUMBER() = \_INTEGER (Write)
      }{
         On exit this holds the number of closed contours at each selected
         height.  Contours are not closed if they intersect a bad pixel or the
         edge of the image.  These values are only computed when
         Parameter STATS is \texttt{TRUE}.
      }
   }
   \sstexamples{
      \sstexamplesubsection{
         contour myfile
      }{
         Contours the data array in the NDF called myfile on the current
         graphics device.  All other settings are defaulted, so for
         example the current mode for determining heights is used, and
         a key is plotted.
      }
      \sstexamplesubsection{
         contour taurus1(100:199,150:269,4)
      }{
         Contours a two-dimensional section of the three-dimensional NDF called
         taurus1 on the \htmlref{current graphics device}{se:devglobal}.  The section extends from
         pixel (100,150,4) to pixel (199,269,4).
      }
      \sstexamplesubsection{
         contour ngc6872 mode=au ncont=5 device=ps\_l pens="style=1;style=2"
      }{
         Contours the data array in the NDF called ngc6872 on the ps\_l
         graphics device.  Five equally spaced contours between the
         maximum and minimum data values are drawn, alternating between
         line styles 1 and 2 (solid and dashed).
      }
      \sstexamplesubsection{
         contour ndf=ngc6872 mode=au ncont=5 penrot style="$\wedge$mysty,grid=1"
      }{
         As above except that the current graphics device is used, pens
         are cycled automatically, and the appearance of the axes is read
         from text file \texttt{mysty}.  The plotting attribute \htmlattref{Grid}{Grid}~ is set
         explicitly to 1 to ensure that a co-ordinate grid is drawn over
         the plot.  The text file \texttt{mysty} could, for instance, contain the
         two lines \texttt{"Title=NGC6872 at 25 microns"} and \texttt{"grid=0"}.
         The Title setting gives the title to display at the top of the axes.
         The Grid setting would normally prevent a co-ordinate grid being
         drawn, but is overridden in this example by the explicit setting
         for Grid which follows the file name.
      }
      \sstexamplesubsection{
         contour m51 mode=li firstcnt=10 stepcnt=2 ncont=4 keystyle=$\wedge$keysty
      }{
         Contours the data array in the NDF called m51 on the
         current graphics device.  Four contours at heights 10, 12, 14,
         and 16 are drawn.  A key is plotted using the style specified
         in the text file \texttt{keysty}.  This file could, for instance, contain
         the two lines \texttt{"font=3"} and \texttt{"digits(2)=4"} to cause all text in
         the key to be drawn using \PGPLOT\  font 3 (an italic fount), and
         4 digits to be used when formatting the contour values.
      }
      \sstexamplesubsection{
         contour ss443 mode=pe percentiles=[80,90,95] stats keypos=0.02
      }{
         Contours the data array in the NDF called ss443 on the current
         graphics device.  Contours at heights corresponding to the 80,
         90 and 95 percentiles are drawn.  The key is placed closer
         to the contour map than usual.  Contour statistics are computed.
      }
      \sstexamplesubsection{
         contour skyflux mode=eq ncont=5 dashed=0 pens='colour=red' noclear
      }{
         Contours the data array in the NDF called skyflux on the current
         graphics device.  The contour map is automatically aligned with
         any existing DATA picture, if possible.  Contours at heights
         corresponding to the 10, 30, 50, 70 and 90 percentiles (of the
         data within the picture) are drawn in red.  Those contours whose
         values are negative will appear as dashed lines.
      }
      \sstexamplesubsection{
         contour comp=d nokey penrot style="grid=1,title=My data" $\backslash$
      }{
         Contours the data array in the current NDF on
         the current graphics device using the current method for
         height selection.  No key is drawn.  The appearance of the
         contours cycles every third contour.  A co-ordinate grid is
         drawn over the plot, and a title of \texttt{"My data"} is displayed at
         the top.
      }
      \sstexamplesubsection{
         contour comp=v mode=fr heights=[10,20,40,80] $\backslash$
      }{
         Contours the variance array in the current NDF on the
         current graphics device.  Contours at 10, 20, 40 and 80 are
         drawn.
      }
   }
   \sstnotes{
      \sstitemlist{

         \sstitem
         If no \htmlattref{Title}{plotel:Title}~ is specified via the
         STYLE parameter, then the \htmlref{TITLE}{apndf:title}
         component in the NDF is used as the default title for the
         annotated axes.  Should the NDF not have a TITLE component,
         then the default title is instead taken from current
         co-ordinate Frame in the NDF, unless this attribute has not
         been set explicitly, whereupon the name of the NDF is used as
         the default title.

         \sstitem
         The application stores a number of pictures in the graphics
         database in the following order: a FRAME picture containing the
         annotated axes, contours, and key; a KEY picture to store
         the key if present; and a DATA picture containing just the contours.
         Note, the FRAME picture is only created if annotated axes or a key
         has been drawn, or if non-zero margins were specified using Parameter
         MARGIN.  The world co-ordinates in the DATA picture will be pixel
         co-ordinates.  A reference to the supplied NDF, together with a copy
         of the WCS information in the NDF are stored in the DATA picture.  On
         exit the current database picture for the chosen device reverts to the
         input picture.
      }
   }
   \sstdiytopic{
      Related Applications
   }{
KAPPA: \htmlref{WCSFRAME}{WCSFRAME},
\htmlref{PICDEF}{PICDEF};
\xref{FIGARO}{sun86}{}: \xref{ICONT}{sun86}{ICONT},
\xref{SPECCONT}{sun86}{SPECCONT}.
   }
   \sstimplementationstatus{
      \sstitemlist{

         \sstitem
         Only real data can be processed directly.  Other \htmlref{non-complex numeric data types}{ap:HDStypes} will undergo a type conversion before the
         contour plot is drawn.

         \sstitem
         \htmlref{Bad pixels}{se:masking} and \htmlref{quality masking}{se:qualitymask} are supported.
      }
   }
}



\sstroutine{
   CONVOLVE
}{
   Convolves a pair of one- or two-dimensional NDFs together
}{
   \sstdescription{
      This application smooths a one- or two-dimensional \NDFref{NDF} using a
      Point-Spread Function given by a second NDF.  The output NDF is
      normalised to the same mean data value as the input NDF (if Parameter
      NORM is set to \texttt{TRUE}), and is the same size as the input NDF.
   }
   \sstusage{
      convolve in psf out xcentre ycentre
   }
   \sstparameters{
      \sstsubsection{
         AXES( 2 ) = \_INTEGER (Read)
      }{
         This parameter is only accessed if the NDF has exactly three
         significant pixel axes.  It should be set to the indices of the
         NDF pixel axes which span the plane in which smoothing is to
         be applied.  All pixel planes parallel to the specified plane
         will be smoothed independently of each other.  The dynamic
         default is the indices of the first two significant axes in
         the NDF.  \texttt{[]}
      }
      \sstsubsection{
         IN = NDF (Read)
      }{
         The input NDF containing the image to be smoothed.
      }
      \sstsubsection{
         NORM = \_LOGICAL (Read)
      }{
         Determines how the output NDF is normalised to take account of
         the total data sum in the PSF, and of the presence of bad pixels
         in the input NDF.  If \texttt{TRUE}, bad pixels are excluded from the
         data sum for each output pixel, and the associated weight for the
         output pixel is reduced appropriately.  The supplied PSF is
         normalised to a total data sum of unity so that the output NDF has
         the same normalisation as the input NDF.  If NORM is \texttt{FALSE},
         bad pixels are replaced by the mean value and then included in the
         convolution as normal.  The normalisation of the supplied PSF is
         left unchanged, and so determines the normalisation of the output
         NDF.  \texttt{[TRUE]}
      }
      \sstsubsection{
         OUT = NDF (Write)
      }{
         The output NDF which is to contain the smoothed image.
      }
      \sstsubsection{
         PSF = NDF (Read)
      }{
         An NDF holding the Point-Spread Function (PSF) with which the
         input image is to be smoothed.  An error is reported if the PSF
         contains any bad pixels.  The PSF can be centred anywhere
         within the image (see Parameters XCENTRE and YCENTRE).  A
         constant background is removed from the PSF before use.  This
         background level is equal to the minimum of the absolute value
         in the four corner pixel values.  The PSF is assumed to be zero
         beyond the bounds of the supplied NDF.  It should have the
         same number of dimensions as the NDF being smoothed, unless
         the input NDF has three significant dimensions, whereupon the
         PSF must be two-dimensional.  It will be normalised to a total
         data sum of unity if Parameter NORM is \texttt{TRUE}.
      }
      \sstsubsection{
         TITLE = LITERAL (Read)
      }{
         A \htmlref{title}{apndf:title} for the output NDF.  A null (\texttt{{!}})
         value means using the title of the input NDF.  \texttt{[!]}
      }
      \sstsubsection{
         WLIM = \_REAL (Read)
      }{
         If the input array contains bad pixels, and NORM is \texttt{TRUE},
         then this parameter may be used to determine the number of good
         pixels that must be present within the smoothing box before a
         valid output pixel is generated.  It can be used, for example, to
         prevent output pixels from being generated in regions where there
         are relatively few good pixels to contribute to the smoothed
         result.

         By default, a null (\texttt{{!}}) value is used for WLIM, which causes
         the pattern of bad pixels to be propagated from the input
         image to the output image unchanged.  In this case, smoothed
         output values are only calculated for those pixels which are
         not bad in the input image.

         If a numerical value is given for WLIM, then it specifies the
         minimum total weight associated with the good pixels in the
         smoothing box required to generate a good output pixel
         (weights for each pixel are defined by the normalised PSF).
         If this specified minimum weight is not present, then a bad
         output pixel will result, otherwise a smoothed output value
         will be calculated.  The value of this parameter should lie
         between 0.0 and 1.0.  A value of \texttt{0.0} will result in a good
         output pixel being created even if only one good input pixel
         contributes to it.  A value of \texttt{1.0} will result in a good output
         pixel being created only if all the input pixels which
         contribute to it are good.  See also Parameter NORM.  \texttt{[!]}
      }
      \sstsubsection{
         XCENTRE = \_INTEGER (Read)
      }{
         The \textit{x} pixel index (column number) of the centre of the PSF
         within the supplied PSF array.  The suggested default is the
         centre of the PSF array.  (This is how the PSF command would
         generate the array.)
      }
      \sstsubsection{
         YCENTRE = \_INTEGER (Read)
      }{
         The \textit{y} pixel index (line number) of the centre of the PSF
         within the supplied PSF array.  The suggested default is the
         centre of the PSF array.  (This is how the PSF command would
         generate the array.)
      }
   }
   \sstexamples{
      \sstexamplesubsection{
         convolve ccdframe iraspsf ccdlores 50 50
      }{
         The image in the NDF called ccdframe is convolved using the
         PSF in NDF iraspsf to create the smoothed image ccdlores.  The
         centre of the PSF image in iraspsf is at pixel indices
         (50,~50).  Any bad pixels in the input image are propagated to
         the output.
      }
      \sstexamplesubsection{
         convolve ccdframe iraspsf ccdlores 50 50 wlim=1.0
      }{
         As above, but good output values are only created for pixels
         which have no contributions from bad input pixels.
      }
      \sstexamplesubsection{
         convolve ccdframe iraspsf ccdlores $\backslash$
      }{
         As in the first example except the centre of the PSF is located
         at the centre of the PSF array.
      }
   }
   \sstnotes{
      \sstitemlist{

         \sstitem
         The algorithm used is based on the multiplication of the
         Fourier transforms of the input image and PSF image.

         \sstitem
         A PSF can be created using the PSF command or MATHS if the
         PSF is an analytic function.

      }
   }
   \sstdiytopic{
      Related Applications
   }{
KAPPA: \htmlref{BLOCK}{BLOCK},
\htmlref{FFCLEAN}{FFCLEAN},
\htmlref{GAUSMOOTH}{GAUSMOOTH},
\htmlref{MATHS}{MATHS},
\htmlref{MEDIAN}{MEDIAN},
\htmlref{PSF}{PSF};
\xref{FIGARO}{sun86}{}: \xref{ICONV3}{sun86}{ICONV3},
\xref{ISMOOTH}{sun86}{ISMOOTH},
\xref{IXSMOOTH}{sun86}{IXSMOOTH},
\xref{MEDFILT}{sun86}{MEDFILT}.
   }
   \sstimplementationstatus{
      \sstitemlist{

         \sstitem
         This routine correctly processes the \htmlref{AXIS}{apndf:axis}, DATA, \htmlref{QUALITY}{apndf:quality},
         \htmlref{VARIANCE}{apndf:variance}, \htmlref{LABEL}{apndf:label}, \htmlref{TITLE}{apndf:title}, \htmlref{UNITS}{apndf:units}, \htmlref{WCS}{apndf:wcs}, and \htmlref{HISTORY}{apndf:history}~ components of the
         input NDF and propagates all \htmlref{extensions}{apndf:extensions}.

         \sstitem
         Processing of \htmlref{bad pixels}{se:masking} and automatic \htmlref{quality masking}{se:qualitymask} are
         supported.

         \sstitem
         All \htmlref{non-complex numeric data types}{ap:HDStypes} can be handled.  Arithmetic
         is performed using double-precision floating point.

      }
   }
}

\sstroutine{
   COPYBAD
}{
   Copies bad pixels from one NDF file to another
}{
   \sstdescription{
      This application copies \htmlref{bad pixels}{se:masking}~ from one \NDFref{NDF} file
      to another.  It takes in two NDFs (Parameters IN and REF), and creates a third
      (Parameter OUT) which is a copy of IN, except that any pixel which
      is set bad in the DATA array of REF, is also set bad in the DATA
      and \htmlref{VARIANCE}{apndf:variance}~ (if available) arrays in OUT.

      By setting the INVERT Parameter \texttt{TRUE}, the opposite effect can be
      produced (\emph{i.e.} any pixel that is not set bad in the DATA array
      of REF, is set bad in OUT and the others are left unchanged).
   }
   \sstusage{
      copybad in ref out [title]
   }
   \sstparameters{
      \sstsubsection{
         IN = NDF (Read)
      }{
         NDF containing the data to be copied to OUT.
      }
      \sstsubsection{
         INVERT = \_LOGICAL (Read)
      }{
         If \texttt{TRUE}, then the bad and good pixels within the reference NDF
         specified by Parameter REF are inverted before being used (that
         is, good pixels are treated as bad and bad pixels are treated as
         good).  \texttt{[FALSE]}
      }
      \sstsubsection{
         OUT = NDF (Write)
      }{
         The output NDF.
      }
      \sstsubsection{
         REF = NDF (Read)
      }{
         NDF containing the bad pixels to be copied to OUT.
      }
      \sstsubsection{
         TITLE = LITERAL (Read)
      }{
         The title for the output NDF.  A null value will cause
         the title of the NDF supplied for Parameter IN to be used
         instead.  \texttt{[!]}
      }
   }
   \sstresparameters{
      \sstsubsection{
         NBAD = \_INTEGER (Write)
      }{
         The number of bad pixels copied to the output NDF.
      }
      \sstsubsection{
         NGOOD = \_INTEGER (Write)
      }{
         The number of pixels not made bad in the output NDF.
      }
   }
   \sstexamples{
      \sstexamplesubsection{
         copybad in=a ref=b out=c title="New image"
      }{
         This creates a NDF called c, which is a copy of the NDF called a.
         Any bad pixels present in the NDF called b are copied into the
         corresponding positions in c (non-bad pixels in b are ignored).
         The title of c is \texttt{"New image"}.
      }
   }
   \sstnotes{
      \sstitemlist{

         \sstitem
         If the two input NDFs have different pixel-index bounds, then
         they will be trimmed to match before being processed.  An error
         will result if they have no pixels in common.
      }
   }
   \sstdiytopic{
      Related Applications
   }{
KAPPA: \htmlref{SUBSTITUTE}{SUBSTITUTE},
\htmlref{NOMAGIC}{NOMAGIC},
\htmlref{FILLBAD}{FILLBAD},
\htmlref{PASTE}{PASTE},
\htmlref{GLITCH}{GLITCH}.
   }
   \sstimplementationstatus{
      \sstitemlist{

         \sstitem
         This routine correctly processes the \htmlref{WCS}{apndf:wcs}, \htmlref{AXIS}{apndf:axis}, DATA, \htmlref{QUALITY}{apndf:quality},
         \htmlref{LABEL}{apndf:label}, \htmlref{TITLE}{apndf:title}, \htmlref{UNITS}{apndf:units}, \htmlref{HISTORY}{apndf:history},
         and \htmlref{VARIANCE}{apndf:variance}~ components of an NDF data
         structure and propagates all \htmlref{extensions}{apndf:extensions}.

         \sstitem
         The \htmlref{BAD\_PIXEL flag}{apndf:bpflag} is set appropriately.

         \sstitem
         All \htmlref{non-complex numeric data types}{ap:HDStypes} can be handled.
      }
   }
}

\sstroutine{
   CREFRAME
}{
   Generates a test two-dimensional NDF with a selection of several forms
}{
   \sstdescription{
      This application creates a two-dimensional output \NDFref{NDF} containing
      artificial data of various forms (see Parameter MODE).  The output
      NDF can, optionally, have a VARIANCE component describing the noise
      in the data array (see Parameter VARIANCE), and additionally a
      randomly generated pattern of bad pixels (see Parameter BADPIX).
      Bad columns or rows of pixels can also be generated.
   }
   \sstusage{
      creframe out mode [lbound] [ubound]
        \newline\hspace*{1.5em}
        $\left\{ {\begin{tabular}{l}
                  mean=? \\
                  background=? distrib=? max=? min=? ngauss=? seeing=? \\
                  mean=? sigma=? \\
                  high=? low=?
                  \end{tabular} }
        \right.$
        \newline\hspace*{1.9em}
        \makebox[0mm][c]{\small mode}
   }
   \sstparameters{
      \sstsubsection{
         BACKGROUND = \_REAL (Read)
      }{
         Background intensity to be used in the generated data array.
         Must not be negative. (GS mode).
      }
      \sstsubsection{
         BADCOL = \_INTEGER (Read)
      }{
         The number of bad columns to include.  Only accessed if
         Parameter BADPIX is \texttt{TRUE}.  The bad columns are distributed
         at random using a uniform distribution.  \texttt{[0]}
      }
      \sstsubsection{
         BADPIX  = \_LOGICAL (Read)
      }{
         Whether or not bad pixels are to be included.  See also
         Parameters FRACTION, BADCOL and BADROW.  \texttt{[FALSE]}
      }
      \sstsubsection{
         BADROW = \_INTEGER (Read)
      }{
         The number of bad rows to include.  Only accessed if
         Parameter BADPIX is \texttt{TRUE}.  The bad rows are distributed
         at random using a uniform distribution.  \texttt{[0]}
      }
      \sstsubsection{
         DIRN = \_INTEGER (Read)
      }{
         Direction of the ramp. 1 means left to right, 2 is right to
         left, 3 is bottom to top, and 4 is top to bottom. (RA mode)
      }
      \sstsubsection{
         DISTRIB = LITERAL (Read)
      }{
         Radial distribution of the Gaussians to be used (GS mode).
         Alternatives weightings are:

         \ssthitemlist{

            \sstitem
            \texttt{"FIX"} --- fixed distance, and

            \sstitem
            \texttt{"RSQ"} --- one over radius squared.

         }
         \texttt{["FIX"]}
      }
      \sstsubsection{
         FRACTION = \_REAL (Read)
      }{
         Fraction of bad pixels to be included.  Only accessed if BADPIX
         is \texttt{TRUE}.  \texttt{[0.01]}
      }
      \sstsubsection{
         HIGH = \_REAL (Read)
      }{
         High value used in the generated data array (RA and RL modes).
      }
      \sstsubsection{
         LBOUND( 2 ) = \_INTEGER (Read)
      }{
         Lower pixel bounds of the output NDF.  Only accessed if Parameter
         LIKE is set to null (\texttt{{!}}).
      }
      \sstsubsection{
         LIKE = NDF (Read)
      }{
         An optional template NDF which, if specified, will be used to
         define the bounds for the output NDF.  If a null value (\texttt{{!}}) is
         given the bounds are obtained via Parameters LBOUND and
         UBOUND.  \texttt{[!]}
      }
      \sstsubsection{
         LOGFILE = LITERAL (Read)
      }{
         Name of a log file in which to store details of the Gaussians
         added to the output NDF (GS mode).  If a null value is supplied
         no log file is created.  \texttt{[!]}
      }
      \sstsubsection{
         LOW  = \_REAL (Read)
      }{
         Low value used in the generated data array (RA and RL modes).
      }
      \sstsubsection{
         MAX = \_REAL (Read)
      }{
         Peak Gaussian intensity to be used in the generated data
         array (GS mode).
      }
      \sstsubsection{
         MEAN = \_REAL (Read)
      }{
         Mean value used in the generated data array (FL, RP and GN modes).
      }
      \sstsubsection{
         MIN = \_REAL (Read)
      }{
         Lowest Gaussian intensity to be used in the generated data
         array (GS mode).
      }
      \sstsubsection{
         MODE = \htmlref{LITERAL}{se:parmenu} (Read)
      }{
         The form of the data to be generated.  The options are as follows.

         \ssthitemlist{

            \sstitem
            \texttt{"RR"} --- Uniform noise between 0 and 1.

            \sstitem
            \texttt{"RL"} --- Uniform noise between specified limits.

            \sstitem
            \texttt{"BL"} --- A constant value of zero.

            \sstitem
            \texttt{"FL"} --- A specified constant value.

            \sstitem
            \texttt{"RP"} --- Poisson noise about a specified mean.

            \sstitem
            \texttt{"GN"} --- Gaussian noise about a specified mean.

            \sstitem
            \texttt{"RA"} --- Ramped between specified minimum and maximum values and a
                  choice of four directions.

            \sstitem
            \texttt{"GS"} --- A random distribution of two-dimensional Gaussians of defined
                  FWHM and range of maximum peak values on a specified
                  background, with Poissonian noise.  There is a choice of
                  spatial distributions for the Gaussians: fixed, or inverse
                  square radially from the array centre. (In essence it is
                  equivalent to a simulated star field.)  The \textit{x}-\textit{y}
                  position and peak value of each Gaussian may be stored in a log file,
                  a positions list catalogue, or reported on the screen.
                  Bad pixels may be included randomly, and/or in a column
                  or line of the array.
         }
      }
      \sstsubsection{
         NGAUSS  = \_INTEGER (Read)
      }{
         Number of Gaussian star-like images to be generated (GS mode).
      }
      \sstsubsection{
         OUT = NDF (Write)
      }{
         The output NDF.
      }
      \sstsubsection{
         OUTCAT = FILENAME (Write)
      }{
         An output catalogue in which to store the pixel co-ordinates of
         the Gausians in the output NDF (GS mode).  If a null value is
         supplied, no output positions list is produced.  \texttt{[!]}
      }
      \sstsubsection{
         SEEING = \_REAL (Read)
      }{
         Seeing (FWHM) in pixels (not the same as the standard deviation)
         (GS mode).
      }
      \sstsubsection{
         SIGMA = \_REAL (Read)
      }{
         Standard deviation of noise to be used in the generated data
         array (GN mode).
      }
      \sstsubsection{
         TITLE = LITERAL (Read)
      }{
         Title for the output NDF.  \texttt{["KAPPA - Creframe"]}
      }
      \sstsubsection{
         UBOUND( 2 ) = \_INTEGER (Read)
      }{
         Upper pixel bounds of the output NDF.  Only accessed if Parameter
         LIKE is set to null (\texttt{{!}}).
      }
      \sstsubsection{
         VARIANCE = \_LOGICAL (Read)
      }{
         If \texttt{TRUE}, a VARIANCE component is added to the output NDF
         representing the noise added to the field.  If a null (\texttt{{!}}) value is
         supplied, a default is used which is \texttt{TRUE} for modes which include
         noise, and \texttt{FALSE} for modes which do not include any noise.  \texttt{[!]}
      }
   }
   \sstexamples{
      \sstexamplesubsection{
         creframe out=file ubound=[128,128] mode=gs ngauss=5 badpix badcol=2 max=200 min=20 background=20 seeing=1.5
      }{
         Produces a 128$\times$128 pixel data array with 5 gaussians with peak
         values of 200 counts and a background of 20 counts.  There will
         be two bad columns added to the resulting data.
      }
   }
   \sstnotes{
      \sstitemlist{

         \sstitem
         The Gaussian parameters (GS mode) are not displayed when the
         message filter environment variable MSG\_FILTER is set to \texttt{QUIET}.

      }
   }
   \sstimplementationstatus{
      \sstitemlist{

         \sstitem
         The DATA and \htmlref{VARIANCE components}{apndf:variance} of the output NDF have a numerical
         type of \texttt{"\_REAL"} (single-precision floating point).

         \sstitem
         This routine does not assign values to any of the following
         components in the output NDF: \htmlref{LABEL}{apndf:label}, \htmlref{UNITS}{apndf:units}, \htmlref{QUALITY}{apndf:quality}, \htmlref{AXIS}{apndf:axis}, \htmlref{WCS}{apndf:wcs}.
      }
   }
}
\sstroutine{
   CSUB
}{
   Subtracts a scalar from an NDF data structure
}{
   \sstdescription{
      The routine subtracts a scalar (\emph{i.e.} constant) value from each
      pixel of an \NDFref{NDF's}  data array to produce a new NDF data structure.
   }
   \sstusage{
      csub in scalar out
   }
   \sstparameters{
      \sstsubsection{
         IN = NDF (Read)
      }{
         Input NDF data structure, from which the value is to be
         subtracted.
      }
      \sstsubsection{
         OUT = NDF (Write)
      }{
         Output NDF data structure.
      }
      \sstsubsection{
         SCALAR = \_DOUBLE (Read)
      }{
         The value to be subtracted from the NDF's data array.
      }
      \sstsubsection{
         TITLE = LITERAL (Read)
      }{
         The title for the output NDF.  A null value will cause
         the title of the NDF supplied for Parameter IN to be used
         instead.  \texttt{[!]}
      }
   }
   \sstexamples{
      \sstexamplesubsection{
         csub a 10 b
      }{
         This subtracts ten from the NDF called a, to make the NDF
         called b.  NDF b inherits its title from a.
      }
      \sstexamplesubsection{
         csub title="HD123456" out=b in=a scalar=21.9
      }{
         This subtracts 21.9 from the NDF called a, to make the NDF
         called b.  NDF b has the title \texttt{"HD123456"}.
      }
   }
   \sstdiytopic{
      Related Applications
   }{
KAPPA: \htmlref{ADD}{ADD},
\htmlref{CADD}{CADD},
\htmlref{CDIV}{CDIV},
\htmlref{CMULT}{CMULT},
\htmlref{DIV}{DIV},
\htmlref{MATHS}{MATHS},
\htmlref{MULT}{MULT},
\htmlref{SUB}{SUB}.
   }
   \sstimplementationstatus{

      \sstitemlist{

         \sstitem
         This routine correctly processes the \htmlref{AXIS}{apndf:axis}, DATA, \htmlref{QUALITY}{apndf:quality},
         \htmlref{LABEL}{apndf:label}, \htmlref{TITLE}{apndf:title}, \htmlref{UNITS}{apndf:units}, \htmlref{HISTORY}{apndf:history}, \htmlref{WCS}{apndf:wcs}, and \htmlref{VARIANCE}{apndf:variance}~ components of an NDF
         data structure and propagates all \htmlref{extensions}{apndf:extensions}.

         \sstitem
         Processing of \htmlref{bad pixels}{se:masking} and automatic \htmlref{quality masking}{se:qualitymask} are supported.

         \sstitem
         All \htmlref{non-complex numeric data types}{ap:HDStypes} can be handled.

         \sstitem
         Huge NDFs are supported.
      }
   }
}
\sstroutine{
   CUMULVEC
}{
   Sums the values cumulatively in a one-dimensional NDF
}{
   \sstdescription{
      This application forms the cumulative sum of the values of a
      one-dimensional NDF starting from the first to the last element.
      thus the first output pixel will be unchanged but the second will
      be the sum of the first two input pixels, third output pixel is the
      sum of the first three input pixels and so on.  Anomalous values
      may be excluded from the summation by setting a threshold.
   }
   \sstusage{
      cumulvec in out [thresh]
   }
   \sstparameters{
      \sstsubsection{
         IN = NDF (Read)
      }{
         The one-dimensional NDF containing the vector to be summed.
      }
      \sstsubsection{
         OUT = NDF (Write)
      }{
         The NDF to contain the summed image.
      }
      \sstsubsection{
         THRESH = \_DOUBLE (Read)
      }{
         The maximum difference between adjacent elements for the
         summation to ocur.  For increments outside the allowed
         range, the increment becomes zero.  If null, \texttt{{!}}, is
         given, then there is no limit.  \texttt{[!]}
      }
      \sstsubsection{
         TITLE = LITERAL (Read)
      }{
         The title of the output NDF.  A null (\texttt{{!}}) value means using the
         title of the input NDF.  \texttt{[!]}
      }
   }
   \sstexamples{
      \sstexamplesubsection{
         cumulvec gradient profile
      }{
         The one-dimensional NDF called gradient is summed cumulatively to
         form NDF profile.
      }
      \sstexamplesubsection{
         cumulvec in=gradient out=profile thresh=20
      }{
         As above but only adjacent values separated by less than 20
         are included in the summation.
      }
   }
   \sstdiytopic{
      Related Applications
   }{
KAPPA: \htmlref{HISTOGRAM}{HISTOGRAM}.
   }
   \sstimplementationstatus{
      \ssthitemlist{

         \sstitem
         This routine correctly processes the \htmlref{AXIS}{apndf:axis}, DATA,
         \htmlref{QUALITY}{apndf:quality}, \htmlref{VARIANCE}{apndf:variance},
         \htmlref{LABEL}{apndf:label}, \htmlref{TITLE}{apndf:title}, \htmlref{UNITS}{apndf:units},
         \htmlref{WCS}{apndf:wcs}, and \htmlref{HISTORY}{apndf:history} components of an NDF
         data structure and propagates all \htmlref{extensions}{apndf:extensions}.

         \sstitem
         Processing of \htmlref{bad pixels}{se:masking} and automatic
         \htmlref{quality masking}{se:qualitymask} are supported.  Bad pixels
         are propagated and excluded from the summation.

         \sstitem
         All \htmlref{non-complex numeric data types}{ap:HDStypes} can be
         handled.  Arithmetic is performed using single- or double-precision
         floating point as appropriate.
      }
   }
}
\sstroutine{
   CURSOR
}{
   Reports the co-ordinates of positions selected using the cursor
}{
   \sstdescription{
      This application reads co-ordinates from the chosen graphics device
      using a cursor and displays them on your terminal.  The selected
      positions may be marked in various ways on the device (see Parameter
      PLOT), and can be written to an output positions list so that subsequent
      applications can make use of them (see Parameter OUTCAT).  The format
      of the displayed positions may be controlled using Parameter STYLE.
      The pixel data value in any associated NDF can also be displayed
      (see Parameter SHOWDATA).

      Positions may be reported in several different
      \htmlref{co-ordinate Frames}{se:domains}~ (see Parameter FRAME).  Optionally,
      the corresponding pixel co-ordinates at each position may also be reported
      (see Parameter SHOWPIXEL).

      The picture or pictures within which positions are required can be
      selected in several ways (see Parameters MODE and NAME).

      Restrictions can be made on the number of positions to be given (see
      Parameters MAXPOS and MINPOS), and screen output can be suppressed
      (see the \htmlref{``Notes''}{notes:cursor}).
   }
   \sstusage{
      cursor [mode] [name] [outcat] [device]
   }
   \sstparameters{
      \sstsubsection{
         CATFRAME = LITERAL (Read)
      }{
         A string determining the co-ordinate Frame in which positions are
         to be stored in the output catalogue associated with Parameter
         OUTCAT.  The string supplied for CATFRAME can be one of the
         following:

         \ssthitemlist{

            \sstitem
            A \htmlref{domain name}{se:domains}~ such as \htmlref{SKY, AXIS, PIXEL}{se:resdoms}.

            \sstitem
            An integer value giving the index of the required Frame.

            \sstitem
            An IRAS90 \emph{Sky Co-ordinate System} (SCS) values such as
            \texttt{"EQUAT(J2000)"} (see \xref{SUN/163}{sun163}{}).

         }
         If a null (\texttt{{!}}) value is supplied, the positions will be stored
         in the current Frame. \texttt{[!]}
      }
      \sstsubsection{
         CATEPOCH = \_DOUBLE (Read)
      }{
         The epoch at which the sky positions stored in the output
         catalogue were determined.  It will only be accessed if an epoch
         value is needed to qualify the co-ordinate Frame specified by
         COLFRAME.  If required, it should be given as a decimal years
         value, with or without decimal places (\texttt{"1996.8"} for example).
         Such values are interpreted as a Besselian epoch if less than
         1984.0 and as a Julian epoch otherwise.
      }
      \sstsubsection{
         CLOSE = \_LOGICAL (Read)
      }{
         This parameter is only accessed if Parameter PLOT is set to
         \texttt{"Chain"} or \texttt{"Poly"}.  If \texttt{TRUE}, polygons will be closed by joining
         the first position to the last position.  \texttt{[}current value\texttt{{]}}
      }

      \sstsubsection{
         COMP = LITERAL (Read)
      }{
         The NDF component to be displayed.  It may be \texttt{"Data"},
         \texttt{"Quality"}, \texttt{"Variance"}, or \texttt{"Error"} (where \texttt{"Error"}
         is an alternative to \texttt{"Variance"} and causes the square root of the
         variance values to be displayed).  If
         \texttt{"Quality"} is specified, then the quality values are treated as
         numerical values (in the range 0 to 255).  \texttt{["Data"]}
      }
      \sstsubsection{
         DESCRIBE = \_LOGICAL (Read)
      }{
         If \texttt{TRUE}, a detailed description of the co-ordinate Frame in which
         subsequent positions will be reported is produced each time a
         position is reported within a new picture.  \texttt{[}current value\texttt{{]}}
      }
      \sstsubsection{
         DEVICE = \htmlref{DEVICE}{se:selgradev} (Read)
      }{
         The graphics workstation.  This device must support cursor
         interaction.  \texttt{[}current graphics device\texttt{{]}}
      }
      \sstsubsection{
         EPOCH = \_DOUBLE (Read)
      }{
         If a `Sky Co-ordinate System' specification is supplied (using
         Parameter FRAME) for a celestial co-ordinate system, then an
         epoch value is needed to qualify it.  This is the epoch at
         which the supplied sky positions were determined.  It should be
         given as a decimal years value, with or without decimal places
         (\texttt{"1996.8"} for example).  Such values are interpreted as a Besselian
         epoch if less than 1984.0 and as a Julian epoch otherwise.
      }
      \sstsubsection{
         FRAME = LITERAL (Read)
      }{
         A string determining the co-ordinate Frame in which positions are
         to be reported.  When a data array is displayed by an application
         such as DISPLAY, CONTOUR the \htmlref{WCS}{apndf:wcs} information describing the co-ordinate
         systems known to the data array are stored with the DATA picture
         in the \htmlref{graphics database}{se:agitate}.  This application can report positions in
         any of the co-ordinate Frames stored with each picture.  The
         string supplied for FRAME can be one of the following:

         \ssthitemlist{

            \sstitem
            A \htmlref{domain name}{se:domains}~ such as \htmlref{SKY, AXIS, PIXEL}{se:resdoms}.  The special domains
            AGI\_WORLD and AGI\_DATA are used to refer to the world and data
            co-ordinate system stored in the AGI graphics database.  They can
            be useful if no WCS information was store with the picture when
            it was created.

            \sstitem
            An integer value giving the index of the required Frame.

            \sstitem
            An IRAS90 \emph{Sky Co-ordinate System} (SCS) values such as
            \texttt{"EQUAT(J2000)"} (see \xref{SUN/163}{sun163}{}).

         }
         If a null value (\texttt{{!}}) is supplied, positions are reported in the
         co-ordinate Frame which was current when the picture was created.
         \texttt{[!]}
      }
      \sstsubsection{
         GEODESIC = \_LOGICAL (Read)
      }{
         This parameter is only accessed if Parameter PLOT is set to
         \texttt{"Chain"} or \texttt{"Poly"}.  It specifies whether the curves drawn between
         positions should be straight lines, or should be geodesic curves.
         In many co-ordinate Frames geodesic curves will be simple straight
         lines.  However, in others (such as the majority of celestial
         co-ordinates Frames) geodesic curves will be more complex curves
         tracing the shortest path between two positions in a non-linear
         projection.  \texttt{[FALSE]}
      }
      \sstsubsection{
         INFO = \_LOGICAL (Read)
      }{
         If \texttt{TRUE}, then messages are displayed describing the use of the
         mouse prior to obtaining the first position.  Note, these
         informational messages are not suppressed by setting MSG\_FILTER
        environment variable to \texttt{QUIET}.  \texttt{[TRUE]}
      }
      \sstsubsection{
         JUST = LITERAL (Read)
      }{
         A string specifying the justification to be used when displaying
         text strings at the supplied cursor positions.  This parameter is
         only accessed if Parameter PLOT is set to \texttt{"Text"}.  The supplied
         string should contain two characters; the first should be \texttt{"B"},
         \texttt{"C"} or \texttt{"T"}, meaning bottom, centre or top.  The second should be
         \texttt{"L"}, \texttt{"C"} or \texttt{"R"}, meaning left, centre or right.  The text is
         displayed so that the supplied position is at the specified
         point within the displayed text string.  \texttt{["CC"]}
      }
      \sstsubsection{
         LOGFILE = FILENAME (Write)
      }{
         The name of the text file in which the formatted co-ordinates of
         positions selected with the cursor may be stored.  This is intended
         primarily for recording the screen output, and not for communicating
         positions to subsequent applications (use Parameter OUTCAT for this
         purpose).  A null string (\texttt{{!}}) means that no file is created.  \texttt{[!]}
      }
      \sstsubsection{
         MARKER = \_INTEGER (Read)
      }{
         This parameter is only accessed if Parameter PLOT is set to
         \texttt{"Chain"} or \texttt{"Mark"}.  It specifies the symbol with which each
         position should be marked, and should be given as an integer
         \PGPLOT\  marker type.  For instance, \texttt{0} gives a box,
         \texttt{1} gives a dot, \texttt{2} gives a cross, \texttt{3} gives an asterisk,
         \texttt{7} gives a triangle.  The value must be larger than or equal to
         $-$31.  \texttt{[}current value\texttt{{]}}
      }
      \sstsubsection{
         MAXPOS = \_INTEGER (Read)
      }{
         The maximum number of positions which may be supplied before the
         application terminates.  The number must be in the range 1 to 200.
         \texttt{[200]}
      }
      \sstsubsection{
         MINPOS = \_INTEGER (Read)
      }{
         The minimum number of positions which may be supplied.  The user
         is asked to supply more if necessary.  The number must be in the
         range 0 to the value of Parameter MAXPOS.  \texttt{[0]}
      }
      \sstsubsection{
         MODE = \htmlref{LITERAL}{se:parmenu} (Read)
      }{
         The method used to select the pictures in which cursor positions are
         to be reported.  There are three options.

         \ssthitemlist{

            \sstitem
            \texttt{"Current"} --- reports positions within the current picture in the
            AGI database.  If a position does not lie within the current picture,
            an extrapolated position is reported, if possible.

            \sstitem
            \texttt{"Dynamic"} --- reports positions within the top-most picture
            under the cursor in the AGI  database.  Thus the second and
            subsequent cursor hits may result in the selection of a new picture.

            \sstitem
            \texttt{"Anchor"} --- lets the first cursor hit select the picture in
            which all positions are to be reported.  If a subsequent cursor hit
            falls outside this picture, an extrapolated position is reported if
            possible.

         }
         \texttt{["Dynamic"]}
      }
      \sstsubsection{
         NAME = LITERAL (Read)
      }{
         Only pictures of this name are to be selected.  For instance, if
         you want positions in a DATA picture which is covered by a
         transparent FRAME picture, then you could specify NAME=\texttt{"DATA"}.
         A null (\texttt{{!}}) or blank string means that pictures of all names may
         be selected.  NAME is ignored when MODE=\texttt{"Current"}.  \texttt{[!]}
      }
      \sstsubsection{
         OUTCAT = FILENAME (Write)
      }{
         An output catalogue in which to store the valid selected positions.
         The catalogue has the form of a positions list such as created by
         application LISTMAKE.  Only positions in the first selected picture
         are recorded.  This application uses the conventions of the
         \CURSAref\ package for determining the format of the catalogue.  If a
         file type of \texttt{.fit} is given, then the catalogue is stored as a FITS
         binary table.  If a file type of \texttt{.txt} is given, then the catalogue
         is stored in a text file in \texttt{"Small Text List"} (STL) format.  If no
         file type is given, then \texttt{.fit} is assumed.  If a null value is
         supplied, no output positions list is produced.  See also
         Parameter CATFRAME.  \texttt{[!]}
      }
      \sstsubsection{
         PLOT = LITERAL (Read)
      }{
         The type of graphics to be used to mark the selected positions
         which have valid co-ordinates.  The appearance of these graphics
         (colour, size, \emph{etc.}) is controlled by the STYLE parameter.  PLOT
         can take any of the following values:

         \ssthitemlist{

            \sstitem
            \texttt{"None"} --- No graphics are produced.

            \sstitem
            \texttt{"Mark"} --- Each position is marked by the symbol specified
            by Parameter MARKER.

            \sstitem
            \texttt{"Poly"} --- Causes each position to be joined by a line to the
            previous position.  These lines may be simple straight lines or
            geodesic curves (see Parameter GEODESIC).  The polygons may
            optionally be closed by joining the last position to the first (see
            Parameter CLOSE).

            \sstitem
            \texttt{"Chain"} --- This is a combination of \texttt{"Mark"} and \texttt{"Poly"}.
            Each position is marked by a symbol and joined by a line to the previous
            position.  Parameters MARKER, GEODESIC and CLOSE are used to
            specify the symbols and lines to use.

            \sstitem
            \texttt{"Box"} --- A rectangular box with edges parallel to the edges of
            the graphics device is drawn with the specified position at one
            corner, and the previously specified position at the diagonally
            opposite corner.

            \sstitem
            \texttt{"Vline"} --- A vertial line is drawn through each specified
            position, extending the entire height of the selected picture.

            \sstitem
            \texttt{"Hline"} --- A horizontal line is drawn through each specified
            position, extending the entire width of the selected picture.

            \sstitem
            \texttt{"Cross"} --- A combination of \texttt{"Vline"} and \texttt{"Hline"}.

            \sstitem
            \texttt{"Text"} --- A text string is used to mark each position.  The string
            is drawn horizontally with the justification specified by Parameter
            JUST.  The strings to use for each position are specified using
            Parameter STRINGS.

         }
         \texttt{[}current value\texttt{{]}}
      }
      \sstsubsection{
         SHOWDATA = \_LOGICAL (Read)
      }{
         If \texttt{TRUE}, the pixel value within the displayed NDF is
         reported for each selected position.  This is only possible if
         the picture within which position are being selected contains a
         reference to an existing NDF.  The NDF array component to be
         displayed is selected via Parameter COMP.  \texttt{[FALSE]}
      }
      \sstsubsection{
         SHOWPIXEL = \_LOGICAL (Read)
      }{
         If \texttt{TRUE}, the pixel co-ordinates of each selected position are
         shown on a separate line, following the co-ordinates requested
         using Parameter FRAME.  If pixel co-ordinates are being displayed
         anyway (see Parameter FRAME) then a value of \texttt{FALSE} is used for.
         SHOWPIXEL.  \texttt{[}current value\texttt{{]}}
      }
      \sstsubsection{
         STRINGS = LITERAL (Read)
      }{
         A group of text strings which are used to mark the supplied positions
         if Parameter PLOT is set to \texttt{"TEXT"}.  The first string in the
         group is used to mark the first position, the second string is
         used to mark the second position, \emph{etc}.  If more positions are
         given than there are strings in the group, then the extra
         positions will be marked with an integer value indicating the
         index within the list of supplied positions.  If a null value (\texttt{{!}})
         is given for the parameter, then all positions will be marked
         with integer indices, starting at 1.

         A comma-separated list should be given in which each element is
         either a marker string, or the name of a text file preceded by an
         up-arrow character \texttt{"$\wedge$"}.  Such text files should contain further
         comma-separated lists which will be read and interpreted in the
         same manner.  Note, strings within text files can be separated by
         new lines as well as commas.
      }
      \sstsubsection{
         STYLE = \htmlref{GROUP}{se:groups} (Read)
      }{
         A group of attribute settings describing the plotting style to use
         when drawing the graphics specified by Parameter PLOT.  The format
         of the positions reported on the screen may also be controlled.

         A comma-separated list of strings should be given in which each
         string is either an attribute setting, or the name of a text
         file preceded by an up-arrow character \texttt{"$\wedge$"}.  Such text files
         should contain further comma-separated lists which will be
         read and interpreted in the same manner.  Attribute settings
         are applied in the order in which they occur within the list,
         with later settings overriding any earlier settings given for
         the same attribute.

         Each individual attribute setting should be of the form:

            $<$name$>$=$<$value$>$


         where $<$name$>$ is the name of a plotting attribute, and $<$value$>$
         is the value to assign to the attribute.  Default values will be
         used for any unspecified attributes.  All attributes will be
         defaulted if a null value (\texttt{{!}})---the initial default---is supplied.
         To apply changes of style to only the current invocation, begin these
         attributes with a plus sign.  A mixture of persistent and temporary
         style changes is achieved by listing all the persistent attributes
         followed by a plus sign then the list of temporary attributes.

         See \slhyperref{Plotting Attributes}{Section~}{}{ap:plotting_attr}
         for a description of the available attributes.  Any unrecognised
         attributes are ignored (no error is reported).

         In addition to the attributes which control the appearance of
         the graphics (\htmlattref{Colour}{Colour(element)}, \htmlattref{Font}{Font(element)}, \emph{etc.}), the following attributes may
         be set in order to control the appearance of the formatted axis
         values reported on the screen: \htmlattref{Format}{Format(axis)},
         \htmlattref{Digits}{Digits/Digits(axis)}, \htmlattref{Symbol}{Symbol(axis)},
         \htmlattref{Unit}{Unit(axis)}.  These
         may be suffixed with an axis number (\emph{e.g.} \texttt{"Digits(2)"}) to refer to
         the values displayed for a specific axis.  \texttt{[}current value\texttt{{]}}
      }
   }
   \sstresparameters{
      \sstsubsection{
         LASTDIM = \_INTEGER (Write)
      }{
         The number of axis values written to Parameter LASTPOS.
      }
      \sstsubsection{
         LASTPOS() = \_DOUBLE (Write)
      }{
         The unformatted co-ordinates of the last valid position selected
         with the cursor, in the co-ordinate Frame which was used to
         report the position.  The number of axis values is written to output
         Parameter LASTDIM.
      }
      \sstsubsection{
         NUMBER = \_INTEGER (Write)
      }{
         The number of positions selected with the cursor (excluding
         invalid positions).
      }
   }
   \sstexamples{
      \sstexamplesubsection{
         cursor frame=pixel
      }{
         This obtains co-ordinates within any visible picture for the
         \htmlref{current graphics device}{se:devglobal} by use of the cursor.  Positions are
         reported in pixel co-ordinates if available, and in the current
         co-ordinate Frame of the picture otherwise.
      }
      \sstexamplesubsection{
         cursor frame=pixel outcat=a catframe=gal
      }{
         Like the previous example, except that, in addition to being
         displayed on the screen, the positions are transformed into
         galactic co-ordinates and stored in FITS binary table called
         \texttt{a.FIT}, together with any associated WCS information.
      }
      \sstexamplesubsection{
         cursor frame=equat(J2010)
      }{
         This obtains co-ordinates within any visible picture for the
         current graphics device by use of the cursor.  Positions are
         reported in equatorial RA/DEC co-ordinates (referenced to the
         J2010 equinox) if available, and in the current co-ordinate Frame
         of the picture otherwise.
      }
      \sstexamplesubsection{
         cursor describe plot=mark marker=3 style="colour=red,size=2"
      }{
         As above except, positions are always reported in the current
         co-ordinate Frame of each picture.  The details of these co-ordinate
         Frames are described as they are used.  Each selected point is
         marked with \PGPLOT\  marker 3 (an asterisk).  The markers are
         red and are twice the default size.
      }
      \sstexamplesubsection{
         cursor current maxpos=2 minpos=2 plot=poly outcat=slice
      }{
         Exactly two positions are obtained within the current picture,
         and are joined with a straight line.  The positions are written to
         a FITS binary catalogue called \texttt{slice.FIT}.
         The catalogue may be used to communicate the positions
         to later applications (LISTSHOW, PROFILE, \emph{etc.}).
      }
      \sstexamplesubsection{
         cursor name=data style="$\wedge$mystyle,digits(1)=5,digits(2)=7"
      }{
         This obtains co-ordinates within any visible DATA picture on
         the current graphics device.  The style to use is read from
         text file mystyle, but is then modified so that five digits are used
         to format axis-1 values, and seven to format axis-2 values.
      }
      \sstexamplesubsection{
         cursor plot=box style="width=3,colour=red" maxpos=2 minpos=2
      }{
         Exactly two positions must be given using the cursor, and a red box
         is drawn joining the two positions.  The lines making up the box
         are three times the default width.
      }
      \sstexamplesubsection{
         cursor plot=text style="size=2,textbackcolour=clear"
      }{
         Positions are marked using integer values, starting at 1 for the
         first position.  The text drawn is twice as large as normal, and
         the background is not cleared before drawing the text.
      }
   }
   \label{notes:cursor}
   \sstnotes{
      \sstitemlist{

         \sstitem
         The unformatted values stored in the output Parameter LASTPOS,
         may not be in the same units as the formatted values shown on
         the screen and logged to the log file.  For instance, unformatted
         celestial co-ordinate values are stored in radians.

         \sstitem
         The current picture is unchanged by this application.

         \sstitem
         In DYNAMIC and ANCHOR modes, if the cursor is situated at a
         position where there are no pictures of the selected name, the
         co-ordinates in the BASE picture are reported.

         \sstitem
         Pixel co-ordinates are formatted with 1 decimal place unless a
         format has already been specified by setting the Format attributes
         for the axes of the PIXEL co-ordinate Frame (\emph{e.g.} using application
         \htmlref{WCSATTRIB}{WCSATTRIB}).

         \sstitem
         Positions can be removed (the instructions state how), starting
         from the most-recent one.  Such positions are excluded from the
         output positions list and log file (if applicable).  If graphics
         are being used to mark the positions, then removed positions will
         be highlighted by drawing a marker of type 8 (a circle containing a
         cross) over the removed positions in a different colour.

         \sstitem
         The positions are not displayed on the screen when the
         message filter environment variable MSG\_FILTER is set to \texttt{QUIET}.
         The creation of output parameters and files is unaffected by MSG\_FILTER.
         The display of informational messages describing the use of the cursor
         is controlled by the Parameter INFO.

      }
   }
   \sstdiytopic{
      Related Applications
   }{
KAPPA: \htmlref{LISTSHOW}{LISTSHOW},
\htmlref{LISTMAKE}{LISTMAKE},
\htmlref{PICCUR}{PICCUR};
\xref{FIGARO}{sun86}{}: \xref{ICUR}{sun86}{ICUR},
\xref{IGCUR}{sun86}{IGCUR}.
   }
}
\sstroutine{
   DISPLAY
}{
   Displays a one- or two-dimensional NDF
}{
   \sstdescription{
      This application displays a one- or two-dimensional \NDFref{NDF} as an image
      on the \htmlref{current graphics device}{se:devglobal}.  The minimum and maximum data
      values to be displayed can be selected in several ways (see
      Parameter MODE).  Data values outside these limits are displayed
      with the colour of the nearest limit.  A key showing the
      relationship between colour and data value can be displayed (see
      Parameter KEY).

      Annotated axes or a simple border can be drawn around the image (see
      Parameters AXES and BORDER).  The appearance of these may be controlled
      in detail (see Parameters STYLE and BORSTYLE).

      A specified colour lookup table may optionally be loaded prior to
      displaying the image (see Parameter LUT).  For devices which reset
      the colour table when opened (such as PostScript files), this may
      be the only way of controlling the colour table.

      The image is produced within the \htmlref{current graphics database picture.}{se:agitate}
      The co-ordinates at the centre of the image, and the scale of the
      image can be controlled using Parameters CENTRE, XMAGN and YMAGN.  Only
      the parts of the image that lie within the current picture are visible;
      the rest is clipped.  The image is padded with bad pixels if necessary.
   }
   \sstusage{
      display in [comp] clear [device] mode [centre] [xmagn] [ymagn] [out]
        \newline\hspace*{1.5em}
        $\left\{ {\begin{tabular}{l}
                  low=? high=? \\
                  percentiles=? \\
                  sigmas=?
                  \end{tabular} }
        \right.$
        \newline\hspace*{1.9em}
        \makebox[0mm][c]{\small mode}
   }
   \sstparameters{
      \sstsubsection{
         AXES = \_LOGICAL (Read)
      }{
         \texttt{TRUE} if labelled and annotated axes are to be drawn around the
         image.  These display co-ordinates in the current \htmlref{co-ordinate Frame}{se:domains}~
         of the supplied NDF, and may be changed using application \htmlref{WCSFRAME}{WCSFRAME}
         (see also Parameter USEAXIS).  The width of the margins left for
         the annotation may be controlled using Parameter MARGIN.  The
         appearance of the axes (colours, founts, \emph{etc.}) can be controlled
         using the STYLE Parameter.  \texttt{{[}}current value\texttt{{]}}
      }
      \sstsubsection{
         BADCOL = LITERAL (Read)
      }{
         The colour with which to mark any bad (\emph{i.e.} missing) pixels in the
         display.  There are a number of options described below.

         \ssthitemlist{

            \sstitem
            \texttt{"MAX"} --- The maximum \htmlref{colour index}{se:coltab} used for the display of the image.

            \sstitem
            \texttt{"MIN"} --- The minimum colour index used for the display of the image.

            \sstitem
            An integer --- The actual colour index.  It is constrained between
            0 and the maximum colour index available on the device.

            \sstitem
            A named colour --- Uses the \htmlref{named colour}{ap:colset} from the \htmlref{palette}{se:palette}, and if it
            is not present, the nearest colour from the palette is selected.

            \sstitem
            An \htmlref{HTML colour code}{htmlcolour} such as \texttt{\#ff002d}.

         }
         If the colour is to remain unaltered as the lookup table is
         manipulated choose an integer between 0 and 15, or a named
         colour.  The suggested default is the current value.  \texttt{{[}}current value\texttt{{]}}
      }
      \sstsubsection{
         BORDER = \_LOGICAL (Read)
      }{
         \texttt{TRUE} if a border is to be drawn around the regions of the
         displayed image containing valid co-ordinates in the current
         co-ordinate Frame of the NDF.  For instance, if the NDF contains
         an Aitoff all-sky map, then an elliptical border will be drawn
         if the current co-ordinate Frame is galactic longitude and
         latitude.  This is because pixels outside this ellipse have
         undefined positions in galactic co-ordinates.  If, instead, the
         current co-ordinate Frame had been pixel co-ordinates, then a
         simple box would have been drawn containing the whole image.
         This is because every pixel has a defined position in pixel
         co-ordinates.  The appearance of the border (colour, width, \emph{etc.})
         can be controlled using Parameter BORSTYLE.  \texttt{{[}}current value\texttt{{]}}
      }
      \sstsubsection{
         BORSTYLE = \htmlref{GROUP}{se:groups} (Read)
      }{
         A group of attribute settings describing the plotting style to use
         for the border (see Parameter BORDER).

         A comma-separated list of strings should be given in which each
         string is either an attribute setting, or the name of a text
         file preceded by an up-arrow character \texttt{"$\wedge$"}.  Such text files
         should contain further comma-separated lists which will be
         read and interpreted in the same manner.  Attribute settings
         are applied in the order in which they occur within the list,
         with later settings overriding any earlier settings given for
         the same attribute.

         Each individual attribute setting should be of the form:

            $<$name$>$=$<$value$>$


         where $<$name$>$ is the name of a plotting attribute, and $<$value$>$
         is the value to assign to the attribute.  Default values will be
         used for any unspecified attributes.  All attributes will be
         defaulted if a null value (\texttt{{!}})---the initial default---is supplied.
         To apply changes of style to only the current invocation, begin these
         attributes with a plus sign.  A mixture of persistent and temporary
         style changes is achieved by listing all the persistent attributes
         followed by a plus sign then the list of temporary attributes.

         See \slhyperref{Plotting Attributes}{Section~}{}{ap:plotting_attr}
         for a description of the available attributes.  Any unrecognised
         attributes are ignored (no error is reported).
         \texttt{[}current value\texttt{{]}}
      }
      \sstsubsection{
         CENTRE = LITERAL (Read)
      }{
         The co-ordinates of the data pixel to be placed at the centre of
         the image, in the current co-ordinate Frame of the NDF (supplying
         a colon \texttt{":"} will display details of the current co-ordinate Frame).
         The position should be supplied as a list of
         \xref{formatted axis values}{sun210}{AST_UNFORMAT}
         separated by spaces or commas.  See also Parameter USEAXIS.  A
         null (\texttt{{!}}) value causes the centre of the image to be used.  \texttt{[!]}
      }
      \sstsubsection{
         CLEAR = \_LOGICAL (Read)
      }{
         \texttt{TRUE} if the current picture is to be cleared before the image is
         displayed.  \texttt{[}current value\texttt{{]}}
      }
      \sstsubsection{
         COMP = \htmlref{LITERAL}{se:parmenu} (Read)
      }{
         The NDF array component to be displayed.  It may be \texttt{"Data"},
         \texttt{"Quality"}, \texttt{"Variance"}, or \texttt{"Error"} (where \texttt{"Error"} is an
         alternative to \texttt{"Variance"} and causes the square root of the
         variance values to be displayed).  If \texttt{"Quality"} is specified,
         then the quality values are treated as numerical values (in
         the range 0 to 255).  \texttt{["Data"]}
      }
      \sstsubsection{
         DEVICE = \htmlref{DEVICE}{se:selgradev} (Read)
      }{
         The name of the graphics device used to display the image.
         The device must have at least 24 colour indices or grey-scale
         intensities.  \texttt{[}current graphics device\texttt{{]}}
      }
      \sstsubsection{
         FILL = \_LOGICAL (Read)
      }{
         If FILL is set to \texttt{TRUE}, then the image will be `stretched' to fill
         the current picture in both directions.  This can be useful when
         displaying images with markedly different dimensions, such as
         two-dimensional spectra.  The dynamic default is \texttt{TRUE} if the array
         being displayed is one-dimensional, and \texttt{FALSE} otherwise.  \texttt{[]}
      }
      \sstsubsection{
         HIGH = \_DOUBLE (Read)
      }{
         The data value corresponding to the highest pen in the colour
         table.  All larger data values are set to the highest colour
         index when HIGH is greater than LOW, otherwise all data values
         greater than HIGH are set to the lowest colour index.  The
         dynamic default is the maximum data value.  There is an
         efficiency gain when both LOW and HIGH are given on the
         command line, because the extreme values need not be computed.
         (Scale mode)
      }
      \sstsubsection{
         IN = NDF (Read)
      }{
         The input NDF structure containing the data to be displayed.
      }
      \sstsubsection{
         KEY = \_LOGICAL (Read)
      }{
         \texttt{TRUE} if a key to the colour table is to be produced to the right
         of the display.  This can take the form of a colour ramp, a
         coloured histogram of pen indices, or graphs of RGB intensities,
         all annotated with data value.  The form and appearance of this key
         can be controlled using Parameter KEYSTYLE, and its horizontal
         position can be controlled using Parameter KEYPOS.  If the key is
         required in a different location, set KEY=NO and use application
         \htmlref{LUTVIEW}{LUTVIEW} after displaying the image.  \texttt{[TRUE]}
      }
      \sstsubsection{
         KEYPOS( 2 ) = \_REAL (Read)
      }{
         The first element gives the gap between the right-hand edge of
         the display and the left-hand edge of the key, as a fraction
         of the width of the current picture.  If a key is produced,
         then the right-hand margin specified by Parameter MARGIN is
         ignored, and the value supplied for KEYPOS is used instead.

         The second element gives the vertical position of the key as a
         fractional value in the range zero to one: zero puts the key
         as low as possible, one puts it as high as possible.  A
         negative value (no lower than \texttt{-1}) causes the key to
         match the height of the display image. This may mean any text,
         like a label, for the horizontal axis may not appear, though
         if AXES is \texttt{TRUE} there is usually room.
         \texttt{[}current value\texttt{{]}}
      }
      \sstsubsection{
         KEYSTYLE = \htmlref{GROUP}{se:groups} (Read)
      }{
         A group of attribute settings describing the plotting style to use
         for the key (see Parameter KEY).

         A comma-separated list of strings should be given in which each
         string is either an attribute setting, or the name of a text
         file preceded by an up-arrow character \texttt{"$\wedge$"}.  Such text files
         should contain further comma-separated lists which will be
         read and interpreted in the same manner.  Attribute settings
         are applied in the order in which they occur within the list,
         with later settings overriding any earlier settings given for
         the same attribute.

         Each individual attribute setting should be of the form:

            $<$name$>$=$<$value$>$


         where $<$name$>$ is the name of a plotting attribute, and $<$value$>$
         is the value to assign to the attribute.  Default values will be
         used for any unspecified attributes.  All attributes will be
         defaulted if a null value (\texttt{{!}})---the initial default---is supplied.
         To apply changes of style to only the current invocation, begin these
         attributes with a plus sign.  A mixture of persistent and temporary
         style changes is achieved by listing all the persistent attributes
         followed by a plus sign then the list of temporary attributes.

         See \slhyperref{Plotting Attributes}{Section~}{}{ap:plotting_attr}
         for a description of the available attributes.  Any unrecognised
         attributes are ignored (no error is reported).

         Axis 1 is always the \emph{data value} axis.  So for instance, to set
         the label for the data-value axis, assign a value to \texttt{"Label(1)"}
         in the supplied style.

         To get a ramp key (the default), specify \texttt{"form=ramp"}.  To
         get a histogram key (a coloured histogram of pen indices),
         specify \texttt{"form=histogram"}.  To get a graph key (three curves of
         RGB intensities), specify \texttt{"form=graph"}.  If a histogram key
         is produced, the population axis can be either logarithmic or
         linear.  To get a logarithmic population axis, specify \texttt{"logpop=1"}.
         To get a linear population axis, specify \texttt{"logpop=0"} (the default).
         To annotate the long axis with pen numbers instead of pixel value,
         specify \texttt{"pennums=1"} (the default, \texttt{"pennums=0"}, shows pixel
         values).  \texttt{[}current value\texttt{{]}}
      }
      \sstsubsection{
         LOW = \_DOUBLE (Read)
      }{
         The data value corresponding to the lowest pen in the colour
         table.  All smaller data values are set to the lowest colour
         index when LOW is less than HIGH, otherwise all data values
         smaller than LOW are set to the highest colour index.  The
         dynamic default is the minimum data value.  There is an
         efficiency gain when both LOW and HIGH are given on the
         command line, because the extreme values need not be computed.
         (Scale mode)
      }
      \sstsubsection{
         LUT = NDF (Read)
      }{
         Name of the NDF containing a colour lookup table in its Data
         array; the lookup table is written to the graphics device's colour
         table.  The purpose of this parameter is to provide a means of
         controlling the appearance of the image on certain devices,
         such as colour printers, that do not have a dynamic colour
         table (\emph{i.e.} the colour table is reset when the device is
         opened).  If used with dynamic devices (such as X-windows),
         the new colour table remains after this application has
         completed.  A null value (\texttt{{!}}) causes the existing colour table to
         be used.

         The LUT must be two-dimensional, the dimension of the first axis
         being 3, and the second being arbitrary.  The method used to
         compress or expand the colour table if the second dimension is
         different from the number of unreserved colour indices is
         controlled by Parameter NN.  Also the LUT's values must lie in
         the range 0.0--1.0.  \texttt{[!]}
      }
      \sstsubsection{
         MARGIN( 4 ) = \_REAL (Read)
      }{
         The widths of the margins to leave around the image for axis
         annotations, given as fractions of the corresponding dimension
         of the current picture.  The actual margins used may be increased to
         preserve the aspect ratio of the data.  Four values may be given, in
         the order: bottom, right, top, left.  If fewer than four values are
         given, extra values are used equal to the first supplied value.  If
         these margins are too narrow any axis annotation may be clipped.
         If a null (\texttt{{!}}) value is supplied, the value used is (for all edges);
         \texttt{0.15} if annotated axes are being produced; 0.04, if a simple border
         is being produced; and 0.0 if neither border nor axes are being
         produced.  \texttt{[}current value\texttt{{]}}
      }
      \sstsubsection{
         MODE = \htmlref{LITERAL}{se:parmenu} (Read)
      }{
         The method by which the maximum and minimum data values to be
         displayed are chosen.  The options are as follows.

         \ssthitemlist{

            \sstitem
            \texttt{"Current"} --- The image is scaled between the upper and lower
            limits that were used by the previous invocation of DISPLAY. If
            the previous scaling limits cannot be determined, the MODE
            value reverts to \texttt{"Scale"}.

            \sstitem
            \texttt{"Faint"} --- The image is scaled between the mean data value minus
            one standard deviation and the mean data value plus seven standard
            deviations.  The scaling values are reported so that the faster
            Scale mode may be utilised later.

            \sstitem
            \texttt{"Flash"} --- The image is flashed on to the screen without any
            scaling at all.  This is the fastest option.

            \sstitem
            \texttt{"Percentiles"} --- The image is scaled between the data values
            corresponding to two percentiles.  The scaling values are reported
            so that the faster Scale mode may be used later.

            \sstitem
            \texttt{"Range"} --- The image is scaled between the minimum and maximum
            data values.

            \sstitem
            \texttt{"Scale"} --- You define the upper and lower limits between which
            the image is to be scaled.  The application reports the maximum
            and the minimum data values for reference and makes these the
            suggested defaults.

            \sstitem
            \texttt{"Sigmas"} --- The image is scaled between two standard-deviation
            limits.  The scaling values used are reported so that the faster
            Scale mode may be utilised later.
         }
      }
      \sstsubsection{
         NN = \_LOGICAL (Read)
      }{
         If \texttt{TRUE} the input lookup table is mapped to the colour table by
         using the nearest-neighbour method.  This preserves sharp
         edges and is better for lookup tables with blocks of colour.
         If NN is \texttt{FALSE}, linear interpolation is used, and this is
         suitable for smoothly varying colour tables.  NN is ignored
         unless LUT is not null.  \texttt{[FALSE]}
      }
      \sstsubsection{
         NUMBIN = \_INTEGER (Read)
      }{
         The number of histogram bins used to compute percentiles for
         scaling. (Percentiles mode) \texttt{[2048]}
      }
      \sstsubsection{
         OUT = NDF (Write)
      }{
         A scaled copy of the displayed section of the image. Values in
         this output image are integer colour indices shifted to exclude
         the indices reserved for the palette (\emph{i.e.} the value zero refers
         to the first colour index following the palette).  The output NDF
         is intended to be used as the input data in conjunction with
         SCALE=\texttt{FALSE}.  If a null value (\texttt{{!}}) is supplied, no output NDF will
         be created.  This parameter is not accessed when SCALE=\texttt{FALSE}.  \texttt{[!]}
      }
      \sstsubsection{
         PENRANGE( 2 ) = \_REAL (Read)
      }{
         The range of colour indices (``pens'') to use. The supplied values
         are fractional values where zero corresponds to the lowest available
         colour index and 1.0 corresponds to the highest available colour
         index. The default value of \texttt{[0.0,1.0]} thus causes the full range
         of colour indices to be used. Note, if Parameter LUT is null
         (!) or Parameter SCALE is \texttt{FALSE} then this parameter is ignored
         and the fill range of pens is used. \texttt{[0.0,1.0]}
      }
      \sstsubsection{
         PERCENTILES( 2 ) = \_REAL (Read)
      }{
         The percentiles that define the scaling limits.  For example,
         \texttt{[25,75]} would scale between the quartile values. (Percentile
         mode)
      }
      \sstsubsection{
         SCALE = \_LOGICAL (Read)
      }{
         If \texttt{TRUE} the input data are to be scaled according to the value of
         Parameter MODE.  If it is \texttt{FALSE}, MODE is ignored, and the input
         data are displayed as is (\emph{i.e.} the data values are simply converted
         to integer type and used as indices into the colour table).  A value
         of zero refers to the first pen following the palette.  A \texttt{FALSE}
         value is intended to be used with data previously scaled by this
         or similar applications which have already performed the required
         scaling (see Parameter OUT).  It provides the quickest method of
         image display within this application.  \texttt{[TRUE]}
      }
      \sstsubsection{
         SIGMAS( 2 ) = \_REAL (Read)
      }{
         The standard-deviation bounds that define the scaling limits.
         To obtain values either side of the mean both a negative and
         a positive value are required.  Thus \texttt{[-2,3]} would scale
         between the mean minus two and the mean plus three standard
         deviations.  \texttt{[3,-2]} would give the negative of that.
      }
      \sstsubsection{
         SQRPIX = \_LOGICAL (Read)
      }{
         If \texttt{TRUE}, then the default value for YMAGN equals the value
         supplied for XMAGN, resulting in all pixels being displayed as
         squares on the display surface.  If a \texttt{FALSE} value is supplied for
         SQRPIX, then the default value for YMAGN is chosen to retain the
         pixels original aspect ratio at the centre of the image.
         \texttt{[}current value\texttt{{]}}
      }
      \sstsubsection{
         STYLE = \htmlref{GROUP}{se:groups} (Read)
      }{
         A group of attribute settings describing the plotting style to use
         for the annotated axes (see Parameter AXES).

         A comma-separated list of strings should be given in which each
         string is either an attribute setting, or the name of a text
         file preceded by an up-arrow character \texttt{"$\wedge$"}.  Such text files
         should contain further comma-separated lists which will be
         read and interpreted in the same manner.  Attribute settings
         are applied in the order in which they occur within the list,
         with later settings overriding any earlier settings given for
         the same attribute.

         Each individual attribute setting should be of the form:

            $<$name$>$=$<$value$>$


         where $<$name$>$ is the name of a plotting attribute, and $<$value$>$
         is the value to assign to the attribute.  Default values will be
         used for any unspecified attributes.  All attributes will be
         defaulted if a null value (\texttt{{!}})---the initial default---is supplied.
         To apply changes of style to only the current invocation, begin these
         attributes with a plus sign.  A mixture of persistent and temporary
         style changes is achieved by listing all the persistent attributes
         followed by a plus sign then the list of temporary attributes.

         See \slhyperref{Plotting Attributes}{Section~}{}{ap:plotting_attr}
         for a description of the available attributes.  Any unrecognised
         attributes are ignored (no error is reported).
         \texttt{[}current value\texttt{{]}}
      }
      \sstsubsection{
         USEAXIS = GROUP (Read)
      }{
         USEAXIS is only accessed if the current co-ordinate Frame of the
         NDF has more than two axes.  A group of two strings should be
         supplied specifying the two axes which are to be used when annotating
         the image, and when supplying a value for Parameter CENTRE.  Each
         axis can be specified using one of the following options.

         \ssthitemlist{

            \sstitem
            Its integer index within the current Frame of the
            input  NDF (in the range 1 to the number of axes in the
            current Frame).

            \sstitem
            Its \htmlattref{Symbol}{Symbol(axis)}~ string such as
            \texttt{"RA"} or \texttt{"VRAD"}.

            \sstitem
            A generic option where \texttt{"SPEC"} requests the spectral axis,
            \texttt{"TIME"} selects the time axis, \texttt{"SKYLON"} and
            \texttt{"SKYLAT"} picks the sky longitude and latitude axes
            respectively.  Only those axis domains present are
            available as options.
         }

         A list of acceptable values is displayed if an illegal value is
         supplied.  If a null (\texttt{{!}}) value is supplied, the axes
         with the same indices as the two used pixel axes within the NDF
         are used.  \texttt{[!]}
      }
      \sstsubsection{
         XMAGN = \_REAL (Read)
      }{
         The horizontal magnification for the image.  The default
         value of \texttt{1.0} corresponds to 'normal' magnification in which the
         the image fills the available space in at least one dimension.
         A value larger than 1.0 makes each data pixel wider.  If this
         results in the image being wider than the available space then
         the image will be clipped to display fewer pixels.  See also
         Parameters YMAGN, CENTRE, SQRPIX, and FILL.  \texttt{[1.0]}
      }
      \sstsubsection{
         YMAGN = \_REAL (Read)
      }{
         The vertical magnification for the image.  A value of \texttt{1.0}
         corresponds to 'normal' magnification in which the image fills
         the available space in at least one dimension.  A value larger than
         1.0 makes each data pixel taller.  If this results in the image
         being taller than the available space then the image will be
         clipped to display fewer pixels.  See also Parameters XMAGN,
         CENTRE, and FILL.  If a null ((\texttt{{!}}) value is supplied, the
         default  value used depends on Parameter SQRPIX.  If SQRPIX is
         \texttt{TRUE}, the default YMAGN value used is the value supplied
         for XMAGN.  This  will result in each pixel occupying a square
         area on the screen.  If SQRPIX is \texttt{FALSE}, then the default
         value for YMAGN is chosen so that each pixel occupies a rectangular
         area on the screen matching the pixel aspect ratio at the centre of
         the image, determined within the \htmlref{current WCS
         Frame.}{se:curframe}~ \texttt{[!]}
      }
   }
   \sstresparameters{
      \sstsubsection{
         SCAHIGH = \_DOUBLE (Write)
      }{
         On exit, this holds the data value which corresponds to the maximum
         colour index in the displayed image.  In Flash mode or when there is
         no scaling the highest colour index is returned.
      }
      \sstsubsection{
         SCALOW = \_DOUBLE (Write)
      }{
         The data value scaled to the minimum colour index for display.
         In Flash mode or when there is no scaling the lowest colour
         index is used.  The current display linear-scaling minimum is
         set to this value.
      }
   }
   \sstexamples{
      \sstexamplesubsection{
         display ngc6872 mode=p percentiles=[10,90] noaxes
      }{
         Displays the NDF called ngc6872 on the current graphics device
         device.  The scaling is between the 10 and 90 per cent
         percentiles of the image.  No annotated axes are produced.
      }
      \sstexamplesubsection{
         display vv256 mode=flash noaxes border borstyle="colour=blue,style=2"
      }{
         Displays the NDF called vv256 on the current graphics device
         device.  There is no scaling of the data; instead the modulus
         of each pixel with respect to the number of colour-table
         indices is shown.  No annotated axes are drawn, but a blue border
         is drawn around the image using \PGPLOT\  line style number 2 (\emph{i.e.}
         dashed lines).
      }
      \sstexamplesubsection{
         display mode=fa axes style="$\wedge$sty,grid=1" margin=0.2 clear out=video $\backslash$
      }{
         Displays the current NDF DATA component with annotated axes
         after clearing the current picture on the current graphics device
         device.  The appearance of the axes is specified in the text
         file \texttt{sty}, but this is modified by setting the \htmlattref{Grid}{Grid}~ attribute
         to 1 so that a co-ordinate grid is drawn across the plot.  The
         margins around the image containing the axes are made slightly
         wider than normal.  The scaling is between the $-$1 and $+$7 standard
         deviations of the image around its mean.  The scaled data are
         stored in an NDF called video.
      }
      \sstexamplesubsection{
         display kn26 axes key keypos=[0.0,-1.0] keystyle=$\wedge$key.sty $\backslash$
      }{
         Displays the NDF called kn26 using the current scaling,
         surrounded by axes.  It adds a colour-table key to the right
         that abuts the data picture and is aligned vertically with
         the image.  The plot attributes set in the text file \texttt{key.sty}
         controls the appearance of the key.
      }
      \sstexamplesubsection{
         display video noscale $\backslash$
      }{
         Displays the DATA component of the NDF called video (created
         in the previous example) without scaling within the current
         picture on the current graphics device.
      }
      \sstexamplesubsection{
         display in=cgs4a comp=v mode=sc low=1 high=5.2 device=xwindows
      }{
         Displays the VARIANCE component of NDF cgs4a on the xwindows
         device, scaling between 1 and 5.2.
      }
      \sstexamplesubsection{
         display mydata centre="12:23:34 -22:12:23" xmagn=2 badcol="red" $\backslash$
      }{
         Displays the NDF called mydata centred on the position RA=12:23:34,
         DEC=-22:12:23.  This assumes that the current co-ordinate Frame in the
         NDF is an equatorial (RA/DEC) Frame.  The image is displayed with
         a magnification of 2 so that each data pixel appears twice as large
         (on each axis) as normal.  Fewer data pixels may be displayed
         to ensure the image fits within the available space in the current
         picture.  The current scaling is used, and bad pixels are shown
         in red.
      }
      \sstexamplesubsection{
         display ngc6872 mode=ra device=lj250 lut=pizza
      }{
         Displays the NDF called ngc6872 on the LJ250 device.  The
         lookup table in the NDF called pizza is mapped on the LJ250's
         colour table.  The scaling is between the minimum and maximum
         of the image.
      }
   }
   \sstnotes{
      \sstitemlist{

         \sstitem
         For large images the resolution of the graphics device may allow
         only a fraction of the detail in the data to be plotted.  Therefore,
         large images will be compressed by block averaging when this can be
         done without loss of resolution in the displayed image.  This saves
         time scaling the data and transmitting them to the graphics device.
         Note that the default values for Parameters LOW and HIGH are
         the minimum and maximum values in the compressed floating-point
         data.

         \sstitem
         If no \htmlattref{Title}{plotel:Title}~ is specified via the
         STYLE parameter, then the \htmlref{TITLE}{apndf:title}
         component in the NDF is used as the default title for the
         annotated axes.  Should the NDF not have a TITLE component,
         then the default title is instead taken from current
         co-ordinate Frame in the NDF, unless this attribute has not
         been set explicitly, whereupon the name of the NDF is used as
         the default title.

         \sstitem
         The application stores a number of pictures in the graphics
         database in the following order: a FRAME picture containing the
         annotated axes, the image area, and the border; if there is a key,
         a KEY picture encompassing the key and its annotations; and a DATA
         picture containing just the image area.  Note, the FRAME picture
         is only created if annotated axes or a border have been drawn, or
         if non-zero margins were specified using Parameter MARGIN.  The
         world co-ordinates in the DATA picture will be pixel co-ordinates.
         A reference to the supplied NDF, together with a copy of the \htmlref{WCS}{apndf:wcs}
         information in the NDF are stored in the DATA picture.  On exit
         the current database picture for the chosen device reverts to the
         input picture.

         \sstitem
         The data type of the output NDF depends on the number of colour
         indices: \_UBYTE for no more than 256, \_UWORD for 257 to 65535,
         and \_INTEGER otherwise.  The output NDF will not contain any
         extensions, UNITS, QUALITY, and VARIANCE; but LABEL, TITLE, WCS and
         AXIS information are propagated from the input NDF.  The output
         NDF does not become the new current data array.  It is a Simple
         NDF (because the \htmlref{bad-pixel flag}{setbad:badpixelflag}~ is
         set to false in order to
         access the maximum colour index, and to handle sections),
         therefore only NDF-compliant applications can process it.
      }
   }
   \sstdiytopic{
      Related Applications
   }{
KAPPA: \htmlref{WCSFRAME}{WCSFRAME},
\htmlref{PICDEF}{PICDEF},
\htmlref{LUTVIEW}{LUTVIEW};
\xref{FIGARO}{sun86}{}: \xref{IGREY}{sun86}{IGREY},
\xref{IMAGE}{sun86}{IMAGE},
\xref{MOVIE}{sun86}{MOVIE}.
   }
   \sstimplementationstatus{
      \sstitemlist{

         \sstitem
         This routine correctly processes the \htmlref{AXIS}{apndf:axis}, DATA, \htmlref{QUALITY}{apndf:quality},
         \htmlref{VARIANCE}{apndf:variance}, \htmlref{LABEL}{apndf:label}, \htmlref{TITLE}{apndf:title}, \htmlref{WCS}{apndf:wcs}, and \htmlref{UNITS}{apndf:units}~ components of the input NDF.

         \sstitem
         Processing of \htmlref{bad pixels}{se:masking} and automatic \htmlref{quality masking}{se:qualitymask} are
         supported.

         \sstitem
         This application will handle data in all numeric types, though
         type conversion to integer will occur for unsigned byte and word
         images.  However, when there is no scaling only integer data will
         not be type converted, but this is not expensive for the expected
         byte-type data.
      }
   }
}



\sstroutine{
   DIV
}{
   Divides one NDF data structure by another
}{
   \sstdescription{
      The routine divides one \NDFref{NDF} data structure by another
      pixel-by-pixel to produce a new NDF.
   }
   \sstusage{
      div in1 in2 out
   }
   \sstparameters{
      \sstsubsection{
         IN1 = NDF (Read)
      }{
         First NDF, to be divided by the second NDF.
      }
      \sstsubsection{
         IN2 = NDF (Read)
      }{
         Second NDF, to be divided into the first NDF.
      }
      \sstsubsection{
         OUT = NDF (Write)
      }{
         Output NDF to contain the ratio of the two input NDFs.
      }
      \sstsubsection{
         TITLE = LITERAL (Read)
      }{
         The title for the output NDF.  A null value will cause
         the title of the NDF supplied for Parameter IN1 to be used
         instead.  \texttt{[!]}
      }
   }
   \sstexamples{
      \sstexamplesubsection{
         div a b c
      }{
         This divides the NDF called a by the NDF called b, to make the
         NDF called c.  NDF c inherits its title from a.
      }
      \sstexamplesubsection{
         div out=c in1=a in2=b title="Normalised data"
      }{
         This divides the NDF called a by the NDF called b, to make the
         NDF called c.  NDF c has the title \texttt{"Normalised data"}.
      }
   }
   \sstnotes{
      If the two input NDFs have different pixel-index bounds, then
      they will be trimmed to match before being divided.  An error will
      result if they have no pixels in common.
   }
   \sstdiytopic{
      Related Applications
   }{
KAPPA: \htmlref{ADD}{ADD},
\htmlref{CADD}{CADD},
\htmlref{CDIV}{CDIV},
\htmlref{CMULT}{CMULT},
\htmlref{CSUB}{CSUB},
\htmlref{MATHS}{MATHS},
\htmlref{MULT}{MULT},
\htmlref{SUB}{SUB}.
   }
   \sstimplementationstatus{
      \sstitemlist{

         \sstitem
         This routine correctly processes the \htmlref{AXIS}{apndf:axis}, DATA, \htmlref{QUALITY}{apndf:quality},
         \htmlref{LABEL}{apndf:label}, \htmlref{TITLE}{apndf:title}, \htmlref{UNITS}{apndf:units},
         \htmlref{HISTORY}{apndf:history}, \htmlref{WCS}{apndf:wcs}, and \htmlref{VARIANCE}{apndf:variance}~
         components of an NDF data structure and propagates all \htmlref{extensions}{apndf:extensions}.

         \sstitem
         Processing of \htmlref{bad pixels}{se:masking} and automatic \htmlref{quality masking}{se:qualitymask} are supported.

         \sstitem
         All \htmlref{non-complex numeric data types}{ap:HDStypes} can be handled.
         Calculations will be performed using either real or double
         precision arithmetic, whichever is more appropriate.  If the input
         NDF structures contain values with other data types, then
         conversion will be performed as necessary.

         \sstitem
         Huge NDFs are supported.

      }
   }
}

\sstroutine{
   DRAWNORTH
}{
   Draws arrows parallel to the axes
}{
   \sstdescription{
      This application draws a pair of arrows on top of a previously displayed
      \htmlref{DATA picture}{se:agiaction}~ which indicate the directions of the labelled axes in the
      underlying picture, at the position specified by Parameter ORIGIN.  For
      instance, if the underlying picture has axes labelled with celestial
      co-ordinates, then the arrows will by default indicate the directions
      of north and east.  The appearance of the arrows, including the labels
      attached to each arrow, may be controlled using the STYLE parameter.
      The picture area behind the arrows may optionally be cleared before
      drawing the arrows (see Parameter BLANK).
   }
   \sstusage{
      drawnorth [device] [length] [origin]
   }
   \sstparameters{
      \sstsubsection{
         ARROW = \_REAL (Read)
      }{
         The size of the arrow heads are specified by this parameter.  Simple
         lines can be drawn by setting the arrow head size to zero.  The value
         should be expressed as a fraction of the largest dimension of the
         underlying DATA picture.  \texttt{[}current value\texttt{{]}}
      }
      \sstsubsection{
         BLANK = \_LOGICAL (Read)
      }{
         If \texttt{TRUE}, then the area behind the arrows is blanked before the
         arrows are drawn.  This is done by drawing a rectangle filled with
         the current background colour of the selected graphics device.
         The size of the blanked area can be controlled using Parameter
         BLANKSIZE.  \texttt{[FALSE]}
      }
      \sstsubsection{
         BLANKSIZE = \_REAL (Read)
      }{
         Specifies the size of the blanked area (see Parameter BLANK).  A
         value of \texttt{1.0} results in the blanked area being just large enough
         to contain the drawn arrows and labels. Values larger than 1.0
         introduce a blank margin around the drawn arrows and labels.  This
         parameter also specifies the size of the picture stored in the
         graphics database.  \texttt{[1.05]}
      }
      \sstsubsection{
         DEVICE = \htmlref{DEVICE}{se:selgradev} (Read)
      }{
         The plotting device.  \texttt{[}Current graphics device\texttt{{]}}
      }
      \sstsubsection{
         EPOCH = \_DOUBLE (Read)
      }{
         If a `Sky Co-ordinate System' specification is supplied (using
         Parameter FRAME) for a celestial co-ordinate system, then an
         epoch value is needed to qualify it.  This is the epoch at
         which the supplied sky positions were determined.  It should be
         given as a decimal years value, with or without decimal places
         (\texttt{"1996.8"} for example).  Such values are interpreted as a
         Besselian epoch if less than 1984.0 and as a Julian epoch otherwise.
      }
      \sstsubsection{
         FRAME = LITERAL (Read)
      }{
         Specifies the \htmlref{co-ordinate Frame}{se:domains}~  to which the
         drawn arrows refer.
         If a null (\texttt{{!}}) value is supplied, the arrows are drawn parallel to
         the two axes which were used to annotate the previously displayed
         picture.  If the arrows are required to be parallel to the axes of
         some other  Frame, the required Frame should be specified using this
         parameter.  The string supplied for FRAME can be one of the following
         options.

         \ssthitemlist{

            \sstitem
            A \htmlref{domain name}{se:domains}~ such as \htmlref{SKY, AXIS, PIXEL}{se:resdoms}.

            \sstitem
            An integer value giving the index of the required Frame.

            \sstitem
            An IRAS90 \emph{Sky Co-ordinate System} (SCS) values such as
            \texttt{"EQUAT(J2000)"} (see \xref{SUN/163}{sun163}{}).

         }
         An error will be reported if a co-ordinate Frame is requested
         which is not available in the previously displayed picture.  If
         the selected Frame has more than two axes, the Parameter USEAXIS
         will determine the two axes which are to be used.  \texttt{[!]}
      }
      \sstsubsection{
         LENGTH( 2 ) = \_REAL (Read)
      }{
         The lengths of the arrows, expressed as fractions of the largest
         dimension of the underlying DATA picture.  If only one value is
         supplied, both arrows will be drawn with the given length.  One of
         the supplied values can be set to zero if only a single arrow is
         required.   \texttt{[}current value\texttt{{]}}
      }
      \sstsubsection{
         OFRAME = \htmlref{LITERAL}{se:parmenu} (Read)
      }{
         Specifies the \htmlref{co-ordinate Frame}{se:domains}~  in which the position of the arrows
         will be supplied (see Parameter ORIGIN).  The following Frames will
         always be available.

         \ssthitemlist{

            \sstitem
            \texttt{"GRAPHICS"} --- gives positions in millimetres from
            the bottom-left corner of the plotting surface.

            \sstitem
            \texttt{"BASEPIC"} --- gives positions in a normalised system in which the
            bottom-left corner of the plotting surface is (0,~0) and the
            shortest dimension of the plotting surface has length 1.0.  The
            scales on the two axes are equal.

            \sstitem
            \texttt{"CURPIC"} --- gives positions in a normalised system in which the
            bottom-left corner of the underlying DATA picture is (0,~0) and
            the shortest dimension of the picture has length 1.0.  The scales
            on the two axes are equal.

            \sstitem
            \texttt{"NDC"} --- gives positions in a normalised system in which the
            bottom-left corner of the plotting surface is (0,~0) and the
            top-right corner is (1,~1).

            \sstitem
            \texttt{"CURNDC"} --- gives positions in a normalised system in which the
            bottom-left corner of the underlying DATA picture is (0,~0) and the
            top-right corner is (1,~1).

         }
         Additional Frames will be available, describing the co-ordinates
         systems known to the data displayed within the underlying picture.
         These could include PIXEL, AXIS, SKY, for instance, but the
         exact list will depend on the displayed data.  If a null value is
         supplied, the ORIGIN position should be supplied in the Frame
         used to annotate the underlying picture (supplying a colon \texttt{":"}
         will display details of this co-ordinate Frame).  \texttt{["CURNDC"]}
      }
      \sstsubsection{
         ORIGIN = LITERAL (Read)
      }{
         The co-ordinates at which to place the origin of the arrows, in the
         Frame specified by Parameter OFRAME.  If a null (\texttt{{!}}) value is
         supplied, OFRAME is ignored and the arrows are situated at a
         default position near one of the corners, or at the centre.  The
         supplied position can be anywhere within the current picture.  An
         error is reported if the arrows and labels cannot be drawn at any
         of these positions.  \texttt{[!]}
      }
      \sstsubsection{
         STYLE = \htmlref{GROUP}{se:groups} (Read)
      }{
         A group of attribute settings describing the plotting style to use
         for the vectors and annotated axes.

         A comma-separated list of strings should be given in which each
         string is either an attribute setting, or the name of a text
         file preceded by an up-arrow character \texttt{"$\wedge$"}.  Such text files
         should contain further comma-separated lists which will be
         read and interpreted in the same manner.  Attribute settings
         are applied in the order in which they occur within the list,
         with later settings overriding any earlier settings given for
         the same attribute.

         Each individual attribute setting should be of the form:

            $<$name$>$=$<$value$>$

         where $<$name$>$ is the name of a plotting attribute, and $<$value$>$
         is the value to assign to the attribute.  Default values will be
         used for any unspecified attributes.  All attributes will be
         defaulted if a null value (\texttt{{!}})---the initial default---is supplied.
         To apply changes of style to only the current invocation, begin these
         attributes with a plus sign.  A mixture of persistent and temporary
         style changes is achieved by listing all the persistent attributes
         followed by a plus sign then the list of temporary attributes.

         See \slhyperref{Plotting Attributes}{Section~}{}{ap:plotting_attr}
         for a description of the available attributes.  Any unrecognised
         attributes are ignored (no error is reported).

         The appearance of the arrows is controlled by the attributes
         \htmlattref{Colour(Axes)}{Colour(element)},
         \htmlattref{Width(Axes)}{Width(element)}, \emph{etc.} (the synonym
         \att{Arrows} may be used in place of \att{Axes}).

         The text of the label to draw against each arrow is specified by
         the \att{Symbol(1)} and \att{Symbol(2)} attributes.  These default to the
         corresponding attributes of the underlying picture.  The
         appearance of these labels can be controlled using the attributes
         \htmlattref{Font(TextLab)}{Font(element)},
         \htmlattref{Size(TextLab)}{Size(element)}, \emph{etc}.  The gap between the end of the
         arrow and the corresponding label can be controlled using attribute
         \htmlattref{TextLabGap}{TextLabGap(axis)}.  The drawing of labels can be suppressed using
         attribute \htmlattref{TextLab}{TextLab(axis)}.   \texttt{[}current value\texttt{{]}}
      }
      \sstsubsection{
         USEAXIS = GROUP (Read)
      }{
         USEAXIS is only accessed if the co-ordinate Frame selected using
         Parameter FRAME has more than two axes.  A group of two strings should
         be supplied specifying the two axes to which the two drawn arrows should
         refer.  Each axis can be specified using one of the following options.

         \ssthitemlist{

            \sstitem
            An integer index of an axis within the current Frame of the
            input NDF (in the range 1 to the number of axes in the current
            Frame).

            \sstitem
            An axis \htmlattref{Symbol}{Symbol(axis)}~ string such as
            \texttt{"RA"} or \texttt{"VRAD"}.

            \sstitem
            A generic option where \texttt{"SPEC"} requests the spectral
            axis, \texttt{"TIME"} selects the time axis, \texttt{"SKYLON"}
            and \texttt{"SKYLAT"} picks the sky longitude and latitude
            axes respectively.  Only those axis domains present are
            available as options.
         }

         A list of acceptable values is displayed if an illegal value is
         supplied.  If a null (\texttt{{!}}) value is supplied, the
         first two axes of the Frame are used.  \texttt{[!]}
      }
   }
   \sstexamples{
      \sstexamplesubsection{
         drawnorth
      }{
         Draws a pair of arrows indicating the directions of the axes of
         the previously displayed image, contour map, \emph{etc}.  The arrows are
         drawn at the top left of the picture.  The current values for all
         other parameters are used.
      }
      \sstexamplesubsection{
         drawnorth blank origin="0.5,0.5" style='TextBackColour=clear'
      }{
         As above, but blanks out the picture area behind the arrows, and
         positions them in the middle of the underlying DATA picture.  In
         addition, the text labels are drawn with a clear background so
         that the underlying image can seen around the text.
      }
      \sstexamplesubsection{
         drawnorth blank blanksize=1.2 oframe=pixel origin="150,250"
      }{
         As above, but positions the arrows at pixel co-ordinates
         (150,250), and blanks out a larger area around the arrows.
      }
      \sstexamplesubsection{
         drawnorth blank oframe=! origin="10:12:34,-12:23:37"
      }{
         As above, but positions the arrows at RA=10:12:34 and
         DEC=-12:23:37 (this assumes the underlying picture was annotated
         with RA and DEC axes).
      }
      \sstexamplesubsection{
         drawnorth length=[0.1,0] style='colour(arrows)=red'
      }{
         Draws the axis-1 arrow with length equal to 0.1 of the longest
         dimension of the underlying picture, but does not draw the axis-2
         arrow.  Both arrows are drawn red.
      }
      \sstexamplesubsection{
         drawnorth style='textlab=0'
      }{
         Draws both arrows but does not draw any text labels.
      }
      \sstexamplesubsection{
         drawnorth style="'Size(TextLab1)=2,Symbol(1)=A,Symbol(2)=B'"
      }{
         Draws arrows with labels \texttt{"A"} and \texttt{"B"}, using characters of twice
         the default size for the label for the first axis.
      }
   }
   \sstnotes{
      \sstitemlist{

         \sstitem
         An error is reported if there is no existing DATA picture within
         the current picture on the selected graphics device.

         \sstitem
         The application stores a picture in the graphics database with name
         KEY which contains the two arrows.  On exit the current database picture
         for the chosen device reverts to the input picture.
      }
   }
}

\sstroutine{
   DRAWSIG
}{
   Draws $\pm$n standard-deviation lines on a line plot
}{
   \sstdescription{
      This routine draws straight lines on an existing plot stored in
      the \htmlref{graphics database}{se:agitate}, such as produced by
      LINPLOT or HISTOGRAM.
      The lines are located at arbitrary multiples of the standard
      deviation (NSIGMA) either side of the mean of a given dataset.
      The default dataset is the one used to draw the existing plot.
      You can plot the lines horizontally or vertically as appropriate.
      The lines extend the full width or height of the plot's data
      area.  Up to five different multiples of the standard deviation
      may be presented in this fashion.  Each line can be drawn with a
      different style (see Parameter STYLE).

      The application also computes statistics for those array values
      that lie between each pair of plotted lines.  In other words it
      finds the statistics between clipping limits defined by each
      2$*$NSIGMA range centred on the unclipped mean.

      The task tabulates NSIGMA, the mean, the standard deviation, and
      the error in the mean after the application of each pair of
      clipping limits.  For comparison purposes the first line of the
      table presents these values without clipping.  The table is
      written at the normal reporting level.
   }
   \sstusage{
      drawsig ndf nsigma [axis] [comp]
   }
   \sstparameters{
      \sstsubsection{
         AXIS = LITERAL (Read)
      }{
         The orientation of the lines, or put another way, the axis
         which represents data value.  Thus the allowed values are
         \texttt{"Horizontal"}, \texttt{"Vertical"}, \texttt{"X"}, or \texttt{"Y"}.  \texttt{"Horizontal"} is
         equivalent to \texttt{"Y"} and \texttt{"Vertical"} is a synonym for \texttt{"X"}.  On
         LINPLOT output AXIS would be \texttt{"Y"}, but on a plot from HISTOGRAM
         it would be \texttt{"X"}.  The suggested default is the current value.
         \texttt{["Y"]}
      }
      \sstsubsection{
         COMP = \htmlref{LITERAL}{se:parmenu} (Read)
      }{
         The name of the \NDFref{NDF} array component from which to derive the
         mean and standard deviation used to draw the lines: \texttt{"Data"},
         \texttt{"Error"}, \texttt{"Quality"} or \texttt{"Variance"} (where \texttt{"Error"} is the
         alternative to \texttt{"Variance"} and causes the square root of the
         variance values to be taken before computing the statistics).
         If \texttt{"Quality"} is specified, then the quality values are treated
         as numerical values (in the range 0 to 255).  \texttt{["Data"]}
      }
      \sstsubsection{
         DEVICE = \htmlref{DEVICE}{se:selgradev} (Read)
      }{
         The graphics device to draw the sigma lines on.
         \texttt{[}Current graphics device\texttt{{]}}
      }
      \sstsubsection{
         NDF = NDF (Read)
      }{
         The NDF structure containing the data array whose error limits
         are to be plotted.  Usually this parameter is not defined
         thereby causing the statistics to be derived from the dataset
         used to draw the plot.  If, however, you had plotted a section
         of a dataset but wanted to plot the statistics from the whole
         dataset, you would specify the full dataset with Parameter NDF.
         \texttt{[}The dataset used to create the existing plot\texttt{]}
      }
      \sstsubsection{
         NSIGMA() = \_REAL (Read)
      }{
         Number of standard deviations about the mean at which the
         lines should be drawn.  The null value or 0.0 causes a line to
         be drawn at the mean value.
      }
      \sstsubsection{
         STYLE = \htmlref{GROUP}{se:groups} (Read)
      }{
         A group of attribute settings describing the plotting style to use
         for the lines.

         A comma-separated list of strings should be given in which each
         string is either an attribute setting, or the name of a text
         file preceded by an up-arrow character \texttt{"$\wedge$"}.  Such text files
         should contain further comma-separated lists which will be
         read and interpreted in the same manner.  Attribute settings
         are applied in the order in which they occur within the list,
         with later settings overriding any earlier settings given for
         the same attribute.

         Each individual attribute setting should be of the form:

            $<$name$>$=$<$value$>$

         where $<$name$>$ is the name of a plotting attribute, and $<$value$>$
         is the value to assign to the attribute.  Default values will be
         used for any unspecified attributes.  All attributes will be
         defaulted if a null value (\texttt{{!}})---the initial default---is supplied.
         To apply changes of style to only the current invocation, begin these
         attributes with a plus sign.  A mixture of persistent and temporary
         style changes is achieved by listing all the persistent attributes
         followed by a plus sign then the list of temporary attributes.

         See \slhyperref{Plotting Attributes}{Section~}{}{ap:plotting_attr}
         for a description of the available attributes.  Any unrecognised
         attributes are ignored (no error is reported).

         The attributes \htmlattref{Colour(Curves)}{Colour(element)},
         \htmlattref{Width(Curves)}{Width(element)}, \emph{etc.}, can be used
         to specify the style for the lines (\att{Lines} is recognised as a
         synonym for \att{Curves}).  These values apply to all
         lines unless subsequent attributes override them.  Attributes for
         individual clipping levels can be given by replacing \att{Curves} above
         by a string of the form \texttt{"Nsig$<$i$>$"} where \texttt{"$<$i$>$"} is an integer
         index into the list of clipping levels supplied for Parameter
         NSIGMA.  Thus, \texttt{"Colour(Nsig1)"} will set the colour for the lines
         associated with the first clipping level, \emph{etc}.  The attribute
         settings can be restricted to one of the two lines by appending
         either a \texttt{"$+$"} or a \texttt{"-"} to the \texttt{"Nsig$<$i$>$"} string.  Thus,
         \texttt{"Width(Nsig2-)"} sets the line width for the lower of the two
         lines associated with the second clipping level, and \texttt{"Width(Nsig2+)"}
         sets the width for the upper of the two lines.  \texttt{[}current value\texttt{{]}}
      }
   }
   \pagebreak
   \sstexamples{
      \sstexamplesubsection{
         drawsig nsigma=3 style='style=1'
      }{
         This draws solid horizontal lines on the last DATA picture on
         the \htmlref{current graphics device}{se:devglobal} located at plus and
         minus 3 standard deviations about the mean.  The statistics come from
         the data array used to draw the DATA picture.
      }
      \sstexamplesubsection{
         drawsig phot 2.5
      }{
         This draws horizontal plus and minus 2.5 standard-deviation
         lines about the mean for the data in the NDF called phot on
         the default graphics device.
      }
      \sstexamplesubsection{
         drawsig phot 2.5 style='"colour(nsig1-)=red,colour(nsig1$+$)=green"'
      }{
         As above, but the lower line is drawn in red and the upper line
         is drawn in green.
      }
      \sstexamplesubsection{
         drawsig cluster [2,3] X Error
      }{
         This draws vertical lines at plus and minus 2 and 3
         standard deviations about the mean for the error data in the
         NDF called cluster on the default graphics device.
      }
      \sstexamplesubsection{
         drawsig device=xwindows phot(20:119) 3 style="'colour=red,style=4'"
      }{
         This draws red dotted horizontal lines on the xwindows device
         at $\pm$3 standard deviations using the 100 pixels in NDF
         phot(20:119).
      }
   }
   \sstnotes{
      There must be an existing DATA picture stored within the graphics
      database for the chosen device.  Lines will only be plotted
      within this picture.
   }
   \sstdiytopic{
      Related Applications
   }{
KAPPA: \htmlref{HISTOGRAM}{HISTOGRAM},
\htmlref{LINPLOT}{LINPLOT},
\htmlref{MLINPLOT}{MLINPLOT},
\htmlref{STATS}{STATS}.
   }
   \sstimplementationstatus{
      \sstitemlist{

         \sstitem
         This routine correctly processes the DATA, \htmlref{VARIANCE}{apndf:variance}, and
         \htmlref{QUALITY}{apndf:quality}, components of the NDF.

         \sstitem
         Processing of \htmlref{bad pixels}{se:masking} and automatic \htmlref{quality masking}{se:qualitymask} are
         supported.

         \sstitem
         All \htmlref{non-complex numeric data types}{ap:HDStypes} can be handled.  The
         statistics are calculated using double-precision floating point.

         \sstitem
         Any number of NDF dimensions is supported.
      }
   }
}
\sstroutine{
   ELPROF
}{
   Creates a radial or azimuthal profile of a two-dimensional image
}{
   \sstdescription{
     This application will bin the input image into elliptical annuli,
     or into a `fan' of adjacent sectors, centred on a specified
     position.  The typical data values in each bin are found (see
     Parameter ESTIMATOR), and stored in a one-dimensional \NDFref{NDF} which
     can be examined using \htmlref{LINPLOT}{LINPLOT}, \htmlref{LOOK}{LOOK},
     \emph{etc}.  A two-dimensional mask
     image can optionally be produced indicating which bin each input
     pixel was placed in.

      The area of the input image which is to be binned is the annulus
      enclosed between the two concentric ellipses defined by Parameters
      RATIO, ANGMAJ, RMIN, and RMAX.  The binned area can be restricted to
      an azimuthal section of this annulus using Parameter ANGLIM.  Input
      data outside the area selected by these parameters is ignored.  The
      selected area can be binned in two ways, specified by Parameter
      RADIAL.

      \sstitemlist{

         \sstitem
         If radial binning is selected (the default), then each bin is
         an elliptical annulus concentric with the ellipses bounding the
         binned area.  The number of bins is specified by Parameter NBIN
         and the radial thickness of each bin is specified by WIDTH.

         \sstitem
         If azimuthal binning is selected, then each bin is a sector
         (\emph{i.e.} a wedge-shape), with its vertex given by Parameters XC and
         YC, and its opening angle given by Parameter WIDTH.  The number of
         bins is specified by NBIN.
      }
   }
   \sstusage{
      elprof in out nbin xc yc
   }
   \sstparameters{
      \sstsubsection{
         ANGLIM( 2 ) = \_REAL (Read)
      }{
         Defines the wedge-shaped sector within which binning is to be
         performed.  The first value should be the azimuthal angle of
         the clockwise boundary of the sector, and the second should be
         the azimuthal angle of the anti-clockwise boundary.  The angles
         are measured in degrees from the \textit{x}-axis, and rotation from the
         \textit{x}-axis to the \textit{y}-axis is positive.  If only a single value is
         supplied, or if both values are equal, the sector starts at
         the given angle and extends for 360 degrees.  \texttt{[0.0]}
      }
      \sstsubsection{
         ANGMAJ = \_REAL (Read)
      }{
         The angle between the \textit{x}-axis and the major axis of the
         ellipse, in degrees.  Rotation from the \textit{x}-axis to the \textit{y}-axis is
         positive.  \texttt{[0.0]}
      }
      \sstsubsection{
         ESTIMATOR = \htmlref{LITERAL}{se:parmenu} (Read)
      }{
        The method to use for estimating the output pixel values.  It
        can be either \texttt{"Mean"} or \texttt{"Weighted Mean"}.  If the weighted mean
        option is selected but no variances are available in the input
        data, the unweighted mean will be used instead.  \texttt{["Mean"]}
      }
      \sstsubsection{
         IN = NDF (Read)
      }{
         The input NDF containing the two-dimensional image from which a
         profile is to be generated.
      }
      \sstsubsection{
         MASK = NDF (Write)
      }{
         An output NDF of the same shape and size as the input NDF
         indicating the bin into which each input pixel was placed.
         For radial profiles, the bins are identified by a mask value
         equal to the radius (in pixels) measured on the major axis, at
         the centre of the annular bin.  For azimuthal profiles, the
         bins are identified by a mask value equal to the angle from
         the \textit{x}-axis to the centre of the sector-shaped bin (in
         degrees).  If a null value is supplied, then no mask NDF is
         produced.  \texttt{[!]}
      }
      \sstsubsection{
         MTITLE = LITERAL (Read)
      }{
         A title for the mask NDF.  If a null value is given, the title
         is propagated from the input NDF.  This is only prompted for
         if MASK is given a non-null value.  \texttt{["Mask created by KAPPA -
         Elprof"]}
      }
      \sstsubsection{
         NBIN = \_INTEGER (Read)
      }{
         The number of radial or azimuthal bins required.
      }
      \sstsubsection{
         OUT = NDF (Write)
      }{
         The output one-dimensional NDF containing the required profile.
         For radial profiles, it has associated axis information
         describing the radius, in pixels, at the centre of each
         annular bin (the radius is measured on the major axis).  For
         azimuthal profiles, the axis information describes the
         azimuthal angle, in degrees, at the centre of each
         sector-shaped bin.  It will contain associated variance
         information if the input NDF has associated variance
         information.
      }
      \sstsubsection{
         RADIAL = \_LOGICAL (Read)
      }{
         Specifies the sort of profile required.  If RADIAL is \texttt{TRUE},
         then a radial profile is produced in which each bin is an
         elliptical annulus.  Otherwise, an azimuthal profile is
         produced in which each bin is a wedge-shaped sector.  \texttt{[TRUE]}
      }
      \sstsubsection{
         RATIO = \_REAL (Read)
      }{
         The ratio of the length of the minor axis of the ellipse to
         the length of the major axis.  It must be in the range 0.0 to
         1.0.  \texttt{[1.0]}
      }
      \sstsubsection{
         RMAX = \_REAL (Read)
      }{
         The radius in pixels, measured on the major axis, at the outer edge
         of the elliptical annulus to be binned.  If a null value (\texttt{{!}}) is
         supplied the value used is the distance from the ellipse centre
         (specified by XC and YC) to the furthest corner of the image.  This
         will cause the entire image to fall within the outer edge of the
         binning area.  \texttt{[!]}
      }
      \sstsubsection{
         RMIN = \_REAL (Read)
      }{
         The radius in pixels, measured on the major axis, at the inner edge
         of the elliptical region to be binned.  \texttt{[0.0]}
      }
      \sstsubsection{
         TITLE = LITERAL (Read)
      }{
         A title for the output profile NDF.  If a null value is
         supplied the title is propagated from the input NDF.  \texttt{["KAPPA -
         Elprof"]}
      }
      \sstsubsection{
         WIDTH = \_REAL (Read)
      }{
         The width of each bin.  If a radial profile is being created
         (see Parameter RADIAL) this is the width of each annulus in
         pixels (measured on the major axis).  If an azimuthal profile
         is being created, it is the opening angle of each sector, in
         degrees.  If a null (\texttt{{!}}) value is supplied, the value used is chosen
         so that there are no gaps between adjacent bins.  Smaller values
         will result in gaps appearing between adjacent bins.  The supplied
         value must be small enough to ensure that adjacent bins do not
         overlap. The supplied value must be at least 1.0.  \texttt{[!]}
      }
      \sstsubsection{
         XC = \_REAL (Read)
      }{
         The \textit{x} pixel co-ordinate of the centre of the ellipse, and the
         vertex of the sectors.
      }
      \sstsubsection{
         YC = \_REAL (Read)
      }{
         The \textit{y} pixel co-ordinate of the centre of the ellipse, and the
         vertex of the sectors.
      }
   }
   \sstexamples{
      \sstexamplesubsection{
         elprof galaxy galprof 20 113 210 angmaj=49 rmin=10 rmax=210 ratio=0.5
      }{
         This example will create a one-dimensional NDF called galprof
         containing a radial profile of the two-dimensional NDF called
         galaxy.  The profile will contain 20 bins and it will be
         centred on the pixel co-ordinates (113,210).  Each bin will be
         an annulus of an ellipse with axis ratio of 0.5 and
         inclination of 49 degrees to the \textit{x}-axis.  The image will be
         binned between radii of 10 pixels, and 210 pixels (measured on
         the major axis), and there will be no gaps between adjacent
         bins (\emph{i.e.} each bin will have a width on the major axis of
         about 10 pixels).
      }
      \sstexamplesubsection{
         elprof galaxy galprof 10 113 210 radial=f anglim=20 rmin=50 rmax=60
      }{
         This example also creates a one-dimensional NDF called galprof,
         this time containing an azimuthal profile of the two-dimensional
         NDF called \texttt{"galaxy"}, containing 10 bins.  Each bin will be a
         wedge-shaped sector with vertex at pixel co-ordinates
         (113,210).  The clockwise edge of the first bin will be at an
         angle of 20 degrees to the \textit{x}-axis, and each bin will have a
         width (opening angle) of 36 degrees (so that 360 degrees are
         covered in total).  Only the section of each sector bounded by
         radii of 50 and 60 pixels is included in the profile.  In this
         case the default value of \texttt{1.0} is accepted for Parameter RATIO
         and so the bins will form a circular annulus of width 10
         pixels.
      }
   }
   \sstdiytopic{
      Related Applications
   }{
\xref{ESP}{sun180}{}: \xref{ELLFOU}{sun180}{ELLFOU},
\xref{ELLPRO}{sun180}{ELLPRO},
\xref{SECTOR}{sun180}{SECTOR}.
   }
   \sstimplementationstatus{
      \sstitemlist{

         \sstitem
         This routine correctly processes the DATA, \htmlref{VARIANCE}{apndf:variance}, \htmlref{TITLE}{apndf:title},
         \htmlref{UNITS}{apndf:units}, \htmlref{WCS}{apndf:wcs} (in radial mode only) and \htmlref{HISTORY}{apndf:history}~ components of the
         input NDF to the output profile NDF.  WCS information is also
         propagated to the output mask NDF.

         \sstitem
         WCS information is currently lost by this application.

         \sstitem
         Processing of \htmlref{bad pixels}{se:masking} and automatic \htmlref{quality masking}{se:qualitymask} are
         supported.

         \sstitem
         All \htmlref{non-complex numeric data types}{ap:HDStypes} can be handled.  Arithmetic
         is performed using single-precision floating point.
      }
   }
}
\sstroutine{
   ERASE
}{
   Erases an HDS object
}{
   \sstdescription{
      This routine erases a specified \HDSref\ object or container file.  If
      the object is a structure, then all the structure's components
      (and sub-components, \emph{etc.}) are also erased.  If a slice or cell
      of an array is specified, then the entire array is erased.
   }
   \sstusage{
      erase object
   }
   \sstparameters{
      \sstsubsection{
         OBJECT = UNIV (Write)
      }{
         The HDS object or container file to be erased.
      }
      \sstsubsection{
         OK = \_LOGICAL (Read)
      }{
         This parameter is used to seek confirmation before an object
         is erased.  If a \texttt{TRUE} value is given, then the HDS object will
         be erased.  If a \texttt{FALSE} value is given, then the object will not
         be erased and a message will be issued to this effect.
      }
      \sstsubsection{
         REPORT = \_LOGICAL (Read)
      }{
         This parameter controls what happens if the named OBJECT does
         not exist.  If \texttt{TRUE}, an error is reported. Otherwise no error
         is reported. \texttt{[TRUE]}
      }
   }
   \sstexamples{
      \sstexamplesubsection{
         erase horse
      }{
         This erases the HDS container file called \texttt{horse.sdf}.
      }
      \sstexamplesubsection{
         erase fig123.axis
      }{
         This erases the \htmlref{AXIS}{apndf:axis}~ component of the HDS file called
         \texttt{fig123.sdf}.
         If AXIS is a structure, all its components are erased too.
      }
      \sstexamplesubsection{
         erase fig123.axis(1).label
      }{
         This erases the \htmlref{LABEL}{apndf:label}~ component within the first element of
         the AXIS structure of the HDS file called \texttt{fig123.sdf}.
      }
      \sstexamplesubsection{
         erase \$AGI\_USER/agi\_restar.agi\_3200\_1
      }{
         This erases the AGIDEV\_3200\_1 structure of the HDS file called \linebreak
         \texttt{\$AGI\_USER/agi\_restar.sdf}.
      }
   }
   \sstdiytopic{
      Related Applications
   }{
\xref{FIGARO}{sun86}{}: \xref{CREOBJ}{sun86}{CREOBJ},
\xref{DELOBJ}{sun86}{DELOBJ},
\xref{RENOBJ}{sun86}{RENOBJ}.
   }
}
\sstroutine{
   ERRCLIP
}{
   Removes pixels with large errors from an NDF
}{
   \sstdescription{
      This application produces a copy of the input \NDFref{NDF} in which pixels
      with errors greater than a specified limit are set invalid in
      both DATA and \htmlref{VARIANCE}{apndf:variance}~ components.  The error limit may be
      specified as the maximum acceptable standard deviation (or
      variance), or the minimum acceptable signal-to-noise ratio.
   }
   \sstusage{
      errclip in out limit [mode]
   }
   \sstparameters{
      \sstsubsection{
         IN = NDF (Read)
      }{
         The input NDF.  An error is reported if it contains no VARIANCE
         component.
      }
      \sstsubsection{
         LIMIT = \_DOUBLE (Read)
      }{
         Either the maximum acceptable standard deviation or variance
         value, or the minimum acceptable signal-to-noise ratio
         (depending on the value given for MODE).  It must be positive.
      }
      \sstsubsection{
         MODE = \htmlref{LITERAL}{se:parmenu} (Read)
      }{
         Determines how the value supplied for LIMIT is to be
         interpreted: \texttt{"Sigma"} for a standard deviation, \texttt{"Variance"}
         for variance, or \texttt{"SNR"} for minimum signal-to-noise ratio.
         \texttt{["Sigma"]}
      }
      \sstsubsection{
         OUT = NDF (Write)
      }{
         The output NDF.
      }
   }
   \sstexamples{
      \sstexamplesubsection{
         errclip m51 m51\_good 2.0
      }{
         The NDF m51\_good is created holding a copy of m51 in which all
         pixels with standard deviation greater than 2 are set invalid.
      }
      \sstexamplesubsection{
         errclip m51 m51\_good 2.0 snr
      }{
         The NDF m51\_good is created holding a copy of m51 in which all
         pixels with a signal-to-noise ratio less than 2 are set
         invalid.
      }
      \sstexamplesubsection{
         errclip m51 m51\_good mode=v limit=100
      }{
         The NDF m51\_good is created holding a copy of m51 in which all
         pixels with a variance greater than 100 are set invalid.
      }
   }
   \sstnotes{
      \sstitemlist{

         \sstitem
         The limit and the number of rejected pixels are reported.

         \sstitem
         A pair of output data and variance values are set bad when either of the input data or variances values is bad.

         \sstitem
         For MODE=\texttt{"SNR"} the comparison is with respect to the absolute
         data value.
      }
   }
   \sstdiytopic{
      Related Applications
   }{
KAPPA: \htmlref{FFCLEAN}{FFCLEAN},
\htmlref{PASTE}{PASTE},
\htmlref{SEGMENT}{SEGMENT},
\htmlref{SETMAGIC}{SETMAGIC},
\htmlref{THRESH}{THRESH}.
   }
   \sstimplementationstatus{
      \sstitemlist{

         \sstitem
         This routine correctly processes the \htmlref{AXIS}{apndf:axis}, DATA, \htmlref{QUALITY}{apndf:quality},
         \htmlref{VARIANCE}{apndf:variance}, \htmlref{LABEL}{apndf:label}, \htmlref{TITLE}{apndf:title}, \htmlref{UNITS}{apndf:units}, \htmlref{WCS}{apndf:wcs}, and \htmlref{HISTORY}{apndf:history}~ components of an NDF data
         structure and propagates all \htmlref{extensions}{apndf:extensions}.

         \sstitem
         Processing of \htmlref{bad pixels}{se:masking} and automatic \htmlref{quality masking}{se:qualitymask} are
         supported.

         \sstitem
         All \htmlref{non-complex numeric data types}{ap:HDStypes} can be handled.  The output
         NDF has the same numeric type as the input NDF.  However, all
         internal calculations are performed in double precision.
      }
   }
}

\sstroutine{
   EXCLUDEBAD
}{
   Excludes bad rows or columns from a two-dimensional NDF
}{
   \sstdescription{
      This application produces a copy of a two-dimensional \NDFref{NDF}, but excludes
      any rows that contain too many \htmlref{bad aata values}{se:masking}.
      Rows with higher pixel indices are shuffled down to fill the gaps left
      by the omission of bad rows. Thus if any bad rows are found, the output
      NDF will have fewer rows than the input NDF, but the order of the
      remaining rows will be unchanged. The number of good pixels required
      in a row for the row to be retained is specified by Parameter WLIM.

      Bad columns may be omitted instead of bad rows (see Parameter ROWS).
   }
   \sstusage{
      excludebad in out [rows] [wlim]
   }
   \sstparameters{
      \sstsubsection{
         IN = NDF (Read)
      }{
         The input two-dimensional NDF.
      }
      \sstsubsection{
         OUT = NDF (Write)
      }{
         The output NDF.
      }
      \sstsubsection{
         ROWS = \_LOGICAL (Read)
      }{
         If \texttt{TRUE}, bad rows are excluded from the output NDF. If \texttt{FALSE},
         bad columns are excluded. \texttt{[TRUE] }
      }
      \sstsubsection{
         WLIM = \_REAL (Read)
      }{
         The minimum fraction of the pixels which must be good in order for
         a row to be retained. A value of \texttt{1.0} results in rows being
         excluded if they contain one or more bad values. A value of
         \texttt{0.0} results in rows being excluded only if they contain no good
         values. \texttt{[0.0] }
      }
   }
   \sstexamples{
      \sstexamplesubsection{
         excludebad ifuframe goodonly false
      }{
         Columns within NDF ifuframe that contain any good data are
         copied to NDF goodonly.
      }
   }
   \sstnotes{
      \sstitemlist{

         \sstitem
         The lower pixel bounds of the output will be the same as those
         of the input, but the upper pixel bounds will be different if any
         bad rows or columns are excluded.
      }
   }
   \sstdiytopic{
      Related Applications
   }{
KAPPA: \htmlref{CHPIX}{CHPIX},
\htmlref{FILLBAD}{FILLBAD},
\htmlref{GLITCH}{GLITCH},
\htmlref{NOMAGIC}{NOMAGIC},
\htmlref{ZAPLIN}{ZAPLIN};
\xref{FIGARO}{sun86}{}: \xref{BCLEAN}{sun86}{BCLEAN},
\xref{CLEAN}{sun86}{CLEAN},
\xref{ISEDIT}{sun86}{ISEDIT},
\xref{REMBAD}{sun86}{REMBAD},
\xref{TIPPEX}{sun86}{TIPPEX}.
   }
   \sstimplementationstatus{
      \sstitemlist{

         \sstitem
         This routine correctly processes the \htmlref{AXIS}{apndf:axis}, DATA, \htmlref{QUALITY}{apndf:quality},
         \htmlref{VARIANCE}{apndf:variance}, \htmlref{LABEL}{apndf:label}, \htmlref{TITLE}{apndf:title}, \htmlref{UNITS}{apndf:units}, \htmlref{WCS}{apndf:wcs}, and \htmlref{HISTORY}{apndf:history}~ components of an NDF data
         structure and propagates all \htmlref{extensions}{apndf:extensions}.
      }
   }
}
\sstroutine{
   EXP10
}{
   Takes the base-10 exponential of an NDF data structure
}{
   \sstdescription{
      This routine takes the base-10 exponential of each
      pixel of a \NDFref{NDF} to produce a new NDF data structure.

      This command is a synonym for \texttt{expon base=10D0}.
   }
   \sstusage{
      exp10 in out
   }
   \sstparameters{
      \sstsubsection{
         IN = NDF (Read)
      }{
         Input NDF data structure.
      }
      \sstsubsection{
         OUT = NDF (Write)
      }{
         Output NDF data structure being the exponential of the input NDF.
      }
      \sstsubsection{
         TITLE = LITERAL (Read)
      }{
         The title for the output NDF.  A null value will cause
         the title of the NDF supplied for Parameter IN to be used
         instead.  \texttt{[!]}
      }
   }
   \sstexamples{
      \sstexamplesubsection{
         exp10 a b
      }{
         This takes exponentials to base ten of the pixels in the NDF
         called a, to make the NDF called b.  NDF b inherits its title
         from a.
      }
      \sstexamplesubsection{
         exp10 title="Abell 4321" out=b in=a
      }{
         This takes exponentials to base ten of the pixels in the NDF
         called a, to make the NDF called b.  NDF b has the title
         \texttt{"Abell 4321"}.
      }
   }
   \sstdiytopic{
      Related Applications
   }{
KAPPA: \htmlref{LOG10}{LOG10},
\htmlref{LOGAR}{LOGAR},
\htmlref{LOGE}{LOGE},
\htmlref{EXPE}{EXPE},
\htmlref{EXPON}{EXPON},
\htmlref{POW}{POW};
\xref{FIGARO}{sun86}{}: \xref{IALOG}{sun86}{IALOG},
\xref{ILOG}{sun86}{ILOG},
\xref{IPOWER}{sun86}{IPOWER}.
   }
   \sstimplementationstatus{
      \sstitemlist{

         \sstitem
         This routine correctly processes the \htmlref{AXIS}{apndf:axis}, DATA, \htmlref{QUALITY}{apndf:quality},
         \htmlref{LABEL}{apndf:label}, \htmlref{TITLE}{apndf:title}, \htmlref{UNITS}{apndf:units}, \htmlref{HISTORY}{apndf:history}, \htmlref{WCS}{apndf:wcs}, and \htmlref{VARIANCE}{apndf:variance}~ components of
         an NDF data structure and propagates all \htmlref{extensions}{apndf:extensions}.

         \sstitem
         Processing of \htmlref{bad pixels}{se:masking} and automatic \htmlref{quality masking}{se:qualitymask} are
         supported.

         \sstitem
         All \htmlref{non-complex numeric data types}{ap:HDStypes} can be handled.
      }
   }
}

\sstroutine{
   EXPE
}{
   Takes the natural exponential of an NDF data structure
}{
   \sstdescription{
      This routine takes the natural exponential of each
      pixel of a \NDFref{NDF} to produce a new NDF data structure.

      This command is a synonym for \texttt{expon base=natural}.
   }
   \sstusage{
      expe in out
   }
   \sstparameters{
      \sstsubsection{
         IN = NDF (Read)
      }{
         Input NDF data structure.
      }
      \sstsubsection{
         OUT = NDF (Write)
      }{
         Output NDF data structure being the exponential of the input NDF.
      }
      \sstsubsection{
         TITLE = LITERAL (Read)
      }{
         The title for the output NDF.  A null value will cause
         the title of the NDF supplied for Parameter IN to be used
         instead.  \texttt{[!]}
      }
   }
   \sstexamples{
      \sstexamplesubsection{
         loge a b
      }{
         This takes the natural exponential of the pixels in the NDF
         called a, to make the NDF called b.  NDF b inherits its title
         from a.
      }
      \sstexamplesubsection{
         loge title="Cas A" out=b in=a
      }{
         This takes natural exponentials of the pixels in the NDF
         called a, to make the NDF called b.  NDF b has the title
         \texttt{"Cas A"}.
      }
   }
   \sstdiytopic{
      Related Applications
   }{
KAPPA: \htmlref{LOG10}{LOG10},
\htmlref{LOGAR}{LOGAR},
\htmlref{LOGE}{LOGE},
\htmlref{EXP10}{EXP10},
\htmlref{EXPON}{EXPON},
\htmlref{POW}{POW};
\xref{FIGARO}{sun86}{}: \xref{IALOG}{sun86}{IALOG},
\xref{ILOG}{sun86}{ILOG},
\xref{IPOWER}{sun86}{IPOWER}.
   }
   \sstimplementationstatus{
      \sstitemlist{

         \sstitem
         This routine correctly processes the \htmlref{AXIS}{apndf:axis}, DATA, \htmlref{QUALITY}{apndf:quality},
         \htmlref{LABEL}{apndf:label}, \htmlref{TITLE}{apndf:title}, \htmlref{UNITS}{apndf:units}, \htmlref{HISTORY}{apndf:history}, \htmlref{WCS}{apndf:wcs}, and \htmlref{VARIANCE}{apndf:variance}~ components of
         an NDF data structure and propagates all \htmlref{extensions}{apndf:extensions}.

         \sstitem
         Processing of \htmlref{bad pixels}{se:masking} and automatic \htmlref{quality masking}{se:qualitymask} are
         supported.

         \sstitem
         All \htmlref{non-complex numeric data types}{ap:HDStypes} can be handled.
      }
   }
}


\sstroutine{
   EXPON
}{
   Takes the exponential (specified base) of an NDF data structure
}{
   \sstdescription{
      This routine takes the exponential to a specified base of each
      pixel of a \NDFref{NDF} to produce a new NDF data structure.
   }
   \sstusage{
      expon in out base
   }
   \sstparameters{
      \sstsubsection{
         BASE = LITERAL (Read)
      }{
         The base of the exponential to be applied.  A special value
         \texttt{"Natural"} gives natural (base-e) exponentiation.
      }
      \sstsubsection{
         IN = NDF (Read)
      }{
         Input NDF data structure.
      }
      \sstsubsection{
         OUT = NDF (Write)
      }{
         Output NDF data structure being the exponential of the input NDF.
      }
      \sstsubsection{
         TITLE = LITERAL (Read)
      }{
         The title for the output NDF.  A null value will cause
         the title of the NDF supplied for Parameter IN to be used
         instead.  \texttt{[!]}
      }
   }
   \sstexamples{
      \sstexamplesubsection{
         expon a b 10
      }{
         This takes exponentials to base ten of the pixels in the NDF
         called a, to make the NDF called b.  NDF b inherits its title
         from a.
      }
      \sstexamplesubsection{
         expon base=8 title="HD123456" out=b in=a
      }{
         This takes expoentials to base eight of the pixels in the NDF
         called a, to make the NDF called b.  NDF b has the title
         \texttt{"HD123456"}.
      }
   }
   \sstdiytopic{
      Related Applications
   }{
KAPPA: \htmlref{LOG10}{LOG10},
\htmlref{LOGAR}{LOGAR},
\htmlref{LOGE}{LOGE},
\htmlref{EXP10}{EXP10},
\htmlref{EXPE}{EXPE},
\htmlref{POW}{POW};
\xref{FIGARO}{sun86}{}: \xref{IALOG}{sun86}{IALOG},
\xref{ILOG}{sun86}{ILOG},
\xref{IPOWER}{sun86}{IPOWER}.
   }
   \sstimplementationstatus{
      \sstitemlist{

         \sstitem
         This routine correctly processes the \htmlref{AXIS}{apndf:axis}, DATA, \htmlref{QUALITY}{apndf:quality},
         \htmlref{LABEL}{apndf:label}, \htmlref{TITLE}{apndf:title}, \htmlref{UNITS}{apndf:units}, \htmlref{HISTORY}{apndf:history}, \htmlref{WCS}{apndf:wcs}, and \htmlref{VARIANCE}{apndf:variance}~ components of
         an NDF data structure and propagates all \htmlref{extensions}{apndf:extensions}.

         \sstitem
         Processing of \htmlref{bad pixels}{se:masking} and automatic \htmlref{quality masking}{se:qualitymask} are
         supported.

         \sstitem
         All \htmlref{non-complex numeric data types}{ap:HDStypes} can be handled.
      }
   }
}

\sstroutine{
   FFCLEAN
}{
   Removes defects from a substantially flat one-, or two-, or three-dimensional NDF
}{
   \sstdescription{
      This application cleans a one- or two-dimensional \NDFref{NDF}
      by removing defects smaller than a specified size.  In addition,
      three-dimensional NDFs can be cleaned by processing each row or
      plane within it using the one- or two-dimensional algorithm (see
      Parameter AXES).

      The defects are \htmlref{flagged}{se:masking}~ with the bad
      value.  The defects are found by looking for pixels that deviate
      from the spectrum or image's smoothed version by more than an
      arbitrary number of standard deviations from the local mean, and
      that lie within a specified range of values.  Therefore, the
      data array must be substantially flat.  The data variances
      provide the local noise estimate for the threshold, but if these
      are not available a variance for the whole of the data array is
      derived from the mean squared deviations of the original and
      smoothed versions.  The smoothed version of the data array is
      obtained by block averaging over a rectangular box.  An
      iterative process progressively removes the outliers from the
      data array.
   }
   \sstusage{
      ffclean in out clip box [thresh] [wlim]
   }
   \sstparameters{
      \sstsubsection{
         AXES( 2 ) = \_INTEGER (Read)
      }{
         The indices of up to two axes that span the rows or planes that
         are to be cleaned.  If only one value is supplied, then the NDF
         is processed as a set of one-dimensional spectra parallel to the
         specified pixel axis.  If two values are supplied, then the NDF
         is processed as a set of two-dimensional images spanned by the
         given axes.  Thus, a two-dimensional NDF can be processed either
         as a single two-dimensional image or as a set of one-dimensional
         spectra.  Likewise, a three-dimensional NDF can be processed
         either as a set of two-dimensional images or a set of
         one-dimensional spectra.  By default, a two-dimensional NDF is
         processed as a single two-dimensional image, and a
         three-dimensional NDF is processed as a set of one-dimensional
         spectra (the spectral axis is chosen by examining the WCS
         component---pixel-axis 1 is used if the current WCS frame does
         not contain a spectral axis).  \texttt{[]}
      }
      \sstsubsection{
         BOX( 2 ) = \_INTEGER (Read)
      }{
         The \textit{x} and \textit{y} sizes (in pixels) of the rectangular box to be
         applied to smooth the image.  If only a single value is given,
         then it will be duplicated so that a square filter is used
         except where the image is one-dimensional for which the box size
         along the insignificant dimension is set to 1.  The values
         given will be rounded up to positive odd integers if
         necessary.
      }
      \sstsubsection{
         CLIP( ) = \_REAL (Read)
      }{
         The number of standard deviations for the rejection threshold
         of each iteration.  Pixels that deviate from their counterpart
         in the smoothed image by more than CLIP times the noise are
         made bad.  The number of values given specifies the number of
         iterations.  Values should lie in the range 0.5--100.  Up to
         one hundred values may be given.  \texttt{[3.0, 3.0, 3.0]}
      }
      \sstsubsection{
         GENVAR = \_LOGICAL (Read)
      }{
         If \texttt{TRUE}, the noise level implied by the deviations from
         the local mean over the supplied box size are stored in the
         output VARIANCE component.  This noise level has a constant
         value over the whole NDF (or over each section of the NDF if the
         NDF is being processed in sections---see Parameter AXES).  This
         constant noise level is also displayed on the screen if
         the current message-reporting level is at least \texttt{NORMAL}.  If
         GENVAR is \texttt{FALSE}, then the output variances will be copied from
         the input variances (if the input NDF has no variances, then
         the output NDF will not have any variances either).   ).  \texttt{[FALSE]}
      }
      \sstsubsection{
         IN = NDF (Read)
      }{
         The one- or two-dimensional NDF containing the input image to be
         cleaned.
      }
      \sstsubsection{
         OUT = NDF (Write)
      }{
         The NDF to contain the cleaned image.
      }
      \sstsubsection{
         THRESH( 2 ) = \_DOUBLE (Read)
      }{
         The range between which data values must lie if cleaning is to
         occur.  Thus it is possible to clean the background without
         removing the cores of images by a judicious choice of these
         thresholds.  If null, \texttt{{!}}, is given, then there is no limit on
         the data range.  \texttt{[!]}
      }
      \sstsubsection{
         TITLE = LITERAL (Read)
      }{
         The title of the output NDF.  A null (\texttt{{!}}) value means using the
         title of the input NDF.  \texttt{[!]}
      }
      \sstsubsection{
         WLIM = \_REAL (Read)
      }{
         If the input image contains bad pixels, then this parameter
         may be used to determine the number of good pixels which must
         be present within the smoothing box before a valid output
         pixel is generated.  It can be used, for example, to prevent
         output pixels from being generated in regions where there are
         relatively few good pixels to contribute to the smoothed
         result.

         By default, a null (\texttt{{!}}) value is used for WLIM, which causes
         the pattern of bad pixels to be propagated from the input
         image to the output image unchanged.  In this case, smoothed
         output values are only calculated for those pixels which are
         not bad in the input image.

         If a numerical value is given for WLIM, then it specifies the
         minimum fraction of good pixels which must be present in the
         smoothing box in order to generate a good output pixel.  If
         this specified minimum fraction of good input pixels is not
         present, then a bad output pixel will result, otherwise a
         smoothed output value will be calculated.  The value of this
         parameter should lie between 0.0 and 1.0 (the actual number
         used will be rounded up if necessary to correspond to at least
         one pixel).  \texttt{[!]}
      }
   }
   \pagebreak
   \sstresparameters{
      \sstsubsection{
         SIGMA = \_DOUBLE (Write)
      }{
         The estimation of the RMS noise per pixel of the output image.
      }
   }
   \sstexamples{
      \sstexamplesubsection{
         ffclean dirty clean $\backslash$
      }{
         The NDF called dirty is filtered such that pixels that
         deviate by more than three standard deviations from the
         smoothed version of dirty are rejected.  Three iterations are performed.
         Each pixel in the smoothed image is the average of the
         neighbouring nine pixels.  The filtered NDF is called clean.
      }
      \sstexamplesubsection{
         ffclean out=clean in=dirty thresh=[-100,200]
      }{
         As above except only those pixels whose values lie between
         $-$100 and 200 can be cleaned.
      }
      \sstexamplesubsection{
         ffclean poxy dazed [2.5,2.8] [5,5]
      }{
         The two-dimensional NDF called poxy is filtered such that pixels that
         deviate by more than 2.5 then 2.8 standard deviations from the
         smoothed version of poxy are rejected.  The smoothing is an
         average of a 5-by-5-pixel neighbourhood.
         The filtered NDF is called dazed.
      }
   }
   \sstnotes{
      \sstitemlist{

         \sstitem
         There are different facts reported, their verbosity depending
         on the current message-reporting level set by environment variable
         MSG\_FILTER.  When the filtering level is at least as verbose as
         NORMAL, the application will report the intermediate results after
         each iteration during processing.  In addition, it will report the
         section of the input NDF currently being processed (but only if
         the NDF is being processed in sections---see Parameter AXES).

      }
   }
   \sstdiytopic{
      Related Applications
   }{
KAPPA: \htmlref{CHPIX}{CHPIX},
\htmlref{FILLBAD}{FILLBAD},
\htmlref{GLITCH}{GLITCH},
\htmlref{MEDIAN}{MEDIAN},
\htmlref{MSTATS}{MSTATS},
\htmlref{ZAPLIN}{ZAPLIN};
\xref{FIGARO}{sun86}{}: \xref{BCLEAN}{sun86}{BCLEAN},
\xref{COSREJ}{sun86}{COSREJ},
\xref{CLEAN}{sun86}{CLEAN},
\xref{ISEDIT}{sun86}{ISEDIT},
\xref{MEDFILT}{sun86}{MEDFILT},
\xref{MEDSKY}{sun86}{MEDSKY},
\xref{TIPPEX}{sun86}{TIPPEX}.
   }
   \sstimplementationstatus{
      \sstitemlist{

         \sstitem
         This routine correctly processes the \htmlref{AXIS}{apndf:axis}, DATA, \htmlref{QUALITY}{apndf:quality},
         \htmlref{VARIANCE}{apndf:variance}, \htmlref{LABEL}{apndf:label}, \htmlref{TITLE}{apndf:title}, \htmlref{UNITS}{apndf:units}, \htmlref{WCS}{apndf:wcs}, and \htmlref{HISTORY}{apndf:history}~ components of an NDF
         data structure and propagates all \htmlref{extensions}{apndf:extensions}.

         \sstitem
         Processing of \htmlref{bad pixels}{se:masking} and automatic \htmlref{quality masking}{se:qualitymask} are
         supported.

         \sstitem
         All \htmlref{non-complex numeric data types}{ap:HDStypes} can be handled.  Arithmetic
         is performed using single- or double-precision floating point as
         appropriate.
      }
   }
}
\sstroutine{
   FILLBAD
}{
   Removes regions of bad values from an NDF
}{
   \sstdescription{
      This application replaces bad values in an \NDFref{NDF} with a
      smooth function which matches the surrounding data.  It can fill
      arbitrarily shaped regions of bad values within n-dimensional arrays.

      It forms a smooth replacement function for the regions of bad
      values by forming successive approximations to a solution of
      Laplace's equation, with the surrounding valid data providing the
      boundary conditions.
   }
   \sstusage{
      fillbad in out [niter] [size]
   }
   \sstparameters{
      \sstsubsection{
         BLOCK = \_INTEGER (Read)
      }{
         The maximum number of pixels along either dimension when the
         array is divided into blocks for processing.  It is ignored
         unless MEMORY=\texttt{TRUE}.  This must be at least 256.
         \texttt{[512]}
      }
      \sstsubsection{
         IN = NDF (Read)
      }{
         The NDF containing the input image with bad values.
      }
      \sstsubsection{
         MEMORY = \_LOGICAL (Read)
      }{
         If this is \texttt{FALSE}, the whole array is processed at the same
         time.  If it is \texttt{TRUE}, the array is divided into chunks whose
         maximum dimension along an axis is given by Parameter BLOCK.
         \texttt{[FALSE]}
      }
      \sstsubsection{
         NITER = \_INTEGER (Read)
      }{
         The number of iterations of the relaxation algorithm.  This
         value cannot be fewer than two, since this is the minimum
         number required to ensure that all bad values are assigned a
         replacement value.  The more iterations used, the finer the
         detail in the replacement function and the closer it will
         match the surrounding good data.  \texttt{[2]}
      }
      \sstsubsection{
         OUT = NDF (Write)
      }{
         The NDF to contain the image free of bad values.
      }
      \sstsubsection{
         SIZE( )  = \_REAL (Read)
      }{
         The initial scale lengths in pixels to be used in the first
         iteration, along each axis.  If fewer values are supplied than
         pixel axes in the NDF, the last value given is repeated for
         the remaining axes.  The size \texttt{0} means no fitting across a
         dimension.  For instance, \texttt{[0,0,5]} would be appropriate if
         the spectra along the third dimension of a cube are independent,
         and the replacement values are to be derived only within each
         spectrum.

         For maximum efficiency, a scale length should normally have a
         value about half the `size' of the largest invalid region to
         be replaced.  (See \htmlref{``Notes''}{notes:fillbad}~ section
         for more details.)
         \texttt{[5.0]}
      }
      \sstsubsection{
         TITLE = LITERAL (Read)
      }{
         The title of the output NDF.  A null (\texttt{{!}}) value means using the
         title of the input NDF.  \texttt{[!]}
      }
      \sstsubsection{
         VARIANCE = \_LOGICAL (Read)
      }{
         If VARIANCE is \texttt{TRUE}, variance information is to be propagated;
         any bad values therein are filled.  Also the variance is used
         to weight the calculation of the replacement data values.  If
         VARIANCE is \texttt{FALSE}, there will be no variance processing thus
         requiring two less arrays in memory.  This parameter is only
         accessed if the input NDF contains a \htmlref{VARIANCE}{apndf:variance}~ component.
         \texttt{[TRUE]}
      }
   }
   \sstresparameters{
      \sstsubsection{
         CNGMAX = \_DOUBLE (Write)
      }{
         The maximum absolute change in output values which occurred in
         the final iteration.
      }
      \sstsubsection{
         CNGRMS = \_DOUBLE (Write)
      }{
         The root-mean-squared change in output values which occurred in the last
         iteration.
      }
   }
   \sstexamples{
      \sstexamplesubsection{
         fillbad aa bb
      }{
         The NDF called aa has its bad pixels replaced by good values
         derived from the surrounding good pixel values using two
         iterations of a relaxation algorithm.  The initial scale length
         is 5 pixels.  The resultant NDF is called bb.
      }
      \sstexamplesubsection{
         fillbad aa bb 6 20 title="Cleaned image"
      }{
         As above except the initial scale length is 20 pixels, 5
         iterations will be performed, and the output title is \texttt{"Cleaned
         image"} instead of the title of NDF aa.
      }
      \sstexamplesubsection{
         fillbad aa bb memory novariance
      }{
         As in the first example except that processing is performed
         with blocks up to 512 by 512 pixels to reduce the memory
         requirements, and no variance information will be used or
         propagated.
      }
      \sstexamplesubsection{
         fillbad in=speccube out=speccube\_fill size=[0,0,128] iter=5
      }{
        Suppose NDF speccube is a spectral imaging cube with the
        spectral axis third.  This example replaces the bad pixels by
        valid values derived from the surrounding good pixel values
        within each spectrm, using an initial scale length of 128
        channels, iterating five times.  The filled NDF is called
        speccube\_fill.
      }
      \sstexamplesubsection{
         fillbad in=speccube out=speccube\_fill size=[5,5,128] iter=5
      }{
        As the previous example, but now the relaxation occurs along
        the spatial axes too, initially with a scale length of five
        pixels.
     }
   }
   \label{notes:fillbad}
   \sstnotes{
      \sstitemlist{

         \sstitem

         The algorithm is based on the relaxation method of repeatedly
         replacing each bad pixel with the mean of its two nearest
         neighbours along each pixel axis.  Such a method converges to
         the required solution, but information about the good regions
         only propagates at a rate of about one pixel per iteration
         into the bad regions, resulting in slow convergence if large
         areas are to be filled.

         This application speeds convergence to an acceptable function
         by forming the replacement mean from all the pixels in the
         same axis (such as row or a column), using a weight which
         decreases exponentially with distance and goes to zero after
         the first good pixel is encountered in any direction.  If
         there is variance information, this is included in the
         weighting so as to give more weight to surrounding values
         with lower variance.  The scale length of the exponential
         weight is initially set large, to allow rapid propagation of
         an approximate `smooth' solution into the bad regions---an
         initially acceptable solution is thus rapidly obtained (often
         in the first one or two iterations).  The scale length is
         subsequently reduced by a factor of 2 whenever the maximum
         absolute change occurring in an iteration has decreased by a
         factor of 4 since the current scale length was first used.
         In this way, later iterations introduce progressively finer
         detail into the solution.  Since this fine detail occurs
         predominantly close to the `crinkly' edges of the bad
         regions, the slower propagation of the solution in the later
         iterations is then less important.

         When there is variance processing the output variance is
         reassigned if either the input variance or data value was
         bad. Where the input value is good but its associated
         variance is bad, the calculation proceeds as if the data
         value were bad, except that only the variance is substituted
         in the output.  The new variance is approximated as twice the
         inverse of the sum of the weights.

         \sstitem
         The price of the above efficiency means that considerable
         workspace is required, typically two or three times the size of
         the input image, but even larger for the one and two-byte integer
         types.  If memory is at a premium, there is an option to process
         in blocks ({\it cf.}\ Parameter MEMORY).  However, this may not give as
         good results as processing the array in full, especially when the
         bad-pixel regions span blocks.

         \sstitem
         The value of the Parameter SIZE is not critical and the
         default value will normally prove effective.  It primarily
         affects the efficiency of the algorithm on various size scales.
         If the smoothing scale is set to a large value, large scale
         variations in the replacement function are rapidly found, while
         smaller scale variations may require many iterations.
         Conversely, a small value will rapidly produce the small scale
         variations but not the larger scale ones.  The aim is to select
         an initial value SIZE such that during the course of a few
         iterations, the range of size scales in the replacement function
         are all used.  In practice this means that the value of SIZE
         should be about half the size of the largest scale variations
         expected.  Unless the valid pixels are very sparse, this is
         usually determined by the `size' of the largest invalid region to
         be replaced.

         \sstitem
         An error results if the input NDF has no bad values to replace.

         \sstitem
         The progress of the iterations is reported at the normal
         reporting level.  The format of the output is slightly
         different if the scale lengths vary with pixel axis; an extra
         axis column is included.
      }
   }
   \sstdiytopic{
      Timing
   }{
      The time taken increases in proportion to the value of NITER.
      Adjusting the SIZE parameter to correspond to the largest regions
      of bad values will reduce the processing time.  See the
      \htmlref{``Notes''}{notes:fillbad}~ section.
   }
   \sstdiytopic{
      Related Applications
   }{
KAPPA: \htmlref{CHPIX}{CHPIX},
\htmlref{GLITCH}{GLITCH},
\htmlref{MEDIAN}{MEDIAN},
\htmlref{ZAPLIN}{ZAPLIN};
\xref{FIGARO}{sun86}{}: \xref{BCLEAN}{sun86}{BCLEAN},
\xref{COSREJ}{sun86}{COSREJ},
\xref{CLEAN}{sun86}{CLEAN},
\xref{ISEDIT}{sun86}{ISEDIT},
\xref{MEDFILT}{sun86}{MEDFILT},
\xref{MEDSKY}{sun86}{MEDSKY},
\xref{REMBAD}{sun86}{REMBAD},
\xref{TIPPEX}{sun86}{TIPPEX}.
   }
   \sstimplementationstatus{
      \sstitemlist{

         \sstitem
         This routine correctly processes the \htmlref{AXIS}{apndf:axis}, DATA, \htmlref{QUALITY}{apndf:quality},
         \htmlref{VARIANCE}{apndf:variance}, \htmlref{LABEL}{apndf:label}, \htmlref{TITLE}{apndf:title}, \htmlref{UNITS}{apndf:units}, \htmlref{WCS}{apndf:wcs}, and \htmlref{HISTORY}{apndf:history}~ components of an NDF
         data structure and propagates all \htmlref{extensions}{apndf:extensions}.

         \sstitem
         Processing of \htmlref{bad pixels}{se:masking} and automatic \htmlref{quality masking}{se:qualitymask} are
         supported.  The output \htmlref{bad-pixel flag}{setbad:badpixelflag}~ is
         set to indicate no bad values in the data and variance arrays.

         \sstitem
         All \htmlref{non-complex numeric data types}{ap:HDStypes} can be handled.  Arithmetic
         is performed using single- or double-precision floating point as
         appropriate.
      }
   }
}

\sstroutine{
   FITSDIN
}{
   Reads a FITS disc file composed of simple, group or table objects
}{
   \sstdescription{
      This application reads selected disc-\FITSref\ ~files.  The files may
      be Basic (simple) FITS, and/or have TABLE extensions (Harten
      {\it et al.}\ 1988).

      The programme reads a simple or a random-groups-format FITS file
      (Wells {\it et al.}\ 1981; Greisen \& Harten 1981), and writes the
      data into an \NDFref{NDF}, and the headers into the NDF's
      \htmlref{FITS extension}{se:fitsairlock}.
      Table-format files (Grosb{\o}l {\it et al.}\ 1988) are read, and the
      application creates two files: a text formatted table/catalogue
      and a FACTS description file (as used by SCAR) based upon the FITS
      header cards.  Composite FITS files can be processed.  You may
      specify a list of files, including wildcards.  A record of the
      FITS headers, and group parameters (for a group-format file) can
      be stored in a text file.

      There is an option to run in automatic mode, where the names of
      output NDF data structures are generated automatically, and you
      can decide whether or not format conversion is to be applied to
      all files (rather than being prompted for each).  This is very
      useful if there is a large number of files to be processed.  Even
      if you want unique file names, format-conversion prompting may be
      switched off globally.
   }
   \sstusage{
      fitsdin files out [auto] fmtcnv [logfile] dscftable=? table=?
   }
   \sstparameters{
      \sstsubsection{
         AUTO = \_LOGICAL (Read)
      }{
         It is \texttt{TRUE} if automatic mode is required, where the name of each
         output NDF structure or table file is to be generated by the
         application, and therefore not prompted; and a global
         format-conversion switch may be set.  In manual mode the
         FITS header is reported, but not in automatic.

         In automatic mode the application generates a filename
         beginning with the input filename, less any extension.  For
         example, if the input file was \texttt{saturn.fits} the filename of the
         output NDF would be \texttt{saturn.sdf}, and an output table would be
         \texttt{saturn.dat} with a description file \texttt{dscfsaturn.dat}.
         If there are sub-files (more than one FITS object in the file)
         a suffix \texttt{\_<subfile>} is appended.  So if \texttt{saturn.fits}
         comprised a simple file followed by a table, the table would
         be called \texttt{SATURN\_2.DAT} and the description file
         \texttt{DSCFSATURN\_2.DAT}.  For group format a suffix
         \texttt{G<groupnumber>} is appended.  Thus if \texttt{saturn.fits}
         is a group format file, the first NDF created would be
         called \texttt{saturn.sdf}, the second would be \texttt{saturnG2.sdf}.
         \texttt{[FALSE]}
      }
      \sstsubsection{
         DSCFTABLE = FILENAME (Read)
      }{
         Name of the text file to contain the FACTS descriptors, which
         defines the table's format for {\footnotesize SCAR}.  Since {\footnotesize SCAR} is now
         deprecated, this parameter has little use, except perhaps to
         give a summary of the format of the file specified by Parameter
         TABLE.  A null value (\texttt{{!}}) means that no description file will
         be created, so this is now the recommended usage.  If your
         FITS file comprises just tables, you should consider other
         tools such as the \CURSAref\  package, which has facilities for
         examining and processing ASCII and binary tables in FITS files.

         A suggested filename for the description file is reported
         immediately prior to prompting in manual mode.  It is the name
         of the catalogue, as written in the FITS header, with a
         \texttt{"dscf"} prefix.
      }
      \sstsubsection{
         ENCODINGS = \htmlref{LITERAL}{se:parmenu} (Read)
      }{
         Determines which FITS keywords should be used to define the
         world co-ordinate systems to be stored in the NDF's \htmlref{WCS}{apndf:wcs}
         component.  The allowed values (case-insensitive) are as
         follows.

         \ssthitemlist{

         \sstitem
            \texttt{"FITS-IRAF"} --- This uses keywords CRVAL\textit{i}
            CRPIX\textit{i}, CD\textit{i\_j}, and is the system commonly
            used by IRAF.  It is described in the document ``World
            Coordinate Systems Representations Within the FITS Format''
            by R.J.~Hanisch and D.G.~Wells, 1988, available by ftp from
            fits.cv.nrao.edu \texttt{/fits/documents/wcs/wcs88.ps.Z}.

         \sstitem
            \texttt{"FITS-WCS"} --- This is the FITS standard WCS
            encoding scheme described in the paper ``Representation of
            celestial coordinates in FITS'' \newline
            (\htmladdnormallink{\texttt{{http://www.cv.nrao.edu/fits/documents/wcs/wcs.html}}}
            {http://www.cv.nrao.edu/fits/documents/wcs/wcs.html}). \newline

            It is very similar to FITS-IRAF but supports a wider range of
            projections and co-ordinate systems.  Once the standard has
            been agreed, this encoding should be understood by any
            FITS-WCS compliant software and it is likely to be adopted
            widely for FITS data in future.

         \sstitem
            \texttt{"FITS-PC"} --- This uses keywords CRVAL\textit{i},
            CDELT\textit{i}, CRPIX\textit{i}, PC\textit{iiijjj}, \emph{etc},
            as in a previous (now superseded) draft of the above
            FITS world co-ordinate system paper by E.W.~Greisen and
            M.~Calabretta.

         \sstitem
            \texttt{"FITS-AIPS"} --- This uses conventions described in the
            document ``Non-linear Coordinate Systems in AIPS'' by Eric W.
            Greisen (revised 9th September, 1994), available by ftp from
            fits.cv.nrao.edu \texttt{/fits/documents/wcs/aips27.ps.Z}.  It is
            currently employed by the AIPS data analysis facility, so its
            use will facilitate data exchange with AIPS.  This encoding
            uses CROTA\textit{i} and CDELT\textit{i} keywords to describe
            axis rotation and scaling.

         \sstitem
            \texttt{"DSS"} --- This is the system used by the Digital Sky Survey,
            and uses keywords AMDX\textit{n}, AMDY\textit{n}, PLTRAH,
            \emph{etc}.

         \sstitem
            \texttt{"Native"} --- This is the native system used by the AST
            library (see \xref{SUN/210}{sun210}{}), and provides a loss-free
            method for transferring WCS information between AST-based
            application.  It allows more complicated WCS information to be
            stored and retrieved than any of the other encodings.
         }

         A comma-separated list of up to six values may be supplied, in
         which case the value actually used is in the first in the list
         for which corresponding keywords can be found in the FITS
         header.

         A FITS header may contain keywords from more than one of these
         encodings, in which case it is possible for the encodings to be
         inconsistent with each other.  This may happen for instance if
         an application modifies the keyword associated with one encoding
         but fails to make equivalent modifications to the others.  If a
         null parameter value (\texttt{{!}}) is supplied for ENCODINGS, then an
         attempt is made to determine the most reliable encoding to use
         as follows.  If both native and non-native encodings are
         available, then the first non-native encoding to be found which
         is inconsistent with the native encoding is used.  If all
         encodings are consistent, then the native encoding is used (if
         present).  \texttt{[!]}
     }
      \sstsubsection{
         FILES() = LITERAL (Read)
      }{
         A list of (optionally wild-carded) file specifications which
         identify the disc-FITS files to be processed.  Up to ten values
         may be given, but only a single specification such as \texttt{"*.fits"}
         is normally required.  Be careful not to include non-FITS files
         in this list.
      }
      \sstsubsection{
         FMTCNV = \_LOGICAL (Read)
      }{
         This specifies whether or not format conversion will occur.
         If \texttt{FALSE}, the HDS type of the data array in the NDF will be
         the equivalent of the FITS data format in the file (\emph{e.g.} BITPIX=\texttt{16}
         creates a \_WORD array).  If \texttt{TRUE}, the data array in the
         current file, or all files in automatic mode, will be
         converted from the FITS data type in the FITS file to \_REAL in the NDF.
         The conversion applies the values of the FITS keywords BSCALE
         and BZERO to the FITS-file data to generate the `true' data values.
         If BSCALE and BZERO are not given in the FITS header, they are
         taken to be 1.0 and 0.0 respectively.
         The suggested default is \texttt{TRUE}.
      }
      \sstsubsection{
         GLOCON = \_LOGICAL (Read)
      }{
         If \texttt{FALSE} a format-conversion query occurs for each FITS file.
         If \texttt{TRUE}, the value of Parameter FMTCNV is obtained before any file
         numbers and will apply to all data arrays.  It is ignored
         in automatic mode---in effect it becomes \texttt{TRUE}.  \texttt{[FALSE]}
      }
      \sstsubsection{
         LOGFILE = FILENAME (Read)
      }{
         The file name of the text log of the FITS header cards.
         For group-format data the group parameters are evaluated
         and appended to the full header.  The log includes the names of
         the output files used to store the data array or table.  A null
         value (\texttt{{!}}) means that no log file is produced.  \texttt{[!]}
      }
      \sstsubsection{
         OUT = NDF (Write)
      }{
         Output NDF structure holding the full contents of the FITS
         file.  If the null value (\texttt{{!}}) is given no NDF will be created.
         This offers an opportunity to review the descriptors before
         deciding whether or not the data are to be extracted.
      }
      \sstsubsection{
         TABLE = FILENAME (Read)
      }{
         Name of the text file to contain the table itself, read from
         the file.  In manual mode, the suggested default filename is the name of
         description file less the \texttt{"dscf"} prefix, or if there is no
         description file or if the description file does not have the
         \texttt{"dscf"} prefix, the suggested name reverts to the catalogue name
         in the FITS header.
      }
   }
   \sstexamples{
      \sstexamplesubsection{
         fitsdin files=*.fit auto nofmtcnv
      }{
         This reads all the files with extension \texttt{"fit"} in the default
         directory.  If the files were \texttt{sao.fit} and \texttt{moimp.fit}
         and each
         contained just an image array, the output NDFs will be sao and
         moimp respectively.  The data will not have format conversion.
      }
      \sstexamplesubsection{
         fitsdin files=ccd.ifits fmtcnv logfile=jkt.log
      }{
         This reads the file \texttt{ccd.ifits} and processes all the FITS
         objects within it.  Integer data arrays are converted to real
         using the scale and zero found in the FITS header.  A record
         of the headers and the names of the output files are written
         to the text file \texttt{jkt.log}.
      }
      \sstexamplesubsection{
         fitsdin files=[*.*fits,*.mt] glocon fmtcnv
      }{
         This reads the files \texttt{*.*fits} and \texttt{*.mt} and processes all the
         FITS objects within them.  Integer data arrays are converted
         to real using the scale and zero found in the FITS header.
         Any IEEE-format data will not be converted although the global
         conversion switch is on.
      }
   }
   \sstdiytopic{
      References
   }{
      \begin{refs}
      \item  Wells, D.C., Greisen, E.W. \& Harten, R.H. 1981,
      {\em Astron.  Astrophys.  Suppl.  Ser.} {\bf 44}, 363.

      \item  Greisen, E.W. \& Harten, R.H. 1981,
      {\em Astron.  Astrophys.  Suppl.  Ser.} {\bf 44}, 371.

      \item  Grosb{\o}l, P., Harten, R.H., Greisen, E.W \& Wells, D.C.
      1988 {\em Astron.  Astrophys.  Suppl.  Ser.} {\bf 73}, 359.

      \item  Harten, R.H., Grosb{\o}l, P., Greisen, E.W \& Wells, D.C.
      1988 {\em Astron.  Astrophys.  Suppl.  Ser.} {\bf 73}, 365.

      \end{refs}
   }
   \sstdiytopic{
      Related Applications
   }{
KAPPA: \htmlref{FITSHEAD}{FITSHEAD},
\htmlref{FITSIMP}{FITSIMP},
\htmlref{FITSIN}{FITSIN},
\htmlref{FITSLIST}{FITSLIST};
\xref{CONVERT}{sun55}{}: \xref{FITS2NDF}{sun55}{FITS2NDF};
\CURSAref;
\xref{FIGARO}{sun86}{}: \xref{RDFITS}{sun86}{RDFITS}.
   }
   \sstimplementationstatus{
      \sstitemlist{

         \sstitem
         The application processes FITS files blocked at other than an
         integer multiple of 2880 bytes up to a maximum of 28800, provided
         it is a multiple of the number of bytes per data value.

         \sstitem
         For simple or group format FITS:

         \ssthitemlist{

            \sstitem
            IEEE floating point is supported.

            \sstitem
            If BUNIT is present its value will appear as the NDF's
            \htmlref{UNITS component}{apndf:units}.

            \sstitem
            If OBJECT is present its value will appear as the NDF's
            \htmlref{TITLE component}{apndf:units}.

            \sstitem
            If the BLANK item is present in the header, undefined pixels
            are converted from the BLANK value to Starlink-standard
            \htmlref{bad value}{se:masking} during data conversion.

            \sstitem
            An \htmlref{AXIS}{apndf:axis}~ component will be stored in the NDF if the CRVAL\textit{n}
            keyword is present.  (\textit{n} is the number of the dimension.)  If the
            CRPIX\textit{n} keyword is absent it defaults to 1, and likewise for the
            CDELT\textit{n} keyword.  The value of CTYPE\textit{n} is made the label of the
            axis structure.
         }

         \sstitem
         For groups format, a new NDF is created for each data array.
         The name of the NDF of the second and subsequent data arrays is
         generated by the application as the \texttt{<filename>G<number>}, where
         \texttt{<filename>} is the name of the first NDF, you supply or
         generated automatically, and \texttt{<number>} is the number of the group.

         Each group NDF contains the full header in the FITS extension,
         appended by the set of group parameters.  The group parameters
         are evaluated using their scales and offsets, and made to look
         like FITS cards, whose keywords are derived from the values of
         PTYPE\textit{m} in the main header.  (\textit{m} is the number of the group
         parameter.) The same format is used in the log file.

         \sstitem
         If there is no data array in the FITS file, \emph{i.e.} the FITS file comprises
         header cards only, then a dummy vector data array of dimension
         two is created to make the output a valid NDF.  This data array
         is undefined.
      }
   }
}
\sstroutine{
   FITSEDIT
}{
   Edits the FITS extension of an NDF
}{
   \sstdescription{
      This procedure allows you to use your favourite editor to
      modify the \FITSref\ ~headers stored in an \NDFref{NDF's}
      \htmlref{FITS extension}{se:fitsairlock}.
      There is limited validation of the FITS headers after editing.
      A FITS extension is created if the NDF does not already have
      one.
   }
   \sstusage{
      fitsedit ndf
   }
   \sstparameters{
      \sstsubsection{
         NDF = NDF (Read)
      }{
         The name of the NDF whose FITS extension is to be edited.
      }
   }
   \sstexamples{
      \sstexamplesubsection{
         fitsedit m51b
      }{
         This allows editing of the FITS headers in the NDF called m51b.
      }
   }
   \sstnotes{
      \sstitemlist{

         \sstitem
         This uses the environmental variable, EDITOR, to select
         the editor.  If this variable is undefined vi is assumed.

         \sstitem
         The script lists the headers to a temporary file; allows text
         editing; and then replaces the former FITS extension with the
         modified version, performing some validation at this stage.
      }
   }
   \sstdiytopic{
      Related Applications
   }{
KAPPA: \htmlref{FITSMOD}{FITSMOD},
\htmlref{FITSEXP}{FITSEXP},
\htmlref{FITSHEAD}{FITSHEAD},
\htmlref{FITSIMP}{FITSIMP},
\htmlref{FITSLIST}{FITSLIST};
\xref{FIGARO}{sun86}{}: \xref{FITSKEYS}{sun86}{FITSKEYS}.
   }
}
\sstroutine{
   FITSEXIST
}{
   Inquires whether or not a keyword exists in a FITS extension
}{
   \sstdescription{
      This application reports whether or not a keyword exists in the
      \htmlref{FITS extension}{se:fitsairlock} of an \NDFref{NDF} file.
   }
   \sstusage{
      fitsexist ndf keyword
   }
   \sstparameters{
      \sstsubsection{
         KEYWORD = LITERAL (Read)
      }{
         The name of the keyword whose existence in the FITS extension
         is to be tested.  A name may be compound to handle hierarchical
         keywords, and it has the form keyword1.keyword2.keyword3
         \emph{etc.}\  The maximum number of keywords per \FITSref\ ~card is 20.
         Each keyword must be no longer than 8 characters, and be a
         valid FITS keyword comprising only alphanumeric characters,
         hyphen, and underscore.  Any lowercase letters are converted to
         uppercase and blanks are removed before comparison with the
         existing keywords.

         KEYWORD may have an occurrence specified in brackets \texttt{[]}
         following the name.  This enables testing for the existence of
         multiple occurrences.  Note that it is not normal to have
         multiple occurrences of a keyword in a FITS header, unless it
         is blank, COMMENT or HISTORY.  Any text between the brackets
         other than a positive integer is interpreted as the first
         occurrence.

         The suggested value is the current value.
      }
      \sstsubsection{
         NDF = NDF (Read)
      }{
         The NDF to be tested for the presence of the FITS keyword.
      }
   }
   \sstresparameters{
      \sstsubsection{
         EXISTS = \_LOGICAL (Write)
      }{
        The result of the final existence test.
      }
   }
   \sstexamples{
      \sstexamplesubsection{
         fitsexist abc bscale
      }{
         This reports \texttt{TRUE} or \texttt{FALSE} depending on whether or not the
         FITS keyword BSCALE exists in the FITS extension of the NDF
         called abc.
      }
      \sstexamplesubsection{
         fitsexist ndf=abc keyword=date[2]
      }{
         This reports \texttt{TRUE} or \texttt{FALSE} depending on whether or not the
         FITS there are at least two occurrences of the keyword DATE.
      }
   }
   \sstdiytopic{
      Related Applications
   }{
KAPPA: \htmlref{FITSEDIT}{FITSEDIT},
\htmlref{FITSHEAD}{FITSHEAD},
\htmlref{FITSLIST}{FITSLIST},
\htmlref{FITSMOD}{FITSMOD},
\htmlref{FITSVAL}{FITSVAL}.
   }
}
\sstroutine{
   FITSEXP
}{
   Exports NDF-extension information into an NDF FITS extension
}{
   \sstdescription{
      This application places the values of components of an \NDFref{NDF}
      extension into the \htmlref{FITS extension}{se:fitsairlock}~ within the same NDF.  This
      operation is needed if auxiliary data are to appear in the header
      of a \FITSref\ ~file converted from the NDF.  The list of extension
      components whose values are to be copied, their corresponding
      FITS keyword names, optional FITS inline comments, and the
      location of the new FITS header are specified in a \emph{keyword
      translation table}~ held in a separate text file.
   }
   \sstusage{
      fitsexp ndf table
   }
   \sstparameters{
      \sstsubsection{
         NDF = NDF (Read and Write)
      }{
         The NDF in which the extension data are to be exported to
         the FITS extension.
      }
      \sstsubsection{
         TABLE = FILE (Read)
      }{
         The text file containing the keyword translation table.  The
         format of this file is described under
         \htmlref{``Table Format''}{table_format:fitsexp}.
      }
   }
   \sstexamples{
      \sstexamplesubsection{
         fitsexp datafile fitstable.txt
      }{
         This writes new FITS-extension elements for the NDF called
         datafile, creating the FITS extension if it does not exist.
         The selection of auxiliary components to export to the FITS
         extension, their keyword names, locations, and comments
         are under the control of a keyword translation table held in
         the file \texttt{fitstable.txt}.
      }
   }
   \sstnotes{
      \sstitemlist{

         \sstitem
         Requests to assign values to the following reserved keywords
         in the FITS extension are ignored: SIMPLE, BITPIX, NAXIS, NAXISn,
         EXTEND, PCOUNT, GCOUNT, XTENSION, BLOCKED, and END.

         \sstitem
         Only scalar or one-element vector components may be
         transferred to the FITS extension.

         \sstitem
         The data type of the component selects the type of the FITS
         value.

         \sstitem
         If the destination keyword exists, the existing value and
         comment are replaced with the new values.

         \sstitem
         If an error is found within a line, processing continues
         to the next line and the error reported.

         \sstitem
         To be sure that the resultant FITS extension is what you
         desired, you should inspect it using the command \htmlref{FITSLIST}{FITSLIST}
         before exporting the data.  If there is something wrong, you may find it
         convenient to use command fitsedit to make minor corrections.
      }
   }
   \sstdiytopic{
      Timing
   }{
      Approximately proportional to the number of FITS keywords to be
      translated.
   }
   \label{table_format:fitsexp}
   \sstdiytopic{
      Table Format
   }{
      The keyword translation table should be held in a text file, with
      one extension component specified per line.  Each line should
      contain two or three fields, separated by spaces and/or tabs, as
      follows.

      \sstitemlist{

         \sstitem
         Field 1:
            The name of the input extension component whose value is to be
            copied to the \htmlref{FITS extension}{se:fitsairlock}.  For example, \texttt{CCDPACK.FILTER}
            would copy the value of the component called FILTER in the
            extension called CCDPACK; and \linebreak
            \texttt{IRAS90.ASTROMETRY.EQUINOX} would
            copy the value of component EQUINOX in the structure
            ASTROMETRY in the extension IRAS90.  The extension may not be
            FITS.

         \sstitem
         Field 2:
            The name of the FITS keyword to which the value is to be
            copied.  Hierarchical keywords are not permissible.  The
            keyword name may be followed by a further keyword name in
            parentheses (and no spaces).  This second keyword defines the
            card before which the new keyword is to be placed.  If this
            second keyword is not present in the FITS extension or is not
            supplied, the new header card is placed at the end of the
            existing cards, but immediately before any END card.  For
            example, \texttt{EQUINOX(EPOCH)} would write the keyword EQUINOX
            immediately before the existing card with keyword EPOCH.  FITS
            keywords are limited to 8 characters and may only comprise
            uppercase alphabetic characters, digits, underscore, and
            hyphen.  While it is possible to have multiple occurrences of
            the same keyword in a FITS header, it is regarded as bad practice.  For this and efficiency reasons, this programme
            only looks for the first appearance of a keyword when
            substituting the values, and so only the last value inserted
            appears in the final FITS extension.  (See \texttt{"Implementation
            Status"}.)

         \sstitem
         Field 3:
            The comment to appear in the FITS header card for the chosen
            keyword.  This field is optional.  As much of the comment will
            appear in the header card as the value permits up to a maximum
            of 47 characters.

      }
      Comments may appear at any point in the table and should begin
      with an exclamation mark.  The remainder of the line will then be
      ignored.
   }
   \sstdiytopic{
      References
   }{
      \begin{refs}
      \item {\em "A User's Guide for the Flexible Image Transport System (FITS)"},
      NASA/Science Office of Science and Technology (1994).

      \end{refs}
   }
   \sstdiytopic{
      Related Applications
   }{
KAPPA: \htmlref{FITSEDIT}{FITSEDIT},
\htmlref{FITSHEAD}{FITSHEAD},
\htmlref{FITSLIST}{FITSLIST},
\htmlref{FITSMOD}{FITSMOD};
\xref{CONVERT}{sun55}{}: \xref{NDF2FITS}{sun55}{NDF2FITS}.
   }
   \sstimplementationstatus{
      \sstitemlist{

         \sstitem
         The replacements are made in blocks of 32 to reduce the number
         of time-consuming shuffles of the FITS extension.  Thus it is
         possible to locate a new keyword before another keyword, provided
         the latter keyword appears in an earlier block, though reliance
         on this feature is discouraged; instead run the application
         twice.

         \sstitem
         For each block the application inserts new cards or relocates
         old ones, marking each with different tokens, and then sorts the
         FITS extension into the requested order, removing the relocated
         cards.  It then inserts the new values.  If there are multiple
         occurrences of a keyword, this process can leave behind cards
         having the token value \texttt{'\{undefined\}'}.
      }
   }
}

\sstroutine{
   FITSHEAD
}{
   Lists the headers of FITS files
}{
   \sstdescription{
      This procedure lists to standard output the headers of the primary
      header and data unit, and any extensions present that are
      contained within a set of input \FITSref\ ~files, or a range of
      specified files on a tape.
   }
   \sstusage{
      fitshead file [block] [start] [finish]
   }
   \sstparameters{
      \sstsubsection{
         BLOCK = \_INTEGER (Read)
      }{
         The FITS blocking factor of the tape to list.  This is the tape
         blocksize in bytes divided by the FITS record length of 2880
         bytes.  BLOCK must be a positive integer, between 1 and 12,
         otherwise you will be prompted for a new value.  Should the first
         argument not be a tape device, this argument will be treated as
         a file name.  \texttt{[1]}
      }
      \sstsubsection{
         FILE = FILENAME (Read)
      }{
         A space-separated list of FITS files whose headers are to be
         listed, or the name of a single no-rewind tape device.  The list
         of files can include wildcard characters.
      }
      \sstsubsection{
         FINISH = \_INTEGER (Read)
      }{
         The last file on the tape to list.  This defaults to the end
         of the tape.  It must be a positive integer and at least equal
         to the value of start, otherwise you will be prompted for a new
         value.  Should the first argument not be a tape device, this
         argument will be treated as a file name.  \texttt{[]}
      }
      \sstsubsection{
         START = \_INTEGER (Read)
      }{
         The first file on the tape to list.  This defaults to 1, \emph{i.e.}
         the start of the tape.  It must be a positive integer,
         otherwise you will be prompted for a new value.  Should the
         first argument not be a tape device, this argument will be
         treated as a file name.  \texttt{[1]}
      }
   }
   \sstexamples{
      \sstexamplesubsection{
         fitshead /dev/nrmt1
      }{
         This lists the FITS headers for all the files of the tape mounted
         on device \texttt{/dev/nrmt1}.  The tape block size is 2880 bytes.
      }
      \sstexamplesubsection{
         fitshead /dev/nrmt1 10 $>$ tape.lis
      }{
         This lists to file \texttt{tape.lis} the FITS headers for all the files of
         the tape mounted on device \texttt{/dev/nrmt1}.  The tape blocking factor is
         10, the tape's blocksize is 28800 bytes.
      }
      \sstexamplesubsection{
         fitshead /dev/rmt/0n 2 3 5 $>$$>$ tape.lis
      }{
         This appends the FITS headers for files 3 to 5 of the tape mounted
         on device \texttt{/dev/rmt/0n} to the file \texttt{tape.lis}.  The tape blocking factor
         is 2, \emph{i.e.} the tape's blocksize is 5760 bytes.
      }
      \sstexamplesubsection{
         fitshead /dev/nrst2 prompt
      }{
         This lists the FITS headers for files of the tape mounted on
         device \texttt{/dev/nrst2}.  The command prompts you for the file limits
         and tape blocking factor.
      }
      \sstexamplesubsection{
         fitshead $\sim$/fits/$*$.fit $\sim$/data/p?.fi$*$ $|$ lpr
      }{
         This prints the FITS headers in the files \texttt{$\sim$/fits/$*$.fit}
         and \texttt{$\sim$/data/p?.fi$*$}.
      }
   }
   \sstnotes{
      \sstitemlist{

         \sstitem
         Prompting is directed to the standard error, so that the listings
         may be redirected to a file.

         \sstitem
         If the blocking factor is unknown it is possible to obtain only
         a part of the headers and some of the FITS data.  Unless the FITS
         file is small, it is usually safe to set Parameter BLOCK higher
         than its true value.
      }
   }
   \sstdiytopic{
      Related Applications
   }{
KAPPA: \htmlref{FITSEDIT}{FITSEDIT},
\htmlref{FITSEXP}{FITSEXP},
\htmlref{FITSIMP}{FITSIMP},
\htmlref{FITSLIST}{FITSLIST};
\xref{FIGARO}{sun86}{}: \xref{FITSKEYS}{sun86}{FITSKEYS}.
   }
}

\sstroutine{
   FITSIMP
}{
   Imports FITS information into an NDF extension
}{
   \sstdescription{
      This application extracts the values of \FITSref\ ~keywords from a
      \htmlref{FITS extension}{se:fitsairlock}~ in an \NDFref{NDF} and uses them to construct another NDF
      extension.  The list of new extension components required, their
      data types and the names of the FITS keywords from which to
      derive their values are specified in a \emph{keyword translation
      table}~ held in a separate text file.
   }
   \sstusage{
      fitsimp ndf table xname xtype
   }
   \sstparameters{
      \sstsubsection{
         NDF = NDF (Read and Write)
      }{
         The NDF in which the new extension is to be created.
      }
      \sstsubsection{
         TABLE = FILENAME (Read)
      }{
         The text file containing the keyword translation table.  The
         format of this file is described under
         \htmlref{``Table Format''}{table_format:fitsimp}.
      }
      \sstsubsection{
         XNAME = LITERAL (Read)
      }{
         The name of the NDF extension which is to receive the values
         read from the FITS extension .  If this extension does not
         already exist, then it will be created.  Otherwise, it should
         be a scalar structure extension within which new components
         may be created (existing components of the same name will be
         over-written).  Extension names may contain up to 15
         alpha-numeric characters, beginning with an alphabetic
         character.
      }
      \sstsubsection{
         XTYPE = LITERAL (Read)
      }{
         The \htmlref{HDS data type}{ap:HDStypes}~ of the output extension.  This value will
         only be required if the extension does not initially exist and
         must be created.  New extensions will be created as scalar
         structures.
      }
   }
   \sstexamples{
      \sstexamplesubsection{
         fitsimp datafile fitstable ccdinfo ccd\_ext
      }{
         Creates a new extension called CCDINFO (with a data type of
         CCD\_EXT) in the NDF structure called datafile.  Keyword values
         are read from the NDF's FITS extension and written into the new
         extension as separate components under control of a keyword
         translation table held in the file \texttt{fitstable}.
      }
      \sstexamplesubsection{
         fitsimp ndf=n1429 table=std\_table xname=std\_extn
      }{
         FITS keyword values are read from the FITS extension in the
         NDF structure n1429 and written into the pre-existing
         extension STD\_EXTN under control of the translation table
         \texttt{std\_table}.  Components which already exist within the
         extension may be over-written by this process.
      }
   }
   \sstdiytopic{
      Timing
   }{
      Approximately proportional to the number of FITS keywords to be
      translated.
   }
   \label{table_format:fitsimp}
   \sstdiytopic{
      Table Format
   }{
      The keyword translation table should be held in a text file, with
      one extension component specified per line.  Each line should
      contain 3 fields, separated by spaces and/or tabs, as follows.

      \sstitemlist{

         \sstitem
         Field 1:
            The name of the component in the output extension for which a
            value is to be obtained.

         \sstitem
         Field 2:
            The data type of the output component, to which the keyword
            value will be converted (one of \texttt{\_INTEGER}, \texttt{\_REAL},
            \texttt{\_DOUBLE}, \texttt{\_LOGICAL} or \texttt{\_CHAR}).

         \sstitem
         Field 3:
            The name of the FITS keyword from which the value is to be
            obtained.  Hierarchical keywords are permissible; the format
            is concatenated keywords joined with full stops and no spaces,
            \emph{e.g.} \texttt{HIERARCH.ESO.NTT.HUMIDITY}, \texttt{ING.DETHEAD}.

      }
      Comments may appear at any point in the table and should begin
      with an exclamation mark.  The remainder of the line will then be
      ignored.
   }
   \sstdiytopic{
   Related Applications
   }{
KAPPA: \htmlref{FITSHEAD}{FITSHEAD},
\htmlref{FITSLIST}{FITSLIST},
\htmlref{FITSDIN}{FITSDIN},
\htmlref{FITSIN}{FITSIN};
\xref{CONVERT}{sun55}{}: \xref{FITS2NDF}{sun55}{FITS2NDF};
\xref{FIGARO}{sun86}{}: \xref{RDFITS}{sun86}{RDFITS}.
   }
}
\sstroutine{
   FITSIN
}{
   Reads a FITS tape composed of simple, group or table files
}{
   \sstdescription{
      This application reads selected files from a \FITSref\ ~tape.  The
      files may be Basic (simple) FITS, and/or have TABLE extensions
      (Harten {\it et al.}\ 1988).

      The programme reads a simple or a random-groups-format FITS file
      (Wells {\it et al.}\ 1981; Greisen \& Harten 1981), and writes the
      data into an \NDFref{NDF}, and the headers into the NDF's \htmlref{FITS extension}{se:fitsairlock}.
      Table-format files (Grosb{\o}l {\it et al.}\ 1988) are read, and the
      application creates two files: a text formatted table/catalogue
      and a FACTS description file (as used by SCAR) based upon the FITS
      header cards.  Composite FITS files can be processed.  You may
      specify a list of files, including wildcards.  A record of the
      FITS headers, and group parameters (for a group-format file) can
      be stored in a text file.

      There is an option to run in automatic mode, where the names of
      output NDF data structures are generated automatically, and you
      can decide whether or not format conversion is to be applied to
      all files (rather than being prompted for each).  This is very
      useful if there is a large number of files to be processed.  Even
      if you want unique file names, format-conversion prompting may be
      switched off globally.
   }
   \sstusage{
      fitsin mt files out [auto] fmtcnv [logfile] more=? dscftable=? table=?
   }
   \sstparameters{
      \sstsubsection{
         AUTO = \_LOGICAL (Read)
      }{
         It is \texttt{TRUE} if automatic mode is required, where the name of
         each output NDF structure or table file is to be generated by the
         application, and therefore not prompted; and a global
         format-conversion switch may be set.  In manual mode the
         FITS header is reported, but not in automatic.

         For simple or group format FITS objects in automatic mode the
         application generates a filename beginning with a defined
         prefix followed by the number of the file on tape.  For
         example, if the prefix was \texttt{"XRAY"} and the 25$^{\textrm{th}}$ file of the
         tape was being processed, the filename of the NDF would be
         XRAY25.

         For table-format FITS objects in the automatic mode the
         application generates a filename beginning with a defined
         prefix followed by the number of the file on tape.  For
         example, if the prefix was \texttt{"cat"} and the 9$^{\textrm{th}}$ file of the tape
         was being processed, the filename of the table and its
         associated FACTS description file would be \texttt{cat9.dat} and
         \texttt{dscfcat9.dat} respectively.
         \texttt{[FALSE]}
      }
      \sstsubsection{
         DSCFTABLE = FILENAME (Read)
      }{
         Name of the text file to contain the FACTS descriptors, which
         defines the table's format for {\footnotesize SCAR}.  Since {\footnotesize SCAR} is now
         deprecated, this parameter has little use, except perhaps to
         give a summary of the format of the file specified by Parameter
         TABLE.  A null value (\texttt{{!}}) means that no description file will
         be created, so this is now the recommended usage.  If your
         FITS file comprises just tables, you should consider other
         tools such as the \CURSAref\  package, which has facilities for
         examining and processing ASCII and binary tables in FITS files.

         A suggested filename for the description file is reported
         immediately prior to prompting in manual mode.  It is the name
         of the catalogue, as written in the FITS header, with a
         \texttt{"dscf"} prefix.
      }
      \sstsubsection{
         ENCODINGS = \htmlref{LITERAL}{se:parmenu} (Read)
      }{
         Determines which FITS keywords should be used to define the
         world co-ordinate systems to be stored in the NDF's \htmlref{WCS}{apndf:wcs}
         component.  The allowed values (case-insensitive) are as follows.

         \ssthitemlist{

         \sstitem
            \texttt{"FITS-IRAF"} --- This uses keywords CRVAL\textit{i}
            CRPIX\textit{i}, CD\textit{i\_j}, and is the system commonly
            used by IRAF.  It is described in the document ``World
            Coordinate Systems Representations Within the FITS Format''
            by R.J.~Hanisch and D.G.~Wells, 1988, available by ftp from
            fits.cv.nrao.edu \texttt{/fits/documents/wcs/wcs88.ps.Z}.

         \sstitem
            \texttt{"FITS-WCS"} --- This is the FITS standard WCS
            encoding scheme described in the paper ``Representation of
            celestial coordinates in FITS'' \newline
            (\htmladdnormallink{\texttt{{http://www.cv.nrao.edu/fits/documents/wcs/wcs.html}}}
            {http://www.cv.nrao.edu/fits/documents/wcs/wcs.html}). \newline

            It is very similar to FITS-IRAF but supports a wider range of
            projections and co-ordinate systems.

         \sstitem
            \texttt{"FITS-PC"} --- This uses keywords CRVAL\textit{i},
            CDELT\textit{i}, CRPIX\textit{i}, PC\textit{iiijjj}, \emph{etc},
            as in a previous (now superceded) draft of the above
            FITS world co-ordinate system paper by E.W.~Greisen and
            M.~Calabretta.

         \sstitem
            \texttt{"FITS-AIPS"} --- This uses conventions described in the
            document ``Non-linear Coordinate Systems in AIPS'' by Eric W.
            Greisen (revised 9th September, 1994), available by ftp from
            fits.cv.nrao.edu \texttt{/fits/documents/wcs/aips27.ps.Z}.  It is
            currently employed by the AIPS data analysis facility, so its
            use will facilitate data exchange with AIPS.  This encoding
            uses CROTA\textit{i} and CDELT\textit{i} keywords to describe
            axis rotation and scaling.

         \sstitem
            \texttt{"DSS"} --- This is the system used by the Digital Sky Survey,
            and uses keywords AMDX\textit{n}, AMDY\textit{n}, PLTRAH,
            \emph{etc}.

         \sstitem
            \texttt{"Native"} --- This is the native system used by the AST
            library (see \xref{SUN/210}{sun210}{}), and provides a loss-free
            method for transferring WCS information between AST-based
            application.  It allows more complicated WCS information to be
            stored and retrieved than any of the other encodings.
         }

         A comma-separated list of up to six values may be supplied, in
         which case the value actually used is in the first in the list
         for which corresponding keywords can be found in the FITS
         header.

         A FITS header may contain keywords from more than one of these
         encodings, in which case it is possible for the encodings to be
         inconsistent with each other.  This may happen for instance if
         an application modifies the keyword associated with one encoding
         but fails to make equivalent modifications to the others.  If a
         null parameter value (\texttt{{!}}) is supplied for ENCODINGS, then an
         attempt is made to determine the most reliable encoding to use
         as follows.  If both native and non-native encodings are
         available, then the first non-native encoding to be found which
         is inconsistent with the native encoding is used.  If all
         encodings are consistent, then the native encoding is used (if
         present).  \texttt{[!]}
      }
      \sstsubsection{
         FILES() = \_CHAR (Read)
      }{
         The list of the file numbers to be processed.  Files are
         numbered consecutively from 1 from the start of the tape.
         Single files or a set of adjacent files may be specified,
         \emph{e.g.} typing \texttt{[4,6-9,12,14-16]} will read files
         4,6,7,8,9,12,14,15,16.  (Note that the
         brackets are required to distinguish this array of characters
         from a single string including commas.  The brackets are
         unnecessary when there only one item.)  For efficiency reasons
         it is sensible to give the file numbers in ascending order.

         If you wish to extract all the files enter the wildcard \texttt{$*$}.
         \texttt{5-$*$} will read from 5 to the last file.  The processing will
         continue until the end of the tape is reached; no error
         will result from this.
      }
      \sstsubsection{
         FMTCNV = \_LOGICAL (Read)
      }{
         This specifies whether or not format conversion will occur.
         If \texttt{FALSE}, the HDS type of the data array in the NDF will be
         the equivalent of the FITS data format on tape (\emph{e.g.} BITPIX=\texttt{16}
         creates a \_WORD array).  If \texttt{TRUE}, the data array in the
         current file, or all files in automatic mode, will be
         converted from the FITS data type on tape to \_REAL in the NDF.
         The conversion applies the values of the FITS keywords BSCALE
         and BZERO to the tape data to generate the `true' data values.
         If BSCALE and BZERO are not given in the FITS header, they are
         taken to be 1.0 and 0.0 respectively.  The suggested default
         is \texttt{TRUE}.
      }
      \sstsubsection{
         GLOCON = \_LOGICAL (Read)
      }{
         If \texttt{FALSE}, a format-conversion query occurs for each FITS file.
         If \texttt{TRUE}, the value of FMTCNV is obtained before any file
         numbers and will apply to all data arrays.  It is ignored
         in automatic mode---in effect it becomes \texttt{TRUE}.  \texttt{[FALSE]}
      }
      \sstsubsection{
         LABEL = \_LOGICAL (Read)
      }{
         It should be \texttt{TRUE} if the tape has labelled files.
         Labelled files are non-standard.  If \texttt{TRUE}, the application
         skips three file marks per file, rather that one.  \texttt{[FALSE]}
      }
      \sstsubsection{
         LOGFILE = FILENAME (Read)
      }{
         The file name of the text log of the FITS header cards.
         For group-format data the group parameters are evaluated
         and appended to the full header.  The log includes the names of
         the output files used to store the data array or table.  A null
         value (\texttt{{!}}) means that no log file is produced.  \texttt{[!]}
      }
      \sstsubsection{
         MORE = \_LOGICAL (Read)
      }{
         A prompt asking if any more files are to be processed once the
         current list has been exhausted.
      }
      \sstsubsection{
         MT = \htmlref{DEVICE}{se:readfitstape} (Read)
      }{
         Tape deck containing the data, usually an explicit device,
         though it can be a pre-assigned environment variable.
      }
      \sstsubsection{
         OUT = NDF (Write)
      }{
         Output NDF structure holding the full contents of the FITS
         file.  If the null value (\texttt{{!}}) is given no NDF will be created.
         This offers an opportunity to review the descriptors before
         deciding whether or not the data are to be extracted.
      }
      \sstsubsection{
         PREFIX = LITERAL (Read)
      }{
         The prefix of the NDF's or table's file name.  It is only used
         in the automatic mode.
      }
      \sstsubsection{
         REWIND = \_LOGICAL (Read)
      }{
         If it is \texttt{TRUE}, the tape drive is rewound before the reading of
         the FITS files commences.  If it is \texttt{FALSE}, the tape is not
         rewound, and the current tape position is read from file
         \texttt{USRDEVDATASET.sdf}.  Note that file numbers are absolute and
         not relative.  REWIND=\texttt{FALSE} is useful if you need to read a
         series of files, process them, then read some more, without
         having to remember the tape's position or apply unnecessary
         wear to the tape.  \texttt{[TRUE]}
      }
      \sstsubsection{
         TABLE = FILENAME (Read)
      }{
         Name of the text file to contain the table itself, read from
         the file.  In manual mode, the suggested default filename is the name of
         description file less the \texttt{"dscf"} prefix, or if there is no
         description file or if the description file does not have the
         \texttt{"dscf"} prefix, the suggested name reverts to the catalogue name
         in the FITS header.
      }
   }
   \sstexamples{
      \sstexamplesubsection{
         fitsin mt=/dev/rmt/1n files=[2-4,9] auto prefix=ccd nofmtcnv
      }{
         This reads files 2, 3, 4, and 9 from the FITS tape on
         device \texttt{/dev/rmt/1n}.  The output NDF names will be ccd2, ccd3, ccd4,
         and ccd9 (assuming there are no groups).  The data will not
         have format conversion.
      }
      \sstexamplesubsection{
         fitsin mt=\$TAPE files=$*$ auto prefix=ccd fmtcnv logfile=jkt.log
      }{
         This reads all the files from the FITS tape on the device
         assigned to the environment variable \texttt{TAPE}.  The output files
         begin with a prefix \texttt{"ccd"}.  Integer data
         arrays are converted to real using the scale and zero
         found in the FITS header.  A record of the headers and the
         names of the output files are written to the text file
         \texttt{jkt.log}.
      }
   }
   \sstdiytopic{
      References
   }{
      \begin{refs}
      \item  Wells, D.C., Greisen, E.W. \& Harten, R.H. 1981,
      {\em Astron.  Astrophys.  Suppl.  Ser.} {\bf 44}, 363.

      \item  Greisen, E.W. \& Harten, R.H. 1981,
      {\em Astron.  Astrophys.  Suppl.  Ser.} {\bf 44}, 371.

      \item  Grosb{\o}l, P., Harten, R.H., Greisen, E.W \& Wells, D.C.
      1988 {\em Astron.  Astrophys.  Suppl.  Ser.} {\bf 73}, 359.

      \item  Harten, R.H., Grosb{\o}l, P., Greisen, E.W \& Wells, D.C.
      1988 {\em Astron.  Astrophys.  Suppl.  Ser.} {\bf 73}, 365.

      \end{refs}
   }
   \sstdiytopic{
      Related Applications
   }{
KAPPA: \htmlref{FITSDIN}{FITSDIN},
\htmlref{FITSHEAD}{FITSHEAD},
\htmlref{FITSIMP}{FITSIMP},
\htmlref{FITSLIST}{FITSLIST};
\xref{CONVERT}{sun55}{}: \xref{FITS2NDF}{sun55}{FITS2NDF};
\CURSAref;
\xref{FIGARO}{sun86}{}: \xref{RDFITS}{sun86}{RDFITS}.
   }
   \sstimplementationstatus{
      \sstitemlist{

         \sstitem
         The application processes tapes blocked at other than an
         integer multiple of 2880 bytes up to a maximum of 63360, provided
         it is a multiple of the number of bytes per data value.

         \sstitem
         For simple or group format FITS:

         \ssthitemlist{

            \sstitem
            IEEE floating point is supported.

            \sstitem
            If BUNIT is present its value will appear as the NDF's
            \htmlref{UNITS component}{apndf:units}.

            \sstitem
            If OBJECT is present its value will appear as the NDF's
            \htmlref{TITLE component}{apndf:title}.

            \sstitem
            If the BLANK item is present in the header, undefined pixels
            are converted from the BLANK value to Starlink-standard bad value during data conversion.

            \sstitem
            An \htmlref{AXIS}{apndf:axis}~ component will be stored in the NDF if the CRVAL\textit{n}
            keyword is present.  (\textit{n} is the number of the dimension.)  If the
            CRPIX\textit{n} keyword is absent it defaults to 1, and likewise for the
            CDELT\textit{n} keyword.  The value of CTYPE\textit{n} is made the label of the
            axis structure.
         }

         \sstitem
         For groups format, a new NDF is created for each data array.
         The name of the NDF of the second and subsequent data arrays is
         generated by the application as the \texttt{<filename>G<number>}, where
         \texttt{<filename>} is the name of the first NDF, supplied by you or
         generated automatically, and \texttt{<number>} is the number of the group.

         Each group NDF contains the full header in the FITS extension,
         appended by the set of group parameters.  The group parameters
         are evaluated using their scales and offsets, and made to look
         like FITS cards, whose keywords are derived from the values of
         PTYPE\textit{m} in the main header.  (\textit{m} is the number of the group
         parameter.) The same format is used in the log file.

         \sstitem
         If there is no data array on tape, \emph{i.e.} the FITS file comprises
         header cards only, then a dummy vector data array of dimension
         two is created to make the output a valid NDF.  This data array
         is undefined.
      }
   }
}

\sstroutine{
   FITSLIST
}{
   Lists the FITS extension of an NDF
}{
   \sstdescription{
      This application lists the \FITSref\ ~header stored in an \NDFref{NDF}
      \htmlref{FITS extension}{se:fitsairlock}.  The list may either be reported
      directly to you, or written to a text file.  The displayed list of headers
      can be augmented, if required, by the inclusion of FITS headers
      representing the current World Co-ordinate System defined by the
      WCS component in the NDF (see Parameter ENCODING).
   }
   \sstusage{
      fitslist in [logfile]
   }
   \sstparameters{
      \sstsubsection{
         ENCODING = LITERAL (Read)
      }{
         If a non-null value is supplied, the NDF WCS component is used
         to generate a set of FITS headers describing the WCS, and these
         headers are added into the displayed list of headers (any WCS
         headers inherited from the FITS extension are first removed).  The
         value supplied for ENCODING controls the FITS keywords that will
         be used to represent the WCS.  The value supplied should be one of
         the encodings listed in the ``World Co-ordinate Systems'' section
         below.  An error is reported if the WCS cannot be represented using
         the supplied encoding.  A trailing minus sign appended to the
         end of the encoding indicates that only the WCS headers should be
         displayed (that is, the contents of the FITS extension are not
         displayed if the encoding ends with a minus sign).  Also see the
         FULLWCS parameter. \texttt{[!]}
      }
      \sstsubsection{
         FULLWCS = \_LOGICAL (Read)
      }{
         Only accessed if ENCODING is non-null. If \texttt{TRUE} then all
         co-ordinate frames in the WCS component are written out.
         Otherwise, only the current Frame is written out. \texttt{[FALSE]}
      }
      \sstsubsection{
         IN = NDF (Read)
      }{
         The NDF whose FITS extension is to be listed.
      }
      \sstsubsection{
         LOGFILE = FILENAME (Read)
      }{
         The name of the text file to store a list of the FITS
         extension.  If it is null (\texttt{{!}}) the list of the FITS extension
         is reported directly to you.  \texttt{[!]}
      }
   }
   \sstexamples{
      \sstexamplesubsection{
         fitslist saturn
      }{
         The contents of the FITS extension in NDF saturn are
         reported to you.
      }
      \sstexamplesubsection{
         fitslist saturn fullwcs encoding=fits-wcs
      }{
         As above but it also lists the standard FITS world-co-ordinate
         headers derived from saturn's WCS component, provided such
         information exists.
      }
      \sstexamplesubsection{
         fitslist saturn fullwcs encoding=fits-wcs-
      }{
         As the previous example except that it only lists the standard
         FITS world-co-ordinate headers derived from saturn's WCS
         component.  The headers in the FITS extension are not listed.
      }
      \sstexamplesubsection{
         fitslist ngc205 logfile=ngcfits.lis
      }{
         The contents of the FITS extension in NDF ngc205 are
         written to the text file \texttt{ngcfits.lis}.
      }
   }
   \sstnotes{
      \sstitemlist{

         \sstitem
         If the NDF does not have a FITS extension the application will
         exit.
      }
   }
   \sstdiytopic{
      World Co-ordinate Systems
   }{
      The ENCODING parameter can take any of the following values.

      \ssthitemlist{

         \sstitem
            \texttt{"FITS-IRAF"} --- This uses keywords CRVAL\textit{i}
            CRPIX\textit{i}, CD\textit{i\_j}, and is the system commonly
            used by IRAF.  It is described in the document ``World
            Coordinate Systems Representations Within the FITS Format''
            by R.J.~Hanisch and D.G.~Wells, 1988, available by ftp from
            fits.cv.nrao.edu \texttt{/fits/documents/wcs/wcs88.ps.Z}.

         \sstitem
            \texttt{"FITS-WCS"} --- This is the FITS standard WCS
            encoding scheme described in the paper ``Representation of
            celestial coordinates in FITS'' \newline
            (\htmladdnormallink{\texttt{{http://www.cv.nrao.edu/fits/documents/wcs/wcs.html}}}
            {http://www.cv.nrao.edu/fits/documents/wcs/wcs.html}).

            It is very similar to \texttt{"FITS-IRAF"} but supports a wider range of
            projections and co-ordinate systems.

         \sstitem
            \texttt{"FITS-WCS(CD)"} --- This is the same as \texttt{"FITS-WCS"} except
            that the scaling and rotation of the data array is described by
            CD matrix instead of a PC matrix with associated CDELT values.

         \sstitem
            \texttt{"FITS-PC"} --- This uses keywords CRVAL\textit{i},
            CDELT\textit{i}, CRPIX\textit{i}, PC\textit{iiijjj}, \emph{etc},
            as in a previous (now superseded) draft of the above
            FITS world co-ordinate system paper by E.W.~Greisen and
            M.~Calabretta.

         \sstitem
            \texttt{"FITS-AIPS"} --- This uses conventions described in the
            document ``Non-linear Coordinate Systems in AIPS'' by Eric W.
            Greisen (revised 9th September, 1994), available by ftp from
            fits.cv.nrao.edu \texttt{/fits/documents/wcs/aips27.ps.Z}.  It is
            currently employed by the AIPS data analysis facility, so its
            use will facilitate data exchange with AIPS.  This encoding
            uses CROTA\textit{i} and CDELT\textit{i} keywords to describe
            axis rotation and scaling.

         \sstitem
            \texttt{"FITS-AIPS++"} --- This is an extension to \texttt{"FITS-AIPS"} that
            allows the use of a wider range of celestial projections, and is
            used by the AIPS++ project.

         \sstitem
            \texttt{"FITS-CLASS"} --- This uses the conventions of the CLASS
            project.  CLASS is a software package for reducing single-dish
            radio and sub-mm spectroscopic data.  It supports double-sideband
            spectra.

            See \htmladdnormallink{\texttt{{http://www.iram.fr/IRAMFR/GILDAS/doc/html/class-html/class.html}}}
            {http://www.iram.fr/IRAMFR/GILDAS/doc/html/class-html/class.html}.

         \sstitem
            \texttt{"DSS"} --- This is the system used by the Digital Sky Survey,
            and uses keywords AMDX\textit{n}, AMDY\textit{n}, PLTRAH,
            \emph{etc}.

         \sstitem
            \texttt{"Native"} --- This is the native system used by the AST
            library (see \xref{SUN/210}{sun210}{}), and provides a loss-free
            method for transferring WCS information between AST-based
            application.  It allows more complicated WCS information to be
            stored and retrieved than any of the other encodings.
      }
   }
   \sstdiytopic{
      Related Applications
   }{
KAPPA: \htmlref{FITSEDIT}{FITSEDIT},
\htmlref{FITSHEAD}{FITSHEAD};
\xref{FIGARO}{sun86}{}: \xref{FITSKEYS}{sun86}{FITSKEYS}.
   }
}
\sstroutine{
   FITSMOD
}{
   Edits an NDF FITS extension via a text file or parameters
}{
   \sstdescription{
      This application edits the \htmlref{FITS extension}{se:fitsairlock} ~of
      an \NDFref{NDF} file in a
      variety of ways.  It permits insersion of new keywords, including
      comment lines; revision of existing keyword, values, and inline
      comments; relocation of keywords; deletion of keywords; printing
      of keyword values; and it can test whether or not a keyword
      exists.  The occurrence of keywords may be defined, when there
      are more than one cards of the same name.  The location of each
      insertion or move is immediately before some occurrence of a
      corresponding keyword.

      Control of the editing can be through parameters, or from a text
      file whose format is described in topic \texttt{"File Format"}.
   }
   \sstusage{
      fitsmod ndf
        $\left\{ {\begin{tabular}{l}
                  keyword edit value comment position \\
                  table=?
                  \end{tabular} }
        \right.$
        \newline\latexhtml{\hspace*{5.8em}}{~~~~~~~~~~}
        \makebox[0mm][c]{\small mode}
   }
   \sstparameters{
      \sstsubsection{
         COMMENT = LITERAL (Read)
      }{
         The comments to be written to the KEYWORD keyword for the
         \texttt{"Update"}, \texttt{"Write"}, and \texttt{"Amend"}editing commands.
         A null value (\texttt{{!}}) gives a blank comment.  The special value
         \texttt{"\$C"} means use the current comment.  In addition
         \texttt{"\$C(}keyword\texttt{)"} requests that the
         comment of the keyword given between the parentheses be
         assigned to the keyword being edited.  If this positional
         keyword does not exist, the comment is unchanged for \texttt{"Update"},
         and is blank for a \texttt{"Write"} edit.  The same applies to the
         \texttt{"Amend"} edit, the choice depending on whether or not the
         KEYWORD keyword exists.
      }
      \sstsubsection{
         EDIT = LITERAL (Read)
      }{
         The editing command to apply to the keyword.  The allowed
         options are listed below.

         \begin{description}
            \item \texttt{"Amend"} --- acts as option \texttt{"Update"} if the
            keyword exists, but as the \texttt{"Write"} option should the
            keyword be absent.

            \item \texttt{"Delete"} --- removes a named keyword.

            \item \texttt{"Exist"} --- reports \texttt{TRUE} to standard
            output if the named keyword exists in the header, and
            \texttt{FALSE} if the keyword is not present.

            \item \texttt{"Move"} --- relocates a named keyword to be
            immediately before a second keyword (see Parameter POSITION).
            When this positional keyword is not supplied, it defaults to the END
            card, and if the END card is absent, the new location is at the end
            of the headers.

            \item \texttt{"Null"} nullifies the value of the named keyword.
            Spaces substitute the keyword's value.

            \item \texttt{"Print"} --- causes the value of a named keyword to be
            displayed to standard output.  This will be a blank for a comment
            card.

            \item \texttt{"Rename"} --- renames a keyword, using Parameter
            NEWKEY to obtain the new keyword.

            \item \texttt{"Update"} --- revises the value and/or the comment.
            If a secondary keyword is defined explicitly (Parameter
            POSITION), the card may be relocated at the same time.  If
            the secondary keyword does not exist, the card being edited
            is not moved.  \texttt{"Update"} requires that the keyword
            being edited exists.

            \item \texttt{"Write"} --- creates a new card given a value and
            an optional comment.  Its location uses the same rules as for
            the \texttt{"Move"} command.  The FITS extension is created
            first should it not exist.
         \end{description}
      }
      \sstsubsection{
         KEYWORD = LITERAL (Read)
      }{
         The name of the keyword to be edited in the FITS extension.  A
         name may be compound to handle hierarchical keywords, and it
         has the form keyword1.keyword2.keyword3  \emph{etc.}\  The maximum
         number of keywords per FITS card is twenty.  Each keyword must be
         no longer than eight characters, and be a valid FITS keyword
         comprising only alphanumeric characters, hyphen, and underscore.
         Any lowercase letters are converted to uppercase and blanks
         are removed before insertion, or comparison with the existing
         keywords.

         The keywords \texttt{" "}, \texttt{"COMMENT"}, and \texttt{"HISTORY"}
         are comment cards and do not have a value.

         The keyword must exist except for the \texttt{"Amend"},
         \texttt{"Write"}, and \texttt{"Exist"} commands.

         Both KEYWORD and POSITION keywords may have an occurrence
         specified in brackets \texttt{[]} following the name. This enables
         editing of a keyword that is not the first occurrence of that
         keyword, or locate a edited keyword not at the first occurrence
         of the positional keyword.  Note that it is not normal to have
         multiple occurrences of a keyword in a FITS header, unless it
         is blank, COMMENT or HISTORY.  Any text between the brackets
         other than a positive integer is interpreted as the first
         occurrence.
      }
      \sstsubsection{
         MODE = \htmlref{LITERAL}{se:parmenu} (Read)
      }{
         The mode by which the editing instructions are supplied.  The
         alternatives are \texttt{"File"}, which uses a text file; and
         \texttt{"Interface"} which uses parameters.  \texttt{["Interface"]}
      }
      \sstsubsection{
         NDF = NDF (Read and Write)
      }{
         The NDF in which the FITS extension is to be edited.
      }
      \sstsubsection{
         NEWKEY = LITERAL (Read)
      }{
         The name of the keyword to replace the KEYWORD keyword.  It is
         only accessed when EDIT=\texttt{"Rename"}.  A name may be compound to
         handle hierarchical keywords, and it has the form
         keyword1.keyword2.keyword3  \emph{etc.}\  The maximum number of
         keywords per FITS card is twenty.  Each keyword must be no longer
         than eight characters, and be a valid FITS keyword comprising only
         alphanumeric characters, hyphen, and underscore.
      }
      \sstsubsection{
         POSITION = LITERAL (Read)
      }{
         The position keyword name.  A position name may be compound to
         handle hierarchical keywords, and it has the form
         keyword1.keyword2.keyword3 \emph{etc.}\  The maximum number of
         keywords per FITS card is twenty.  Each keyword must be no longer
         than eight characters.  When locating the position card,
         comparisons are made in uppercase and with the blanks removed.
         An occurrence may be specified (see Parameter KEYWORD for
         details).

         The new keywords are inserted immediately before each
         corresponding position keyword.  If any name in it does not
         exist in FITS array, or the null value (\texttt{{!}}) is supplied the
         consequences will be as follows.  For a \texttt{"Write"},
         \texttt{"Amend"} (new keyword), or
         \texttt{"Move"} edit, the KEYWORD keyword will be inserted just
         before the END card or appended to FITS array when the END card
         does not exist; for an \texttt{"Update"} or \texttt{"Amend"} (new
         keyword) edit, the edit keyword is not relocated.

         A positional keyword is only accessed by the \texttt{"Move"},
         \texttt{"Amend"}, \texttt{"Write"}, and \texttt{"Update"} editing commands.
      }
      \sstsubsection{
         READONLY = \_LOGICAL (Read)
      }{
         Determines if read or write access is requested for the NDF. If
         a \texttt{TRUE} value is supplied for READONLY, the NDF is opened for
         reading only. An error will then be reported if any of the
         requested editing operations would change the contents of the
         NDF. If a \texttt{FALSE} value is supplied for READONLY, the NDF is
         opened for both reading and writing, but an error will be
         reported if the NDF file is write-protected on disk. If the MODE
         parameter is set to \texttt{"File"}, the dynamic default value for
         READONLY is \texttt{FALSE}. If MODE is set to \texttt{"Interface"}, the
         dynamic default value for READONLY depends on the value of the
         EDIT parameter: \texttt{TRUE} for \texttt{"Print"} and \texttt{"Exist"}, and
         \texttt{FALSE} for all other editing commands.  \texttt{[]}
      }
      \sstsubsection{
         STRING = \_LOGICAL (Read)
      }{
         When STRING is \texttt{FALSE}, inferred data typing is used for the
         \texttt{"Write"}, \texttt{"Update"}, and \texttt{"Amend"} editing commands.  So for
         instance, if Parameter VALUE = \texttt{"Y"}, it would appears as logical
         \texttt{TRUE} rather than the string \texttt{'Y~~~~~~~~~'} in the FITS header.
         See topic \texttt{"Value Data Type"}.  When STRING is \texttt{TRUE},
         the value will be treated as a string for the purpose of
         writing the FITS header.  \texttt{[FALSE]}
      }
      \sstsubsection{
         TABLE = FILENAME (Read)
      }{
         The text file containing the keyword translation table.  The
         format of this file is described under \texttt{"File Format"}.  For
         illustrations, see under \texttt{"Examples of the File Format"}.
      }
      \sstsubsection{
         VALUE = LITERAL (Read)
      }{
         The new value of the KEYWORD keyword for the \texttt{"Update"},
         \texttt{"Write"}, and \texttt{"Amend"} editing commands.  The special
         value \texttt{"\$V"} means use the
         current value of the KEYWORD keyword.  This makes it possible
         to modify a comment, leaving the value unaltered.  In addition
         \texttt{"\$V(}keyword\texttt{)"} requests that the value of the
         reference keyword given between the parentheses be assigned
         to the keyword being edited.  This reference keyword must exist
         and have a value for a \texttt{"Write"} or \texttt{"Amend"} (new keyword)
         edit; whereas the FITS-header
         value is unchanged for \texttt{"Update"} or \texttt{"Amend"}
         (keyword exists) if there are problems with this reference keyword.
      }
   }
   \sstresparameters{
      \sstsubsection{
         EXISTS = \_LOGICAL (Write)
      }{
        The result of the final \texttt{"Exist"} operation (see Parameter EDIT).
      }
   }
   \sstexamples{
      \sstexamplesubsection{
         fitsmod dro42 bscale exist
      }{
         This reports \texttt{TRUE} or \texttt{FALSE} depending on whether or not the
         FITS keyword BSCALE exists in the FITS extension of the NDF
         called dro42.
      }
      \sstexamplesubsection{
         fitsmod dro42 bscale p
      }{
         This reports the value of the keyword BSCALE stored in the
         FITS extension of the NDF called dro42.  An error message will
         appear if BSCALE does not exist.
      }
      \sstexamplesubsection{
         fitsmod abc edit=move keyword=bscale position=bzero
      }{
         This moves the keyword BSCALE to lie immediately before keyword
         BZERO in the FITS extension of the NDF called abc.  An error
         will result if either BSCALE or BZERO does not exist.
      }
      \sstexamplesubsection{
         fitsmod dro42 airmass dele
      }{
         This deletes the keyword AIRMASS, if it exists, in the FITS
         extension of the NDF called dro42.
      }
      \sstexamplesubsection{
         fitsmod ndf=dro42 edit=d keyword=airmass[2]
      }{
         This deletes the second occurrence of keyword AIRMASS, if it
         exists, in the FITS extension of the NDF called dro42.
      }
      \sstexamplesubsection{
         fitsmod @100 airmass w 1.456 "Airmass at mid-observation"
      }{
         This creates the keyword AIRMASS in the FITS extension of the
         NDF called 100, assigning the keyword the real value 1.456 and
         comment \texttt{"Airmass at mid-observation"}.  The header is
         located just before the end.  The FITS extension is created if
         it does not exist.
      }
      \sstexamplesubsection{
         fitsmod @100 airmass w 1.456 "Airmass at mid-observation" phase
      }{
         As the previous example except that the new keyword is written
         immediately before keyword PHASE.
      }
      \sstexamplesubsection{
         fitsmod obe observer u value="O'Leary" comment=\$C
      }{
         This updates the keyword OBSERVER with value \texttt{"O'Leary"},
         retaining its old comment.  The modified FITS extension lies
         within the NDF called obe.
      }
      \sstexamplesubsection{
         fitsmod test filter w position=end value=27 comment=! string
      }{
         This creates the keyword FILTER in the FITS extension of the
         NDF called test, assigning the keyword the string value \texttt{"27"}.
         There is no comment.  The keyword is located at the end of the
         headers, but before any END card.  The FITS extension is
         created if it does not exist.
      }
      \sstexamplesubsection{
         fitsmod test edit=w keyword=detector comment=" ~~~ Detector name"
                 value=\$V(ing.dethead) accept
      }{
         This creates the keyword DETECTOR in the FITS extension of the
         NDF called test, assigning the keyword the value of the
         existing hierarchical keyword ING.DETHEAD.  The comment is
         \texttt{" ~~~ Detector name"}, the leading spaces are significant.  The
         keyword is located at the current position keyword.  The FITS
         extension is created if it does not exist.
      }
      \sstexamplesubsection{
         fitsmod datafile mode=file table=fitstable.txt
      }{
         This edits the FITS-extension of the NDF called
         datafile, creating the FITS extension if it does not exist.
         The editing instructions are stored in the text file called
         \texttt{fitstable.txt}.
      }
   }
   \sstnotes{
      \sstitemlist{

         \sstitem
         Requests to move, assign values or comments, the following
         reserved keywords in the FITS extension are ignored: SIMPLE,
         BITPIX, NAXIS, NAXIS\textit{n}, EXTEND, PCOUNT, GCOUNT, XTENSION, BLOCKED,
         and END.

         \sstitem
         When an error occurs during editing, warning messages are sent
         at the normal reporting level, and processing continues to the
         next editing command.

         \sstitem
         The FITS fixed format is used for writing or updating
         headers, except for double-precision values requiring more space.
         The comment is delineated from the value by the string \texttt{" / "}.

         \sstitem
         The comments in comment cards begin one space following the
         keyword or from column 10 whichever is greater.

         \sstitem
         To be sure that the resultant FITS extension is what you
         desired, you should inspect it using the command FITSLIST before
         exporting the data.  If there is something wrong, you may find it
         convenient to use command FITSEDIT to make minor corrections.
      }
   }
   \sstdiytopic{
      Parameter Defaults
   }{
      All the parameters have a suggested default of their current
      value, except NDF, which uses the global current dataset.
   }
   \sstdiytopic{
      Timing
   }{
      Approximately proportional to the number of FITS keywords to be
      edited.  \texttt{"Update"} and \texttt{"Write"}~ edits require the most time.
   }
   \sstdiytopic{
      File Format
   }{
      The file consists of a series of lines, one per editing
      instruction, although blank lines and lines beginning with a \texttt{{!}} or
      \texttt{\#} are treated as comments.  Note that the order does matter, as
      the edits are performed in the order given.

      The format is summarised below:

       \texttt{command keyword\{[occurrence]\}\{(keyword\{[occurrence]\})\} \{value \{comment\}\}}

      where braces indicate optional values, and occur is the
      occurrence of the keyword.  In effect there are four fields
      delineated by spaces that define the edit operation, keyword,
      value and comment.

      \sstitemlist{

         \sstitem
         Field 1:
            This specifies the editing operation.  Allowed values are
            \texttt{Amend}, \texttt{Delete}, \texttt{Exist}, \texttt{Move},
            \texttt{Null}, \texttt{Print}, \texttt{Rename}, \texttt{Write}, and
            \texttt{Update}, and can be abbreviated to the initial upper-case letter.
            It is not case insensitive to afford some protection against typing errors.

            \sstitemlist{

               \sstitem
               \texttt{Delete} removes a named keyword.

               \sstitem \texttt{Read} causes the value of a named keyword to be
               displayed to standard output.

               \sstitem \texttt{Exist} reports \texttt{TRUE} to standard output if
               the named keyword exists in the header, and \texttt{FALSE} if
               the keyword is not present.

               \sstitem
               \texttt{Move} relocates a named keyword to be immediately before
               a second keyword.  When this positional keyword is not supplied,
               it defaults to the END card, and if the END card is absent, the new location is at the end of the
               headers.

               \sstitem
               \texttt{Write} creates a new card given a value and an optional
               comment.  Its location uses the same rules as for the
               \texttt{Move} command.

               \sstitem
               \texttt{Update} revises the value and/or the comment.
               If a secondary keyword is defined explicitly, the card may be
               relocated at the same time.  \texttt{Update} requires that the
               keyword exists.

               \sstitem
               \texttt{Amend} acts like \texttt{Update} if the keyword supplied in
               \texttt{"Field 2"} exists, and like \texttt{Write} otherwise.

               \sstitem
               \texttt{Null} replaces the value of a named keyword with blanks.

           }

         \sstitem
         Field 2:
            This specifies the keyword to edit, and optionally the
            position of that keyword in the header after the edit (for
            \texttt{Move}, \texttt{Write}, \texttt{Update}, and \texttt{Amend} edits).
            The new position in the header is immediately before a
            positional keyword, whose name is
            given in parentheses concatenated to the edit keyword.  See
            \texttt{"Field 1"} for defaulting when the position
            parameter is not defined or is null.

            Both the editing keyword and position keyword may be
            compound to handle hierarchical keywords.  In this case the
            form is keyword1.keyword2.keyword3 \emph{etc.}\  All
            keywords must be valid FITS keywords.  This means they must
            be no more than eight characters long, and the only permitted
            characters are uppercase alphabetic, numbers, hyphen,
            and underscore.  Invalid keywords will be rejected.

            Both the edit and position keyword may have an occurrence
            specified in brackets \texttt{[]}.  This enables editing of a keyword
            that is not the first occurrence of that keyword, or locate a
            edited keyword not at the first occurrence of the positional
            keyword.  Note that it is not normal to have multiple
            occurrences of a keyword in a FITS header, unless it is blank,
            COMMENT or HISTORY.  Any text between the brackets other than
            a positive integer is interpreted as the first occurrence.

            Use a null value (\texttt{' '} or \texttt{" "}) if you want the
            card to be a comment with keyword other than COMMENT or
            HISTORY.  As blank keywords are used for hierarchical
            keywords, to write a comment in a blank keyword you must
            give a null edit keyword.  These have no keyword appearing
            before the left parenthesis or bracket, such as
            \texttt{()}, \texttt{[]}, \texttt{[2]}, or \texttt{(EPOCH)}.

         \sstitem
         Field 3:
            This specifies the value to assign to the edited keyword in
            the \texttt{Write}, \texttt{Update}, and \texttt{Amend} operations, or the
            name of the new keyword in the \texttt{Rename} modification.  If the
            keyword exists, the existing value or keyword is replaced, as
            appropriate.  The data type used to store the value is inferred
            from the value itself.  See topic \texttt{"Value Data Type"}.

            For the \texttt{Update}, \texttt{Write}, and \texttt{Amend} modifications
            there is a special value, \texttt{\$V}, which means use the current
            value of the edited keyword, provided that keyword exists.  This
            makes it possible to modify a comment, leaving the value
            unaltered.  In addition \texttt{\$V(}keyword\texttt{)} requests that
            the value of the keyword given between the parentheses be
            assigned to the keyword being edited.

            The value field is ignored when the keyword is COMMENT,
            HISTORY or blank, and the modification is to \texttt{Update},
            \texttt{Write}, or \texttt{Amend}.

         \sstitem
         Field 4:
            This specifies the comment to assign to the edited keyword for
            the \texttt{Write}, \texttt{Update}, and \texttt{Amend} operations.  A
            leading \texttt{"/"} should not be supplied.

            There is a special value, \texttt{\$C}, which means use the current
            comment of the edited keyword, provided that keyword exists.
            This makes it possible to modify a value, leaving the comment
            unaltered.  In addition \texttt{\$C(}keyword\texttt{)} requests that the
            comment of the keyword given between the parentheses be assigned to
            the edited keyword.

            To obtain leading spaces before some commentary, use a quote
            (\texttt{{'}}) or double quote (\texttt{{"}}) as the first character
            of the comment.
            There is no need to terminate the comment with a trailing and
            matching quotation character.  Also do not double quotes
            should one form part of the comment.
      }
   }
   \sstdiytopic{
      Value Data Type
   }{
      The data type of a value is determined as follows:
      \ssthitemlist{

         \sstitem
            For the text-file, values enclosed in quotes (\texttt{{'}}) or doubled
            quotes (\texttt{{"}}) are strings.  Note that numeric or logical string
            values must be quoted to prevent them being converted to a
            numeric or logical value in the FITS extension.

         \sstitem
            For prompting the value is a string when Parameter STRING
            is \texttt{TRUE}.

         \sstitem
            Otherwise type conversions of the first word after the
            keywords are made to integer, double precision, and logical
            types in turn.  If a conversion is successful, that becomes the
            data type.  In the case of double precision, the type is set
            to real when the number of significant digits only warrants
            single precision.  If all the conversions failed the value
            is deemed to be a string.
      }
   }
   \sstdiytopic{
      Examples of the File Format
   }{
      The best way to illustrate the options is by listing some example
      lines.

      \begin{description}

      \sstexamplesubsection{
         P AIRMASS
      }{
         This reports the value of keyword AIRMASS to standard output.
      }
      \sstexamplesubsection{
         E FILTER
      }{
         This determines whether keyword FILTER exists and reports
         \texttt{TRUE} or \texttt{FALSE} to standard output.
      }
      \sstexamplesubsection{
         D OFFSET
      }{
         This deletes the keyword OFFSET.
      }
      \sstexamplesubsection{
         Delete OFFSET[2]
      }{
         This deletes any second occurrence of keyword OFFSET.
      }
      \sstexamplesubsection{
         Rename OFFSET1[2] OFFSET2
      }{
         This renames the second occurrence of keyword OFFSET1 to have
         keyword OFFSET2.
      }
      \sstexamplesubsection{
         W AIRMASS 1.379
      }{
         This writes a real value to new keyword AIRMASS, which will be
         located at the end of the FITS extension.
      }
      \sstexamplesubsection{
         A AIRMASS 1.379
      }{
         This writes a real value to keyword AIRMASS if it exists,
         otherwise it writes a real value to new keyword AIRMASS
         located at the end of the FITS extension.
      }
      \sstexamplesubsection{
         N AIRMASS
      }{
         This blanks the value of the AIRMASS keyword, if it exists.
      }
      \sstexamplesubsection{
         W FILTER(AIRMASS) Y
      }{
         This writes a logical true value to new keyword FILTER, which
         will be located just before the AIRMASS keyword, if it exists.
      }
      \sstexamplesubsection{
         Write FILTER(AIRMASS) 'Y'
      }{
         As the preceding example except that this writes a character
         value \texttt{"Y"}.
      }
      \sstexamplesubsection{
         W COMMENT(AIRMASS) .  Following values apply to mid-observation
      }{
         This writes a COMMENT card immediately before the AIRMASS card,
         the comment being \texttt{"Following values apply to mid-observation"}.
         Note the full stop.
      }
      \sstexamplesubsection{
         W DROCOM(AIRMASS) '' Following values apply to mid-observation
      }{
         As the preceding example but this writes to a non-standard
         comment keyword called DROCOM.  Note the need to supply a null
         value.
      }
      \sstexamplesubsection{
         W (AIRMASS) '' Following values apply to mid-observation
      }{
        As the preceding example but this writes to a blank-keyword
        comment.
      }
      \sstexamplesubsection{
         U OBSERVER "Dr.\ Peter O'Leary" Name of principal observer
      }{
         This updates the OBSERVER keyword with the string value
         \texttt{"Dr.\ Peter O'Leary"}, and comment \texttt{"Name of principal observer"}.
         Note that had the value been enclosed in single quotes ('), the
         apostrophe would need to be doubled.
      }
      \sstexamplesubsection{
         M OFFSET
      }{
         This moves the keyword OFFSET to just before the END card.
      }
      \sstexamplesubsection{
         Move OFFSET(SCALE)
      }{
         This moves the keyword OFFSET to just before the SCALE card.
      }
      \sstexamplesubsection{
         Move OFFSET[2](COMMENT[3])
      }{
         This moves the second occurrence of keyword OFFSET to just
         before the third COMMENT card.
      }
      \end{description}
   }
   \sstdiytopic{
      References
   }{
      \begin{refs}
      \item {\em "A User's Guide for the Flexible Image Transport System (FITS)"},
      NASA/Science Office of Science and Technology (1994).

      \end{refs}
   }
   \sstdiytopic{
      Related Applications
   }{
KAPPA: \htmlref{FITSEDIT}{FITSEDIT},
\htmlref{FITSEXIST}{FITSEXIST},
\htmlref{FITSEXP}{FITSEXP},
\htmlref{FITSHEAD}{FITSHEAD},
\htmlref{FITSIMP}{FITSIMP},
\htmlref{FITSLIST}{FITSLIST},
\htmlref{FITSVAL}{FITSVAL},
\htmlref{FITSWRITE}{FITSWRITE}.
   }
}
\sstroutine{
   FITSTEXT
}{
   Creates an NDF FITS extension from a text file
}{
   \sstdescription{
      This application takes a version of a \FITSref\ ~header stored in a
      text file, and inserts it into the \htmlref{FITS
      extension}{se:fitsairlock}~ of an \NDFref{NDF}.  The
      header is not copied verbatim as some validation of the headers
      as legal FITS occurs.  An existing FITS extension is removed.
   }
   \sstusage{
      fitstext ndf file
   }
   \sstparameters{
      \sstsubsection{
         NDF = NDF (Read and Write)
      }{
         The name of the NDF to store the FITS header information.
      }
      \sstsubsection{
         FILE = FILENAME (Read)
      }{
         The text file containing the FITS headers.  Each record should
         be the standard 80-character `card image'.  If the file has
         been edited care is needed to ensure that none of the cards
         are wrapped on to a second line.
      }
   }
   \sstexamples{
      \sstexamplesubsection{
         fitstext hh73 headers.lis
      }{
         This places the FITS headers stored in the text file called
         \texttt{headers.lis} in the FITS extension of the NDF called hh73.
      }
   }
   \sstnotes{
      \sstitemlist{

         \sstitem
         The validation process performs the following checks on each
         header `card': \\
           a) the length of the header is no more than 80 characters,
           otherwise it is truncated; \\
           b) the keyword only contains uppercase Latin alphabetic
           characters, numbers, underscore, and hyphen (the header will
           not be copied to the extension except when the invalid
           characters are lowercase letters); \\
           c) value cards have an equals sign in column 9 and a space in
           column 10; \\
           d) quotes enclose character values; \\
           e) single quotes inside string values are doubled; \\
           f) character values are left justified to column 11 (retaining
           leading blanks) and contain at least 8 characters (padding with
           spaces if necessary); \\
           g) non-character values are right justified to column 30, except
           for non-mandatory keywords which have a double-precision value
           requiring more than 20 digits; \\
           h) the comment delimiter is in column 32 or two characters
           following the value, whichever is greater; \\
           i) an equals sign in column 9 of a commentary card is replaced
           by a space; and \\
           j) comments begin at least two columns after the end of the
           comment delimiter.

         \sstitem
         The validation issues warning messages at the normal reporting
         level for violations a), b), c), d), and i).

         \sstitem
         The validation can only go so far.  If any of your header lines
         are ambiguous, the resulting entry in the FITS extension may not
         be what you intended.  Therefore, you should inspect the
         resulting FITS extension using the command FITSLIST before
         exporting the data.  If there is something wrong, you may find it
         convenient to use command FITSEDIT to make minor corrections.
      }
   }
   \sstdiytopic{
      Related Applications
   }{
KAPPA: \htmlref{FITSEDIT}{FITSEDIT},
\htmlref{FITSEXP}{FITSEXP},
\htmlref{FITSLIST}{FITSLIST};
\xref{CONVERT}{sun55}{}: \xref{NDF2FITS}{sun55}{NDF2FITS}.
   }
}
\sstroutine{
   FITSURFACE
}{
   Fits a polynomial surface to two-dimensional data array
}{
   \sstdescription{
      This task fits a surface to a two-dimensional data array stored
      array within an \NDFref{NDF} data structure.  At present it only
      permits a fit with a polynomial, and the coefficients of that
      surface are stored in a POLYNOMIAL structure (\xref{SGP/38}{sgp38}{}) as an
      \htmlref{extension}{se:ndfext}~ to that NDF.

      Unlike SURFIT, neither does it bin the data nor does it reject
      outliers.
   }
   \sstusage{
      fitsurface ndf [fittype]
        $\left\{ {\begin{tabular}{l}
                  nxpar nypar \\
                  \mbox{[knots]}
                  \end{tabular} }
        \right.$
        \newline\latexhtml{\hspace*{12.5em}}{~~~~~~~~~~~~~~~~~~~~~~}
        \makebox[0mm][c]{\small fittype}
   }
   \sstparameters{
      \sstsubsection{
         COSYS = \htmlref{LITERAL}{se:parmenu} (Read)
      }{
         The \htmlref{co-ordinate system}{se:co-ordsystem} to be used.  This can be either \texttt{"World"}
         or \texttt{"Data"}.  If COSYS=\texttt{"World"} the co-ordinates used to fits
         the surface are pixel co-ordinates.  If COSYS=\texttt{"Data"} the
         data co-ordinates used are used in the fit, provided there are
         axis centres present in the NDF.  COSYS=\texttt{"World"} is
         recommended.  \texttt{[}Current co-ordinate system\texttt{{]}}
      }
      \sstsubsection{
         FITTYPE = LITERAL (Read)
      }{
         The type of fit.  It must be either \texttt{"Polynomial"} for a
         polynomial or \texttt{"Spline"} for a bi-cubic spline.  \texttt{["Polynomial"]}
      }
      \sstsubsection{
         KNOTS( 2 ) = \_INTEGER (Read)
      }{
         The number of interior knots used for the bi-cubic-spline fit
         along the \textit{x} and \textit{y} axes.  These knots are
         equally spaced within the image.  Both values must be in the
         range 0 to 11.  If you supply a single value, it applies to
         both axes.  Thus \texttt{1} creates one interior knot, \texttt{[5,4]}
         gives five along the \textit{x} axis and four along
         the \textit{y} direction.  Increasing this parameter values
         increases the flexibility of the surface.  Normally, \texttt{4}
         is a reasonable value.  The upper limit of acceptable values
         will be reduced along each axis when its binned array
         dimension is fewer than 29.  KNOTS is only accessed when
         FITTYPE=\texttt{"Spline"}.  The default is the current value,
         which is \texttt{4} initially.  \texttt{[]}
      }
      \sstsubsection{
         NDF  = NDF (Update)
      }{
         The NDF containing the two-dimensional data array to be fitted.
      }
      \sstsubsection{
         NXPAR = \_INTEGER (Read)
      }{
         The number of fitting parameters to be used in the \textit{x}
         direction.  It must be in the range 1 to 15 for a polynomial
         fit.  Thus \texttt{1} gives a constant, \texttt{2} a linear fit,
         \texttt{3} a quadratic \emph{etc}.  Increasing this parameter
         increases the flexibility of the surface in the \textit{x}
         direction.  The upper limit of acceptable values will be
         reduced for arrays with an \textit{x} dimension fewer than 29.
         NXPAR is only accessed when FITTYPE=\texttt{"Polynomial"}.
      }
      \sstsubsection{
         NYPAR = \_INTEGER (Read)
      }{
         The number of fitting parameters to be used in the \textit{y}
         direction.  It must be in the range 1 to 15 for a polynomial
         fit.  Thus \texttt{1} gives a constant, \texttt{2} a linear fit,
         \texttt{3} a quadratic \emph{etc}.  Increasing this parameter
         increases the flexibility of the surface in the \textit{y}
         direction.  The upper limit of acceptable values will be
         reduced for arrays with a \textit{y} dimension fewer than 29.
         NYPAR is only accessed when FITTYPE=\texttt{"Polynomial"}.
      }
      \sstsubsection{
         OVERWRITE = \_LOGICAL (Read)
      }{
         OVERWRITE=\texttt{TRUE}, allows an NDF extension containing an existing
         surface fit to be overwritten.  OVERWRITE=\texttt{FALSE} protects an
         existing surface-fit extension, and should one exist, an error
         condition will result and the task terminated.  \texttt{[TRUE]}
      }
      \sstsubsection{
         VARIANCE = \_LOGICAL (Read)
      }{
         A flag indicating whether any variance array present in the
         NDF is used to define the weights for the fit.  If VARIANCE
         is \texttt{TRUE} and the NDF contains a variance array this will be
         used to define the weights, otherwise all the weights will be
         set equal.  \texttt{[TRUE]}
      }
      \sstsubsection{
         XMAX = \_DOUBLE (Read)
      }{
         The maximum \textit{x} value to be used in the fit.  This
         must be greater than or equal to the \textit{x} co-ordinate
         of the right-hand pixel in the data array.  Normally this
         parameter is automatically set to the maximum \textit{x}
         co-ordinate found in the data, but this mechanism can be
         overridden by specifying XMAX on the command line.  The
         parameter is provided to allow the fit limits to be fine
         tuned for special purposes.  It should not normally be
         altered.  If a null (\texttt{{!}}) value is supplied, the value
         used is the maximum \textit{x} co-ordinate of the fitted
         data.  \texttt{[!]}
      }
      \sstsubsection{
         XMIN = \_DOUBLE (Read)
      }{
         The minimum \textit{x} value to be used in the fit.  This
         must be smaller than or equal to the \textit{x} co-ordinate
         of the left-hand pixel in the data array.  Normally this
         parameter is automatically set to the minimum \textit{x}
         co-ordinate found in the data, but this mechanism can be
         overridden by specifying XMIN on the command line.  The
         parameter is provided to allow the fit limits to be fine
         tuned for special purposes.  It should not normally be
         altered.  If a null (\texttt{{!}}) value is supplied, the value
         used is the minimum \textit{x} co-ordinate of the fitted
         data.  \texttt{[!]}
      }
      \sstsubsection{
         YMAX = \_DOUBLE (Read)
      }{
         The maximum \textit{y} value to be used in the fit.  This
         must be greater than or equal to the \textit{y} co-ordinate
         of the top pixel in the data array.  Normally this parameter
         is automatically set to the maximum \textit{y} co-ordinate
         found in the data, but this mechanism can be overridden by
         specifying YMAX on the command line.  The parameter is
         provided to allow the fit limits to be fine tuned for special
         purposes.  It should not normally be altered.  If a null
         (\texttt{{!}}) value is supplied, the value used is the maximum
         \textit{y} co-ordinate of the fitted data.  \texttt{[!]}
      }
      \sstsubsection{
         YMIN = \_DOUBLE (Read)
      }{
         The minimum \textit{y} value to be used in the fit.  This
         must be smaller than or equal to the \textit{y} co-ordinate
         of the bottom pixel in the data array.  Normally this
         parameter is automatically set to the minimum \textit{y}
         co-ordinate found in the data, but this mechanism can be
         overridden by specifying YMIN on the command line.  The
         parameter is provided to allow the fit limits to be fine
         tuned for special purposes.  It should not normally be
         altered.  If a null (\texttt{{!}}) value is supplied, the value
         used is the minimum \textit{y} co-ordinate of the fitted
         data.  \texttt{[!]}
      }
   }
   \sstexamples{
      \sstexamplesubsection{
         fitsurface virgo nxpar=4 nypar=4 novariance
      }{
         This fits a bi-cubic polynomial surface to the data array
         in the NDF called virgo.  All the data values are given
         equal weight.  The coefficients of the fitted surface are
         stored in an extension of virgo.
      }
      \sstexamplesubsection{
         fitsurface virgo nxpar=4 nypar=4
      }{
         As the first example except the data variance, if present,
         is used to weight the data values.
      }
      \sstexamplesubsection{
         fitsurface virgo fittype=spl
      }{
         As the previous example except a B-spline fit is made using
         four interior knots along both axes.
      }
      \sstexamplesubsection{
         fitsurface virgo fittype=spl knots=[10,7]
      }{
         As the previous example except now there are ten interior knots
         along the \textit{x} axis and seven along the \textit{y} axis.
      }
      \sstexamplesubsection{
         fitsurface mkn231 nxpar=6 nypar=2 cosys=d xmin=-10.0 xmax=8.5
      }{
         This fits a polynomial surface to the data array in the NDF
         called mkn231.  A fifth order is used along the \textit{x} direction,
         but only a linear fit along the \textit{y} direction.  The fit is made
         between \textit{x} data co-ordinates $-$10.0 to 8.5.  The variance
         weights the data values.  The coefficients of the fitted
         surface are stored in an extension of mkn231.
      }
   }
   \sstnotes{
      A polynomial surface fit is stored in a SURFACEFIT extension,
      component FIT of type POLYNOMIAL, variant CHEBYSHEV or BSPLINE.
      This is read by MAKESURFACE to create a NDF of the fitted surface.

      For further details of the CHEBYSHEV variant see
      \xref{SGP/38}{sgp38}{}.  The CHEBYSHEV variant includes the fitting
      variance for each coefficient.

      The BSPLINE variant structure is provisional.  It contain the
      spline coefficients in the two-dimensional DATA\_ARRAY component,
      the knots in XKNOTS and YKNOTS arrays, and a scaling factor to
      restore the original values after spline evaluation recorded in
      component SCALE.  All of these components have type \_REAL.

      Also stored in the SURFACEFIT extension are the r.m.s. deviation
      to the fit (component RMS), the maximum absolute deviation
      (component RSMAX), and the co-ordinate system (component COSYS)
      translated to AST Domain names AXIS (for Parameter COSYS=\texttt{"Data"})
      and PIXEL (\texttt{"World"}).
   }
   \sstdiytopic{
      Related Applications
   }{
KAPPA: \htmlref{MAKESURFACE}{MAKESURFACE},
\htmlref{SURFIT}{SURFIT}.
   }
   \sstimplementationstatus{
      \sstitemlist{

         \sstitem
         This routine correctly processes the \htmlref{AXIS}{apndf:axis}, DATA, \htmlref{QUALITY}{apndf:quality},
         \htmlref{VARIANCE}{apndf:variance}, and \htmlref{HISTORY}{apndf:history}~ components of an NDF data structure.

         \sstitem
         Processing of \htmlref{bad pixels}{se:masking} and automatic \htmlref{quality masking}{se:qualitymask} are
         supported.

         \sstitem
         All \htmlref{non-complex numeric data types}{ap:HDStypes} can be handled.  Arithmetic
         is performed using double-precision floating point.
      }
   }
}
\sstroutine{
   FITSVAL
}{
   Reports the value of a keyword in the FITS extension.
}{
   \sstdescription{
     This application reports the value of a keyword in the
     \htmlref{FITS extension}{se:fitsairlock} (`airlock') of an
     \NDFref{NDF} file.  The keyword's value and comment are also
     stored in output parameters.
   }
   \sstusage{
      fitsval ndf keyword
   }
   \sstparameters{
      \sstsubsection{
         KEYWORD = LITERAL (Read)
      }{

         The name of an existing keyword in the FITS extension whose
         value is to be reported.  A name may be compound to handle
         hierarchical keywords, and it has the form
         keyword1.keyword2.keyword3 \emph{etc.}\  The maximum number of
         keywords per \FITSref\ ~card is 20.  Each keyword must be no
         longer than 8 characters, and be a valid FITS keyword
         comprising only alphanumeric characters, hyphen, and
         underscore.  Any lowercase letters are converted to uppercase
         and blanks are removed before comparison with the existing
         keywords.

         KEYWORD may have an occurrence specified in brackets \texttt{[]}
         following the name.  This enables the values to be obtained
         for keywords that appear more than once.  Note that it is not
         normal to have multiple occurrences of a keyword in a FITS
         header, unless it is blank, COMMENT or HISTORY.  Any text
         between the brackets other than a positive integer is
         interpreted as the first occurrence.

         The suggested value is the current value.
      }
      \sstsubsection{
         NDF = NDF (Read)
      }{
         The NDF containing the FITS keyword.
      }
   }
   \sstresparameters{
      \sstsubsection{
         COMMENT = LITERAL (Write)
      }{
         The comment of the keyword.
      }
      \sstsubsection{
         VALUE = LITERAL (Write)
      }{
         The value of the keyword.
      }
   }
   \sstexamples{
      \sstexamplesubsection{
         fitsval abc bscale
      }{
         This reports the value of the FITS keyword BSCALE, which is
         located within the FITS extension of the NDF called abc.
      }
      \sstexamplesubsection{
         fitsval ndf=abc keyword=date[2]
      }{
         This reports the value of the second occurrence FITS keyword
         DATE, which is located within the FITS extension of the NDF
         called abc.
      }
   }
   \sstdiytopic{
      Related Applications
   }{
KAPPA: \htmlref{FITSEDIT}{FITSEDIT},
\htmlref{FITSEXIST}{FITSEXIST},
\htmlref{FITSHEAD}{FITSHEAD},
\htmlref{FITSLIST}{FITSLIST},
\htmlref{FITSMOD}{FITSMOD}.
   }
}
\sstroutine{
   FITSWRITE
}{
   Writes a new keyword to the FITS extension
}{
   \sstdescription{
      This application writes a new keyword in an \NDFref{NDF's}
      \htmlref{FITS extension}{se:fitsairlock}~ given a value and an optional
      inline comment.  It allows the location of the new keyword to be
      specified.  The FITS extension is created if it does not exist.

      It is a synonym for \texttt{fitsmod edit=write mode=interface position=!}.
   }
   \sstusage{
      fitswrite ndf keyword value=? comment=?
   }
   \sstparameters{
      \sstsubsection{
         COMMENT = LITERAL (Read)
      }{
         The comments to be written to the KEYWORD keyword.  A null value
         (\texttt{{!}}) gives a blank comment.  The special value \texttt{"\$C"}
         means use the current comment.  In addition \texttt{"\$C(}keyword\texttt{)"} requests that the comment of the
         keyword given between the parentheses be assigned to the
         keyword being edited.  If this positional keyword does not exist,
         the comment is is blank.
      }
      \sstsubsection{
         KEYWORD = LITERAL (Read)
      }{
         The name of the new keyword in the FITS extension.  A name may
         be compound to handle hierarchical keywords, and it has the
         form keyword1.keyword2.keyword3 \emph{etc.}\  The maximum number
         of keywords per FITS card is 20.  Each keyword must be no
         longer than 8 characters, and be a valid FITS keyword comprising
         only alphanumeric characters, hyphen, and underscore.  Any
         lowercase letters are converted to uppercase and blanks are
         removed before comparison with the existing keywords.

         Note that it is not normal to have multiple occurrences of a
         keyword in a FITS header, unless it is blank, COMMENT or HISTORY.

         The suggested value is the current value.
      }
      \sstsubsection{
         NDF = NDF (Read and Write)
      }{
         The NDF containing the FITS extension into which the new FITS
         keyword.
      }
      \sstsubsection{
         POSITION = LITERAL (Read)
      }{
         The position keyword name.  A position name may be compound to
         handle hierarchical keywords, and it has the form
         keyword1.keyword2.keyword3 \emph{etc.}\  The maximum number of
         keywords per FITS card is 20.  Each keyword must be no longer
         than 8 characters.  When locating the position card,
         comparisons are made in uppercase and with the blanks removed.
         An occurrence may be specified (see Parameter KEYWORD for
         details).

         The new keywords are inserted immediately before each
         corresponding position keyword.  If any name in it does not
         exist in FITS array, or the null value (\texttt{{!}}) is supplied,
         the KEYWORD keyword will be inserted just before the END card
         or appended to FITS array when the END card does not exist.
         \texttt{[!]}
      }
      \sstsubsection{
         STRING = \_LOGICAL (Read)
      }{
         When STRING is \texttt{FALSE}, inferred data typing is used.  So for
         instance if Parameter VALUE = \texttt{"Y"}, it would appears as logical
         \texttt{TRUE} rather than the string \texttt{'Y~~~~~~~~~'} in the FITS header.
         See topic \texttt{"Value Data Type"}.  When STRING is \texttt{TRUE},
         the value will be treated as a string for the purpose of
         writing the FITS header.  \texttt{[FALSE]}
      }
      \sstsubsection{
         VALUE = LITERAL (Read)
      }{
         The new value of the KEYWORD keyword.  The special value \texttt{"\$V"}
         means use the current value of the KEYWORD keyword.  This makes
         it possible to modify a comment, leaving the value unaltered.
         In addition \texttt{"\$V(}keyword\texttt{)"} requests that the value
         of the reference keyword given between the parentheses be
         assigned to the keyword being written.  This reference keyword
         must exist and have a value.
      }
   }
   \sstexamples{
      \sstexamplesubsection{
         fitswrite abc bscale value=1.234
      }{
         This writes the FITS keyword BSCALE just before the end of the
         FITS extension, which is located within the NDF called abc.  It
         assigns BSCALE a value of 1.234.  There is no inline comment.
      }
      \sstexamplesubsection{
         fitswrite @100 airmass value=1.456 comment="Airmass at mid-observation"
      }{
         This creates the keyword AIRMASS in the FITS extension of the
         NDF called 100, assigning the keyword the real value 1.456 and
         comment \texttt{"Airmass at mid-observation"}.  The header is
         located just before the end.
      }
      \sstexamplesubsection{
         fitswrite @100 airmass value=1.456 "Airmass at mid-observation"
         position=phase
      }{
         As the previous example except that the new keyword is written
         immediately before keyword PHASE.
      }
      \sstexamplesubsection{
         fitswrite afcyg observer value="O'Leary" comment=\$C(prininv)
      }{
         This writes the keyword OBSERVER with value \texttt{"O'Leary"},
         and its comment is copied from keyword PRININV.  The modified
         FITS extension lies within the NDF called afcyg.
      }
      \sstexamplesubsection{
         fitswrite test filter position=end value=27 comment=! string
      }{
         This creates the keyword FILTER in the FITS extension of the
         NDF called test, assigning the keyword the string value \texttt{"27"}.
         There is no comment.  The keyword is located at the end of the
         headers, but before any END card.
      }
      \sstexamplesubsection{
         fitswrite ndf=test keyword=detector comment=" ~~~ Detector name"
                 value=\$V(ing.dethead) accept
      }{
         This creates the keyword DETECTOR in the FITS extension of the
         NDF called test, assigning the keyword the value of the
         existing hierarchical keyword ING.DETHEAD.  The comment is
         \texttt{" ~~~ Detector name"}, the leading spaces are significant.  The
         keyword is located at the current position keyword.
      }
   }
   \sstdiytopic{
      Value Data Type
   }{
      The data type of a value is determined as follows:
      \ssthitemlist{

         \sstitem
            For the text-file, values enclosed in quotes (\texttt{{'}}) or doubled
            quotes (\texttt{{"}}) are strings.  Note that numeric or logical string
            values must be quoted to prevent them being converted to a
            numeric or logical value in the FITS extension.

         \sstitem
            For prompting the value is a string when Parameter STRING
            is \texttt{TRUE}.

         \sstitem
            Otherwise type conversions of the first word after the
            keywords are made to integer, double precision, and logical
            types in turn.  If a conversion is successful, that becomes the
            data type.  In the case of double precision, the type is set
            to real when the number of significant digits only warrants
            single precision.  If all the conversions failed the value
            is deemed to be a string.
      }
   }
   \sstdiytopic{
      Related Applications
   }{
KAPPA: \htmlref{FITSEDIT}{FITSEDIT},
\htmlref{FITSEXP}{FITSEXP},
\htmlref{FITSMOD}{FITSMOD}.
   }
}
\sstroutine{
   FLIP
}{
   Reverses an NDF's pixels along a specified dimension
}{
   \sstdescription{
      This application reverses the order of an \NDFref{NDF's}  pixels along a
      specified dimension, leaving all other aspects of the data
      structure unchanged.
   }
   \sstusage{
      flip in out dim
   }
   \sstparameters{
      \sstsubsection{
         AXIS = \_LOGICAL (Read)
      }{
         If a \texttt{TRUE} value is given for this parameter (the default),
         then any axis values associated with the NDF dimension being
         reversed will also be reversed in the same way.  If a \texttt{FALSE}
         value is given, then all axis values will be left unchanged.
         \texttt{[TRUE]}
      }
      \sstsubsection{
         DIM = \_INTEGER (Read)
      }{
         The number of the dimension along which the NDF's pixels
         should be reversed.  The value should lie between 1 and the
         total number of NDF dimensions.  If the NDF has only a single
         dimension, then this parameter is not used, a value of \texttt{1}
         being assumed.
      }
      \sstsubsection{
         IN = NDF (Read)
      }{
         The input NDF data structure whose pixel order is to be
         reversed.
      }
      \sstsubsection{
         OUT = NDF (Write)
      }{
         The output NDF data structure.
      }
      \sstsubsection{
         TITLE = LITERAL (Read)
      }{
         A \htmlref{title}{apndf:title} for the output NDF.  A null value will cause the title
         of the NDF supplied for Parameter IN to be used instead.
         \texttt{[!]}
      }
   }
   \sstexamples{
      \sstexamplesubsection{
         flip a b 2
      }{
         Reverses the pixels in the NDF called a along its second
         dimension to create the new NDF called b.
      }
      \sstexamplesubsection{
         flip specin specout
      }{
         If specin is a one-dimensional spectrum, then this example
         reverses the order of its pixels to create a new spectrum
         specout.  Note that no value for the DIM parameter need be
         supplied in this case.
      }
      \sstexamplesubsection{
         flip in=cube out=newcube dim=2 noaxis
      }{
         Reverses the order of the pixels along dimension 2 of the NDF
         called cube to give newcube, but leaves the associated axis
         values in their original order.
      }
   }
   \sstnotes{
      The pixel-index bounds of the NDF are unchanged by this routine.
   }
   \sstdiytopic{
      Related Applications
   }{
KAPPA: \htmlref{ROTATE}{ROTATE},
\htmlref{REGRID}{REGRID};
\xref{FIGARO}{sun86}{}: \xref{IREVX}{sun86}{IREVX},
\xref{IREVY}{sun86}{IREVY},
\xref{IROT90}{sun86}{IROT90}.
   }
   \sstimplementationstatus{
      \sstitemlist{

         \sstitem
         This routine correctly processes the \htmlref{AXIS}{apndf:axis}, DATA, \htmlref{QUALITY}{apndf:quality},
         \htmlref{VARIANCE}{apndf:variance}, \htmlref{LABEL}{apndf:label}, \htmlref{TITLE}{apndf:title}, \htmlref{UNITS}{apndf:units}, \htmlref{WCS}{apndf:wcs}, and \htmlref{HISTORY}{apndf:history}~ components of the
         input NDF and propagates all \htmlref{extensions}{apndf:extensions}.

         \sstitem
         Processing of \htmlref{bad pixels}{se:masking} and automatic \htmlref{quality masking}{se:qualitymask} are
         supported.

         \sstitem
         All \htmlref{non-complex numeric data types}{ap:HDStypes} can be handled.  The data
         type of the input pixels is preserved in the output NDF.
      }
   }
}
\sstroutine{
   FOURIER
}{
   Performs forward and inverse Fourier transforms of one- or two-dimensional NDFs
}{
   \sstdescription{
      This application performs forward or reverse Fast Fourier
      Transforms (FFTs) of one- or two-dimensional \NDFref{NDFs}.  The output in the
      forward transformation (from the space domain to the Fourier) can
      be produced in Hermitian form in a single NDF, or as two NDFs giving
      the real and imaginary parts of the complex transform, or as two
      NDFs giving the power and phase of the complex transform.  Any
      combination of these may also be produced.  The inverse procedure
      accepts any of these NDFs and produces a purely real output NDF.

      Any \htmlref{bad pixels}{se:masking}~ in the input NDF may be replaced
      by a constant value.
      Input NDFs need neither be square, nor be a power of 2 in size in
      either dimension; their shape is arbitrary.

      The Hermitian transform is a single image in which each quadrant
      consisting of a linear combination of the real and imaginary
      parts of the transform.  This form is useful if you just want to
      multiply the Fourier transform by some known purely real mask and
      then invert it to get a filtered image.  However, if you want to
      multiply the Fourier transform by a complex mask (\emph{e.g.} the
      Fourier transform of another NDF), or do any other operation
      involving combining complex values, then the Hermitian NDF must
      be untangled into separate real and imaginary parts.

      There is an option to swap the quadrants of the input NDF around
      before performing a forward FFT.  This is useful if you want to
      perform convolutions with the FFTs, since the point-spread
      function (PSF) image can be created with the PSF centre at the
      array centre, rather than at pixel (1,~1) as is usually required.
   }
   \sstusage{
      fourier in hermout
   }
   \sstparameters{
      \sstsubsection{
         FILLVAL = LITERAL (Read)
      }{
         A value to replace bad pixels before performing the transform.
         The input image is also padded with this value if necessary to
         form an image of acceptable size.  A value of \texttt{"Mean"} will cause
         the mean value in the array to be used.  \texttt{[0.0]}
      }
      \sstsubsection{
         HERMIN = NDF (Read)
      }{
         Hermitian frequency-domain input NDF containing the complex
         transform.  If null is entered no Hermitian NDF is read and
         then the application should be supplied either separate real
         and imaginary NDFs, or the power and phase NDFs.  Prompting
         will not occur if one of the other (inverse) input NDFs has
         been given on the command line, but not HERMIN as well.  This
         parameter is only relevant for an inverse transformation.
      }
      \sstsubsection{
         HERMOUT = NDF (Write)
      }{
         Hermitian output NDF from a forward transform.  If a null value
         is given then this NDF is not produced.
      }
      \sstsubsection{
         HM\_TITLE = LITERAL (Read)
      }{
         Title for the Hermitian Fourier-transform output NDF.
         A null (\texttt{{!}}) value means using the title of the input NDF.
         \texttt{["KAPPA - Fourier - Hermitian"]}
      }
      \sstsubsection{
         IM\_TITLE = LITERAL (Read)
      }{
         Title for the frequency-domain imaginary output NDF.
         A null (\texttt{{!}}) value means using the title of the input NDF.
         \texttt{["KAPPA - Fourier - Imaginary"]}
      }
      \sstsubsection{
         IMAGIN = NDF (Read)
      }{
         Input frequency-domain NDF containing the imaginary part of the
         complex transform.  If a null is given then an image of zeros is
         assumed unless a null is also given for REALIN, in which case
         the input is requested in power and phase form.  This parameter
         is only available if HERMIN is not used.  One way to achieve
         that is to supply IMAGIN, but not HERMIN, on the command
         line.  This parameter is only relevant for an inverse
         transformation.
      }
      \sstsubsection{
         IMAGOUT = NDF (Write)
      }{
         Frequency-domain output NDF containing the imaginary part of
         the complex Fourier transform.  If a null value is given then
         this NDF is not produced.  \texttt{[!]}
      }
      \sstsubsection{
         IN = NDF (Read)
      }{
         Real (space-domain) input NDF for a forward transformation.
         There are no restrictions on the size or shape of the input
         NDF, although the it may have to be padded or trimmed before
         being transformed.  This parameter is only used if a forward
         transformation was requested.
      }
      \sstsubsection{
         INVERSE = \_LOGICAL (Read)
      }{
         If \texttt{TRUE}, then the inverse transform---frequency domain to
         space domain---is required, otherwise a transform from the
         space to the frequency domain is undertaken.  \texttt{[FALSE]}
      }
      \sstsubsection{
         OUT = NDF (Write)
      }{
         Real space-domain output NDF.  This parameter is only used if
         an inverse transformation is requested.
      }
      \sstsubsection{
         PH\_TITLE = LITERAL (Read)
      }{
         Title for the frequency-domain phase output NDF.
         A null (\texttt{{!}}) value means using the title of the input NDF.
         \texttt{["KAPPA - Fourier - Phase"]}
      }
      \sstsubsection{
         PHASEIN = NDF (Read)
      }{
         Input frequency-domain NDF containing the phase of the complex
         transform.  If a null is given then an image of zeros is
         assumed unless a null is also given for PHASEIN, in which
         case the application quits.  This parameter is only available
         if HERMIN, REALIN, and IMAGIN are all not used.  One way to
         achieve that is to supply PHASEIN, but none of the
         aforementioned parameters, on the command line.  This
         parameter is only relevant for an inverse transformation.
      }
      \sstsubsection{
         PHASEOUT = NDF (Write)
      }{
         Frequency-domain output NDF containing the phase of the
         complex Fourier transform.  If a null value is given then this
         NDF is not produced.  \texttt{[!]}
      }
      \sstsubsection{
         POWERIN = NDF (Read)
      }{
         Input frequency-domain NDF containing the modulus of the
         complex transform.  Note, this should be the square root of the
         power rather than the power itself.  If a null is given then an
         image of zeros is assumed unless a null is also given for
         PHASEIN, in which case the application quits.  This parameter
         is only available if HERMIN, REALIN, and IMAGIN are all not
         used.  One way to achieve that is to supply POWERIN, but none
         of the aforementioned parameters, on the command line.  This
         parameter is only relevant for an inverse transformation.
      }
      \sstsubsection{
         POWEROUT = NDF (Write)
      }{
         Frequency-domain output NDF containing the modulus of the
         complex Fourier transform.  Note, this is the square root of
         the power rather than the power itself.  If a null value is
         given then this NDF is not produced.  \texttt{[!]}
      }
      \sstsubsection{
         PW\_TITLE = LITERAL (Read)
      }{
         Title for the frequency-domain power output NDF.
         A null (\texttt{{!}}) value means using the title of the input NDF.
         \texttt{["KAPPA - Fourier - Power"]}
      }
      \sstsubsection{
         REALIN = NDF (Read)
      }{
         Input frequency-domain NDF containing the real part of the
         complex transform.  If a null is given then an image of zeros is
         assumed unless a null is also given for IMAGIN, in which case
         the input is requested in power and phase form.  This parameter
         is only available if HERMIN is not used.  One way to achieve
         that is to supply REALIN, but not HERMIN, on the command
         line.  This parameter is only relevant for an inverse
         transformation.
      }
      \sstsubsection{
         REALOUT = NDF (Write)
      }{
         Frequency-domain output NDF containing the real part of the
         complex Fourier transform.  If a null value is given then this
         NDF is not produced.  \texttt{[!]}
      }
      \sstsubsection{
         RL\_TITLE = LITERAL (Read)
      }{
         Title for the frequency-domain real output NDF.
         A null (\texttt{{!}}) value means using the title of the input NDF.
         \texttt{["KAPPA - Fourier - Real"]}
      }
      \sstsubsection{
         SHIFT = \_LOGICAL (Read)
      }{
         If \texttt{TRUE}, the transform origin is to be located at the array's
         centre.  This is implemented by swapping bottom-left and
         top-right, and bottom-right and top-left array quadrants,
         before doing the transform.  This results in the transformation
         effectively being done about pixel \textit{x} = INT(NAXIS1/2)$+$1 and
         \textit{y} = INT(NAXIS2/2)$+$1, where NAXIS\textit{n} are the padded or trimmed
         dimensions of the NDF.  \texttt{[FALSE]}
      }
      \sstsubsection{
         TRIM = \_LOGICAL (Read)
      }{
         If \texttt{TRUE}, when the input array dimension cannot be processed by
         the transform, the output arrays will be trimmed rather than
         padded with the fill value.  \texttt{[FALSE]}
      }
      \sstsubsection{
         TITLE = LITERAL (Read)
      }{
         Title for the real space-domain output NDF.
         A null (\texttt{{!}}) value means using the title of the input NDF.
         \texttt{["KAPPA - Fourier"]}
      }
   }
   \sstexamples{
      \sstexamplesubsection{
         fourier galaxy ft\_gal
      }{
         Makes an Hermitian Fourier transform stored in an NDF called
         ft\_gal from the two-dimensional NDF called galaxy.
      }
      \sstexamplesubsection{
         fourier hermin=ft\_gal out=galaxy inverse
      }{
         Takes an Hermitian Fourier transform stored in an NDF called
         ft\_gal and performs the inverse transformation to yield a
         normal (spatial domain) image in NDF galaxy.
      }
      \sstexamplesubsection{
         fourier in=galaxy powerout=galpow hermout=ft\_gal fillval=mean
      }{
         Makes an Hermitian Fourier transform stored in an NDF called
         ft\_gal from the two-dimensional NDF called galaxy.  Any bad values in
         galaxy are replaced by the mean data value of galaxy.  In
         addition the power of the transform is written to an NDF
         called galpow.
      }
      \sstexamplesubsection{
         fourier realin=real\_gal out=galaxy inverse
      }{
         Takes the real component of a Fourier transform stored in an
         NDF called real\_gal and performs the inverse transformation to
         yield a normal image in NDF galaxy.
      }
   }
   \sstnotes{
      \sstitemlist{

         \sstitem
         See the NAG documentation, Chapter C06, and/or \KAPLIBS\ routine
         \texttt{kpg1\_hmltx.gen} for more details of Hermitian Fourier transforms.
       }
   }
   \sstdiytopic{
      Related Applications
   }{
KAPPA: \htmlref{CONVOLVE}{CONVOLVE},
\htmlref{LUCY}{LUCY},
\htmlref{MEM2D}{MEM2D},
\htmlref{WIENER}{WIENER};
\xref{FIGARO}{sun86}{}: \xref{BFFT}{sun86}{BFFT},
CMPLX$\lsk$,
\xref{COSBELL}{sun86}{COSBELL},
\xref{FFT}{sun86}{FFT},
$\lsk$2CMPLX.
   }
   \sstimplementationstatus{
      \sstitemlist{

         \sstitem
         \htmlref{AXIS}{apndf:axis}, \htmlref{VARIANCE}{apndf:variance}~ and \htmlref{QUALITY}{apndf:quality}~ are not propagated from the input to
         output NDFs, but the \htmlref{LABEL}{apndf:label}, \htmlref{TITLE}{apndf:title}, \htmlref{HISTORY}{apndf:history}~ components and all
         \htmlref{extensions}{se:ndfext} are.  Arithmetic is performed using single- or
         double-precision floating point, as appropriate for the type of
         the data array.
      }
   }
}
\sstroutine{
   GAUSMOOTH
}{
   Smooths a one- or two-dimensional image using a Gaussian filter
}{
   \sstdescription{
      This application smooths an  \NDFref{NDF} using a one- or
      two-dimensional symmetrical Gaussian point spread function (PSF)
      of specified  width, or widths and orientation.  Each output pixel
      is the PSF-weighted mean of the input pixels within the filter box.

      The NDF may have up to three dimensions.  If it has three
      dimensions, then the filter is applied in turn to each plane in
      the cube and the result written to the corresponding plane in the
      output cube.  The orientation of the smoothing plane can be
      specified using the AXES parameter.

   }
   \sstusage{
      gausmooth in out fwhm
   }
   \sstparameters{
      \sstsubsection{
         AXES(2) = \_INTEGER (Read)
      }{
         This parameter is only accessed if the NDF has exactly three
         significant pixel axes.  It should be set to the indices of the NDF
         pixel axes which span the plane in which smoothing is to be
         applied. All pixel planes parallel to the specified plane will
         be smoothed independently of each other.  The dynamic default
         is the indices of the first two significant axes in the NDF.  \texttt{[]}
      }
      \sstsubsection{
         BOX() = \_INTEGER (Read)
      }{
         The \textit{x} and \textit{y} sizes (in pixels) of the rectangular region over
         which the Gaussian PSF should be applied at each point.  The
         smoothing PSF will be set to zero outside this rectangle,
         which should therefore be sufficiently large not to truncate
         the PSF too early.  A square region is defined should only one
         size be given.  For a one-dimensional or circular Gaussian a
         second size is ignored.  Two values are expected when an
         elliptical PSF is requested (see the description of Parameter
         FWHM).

         The values given will be rounded up to positive odd integers
         if necessary.  If a null (\texttt{{!}}) value is supplied, the value used
         is just sufficient to accommodate the Gaussian PSF out to a
         radius of 3 standard deviations.  Note that the time taken to
         perform the smoothing increases in approximate proportion to
         the value of this parameter for a circular Gaussian, and in
         proportion to the product of the two box sizes for an
         elliptical Gaussian.  \texttt{[!]}
      }
      \sstsubsection{
         FWHM() = \_REAL (Read)
      }{
         This specifies whether a circular or elliptical Gaussian
         point-spread function is used in smoothing a two-dimensional
         image.  If one value is given it is the full-width at
         half-maximum of a one-dimensional or circular Gaussian PSF.
         (Indeed only one value is permitted for a one-dimensional
         array.)  If two values are supplied, this parameter becomes the
         full-width at half-maximum of the major and minor axes of an
         elliptical Gaussian PSF.  Values between 0.1 and 10000.0 pixels
         should be given.  Note that unless a non-default value is
         specified for the BOX parameter, the time taken to perform the
         smoothing will increase in approximate proportion to the
         value(s) of FWHM.  The suggested default is the current value.
      }
      \sstsubsection{
         IN = NDF (Read)
      }{
         The input NDF containing the one-, two-, or three-dimensional
         image to which Gaussian smoothing is to be applied.
      }
      \sstsubsection{
         ORIENT = \_REAL (Read)
      }{
         The orientation of the major axis of the elliptical Gaussian
         PSF, measured in degrees in an anti-clockwise direction from
         the \textit{x} axis of the NDF.  ORIENT is not obtained if FWHM has one
         value, \emph{i.e.} a circular Gaussian PSF will be used to smooth the
         image, or the input NDF is one-dimensional.  The suggested
         default is the current value.
      }
      \sstsubsection{
         OUT = NDF (Write)
      }{
         The output NDF which is to contain the smoothed image.
      }
      \sstsubsection{
         TITLE = LITERAL (Read)
      }{
         The title for the output NDF.  A null value will cause
         the title of the input NDF to be used.  \texttt{[!]}
      }
      \sstsubsection{
         WLIM = \_DOUBLE (Read)
      }{
         If the input image contains bad pixels, then this parameter
         may be used to determine the number of good pixels which must
         be present within the PSF area before a valid output pixel is
         generated.  It can be used, for example, to prevent output
         pixels from being generated in regions where good pixels are
         only present in the wings of the PSF.

         By default, a null (\texttt{{!}}) value is used for WLIM, which causes
         the pattern of bad pixels to be propagated from the input
         image to the output image unchanged.  In this case, smoothed
         output values are only calculated for those pixels which are
         not bad in the input image.

         If a numerical value is given for WLIM, then it specifies the
         minimum PSF-weighted fraction of good pixels which must be
         present in the PSF area (\emph{i.e.} box) in order to generate a good
         output pixel.  The maximum value, in the absence of bad
         pixels, is unity.  If the specified minimum fraction of good
         input pixels is not present, then a bad output pixel will
         result, otherwise a smoothed output value will be calculated.
         The value of this parameter should lie between 1E-6 and 1.0.
         \texttt{[!]}
      }
   }
   \sstexamples{
      \sstexamplesubsection{
         gausmooth image1 image2 5.0
      }{
         Smooths the two-dimensional image held in the NDF structure
         image1 using a symmetrical Gaussian PSF with a full-width at
         half-maximum of 5 pixels.  The smoothed image is written to
         image2.  If any pixels in the input image are bad, then the
         corresponding pixels in the output image will also be bad.
      }
      \sstexamplesubsection{
         gausmooth spectrum1 spectrum2 5.0 box=9
      }{
         Smooths the one-dimensional image held in the NDF structure
         spectrum1 using a symmetrical Gaussian PSF with a full-width
         at half-maximum of 5, and is evaluated over a length of 9
         pixels.  The smoothed image is written to spectrum2.  If any
         pixels in the input image are bad, then the corresponding
         pixels in the output image will also be bad.
      }
      \sstexamplesubsection{
         gausmooth in=a out=b fwhm=3.5 box=31
      }{
         Smooths the two-dimensional image held in the NDF structure a,
         writing the result into the structure b.  The Gaussian
         smoothing PSF has a full-width at half-maximum of 3.5 pixels
         and is evaluated over a large square of size 31x31 pixels.
      }
      \sstexamplesubsection{
         gausmooth in=a out=b fwhm=[4,3] orient=52.7 box=[29,33]
      }{
         Smooths the two-dimensional image held in the NDF structure a,
         writing the result into the structure b.  The elliptical
         Gaussian smoothing PSF has full-width at half-maximum of 4
         pixels along its major axis and three pixels along its minor
         axis, and is evaluated over a large rectangle of size 29x33
         pixels.  The major axis of the PSF is oriented 52.7 degrees
         anti-clockwise from the \textit{x} axis of the data array.
      }
      \sstexamplesubsection{
         gausmooth ngc1097 ngc1097s fwhm=7.2 wlim=0.1
      }{
         Smooths the specified image data using a Gaussian PSF with a
         full-width at half-maximum of 7.2.  An output value is
         calculated for any pixel for which the PSF-weighted fraction
         of good input pixels is at least 0.1.  This will cause the
         smoothing operation to fill in moderately sized regions of bad
         pixels.
      }
   }
   \sstdiytopic{
      Timing
   }{
      For a circular PSF, the execution time is approximately
      proportional to the number of pixels in the image to be smoothed
      and to the value given for the BOX parameter.  By default, this
      latter value is proportional to the value given for FWHM.  For an
      elliptical PSF, the execution time is approximately proportional
      to the number of pixels in the image to be smoothed and to the
      product of the values given for the BOX parameter.  By default,
      these latter values are approximately proportional to the values
      given for FWHM.  Execution time will be approximately doubled if
      a variance array is present in the input NDF.
   }
   \sstdiytopic{
      Related Applications
   }{
KAPPA: \htmlref{BLOCK}{BLOCK},
\htmlref{CONVOLVE}{CONVOLVE},
\htmlref{FFCLEAN}{FFCLEAN},
\htmlref{MATHS}{MATHS},
\htmlref{MEDIAN}{MEDIAN},
\htmlref{PSF}{PSF};
\xref{FIGARO}{sun86}{}: \xref{ICONV3}{sun86}{ICONV3},
\xref{ISMOOTH}{sun86}{ISMOOTH},
\xref{IXSMOOTH}{sun86}{IXSMOOTH},
\xref{MEDFILT}{sun86}{MEDFILT}.
   }
   \sstimplementationstatus{
      \sstitemlist{

         \sstitem
         This routine correctly processes the \htmlref{AXIS}{apndf:axis}, DATA, \htmlref{QUALITY}{apndf:quality},
         \htmlref{VARIANCE}{apndf:variance}, \htmlref{LABEL}{apndf:label}, \htmlref{TITLE}{apndf:title}, \htmlref{UNITS}{apndf:units}, \htmlref{WCS}{apndf:wcs}, and \htmlref{HISTORY}{apndf:history}~ components of the
         input NDF and propagates all \htmlref{extensions}{apndf:extensions}.

         \sstitem
         Processing of \htmlref{bad pixels}{se:masking} and automatic \htmlref{quality masking}{se:qualitymask} are
         supported.  The \htmlref{bad-pixel flag}{setbad:badpixelflag}~ is also written for the data and variance arrays.

         \sstitem
         All \htmlref{non-complex numeric data types}{ap:HDStypes} can be handled.  Arithmetic
         is performed using single-precision floating point, or double
         precision, if appropriate.
      }
   }
}

\sstroutine{
   GDCLEAR
}{
   Clears a graphics device and purges its database entries
}{
   \sstdescription{
      This application software resets a graphics device.  In effect
      the device is cleared.  It purges the graphics-database entries
      for the device.  Optionally, only the current picture is cleared
      and the database unchanged. (Note the clearing of the current
      picture may not work on some graphics devices.)
   }
   \sstusage{
      gdclear [device] [current]
   }
   \sstparameters{
      \sstsubsection{
         CURRENT = \_LOGICAL (Read)
      }{
         If \texttt{TRUE} then only the current picture is cleared.  \texttt{[FALSE]}
      }
      \sstsubsection{
         DEVICE = \htmlref{DEVICE}{se:selgradev} (Read)
      }{
         The graphics device to be cleared.
         \texttt{[}Current graphics device\texttt{{]}}
      }
   }
   \sstexamples{
      \sstexamplesubsection{
         gdclear
      }{
         Clears the \htmlref{current graphics device}{se:devglobal} and purges its graphics-database
         entries.
      }
      \sstexamplesubsection{
         gdclear current
      }{
         Clears the current picture on the current graphics device.
      }
      \sstexamplesubsection{
         gdclear xw
      }{
         Clears the xw device and purges its graphics-database entries.
      }
   }
   \sstdiytopic{
      Related Applications
   }{
KAPPA: \htmlref{GDSET}{GDSET},
\htmlref{GDSTATE}{GDSTATE}.
   }
}

\sstroutine{
   GDNAMES
}{
   Shows which graphics devices are available
}{
   \sstdescription{
      The routine displays a list of the graphics devices available and
      the names (both traditional Starlink GNS names and the equivalent
      \PGPLOT\  ~names) which identify them.  Each name is accompanied by a
      brief descriptive comment.
   }
   \sstusage{
      gdnames
   }
}

\sstroutine{
   GDSET
}{
   Selects a current graphics device
}{
   \sstdescription{
      This application selects a \htmlref{current graphics device}{se:devglobal}.  This
      device will be used for all applications requiring an
      graphics device until changed explicitly.
   }
   \sstusage{
      gdset device
   }
   \sstparameters{
      \sstsubsection{
         DEVICE = \htmlref{DEVICE}{se:selgradev} (Read)
      }{
         The graphics device to become the current graphics device.
      }
   }
   \sstexamples{
      \sstexamplesubsection{
         gdset xwindows
      }{
         Makes the xwindows device the current graphics device.
      }
   }
}

\sstroutine{
   GDSTATE
}{
   Shows the current status of a graphics device
}{
   \sstdescription{
      This application displays information about the current
      \htmlref{graphics database}{se:agitate} picture on a graphics device,
      including the extreme axis values in any requested
      \htmlref{co-ordinate Frame}{se:domains}~ (see Parameter FRAME).
      Information is written to various output parameters for use by
      other applications, and is also written to the screen by default
      (see Parameter REPORT).  An outline may be drawn around the current
      picture if required (see Parameter OUTLINE).

      A list of the colours in the current palette is also produced.

   }
   \sstusage{
      gdstate [device] [frame]
   }
   \sstparameters{
      \sstsubsection{
         COMMENT = LITERAL (Write)
      }{
         The comment of the current picture.  Up to 132 characters
         will be written.
      }
      \sstsubsection{
         DESCRIBE = \_LOGICAL (Read)
      }{
         If \texttt{TRUE}, a detailed description is displayed of the co-ordinate
         Frame in which the picture bounds are reported (see Parameter
         FRAME).  \texttt{[}current value\texttt{{]}}
      }
      \sstsubsection{
         DEVICE = \htmlref{DEVICE}{se:selgradev} (Read)
      }{
         Name of the graphics device about which information is
         required.  \texttt{[}Current graphics device\texttt{{]}}
      }
      \sstsubsection{
         EPOCH = \_DOUBLE (Read)
      }{
         If a `Sky Co-ordinate System' specification is supplied (using
         Parameter FRAME) for a celestial co-ordinate system, then an
         epoch value is needed to qualify it.  This is the epoch at
         which the displayed sky co-ordinates were determined.  It should
         be given as a decimal years value, with or without decimal places
         (\texttt{"1996.8"} for example).  Such values are interpreted as a Besselian
         epoch if less than 1984.0 and as a Julian epoch otherwise.
      }
      \sstsubsection{
         FRAME = LITERAL (Read)
      }{
         A string determining the co-ordinate Frame in which the bounds
         of the current picture are to be reported.  When a picture is
         created by an application such as \htmlref{PICDEF}{PICDEF},
         \htmlref{DISPLAY}{DISPLAY}, the \htmlref{WCS}{apndf:wcs}
         information describing the available co-ordinate systems are stored
         with the picture in the graphics database.  This application can
         report bounds in any of the co-ordinate Frames stored with the
         current picture.  The string supplied for FRAME can be one of the
         following:

         \ssthitemlist{

            \sstitem
            A \htmlref{domain name}{re:domains}~ such as \htmlref{SKY, AXIS, PIXEL, NDC, BASEPIC, CURPIC}{se:resdoms}.
            The special domain AGI\_WORLD is used to refer to the world co-ordinate
            system stored in the AGI graphics database.  This can be useful if
            no WCS information was store with the picture when it was created.

            \sstitem
            An integer value giving the index of the required Frame.

            \sstitem
            An IRAS90 \emph{Sky Co-ordinate System} (SCS) values such as
            \texttt{"EQUAT(J2000)"} (see \xref{SUN/163}{sun163}{}).

         }
         If a null value (\texttt{{!}}) is supplied, bounds are reported in the
         co-ordinate Frame which was current when the picture was created.
         \texttt{[!]}
      }
      \sstsubsection{
         OUTLINE = \_LOGICAL (Read)
      }{
         If OUTLINE is \texttt{TRUE}, then an outline will be drawn around the
         current picture to indicate its position.  \texttt{[FALSE]}
      }
      \sstsubsection{
         REPORT = \_LOGICAL (Read)
      }{
         If this is \texttt{FALSE}, the state of the graphics device is not
         reported, merely the results are written to the output
         parameters.  It is intended for use within procedures.  \texttt{[TRUE]}
      }
      \sstsubsection{
         STYLE = \htmlref{GROUP}{se:groups} (Read)
      }{
         A group of attribute settings describing the plotting style to use
         when drawing the outline (see Parameter OUTLINE).  The format
         of the axis values reported on the screen may also be controlled.

         A comma-separated list of strings should be given in which each
         string is either an attribute setting, or the name of a text
         file preceded by an up-arrow character \texttt{"$\wedge$"}.  Such text files
         should contain further comma-separated lists which will be
         read and interpreted in the same manner.  Attribute settings
         are applied in the order in which they occur within the list,
         with later settings overriding any earlier settings given for
         the same attribute.

         Each individual attribute setting should be of the form:

            $<$name$>$=$<$value$>$

         where $<$name$>$ is the name of a plotting attribute, and $<$value$>$
         is the value to assign to the attribute.  Default values will be
         used for any unspecified attributes.  All attributes will be
         defaulted if a null value (\texttt{{!}})---the initial default---is supplied.
         To apply changes of style to only the current invocation, begin these
         attributes with a plus sign.  A mixture of persistent and temporary
         style changes is achieved by listing all the persistent attributes
         followed by a plus sign then the list of temporary attributes.

         See \slhyperref{Plotting Attributes}{Section~}{}{ap:plotting_attr}
         for a description of the available attributes.  Any unrecognised
         attributes are ignored (no error is reported).

         The appearance of the outline is controlled by the attributes
         \htmlattref{Colour(Border)}{Colour(element)}, \latex{\goodbreak}
         \htmlattref{Width(Border)}{Width(element)}, \emph{etc.} (the synonym
         \att{Outline} may be used in place of \att{Border}).  In addition,
         the following attributes
         may be set in order to control the appearance of the formatted axis
         values reported on the screen: \htmlattref{Format}{Format(axis)},
         \htmlattref{Digits}{Digits/Digits(axis)},
         \htmlattref{Symbol}{Symbol(axis)}, \htmlattref{Unit}{Unit(axis)}.  These
         may be suffixed with an axis number (\emph{e.g.} \att{Digits(2)}) to refer to
         the values displayed for a specific axis.  \texttt{[}current value\texttt{{]}}
      }
   }
   \sstresparameters{
      \sstsubsection{
         DOMAIN = LITERAL (Write)
      }{
         The Domain name of the current co-ordinate Frame for the current
         picture.
      }
      \sstsubsection{
         LABEL = LITERAL (Write)
      }{
         The label of the current picture.  It is blank if there is no
         label.
      }
      \sstsubsection{
         NAME = LITERAL (Write)
      }{
         The name of the current picture.
      }
      \sstsubsection{
         REFNAM = LITERAL (Write)
      }{
         The reference object associated with the current picture.  It
         is blank if there is no reference object.  Up to 132 characters
         will be written.
      }
      \sstsubsection{
         X1 = LITERAL (Write)
      }{
         The lowest value found within the current picture for Axis 1 of the
         requested co-ordinate Frame (see Parameter FRAME).
      }
      \sstsubsection{
         X2 = LITERAL (Write)
      }{
         The highest value found within the current picture for Axis 1 of the
         requested co-ordinate Frame (see Parameter FRAME).
      }
      \sstsubsection{
         Y1 = LITERAL (Write)
      }{
         The lowest value found within the current picture for Axis 2 of the
         requested co-ordinate Frame (see Parameter FRAME).
      }
      \sstsubsection{
         Y2 = LITERAL (Write)
      }{
         The highest value found within the current picture for Axis 2 of the
         requested co-ordinate Frame (see Parameter FRAME).
      }
   }
   \sstexamples{
      \sstexamplesubsection{
         gdstate
      }{
         Shows the status of the \htmlref{current graphics device}{se:devglobal}.  The bounds of
         the picture are displayed in the current co-ordinate Frame of
         the picture.
      }
      \sstexamplesubsection{
         gdstate ps\_l basepic
      }{
         Shows the status of the ps\_l device.  The bounds of the picture
         are displayed in the BASEPIC Frame (normalised device co-ordinates
         in which the short of the two dimensions of the display surface
         has length 1.0).
      }
      \sstexamplesubsection{
         gdstate outline frame=pixel style="'colour=red,width=3'"
      }{
         Shows the status of the current graphics device and draws a
         thick, red outline around the current database picture.  The
         bounds of the picture are displayed in the PIXEL co-ordinate
         Frame (if available).
      }
      \sstexamplesubsection{
         gdstate refnam=(ndfname)
      }{
         Shows the status of the current graphics device.  If there
         is a reference data object, its name is written to the ICL
         variable NDFNAME.
      }
      \sstexamplesubsection{
         gdstate x1=(x1) x2=(x2) y1=(y1) y2=(y2) frame=basepic
      }{
         Shows the status of the current graphics device.  The bounds
         of the current picture in normalised device co-ordinates
         are written to the ICL variables: X1, X2, Y1, Y2.
      }
   }
   \sstnotes{
      \sstitemlist{

         \sstitem
         The displayed bounds are the extreme axis values found anywhere
         within the current picture.  In some situations these extreme
         values may not occur on the edges of the picture.  For instance, if
         the current picture represents a region including the north
         celestial pole, then displaying the picture bounds in celestial
         co-ordinates will give a declination upper limit of $+$90 degrees,
         whilst the RA limits will be 0 hours and (close to) 24 hours.

         \sstitem
         Previous versions of this application reported bounds in
         \htmlref{`Normalised Device Co-ordinates'}{se:agiframes}
         \latex{(see Section~\ref{se:agiframes})}.  Similar functionality is now
         provided by setting Parameter FRAME to \texttt{"BASEPIC"}.  Be aware though,
         that Normalised Device Co-ordinates were normalised so that the
         longer of the two axes had a length of 1.0, but BASEPIC co-ordinates
         are normalised so that the shorter of the two axes has length 1.0.

         \sstitem
         The `NDC' Frame is now a normalized co-ordinate system
         in which each axis of the graphics device has unit length.

      }
   }
   \sstdiytopic{
      Related Applications
   }{
KAPPA: \htmlref{GDSET}{GDSET},
\htmlref{GDCLEAR}{GDCLEAR}.
   }
}
\sstroutine{
   GLITCH
}{
   Replaces bad pixels in a two-dimensional NDF with the local median
}{
   \sstdescription{
      This routine removes bad pixels from a two-dimensional \NDFref{NDF}, replacing them with
      the median of the eight (or less at the edges) neighbouring pixels.
      At least three of these eight neighbouring pixels must have good
      values (that is, they must not set to the bad value) otherwise the
      resultant pixel becomes bad.

      The positions of the pixels to be removed can be supplied in four
      ways (see Parameter MODE):

      \sstitemlist{

         \sstitem
         In response to parameter prompts.  A single bad pixel position is
         supplied at each prompt, and the user is re-prompted until a null value
         is supplied.

         \sstitem
         Within a positions list such as produced by applications CURSOR,
         LISTMAKE.

         \sstitem
         Within a simple text file.  Each line contains the position of a
         pixel to be replaced.

         \sstitem
         Alternatively, each bad pixel in the input NDF can be used
         (subject to the above requirement that at least three out of the
         eight neighbouring pixels are not bad).
      }
   }
   \sstusage{
      glitch in out [title]
        $\left\{ {\begin{tabular}{l}
                  incat=? \\
                  infile=? \\
                  pixpos=?
                  \end{tabular} }
        \right.$
        \newline\latexhtml{\hspace*{11em}}{~~~~~~~~~~~~~~~~~~~~}
        \makebox[0mm][c]{\small mode}
   }
   \sstparameters{
      \sstsubsection{
         IN  =  NDF (Read)
      }{
         The input image.
      }
      \sstsubsection{
         INCAT = FILENAME (Read)
      }{
         A catalogue containing a positions list giving the pixels
         to be replaced, such as produced by applications CURSOR, LISTMAKE.
         Only accessed if Parameter MODE is given the value \texttt{"Catalogue"}.
      }
      \sstsubsection{
         INFILE = FILENAME (Read)
      }{
         The name of a text file containing the positions of the pixels
         to be replaced.  The positions should be given in the current
         \htmlref{co-ordinate Frame}{se:domains}~  of the input NDF, one per line.  Spaces or
         commas can be used as delimiters between axis values.  The file
         may contain comment lines with the first character \texttt{\#}
         or \texttt{!}.  This parameter is only used if Parameter MODE is
         set to \texttt{"File"}.
      }
      \sstsubsection{
         MODE = \htmlref{LITERAL}{se:parmenu} (Read)
      }{
         The method used to obtain the positions of the pixels to be
         replaced.  The supplied string can be one of the following
         options.

         \ssthitemlist{

            \sstitem
            \texttt{"Bad"} --- The bad pixels in the input NDF are used.

            \sstitem
            \texttt{"Catalogue"} --- Positions are obtained from a positions list
               using Parameter INCAT.

            \sstitem
            \texttt{"File"} --- The pixel positions are read from a text file specified
               by Parameter INFILE.

            \sstitem
            \texttt{"Interface"} --- The position of each pixel is obtained using
               Parameter PIXPOS.  The number of positions supplied must not
               exceed 200.

         }
         \texttt{[}current value\texttt{{]}}
      }
      \sstsubsection{
         OUT  =  NDF (Write)
      }{
         The output image.
      }
      \sstsubsection{
         PIXPOS = LITERAL (Read)
      }{
         The position of a pixel to be replaced, in the current
         co-ordinate Frame of the input NDF.  Axis values should be
         separated by spaces or commas.  This parameter is only used if
         Parameter MODE is set to \texttt{"Interface"}.  If a value is supplied on
         the command line, then the application exits after processing the
         single specified pixel.  Otherwise, the application loops to
         obtain multiple pixels to replace, until a null (\texttt{{!}}) value is
         supplied.  Entering a colon (\texttt{":"}) will result in a description of
         the required co-ordinate Frame being displayed, followed by a
         prompt for a new value.
      }
      \sstsubsection{
         TITLE = LITERAL (Read)
      }{
         Title for the output image.  A null value (\texttt{{!}}) propagates the title from the input image to the output image.  \texttt{[!]}
      }
   }
   \sstexamples{
      \sstexamplesubsection{
         glitch m51 cleaned mode=cat incat=badpix.FIT
      }{
         Reads pixel positions from the positions list stored in the FITS
         file badpix.FIT, and replaces the corresponding pixels in the
         two-dimensional NDF m51 by the median of the surrounding neighbouring
         pixels.  The cleaned image is written to cleaned.sdf.
      }
   }
   \sstnotes{
      \sstitemlist{

         \sstitem
         If the current co-ordinate Frame of the input NDF is not PIXEL,
         then the supplied positions are first mapped into the PIXEL Frame
         before being used.
      }
   }
   \sstdiytopic{
      Related Applications
   }{
KAPPA: \htmlref{ARDMASK}{ARDMASK},
\htmlref{CHPIX}{CHPIX},
\htmlref{FILLBAD}{FILLBAD},
\htmlref{ZAPLIN}{ZAPLIN},
\htmlref{NOMAGIC}{NOMAGIC},
\htmlref{REGIONMASK}{REGIONMASK},
\htmlref{SEGMENT}{SEGMENT},
\htmlref{SETMAGIC}{SETMAGIC};
\xref{FIGARO}{sun86}{}: \xref{CSET}{sun86}{CSET},
\xref{ICSET}{sun86}{ICSET},
\xref{NCSET}{sun86}{NCSET},
\xref{TIPPEX}{sun86}{TIPPEX}.
   }
   \sstimplementationstatus{
      \sstitemlist{

         \sstitem
         This routine correctly processes the \htmlref{AXIS}{apndf:axis}, DATA, \htmlref{QUALITY}{apndf:quality},
         \htmlref{VARIANCE}{apndf:variance}, \htmlref{LABEL}{apndf:label}, \htmlref{TITLE}{apndf:title}, \htmlref{UNITS}{apndf:units}, \htmlref{WCS}{apndf:wcs}, and \htmlref{HISTORY}{apndf:history}~ components of the
         input NDF and propagates all \htmlref{extensions}{apndf:extensions}.

         \sstitem
         Processing of \htmlref{bad pixels}{se:masking} and automatic \htmlref{quality masking}{se:qualitymask} are
         supported.

         \sstitem
         Only single- and double-precision floating-point data can be
         processed directly.  All integer data will be converted to floating
         point before being processed.
      }
   }
}

\sstroutine{
   GLOBALS
}{
   Displays the values of the \KAPPA\ global parameters
}{
   \sstdescription{
      This procedure lists the meanings and values of the \KAPPA\ ~global
      parameters.  If a global parameter does not have a value, the
      string \texttt{"<undefined>"} is substituted where the value would have been
      written.
   }
   \sstusage{
      globals
   }
}
\sstroutine{
   HISCOM
}{
   Adds commentary to the history of an NDF
}{
   \sstdescription{
      This task allows application-independent commentary to be added
      to the \htmlref{history}{se:ndfhistory}~ records of an \NDFref{NDF}.  The
      text may be read from a text file or obtained through a parameter.
   }
   \sstusage{
      hiscom ndf [mode]
        $\left\{ {\begin{tabular}{l}
                  file=? \\
                  comment=?
                  \end{tabular} }
        \right.$
        \newline\latexhtml{\hspace*{8.85em}}{~~~~~~~~~~~~~~~~}
        \makebox[0mm][c]{\small mode}
   }
   \sstparameters{
      \sstsubsection{
         APPNAME = LITERAL (Read)
      }{
         The application name to be recorded in the new history record.
         If a null value (\texttt{{!}}) is supplied, a default of
         \texttt{"HISCOM"} is used and the new history record describes the
         parameter values supplied when HISCOM was invoked. If a non-null
         value is supplied, the new history record refers to the specified
         application name instead of \texttt{"HISCOM"} and does not describe the
         HISCOM parameter values. \texttt{[!]}
      }
      \sstsubsection{
         COMMENT = LITERAL (Read)
      }{
         A line of commentary limited to 72 characters.  If the value is
         supplied on the command line only that line of commentary will
         be written into the history.  Otherwise repeated prompting
         enables a series of commentary lines to be supplied.  A null
         value (\texttt{{!}}) terminates the loop.  Blank lines delimit
         paragraphs.  Paragraph wrapping is enabled by Parameter WRAP.
         There is no suggested default to allow more room for entering
         the value.
      }
      \sstsubsection{
         DATE =  LITERAL (Read)
      }{
         The date and time to associated with the new history record.
         Normally, a null (\texttt{{!}}) value should be supplied, in which case
         the current UTC date and time will be used.  If a value is
         supplied, it should be in one of the following forms.

         \ssthitemlist{

            \sstitem
            Gregorian Calendar Date --- With the month expressed either
            as an integer or a three-character abbreviation, and with
            optional decimal places to represent a fraction of a day
            (\texttt{"1996-10-2"} or \texttt{"1996-Oct-2.6"} for example).
            If no fractional part of a day is given, the time refers to
            the start of the day (zero hours).

            \sstitem
            Gregorian Date and Time --- Any calendar date (as above)
            but with a fraction of a day expressed as hours, minutes and
            seconds (\texttt{"1996-Oct-2 12:13:56.985"} for example).
            The date and  time can be separated by a space or by a
            \texttt{"T"} (as used by ISO~8601 format).

            \sstitem
            Modified Julian Date --- With or without decimal places
            (\texttt{"MJD 54321.4"} for example).

            \sstitem
            Julian Date --- With or without decimal places
            (\texttt{"JD 2454321.9"} for example).
         }
         \texttt{[!]}
      }
      \sstsubsection{
         FILE =  FILENAME (Read)
      }{
         Name of the text file containing the commentary.  It is only
         accessed if MODE=\texttt{"File"}.
      }
      \sstsubsection{
         MODE = \htmlref{LITERAL}{se:parmenu} (Read)
      }{
         The \htmlref{interaction mode}{se:interaction}.  The allowed values are
         described below.

            \texttt{"File"}      ---  The commentary is to be read from a text
                             file.  The formatting and layout of the
                             text is preserved in the history unless
                             WRAP=\texttt{TRUE} and there are lines longer than
                             the width of the history records.

            \texttt{"Interface"} ---  The commentary is to be supplied through a
                             parameter.  See Parameter COMMENT.

         \texttt{["Interface"]}
      }
      \sstsubsection{
         NDF = (Read and Write)
      }{
         The NDF for which commentary is to be added to the history.
      }
      \sstsubsection{
         WRAP = \_LOGICAL (Read)
      }{
         WRAP=\texttt{TRUE} requests that the paragraphs of comments are wrapped
         to make as much text fit on to each line of the history record
         as possible.  WRAP=\texttt{FALSE} means that the commentary text beyond
         the width of the history records (72 characters) is lost.  If a
         null (\texttt{{!}}) value is supplied, the value used is \texttt{TRUE} when
         MODE=\texttt{"Interface"} and \texttt{FALSE} if MODE=\texttt{"File"}.  \texttt{[!]}
      }
   }
   \sstexamples{
      \sstexamplesubsection{
         hiscom frame256 comment="This image has a non-uniform background"
      }{
         This adds the comment \texttt{"This image has a non-uniform background"}
         to the history records of the NDF called frame256.
      }
      \sstexamplesubsection{
         hiscom ndf=eso146-g14 comment="This galaxy is retarded" mode=i
      }{
         This adds the comment \texttt{"This galaxy is retarded"} to the history
         records of the NDF called eso146-g14.
      }
      \sstexamplesubsection{
         hiscom hh14\_k file file=ircam\_info.lis
      }{
         This reads the file \texttt{ircam\_info.lis} and places the text
         contained therein into the history records of the NDF called
         hh14\_k.  Any lines longer than 72 characters are truncated to
         that length.
      }
      \sstexamplesubsection{
         hiscom hh14\_k file file=ircam\_info.lis wrap
      }{
         As the previous example except the text in each paragraph is
         wrapped to a width of 72 characters within the history
         records.
      }
   }
   \sstnotes{
      \sstitemlist{

         \sstitem
         A \htmlref{HISTORY component}{apndf:history} is created if it
         does not exist within the NDF.  The width of the history record is
         72 characters.

         \sstitem
         An error will result if the current history update mode of the
         NDF is \texttt{"Disabled"}, and no commentary is written.  Otherwise the
         commentary is written at the priority equal to the current
         history update mode.

         \sstitem
         A warning messages (at the normal reporting level) is issued
         if lines in the text file are too long for the history record and
         WRAP=\texttt{FALSE}, though the first 72 characters are stored.

         \sstitem
         The maximum line length in the file is 200 characters.

         \sstitem
         Paragraphs should have fewer than 33 lines.  Longer ones will
         be divided.
      }
   }
}
\sstroutine{
   HISLIST
}{
   Lists NDF history records
}{
   \sstdescription{
      This lists all the \htmlref{history}{se:ndfhistory}~ records in an \NDFref{NDF}.  The reported
      information comprises the date, time, and application name,
      and optionally the history text.
   }
   \sstusage{
      hislist ndf
   }
   \sstparameters{
      \sstsubsection{
         BRIEF = \_LOGICAL (Read)
      }{
         This controls whether a summary or the full history information
         is reported.\latex{\goodbreak}  BRIEF=\texttt{TRUE} requests that only the date and
         application name in each history record is listed.  BRIEF=\texttt{FALSE}
         causes the task to report the history text in addition.
         \texttt{[FALSE]}
      }
      \sstsubsection{
         NDF = NDF (Read)
      }{
         The NDF whose history information is to be reported.
      }
   }
   \sstexamples{
      \sstexamplesubsection{
         hislist vcc953
      }{
         This lists the full history information for the NDF called
         vcc935.  The information comprises the names of the
         applications and the times they were used, and the associated
         history text.
      }
      \sstexamplesubsection{
         hislist vcc953 brief
      }{
         This gives a summary of the history information for the NDF
         called vcc935.  It comprises the names of the applications
         and the times they were used.
      }
   }
   \sstdiytopic{
      Related Applications
   }{
KAPPA: \htmlref{HISCOM}{HISCOM},
\htmlref{HISSET}{HISSET},
\htmlref{NDFTRACE}{NDFTRACE}.
   }
}
\sstroutine{
   HISSET
}{
   Sets the NDF history update mode
}{
   \sstdescription{
      This task controls the level of \htmlref{history recording}{se:ndfhistory}~ in an \NDFref{NDF},
      and can also erase the history information.

      The level is called the history update mode and it is a permanent
      attribute of the \htmlref{HISTORY component}{apndf:history}~ of the NDF, and remains with
      the NDF and any NDF created therefrom until the history is erased
      or the update mode is modified (say by this task).
   }
   \sstusage{
      hisset ndf [mode] ok=?
   }
   \sstparameters{
      \sstsubsection{
         MODE = \htmlref{LITERAL}{se:parmenu} (Read)
      }{
         The history update mode.  It can take one of the following
         values.
            \begin{description}
            \item \texttt{"Disabled"}  ---  No history recording is to take place.
            \item \texttt{"Erase"}     ---  Erases the history of the NDF.
            \item \texttt{"Normal"}    ---  Normal history recording is required.
            \item \texttt{"Quiet"}     ---  Only brief history information is to be
                             recorded.
            \item \texttt{"Verbose"}   ---  The fullest-possible history information
                             is to be recorded.
            \end{description}

         The suggested default is \texttt{"Normal"}.  \texttt{["Normal"]}
      }
      \sstsubsection{
         NDF = (Read and Write)
      }{
         The NDF whose history update mode to be modified or history
         information erased.
      }
      \sstsubsection{
         OK = \_LOGICAL (Read)
      }{
         This is used to confirm whether or not the history should be
         erased.  OK=\texttt{TRUE} lets the history records be erased; if
         OK=\texttt{FALSE} the history is retained and a message will be issued
         to this effect.
      }
   }
   \sstexamples{
      \sstexamplesubsection{
         hisset final
      }{
         This sets the history-recording level to be normal for the NDF
         called final.
      }
      \sstexamplesubsection{
         hisset final erase ok
      }{
         This erases the history information from the NDF called final.
      }
      \sstexamplesubsection{
         hisset mode=disabled ndf=spectrum
      }{
         This disables history recording in the NDF called spectrum.
      }
      \sstexamplesubsection{
         hisset test42 v
      }{
         This sets the history-recording level to be verbose for the NDF
         called test42 so that the fullest-possible history is included.
      }
      \sstexamplesubsection{
         hisset ndf=test42 mode=q
      }{
         This sets the history-recording level to be quiet for the NDF
         called test42, so that only brief information is recorded.
      }
   }
   \sstnotes{
      \sstitemlist{

         \sstitem
         A HISTORY component is created if it does not exist within the
         NDF, except for MODE=\texttt{"Erase"}.

         \sstitem
         The task records the new history update mode within the
         history records, even if MODE=\texttt{"Disabled"} provided the mode has
         changed.  Thus the history information will show where there may
         be gaps in the recording.
      }
   }
   \sstdiytopic{
      Related Applications
   }{
KAPPA: \htmlref{HISCOM}{HISCOM},
\htmlref{HISLIST}{HISLIST},
\htmlref{NDFTRACE}{NDFTRACE}.
   }
}
\sstroutine{
   HISTAT
}{
   Computes ordered statistics for an NDF's pixels using an
   histogram
}{
   \sstdescription{
      This application computes and displays simple ordered statistics
      for the pixels in an \NDFref{NDF's}  data, quality, error, or variance
      array.  The statistics available are:

      \sstitemlist{

         \sstitem
         the pixel sum,

         \sstitem
         the pixel mean,

         \sstitem
         the pixel median,

         \sstitem
         the pixel mode,

         \sstitem
         the pixel value at selected percentiles,

         \sstitem
         the value and position of the minimum- and maximum-valued
         pixels,

         \sstitem
         the total number of pixels in the NDF,

         \sstitem
         the number of pixels used in the statistics, and

         \sstitem
         the number of pixels omitted.

         The mode may be obtained in different ways (see Parameter METHOD).
      }
   }
   \sstusage{
      histat ndf [comp] [percentiles] [logfile]
   }
   \sstparameters{
      \sstsubsection{
         COMP = \htmlref{LITERAL}{se:parmenu} (Read)
      }{
         The name of the NDF array component for which statistics are
         required.  The options are limited to the arrays within the
         supplied NDF.  In general the value may \texttt{"Data"}, \texttt{"Error"},
         \texttt{"Quality"} or \texttt{"Variance"} (note that \texttt{"Error"} is the alternative
         to \texttt{"Variance"} and causes the square root of the variance
         values to be taken before computing the statistics).  If
         \texttt{"Quality"} is specified, then the quality values are treated as
         numerical values (in the range 0 to 255).  \texttt{["Data"]}
      }
      \sstsubsection{
         LOGFILE = FILENAME (Write)
      }{
         A text file into which the results should be logged.  If a null
         value is supplied (the default), then no logging of results
         will take place.  \texttt{[!]}
      }
      \sstsubsection{
         METHOD = LITERAL (Read)
      }{
         The method used to evaluate the mode.  The choices are as
         follows.

         \ssthitemlist{

            \sstitem
             \texttt{"Histogram"} --- This finds the peak of an optimally
             binned histogram, the mode being the central value of
             that bin.  The number of bins may be altered given
             through Parameter NUMBIN, however it is recommended to
             use the optimal binsize derived from the prescription of
             Freedman \& Diatonis.

            \sstitem
            \texttt{"Moments"} --- As \texttt{"Histogram"} but the mode is
            the weighted centroid from the moments of the peak bin and
            its neighbours.  The neighbours are those bins either side
            of the peak in a continuous sequence whose membership
            exceeds the peak value less three times the Poisson error
            of the peak bin.  Thus it gives an interpolated mode and
            does reduce the effect of noise.

            \sstitem
            \texttt{"Pearson"} --- This uses the 3 * median - 2 * mean
            formula devised by Pearson.  See the first two References.
            This assumes that the median is bracketed by the mode and
            mean and only a mildly skew unimodal distribution.  This
            often applies to an image of the sky.
         }
         \texttt{["Pearson"]}
      }
      \sstsubsection{
         NDF = NDF (Read)
      }{
         The NDF data structure to be analysed.
      }
      \sstsubsection{
         NUMBIN = \_INTEGER (Read)
      }{
         The number of histogram bins to be used for the coarse
         histogram to evaluate the mode.  It is only accessed when
         METHOD={"Histogram"} or \texttt{"Moments"}.  This must lie in
         the range 10 to 10000.  The suggested default is calculated
         dynamically depending on the data spread and number of values
         (using the prescription of Freedman \& Diaconis).  For
         integer data it is advisble to use the dynamic default or an
         integer multiple thereof to avoid creating non-integer wide
         bins.  \texttt{[]}
      }
      \sstsubsection{
         PERCENTILES( 100 ) = \_REAL (Read)
      }{
         A list of percentiles to be found.  None are computed if this
         parameter is null (\texttt{{!}}).  The percentiles must be in the range
         0.0 to 100.0 \texttt{[!]}
      }
   }
   \sstresparameters{
      \sstsubsection{
         MAXCOORD( ) = \_DOUBLE (Write)
      }{
         A one-dimensional array of values giving the \htmlref{WCS}{apndf:wcs}
         co-ordinates of the centre of the (first) maximum-valued pixel found in
         the NDF array.  The number of co-ordinates is equal to the number
         of NDF dimensions.
      }
      \sstsubsection{
         MAXIMUM = \_DOUBLE (Write)
      }{
         The maximum pixel value found in the NDF array.
      }
      \sstsubsection{
         MAXPOS( ) = \_INTEGER (Write)
      }{
         A one-dimensional array of \htmlref{pixel indices}{se:pixgrd}~
         identifying the (first)
         maximum-valued pixel found in the NDF array.  The number of
         indices is equal to the number of NDF dimensions.
      }
      \sstsubsection{
         MAXWCS = LITERAL (Write)
      }{
         The formatted WCS co-ordinates at the maximum pixel value.  The
         individual axis values are comma separated.
      }
      \sstsubsection{
         MEAN = \_DOUBLE (Write)
      }{
         The mean value of all the valid pixels in the NDF array.
      }
      \sstsubsection{
         MEDIAN = \_DOUBLE (Write)
      }{
         The median value of all the valid pixels in the NDF array.
      }
      \sstsubsection{
         MINCOORD( ) = \_DOUBLE (Write)
      }{
         A one-dimensional array of values giving the user co-ordinates of
         the centre of the (first) minimum-valued pixel found in the
         NDF array.  The number of co-ordinates is equal to the number
         of NDF dimensions.
      }
      \sstsubsection{
         MINIMUM = \_DOUBLE (Write)
      }{
         The minimum pixel value found in the NDF array.
      }
      \sstsubsection{
         MINPOS( ) = \_INTEGER (Write)
      }{
         A one-dimensional array of pixel indices identifying the (first)
         minimum-valued pixel found in the NDF array.  The number of
         indices is equal to the number of NDF dimensions.
      }
      \sstsubsection{
         MINWCS = LITERAL (Write)
      }{
         The formatted WCS co-ordinates at the minimum pixel value.  The
         individual axis values are comma separated.
      }
      \sstsubsection{
         MODE = \_DOUBLE (Write)
      }{
         The modal value of all the valid pixels in the NDF array.
         The method used to obtain the mode is governed by Parameter
         METHOD.
      }
      \sstsubsection{
         NUMBAD = \_INTEGER (Write)
      }{
         The number of pixels which were either not valid or were
         rejected from the statistics during iterative $\kappa$-sigma
         clipping.
      }
      \sstsubsection{
         NUMGOOD = \_INTEGER (Write)
      }{
         The number of NDF pixels which actually contributed to the
         computed statistics.
      }
      \sstsubsection{
         NUMPIX = \_INTEGER (Write)
      }{
         The total number of pixels in the NDF (both good and bad).
      }
      \sstsubsection{
         PERVAL() = \_DOUBLE (Write)
      }{
         The values of the percentiles of the good pixels in the NDF
         array.  This parameter is only written when one or more
         percentiles have been requested.
      }
      \sstsubsection{
         TOTAL = \_DOUBLE (Write)
      }{
         The sum of the pixel values in the NDF array.
      }
   }
   \sstexamples{
      \sstexamplesubsection{
         histat image
      }{
         Computes and displays simple ordered statistics for the data
         array in the NDF called image.
      }
      \sstexamplesubsection{
         histat image method=his
      }{
         As above but the mode is the centre of peak bin in the
         optimally distributed histogram rather than sub-bin
         interpolated using neighbouring bins.
      }
      \sstexamplesubsection{
         histat ndf=spectrum variance
      }{
         Computes and displays simple ordered statistics for the
         variance array in the NDF called spectrum.
      }
      \sstexamplesubsection{
         histat spectrum error
      }{
         Computes and displays ordered statistics for the variance
         array in the NDF called spectrum, but takes the square root of
         the variance values before doing so.
      }
      \sstexamplesubsection{
         histat halley logfile=stats.dat method=pearson
      }{
         Computes ordered statistics for the data array in the NDF
         called halley, and writes the results to a logfile called
         \texttt{stats.dat}.  The mode is derived using the Pearson
         formula.
      }
      \sstexamplesubsection{
         histat ngc1333 percentiles=[0.25,0.75]
      }{
         Computes ordered statistics for the data array in the NDF
         called ngc1333, including the quartile values.
      }
   }
   \sstnotes{
      \sstitemlist{

         \sstitem
         Where the histogram contains a few extreme outliers, the
         histogram limits are adjusted to reduce greatly the bias upon
         the statistics, even if a chosen percentile corresponds to an
         extreme outlier.  The outliers are still accounted in the median
         and percentiles.  The histogram normally uses 10000 bins.  For
         small arrays the number of bins is at most a half of the number
         of array elements.  Integer arrays have a minimum bin width of
         one; this can also reduce the number of bins.  The goal is to
         avoid most histogram bins being empty artificially, since the
         sparseness of the histogram is the main criterion for detecting
         outliers.  Outliers can also be removed (flagged) via application
         \htmlref{THRESH}{THRESH} prior to using this application.

         \sstitem
         There is quantisation bias in the statistics, but for
         non-pathological distributions this should be insignificant.
         Accuracy to better than 0.01 of a percentile is normal.  Linear
         interpolation within a bin is used, so the largest errors arise
         near the median.
      }
   }
   \sstdiytopic{
      References
   }{
      \begin{refs}
      \item Moroney, M.J., 1957, {\em Facts from Figures} (Pelican)
      \item Goad, L.E. 1980, {\em Statistical Filtering of Cosmic-Ray Events
        from Astronomical CCD Images} in {\em Applications of Digital Image
        Processing to Astronomy}, SPIE {\bf 264}, 136.
      \item Freedman, D. \& Diaconis, P. 1981, {\em On the histogram as a density
        estimator: L2 theory}, Zeitschrift f"{u}r
        Wahrscheinlichkeitstheorie und verwandte Gebiete {\bf 57}, 453.
      \end{refs}
   }
   \sstdiytopic{
      Related Applications
   }{
KAPPA: \htmlref{HISTOGRAM}{HISTOGRAM},
\htmlref{MSTATS}{MSTATS},
\htmlref{NDFTRACE}{NDFTRACE},
\htmlref{NUMB}{NUMB},
\htmlref{STATS}{STATS};
\xref{ESP}{sun180}{}: \xref{HISTPEAK}{sun180}{HISTPEAK};
\xref{FIGARO}{sun86}{}: \xref{ISTAT}{sun86}{ISTAT}.
   }
   \sstimplementationstatus{
      \sstitemlist{

         \sstitem
         This routine correctly processes the \htmlref{AXIS}{apndf:axis}, DATA, \htmlref{VARIANCE}{apndf:variance},
         \htmlref{QUALITY}{apndf:quality}, \htmlref{TITLE}{apndf:title}, and \htmlref{HISTORY}{apndf:history}~ components of the NDF.

         \sstitem
         Processing of \htmlref{bad pixels}{se:masking} and automatic \htmlref{quality masking}{se:qualitymask} are
         supported.

         \sstitem
         All \htmlref{non-complex numeric data types}{ap:HDStypes} can be handled.  Arithmetic
         is performed using single- or double-precision floating point,
         as appropriate.

         \sstitem
         Any number of NDF dimensions is supported.
      }
   }
}
\sstroutine{
   HISTEQ
}{
   Performs an histogram equalisation on an NDF
}{
   \sstdescription{
      This application transforms an \NDFref{NDF} via histogram equalisation.
      Histogram equalisation is an image-processing technique in which
      the distribution (between limits) of data values in the input
      array is adjusted so that in the output array there are
      approximately equal numbers of elements in each histogram bin.
      To achieve this the histogram bin size is no longer a constant.
      This technique is commonly known as histogram equalisation.  It
      is useful for displaying features across a wide dynamic range,
      sometimes called a maximum-information picture.  The transformed
      array is output to a new NDF.
   }
   \sstusage{
      histeq in out [numbin]
   }
   \sstparameters{
      \sstsubsection{
         IN  = NDF (Read)
      }{
         The NDF structure to be transformed.
      }
      \sstsubsection{
         NUMBIN = \_INTEGER (Read)
      }{
         The number of histogram bins to be used.  This should be a
         large number, say 2000, to reduce quantisation errors.  It
         must be in the range 100 to 10000.  \texttt{[2048]}
      }
      \sstsubsection{
         OUT = NDF (Write)
      }{
         The NDF structure to contain the transformed data array.
      }
      \sstsubsection{
         TITLE = LITERAL (Read)
      }{
         \htmlref{Title}{apndf:title} for the output NDF structure.  A null value (\texttt{{!}})
         propagates the title from the input NDF to the output NDF.  \texttt{[!]}
      }
   }
   \sstexamples{
      \sstexamplesubsection{
         histeq halley maxinf
      }{
         The data array in the NDF called halley is remapped via
         histogram equalisation to form the new NDF called maxinf.
      }
      \sstexamplesubsection{
         histeq halley maxinf 10000 title="Maximum information of Halley"
      }{
         The data array in the NDF called halley is remapped via
         histogram equalisation to form the new NDF called maxinf.
         Ten thousand bins in the histogram are required rather than
         the default of 2048.  The title of NDF maxinf is
         \texttt{"Maximum information of Halley"}.
      }
   }
   \sstnotes{
      If there are a few outliers in the data and most of the points
      concentrated about a value it may be wise to truncate the
      data array via THRESH, or have a large number of histogram bins.
   }
   \sstdiytopic{
      Related Applications
   }{
KAPPA: \htmlref{LAPLACE}{LAPLACE},
\htmlref{LUTABLE}{LUTABLE},
\htmlref{LUTEDIT}{LUTEDIT},
\htmlref{SHADOW}{SHADOW},
\htmlref{THRESH}{THRESH};
\xref{FIGARO}{sun86}{}: \xref{HOPT}{sun86}{HOPT}.
   }
   \sstimplementationstatus{
      \sstitemlist{

         \sstitem
         This routine correctly processes the \htmlref{AXIS}{apndf:axis}, DATA, \htmlref{QUALITY}{apndf:quality},
         \htmlref{LABEL}{apndf:label}, \htmlref{TITLE}{apndf:title}, \htmlref{WCS}{apndf:wcs}, and \htmlref{HISTORY}{apndf:history}~ components of an NDF data structure and
         propagates all \htmlref{extensions}{apndf:extensions}.  \htmlref{UNITS}{apndf:units}~ and \htmlref{VARIANCE}{apndf:variance}~ become undefined
         by the transformation, and so are not propagated.

         \sstitem
         Processing of \htmlref{bad pixels}{se:masking} and automatic \htmlref{quality masking}{se:qualitymask} are
         supported.

         \sstitem
         All \htmlref{non-complex numeric data types}{ap:HDStypes} can be handled.

         \sstitem
         Any number of NDF dimensions is supported.
      }
   }
}



\sstroutine{
   HISTOGRAM
}{
   Computes an histogram of an NDF's values
}{
   \sstdescription{
      This application derives histogram information for an \NDFref{NDF} array
      between specified limits, using either a set number of bins
      (Parameter NUMBIN) or a chosen bin width (Parameter WIDTH).  The
      histogram is reported, and may optionally be written to a text log
      file, and/or plotted graphically.

      By default, each data value contributes a value of one to the
      corresponding histogram bin, but alternative weights may be
      supplied via Parameter WEIGHTS.
   }
   \sstusage{
      histogram in numbin range [comp] [logfile]
   }
   \sstparameters{
      \sstsubsection{
         AXES = \_LOGICAL (Read)
      }{
         \texttt{TRUE} if labelled and annotated axes are to be drawn around the
         plot.  The width of the margins left for the annotation may be
         controlled using Parameter MARGIN.  The appearance of the axes
         (colours, founts, \emph{etc.}) can be controlled using the Parameter
         STYLE.  The dynamic default is \texttt{TRUE} if CLEAR is \texttt{TRUE}, and
         \texttt{FALSE} otherwise. \texttt{[]}
      }
      \sstsubsection{
         CLEAR = \_LOGICAL (Read)
      }{
         If \texttt{TRUE} the current picture is cleared before the plot is
         drawn.  If CLEAR is \texttt{FALSE} not only is the existing plot
         retained, but also an attempt is made to align the new
         picture with the existing picture.  Thus you can generate a
         composite plot within a single set of axes, say using
         different colours or modes to distinguish data from different
         datasets.  \texttt{[TRUE]}
      }
      \sstsubsection{
         COMP = \htmlref{LITERAL}{se:parmenu} (Read)
      }{
         The name of the NDF array component to have its histogram
         computed: \texttt{"Data"}, \texttt{"Error"}, \texttt{"Quality"} or
         \texttt{"Variance"} (where \texttt{"Error"} is the alternative to
         \texttt{"Variance"} and causes the square
         root of the variance values to be taken before computing the
         statistics).  If \texttt{"Quality"} is specified, then the quality
         values are treated as numerical values (in the range 0 to
         255).  \texttt{["Data"]}
      }
      \sstsubsection{
         CUMUL = \_LOGICAL (Read)
      }{
         If \texttt{TRUE} then a cumulative histogram is reported.  \texttt{[FALSE]}
      }
      \sstsubsection{
         DEVICE = \htmlref{DEVICE}{se:selgradev} (Read)
      }{
         The graphics workstation on which to produce the plot.  If it
         is null (\texttt{{!}}), no plot will be made.  \texttt{[}Current graphics device\texttt{{]}}
      }
      \sstsubsection{
         IN = NDF (Read)
      }{
         The NDF data structure to be analysed.
      }
      \sstsubsection{
         LOGFILE = FILENAME (Write)
      }{
         A text file into which the results should be logged.  If a null
         value is supplied (the default), then no logging of results
         will take place.  \texttt{[!]}
      }
      \sstsubsection{
         MARGIN( 4 ) = \_REAL (Read)
      }{
         The widths of the margins to leave for axis annotation, given
         as fractions of the corresponding dimension of the current
         picture.  Four values may be given, in the order bottom, right,
         top, left.  If fewer than four values are given, extra values
         are used equal to the first supplied value.  If these margins
         are too narrow any axis annotation may be clipped.  If a null
         (\texttt{{!}}) value is supplied, the value used is \texttt{0.15} (for all edges)
         if either annotated axes or a key are produced, and zero
         otherwise.  \texttt{[}current value\texttt{{]}}
      }
      \sstsubsection{
         NUMBIN = \_INTEGER (Read)
      }{
         The number of histogram bins to be used.  This must lie in the
         range 2 to 10000.  The suggested default is the current value.
         It is ignored if WIDTH is not null.
      }
      \sstsubsection{
         OUT = NDF (Read)
      }{
         Name of the NDF structure to save the histogram in its data
         array.  If null (\texttt{{!}}) is entered the histogram NDF is not
         created.  \texttt{[!]}
      }
      \sstsubsection{
         RANGE = LITERAL (Read)
      }{
         RANGE specifies the range of values for which the histogram is
         to be computed.  The supplied string should consist of up to
         three sub-strings, separated by commas.  For all but the option
         where you give explicit numerical limits, the first sub-string
         must specify the method to use.  If supplied, the other two
         sub-strings should be numerical values as described below
         (default values will be used if these sub-strings are not
         provided).  The following options are available.

         \ssthitemlist{

            \sstitem
            lower,upper --- You can supply explicit lower and upper
            limiting values.  For example, \texttt{"10,200"} would set the histogram
            lower limit to 10 and its upper limit to 200.  No method name
            prefixes the two values.  If only one value is supplied,
            the \texttt{"Range"} method is adopted.  The limits must be within the
            dynamic range for the data type of the NDF array component.

            \sstitem
            \texttt{"Percentiles"} --- The default values for the histogram data
            range are set to the specified percentiles of the data.  For
            instance, if the value \texttt{"Per,10,99"} is supplied, then the lowest
            10\% and highest 1\% of the data values are excluded from the
            histogram.  If only one value, \textit{p}1, is supplied, the second
            value, $p2$, defaults to (100 - \textit{p}1).  If no values are supplied,
            the values default to \texttt{"5,95"}.  Values must be in the range 0 to
            100.

            \sstitem
            \texttt{"Range"} --- The minimum and maximum array values are used.  No
            other sub-strings are needed by this option.  Null (\texttt{{!}}) is a
            synonym for the \texttt{"Range"} method.

            \sstitem
            \texttt{"Sigmas"} --- The histogram limiting values are set to the
            specified numbers of standard deviations below and above the
            mean of the data.  For instance, if the supplied value is
            \texttt{"sig,1.5,3.0"}, then the histogram extends from the mean of the
            data minus 1.5 standard deviations to the mean plus 3 standard
            deviations.  If only one value is supplied, the second value
            defaults to the supplied value.  If no values are supplied,
            both default to \texttt{"3.0"}.

         }
         The \texttt{"Percentiles"} and \texttt{"Sigmas"} methods are useful to generate
         a first pass at the histogram.  They reduce the likelihood
         that all but a small number of values lie within a few
         histogram bins.

         The extreme values are reported unless Parameter RANGE is
         specified on the command line.  In this case extreme values
         are only calculated where necessary for the chosen method.

         The method name can be abbreviated to a single character, and is
         case insensitive.  The initial value is \texttt{"Range"}.  The suggested
         defaults are the current values, or \texttt{{!}} if these do not exist.
         \texttt{[}current value\texttt{{]}}
      }
      \sstsubsection{
         STYLE = \htmlref{GROUP}{se:groups} (Read)
      }{
         A group of attribute settings describing the plotting style to
         use when drawing the annotated axes and data values.

         A comma-separated list of strings should be given in which each
         string is either an attribute setting, or the name of a text
         file preceded by an up-arrow character \texttt{"$\wedge$"}.  Such text files
         should contain further comma-separated lists which will be
         read and interpreted in the same manner.  Attribute settings
         are applied in the order in which they occur within the list,
         with later settings overriding any earlier settings given for
         the same attribute.

         Each individual attribute setting should be of the form:

            $<$name$>$=$<$value$>$

         where $<$name$>$ is the name of a plotting attribute, and $<$value$>$
         is the value to assign to the attribute.  Default values will be
         used for any unspecified attributes.  All attributes will be
         defaulted if a null value (\texttt{{!}})---the initial default---is supplied.
         To apply changes of style to only the current invocation, begin these
         attributes with a plus sign.  A mixture of persistent and temporary
         style changes is achieved by listing all the persistent attributes
         followed by a plus sign then the list of temporary attributes.

         See \slhyperref{Plotting Attributes}{Section~}{}{ap:plotting_attr}
         for a description of the available attributes.  Any unrecognised
         attributes are ignored (no error is reported).

         The appearance of the histogram curve is controlled by the
         attributes \htmlattref{Colour(Curves)}{Colour(element)},
         \htmlattref{Width(Curves)}{Width(element)}, {\em{etc}}.  (The
         synonym \att{Line} may be used in place of \att{Curves}.)
         \texttt{[}current value\texttt{{]}}
      }
      \sstsubsection{
         TITLE = LITERAL (Read)
      }{
         Title for the histogram NDF.  \texttt{["KAPPA - Histogram"]}
      }
      \sstsubsection{
         WEIGHTS = NDF (Read)
      }{
         An optional NDF holding weights associated with each input pixel
         value (supplied via Parameter IN). Together with Parameter
         WEIGHTSTEP, these determine the count added to the corresponding
         histogram bin for each input pixel value. For instance, weights
         could be related to the variance of the data values, or to the
         position of the data values within the input NDF. If a null value
         (\texttt{{!}}) is supplied for WEIGHTS, all input values contribute a
         count of one to the corresponding histogram bin. If an NDF is
         supplied, the histogram count for a particular input pixel is
         formed by dividing its weight value (supplied in the WEIGHTS NDF)
         by the value of Parameter WEIGHTSTEP, and then taking the nearest
         integer. Input pixels with bad or zero weights are excluded from
         the histogram. \texttt{[!]}
      }
      \sstsubsection{
         WEIGHTSTEP = \_DOUBLE (Read)
      }{
         Only accessed if a value is supplied for Parameter WEIGHTS.
         WEIGHTSTEP is the increment in weight value that corresponds to
         a unit increment in histogram count.
      }
      \sstsubsection{
         WIDTH = \_DOUBLE (Read)
      }{
         The bin width.  This is the alternative to setting the number of
         bins.  The bins of the chosen width start from the minimum value
         and do not exceed the maximum value.  Values are constrained
         to give between 2 and 10000 bins. If this parameter is set to
         null (\texttt{{!}}), the data range and Parameter NUMBIN are used to
         specify the bin width.  \texttt{[!]}
      }
      \sstsubsection{
         XLEFT = \_DOUBLE (Read)
      }{
         The axis value to place at the left hand end of the horizontal
         axis of the plot.  If a null (\texttt{{!}}) value is supplied, the minimum
         data value in the histogram is used.  The value supplied may be
         greater than or less than the value supplied for XRIGHT.  \texttt{[!]}
      }
      \sstsubsection{
         XLOG = \_LOGICAL (Read)
      }{
         \texttt{TRUE} if the plot \textit{x} axis is to be logarithmic.  Any histogram
         bins which have negative or zero central data values are
         omitted from the plot.  \texttt{[FALSE]}
      }
      \sstsubsection{
         XRIGHT = \_DOUBLE (Read)
      }{
         The axis value to place at the right hand end of the horizontal
         axis of the plot.  If a null (\texttt{{!}}) value is supplied, the maximum
         data value in the histogram is used.  The value supplied may be
         greater than or less than the value supplied for XLEFT.  \texttt{[!]}
      }
      \sstsubsection{
         YBOT = \_DOUBLE (Read)
      }{
         The axis value to place at the bottom end of the vertical axis
         of the plot.  If a null (\texttt{{!}}) value is supplied, the lowest count
         the histogram is used.  The value supplied may be greater than
         or less than the value supplied for YTOP.  \texttt{[!]}
      }
      \sstsubsection{
         YLOG = \_LOGICAL (Read)
      }{
         \texttt{TRUE} if the plot \textit{y} axis is to be logarithmic.  Empty bins are
         removed from the plot if the \textit{y} axis is logarithmic.  \texttt{[FALSE]}
      }
      \sstsubsection{
         YTOP = \_DOUBLE (Read)
      }{
         The axis value to place at the top end of the vertical axis of
         the plot.  If a null (\texttt{{!}}) value is supplied, the largest count
         in the histogram is used.  The value supplied may be greater
         than or less than the value supplied for YBOT.  \texttt{[!]}
      }
   }
   \sstexamples{
      \sstexamplesubsection{
         histogram image 100 ! device=!
      }{
         Computes and reports the histogram for the data array in the
         NDF called image.  The histogram has 100 bins and spans the
         full range of data values.
      }
      \sstexamplesubsection{
         histogram ndf=spectrum comp=variance range="100,200" numbin=20
      }{
         Computes and reports the histogram for the variance array in
         the NDF called spectrum.  The histogram has 20 bins and spans
         the values between 100 and 200.  A plot is made to the current
         graphics device.
      }
      \sstexamplesubsection{
         histogram ndf=spectrum comp=variance range="100,204" width=5
      }{
         This behaves the same as the previous example, even though it
         specifies a larger maximum, as the same number of width=5 bins
         are used.
      }
      \sstexamplesubsection{
         histogram cube(3,4,) 10 si out=c3\_4\_hist device=!
      }{
         Computes and reports the histogram for the \textit{z}-vector at (\textit{x},\textit{y})
         element (3,4) of the data array in the three-dimensional NDF called
         cube.  The histogram has 10 bins and spans a range three
         standard deviations either side of the mean of the data values.
         The histogram is written to a one-dimensional NDF called
         c3\_4\_hist.
      }
      \sstexamplesubsection{
         histogram cube numbin=32 ! device=xwindows style="title=cube"
      }{
         Computes and reports the histogram for the data array in
         the NDF called cube.  The histogram has 32 bins and spans the
         full range of data values.  A plot of the histogram is made to
         the XWINDOWS device, and is titled \texttt{"cube"}.
      }
      \sstexamplesubsection{
         histogram cube numbin=32 ! device=xwindows ylog style=$\wedge$style.dat
      }{
         As in the previous example except the logarithm of the number
         in each histogram bin is plotted, and the contents of the text
         file \texttt{style.dat} control the style of the resulting graph.
         The plotting style specified in file \texttt{style.dat} becomes
         the default plotting style for future invocations of HISTOGRAM.
      }
      \sstexamplesubsection{
         histogram cube numbin=32 ! device=xwindows ylog tempstyle=$\wedge$style.dat
      }{
         This is the same as the previous example, except that the style
         specified in file \texttt{style.dat} does not become the default style
         for future invocations of HISTOGRAM.
      }
      \sstexamplesubsection{
         histogram halley($\sim$200,$\sim$300) "pe,10,90" logfile=hist.dat $\backslash$
      }{
         Computes the histogram for the central 200 by 300 elements of
         the data array in the NDF called halley, and writes the
         results to a logfile called \texttt{hist.dat}.  The histogram uses the
         current number of bins, and includes data values between the 10
         and 90 percentiles.  A plot appears on the current graphics
         device.
      }
   }
   \sstdiytopic{
      Related Applications
   }{
KAPPA: \htmlref{HISTAT}{HISTAT},
\htmlref{MSTATS}{MSTATS},
\htmlref{NUMB}{NUMB},
\htmlref{STATS}{STATS};
\xref{FIGARO}{sun86}{}: \xref{HIST}{sun86}{HIST},
\xref{ISTAT}{sun86}{ISTAT}.
   }
   \sstimplementationstatus{
      \sstitemlist{

         \sstitem
         This routine correctly processes the \htmlref{AXIS}{apndf:axis}, DATA, \htmlref{VARIANCE}{apndf:variance},
         \htmlref{QUALITY}{apndf:quality}, \htmlref{LABEL}{apndf:label}, \htmlref{TITLE}{apndf:title}, \htmlref{UNITS}{apndf:units}, and \htmlref{HISTORY}{apndf:history}~ components of the input
         NDF.

         \sstitem
         Processing of \htmlref{bad pixels}{se:masking} and automatic \htmlref{quality masking}{se:qualitymask} are
         supported.

         \sstitem
         All \htmlref{non-complex numeric data types}{ap:HDStypes} can be handled.

         \sstitem
         Any number of NDF dimensions is supported.
      }
   }
}




\sstroutine{
   INTERLEAVE
}{
   Forms a higher-resolution NDF by interleaving a set of NDFs
}{
   \sstdescription{
      This routine performs interleaving, also known as interlacing,
      in order to restore resolution where the pixel dimension
      undersamples data.  Resolution may be improved by integer
      factors along one or more dimensions.  For an $N$-fold increase in
      resolution along a dimension, INTERLEAVE demands N NDF
      structures that are displaced from each other by $i/N$ pixels,
      where $i$ is an integer from 1 to $N-1$.  It creates an NDF whose
      dimensions are enlarged by $N$ along that dimension.

      The supplied NDFs should have the same dimensionality.
   }
   \sstusage{
      interleave in out expand
   }
   \sstparameters{
      \sstsubsection{
         EXPAND() = \_INTEGER (Read)
      }{
         Linear expansion factors to be used to create the new data
         array.  The number of factors should equal the number of
         dimensions in the input NDF.  If fewer are supplied the last
         value in the list of expansion factors is given to the
         remaining dimensions.  Thus if a uniform expansion is required
         in all dimensions, just one value need be entered.  If the net
         expansion is one, an error results.  The suggested default is
         the current value.
      }
      \sstsubsection{
         FILL = \htmlref{LITERAL}{se:parmenu} (Read)
      }{
         Specifies the value to use where the interleaving does not
         fill the array, say because the shapes of the input NDFs are
         not the same, or have additional shifts of origin.  Allowed
         values are \texttt{"Bad"} or \texttt{"Zero"}.   \texttt{["Bad"]}
      }
      \sstsubsection{
         IN = NDF (Read)
      }{
         A \htmlref{group}{se:groups} of input NDFs to be interweaved.  They
         may have different shapes, but must all have the same number of
         dimensions.  This should be given as a comma-separated list,
         in which each list element can be:

         \sstitemlist{

            \sstitem
            an NDF name, optionally containing wild-cards and/or regular
            expressions (\texttt{"$*$"}, \texttt{"?"}, \texttt{"[a-z]"} \emph{etc.}).

            \sstitem
            the name of a text file, preceded by an up-arrow character
            \texttt{"$\wedge$"}.  Each line in the text file should contain a
            comma-separated list of elements, each of which can in turn
            be an NDF name (with optional wild-cards, \emph{etc.}), or another
            file specification (preceded by a caret).  Comments can be
            included in the file by commencing lines with a hash
            character \texttt{"\#"}.

         }
         If the value supplied for this parameter ends with a hyphen
         \texttt{"-"}, then you are re-prompted for further input until
         a value is given which does not end with a hyphen.  All
         the datasets given in this way are concatenated into a single
         group.
      }
      \sstsubsection{
         OUT = NDF (Write)
      }{
         Output NDF structure.
      }
      \sstsubsection{
         TITLE = LITERAL (Read)
      }{
         Title for the output NDF structure.  A null value (\texttt{{!}})
         propagates the title from the input NDF to the output NDF.
         \texttt{[!]}
      }
      \sstsubsection{
         TRIM = \_LOGICAL (Read)
      }{
         This parameter controls the shape of the output NDF before the
         application of the expansion.  If TRIM=\texttt{{TRUE}}, then the output
         NDF reflects the shape of the intersection of all the input
         NDFs, \emph{i.e.} only pixels which appear in all the input arrays
         will be represented in the output.  If TRIM=\texttt{{FALSE}}, the output
         reflects shape of the union of the inputs, \emph{i.e.} every pixel
         which appears in the input arrays will be represented in the
         output.  \texttt{[TRUE]}
      }
   }
   \sstexamples{
      \sstexamplesubsection{
         interleave "vector1,vector2" weave 2
      }{
         This interleaves the one-dimensional NDFs called vector1 and
         vector2 and stores the result in NDF weave.  Only the
         intersection of the two input NDFs is used.
      }
      \sstexamplesubsection{
         interleave 'image$*$' weave [3,2] title="Interlaced image"
      }{
         This interleaves the two-dimensional NDFs with names beginning
         with \texttt{"image"} into an NDF called weave.  The interleaving has
         three datasets along the first dimension and two along the
         second.  Therefore there should be six input NDFs.  The output
         NDF has title \texttt{"Interlaced image"}.
      }
      \sstexamplesubsection{
         interleave in='image$*$' out=weave expand=[3,2] notrim
      }{
         As above except the title is not set and the union of the
         bounds of the input NDFs is expanded to form the shape of the
         weave NDF.
      }
      \sstexamplesubsection{
         interleave $\wedge$frames.lis finer 2
      }{
         This interleaves the NDFs listed in the text file \texttt{frames.lis}
         to form an enlarged NDF called finer.  The interleaving is
         twofold along each axis of those NDFs.
      }
   }
   \sstdiytopic{
      Related Applications
   }{
      KAPPA: \htmlref{PIXDUPE}{PIXDUPE};
      \xref{CCDPACK}{sun139}{}: \xref{DRIZZLE}{sun139}{DRIZZLE}.
   }
   \sstimplementationstatus{
      \sstitemlist{

         \sstitem
         This routine processes the \htmlref{AXIS}{apndf:axis},
         DATA, \htmlref{QUALITY}{apndf:quality}, and
         \htmlref{VARIANCE}{apndf:variance} from the all input NDF data
         structures.  It also processes the \htmlref{WCS}{apndf:wcs},
         \htmlref{LABEL}{apndf:label}, \htmlref{TITLE}{apndf:title},
         \htmlref{UNITS}{apndf:units}, and \htmlref{HISTORY}{apndf:history}~
         components of the primary NDF data structure, and propagates
         all of its extensions.

         \sstitem
         The AXIS centre values along each axis are formed by
         interleaving the corresponding centres from the first NDF, and
         linearly interpolating between those to complete the array.

         \sstitem
         The AXIS width and variance values in the output are formed by
         interleaving the corresponding input AXIS values.  Each array
         element is assigned from the first applicable NDF.  For example,
         for a two-dimensional array with expansion factors of 2 and 3
         respectively, the first two NDFs would be used to define the
         array elements for the first axis.  The second axis's elements
         come from the first, third, and fifth NDFs.

         \sstitem
         All \htmlref{non-complex numeric data types}{ap:HDStypes} can be handled.

         \sstitem
         Any number of NDF dimensions is supported.
      }
   }
}

\sstroutine{
   KAPHELP
}{
   Gives help about \KAPPA
}{
   \sstdescription{
      Displays help about \KAPPA.  The help information
      has classified and alphabetical lists of commands, general
      information about \KAPPA\ and related material;
      it describes individual commands in detail.

      Here are some of the main options.
      \begin{description}
      \item \texttt{kaphelp} \\
         No parameter is given so the introduction and the top-level
         help index is displayed.

      \item \texttt{kaphelp application/topic} \\
         This gives help about the specified application or topic.

      \item \texttt{kaphelp application/topic subtopic} \\
         This lists help about a subtopic of the specified application
         or topic.  The hierarchy of topics has a maximum of four levels.

      \item \texttt{kaphelp Hints} \\
         This gives hints for new and intermediate users.

      \item \texttt{kaphelp summary} \\
         This shows a one-line summary of each application.

      \item \texttt{kaphelp classified classification} \\
         This lists a one-line summary of each application in the
         given functionality classification.
      \end{description}

      See the Section \texttt{"Navigating the Help Library"} for details
      how to move around the help information, and to select the topics
      you want to view.
   }
   \sstusage{
      kaphelp [topic] [subtopic] [subsubtopic] [subsubsubtopic]
   }
   \sstparameters{
      \sstsubsection{
         TOPIC = LITERAL (Read)
      }{
         Topic for which help is to be given.  \texttt{[" "]}
      }
      \sstsubsection{
         SUBTOPIC = LITERAL (Read)
      }{
         Subtopic for which help is to be given.  \texttt{[" "]}
      }
      \sstsubsection{
         SUBSUBTOPIC = LITERAL (Read)
      }{
         Subsubtopic for which help is to be given.  \texttt{[" "]}
      }
      \sstsubsection{
         SUBSUBSUBTOPIC = LITERAL (Read)
      }{
         Subsubsubtopic for which help is to be given.  \texttt{[" "]}
      }
   }
   \sstdiytopic{
      Navigating the Help Library
   }{
      The help information is arranged hierarchically.  You can
      move around the help information whenever KAPHELP prompts.  This
      occurs when it has either presented a screen's worth of text or
      has completed displaying the previously requested help.  The
      information displayed by KAPHELP on a particular topic includes
      a description of the topic and a list of subtopics that further
      describe the topic.

      At a prompt you may enter:

      \sstitemlist{

         \sstitem
         a topic and/or subtopic name(s) to display the help for that
         topic or subtopic, so for example, \texttt{block parameters box}
         gives help on \texttt{BOX}, which is a subtopic of \texttt{Parameters},
         which in turn is a subtopic of \texttt{BLOCK};

         \sstitem
         a \texttt{$<$CR$>$} to see more text at a \texttt{Press RETURN to
         continue ...} request;

         \sstitem
         a \texttt{$<$CR$>$} at topic and subtopic prompts to move up one
         level in the hierarchy, and if you are at the top level it will
         terminate the help session;

         \sstitem
         a \texttt{CTRL/D }(pressing the CTRL and D keys simultaneously) in
         response to any prompt will terminate the help session;

         \sstitem
         a question mark \texttt{?} to redisplay the text for the current topic,
         including the list of topic or subtopic names; or

         \sstitem
         an ellipsis \texttt{...} to display all the text below the current
         point in the hierarchy.  For example, \texttt{BLOCK...} displays
         information on the BLOCK topic as well as information on all the
         subtopics under BLOCK.
      }

      You can abbreviate any topic or subtopic using the following rules.

      \sstitemlist{

         \sstitem
         Just give the first few characters, \emph{e.g.} \texttt{PARA} for
         \texttt{Parameters}.

         \sstitem
         Some topics are composed of several words separated by
         underscores.  Each word of the keyword may be abbreviated,
         \emph{e.g.} \texttt{Colour\_Set} can be shortened to \texttt{C\_S}.

         \sstitem
         The characters \texttt{\%} and \texttt{$\lsk$} act as wildcards, where
         the percent sign matches any single character, and asterisk
         matches any sequence of characters.  Thus to display information
         on all available topics, type an asterisk in reply to a prompt.

         \sstitem
         If a word contains, but does end with an asterisk wildcard, it
         must not be truncated.

         \sstitem
         The entered string must not contain leading or embedded spaces.
      }

      Ambiguous abbreviations result in all matches being displayed.
   }
   \sstimplementationstatus{
      \sstitemlist{

         \sstitem
         Uses the \xref{portable help system}{sun124}{}.

      }
   }
}
\sstroutine{
   KAPVERSION
}{
   Checks the package version number
}{
   \sstdescription{
      This application will display the installed package version number,
      or compare the version number of the installed package against a
      specified version number, reporting whether the installed package
      is older, or younger, or equal to the specified version.
   }
   \sstusage{
      kapversion [compare]
   }
   \sstparameters{
      \sstsubsection{
         COMPARE = LITERAL (Read)
      }{
         A string specifying the version number to be compared to the
         version of the installed package.  If a null (\texttt{{!}}) value is supplied,
         the version string of the installed package is displayed, but no
         comparison takes place.  If a non-null value is supplied, the
         version of the installed package is not displayed.

         The supplied string should be in the \texttt{"V$<$ddd$>$.$<$ddd$>$-$<$ddd$>$"},
         where \texttt{"$<$ddd$>$"} represents a set of digits.  The leading \texttt{"V"} can be
         omitted, as can any number of trailing fields (missing trailing
         fields default to zero).  \texttt{[!]}
      }
   }
   \sstresparameters{
      \sstsubsection{
         RESULT = \_INTEGER (Write)
      }{
         If a value is given for the COMPARE parameter, then RESULT is
         set to one of the following values:

         \ssthitemlist{

            \sstitem
            \texttt{1} --- The installed package is older than the version number
            specified by the COMPARE parameter.

            \sstitem
            \texttt{0} --- The version of the installed package is equal to the
            version specified by the COMPARE parameter.

            \sstitem
            \texttt{-1} --- The installed package is younger than the version number
            specified by the COMPARE parameter.

         }
         The same value is also written to standard output.
      }
   }
   \sstexamples{
      \sstexamplesubsection{
         kapversion
      }{
         Displays the version number of the installed package.
      }
      \sstexamplesubsection{
         kapversion compare="V0.14-1"
      }{
         Compares the version of the installed package with the version
         \texttt{"V0.14-1"}, and sets the RESULT parameter appropriately.  For
         instance, if the installed package was \texttt{"V0.13-6"} then RESULT
         would be set to $-$1.  If the installed package was \texttt{"V0.14-1"},
         RESULT would be set to 0.  If the installed package was \texttt{"V0.14-5"}
         RESULT would be set to $+$1.
      }
   }
   \sstnotes{
      \sstitemlist{

         \sstitem
         The package version number is obtained from the \texttt{version} file
         in the directory containing the package's installed executable files.
         This file is created when the package is installed using the \texttt{"mk
         install"} command.  An error will be reported if this file cannot be
         found.
      }
   }
}
\sstroutine{
   KSTEST
}{
   Compares data sets using the Kolmogorov-Smirnov test
}{
   \sstdescription{
      This routine reads in a data array and performs a two sided
      Kolmogorov-Smirnov test on the vectorised data.  It does this in
      two ways:

      \begin{enumerate}
      \item If only one dataset is to be tested the data array is
           divided into subsamples.  First it compares subsample 1 with
           subsample 2, if they are thought to be from the same sample
           they are concatenated.  This enlarged sample is then
           compared with subsample 3 {\em{etc.}}, concatenating if consistent,
           until no more subsamples remain.

       \item If more than one dataset is specified, the datasets are
           compared to the reference dataset in turn.  If the
           probability the two are from the same sample is greater than
           the specified confidence level, the datasets are
           concatenated, and the next sample is tested against this
           enlarged reference dataset.
       \end{enumerate}

      The probability and maximum separation of the cumulative
      distribution function is written for each comparison (at the
      normal reporting level).  The mean value of the consistent data
      and its error are also reported.  In all cases the consistent
      data can be output to a new dataset.  The statistics and
      probabilities are written to results parameters.

   }
   \sstusage{
      kstest in out [limit]
   }
   \sstparameters{
      \sstsubsection{
         COMP = \htmlref{LITERAL}{se:parmenu} (Read)
      }{
         The name of the NDF array component to be tested for
         consistency: \texttt{"Data"}, \texttt{"Error"}, \texttt{"Quality"} or
         \texttt{"Variance"} (where \texttt{"Error"} is the alternative to
         \texttt{"Variance"} and causes the square
         root of the variance values to be taken before performing the
         comparisons).  If \texttt{"Quality"} is specified, then the quality
         values are treated as numerical values (in the range 0 to
         255).  \texttt{["Data"]}
      }
      \sstsubsection{
         LIMIT = \_REAL (Read)
      }{
         Confidence level at which samples are thought to be
         consistent.  This must lie in the range 0 to 1.  \texttt{[0.05]}
      }
      \sstsubsection{
         IN = LITERAL (Read)
      }{
         The names of the \NDFref{NDFs} to be tested.  If just one dataset is
         supplied, it is divided into subsamples, which are compared
         (see Parameter NSAMPLE).  When more than one dataset is
         provided, the first becomes the reference dataset to which all
         the remainder are compared.

         It may be a list of NDF names or direction specifications
         separated by commas.  If a list is supplied on the command
         line, the list must be enclosed in double quotes.  NDF names
         may include the regular expressions (\texttt{"$*$"}, \texttt{"?"},
         \texttt{"[a-z]"} {\em{etc.}}).
         Indirection may occur through text files (nested up to seven
         deep).  The indirection character is \texttt{"$\wedge$"}.  If extra prompt
         lines are required, append the continuation character \texttt{"-"} to
         the end of the line.  Comments in the indirection file begin
         with the character \texttt{"\#"}.
      }
      \sstsubsection{
         NSAMPLE = \_INTEGER (Read)
      }{
         The number of the subsamples into which to divide the reference
         dataset.  This parameter is only requested when a single NDF
         is to be analysed, {\emph{i.e.}} when only one dataset name is supplied
         via Parameter IN.  The allowed range is 2 to 20.  \texttt{[3]}
      }
      \sstsubsection{
         OUT = NDF (Write)
      }{
         Output one-dimensional NDF to which the consistent data are
         written.  A null value (\texttt{{!}})---the suggested default---prevents
         creation of this output dataset.
      }
   }
   \sstresparameters{
      \sstsubsection{
         DIST() = \_REAL (Write)
      }{
         Maximum separation found in the cumulative distributions for
         each comparison subsample.  Note that it excludes the
         reference dataset.
      }
      \sstsubsection{
         ERRMEAN = \_DOUBLE (Write)
      }{
         Error in the mean value of the consistent data.
      }
      \sstsubsection{
         FILES() = LITERAL (Write)
      }{
         The names of the datasets intercompared.  The first is the
         reference dataset.
      }
      \sstsubsection{
         MEAN = \_DOUBLE (Write)
      }{
         Mean value of the consistent data.
      }
      \sstsubsection{
         NKEPT = \_INTEGER (Write)
      }{
         Number of consistent data.
      }
      \sstsubsection{
         PROB() = \_REAL (Write)
      }{
         Probability that each comparison subsample is drawn from the
         same sample.  Note that this excludes the reference sample.
      }
      \sstsubsection{
         SIGMA = \_DOUBLE (Write)
      }{
         Standard deviation of the consistent data.
      }
   }
   \sstexamples{
      \sstexamplesubsection{
         kstest arlac accept
      }{
         This tests the NDF called arlac for self-consistency at the 95\%
         confidence level using three subsamples.  No output dataset is
         created.

         The following applies to all the examples.  If the reference
         dataset and a comparison subsample are consistent, the two
         merge to form an expanded reference dataset, which is then
         used for the next comparison.  Details of the comparisons are
         presented.
      }
      \sstexamplesubsection{
         kstest arlac arlac\_filt 0.10 nsample=10
      }{
         As above except data are retained if they exceed the 90\%
         probability level, the comparisons are made with ten
         subsamples, and the consistent data are written to the
         one-dimensional NDF called arlac\_filt.
      }
      \sstexamplesubsection{
         kstest in="ref,obs$*$" comp=v out=master
      }{
         This compares the variance in the NDF called ref with that in
         a series of other NDFs whose names begin \texttt{"obs"}.  The variance
         consistent with the reference dataset are written to the data
         array in the NDF called master.  To be consistent, they must be
         the same at 95\% probability.
      }
      \sstexamplesubsection{
         kstest "ref,$\wedge$96lc.lis,obs$*$" master comp=v
      }{
         As the previous example, except the comparison files include
         those listed in the text file \texttt{96lc.lis}.
      }
   }
   \sstnotes{
      \sstitemlist{

         \sstitem
         The COMP array MUST exist in each NDF to be compared.  The
         COMP array becomes the data array in the output dataset.  When
         COMP=\texttt{"Data"}, the variance values corresponding to
         consistent data are propagated to the output dataset.

         \sstitem
         Pixel bounds are ignored for the comparisons.

         \sstitem
         The internal comparison of a single dataset follows the method
         outlined in Hughes D., 1993, {\em JCMT-UKIRT Newsletter}, \#4, p32.

         \sstitem
         The maximum number of files is 20.
      }
   }
   \sstimplementationstatus{
      \sstitemlist{

         \sstitem
         This routine correctly processes DATA, \htmlref{VARIANCE}{apndf:variance}, \htmlref{HISTORY}{apndf:history}, \htmlref{LABEL}{apndf:label},
         \htmlref{TITLE}{apndf:title}, and \htmlref{UNITS}{apndf:units}~ components, and propagates
         all \htmlref{extensions}{apndf:extensions}.  \htmlref{AXIS}{apndf:axis}
         information is lost.  Propagation is from the reference dataset.

         \sstitem
         Processing of \htmlref{bad pixels}{se:masking} and automatic \htmlref{quality masking}{se:qualitymask} are
         supported.

         \sstitem
         All \htmlref{numeric data types}{ap:HDStypes} are supported, however, processing uses
         the \_REAL data type, and the output dataset has this type.
      }
   }
}
\sstroutine{
   LAPLACE
}{
   Performs a Laplacian convolution as an edge detector in a two-dimensional NDF
}{
   \sstdescription{
      This routine calculates the Laplacian of the supplied two-dimensional \NDFref{NDF}, and
      subtracts it from the original array to create the output NDF.  The
      subtractions can be done a specified integer number of times.
      This operation can be approximated by a convolution with the kernel:

      $
      \begin{array}{ccc}
                     N & -N  &  -N \\
                     N & +8N &  -N \\
                     N & -N  &  -N
      \end{array}
      $

      where \textit{N}\ is the integer number of times the Laplacian is
      subtracted.  This convolution is used as a uni-directional edge
      detector.  Areas where the input data array is flat become zero
      in the output data array.
   }
   \sstusage{
      laplace in [number] out [title]
   }
   \sstparameters{
      \sstsubsection{
         IN = NDF (Read)
      }{
         Input NDF.
      }
      \sstsubsection{
         NUMBER = \_INTEGER (Read)
      }{
         Number of Laplacians to remove.  \texttt{[1]}
      }
      \sstsubsection{
         OUT = NDF (Write)
      }{
         Output NDF.
      }
      \sstsubsection{
         TITLE = LITERAL (Read)
      }{
         The title for the output NDF.  A null value will cause
         the title of the NDF supplied for Parameter IN to be used
         instead.  \texttt{[!]}
      }
   }
   \sstexamples{
      \sstexamplesubsection{
         laplace a 10 b
      }{
         This subtracts ten Laplacians from the NDF called a, to make the
         NDF called b.  NDF b inherits its title from a.
      }
   }
   \sstdiytopic{
      Related Applications
   }{
KAPPA: \htmlref{SHADOW}{SHADOW},
\htmlref{MEDIAN}{MEDIAN};
\xref{FIGARO}{sun86}{}: \xref{ICONV3}{sun86}{ICONV3}.
   }
   \sstimplementationstatus{
      \sstitemlist{

         \sstitem
         This routine correctly processes the \htmlref{WCS}{apndf:wcs}, \htmlref{AXIS}{apndf:axis}, DATA, and \htmlref{VARIANCE}{apndf:variance}
         components of an NDF data structure. QUALITY is propagated.

         \sstitem
         Processing of \htmlref{bad pixels}{se:masking} and automatic \htmlref{quality masking}{se:qualitymask} are
         supported.

         \sstitem
         All \htmlref{non-complex numeric data types}{ap:HDStypes} can be handled.
      }
   }
}
\sstroutine{
   LINPLOT
}{
   Draws a line plot of the data values in a one-dimensional NDF
}{
   \sstdescription{
      This application creates a plot of array value against position for
      a one-dimensional \NDFref{NDF}.  The vertical axis of the plot represents array
      value, and the horizontal axis represents position.  These can be
      mapped in various ways on to the graphics surface (\emph{e.g.} linearly,
      logarithmically); see Parameters XMAP and YMAP.

      The plot may take several different forms such as a \texttt{"join-the-dots"}
      plot, a \texttt{"staircase"} plot, a \texttt{"chain"} plot (see Parameter MODE).
      Errors on both the data values and the data positions may be represented
      in several different ways (see Parameters ERRBAR and SHAPE).  The
      plotting style (colour, founts, text size, \emph{etc.}) may be specified in
      detail using Parameter STYLE.

      The bounds of the plot on both axes can be specified using
      Parameters XLEFT, XRIGHT, YBOT, and YTOP.  If not specified they take
      default values which encompass the entire supplied data set.  The
      current picture is usually cleared before plotting the new picture,
      but Parameter CLEAR can be used to prevent this, allowing several
      plots to be `stacked' together.  If a new plot is drawn over an
      existing plot, then there is an option to allow the new plot to be
      aligned with the existing plot (see Parameter ALIGN).

      The input NDF may, for instance, contain a spectrum of data values
      against wavelength, or it may contain data values along a
      one-dimensional profile through an NDF of higher dimensionality.  In
      the latter case, the current \htmlref{co-ordinate Frame}{se:domains}~  of the NDF may have
      more than one axis.  Any of the axes in the current co-ordinate Frame
      of the input NDF may be used to annotate the horizontal axis of the
      plot produced by this application.  Alternatively, the horizontal
      axis may be annotated with offset from the first array element
      measured within the current co-ordinate Frame of the NDF.  For
      instance, a one-dimensional slice through a two-dimensional image calibrated in RA/DEC could
      be annotated with RA, or DEC, or offset from the first element (in
      arcminutes, degrees, \emph{etc}).  This offset is measured along the path
      of the profile.  The choice of annotation for the horizontal axis is
      controlled by Parameter USEAXIS.
   }
   \sstusage{
      linplot ndf [comp] [mode] [xleft] [xright] [ybot] [ytop] [device]
   }
   \sstparameters{
      \sstsubsection{
         ALIGN = \_LOGICAL (Read)
      }{
         Controls whether or not the new data plot should be aligned with an
         existing data plot.  If ALIGN is \texttt{TRUE}, the \textit{x} axis values of the
         new plot will be mapped into the co-ordinate system of the \textit{x} axis in
         the existing plot before being used (if this is not possible an
         error is reported).  In this case, the XLEFT, XRIGHT, YBOT and YTOP
         parameters are ignored and the bounds of the existing plot are used
         instead.  If ALIGN is \texttt{FALSE}, the new \textit{x} axis values are used without
         change.  The bounds of the new plot are specified using Parameters
         XLEFT, XRIGHT, YBOT, and YTOP as usual, and these bounds are mapped
         to the edges of the existing picture.  The ALIGN parameter is only
         accessed if Parameter CLEAR is \texttt{FALSE}, and if there is another line
         plot within the current picture.  If a null (\texttt{{!}}) value is supplied,
         a value of \texttt{TRUE} will be used if and only if a mapping can be found
         between the existing and the new plots.  A value of \texttt{FALSE} will
         be used otherwise.  \texttt{[!]}
      }
      \sstsubsection{
         ALIGNSYS = LITERAL (Read)
      }{
         This is only used if a \texttt{TRUE} value is supplied for Parameter
         ALIGN.  It specifies the co-ordinate system in which the new plot and
         the existing plot are aligned (for further details see the
         description of the {\xref{\att{AlignSystem}}{sun210}{AlignSystem}}
         attribute in \xref{SUN/210}{sun210}{}.).  The supplied name should be
         a valid co-ordinate system name for the horizontal axis (see the
         description of the {\xref{\att{System}}{sun210}{System}} attribute in
         SUN/210 for a list of these names).  It may also take the special
         value \texttt{"Data"}, in which case alignment occurs in the co-ordinate
         system represented by the \htmlref{current WCS Frame}{se:curframe}~
         in the supplied NDF.  If a null (\texttt{{!}}) value is supplied.  The
         alignment system is determined by the current value of {\att
         AlignSystem} attribute in the supplied NDF.  \texttt{["Data"]}
      }
      \sstsubsection{
         AXES = \_LOGICAL (Read)
      }{
         \texttt{TRUE} if labelled and annotated axes are to be drawn around the
         plot.  If a null (\texttt{{!}}) value is supplied, the value used is \texttt{FALSE} if
         the plot is being aligned with an existing plot (see Parameter
         ALIGN), and \texttt{TRUE} otherwise.  Parameter USEAXIS determines the
         quantity used to annotated the horizontal axis.  The width of the
         margins left for the annotation may be controlled using Parameter
         MARGIN.  The appearance of the axes (colours, founts, \emph{etc.}) can be
         controlled using the Parameter STYLE.  \texttt{[!]}
      }
      \sstsubsection{
         CLEAR = \_LOGICAL (Read)
      }{
         If \texttt{TRUE} the current picture is cleared before the plot is
         drawn.  If CLEAR is \texttt{FALSE} not only is the existing plot retained,
         but also the previous plot can be used to specify the axis
         limits (see Parameter ALIGN).  Thus you can generate a composite
         plot within a single set of axes, say using different colours or
         modes to distinguish data from different datasets.  Note,
         alignment between the two plots is controlled by the \att{AlignSystem}
         attribute of the data being displayed.  For instance, if you have
         an existing plot showing a spectrum plotted against radio
         velocity and you overlay another spectrum, also in radio velocity
         but with a different rest frequency, the appearance of the final
         plot will depend on the value of the \att{AlignSystem} attribute of the
         second spectrum.  If \att{AlignSystem} is \texttt{"Wavelen"} (this is the default)
         then the two spectra will be aligned in wavelength, but if
         \att{AlignSystem} is \texttt{"vrad"} they will be aligned in radio velocity.
         There will be no difference in effect between these two forms of
         alignment unless the rest frequency is different in the two
         spectra.  Likewise, the \att{AlignStdOfRest} attribute of the second
         spectrum controls the standard of rest in which alignment occurs.
         These attributes (like all other attributes) may be examined and
         modified using \htmlref{WCSATTRIB}{WCSATTRIB}.
      }
      \sstsubsection{
         COMP = \htmlref{LITERAL}{se:parmenu} (Read)
      }{
         The NDF component to be plotted.  It may be \texttt{"Data"}, \texttt{"Quality"},
         \texttt{"Variance"}, or \texttt{"Error"} (where \texttt{"Error"} is an alternative to
         \texttt{"Variance"} and causes the square root of the variance values
         to be displayed).  If \texttt{"Quality"} is specified, then the quality
         values are treated as numerical values (in the range 0 to
         255).  \texttt{["Data"]}
      }
      \sstsubsection{
         DEVICE = \htmlref{DEVICE}{se:selgradev} (Read)
      }{
         The plotting device.  \texttt{[}Current graphics device\texttt{{]}}
      }
      \sstsubsection{
         ERRBAR = \_LOGICAL (Read)
      }{
         \texttt{TRUE} if error bars are to be drawn.  The error bars can
         comprise either or both of the data and axis-centre errors,
         depending on what is available in the supplied dataset.  The
         Parameter SHAPE controls the appearance of the error bars, and
         XSIGMA and YSIGMA control their lengths.  The ERRBAR parameter is
         ignored unless the Parameter COMP is set to \texttt{"Data"}.  \texttt{[FALSE]}
      }
      \sstsubsection{
         FREQ = \_INTEGER (Read)
      }{
         The frequency at which error bars are to be plotted.  For
         instance, a value of \texttt{2} would mean that alternate points have
         error bars plotted.  This lets some plots be less cluttered.
         FREQ must lie in the range 1 to half of the number of points
         to be plotted.  FREQ is only accessed when Parameter ERRBAR is
         \texttt{TRUE}.  \texttt{[1]}
      }
      \sstsubsection{
         KEY = \_LOGICAL (Read)
      }{
         \texttt{TRUE} if a key is to be plotted below the horizontal axis giving
         the positions at the start and end of the plot, within the
         current co-ordinate Frame of the NDF.  If Parameter USEAXIS is
         zero (\emph{i.e.} if the horizontal axis is annotated with offset from
         the first array element), then the positions refer to the centres
         of the first and last elements in the supplied NDF, whether or not
         these elements are actually visible in the plot.  If USEAXIS is not
         zero (\emph{i.e.} if the horizontal axis is annotated with the value on
         one of the axes of the NDF's current co-ordinate Frame), then the
         displayed positions correspond to the two ends of the visible
         section of the horizontal axis.  The appearance of the key can be
         controlled using Parameter KEYSTYLE.  If a null (\texttt{{!}}) value is
         supplied, a key is produced if the current co-ordinate Frame of
         the supplied NDF has two or more axes, but no key is produced if it
         only has one axis.  \texttt{[!]}
      }
      \sstsubsection{
         KEYSTYLE = GROUP (Read)
      }{
         A group of attribute settings describing the plotting style to use
         for the key (see Parameter KEY).

         A comma-separated list of strings should be given in which each
         string is either an attribute setting, or the name of a text
         file preceded by an up-arrow character \texttt{"$\wedge$"}.  Such text files
         should contain further comma-separated lists which will be
         read and interpreted in the same manner.  Attribute settings
         are applied in the order in which they occur within the list,
         with later settings overriding any earlier settings given for
         the same attribute.

         Each individual attribute setting should be of the form:

            $<$name$>$=$<$value$>$


         where $<$name$>$ is the name of a plotting attribute, and $<$value$>$
         is the value to assign to the attribute.  Default values will be
         used for any unspecified attributes.  All attributes will be
         defaulted if a null value (\texttt{{!}})---the initial default---is supplied.
         To apply changes of style to only the current invocation, begin these
         attributes with a plus sign.  A mixture of persistent and temporary
         style changes is achieved by listing all the persistent attributes
         followed by a plus sign then the list of temporary attributes.

         See \slhyperref{Plotting Attributes}{Section~}{}{ap:plotting_attr}
         for a description of the available attributes.  Any unrecognised
         attributes are ignored (no error is reported).
         \texttt{[}current value\texttt{{]}}
      }
      \sstsubsection{
         LMODE = \htmlref{LITERAL}{se:parmenu} (Read)
      }{
         LMODE specifies how the defaults for Parameters YBOT and YTOP (the
         lower and upper limit of the vertical axis of the plot) should be
         found.  The supplied string should consist of up to three sub-strings,
         separated by commas.  The first sub-string must specify the method
         to use.  If supplied, the other two sub-strings should be numerical
         values as described below (default values will be used if these
         sub-strings are not provided).  The following methods are available.

         \ssthitemlist{

            \sstitem
            \texttt{"Range"} --- The minimum and maximum data values (including any
            error bars) are used as the defaults for YBOT and YTOP.  No other
            sub-strings are needed by this option.

            \sstitem
            \texttt{"Extended"} --- The minimum and maximum data values (including error
            bars) are extended by percentages of the data range, specified by
            the second and third sub-strings.  For instance, if the value
            \texttt{"Ex,10,5"} is supplied, then the default for YBOT is set to the
            minimum data value minus 10\% of the data range, and the default
            for YTOP is set to the maximum data value plus 5\% of the data range.
            If only one value is supplied, the second value defaults to the
            supplied value.  If no values are supplied, both values default to
            \texttt{"2.5"}.  Care should be taken with this mode if YMAP is set to \texttt{"Log"}
            since the extension to the data range caused by this mode may result
            in the axis encompassing the value zero.

            \sstitem
            \texttt{"Percentile"} --- The default values for YBOT and YTOP are set to
            the specified percentiles of the data (excluding error bars).  For
            instance, if the value \texttt{"Per,10,99"} is supplied, then the default
            for YBOT is set so that the lowest 10\% of the plotted points are
            off the bottom of the plot, and the default for YTOP is set so
            that the highest 1\% of the points are off the top of the plot.
            If only one value, $p1$, is supplied, the second value, $p2$, defaults
            to $(100 - p1)$.  If no values are supplied, the values default to
            \texttt{"5,95"}.

            \sstitem
            \texttt{"Sigma"} --- The default values for YBOT and YTOP are set to the
            specified numbers of standard deviations below and above the mean
            of the data.  For instance, if the value \texttt{"sig,1.5,3.0"} is supplied,
            then the default for YBOT is set to the mean of the data minus 1.5
            standard deviations, and the default for YTOP is set to the mean
            plus 3 standard deviations.  If only one value is supplied, the
            second value defaults to the supplied value.  If no values are
            provided, both default to \texttt{"3.0"}.

         }
         The method name can be abbreviated to a single character, and is
         case insensitive.  The initial value is \texttt{"Extended"}.   \texttt{[}current value\texttt{{]}}
      }
      \sstsubsection{
         MARGIN( 4 ) = \_REAL (Read)
      }{
         The widths of the margins to leave for axis annotation, given
         as fractions of the corresponding dimension of the current picture.
         Four values may be given, in the order: bottom, right, top, left.
         If fewer than four values are given, extra values are used equal to
         the first supplied value.  If these margins are too narrow any axis
         annotation may be clipped.  If a null (\texttt{{!}}) value is supplied, the
         value used is \texttt{0.15} (for all edges) if either annotated axes or a
         key are produced, and zero otherwise.   \texttt{[}current value\texttt{{]}}
      }
      \sstsubsection{
         MARKER = \_INTEGER (Read)
      }{
         This parameter is only accessed if Parameter MODE is set to
         \texttt{"Chain"} or \texttt{"Mark"}.  It specifies the symbol with which each
         position should be marked, and should be given as an integer
         \PGPLOT\  marker type.  For instance, \texttt{0} gives a box,
         \texttt{1} gives a dot, \texttt{2} gives a cross, \texttt{3} gives an asterisk,
         \texttt{7} gives a triangle.  The value must be larger than or equal
         to $-$31.   \texttt{[}current value\texttt{{]}}
      }
      \sstsubsection{
         MODE = \htmlref{LITERAL}{se:parmenu} (Read)
      }{
         Specifies the way in which data values are represented.  MODE
         can take the following values.

         \ssthitemlist{

            \sstitem
            \texttt{"Histogram"} --- An histogram of the points is plotted in the
            style of a `staircase' (with vertical lines only joining the
            \textit{y}-axis values and not extending to the base of the plot).  The
            vertical lines are placed midway between adjacent \textit{x}-axis
            positions.  Bad values are flanked by vertical lines to the
            lower edge of the plot.

            \sstitem
            \texttt{"GapHistogram"} --- The same as the \texttt{"Histogram"} option except bad
            values are not flanked by vertical lines to the lower edge of
            the plot, leaving a gap.

            \sstitem
            \texttt{"Line"} --- The points are joined by straight lines.

            \sstitem
            \texttt{"Point"} --- A dot is plotted at each point.

            \sstitem
            \texttt{"Mark"} --- Each point is marker with a symbol specified by
            Parameter MARKER.

            \sstitem
            \texttt{"Chain"} --- A combination of \texttt{"Line"} and \texttt{"Mark"}.

            \sstitem
            \texttt{"Step"} --- Each point is displayed as a horizontal line, whose
            length is specified by the axis width of the pixel.

         }
         The initial default is \texttt{"Line"}.  \texttt{[}current value\texttt{{]}}
      }
      \sstsubsection{
         NDF = NDF (Read)
      }{
         NDF structure containing the array to be plotted.
      }
      \sstsubsection{
         SHAPE = LITERAL (Read)
      }{
         Specifies the way in which errors are represented.  SHAPE
         can take the following values.

         \ssthitemlist{

            \sstitem
            \texttt{"Bars"} --- Bars with serifs (\emph{i.e.} cross pieces at each end) are
            drawn joining the \textit{x}-error limits and the \textit{y}-error limits.  The plotting
            attribute \htmlattref{Size(ErrBars)}{Size(element)}~ (see Parameter STYLE) can be used to
            control the size of these serifs (the attribute value should be
            a magnification factor; 1.0 gives default serifs).

            \sstitem
            \texttt{"Cross"} --- San-serif bars are drawn joining the \textit{x}-error limits and
            the \textit{y}-error limits.

            \sstitem
            \texttt{"Diamond"} --- Adjacent error limits are joined to form an
            error diamond.

         }
         The length of the error bars can be controlled using Parameters
         XSIGMA and YSIGMA.  The colour, line width and line style used to
         draw them can be controlled using the plotting attributes
         \att{Colour(ErrBars)}, \att{Width(ErrBars)} and
         \htmlattref{Style(ErrBars)}{Style(element)}~
         (see Parameter STYLE).  SHAPE is only accessed when Parameter ERRBAR
         is \texttt{TRUE}.    \texttt{[}current value\texttt{{]}}
      }
      \sstsubsection{
         STYLE = \htmlref{GROUP}{se:groups} (Read)
      }{
         A group of attribute settings describing the plotting style to
         use when drawing the annotated axes, data values, and error
         markers.

         A comma-separated list of strings should be given in which each
         string is either an attribute setting, or the name of a text
         file preceded by an up-arrow character \texttt{"$\wedge$"}.  Such text files
         should contain further comma-separated lists which will be
         read and interpreted in the same manner.  Attribute settings
         are applied in the order in which they occur within the list,
         with later settings overriding any earlier settings given for
         the same attribute.

         Each individual attribute setting should be of the form:

            $<$name$>$=$<$value$>$

         where $<$name$>$ is the name of a plotting attribute, and $<$value$>$
         is the value to assign to the attribute.  Default values will be
         used for any unspecified attributes.  All attributes will be
         defaulted if a null value (\texttt{{!}})---the initial default---is supplied.
         To apply changes of style to only the current invocation, begin these
         attributes with a plus sign.  A mixture of persistent and temporary
         style changes is achieved by listing all the persistent attributes
         followed by a plus sign then the list of temporary attributes.

         See \slhyperref{Plotting Attributes}{Section~}{}{ap:plotting_attr}
         for a description of the available attributes.  Any unrecognised
         attributes are ignored (no error is reported).

         The appearance of the data values is controlled by the attributes
         \htmlattref{Colour(Curves)}{Colour(element)},
         \htmlattref{Width(Curves)}{Width(element)}, \emph{etc.} (the synonym
         \att{Lines} may be used in place of \att{Curves}).  The appearance
         of markers used if Parameter MODE is set to \texttt{"Point"},
         \texttt{"Mark"} or \texttt{"Chain"} is controlled by
         \att{Colour(Markers)}, \att{Width(Markers)}, \emph{etc.} (the
         synonym \att{Symbols} may be used in place of \att{Markers}).  The
         appearance of the error symbols is
         controlled using \att{Colour(ErrBars)}, \att{Width(ErrBars)}, \emph{etc}, (see
         Parameter SHAPE).   \texttt{[}current value\texttt{{]}}
      }
      \sstsubsection{
         USEAXIS = LITERAL (Read)
      }{
         Specifies the quantity to be used to annotate the horizontal axis
         of the plot using one of the following options.

         \ssthitemlist{

            \sstitem
            An integer index of an axis within the current Frame of the
            input NDF (in the range 1 to the number of axes in the current
            Frame).

            \sstitem
            An axis \htmlattref{Symbol}{Symbol(axis)}~ string such as
            \texttt{"RA"} or \texttt{"VRAD"}.

            \sstitem
            A generic option where \texttt{"SPEC"} requests the spectral
            axis, \texttt{"TIME"} selects the time axis, \texttt{"SKYLON"}
            and \texttt{"SKYLAT"} picks the sky longitude and latitude
            axes respectively.  Only those axis domains present are
            available as options.

            \sstitem
            The special value \texttt{{0}} (zero), asks for the distance
            along the profile from the centre of the first element in
            the supplied NDF to be used to annotate the axis.  This
            will be measured in the current co-ordinate Frame of the
            NDF.
         }

         The quantity used to annotate the horizontal axis must have a
         defined value at all points in the array, and must increase or
         decrease monotonically along the array.  For instance, if RA is
         used to annotate the horizontal axis, then an error will be
         reported if the profile passes through RA=0 because it will
         introduce a non-monotonic jump in axis value (from 0h to 24h, or
         24h to 0h).  If a null (\texttt{{!}}) value is supplied, the value used is
         \texttt{{1}} if the current co-ordinate Frame in the NDF is one-dimensional
         and \texttt{{0}} otherwise.  \texttt{[!]}
      }
      \sstsubsection{
         XLEFT = LITERAL (Read)
      }{
         The axis value to place at the left hand end of the horizontal
         axis.  If a null (\texttt{{!}}) value is supplied, the value used is the value
         for the first element in the supplied NDF (with a margin to include
         any horizontal error bar).  The value supplied may be greater than or
         less than the value supplied for XRIGHT.  A formatted value for the
         quantity specified by Parameter USEAXIS should be supplied.  See also
         Parameter ALIGN.  \texttt{[!]}
      }
      \sstsubsection{
         XMAP = LITERAL (Read)
      }{
         Specifies how the quantity represented by the \textit{x} axis is mapped
         on to the plot.  The options are as follows.

         \ssthitemlist{

            \sstitem
            \texttt{"Pixel"} --- The mapping is such that pixel index within the
            input NDF increases linearly across the plot.

            \sstitem
            \texttt{"Distance"} --- The mapping is such that distance along the curve
            within the current \htmlref{WCS}{apndf:wcs} Frame of the input NDF increases linearly
            across the plot.

            \sstitem
            \texttt{"Log"} --- The mapping is such that the logarithm (base 10) of
            the value used to annotate the axis increases linearly across
            the plot.  An error will be reported if the dynamic range of
            the axis is less than 100, or if the range specified by XLEFT
            and XRIGHT encompasses the value zero.

            \sstitem
            \texttt{"Linear"} --- The mapping is such that the value used to
            annotate the axis increases linearly across the plot. Note the
            corresponding pixel indices always increase left to right in
            this mode so the annotated values may possibly increase right
            to left depending on the nature of the WCS mapping.

            \sstitem
            \texttt{"LRLinear"} --- Like \texttt{"Linear"} except that the pixel
            indices are reversed if necessary to ensure that the annotated values
            always increases left to right.

            \sstitem
            \texttt{"Default"} --- One of \texttt{"Linear"} or \texttt{"log"} is chosen automatically,
            depending on which one produces a more-even spread of values on the
            plot.
         }
         \texttt{["Default"]}
      }
      \sstsubsection{
         XRIGHT = LITERAL (Read)
      }{
         The axis value to place at the right hand end of the horizontal
         axis.  If a null (\texttt{{!}}) value is supplied, the value used is the value
         for the last element in the supplied NDF (with a margin to include
         any horizontal error bar).  The value supplied may be greater than
         or less than the value supplied for XLEFT.  A formatted value for
         the quantity specified by Parameter USEAXIS should be supplied.
         See also Parameter ALIGN.  \texttt{[!]}
      }
      \sstsubsection{
         XSIGMA = LITERAL (Read)
      }{
         If horizontal error bars are produced (see Parameter ERRBAR), then
         XSIGMA gives the number of standard deviations that the error
         bars are to represent.   \texttt{[}current value\texttt{{]}}
      }
      \sstsubsection{
         YBOT = LITERAL (Read)
      }{
         The axis value to place at the bottom end of the vertical
         axis.  If a null (\texttt{{!}}) value is supplied, the value used is
         determined by Parameter LMODE.  The value of YBOT may be
         greater than or less than the value supplied for YTOP.  If
         Parameter YMAP is set to \texttt{"ValueLog"}, then the supplied value
         should be the logarithm (base 10) of the bottom data value.
         See also Parameter ALIGN.  \texttt{[!]}
      }
      \sstsubsection{
         YMAP = LITERAL (Read)
      }{
         Specifies how the quantity represented by the \textit{y} axis is mapped
         on to the screen.  The options are as follows.

         \ssthitemlist{

            \sstitem
            \texttt{"Linear"} --- The data values are mapped linearly on to the
            screen.

            \sstitem
            \texttt{"Log"} --- The data values are logged logarithmically on to the
            screen.  An error will be reported if the dynamic range of
            the axis is less than 100, or if the range specified by YTOP and
            YBOT encompasses the value zero.  For this reason, care should
            be taken over the choice of value for Parameter LMODE, since
            some choices could result in the \textit{y} range being extended so far
            that it encompasses zero.

            \sstitem
            \texttt{"ValueLog"} --- This is similar to \texttt{"Log"} except that, instead
            of mapping the data values logarithmically on to the screen,
            this option maps the log (base 10) of the data values linearly
            on to the screen.  If this option is selected, the values supplied
            for Parameters YTOP and YBOT should be values for the logarithm of
            the data value, not the data value itself.
         }
         \texttt{["Linear"]}
      }
      \sstsubsection{
         YSIGMA = LITERAL (Read)
      }{
         If vertical error bars are produced (see Parameter ERRBAR), then
         YSIGMA gives the number of standard deviations that the error
         bars are to represent.   \texttt{[}current value\texttt{{]}}
      }
      \sstsubsection{
         YTOP = LITERAL (Read)
      }{
         The axis value to place at the top end of the vertical
         axis.  If a null (\texttt{{!}}) value is supplied, the value used is
         determined by Parameter LMODE.  The value of LTOP may be
         greater than or less than the value supplied for YBOT.  If
         Parameter YMAP is set to \texttt{"ValueLog"}, then the supplied value
         should be the logarithm (base 10) of the bottom data value.
         See also Parameter ALIGN.  \texttt{[!]}
      }
   }
   \sstexamples{
      \sstexamplesubsection{
         linplot spectrum
      }{
         Plots data values versus position for the whole of the
         one-dimensional NDF called spectrum on the current graphics
         device.  If the current co-ordinate Frame of the NDF is also
         one-dimensional, then the horizontal axis will be labelled with
         values on Axis 1 of the current co-ordinate Frame.  Otherwise, it
         will be labelled with offset from the first element.
      }
      \sstexamplesubsection{
         linplot map(,100)
      }{
         Plots data values versus position for row 100 in the two-dimensional
         NDF called map on the \htmlref{current graphics device}{se:devglobal}.
      }
      \sstexamplesubsection{
         linplot spectrum(1:500) device=ps\_l
      }{
         Plots data values versus position for the first 500 elements
         of the one-dimensional NDF called spectrum.  The output goes to a
         text file which can be printed on a PostScript printer.
      }
      \sstexamplesubsection{
         linplot ironarc v style="title=Fe Arc variance"
      }{
         Plots variance values versus position for the whole of the
         one-dimensional NDF called ironarc on the current graphics device.
         The plot has a title of \texttt{"Fe Arc variance"}.
      }
      \sstexamplesubsection{
         linplot prof useaxis=dec xleft="23:30:22" xright="23:30:45"
      }{
         This plots data values versus declination for those elements of the
         one-dimensional NDF called prof with declination value between 23d
         30m 22s, and 23d 30m 45s.  This assumes that the current
         co-ordinate Frame in the NDF has an axis with symbol \texttt{"dec"}.
      }
      \sstexamplesubsection{
         linplot prof useaxis=2 ybot=10 ytop=1000.0 ymap=log xmap=log
      }{
         This plots the data values in the entire one-dimensional NDF called
         prof, against the value described by the second axis in the current
         co-ordinate Frame of the NDF.  The values represented by both
         axes are mapped logarithmically on to the screen.  The bottom of the
         vertical axis corresponds to a data value of 10.0 and the top
         corresponds to a data value of 1000.0.
      }
      \sstexamplesubsection{
         linplot xspec mode=p errbar xsigma=3 ysigma=3 shape=d style=$\wedge$my\_sty
      }{
         This plots the data values versus position for the dataset
         called xspec.  Each pixel is plotted as a point surrounded by
         diamond-shaped error bars.  The error bars are 3-sigma error
         bars.  The plotting style is read from text file \texttt{
         my\_sty}.  This could, for instance, contain strings such as:
         \texttt{colour(err)=pink, colour(sym)=red, tickall=0,
         edge(2)=right}.  These cause the error bars to be drawn in
         pink, the points to be drawn in red, tick marks to be
         restricted to the labelled edges of the plot, and the
         vertical axis (Axis 2) to be annotated on the right-hand edge
         of the plot.   The plotting style specified in file
         \texttt{my\_sty} becomes the default plotting style for future
         invocations of LINPLOT.
      }
      \sstexamplesubsection{
         linplot xspec mode=p errbar xsigma=3 ysigma=3 shape=d style=+$\wedge$my\_sty
      }{
         This is the same as the previous example, except that the style
         specified in file \texttt{my\_sty} does not become the default
         style for future invocations of LINPLOT.
      }
      \sstexamplesubsection{
         linplot ndf=spectrum noclear align
      }{
         Plots data values versus pixel co-ordinate for the whole of
         the one-dimensional NDF called spectrum on the current graphics
         device.  The plot is drawn over any existing plot and inherits
         the bounds of the previous plot on both axes.  A warning will be
         reported if the labels for the horizontal axes of the two plots
         are different.
      }
      \sstexamplesubsection{
         linplot spectrum system="'system(1)=freq,unit(1)=GHz'"
      }{
         This example assumes that the current co-ordinate Frame of NDF
         \texttt{"spectrum"} is a \xref{SpecFrame}{sun210}{SpecFrame}.  The
         horizontal axis (\texttt{"Axis 1"}) is labelled
         with frequency values, in units of GHz.  If the SpecFrame represents
         some other system (such as wavelength, velocity, energy)
         or has some other units, then the conversion is done automatically.
         Note, a SpecFrame is a specialised class of Frame which knows how to
         do these conversions; the above command will fail if the current
         co-ordinate Frame in the NDF is a simple Frame (such as the AXIS
         Frame).  A SpecFrame can be created from an AXIS Frame using
         application \htmlref{WCSADD}{WCSADD}.
      }
   }
   \sstnotes{
      \sstitemlist{

         \sstitem
         If no \htmlattref{Title}{plotel:Title}~ is specified via the
         STYLE parameter, then the \htmlref{TITLE}{apndf:title}
         component in the NDF is used as the default title for the
         annotated axes.  Should the NDF not have a TITLE component,
         then the default title is instead taken from current
         co-ordinate Frame in the NDF, unless this attribute has not
         been set explicitly, whereupon the name of the NDF is used as
         the default title.

         \sstitem
         Default axis errors and widths are used, if none are present in
         the NDF.  The defaults are the constants 0 and 1 respectively.

         \sstitem
         The application stores a number of pictures in the
         \htmlref{graphics database}{se:agitate}~ in the following order:
         a FRAME picture containing the
         annotated axes, data plot, and optional key; a KEY picture to store
         the key if present; and a DATA picture containing just the data plot.
         Note, the FRAME picture is only created if annotated axes or a key
         has been drawn, or if non-zero margins were specified using Parameter
         MARGIN.  The world co-ordinates in the DATA picture will correspond
         to offset along the profile on the horizontal axis, and data value
         (or logarithm of data value) on the vertical axis.  On exit the current
         database picture for the chosen device reverts to the input picture.
      }
   }
   \sstdiytopic{
      Related Applications
   }{
KAPPA: \htmlref{PROFILE}{PROFILE},
\htmlref{CLINPLOT}{CLINPLOT};
\htmlref{MLINPLOT}{MLINPLOT};
\xref{FIGARO}{sun86}{}: \xref{ESPLOT}{sun86}{ESPLOT},
\xref{IPLOTS}{sun86}{IPLOTS},
\xref{MSPLOT}{sun86}{MSPLOT},
\xref{SPECGRID}{sun86}{SPECGRID};
\xref{SPLOT}{sun86}{SPLOT},
\xref{SPLAT}{sun243}{}.
   }
   \sstimplementationstatus{
      \sstitemlist{

         \sstitem
         This routine correctly processes the \htmlref{AXIS}{apndf:axis}, DATA, \htmlref{VARIANCE}{apndf:variance},
         \htmlref{QUALITY}{apndf:quality}, \htmlref{LABEL}{apndf:label}, \htmlref{TITLE}{apndf:title}, \htmlref{WCS}{apndf:wcs}, and \htmlref{UNITS}{apndf:units}~ components of the NDF.

         \sstitem
         Processing of \htmlref{bad pixels}{se:masking} and automatic \htmlref{quality masking}{se:qualitymask} are
         supported.

         \sstitem
         All \htmlref{non-complex numeric data types}{ap:HDStypes} can be handled.  Only
         double-precision floating-point data can be processed directly.
         Other non-complex data types will undergo a type conversion before the plot is drawn.
      }
   }
}



\sstroutine{
   LISTMAKE
}{
   Creates a catalogue holding a positions list
}{
   \sstdescription{
      This application creates a catalogue containing a list of positions
      supplied by the user, together with information describing the
      \htmlref{co-ordinate Frames}{se:domains}~ in which the positions are defined.  Integer
      position identifiers which allow positions to be distinguished are
      also stored in the catalogue.  The catalogue may be manipulated
      using the \CURSAref\ ~package, and is stored in either FITS
      binary format or the \emph{Small Text List}~ (\STLref) format
      defined by \CURSA.

      If an \NDFref{NDF} is specified using Parameter NDF, then the positions should
      be given in the current co-ordinate Frame of the NDF.  Information
      describing the co-ordinate Frames available within the NDF will be
      copied to the output positions list.  Subsequent applications can
      use this information in order to align the positions with other
      data sets.

      If no NDF is specified, then the user must indicate the co-ordinate
      Frame in which the positions will be supplied using Parameter FRAME.
      A description of this Frame will be written to the output positions
      list for use by subsequent applications.

      The positions themselves may be supplied within a text file,
      or may be given in response to repeated prompts for a parameter.
      Alternatively, pixel centres in the NDF supplied for Parameter
      NDF can be used (see Parameter MODE).

      The output can be initialised by copying positions from an existing
      positions list.  Any positions supplied directly by the user are then
      appended to the end of this initial list (see Parameter INCAT).
   }
   \sstusage{
      listmake outcat [ndf] [mode]
        $\left\{ {\begin{tabular}{l}
                  file=? \\
                  position=?
                  \end{tabular} }
        \right.$
        \newline\latexhtml{\hspace*{14.4em}}{~~~~~~~~~~~~~~~~~~~~~~~~~~~}
        \makebox[0mm][c]{\small mode}
   }
   \sstparameters{
      \sstsubsection{
         CATFRAME = LITERAL (Read)
      }{
         A string determining the co-ordinate Frame in which positions
         are to be stored in the output catalogue associated with
         Parameter OUTCAT.  The string supplied for CATFRAME can be one
         of the following options.

         \ssthitemlist{

            \sstitem
            A \htmlref{domain name}{se:domains}~ such as
            \htmlref{SKY, AXIS, PIXEL}{se:resdoms}.

            \sstitem
            An integer value giving the index of the required Frame.

            \sstitem
            An IRAS90 \emph{Sky Co-ordinate System} (SCS) values such as
            \texttt{"EQUAT(J2000)"} (see \xref{SUN/163}{sun163}{}).

         }

         If a null (\texttt{{!}}) value is supplied, the positions will be
         stored in the current Frame. \texttt{[!]}
      }
      \sstsubsection{
         CATEPOCH = \_DOUBLE (Read)
      }{
         The epoch at which the sky positions stored in the output
         catalogue were determined.  It will only be accessed if an
         epoch value is needed to qualify the co-ordinate Frame
         specified by COLFRAME.  If required, it should be given as a
         decimal years value, with or without decimal places
         (\texttt{"1996.8"} for example).  Such values are interpreted as
         a Besselian epoch if less than 1984.0 and as a Julian epoch
         otherwise.
      }
      \sstsubsection{
         DESCRIBE = \_LOGICAL (Read)
      }{
         If \texttt{TRUE}, a detailed description of the co-ordinate Frame
         in which positions are required will be displayed before the
         positions are obtained using either Parameter POSITION or
         FILE.  \texttt{[}current value\texttt{{]}}
      }
      \sstsubsection{
         DIM = \_INTEGER (Read)
      }{
         The number of axes for each position.  It is only accessed if
         a null value is supplied for Parameter NDF.
      }
      \sstsubsection{
         EPOCH = \_DOUBLE (Read)
      }{
         If an IRAS90 Sky Co-ordinate System specification is supplied
         (using Parameter DOMAIN) for a celestial co-ordinate system,
         then an epoch value is needed to qualify it.  This is the epoch at
         which the supplied sky positions were determined.  It should be
         given as a decimal years value, with or without decimal places
         (\texttt{"1996.8"} for example).  Such values are interpreted as a
         Besselian epoch if less than 1984.0 and as a Julian epoch otherwise.
      }
      \sstsubsection{
         FILE = FILENAME (Read)
      }{
         A text file containing the positions to be stored in the output
         positions list.  Each line should contain the
         \xref{formatted axis values}{sun210}{AST_UNFORMAT}
         for a single position, separated by white space.  It is only
         accessed  if Parameter MODE is given the value \texttt{"File"}.
      }
      \sstsubsection{
         FRAME = LITERAL (Read)
      }{
         Specifies the co-ordinate Frame of the positions supplied through
         Parameters POSITION or FILE.  There is a cascade of allowed
         interpretations of this parameter value; the search for the
         co-ordinate Frame ends once there is a successful interpretation,
         otherwise the search moves on to the next possible meaning in
         the following order.

         \begin{enumerate}
         \item An HDS path containing a WCS FrameSet, whose current
         Frame defines the co-ordinate Frame.

         \item The name of an NDF, whose current WCS co-ordinate Frame
         is used.

         \item If the parameter value ends with \texttt{.FIT}, an attempt is
         made to interpret the parameter value as the name of a FITS file.
         If successful, the primary WCS co-ordinate system from the primary
         HDU headers is used.

         \item A text file containing either an AST Frame dump (such as
         produced by commands in the ATOOLS package), or a set of FITS
         WCS headers.

         \item An IRAS90 \emph{Sky Co-ordinate System} (SCS) string
         such as \texttt{"EQUAT(J2000)"} (see \xref{SUN/163}{sun163}{}),
         whereupon the positions are assumed to be two-dimensional
         celestial co-ordinates in the specified system.

         \item Domain name without any interpretation.   Any Domain name may
         be supplied, but normally one of the standard
         \htmlref{Domain names}{se:domains}, such as
         \htmlref{GRID, PIXEL, GRAPHICS}{se:resdoms}~ should be given.
         Parameter DIM is used to determine the number of axes in the Frame.
         \end{enumerate}

         This parameter is only accessed if the parameter NDF is given a
         null value.
      }
      \sstsubsection{
         INCAT = FILENAME (Read)
      }{
         A catalogue containing an existing positions list which is to be
         included at the start of the output positions list.  These positions
         are mapped into the current co-ordinate Frame of the supplied
         NDF, or into the Frame specified by Parameter FRAME if no NDF was
         supplied.  A message is displayed indicating the Frame in which
         alignment occurred.  They are then stored in the output list before
         any further positions are added.  A null value may be supplied
         if there is no input positions list.  \texttt{[!]}
      }
      \sstsubsection{
         MODE = \htmlref{LITERAL}{se:parmenu} (Read)
      }{
         The mode by which the positions are to be obtained.  The options are
         as follows.

         \ssthitemlist{

            \sstitem
            \texttt{"Interface"} --- The positions are obtained using Parameter
            POSITION.

            \sstitem
            \texttt{"File"} --- The positions are to be read from a text file
            specified using Parameter FILE.

            \sstitem
            \texttt{"Good"} --- The positions used are the pixel centres in the data
            file specified by Parameter NDF.  Only the pixels that have good
            values in the Data array of the NDF are used.

            \sstitem
            \texttt{"Pixel"} --- The positions used are the pixel centres in the data
            file specified by Parameter NDF.  All pixel are used, whether the
            pixel values are good or not.

         }
         \texttt{["Interface"]}
      }
      \sstsubsection{
         NDF = NDF (READ)
      }{
         The NDF which defines the available co-ordinate Frames in the
         output positions list.  If an NDF is supplied, the positions
         obtained using Parameter POSITION or FILE are assumed to be
         in the current co-ordinate Frame of the NDF, and the
         \htmlref{WCS}{apndf:wcs} component of the NDF is copied to
         the output positions list.  If a null value is supplied, the
         single co-ordinate Frame defined by Parameter FRAME is stored
         in the output positions list, and supplied positions are
         assumed to be in the same Frame.  \texttt{[!]}
      }
      \sstsubsection{
         OUTCAT = FILENAME (Write)
      }{
         The catalogue holding the output positions list.
      }
      \sstsubsection{
         POSITION = LITERAL (Read)
      }{
         The co-ordinates of a single position to be stored in the
         output positions list.  Supplying \texttt{":"} will display
         details of the co-ordinate Frame in which the position is
         required.  The position should be given as a list of
         \xref{formatted axis values}{sun210}{AST_UNFORMAT} separated
         by white space. You are prompted for new values for this
         parameter until a null value is entered.  It is only accessed
         if Parameter MODE is given the value \texttt{"Interface"}.
      }
      \sstsubsection{
         TITLE = LITERAL (Read)
      }{
         A title for the output positions list.  If a null (\texttt{{!}})
         value is supplied, the value used is obtained from the input
         positions list if one is supplied.  Otherwise, it is obtained
         from the NDF if one is supplied.  Otherwise, it is \texttt{
         "Output from LISTMAKE"}.  \texttt{[!]}
      }
   }
   \sstexamples{
      \sstexamplesubsection{
         listmake newlist frame=pixel dim=2
      }{
         This creates a FITS binary catalogue called \texttt{newlist.FIT} containing a
         list of positions, together with a description of a single
         two-dimensional pixel co-ordinate Frame.  The positions are supplied as
         a set of space-separated pixel co-ordinates in response to repeated
         prompts for the Parameter POSITION.
      }
      \sstexamplesubsection{
         listmake stars.txt frame=equat(B1950) epoch=1962.3
      }{
         This creates a catalogue called \texttt{stars.txt} containing a list of
         positions, together with a description of a single FK4 equatorial
         RA/DEC co-ordinate Frame (referenced to the B1950 equinox).  The
         catalogue is stored in a text file using the CAT \emph{Small Text List}
         format (\texttt{"STL"}---see \xref{SUN/190}{sun190}{}).  The positions
         were determined at epoch
         B1962.3.  The epoch of observation is required since the underlying
         model on which the FK4 system is based is non-inertial and rotates
         slowly with time, introducing fictitious proper motions.  The
         positions are supplied hours and degrees values in reponse to
         repeated prompts for Parameter POSITIONS.
      }
      \sstexamplesubsection{
         listmake outlist ndf=allsky mode=file file=stars
      }{
         This creates a FITS binary catalogue called \texttt{outlist.FIT} containing a
         list of positions, together with descriptions of all the co-ordinate
         Frames contained in the NDF allsky.  The positions are supplied
         as co-ordinates within the current co-ordinate Frame of the NDF.
         Application \htmlref{WCSFRAME}{WCSFRAME} can be used to find out what this Frame is.
         The positions are supplied in a text file called \texttt{stars}.
      }
      \sstexamplesubsection{
         listmake out.txt incat=old.fit frame=gal
      }{
         This creates an STL format catalogue stored in a text file called
         \texttt{out.txt} containing a list of positions, together with a description
         of a single galactic co-ordinate Frame.  The positions contained in
         the existing binary FITS catalogue \texttt{old.fit} are mapped into galactic
         co-ordinates (if possible) and stored in the output positions list.
         Further galactic co-ordinate positions are then obtained by repeated
         prompting for the Parameter POSITION.  These positions are
         appended to the positions obtained from file \texttt{old.fit}.
      }
      \sstexamplesubsection{
         listmake out.txt incat=old.fit ndf=cobe
      }{
         As above but the output positions list contains copies of all the
         Frames in the NDF cobe.  The positions in \texttt{old.fit} are mapped into
         the current co-ordinate Frame of the NDF (if possible) before
         being stored in the output positions list.  The new positions must
         also be supplied in the same Frame (using Parameter POSITION).
      }
      \sstexamplesubsection{
         listmake profpos.fit ndf=prof1 mode=pixel
      }{
         This creates a positions list called \texttt{profpos.fit} containing the
         positions of all the pixel centres in the one-dimensional NDF
         called prof.  This could for instance be used as input to
         application PROFILE in order to produce another profile in which
         the samples are at the same positions as those in NDF prof.
      }
   }
   \sstnotes{
      \sstitemlist{

         \sstitem
         This application uses the conventions of the \CURSAref\ package
         for determining the formats of input and output catalogues.  If a file
         type of .fit is given, then the catalogue is assumed to be a FITS
         binary table.  If a file type of .txt is given, then the catalogue is
         assumed to be stored in a text file in STL format.  If no file type is
         given, then \texttt{".fit"} is assumed.

         \sstitem
         There is a limit of 200 on the number of positions which can be
         given using Parameter POSITION.  There is no limit on the number of
         positions which can be given using Parameter FILE.

         \sstitem
         Position identifiers are asigned to the supplied positions in
         the order in which they are supplied.  If no input positions list is
         given using Parameter INCAT, then the first supplied position will
         be assigned the identifier \texttt{"1"}.  If an input positions list is
         given, then the first supplied position is assigned an identifier
         one greater than the largest identifier in the input positions list.
      }
   }
   \sstdiytopic{
      Related Applications
   }{
KAPPA: \htmlref{CURSOR}{CURSOR},
\htmlref{LISTSHOW}{LISTSHOW};
\CURSA: \xref{XCATVIEW}{sun190}{XVIEW},
\xref{CATSELECT}{sun190}{CATSELECT}.
   }
}
\sstroutine{
   LISTSHOW
}{
   Reports the positions stored in a positions list
}{
   \sstdescription{
      This application reports positions contained in a catalogue.  The
      catalogue should have the form of a positions list as produced, for
      instance, by applications \htmlref{LISTMAKE}{LISTMAKE} and
      \htmlref{CURSOR}{CURSOR}.  By default all positions
      in the catalogue are reported, but a subset may be reported by
      specifying a range of \emph{position identifiers}~ (see Parameters FIRST,
      LAST, and STEP).

      An \NDFref{NDF} may be supplied (see Parameter NDF) in which case the NDF
      pixel values at the positions listed in the catalogue are reported,
      using the interpolation method specified by Parameter METHOD. The
      pixel values are also written to an output parameter (see Parameter
      PIXVALS).

      Positions may be reported in a range of \htmlref{co-ordinate Frames}{se:domains}~ dependent
      on the information stored in the supplied positions list (see Parameter
      FRAME).  The selected positions are written to an output parameter
      (Parameter POSNS), and may also be written to an output positions
      list (see Parameter OUTCAT).  The formatted screen output can be saved
      in a logfile (see Parameter LOGFILE).  The formats used to report the
      axis values can be controlled using Parameter STYLE.

      Graphics may also be drawn marking the selected positions (see
      Parameters PLOT and LABEL).  The supplied positions are aligned with the
      picture specified by Parameter NAME.  If possible, this alignment occurs
      within the  co-ordinate Frame specified using Parameter FRAME.  If this
      is not possible, alignment may occur in some other suitable Frame.  A
      message is displayed indicating the Frame in which alignment occurred.
      If the supplied positions are aligned successfully with a picture, then
      the range of Frames in which the positions may be reported on the screen
      is extended to include all those associated with the picture.
   }
   \sstusage{
      listshow incat [frame] [first] [last] [plot] [device]
   }
   \sstparameters{
      \sstsubsection{
         CATFRAME = LITERAL (Read)
      }{
         A string determining the co-ordinate Frame in which positions are
         to be stored in the output catalogue associated with Parameter
         OUTCAT.  See Parameter FRAME for a description of the allowed
         values for this parameter.  If a null (\texttt{{!}}) value is supplied,
         the positions will be stored the Frame used to specify positions
         within the input catalogue.  \texttt{[!]}
      }
      \sstsubsection{
         CATEPOCH = \_DOUBLE (Read)
      }{
         The epoch at which the sky positions stored in the output
         catalogue were determined.  It will only be accessed if an epoch
         value is needed to qualify the co-ordinate Frame specified by
         COLFRAME.  If required, it should be given as a decimal-years
         value, with or without decimal places (\texttt{"1996.8"}, for example).
         Such values are interpreted as a Besselian epoch if less than
         1984.0 and as a Julian epoch otherwise.
      }
      \sstsubsection{
         CLOSE = \_LOGICAL (Read)
      }{
         This parameter is only accessed if Parameter PLOT is set to
         \texttt{"Chain"} or \texttt{"Poly"}.  If \texttt{TRUE}, polgons will be closed by joining
         the first position to the last position.   \texttt{[}current value\texttt{{]}}
      }
      \sstsubsection{
         COMP = LITERAL (Read)
      }{
         The NDF array component to be displayed if a non-null value is
         supplied for Parameter NDF. It may be \texttt{"Data"},
         \texttt{"Variance"}, \texttt{"Error"}, or \texttt{"Quality"}. \texttt{["Data"]}
      }
      \sstsubsection{
         DESCRIBE = \_LOGICAL (Read)
      }{
         If \texttt{TRUE}, a detailed description of the co-ordinate Frame in which
         the positions will be reported is displayed before the
         positions.   \texttt{[}current value\texttt{{]}}
      }
      \sstsubsection{
         DEVICE = \htmlref{DEVICE}{se:selgradev} (Read)
      }{
         The graphics workstation.  Only accessed if Parameter PLOT
         indicates that graphics are required.  \texttt{[}current graphics device\texttt{]}
      }
      \sstsubsection{
         EPOCH = \_DOUBLE (Read)
      }{
         If an IRAS90 Sky Co-ordinate System specification is supplied
         (using Parameter FRAME) for a celestial co-ordinate system,
         then an epoch value is needed to qualify it.  This is the epoch at
         which the supplied sky positions were determined.  It should be
         given as a decimal years value, with or without decimal places
         (\texttt{"1996.8"} for example).  Such values are interpreted as a Besselian
         epoch if less than 1984.0 and as a Julian epoch otherwise.
      }
      \sstsubsection{
         FIRST = \_INTEGER (Read)
      }{
         The identifier for the first position to be displayed.  Positions
         are only displayed which have identifiers in the range given by
         Parameters FIRST and LAST.  If a null (\texttt{{!}}) value is supplied, the
         value used is the lowest identifier value in the positions list.  \texttt{[!]}
      }
      \sstsubsection{
         FRAME = LITERAL (Read)
      }{
         A string determining the co-ordinate Frame in which positions are
         to be reported.  This application can report positions in
         any of the co-ordinate Frames stored with the positions list.  The
         string supplied for FRAME can be one of the following.

         \ssthitemlist{

            \sstitem
            A \htmlref{domain name}{se:domains}~ such as \htmlref{SKY, AXIS, PIXEL}{se:resdoms}.

            \sstitem
            An integer value giving the index of the required Frame.

            \sstitem
            An IRAS90 \emph{Sky Co-ordinate System} (SCS) values such as
            \texttt{"EQUAT(J2000)"} (see \xref{SUN/163}{sun163}{}).

         }
         If a null value (\texttt{{!}}) is supplied, positions are reported in the
         co-ordinate Frame which was current when the positions list was
         created.  The user is re-prompted if the specified Frame is not
         available within the positions list.  The range of Frames available
         will include all those read from the supplied positions list.  In
         addition, if a graphics device is opened (\emph{i.e.} if Parameter PLOT
         is set to anything other than \texttt{"None"}), then all the Frames associated
         with the picture specified by Parameter NAME will also be available.
         \texttt{[!]}
      }
      \sstsubsection{
         GEODESIC = \_LOGICAL (Read)
      }{
         This parameter is only accessed if Parameter PLOT is set to
         \texttt{"Chain"} or \texttt{"Poly"}.  It specifies whether the curves drawn between
         positions should be stright lines, or should be geodesic curves.
         In many co-ordinate Frames geodesic curves will be simple straight
         lines.  However, in others (such as the majority of celestial
         co-ordinate Frames) geodesic curves will be more complex curves
         tracing the shortest path between two positions in a non-linear
         projection.  \texttt{[FALSE]}
      }
      \sstsubsection{
         INCAT = FILENAME (Read)
      }{
         A catalogue containing a positions list such as produced by
         applications LISTMAKE and CURSOR.
      }
      \sstsubsection{
         JUST = LITERAL (Read)
      }{
         A string specifying the justification to be used when displaying
         text strings at the supplied positions.  This parameter is
         only accessed if Parameter PLOT is set to \texttt{"Text"}.  The supplied
         string should contain two characters; the first should be \texttt{"B"},
         \texttt{"C"}, or \texttt{"T"}, meaning bottom, centre, or top
         respectively.  The second should be \texttt{"L"}, \texttt{"C"}, or
         \texttt{"R"}, meaning left, centre, or right respectively.  The text is
         displayed so that the supplied position is at the specified
         point within the displayed text string.  \texttt{[CC]}
      }
      \sstsubsection{
         LABEL = \_LOGICAL (Read)
      }{
         If \texttt{TRUE} the positions are labelled on the graphics device
         specified by Parameter DEVICE.  The offset of the centre of each
         label from the corresponding position is controlled using the
         \htmlattref{NumLabGap(1)}{NumLabGap(axis)}~ and \att{NumLabGap(2)}
         plotting attributes, and the appearance of the labels is controlled
         using attributes \htmlattref{Colour(NumLab)}{Colour(element)},
         \htmlattref{Size(NumLab)}{Size(element)}, \emph{etc}.  These
         attributes may be specified using Parameter STYLE.  The content of
         the label is determined by Parameter LABTYPE.  \texttt{[FALSE]}
      }
      \sstsubsection{
         LABTYPE = LITERAL (Read)
      }{
         Determines what sort of labels are drawn if the LABEL parameter
         is set \texttt{TRUE}.  It can be either of the following.

         \ssthitemlist{

            \sstitem
            \texttt{"ID"} --- causes the integer identifier associated with
            each row to be used as the label for the row.

            \sstitem
            \texttt{"LABEL"} --- causes the textual label associated with each row
            to be used as the label for the row.  These strings are read from
            the "LABEL" column of the supplied catalogue.
         }

         If a null (\texttt{{!}}) value is supplied, a default of \texttt{"LABEL"}
         will be used if the input catalogue contains a \texttt{"LABEL"} column.
         Otherwise, a default of "ID" will be used.  \texttt{[!]}
      }
      \sstsubsection{
         LAST = \_INTEGER (Read)
      }{
         The identifier for the last position to be displayed.  Positions
         are only displayed which have identifiers in the range given by
         Parameters FIRST and LAST.  If a null (\texttt{{!}}) value is supplied,
         the value used is the highest identifier value in the positions list.  \texttt{[!]}
      }
      \sstsubsection{
         LOGFILE = FILENAME (Write)
      }{
         The name of the text file in which the formatted co-ordinates of
         the selected positions may be stored.  This is intended primarily
         for recording the screen output, and not for communicating
         positions to subsequent applications.  A null string (\texttt{{!}}) means that
         no file is created.  \texttt{[!]}
      }
      \sstsubsection{
         MARKER = \_INTEGER (Read)
      }{
         This parameter is only accessed if Parameter PLOT is set to
         \texttt{"Chain"} or \texttt{"Mark"}.  It specifies the type of marker with which each
         position should be marked, and should be given as an integer
         \PGPLOT\  marker type.  For instance, \texttt{0} gives a box,
         \texttt{1} gives a dot, \texttt{2} gives a cross, \texttt{3} gives an asterisk,
         \texttt{7} gives a triangle.  The value must be larger than or equal
         to $-$31.  \texttt{[}current value\texttt{{]}}
      }
      \sstsubsection{
         METHOD = LITERAL (Read)
      }{
         The method to use when sampling the input NDF (if any) specified
         by Parameter NDF at each of the positions in the catalogue. For
         details on these schemes, see the description of routine
         \xref{AST\_RESAMPLEx}{sun210}{AST_RESAMPLE\$<X>\$} in
         \xref{SUN/210}{sun210}{}. Note, \texttt{"Nearest"} is always used if
         Parameter COMP is \texttt{"Quality"} . METHOD can take
         the following values.

         \sstitemlist{

            \sstitem
            \texttt{"Bilinear"} --- The displayed pixel values are calculated by
            bi-linear interpolation among the four nearest pixels values
            in the input NDF.  This produces smoother output NDFs than the
            nearest-neighbour scheme, but is marginally slower.

            \sstitem
            \texttt{"Nearest"} --- Each displayed pixel value is the value of the
            nearest input pixel.

            \sstitem
            \texttt{"Sinc"} --- Uses the ${\textrm{sinc}}({\pi}x)$ kernel, where
            $x$ is the pixel offset from the interpolation point, and
            ${\textrm{sinc}}(z)=\sin(z)/z$.  Use of this scheme is not recommended.

            \sstitem
            \texttt{"SincSinc"} --- Uses the ${\textrm{sinc}}({\pi}x){\textrm{sinc}}(k{\pi}x)$
            A valuable general-purpose scheme, intermediate in its visual
            effect on NDFs between the bi-linear and nearest-neighbour schemes.

            \sstitem
            \texttt{"SincCos"} --- Uses the ${\textrm{sinc}}({\pi}x)\cos(k{\pi}x)$
            kernel.  Gives similar results to the \texttt{"SincSinc"} scheme.

            \sstitem
            \texttt{"SincGauss"} --- Uses the ${\textrm{sinc}}({\pi}x)e^{-kx^2}$
            kernel.  Good results can be obtained by matching the FWHM of the
            envelope function to the point-spread function of the
            input data (see Parameter PARAMS).

            \sstitem
            \texttt{"Somb"} --- Uses the  ${\textrm{somb}}({\pi}x)$ kernel, where
            $x$ is the pixel offset from the interpolation point, and
            ${\textrm{somb}}(z)=2*J_{1}(z)/z$. $J_1$ is the first-order Bessel
            function of the first kind.  This scheme is similar to the
            \texttt{"Sinc"} scheme.

            \sstitem
            \texttt{"SombCos"} --- Uses the ${\textrm{somb}}({\pi}x)\cos(k{\pi}x)$
            kernel.  This scheme is similar to the \texttt{"SincCos"} scheme.

            \sstitem
            \texttt{"Gauss"} --- Uses the $e^{-kx^2}$ kernel.  The FWHM of the
            Gaussian is given by Parameter PARAMS(2), and the point at
            which to truncate the Gaussian to zero is given by Parameter
            PARAMS(1).

         }
         The initial default is \texttt{"} Nearest\texttt{"} .  [current value]
      }
      \sstsubsection{
         NAME = LITERAL (Read)
      }{
         Determines the graphics database picture with which the supplied
         positions are to be aligned.  Only accessed if Parameter PLOT
         indicates that some graphics are to be produced.  A search is made
         for the most recent picture with the specified name (\emph{e.g.} DATA,
         FRAME or KEY) within the current picture.  If no such picture can
         be found, or if a null value is supplied, the current picture itself
         is used.  The name BASE can also be supplied as a special case, which
         causes the BASE picture to be used even though it will not in
         general fall within the current picture.  \texttt{["DATA"]}
      }
      \sstsubsection{
         NDF = NDF (Read)
      }{
         If an NDF is supplied, the values within the NDF at the positions
         specified in the input catalogue are displayed on the screen and
         written to Output Parameter PIXVALS. The displayed values are
         calculated by interpolating between the NDF pixel values using
         the interpolation method specified by Parameter METHOD. The NDF
         array component to be displayed is specified by Parameter COMP. [!]
      }
      \sstsubsection{
         OUTCAT = FILENAME (Write)
      }{
         The output catalogue in which to store the selected positions.
         If a null value is supplied, no output catalogue is produced.  See
         Parameter COLFRAME.  \texttt{[!]}
      }
      \sstsubsection{
         PARAMS( 2 ) = \_DOUBLE (Read)
      }{
         An optional array which consists of additional parameters
         required by the Sinc, SincSinc, SincCos, SincGauss, Somb,
         SombCos, and Gauss methods (see Parameter METHOD).

         PARAMS( 1 ) is required by all the above schemes.
         It is used to specify how many pixels are to contribute to the
         interpolated result on either side of the interpolation or
         binning point in each dimension. Typically, a value of \texttt{2} is
         appropriate and the minimum allowed value is \texttt{1} (i.e. one pixel
         on each side). A value of zero or fewer indicates that a
         suitable number of pixels should be calculated automatically.
         [0]

         PARAMS( 2 ) is required only by the Gauss, SombCos, SincSinc,
         SincCos, and SincGauss schemes. For the SombCos, SincSinc, and
         SincCos schemes, it specifies the number of pixels at which the
         envelope of the function goes to zero. The minimum value is
         \texttt{1.0}, and the run-time default value is \texttt{2.0}. For the Gauss and
         SincGauss schemes, it specifies the full-width at half-maximum
         (FWHM) of the Gaussian envelope measured in output pixels.  The
         minimum value is \texttt{0.1}, and the run-time default is \texttt{1.0}.  On
         astronomical NDFs and spectra, good results are often obtained
         by approximately matching the FWHM of the envelope function,
         given by PARAMS(2), to the point-spread function of the input
         data. []
      }
      \sstsubsection{
         PLOT = \htmlref{LITERAL}{se:parmenu} (Read)
      }{
         The type of graphics to be used to mark the positions on the
         graphics device specified by Parameter DEVICE.  The appearance of
         these graphics (colour, size, \emph{etc.}) is controlled by the STYLE
         parameter.  PLOT can take any of the following values.

         \ssthitemlist{

            \sstitem
            \texttt{"None"} --- No graphics are produced.

            \sstitem
            \texttt{"Mark"} --- Each position is marked with a marker of type specified
            by Parameter MARKER.

            \sstitem
            \texttt{"Poly"} --- Causes each position to be joined by a line to the
            previous position.  These lines may be simple straight lines or
            geodesic curves (see Parameter GEODESIC).  The polygons may
            optionally be closed by joining the last position to the first (see
            Parameter CLOSE).

            \sstitem
            \texttt{"Chain"} --- This is a combination of \texttt{"Mark"} and \texttt{"Poly"}.  Each
            position is marked by a marker and joined by a line to the previous
            position.  Parameters MARKER, GEODESIC, and CLOSE are used to
            specify the markers and lines to use.

            \sstitem
            \texttt{"Box"} --- A rectangular box with edges parallel to the edges of
            the graphics device is drawn between each pair of positions.

            \sstitem
            \texttt{"Vline"} --- A vertical line is drawn through each position,
            extending the entire height of the selected picture.

            \sstitem
            \texttt{"Hline"} --- A horizontal line is drawn through each position,
            extending the entire width of the selected picture.

            \sstitem
            \texttt{"Cross"} --- A combination of \texttt{"Vline"} and \texttt{"Hline"}.

            \sstitem
            \texttt{"STCS"} --- Indicates that each position should be
            marked using the two-dimensional STC-S shape read from the
            catalogue column specified by Parameter STCSCOL.

            \sstitem
            \texttt{"Text"} --- A text string is used to mark each position.  The string
            is drawn horizontally with the justification specified by Parameter
            JUST.  The strings to use for each position are specified using
            Parameter STRINGS.

            \sstitem
            \texttt{"Blank"} --- The graphics device is opened and the picture specified
            by Parameter NAME is found, but no actual graphics are drawn to mark
            the positions.  This can be useful if you just want to transform
            the supplied positions into one of the co-ordinate Frames associated
            with the picture, without drawing anything (see Parameter FRAME).

         }
         Each position may also be separately labelled with its integer
         identifier value by giving a \texttt{TRUE} value for Parameter LABEL.  \texttt{["None"]}
      }
      \sstsubsection{
         POSNS() = \_DOUBLE (Write)
      }{
         The unformatted co-ordinates of the positions selected by
         Parameters FIRST and LAST, in the co-ordinate Frame selected by
         FRAME.  The axis values are stored as a one-dimensional vector.  All
         the Axis-1 values for the selected positions are stored first,
         followed by the Axis-2 values, \emph{etc}.  The number of positions in
         the vector is written to the output Parameter NUMBER, and the
         number of axes per position is written to the output Parameter
         DIM.  The axis values may not be in the same units as the
         formatted values shown on the screen.  For instance, unformatted
         celestial co-ordinate values are stored in units of radians.
      }
      \sstsubsection{
         STEP = \_INTEGER (Read)
      }{
         The increment between position identifiers to be displayed.
         Specifying a value larger than 1 causes a subset of the position
         identifiers between FIRST and LAST to be displayed. \texttt{[1]}
      }
      \sstsubsection{
         STCSCOL = LITERAL (Read)
      }{
         The name of a catalogue column containing an STC-S description of
         a two-dimensional spatial shape associated with each position. The
         STC-S format is an IVOA proposal for describing regions of space,
         time and spectral position. For further details, see the STC-S
         document on the IVOA web site
         (\htmladdnormallink{\texttt{{http://www.ivoa.net/Documents/}}}
         {http://www.ivoa.net/Documents/}). An STC-S description of a
         shape includes the co-ordinate system in which the shape is
         defined. This application assumes that all the STC-S shapes
         read from the specified column will be defined within the same
         co-ordinate system. The transformation from the STC-S co-ordinate
         system to the co-ordinate system of the displayed image is
         determined once from the first shape plotted, and then re-used
         for all later shapes.  \texttt{["Shape"]}
      }
      \sstsubsection{
         STRINGS = LITERAL (Read)
      }{
         A group of text strings which are used to mark the supplied positions
         if Parameter PLOT is set to \texttt{"Text"}.  The first string in the group
         is used to mark the first position, the second string is used to
         mark the second position, \emph{etc}.  If more positions are given than there
         are strings in the group, then the extra positions will be marked
         with an integer value indicating the index within the list of
         supplied positions. (Note, these integers may be different from the
         identifiers in the supplied positions list).  If a null value (\texttt{{!}})
         is given for the parameter, then all positions will be marked with
         the integer indices, starting at 1.

         A comma-separated list should be given in which each element is
         either a marker string, or the name of a text file preceded by an
         up-arrow character \texttt{"$\wedge$"}.  Such text files should contain further
         comma-separated lists which will be read and interpreted in the
         same manner.  Note, strings within text files can be separated by
         new lines as well as commas.
      }
      \sstsubsection{
         STYLE = \htmlref{GROUP}{se:groups} (Read)
      }{
         A group of attribute settings describing the style to use when
         formatting the co-ordinate values displayed on the screen, and
         when drawing the graphics specified by Parameter PLOT.

         A comma-separated list of strings should be given in which each
         string is either an attribute setting, or the name of a text
         file preceded by an up-arrow character \texttt{"$\wedge$"}.  Such text files
         should contain further comma-separated lists which will be
         read and interpreted in the same manner.  Attribute settings
         are applied in the order in which they occur within the list,
         with later settings overriding any earlier settings given for
         the same attribute.

         Each individual attribute setting should be of the form:

            $<$name$>$=$<$value$>$


         where $<$name$>$ is the name of a plotting attribute, and $<$value$>$
         is the value to assign to the attribute.  Default values will be
         used for any unspecified attributes.  All attributes will be
         defaulted if a null value (\texttt{{!}})---the initial default---is supplied.
         To apply changes of style to only the current invocation, begin these
         attributes with a plus sign.  A mixture of persistent and temporary
         style changes is achieved by listing all the persistent attributes
         followed by a plus sign then the list of temporary attributes.

         See \slhyperref{Plotting Attributes}{Section~}{}{ap:plotting_attr}
         for a description of the available attributes.  Any unrecognised
         attributes are ignored (no error is reported).

         In addition to the attributes which control the appearance of
         the graphics (\htmlattref{Colour}{Colour(element)}, \htmlattref{Font}{Font(element)}, \emph{etc.}), the following attributes may
         be set in order to control the appearance of the formatted axis
         values reported on the screen: \htmlattref{Format}{Format(axis)}, \htmlattref{Digits}{Digits/Digits(axis)},
         \htmlattref{Symbol}{Symbol(axis)}, \htmlattref{Unit}{Unit(axis)}.  These
         may be suffixed with an axis number (\emph{e.g.} \att{Digits(2)}) to refer to
         the values displayed for a specific axis.  \texttt{[}current value\texttt{{]}}
      }
   }
   \sstresparameters{
      \sstsubsection{
         DIM = \_INTEGER (Write)
      }{
         The number of axes for each position written to output Parameter
         POSNS.
      }
      \sstsubsection{
         NUMBER = \_INTEGER (Write)
      }{
         The number of positions selected.
      }
      \sstsubsection{
         PIXVALS() = \_DOUBLE (Write)
      }{
         The pixel values at the listed positions. Only used if a
         non-null value is supplied for Parameter NDF.
      }
   }
   \sstexamples{
      \sstexamplesubsection{
         listshow stars pixel
      }{
         This displays the pixel co-ordinates of all the positions
         stored in the FITS binary catalogue \texttt{stars.fit}.  They are all written
         to the output Parameter POSNS.
      }
      \sstexamplesubsection{
         listshow star outcat=star-gal catframe=gal
      }{
         This copies a position list from catalogue \texttt{star} to a new
         catalogue called \texttt{star-gal}.  The positions are stored in galactic
         co-ordinates in the output catalogue.
      }
      \sstexamplesubsection{
         listshow stars.fit equat(J2010) first=3 last=3
      }{
         This extracts Position 3 from the catalogue \texttt{stars.fit} transforming
         it into FK5 equatorial RA/DEC co-ordinates (referenced to the
         J2010 equinox), if possible.  The RA/DEC values (in radians) are
         written to the output Parameter POSNS.
      }
      \sstexamplesubsection{
         listshow stars\_2.txt style="digits(1)=5,digits(2)=7"
      }{
         This lists the positions in the STL format catalogue contained
         in text file \texttt{stars\_2.txt} in their original co-ordinate Frame.  By
         default, five digits are used to format Axis-1 values, and seven to format
         Axis-2 values.  These defaults are overridden if the attributes
         \htmlattref{Format(1)}{Format(axis)}~ and/or \att{Format(2)} are assigned
         values in the description of the current Frame stored in the positions list.
      }
      \sstexamplesubsection{
         listshow s.txt plot=marker marker=3 style="colour(marker)=red,size=2"
      }{
         This marks the positions in \texttt{s.txt} on the currently selected graphics
         device using \PGPLOT\  Marker 3 (an asterisk).  The positions are aligned
         with the most recent DATA picture in the current picture.  The markers
         are red and are twice the default size.  The positions are likely not
         to be reported on the screen.
      }
   }
   \sstnotes{
      \sstitemlist{

         \sstitem
         This application uses the conventions of the \CURSAref\ package
         for determining the formats of input and output catalogues.  If a file
         type of .fits is given, then the catalogue is assumed to be a FITS
         binary table.  If a file type of .txt is given, then the catalogue is
         assumed to be stored in a text file in \emph{Small Text List} (STL) format.
         If no file type is given, then \texttt{.fit} is assumed.

         \sstitem
         The positions are not displayed on the screen when either the
         message filter environment variable MSG\_FILTER is set to \texttt{NORMAL}
         and any graphics or labels are being plotted (see Parameters PLOT
         and LABEL); or when MSG\_FILTER is set to \texttt{QUIET} and no graphics
         are produced.  The creation of output parameters and files is
         unaffected by MSG\_FILTER.

      }
   }
   \sstdiytopic{
      Related Applications
   }{
KAPPA: \htmlref{CURSOR}{CURSOR},
\htmlref{LISTMAKE}{LISTMAKE};
\CURSA: \xref{XCATVIEW}{sun190}{XVIEW},
\xref{CATSELECT}{sun190}{CATSELECT}.
   }
}
\sstroutine{
   LOG10
}{
   Takes the base-10 logarithm of an NDF data structure
}{
   \sstdescription{
      This routine takes the base-10 logarithm of each
      pixel of a \NDFref{NDF} to produce a new NDF data structure.

      This command is a synonym for \texttt{logar base=10D0}.
   }
   \sstusage{
      log10 in out
   }
   \sstparameters{
      \sstsubsection{
         IN = NDF (Read)
      }{
         Input NDF data structure.
      }
      \sstsubsection{
         OUT = NDF (Write)
      }{
         Output NDF data structure being the logarithm of the input NDF.
      }
      \sstsubsection{
         TITLE = LITERAL (Read)
      }{
         The title for the output NDF.  A null value will cause
         the title of the NDF supplied for Parameter IN to be used
         instead.  \texttt{[!]}
      }
   }
   \sstexamples{
      \sstexamplesubsection{
         log10 a b
      }{
         This takes logarithms to base ten of the pixels in the NDF
         called a, to make the NDF called b.  NDF b inherits its title
         from a.
      }
      \sstexamplesubsection{
         log10 title="Abell 4321" out=b in=a
      }{
         This takes logarithms to base ten of the pixels in the NDF
         called a, to make the NDF called b.  NDF b has the title
         \texttt{"Abell 4321"}.
      }
   }
   \sstdiytopic{
      Related Applications
   }{
KAPPA: \htmlref{LOGAR}{LOGAR},
\htmlref{LOGE}{LOGE},
\htmlref{EXP10}{EXP10},
\htmlref{EXPE}{EXPE},
\htmlref{EXPON}{EXPON},
\htmlref{POW}{POW};
\xref{FIGARO}{sun86}{}: \xref{IALOG}{sun86}{IALOG},
\xref{ILOG}{sun86}{ILOG},
\xref{IPOWER}{sun86}{IPOWER}.
   }
   \sstimplementationstatus{
      \sstitemlist{

         \sstitem
         This routine correctly processes the \htmlref{AXIS}{apndf:axis}, DATA, \htmlref{QUALITY}{apndf:quality},
         \htmlref{LABEL}{apndf:label}, \htmlref{TITLE}{apndf:title}, \htmlref{UNITS}{apndf:units}, \htmlref{HISTORY}{apndf:history}, \htmlref{WCS}{apndf:wcs}, and \htmlref{VARIANCE}{apndf:variance}~ components of
         an NDF data structure and propagates all \htmlref{extensions}{apndf:extensions}.

         \sstitem
         Processing of \htmlref{bad pixels}{se:masking} and automatic \htmlref{quality masking}{se:qualitymask} are
         supported.

         \sstitem
         All \htmlref{non-complex numeric data types}{ap:HDStypes} can be handled.
      }
   }
}

\sstroutine{
   LOGAR
}{
   Takes the logarithm (specified base) of an NDF data structure
}{
   \sstdescription{
      This routine takes the logarithm to a specified base of each
      pixel of a \NDFref{NDF} to produce a new NDF data structure.
   }
   \sstusage{
      logar in out base
   }
   \sstparameters{
      \sstsubsection{
         BASE = LITERAL (Read)
      }{
         The base of the logarithm to be applied.  A special value
         \texttt{"Natural"} gives natural (base-e) logarithms.
      }
      \sstsubsection{
         IN = NDF (Read)
      }{
         Input NDF data structure.
      }
      \sstsubsection{
         OUT = NDF (Write)
      }{
         Output NDF data structure being the logarithm of the input NDF.
      }
      \sstsubsection{
         TITLE = LITERAL (Read)
      }{
         The title for the output NDF.  A null value will cause
         the title of the NDF supplied for Parameter IN to be used
         instead.  \texttt{[!]}
      }
   }
   \sstexamples{
      \sstexamplesubsection{
         logar a b 10
      }{
         This takes logarithms to base ten of the pixels in the NDF
         called a, to make the NDF called b.  NDF b inherits its title
         from a.
      }
      \sstexamplesubsection{
         logar base=8 title="HD123456" out=b in=a
      }{
         This takes logarithms to base eight of the pixels in the NDF
         called a, to make the NDF called b.  NDF b has the title
         \texttt{"HD123456"}.
      }
   }
   \sstdiytopic{
      Related Applications
   }{
KAPPA: \htmlref{LOG10}{LOG10},
\htmlref{LOGE}{LOGE},
\htmlref{EXP10}{EXP10},
\htmlref{EXPE}{EXPE},
\htmlref{EXPON}{EXPON},
\htmlref{POW}{POW};
\xref{FIGARO}{sun86}{}: \xref{IALOG}{sun86}{IALOG},
\xref{ILOG}{sun86}{ILOG},
\xref{IPOWER}{sun86}{IPOWER}.
   }
   \sstimplementationstatus{
      \sstitemlist{

         \sstitem
         This routine correctly processes the \htmlref{AXIS}{apndf:axis}, DATA, \htmlref{QUALITY}{apndf:quality},
         \htmlref{LABEL}{apndf:label}, \htmlref{TITLE}{apndf:title}, \htmlref{UNITS}{apndf:units}, \htmlref{HISTORY}{apndf:history}, \htmlref{WCS}{apndf:wcs}, and \htmlref{VARIANCE}{apndf:variance}~ components of
         an NDF data structure and propagates all \htmlref{extensions}{apndf:extensions}.

         \sstitem
         Processing of \htmlref{bad pixels}{se:masking} and automatic \htmlref{quality masking}{se:qualitymask} are
         supported.

         \sstitem
         All \htmlref{non-complex numeric data types}{ap:HDStypes} can be handled.
      }
   }
}
\sstroutine{
   LOGE
}{
   Takes the natural logarithm of an NDF data structure
}{
   \sstdescription{
      This routine takes the natural logarithm of each
      pixel of a \NDFref{NDF} to produce a new NDF data structure.

      This command is a synonym for \texttt{logar base=natural}.
   }
   \sstusage{
      loge in out
   }
   \sstparameters{
      \sstsubsection{
         IN = NDF (Read)
      }{
         Input NDF data structure.
      }
      \sstsubsection{
         OUT = NDF (Write)
      }{
         Output NDF data structure being the logarithm of the input NDF.
      }
      \sstsubsection{
         TITLE = LITERAL (Read)
      }{
         The title for the output NDF.  A null value will cause
         the title of the NDF supplied for Parameter IN to be used
         instead.  \texttt{[!]}
      }
   }
   \sstexamples{
      \sstexamplesubsection{
         loge a b
      }{
         This takes the natural logarithm of the pixels in the NDF
         called a, to make the NDF called b.  NDF b inherits its title
         from a.
      }
      \sstexamplesubsection{
         loge title="Cas A" out=b in=a
      }{
         This takes natural logarithms of the pixels in the NDF
         called a, to make the NDF called b.  NDF b has the title
         \texttt{"Cas A"}.
      }
   }
   \sstdiytopic{
      Related Applications
   }{
KAPPA: \htmlref{LOG10}{LOG10},
\htmlref{LOGAR}{LOGAR},
\htmlref{EXP10}{EXP10},
\htmlref{EXPE}{EXPE},
\htmlref{EXPON}{EXPON},
\htmlref{POW}{POW};
\xref{FIGARO}{sun86}{}: \xref{IALOG}{sun86}{IALOG},
\xref{ILOG}{sun86}{ILOG},
\xref{IPOWER}{sun86}{IPOWER}.
   }
   \sstimplementationstatus{
      \sstitemlist{

         \sstitem
         This routine correctly processes the \htmlref{AXIS}{apndf:axis}, DATA, \htmlref{QUALITY}{apndf:quality},
         \htmlref{LABEL}{apndf:label}, \htmlref{TITLE}{apndf:title}, \htmlref{UNITS}{apndf:units}, \htmlref{HISTORY}{apndf:history}, \htmlref{WCS}{apndf:wcs}, and \htmlref{VARIANCE}{apndf:variance}~ components of
         an NDF data structure and propagates all \htmlref{extensions}{apndf:extensions}.

         \sstitem
         Processing of \htmlref{bad pixels}{se:masking} and automatic \htmlref{quality masking}{se:qualitymask} are
         supported.

         \sstitem
         All \htmlref{non-complex numeric data types}{ap:HDStypes} can be handled.
      }
   }
}
\sstroutine{
   LOOK
}{
   List pixel values in a two-dimensional NDF
}{
   \sstdescription{
      This application lists pixel values within a region of
      a two-dimensional \NDFref{NDF}.  The listing may be displayed on the screen
      and logged in a text file (see Parameter LOGFILE).  The region
      to be listed can be specified either by giving its centre and size or
      its corners, or by giving an `ARD Description' for the region (see
      Parameter MODE).  The top-right pixel value is also written to an output
      parameter (VALUE).  The listing may be produced in several different
      formats (see Parameter FORMAT), and the format of each individual
      displayed data value can be controlled using Parameter STYLE.
   }
   \sstusage{
      look ndf centre [size] [logfile] [format] [comp] [mode]
        \newline\hspace*{1.5em}
        $\left\{ {\begin{tabular}{l}
                  arddesc=? \\
                  ardfile=? \\
                  lbound=? ubound=? \\
                  centre=?
                  \end{tabular} }
        \right.$
        \newline\hspace*{1.9em}
%        \newline\latexhtml{\hspace*{27.9em}}{~~~~~~~~~~~~~~~~~~~~~~~~~~~~~~~~~~~~~~~~~~~~~~}
        \makebox[0mm][c]{\small mode}
   }
   \sstparameters{
      \sstsubsection{
         AGAIN = \_LOGICAL (Read)
      }{
         If \texttt{TRUE}, the user is prompted for further regions to list until
         a \texttt{FALSE} value is obtained.  \texttt{[FALSE]}
      }
      \sstsubsection{
         ARDDESC = LITERAL (Read)
      }{
         An \xref{`ARD Description'}{sun183}{}~ for the parts of the image to
         be listed.  Multiple lines can be supplied by ending each line with
         a hyphen, in which case further prompts for ARDDESC are made until
         a value is supplied which does not end with a hyphen.  All
         the supplied values are then concatenated together (after removal
         of the trailing hyphens).  ARCDESC is only acessed if MODE is \texttt{"ARD"}.
         Positions in the ARD description are assumed to be in the
         current \htmlref{co-ordinate Frame}{se:domains}~  of the NDF unless
         there are COFRAME or WCS statements which indicate a different system.  See
         \htmlref{``Notes''}{notes:look} below.
      }
      \sstsubsection{
         ARDFILE = FILENAME (Read)
      }{
         The name of an existing text file containing an `ARD Description'
         for the parts of the image to be listed.  ARDFILE is only accessed if
         MODE is \texttt{"ARDFile"}.  Positions in the ARD description are
         assumed to be in pixel co-ordinates unless there are COFRAME or
         WCS statements that indicate a different system.  See
         \htmlref{``Notes''}{notes:look} below.
      }
      \sstsubsection{
         CENTRE = LITERAL (Read)
      }{
         The co-ordinates of the data pixel at the centre of the area to
         be displayed, in the current co-ordinate Frame of the NDF (supplying
         a colon \texttt{":"} will display details of the current co-ordinate Frame).
         The position should be supplied as a list of
         \xref{formatted axis values}{sun210}{AST_UNFORMAT}
         separated by spaces or commas.  See also Parameter USEAXIS.  Only
         acessed if MODE is \texttt{"Centre"}.
      }
      \sstsubsection{
         COMP = \htmlref{LITERAL}{se:parmenu} (Read)
      }{
         The NDF array component to be displayed.  It may be \texttt{"Data"},
         \texttt{"Quality"}, \texttt{"Variance"}, or \texttt{"Error"} (where \texttt{"Error"}
         is an alternative to \texttt{"Variance"} and causes the square root of the
         variance values to be displayed).  If \texttt{"Quality"} is specified,
         then the quality values are treated as numerical values (in
         the range 0 to 255).  \texttt{["Data"]}
      }
      \sstsubsection{
         FORMAT = LITERAL (Read)
      }{
         Specifies the format for the listing from the following options.

         \ssthitemlist{

            \sstitem
            \texttt{"strips"} --- The area being displayed is divided up into
            vertical strips of limited width.  Each strip is displayed in
            turn, with \textit{y} pixel index at the left of each row, and
            \textit{x} pixel index at the top of each column.  The highest row is
            listed first in each strip.  This format is intended for human
            readers --- the others are primarily intended for being read by
            other software.

            \sstitem
            \texttt{"clist"} --- Each row of textual output consists of an \textit{x} pixel
            index, followed by a \textit{y} pixel index, followed by the pixel data
            value.  No headers or blank lines are included.  The pixels are
            listed in `Fortran order'---the lower-left pixel first, and the
            upper-right pixel last.

            \sstitem
            \texttt{"cglist"} --- Like \texttt{"clist"} except that bad pixel
            are omitted from the list.

            \sstitem
            \texttt{"vlist"} --- Each row of textual output consists of just the
            pixel data value.  No headers or blank lines are included.  The
            pixels are listed in `Fortran order'---the lower-left pixel first,
            and the upper-right pixel last.

            \sstitem
            \texttt{"wlist"} --- Each row of textual output consists of the
            WCS co-ordinate values, followed by the pixel data value.  No
            headers or blank lines are included.  The pixels are listed in
            `Fortran order'---the lower-left pixel first, and the
            upper-right pixel last.

            \sstitem
            \texttt{"wglist"} --- Like \texttt{"wlist"} except that bad pixel
            are omitted from the list.

            \sstitem
            \texttt{"region"} --- The pixel data values are listed as a two-dimensional
            region.  Each row of textual output contains a whole row of data
            values.  The textual output may be truncated if it is too wide.  The
            lowest row is listed first.

         }
         In all cases, adjacent values are separated by spaces, and bad
         pixel values are represented by the string \texttt{"BAD"}.  \texttt{["strips"]}
      }
      \sstsubsection{
         LBOUND = LITERAL (Read)
      }{
         The co-ordinates of the data pixel at the bottom-left of the
         area to be displayed, in the current co-ordinate Frame of the NDF
         (supplying a colon \texttt{":"} will display details of the current
         co-ordinate Frame).  The position should be supplied as a list of
         formatted axis values separated by spaces or commas.  See also
         Parameter USEAXIS.  A null (\texttt{{!}}) value causes the bottom-left corner
         of the supplied NDF to be used.  LBOUND is only accessed if
         MODE is \texttt{"Bounds"}.
      }
      \sstsubsection{
         LOGFILE = FILENAME (Write)
      }{
         The name of the text file in which the textual output may be stored.
         See MAXLEN.  A null string (\texttt{{!}}) means that no file is created.  \texttt{[!]}
      }
      \sstsubsection{
         MAXLEN = \_INTEGER (Read)
      }{
         The maximum number of characters in a line of textual output.  The
         line is truncated after the last complete value if it would extend
         beyond this value.  \texttt{[80]}
      }
      \sstsubsection{
         MODE = \htmlref{LITERAL}{se:parmenu} (Read)
      }{
         Indicates how the region to be listed will be specified:

         \ssthitemlist{

            \sstitem
            \texttt{"All"} --- The entire NDF is used.

            \sstitem
            \texttt{"Centre"} --- The centre and size of the region are specified
            using Parameters CENTRE and SIZE.

            \sstitem
            \texttt{"Bounds"} --- The bounds of the region are specified using
            Parameters LBOUND and UBOUND.

            \sstitem
            \texttt{"ARDFile"} --- The region is given by an `ARD Description'
            supplied within a text file specified using Parameter ARDFILE.
            Pixels outside the ARD region are represented by the string \texttt{"OUT"}.

            \sstitem
            \texttt{"ARD"} --- The region is given using an ARD description
            supplied directly using Parameter ARDDESC.  Pixels outside the
            ARD region are represented by the string \texttt{"OUT"}.

         }
         \texttt{["Centre"]}
      }
      \sstsubsection{
         NDF = NDF (Read)
      }{
         The input NDF structure containing the data to be displayed.
      }
      \sstsubsection{
         SIZE( 2 ) = \_INTEGER (Read)
      }{
         The dimensions of the rectangular area to be displayed, in pixels.
         If a single value is given, it is used for both axes.  The area
         is centred on the position specified by Parameter CENTRE.
         It is only accessed if MODE is \texttt{"Centre"}.  \texttt{[7]}
      }
      \sstsubsection{
         STYLE = \htmlref{GROUP}{se:groups} (Read)
      }{
         A group of attribute settings describing the format to use
         for individual data values.

         A comma-separated list of strings should be given in which each
         string is either an attribute setting, or the name of a text
         file preceded by an up-arrow character \texttt{"$\wedge$"}.  Such text files
         should contain further comma-separated lists which will be
         read and interpreted in the same manner.  Attribute settings
         are applied in the order in which they occur within the list,
         with later settings overriding any earlier settings given for
         the same attribute.

         Each individual attribute setting should be of the form:

            $<$name$>$=$<$value$>$

         where $<$name$>$ is the name of a plotting attribute, and $<$value$>$
         is the value to assign to the attribute.  Default values will be
         used for any unspecified attributes.  All attributes will be
         defaulted if a null value (\texttt{{!}})---the initial default---is supplied.
         To apply changes of style to only the current invocation, begin these
         attributes with a plus sign.  A mixture of persistent and temporary
         style changes is achieved by listing all the persistent attributes
         followed by a plus sign then the list of temporary attributes.

         See \slhyperref{Plotting Attributes}{Section~}{}{ap:plotting_attr}
         for a description of the available attributes.  Any unrecognised
         attributes are ignored (no error is reported).

         Data values are formatted using attributes \htmlattref{Format(1)}{Format(axis)}
         and  \htmlattref{Digits(1)}{Digits/Digits(axis)}.  \texttt{[}current value\texttt{{]}}
      }
      \sstsubsection{
         UBOUND = LITERAL (Read)
      }{
         The co-ordinates of the data pixel at the top-right corner of the
         area to be displayed, in the current co-ordinate Frame of the NDF
         (supplying a colon \texttt{":"} will display details of the current
         co-ordinate Frame).  The position should be supplied as a list of
         formatted axis values separated by spaces or commas.  See also
         Parameter USEAXIS.  A null (\texttt{{!}}) value causes the top-right corner
         of the supplied NDF to be used.  Only acessed if MODE is \texttt{"Bounds"}.
      }
      \sstsubsection{
         USEAXIS = GROUP (Read)
      }{
         USEAXIS is only accessed if the current co-ordinate Frame of the
         NDF has more than two axes.  A group of two strings should be
         supplied specifying the two axes which are to be used when supplying
         positions for Parameters CENTRE, LBOUND, and UBOUND.  Each axis can
         be specified using one of the following options.

         \ssthitemlist{

            \sstitem
            Its integer index within the \htmlref{current Frame}{se:curframe}~ of
            the input NDF (in the range 1 to the number of axes in the
            current Frame).

            \sstitem
            Its \htmlattref{Symbol}{Symbol(axis)}~ string such as
            \texttt{"RA"} or \texttt{"VRAD"}.

            \sstitem
            A generic option where \texttt{"SPEC"} requests the spectral axis,
            \texttt{"TIME"} selects the time axis, \texttt{"SKYLON"} and
            \texttt{"SKYLAT"} picks the sky longitude and latitude axes
            respectively.  Only those axis domains present are
            available as options.
         }

         A list of acceptable values is displayed
         if you supply an illegal value.  If a null (\texttt{{!}}) value is
         supplied, the axes with the same indices as the two used pixel
         axes within the NDF are selected.  \texttt{[!]}
      }
   }
   \sstresparameters{
      \sstsubsection{
         VALUE = \_DOUBLE (Write)
      }{
         The data value at the top-right pixel in the displayed rectangle.
      }
   }
   \sstexamples{
      \sstexamplesubsection{
         look ngc6872 "1:27:23 -22:41:12" logfile=log
      }{
         Lists a 7$\times$7 block of pixel values centred on RA/DEC 1:27:23,
         $-$22:41:12 (this assumes that the current co-ordinate Frame in
         the NDF is an RA/DEC Frame).  The listing is written to the text
         file \texttt{log}.
      }
      \sstexamplesubsection{
         look m57 mode=bo lbound="18 20" ubound="203 241"
      }{
         Lists the pixel values in an NDF called m57, within a
         rectangular region from pixel (18,20) to (203,241) (this
         assumes that the current co-ordinate Frame in the NDF is pixel
         co-ordinates).  The listing is displayed on the screen only.
      }
      \sstexamplesubsection{
         look ngc6872 "10 11" 1
      }{
         Stores the value of pixel (10,~11) in output Parameter VALUE, but
         does not store it in a log file.  This assumes that the current
         co-ordinate Frame in the NDF is pixel co-ordinates.
      }
      \sstexamplesubsection{
         look ngc6872 mode=ard arddesc="circle(1:27:23,-22:41:12,0:0:10)"
      }{
         Lists the pixel values within a circle of radius 10 ard-seconds,
         centred on RA=1:27:23 DEC=$-$22:41:12.  This assumes that the
         current co-ordinate Frame in the NDF is an RA/DEC Frame.
      }
      \sstexamplesubsection{
         look ngc6872 mode=ardfile ardfile=central.ard
      }{
         Lists the pixel values specified by the ARD description stored
         in the text file \texttt{central.ard}.
      }
   }
   \label{notes:look}
   \sstnotes{
      \sstitemlist{

         \sstitem
         ARD files may be created by ARDGEN or written manually.  In the
         latter case consult SUN/183 for full details of the ARD
         descriptors and syntax; however, much may be learnt from looking
         at the ARD files created by ARDGEN and the ARDGEN documentation.
         There is also a \slhyperref{summary with examples}{in Section~}{}{se:ardwork}.

         \sstitem
         The co-ordinate system in which positions are given within ARD
         descriptions can be indicated by including suitable
         COFRAME or WCS statements within the description (see SUN/183).
         For instance, starting the description with the text
         \texttt{"COFRAME(PIXEL)"} will indicate that positions are specified in
         pixel co-ordinates.  The statement \latex{\goodbreak}\texttt{"COFRAME(SKY,System=FK5)"} would
         indicate that positions are specified in RA/DEC (FK5,J2000).  If
         no such statements are included, then a default co-ordinate system
         is used as specified in the parameter description above.

         \sstitem
         Output messages are not displayed on the screen when the
         message filter environment variable MSG\_FILTER is set to \texttt{QUIET}.
         The creation of output parameters and the log file is unaffected
         by MSG\_FILTER.

      }
   }
   \sstdiytopic{
      Related Applications
   }{
KAPPA: \htmlref{ARDGEN}{ARDGEN},
\htmlref{ARDMASK}{ARDMASK},
\htmlref{ARDPLOT}{ARDPLOT},
\htmlref{TRANDAT}{TRANDAT}.
   }
   \sstimplementationstatus{
      \sstitemlist{

         \sstitem
         This routine correctly processes the DATA, \htmlref{QUALITY}{apndf:quality}~ and
         \htmlref{VARIANCE}{apndf:variance}~ components of the input NDF.

         \sstitem
         Processing of \htmlref{bad pixels}{se:masking} and automatic \htmlref{quality masking}{se:qualitymask} are
         supported.
      }
   }
}
\sstroutine{
   LUCY
}{
   Performs a Richardson-Lucy deconvolution of a one- or two-dimensional
   array
}{
   \sstdescription{
      This application deconvolves the supplied one- or two-dimensional array
      using the Richardson-Lucy (R-L) algorithm.  It takes an array holding
      observed data and another holding a Point-Spread Function (PSF) as
      input and produces an output array with higher resolution.  The
      algorithm is iterative, each iteration producing a new estimate
      of the restored array which (usually) fits the observed data more
      closely than the previous estimate (in the sense that simulated
      data generated from the restored array is closer to the observed
      data).  The closeness of the fit is indicated after each iteration
      by a normalised $\chi^{2}$ value (\emph{i.e.} the $\chi^{2}$ per
      pixel).  The algorithm terminates when the normalised $\chi^{2}$
      given by Parameter AIM is reached, or the maximum number of
      iterations given by Parameter NITER have been performed.  The
      current estimate of the restored array is then written to the
      output \NDFref{NDF}.

      Before the first iteration, the restored array is initialised
      either to the array given by Parameter START, or, if no array is
      given, to the difference between the mean value in the input data
      array and the mean value in the background (specified by
      Parameters BACK and BACKVAL).  Simulated data are then created from
      this trial array by smoothing it with the supplied PSF, and then
      adding the background on.  The $\chi^{2}$ value describing the
      deviation of this simulated data from the observed data are then
      found and displayed.  If the required $\chi^{2}$ is not reached
      by this simulated data, the first iteration commences, which
      consists of creating a new version of the restored array and then
      creating new simulated data from this new restored array (the
      corresponding $\chi^{2}$ value is displayed).  Repeated
      iterations are performed until the required $\chi^{2}$ is
      reached, or the iteration limit is reached.  The new version of
      the restored array is created as follows.

      \begin{enumerate}

         \item A correction factor is found for each data value.  This is
         the ratio of the observed data value to the simulated data
         value.  An option exists to use the Snyder modification as
         used by the LUCY program in the {\footnotesize STSDAS} package within
         \IRAF.  With this option selected, the variance
         of the observed
         data value is added to both the numerator and the denominator
         when finding the correction factors.

         \item These correction factors are mapped into an array by
         smoothing the array of correction factors with the transposed
         PSF.

         \item The current version of the restored array is multiplied by
         this correction factor array to produce the new version of the
         restored array.

      \end{enumerate}

      For further background to the algorithm, see L.B.~Lucy, {\it
      Astron.J.}\ 1974, Vol 79, No. 6.
   }
   \sstusage{
      lucy in psf out [aim]
   }
   \sstparameters{
      \sstsubsection{
         AIM = \_REAL (Read)
      }{
         The $\chi^{2}$ value at which the algorithm should terminate.
         Smaller values of AIM will result in higher apparent resolution
         in the output array but will also cause noise in the observed
         data to be interpreted as real structure.  Small values will
         require larger number of iterations, so NITER may need to be
         given a larger value.  Very-small values may be completely
         un-achievable, indicated by $\chi^{2}$ not decreasing (or
         sometimes increasing) between iterations.  Larger values will
         result in smoother output arrays with less noise.  \texttt{[1.0]}
      }
      \sstsubsection{
         BACK = NDF (Read)
      }{
         An NDF holding the background value for each observed data
         value.  If a null value is supplied, a constant background
         value given by Parameter BACKVAL is used.  \texttt{[!]}
      }
      \sstsubsection{
         BACKVAL = \_REAL (Read)
      }{
         The constant background value to use if BACK is given a null
         value.  \texttt{[0.0]}
      }
      \sstsubsection{
         CHIFAC = \_REAL (Read)
      }{
         The normalised $\chi^{2}$ value which is used to determine if
         the algorithm should terminate is defined as follows:

         \begin{Large}
         \vspace{5mm}
         \hspace{20mm} $\chi^{2} = \frac{1}{N}.\sum \frac{ ( d - s )^{2}}{( CHIFAC.s - \sigma^{2} )}$
         \vspace{5mm}
         \end{Large}

         where the sum is taken over the entire input array (excluding
         the margins used to pad the input array), \textit{n} is the number of
         values summed, $d$ is the observed data value, $s$ is the simulated
         data value based on the current version of the restored array,
         $\sigma^{2}$ is the variance of the error associated with $d$,
         and $CHIFAC$ is the value of Parameter CHIFAC.  Using 0 for CHIFAC
         results in the standard expression for $\chi^{2}$.  However, the
         algorithm sometimes has difficulty fitting bright features and so
         may not reach the required normalised $\chi^{2}$ value.  Setting
         CHIFAC to 1 (as is done by the LUCY program in the STSDAS
         package within IRAF) causes larger data values to be given
         less weight in the $\chi^{2}$ calculation, and so encourages
         lower $\chi^{2}$ values.  \texttt{[1.0]}
      }
      \sstsubsection{
         IN= NDF (Read)
      }{
         The input NDF containing the observed data.
      }
      \sstsubsection{
         NITER = \_INTEGER (Read)
      }{
         The maximum number of iterations to perform.  \texttt{[50]}
      }
      \sstsubsection{
         OUT = NDF (Write)
      }{
         The restored output array.  The background specified by
         Parameters BACK and BACKVAL will have been removed from this
         array.  The output is the same size as the input.  There is no
         \htmlref{VARIANCE}{apndf:variance}~ component in the output, but any QUALITY values are
         propagated from the input to the output.
      }
      \sstsubsection{
         PSF = NDF (Read)
      }{
         An NDF holding an estimate of the Point-Spread Function (PSF)
         of the input array.  This could, for instance, be produced
         using the \KAPPA\ application PSF.  There should be no bad
         pixels in the PSF otherwise an error will be reported.  The
         PSF can be centred anywhere within the array, but the location
         of the centre must be specified using Parameters XCENTRE and
         YCENTRE.  The PSF is assumed to have the value zero outside
         the supplied NDF.
      }
      \sstsubsection{
         SIGMA = \_REAL (Read)
      }{
         The standard deviation of the noise in the observed data.
         This is only used if Parameter VARIANCE is given the value
         \texttt{FALSE}.  If a null (\texttt{{!}}) value is supplied, the value used is
         an estimate of the noise based on the difference between
         adjacent pixel values in the observed data.  \texttt{[!]}
      }
      \sstsubsection{
         START = NDF (Read)
      }{
         An NDF containing an initial guess at the restored array.
         This could, for instance, be the output from a previous run of
         LUCY, in which case the deconvolution would continue from the
         point it had previously reached.  If a null value is given,
         then the restored array is initialised to a constant value
         equal to the difference between the mean observed data value
         and the mean background value.  \texttt{[!]}
      }
      \sstsubsection{
         SNYDER = \_LOGICAL (Read)
      }{
         If \texttt{TRUE} then the variance of the observed data sample is added
         to both the numerator and denominator when evaluating the
         correction factor for each data sample.  This is the modified
         form of the R-L algorithm used by the LUCY program in the
         STSDAS package within IRAF.  \texttt{[TRUE]}
      }
      \sstsubsection{
         THRESH = \_REAL (Read)
      }{
         The fraction of the PSF peak amplitude at which the extents of
         the PSF are determined.  These extents are used to determine
         the size of the margins used to pad the supplied input array.
         Lower values of THRESH will result in larger margins being
         used.  THRESH must be positive and less than 0.5.  \texttt{[0.0625]}
      }
      \sstsubsection{
         TITLE = LITERAL (Read)
      }{
         A \htmlref{title}{apndf:title} for the output NDF.  A null (\texttt{{!}}) value means using the
         title of the input NDF.  \texttt{[!]}
      }
      \sstsubsection{
         VARIANCE = \_LOGICAL (Read)
      }{
         If \texttt{TRUE}, then the variance of each input data sample will be
         obtained from the \htmlref{VARIANCE}{apndf:variance}~ component of
         the input NDF.  An
         error is reported if this option is selected and the NDF has
         no VARIANCE component.  If \texttt{FALSE}, then a constant variance
         equal to the square of the value given for Parameter SIGMA is
         used for all data samples.  If a null (\texttt{{!}}) value is supplied,
         the value used is \texttt{TRUE} if the input NDF has a VARIANCE
         component, and \texttt{FALSE} otherwise.  \texttt{[!]}
      }
      \sstsubsection{
         WLIM = \_REAL (Read)
      }{
         If the input array contains bad pixels, then this parameter
         may be used to determine the number of good data values which
         must contribute to an output pixel before a valid value is
         stored in the restored array.  It can be used, for example, to
         prevent output pixels from being generated in regions where
         there are relatively few good data values to contribute to the
         restored result.  It can also be used to `fill in' small areas
         (\emph{i.e.} smaller than the PSF) of bad pixels.

         The numerical value given for WLIM specifies the minimum total
         weight associated with the good pixels in a smoothing box
         required to generate a good output pixel (weights for each
         pixel are defined by the normalised PSF).  If this specified
         minimum weight is not present, then a bad output pixel will
         result, otherwise a smoothed output value will be calculated.
         The value of this parameter should lie between 0.0 and
         1.0.  WLIM=0 causes a good output value to be created even if
         there is only one good input value, whereas WLIM=1 causes a
         good output value to be created only if all input values are
         good.  Values less than 0.5 will tend to reduce the number of
         bad pixels, whereas values larger than 0.5 will tend to
         increase the number of bad pixels.

         This threshold is applied each time a smoothing operation is
         performed.  Many smoothing operations are typically performed
         in a run of LUCY, and if WLIM is larger than 0.5 the effects
         of bad pixels will propagate further through the array at each
         iteration.  After several iterations this could result in there
         being no good data left.  An error is reported if this
         happens.  \texttt{[0.001]}
      }
      \sstsubsection{
         XCENTRE = \_INTEGER (Read)
      }{
         The \textit{x} pixel index of the centre of the PSF within the supplied
         PSF array.  If a null (\texttt{{!}}) value is supplied, the value used is
         the middle pixel (rounded down if there are an even number of
         pixels per line).  \texttt{[!]}
      }
      \sstsubsection{
         YCENTRE = \_INTEGER (Read)
      }{
         The \textit{y} pixel index of the centre of the PSF within the supplied
         PSF array.  If a null (\texttt{{!}}) value is supplied, the value used is
         the middle line (rounded down if there are an even number of
         lines).  \texttt{[!]}
      }
   }
   \sstexamples{
      \sstexamplesubsection{
         lucy m51 star m51\_hires
      }{
         This example deconvolves the array in the NDF called m51,
         putting the resulting array in the NDF called m51\_hires.  The
         PSF is defined by the array in NDF star (the centre of the
         PSF is assumed to be at the central pixel).  The deconvolution
         terminates when a normalised chi-squared value of 1.0 is
         reached.
      }
      \sstexamplesubsection{
         lucy m51 star m51\_hires 0.5 niter=60
      }{
         This example performs the same function as the previous
         example, except that the deconvolution terminates when a
         normalised chi-squared value of 0.5 is reached, giving higher
         apparent resolution at the expense of extra spurious
         noise-based structure.  The maximum number of iterations is
         increased to 60 to give the algorithm greater opportunity to
         reach the reduced chi-squared value.
      }
      \sstexamplesubsection{
         lucy m51 star m51\_hires2 0.1 start=m51\_hires
      }{
         This example continues the deconvolution started by the
         previous example in order to achieve a normalised chi-squared
         of 0.1.  The output array from the previous example is used to
         initialise the restored array.
      }
   }
   \sstnotes{
      \sstitemlist{

         \sstitem
         The convolutions required by the R-L algorithm are performed by
         the multiplication of Fourier transforms.  The supplied input
         array is extended by a margin along each edge to avoid problems
         of wrap-around between opposite edges of the array.  The width of
         this margin is about equal to the width of the significant part
         of the PSF (as determined by Parameter THRESH).  The application
         displays the width of these margins.  The margins are filled by
         replicating the edge pixels from the supplied input NDFs.

         \sstitem
         The R-L algorithm works best for arrays which have zero
         background.  Non-zero backgrounds cause dark rings to appear
         around bright, compact sources.  To avoid this a background array
         should be created before running LUCY and assigned to the
         Parameter BACK.  The \htmlref{SEGMENT}{SEGMENT} and
         \htmlref{SURFIT}{SURFIT} applications within
         \KAPPA\ can be used to create such a background array.
      }
   }
   \sstdiytopic{
      Related Applications
   }{
KAPPA: \htmlref{FOURIER}{FOURIER},
\htmlref{MEM2D}{MEM2D},
\htmlref{WIENER}{WIENER}.
   }
   \sstimplementationstatus{
      \sstitemlist{

         \sstitem
         This routine correctly processes the \htmlref{AXIS}{apndf:axis}, DATA, \htmlref{QUALITY}{apndf:quality},
         \htmlref{VARIANCE}{apndf:variance}, \htmlref{LABEL}{apndf:label}, \htmlref{TITLE}{apndf:title}, \htmlref{UNITS}{apndf:units}, \htmlref{WCS}{apndf:wcs}, and \htmlref{HISTORY}{apndf:history}~ components of the
         input NDF and propagates all \htmlref{extensions}{apndf:extensions}.

         \sstitem
         Processing of \htmlref{bad pixels}{se:masking} and automatic \htmlref{quality masking}{se:qualitymask} are
         supported.

         \sstitem
         All \htmlref{non-complex numeric data types}{ap:HDStypes} can be handled.  Arithmetic
         is performed using single-precision floating point.
      }
   }
}
\sstroutine{
   LUTABLE
}{
   Manipulates an graphics device colour table
}{
   \sstdescription{
      This application allows manipulation of the
      \htmlref{colour table}{se:coltab}~ of an
      graphics device provided some data are, according to the
      graphics database, already displayed upon the device.  A
      two-dimensional data array, stored in the input \NDFref{NDF} structure, may
      be nominated to assist in defining the colour table via an histogram
      equalisation.  There are two stages to the running of this
      application.
      \begin{enumerate}
      \item The way in which the lookup table (LUT) is to distributed
      amongst the pens (colour indices) of the colour table is
      required.  Some pens are reserved by \KAPPA\ as a
      \htmlref{palette}{se:palette}, particularly
      for annotation.  This application only modifies the unreserved
      portion of the colour table.

      \item The lookup table is now chosen from a programmed selection or
      read from an NDF.
      \end{enumerate}

      The two stages may be repeated cyclically if desired.  To exit the
      loop give the null response, \texttt{{!}}, to a prompt.  Looping will not
      occur if the lookup table and the distribution method are supplied
      on the command line.
   }
   \sstusage{
      lutable mapping coltab lut [device] ndf percentiles shade
   }
   \sstparameters{
      \sstsubsection{
         DEVICE = \htmlref{DEVICE}{se:selgradev} (Read)
      }{
         Name of the graphics device to be used.  \texttt{[}Current graphics device\texttt{{]}}
      }
      \sstsubsection{
         COLTAB = \htmlref{LITERAL}{se:parmenu} (Read)
      }{
         The lookup table required.  The options are listed below.
         \begin{description}
            \item \texttt{"Negative"} --- This is negative grey scale with black assigned
                         to the highest pen, and white assigned to the
                         lowest available pen.
            \item \texttt{"Colour"} --- This consists of eighteen standard colour
                         blocks.
            \item \texttt{"Grey"} --- This a standard grey scale.
            \item \texttt{"External"} --- Obtain a lookup table stored in an NDF's data
                         array.  If the table cannot be found in the
                         specified NDF or if it is not a LUT then a
                         grey scale is used.
         \end{description}
      }
      \sstsubsection{
         FULL = \_LOGICAL (Read)
      }{
         If \texttt{TRUE} the whole colour-table for the device is stored
         including the reserved pens.  This is necessary to save a
         colour table written by another package that does not reserve
         colour indices.  For colour tables produced by
         \KAPPA\ this should be \texttt{FALSE}.  \texttt{[FALSE]}
      }
      \sstsubsection{
         LUT = NDF (Read)
      }{
         Name of the NDF containing the lookup table as its data
         array.  The LUT must be two-dimensional, the first dimension
         being 3, and the second being arbitrary.  The method used to
         compress or expand the colour table if the second dimension is
         different from the number of unreserved colour indices is
         controlled by Parameter NN.  Also the LUT's values must lie in
         the range 0.0--1.0.
      }
      \sstsubsection{
         MAPPING = LITERAL (Read)
      }{
         The way in which the colours are to be distributed among
         the pens.  If NINTS is the number of unreserved colour indices
         the mapping options are described below.
         \begin{description}
            \item \texttt{"Histogram"} --- The colours are fitted to the pens using
                           histogram equalisation of an NDF, given by
                           Parameter IN, so that the colours
                           approximately have an even distribution.  In
                           other words each pen is used approximately
                           an equal number of times to display the
                           two-dimensional NDF array.  There must be an existing
                           graphics deviceed.  This is determined by
                           looking for a DATA picture in the database.
                           This is not foolproof as this may be a line
                           plot rather an image.
            \item \texttt{"Linear"} --- The colours are fitted directly to the pens.
            \item \texttt{"Logarithmic"} --- The colours are fitted
                           logarithmically to the pens, with colour 1
                           given to the first available pen and colour
                           NINTS given to the last pen.
         \end{description}
      }
      \sstsubsection{
         NDF = NDF (Read)
      }{
         The input NDF structure containing the two-dimensional data array to be
         used for the histogram-equalisation mapping of the pens.  The
         the data object referenced by the last DATA picture in the
         graphics database is reported.  This assumes that the
         displayed data picture was derived from the nominated NDF data
         array.
      }
      \sstsubsection{
         NN = \_LOGICAL (Read)
      }{
         If \texttt{TRUE} the input lookup table is mapped to the colour table by
         using the nearest-neighbour method.  This preserves sharp
         edges and is better for lookup tables with blocks of colour.
         If NN is \texttt{FALSE} linear interpolation is used, and this is
         suitable for smoothly varying colour tables.  \texttt{[FALSE]}
      }
      \sstsubsection{
         PERCENTILES( 2 ) = \_REAL (Read)
      }{
         The percentiles that define the range of the histogram to be
         equalised.  For example, \texttt{[25,75]} would scale between the
         quartile values.  It is advisable not to choose the limits
         less than 3 per cent and greater than 97.  The percentiles are
         only required for histogram mapping.  All values in the NDF's
         data array less than the value corresponding to the lower
         percentile will have the colour of the first unreserved pen.
         All values greater than the value corresponding to the upper
         percentile will have the colour of the last unreserved pen.
      }
      \sstsubsection{
         SHADE = \_REAL (Read)
      }{
         The type of shading.  This only required for the histogram
         mapping.  A value of \texttt{$-$1} emphasises low values;
         \texttt{$+$1} emphasises
         high values; \texttt{0} is neutral, all values have equal weight.  The
         shade must lie in the range $-$1 to $+$1.
      }
   }
   \sstexamples{
      \sstexamplesubsection{
         lutable lo co
      }{
         Changes the colour table on the \htmlref{current graphics device}{se:devglobal}
         to a series of coloured blocks whose size increase
         logarithmically with the table index number.
      }
      \sstexamplesubsection{
         lutable li ex rococo
      }{
         This maps the lookup table stored in the NDF called rococo
         linearly to the colour table on the current graphics device
         device.
      }
      \sstexamplesubsection{
         lutable li ex rococo full
      }{
         This maps the lookup table stored in the NDF called rococo
         linearly to the full colour table on the current graphics device
         device, \emph{i.e.} ignoring the reserved pens.
      }
      \sstexamplesubsection{
         lutable hi gr ndf=nebula shade=0 percentiles=[5,90]
      }{
         This maps the grey-scale lookup table via histogram
         equalisation between the 5 and 90 percentiles of an NDF called
         nebula to the colour table on the current graphics device
         device.  There is no bias or shading to white or black.
      }
   }
   \sstnotes{
      \sstitemlist{

         \sstitem
         The effects of this command will only be immediately apparent
         when run on X windows which have 256 colours (or other similar
         pseudocolour devices).  On other devices (for instance, X windows
         with more than 256 colours) the effects will only become
         apparent when subsequent graphics applications are run.
      }
   }
   \sstdiytopic{
      Related Applications
   }{
KAPPA: \htmlref{LUTEDIT}{LUTEDIT},
\htmlref{LUTREAD}{LUTREAD},
\htmlref{LUTSAVE}{LUTSAVE},
\htmlref{LUTVIEW}{LUTVIEW};
\xref{FIGARO}{sun86}{}: \xref{COLOUR}{sun86}{COLOUR}.
   }
   \sstimplementationstatus{
      \sstitemlist{

         \sstitem
         Processing of \htmlref{bad pixels}{se:masking} and automatic \htmlref{quality masking}{se:qualitymask} are
         supported for the image NDF

         \sstitem
         All \htmlref{non-complex numeric data types}{ap:HDStypes} can be handled.  Processing
         is performed using single- or double-precision floating point,
         as appropriate.
      }
   }
}
\sstroutine{
   LUTBGYRW
}{
   Loads the {\it BGYRW} lookup table
}{
   \sstdescription{
      This procedure loads the {\it BGYRW\/}~ \htmlref{lookup table}{se:lookuptables}~ with linear scaling
      into the \htmlref{current graphics device}{se:devglobal}.  It is a continuous LUT
      starting with blue, followed by green, yellow, red and a splash of
      white.
   }
   \sstusage{
      lutbgyrw
   }
   \sstparameters{
      \sstsubsection{
         DEVICE = \htmlref{DEVICE}{se:selgradev} (Read)
      }{
         Name of the graphics device whose colour table is to be changed.
         \texttt{[}Current graphics device\texttt{{]}}
      }
   }
   \sstnotes{
      This is a procedure that calls LUTABLE.  Therefore, the
      parameter cannot be specified on the command line.  You will
      only be prompted for the DEVICE parameter if the current
      graphics device is not suitable or not available.
   }
}
\sstroutine{
   LUTCOL
}{
   Loads the standard colour lookup table
}{
   \sstdescription{
      Procedure for loading the standard colour \htmlref{lookup table}{se:lookuptables}~ into
      the \htmlref{current graphics device}{se:devglobal}~ with linear scaling.
   }
   \sstusage{
      lutcol
   }
   \sstparameters{
      \sstsubsection{
         DEVICE = \htmlref{DEVICE}{se:selgradev} (Read)
      }{
         Name of the graphics device whose colour table is to be changed.
         \texttt{[}Current graphics device\texttt{{]}}
      }
   }
   \sstnotes{
      This is a procedure that calls LUTABLE.  Therefore, the parameter
      cannot be specified on the command line.  You will only be
      prompted for the DEVICE parameter if the current graphics device
      is not suitable or not available.
   }
}
\sstroutine{
   LUTCOLD
}{
   Loads the {\it cold} lookup table
}{
   \sstdescription{
      This procedure loads the {\it cold\/}~ \htmlref{lookup table}{se:lookuptables}~ with linear scaling
      into the \htmlref{current graphics device}{se:devglobal}.  It is a continuous LUT
      going from black to white, passing through cold shades of pale blue
      and grey.
   }
   \sstusage{
      lutcold
   }
   \sstparameters{
      \sstsubsection{
         DEVICE = \htmlref{DEVICE}{se:selgradev} (Read)
      }{
         Name of the graphics device whose colour table is to be changed.
         \texttt{[}Current graphics device\texttt{{]}}
      }
   }
   \sstnotes{
      This is a procedure that calls LUTABLE.  Therefore, the
      parameter cannot be specified on the command line.  You will
      only be prompted for the DEVICE parameter if the current
      graphics device is not suitable or not available.
   }
}
\sstroutine{
   LUTCONT
}{
   Loads a lookup table to give the display the appearance of a
   contour plot
}{
   \sstdescription{
      This procedure loads a \htmlref{lookup table}{se:lookuptables}~ that gives a contour-plot
      appearance into the \htmlref{current graphics device}{se:devglobal}.  The lookup table
      is mainly black with a set of white stripes and it is loaded with
      linear scaling.
   }
   \sstusage{
      lutcont
   }
   \sstparameters{
      \sstsubsection{
         DEVICE = \htmlref{DEVICE}{se:selgradev} (Read)
      }{
         Name of the graphics device whose colour table is to be changed.
         \texttt{[}Current graphics device\texttt{{]}}
      }
   }
   \sstnotes{
      This is a procedure that calls LUTABLE.  Therefore, the parameter
      cannot be specified on the command line.  You will only be
      prompted for the DEVICE parameter if the current graphics device
      is not suitable or not available.
   }
}
\sstroutine{
   LUTEDIT
}{
   Creates or edits an graphics device colour lookup table
}{
   \sstdescription{
      This application allows a \htmlref{lookup table}{se:lookuptables}~ to be created or edited
      interactively.  The process is controlled through a graphical user
      interface which presents curves of intensity against pen number, and
      allows the user to change them in various ways.  A specified image
      is displayed as part of the interface in order to see the effects of
      the changes.  A histogram of pen values is also included.  The colour
      of each pen can be displayed either as red, green and blue intensity,
      or as hue, saturation and value.  More information on the use of the
      GUI is available through the Help menu within the GUI.
   }
   \sstusage{
      lutedit lut image device
   }
   \sstparameters{
      \sstsubsection{
         DEVICE = \htmlref{DEVICE}{se:selgradev} (Read)
      }{
         The name of an graphics device.  If a null (\texttt{{!}}) value is
         supplied for Parameter LUT, then the current LUT associated with
         the specified device will be loaded into the editor initially.
         On exit, the final contents of the editor (if saved) are
         established as the current LUT for the specified device.
         \texttt{[}Current graphics device\texttt{{]}}
      }
      \sstsubsection{
         LUT = NDF (Read)
      }{
         Name of an exiting colour table to be edited.  This should be an
         \NDFref{NDF} containing an array of red, green and blue intensities.  The
         NDF must be two-dimensional, the first dimension being 3, and the
         second being arbitrary.  The method used to compress or expand the
         colour table if the second dimension is different from the number
         of unreserved colour indices is controlled by the \texttt{"Interpolation"}
         option in the GUI.  If LUT is null (\texttt{{!}}) the current \KAPPA\ colour
         table for the xwindows graphics display is used.  \texttt{[!]}
      }
      \sstsubsection{
         IMAGE = NDF (Read)
      }{
         Input NDF data structure containing the image to be displayed
         to show the effect of the created colour table.  If a null value
         is supplied a default image is used.
      }
   }
   \sstdiytopic{
      Related Applications
   }{
KAPPA: \htmlref{LUTABLE}{LUTABLE},
\htmlref{LUTREAD}{LUTREAD},
\htmlref{LUTSAVE}{LUTSAVE},
\htmlref{LUTVIEW}{LUTVIEW},
\htmlref{PALREAD}{PALREAD},
\htmlref{PALSAVE}{PALSAVE};
\xref{FIGARO}{sun86}{}: \xref{COLOUR}{sun86}{COLOUR}.
   }
}
\sstroutine{
   LUTFC
}{
   Loads the standard false-colour lookup table
}{
   \sstdescription{
      This procedure loads the standard false-colour \htmlref{lookup table}{se:lookuptables}~ with
      linear scaling into the \htmlref{current graphics device}{se:devglobal}.
   }
   \sstusage{
      lutfc
   }
   \sstparameters{
      \sstsubsection{
         DEVICE = \htmlref{DEVICE}{se:selgradev} (Read)
      }{
         Name of the graphics device whose colour table is to be changed.
         \texttt{[}Current graphics device\texttt{{]}}
      }
   }
   \sstnotes{
      This is a procedure that calls LUTABLE.  Therefore, the parameter
      cannot be specified on the command line.  You will only be
      prompted for the DEVICE parameter if the current graphics device
      is not suitable or not available.
   }
}

\sstroutine{
   LUTGREY
}{
   Loads the standard grey-scale lookup table
}{
   \sstdescription{
      Procedure for loading the standard grey-scale \htmlref{lookup table}{se:lookuptables}~ into
      the \htmlref{current graphics device}{se:devglobal}~ with linear scaling.
   }
   \sstusage{
      lutgrey
   }
   \sstparameters{
      \sstsubsection{
         DEVICE = \htmlref{DEVICE}{se:selgradev} (Read)
      }{
         Name of the graphics device whose colour table is to be changed.
         \texttt{[}Current graphics device\texttt{{]}}
      }
   }
   \sstnotes{
      This is a procedure that calls LUTABLE.  Therefore, the parameter
      cannot be specified on the command line.  You will only be
      prompted for the DEVICE parameter if the current graphics device
      is not suitable or not available.
   }
}
\sstroutine{
   LUTHEAT
}{
   Loads the {\it heat} lookup table
}{
   \sstdescription{
      This procedure loads the {\it heat}~ \htmlref{lookup table}{se:lookuptables}~ with linear scaling
      into the \htmlref{current graphics device}{se:devglobal}.
   }
   \sstusage{
      lutheat
   }
   \sstparameters{
      \sstsubsection{
         DEVICE = \htmlref{DEVICE}{se:selgradev} (Read)
      }{
         Name of the graphics device whose colour table is to be changed.
         \texttt{[}Current graphics device\texttt{{]}}
      }
   }
   \sstnotes{
      This is a procedure that calls LUTABLE.  Therefore, the parameter
      cannot be specified on the command line.  You will only be
      prompted for the DEVICE parameter if the current graphics device
      is not suitable or not available.
   }

}
\sstroutine{
   LUTIKON
}{
   Loads the default {\it Ikon} lookup table
}{
   \sstdescription{
      This procedure loads the default {\it Ikon\/}~ \htmlref{lookup table}{se:lookuptables}~ with linear
      scaling into the \htmlref{current graphics device}{se:devglobal}.
   }
   \sstusage{
      lutikon
   }
   \sstparameters{
      \sstsubsection{
         DEVICE = \htmlref{DEVICE}{se:selgradev} (Read)
      }{
         Name of the graphics device whose colour table is to be changed.
         \texttt{[}Current graphics device\texttt{{]}}
      }
   }
   \sstnotes{
      \sstitemlist{

         \sstitem
         This is a procedure that calls LUTABLE.  Therefore, the parameter
         cannot be specified on the command line.  You will only be
         prompted for the DEVICE parameter if the current graphics device
         is not suitable or not available.

         \sstitem
         The device need not be an Ikon.
      }
   }
}
\sstroutine{
   LUTNEG
}{
   Loads the standard negative grey-scale lookup table
}{
   \sstdescription{
      Procedure for loading the standard grey-scale \htmlref{lookup table}{se:lookuptables}~ into
      the \htmlref{current graphics device}{se:devglobal}~ with negative linear scaling.
   }
   \sstusage{
      lutneg
   }
   \sstparameters{
      \sstsubsection{
         DEVICE = \htmlref{DEVICE}{se:selgradev} (Read)
      }{
         Name of the graphics device whose colour table is to be changed.
         \texttt{[}Current graphics device\texttt{{]}}
      }
   }
   \sstnotes{
      This is a procedure that calls LUTABLE.  Therefore, the parameter
      cannot be specified on the command line.  You will only be
      prompted for the DEVICE parameter if the current graphics device
      is not suitable or not available.
   }
}
\sstroutine{
   LUTRAMPS
}{
   Loads the coloured-ramps lookup table
}{
   \sstdescription{
      This procedure loads the coloured-ramps \htmlref{lookup table}{se:lookuptables}~ with linear
      scaling into the \htmlref{current graphics device}{se:devglobal}.
   }
   \sstusage{
      lutramps
   }
   \sstparameters{
      \sstsubsection{
         DEVICE = \htmlref{DEVICE}{se:selgradev} (Read)
      }{
         Name of the graphics device whose colour table is to be changed.
         \texttt{[}Current graphics device\texttt{{]}}
      }
   }
   \sstnotes{
      This is a procedure that calls LUTABLE.  Therefore, the parameter
      cannot be specified on the command line.  You will only be
      prompted for the DEVICE parameter if the current graphics device
      is not suitable or not available.
   }
}
\sstroutine{
   LUTREAD
}{
   Loads an graphics device lookup table from an NDF
}{
   \sstdescription{
      This application reads a \htmlref{lookup table}{se:lookuptables}~ stored in an \NDFref{NDF} with
      the standard format, and loads it into the \htmlref{current graphics device}{se:devglobal}.
      device.
   }
   \sstusage{
      lutread lut
   }
   \sstarguments{
      \sstsubsection{
        LUT = LITERAL (Read)
      }{
           The file containing the lookup table.  It is passed to the
           Parameter LUT but not validated.
      }
   }
   \sstparameters{
      \sstsubsection{
         DEVICE = \htmlref{DEVICE}{se:selgradev} (Read)
      }{
         Name of the graphics device whose colour table is to be changed.
         \texttt{[}Current graphics device\texttt{{]}}
      }
      \sstsubsection{
         LUT = NDF (Read)
      }{
         Name of the NDF containing the lookup table as its data
         array.  The LUT must be two-dimensional, the first dimension
         being 3, and the second being arbitrary.  Linear interpolation
         is used to compress or expand the colour table if the second
         dimension is different from the number of unreserved colour
         indices.  Also the LUT's values must lie in the range 0.0--1.0.
      }
   }
   \sstnotes{
      This is a procedure that calls LUTABLE.  Therefore, the parameters
      cannot be specified on the command line.  You will only be
      prompted for the parameters if the device is not suitable or not
      available, and/or the lookup table file could not be accessed.
   }
}
\sstroutine{
   LUTSAVE
}{
   Saves the current colour table of an graphics device in an
   NDF
}{
   \sstdescription{
      This routine saves the colour table of a nominated graphics device to
      an \NDFref{NDF} LUT file and/or a text file.
   }
   \sstusage{
      lutsave lut [device]
   }
   \sstparameters{
      \sstsubsection{
         DEVICE = \htmlref{DEVICE}{se:selgradev} (Read)
      }{
         The name of the graphics device whose colour table is to
         be saved.  \texttt{[}Current graphics device\texttt{{]}}
      }
      \sstsubsection{
         FULL = \_LOGICAL (Read)
      }{
         If \texttt{TRUE} the whole colour-table for the device is stored
         including the reserved pens.  This is necessary to save a
         colour table written by another package that does not reserve
         colour indices.  For colour tables produced by \KAPPA\ this
         should be \texttt{FALSE}.  \texttt{[FALSE]}
      }
      \sstsubsection{
         LOGFILE = FILENAME (Write)
      }{
         The name of a text file to receive the formatted values in the
         colour table.  Each line i the file contains the red, green and
         blue intensities for a single pen, separated by spaces.  A null
         string (\texttt{{!}}) means that no file is created.  \texttt{[!]}
      }
      \sstsubsection{
         LUT = NDF (Write)
      }{
         The output NDF into which the colour table is to be stored.
         Its second dimension equals the number of colour-table
         entries that are stored.  This will be fewer than the
         total number of colour indices on the device if FULL is \texttt{FALSE}.
         No NDF is created if a null (\texttt{{!}}) value is given.
      }
      \sstsubsection{
         TITLE = LITERAL (Read)
      }{
         The title for the output NDF.  \texttt{["KAPPA - Lutsave"]}
      }
   }
   \sstexamples{
      \sstexamplesubsection{
         lutsave pizza
      }{
         This saves the current colour table on the current
         graphics device to an NDF called pizza.
      }
      \sstexamplesubsection{
         lutsave redshift full
      }{
         This saves in full the current colour table on the current
         graphics device to an NDF called redshift.
      }
   }
   \sstdiytopic{
      Related Applications
   }{
KAPPA: \htmlref{LUTEDIT}{LUTEDIT},
\htmlref{LUTABLE}{LUTABLE},
\htmlref{LUTREAD}{LUTREAD}.
   }
}

\sstroutine{
   LUTSPEC
}{
   Loads a spectrum-like lookup table
}{
   \sstdescription{
      This procedure loads an optical-spectrum-like \htmlref{lookup table}{se:lookuptables}~ with linear
      scaling into the \htmlref{current graphics device}{se:devglobal}.
   }
   \sstusage{
      lutspec
   }
   \sstparameters{
      \sstsubsection{
         DEVICE = \htmlref{DEVICE}{se:selgradev} (Read)
      }{
         Name of the graphics device whose colour table is to be changed.
         \texttt{[}Current graphics device\texttt{{]}}
      }
   }
   \sstnotes{
      This is a procedure that calls LUTABLE.  Therefore, the parameter
      cannot be specified on the command line.  You will only be
      prompted for the DEVICE parameter if the current graphics device
      is not suitable or not available.
   }
}

\sstroutine{
   LUTVIEW
}{
   Draws a colour-table key
}{
   \sstdescription{
      This application displays a key to the current colour table on the
      specified graphics device using the whole of the current colour
      table (excluding the low 16 pens which are reserved for axis
      annotation, \emph{etc.}).  The key can either be a simple rectangular block
      of colour which ramps through the colour table, a histogram-style
      key in which the width of the block reflects the number of pixels
      allocated to each colour index, or a set of RGB intensity curves.
      The choice is made using the STYLE parameter.

      By default, numerical data values are displayed along the long edge of
      the key.  The values corresponding to the maximum and minimum colour
      index are supplied using Parameters HIGH and LOW.  Intermediate colour
      indices are labelled with values which are linearly interpolated
      between these two extreme values.

      The rectangular area in which the key (plus annotations) is drawn
      may be specified either using a graphics cursor, or by specifying the
      co-ordinates of two corners using Parameters LBOUND and UBOUND.
      Additionally, there is an option to make the key fill the current
      picture.  See Parameter MODE.  The key may be constrained to the
      current picture using Parameter CURPIC.

      The appearance of the annotation my be controlled in detail using
      the STYLE parameter.
   }
   \sstusage{
      lutview [mode] [low] [high] [curpic] [device] lbound=? ubound=?
   }
   \sstparameters{
      \sstsubsection{
         COMP = \htmlref{LITERAL}{se:parmenu} (Read)
      }{
         The component (within the NDF given by Parameter NDF) which is
         currently displayed.  It may be \texttt{"Data"}, \texttt{"Quality"}, \texttt{"Variance"}, or
         \texttt{"Error"} (where \texttt{"Error"} is an alternative to \texttt{"Variance"} and causes
         the square root of the variance values to be used).  If \texttt{"Quality"} is
         specified, then the quality values are treated as numerical values (in
         the range 0 to 255).  The dynamic default is obtained from global
         Parameter COMP which is set by applications such as DISPLAY.  \texttt{[]}
      }
      \sstsubsection{
         CURPIC = \_LOGICAL (Read)
      }{
         If CURPIC is \texttt{TRUE}, the colour table key is to lie within the
         current picture, otherwise the new picture can lie anywhere
         within the BASE picture.  This parameter ignored if the
         current-picture mode is selected.  \texttt{[FALSE]}
      }
      \sstsubsection{
         DEVICE = \htmlref{DEVICE}{se:selgradev} (Read)
      }{
         The graphics device on which the colour table is to be
         drawn.  \texttt{[}Current graphics device\texttt{{]}}
      }
      \sstsubsection{
         FRAME = LITERAL (Read)
      }{
         Specifies the \htmlref{co-ordinate Frame}{se:domains}~  of the positions supplied using
         Parameters LBOUND and UBOUND.  The following Frames will always
         be available.

         \ssthitemlist{

            \sstitem
            \texttt{"GRAPHICS"} --- gives positions in millimetres from
            the bottom-left corner of the plotting surface.

            \sstitem
            \texttt{"BASEPIC"} --- gives positions in a normalised system in which the
            bottom-left corner of the plotting surface is (0,~0) and the
            shortest dimension of the plotting surface has length 1.0.  The
            scales on the two axes are equal.

            \sstitem
            \texttt{"CURPIC"} --- gives positions in a normalised system in which the
            bottom-left corner of the underlying DATA picture is (0,~0) and
            the shortest dimension of the picture has length 1.0.  The scales
            on the two axes are equal.

            \sstitem
            \texttt{"NDC"} --- gives positions in a normalised system in which the
            bottom-left corner of the plotting surface is (0,~0) and the
            top-right corner is (1,~1).

            \sstitem
            \texttt{"CURNDC"} --- gives positions in a normalised system in which the
            bottom-left corner of the current picture is (0,~0) and the
            top-right corner is (1,~1).

         }
         There may be additional Frames available, describing previously
         displayed data.  If a null value is supplied, the
         \htmlref{current Frame}{se:curframe}
         associated with the displayed data (if any) is used.  This parameter
         is only accessed if Parameter MODE is set to \texttt{"XY"}.  \texttt{["BASEPIC"]}
      }
      \sstsubsection{
         HIGH = \_REAL (Read)
      }{
         The value corresponding to the maximum colour index.  It is
         used to calculate the annotation scale for the key.  If it
         is null (\texttt{{!}}) the maximum colour index is used, and histogram
         style keys are not available.
         \texttt{[}Current display linear-scaling maximum\texttt{{]}}
      }
      \sstsubsection{
         LBOUND = LITERAL (Read)
      }{
         Co-ordinates of the lower-left corner of the rectangular region
         containing the colour ramp and annotation, in the co-ordinate
         Frame specified by Parameter FRAME (supplying a colon \texttt{":"} will
         display details of the selected co-ordinate Frame).  The position
         should be supplied as a list of
         \xref{formatted axis values}{sun210}{AST_UNFORMAT} separated
         by spaces or commas.  A null (\texttt{{!}}) value causes the lower-left corner
         of the BASE or (if CURPIC is \texttt{TRUE}) current picture to be used.
      }
      \sstsubsection{
         LOW = \_REAL (Read)
      }{
         The value corresponding to the minimum colour index.  It is
         used to calculate the annotation scale for the key.  If it
         is null (\texttt{{!}}) the minimum colour index is used, and histogram
         style keys are not available.
         \texttt{}[Current display linear-scaling minimum\texttt{{]}}
      }
      \sstsubsection{
         LUT = NDF (Read)
      }{
         Name of the NDF containing a lookup table as its data array;
         the lookup table is written to the graphics device's colour
         table.  The purpose of this parameter is to provide a means of
         controlling the appearance of the image on certain devices,
         such as colour printers, that do not have a dynamic colour
         table, \emph{i.e.} the colour table is reset when the device is
         opened.  If used with dynamic devices, such as windows or
         Ikons, the new colour table remains after this application has
         completed.  A null, !, means that the existing colour table
         will be used.

         The LUT must be two-dimensional, the first dimension
         being 3, and the second being arbitrary.  The method used to
         compress or expand the colour table if the second dimension is
         different from the number of unreserved colour indices is
         controlled by Parameter NN.  Also the LUT's values must lie in
         the range 0.0--1.0.  \texttt{[!]}
      }
      \sstsubsection{
         MODE = \htmlref{LITERAL}{se:parmenu} (Read)
      }{
         Method for defining the position, size and shape of the
         rectangular region containing the colour ramp and annotation.
         The options are:

         \ssthitemlist{

            \sstitem
            \texttt{"Cursor"} ---  The graphics cursor is used to supply two
            diametrically opposite corners or the region.

            \sstitem
            \texttt{"XY"} --- The Parameters LBOUND and UBOUND are used to get the
            limits.

            \sstitem
            \texttt{"Picture"} --- The whole of the current picture is used.
            Additional positioning options are available by using other
            \KAPPA\ applications to create new pictures and then specifying
            the picture mode.

         }
         \texttt{["Cursor"]}
      }
      \sstsubsection{
         NDF = NDF (Read)
      }{
         The \NDFref{NDF} defining the image values to be used if a
         histogram-style key is requested.  This should normally be the
         NDF currently displayed in the most recently created DATA picture.
         If a value is supplied on the command line for this parameter it
         will be used.  Otherwise, the NDF to used is found by
         interrogating the graphics database (which contains references to
         displayed images).  If no reference NDF can be obtained from the
         graphics database, the user will be prompted for a value.
      }
      \sstsubsection{
         NN = \_LOGICAL (Read)
      }{
         If NN is \texttt{TRUE}, the input lookup table is mapped to the colour
         table by using the nearest-neighbour method.  This preserves
         sharp edges and is better for lookup tables with blocks of
         colour.  If NN is \texttt{FALSE}, linear interpolation is used, and
         this is suitable for smoothly varying colour tables.  NN is
         ignored unless LUT is not null.  \texttt{[FALSE]}
      }
      \sstsubsection{
         STYLE = \htmlref{GROUP}{se:groups} (Read)
      }{
         A group of attribute settings describing the plotting style to use
         for the annotation.

         A comma-separated list of strings should be given in which each
         string is either an attribute setting, or the name of a text
         file preceded by an up-arrow character \texttt{"$\wedge$"}.  Such text files
         should contain further comma-separated lists which will be
         read and interpreted in the same manner.  Attribute settings
         are applied in the order in which they occur within the list,
         with later settings overriding any earlier settings given for
         the same attribute.

         Each individual attribute setting should be of the form:

            $<$name$>$=$<$value$>$

         where $<$name$>$ is the name of a plotting attribute, and $<$value$>$
         is the value to assign to the attribute.  Default values will be
         used for any unspecified attributes.  All attributes will be
         defaulted if a null value (\texttt{{!}})---the initial default---is supplied.
         To apply changes of style to only the current invocation, begin these
         attributes with a plus sign.  A mixture of persistent and temporary
         style changes is achieved by listing all the persistent attributes
         followed by a plus sign then the list of temporary attributes.

         See \slhyperref{Plotting Attributes}{Section~}{}{ap:plotting_attr}
         for a description of the available attributes.  Any unrecognised
         attributes are ignored (no error is reported).

         Axis 1 is always the \emph{data value} axis, whether it is displayed
         horizontally or vertically.  So for instance, to set the label
         for the data value axis, assign a value to \att{Label(1)} in the
         supplied style.

         To get a ramp key (the default), specify \texttt{"form=ramp"}.  To
         get a histogram key (a coloured histogram of pen indices),
         specify \texttt{"form=histogram"}.  To get a graph key (three curves of
         RGB intensities), specify \texttt{"form=graph"}.  If a histogram key
         is produced, the population axis can be either logarithmic or
         linear.  To get a logarithmic population axis, specify \texttt{"logpop=1"}.
         To get a linear population axis, specify \texttt{"logpop=0"} (the default).
         To annotate the long axis with pen numbers instead of pixel value,
         specify \texttt{"pennums=1"} (the default, \texttt{"pennums=0"}, shows pixel
         values).  \texttt{[}current value\texttt{{]}}
      }
      \sstsubsection{
         UBOUND = LITERAL (Read)
      }{
         Co-ordinates of the upper-right corner of the rectangular region
         containing the colour ramp and annotation, in the co-ordinate
         Frame specified by Parameter FRAME (supplying a colon \texttt{":"} will
         display details of the selected co-ordinate Frame).  The position
         should be supplied as a list of formatted axis values separated
         by spaces or commas.  A null (\texttt{{!}}) value causes the lower-left corner
         of the BASE or (if CURPIC is \texttt{TRUE}) the current picture to be used.
      }
   }
   \sstexamples{
      \sstexamplesubsection{
         lutview
      }{
         Draws an annotated colour table at a position selected via
         the cursor on the \htmlref{current graphics device}{se:devglobal}.
      }
      \sstexamplesubsection{
         lutview style="form=hist,logpop=1"
      }{
         As above, but the key has the form of a coloured histogram of
         the pen numbers in the most recently displayed image.  The second
         axis displays the logarithm (base 10) of the bin population.
      }
      \sstexamplesubsection{
         lutview style="form=graph,pennums=1"
      }{
         The key is drawn as a set of three (or one if a monochrome
         colour table is in use) curves indicating the red, green and
         blue intensity for each pen.  The first axis is annotated with
         pen numbers instead of data values.
      }
      \sstexamplesubsection{
         lutview style="edge(1)=right,label(1)=Data value in m31"
      }{
         As above, but the data values are labelled on the right edge of
         the box, and the values are labelled with the string \texttt{"Data value
         in m31"}.
      }
      \sstexamplesubsection{
         lutview style="textlab(1)=0,width(border)=3,colour(border)=white"
      }{
         No textual label is drawn for the data values, and a thicker than
         usual white box is drawn around the colour ramp.
      }
      \sstexamplesubsection{
         lutview style="textlab(1)=0,numlab(1)=0,majticklen(1)=0"
      }{
         Only the border is drawn around the colour ramp.
      }
      \sstexamplesubsection{
         lutview style="textlab(1)=0,numlab(1)=0,majticklen(1)=0,border=0"
      }{
         No annotation at all is drawn.
      }
      \sstexamplesubsection{
         lutview p
      }{
         Draws a colour table that fills the current picture on the
         current graphics device.
      }
      \sstexamplesubsection{
         lutview curpic
      }{
         Draws a colour table within the current picture positioned
         via the cursor.
      }
      \sstexamplesubsection{
         lutview xy lut=my\_lut device=ps\_p lbound="0.92,0.2" ubound="0.98,0.8"
      }{
         Draws the colour table in the NDF called my\_lut with an
         outline within the BASE picture on the device ps\_p, defined
         by the \textit{x}-\textit{y} bounds (0.92,~0.2) and (0.98,~0.8).  In other words
         the plot is to the right-hand side with increasing colour
         index with increasing \textit{y} position.
      }
   }
   \sstdiytopic{
      Related Applications
   }{
KAPPA: \htmlref{DISPLAY}{DISPLAY},
\htmlref{LUTABLE}{LUTABLE},
\htmlref{LUTEDIT }{LUTEDIT };
\xref{FIGARO}{sun86}{}: \xref{COLOUR}{sun86}{COLOUR}.
   }
}
\sstroutine{
   LUTWARM
}{
   Loads the {\it warm} lookup table
}{
   \sstdescription{
      This procedure loads the {\it warm\/}~ \htmlref{lookup table}{se:lookuptables}~ with linear scaling
      into the \htmlref{current graphics device}{se:devglobal}.  It is a continuous LUT
      going from black to white, passing through warm shades of yellow
      and brown.
   }
   \sstusage{
      lutwarm
   }
   \sstparameters{
      \sstsubsection{
         DEVICE = \htmlref{DEVICE}{se:selgradev} (Read)
      }{
         Name of the graphics device whose colour table is to be changed.
         \texttt{[}Current graphics device\texttt{{]}}
      }
   }
   \sstnotes{
      This is a procedure that calls LUTABLE.  Therefore, the
      parameter cannot be specified on the command line.  You will
      only be prompted for the DEVICE parameter if the current
      graphics device is not suitable or not available.
   }
}
\sstroutine{
   LUTZEBRA
}{
   Loads a pseudo-contour lookup table
}{
   \sstdescription{
      This procedure loads a pseudo-contour \htmlref{lookup table}{se:lookuptables}~ with linear
      scaling into the \htmlref{current graphics device}{se:devglobal}.  The lookup table
      is mainly black with a set of white stripes.
   }
   \sstusage{
      lutzebra
   }
   \sstparameters{
      \sstsubsection{
         DEVICE = \htmlref{DEVICE}{se:selgradev} (Read)
      }{
         Name of the graphics device whose colour table is to be changed.
         \texttt{[}Current graphics device\texttt{{]}}
      }
   }
   \sstnotes{
      This is a procedure that calls LUTABLE.  Therefore, the parameter
      cannot be specified on the command line.  You will only be
      prompted for the DEVICE parameter if the current graphics device
      is not suitable or not available.
   }
}
\sstroutine{
   MAKESNR
}{
   Creates a signal-to-noise array from an NDF with defined variances
}{
   \sstdescription{
      This application creates a new \NDFref{NDF} from an existing NDF by
      dividing the \htmlref{DATA}{apndf:data}~ component of the input NDF by
      the square root of its \htmlref{VARIANCE}{apndf:variance}~ component.
      The DATA array in the output NDF thus measures the signal to noise
      ratio in the input NDF.

      Anomalously small variance values in the input can cause very
      large spurious values in the output signal to noise array. To avoid
      this, pixels that have a variance value below a given threshold are
      set bad in the output NDF.
   }
   \sstusage{
      makesnr in out [minvar]
   }
   \sstparameters{
      \sstsubsection{
         IN = NDF (Read)
      }{
         The input NDF. An error is reported if this NDF does not have a
         VARIANCE component.
      }
      \sstsubsection{
         MINVAR = \_REAL (Read)
      }{
         The minimum variance value to be used. Input pixels that have
         variance values smaller than this value will be set bad in the
         output. The suggested default is determined by first forming a
         histogram of the logarithm of the input variance values. The highest
         peak is then found in this histogram. The algorithm then moves
         down from this peak towards lower variance values until the
         histogram has dropped to a value equal to the square root of
         the peak value, or a significant minimum is encountered in the
         histogram. The corresponding variance value is used as the
         suggested default.  \texttt{[]}
      }
      \sstsubsection{
         OUT = NDF (Write)
      }{
         The output signal to noise NDF. The VARIANCE component of this NDF
         will be filled with the value 1.0 (except that bad DATA values will
         also have bad VARIANCE values).
      }
   }
   \sstexamples{
      \sstexamplesubsection{
         makesnr m51 m51\_snr
      }{
         This example divides the DATA component of the NDF called m51,
         by the square root of its own VARIANCE component, rejecting
         pixels below the default MINVAR value, and writes the resulting
         signal-to-noise values to an NDF called m51\_hires.
      }
   }
   \sstimplementationstatus{
      \sstitemlist{

         \sstitem
         This routine correctly processes the \htmlref{AXIS}{apndf:axis}, DATA, \htmlref{QUALITY}{apndf:quality},
         \htmlref{LABEL}{apndf:label}, \htmlref{TITLE}{apndf:title}, \htmlref{HISTORY}{apndf:history}, \htmlref{WCS}{apndf:wcs}, and \htmlref{VARIANCE}{apndf:variance} components of an NDF
         data structure and propagates all \htmlref{extensions}{apndf:extensions}.

         \sstitem
         The DATA values in the output NDF represent dimensionless
         ratios, and therefore the \htmlref{UNITS}{apndf:units}~ component
         is not propagated.
      }
   }
}

\sstroutine{
   MAKESURFACE
}{
   Creates a two-dimensional NDF from the coefficients of a polynomial
   surface
}{
   \sstdescription{
      The coefficients describing a two-dimensional polynomial surface
      are read from a SURFACEFIT extension in an \NDFref{NDF} (written by
      FITSURFACE), and are used to create a two-dimensional surface of
      specified size and extent.  The surface is written to a new NDF.

      The size and extent of the surface may be obtained from a template
      NDF or given explicitly.

      Elements in the new NDF outside the defined range of the
      polynomial or spline will be set to bad values.
   }
   \sstusage{
      makesurface in out [like] type=? lbound=? ubound=? xlimit=? ylimit=?
   }
   \sstparameters{
      \sstsubsection{
         IN  = NDF (Read)
      }{
         The NDF containing the SURFACEFIT extension.
      }
      \sstsubsection{
         LBOUND( 2 ) = \_INTEGER (Read)
      }{
         Lower bounds of new NDF (if LIKE=\texttt{{!}}).  The suggested defaults
         are the lower bounds of the IN NDF.
      }
      \sstsubsection{
         LIKE = NDF (Read)
      }{
         An optional template NDF which, if specified, will be used to
         define the labels, size, shape, data type and axis range of
         the new NDF.  If a null response (\texttt{{!}}) is given, the label,
         units, axis labels, and axis units are taken from the IN NDF.
         The task prompts for the data type and bounds, using those of
         the IN NDF as defaults, and the axis ranges.  \texttt{[!]}
      }
      \sstsubsection{
         OUT = NDF (Write)
      }{
         The new NDF to contain the surface fit.
      }
      \sstsubsection{
         TITLE = LITERAL (Read)
      }{
         A title for the new NDF.  If a null response (\texttt{{!}}) is given,
         the title will be propagated either from LIKE, or from IN
         if LIKE=\texttt{{!}}.  \texttt{[!]}
      }
      \sstsubsection{
         TYPE = LITERAL (Read)
      }{
         Data type for the new NDF (if LIKE=\texttt{{!}}).  It must be one of
         the following: \texttt{"\_DOUBLE"}, \texttt{"\_REAL"}, \texttt{"\_INTEGER"},
         \texttt{"\_WORD"}, \texttt{"\_BYTE"}, \texttt{"\_UBYTE"}.  The suggested default is the data type of
         the data array in the IN NDF.
      }
      \sstsubsection{
         UBOUND( 2 ) = \_INTEGER (Read)
      }{
         Upper bounds of new NDF (if LIKE=\texttt{{!}}).  The suggested defaults
         are the upper bounds of the IN NDF.
      }
      \sstsubsection{
         VARIANCE = \_LOGICAL (Read)
      }{
         If \texttt{TRUE}, a uniform variance array equated to the mean squared
         residual of the fit is created in the output NDF, provided the
         SURFACEFIT structure contains the RMS component.  \texttt{[FALSE]}
      }
      \sstsubsection{
         XLIMIT( 2 ) = \_DOUBLE (Read)
      }{
         Co-ordinates of the left then right edges of the \textit{x} axis (if
         LIKE=\texttt{{!}}).  The suggested defaults are respectively the
         minimum and maximum \textit{x} co-ordinates of the IN NDF.
      }
      \sstsubsection{
         YLIMIT( 2 ) = \_DOUBLE (Read)
      }{
         Co-ordinates of the bottom then top edges of the \textit{y} axis (if
         LIKE=\texttt{{!}}).  The suggested defaults are respectively the
         minimum and maximum \textit{y} co-ordinates of the IN NDF.
      }
   }
   \sstexamples{
      \sstexamplesubsection{
         makesurface flatin flatout $\backslash$
      }{
         This generates a two-dimensional image in the NDF called flatout
         using the surface fit stored in the two-dimensional NDF flatin.
         The created image has the same data type, bounds, and
         co-ordinate limits as the data array of flatin.
      }
      \sstexamplesubsection{
         makesurface flatin flatout type=\_wo lbound=[1,1] ubound=[320,512]
      }{
         As the previous example, except that the data array in flatout
         has data type \_WORD, and the bounds of flatout are 1:320,
         1:512.
      }
      \sstexamplesubsection{
         makesurface flatin flatout like=flatin
      }{
         This has the same effect as the first example, except it has
         an advantage.  If the current co-ordinate system is \texttt{"Data"} and
         either or both of the axes are inverted (values decrease with
         increasing pixel index), the output image will be correctly
         oriented.
      }
      \sstexamplesubsection{
         makesurface flatin flatout like=template title="Surface fit"
      }{
         This generates a two-dimensional image in the NDF called flatout
         using the surface fit stored in the two-dimensional NDF flatin.
         The created image inherits the attributes of the NDF called
         template.  The title of flatout is \texttt{"Surface fit"}.
      }
   }
   \sstnotes{
      \sstitemlist{

         \sstitem
         The polynomial surface fit is stored in SURFACEFIT extension,
         component FIT of type POLYNOMIAL, variant CHEBYSHEV or
         BSPLINE.  This extension is created by FITSURFACE.  Also read
         from the SURFACEFIT extension is the co-ordinate system
         (component COSYS), and the fit RMS (component RMS).

         \sstitem
         When LIKE=\texttt{{!}}, COSYS=\texttt{"Data"} or \texttt{"Axis"} and
         the original NDF had an axis that decreased with increasing
         pixel index, you may want to flip the co-ordinate limits (via
         Parameters XLIMIT or YLIMIT) to match the original sense of
         the axis, otherwise the created surface will be flipped with
         respect to the image from which it was fitted.
      }
   }
   \sstdiytopic{
      Related Applications
   }{
KAPPA: \htmlref{FITSURFACE}{FITSURFACE},
\htmlref{SURFIT}{SURFIT}.
   }
   \sstimplementationstatus{
      \sstitemlist{

         \sstitem
         This routine correctly processes the \htmlref{AXIS}{apndf:axis}, DATA, \htmlref{QUALITY}{apndf:quality},
         \htmlref{VARIANCE}{apndf:variance}, \htmlref{LABEL}{apndf:label}, \htmlref{TITLE}{apndf:title}, \htmlref{UNITS}{apndf:units}, \htmlref{WCS}{apndf:wcs}, and \htmlref{HISTORY}{apndf:history}~ components of an NDF
         data structure and propagates all \htmlref{extensions}{apndf:extensions}.  However, neither
         QUALITY nor a SURFACEFIT extension is propagated when LIKE is not
         null.

         \sstitem
         All \htmlref{non-complex numeric data types}{ap:HDStypes} can be handled.  Processing
         is performed in single- or double-precision floating point, as
         appropriate.
      }
   }
}
\sstroutine{
   MANIC
}{
   Change the dimensionality of all or part of an NDF
}{
   \sstdescription{
      This application manipulates the dimensionality of an \NDFref{NDF}.
      The input NDF can be projected on to any \textit{n}-dimensional surface 
      (line, plane, \emph{etc.}) by averaging or taking the median the pixels in
      perpendicular directions, or grown into new dimensions by duplicating an
      existing \textit{n}-dimensional surface.  The order of the axes can also
      be changed at the same time.  Any combination of these operations is also
      possible.

      The shape of the output NDF is specified using Parameter AXES.  This
      is a list of integers, each element of which identifies the source
      of the corresponding axis of the output---either the index of one of
      the pixel axes of the input, or a zero indicating that the input
      should be expanded with copies of itself along that axis.  If any
      axis of the input NDF is not referenced in the AXES list, the
      missing dimensions will be collapsed to form the resulting data.
      Dimensions are collapsed by averaging all the non-bad pixels along
      the relevant pixel axis (or axes).
   }
   \sstusage{
      manic in out axes
   }
   \sstparameters{
      \sstsubsection{
         AXES( ) = \_INTEGER (Read)
      }{
         An array of integers which define the pixel axes of the output
         NDF.  The array should contain one value for each pixel axis in
         the output NDF.  Each value can be either a positive integer or
         zero.  If positive, it is taken to be the index of a pixel axis
         within the input NDF which is to be used as the output axis.  If
         zero, the output axis will be formed by replicating the entire
         output NDF a specified number of times (see Parameters LBOUND and
         UBOUND).  At least one non-zero value must appear in the list, and
         no input axis may be used more than once.
      }
      \sstsubsection{
         ESTIMATOR = \htmlref{LITERAL}{se:parmenu} (Read)
      }{
         The method by which data values in collapsed axes are combined.
         The permittted options are \texttt{"Mean"} to form the average, or
         \texttt{"Median"} to use the median.  \texttt{["Mean"]}
      }
      \sstsubsection{
         IN = NDF (Read)
      }{
         The input NDF.
      }
      \sstsubsection{
         LBOUND( ) = \_INTEGER (Read)
      }{
         An array holding the lower pixel bounds of any new axes in the
         output NDF (that is, output axes which have a zero value in the
         corresponding element of the AXES parameter).  One element must
         be given for each zero-valued element within AXES, in order of
         appearance within AXES.  The dynamic default is to use 1 for
         every element.  \texttt{[]}
      }
      \sstsubsection{
         OUT = NDF (Write)
      }{
         The output NDF.
      }
      \sstsubsection{
         TITLE = LITERAL (Read)
      }{
         Title for the output NDF.  A null (\texttt{{!}}) means use the title from the input
         NDF.  \texttt{[!]}
      }
      \sstsubsection{
         UBOUND( ) = \_INTEGER (Read)
      }{
         An array holding the upper pixel bounds of any new axes in the
         output NDF (that is, output axes which have a zero value in the
         corresponding element of the AXES parameter).  One element must
         be given for each zero-valued element within AXES, in order of
         appearance within AXES.  The dynamic default is to use 1 for
         every element.  \texttt{[]}
      }
   }
   \sstexamples{
      \sstexamplesubsection{
         manic image transim [2,1]
      }{
         This transposes the two-dimensional NDF image so that its \textit{x} pixel
         co-ordinates are in the \textit{y} direction and vice versa.  The ordering
         of the axes within the \htmlref{current WCS Frame}{se:curframe}~ will only
         be changed if the \htmlref{Domain}{se:domains} of the
         \htmlref{current Frame}{se:curframe}~ is PIXEL or AXES.  For instance,
         if the current Frame has Domain \texttt{"SKY"}, with Axis 1 being RA and
         Axis 2 being DEC, then these will be unchanged in the output NDF.
         However, the Mapping which is used to relate (RA,DEC) positions
         to pixel positions will be modified to take the permutation of
         the pixel axes into account.
      }
      \sstexamplesubsection{
         manic cube summ 3
      }{
         This creates a one dimensional output NDF called summ, in which
         the single pixel axis corresponds to the \textit{z} (third) axis in an input
         NDF called (cube).  Each element in the output is equal to the
         average data value in the corresponding \textit{xy} plane of the input.
      }
      \sstexamplesubsection{
         manic in=cube out=summ axis=3 estimator=median
      }{
         The same as the previous example, except each output value is 
         equal to the median data value in the corresponding \textit{xy} plane of
         the input cube.
      }
      \sstexamplesubsection{
         manic line plane [0,1] lbound=1 ubound=25
      }{
         This takes a one-dimensional NDF called line and expands it into a
         two-dimensional NDF called plane.  The second pixel axis of the output
         NDF corresponds to the first (and only) pixel axis in the input NDF.
         The first pixel axes of the output is formed by replicating the
         the input NDF 25 times.
      }
      \sstexamplesubsection{
         manic line plane [1,0] lbound=1 ubound=25
      }{
         This does the same as the last example except that the output
         NDF is transposed.  That is, the input NDF is copied into the
         output NDF so that it is parallel to pixel Axis 1 (\textit{x}) in the
         output NDF, instead of pixel Axis 2 (\textit{y}) as before.
      }
      \sstexamplesubsection{
         manic cube hyper [1,0,0,0,0,0,3] ubound=[2,4,2,2,1] accept
      }{
         This manic example projects the second dimension of an input
         three-dimensional NDF on to the plane formed by its first and third
         dimensions by averaging, and grows the resulting plane
         up through five new dimensions with a variety of extents.
      }
   }
   \sstnotes{
      \sstitemlist{

         \sstitem
         This application permutes the NDF pixel axes, and any associated AXIS
         structures.  It does not change the axes of the current WCS co-ordinate
         Frame, either by permuting, adding or deleting, unless that frame has
         Domain \texttt{"PIXEL"} or \texttt{"AXES"}.  See the first example in the \texttt{"Examples"}
         section.
      }
   }
   \sstdiytopic{
      Related Applications
   }{
KAPPA: \htmlref{COLLAPSE}{COLLAPSE},
\htmlref{PERMAXES}{PERMAXES}.
   }
   \sstimplementationstatus{
      \sstitemlist{

         \sstitem
         This routine correctly processes the \htmlref{AXIS}{apndf:axis}, DATA, \htmlref{VARIANCE}{apndf:variance},
         \htmlref{LABEL}{apndf:label}, \htmlref{TITLE}{apndf:title}, \htmlref{UNITS}{apndf:units}, \htmlref{WCS}{apndf:wcs}, and \htmlref{HISTORY}{apndf:history}~ components of the input NDF and
         propagates all \htmlref{extensions}{apndf:extensions}.  QUALITY is also propagated if possible
 (\emph{i.e.} if no input axes are collapsed).

         \sstitem
         Processing of \htmlref{bad pixels}{se:masking} and automatic \htmlref{quality masking}{se:qualitymask} are
         supported.

         \sstitem
         All \htmlref{non-complex numeric data types}{ap:HDStypes} can be handled.

         \sstitem
         Any number of NDF dimensions is supported, up to a maximum of 7.
      }
   }
}
\sstroutine{
   MATHS
}{
   Evaluates mathematical expressions applied to NDF data structures
}{
   \sstdescription{
      This application allows arithmetic and mathematical functions to
      be applied pixel-by-pixel to a number of \NDFref{NDF} data structures and
      constants so as to produce a new NDF.  The operations to be
      performed are specified using a Fortran-like mathematical
      expression.  Up to 26 each input NDF data and variance arrays, 26
      parameterised `constants', and pixel and data co-ordinates along
      up to 7 dimensions may be combined in wide variety of ways using
      this application.  The task can also calculate variance estimates
      for the result when there is at least one input NDF array.
   }
   \sstusage{
      maths exp out ia-iz=? va-vz=? fa-fz=? pa-pz=? lbound=? ubound=?
   }
   \sstparameters{
      \sstsubsection{
         EXP = LITERAL (Read)
      }{
         The mathematical expression to be evaluated for each NDF
         pixel, \emph{e.g.} \texttt{"(IA-IB$+$2)$*$PX"}.  In this
         expression, input NDFs are denoted by the variables IA, IB,
         \ldots IZ, while constants may either be given literally or
         represented by the variables PA, PB, \ldots PZ.  Values for
         those NDFs and constants which appear in the expression will
         be requested via the application's parameter of the same
         name.

         Fortran-77 syntax is used for specifying the expression,
         which may contain the usual intrinsic functions, plus a few
         extra ones.  An appendix in \xref{SUN/61}{sun61}{} gives a
         full description of the syntax used and an up to date list of
         the functions available. The expression may be up to 132
         characters long and is case insensitive.
      }
      \sstsubsection{
         FA-FZ = LITERAL (Read)
      }{
         These parameters supply the values of `sub-expressions' used
         in the expression EXP.  Any of the 26 (FA, FB, \ldots FZ) may
         appear; there is no restriction on order.  These parameters
         should be used when repeated expressions are present in
         complex expressions, or to shorten the value of EXP to fit
         within the 132-character limit.  Sub-expressions may contain
         references to other sub-expressions and constants (PA-PZ).  An
         example of using sub-expressions is:
         \begin{description}
         \item EXP $>$ \texttt{PA$*$ASIND(FA/PA)$*$XA/FA}
         \item FA $>$ \texttt{SQRT(XA$*$XA$+$XB$*$XB)}
         \item PA $>$ \texttt{10.1}
         \end{description}
         where the parameter name is to the left of $>$ and its value is
         to the right of the $>$.
      }
      \sstsubsection{
         IA-IZ = NDF (Read)
      }{
         The set of 26 parameters named IA, IB, \ldots IZ is used to
         obtain the input NDF data structure(s) to which the
         mathematical expression is to be applied.  Only those
         parameters which actually appear in the expression are used,
         and their values are obtained in alphabetical order.  For
         instance, if the expression were \texttt{"SQRT(IB$+$IA)"}, then the
         Parameters IA and IB would be used (in this order) to obtain
         the two input NDF data structures.
      }
      \sstsubsection{
         LBOUND( ) = \_INTEGER (Read)
      }{
         Lower bounds of new NDF, if LIKE=\texttt{{!}} and there is no input NDF
         referenced in the expression.  The number of values required
         is the number of pixel co-ordinate axes in the expression.
      }
      \sstsubsection{
         LIKE = NDF (Read)
      }{
         An optional template NDF which, if specified, will be used to
         define bounds and data type of the new NDF, when the expression
         does not contain a reference to an NDF.  If a null response
         (\texttt{{!}}) is given the bounds are obtained via Parameters LBOUND
         and UBOUND, and the data type through Parameter TYPE.  \texttt{[!]}
      }
      \sstsubsection{
         OUT = NDF (Write)
      }{
         Output NDF to contain the result of evaluating the expression
         at each pixel.
      }
      \sstsubsection{
         PA-PZ = \_DOUBLE (Read)
      }{
         The set of 26 parameters named PA, PB, \ldots PZ is used to
         obtain the numerical values of any parameterised `constants'
         which appear in the expression being evaluated.  Only those
         parameters which actually appear in the expression are used,
         and their values are obtained in alphabetical order.  For
         instance, if the expression were \texttt{"PT$*$SIN(IA/PS)"}, then the
         Parameters PS and PT (in this order) would be used to obtain
         numerical values for substitution into the expression at the
         appropriate points.

         These parameters are particularly useful for supplying the
         values of constants when writing procedures, where the
         constant may be determined by a command-language variable, or
         when the constant is stored in a data structure such as a
         global parameter.  In other cases, constants should normally be
         given literally as part of the expression, as in \texttt{"IZ$*$$*$2.77"}.
      }
      \sstsubsection{
         QUICK = \_LOGICAL (Read)
      }{
         Specifies the method by which values for the variance
         component of the output NDF are calculated.  The algorithm used
         to determine these values involves perturbing each of the
         input NDF data arrays in turn by an appropriate amount, and
         then combining the resulting output perturbations.  If QUICK
         is set to \texttt{TRUE}, then each input data array will be perturbed
         once, in the positive direction only.  If QUICK is set to
         \texttt{FALSE}, then each will be perturbed twice, in the positive and
         negative directions, and the maximum resultant output
         perturbation will be used to calculate the output variance.
         The former approach (the normal default) executes more
         quickly, but the latter is likely to be more accurate in cases
         where the function being evaluated is highly non-linear,
         and/or the errors on the data are large.  This parameter is
         ignored if the expression does not contain a token to at least
         one input NDF structure.  \texttt{[TRUE]}
      }
      \sstsubsection{
         TITLE = LITERAL (Read)
      }{
         The title for the output NDF.  A null value will cause
         the title of the (alphabetically) first input NDF to be used
         instead.  \texttt{[!]}
      }
      \sstsubsection{
         TYPE = LITERAL (Read)
      }{
         Data type for the new NDF, if LIKE=\texttt{{!}} and no input NDFs are
         referenced in the expression.  It must be one either
         \texttt{"\_DOUBLE"} or \texttt{"\_REAL"}.
      }
      \sstsubsection{
         UBOUND( ) = \_INTEGER (Read)
      }{
         Upper bounds of new NDF, if LIKE=\texttt{{!}} and there is no input NDF
         referenced in the expression.  These must not be smaller
         than the corresponding LBOUND.  The number of values required
         is the number of pixel co-ordinate axes in the expression.
      }
      \sstsubsection{
         UNITS = \_LOGICAL (Read)
      }{
         Specifies whether the UNITS component of the (alphabetically)
         first input NDF or the template NDF will be propagated to the
         output NDF.  By default this component is not propagated since,
         in most cases, the units of the output data will differ from
         those of any of the input data structures.  In simple cases,
         however, the units may be unchanged, and this parameter then
         allows the UNITS component to be preserved.  This parameter is
         ignored if the expression does not contain a token to at least
         one input NDF structure and LIKE=\texttt{{!}}.  \texttt{[FALSE]}
      }
      \sstsubsection{
         VA-VZ = NDF (Read)
      }{
         The set of 26 parameters named VA, VB, \ldots VZ is used to
         obtain the input NDF \htmlref{variance array(s)}{apndf:variance}~ to
         which the mathematical expression is to be applied.  The variance VA
         corresponds to the data array specified by Parameter IA, and
         so on.  Only those parameters which actually appear in the
         expression, and do not have their corresponding data-array
         Parameter IA-IZ present, have their values obtained in
         alphabetical order.  For instance, if the expression were
         \texttt{"IB$+$SQRT(VB$+$VA)"}, then the Parameters VA and IB would be used
         (in this order) to obtain the two input NDF data structures.
         The first would use just the variance array, whilst the second
         would read both data and variance arrays.
      }
      \sstsubsection{
         VARIANCE = \_LOGICAL (Read)
      }{
         Specifies whether values for the VARIANCE component of the
         output NDF should be calculated.  If this parameter is set to
         \texttt{TRUE} (the normal default), then output variance values will be
         calculated if any of the input NDFs contain variance
         information.  Any which do not are regarded as having zero
         variance.  Variance calculations will normally be omitted only
         if none of the input NDFs contain variance information.
         However, if VARIANCE is set to \texttt{FALSE}, then calculation of
         output variance values will be disabled under all
         circumstances, with a consequent saving in execution time.
         This parameter is ignored if the expression does not contain
         at least one token to an input NDF structure.
         \texttt{[TRUE]}
      }
   }
   \sstexamples{
      \sstexamplesubsection{
         maths "ia-1" dat2 ia=dat1
      }{
         The expression \texttt{"ia-1"} is evaluated to subtract 1 from each
         pixel of the input NDF referred to as IA, whose values reside
         in the data structure dat1.  The result is written to the NDF
         structure dat2.
      }
      \sstexamplesubsection{
         maths "(ia-ib)/ic" ia=data ib=back ic=flat out=result units
      }{
         The expression \texttt{"(ia-ib)/ic"} is evaluated to remove a
         background from an image and to divide it by a flat-field.
         All the images are held in NDF data structures, the input
         image being obtained from the data structure data, the
         background image from back and the flat-field from flat.  The
         result is written to the NDF structure result.  The data units
         are unchanged and are therefore propagated to the output NDF.
      }
      \sstexamplesubsection{
         maths "-2.5$*$log10(ii)$+$25.7" ii=file1 out=file2
      }{
         The expression \texttt{"-2.5$*$log10(ii)$+$25.7"} is evaluated to convert
         intensity measurements into magnitudes, including a zero
         point.  Token II represents the input measurements held in the
         NDF structure file1.  The result is written to the NDF
         structure file2.  If file1 contains variance values, then
         corresponding variance values will also be calculated for
         file2.
      }
      \sstexamplesubsection{
         maths exp="pa$*$exp(ia$+$pb)" out=outfile pb=13.7 novariance
      }{
         The expression \texttt{"pa$*$exp(ia$+$pb)"} is evaluated with a value of
         \texttt{13.7} for the constant PB, and output is written to the NDF
         structure outfile.  The input NDF structure to be used for
         token IA and the value of the other numerical constant PA will
         be prompted for.  \texttt{NOVARIANCE} has been specified so that output
         variance values will not be calculated.
      }
      \sstexamplesubsection{
         maths exp="mod(XA,32)$+$mod(XB,64)" out=outfile like=comwest
      }{
         The expression \texttt{"mod(XA,32)$+$mod(XB,64)"} is evaluated, and
         output is written to the NDF structure outfile.  The output
         NDF inherits the shape, bounds, and other properties (except the
         variance) of the NDF called comwest.  The data type of outfile
         is \_REAL unless comwest has type \_DOUBLE.  XA and XB represent
         the pixel co-ordinates along the \textit{x} and \textit{y} axes respectively.
      }
      \sstexamplesubsection{
         maths "xf$*$xf$+$0$*$xa" ord2 lbound=[-20,10] ubound=[20,50]
      }{
         The expression \texttt{"xf$*$xf$+$0$*$xa"} is evaluated, and output is
         written to the NDF structure ord2.  The output NDF has data
         type \_REAL, is two-dimensional with bounds $-$20:20, 10:50.  The
         XA is needed to indicate that XF represents pixel co-ordinates
         along the \textit{y} axis.
      }
      \sstexamplesubsection{
         maths "xa/max(1,xb)$+$sqrt(va)" ord2 va=fuzz title="Fuzz correction"
      }{
         The expression \texttt{"xa/max(1,xb)$+$sqrt(va)"} is evaluated, and output
         is written to the NDF structure ord2.  Token VA represents the
         input variance array held in the NDF structure fuzz.  The
         output NDF inherits the shape, bounds, and other properties of
         fuzz.  The title of ord2 is \texttt{"Fuzz correction"}.  The data type
         of ord2 is \_REAL unless fuzz has type \_DOUBLE.  XA and XB
         represent the pixel co-ordinates along the \textit{x} and \textit{y} axes
         respectively.
      }
   }
   \sstnotes{
      \sstitemlist{

         \sstitem
         The alphabetically first input NDF is regarded as the primary
         input dataset.  NDF components whose values are not changed by
         this application will be propagated from this NDF to the output.
         The same propagation rules apply to the LIKE template NDF,
         except that the output NDF does have inherit any variance
         information.

         \sstitem
         There are additional tokens which can appear in the expression.

         The set of seven tokens named CA, CB, \ldots CG is used to obtain the
         data co-ordinates from the primary input NDF data structure.  Any
         of the seven parameters may appear in the expression.  The order
         defines which axis is which, so for example, \texttt{"2$*$CF$+$CB$*$CB"} means
         the first-axis data co-ordinates squared, plus twice the
         co-ordinates along the second axis.  There must be at least one
         input NDF in the expression to use the CA-CG tokens, and it must
         have dimensionality of at least the number of CA-CG tokens given.

         The set of seven tokens named XA, XB, \ldots XG is used to obtain the
         pixel co-ordinates from the primary input NDF data structure.  Any
         of the seven parameters may appear in the expression.  The order
         defines which axis is which, so for example, \texttt{"SQRT(XE)$+$XC"} means
         the first-axis pixel co-ordinates plus the square root of the
         co-ordinates along the second axis.  Here no input NDF need be
         supplied.  In this case the dimensionality of the output NDF is equal to the
         number of XA-XG tokens in the expression.  However, if there is
         at least one NDF in the expression, there should not be more
         XA-XG tokens than the dimensionality of the output NDF (given
         as the intersection of the bounds of the input NDFs).

         \sstitem
         If illegal arithmetic operations (\emph{e.g.} division by zero, or
         square root of a negative number) are attempted, then a bad pixel will be generated as a result.  (However, the infrastructure
         software that detects this currently does not work on OSF/1
         systems, and therefore MATHS will crash in this circumstance.)

         \sstitem
         All arithmetic performed by this application is floating
         point.  Single-precision will normally be used, but
         double-precision will be employed if any of the input NDF arrays
         has a numeric type of \_DOUBLE.

      }
   }
   \sstdiytopic{
      Calculating Variance
   }{
      The algorithm used to calculate output variance values is
      general-purpose and will give correct results for any reasonably
      well-behaved mathematical expression.  However, this application
      as a whole, and the variance calculations in particular, are
      likely to be less efficient than a more specialised application
      written knowing the form of the mathematical expression in
      advance.  For simple operations (addition, subtraction, \emph{etc.}) the
      use of other applications (ADD, SUB, \emph{etc.}) is therefore
      recommended, particularly if variance calculations are required.

      The main value of the variance-estimation algorithm used here
      arises when the expression to be evaluated is too complicated, or
      too infrequently used, to justify the work of deriving a direct
      formula for the variance.  It is also of value when the data
      errors are especially large, so that the linear approximation
      normally used in error analysis breaks down.

      There is no variance processing when there are no tokens for
      input NDF structures.
   }
   \sstdiytopic{
      Timing
   }{
      If variance calculations are not being performed, then the time
      taken is approximately proportional to the number of NDF pixels
      being processed.  The execution time also increases with the
      complexity of the expression being evaluated, depending in the
      usual way on the nature of any arithmetic operations and
      intrinsic functions used.  If certain parts of the expression will
      often give rise to illegal operations (resulting in bad pixels), then execution time may be minimised by placing these operations
      near the beginning of the expression, so that later parts may not
      need to be evaluated.

      If output variance values are being calculated and the QUICK
      parameter is set to \texttt{TRUE}, then the execution time will be
      multiplied by an approximate factor ($N+$1), where \textit{n} is the number
      of input NDFs which contain a VARIANCE component.  If QUICK is set
      to \texttt{FALSE}, then the execution time will be multiplied by an
      approximate factor ($2N+$1).
   }
   \sstdiytopic{
      Related Applications
   }{
KAPPA: \htmlref{CREFRAME}{CREFRAME},
\htmlref{SETAXIS}{SETAXIS},
and numerous arithmetic tasks;
\xref{FIGARO}{sun86}{}: numerous arithmetic tasks.
   }
   \sstimplementationstatus{

      \sstitemlist{

         \sstitem
         This routine correctly processes the \htmlref{AXIS}{apndf:axis}, DATA, \htmlref{QUALITY}{apndf:quality},
         \htmlref{VARIANCE}{apndf:variance}, \htmlref{LABEL}{apndf:label}, \htmlref{TITLE}{apndf:title}, \htmlref{UNITS}{apndf:units}, \htmlref{WCS}{apndf:wcs}, and \htmlref{HISTORY}{apndf:history}~ components of the
         input NDFs.  HISTORY and \htmlref{extensions}{apndf:extensions}~ are propagated from both the
         primary NDF and template NDF.

         \sstitem
         Processing of \htmlref{bad pixels}{se:masking} and automatic \htmlref{quality masking}{se:qualitymask} are
         supported.

         \sstitem
         All \htmlref{non-complex numeric data types}{ap:HDStypes} can be handled.

         \sstitem
         NDFs with any number of dimensions can be processed.  The NDFs
         supplied as input need not all be the same shape.
      }
   }
}
\sstroutine{
   MEDIAN
}{
   Smooths a two-dimensional data array using a weighted median filter
}{
   \sstdescription{
      This task filters the two-dimensional data array in the input \NDFref{NDF}
      structure with a Weighted Median Filter (WMF) in a 3-by-3-pixel
      kernel to create a new NDF.  There are a number of predefined
      weighting functions and parameters that permit other symmetric
      weighting functions.  See Parameter MODE and the topic
      \htmlref{``User-defined Weighting Functions''}{weighting:median}.

      A threshold for replacement of a value by the median can be set.
      If the absolute value of the difference between the actual value
      and the median is less than the threshold, the replacement will
      not occur.  The array boundary is dealt by either pixel
      replication or a reflection about the edge pixels of the array.

      The WMF can be repeated iteratively a specified number of times,
      or it can be left to iterate continuously until convergence is
      achieved and no further changes are made to the data.  In the
      latter case a damping algorithm is used if the number of
      iterations exceeds some critical value, which prevents the result
      oscillating between two solutions (which can sometimes happen).
      When damping is switched on data values are replaced not by the
      median value, but by a value midway between the original and the
      median.

      Bad pixels are not included in the calculation of the median.
      There is a defined threshold which specifies minimum-allowable
      median position as a fraction of the median position when there
      are no bad pixels.  For neighbourhoods with too many bad pixels,
      and so the median position is too small, the resulting output
      pixel is bad.
   }
   \sstusage{
      median in out [mode] [diff] [bound] [numit] [corner] [side] [centre]
   }
   \sstparameters{
      \sstsubsection{
         BOUND = LITERAL (Read)
      }{
         Determines the type of padding required at the array edges
         before the filtering starts.  The alternatives are described
         below.

         \begin{description}
         \item \texttt{"Replication"} --- The values at the edge of the data array
                           are replicated into the padded area.  For
                           example, with STEP=\texttt{2} one corner of the
                           original and padded arrays would appear
                           as follows:

                \parbox{29mm}{corner of original array:}
                $
                \begin{array}{ccccc}
                1 & 1 & 1 & 1 & 1 \\
                1 & 2 & 2 & 2 & 2 \\
                1 & 2 & 3 & 3 & 3 \\
                1 & 2 & 3 & 4 & 4 \\
                1 & 2 & 3 & 4 & 5
                \end{array}
                $
                \hspace{1em}\parbox{35mm}{corresponding corner of padded array:}
                $
                \begin{array}{ccccccc}
                1 & 1 & 1 & 1 & 1 & 1 & 1 \\
                1 & 1 & 1 & 1 & 1 & 1 & 1 \\
                1 & 1 & 1 & 1 & 1 & 1 & 1 \\
                1 & 1 & 1 & 1 & 1 & 1 & 1 \\
                1 & 1 & 1 & 2 & 2 & 2 & 2 \\
                1 & 1 & 1 & 2 & 3 & 3 & 3 \\
                1 & 1 & 1 & 2 & 3 & 4 & 4 \\
                1 & 1 & 1 & 2 & 3 & 4 & 5
                \end {array}
                $

         \item \texttt{"Reflection"} --- The values near the edge of the data array
                           are reflected about the array's edge pixels.
                           For example, with STEP=\texttt{2} one corner of the
                           original and padded arrays would appear as
                           follows:

                \parbox{29mm}{corner of original array:}
                $
                \begin{array}{ccccc}
                1 & 1 & 1 & 1 & 1 \\
                1 & 2 & 2 & 2 & 2 \\
                1 & 2 & 3 & 3 & 3 \\
                1 & 2 & 3 & 4 & 4 \\
                1 & 2 & 3 & 4 & 5
                \end{array}
                $
                \hspace{1em}\parbox{35mm}{corresponding corner of padded array:}
                $
                \begin{array}{ccccccc}
                3 & 2 & 1 & 2 & 3 & 3 & 3 \\
                2 & 2 & 1 & 2 & 2 & 2 & 2 \\
                1 & 1 & 1 & 1 & 1 & 1 & 1 \\
                2 & 2 & 1 & 2 & 2 & 2 & 2 \\
                3 & 2 & 1 & 2 & 3 & 3 & 3 \\
                3 & 2 & 1 & 2 & 3 & 4 & 4 \\
                3 & 2 & 1 & 2 & 3 & 4 & 5 \\
                \end {array}
                $
         \end{description}

          \texttt{["Replication"]}
      }
      \sstsubsection{
         CENTRE = \_INTEGER (Read)
      }{
         Central value for weighting function, required if MODE = \texttt{$-$1}.
         It must be an odd value in the range 1 to 21.  \texttt{[1]}
      }
      \sstsubsection{
         CORNER = \_INTEGER (Read)
      }{
         Corner value for weighting function, required if MODE = \texttt{$-$1}.
         It must be in the range 0 to 10.  \texttt{[1]}
      }
      \sstsubsection{
         DIFF = \_DOUBLE (Read)
      }{
         Replacement of a value by the median occurs if the absolute
         difference of the value and the median is greater than DIFF.
         \texttt{[0.0]}
      }
      \sstsubsection{
         IN = NDF (Read)
      }{
         NDF structure containing the two-dimensional data array to be
         filtered.
      }
      \sstsubsection{
         ITERATE = \htmlref{LITERAL}{se:parmenu} (Read)
      }{
         Determines the type of iteration used.  The alternatives are
         described below.

         \begin{description}
         \item \texttt{"Specified"} --- You specify the number of iterations
                           at each step size in the Parameter NUMIT.

         \item \texttt{"Continuous"} --- The filter iterates continuously until
                           convergence is achieved and the array is no
                           longer changed by the filter.  A damping
                           algorithm comes into play after MAXIT
                           iterations, and the filter will give up
                           altogether after MAXIT $\times$ 1.5 iterations
                           (rounded up to the next highest integer).
         \end{description}

         \texttt{"Continuous"} mode is recommended only for images which are
         substantially smooth to start with (such as a sky background
         frame from a measuring machine).  Complex images may take many
         iterations, and a great deal of time, to converge.
         \texttt{["Specified"]}
      }
      \sstsubsection{
         MAXIT = \_INTEGER (Read)
      }{
         The maximum number of iterations of the filter before the
         damping algorithm comes into play, when ITERATE =
         \texttt{"Continuous"}.  It must lie in the range 1 to 30.  \texttt{[10]}
      }
      \sstsubsection{
         MEDTHR = \_REAL (Read)
      }{
         Minimum-allowable actual median position as a fraction of the
         median position when there are no bad pixels, for the
         computation of the median at a given pixel.  \texttt{[0.8]}
      }
      \sstsubsection{
         MODE = \_INTEGER (Read)
      }{
         Determines type of weighting used, \texttt{$-$1} allows you to define the
         weighting, and \texttt{0} to \texttt{7} the predefined filters.  The predefined
         modes have the following weighting functions:

\[
\begin{array}{lccclccclccclccc}
{\bf 0:}&1&1&1\hspace{4ex}&{\bf 1:}&0&1&0\hspace{4ex}&{\bf 2:}&1&0&1\hspace{4ex}&{\bf 3:}&1&1&1 \\
  &  1&1&1\hspace{4ex}&&  1&1&1\hspace{4ex}&&  0&1&0\hspace{4ex}&&  1&3&1 \\
  &  1&1&1\hspace{4ex}&&  0&1&0\hspace{4ex}&&  1&0&1\hspace{4ex}&&  1&1&1 \\
\\
\\
{\bf 4:}&0&1&0\hspace{4ex}&{\bf 5:}&1&0&1\hspace{4ex}&{\bf 6:}&1&2&1\hspace{4ex}&{\bf 7:}&1&3&1 \\
  &  1&3&1\hspace{4ex}&&  0&3&0\hspace{4ex}&&  2&3&2\hspace{4ex}&&  3&3&3 \\
  &  0&1&0\hspace{4ex}&&  1&0&1\hspace{4ex}&&  1&2&1\hspace{4ex}&&  1&3&1
\end{array}
\]

         \texttt{[0]}
      }
      \sstsubsection{
         NUMIT = \_INTEGER (Read)
      }{
         The specified number of iterations of the filter, when
         ITERATE=\texttt{"Specified"}.  \texttt{[1]}
      }
      \sstsubsection{
         OUT = NDF (Write)
      }{
         NDF structure to contain the two-dimensional data array after
         filtering.
      }
      \sstsubsection{
         SIDE = \_INTEGER (Read)
      }{
         Side value for weighting function, required if MODE = \texttt{$-$1}.
         It must be in the range 0 to 10.  \texttt{[1]}
      }
      \sstsubsection{
         STEP() = \_INTEGER (Read)
      }{
         The spacings between the median filter elements to be used.
         The data may be filtered at one particular spacing by
         specifying a single value, such as STEP=\texttt{4}, or may be filtered
         at a whole series of spacings in turn by specifying a list of
         values, such as STEP=\texttt{[4,3,2,1]}.  There is a limit of 32 values.
         \texttt{[1]}
      }
      \sstsubsection{
         TITLE = LITERAL (Read)
      }{
         The title for the output NDF.  A null value will cause
         the title of the NDF supplied for Parameter IN to be used
         instead.  \texttt{[!]}
      }
   }
   \sstexamples{
      \sstexamplesubsection{
         median a100 a100med
      }{
         This applies an equally weighted median filter to the NDF
         called a100 and writes the result to the NDF a100med.  It uses
         the default settings, which are a single step size of one
         pixel, and a difference threshold of 0.0.  The task pads the
         array by replication to deals with the edge pixels, and runs
         the filter once only.
      }
      \sstexamplesubsection{
         median a100 a100med bound=ref
      }{
         As in the previous example except that it uses reflection
         rather than replication when padding the array.
      }
      \sstexamplesubsection{
         median abc sabc mode=3 step=4 diff=1.0 numit=2
      }{
         This applies a median filter to the NDF called abc with a
         $
         \begin{array}{ccc}
         1 & 1 & 1 \\
         1 & 3 & 1 \\
         1 & 1 & 1
         \end{array}
         $
         weighting mask (MODE=\texttt{3}), a step size of 4 pixels
         (STEP=\texttt{4}) and a difference threshold of 1.0 (DIFF=\texttt{1.0}).  It
         runs the filter twice (NUMIT=\texttt{2}) and writes the result to
         the NDF called sabc.
      }
      \sstexamplesubsection{
         median abc sabc mode=3 step=[4,3,2,1] diff=1.0 numit=2
      }{
         This applies a median filter as in the previous example,
         only this time run the filter at step sizes of 4, 3, 2,
         and 1 pixels, in that order (STEP=\texttt{[4,3,2,1]}).  It runs the
         filter twice at each step size (NUMIT=\texttt{2}).  Note that the
         filter will be run a total of {\em eight\/} times (number of step
         sizes times the number of iterations).
      }
      \sstexamplesubsection{
         median in=spotty step=[4,3,2,1] iterate=cont maxit=6 out=clean
      }{
         This applies a median filter to the NDF called spotty with
         the default settings for the mode and difference threshold.
         It runs the filter at step sizes of 4, 3, 2 and 1 pixels,
         operating continuously at each step size until the result
         converges (ITERATE=\texttt{CONT}).  Damping will begin after 6
         iterations (MAXIT=\texttt{6}), and the filtering will stop regardless
         after 10 iterations (1 $+$ INT(1.5 $*$ MAXIT)).  Note that the
         filter will run an indeterminate number of times, up to a
         maximum of 40 (number of step sizes $\times$ maximum number of
         iterations), and may take a long time.  The resultant data
         array are written to the NDF called clean.
      }
   }
   \label{weighting:median}
   \sstdiytopic{
      User-defined Weighting Functions
   }{
      Parameters CORNER, SIDE, and CENTRE allow other symmetric
      functions in addition to those offered by MODE=\texttt{0}~ to \texttt{7}.  A step
      size has to be specified too; this determines the spacing of the
      elements of the weighting function.  The data can be filtered at
      one step size only, or using a whole series of step sizes in
      sequence.  The weighting function has the form:
\[
\begin{array}{ccccc}

  \%\-\texttt{CORNER} & . &  \%\-\texttt{SIDE}  & . &  \%\-\texttt{CORNER} \\
  . &  & . &  & . \\
  \%\-\texttt{SIDE}  & . &  \%\-\texttt{CENTRE} & . &  \%\-\texttt{SIDE} \\
  . &  & . &  & . \\
  \%\-\texttt{CORNER} & . &  \%\-\texttt{SIDE}  & . &  \%\-\texttt{CORNER}

\end{array}
\]

      The .  Indicates that the weights are separated by the
      stepsize-minus-one zeros.
   }
   \sstdiytopic{
      Related Applications
   }{
KAPPA: \htmlref{BLOCK}{BLOCK},
\htmlref{CONVOLVE}{CONVOLVE},
\htmlref{FFCLEAN}{FFCLEAN},
\htmlref{GAUSMOOTH}{GAUSMOOTH};
\xref{ESP}{sun180}{}: \xref{FASTMED}{sun180}{FASTMED};
\xref{FIGARO}{sun86}{}: \xref{ICONV3}{sun86}{ICONV3},
\xref{ISMOOTH}{sun86}{ISMOOTH},
\xref{IXSMOOTH}{sun86}{IXSMOOTH},
\xref{MEDFILT}{sun86}{MEDFILT}.
   }
   \sstimplementationstatus{
      \sstitemlist{

         \sstitem
         This routine correctly processes the \htmlref{AXIS}{apndf:axis}, DATA, \htmlref{LABEL}{apndf:label}, \htmlref{TITLE}{apndf:title},
         \htmlref{UNITS}{apndf:units}, \htmlref{WCS}{apndf:wcs}, and \htmlref{HISTORY}{apndf:history}~ components of an NDF data structure and
         propagates all \htmlref{extensions}{apndf:extensions}.  \htmlref{VARIANCE}{apndf:variance}~ is not used to weight the
         median filter and is not propagated.
         \htmlref{QUALITY}{apndf:quality}~ is also lost.

         \sstitem
         Processing of \htmlref{bad pixels}{se:masking} and automatic \htmlref{quality masking}{se:qualitymask} are
         supported.

         \sstitem
         All \htmlref{non-complex numeric data types}{ap:HDStypes} can be handled.
      }
   }
}
\sstroutine{
   MEM2D
}{
   Performs a Maximum-Entropy deconvolution of a two-dimensional NDF
}{
   \sstdescription{
      MEM2D is based on the Gull and Skilling Maximum Entropy package
      {\footnotesize \xref{MEMSYS3}{sun117}{}}.  It takes an image and a Point-Spread Function as input
      and produces an equal-sized image as output with higher
      resolution.  Facilities are provided to `analyse' the resulting
      deconvolved image, \emph{i.e.} to calculate an integrated flux in some
      area of the deconvolved image and also an estimate of the
      uncertainty in the integrated flux.  This allows the significance
      of structure visible in the deconvolution to be checked.

      For a detailed description of the algorithm, and further
      references, see the {\footnotesize MEMSYS} users manual, and
      \xref{SUN/117}{sin117}{}.
   }
   \sstusage{
      mem2d in out mask=?
      $\left\{ {\begin{tabular}{l}
                fwhmpsf=? \\
                psf=?
                \end{tabular} }
      \right.$
      \newline\latexhtml{\hspace*{10.5em}}{~~~~~~~~~~~~~~~~~}
      \makebox[0mm][c]{\small psftype}
   }
   \sstparameters{
      \sstsubsection{
         ANALYSE = \_LOGICAL (Read)
      }{
         ANALYSE should be given a \texttt{TRUE} value if an analysis of a
         previously generated deconvolution is to be performed, instead
         of a whole new deconvolution being started.  An analysis
         returns the integrated flux in some area of the deconvolved
         image you specify, together with the standard deviation on the
         integrated flux value.  The area to be integrated over is
         specified by an image associated with Parameter MASK.  This
         facility can, for instance, be used to assess the significance
         of structure seen in the deconvolution.  An analysis can only
         be performed if the input NDF (see Parameter IN) contains a
         MEM2D \htmlref{extension}{apndf:extensions}~ (see Parameter
         EXTEND).  If the input does
         contain such an extension, and if the extension shows that the
         deconvolution was completed, then ANALYSE is defaulted to
         \texttt{TRUE}, otherwise it is defaulted to \texttt{FALSE}.  \texttt{[]}
      }
      \sstsubsection{
         DEF = \_REAL (Read)
      }{
         This is the value to which the output image will default in
         areas for which there is no valid data in the input.  The `zero
         entropy' image is defined to be a flat surface with value
         given by Parameter DEF.  Any deviation of the output image away
         from this image will cause its entropy to become negative.
         Thus a maximum-entropy criterion causes the output image to be
         as similar as possible to a flat surface with value DEF
         (within the constraints of the data).  DEF is defaulted to the
         mean data value in the input image and must always be strictly
         positive.  \texttt{[]}
      }
      \sstsubsection{
         EXTEND = \_LOGICAL (Read)
      }{
         If EXTEND has a \texttt{TRUE} value, then the output NDF will contain
         an extension called MEM2D which will contain all the
         information required to either restart or analyse the
         deconvolution.  Note, including this extension makes the output
         file much bigger (by about a factor of seven).  \texttt{[TRUE]}
      }
      \sstsubsection{
         FWHMICF = \_REAL (Read)
      }{
         This is the Full Width at Half Maximum (in pixels) of a
         Gaussian Intrinsic Correlation Function (ICF) to be used in
         the deconvolution.  The ICF can be used to encode prior
         knowledge of pixel-to-pixel correlations in the output image.
         A value of \texttt{0} for FWHMICF causes no ICF to be used, and so
         no correlations are expected in the output.  Larger values
         encourage smoothness in the output on the scale of the ICF.  If
         a non-zero ICF is used, the image entropy which is maximised
         is not the output image, but a `hidden' image.  This hidden
         image is the deconvolution of the output image with the ICF,
         and is assumed to have no pixel-to-pixel correlations.  \texttt{[2]}
      }
      \sstsubsection{
         FWHMPSF = \_REAL (Read)
      }{
         This is the Full Width at Half Maximum (in pixels) of a
         Gaussian Point Spread Function (PSF).  This PSF is used to
         deconvolve the input only if Parameter PSFTYPE has the value
         \texttt{"Gaussian"}.
      }
      \sstsubsection{
         ILEVEL = \_INTEGER (Read)
      }{
         ILEVEL controls the amount of information displayed as MEM2D
         runs.  If set to \texttt{0} then no information is displayed.  Larger
         values up to a maximum of 3, give larger amounts of
         information.  A value of \texttt{3} gives full {\footnotesize MEMSYS3} diagnostics
         after each iteration.  \texttt{[1]}
      }
      \sstsubsection{
         IN = NDF (Read)
      }{
         The input \NDFref{NDF}.  This can either contain an image to be
         deconvolved, or the output from a previous run of MEM2D.  The
         NDF is considered to be an output from MEM2D if it contains an
         extension called MEM2D (see Parameter EXTEND).  If such an
         extension is found, a check is made to see if the NDF contains
         a completed deconvolution or a partial deconvolution.  If the
         deconvolution is complete, the ANALYSE parameter is defaulted
         to \texttt{TRUE}, and unless you override this default, an
         analysis of the deconvolution contained in the input NDF is
         performed.  If the input deconvolution is not complete, then
         the deconvolution process is restarted from where it left off.
         If no MEM2D extension is found, then a new deconvolution
         is started from scratch.
      }
      \sstsubsection{
         MASK = NDF (Read)
      }{
         An image to use as a mask to define the areas to be integrated
         when performing an analysis (see Parameter ANALYSE).  The
         integrated-flux value calculated by the analysis is actually
         the total data sum in the product of the mask and the
         deconvolved image.  Mask pixel values can be positive or
         negative (or zero) and so, for instance, masks can be arranged
         which subtract off a background brightness from a source
         before returning the integrated source flux.
      }
      \sstsubsection{
         MODEL = NDF (Read)
      }{
         An image to use as the default model for the reconstruction.
         If a null value is given, then a constant value given by the
         Parameter DEF is used to define a flat default model.  The
         section of the given image which matches the bounds of the
         input image is used.  Any bad pixels in the image cause the
         corresponding pixels in the input image to be ignored.  Such
         pixels are set bad in the output.  The model image should
         contain no pixels with a value of zero or less.  The default
         model is defined to have zero entropy.  The hidden image will
         tend to the default model in the absence of data.  It should be
         noted that this model applies to the `hidden' image, not the
         actually required reconstructed image.  The reconstructed image
         is obtained from the hidden image by blurring the hidden image
         with the ICF.  \texttt{[!]}
      }
      \sstsubsection{
         MODELOUT = NDF (Write)
      }{
         An image which can be used for the default model in a further
         run of MEM2D.  Each pixel value in the created image is a
         linear combination of the model value at the corresponding
         pixel in the current reconstruction, and the hidden image
         pixel value.  Pixels for which the hidden image is well away
         from the current model, tend towards the value of the hidden
         image; pixels for which the hidden image is close to the
         current model tend towards the model.  Running MEM2D several
         times, using the new model created on the previous run as the
         model for the current run, can reduce the `mottling' often
         seen in MEM2D reconstructions.  \texttt{[!]}
      }
      \sstsubsection{
         NITER = \_INTEGER (Read)
      }{
         The maximum number of maximum-entropy iterations to perform.  MEM2D
         continues the deconvolution until either {\footnotesize MEMSYS3} indicates
         that the termination criterion ($\Omega=1.0$) has been reached,
         or the maximum number of iterations is reached.  If a
         deconvolution requires more iterations than was allowed by
         NITER, then you can choose to continue the deconvolution
         by giving the prematurely terminated output from MEM2D as the
         input to another run of MEM2D, specifying a larger value for
         NITER.  \texttt{[50]}
      }
      \sstsubsection{
         NOISE = \htmlref{LITERAL}{se:parmenu} (Read)
      }{
         NOISE defines the noise statistics within the input image.  It
         can take the value \texttt{"Gaussian"} or \texttt{"Poisson"}.  If Gaussian noise
         is selected, the data variances are set initially to the
         values stored in the \htmlref{VARIANCE}{apndf:variance}~ component of the input NDF.  If
         no such component exists, then the data variances are set to a
         constant value equal to the RMS difference between adjacent
         pixels in the \textit{x} direction.  {\footnotesize MEMSYS3} scales these initial noise
         estimates to maximise the data `evidence'.  The evidence is
         displayed as \texttt{"LOG(PROB)"} and the noise scaling factor as
         \texttt{"SIGMA"}, if Parameter ILEVEL is set to \texttt{2} or more.  If Poisson
         statistics are selected the uncertainty in each data value is,
         as usual, of the order of the square root of the data value.
         When using Poisson statistics, there is no equivalent to the
         noise scaling performed when using Gaussian statistics.  Any
         input VARIANCE component is ignored.  \texttt{["Gaussian"]}
      }
      \sstsubsection{
         OUT = NDF (Write)
      }{
         The output image in a `primitive' NDF.  The output is the same
         size as the input.  Any pixels which were flagged as bad in the
         input will also be bad in the output.  If Parameter EXTEND is
         \texttt{TRUE}, then the output NDF contains an extension called
         MEM2D containing information which allows the deconvolution to
         be either continued or analysed.  There is no VARIANCE
         component in the output, but any QUALITY values are propagated
         from the input to the output.  If Parameter UPDATE is \texttt{TRUE},
         then the output NDF is created after the first iteration and is
         updated after each subsequent iteration.
      }
      \sstsubsection{
         PSF = NDF (Read)
      }{
         An NDF holding an estimate of the Point Spread Function (PSF)
         of the input image.  This PSF is used to deconvolve the input
         only if Parameter PSFTYPE has the value \texttt{"NDF"}.  The PSF can be
         centred anywhere within the image, the location of the centre
         is specified using Parameters XCENTRE and YCENTRE.  The
         extent of the PSF actually used is controlled by Parameter
         THRESH.
      }
      \sstsubsection{
         PSFTYPE = LITERAL (Read)
      }{
         PSFTYPE determines if the Point Spread Function used in the
         deconvolution is to be Gaussian (if PSFTYPE=\texttt{"Gaussian"}), or
         is to be defined by an image you supply (if PSFTYPE=\texttt{"NDF"}).
         \texttt{["NDF"]}
      }
      \sstsubsection{
         RATE = \_REAL (Read)
      }{
         This is the value to use for the {\footnotesize MEMSYS3} RATE parameter.  It
         determines the rate at which the convergence is allowed to
         proceed.  If RATE is high, each maximum-entropy iteration is allowed to
         make a big change to the current reconstruction.  This can
         cause numeric problems within {\footnotesize MEMSYS3} resulting in MEM2D
         crashing with a \texttt{"floating overflow"} error.  If this happens,
         try reducing RATE.  Useful values will normally be of the order
         of unity, and must lie in the interval 0.0001 to 100.  \texttt{[0.5]}
      }
      \sstsubsection{
         THRESH = \_REAL (Read)
      }{
         The fraction of the PSF peak amplitude at which the extents of
         the NDF PSF are determined.  It must be positive and less than
         0.5.  This parameter is only used when PSFTYPE=\texttt{"NDF"}.  An
         error will result if the input PSF is truncated above this
         threshold.  \texttt{[0.0625]}
      }
      \sstsubsection{
         TITLE = LITERAL (Read)
      }{
         A \htmlref{title}{apndf:title} for the output NDF.  A null (\texttt{{!}}) value means using the
         title of the input NDF.  \texttt{[!]}
      }
      \sstsubsection{
         UPDATE = \_LOGICAL (Read)
      }{
         If UPDATE is given a \texttt{TRUE} value, then the output NDF will be
         created after the first iteration, and will then be updated
         after each subsequent iteration.  This means that the current
         reconstruction can be examined without aborting the
         application.  Also, if Parameter EXTEND is \texttt{TRUE}, then
         if the job aborts for any reason, it can be restarted from the
         last completed iteration (see Parameter IN).  \texttt{[TRUE]}
      }
      \sstsubsection{
         XCENTRE = \_INTEGER (Read)
      }{
         The \textit{x} pixel index of the centre of the PSF within the supplied
         PSF image.  This is only required if PSFTYPE is \texttt{"NDF"}.  XCENTRE
         is defaulted to the middle pixel (rounded down if there are an
         even number of pixels per line).  \texttt{[]}
      }
      \sstsubsection{
         YCENTRE = \_INTEGER (Read)
      }{
         The \textit{y} pixel index (line number) of the centre of the PSF
         within the supplied PSF image.  This is only required if
         PSFTYPE is \texttt{"NDF"}.  YCENTRE is defaulted to the middle line
         (rounded down if there are an even number of lines).  \texttt{[]}
      }
   }
   \sstresparameters{
      \sstsubsection{
         DSUM = \_REAL (Write)
      }{
         The standard deviation of the integrated-flux value calculated if an
         analysis is performed (see Parameter ANALYSE).
      }
      \sstsubsection{
         SUM = \_REAL (Write)
      }{
         The integrated-flux value calculated if an analysis is performed
         (see Parameter ANALYSE).
      }
   }
   \sstexamples{
      \sstexamplesubsection{
         mem2d m51 m51\_hires psftype=gaussian fwhmpsf=3
      }{
         This example deconvolves the data array in the NDF called m51,
         putting the resulting image in the data array of the NDF called
         m51\_hires.  A circular Gaussian Point-Spread Function is used
         with a Full Width at Half Maximum of 3 pixels.
      }
      \sstexamplesubsection{
         mem2d m51 m51\_hires psf=star xcentre=20 ycentre=20
      }{
         This example performs the same function as the previous
         example, but the PSF is defined by the data array of the NDF
         called star, instead of being defined to be Gaussian.  This
         allows the PSF to be any arbitrary two-dimensional function.  NDF
         star could be produced for example, by the \KAPPA
         application called \htmlref{PSF}{PSF}.
         Parameters XCENTRE and YCENTRE give the pixel indices of
         the centre of the beam defined by the PSF in star.  The PSF is
         truncated to one sixteenth of its peak amplitude.
      }
      \sstexamplesubsection{
         mem2d m51\_hires m51\_hires niter=70 psf=star
      }{
         If the previous example failed to converge within the default
         50 iterations, the deconvolution can be started again from
         its current state, rather than having to start again from
         scratch.  Here NITER gives the upper limit on the total number
         of iterations which can be performed (including those performed
         in the previous run of MEM2D), {\bf not} just the number performed in
         this single run of MEM2D.  This facility can also be used if a
         MEM2D run is interrupted for any reason, such as the host
         computer going down, or a batch-queue CPU limit being reached.
         To use this facility the Parameters EXTEND and UPDATE should
         have the default values of \texttt{TRUE}.
      }
      \sstexamplesubsection{
         mem2d m51\_hires mask=nucleus
      }{
         Once a deconvolved image has been produced, the significance
         of features seen in the deconvolution can be assessed.  This
         example takes in the NDF m51\_hires produced by a previous run
         of MEM2D.  If this is a completed deconvolution then the
         Parameter ANALYSE will be defaulted to \texttt{TRUE}, and an analysis
         will be performed.  This effectively results in the
         deconvolution being multiplied by the data array of the NDF
         called nucleus, and the total data sum in the resulting image
         being displayed, together with the standard deviation on the
         total data sum.  The image in m51\_hires is the most probable
         deconvolution, but there may be other deconvolutions only
         slightly less probable than m51\_hires.  The standard deviation
         produced by an analysis takes account of the spread between
         such deconvolutions.  If the total data sum is not significantly
         greater than the standard deviation, then the feature selected
         by the mask image (called nucleus in this case) may well be
         spurious.  The mask image itself may for instance consist of an
         area of uniform value $+$1 covering some feature of interest,
         and the bad value (or equivalently the value zero) everywhere
         else.  The analysis would then give the integrated flux in the
         feature, assuming that the background is known to be zero.  If
         the background is not zero, then the mask may contain a
         background region containing the value $-$1, of equal area to
         the region containing the value $+$1.  The resulting integrated
         flux would then be the total flux in the source minus the flux
         in a background region of equal area.
      }
   }
   \sstnotes{
      \sstitemlist{

         \sstitem
         MEM2D requires a large quantity of memory---almost as much as
         the rest of \KAPPA.  In order for the \KAPPA\ monolith to
         load without you having to increase your memory or datasize
         resources, and because MEM2D is batch oriented (see Timing) it
         is only available as a separate application.

         \sstitem
         Memory is required to store several intermediate images while
         the deconvolution is in progress.  If the input image is small
         enough, these images are stored in a statically declared, internal
         array.  Otherwise, they are stored in dynamically mapped external
         arrays.  There is no limit on the size of image which can be
         processed by MEM2D (other than those imposed by limited resources
         on the host computer).

         \sstitem
         It is sometimes desirable for the pixels in the output image
         to be smaller than those in the input image.  For instance, if the
         input data are critically sampled (two samples per PSF), the output
         image may not be a very good deconvolution.  In such cases
         sub-dividing the output pixels would give better results.  At the
         moment MEM2D cannot do this.  Be warned that sub-dividing the
         input pixels and then running the current version of MEM2D will not
         have the same effect, since the noise in the input image will then
         have pixel-to-pixel correlations, and be interpreted as real structure.
      }
   }
   \sstdiytopic{
      Timing
   }{
      MEM deconvolution is extremely CPU intensive.  The total CPU time
      taken depends partly on the size of the image, and partly on the
      complexity of the structure within the image.  As a typical
      example, a 100$\times$100 image containing 20 Gaussians on a flat
      background took about 34 minutes of elapsed time on an unloaded
      DEC Alpha 2000.  Deconvolution jobs should therefore always be done
      in batch.  To perform an analysis on a deconvolution takes about the
      same length of time as a single deconvolution iteration.
   }
   \sstdiytopic{
      Related Applications
   }{
KAPPA: \htmlref{FOURIER}{FOURIER},
\htmlref{LUCY}{LUCY},
\htmlref{WIENER}{WIENER}.
   }
   \sstimplementationstatus{
      \sstitemlist{

         \sstitem
         This routine correctly processes the \htmlref{AXIS}{apndf:axis}, DATA, \htmlref{QUALITY}{apndf:quality},
         \htmlref{VARIANCE}{apndf:variance}, \htmlref{LABEL}{apndf:label}, \htmlref{TITLE}{apndf:title}, \htmlref{UNITS}{apndf:units}, \htmlref{WCS}{apndf:wcs}, and \htmlref{HISTORY}{apndf:history}~ components of an NDF
         data structure and propagates all \htmlref{extensions}{apndf:extensions}.

         \sstitem
         Processing of \htmlref{bad pixels}{se:masking} and automatic \htmlref{quality masking}{se:qualitymask} are
         supported, though only to remove them by the DEF value.

         \sstitem
         All \htmlref{non-complex numeric data types}{ap:HDStypes} can be handled.  Arithmetic is
         performed using single-precision floating point.
      }
   }
}

\sstroutine{
   MFITTREND
}{
   Fits independent trends to data lines that are parallel to an
   axis
}{
   \sstdescription{
      This routine fits trends to all lines of data in an NDF that lie
      parallel to a chosen axis.  The trends are characterised by
      polynomials of order up to 15, or by cubic splines.  The fits can
      be restricted to use data that only lies within a series of
      co-ordinate ranges along the selected axis.

      The ranges may be determined automatically.  There is a choice
      of tunable approaches to mask regions to be excluded from the
      fitting to cater for a variety of data sets.  The actual ranges
      used are reported in the current co-ordinate Frame and pixels,
      provided they apply to all lines being fitted.

      Once the trends have been determined they can either be stored
      directly or subtracted from the input data.  If stored directly
      they can be subtracted later.  The advantage of that approach is
      the subtraction can be undone, but at some cost in efficiency.

      Fits may be rejected if their root-mean squared (rms) residuals
      are more than a specified number of standard deviations from the
      the mean rms residuals of the fits.  Rejected fits appear as
      bad pixels in the output data.

      Fitting independent trends can be useful when you need to remove
      the continuum from a spectral cube, where each spectrum is
      independent of the others (that is you need an independent
      continuum determination for each position on the sky).  It can
      also be used to de-trend individual spectra and perform functions
      like debiassing a CCD which has bias strips.
   }
   \sstusage{
      mfittrend in axis ranges out
        $\left\{ {\begin{tabular}{l}
                  order \\
                  knots=? \\
                  \end{tabular} }
        \right.$
        \newline\latexhtml{\hspace*{14.1em}}{~~~~~~~~~~~~~~~~~~~~~~~~~~~}
        \makebox[0mm][c]{\small fittype}
   }
   \sstparameters{
      \sstsubsection{
         AUTO = \_LOGICAL (Read)
      }{
         If \texttt{TRUE}, the ranges that define the trends are determined
         automatically, and Parameter RANGES is ignored.  \texttt{[FALSE]}
      }
      \sstsubsection{
         AXIS =  \htmlref{LITERAL}{se:parmenu} (Read)
      }{
         The axis of the current co-ordinate system that defines the
         direction of the trends.  This is specified using one of the
         following options.

         \ssthitemlist{

            \sstitem
            Its integer index within the \htmlref{current Frame}{se:curframe}~ of
            the input NDF (in the range 1 to the number of axes in the current
            Frame).

            \sstitem
            Its \htmlattref{Symbol}{Symbol(axis)}~ string such as
            \texttt{"RA"} or \texttt{"VRAD"}.

            \sstitem
            A generic option where \texttt{"SPEC"} requests the spectral axis,
            \texttt{"TIME"} selects the time axis, \texttt{"SKYLON"} and
            \texttt{"SKYLAT"} picks the sky longitude and latitude axes
            respectively.  Only those axis domains present are
            available as options.
         }

         A list of acceptable values is displayed if an illegal
         value is supplied.  If the axes of the current Frame are not
         parallel to the NDF pixel axes, then the pixel axis which is
         most nearly parallel to the specified current Frame axis will
         be used.  AXIS defaults to the last axis.  \texttt{[!]}
      }
      \sstsubsection{
         CLIP() = \_REAL (Read)
      }{
         Array of standard-deviation limits for progressive clipping
         of outlying binned (see NUMBIN) pixel values while determining
         the fitting ranges automatically.  It is therefore only
         applicable when AUTO=\texttt{TRUE}.  Its purpose is to exclude
         features that are not part of the trends.

         Pixels are rejected at the $i$th clipping cycle if they lie
         beyond plus or minus CLIP($i$) times the dispersion about the
         median of the remaining good pixels.  Thus lower values of CLIP
         will reject more pixels.  The normal approach is to start low
         and progressivley increase the clipping factors, as the
         dispersion decreases after the exclusion of features.  The
         source of the dispersion depends on the value the METHOD
         parameter.  Between one and five values may be supplied.
         Supplying the null value (\texttt{{!}}), results in 2, 2.5, and 3
         clipping factors.  \texttt{[2,2,2.5,3]}
      }
      \sstsubsection{
         FITTYPE = LITERAL (Read)
      }{
         The type of fit.  It must be either \texttt{"Polynomial"} for a
         polynomial or \texttt{"Spline"} for a bi-cubic B-spline.
         \texttt{["Polynomial"]}
      }
      \sstsubsection{
         FOREST = \_LOGICAL (Read)
      }{
         Set this \texttt{TRUE} if the data may contain spectral data with
         many lines---a line forest---when using the automatic range mode
         (AUTO=\texttt{TRUE}).  A different approach using the histogram
         determines the baseline mode and noise better in the presence
         of multiple lines.  This leads to improved masking of the
         spectral lines and affords a better determination of the
         baseline.  In a lineforest the ratio of baseline to line
         regions is much reduced and hence normal sigma clipping,
         when FOREST=\texttt{FALSE}, is biased.  \texttt{[FALSE]}
      }
      \sstsubsection{
         KNOTS = \_INTEGER (Read)
      }{
         The number of interior knots used for the cubic-spline fit
         along the trend axis.  Increasing this parameter value
         increases the flexibility of the surface.  KNOTS is only
         accessed when FITTYPE=\texttt{"Spline"}.  See INTERPOL for how
         the knots are arranged.  The default is the current value.

         For INTERPOL=\texttt{TRUE}, the value must be in the range 1 to
         11, and \texttt{4} is a reasonable value for flatish trends.  The
         initial default is \texttt{4}.

         For INTERPOL=\texttt{FALSE} the allowed range is 1 to 60 with an
         initial default of \texttt{8}.  In this mode, KNOTS is the
         maximum number of interior knots.

         The upper limit of acceptable values for a trend axis is no
         more than half of the axis dimension.   \texttt{[]}
      }
      \sstsubsection{
         IN = NDF (Read \& Write)
      }{
         The input NDF.  On successful completion this may have the
         trends subtracted, but only if SUBTRACT and MODIFYIN are both
         set \texttt{TRUE}.
      }
      \sstsubsection{
         INTERPOL = \_LOGICAL (Read)
      }{
         The type of spline fit to use when FITTYPE=\texttt{"Spline"}.

         If set \texttt{TRUE}, an interpolating spline is fitted by least
         squares that ensures the fit is exact at the knots.
         Therefore the knot locations may be set by the POSKNOT
         parameter.

         If set \texttt{FALSE}, a smoothing spline is fitted.  A
         smoothing factor controls the degree of smoothing.  The
         factor is determined iteratively between limits, hence it is
         the slower option of the two, but generally gives better
         fits, especially for curvy trends.  The location of of the
         knots is decided automatically by Dierckx's algorithm,
         governed where they are most needed.  Knots are added when
         the weighted sum of the squared residuals exceeds the
         smoothing factor.  A final fit is made with the chosen
         smoothing, but finding the knots afresh.

         The few iterations commence from the upper limit and progress
         more slowly at each iteration towards the lower limit.  The
         iterations continue until the residuals stabilise or the
         maximum number of interior knots is reached or the lower limit
         is reached.  The upper limit is the weighted sum of the squares
         of the residuals of the least-squares cubic polynomial fit.
         The lower limit is the estimation of the overall noise obtained
         from a clipped mean the standard deviation in short segments
         that diminish bias arising from the shape of the trend.  The
         lower limit prevents too many knots being created and fitting
         to the noise or fine features.

         The iteration to a smooth fit makes a smoothing spline somewhat
         slower.  \texttt{[FALSE]}
      }
      \sstsubsection{
         MASK = NDF (Write)
      }{
         The name of the NDF to contain the feature mask.  It is only
         accessed for automatic mode and METHOD=\texttt{"Single"} or
         \texttt{"Global"}.  It has the same bounds as the input NDF and the
         data array is type \_BYTE.  No mask NDF is created if null
         (\texttt{{!}}) is supplied.  \texttt{[!]}
      }
      \sstsubsection{
         METHOD = LITERAL (Given)
      }{
         The method used to define the masked regions in automatic
         mode.  Allowed values are as follows.

         \ssthitemlist{

            \sstitem
            \texttt{"Region"} --- This averages trend lines from a selected
            representative region given by Parameter SECTION and bins
            neighbouring elements within this average line.  Then it
            performs a linear fit upon the binned line, and rejects the
            outliers, iteratively with standard-deviation clipping
            (Parameter CLIP).  The standard deviation is that of the
            average line within the region.  The ranges are the
            intervals between the rejected points, rescaled to the
            original pixels.  They are returned in Parameter ARANGES.

            This is best suited to a low dispersion along the trend axis
            and a single concentrated region containing the bulk of the
            signal to be excluded from the trend fitting.

            \sstitem
            \texttt{"Single"} --- This is like \texttt{"Region"} except
            there is neither averaging of lines nor a single set of
            ranges.  Each line is masked independently.  The
            dispersion for the clipping of outliers within a line is
            the standard deviation within that line.

            This is more appropriate when the features being masked
            vary widely across the image, and significantly between
            adjacent lines.  Some prior smoothing or background tracing
            (\xref{CUPID}{sun255}{}: \xref{FINDBACK}{sun255}{FINDBACK})
            will usually prove beneficial.

            \sstitem
            \texttt{"Global"} --- This is a variant of \texttt{"Single"}.  The
            only difference is that the dispersion used to reject features
            using the standard deviation of the whole data array.  This
            is more robust than \texttt{"Single"}, however it does not perform
            rejections well for lines with anomalous noise.
         }

         \texttt{["Single"]}
      }
      \sstsubsection{
         MODIFYIN = \_LOGICAL (Read)
      }{
         Whether or not to subtract the trends from the input NDF.  It is only
         used when SUBTRACT is \texttt{TRUE}.  If MODIFYIN is \texttt{FALSE}, then an
         NDF name must be supplied by the OUT parameter.  \texttt{[FALSE]}
      }
      \sstsubsection{
         NUMBIN = \_INTEGER (Read)
      }{
         The number of bins in which to compress the trend line for the
         automatic range-determination mode.  A single line or even the
         average over a region will often be noisy; this compression
         creates a better signal-to-noise ratio from which to detect
         features to be excluded from the trend fitting.  If NUMBIN is
         made too large, weaker features will be lost or stronger
         features will be enlarged and background elements excluded from
         the fitting.  The minimum value is \texttt{16}, and the maximum is such
         that there will be a factor of two compression.  NUMBIN is
         ignored when there are fewer than 32 elements in each line to
         be de-trended.  \texttt{[32]}
      }
      \sstsubsection{
         ORDER = \_INTEGER (Read)
      }{
         The order of the polynomials to be used when trend fitting.
         A polynomial of order \texttt{0} is a constant and \texttt{1} a line,
         \texttt{2} a quadratic \emph{etc.}  The maximum value is \texttt{15}.
         ORDER is only accessed when FITTYPE=\texttt{"Polynomial"}.  \texttt{[3]}
      }
      \sstsubsection{
         OUT = NDF (Read)
      }{
         The output NDF containing either the difference between the
         input NDF and the various trends, or the values of the trends
         themselves.  This will not be used if SUBTRACT and MODIFYIN
         are \texttt{TRUE} (in that case the input NDF will be modified).
      }
      \sstsubsection{
         POSKNOT( ) = LITERAL (Read)
      }{
         The co-ordinates of the interior knots for all trends.  KNOTS
         values should be supplied, or just the null (\texttt{{!}}) value to
         request equally spaced knots.  The units of these co-ordinates
         is determined by the axis of the current world co-ordinate
         system of the input NDF that corresponds to the trend axis.
         Supplying a colon \texttt{":"} will display details of the current
         co-ordinate Frame.  \texttt{[!]}
      }
      \sstsubsection{
         PROPBAD = \_LOGICAL (Read)
      }{
         Only used if SUBTRACT is \texttt{FALSE}.  If PROPBAD is \texttt{TRUE},
         the returned fitted values are set bad if the corresponding
         input value is bad.  If PROPBAD is \texttt{FALSE}, the fitted
         values are retained.  \texttt{[TRUE]}
      }
      \sstsubsection{
         RANGES() = LITERAL (Read)
      }{
         Pairs of co-ordinates that define ranges along the trend
         axis.  When given these ranges are used to select the values
         that are used in the fits.  The null value (\texttt{{!}}),
         causes all the values along each data line to be used.
         The units of these ranges is determined by the axis of the
         current \htmlref{world co-ordinate system}{se:domains}~  of the
         input NDF that corresponds to the trend axis.  Supplying a
         colon \texttt{":"} will display details of the current co-ordinate
         Frame.  Up to ten pairs of values are allowed.
         This parameter is not accessed when AUTO=\texttt{TRUE}.  \texttt{[!]}
      }
      \sstsubsection{
         RMSCLIP = \_REAL (Read)
      }{
         The number of standard deviations exceeding the mean of the
         root-mean-squared residuals of the fits at which a fit is
         rejected.  A null value (\texttt{{!}}) means perform no rejections.
         Allowed values are between 2 and 15.  \texttt{[!]}
      }
      \sstsubsection{
         SECTION = LITERAL (Read)
      }{
         The region from which representative lines are averaged in
         automatic mode to determine the regions to fit trends.  It is
         therefore only accessed when AUTO=\texttt{TRUE}, METHOD=\texttt{
         "Region"}, and the dimensionality of the input NDF is more
         than 1.  The value is defined as an NDF section, so that
         ranges can be defined along any axis, and be given as pixel
         indices or axis (data) co-ordinates.  The pixel axis
         corresponding to Parameter AXIS is ignored.  So for example,
         if the pixel axis were three in a cube, the value \texttt{"3:5,4,"}
         would average all the lines in elements in Columns 3 to 5 and
         Row 4.  See \slhyperref{NDF Sections}{Section~}{}{se:ndfsect}
         for details.

         A null value (\texttt{{!}}) requests that a representative region
         around the centre be used.  \texttt{[!]}
      }
      \sstsubsection{
         SUBTRACT = \_LOGICAL (Read)
      }{
         Whether not to subtract the trends from the input NDF or not.
         If not, then the trends will be evaluated and written to a new
         NDF (see also Parameter PROPBAD).  \texttt{[FALSE]}
      }
      \sstsubsection{
         TITLE = LITERAL (Read)
      }{
         Value for the title of the output NDF.  A null value will cause
         the title of the NDF supplied for Parameter IN to be used
         instead.  \texttt{[!]}
      }
      \sstsubsection{
         VARIANCE = \_LOGICAL (Read)
      }{
         If \texttt{TRUE} and the input NDF contains variances, then the
         polynomial or spline fits will be weighted by the variances.
      }
   }
   \sstresparameters{
      \sstsubsection{
         ARANGES() = \_INTEGER (Write)
      }{
         This parameter is only written when AUTO=\texttt{TRUE}, recording the
         trend-axis fitting regions determined automatically.  They
         comprise pairs of pixel co-ordinates.
      }
   }
   \sstexamples{
      \sstexamplesubsection{
         mfittrend in=cube axis=3 ranges="1000,2000,3000,4000" order=4 out=detrend
      }{
         This example fits cubic polynomials to the spectral axis of
         a data cube.  The fits only use the data lying within the
         ranges 1000 to 2000 and 3000 to 4000 {\AA}ngstroms (assuming
         the spectral axis is calibrated in {\AA}ngstroms and that is the
         current co-ordinate system).  The fit is evaluated and
         written to the data cube called detrend.
      }
      \sstexamplesubsection{
         mfittrend in=cube axis=3 auto clip=[2,3] order=4 out=detrend
      }{
         As above except the fitting ranges are determined automatically
         with 2- then 3-sigma clipping.
      }
      \sstexamplesubsection{
         mfittrend in=cube axis=3 auto clip=[2,3] fittype=spline out=detrend interpol
      }{
         As the previous example except that interpolation cubic-spline
         fits with four equally spaced interior knots are used
         to characterise the trends.
      }
      \sstexamplesubsection{
         mfittrend m51 3 out=m51\_bsl mask=m51\_msk auto fittype=spl
      }{
         This example fits to trends along the third axis of NDF m51
         and writes the evaluated fits to NDF m51\_bsl.  The fits use a
         smoothing cubic spline with the placement and number of
         interior knots determined automatically.  Features are determined
         automatically, and a mask of excluded features is written to
         NDF m51\_msk.
      }
      \sstexamplesubsection{
         mfittrend cube axis=3 auto method=single order=1 subtract
                   out=cube\_dt mask=cube\_mask
      }{
         This fits linear trends to the spectral axis of a data cube
         called cube, masking spectral features along each line
         independently.  The mask pixels are recorded in NDF cube\_mask.
         The fitted trend are subtracted and stored in NDF cube\_dt.
      }
   }
   \sstnotes{
      \sstitemlist{

         \sstitem
         This application attempts to solve the problem of fitting
         numerous polynomials in a least-squares sense and that do not
         follow the natural ordering of the NDF data, in the most
         CPU-time-efficient way possible.

         To do this requires the use of additional memory (of order
         one fewer than the dimensionality of the NDF itself, times the
         polynomial order squared).  To minimise the use of memory and
         get the fastest possible determinations you should not use
         weighting and assert that the input data do not have any BAD
         values (use the application SETBAD to set the appropriate
         flag).

         \sstitem
         If you choose to use the automatic range determination.  You
         may need to determine empirically what are the best clipping
         limits, binning factor, and for METHOD=\texttt{"Region"} the
         region to average.

         \sstitem
         You are advised to inspect the fits, especially the spline
         fits or high-order polynomials.  A given set of trends may require
         more than one pass through this task, if they exhibit varied
         morphologies.  Use masking or NDF sections to select different
         regions that are fit with different parameters.  The various trend
         maps are then integrated with \htmlref{PASTE}{PASTE} to form the
         final composite set of trends that you can subtract.
      }
   }
   \sstdiytopic{
      Related Applications
   }{
\xref{FIGARO}{sun86}{}: \xref{FITCONT}{sun86}{FITCONT},
\xref{FITPOLY}{sun86}{FITPOLY};
\xref{CCDPACK}{sun139}{}: \xref{DEBIAS}{sun139}{DEBIAS};
KAPPA: \htmlref{SETBAD}{SETBAD}.
   }
   \sstimplementationstatus{
      \sstitemlist{

         \sstitem
         This routine correctly processes the
         \htmlref{AXIS}{apndf:axis}, DATA,
         \htmlref{QUALITY}{apndf:quality},
         \htmlref{LABEL}{apndf:label}, \htmlref{TITLE}{apndf:title},
         \htmlref{UNITS}{apndf:units},
         \htmlref{HISTORY}{apndf:history}, \htmlref{WCS}{apndf:wcs},
         and \htmlref{VARIANCE}{apndf:variance}~ components of an NDF
         data structure and propagates all
         \htmlref{extensions}{apndf:extensions}.

         \sstitem
         Processing of \htmlref{bad pixels}{se:masking} and automatic
         \htmlref{quality masking}{se:qualitymask} are supported.

         \sstitem
         All \htmlref{non-complex numeric data types}{ap:HDStypes} can
         be handled.

         \sstitem
         Handles data of up to 7 dimensions.

         \sstitem
         Huge NDFs are supported.
      }
   }
}

\sstroutine{
   MLINPLOT
}{
   Draws a multi-line plot of the data values in a two-dimensional NDF
}{
   \sstdescription{
      This application plots a set of curves giving array value against
      position in a two-dimensional \NDFref{NDF}.  All the curves are drawn within a
      single set of annotated axes.  Each curve is displaced vertically by
      a specified offset to minimise overlap between the curves.  These
      offsets may be chosen automatically or specified by the user (see
      Parameter SPACE).  The curves may be drawn in several different ways
      such as a \texttt{"join-the-dots"}~ plot, a \texttt{"staircase"}~ plot,
      a \texttt{"chain"}~ plot
      (see Parameter MODE).

      The data represented by each curve can be either a row or column
      (chosen using Parameter ABSAXS) of any array component within the
      supplied NDF (see Parameter COMP). Vertical error bars may be drawn if
      the NDF contains a VARIANCE component (see Parameter ERRBAR).  The
      vertical axis of the plot represents array value (or the logarithm of
      the array value---see Parameter YLOG).  The horizontal axis represents
      position, and may be annotated using an axis selected from the Current
      Frame of the NDF (see Parameter USEAXIS).

      Each curve may be labelled using its pixel index or a label specified
      by the user (see Parameters LINLAB and LABELS).  The appearance of these
      labels (size, colour, fount, horizontal position, \emph{etc.}) can be
      controlled using Parameter STYLE.  A key may be produced to the left
      of the main plot listing the vertical offsets of the curves (see
      Parameter KEY).  The appearance of the key may be controlled using
      Parameter KEYSTYLE.  Its position may be controlled using Parameter
      KEYOFF.  Markers indicating the zero point for each curve may also be
      drawn within the main plot (see Parameter ZMARK).

      The bounds of the plot on both axes can be specified using
      Parameters XLEFT, XRIGHT, YBOT, and YTOP.  If not specified they take
      default values which encompass the entire supplied data set.  The
      current picture is usually cleared before plotting the new picture,
      but Parameter CLEAR can be used to prevent this, allowing several
      plots to be `stacked' together.  If a new plot is drawn over an
      existing plot, then the bounds of the new plot are set automatically
      to the bounds of the existing plot (XLEFT, XRIGHT, YBOT, and YTOP are
      then ignored).
   }
   \sstusage{
      mlinplot ndf [comp] lnindx [mode] [xleft] [xright] [ybot] [ytop]
               [device]
   }
   \sstparameters{
      \sstsubsection{
         ABSAXS = \_INTEGER (Read)
      }{
         This selects whether to plot rows or columns within the NDF.
         If ABSAXS is \texttt{1}, each curve will represent the array
         values within a single row of pixels within the NDF.  If it
         is \texttt{2}, each curve will represent the array values within
         a single column of pixels within the NDF.  \texttt{[1]}
      }
      \sstsubsection{
         AXES = \_LOGICAL (Read)
      }{
         \texttt{TRUE} if labelled and annotated axes are to be drawn around the
         plot.  If a null (\texttt{{!}}) value is supplied, \texttt{FALSE} is used if the plot
         is being aligned with an existing plot (see Parameter CLEAR), and
         \texttt{TRUE} is used otherwise.  Parameters USEAXIS and YLOG determine the
         quantities used to annotated the horizontal and vertical axes
         respectively.  The width of the margins left for the annotation
         may be controlled using Parameter MARGIN.  The appearance of the
         axes (colours, founts, \emph{etc.}) can be controlled using the Parameter
         STYLE.  \texttt{[!]}
      }
      \sstsubsection{
         CLEAR = \_LOGICAL (Read)
      }{
         If \texttt{TRUE} the current picture is cleared before the plot is
         drawn.  If CLEAR is \texttt{FALSE} not only is the existing plot retained,
         but also the previous plot is used to specify the axis limits.
         \texttt{[TRUE]}
      }
      \sstsubsection{
         COMP = \htmlref{LITERAL}{se:parmenu} (Read)
      }{
         The NDF component to be plotted.  It may be \texttt{"Data"}, \texttt{"Quality"},
         \texttt{"Variance"}, or \texttt{"Error"} (where \texttt{"Error"} is an alternative to
         \texttt{"Variance"} and causes the square root of the variance values
         to be displayed).  If \texttt{"Quality"} is specified, then the quality
         values are treated as numerical values (in the range 0 to
         255).  \texttt{["Data"]}
      }
      \sstsubsection{
         DEVICE = \htmlref{DEVICE}{se:selgradev} (Read)
      }{
         The plotting device.  \texttt{[}current graphics device\texttt{{]}}
      }
      \sstsubsection{
         ERRBAR = \_LOGICAL (Read)
      }{
         \texttt{TRUE} if vertical error bars are to be drawn.  This is only
         possible if the NDF contains a VARIANCE component, and Parameter
         COMP is set to \texttt{"Data"}.  The length of the error bars (in terms of
         standard deviations) is set by Parameter SIGMA.  The appearance
         of the error bars (width, colour, \emph{etc.}) can be controlled using
         Parameter STYLE.  See also Parameter FREQ.  \texttt{[FALSE]}
      }
      \sstsubsection{
         FREQ = \_INTEGER (Read)
      }{
         The frequency at which error bars are to be plotted.  For
         instance, a value of \texttt{2} would mean that alternate points have
         error bars plotted.  This lets some plots be less cluttered.
         FREQ must lie in the range 1 to half of the number of points
         to be plotted.  FREQ is only accessed when Parameter ERRBAR is
         \texttt{TRUE}.  \texttt{[1]}
      }
      \sstsubsection{
         KEY = \_LOGICAL (Read)
      }{
         \texttt{TRUE} if a key giving the offset of each curve is to be produced.
         The appearance of this key can be controlled using Parameter
         KEYSTYLE, and its position can be controlled using Parameter
         KEYPOS.  \texttt{[TRUE]}
      }
      \sstsubsection{
         KEYPOS() = \_REAL (Read)
      }{
         Two values giving the position of the key.  The first value
         gives the gap between the right-hand edge of the
         multiple-line plot and the left-hand edge of the key (0.0 for
         no gap, 1.0 for the largest gap).  The second value gives the
         vertical position of the top of the key (1.0 for the highest
         position, 0.0 for the lowest).  If the second value is not
         given, the top of the key is placed level with the top of the
         multiple-line plot.  Both values should be in the range 0.0 to 1.0.
         If a key is produced, then the right hand margin specified by
         Parameter MARGIN is ignored.  \texttt{[}current value\texttt{{]}}
      }
      \sstsubsection{
         KEYSTYLE = \htmlref{GROUP}{se:groups} (Read)
      }{
         A group of attribute settings describing the plotting style to use
         for the key (see Parameter KEY).

         A comma-separated list of strings should be given in which each
         string is either an attribute setting, or the name of a text
         file preceded by an up-arrow character \texttt{"$\wedge$"}.  Such text files
         should contain further comma-separated lists which will be
         read and interpreted in the same manner.  Attribute settings
         are applied in the order in which they occur within the list,
         with later settings overriding any earlier settings given for
         the same attribute.

         Each individual attribute setting should be of the form:

            $<$name$>$=$<$value$>$

         where $<$name$>$ is the name of a plotting attribute, and $<$value$>$
         is the value to assign to the attribute.  Default values will be
         used for any unspecified attributes.  All attributes will be
         defaulted if a null value (\texttt{{!}})---the initial default---is supplied.
         To apply changes of style to only the current invocation, begin these
         attributes with a plus sign.  A mixture of persistent and temporary
         style changes is achieved by listing all the persistent attributes
         followed by a plus sign then the list of temporary attributes.

         See \slhyperref{Plotting Attributes}{Section~}{}{ap:plotting_attr}
         for a description of the available attributes.  Any unrecognised
         attributes are ignored (no error is reported).

         The heading in the key can be changed by setting a value for the
         Title attribute (the supplied heading is split into lines of no more
         than 17 characters).  The appearance of the heading is controlled
         by attributes \htmlattref{Colour(Title)}{Colour(element)},
         \htmlattref{Font(Title)}{Font(element)}, \emph{etc}.  The appearance of
         the curve labels is controlled by attributes
         \att{Colour(TextLab)}, {Font(TextLab)}, \emph{etc.} (the synonym
         \att{Labels} can be used in place of \htmlattref{TextLab}{TextLab(axis)}).  The
         appearance of the offset values is controlled by attributes
         \att{Colour(NumLab)}, \att{Font(NumLab)}, \emph{etc.} (the
         synonym \att{Offset} can be used in place of
         \htmlattref{NumLab}{NumLab(axis)}).  Offset values
         are formatted using attributes \att{Format(2)}, \emph{etc.} (the
         synonym \att{Offset} can be used in place of the value 2).
         \texttt{[}current value\texttt{{]}}
      }
      \sstsubsection{
         LABELS = LITERAL (Read)
      }{
         A group of strings with which to label the plotted curves.  A
         comma-separated list of strings should be given, or the name
         of a text file preceded by an up-arrow character \texttt{"$\wedge$"}.  Such text
         files should contain further comma-separated lists which will be
         read and interpreted in the same manner.  The first string
         obtained is used as the label for the first curve requested
         using Parameter LNINDX, the second string is used as the label
         for the second curve, \emph{etc}.  If the number of supplied strings is
         less than the number of curves requested using LNINDX, then
         extra default labels are used.  These are equal to the NDF pixel
         index of the row or column, preceded by a hash character (\texttt{"\#"}).
         If a null (\texttt{{!}}) value is supplied for LABELS, then default labels are
         used for all curves.  \texttt{[!]}
      }
      \sstsubsection{
         LINLAB = \_LOGICAL (Read)
      }{
         If \texttt{TRUE}, the curves in the plot will be labelled using the labels
         specified by Parameter LABELS.  A single label is placed in-line
         with the curve.  The horizontal position and appearance of these
         labels can be controlled using Parameter STYLE.  \texttt{[TRUE]}
      }
      \sstsubsection{
         LNINDX = LITERAL (Read)
      }{
         Specifies the NDF pixel indices of the rows or columns to be
         displayed (see Parameter ABSAXS).  A maximum of 100 lines may be
         selected.  It can take any of the following values.

         \ssthitemlist{

            \sstitem
            \texttt{"ALL"} or \texttt{"$*$"} ---  All lines (rows or columns).

            \sstitem
            \texttt{"xx,yy,zz"} --- A list of line indices.

            \sstitem
            \texttt{"xx:yy"} ---  Line indices between \textit{xx} and
            \textit{yy} inclusively.  When
            \textit{xx} is omitted the range begins from the lower bound of the line
            dimension; when \textit{yy} is omitted the range ends with the maximum
            value it can take, that is the upper bound of the line dimension
            or the maximum number of lines this routine can plot.

            \sstitem
            Any reasonable combination of above values separated by commas.
         }
      }
      \sstsubsection{
         MARGIN( 4 ) = \_REAL (Read)
      }{
         The widths of the margins to leave around the multiple-line plot for
         axis annotation.  The widths should be given as fractions of the
         corresponding dimension of the current picture.  Four values may be
         given, in the order; bottom, right, top, left.  If fewer than four
         values are given, extra values are used equal to the first supplied
         value.  If these margins are too narrow any axis annotation may be
         clipped.  See also Parameter KEYPOS.  \texttt{[}current value\texttt{{]}}
      }
      \sstsubsection{
         MARKER = \_INTEGER (Read)
      }{
         This parameter is only accessed if Parameter MODE is set to
         \texttt{"Chain"} or \texttt{"Mark"}.  It specifies the symbol with which each
         position should be marked, and should be given as an integer
         \PGPLOT\  marker type.  For instance, \texttt{0} gives a box,
         \texttt{1} gives a dot, \texttt{2} gives a cross, \texttt{3} gives an asterisk,
         \texttt{7} gives a triangle.  The value must be larger than or equal
         to $-$31.  \texttt{[}current value\texttt{{]}}
      }
      \sstsubsection{
         MODE = \htmlref{LITERAL}{se:parmenu} (Read)
      }{
         Specifies the way in which each curve is drawn.  MODE can take the
         following values.

         \ssthitemlist{

            \sstitem
            \texttt{"Histogram"} --- An histogram of the points is plotted in the
            style of a `staircase' (with vertical lines only joining the \textit{y}
            values and not extending to the base of the plot).  The vertical
            lines are placed midway between adjacent \textit{x} positions.

            \sstitem
            \texttt{"Line"} --- The points are joined by straight lines.

            \sstitem
            \texttt{"Point"} --- A dot is plotted at each point.

            \sstitem
            \texttt{"Mark"} --- Each point is marker with a symbol specified by
            Parameter MARKER.

            \sstitem
            \texttt{"Chain"} --- A combination of \texttt{"Line"} and \texttt{"Mark"}.

         }
         \texttt{[}current value\texttt{{]}}
      }
      \sstsubsection{
         NDF = NDF (Read)
      }{
         NDF structure containing the array to be plotted.
      }
      \sstsubsection{
         OFFSET() = \_DOUBLE (Read)
      }{
         This parameter is used to obtain the vertical offsets for the data
         curve when Parameter SPACE is given the value \texttt{"Free"}.  The number
         of values supplied should equal the number of curves being drawn.
      }
      \sstsubsection{
         PENS = \htmlref{GROUP}{se:groups} (Read)
      }{
         A group of strings, separated by semicolons, each of which specifies
         the appearance of a pen to be used to draw a curve.  The first
         string in the group describes the pen to use for the first curve,
         the second string describes the pen for the second curve, \emph{etc}.  If
         there are fewer strings than curves, then the supplied pens are
         cycled through again, starting at the beginning.  Each string should
         be a comma-separated list of plotting attributes to be used when drawing
         the curve.  For instance, the string \texttt{"width=0.02,colour=red,style=2"}
         produces a thick, red, dashed curve.  Attributes which are
         unspecified in a string default to the values implied by Parameter
         STYLE.  I f a null value (\texttt{{!}}) is given for PENS, then the pen
         attributes implied by Parameter STYLE are used.  \texttt{[!]}
      }
      \sstsubsection{
         SIGMA = LITERAL (Read)
      }{
         If vertical error bars are produced (see Parameter ERRBAR), then
         SIGMA gives the number of standard deviations that the error
         bars are to represent.  \texttt{[}current value\texttt{{]}}
      }
      \sstsubsection{
         SPACE = LITERAL (Read)
      }{
         The value of this parameter specifies how the vertical offset for
         each data curve is determined.  It should be given one of
         the following values:

         \ssthitemlist{

            \sstitem
            \texttt{"Average"} --- The offsets are chosen automatically so that
            the average data values of the curves are evenly spaced between
            the upper and lower limits of the plotting area.  Any line-
            to-line striping is thus hidden and the amount of overlap of
            adjacent traces is minimised.

            \sstitem
            \texttt{"Constant"} --- The offsets are chosen automatically so that
            the zero points of the curves are evenly spaced between the upper
            and lower limits of the plotting area.  The width of any line-
            to-line strip is constant, which could result in the curves
            becoming confused if the bias of a curve from its zero point is
            so large that it overlaps another curve.

            \sstitem
            \texttt{"Free"} --- The offsets to use are obtained explicitly using
            Parameter OFFSET.

            \sstitem
            \texttt{"None"} --- No vertical offsets are used.  All curves are
            displayed with the same zero point.

         }
         The input can be abbreviated to an unambiguous length and
         is case insensitive.  \texttt{["Average"]}
      }
      \sstsubsection{
         STYLE = \htmlref{GROUP}{se:groups} (Read)
      }{
         A group of attribute settings describing the plotting style to use
         when drawing the annotated axes, data curves, error bars, zero
         markers, and curve labels.

         A comma-separated list of strings should be given in which each
         string is either an attribute setting, or the name of a text
         file preceded by an up-arrow character \texttt{"$\wedge$"}.  Such text files
         should contain further comma-separated lists which will be
         read and interpreted in the same manner.  Attribute settings
         are applied in the order in which they occur within the list,
         with later settings overriding any earlier settings given for
         the same attribute.

         Each individual attribute setting should be of the form:

            $<$name$>$=$<$value$>$

         where $<$name$>$ is the name of a plotting attribute, and $<$value$>$
         is the value to assign to the attribute.  Default values will be
         used for any unspecified attributes.  All attributes will be
         defaulted if a null value (\texttt{{!}})---the initial default---is supplied.
         To apply changes of style to only the current invocation, begin these
         attributes with a plus sign.  A mixture of persistent and temporary
         style changes is achieved by listing all the persistent attributes
         followed by a plus sign then the list of temporary attributes.

         See \slhyperref{Plotting Attributes}{Section~}{}{ap:plotting_attr}
         for a description of the available attributes.  Any unrecognised
         attributes are ignored (no error is reported).

         The appearance of the data curves is controlled by the attributes
         \htmlattref{Colour(Curves)}{Colour(element)},
         \htmlattref{Width(Curves)}{Width(element)}, \emph{etc.} (the synonym
         \att{Lines} may be used in place of \att{Curves}).  The appearance
         of markers used if Parameter MODE is set to \texttt{"Point"}, \texttt{"Mark"}
         or \texttt{"Chain"} is controlled by \att{Colour(Markers)},
         \att{Width(Markers)}, \emph{etc.} (the synonym \att{Symbols} may
         be used in place of \att{Markers}).  The appearance of the
         error bars is  controlled using \att{Colour(ErrBars)},
         \att{Width(ErrBars)}, \emph{etc.} (see Parameter ERRBAR).  The appearance of
         the zero-point markers is controlled using \att{Colour(ZeroMark)},
         \att{Size(ZeroMark}),  \emph{etc}.  The appearance of the curve labels is
         controlled using \att{Colour(Labels)}, \att{Size(Labels)}, \emph{etc}.
         \att{LabPos(Left)} controls the horizontal position
         of the in-line curve label (see Parameter LINLAB), and
         \att{LabPos(Right)} controls the horizontal position of the curve
         label associated with the right-hand zero-point marker (see
         Parameter ZMARK).  \att{LabPos} without any qualifier is equivalent to
         \att{LabPos(Left)}.  \att{LabPos} values are floating point, with
         \texttt{0.0} meaning
         the left edge of the plotting area, and \texttt{1.0} the right edge. Values
         outside the range 0 to 1 may be used.  \texttt{[}current value\texttt{{]}}
      }
      \sstsubsection{
         USEAXIS = LITERAL (Read)
      }{
         The quantity to be used to annotate the horizontal axis
         of the plot specified by using one of the following options.

         \ssthitemlist{

            \sstitem
            An integer index of an axis within the current \htmlref{co-ordinate
            Frame}{se:domains}~ of the input NDF (in the range 1 to the number
            of axes in the current Frame).

            \sstitem
            An axis \htmlattref{Symbol}{Symbol(axis)}~ string such as
            \texttt{"RA"} or \texttt{"VRAD"}.

            \sstitem
            A generic option where \texttt{"SPEC"} requests the spectral
            axis, \texttt{"TIME"} selects the time axis, \texttt{"SKYLON"}
            and \texttt{"SKYLAT"} picks the sky longitude and latitude
            axes respectively.  Only those axis domains present are
            available as options.
         }

         The quantity used to annotate the horizontal axis must have a
         defined value at all points in the array, and must increase or
         decrease monotonically along the array.  For instance, if RA is
         used to annotate the horizontal axis, then an error will be
         reported if the profile passes through RA=0 because it will
         introduce a non-monotonic jump in axis value (from 0h to 24h, or
         24h to 0h).  If a null (\texttt{{!}}) value is supplied, the value
         of Parameter ABSAXS is used.  \texttt{[!]}
      }
      \sstsubsection{
         XLEFT = LITERAL (Read)
      }{
         The axis value to place at the left-hand end of the horizontal
         axis.  If a null (\texttt{{!}}) value is supplied, the value used is the first
         element in the data being displayed.  The value supplied may be
         greater than or less than the value supplied for XRIGHT.  A formatted
         value for the quantity specified by Parameter USEAXIS should be
         supplied.  \texttt{[!]}
      }
      \sstsubsection{
         XRIGHT = LITERAL (Read)
      }{
         The axis value to place at the right-hand end of the horizontal
         axis.  If a null (\texttt{{!}}) value is supplied, the value used is the
         last element in the data being displayed.  The value supplied may be
         greater than or less than the value supplied for XLEFT.  A formatted
         value for the quantity specified by Parameter USEAXIS should be
         supplied.  \texttt{[!]}
      }
      \sstsubsection{
         YBOT = \_DOUBLE (Read)
      }{
         The data value to place at the bottom end of the vertical axis.
         If a null (\texttt{{!}}) value is supplied, the value used is the lowest data
         value to be displayed, after addition of the vertical offsets.  The
         value supplied may be greater than or less than the value supplied
         for YTOP.  \texttt{[!]}
      }
      \sstsubsection{
         YLOG = \_LOGICAL (Read)
      }{
         \texttt{TRUE} if the value displayed on the vertical axis is to be the
         logarithm of the supplied data values.  If \texttt{TRUE}, then the values
         supplied for Parameters YTOP and YBOT should be values for the
         logarithm of the data value, not the data value itself.  \texttt{[FALSE]}
      }
      \sstsubsection{
         YTOP = \_DOUBLE (Read)
      }{
         The data value to place at the top end of the vertical axis.
         If a null (\texttt{{!}}) value is supplied, the value used is the highest data
         value to be displayed, after addition of the vertical offsets.  The
         value supplied may be greater than or less than the value supplied
         for YBOT.  \texttt{[!]}
      }
      \sstsubsection{
         ZMARK = \_LOGICAL (Read)
      }{
         If \texttt{TRUE}, then a pair of short horizontal lines are drawn at the left
         and right edges of the main plot for each curve.  The vertical
         position of these lines corresponds to the zero point for the
         corresponding curve.  The right-hand marker is annotated with the
         curve label (see Parameter LABELS).  The appearance of these
         markers can be controlled using the Parameter STYLE.  \texttt{[TRUE]}
      }
   }
   \sstexamples{
      \sstexamplesubsection{
         mlinplot rcw3\_b1 reset $\backslash$
      }{
         Plot the first five rows of the two-dimensional NDF file,
         rcw3\_b1 on the \htmlref{current graphics device}{se:devglobal}.  The lines are offset
         such that the averages of the rows are evenly separated in the
         direction of the vertical axis.
      }
      \sstexamplesubsection{
         mlinplot rcw3\_b1 lnindx="1,3,5,7:10" $\backslash$
      }{
         Plot the rows 1, 3, 5, 7, 8, 9 and 10 of the two-dimensional
         NDF file, rcw3\_b1, on the current graphics device.
      }
      \sstexamplesubsection{
         mlinplot rcw3\_b1 lnindx=$*$ $\backslash$
      }{
         Plot all rows of the two-dimensional NDF file, rcw3\_b1, on the
         current graphics device.
      }
      \sstexamplesubsection{
         mlinplot rcw3\_b1 lnindx=* style="colour(curve)=red+width(curve)=4" \
      }{
         As the previous example, but the rows are drawn in red at four
         times normal thickness.  The change of line coluor persists
         to the next invocation, but not the temporary widening of the
         lines.
      }
      \sstexamplesubsection{
         mlinplot rcw3\_b1 lnindx=* style="+width(curve)=4" \
      }{
         As the previous example, but now the rows are drawn in the
         current line colour.
      }
      \sstexamplesubsection{
         mlinplot rcw3\_b1 absaxs=2 lnindx="20:25,30,31" $\backslash$
      }{
         Plot columns 20, 21, 22, 23, 24, 25, 30 and 31 of the
         two-dimensional NDF file, rcw3\_b1, on the current graphics device.
      }
      \sstexamplesubsection{
         mlinplot rcw3\_b1 style="Title=CRDD rcw3\_b1" $\backslash$
      }{
         Plot the currently selected rows of the two-dimensional NDF
         file, rcw3\_b1, on the current graphics device.  The plot has a
         title of \texttt{"CRDD rcw3\_b1"}.
      }
      \sstexamplesubsection{
         mlinplot rcw3\_b1(100:500,) ybot=0.0 ytop=1.0E-3 $\backslash$
      }{
         Plot the currently selected rows of the two-dimensional NDF, rcw3\_b1,
         between column 100 and column 500.  The vertical display range is
         from 0.0 to 1.0E-3.
      }
      \sstexamplesubsection{
         mlinplot rcw3\_b1 space=constant device=ps\_p $\backslash$
      }{
         Plot the currently selected rows of the two-dimensional NDF
         file, rcw3\_b1, on the ps\_p device.  The base lines are evenly
         distributed over the range of the vertical axis.
      }
      \sstexamplesubsection{
         mlinplot rcw3\_b1 space=free offset=[0.,2.0E-4,4.0E-4,6.0E-4,0.1] $\backslash$
      }{
         Plot the currently selected rows of the two-dimensional NDF
         file, rcw3\_b1.  The base lines are set at 0.0 for the first row,
         2.0E-4 for the second, 4.0E-4 for the third, 6.0E-4 for the fourth,
         and 0.1 for the fifth.
      }
   }
   \sstnotes{
      \sstitemlist{

         \sstitem
         If no \htmlattref{Title}{plotel:Title}~ is specified via the
         STYLE parameter, then the \htmlref{TITLE}{apndf:title}
         component in the NDF is used as the default title for the
         annotated axes.  Should the NDF not have a TITLE component,
         then the default title is instead taken from current
         co-ordinate Frame in the NDF, unless this attribute has not
         been set explicitly, whereupon the name of the NDF is used as
         the default title.

         \sstitem
         The application stores a number of pictures in the graphics
         database in the following order: a FRAME picture containing the
         annotated axes, data plot, and optional key; a KEY picture to store
         the key if present; and a DATA picture containing just the data plot.
         Note, the FRAME picture is only created if annotated axes or a key
         has been drawn, or if non-zero margins were specified using Parameter
         MARGIN.
      }
   }
   \sstdiytopic{
      Related Applications
   }{
KAPPA: \htmlref{CLINPLOT}{CLINPLOT},
\htmlref{LINPLOT}{LINPLOT};
\xref{FIGARO}{sun86}{}: \xref{ESPLOT}{sun86}{ESPLOT},
\xref{IPLOTS}{sun86}{IPLOTS},
\xref{MSPLOT}{sun86}{MSPLOT},
\xref{SPLOT}{sun86}{SPLOT},
\xref{SPECGRID}{sun86}{SPECGRID};
\xref{SPLAT}{sun243}{}.
   }
   \sstimplementationstatus{
      \sstitemlist{

         \sstitem
         This routine correctly processes the \htmlref{AXIS}{apndf:axis}, DATA, \htmlref{VARIANCE}{apndf:variance},
         \htmlref{QUALITY}{apndf:quality}, \htmlref{LABEL}{apndf:label}, \htmlref{TITLE}{apndf:title}, \htmlref{WCS}{apndf:wcs}, and \htmlref{UNITS}{apndf:units}~ components of the NDF.

         \sstitem
         Processing of \htmlref{bad pixels}{se:masking} and automatic \htmlref{quality masking}{se:qualitymask} are
         supported.

         \sstitem
         All \htmlref{non-complex numeric data types}{ap:HDStypes} can be handled.  Only
         double-precision floating-point data can be processed directly.
         Other non-complex data types will undergo a type conversion before the plot is drawn.
      }
   }
}

\sstroutine{
   MOCGEN
}{
   Creates a Multi-Order Coverage map describing regions of an image
}{
   \sstdescription{
      This application creates a Multi-Order Coverage (MOC) map describing
      selected regions of the sky, using the scheme described in Version
      1.1 of the \htmladdnormallink{MOC recommendation}{http://ivoa.net/documents/MOC/}
      published by the International Virtual Observatory Alliance (IVOA).

      The regions of sky to be included in the MOC may be specified in
      several ways (see Parameter MODE).
   }
   \sstusage{
      mocgen in out mode
   }
   \sstparameters{
      \sstsubsection{
         COMP = LITERAL (Read)
      }{
         The NDF component to be used if Parameter MODE is \texttt{"Good"}  or
         \texttt{"Bad"}.  It may be \texttt{"Data"}  or \texttt{"Variance"}. \texttt{["Data"]}
      }
      \sstsubsection{
         FORMAT = LITERAL (Read)
      }{
         The format to use when generating the output MOC specified by
         Parameter OUT.

         \sstitemlist{

            \sstitem
             \texttt{"FITS"}  --- The output MOC is stored as a binary-table extension
            in a FITS file, using the conventions described in Version 1.1
            of the IVOA's MOC recommendation.

            \sstitem
             \texttt{"AST"}  --- The output MOC is stored as a text file using native
            AST encoding.

            \sstitem
             \texttt{"String"}  --- The output MOC is stored as a text file using the
            \texttt{"string"}  encoding described in Version 1.1 of the IVOA's MOC
            recommendation.

            \sstitem
             \texttt{"JSON"}  --- The output MOC is stored as a text file using the
            JSON encoding described in Version 1.1 of the IVOA's MOC
            recommendation.

         }
         \texttt{["FITS"]}
      }
      \sstsubsection{
         IN = NDF (Read)
      }{
         The input NDF. Must be two-dimensional.
      }
      \sstsubsection{
         MAXRES = \_REAL (Read)
      }{
         The size of the smallest cells in the returned MOC, in
         arcseconds. The nearest of the valid values defined in the MOC
         recommendation is used. The default value is the largest legal
         value that results in the cells in the MOC being no larger than
         the size of the pixels in the centre of the supplied NDF. [!]
      }
      \sstsubsection{
         MINRES = \_REAL (Read)
      }{
         The size of the largest feature that may be missed in the
         supplied NDF, in arcseconds. It gives the resolution of the
         initial grid used to identify areas that are inside the MOC.
         Bounded 'holes'  or 'islands'  in the NDF that are smaller than
         one cell of this initial grid may be missed (i.e. such holes may
         be 'filled in' and islands omitted in the resulting MOC). The
         default value is 16 times Parameter MaxRes. [!]
      }
      \sstsubsection{
         MODE = LITERAL (Read)
      }{
         The mode used to specify the sky regions to include in the
         output MOC.

         \sstitemlist{

            \sstitem
             \texttt{"Good"}  --- The output MOC contains the good pixels in the input
            NDF specified by Parameter IN.

            \sstitem
             \texttt{"Bad"}  --- The output MOC contains the bad pixels in the input
            NDF specified by Parameter IN.

            \sstitem
             \texttt{"Qual"}  --- The output MOC contains the pixels that have the
            quality specified by Parameter QEXP within the input NDF
            specified by Parameter IN.

         }
         \texttt{["Good"]}
      }
      \sstsubsection{
         OUT = FILENAME (Write)
      }{
         Name of the file in which to store the MOC description of the
         selected regions. The format to use is specified by Parameter
         FORMAT. If Parameter FORMAT is \texttt{"FITS"}, \texttt{".fits"}  is
         appended to the supplied file name, provided no other file type is
         included in the supplied string.
      }
      \sstsubsection{
         QEXP = LITERAL (Read)
      }{
         The quality expression. Only used if Parameter MOD is \texttt{"QUAL"}.
      }
      \sstsubsection{
         USEAXIS = \htmlref{GROUP}{se:groups} (Read)
      }{
         USEAXIS is only accessed if the current co-ordinate Frame of the
         NDF has too many axes.  A group of strings should be supplied
         specifying the axes which are to be used.  Each axis can
         be specified using one of the following options.

         \ssthitemlist{

            \sstitem
            Its integer index within the current Frame of the
            input  NDF (in the range 1 to the number of axes in the
            current Frame).

            \sstitem
            Its \htmlattref{Symbol}{Symbol(axis)}~ string such as
            \texttt{"RA"} or \texttt{"VRAD"}.

            \sstitem
            A generic option where \texttt{"SPEC"} requests the spectral axis,
            \texttt{"TIME"} selects the time axis, \texttt{"SKYLON"} and
            \texttt{"SKYLAT"} picks the sky longitude and latitude axes
            respectively.  Only those axis domains present are
            available as options.
         }

         A list of acceptable values is displayed if an illegal value is
         supplied.  If a null (\texttt{{!}}) value is supplied, the
         axes with the same indices as the two used pixel axes within the
         NDF are used.  \texttt{[!]}
      }
   }
   \sstexamples{
      \sstexamplesubsection{
         mocgen m31 m31.fits
      }{
         Generates a a MOC description of the good pixels in NDF \texttt{"m31"},
         storing the MOC as a binary table in FITS file \texttt{"m31.fits"}.
      }
   }
   \sstdiytopic{
      Related Applications
   }{
      KAPPA: \htmlref{REGIONMASK}{REGIONMASK}
   }
}

\sstroutine{
   MSTATS
}{
   Calculate statistics over a group of data arrays or points
}{
   \sstdescription{
      This application calculates cumulative statistics over a group
      of \NDFref{NDFs}.  It can either generate the statistics of each
      corresponding pixel in the input array components and output
      a new NDF with array components containing the result, or
      calculate statistics at a single point specified in the
      current \htmlref{co-ordinate Frame}{se:domains}~  of the input NDFs.

      In array mode (SINGLE=\texttt{FALSE}), statistics are calculated for each
      pixel in one of the array components (DATA, VARIANCE or QUALITY)
      accumulated over all the input NDFs and written to an output
      NDF; each pixel of the output NDF is a result of combination
      of pixels with the same Pixel co-ordinates in all the input NDFs.
      There is a selection of statistics available to form the output
      values.

      The input NDFs must all have the same number of dimensions, but
      need not all be the same shape.  The shape of the output NDF can
      be set to either the intersection or the union of the shapes of
      the input NDFs using the TRIM parameter.

      In single pixel mode (SINGLE=\texttt{TRUE}) a position in the current
      co-ordinate Frame of all the NDFs is given, and the value at
      the pixel covering this point in each of the input NDFs is
      accumulated to form the results that comprise the mean, variance,
      and median.  These statistics, and if environment variable
      MSG\_FILTER is set to \texttt{VERBOSE}, the value of each contributing
      pixel, is reported directly to you.
   }
   \sstusage{
      mstats in out [estimator]
   }
   \sstparameters{
      \sstsubsection{
         CLIP = \_REAL (Read)
      }{
         The number of standard deviations about the mean at which to
         clip outliers for the \texttt{"Mode"}, \texttt{"Cmean"} and \texttt{"Csigma"}
         statistics (see Parameter ESTIMATOR).  The application first computes
         statistics using all the available pixels.  It then rejects
         all those pixels whose values lie beyond CLIP standard
         deviations from the mean and will then re-evaluate the
         statistics.  For \texttt{"Cmean"} and \texttt{"Csigma"} there is currently
         only one iteration, but up to seven for "Mode".

         The value must be positive.  \texttt{[3.0]}
      }

      \sstsubsection{
         COMP = \htmlref{LITERAL}{se:parmenu} (Read)
      }{
         The NDF array component to be analysed.  It may be \texttt{"Data"},
         \texttt{"Quality"}, \texttt{"Variance"}, or \texttt{"Error"} (where \texttt{"Error"} is an
         alternative to \texttt{"Variance"} and causes the square root of the
         variance values to be used).  If \texttt{"Quality"} is specified,
         then the quality values are treated as numerical values (in
         the range 0 to 255).  In cases other than \texttt{"Data"}, which is
         always present, a missing component will be treated as having
         all pixels set to the `bad' value.  \texttt{["Data"]}
      }
      \sstsubsection{
         ESTIMATOR = LITERAL (Read)
      }{
         The method to use for estimating the output pixel values.  It
         can be one of the following options.  The first four are
         more for general collapsing, and the remainder are for cube
         analysis.

         \ssthitemlist{

            \sstitem
            \texttt{"Mean"} --- Mean value

            \sstitem
            \texttt{"WMean"}  --- Weighted mean in which each data value is weighted
                        by the reciprocal of the associated variance.  (2)

            \sstitem
            \texttt{"Mode"}   --- Modal value.  (4)

            \sstitem
            \texttt{"Median"} --- Median value.  Note that this is extremely memory
                        and CPU intensive for large datasets; use with
                        care!  If strange things happen, use \texttt{"Mean"}.  (3) \\

            \sstitem
            \texttt{"Absdev"} --- Mean absolute deviation from the unweighted mean.  (2)

            \sstitem
            \texttt{"Cmean"}  --- Sigma-clipped mean.  (4)

            \sstitem
            \texttt{"Csigma"} --- Sigma-clipped standard deviation.  (4)

            \sstitem
            \texttt{"Comax"}  --- Co-ordinate of the maximum value.

            \sstitem
            \texttt{"Comin"}  --- Co-ordinate of the minimum value.

            \sstitem
            \texttt{"FBad"}   --- Fraction of bad pixel values.

            \sstitem
            \texttt{"FGood"}  --- Fraction of good pixel values.

            \sstitem
            \texttt{"Integ"}  --- Integrated value, being the sum of the products
                        of the value and pixel width in world co-ordinates.  Note
                        that for sky co-ordinates the width is measured in radians.

            \sstitem
            \texttt{"Iwc"}    --- Intensity-weighted co-ordinate, being the sum of
                        each value times its co-ordinate, all divided by
                        the integrated value (see the \texttt{"Integ"} option).

            \sstitem
            \texttt{"Iwd"}    --- Intensity-weighted dispersion of the
                        co-ordinate, normalised like \texttt{"Iwc"} by the
                        integrated value.  (4)

            \sstitem
            \texttt{"Max"}    --- Maximum value.

            \sstitem
            \texttt{"Min"}    --- Minimum value.

            \sstitem
            \texttt{"FBad"}   --- Fraction of bad pixel values.

            \sstitem
            \texttt{"FGood"}  --- Fraction of good pixel values.

            \sstitem
            \texttt{"NBad"}   --- Count of bad pixel values.

            \sstitem
            \texttt{"NGood"}  --- Count of good pixel values.

            \sstitem
            \texttt{"Rms"}    --- Root-mean-square value.  (4)

            \sstitem
            \texttt{"Sigma"}  --- Standard deviation about the unweighted mean.  (4)

            \sstitem
            \texttt{"Sum"}    --- The total value.
         }

         Where needed, the co-ordinates are the indices of the input
         NDFs in the supplied order.  Thus the calculations behave like
         the NDFs were stacked one upon another to form an extra axis,
         and that axis had GRID co-ordinates.  Care using wildcards is
         necessary, to achieve a specific order, say for a time series,
         and hence assign the desired co-ordinate for a each NDF.
         Indirection through a text file is recommended.

         The selection is restricted if there are only a few input NDFs. For
         instance, measures of dispersion like \texttt{"Sigma"} and \texttt{"Iwd"}
         are meaningless for combining only two NDFs.  The minimum number of
         input NDFs for each estimator is given in parentheses in the list
         above.  Where there is no number, there is no restriction.  If you
         supply an unavailable option, you will be informed, and presented
         with the available options.   \texttt{["Mean"]}
      }
      \sstsubsection{
         IN = \htmlref{GROUP}{se:groups} (Read)
      }{
         A group of input NDFs.  They may have different shapes, but
         must all have the same number of dimensions.  This should
         be given as a comma-separated list, in which each list element
         can be one of the following.

         \ssthitemlist{

            \sstitem
            An NDF name, optionally containing wild-cards and/or regular
            expressions (\texttt{"$*$"}, \texttt{"?"}, \texttt{"[a-z]"} \emph{etc.});

            \sstitem
            the name of a text file, preceded by an up-arrow character \texttt{"$\wedge$"}.
            Each line in the text file should contain a comma-separated list
            of elements, each of which can in turn be an NDF name (with
            optional wild-cards, \emph{etc}), or another file specification
            (preceded by an up-arrow).  Comments can be included in the file
            by commencing lines with a hash character \texttt{"\#"}.

         }
         If the value supplied for this parameter ends with a minus
         sign \texttt{"-"}, then the user is re-prompted for further input until
         a value is given which does not end with a minus sign.  All the
         images given in this way are concatenated into a single group.
      }
      \sstsubsection{
         OUT = NDF (Read)
      }{
         The name of an NDF to receive the results.  Each pixel of
         the DATA (and perhaps VARIANCE) component represents the
         statistics of the corresponding pixels of the input NDFs.
         Only used if SINGLE=\texttt{FALSE}.
      }
      \sstsubsection{
         POS = LITERAL (Read)
      }{
         In Single pixel mode (SINGLE=\texttt{TRUE}), this parameter gives the
         position in the current co-ordinate Frame at which the
         statistics should be calculated (supplying a colon \texttt{":"} will
         display details of the required co-ordinate Frame).  The
         position should be supplied as a list of
         \xref{formatted axis values}{sun210}{AST_UNFORMAT}
         separated by spaces or commas.  The pixel covering this point
         in each input array, if any, will be used.
      }
      \sstsubsection{
         SINGLE = \_LOGICAL (Read)
      }{
         Whether the statistics should be calculated in Single pixel
         mode or Array mode.  If SINGLE=\texttt{TRUE}, then the POS parameter
         will be used to get the point to which the statistics refer,
         but if SINGLE=\texttt{FALSE} an output NDF will be generated containing
         the results for all the pixels.  \texttt{[FALSE]}
      }
      \sstsubsection{
         TITLE = LITERAL (Read)
      }{
         Title for the output NDF.  \texttt{["KAPPA - Mstats"]}
      }
      \sstsubsection{
         TRIM = \_LOGICAL (Read)
      }{
         This parameter controls the shape of the output NDF.
         If TRIM=\texttt{TRUE}, then the output NDF is the shape of the
         intersection of all the input NDFs, \emph{i.e.} only pixels which
         appear in all the input arrays will be represented in the output.
         If TRIM=\texttt{FALSE}, the output is the shape of the union of the
         inputs, \emph{i.e.} every pixel which appears in the input arrays
         will be represented in the output.  \texttt{[TRUE]}
      }
      \sstsubsection{
         VARIANCE = \_LOGICAL (Read)
      }{
         A flag indicating whether a variance array present in the
         NDF is used to weight the array values while forming the
         estimator's statistic, and to derive output variance.  If
         VARIANCE is \texttt{TRUE} and all the input NDFs contain a variance
         array, this array will be used to define the weights, otherwise
         all the weights will be set equal.  \texttt{[TRUE]}
      }
      \sstsubsection{
         WLIM = \_REAL (Read)
      }{
         If the input NDFs contain bad pixels, then this parameter
         may be used to determine at a given pixel location the number
         of good pixels which must be present within the input NDFs
         before a valid output pixel is generated.  It can be used, for
         example, to prevent output pixels from being generated in
         regions where there are relatively few good pixels to
         contribute to the result of combining the input NDFs.
      }
   }
   \sstresparameters{
      \sstsubsection{
         MEAN = \_DOUBLE (Write)
      }{
         The mean pixel value, if SINGLE=\texttt{TRUE}.
      }
      \sstsubsection{
         MEDIAN = \_DOUBLE (Write)
      }{
         The median pixel value, if SINGLE=\texttt{TRUE}.
      }
      \sstsubsection{
         VAR = \_DOUBLE (Write)
      }{
         The variance of the pixel values, if SINGLE=\texttt{TRUE}.
      }
   }
   \sstexamples{
      \sstexamplesubsection{
         mstats idat$*$ ostats
      }{
         This calculates the mean of each pixel in the
         Data arrays of all the NDFs in the current directory with names
         which start \texttt{"idat"}, and writes the result in a new NDF called
         ostats.  The shape of ostats will be the intersection of the
         volumes of all the indat$*$ NDFs.
      }
      \sstexamplesubsection{
         mstats idat$*$ ostats trim=false
      }{
         This does the same as the previous example, except that the
         output NDF will be the `union' of the volumes of the input
         NDFs, that is a cuboid with lower bounds as low as the
         lowest pixel bound of the input NDFs in each dimension and
         with upper bounds as high as the highest pixel bound in
         each dimension.
      }
      \sstexamplesubsection{
         mstats idat$*$ ostats variance
      }{
         This is like the first example except variance information
         present is used to weight the data values.
      }
      \sstexamplesubsection{
         mstats idat$*$ ostats comp=variance variance
      }{
         This does the same as the first example except that statistics
         are calculated on the VARIANCE components of all the input
         NDFs.  Thus the pixels of the VARIANCE component of ostats
         will be the variances of the variances of the input data.
      }
      \sstexamplesubsection{
         mstats m31$*$ single=true pos="0:42:38,40:52:20"
      }{
         This example is analysing the pixel brightness at the indicated
         sky position in a number of NDFs whose name start with \texttt{"m31"},
         which all have SKY as their current co-ordinate Frame.
         The mean and variance of the pixels at that position in all
         the NDFs are printed to the screen.  If the reporting level is
         verbose, the command also prints the value of the sampled pixel
         in each of the NDFs.  For those in which the pixel at the
         selected position is bad or falls outside the NDF, this is also
         indicated.
      }
      \sstexamplesubsection{
         mstats in="arr1,arr2,arr3" out=middle estimator=median wlim=1.0
      }{
         This example calculates the medians of the DATA components of
         the three named NDFs and writes them into a new NDF called
         middle.  All input values must be good to form a non-bad
         output value.
      }
   }
   \sstnotes{
      \sstitemlist{

         \sstitem
         A warning is issued (at the normal reporting level) whenever
         any output values are set bad because there are too few
         contributing data values.  This reports the fraction of flagged
         output data generated by the WLIM parameter's threshold.

         No warning is given when Parameter WLIM=\texttt{0}.  Input data
         containing only bad values are not counted in the flagged
         fraction, since no potential good output value has been lost.

         \sstitem
         For SINGLE=\texttt{TRUE} the value of the MSG\_FILTER environment
         variable is used to output messages.  If it is \texttt{QUIET}, nothing
         is reported on the screen.  If it is undefined, \texttt{NORMAL} or
         \texttt{VERBOSE}, the statistics are reported.  If it is \texttt{VERBOSE},
         the individual pixel values are also reported.

      }
   }
   \sstdiytopic{
      Related Applications
   }{
\xref{CCDPACK}{sun139}{}: \xref{MAKEMOS}{sun139}{MAKEMOS},
\xref{MAKECAL}{sun139}{MAKECAL},
\xref{MAKEFLAT}{sun139}{MAKEFLAT}.
   }
   \sstimplementationstatus{
      \sstitemlist{

         \sstitem
         This routine correctly processes the \htmlref{AXIS}{apndf:axis}, DATA,
         \htmlref{VARIANCE}{apndf:variance}, \htmlref{QUALITY}{apndf:quality},
         \htmlref{LABEL}{apndf:label}, \htmlref{TITLE}{apndf:title},
         \htmlref{UNITS}{apndf:units}, \htmlref{WCS}{apndf:wcs}, and
         \htmlref{HISTORY}{apndf:history}~ components of the first input NDF,
         and propagates all its \htmlref{extensions}{apndf:extensions}.

         \sstitem
         Processing of \htmlref{bad pixels}{se:masking} and automatic \htmlref{quality masking}{se:qualitymask} are supported.

         \sstitem
         All \htmlref{non-complex numeric data types}{ap:HDStypes} can be handled.
         Calculations are performed using the most appropriate of the
         data types integer, real or double precision.  If the input NDFs'
         structures contain values with other data types, then conversion
         will be performed as necessary.

         \sstitem
         Up to six NDF dimensions are supported.

         \sstitem
         Huge NDFs are supported.
      }
   }
}

\sstroutine{
   MULT
}{
   Multiplies two NDF data structures
}{
   \sstdescription{
      The routine multiplies two \NDFref{NDF} data structures pixel-by-pixel to
      produce a new NDF.
   }
   \sstusage{
      mult in1 in2 out
   }
   \sstparameters{
      \sstsubsection{
         IN1 = NDF (Read)
      }{
         First NDF to be multiplied.
      }
      \sstsubsection{
         IN2 = NDF (Read)
      }{
         Second NDF to be multiplied.
      }
      \sstsubsection{
         OUT = NDF (Write)
      }{
         Output NDF to contain the product of the two input NDFs.
      }
      \sstsubsection{
         TITLE = LITERAL (Read)
      }{
         The title for the output NDF.  A null value will cause
         the title of the NDF supplied for Parameter IN1 to be used
         instead.  \texttt{[!]}
      }
   }
   \sstexamples{
      \sstexamplesubsection{
         mult a b c
      }{
         This multiplies the NDF called a by the NDF called b, to make
         the NDF called c.  NDF c inherits its title from a.
      }
      \sstexamplesubsection{
         mult out=c in1=a in2=b title="Normalised spectrum"
      }{
         This multiplies the NDF called a by the NDF called b, to make
         the NDF called c.  NDF c has the title \texttt{"Normalised spectrum"}.
      }
   }
   \sstnotes{
      If the two input NDFs have different pixel-index bounds, then
      they will be trimmed to match before being multiplied.  An error
      will result if they have no pixels in common.
   }
   \sstdiytopic{
      Related Applications
   }{
KAPPA: \htmlref{ADD}{ADD},
\htmlref{CADD}{CADD},
\htmlref{CDIV}{CDIV},
\htmlref{CMULT}{CMULT},
\htmlref{CSUB}{CSUB},
\htmlref{DIV}{DIV},
\htmlref{MATHS}{MATHS},
\htmlref{SUB}{SUB}.
   }
   \sstimplementationstatus{
      \sstitemlist{

         \sstitem
         This routine correctly processes the \htmlref{AXIS}{apndf:axis}, DATA, \htmlref{QUALITY}{apndf:quality},
         \htmlref{LABEL}{apndf:label}, \htmlref{TITLE}{apndf:title}, \htmlref{UNITS}{apndf:units},
         \htmlref{HISTORY}{apndf:history}, \htmlref{WCS}{apndf:wcs}, and \htmlref{VARIANCE}{apndf:variance}~
         components of an NDF data structure and propagates all \htmlref{extensions}{apndf:extensions}.

         \sstitem
         Processing of \htmlref{bad pixels}{se:masking} and automatic \htmlref{quality masking}{se:qualitymask} are supported.

         \sstitem
         All \htmlref{non-complex numeric data types}{ap:HDStypes} can be handled.
         Calculations are performed using the most appropriate of the
         data types integer, real or double precision.  If the input NDF
         structures contain values with other data types, then conversion
         will be performed as necessary.

         \sstitem
         Huge NDFs are supported.
      }
   }
}
\sstroutine{
   NATIVE
}{
   Converts an HDS object to native machine data representation
}{
   \sstdescription{
      This application converts an \HDSref\ object (or structure) so that
      all primitive data values within it are represented using the
      appropriate native data representation for the machine in use
      (this includes the appropriate number format and byte ordering).
      This may typically be required after moving HDS files from
      another machine which uses a different number format and/or byte
      order, and will minimise the subsequent access time on the new
      machine.  Conversion is performed by modifying the data {\it in situ}.
      No separate output file is produced.

      This application can also be used to replace any IEEE floating-point
      NaN or Inf values in an HDS object with the appropriate Starlink
      bad value.  This conversion is performed even if the data values
      within the object are already represented using the appropriate
      native data representation for the machine in use.
   }
   \sstusage{
      native object
   }
   \sstparameters{
      \sstsubsection{
         OBJECT = UNIVERSAL (Read and Write)
      }{
         The HDS structure to be converted; either an entire container
         file or a particular object or structure within the file may
         be specified.  If a structure is given, all components (and
         sub-components, \emph{etc.}) within it will also be converted.
      }
   }
   \sstexamples{
      \sstexamplesubsection{
         native myfile
      }{
         Converts all the primitive data in the HDS container file
         myfile to be held using the appropriate native machine
         representation for faster subsequent access.
      }
      \sstexamplesubsection{
         native yourfile.data\_array
      }{
         Converts just the DATA\_ARRAY component (and its contents, if a
         structure) in the container file yourfile to the appropriate
         native machine data representation.  Other file contents remain
         unchanged.
      }
   }
}

\sstroutine{
   NDFCOMPARE
}{
   Compares a pair of NDFs for equivalence
}{
   \sstdescription{
      This application compares two supplied NDFs, and sets the
      Parameter SIMILAR to \texttt{"FALSE"} if they are significantly
      different in any way, and to \texttt{"TRUE"} if they are not
      significantly different.

      If they are not similar, a textual description of the differences
      is written to standard output, and to any file specified by
      Parameter REPORT.

      The two NDFS are compared in the following ways. Each test has
      an integer identifier, and the list of tests to be used can be
      controlled by Parameters DOTESTS and SKIPTESTS. Tests that are
      not included by default are indicated by the test number being in
      square brackets. Some tests have parameters that control the exact
      nature of the test.  These are listed in parentheses at the end of
      the description test listed below.

      \begin{itemize}
         \item 1   --- The number of pixel axes are compared.
         \item 2   --- The pixel bounds are compared.
         \item 3   --- The list of co-ordinate systems in the WCS FrameSet are compared.
         \item 4   --- The presence or absence of NDF components are compared (COMP).
         \item 5   --- The sky positions of a grid of pixels are compared (ACCPOS).
         \item 6   --- The data units strings are compared (WHITE).
         \item 7   --- The label strings are compared (CASE,WHITE).
         \item 8   --- The title strings are compared (CASE,WHITE).
         \item 9   --- The data types are compared.
         \item 10  --- The lists of NDF extensions are compared.
         \item 11  --- The number of bad DATA values are compared (NBAD).
         \item 12  --- The number of bad VARIANCE values are compared (NBAD).
         \item 13  --- The pixel DATA values are compared (ACCDAT).
         \item 14  --- The pixel VARIANCE values (if any) are compared (ACCVAR).
         \item 15  --- The pixel QUALITY values (if any) are compared (NBAD).
         \item 16  --- The QUALITY names (if any) are compared.
         \item $[$17$]$ --- The lists of root ancestor NDFs that were used to create each NDF are compared.
      \end{itemize}
   }
   \sstusage{
      ndfcompare in1 in2 [report]
   }
   \sstparameters{
      \sstsubsection{
         ACCDAT = LITERAL (Read)
      }{
         The maximum difference allowed between two pixel data values
         for them to be considered equivalent. The supplied string should
         contain a numerical value followed by a single character (case
         insensitive) from the list below indicating how the numerical
         value is to be used.

         \sstitemlist{

            \sstitem
            \texttt{"V"} --- The numerical value is a signal-to-noise value. The
            absolute difference in pixel data value is divided by the square
            root of the smaller of the two variances associated with the
            pixels (one from each input NDF). If the resulting ratio is
            smaller than the ACCDAT value, then the two pixel data values are
            considered to be equivalent. An error is reported if either NDF
            does not have a VARIANCE component.

            \sstitem
            \texttt{"R"} --- The numerical value is a relative error. The absolute
            difference between the two pixel data values is divided by the
            absolute mean of the two data values. If the resulting ratio is
            smaller than the ACCDAT value, then the two pixel data values are
            considered to be equivalent. To avoid problems with pixels where
            the mean is close to zero, a lower limit equal to the RMS of
            the data values is placed on the mean value used in the above ratio.

            \sstitem
            \texttt{"A"} --- The numerical value is an absolute error. If the absolute
            difference in pixel data value is smaller than the ACCDAT value,
            then the two pixel data values are considered to be equivalent.

         }
         If no character is included in the ACCDAT string, \texttt{"R"} is assumed.
         \texttt{["1E-6 R"]}
      }
      \sstsubsection{
         ACCPOS = \_DOUBLE (Read)
      }{
         The maximum difference allowed between two axis values for
         them to be considered equivalent, in units of pixels on the
         corresponding pixel axes.  \texttt{[0.2]}
      }
      \sstsubsection{
         ACCVAR = LITERAL (Read)
      }{
         The maximum difference allowed between two pixel variance values
         for them to be considered equivalent. The supplied string should
         contain a numerical value followed by a single character (case
         insensitive) from the list below indicating how the numerical value
         is to be used.

         \sstitemlist{

            \sstitem
            \texttt{"R"} --- The numerical value is a relative error. The absolute
            difference in variance value is divided by the absolute mean of
            the two variance values. If the resulting ratio is smaller than
            the ACCVAR value, then the two pixel variances are considered to
            be equivalent.

            \sstitem
            \texttt{"A"} --- The numerical value is an absolute error. If the absolute
            difference in variance values is smaller than the ACCVAR value,
            then the two pixel variances are considered to be equivalent.

         }
         If no character is included in the ACCVAR string, \texttt{"R"} is assumed.
         \texttt{["1E-6 R"]}
      }
      \sstsubsection{
         CASE = \_LOGICAL (Read)
      }{
         If \texttt{TRUE}, then string comparisons are case sensitive. Otherwise
         they are case insensitive.   \texttt{[TRUE]}
      }
      \sstsubsection{
         COMP = \_LITERAL (Read)
      }{
         A comma separated list of the NDF components to include in the
         test. If a null (!) value is supplied, all NDF components are
         included. \texttt{[!]}
      }
      \sstsubsection{
         DOTESTS() = \_INTEGER (Read)
      }{
         An initial list of indices for the tests to be performed, or
         null (\texttt{!}) if all tests are to be included in the initial list.
         This initial list is modified by excluding any tests specified
         by Parameter SKIPTESTS.  \texttt{[!]}
      }
      \sstsubsection{
         IN1 = NDF (Read)
      }{
         The first NDF.
      }
      \sstsubsection{
         IN2 = NDF (Read)
      }{
         The second NDF.
      }
      \sstsubsection{
         NBAD = LITERAL (Read)
      }{
         The maximum difference allowed between the number of bad values
         in each NDF. The same value is used for both DATA and VARIANCE
         arrays. It is also used as the maximum number of pixel that can
         have different QUALITY values. The supplied string should contain
         a numerical value followed by a single character (case insensitive)
         from the list below indicating how the numerical value is to
         be used.

         \sstitemlist{

            \sstitem
            \texttt{"R"} --- The numerical value is a relative error. The absolute
            difference in the number of bad values is divided by the mean
            number of bad values in both NDFs (for the QUALITY array, the
            total number of pixels in the NDF is used as the denominator in
            this ratio). If the resulting ratio is smaller than the NBAD value,
            then the two NDFs are considered to be equivalent for the purposes
            of this test.

            \sstitem
            \texttt{"A"} --- The numerical value is an absolute error. If the absolute
            difference in the number of bad values is smaller than the NBAD
            value, then the two NDFs are considered to be equivalent for the
            purposes of this test.

         }
         If no character is included in the NBAD string, \texttt{"R"} is assumed.
         \texttt{["0.001 R"]}
      }
      \sstsubsection{
         REPORT = LITERAL (Read)
      }{
         The name of a text file to create in which details of the
         differences found between the two NDFs will be store. \texttt{[!]}
      }
      \sstsubsection{
         SKIPTESTS() = \_INTEGER (Read)
      }{
         A list of indices for tests that are to removed from the initial
         list of tests specified by Parameter DOTESTS. If a null (\texttt{!}) value
         is supplied, the initial list is left unchanged.  \texttt{[15]}
      }
      \sstsubsection{
         SIMILAR = \_LOGICAL (Write)
      }{
         Set to \texttt{FALSE} on exit if any of the used tests indicate that the
         two NDFs differ.
      }
      \sstsubsection{
         WHITE = \_LOGICAL (Read)
      }{
         If \texttt{TRUE}, then trailing or leading white space is ignored when
         comparing strings.  \texttt{[FALSE]}
      }
   }
   \sstdiytopic{
      Related Applications
   }{
KAPPA: \htmlref{NDFTRACE}{NDFTRACE},
\htmlref{NORMALIZE}{NORMALIZE}.
   }
}

\sstroutine{
   NDFCOMPRESS
}{
   Compresses an NDF so that it occupies less disk space
}{
   \sstdescription{
      This application creates a copy of an \NDFref{NDF} that occupies less
      disk space.  This compression does not affect the data values seen by
      subsequent application, since all applications will automatically
      uncompress the data.

      Two compression methods are available: SCALE or DELTA (see
      Parameter METHOD).
   }
   \sstusage{
      ndfcompress in out method
   }
   \sstparameters{
      \sstsubsection{
         DSCALE = \_DOUBLE (Read)
      }{
         The scale factor to use for the Data component, when compressing
         with METHOD set to SCALE. If a null (\texttt{{!}}) value is supplied for
         DSCALE or DZERO, default values will be used for both that cause
         the scaled data values to occupy 96\% of the available range of the
         data type selected using Parameter SCALEDTYPE.  \texttt{[!]}
      }
      \sstsubsection{
         DZERO = \_DOUBLE (Read)
      }{
         The zero offset to use for the Data component, when compressing
         with METHOD set to SCALE. If a null (\texttt{{!}}) value is supplied for
         DSCALE or DZERO, default values will be used for both that cause
         the scaled data values to occupy 96\% of the available range of the
         data type selected using Parameter SCALEDTYPE.  \texttt{[!]}
      }
      \sstsubsection{
         IN = NDF (Read)
      }{
         The input NDF.
      }
      \sstsubsection{
         METHOD = LITERAL (Read)
      }{
         The compression method to use.  The options are as follows.

         \ssthitemlist{

            \sstitem
            \texttt{"BOTH"} --- A lossy compression scheme for all data types.
            It first creates an intermediate NDF from the supplied NDF using
            \texttt{"SCALED"} compression and then creates the final ouput NDF
            by applying \texttt{"DELTA"} compression to the intermediate NDF.
            The intermediate NDF is then deleted.

            \sstitem
            \texttt{"SCALED"} --- A lossy compression scheme for all data types. See
            \htmlref{``Scaled Compression''}{scaled_compression:ndfcompress}
            below, and Parameters DSCALE, DZERO, VSCALE, VZERO, and SCALEDTYPE.

            \sstitem
            \texttt{"DELTA"} --- A lossless compression scheme for integer data types.
            See \htmlref{``Delta Compression''}{delta_compression:ndfcompress}
            below, and Parameters ZAXIS, ZMINRATIO, and ZTYPE.
         }
         The current value is the default, which is initially \texttt{"DELTA"}.  \texttt{[]}
      }
      \sstsubsection{
         OUT = NDF (Write)
      }{
         The output NDF.
      }
      \sstsubsection{
         SCALEDTYPE = \htmlref{LITERAL}{se:parmenu} (Read)
      }{
         The \htmlref{data type}{ap:HDStypes} to use for the scaled data
         values.  This is only used if METHOD is \texttt{"SCALED"}.  It can be
         one of the following options.

         \ssthitemlist{

            \sstitem
            \texttt{"\_INTEGER"} --- four-byte signed integers

            \sstitem
            \texttt{"\_WORD"} --- two-byte signed integers

            \sstitem
            \texttt{"\_UWORD"} --- two-byte unsigned integers

            \sstitem
            \texttt{"\_BYTE"} --- one-byte signed integers

            \sstitem
            \texttt{"\_UBYTE"} --- one-byte unsigned integers

         }
         The same data type is used for both DATA and (if required)
         VARIANCE components of the output NDF.  The initial default
         value is \texttt{"\_WORD"}.  \texttt{[}current value\texttt{{]}}
      }
      \sstsubsection{
         VSCALE = \_DOUBLE (Read)
      }{
         The scale factor to use for the VARIANCE component, when
         compressing with METHOD set to SCALE.  If a null (\texttt{{!}}) value is
         supplied for VSCALE or VZERO, default values will be used for
         both that cause the scaled variance values to occupy 96\% of
         the available range of the data type selected using Parameter
         SCALEDTYPE.  \texttt{[!]}
      }
      \sstsubsection{
         VZERO = \_DOUBLE (Read)
      }{
         The zero factor to use for the VARIANCE component, when
         compressing with METHOD set to SCALE.  If a null (\texttt{{!}}) value is
         supplied for VSCALE or VZERO, default values will be used for
         both that cause the scaled variance values to occupy 96\% of
         the available range of the data type selected using Parameter
         SCALEDTYPE.  \texttt{[!]}
      }
      \sstsubsection{
         ZAXIS = \_INTEGER (Read)
      }{
         The index of the pixel axis along which differences are to be
         taken, when compressing with METHOD set to \texttt{"DELTA"}. If this is
         zero, a default value will be selected that gives the greatest
         compression.  \texttt{[0]}
      }
      \sstsubsection{
         ZMINRATIO = \_REAL (Read)
      }{
         The minimum allowed compression ratio for an array (the ratio
         of the supplied array size to the compressed array size), when
         compressing with METHOD set to \texttt{"DELTA"}. If compressing an array
         results in a compression ratio smaller than or equal to
         ZMINRATIO, then the array is left uncompressed in the new NDF.
         If the supplied value is zero or negative, then each array will
         be compressed regardless of the compression ratio.  \texttt{[1.0]}
      }
      \sstsubsection{
         ZTYPE = LITERAL (Read)
      }{
         The data type to use for storing differences between adjacent
         uncompressed data values, when compressing with METHOD set to
         \texttt{"DELTA"}.  Must be one of \_INTEGER, \_WORD, \_BYTE or blank. If a null
         (\texttt{{!}}) value or blank value is supplied, the data type that gives
         the best compression is determined and used.  \texttt{[!]}
      }
   }
   \sstexamples{
      \sstexamplesubsection{
         ndfcompress infile outfile scale scaledtype=\_uword
      }{
         Copies the contents of the NDF structure infile to the new
         structure outfile, scaling the values so that they fit into
         unsigned two-byte integers.  The scale and zero values used are
         chosen automatically.
      }
   }
   \label{scaled_compression:ndfcompress}
   \sstdiytopic{
      Scaled Compression
   }{
      The SCALE compression method scales the supplied data values using a
      linear transformation so that they fit into a smaller (integer) data
      type. A description of the scaling uses is stored with the output NDF
      so that later application can reconstruct the original unscaled values.
      This method is not lossless, due to the truncation involved in
      converting floating-point values to integers.
   }
   \label{delta_compression:ndfcompress}
   \sstdiytopic{
      Delta Compression
   }{
      DELTA compression is lossless, but can only be used on integer
      values. It assumes that adjacent integer values in the input tend to
      be close in value, and so differences between adjacent values can be
      represented in fewer bits than the absolute values themselves. The
      differences are taken along a nominated pixel axis within the
      supplied array (specified by Parameter ZAXIS). Any input value that
      differs from its earlier neighbour by more than the data range of
      the selected data type is stored explicitly using the data type of
      the input array.

      Further compression is achieved by replacing runs of equal input
      values by a single occurrence of the value with a corresponding
      repetition count.

      It should be noted that the degree of compression achieved is
      dependent on the nature of the data, and it is possible for a
      compressed array to occupy more space than the uncompressed array.
      The mean compression factor actually achieved is displayed (the
      ratio of the supplied NDF size to the compressed NDF size).

      It is possible to delta compress an NDF that has already been scale
      compressed. This provides a means of further compressing floating-point
      arrays. However, note that the default values supplied for DSCALE, DZERO,
      VSCALE, and VZERO may not be appropriate as they are chosen to maximise
      the spread of the scaled integer values in order to minimise the integer
      truncation error, but delta compression works best on arrays of integers
      in which the spread of values is small.

      If the input NDF is already DELTA compressed, it will be
      uncompressed and then recompressed using the supplied parameter values.

      More details of delta compression can be found in
      \latex{SUN/11 (\emph{ARY - A Subroutine Library for Accessing ARRAY
      Data Structures}), subsection \emph{Delta Compressed Array Form}.}
      \html{\xref{Delta Compressed Array Form}{sun11}{ddelta_form}.}
   }
   \sstdiytopic{
      Related Applications
   }{
      KAPPA: \htmlref{NDFCOPY}{NDFCOPY}.
   }
   \sstimplementationstatus{
      The TITLE, LABEL, UNITS, DATA, VARIANCE, QUALITY, AXIS, WCS, and
      HISTORY components are copied by this routine, together with all
      extensions.
   }
}

\sstroutine{
   NDFCOPY
}{
   Copies an NDF (or NDF section) to a new location
}{
   \sstdescription{
      This application copies an \NDFref{NDF} to a new location.  By supplying an
      \htmlref{NDF section}{se:ndfsect}~ as input it may be used to extract a subset,
      or to change the size or dimensionality of an NDF.  A second NDF may
      also be supplied to act as a shape template, and hence to define
      the region of the first NDF which is to be copied.

      Any unused space will be eliminated by the copying operation
      performed by this routine, so it may be used as a way of
      compressing NDF structures from which components have been
      deleted.  This ability also makes NDFCOPY a useful alternative to
      SETBOUND in cases where an NDF's size is to be reduced.
   }
   \sstusage{
      ndfcopy in out
   }
   \sstparameters{
      \sstsubsection{
         COMP = LITERAL (Read)
      }{
         The name of an array component in the input NDF (specified by
         Parameter IN) that will become the DATA\_ARRAY in the output NDF
         (specified by Parameter OUT).  It has the following options.

         \ssthitemlist{

            \sstitem
            \texttt{"Data"}  --- Each array component present is propagated to
            its counterpart.

            \sstitem
            \texttt{"Variance"} --- The VARIANCE component in the input NDF
            becomes the \linebreak
            DATA\_ARRAY in the output NDF and retains
            its data type.  The original DATA\_ARRAY is not copied.

            \sstitem
            \texttt{"Error"}  --- The square root of the VARIANCE component
            in the input NDF becomes the DATA\_ARRAY in the
            output NDF and retains the VARIANCE's data
            type.  The original DATA\_ARRAY and VARIANCE components
            are not copied.

            \sstitem
            \texttt{"Quality"} ---  The QUALITY component in the input NDF
            becomes the \linebreak DATA\_ARRAY in the output NDF and will be data type
            \_UBYTE.  The original DATA\_ARRAY and VARIANCE components are
            not copied.
         }
         \texttt{["Data"]}
      }
      \sstsubsection{
         EXTEN = \_LOGICAL (Read)
      }{
         If set to \texttt{FALSE} (the default), any NDFs contained within
         extensions of the input NDF are copied to equivalent places
         within the output NDF without change. If set \texttt{TRUE}, then any
         extension NDFs which have the same bounds as the base input
         NDF are padded or trimmed as necessary in order to ensure that
         they have the same bounds as the output NDF.  \texttt{[FALSE]}
      }
      \sstsubsection{
         IN = NDF (Read)
      }{
         The input NDF (or section) which is to be copied.
      }
      \sstsubsection{
         LIKE = NDF (Read)
      }{
         This parameter may be used to supply an NDF to be used as a
         shape template during the copying operation.  If such a
         template is supplied, then its shape will be used to select a
         matching section from the input NDF before copying takes
         place.  By default, no template will be used and the shape of
         the output NDF will therefore match that of the input NDF (or
         NDF section). The shape of the template in either pixel indices
         or the current WCS Frame may be used, as selected by Parameter
         LIKEWCS.  \texttt{[!]}
      }
      \sstsubsection{
         LIKEWCS = \_LOGICAL (Read)
      }{
         If \texttt{TRUE}, then the WCS bounds of the template supplied
         via Parameter LIKE are used to decide on the bounds of the
         output NDF.  Otherwise, the pixel bounds of the template are
         used.  \texttt{[FALSE]}
      }
      \sstsubsection{
         OUT = NDF (Write)
      }{
         The output NDF data structure.
      }
      \sstsubsection{
         TITLE = LITERAL (Read)
      }{
         A \htmlref{title}{apndf:title} for the output NDF.  A null value
         (the default) will cause the title of the NDF supplied for
         Parameter IN to be used instead.  \texttt{[!]}
      }
      \sstsubsection{
         TRIM = \_LOGICAL (Read)
      }{
         If \texttt{TRUE}, then the number of pixel axes in the output NDF will
         be reduced if necessary to remove any pixel axes which span only a
         single pixel.  For instance if \texttt{stokes} is a three-dimensional
         data cube with pixel bounds (1:100,-50:40,1:3), and the Parameter IN
         is given the value \texttt{"stokes(,,2)"}, then the dimensionality of
         the output depends on the setting of TRIM: if TRIM=\texttt{FALSE} the
         output is three-dimensional with pixel bounds (1:100,-50:40,2:2) and
         if TRIM=\texttt{TRUE} the output is two-dimensional with pixel bounds
         (1:100,-50:40).  In this example, the third pixel axis spans only a
         single pixel and is consequently removed if TRIM=\texttt{TRUE}.  \texttt{
         [FALSE]}
      }
       \sstsubsection{
          TRIMBAD = \_LOGICAL (Read)
      }{
          If \texttt{TRUE}, then the pixel bounds of the output NDF are trimmed
          to exclude any border of bad pixels within the input NDF.  That is,
          the output NDF will be the smallest NDF that encloses all good
          data values in the input NDF.  \texttt{[FALSE]}
      }
      \sstsubsection{
         TRIMWCS = \_LOGICAL (Read)
      }{
         This parameter is only accessed if Parameter TRIM is \texttt{TRUE}.  It
         controls the number of axes in the current WCS \htmlref{co-ordinate
         Frame}{se:domains}~ of the output NDF. If TRIMWCS=\texttt{YES}, then the
         \htmlref{current Frame}{se:curframe}~ in the output NDF will have the same number of axes as
         there are pixel axes in the output NDF.  If this involves removing
         axes, then the axes to retain are specified by Parameter USEAXIS.  If
         TRIMWCS=\texttt{NO} then all axes are retained in the \htmlref{current
         WCS Frame}{se:curframe}~ of the output NDF. Using the example in the
         description of the TRIM parameter, if the input NDF \texttt{stokes} has
         a three-dimensional current WCS Frame with axes (RA,Dec,Stokes) and
         TRIMWCS=YES, then an axis will be removed from the current Frame to
         make it two-dimensional (that is, to match the number of pixel axes
         remaining after the removal of insignificant pixel axes). The choice
         of which two axes to retain is controlled by Parameter USEAXIS.  If,
         on the other hand, TRIMWCS was set to \texttt{FALSE}, then the output
         NDF would still have two pixel axes, but the current WCS Frame would
         retain all three axes from the input NDF.  If one or more
         current-Frame axes are removed, the transformation from the current
         Frame to pixel Frame may become undefined resulting in some WCS
         operations being unusable.  The inverse of this transformation (from
         pixel Frame to current Frame) is unchanged however.  \texttt{[TRUE]}
      }
      \sstsubsection{
         USEAXIS = LITERAL (Read)
      }{
         This parameter is only accessed if TRIM and TRIMWCS are both \texttt{TRUE}
         and some axes need to be removed from the current WCS Frame of
         the output NDF.  It gives the axes which are to be retained in
         the current WCS Frame of the output NDF.  Each axis can be
         specified using one of the following options.

         \ssthitemlist{

            \sstitem
            An integer index of an axis within the current Frame of the
            input NDF (in the range 1 to the number of axes in the current
            Frame).

            \sstitem
            An axis \htmlattref{Symbol}{Symbol(axis)}~ string such as
            \texttt{"RA"} or \texttt{"VRAD"}.

            \sstitem
            A generic option where \texttt{"SPEC"} requests the spectral
            axis, \texttt{"TIME"} selects the time axis, \texttt{"SKYLON"}
            and \texttt{"SKYLAT"} picks the sky longitude and latitude
            axes respectively.  Only those axis domains present are
            available as options.
         }

         The dynamic default selects the axes with the same indices as
         the pixel axes being copied.  The value should be given as a
         comma-separated list.  \texttt{[]}
      }
   }
   \sstexamples{
      \sstexamplesubsection{
         ndfcopy infile outfile
      }{
         Copies the contents of the NDF structure infile to the new
         structure outfile.  Any unused space will be eliminated during
         the copying operation.
      }
      \sstexamplesubsection{
         ndfcopy infile outfile comp=var
      }{
         As the previous example except that the VARIANCE component of
         NDF infile becomes the DATA\_ARRAY of NDF outfile.
      }
      \sstexamplesubsection{
         ndfcopy in=data1(3:40,-3:17) out=data2 title="Extracted section"
      }{
         Copies the section (3:40,-3:17) of the NDF called data1 to a
         new NDF called data2.  The output NDF is assigned the new title
         \texttt{"Extracted section"}, which replaces the title derived from the
         input NDF.
      }
      \sstexamplesubsection{
         ndfcopy galaxy newgalaxy like=oldgalaxy
      }{
         Copies a section of the NDF called galaxy to form a new NDF
         called newgalaxy.  The section which is copied will correspond
         in shape with the template oldgalaxy.  Thus, after the copying
         operation, both newgalaxy and oldgalaxy will have the same
         pixel-index bounds.
      }
      \sstexamplesubsection{
         ndfcopy aa(20$\sim$11,20$\sim$11) bb like=aa
      }{
         Copies from the NDF section consisting of an 11$\times$11 pixel
         region of aa centred on pixel (20,20), into a new NDF called
         bb.  The shape of the region copied is made to match the
         original shape of aa.  The effect is to extract the selected
         square region of pixels into a new NDF of the same shape as
         the original, setting the surrounding region to the bad-pixel
         value.
      }
      \sstexamplesubsection{
         ndfcopy survey(12h23m:12h39m,11d:13d50m,) virgo trimwcs trim
      }{
         Copies a section specified by equatorial co-ordinate ranges
         from the three-dimensional NDF called survey, whose third pixel
         axis has only one element, to a two-dimensional NDF called
         virgo.  Information on the third WCS axis is removed too.
      }
   }
   \sstdiytopic{
      Related Applications
   }{
KAPPA: \htmlref{SETBOUND}{SETBOUND};
\xref{FIGARO}{sun86}{}: \xref{ISUBSET}{sun86}{ISUBSET}.
   }
   \sstimplementationstatus{
      \sstitemlist{
         \sstitem
         If present, an NDF's \htmlref{TITLE}{apndf:title}, \htmlref{LABEL}{apndf:label},
         \htmlref{UNITS}{apndf:units}, DATA, \htmlref{VARIANCE}{apndf:variance},
         \htmlref{QUALITY}{apndf:quality}, \htmlref{AXIS}{apndf:axis}~ \htmlref{WCS}{apndf:wcs},
         and \htmlref{HISTORY}{apndf:history}~ components are copied by this routine,
         together with all \htmlref{extensions}{apndf:extensions}.  The output NDF's title may
         be modified, if required, by specifying a new value via the TITLE parameter.

         \sstitem
         Huge NDFs are supported.
      }
   }
}
\sstroutine{
   NDFECHO
}{
   Displays a group of NDF names
}{
   \sstdescription{
      This application lists the names of the supplied NDFs to the
      screen, optionally filtering them using a regular expression. Its
      primary use is within scripts that need to process groups of NDFs.
      Instead of the full name, a required component of the name may be
      displayed instead (see Parameter SHOW).

      Two modes are available.

      \ssthitemlist{

         \sstitem
         If the NDFs are specified via the NDF parameter,
         then the NDFs must exist and be accessible (an error is reported
         otherwise). The NDF names obtained can then be modified by
         supplying a suitable GRP modification expression such as
         \texttt{"$*$\_A"} for Parameter MOD.

         \sstitem
         To list NDFs that may not exist, supply a null (\texttt{{!}}) value for
         Parameter NDF and the main group expression to Parameter MOD.
      }
   }
   \sstusage{
      ndfecho ndf [mod] [first] [last] [show]
   }
   \sstparameters{
      \sstsubsection{
         ABSPATH = \_LOGICAL (Read)
      }{
         If \texttt{TRUE}, any relative NDF paths are converted to absolute, using
         the current working directory. \texttt{[FALSE]}
      }
      \sstsubsection{
         EXISTS = \_LOGICAL (Read)
      }{
         If \texttt{TRUE}, then only display paths for NDFs specified by Parameter
         MOD that actually exist and are accessible. \texttt{[FALSE]}
      }
      \sstsubsection{
         FIRST = \_INTEGER (Read)
      }{
         The index of the first NDF to be tested.  A null (\texttt{{!}}) value causes
         the first NDF to be used (Index 1). \texttt{[!]}
      }
      \sstsubsection{
         LAST = \_INTEGER (Read)
      }{
         The index of the last NDF to be tested. If a non-null value is
         supplied for FIRST, then the run-time default for LAST is equal
         to the supplied FIRST value (so that only a single NDF will be
         tested). If a null value is supplied for FIRST, then the
         run-time default for LAST is the last NDF in the supplied group.
         \texttt{[]}
      }
      \sstsubsection{
         LOGFILE = FILENAME (Write)
      }{
         The name of a text file in which to store the listed NDF names.
         If a null (!) value is supplied, no log file is created. \texttt{[!]}
      }
      \sstsubsection{
         MOD = LITERAL (Read)
      }{
         An optional GRP modification expression that will be used to
         modify any names obtained via the NDF parameter.  For instance,
         if MOD is \texttt{"$*$\_A"} then the supplied NDF names will be
         modified by appending \texttt{"\_A"} to them. No modification occurs
         if a null (\texttt{{!}}) value is supplied.

         If a null value is supplied for Parameter NDF then the value
         supplied for Parameter MOD should not include an asterisk,
         since there are no names to be modified. Instead, the
         MOD value should specify an explicit group of NDF
         names do not need to exist.

         The list can be filtered to remove any NDFs that do not exist
         (see Parameter EXISTS). \texttt{[!]}
      }
      \sstsubsection{
         NDF = NDF (Read)
      }{
         A group of existing NDFs.  This should be given as a
         comma-separated list, in which each list element can be one of
         the following options.

         \ssthitemlist{

            \sstitem
            An NDF name, optionally containing wild-cards and/or regular
            expressions (\texttt{"$*$"}, \texttt{"?"}, \texttt{"[a-z]"}
            \emph{etc.}).

            \sstitem
            The name of a text file, preceded by an up-arrow character
            \texttt{"$\wedge$"}.  Each line in the text file should
            contain a comma-separated list of elements, each of which
            can in turn be an NDF name (with optional wild-cards,
            \emph{etc.}), or another file specification (preceded by an
            up-arrow).  Comments can be included in the file by
            commencing lines with a hash character \texttt{"\#"}.

         }
         If the value supplied for this parameter ends with a hyphen,
         then you are re-prompted for further input until a value is
         given which does not end with a hyphen.  All the NDFs given in
         this way are concatenated into a single group.

         If a null (\texttt{{!}}) value is supplied, then the displayed list of
         NDFs is determined by the value supplied for the MOD parameter.
      }
      \sstsubsection{
         PATTERN = LITERAL (Read)
      }{
         Specifies a pattern matching template using the syntax
         described below in \htmlref{``Pattern Matching
         Syntax''}{pattern_matching:ndfecho}.  Each NDF is
         displayed only if a match is found between this pattern and the
         item specified by Parameter SHOW.  A null (\texttt{{!}}) value
         causes all NDFs to be displayed.  \texttt{[!]}
      }
      \sstsubsection{
         SHOW = LITERAL (Read)
      }{
         Specifies the information to be displayed about each NDF.  The
         options are as follows.

         \sstitemlist{

            \sstitem
            \texttt{"Base"} --- The base file name.

            \sstitem
            \texttt{"Dir"} --- The directory path (if any).

            \sstitem
            \texttt{"Fspec"} --- The directory, base name and file type
            concatenated to form a full file specification.

            \sstitem
            \texttt{"Ftype"} --- The file type (usually \texttt{.sdf} but may not be if
            any foreign NDFs are supplied).

            \sstitem
            \texttt{"HDSpath"} --- The HDS path within the container file (if any).

            \sstitem
            \texttt{"Path"} --- The full name of the NDF as supplied by the user.

            \sstitem
            \texttt{"Slice"} --- The NDF slice specification (if any).

         }
         Note, the fields are extracted from the NDF names as supplied by
         the user. No missing fields (except for \texttt{"Ftype"}) are filled in.
         \texttt{["Path"]}
      }
   }
   \sstresparameters{
      \sstsubsection{
         NMATCH = \_INTEGER (Write)
      }{
         An output parameter to which is written the number of NDFs
         between FIRST and LAST that match the pattern supplied by
         Parameter PATTERN.
      }
      \sstsubsection{
         SIZE = \_INTEGER (Write)
      }{
         An output parameter to which is written the total number of NDFs
         in the specified group.
      }
      \sstsubsection{
         VALUE = LITERAL (Write)
      }{
         An output parameter to which is written information about the NDF
         specified by Parameter FIRST, or the first NDF in the group if
         FIRST is not specified. The information to write is specified by
         the SHOW parameter.
      }
   }
   \sstexamples{
      \sstexamplesubsection{
         ndfecho mycont
      }{
         Report the full path of all the NDFs within the HDS container file
         \texttt{mycont.sdf}. The NDFs must all exist.
      }
      \sstexamplesubsection{
         ndfecho $\wedge$files.lis first=4 show=base
      }{
         This reports the file base name for just the fourth NDF in the list
         specified within the text file \texttt{files.lis}. The NDFs must all
         exist.
      }
      \sstexamplesubsection{
         ndfecho $\wedge$files.lis $*$\_a logfile=log.lis
      }{
         This reports the names of the NDFs listed in text file \texttt{files.lis},
         but appending \texttt{"\_a"} to the end of each name. The NDFs must all exist.
         The listed NDF names are written to a new text file called \texttt{log.lis}.
      }
      \sstexamplesubsection{
         ndfecho in=! mod=\{$\wedge$base\}$|$\_a$|$\_b$|$
      }{
         This reports the names of the NDFs listed in text file \texttt{base}, but
         replacing \texttt{"\_a"} with \texttt{"\_b"} in their names. The NDFs need not
         exist since they are completely specified by Parameter MOD and not by
         Parameter NDF.
      }
   }
   \label{pattern_matching:ndfecho}
   \sstdiytopic{
      Pattern Matching Syntax
   }{
      The syntax for the PATTERN parameter value is a minimal form of
      regular expression. The following atoms are allowed.

% This conditional text is to avoid a space appearing before the backslash
% in the hypertext, such as " \W" instead of "\W".
      \latex{
         \sstitemlist{

         \sstitem
         \texttt{"[chars]"} --- Matches any of the characters within the brackets.

         \sstitem
         \texttt{"[$\wedge$chars]"} --- Matches any character that is not within the
                       brackets (ignoring the initial \texttt{"$\wedge$"} character).
         \sstitem
         \texttt{"."} --- Matches any single character.

         \sstitem
         \texttt{"$\backslash$d"} --- Matches a single digit.

         \sstitem
         \texttt{"$\backslash$D"} --- Matches anything but a single digit.

         \sstitem
         \texttt{"$\backslash$w"} --- Matches any alphanumeric character, and \texttt{"\_"}.

         \sstitem
         \texttt{"$\backslash$W"} --- Matches anything but alphanumeric characters, and \texttt{"\_"}.

         \sstitem
         \texttt{"$\backslash$s"} --- Matches white space.

         \sstitem
         \texttt{"$\backslash$S"} --- Matches anything but white space.
         }
      }
      % \html{
      %   \sstitemlist{

      %    \sstitem
      %    \texttt{"[chars]"} --- Matches any of the characters within the brackets.

      %    \sstitem
      %    \texttt{"[$\wedge$chars]"} --- Matches any character that is not within the
      %                  brackets (ignoring the initial \texttt{"$\wedge$"} character).
      %    \sstitem
      %    \texttt{"."} --- Matches any single character.

      %    \sstitem
      %    \verb+"\d"+ --- Matches a single digit.

      %    \sstitem
      %    \verb+"\D"+ --- Matches anything but a single digit.

      %    \sstitem
      %    \verb+"\w"+ --- Matches any alphanumeric character, and \texttt{"\_"}.

      %    \sstitem
      %    \verb+"\W"+ --- Matches anything but alphanumeric characters, and \texttt{"\_"}.

      %    \sstitem
      %    \verb+"\s"+ --- Matches white space.

      %    \sstitem
      %    \verb+"\S"+ --- Matches anything but white space.
      %   }
      % }

      Any other character that has no special significance within a
      regular expression matches itself.  Characters that have special
      significance can be matched by preceding them with a backslash
      ($\backslash$) in which case their special significance is ignored (note,
      this does not apply to the characters in the set dDsSwW).

      Note, minus signs (\texttt{{"-"}}) within brackets have no special
      significance, so ranges of characters must be specified
      explicitly.

      The following quantifiers are allowed.
      \ssthitemlist{

         \sstitem
         \texttt{"*"} --- Matches zero or more of the preceding atom, choosing the
             largest possible number that gives a match.

         \sstitem
         \texttt{"*?"}--- Matches zero or more of the preceding atom, choosing the
            smallest possible number that gives a match.

         \sstitem
         \texttt{"+"} --- Matches one or more of the preceding atom, choosing the
             largest possible number that gives a match.

         \sstitem
         \texttt{"+?"}--- Matches one or more of the preceding atom, choosing the
             smallest possible number that gives a match.

         \sstitem
         \texttt{"?"} --- Matches zero or one of the preceding atom.

         \sstitem
         \texttt{"\{n\}"} --- Matches exactly $n$ occurrences of the preceding atom.
      }

      The following constraints are allowed.
      \ssthitemlist{

         \sstitem
         \texttt{"$\wedge$"} --- Matches the start of the test string.

         \sstitem
         \texttt{"\$"} --- Matches the end of the test string.
      }

      Multiple templates can be concatenated, using the \texttt{"|"} character to
      separate them.  The test string is compared against each one in
      turn until a match is found.
   }
}

\sstroutine{
   NDFTRACE
}{
   Displays the attributes of an NDF data structure
}{
   \sstdescription{
      This routine displays the attributes of an \NDFref{NDF} data structure
      including:

      \sstitemlist{

         \sstitem
         its name;

         \sstitem
         the values of its character components (\htmlref{title}{apndf:title},
         \htmlref{label}{apndf:label}, and \htmlref{units}{apndf:units});

         \sstitem
         its shape (pixel bounds, dimension sizes, number of dimensions
         and total number of pixels);

         \sstitem
         \htmlref{axis}{apndf:axis}~ co-ordinate information (axis labels, units and extents);

         \sstitem
         optionally, axis array attributes (type and storage form) and
         the values of the axis normalisation flags;

         \sstitem
         attributes of the main data array and any other array
         components present (including the type and storage form and an
         indication of whether \htmlref{`bad'}{se:masking}~ pixels may be present);

         \sstitem
         attributes of the current \htmlref{co-ordinate Frame}{se:domains}~ in the \htmlref{WCS}{apndf:wcs} component
         (title, domain, and, optionally, axis labels and axis units, plus the
         system epoch and projection for sky co-ordinate Frames). In addition
         the bounding box of the NDF within the Frame is displayed.


         \sstitem
         optionally, attributes of all other co-ordinate Frames in the WCS
         component.

         \sstitem
         a list of any NDF \htmlref{extensions}{apndf:extensions}~ present, together with their data
         types; and

         \sstitem
         \htmlref{history}{se:ndfhistory}~ information (creation and last-updated dates, the
         update mode and the number of history records).

      }
      Most of this information is output to parameters.
   }
   \sstusage{
      ndftrace ndf
   }
   \sstparameters{

      \sstsubsection{
         FULLAXIS = \_LOGICAL (Read)
      }{
         If the NDF being examined has an axis co-ordinate system
         defined, then by default only the label, units and extent of
         each axis will be displayed.  However, if a \texttt{TRUE} value is given
         for this parameter, full details of the attributes of all the
         axis arrays will also be given.  \texttt{[FALSE]}
      }
      \sstsubsection{
         FULLFRAME = \_LOGICAL (Read)
      }{
         If a \texttt{FALSE} value is given for this parameter then only the
         Title and Domain attributes are displayed for a co-ordinate Frame.
         Otherwise, a more complete description is given.  \texttt{[FALSE]}
      }
      \sstsubsection{
         FULLWCS = \_LOGICAL (Read)
      }{
         If a \texttt{TRUE} value is given for this parameter then all co-ordinate
         Frames in the WCS component of the NDF are displayed.  Otherwise,
         only the current co-ordinate Frame is displayed.  \texttt{[FALSE]}
      }
      \sstsubsection{
         NDF = NDF (Read)
      }{
         The NDF data structure whose attributes are to be displayed.
      }
   }
   \sstresparameters{
      \sstsubsection{
         AEND( ) = \_DOUBLE (Write)
      }{
         The axis upper extents of the NDF.  For non-monotonic axes,
         zero is used.  See Parameter AMONO.  This is not assigned if
         AXIS is \texttt{FALSE}.
      }
      \sstsubsection{
         AFORM( ) = LITERAL (Write)
      }{
         The storage forms of the axis centres of the NDF.  This is
         only written when FULLAXIS is \texttt{TRUE} and AXIS is \texttt{TRUE}.
      }
      \sstsubsection{
         ALABEL( ) = LITERAL (Write)
      }{
         The axis labels of the NDF.  This is not assigned if AXIS is
         \texttt{FALSE}.
      }
      \sstsubsection{
         AMONO( ) = \_LOGICAL (Write)
      }{
         These are \texttt{TRUE} when the axis centres are monotonic, and \texttt{FALSE}
         otherwise.  This is not assigned if AXIS is \texttt{FALSE}.
      }
      \sstsubsection{
         ANORM( ) = \_LOGICAL (Write)
      }{
         The axis normalisation flags of the NDF.  This is only written
         when FULLAXIS is \texttt{TRUE} and AXIS is \texttt{TRUE}.
      }
      \sstsubsection{
         ASTART( ) = \_DOUBLE (Write)
      }{
         The axis lower extents of the NDF.  For non-monotonic axes,
         zero is used.  See Parameter AMONO.  This is not assigned if
         AXIS is \texttt{FALSE}.
      }
      \sstsubsection{
         ATYPE( ) = LITERAL (Write)
      }{
         The data types of the axis centres of the NDF.  This is only
         written when FULLAXIS is \texttt{TRUE} and AXIS is \texttt{TRUE}.
      }
      \sstsubsection{
         AUNITS( ) = LITERAL (Write)
      }{
         The axis units of the NDF.  This is not assigned if AXIS is
         \texttt{FALSE}.
      }
      \sstsubsection{
         AVARIANCE( ) = \_LOGICAL (Write)
      }{
         Whether or not there are axis variance arrays present in the
         NDF.  This is only written when FULLAXIS is \texttt{TRUE} and AXIS is
         \texttt{TRUE}.
      }
      \sstsubsection{
         AXIS = \_LOGICAL (Write)
      }{
         Whether or not the NDF has an axis system.
      }
      \sstsubsection{
         BAD = \_LOGICAL (Write)
      }{
         If \texttt{TRUE}, the NDF's data array may contain bad values.
      }
      \sstsubsection{
         BADBITS = LITERAL (Write)
      }{
         The BADBITS mask.  This is only valid when QUALITY is \texttt{TRUE}.
      }
      \sstsubsection{
         CURRENT = \_INTEGER (Write)
      }{
         The integer Frame index of the current co-ordinate Frame in the
         \htmlref{WCS}{apndf:wcs} component.
      }
      \sstsubsection{
         DIMS( ) = \_INT64 (Write)
      }{
         The dimensions of the NDF.
      }
      \sstsubsection{
         EXTNAME( ) = LITERAL (Write)
      }{
         The names of the extensions in the NDF.  It is only written
         when NEXTN is positive.
      }
      \sstsubsection{
         EXTTYPE( ) = LITERAL (Write)
      }{
         The types of the extensions in the NDF.  Their order
         corresponds to the names in EXTNAME.  It is only written when
         NEXTN is positive.
      }
      \sstsubsection{
         FDIM( ) = \_INTEGER (Write)
      }{
         The numbers of axes in each co-ordinate Frame stored in the WCS
         component of the NDF.  The elements in this parameter correspond to
         those in the FDOMAIN and FTITLE parameters.  The number of elements
         in each of these parameters is given by NFRAME.
      }
      \sstsubsection{
         FDOMAIN( ) = LITERAL (Write)
      }{
         The domain of each co-ordinate Frame stored in the WCS component
         of the NDF.  The elements in this parameter correspond to
         those in the FDIM and FTITLE parameters.  The number of elements
         in each of these parameters is given by NFRAME.
      }
      \sstsubsection{
         FLABEL( ) = LITERAL (Write)
      }{
         The axis labels from the \htmlref{current WCS
         Frame}{se:curframe}~ of the NDF.
      }
      \sstsubsection{
         FLBND( ) = \_DOUBLE (Write)
      }{
         The lower bounds of the bounding box enclosing the NDF in the
         current WCS Frame. The number of elements in this parameter is
         equal to the number of axes in the current WCS Frame (see FDIM).
         Celestial axis values will be in units of radians.
      }
      \sstsubsection{
         FORM = LITERAL (Write)
      }{
         The storage form of the NDF's data array.   This will be
         \texttt{"SIMPLE"}, \texttt{"PRIMITIVE"}, or \texttt{"SCALED"}.
      }
      \sstsubsection{
         FPIXSCALE( ) = LITERAL (Write)
      }{
         The nominal WCS pixel scale for each axis in the current WCS
         Frame.  For celestial axes, the value stored will be in
         arcseconds.  For other axes, the value stored will be in the
         units given by the corresponding element of FUNIT.
      }
      \sstsubsection{
         FTITLE( ) = LITERAL (Write)
      }{
         The title of each co-ordinate Frame stored in the WCS component of
         the NDF.  The elements in this parameter correspond to those in the
         FDOMAIN and FDIM parameters.  The number of elements in each
         of these parameters is given by NFRAME.
      }
     \sstsubsection{
         FUBND( ) = \_DOUBLE (Write)
      }{
         The upper bounds of the bounding box enclosing the NDF in the
         current WCS Frame. The number of elements in this parameter is
         equal to the number of axes in the current WCS Frame (see FDIM).
         Celestial axis values will be in units of radians.
      }
      \sstsubsection{
         FUNIT( ) = LITERAL (Write)
      }{
         The axis units from the current WCS Frame of the NDF.
      }
      \sstsubsection{
         HISTORY = \_LOGICAL (Write)
      }{
         Whether or not the NDF contains HISTORY records.
      }
      \sstsubsection{
         LABEL = LITERAL (Write)
      }{
         The label of the NDF.
      }
      \sstsubsection{
         LBOUND( ) = \_INT64 (Write)
      }{
         The lower bounds of the NDF.
      }
      \sstsubsection{
         NDIM = \_INTEGER (Write)
      }{
         The number of dimensions of the NDF.
      }
      \sstsubsection{
         NEXTN = \_INTEGER (Write)
      }{
         The number of extensions in the NDF.
      }
      \sstsubsection{
         NFRAME = \_INTEGER (Write)
      }{
         The number of \htmlref{WCS domains}{se:domains} described by Parameters FDIM, FDOMAIN, and
         FTITLE.  Set to zero if WCS is \texttt{FALSE}.
      }
      \sstsubsection{
         QUALITY = \_LOGICAL (Write)
      }{
         Whether or not the NDF contains a QUALITY array.
      }
      \sstsubsection{
         TITLE = LITERAL (Write)
      }{
         The title of the NDF.
      }
      \sstsubsection{
         TYPE = LITERAL (Write)
      }{
         The data type of the NDF's data array.
      }
      \sstsubsection{
         UBOUND( ) = \_INT64 (Write)
      }{
         The upper bounds of the NDF.
      }
      \sstsubsection{
         UNITS = LITERAL (Write)
      }{
         The units of the NDF.
      }
      \sstsubsection{
         VARIANCE = \_LOGICAL (Write)
      }{
         Whether or not the NDF contains a variance array.
      }
      \sstsubsection{
         WCS = \_LOGICAL (Write)
      }{
         Whether or not the NDF has any WCS co-ordinate Frames, over
         and above the default GRID, PIXEL and AXIS Frames.
      }
      \sstsubsection{
         WIDTH( ) = \_LOGICAL (Write)
      }{
         Whether or not there are axis width arrays present in the NDF.
         This is only written when FULLAXIS is \texttt{TRUE} and AXIS is \texttt{TRUE}.
      }
   }
   \sstexamples{
      \sstexamplesubsection{
         ndftrace mydata
      }{
         Displays information about the attributes of the NDF structure
         called mydata.
      }
      \sstexamplesubsection{
         ndftrace ndf=r106 fullaxis
      }{
         Displays information about the NDF structure r106, including
         full details of any axis arrays present.
      }
      \sstexamplesubsection{
         ndftrace mydata ndim=(mdim)
      }{
         Passes the number of dimensions of the NDF called mydata
         into the {\ICLref}~ variable mdim.
      }
   }
   \sstnotes{
      \sstitemlist{

         \sstitem
         If the WCS component of the NDF is undefined, then an attempt is
         made to find WCS information from two other sources: first, an
         \xref{IRAS90 astrometry structure}{sun163}{}, and secondly,
         the \htmlref{FITS extension}{se:fitsairlock}.  If
         either of these sources yield usable WCS information, then it is
         displayed in the same way as the NDF WCS component.  Other
         \KAPPA\ applications will use this WCS information as if it were
         stored in the WCS component.

         \sstitem
         The reporting of NDF attributes is suppressed when the message
         filter environment variable MSG\_FILTER is set to \texttt{QUIET}.  It
         benefits procedures and scripts where only the output parameters
         are needed.  The creation of output parameters is unaffected
         by MSG\_FILTER.

      }
   }
   \sstdiytopic{
      Related Applications
   }{
KAPPA: \htmlref{FITSLIST}{FITSLIST},
\htmlref{WCSFRAME}{WCSFRAME};
\xref{HDSTRACE}{sun102}{}.
   }
   \sstimplementationstatus{
      Huge NDFs are supported.
   }
}
\sstroutine{
   NOGLOBALS
}{
   Resets the \KAPPA\ global parameters
}{
   \sstdescription{
      This application resets the {\KAPPA}~ \htmlref{global parameters}{se:parglobals},
      and so makes their values undefined.
   }
   \sstusage{
      noglobals
   }
}
\sstroutine{
   NOMAGIC
}{
   Replaces all occurrences of magic value pixels in an NDF array
   with a new value
}{
   \sstdescription{
      This function replaces the standard \htmlref{`magic value'}{se:masking}~ assigned
      to bad pixels in an \NDFref{NDF} with an alternative value, or with
      random samples taken from a Normal distribution.  Input pixels which
      do not have the magic value are left unchanged.  The number of
      replacements is reported.  NOMAGIC's applications include the
      export of data to software that has different magic values or
      does not support bad values.

      If a constant value is used to replace magic values (which will
      be the case if Parameter SIGMA is given the value zero), then the
      same replacement value is used for both the data and variance
      arrays when COMP=\texttt{"All"}.  If the variance is being processed, the
      replacement value is constrained to be non-negative.

      Magic values are replaced by random values if the Parameter SIGMA
      is given a non-zero value.  If both DATA and VARIANCE components
      are being processed, then the random values are only stored in
      the DATA component; a constant value equal to SIGMA squared is
      used to replace all magic values in the VARIANCE component.  If
      only a single component is being processed (whether it be DATA,
      VARIANCE, or Error), then the random values are used to replace
      the magic values.  If random values are generated which will not
      fit into the allowed numeric range of the output NDF, then they
      are discarded and new random values are obtained instead.  This
      continues until a usable value is obtained.  This could introduce
      some statistical bias if many such re-tries are performed.  For
      this reason SIGMA is restricted so that there are at least 4
      standard deviations between the mean (given by REPVAL) and the
      nearest limit.  NOMAGIC notifies of any re-tries that are
      required.
   }
   \sstusage{
      nomagic in out repval [sigma] [comp]
   }
   \sstparameters{
      \sstsubsection{
         COMP = \htmlref{LITERAL}{se:parmenu} (Read)
      }{
         The components whose flagged values are to be substituted.  It
         may be:

         \begin{itemize}
            \item \texttt{"Data"}
            \item \texttt{"Error"}
            \item \texttt{"Variance"}
            \item \texttt{"All"}
         \end{itemize}

         The last of the
         options forces substitution of bad pixels in both the data and
         variance arrays.  This parameter is ignored if the data array
         is the only array component within the NDF.  \texttt{["Data"]}
      }
      \sstsubsection{
         IN = NDF  (Read)
      }{
         Input NDF structure containing the data and/or variance array
         to have its elements flagged with the magic value replaced by
         another value.
      }
      \sstsubsection{
         OUT = NDF (Write)
      }{
         Output NDF structure containing the data and/or variance array
         without any elements flagged with the magic value.
      }
      \sstsubsection{
         REPVAL = \_DOUBLE (Read)
      }{
         The constant value to substitute for the magic values, or (if
         Parameter SIGMA is given a non-zero value) the mean of the
         distribution from which replacement values are obtained.  It
         must lie within the minimum and maximum values of the data
         type of the array with higher precision, except when variance
         is being processed, in which case the minimum is constrained
         to be non-negative.  The replacement value is converted to the
         data type of the array being converted.  The suggested default
         is the current value.
      }
      \sstsubsection{
         SIGMA = \_DOUBLE (Read)
      }{
         The standard deviation of the random values used to replace
         magic values in the input NDF.  If this is zero (or if a null
         value is given), then a constant replacement value is
         used.  The supplied value must be positive and must be small
         enough to allow at least 4 standard deviations between the
         mean value (given by REPVAL) and the closest limit.  \texttt{[!]}
      }
      \sstsubsection{
         TITLE = LITERAL (Read)
      }{
         \htmlref{Title}{apndf:title} for the output NDF structure.  A null value (\texttt{{!}})
         propagates the title from the input NDF to the output NDF.  \texttt{[!]}
      }
   }
   \sstexamples{
      \sstexamplesubsection{
         nomagic aitoff irasmap repval=$-$2000000
      }{
         This copies the NDF called aitoff to the NDF irasmap, except
         that any bad values in the data array are replaced with the
         IPAC blank value, $-$2000000, in the NDF called irasmap.
      }
      \sstexamplesubsection{
         nomagic saturnb saturn 9999.0 comp=all
      }{
         This copies the NDF called saturnb to the NDF saturn, except
         that any bad values in the data and variance arrays are
         replaced with 9999 in the NDF called saturn.
      }
      \sstexamplesubsection{
         nomagic in=cleaned out=filled repval=0 sigma=10 comp=all
      }{
         This copies the NDF called cleaned to the NDF filled, except
         that any bad values in the data array are replaced by random
         samples taken from a Normal distribution of mean zero and
         standard deviation 10.  Bad values in the variance array are
         replaced by the constant value 100.
      }
   }
   \sstnotes{
      \sstitemlist{

         \sstitem
         If the NDF arrays have no bad pixels the application will abort.

         \sstitem
         Use GLITCH if a neighbourhood context is required to remove
         the bad values. }
   }
   \sstdiytopic{
      Related Applications
   }{
KAPPA: \htmlref{CHPIX}{CHPIX},
\htmlref{FILLBAD}{FILLBAD},
\htmlref{GLITCH}{GLITCH},
\htmlref{SEGMENT}{SEGMENT},
\htmlref{SETMAGIC}{SETMAGIC},
\htmlref{SUBSTITUTE}{SUBSTITUTE},
\linebreak
\htmlref{ZAPLIN}{ZAPLIN};
\xref{FIGARO}{sun86}{}: \xref{GOODVAR}{sun86}{GOODVAR}.
   }
   \sstimplementationstatus{
      \sstitemlist{

         \sstitem
         This routine correctly processes the \htmlref{AXIS}{apndf:axis}, DATA, \htmlref{QUALITY}{apndf:quality},
         \htmlref{VARIANCE}{apndf:variance}, \htmlref{LABEL}{apndf:label}, \htmlref{TITLE}{apndf:title}, \htmlref{UNITS}{apndf:units}, \htmlref{WCS}{apndf:wcs}, and \htmlref{HISTORY}{apndf:history}~ components of an NDF
         data structure and propagates all \htmlref{extensions}{apndf:extensions}.

         \sstitem
         Processing of \htmlref{bad pixels}{se:masking} and automatic \htmlref{quality masking}{se:qualitymask} are
         supported.

         \sstitem
         All \htmlref{non-complex numeric data types}{ap:HDStypes} can be handled.

         \sstitem
         Any number of NDF dimensions is supported.
      }
   }
}
\sstroutine{
   NORMALIZE
}{
   Normalises one NDF to a similar NDF by calculating a scale factor
   and zero-point difference
}{
   \sstdescription{
      This application compares the data values in one \NDFref{NDF} against the
      corresponding values in the other NDF.  A least-squares
      straight-line is then fitted to the relationship between the two
      sets of data values in order to determine the relative scale
      factor and any zero-level offset between the NDFs (the offset may
      optionally be fixed at zero---see Parameter ZEROFF).  To reduce
      computation time, the data points are binned according to the
      data value in the second NDF.  The mean data value within each bin
      is used to find the fit and weights are applied according to the
      number of pixels which contribute to each bin.

      To guard against erroneous data values, which can corrupt the fit
      obtained, the application then performs a number of iterations.
      It calculates a noise estimate for each bin according to the rms
      deviation of data values in the bin from the straight-line fit
      obtained previously.  It then re-bins the data, omitting values
      which lie more than a specified number of standard deviations
      from the expected value in each bin.  The straight-line fit is
      then re-calculated.  You can specify the number of standard
      deviations and the number of iterations used.

      A plot is produced after the final iteration showing the bin
      centres, with error bars representing the spread of values in each
      bin.  The best fitting straight line is overlayed on this plot.

      Optionally, an output NDF can be created containing a normalised
      version of the data array from the first input NDF.

      For the special case of two-dimensional images, if IN2 (or IN1)
      spans only a single row or column, it can be used to normalize each
      row or column of IN1 (or IN2) in turn.  See Parameter LOOP.

   }
   \sstusage{
      normalize in1 in2 out
   }
   \sstparameters{
      \sstsubsection{
         AXES = \_LOGICAL (Read)
      }{
         \texttt{TRUE} if labelled and annotated axes are to be drawn around the
         plot.  The width of the margins left for the annotation may be
         controlled using Parameter MARGIN.  The appearance of the axes
         (colours, founts, \emph{etc.}) can be controlled using the Parameter
         STYLE.  The dynamic default is \texttt{TRUE} if CLEAR is
         \texttt{TRUE}, and \texttt{FALSE} otherwise. \texttt{[]}
      }
      \sstsubsection{
         CLEAR = \_LOGICAL (Read)
      }{
         If \texttt{TRUE} the current picture is cleared before the plot is
         drawn.  If CLEAR is \texttt{FALSE} not only is the existing plot retained,
         but also an attempt is made to align the new picture with the
         existing picture.  Thus you can generate a composite plot within
         a single set of axes, say using different colours or modes to
         distinguish data from different datasets.  \texttt{[TRUE]}
      }
      \sstsubsection{
         DATARANGE( 2 ) = \_REAL (Read)
      }{
         This parameter may be used to override the auto-scaling
         feature.  If given, two real numbers should be supplied
         specifying the lower and upper data values in IN2, between
         which data will be used.  If a null (\texttt{{!}}) value is supplied, the
         values used are the auto-scaled values, calculated according to
         the value of PCRANGE.  Note, this parameter controls the range of
         data used in the fitting algorithm.  The range of data displayed
         in the plot can be specified separately using Parameters XLEFT,
         XRIGHT, YBOT, and YTOP.  \texttt{[!]}
      }
      \sstsubsection{
         DEVICE = \htmlref{DEVICE}{se:selgradev} (Read)
      }{
         The graphics workstation on which to produce the plot.  If a
         null value (\texttt{{!}}) is supplied no plot will be made.  \texttt{[}Current graphics
         device\texttt{}]
      }
      \sstsubsection{
          DRAWMARK = \_LOGICAL (Read)
      }{
         The central markers for each bin are not included in the plot
         if this parameter is set to \texttt{FALSE}. \texttt{[TRUE]}
      }
      \sstsubsection{
          DRAWWIDTH = \_LOGICAL (Read)
      }{
         The ``error bars'' marking the width of each bin are not included
         in the plot if this parameter is set to \texttt{FALSE}. \texttt{[TRUE]}
      }
      \sstsubsection{
         IN1 = NDF (Read)
      }{
         The NDF to be normalised.
      }
      \sstsubsection{
         IN2 = NDF (Read)
      }{
         The NDF to which IN1 will be normalised.
      }
      \sstsubsection{
          LOOP = \_LOGICAL (Read)
      }{
         If both IN1 and IN2 are two-dimensional, but one of them spans
         only a single row or column, then setting LOOP to \texttt{TRUE} will
         cause every row or column in to be normalised independently of
         each other. Specifically, if IN2 spans only a single row or
         column, then it will be used to normalise each row or column of
         IN1 in turn. Any output NDF (see Parameter OUT) will have the
         shape and size of IN1. If IN1 spans only a single row or column,
         then it will be normalised in turn by each row or column of IN2.
         Any output NDF (see Parameter OUT) will then have the shape and
         size of IN2. In either case, the details of the fit for each row
         or column will be displayed separately.  Also see Parameters
         OUTSLOPE, OUTOFFSET, and OUTCORR. \texttt{[FALSE]}
      }
      \sstsubsection{
         MARGIN( 4 ) = \_REAL (Read)
      }{
         The widths of the margins to leave for axis annotation, given
         as fractions of the corresponding dimension of the current picture.
         Four values may be given, in the order bottom, right, top, left.
         If fewer than four values are given, extra values are used equal to
         the first supplied value.  If these margins are too narrow any axis
         annotation may be clipped.  If a null (\texttt{{!}}) value is supplied, the
         value used is \texttt{0.15} (for all edges) if annotated axes are produced,
         and zero otherwise.  \texttt{[}current value\texttt{{]}}
      }
      \sstsubsection{
         MARKER = \_INTEGER (Read)
      }{
         Specifies the symbol with which each position should be marked in
         the plot.  It should be given as an integer \PGPLOT\  marker type.  For
         instance, \texttt{0} gives a box, \texttt{1} gives a dot, \texttt{2} gives a
         cross, \texttt{3} gives an asterisk, \texttt{7} gives a triangle.  The
         value must be larger than or equal to $-$31.  \texttt{[}current value\texttt{{]}}
      }
      \sstsubsection{
         MINPIX = \_INTEGER (Read)
      }{
         The minimum number of good pixels required in a bin before it
         contributes to the fitted line.  It must be in the range 1 to
         the number of pixels per bin.  \texttt{[2]}
      }
      \sstsubsection{
         NBIN = \_INTEGER (Read)
      }{
         The number of bins to use when binning the scatter plot prior
         to fitting a straight line, in the range 2 to 10000.  \texttt{[50]}
      }
      \sstsubsection{
         NITER = \_INTEGER (Read)
      }{
         The number of iterations performed to reject bad data values
         in the range 0 to 100.  \texttt{[2]}
      }
      \sstsubsection{
         NSIGMA = \_REAL (Read)
      }{
         The number of standard deviations at which bad data are
         rejected.  It must lie in the range 0.1 to 1.0E6.  \texttt{[3.0]}
      }
      \sstsubsection{
         OUT = NDF (Write)
      }{
         An optional output NDF to hold a version of IN1 which is
         normalised to IN2.  A null (\texttt{{!}}) value indicates that an output
         NDF is not required. See also Parameter LOOP.
      }
      \sstsubsection{
         OUTCORR = NDF (Write)
      }{
         An optional 1-dimensonal output NDF to hold the correlation
         coefficient for each row or column when LOOP=YES. See Parameter
         CORR. Ignored if LOOP=NO.  \texttt{[!]}
      }
      \sstsubsection{
         OUTOFFSET = NDF (Write)
      }{
         An optional 1-dimensonal output NDF to hold the offset used for
         each row or column when LOOP=YES. See Parameter OFFSET. Ignored
         if LOOP=NO.  \texttt{[!]}
      }
      \sstsubsection{
         OUTSLOPE = NDF (Write)
      }{
         An optional 1-dimensonal output NDF to hold the slope used for
         each row or column when LOOP=YES. See Parameter SLOPE. Ignored
         if LOOP=NO.  \texttt{[!]}
      }
      \sstsubsection{
         PCRANGE( 2 ) = \_REAL (Read)
      }{
         This parameter takes two real values in the range 0 to 100 and
         is used to modify the action of the auto-scaling algorithm
         which selects the data to use in the fitting algorithm.  The two
         values correspond to the percentage points in the histogram of
         IN2 at which the lower and upper cuts on data value are placed.
         With the default value, the plots will omit those pixels that
         lie in the lower and upper two-percent intensity range of IN2.
         Note, this parameter controls the range of data used in the
         fitting algorithm.  The range of data displayed in the plot can
         be specified separately using Parameters XLEFT, XRIGHT, YBOT, and
         YTOP.  \texttt{[2,98]}
      }
      \sstsubsection{
         STYLE = \htmlref{GROUP}{se:groups} (Read)
      }{
         A group of attribute settings describing the plotting style to use
         when drawing the annotated axes, data values, error bars, and
         best-fitting line.

         A comma-separated list of strings should be given in which each
         string is either an attribute setting, or the name of a text
         file preceded by an up-arrow character \texttt{"$\wedge$"}.  Such text files
         should contain further comma-separated lists which will be
         read and interpreted in the same manner.  Attribute settings
         are applied in the order in which they occur within the list,
         with later settings overriding any earlier settings given for
         the same attribute.

         Each individual attribute setting should be of the form:

            $<$name$>$=$<$value$>$


         where $<$name$>$ is the name of a plotting attribute, and $<$value$>$
         is the value to assign to the attribute.  Default values will be
         used for any unspecified attributes.  All attributes will be
         defaulted if a null value (\texttt{{!}})---the initial default---is supplied.
         To apply changes of style to only the current invocation, begin these
         attributes with a plus sign.  A mixture of persistent and temporary
         style changes is achieved by listing all the persistent attributes
         followed by a plus sign then the list of temporary attributes.

         See \slhyperref{Plotting Attributes}{Section~}{}{ap:plotting_attr}
         for a description of the available attributes.  Any unrecognised
         attributes are ignored (no error is reported).

         The appearance of the best-fitting straight line is controlled by
         the attributes \latex{\goodbreak} \htmlattref{Colour(Curves)}{Colour(element)},
         \htmlattref{Width(Curves)}{Width(element)}, \emph{etc.} (the synonym \att{Line}
         may be used in place of \att{Curves}).  The appearance of markers is
         controlled by \att{Colour(Markers)}, \att{Width(Markers)},
         \emph{etc.} (the synonym \att{Symbols} may be used in place of
         \att{Markers}).  The appearance of the error bars is controlled
         using \att{Colour(ErrBars)}, \att{Width(ErrBars)},
         \emph{etc}.  Note, \htmlattref{Size(ErrBars)}{Size(element)}~ controls
         the length of the serifs (\emph{i.e.} the cross pieces at each end of the
         error bar), and defaults to 1.0.  \texttt{[}current value\texttt{{]}}
      }
      \sstsubsection{
         TITLE = LITERAL (Read)
      }{
         The title for the output NDF.  A null value will cause
         the title of the NDF supplied for Parameter IN1 to be used
         instead.  \texttt{[!]}
      }
      \sstsubsection{
         XLEFT = \_DOUBLE (Read)
      }{
         The axis value to place at the left hand end of the horizontal
         axis of the plot.  If a null (\texttt{{!}}) value is supplied, the value used
         is the minimum data value used by the fitting algorithm from IN2
         (with a small margin).  The value supplied may be greater than or
         less than the value supplied for XRIGHT.  \texttt{[!]}
      }
      \sstsubsection{
         XRIGHT = \_DOUBLE (Read)
      }{
         The axis value to place at the right hand end of the horizontal
         axis of the plot.  If a null (\texttt{{!}}) value is supplied, the value used
         is the maximum data value used by the fitting algorithm from IN2
         (with a small margin).  The value supplied may be greater than or
         less than the value supplied for XLEFT.  \texttt{[!]}
      }
      \sstsubsection{
         YBOT = \_DOUBLE (Read)
      }{
         The axis value to place at the bottom end of the vertical axis of
         the plot.  If a null (\texttt{{!}}) value is supplied, the value used is
         the minimum data value used by the fitting algorithm from IN1 (with
         a small margin).  The value supplied may be greater than or less than
         the value supplied for YTOP.  \texttt{[]}
      }
      \sstsubsection{
         YTOP = \_DOUBLE (Read)
      }{
         The axis value to place at the top end of the vertical axis of
         the plot.  If a null (\texttt{{!}}) value is supplied, the value used is
         the maximum data value used by the fitting algorithm from IN1
         (with a small margin).  The value supplied may be greater than or
         less than the value supplied for YBOT.  \texttt{[!]}
      }
      \sstsubsection{
         ZEROFF = \_LOGICAL (Read)
      }{
         If \texttt{TRUE}, the offset of the linear fit is constrained to be
         zero.  \texttt{[FALSE]}
      }
   }
   \sstresparameters{
      \sstsubsection{
         CORR = \_REAL (Write)
      }{
         Pearson's coefficient of linear correlation for the data included
         in the last fit.
      }
      \sstsubsection{
         OFFSET = \_REAL (Write)
      }{
         The offset in the linear normalisation expression: IN1 = SLOPE $*$ IN2 $+$ OFFSET.
      }
      \sstsubsection{
         SLOPE = \_REAL (Write)
      }{
         The slope of the linear normalisation expression: IN1 = SLOPE $*$ IN2 $+$ OFFSET.
      }
   }
   \sstexamples{
      \sstexamplesubsection{
         normalize cl123a cl123b cl123c
      }{
         This normalises NDF cl123a to the NDF cl123b.  A plot of the
         fit is made on the \htmlref{current graphics device}{se:devglobal}, and the resulting
         normalisation scale and offset are written only to the
         \texttt{normalize.sdf} parameter file (as in the all the examples below
         except where noted).  The NDF cl123c is the normalised version
         of the input cl123a.
      }
      \sstexamplesubsection{
         normalize cl123a cl123b style="'size(errba)=0,title=Gain calibration'"
      }{
         This normalises NDF cl123a to the NDF cl123b.  A plot of the
         fit is made on the current graphics device with the title
         \texttt{"Gain calibration"}.  The error bars are drawn with no serifs.
      }
      \sstexamplesubsection{
         normalize cl123a cl123b cl123c offset=(shift) slope=(scale)
      }{
         This normalises NDF cl123a to the NDF cl123b.  A plot of the
         fit is made on the current graphics device.  The resulting
         normalisation scale and offset are written to the ICL
         variables SCALE and SHIFT respectively, where they could be
         passed to another application via an ICL procedure.  The NDF
         cl123c is the normalised version of the input cl123a.
      }
      \sstexamplesubsection{
         normalize in2=old in1=new out=! device=xwindows style=$\wedge$normstyle
      }{
         This normalises NDF new to the NDF old.  A plot of the fit is
         made on the xwindows device, using the plotting style defined in
         text file normstyle.  No output NDF is produced.
      }
      \sstexamplesubsection{
         normalize in1=new in2=old out=young niter=5 pcrange=[3,98.5]
      }{
         This normalises NDF new to the NDF old.  It has five iterations
         to reject outliers from the linear regression, and forms the
         regression using pixels in old whose data values lie between
         the 3 and 98.5 percentiles, comparing with the corresponding
         pixels in new.  A plot of the fit is made on the current
         graphics device.  The NDF young is the normalised version of
         the input new.
      }
   }
   \sstnotes{
      \sstitemlist{

         \sstitem
         The application stores two pictures in the
         \htmlref{graphics database}{se:agitate}~ in
         the following order: a FRAME picture containing the annotated axes
         and data plot, and a DATA picture containing just the data plot.
         Note, the FRAME picture is only created if annotated axes have been
         drawn, or if non-zero margins were specified using Parameter
         MARGIN.  The world co-ordinates in the DATA picture will correspond
         to data value in the two NDFs.
      }
   }
   \sstdiytopic{
      Related Applications
   }{
\xref{CCDPACK}{sun139}{}: \xref{MAKEMOS}{sun139}{MAKEMOS}.
   }
   \sstimplementationstatus{
      \sstitemlist{

         \sstitem
         The routine correctly processes the \htmlref{AXIS}{apndf:axis}, DATA, \htmlref{QUALITY}{apndf:quality},
         \htmlref{VARIANCE}{apndf:variance}, \htmlref{LABEL}{apndf:label}, \htmlref{TITLE}{apndf:title}, \htmlref{UNITS}{apndf:units}, \htmlref{WCS}{apndf:wcs}, and \htmlref{HISTORY}{apndf:history}~ components of an NDF,
         and propagates all \htmlref{extensions}{apndf:extensions} to the output NDF.  All propagated
 components come from the NDF to be normalised.

         \sstitem
         At the moment, variance values are not used in the fitting
         algorithm but are modified in the output NDF to take account of
         the scaling introduced by the normalisation.  (A later version may
         take account of variances in the fitting algorithm.)

         \sstitem
         Processing of \htmlref{bad pixels}{se:masking} and automatic \htmlref{quality masking}{se:qualitymask} are
         supported.

         \sstitem
         Only \_REAL data can be processed directly.  Other \htmlref{non-complex numeric data types}{ap:HDStypes} will undergo a type conversion before
         processing occurs.  \_DOUBLE data cannot be processed due to a
         loss of precision.

         \sstitem
         The pixel bounds of the two input NDFs are matched by trimming
         before calculating the normalisation constants, and are mapped as
         vectors to allow processing of NDFs of any dimensionality.  An
         output NDF may optionally be produced which is based on the
         first input NDF (IN1) by applying the calculated normalisation
         constants to IN1.
      }
   }
}
\sstroutine{
   NUMB
}{
   Counts the number of elements of an NDF with values or absolute
   values above or below a threshold
}{
   \sstdescription{
      This routine counts and reports the number of elements of an
      array within an input \NDFref{NDF} structure that have a value or absolute
      value greater or less than a specified threshold.  This statistic
      is also shown as a percentage of the total number of array
      elements.
   }
   \sstusage{
      numb in value [comp]
   }
   \sstparameters{
      \sstsubsection{
         ABS = \_LOGICAL (Read)
      }{
         If ABS=\texttt{TRUE}, the criterion is a comparison of the absolute value
         with the threshold; if \texttt{FALSE}, the criterion is a comparison of
         the actual value with the threshold.  The current value is the
         suggested default.  \texttt{[FALSE]}
      }
      \sstsubsection{
         ABOVE = \_LOGICAL (Read)
      }{
         If ABOVE=\texttt{TRUE} the criterion tests whether values are greater
         than the threshold; if \texttt{FALSE} the criterion tests whether values
         are less than the threshold.  The current value of ABOVE is the
         suggested default.  \texttt{[TRUE]}
      }
      \sstsubsection{
         COMP = \htmlref{LITERAL}{se:parmenu} (Read)
      }{
         The components whose flagged values are to be substituted.
         It may be \texttt{"Data"}, \texttt{"Error"}, \texttt{"Variance"}, or
         \texttt{"Quality"}.  If \texttt{"Quality"} is specified, then the quality values are treated as
         numerical values in the range 0 to 255.  \texttt{["Data"]}
      }
      \sstsubsection{
         IN  = NDF (Read)
      }{
         Input NDF structure containing the array to be tested.
      }
      \sstsubsection{
         VALUE = \_DOUBLE (Read)
      }{
         Threshold against which the values of the array elements will
         be tested.  It must lie in within the minimum and maximum
         values of the data type of the array being processed, unless
         ABS=\texttt{TRUE} or the component is the variance or quality
         array, in which case the minimum is zero.  The suggested
         default is the current value.
      }
   }
   \sstresparameters{
      \sstsubsection{
         NUMBER = \_INTEGER (Write)
      }{
         The number of elements that satisfied the criterion.
      }
   }
   \sstexamples{
      \sstexamplesubsection{
         numb image 100
      }{
         This counts the number of elements in the data array of the NDF
         called image that exceed 100.
      }
      \sstexamplesubsection{
         numb spectrum 100 noabove
      }{
         This counts the number of elements in the data array of the NDF
         called spectrum that are less than 100.
      }
      \sstexamplesubsection{
         numb cube 100 abs
      }{
         This counts the number of elements in the data array of the NDF
         called cube whose absolute values exceed 100.
      }
      \sstexamplesubsection{
         numb image $-$100 number=(count)
      }{
         This counts the number of elements in the data array of the NDF
         called image that exceed $-$100 and write the number to
         \ICL\ variable COUNT.
      }
      \sstexamplesubsection{
         numb image 200 v
      }{
         This counts the number of elements in the variance array of
         the NDF called image that exceed 200.
      }
   }
   \sstimplementationstatus{
      \sstitemlist{

         \sstitem
         This routine correctly processes the DATA, \htmlref{QUALITY}{apndf:quality},
         \htmlref{TITLE}{apndf:title}, and \htmlref{VARIANCE}{apndf:variance}~ components of an NDF data structure.

         \sstitem
         Processing of \htmlref{bad pixels}{se:masking} and automatic \htmlref{quality masking}{se:qualitymask} are
         supported.

         \sstitem
         All \htmlref{non-complex numeric data types}{ap:HDStypes} can be handled.

         \sstitem
         Any number of NDF dimensions is supported.

         \sstitem
         Huge NDFs are supported.
      }
   }
}
\sstroutine{
   ODDEVEN
}{
   Removes odd-even defects from a one-dimensional NDF
}{
   \sstdescription{
      This application forms a smoothed signal for a one-dimensional NDF
      whose elements have oscillating biases.  It averages the signal
      levels of alternating pixels.  Both elements must be good and not
      deviate from each other by more than a threshold for the averaging
      to take place.

      This application is intended for images exhibiting alternating
      patterns in columns or rows, the so called odd-even noise, arising
      from electronic interference or readout through different
      channels.  However, you must supply a vector collapsed along the
      unaffected axis, such that the vector exhibits the pattern.  See
      task \htmlref{COLLAPSE}{COLLAPSE} using the Mode or Median
      estimators.  The smoothed image is then subtracted from the supplied
      vector to yield the odd-even pattern.
   }
   \sstusage{
      oddeven in out [thresh]
   }
   \sstparameters{
      \sstsubsection{
         IN = NDF (Read)
      }{
         The one-dimensional NDF containing the input array to be
         filtered.
      }
      \sstsubsection{
         OUT = NDF (Write)
      }{
         The NDF to contain the filtered image.
      }
      \sstsubsection{
         THRESH = \_DOUBLE (Read)
      }{
         The maximum difference between adjacent elements for the
         averaging filter to be applied.  This allows anomalous pixels
         to be excluded.  If null, \texttt{{!}}, is given, then there is no limit.
         \texttt{[!]}
      }
      \sstsubsection{
         TITLE = LITERAL (Read)
      }{
         The title of the output NDF.  A null (\texttt{{!}}) value means using the
         title of the input NDF.  \texttt{[!]}
      }
   }
   \sstexamples{
      \sstexamplesubsection{
         oddeven raw clean
      }{
         The one-dimensional NDF called raw is filtered such that
         adjacent pixels are averaged to form the output NDF call clean.
      }
      \sstexamplesubsection{
         oddeven out=clean in=raw thresh=20
      }{
         As above except only those adjacent pixels whose values differ
         by no more than 20 are averaged.
      }
   }
   \sstdiytopic{
      Related Applications
   }{
KAPPA: \htmlref{BLOCK}{BLOCK},
\htmlref{CHPIX}{CHPIX},
\htmlref{FFCLEAN}{FFCLEAN},
\htmlref{GLITCH}{GLITCH},
\htmlref{ZAPLIN}{ZAPLIN};
\xref{FIGARO}{sun86}{}: \xref{BCLEAN}{sun86}{BCLEAN},
\xref{CLEAN}{sun86}{CLEAN},
\xref{ISEDIT}{sun86}{ISEDIT},
\xref{TIPPEX}{sun86}{TIPPEX}.
   }
   \sstimplementationstatus{
      \ssthitemlist{

         \sstitem
         This routine correctly processes the \htmlref{AXIS}{apndf:axis}, DATA, \htmlref{QUALITY}{apndf:quality},
         \htmlref{LABEL}{apndf:label}, \htmlref{TITLE}{apndf:title}, \htmlref{UNITS}{apndf:units},
         \htmlref{WCS}{apndf:wcs}, and \htmlref{HISTORY}{apndf:history} components of an NDF
         data structure and propagates all \htmlref{extensions}{apndf:extensions}.

         \sstitem
         \htmlref{VARIANCE}{apndf:variance} is merely propagated.

         \sstitem
         Processing of \htmlref{bad pixels}{se:masking} and automatic
         \htmlref{quality masking}{se:qualitymask} are supported.

         \sstitem
         All \htmlref{non-complex numeric data types}{ap:HDStypes} can be
         handled.  Arithmetic is performed using single- or double-precision
         floating point as appropriate.
      }
   }
}
\sstroutine{
   OUTLINE
}{
   Draws an outline of a two-dimensional NDF
}{
   \sstdescription{
      This application draws an outline of a two-dimensional \NDFref{NDF} on
      the \htmlref{current graphics device}{se:devglobal}, aligning it with any existing plot.

      Annotated axes can be produced (see Parameter AXES), and the appearance
      of the axes and curve can be controlled in detail (see Parameter STYLE).
      The axes show co-ordinates in the current \htmlref{co-ordinate Frame}{se:domains}~  of the
      supplied NDF.

      This command is a synonym for \texttt{contour mode=bounds penrot=yes
      clear=no}.
   }
   \sstusage{
      outline ndf
   }
   \sstparameters{
      \sstsubsection{
         AXES = \_LOGICAL (Read)
      }{
         \texttt{TRUE} if labelled and annotated axes are to be drawn around the
         plot, showing the current co-ordinate Frame of the
         supplied NDF.  The appearance of the axes can be controlled using
         the STYLE parameter.  \texttt{[TRUE]}
      }
      \sstsubsection{
         DEVICE = \htmlref{DEVICE}{se:selgradev} (Read)
      }{
         The plotting device.  \texttt{[}current graphics device\texttt{{]}}
      }
      \sstsubsection{
         LABOUT = \_LOGICAL (Read)
      }{
         Specifies the position at which to place a label identifying the
         input NDF within the plot.  The label is drawn parallel to the
         first pixel axis.  Two values should be supplied for LABPOS.  The
         first value specifies the distance in millimetres along the first
         pixel axis from the centre of the bottom-left pixel to the left
         edge of the label.  The second value specifies the distance in
         millimetres along the second pixel axis from the centre of the
         bottom-left pixel to the baseline of the label.  If a null (\texttt{{!}})
         value is given, no label is produced.  The appearance of the label
         can be set by using the STYLE parameter
         (for instance \texttt{"Size(strings)=2"}).  \texttt{[}current value\texttt{{]}}
      }
      \sstsubsection{
         MARGIN( 4 ) = \_REAL (Read)
      }{
         The widths of the margins to leave around the outline for
         axis annotation.  The widths should be given as fractions of the
         corresponding dimension of the current picture.  The actual margins
         used may be increased to preserve the aspect ratio of the DATA
         picture.  Four values may be given, in the order bottom, right,
         top, left.  If fewer than four values are given, extra values are
         used equal to the first supplied value.  If these margins are too
         narrow any axis annotation may be clipped.  If a null (\texttt{{!}}) value is
         supplied, the value used is \texttt{0.15} (for all edges) if annotated axes
         are being produced, and zero otherwise.  \texttt{[}current value\texttt{{]}}
      }
      \sstsubsection{
         NDF = NDF (Read)
      }{
         NDF structure containing the two-dimensional image to be
         outlined.
      }
      \sstsubsection{
         STYLE = \htmlref{GROUP}{se:groups} (Read)
      }{
         A group of attribute settings describing the plotting style to use
         for the outline and annotated axes.

         A comma-separated list of strings should be given in which each
         string is either an attribute setting, or the name of a text
         file preceded by an up-arrow character \texttt{"$\wedge$"}.  Such text files
         should contain further comma-separated lists which will be
         read and interpreted in the same manner.  Attribute settings
         are applied in the order in which they occur within the list,
         with later settings overriding any earlier settings given for
         the same attribute.

         Each individual attribute setting should be of the form:

            $<$name$>$=$<$value$>$


         where $<$name$>$ is the name of a plotting attribute, and $<$value$>$
         is the value to assign to the attribute.  Default values will be
         used for any unspecified attributes.  All attributes will be
         defaulted if a null value (\texttt{{!}})---the initial default---is supplied.
         To apply changes of style to only the current invocation, begin these
         attributes with a plus sign.  A mixture of persistent and temporary
         style changes is achieved by listing all the persistent attributes
         followed by a plus sign then the list of temporary attributes.

         See \slhyperref{Plotting Attributes}{Section~}{}{ap:plotting_attr}
         for a description of the available attributes.  Any unrecognised
         attributes are ignored (no error is reported).

         The appearance of the outline is controlled by the attributes
         \htmlattref{Colour(Curves)}{Colour(element)}, \goodbreak
         \htmlattref{Width(Curves)}{Width(element)}, \emph{etc}.  \texttt{[}current value\texttt{{]}}
      }
   }
   \sstdiytopic{
      Related Applications
   }{
KAPPA: \htmlref{WCSFRAME}{WCSFRAME},
\htmlref{CONTOUR}{CONTOUR},
\htmlref{PICDEF}{PICDEF};
\xref{CCDPACK}{sun139}{}: \xref{DRAWNDF}{sun139}{DRAWNDF}.
   }
}
\sstroutine{
   OUTSET
}{
   Mask pixels inside or outside a specified circle in a two-dimensional NDF
}{
   \sstdescription{
      This routine assigns a specified value (which may be
      \htmlref{\emph{bad}}{se:masking}) to
      either the outside or inside of a specified circle within a
      specified component of a given two-dimensional \NDFref{NDF}.
   }
   \sstusage{
      outset in out centre diam
   }
   \sstparameters{
      \sstsubsection{
         CENTRE = LITERAL (Read)
      }{
         The co-ordinates of the centre of the circle.  The position must
         be given in the current \htmlref{co-ordinate Frame}{se:domains}~  of the NDF (supplying
         a colon \texttt{":"} will display details of the current co-ordinate
         Frame).  The position should be supplied as a list of formatted
         axis values separated by spaces or commas.  See also Parameter
         USEAXIS.  The \htmlref{current co-ordinate Frame}{se:curframe}~ can be
         changed using task \htmlref{WCSFRAME}{WCSFRAME}.
      }
      \sstsubsection{
         COMP = \htmlref{LITERAL}{se:parmenu} (Read)
      }{
         The NDF array component to be masked.  It may be \texttt{"Data"}, or
         \texttt{"Variance"}, or \texttt{"Error"} (where \texttt{"Error"} is equivalent to
         \texttt{"Variance"}).  \texttt{["Data"]}
      }
      \sstsubsection{
         CONST = LITERAL (Given)
      }{
         The constant numerical value to assign to the masked pixels, or
         the string \texttt{"bad"}.  \texttt{["bad"]}
      }
      \sstsubsection{
         DIAM = LITERAL (Read)
      }{
         The diameter of the circle.  If the current co-ordinate Frame of
         the NDF is a SKY Frame (\emph{e.g.} RA and DEC), then the value should be
         supplied as an increment of celestial latitude (\emph{e.g.} DEC).  Thus,
         \texttt{"10.2"} means 10.2 arcseconds, \texttt{"30:0"} would mean 30 arcminutes,
         and \texttt{"1:0:0"} would mean 1 degree.  If the current co-ordinate
         Frame is not a SKY Frame, then the diameter should be specified
         as an increment along Axis 1 of the current co-ordinate Frame.
         Thus, if the current Frame is PIXEL, the value should be given
         simply as a number of pixels.
      }
      \sstsubsection{
         IN = NDF (Read)
      }{
         The name of the source NDF.
      }
      \sstsubsection{
         INSIDE = \_LOGICAL (Read)
      }{
         If a \texttt{TRUE} value is supplied, the constant value is assigned to the
         inside of the circle.  Otherwise, it is assigned to the outside.  \texttt{[FALSE]}
      }
      \sstsubsection{
         OUT = NDF (Write)
      }{
         The name of the masked NDF.
      }
      \sstsubsection{
         TITLE = LITERAL (Read)
      }{
         \htmlref{Title}{apndf:title} for the output NDF structure.  A null value (\texttt{{!}})
         propagates the title from the input NDF to the output NDF.  \texttt{[!]} }
     \sstsubsection{
         USEAXIS = \htmlref{GROUP}{se:groups} (Read)
     }{
         USEAXIS is only accessed if the current co-ordinate Frame of
         the NDF has more than two axes. A group of strings should be
         supplied designating the axes that are to be used when
         specifying the circle via Parameters CENTRE and DIAM.  Each axis
         can be specified using one of the following options.

         \ssthitemlist{
           \sstitem
           Its integer index within the current Frame of the input
           NDF (in the range 1 to the number of axes in the current
           Frame).

           \sstitem
           Its \htmlattref{Symbol}{Symbol(axis)}~ string such as
           \texttt{"RA"} or \texttt{"VRAD"}.

           \sstitem
           A generic option where \texttt{"SPEC"} requests the spectral axis,
           \texttt{"TIME"} selects the time axis, \texttt{"SKYLON"} and
           \texttt{"SKYLAT"} picks the sky longitude and latitude axes
           respectively.  Only those axis domains present are
           available as options.
         }

         A list of acceptable values is displayed if an illegal value is
         supplied.  If a null (\texttt{!}) value is supplied, the axes with the
         same indices as the two used pixel axes within the NDF are
         used. \texttt{[!]}
     }
   }
   \sstexamples{
      \sstexamplesubsection{
         outset neb1 nebm "13.5,201.3" 20 const=0
      }{
         This copies NDF neb1 to nebm, setting pixels to zero in the
         DATA array if they fall outside the specified circle.  Assuming the
         current co-ordinate Frame of neb1 is PIXEL, the circle is centred
         at pixel co-ordinates (13.5, 201.3) and has a diameter of 20 pixels.
      }
      \sstexamplesubsection{
         outset neb1 nebm "15:23:43.2 -22:23:34.2" "10:0" inside comp=var
      }{
         This copies NDF neb1 to nebm, setting pixels bad in the
         variance array if they fall inside the specified circle.  Assuming
         the current co-ordinate Frame of neb1 is a SKY Frame describing RA
         and DEC, the aperture is centred at RA 15:23:43.2 and
         DEC -22:23:34.2, and has a diameter of 10 arcminutes.
      }
   }
   \sstdiytopic{
      Related Applications
   }{
KAPPA: \htmlref{ARDMASK}{ARDMASK},
\htmlref{REGIONMASK}{REGIONMASK}.
   }
   \sstimplementationstatus{
      \sstitemlist{

         \sstitem
         This routine correctly processes the \htmlref{WCS}{apndf:wcs}, \htmlref{AXIS}{apndf:axis}, DATA, \htmlref{QUALITY}{apndf:quality},
         \htmlref{LABEL}{apndf:label}, \htmlref{TITLE}{apndf:title}, \htmlref{UNITS}{apndf:units}, \htmlref{HISTORY}{apndf:history}, and \htmlref{VARIANCE}{apndf:variance}~ components of an NDF
         data structure and propagates all \htmlref{extensions}{apndf:extensions}.

         \sstitem
         Processing of \htmlref{bad pixels}{se:masking} and automatic \htmlref{quality masking}{se:qualitymask} are
         supported.

         \sstitem
         \htmlref{Bad pixels}{se:masking} and \htmlref{quality masking}{se:qualitymask} are supported.

         \sstitem
         All \htmlref{non-complex numeric data types}{ap:HDStypes} can be handled.
      }
   }
}

\sstroutine{
   PALDEF
}{
   Loads the default palette to a colour table
}{
   \sstdescription{
      This application loads the standard \htmlref{palette}{se:palette}~ of colours to fill
      the portion of the \htmlref{current graphics device's}{se:devglobal}~ colour table which is
      reserved for the palette.  The palette comprises 16 colours and
      is intended to provide coloured annotations, borders, axes,
      graphs \emph{etc.} that are unaffected by changes to the \htmlref{lookup table}{se:lookuptables}~
      used for images.

      The standard palette is as follows
        0: Black     1: White     2: Red       3: Green     4: Blue
        5: Yellow    6: Magenta   7: Cyan      8--15: Black
   }
   \sstusage{
      paldef [device]
   }
   \sstparameters{
      \sstsubsection{
         DEVICE = \htmlref{DEVICE}{se:selgradev} (Read)
      }{
         Name of the graphics device to be used.  \texttt{[}Current graphics device\texttt{{]}}
      }
   }
   \sstexamples{
      \sstexamplesubsection{
         paldef
      }{
         This loads the standard palette into the reserved portion of
         the colour table of the current graphics device.
      }
      \sstexamplesubsection{
         paldef xwindows
      }{
         This loads the standard palette into the reserved portion of
         the colour table of the xwindows device.
      }
   }
   \sstnotes{
      \sstitemlist{

         \sstitem
         The effects of this command will only be immediately apparent
         when run on X windows which have 256 colours (or other similar
         pseudocolour devices).  On other devices (for instance, X windows
         with more than 256 colours) the effects will only become
         apparent when subsequent graphics applications are run.
      }
   }
   \sstdiytopic{
      Related Applications
   }{
KAPPA: \htmlref{PALENTRY}{PALENTRY},
\htmlref{PALREAD}{PALREAD},
\htmlref{PALSAVE}{PALSAVE}.
   }
}

\sstroutine{
   PALENTRY
}{
   Enters a colour into an graphics device's palette
}{
   \sstdescription{
      This application obtains a colour and enters it into the
      \htmlref{palette}{se:palette}~ portion of the
      \htmlref{current graphics device's}{se:devglobal}~ colour table.  The palette
      comprises up to 16 colours and is intended to provide coloured
      annotations, borders, axes, graphs \emph{etc.} that are unaffected by
      changes to the \htmlref{lookup table}{se:lookuptables}~ used for images.

      A colour is specified either by the giving the red, green, blue
      intensities; or named colours.
   }
   \sstusage{
      palentry palnum colour [device]
   }
   \sstparameters{
      \sstsubsection{
         COLOUR() = LITERAL (Read)
      }{
         A colour to be added to the palette at the entry given by
         Parameter PALNUM.  It is one of the following options.

         \ssthitemlist{

            \sstitem
             A \htmlref{named colour}{ap:colset} from the standard colour set,
             which may be abbreviated.  If the abbreviated name is ambiguous the
             first match (in alphabetical order) is selected.  The case
             of the name is ignored.  Some examples are \texttt{"Pink"},
              \texttt{"Yellow"}, \texttt{"Aquamarine}", and \texttt{"Orchid"}.

             \sstitem
             Normalised red, green, and blue intensities separated by
             commas or spaces.  Each value must lie in the range 0.0--1.0.
             For example, \texttt{"1.0,1.0,0.5"} would give a pale yellow.

             \sstitem
             An \htmlref{HTML colour code}{htmlcolour} such as \texttt{\#ff002d}.
         }
      }
      \sstsubsection{
         DEVICE = \htmlref{DEVICE}{se:selgradev} (Read)
      }{
         Name of the graphics device to be used.  \texttt{[}Current graphics device\texttt{{]}}
      }
      \sstsubsection{
         PALNUM = \_INTEGER (Read)
      }{
         The number of the palette entry whose colour is to be
         modified.  PALNUM must lie in the range zero to the minimum
         of 15 or the number of colour indices minus one.  The
         suggested default is 1.
      }
   }
   \sstexamples{
      \sstexamplesubsection{
         palentry 5 gold
      }{
         This makes palette entry number 5 have the colour gold in the
         reserved portion of the colour table of the current image
         display.
      }
      \sstexamplesubsection{
         palentry 12 [1.0,1.0,0.3] xwindows
      }{
         This makes the xwindows device's palette entry number 12 have a
         pale-yellow colour.
      }
   }
   \sstnotes{
      \sstitemlist{

         \sstitem
         The effects of this command will only be immediately apparent
         when run on X windows which have 256 colours (or other similar
         pseudocolour devices).  On other devices (for instance, X windows
         with more than 256 colours) the effects will only become
         apparent when subsequent graphics applications are run.
      }
   }
   \sstdiytopic{
      Related Applications
   }{
KAPPA: \htmlref{PALDEF}{PALDEF},
\htmlref{PALREAD}{PALREAD},
\htmlref{PALSAVE}{PALSAVE}.
   }
}

\sstroutine{
   PALREAD
}{
   Fills the palette of a colour table from an NDF
}{
   \sstdescription{
      This application reads a \htmlref{palette}{se:palette}~ of colours from an \NDFref{NDF}, stored as
      red, green and blue intensities, to fill the portion of
      the \htmlref{current graphics device's}{se:devglobal}~ colour table which is reserved for
      the palette.  The palette comprises 16 colours and is intended
      to provide coloured annotations, borders, axes, graphs \emph{etc.} that
      are unaffected by changes to the \htmlref{lookup table}{se:lookuptables}~ used for images.
   }
   \sstusage{
      palread palette [device]
   }
   \sstparameters{
      \sstsubsection{
         DEVICE = \htmlref{DEVICE}{se:selgradev} (Read)
      }{
         Name of the graphics device to be used.  \texttt{[}Current graphics device\texttt{{]}}
      }
      \sstsubsection{
         PALETTE = NDF (Read)
      }{
         The name of the NDF containing the palette of reserved colours
         as its data array.  The palette must be two-dimensional, the
         first dimension being 3, and the second 16.  If the second
         dimension is greater than 16 only the first 16 colours are
         used; if it has less than 16 just fill as much of the palette
         as is possible starting from the first colour.  The palette's
         values must lie in the range 0.0--1.0.
      }
   }
   \sstexamples{
      \sstexamplesubsection{
         palread rustic
      }{
         This loads the palette stored in the NDF called rustic into
         the reserved portion of the colour table of the current
         graphics device.
      }
      \sstexamplesubsection{
         palread rustic xwindows
      }{
         This loads the palette stored in the NDF called rustic into
         the reserved portion of the colour table of the xwindows
         device.
      }
   }
   \sstnotes{
      \sstitemlist{

         \sstitem
         The effects of this command will only be immediately apparent
         when run on X windows which have 256 colours (or other similar
         pseudocolour devices).  On other devices (for instance, X windows
         with more than 256 colours) the effects will only become
         apparent when subsequent graphics applications are run.
      }
   }
   \sstdiytopic{
      Related Applications
   }{
KAPPA: \htmlref{PALDEF}{PALDEF},
\htmlref{PALENTRY}{PALENTRY},
\htmlref{PALSAVE}{PALSAVE}.
   }
}

\sstroutine{
   PALSAVE
}{
   Saves the current palette of a colour table to an NDF
}{
   \sstdescription{
      This application reads the \htmlref{palette}{se:palette}~ portion of the
      \htmlref{current graphics device's}{se:devglobal}~
      colour table and saves it in an \NDFref{NDF}.  The palette
      comprises 16 colours and is intended to provide coloured
      annotations, borders, axes, graphs \emph{etc.} that are unaffected by
      changes to the \htmlref{lookup table}{se:lookuptables}~ used for images.  Thus once you have
      established a palette of colours you prefer, it is straightforward
      to recover the palette at a future time.
   }
   \sstusage{
      palsave palette [device] [title]
   }
   \sstparameters{
      \sstsubsection{
         DEVICE = \htmlref{DEVICE}{se:selgradev} (Read)
      }{
         Name of the graphics device to be used.  \texttt{[}Current graphics device\texttt{{]}}
      }
      \sstsubsection{
         PALETTE = NDF (Write)
      }{
         The NDF in which the current colour-table reserved pens are
         to be stored.  Thus if you have created non-standard colours
         for annotation, doodling, colour of axes \emph{etc.} they may be
         stored for future use.
      }
      \sstsubsection{
         TITLE = LITERAL (Read)
      }{
         Title for the output NDF.  \texttt{["KAPPA - Palsave"]}
      }
   }
   \sstexamples{
      \sstexamplesubsection{
         palsave rustic
      }{
         This saves the palette of the colour table of the current
         graphics device into the NDF called rustic.
      }
      \sstexamplesubsection{
         palsave hitec xwindows title="Hi-tech-look palette"
      }{
         This saves the palette of the colour table of the xwindows
         device in the NDF called hitec.  The NDF has a title called
         \texttt{"Hi-tech-look palette"}.
      }
   }
   \sstdiytopic{
      Related Applications
   }{
KAPPA: \htmlref{PALDEF}{PALDEF},
\htmlref{PALENTRY}{PALENTRY},
\htmlref{PALREAD}{PALREAD}.
   }
}

\sstroutine{
   PARGET
}{
   Obtains the value or values of an application parameter
}{
   \sstdescription{
      This application reports the value or values of a parameter from
      a named task.  It does this by searching the
      \htmlref{parameter file}{se:defaults}~ of
      the task.  The purpose is to offer an easier-to-use interface for
      passing values (especially \htmlref{results parameters}{se:parout})
      between tasks in
      shell scripts.  The values are formatted in lines with as many
      values as can be accommodated across the screen up to a maximum of
      132 characters; values are space separated.  However, in scripts
      the values are likely to be written to a script variable.  Thus
      for example in the C-shell:

         \hspace{2.1em}\texttt{set med = `parget median histat`}

      would redirect the output to shell variable med, and thus a
      reference to \$med would substitute the median value obtained the
      last time application \htmlref{HISTAT}{HISTAT} was invoked.  If the parameter
      comprises a vector of values these can be stored in a C-shell
      array.  For instance,

         \hspace{2.1em}\texttt{set perval = `parget perval histat`}

      would assign elements of the shell array perval[1], perval[2],
      \emph{etc.}\ to the last-computed percentile values of HISTAT.
      For other scripting languages such as Python, the alternative
      vector format produced by setting Parameter VECTOR to \texttt{TRUE}
      may be more appropriate.

      Single elements of an parameter array may also be accessed using
      the array index in parentheses.
   }
   \sstusage{
      parget parname applic
   }
   \sstparameters{
      \sstsubsection{
         APPLIC = LITERAL (Read)
      }{
         The name of the application from which the parameter comes.
      }
      \sstsubsection{
         PARNAME = LITERAL (Read)
      }{
         The parameter whose value or values are to be reported.
      }
      \sstsubsection{
         VECTOR = \_LOGICAL (Read)
      }{
         If \texttt{TRUE}, then vector parameters will be displayed as a
         comma-separated list of values enclosed in square brackets. If
         \texttt{FALSE}, vector values are printed as a space-separated list with
         no enclosing brackets. Additionally, if VECTOR is \texttt{TRUE}, string
         values (whether vector or scalar) are enclosed in single quotes
         and any embedded quotes are escaped using a backslash. \texttt{[FALSE]}
      }
   }
   \sstexamples{
      \sstexamplesubsection{
         parget mean stats
      }{
         Report the value of Parameter MEAN for the application \htmlref{STATS}{STATS}.
      }
      \sstexamplesubsection{
         parget mincoord $\backslash$
      }{
         This reports the values of Parameter MINCOORD of the current
         application, in this case STATS.
      }
      \sstexamplesubsection{
         parget applic=ndftrace parname=flabel(2)
      }{
         This reports the value of the second element of Parameter
         FLABEL for the application NDFTRACE.
      }
   }
   \sstnotes{
      \sstitemlist{

         \sstitem
         The parameter file is located in the \texttt{\$ADAM\_USER} directory, if
         defined, otherwise in the \texttt{adam} subdirectory of \texttt{\$HOME}.  If it
         cannot be located there, the task reports an error.

         \sstitem
         The parameter must exist in the selected application parameter
         file and not be a structure, except one of type ADAM\_PARNAME.

         \sstitem
         This task is not designed for use with \ICL,
         where passing parameter values is quite straightforward.  It
         does not operate with monolith parameter files.
      }
   }
}

\sstroutine{
   PASTE
}{
   Pastes a series of NDFs upon each other
}{
   \sstdescription{
      This application copies a series of \NDFref{NDFs}, in the order supplied
      and taking account of origin information, on to a `base' NDF to
      produce an output NDF.  The output NDF is therefore a copy of the
      base NDF obscured wholly or partially by the other input NDFs.
      This operation is analogous to pasting in publishing.  It is
      intended for image editing and the creation of insets.

      The dimensions of the NDFs may be different, and indeed so may
      their dimensionalities.  The output NDF can be constrained to
      have the dimensions of the base NDF, so the pasted NDFs are
      clipped.  Normally, the output NDF will have dimensions such
      that all the input NDFs are accommodated in full.

      Bad values in the pasted NDFs are by default transparent, so the
      underlying data are not replaced during the copying.

      Input NDFs can be shifted in pixel space before pasting them into
      the output NDF (see Parameter SHIFT).
   }
   \sstusage{
      paste in p1 [p2] ... [p25] out=?
   }
   \sstparameters{
      \sstsubsection{
         CONFINE = \_LOGICAL (Read)
      }{
         This parameter controls the dimensions of the output NDF.  If
         CONFINE is \texttt{FALSE} the output NDF just accommodates all the input
         NDFs.  If CONFINE is \texttt{TRUE}, the output NDF's dimensions matches
         those of the base NDF.  \texttt{[FALSE]}
      }
      \sstsubsection{
         IN = NDF (Read)
      }{
         This parameter is either: \\
         a) the base NDF on to which the other input NDFs supplied via
         Parameters P1 to P25 will be pasted; or \\
         b) a group of input NDFs (of any dimensionality) comprising all
         the input NDFs, of which the first is deemed to be the base
         NDF, and the remainder are to be pasted in the order supplied.

         The group should be given as a comma-separated list, in which
         each list element can be:

         \ssthitemlist{

            \sstitem
            an NDF name, optionally containing wild-cards and/or regular
            expressions (\texttt{"$*$"}, \texttt{"?"}, \texttt{"[a-z]"} \emph{etc.}).

            \sstitem
            the name of a text file, preceded by an up-arrow character \texttt{"$\wedge$"}.
            Each line in the text file should contain a comma-separated list
            of elements, each of which can in turn be an NDF name (with
            optional wild-cards, \emph{etc.}), or another file specification
            (preceded by an up-arrow).  Comments can be included in the file
            by commencing lines with a hash character \texttt{"\#"}.

         }
         If the value supplied for this parameter ends with a hyphen
         \texttt{"-"}, then you are re-prompted for further input until
         a value is given which does not end with a hyphen.  All the
         NDFs given in this way are concatenated into a single group.

         The group can contain no more than 1000 names.
      }
      \sstsubsection{
         OUT = NDF (Write)
      }{
         The NDF resulting from pasting of the input NDFs on to the base
         NDF.  Its dimensions may be different from the base NDF.  See
         Parameter CONFINE.
      }
      \sstsubsection{
         P1-P25 = NDF (Read)
      }{
         The NDFs to be pasted on to the base NDF.  The NDFs are pasted
         in the order P1, P2, ...  P25.  There can be no missing NDFs,
         \emph{e.g.} in order for P3 to be processed there must be a P2 given
         as well.  A null value (\texttt{{!}}) indicates that there is no NDF.
         NDFs P2 to P25 are defaulted to \texttt{{!}}.  At least one NDF must be
         pasted, therefore P1 may not be null.

         P1 to P25 are ignored if the group specified through Parameter IN
         comprises more than one NDF.
      }
      \sstsubsection{
         SHIFT( * ) = \_INTEGER (Read)
      }{
         An incremental shift to apply to the pixel origin of each input
         NDF before pasting it into the output NDF.  If supplied, this
         parameter allows a set of NDFs with the same pixel bounds to be
         placed `side-by-side' in the output NDF.  For instance, this
         allows a set of images to be pasted into a cube.  The first
         input NDF is not shifted.  The pixel origin of the second NDF
         is shifted by the number of pixels given in SHIFT.  The pixel
         origin of the third NDF is shifted by twice the number of
         pixels given in SHIFT.  Each subsequent input NDF is shifted by
         a further multiple of SHIFT.  If null (\texttt{{!}}) is supplied, no
         shifts are applied.  \texttt{[!]}
      }
      \sstsubsection{
         TITLE = LITERAL (Read)
      }{
         \htmlref{Title}{apndf:title} for the output NDF structure.  A null value (\texttt{{!}})
         propagates the title from the base NDF to the output NDF.  \texttt{[!]}
      }
      \sstsubsection{
         TRANSP = \_LOGICAL (Read)
      }{
         If TRANSP is \texttt{TRUE}, bad values within the pasted NDFs are not
         copied to the output NDF as if the bad values were transparent.
         If TRANSP is \texttt{FALSE}, all values are copied during the paste
         and a bad value will obscure an underlying value.  \texttt{[TRUE]}
      }
   }
   \sstexamples{
      \sstexamplesubsection{
         paste aa inset out=bb
      }{
         This pastes the NDF called inset on to the arrays in the NDF
         called aa to produce the NDF bb.  Bad values are transparent.
         The bounds and dimensionality of bb may be larger than those of
         aa.
      }
      \sstexamplesubsection{
         paste aa inset out=bb notransp
      }{
         As above except that bad values are copied from the NDF inset
         to NDF bb.
      }
      \sstexamplesubsection{
         paste aa inset out=bb confine
      }{
         As the first example except that the bounds of NDF bb match
         those of NDF aa.
      }
      \sstexamplesubsection{
         paste in="aa,inset" out=bb
      }{
         The same as the first example.
      }
      \sstexamplesubsection{
         paste in="aa,inset,inset2,inset3" out=bb
      }{
         Similar to first example, but now two further NDFs
         inset2 and inset3 are also pasted.
      }
      \sstexamplesubsection{
         paste ccd fudge inset out=ccdc
      }{
         This pastes the NDF called fudge, followed by NDF inset on to
         the arrays in the NDF called ccd to produce the NDF ccdc.  Bad
         values are transparent.  The bounds and dimensionality of ccd
         may be larger than those of ccdc.
      }
      \sstexamplesubsection{
         paste in="canvas,$\wedge$shapes.lis" out=collage confine
      }{
         This pastes the NDFs listed in the text file \texttt{shapes.lis} in
         the order given on the NDF called canvas.  Bad values are
         transparent.  The bounds of NDF collage match those of NDF
         canvas.
      }
      \sstexamplesubsection{
         paste in=$\wedge$planes out=cube shift=[0,0,1]
      }{
         Assuming the text file \texttt{planes} contains a list of two-dimensional
         NDFs, this arranges them into a cube, one behind the other.
      }
   }
   \sstimplementationstatus{
      \sstitemlist{

         \sstitem
         This routine correctly processes the \htmlref{AXIS}{apndf:axis}, DATA, \htmlref{QUALITY}{apndf:quality},
         \htmlref{VARIANCE}{apndf:variance}, \htmlref{LABEL}{apndf:label}, \htmlref{TITLE}{apndf:title}, \htmlref{UNITS}{apndf:units}, \htmlref{WCS}{apndf:wcs}, and \htmlref{HISTORY}{apndf:history}, components of an NDF
         data structure and propagates all \htmlref{extensions}{apndf:extensions}.  Propagation is from
 the base NDF.

         \sstitem
         Processing of \htmlref{bad pixels}{se:masking} and automatic \htmlref{quality masking}{se:qualitymask} are
         supported.

         \sstitem
         All \htmlref{non-complex numeric data types}{ap:HDStypes} can be handled.

         \sstitem
         Any number of NDF dimensions is supported.
       }
   }
}

\sstroutine{
   PERMAXES
}{
   Permute an NDF's pixel axes
}{
   \sstdescription{
     This application re-orders the pixel axes of an \NDFref{NDF}, together with
     all related information (AXIS structures, and the axes of all
     \htmlref{co-ordinate Frames}{se:domains}~ stored in the
     \htmlref{WCS}{apndf:wcs}~ component of the NDF).
   }
   \sstusage{
      permaxes in out perm
   }
   \sstparameters{
      \sstsubsection{
         IN = NDF (Read)
      }{
         The input NDF data structure.
      }
      \sstsubsection{
         OUT = NDF (Write)
      }{
         The output NDF data structure.
      }
      \sstsubsection{
         PERM() = \_INTEGER (Read)
      }{
         A list of integers defining how the pixel axes are to be permuted.
         The list must contain one element for each pixel axis in the NDF.
         The first element is the index of the pixel axis within the input
         NDF which is to become Axis 1 in the output NDF.  The second element
         is the index of the pixel axis within the input NDF which is to
         become Axis 2 in the output NDF, \emph{etc}.  Axes are numbered from 1.
      }
      \sstsubsection{
         TITLE = LITERAL (Read)
      }{
         A \htmlref{title}{apndf:title} for the output NDF.  A null value will cause the title
         of the NDF supplied for Parameter IN to be used instead.  \texttt{[!]}
      }
   }
   \sstexamples{
      \sstexamplesubsection{
         permaxes a b [2,1]
      }{
         Swaps the axes in the two-dimensional NDF called a, to produce a
         new two-dimensional NDF called b.
      }
      \sstexamplesubsection{
         permaxes a b [3,1,2]
      }{
         Creates a new three-dimensional NDF called b in which Axis 1
         corresponds to Axis 3 in the input three-dimensional NDF called a,
         Axis 2 corresponds to input Axis 1, Axis 3 corresponds to input
         Axis 2.
      }
   }
   \sstnotes{
      \sstitemlist{
         \sstitem
         If any WCS co-ordinate Frame has more axes then the number of
         pixel axes in the NDF, then the high numbered surplus axes in
         the WCS Frame are left unchanged.

         \sstitem
         If any WCS co-ordinate Frame has fewer axes then the number of
         pixel axes in the NDF, then the Frame is left unchanged if the
         specified permutation would change any of the high numbered
         surplus pixel axes.  A warning message is issued if this occurs.
      }
   }
   \sstdiytopic{
      Related Applications
   }{
KAPPA: \htmlref{ROTATE}{ROTATE},
\htmlref{FLIP}{FLIP};
\xref{FIGARO}{sun86}{}: \xref{IREVX}{sun86}{IREVX},
\xref{IREVY}{sun86}{IREVY},
\xref{IROT90}{sun86}{IROT90}.
   }
   \sstimplementationstatus{
      \sstitemlist{

         \sstitem
         This routine correctly processes the \htmlref{AXIS}{apndf:axis}, DATA, \htmlref{QUALITY}{apndf:quality},
         \htmlref{VARIANCE}{apndf:variance}, \htmlref{LABEL}{apndf:label}, \htmlref{TITLE}{apndf:title}, \htmlref{UNITS}{apndf:units}, \htmlref{WCS}{apndf:wcs}, and \htmlref{HISTORY}{apndf:history}~ components of the
         input NDF and propagates all \htmlref{extensions}{apndf:extensions}.

         \sstitem
         Processing of \htmlref{bad pixels}{se:masking} and automatic \htmlref{quality masking}{se:qualitymask} are
         supported.

         \sstitem
         All \htmlref{non-complex numeric data types}{ap:HDStypes} can be handled.  The data
         type of the input pixels is preserved in the output NDF.

         \sstitem
         Huge NDFs are supported.
      }
   }
}
\sstroutine{
   PICBASE
}{
   Selects the BASE picture from the graphics database.
}{
   \sstdescription{
      This command selects the \htmlref{BASE picture}{se:agitate}.  Subsequent
      plotting for the chosen device will be in this new current picture.  The BASE
      picture is the largest picture available, and encompasses all
      other pictures.  By default the chosen device is the current one.

      This command is a synonym for \texttt{piclist picnum=1}.
   }
   \sstusage{
      picbase
   }
   \sstparameters{
      \sstsubsection{
         DEVICE = \htmlref{DEVICE}{se:selgradev} (Read)
      }{
         The graphics workstation.  \texttt{[}The current graphics device\texttt{{]}}
      }
   }
   \sstexamples{
      \sstexamplesubsection{
         picbase
      }{
         This selects the BASE picture for the \htmlref{current graphics device}{se:devglobal}.
      }
      \sstexamplesubsection{
         picbase device=x2w
      }{
         This selects the BASE picture for the x2w device.
      }
   }
   \sstdiytopic{
      Related Applications
   }{
KAPPA: \htmlref{PICCUR}{PICCUR},
\htmlref{PICDATA}{PICDATA},
\htmlref{PICFRAME}{PICFRAME},
\htmlref{PICLAST}{PICLAST},
\htmlref{PICLIST}{PICLIST},
\htmlref{PICSEL}{PICSEL}.
   }
}

\sstroutine{
   PICCUR
}{
   Uses a graphics cursor to change the current picture
}{
   \sstdescription{
      This application allows you to select a new current picture in
      the \htmlref{graphics database}{se:agitate}~ using the cursor.  Each
      time a position is selected (usually by pressing a button on the mouse),
      details of the topmost picture in the AGI database which encompasses that
      position are displayed, together with the cursor position (in
      millimetres from the bottom -left corner of the graphics device).
      On exit the last picture selected becomes the current picture.
   }
   \sstusage{
      piccur [device] [name]
   }
   \sstparameters{
      \sstsubsection{
         DEVICE = \htmlref{DEVICE}{se:selgradev} (Read)
      }{
         The graphics workstation.  \texttt{[The current graphics device]}
      }
      \sstsubsection{
         NAME = LITERAL (Read)
      }{
         Only pictures of this name are to be selected.  For instance, if
         you want to select a DATA picture which is covered by a
         transparent FRAME picture, then you could specify NAME=\texttt{"DATA"}.
         A null (\texttt{{!}}) or blank string means that pictures of all names may
         be selected.  \texttt{[!]}
      }
      \sstsubsection{
         SINGLE = \_LOGICAL (Read)
      }{
         If \texttt{TRUE} then the user can supply only one position using the
         cursor, where-upon the application immediately exits, leaving
         the selected picture as the current picture.  If \texttt{FALSE} is supplied,
         then the user can supply multiple positions.  Once all positions
         have been supplied, a button is pressed to indicate that no more
         positions are required.  \texttt{[FALSE]}
      }
   }
   \sstexamples{
      \sstexamplesubsection{
         piccur
      }{
         This selects a picture on the \htmlref{current graphics device}{se:devglobal}~ by use of
         the cursor.  The picture containing the last-selected point becomes
         the current picture.
      }
      \sstexamplesubsection{
         piccur name=data
      }{
         This is like the previous example, but only DATA pictures can be
         selected.
      }
   }
   \sstnotes{
      \sstitemlist{

         \sstitem
         Nothing is displayed on the screen when the message filter
         environment variable MSG\_FILTER is set to \texttt{QUIET}.
      }
   }
   \sstdiytopic{
      Related Applications
   }{
KAPPA: \htmlref{CURSOR}{CURSOR},
\htmlref{PICBASE}{PICBASE},
\htmlref{PICDATA}{PICDATA},
\htmlref{PICEMPTY}{PICEMPTY},
\htmlref{PICENTIRE}{PICENTIRE},
\htmlref{PICFRAME}{PICFRAME},
\htmlref{PICLIST}{PICLIST},
\htmlref{PICSEL}{PICSEL},
\htmlref{PICVIS}{PICVIS}.
   }
}
\sstroutine{
   PICDATA
}{
   Selects the last DATA picture from the graphics database.
}{
   \sstdescription{
      This command selects the last-created \htmlref{DATA picture}{se:agitate}.
      Subsequent plotting for the chosen device will be in this new current
      picture.  By default the chosen device is the current one.

      This command is a synonym for \texttt{piclist name=data picnum=last}.
   }
   \sstusage{
      picdata
   }
   \sstparameters{
      \sstsubsection{
         DEVICE = \htmlref{DEVICE}{se:selgradev} (Read)
      }{
         The graphics workstation.  \texttt{[}The current graphics device\texttt{{]}}
      }
   }
   \sstexamples{
      \sstexamplesubsection{
         picdata
      }{
         This selects the last DATA picture for the \htmlref{current
         graphics device}{se:devglobal}.
      }
      \sstexamplesubsection{
         picdata device=xw
      }{
         This selects the last DATA picture for the xw device.
      }
   }
   \sstdiytopic{
      Related Applications
   }{
KAPPA: \htmlref{PICCUR}{PICCUR},
\htmlref{PICBASE}{PICBASE},
\htmlref{PICFRAME}{PICFRAME},
\htmlref{PICLAST}{PICLAST},
\htmlref{PICLIST}{PICLIST},
\htmlref{PICSEL}{PICSEL}.
   }
}
\sstroutine{
   PICDEF
}{
   Defines a new graphics-database FRAME picture or an array of
   FRAME pictures
}{
   \sstdescription{
      This application creates either one new
      \htmlref{graphics-database}{se:agitate}~ FRAME
      picture or a grid of new FRAME pictures.  It offers a variety of
      ways by which you can define a new picture's location and extent.
      You may constrain a new picture to lie within either the current
      or the BASE picture, and the new picture adopts the world
      co-ordinate system of that reference picture.

      You may specify a single new picture using one of three methods:
        1.  moving a cursor to define the lower and upper bounds via
            pressing choice number 1 (the application will instruct what
            to do for the specific graphics device), provided a cursor
            is available on the chosen graphics workstation;
        2.  obtaining the bounds from the environment (in world
            co-ordinates of the reference picture);
        3.  or by giving a position and size for the new picture.  The
            position is specified by a two-character code.  The first
            controls the vertical location, and may be \texttt{"T"}, \texttt{"B"}, or \texttt{"C"}
            to create the new picture at the top, bottom, or in the
            centre respectively.  The second defines the horizontal
            situation, and may be \texttt{"L"}, \texttt{"R"}, or \texttt{"C"} to define a new
            picture to the left, right, or in the centre respectively.
            Thus a code of \texttt{"BR"} will make a new picture in the
            bottom-right corner.  The size of the new picture along each
            axis may be specified either in centimetres, or as a fraction
            of the corresponding axis of the reference picture.  The picture
            may also be forced to have a specified aspect ratio.

      The picture created becomes the current picture on exit.

      Alternatively, you can create an array of
      \textit{n}-by-\textit{m} equal-sized
      pictures by giving the number of pictures in the horizontal and
      vertical directions.  These may or may not be abutted.  For easy
      reference in later processing the pictures may be labelled
      automatically.  The label consists of a prefix you define,
      followed by the number of the picture.  The numbering starts at a
      defined value, usually one, and increments by one for each new
      picture starting from the bottom-left corner and moving from left
      to right to the end of the line.  This is repeated in each line
      until the top-right picture.  Thus if the prefix were \texttt{"GALAXY"},
      the start number is one and the array comprises three pictures
      horizontally and two vertically, the top-left picture would have
      the label \texttt{"GALAXY4"}.  On completion the bottom-left picture in
      the array becomes the current picture.
   }
   \sstusage{
      picdef [mode] [fraction]
        $\left\{ {\begin{tabular}{l}
                  xpic ypic prefix=? \\
                  lbound=? ubound=?
                  \end{tabular} }
        \right.$
        \newline\latexhtml{\hspace*{12.4em}}{~~~~~~~~~~~~~~~~~~~~~~~}
        \makebox[0mm][c]{\small mode}
   }
   \sstparameters{
      \sstsubsection{
         ASPECT = \_REAL (Read)
      }{
         The aspect ratio (\textit{x}/\textit{y}) of the picture to be created
         in modes other than Cursor, Array, and XY.  The new picture is the
         largest possible with the chosen aspect ratio that will fit
         within the part of the reference picture defined by the
         fractional sizes (see Parameter FRACTION).  The justification
         comes from the value of MODE.  Thus to obtain the largest
         picture, set FRACTION=\texttt{1.0}.  A null value (\texttt{{!}})
         does not apply an aspect-ratio constraint, and therefore the new
         picture fills the part of the reference picture defined by the
         fractional sizes.  \texttt{[!]}
      }
      \sstsubsection{
         CURRENT = \_LOGICAL (Read)
      }{
         \texttt{TRUE} if the new picture is to lie within the current picture,
         otherwise the new picture can lie anywhere within the BASE
         picture.  In other words, when it is \texttt{TRUE} the current picture
         is the reference picture, and when \texttt{FALSE}, the base is the
         reference picture.  \texttt{[FALSE]}
      }
      \sstsubsection{
         DEVICE = \htmlref{DEVICE}{se:selgradev} (Read)
      }{
         The graphics device.  \texttt{[}Current graphics device\texttt{{]}}
      }
      \sstsubsection{
         FILL = \_REAL (Read)
      }{
         The linear filling factor for the Array mode.  In other words
         the fractional size (applied to both co-ordinates) of the new
         picture within each of the XPIC $*$ YPIC abutted sections of
         the picture being sub-divided.  Each new picture is located
         centrally within the section.  A filling factor of 1.0 means
         that the pictures in the array abut.  Smaller factors permit a
         gap between the pictures.  For example, FILL = 0.9 would give
         a gap between the created pictures of 10 per cent of the
         height and width of each picture, with exterior borders of 5
         per cent.  FILL must lie between 0.1 and 1.0.  \texttt{[1.0]}
      }
      \sstsubsection{
         FRACTION( ) = \_REAL (Read)
      }{
         The fractional size of the new picture along each axis,
         applicable for modes other than Array, Cursor, and XY.  Thus
         FRACTION controls the relative shape as well as the size of
         the new picture.  If only a single value is given then it is
         applied to both \textit{x} and \textit{y} axes, whereupon the new picture has
         the shape of the reference picture.  So a value of \texttt{0.5} would
         create a picture 0.25 the area of the current or BASE picture.
         The default is \texttt{0.5}, unless Parameter ASPECT is not null, when
         the default is \texttt{1.0}.  This parameter is not used if the picture
         size is specified in centimetres using Parameter SIZE.  \texttt{[]}
      }
      \sstsubsection{
         LABELNO = \_INTEGER (Read)
      }{
         The number used to form the label for the first (bottom-left)
         picture in Array mode.  It cannot be negative.  \texttt{[1]}
      }
      \sstsubsection{
         LBOUND( 2 ) = \_REAL (Read)
      }{
         Co-ordinates of the lower bound that defines the new picture.
         The suggested default is the bottom-left of the current
         picture.  (XY mode)
      }
      \sstsubsection{
         MODE = \htmlref{LITERAL}{se:parmenu} (Read)
      }{
         Method for selecting the new picture.  The options are \texttt{"Cursor"}
         for cursor mode (provided the graphics device has one), \texttt{"XY"}
         to select \textit{x}-\textit{y} limits, and \texttt{"Array"} to create a grid of
         equal-sized FRAME pictures.  The remainder are locations specified
         by two characters, the first corresponding to the vertical
         position and the second the horizontal.  For the vertical,
         valid positions are \texttt{T}(op), \texttt{B}(ottom), or \texttt{C}(entre);
         and for the horizontal the options are \texttt{L}(eft), \texttt{R}(ight),
         or \texttt{C}(entre).  \texttt{["Cursor"]}
      }
      \sstsubsection{
         OUTLINE = \_LOGICAL (Read)
      }{
         If \texttt{TRUE}, a box that delimits the new picture is drawn.  \texttt{[TRUE]}
      }
      \sstsubsection{
         PREFIX = LITERAL (Read)
      }{
         The prefix to be given to the labels of picture created in
         Array mode.  It should contain no more than twelve characters.
         If the empty string \texttt{""} is given, the pictures will have
         enumerated labels.  Note that the database can contain only
         one picture with a given label, and so merely numbering labels
         increases the chance of losing existing labels.  A \texttt{{!}} response
         means no labelling is required.  The suggested default is the
         last-used prefix.
      }
      \sstsubsection{
         SIZE( 2 ) = \_REAL (Read)
      }{
         The size of the new picture along both axes, in centimetres,
         applicable for modes other than Array, Cursor, and XY.  If a
         single value is given, it is used for both axes.  If a null
         value (\texttt{{!}}) is given, then the size of the picture is determined
         by Parameter FRACTION.  \texttt{[!]}
      }
      \sstsubsection{
         UBOUND( 2 ) = \_REAL (Read)
      }{
         Co-ordinates of the upper bound that defines the new picture.
         The suggested default is the top-right of the current picture.
         (XY mode)
      }
      \sstsubsection{
         XPIC = \_INTEGER (Read)
      }{
         The number of new pictures to be formed horizontally in the
         BASE or current picture in Array mode.  The total number of
         new pictures is XPIC $*$ YPIC.   The value must lie in the
         range 1--20.  The suggested default is \texttt{2}.
      }
      \sstsubsection{
         YPIC = \_INTEGER (Read)
      }{
         The number of new pictures to be formed vertically in the BASE
         or current picture in Array mode.  The value must lie in the
         range 1--20.  The suggested default is the value of Parameter
         XPIC.
      }
   }
   \sstexamples{
      \sstexamplesubsection{
         picdef tr
      }{
         Creates a new FRAME picture in the top-right quarter of the
         full display area on the \htmlref{current graphics device}{se:devglobal}, and an
         outline is drawn around the new picture.  This picture becomes
         the current picture.
      }
      \sstexamplesubsection{
         picdef bl aspect=1.0
      }{
         Creates a new FRAME picture within the full display area on
         the current graphics device, and an outline is drawn around
         the new picture.  This picture is the largest square possible,
         and it is justified to the bottom-left corner.  It becomes the
         current picture.
      }
      \sstexamplesubsection{
         picdef bl size=[15,10]
      }{
         Creates a new FRAME picture within the full display area on
         the current graphics device, and an outline is drawn around
         the new picture.  This picture is 15 by 10 centimetres in size
         and it is justified to the bottom-left corner.  It becomes the
         current picture.
      }
      \sstexamplesubsection{
         picdef cc 0.7 current nooutline
      }{
         Creates a new FRAME picture situated in the centre of the
         current picture on the current graphics device.  The new
         picture has the same shape as the current one, but it is
         linearly reduced by a factor of 0.7.  No outline is drawn
         around it.  The new picture becomes the current picture.
      }
      \sstexamplesubsection{
         picdef cc [0.8,0.5] current nooutline
      }{
         As above except that its height is half that of the current
         picture, and its width is 0.8 of the current picture.
      }
      \sstexamplesubsection{
         picdef cu device=graphon
      }{
         Creates a new FRAME picture within the full display area of
         the Graphon device.  The bounds of the new picture are defined
         by cursor interaction.  An outline is drawn around the new
         picture which becomes the current picture.
      }
      \sstexamplesubsection{
         picdef mode=a prefix=M xpic=3 ypic=2
      }{
         Creates six new equally sized and abutting FRAME pictures
         within the full display area of the current graphics device.
         All are outlined.  They have labels M1, M2, M3, M4, M5, and
         M6.  The bottom-left picture (M1) becomes the current picture.
      }
      \sstexamplesubsection{
         picdef mode=a prefix="" xpic=3 ypic=2 fill=0.8
      }{
         As above except that the pictures do not abut since each is
         0.8 times smaller in both dimensions, and the labels are 1,
         2, 3, 4, 5, and 6.
      }
   }
   \sstdiytopic{
      Related Applications
   }{
KAPPA: \htmlref{PICBASE}{PICBASE},
\htmlref{PICCUR}{PICCUR},
\htmlref{PICDATA}{PICDATA},
\htmlref{PICFRAME}{PICFRAME},
\htmlref{PICGRID}{PICGRID},
\htmlref{PICLABEL}{PICLABEL},
\htmlref{PICLIST}{PICLIST},
\htmlref{PICSEL}{PICSEL},
\htmlref{PICXY}{PICXY}.
   }
}
\sstroutine{
   PICEMPTY
}{
   Finds the first empty FRAME picture in the graphics database
}{
   \sstdescription{
      This application selects the first, \emph{i.e.} oldest, empty FRAME
      picture in the \htmlref{graphics database}{se:agitate}~ for a
      graphics device.  Empty
      means that there is no additional picture lying completely with
      its bounds.  This implies that the FRAME is clear for plotting.
      This task is probably most useful for plotting data in a grid of
      FRAME pictures.
   }
   \sstusage{
      picempty [device]
   }
   \sstparameters{
      \sstsubsection{
         DEVICE = \htmlref{DEVICE}{se:selgradev} (Read)
      }{
         The graphics workstation.  \texttt{[}The current graphics device\texttt{{]}}
      }
   }
   \sstexamples{
      \sstexamplesubsection{
         picempty
      }{
         This selects the first empty FRAME picture for the
         \htmlref{current graphics device}{se:devglobal}.
      }
      \sstexamplesubsection{
         picempty xwindows
      }{
         This selects the first empty FRAME picture for the xwindows
         graphics device.
      }
   }
   \sstnotes{
      \sstitemlist{

         \sstitem
         An error is returned if there is no empty FRAME picture, and
         the current picture remains unchanged.
      }
   }
   \sstdiytopic{
      Related Applications
   }{
KAPPA: \htmlref{PICENTIRE}{PICENTIRE},
\htmlref{PICGRID}{PICGRID},
\htmlref{PICLAST}{PICLAST},
\htmlref{PICLIST}{PICLIST},
\htmlref{PICSEL}{PICSEL},
\htmlref{PICVIS}{PICVIS}.
   }
   \sstdiytopic{
      Timing
   }{
      The execution time is approximately proportional to the number of
      pictures in the database before the first empty FRAME picture is
      identified.
   }
}

\sstroutine{
   PICENTIRE
}{
   Finds the first unobscured and unobscuring FRAME picture in the
   graphics database
}{
   \sstdescription{
      This application selects the first, \emph{i.e.} oldest, FRAME
      picture in the \htmlref{graphics database}{se:agitate}~ for a
      graphics device, subject to the
      following criterion.  The picture must not obstruct any other
      picture except the BASE, and must itself not be obstructed.
      Unobstructed means that there is no younger picture overlying it
      either wholly or in part.  This means that plotting can occur
      within the selected FRAME picture without overwriting or
      obscuring earlier plots.
   }
   \sstusage{
      picentire [device]
   }
   \sstparameters{
      \sstsubsection{
         DEVICE = \htmlref{DEVICE}{se:selgradev} (Read)
      }{
         The graphics workstation.  \texttt{[}The current graphics device\texttt{{]}}
      }
   }
   \sstexamples{
      \sstexamplesubsection{
         picentire
      }{
         This selects the first unobscured and unobscuring FRAME
         picture for the \htmlref{current graphics device}{se:devglobal}.
      }
      \sstexamplesubsection{
         picentire xwindows
      }{
         This selects the first unobscured and unobscuring FRAME picture
         for the xwindows graphics device.
      }
   }
   \sstnotes{
      \sstitemlist{

         \sstitem
         An error is returned if there is no suitable FRAME picture,
         and the current picture remains unchanged.

         \sstitem
         This routine cannot know whether or a picture has been cleared,
         and hence is safe to reuse, as such information is not stored in
         the graphics database.
      }
   }
   \sstdiytopic{
      Related Applications
   }{
KAPPA: \htmlref{PICEMPTY}{PICEMPTY},
\htmlref{PICGRID}{PICGRID},
\htmlref{PICLAST}{PICLAST},
\htmlref{PICLIST}{PICLIST},
\htmlref{PICSEL}{PICSEL},
\htmlref{PICVIS}{PICVIS}.
   }
   \sstdiytopic{
      Timing
   }{
      The execution time is approximately proportional to a linear
      combination of the number of pictures in the database before the
      unobstructed FRAME picture is found, and the square of the number
      of pictures in the database.
   }
}

\
\sstroutine{
   PICFRAME
}{
   Selects the last FRAME picture from the graphics database.
}{
   \sstdescription{
      This command selects the last-created \htmlref{FRAME picture}{se:agitate}.  Subsequent
      plotting for the chosen device will be in this new current
      picture.  By default the chosen device is the current one.

      This command is a synonym for \texttt{piclist name=frame picnum=last}.
   }
   \sstusage{
      picframe
   }
   \sstparameters{
      \sstsubsection{
         DEVICE = \htmlref{DEVICE}{se:selgradev} (Read)
      }{
         The graphics workstation.  \texttt{[}The current graphics device\texttt{{]}}
      }
   }
   \sstexamples{
      \sstexamplesubsection{
         picframe
      }{
         This selects the last FRAME picture for the
         \htmlref{current graphics device}{se:devglobal}.
      }
      \sstexamplesubsection{
         picframe device=xw
      }{
         This selects the last FRAME picture for the xw device.
      }
   }
   \sstdiytopic{
      Related Applications
   }{
KAPPA: \htmlref{PICBASE}{PICBASE},
\htmlref{PICCUR}{PICCUR},
\htmlref{PICDATA}{PICDATA},
\htmlref{PICLAST}{PICLAST},
\htmlref{PICLIST}{PICLIST},
\htmlref{PICSEL}{PICSEL}.
   }
}

\sstroutine{
   PICGRID
}{
   Creates an array of FRAME pictures
}{
   \sstdescription{
      This command creates a two-dimensional grid of equally sized
      new FRAME pictures in the \htmlref{graphics database}{se:agitate}.  The
      array of pictures do not have to abut, but abutting is the default.  The
      new pictures are formed within either the current or BASE picture, and
      they adopt the world co-ordinate system of that enclosing picture.  On
      completion, the bottom-left picture in the array becomes the
      current picture.

      For easy reference in later processing the pictures have integer
      labels.  The numbering starts at a defined value, usually one,
      and increments by one for each new picture starting from the
      bottom-left corner and moving from left to right to the end of
      the line.  This is repeated in each line until the top-right
      picture.

      This command is a synonym for \texttt{picdef array 1.0 prefix=""}.
   }
   \sstusage{
      picgrid xpic ypic
   }
   \sstparameters{
      \sstsubsection{
         CURRENT = \_LOGICAL (Read)
      }{
         \texttt{TRUE} if the new pictures are to lie within the current picture,
         otherwise the new pictures can lie anywhere within the BASE
         picture.  In other words, when CURRENT is \texttt{TRUE} the
         current picture is the reference picture, and when it is
         \texttt{FALSE} the BASE is the reference picture.  \texttt{[FALSE]}
      }
      \sstsubsection{
         DEVICE = \htmlref{DEVICE}{se:selgradev} (Read)
      }{
         The graphics device.  \texttt{[}Current graphics device\texttt{{]}}
      }
      \sstsubsection{
         FILL = \_REAL (Read)
      }{
         The linear filling factor for the array.  In other words
         the fractional size (applied to both co-ordinates) of the new
         picture within each of the XPIC $*$ YPIC abutted sections of
         the picture being sub-divided.  Each new picture is located
         centrally within the section.  A filling factor of 1.0 means
         that the pictures in the array abut.  Smaller factors permit a
         gap between the pictures.  For example, FILL=\texttt{0.9} would give
         a gap between the created pictures of 10 per cent of the
         height and width of each picture, with exterior borders of 5
         per cent.  FILL must lie between 0.1 and 1.0.  \texttt{[1.0]}
      }
      \sstsubsection{
         LABELNO = \_INTEGER (Read)
      }{
         The number used to form the label for the first (bottom-left)
         picture.  It cannot be negative.  \texttt{[1]}
      }
      \sstsubsection{
         OUTLINE = \_LOGICAL (Read)
      }{
         If \texttt{TRUE}, a box that delimits the new picture is drawn.  \texttt{[TRUE]}
      }
      \sstsubsection{
         XPIC = \_INTEGER (Read)
      }{
         The number of new pictures to be formed horizontally in the
         BASE picture.  The total number of new pictures is XPIC $*$ YPIC.
         The value must lie in the range 1--20.  The suggested default
         is \texttt{2}.
      }
      \sstsubsection{
         YPIC = \_INTEGER (Read)
      }{
         The number of new pictures to be formed vertically in the BASE
         picture.  The total number of new pictures is XPIC $*$ YPIC.
         The value must lie in the range 1--20.  The suggested default
         is the value of Parameter XPIC.
      }
   }
   \sstexamples{
      \sstexamplesubsection{
         picgrid 3 2
      }{
         Creates six new equally sized and abutting FRAME pictures within the
         full display area of the \htmlref{current graphics device}{se:devglobal}.  All the
         pictures are outlined.  They have labels 1, 2, 3, 4, 5, and 6.
         The bottom-left picture (1) becomes the current picture.
      }
      \sstexamplesubsection{
         picgrid xpic=3 ypic=2 fill=0.8 labelno=11 nooutline
      }{
         As above except that the pictures do not abut since each is
         0.8 times smaller in both dimensions, the labels are 11 to
         16, and there are no picture outlines drawn.
      }
      \sstexamplesubsection{
         picgrid device=xw current \
      }{
         This creates a 2-by-2 grid of new FRAME pictures within the current
         picture on device xw.
      }
   }
   \sstdiytopic{
      Related Applications
   }{
KAPPA: \htmlref{PICCUR}{PICCUR},
\htmlref{PICDEF}{PICDEF},
\htmlref{PICLABEL}{PICLABEL},
\htmlref{PICSEL}{PICSEL},
\htmlref{PICXY}{PICXY}.
   }
}
\sstroutine{
   PICIN
}{
   Finds the attributes of a picture interior to the current picture
}{
   \sstdescription{
      This application finds the attributes of a
      \htmlref{picture}{se:agitate}, selected by
      name, that was created since the current picture and lies within
      the bounds of the current picture.  The search starts from the
      most-recent picture, unless the current picture is included,
      whereupon the current picture is tested first.

      The attributes reported are the name, comment, label, name of the
      reference data object, the bounds in the \htmlref{co-ordinate Frame}{se:domains}~  selected
      by Parameter FRAME.
   }
   \sstusage{
      picin [name] [device] [frame]
   }
   \sstparameters{
      \sstsubsection{
         CURRENT = \_LOGICAL (Read)
      }{
         If this is \texttt{TRUE}, the current picture is compared against the
         chosen name before searching from the most-recent picture
         within the current picture.  \texttt{[FALSE]}
      }
      \sstsubsection{
         DESCRIBE = \_LOGICAL (Read)
      }{
         This controls whether or not the report (when REPORT=\texttt{TRUE})
         should contain a description of the Frame being used.  \texttt{[FALSE]}
      }
      \sstsubsection{
         DEVICE = \htmlref{DEVICE}{se:selgradev} (Read)
      }{
         Name of the graphics device about which information is
         required.  \texttt{[}Current graphics device\texttt{{]}}
      }
      \sstsubsection{
         EPOCH = \_DOUBLE (Read)
      }{
         If a `Sky Co-ordinate System' specification is supplied (using
         Parameter FRAME) for a celestial co-ordinate system, then an
         epoch value is needed to qualify it.  This is the epoch at
         which the displayed sky co-ordinates were determined.  It should
         be given as a decimal years value, with or without decimal places
         (\texttt{"1996.8"} for example).  Such values are interpreted as a Besselian
         epoch if less than 1984.0 and as a Julian epoch otherwise.
      }
      \sstsubsection{
         FRAME = LITERAL (Read)
      }{
         A string determining the co-ordinate Frame in which the bounds
         of the picture are to be reported.  When a picture is
         created by an application such as \htmlref{PICDEF}{PICDEF},
         \htmlref{DISPLAY}{DISPLAY}, the \htmlref{WCS}{apndf:wcs}
         information describing the available co-ordinate systems are stored
         with the picture in the graphics database.  This application can
         report bounds in any of the co-ordinate Frames stored with the
         current picture.  The string supplied for FRAME can be one of the
         following:

         \ssthitemlist{

            \sstitem
            A \htmlref{domain name}{se:domains}~ such as \htmlref{SKY, AXIS, PIXEL, NDC, BASEPIC, CURPIC}{se:resdoms}.
            the special domain AGI\_WORLD is used to refer to the world co-ordinate
            system stored in the AGI graphics database.  This can be useful if
            no WCS information was store with the picture when it was created.

            \sstitem
            An integer value giving the index of the required Frame.

            \sstitem
            An IRAS90 \emph{Sky Co-ordinate System} (SCS) values such as
            \texttt{"EQUAT(J2000)"} (see \xref{SUN/163}{sun163}{}).

         }
         If a null value (\texttt{{!}}) is supplied, bounds are reported in the
         co-ordinate Frame which was current when the picture was created.
         \texttt{[!]}
      }
      \sstsubsection{
         NAME = LITERAL (Read)
      }{
         The name of the picture to be found within the current picture.
         If it is null (\texttt{{!}}), the first interior picture is selected.
         \texttt{[DATA]}
      }
      \sstsubsection{
         PNAME = LITERAL (Write)
      }{
         The name of the picture.
      }
      \sstsubsection{
         REPORT = \_LOGICAL (Read)
      }{
         If this is \texttt{FALSE} details of the picture are not reported, merely the
         results are written to the output parameters.  It is intended for
         use within procedures.  \texttt{[TRUE]}
      }
   }
   \sstresparameters{
      \sstsubsection{
         COMMENT = LITERAL (Write)
      }{
         The comment of the picture.  Up to 132 characters will be written.
      }
      \sstsubsection{
         DOMAIN = LITERAL (Write)
      }{
         The Domain name of the current co-ordinate Frame for the picture.
      }
      \sstsubsection{
         LABEL = LITERAL (Write)
      }{
         The label of the picture.  It is blank if there is no label.
      }
      \sstsubsection{
         REFNAM = LITERAL (Write)
      }{
         The reference object associated with the picture.  It is blank if
         there is no reference object.  Up to 132 characters will be written.
      }
      \sstsubsection{
         X1 = LITERAL (Write)
      }{
         The lowest value found within the  picture for Axis 1 of the
         requested co-ordinate Frame (see Parameter FRAME).
      }
      \sstsubsection{
         X2 = LITERAL (Write)
      }{
         The highest value found within the  picture for Axis 1 of the
         requested co-ordinate Frame (see Parameter FRAME).
      }
      \sstsubsection{
         Y1 = LITERAL (Write)
      }{
         The lowest value found within the  picture for Axis 2 of the
         requested co-ordinate Frame (see Parameter FRAME).
      }
      \sstsubsection{
         Y2 = LITERAL (Write)
      }{
         The highest value found within the  picture for Axis 2 of the
         requested co-ordinate Frame (see Parameter FRAME).
      }
   }
   \sstexamples{
      \sstexamplesubsection{
         picin
      }{
         This reports the attributes of the last DATA picture within
         the current picture for the \htmlref{current graphics device}{se:devglobal}.  The bounds
         of the picture in its current co-ordinate Frame are reported.
      }
      \sstexamplesubsection{
         picin frame=pixel
      }{
         As above but the bounds of the picture in the PIXEL Frame are
         reported.
      }
      \sstexamplesubsection{
         picin refnam=(object) current
      }{
         This reports the attributes of the last data picture within
         the current picture for the current graphics device.  If there
         is a reference data object, its name is written to the ICL
         variable OBJECT.  The search includes the current picture.
      }
      \sstexamplesubsection{
         picin x1=(x1) x2=(x2) y1=(y1) y2=(y2)
      }{
         This reports the attributes of the last DATA picture within
         the current picture for the current graphics device.  The
         bounds of the current picture are written to the
         {\ICLref\normalsize}~ variables: X1, X2, Y1, Y2.
      }
   }
   \sstnotes{
      This application is intended for use within procedures.  Also if
      a DATA picture is selected and the current picture is included in
      the search, this application informs about the same picture that
      an application that works in a cursor
      \htmlref{interaction mode}{se:interaction}~ would select, and so
      acts as a check that the correct picture will be accessed.
   }
   \sstdiytopic{
      Related Applications
   }{
KAPPA: \htmlref{GDSTATE}{GDSTATE},
\htmlref{PICDEF}{PICDEF},
\htmlref{PICLIST}{PICLIST},
\htmlref{PICTRANS}{PICTRANS},
\htmlref{PICXY}{PICXY}.
   }
}
\sstroutine{
   PICLABEL
}{
   Labels the current graphics-database picture
}{
   \sstdescription{
      This application annotates the current
      \htmlref{graphics-database}{se:agitate}~ picture
      of a specified device with a label you define.  This provides an
      easy-to-remember handle for selecting pictures in subsequent
      processing.
   }
   \sstusage{
      piclabel label [device]
   }
   \sstparameters{
      \sstsubsection{
         DEVICE = \htmlref{DEVICE}{se:selgradev} (Read)
      }{
         The graphics device.  \texttt{[}Current graphics device\texttt{{]}}
      }
      \sstsubsection{
         LABEL = LITERAL (Read)
      }{
         The label to be given to the current picture.  It is limited
         to 15 characters, but may be in mixed case.  If it is null (\texttt{{!}})
         a blank label is inserted in the database.
      }
   }
   \sstexamples{
      \sstexamplesubsection{
         piclabel GALAXY
      }{
         This makes the current picture of the \htmlref{current graphics device}{se:devglobal}
         have a label of \texttt{"GALAXY"}.
      }
      \sstexamplesubsection{
         piclabel A3 x2w
      }{
         This labels the current picture on the x2w device \texttt{"A3"}.
      }
   }
   \sstnotes{
      The label must be unique for the chosen device.  If the new label
      clashes with an existing one, then the existing label is deleted.
   }
   \sstdiytopic{
      Related Applications
   }{
KAPPA: \htmlref{PICDEF}{PICDEF},
\htmlref{PICLIST}{PICLIST},
\htmlref{PICSEL}{PICSEL}.
   }
}

\sstroutine{
   PICLAST
}{
   Selects the last picture from the graphics database.
}{
   \sstdescription{
      This command selects the last-created
      \htmlref{picture}{se:agitate}.  Subsequent
      plotting for the chosen device will be in this new current picture.
      By default the chosen device is the current one.

      This command is a synonym for \texttt{piclist picnum=last}.
   }
   \sstusage{
      piclast
   }
   \sstparameters{
      \sstsubsection{
         DEVICE = \htmlref{DEVICE}{se:selgradev} (Read)
      }{
         The graphics workstation.  \texttt{[}current graphics device\texttt{{]}}
      }
   }
   \sstexamples{
      \sstexamplesubsection{
         piclast
      }{
         This selects the last picture for the
         \htmlref{current graphics device}{se:devglobal}.
      }
      \sstexamplesubsection{
         piclast device=x2w
      }{
         This selects the last picture for the x2w device.
      }
   }
   \sstdiytopic{
      Related Applications
   }{
KAPPA: \htmlref{PICBASE}{PICBASE},
\htmlref{PICCUR}{PICCUR},
\htmlref{PICDATA}{PICDATA},
\htmlref{PICFRAME}{PICFRAME},
\htmlref{PICLIST}{PICLIST},
\htmlref{PICSEL}{PICSEL}.
   }
}

\sstroutine{
   PICLIST
}{
   Lists the pictures in the graphics database for a device
}{
   \sstdescription{
      This application produces a summary of the contents of the
      \htmlref{graphics database}{se:agitate}~ for a graphics device,
      and/or permits a picture
      specified by its order in the list to be made the new current
      picture.  The list may either be reported or written to a text file.

      The headed list has one line per picture.  Each line comprises
      a reference number; the picture's name, comment (up to 24
      characters), and label; and a flag to indicate whether or not
      there is a reference data object associated with the picture.  A
      \texttt{"C"} in the first column indicates that the picture that was
      current when this application was invoked.  In the text file,
      because there is more room, the name of the reference object is
      given (up to 64 characters) instead of the reference flag.
      Pictures are listed in chronological order of their creation.
   }
   \sstusage{
      piclist [name] [logfile] [device] picnum=?
   }
   \sstparameters{
      \sstsubsection{
         DEVICE = \htmlref{DEVICE}{se:selgradev} (Read)
      }{
         The graphics workstation.  \texttt{[}The current graphics device\texttt{{]}}
      }
      \sstsubsection{
         LOGFILE = FILENAME (Write)
      }{
         The name of the text file in which the list of pictures will
         be made.  A null string (\texttt{{!}}) means the list will be reported
         to you.  The suggested default is the current value.  \texttt{[!]}
      }
      \sstsubsection{
         NAME = LITERAL (Read)
      }{
         Only pictures of this name are to be selected.  A null string
         (\texttt{{!}}) or blanks means that pictures of all names may be selected.
         \texttt{[!]}
      }
      \sstsubsection{
         PICNUM = LITERAL (Read)
      }{
         The reference number of the picture to be made the current
         picture when the application exits.  PICNUM=\texttt{"Last"} selects the
         last picture in the database.  Parameter PICNUM is not accessed
         if the list is written to the text file.  A null (\texttt{{!}}) or any
         other error causes the current picture on entry to be current
         again on exit.  The suggested default is null.
      }
   }
   \sstexamples{
      \sstexamplesubsection{
         piclist
      }{
         This reports all the pictures in the graphics database for the
         \htmlref{current graphics device}{se:devglobal}.
      }
      \sstexamplesubsection{
         piclist device=ps\_l
      }{
         This reports all the pictures in the graphics database for the
         ps\_l device.
      }
      \sstexamplesubsection{
         piclist data
      }{
         This reports all the DATA pictures in the graphics database for
         the current graphics device.
      }
      \sstexamplesubsection{
         piclist data logfile=datapic.dat
      }{
         This lists all the DATA pictures in the graphics database for
         the current graphics device into the text file \texttt{datapic.dat}.
      }
      \sstexamplesubsection{
         piclist frame picnum=5
      }{
         This selects the fifth most ancient FRAME picture (in the
         graphics database for the current graphics device) as the
         current picture.  The pictures are not listed.
      }
      \sstexamplesubsection{
         piclist picnum=last
      }{
         This makes the last picture in the graphics database for the
         current graphics device the current picture.  The pictures are
         not listed.
      }
   }
   \sstdiytopic{
      Related Applications
   }{
KAPPA: \htmlref{PICBASE}{PICBASE},
\htmlref{PICDATA}{PICDATA},
\htmlref{PICEMPTY}{PICEMPTY},
\htmlref{PICENTIRE}{PICENTIRE},
\htmlref{PICFRAME}{PICFRAME},
\htmlref{PICIN}{PICIN},
\htmlref{PICLAST}{PICLAST},
\htmlref{PICSEL}{PICSEL},
\htmlref{PICVIS}{PICVIS}.
   }
   \sstdiytopic{
      Timing
   }{
      The execution time is approximately proportional to the number of
      pictures in the database for the chosen graphics device.
      Selecting only a subset by name is slightly faster.
   }
}

\sstroutine{
   PICSEL
}{
   Selects a graphics-database picture by its label
}{
   \sstdescription{
      This application selects by label a
      \htmlref{graphics-database}{se:agitate}~ picture of a
      specified device.  If such a picture is found then it becomes the
      current picture on exit, otherwise the input picture remains
      current.  Labels in the database are stored in the case supplied
      when they were created.  However, the comparisons of the label you
      supply with the labels in the database are made in uppercase, and
      leading spaces are ignored.

      Should the label not be found the current picture is unchanged.
   }
   \sstusage{
      picsel label [device]
   }
   \sstparameters{
      \sstsubsection{
         DEVICE = \htmlref{DEVICE}{se:selgradev} (Read)
      }{
         The graphics device.  \texttt{[}Current graphics device\texttt{{]}}
      }
      \sstsubsection{
         LABEL = LITERAL (Read)
      }{
         The label of the picture to be selected.
      }
   }
   \sstexamples{
      \sstexamplesubsection{
         picsel GALAXY
      }{
         This makes the picture labelled \texttt{"GALAXY"} the current picture on
         the \htmlref{current graphics device}{se:devglobal}.  Should there be no picture of
         this name, the current picture is unchanged.
      }
      \sstexamplesubsection{
         picsel A3 xwindows
      }{
         This makes the picture labelled \texttt{"A3"} the current picture on the
         xwindows device.  Should there be no picture of this name, the current
         picture is unchanged.
      }
   }
   \sstdiytopic{
      Related Applications
   }{
KAPPA: \htmlref{PICDATA}{PICDATA},
\htmlref{PICDEF}{PICDEF},
\htmlref{PICEMPTY}{PICEMPTY},
\htmlref{PICENTIRE}{PICENTIRE},
\htmlref{PICFRAME}{PICFRAME},
\htmlref{PICLABEL}{PICLABEL},
\htmlref{PICLAST}{PICLAST},
\htmlref{PICVIS}{PICVIS}.
   }
}
\sstroutine{
   PICTRANS
}{
   Transforms a graphics position from one picture co-ordinate Frame to
   another
}{
   \sstdescription{
      This application transforms a position on a graphics device from one
      \htmlref{co-ordinate Frame}{se:domains}~  to another.  The input and output Frames may be chosen
      freely from the Frames available in the \htmlref{WCS}{apndf:wcs}~ information stored with
      the current picture in the \htmlref{AGI graphics database}{se:agitate}.  The transformed
      position is formatted for display and written to the screen and
      also to an output parameter.
   }
   \sstusage{
      pictrans posin framein [frameout] [device]
   }
   \sstparameters{
      \sstsubsection{
         DEVICE = \htmlref{DEVICE}{se:selgradev} (Read)
      }{
         The graphics workstation.  \texttt{[The current graphics device]}
      }
      \sstsubsection{
         EPOCHIN = \_DOUBLE (Read)
      }{
         If a `Sky Co-ordinate System' specification is supplied (using
         Parameter FRAMEIN) for a celestial co-ordinate system, then an epoch
         value is needed to qualify it.  This is the epoch at which the
         supplied sky position was determined.  It should be given as a
         decimal years value, with or without decimal places  (\texttt{"1996.8"} for
         example).  Such values are interpreted as a Besselian epoch if less
         than 1984.0 and as a Julian epoch otherwise.
      }
      \sstsubsection{
         EPOCHOUT = \_DOUBLE (Read)
      }{
         If a `Sky Co-ordinate System' specification is supplied (using
         Parameter FRAMEOUT) for a celestial co-ordinate system, then an epoch
         value is needed to qualify it.  This is the epoch at which the
         transformed sky position is required.  It should be given as a
         decimal years value, with or without decimal places  (\texttt{"1996.8"} for
         example).  Such values are interpreted as a Besselian epoch if less
         than 1984.0 and as a Julian epoch otherwise.
      }
      \sstsubsection{
         FRAMEIN = LITERAL (Read)
      }{
         A string specifying the co-ordinate Frame in which the input
         position is supplied (see Parameter POSIN).  The string can be
         one of the following options.

         \ssthitemlist{

            \sstitem
            A \htmlref{domain name}{se:domains}~ such as \htmlref{SKY, AXIS, PIXEL, NDC, BASEPIC, CURPIC}{se:resdoms}.

            \sstitem
            An integer value giving the index of the required Frame within
            the \htmlref{WCS}{apndf:wcs} component.

            \sstitem
            An IRAS90 \emph{Sky Co-ordinate System} (SCS) values such as
            \texttt{"EQUAT(J2000)"} (see \xref{SUN/163}{sun163}{}).

         }
         If a null parameter value is supplied, then the
         \htmlref{current Frame}{se:curframe}~ in
         the current picture is used.  \texttt{[!]}
      }
      \sstsubsection{
         FRAMEOUT = LITERAL (Read)
      }{
         A string specifying the co-ordinate Frame in which the transformed
         position is required.  If a null parameter value is supplied, then
         the current Frame in the picture is used.  The string can be one of
         the following options.

         \ssthitemlist{

            \sstitem
            A domain name such as SKY, AXIS, PIXEL, GRAPHICS, CURPIC,
            NDC, BASEPIC.

            \sstitem
            An integer value giving the index of the required Frame within
            the WCS component.

            \sstitem
            An IRAS90 \emph{Sky Co-ordinate System} (SCS) values such as
            \texttt{"EQUAT(J2000)"} (see SUN/163).

         }
         If a null parameter value is supplied, then the BASEPIC Frame is
         used.  \texttt{["BASEPIC"]}
      }
      \sstsubsection{
         POSIN = LITERAL (Read)
      }{
         The co-ordinates of the position to be transformed, in the
         co-ordinate Frame specified by Parameter FRAMEIN (supplying
         a colon \texttt{":"} will display details of the required co-ordinate Frame).
         The position should be supplied as a list of
         \xref{formatted axis values}{sun210}{AST_UNFORMAT}
         separated by spaces or commas.
      }
   }
   \sstresparameters{
      \sstsubsection{
         BOUND = \_LOGICAL (Write)
      }{
         BOUND is \texttt{TRUE} when the supplied point lies within the bounds of
         the current picture.
      }
      \sstsubsection{
         POSOUT = LITERAL (Write)
      }{
         The formatted co-ordinates of the transformed position, in the
         co-ordinate Frame specified by Parameter FRAMEOUT.  The position
         will be stored as a list of formatted axis values separated by
         spaces.
      }
   }
   \sstexamples{
      \sstexamplesubsection{
         pictrans "100.3,-20.1" framein=pixel
      }{
         This converts the position (100.3,~$-$20.1), in pixel co-ordinates
         within the current picture of the \htmlref{current graphics device}{se:devglobal}, to the
         BASEPIC co-ordinates of that point in the BASE picture.
      }
      \sstexamplesubsection{
         pictrans "100.3,-20.1" framein=pixel frameout=graphics
      }{
         This converts the position (100.3,~$-$20.1), in pixel co-ordinates
         within the current picture of the current graphics device, to the
         GRAPHICS co-ordinates of that point (\emph{i.e.} millimetres from the
         bottom-left corner of the graphics device).
      }
      \sstexamplesubsection{
         pictrans "10 10" framein=graphics frameout=basepic
      }{
         This converts the position (10,~10), in graphics co-ordinates
         (\emph{i.e.} the point which is 10mm above and to the right of the
         lower-left corner of the graphics device), into BASEPIC
         co-ordinates.
      }
   }
   \sstnotes{

      \ssthitemlist{

         \sstitem
         BASEPIC co-ordinates locate a position within the entire graphics
         device.  The bottom-left corner of the device screen has BASEPIC
         co-ordinates of (0,~0).  The shorter dimension of the screen has
         length 1.0, and the other axis has a length greater than 1.0.

         \sstitem
         NDC co-ordinates are like BASEPIC co-ordinates except that
         the top-right corner of the screen has NDC co-ordinates
         of (1,~1).  That is, both axes of the screen have unit length.

         \sstitem
         GRAPHICS co-ordinates also span the entire graphics device but
         are measured in millimetres from the bottom-left corner.

         \sstitem
         CURPIC co-ordinates locate a point within the current picture.  The
         bottom-left corner of the current picture has CURPIC co-ordinates of
         (0,~0).  The shorter dimension of the current picture has length 1.0, and
         the other axis has a length greater than 1.0.

         \sstitem
         The transformed position is not written to the screen when the
         message filter environment variable MSG\_FILTER is set to \texttt{QUIET}.
         The creation of the output Parameter POSOUT is unaffected
         by MSG\_FILTER.

      }
   }
   \sstdiytopic{
      Related Applications
   }{
KAPPA: \htmlref{GDSTATE}{GDSTATE},
\htmlref{PICIN}{PICIN},
\htmlref{PICXY}{PICXY}.
   }
}
\sstroutine{
   PICVIS
}{
   Finds the first unobscured FRAME picture in the graphics database
}{
   \sstdescription{
      This application selects the first, \emph{i.e.} oldest, unobstructed
      FRAME picture in the \htmlref{graphics database}{se:agitate}~ for a graphics device.
      Unobstructed means that there is no younger picture overlying it
      either wholly or in part.
   }
   \sstusage{
      picvis [device]
   }
   \sstparameters{
      \sstsubsection{
         DEVICE = \htmlref{DEVICE}{se:selgradev} (Read)
      }{
         The graphics workstation.  \texttt{[}The current graphics device\texttt{{]}}
      }
   }
   \sstexamples{
      \sstexamplesubsection{
         picvis
      }{
         This selects the first unobscured FRAME picture for the
         \htmlref{current graphics device}{se:devglobal}.
      }
      \sstexamplesubsection{
         picvis xwindows
      }{
         This selects the first unobscured FRAME picture for the
         xwindows graphics device.
      }
   }
   \sstnotes{
      \sstitemlist{

         \sstitem
         An error is returned if there is no unobscured FRAME picture,
         and the current picture remains unchanged.

         \sstitem
         This routine cannot know whether or a picture has been cleared,
         and hence is safe to reuse, as such information is not stored in
         the graphics database.
      }
   }
   \sstdiytopic{
      Related Applications
   }{
KAPPA: \htmlref{PICEMPTY}{PICEMPTY},
\htmlref{PICENTIRE}{PICENTIRE},
\htmlref{PICGRID}{PICGRID},
\htmlref{PICLAST}{PICLAST},
\htmlref{PICLIST}{PICLIST},
\htmlref{PICSEL}{PICSEL}.
   }
   \sstdiytopic{
      Timing
   }{
      The execution time is approximately proportional to a linear
      combination of the number of pictures in the database before the
      unobstructed FRAME picture is found, and the square of the number
      of pictures in the database.
   }
}

\sstroutine{
   PICXY
}{
   Creates a new FRAME picture defined by co-ordinate bounds
}{
   \sstdescription{
      This command creates a new FRAME picture in the
      \htmlref{graphics database}{se:agitate}.
      The bounds of the new picture are defined through two parameters.
      The new picture is formed within either the current or BASE
      picture, and it adopts the world co-ordinate system of that
      reference picture.  On completion the new picture becomes the
      current picture.

      This command is a synonym for \texttt{picdef xy 1.0}.
   }
   \sstusage{
      picxy lbound ubound
   }
   \sstparameters{
      \sstsubsection{
         CURRENT = \_LOGICAL (Read)
      }{
         \texttt{TRUE} if the new picture is to lie within the current picture,
         otherwise the new picture can lie anywhere within the BASE
         picture.  In other words, when CURRENT is \texttt{TRUE} the
         current picture is the reference picture, and when it is
         \texttt{FALSE} the base is the reference picture.  \texttt{[FALSE]}
      }
      \sstsubsection{
         DEVICE = \htmlref{DEVICE}{se:selgradev} (Read)
      }{
         The graphics device.  \texttt{[}\htmlref{current graphics device}{se:devglobal}\texttt{{]}}
      }
      \sstsubsection{
         LBOUND( 2 ) = \_REAL (Read)
      }{
         Co-ordinates of the lower bound that defines the new picture.
         The suggested default is the bottom-left of the current
         picture.
      }
      \sstsubsection{
         OUTLINE = \_LOGICAL (Read)
      }{
         If \texttt{TRUE}, a box that delimits the new picture is drawn.  \texttt{[TRUE]}
      }
      \sstsubsection{
         UBOUND( 2 ) = \_REAL (Read)
      }{
         Co-ordinates of the upper bound that defines the new picture.
         The suggested default is the top-right of the current picture.
      }
   }
   \sstexamples{
      \sstexamplesubsection{
         picxy [0.1,0.2] [0.9,0.6]
      }{
         This creates a new FRAME picture in the BASE picture extending from
         (0.1,~0.2) to (0.9,~0.6), which becomes the new current picture.
         An outline is drawn around the picture.
      }
      \sstexamplesubsection{
         picxy ubound=[1.1,0.9] lbound=[0.1,0.2] current nooutline
      }{
         This creates a new FRAME picture in the current picture extending
         from (0.1,~0.2) to (1.1,~0.9), which becomes the new current
         picture.  No outline is drawn.
      }
   }
   \sstdiytopic{
      Related Applications
   }{
KAPPA: \htmlref{PICCUR}{PICCUR},
\htmlref{PICDEF}{PICDEF},
\htmlref{PICGRID}{PICGRID},
\htmlref{PICSEL}{PICSEL}.
   }
}

\sstroutine{
   PIXBIN
}{
   Places each pixel value in an input NDF into an output bin
}{
   \sstdescription{
      This application collects groups of pixel values together from an
      input NDF and places each group into a single column of an output
      NDF. Each such output column represents a ``bin'' into
      which a group of input pixels is placed.

      If the input NDF has $N$ pixel axes, the user provides a set of $M$
      $N$-dimensional ``index'' NDFs (where $M$ is between 1
      and 6). For each pixel in the input NDF, the corresponding value in
      each of the $M$ index NDFs is found. This vector of $M$ values is used
      (after rounding them to the nearest integer) to determine the pixel
      indices within the output ($M$-dimensional) NDF at which to store
      the input pixel value. Thus each output pixel corresponds to a bin
      into which one or more input pixels can be placed, as selected by
      the index NDFs.

      There are many possible ways in which the input pixels values
      that fall in a single bin could be combined to create a single
      representative output value for each bin. For instance, the output
      NDF could contain the mean of the input values that fall in each
      bin, or the maximum, or the standard deviation, etc. However, this
      application does not store a single representative value for each bin.
      Instead it stores all the separate input pixel values that fall in
      each bin. This requires an extra trailing pixel axis in the output
      NDF, with a lower pixel bounds of 1 and and upper pixel bound equal to
      the maximum population of any bin. Each ``column'' of values parallel
      to this final output pixel axis represents one bin, and
      contains the corresponding input pixel values at its lower end, with
      bad values filling any unused higher pixels. The
      \htmlref{COLLAPSE}{COLLAPSE} application
      could then be used to get a representative value for each bin by
      collapsing this final pixel axis using any of the many estimators
      provided by COLLAPSE.

      An extra group of $M$ NDFs can be supplied that define the WCS to
      be stored in the output NDF---see Parameter WCS.
   }
   \sstusage{
      pixbin in out index [wcs]
   }
   \sstparameters{
      \sstsubsection{
         IN = NDF (Read)
      }{
         The input $N$-dimensional NDF.
      }
      \sstsubsection{
         INDEX = NDF (Read)
      }{
         A group of index NDFs (all with $N$-dimensions). The number of index
         NDFs (referred to below as ``$M$'') supplied should be in the range
         1--6 and determines the dimensionality of the output NDF. A section
         is taken from each one so that it matches the input NDF supplied by
         Parameter IN. The data values in the $J$th index NDF are
         converted to \_INTEGER (by finding the nearest integer) and then
         used as the pixel indices on the $J$th output pixel axis.
      }
      \sstsubsection{
         OUT = NDF (Write)
      }{
         The output NDF containing all the values from the input NDF
         collected into a set of bins. This NDF will have $M+1$ pixel axes,
         where $M$ is the number of index NDF supplied using Parameter INDEX.
         The final pixel axis enumerates the individual input pixels that
         fall within each bin.
      }
      \sstsubsection{
         WCS = NDF (Read)
      }{
         An optional group of NDFs (all with $N$-dimensions) that define the
         WCS to be stored in the output NDF. The number of NDFs in this
         group should be $M$, the number of index NDFs (see Parameter INDEX).
         The data values in the $J$th WCS NDF determine the values
         to be stored for the $J$th axis in the WCS of the output
         NDF (the WCS values on the final trailing axis in the output NDF,
         axis $M+1$, are just equal to pixel index). If a null (\texttt{!})
         value is supplied, no WCS is stored in the output NDF. The WCS
         values for each of the first $M$ output axes are described using a
         look-up-table (one for each axis) that converts value in an index
         NDF into the corresponding value in a WCS NDF. For all pixels with
         the same integer index value, the mean of the corresponding WCS
         values is found and stored in the look-up-table. The label and unit
         for each axis is taken from the Label and Unit components of the
         corresponding WCS NDF. \texttt{[!]}
      }
   }
   \sstexamples{
      \sstexamplesubsection{
         pixbin m31 binned radius
      }{
         Here the pixel values in a two-dimensional NDF called m31 are
         placed into bins as defined by the contents of a single two-dimensional
         NDF called radius, to create a two-dimensional output NDF called
         binned.  (The number of pixel axes in the output is always one
         more than the number of index NDFs.) The data values in NDF radius
         are used as the pixel indices along the first axis of the output
         NDF, at which to store each input pixel value. Each column in the
         output NDF contains the individual input pixel values assigned
         to that bin, padded with bad values if necessary to fill the
         column.
      }
   }
   \sstdiytopic{
      Related Applications
   }{
      KAPPA: \htmlref{COLLAPSE}{COLLAPSE}
   }
   \sstimplementationstatus{
      \sstitemlist{

         \sstitem
         This routine correctly processes the DATA, \htmlref{QUALITY}{apndf:quality},
         \htmlref{VARIANCE}{apndf:variance}, \htmlref{LABEL}{apndf:label}, \htmlref{TITLE}{apndf:title},
         \htmlref{UNITS}{apndf:units}, and \htmlref{HISTORY}{apndf:history}, components of the input NDF
         and propagates all \htmlref{extensions}{apndf:extensions}.

         \sstitem
          Processing of bad pixels and automatic quality masking are
         supported.

         \sstitem
          All non-complex numeric data types can be handled.
      }
   }
}

\sstroutine{
   PIXDUPE
}{
   Expands an NDF by pixel duplication
}{
   \sstdescription{
      This routine expands the size of an NDF structure by duplicating
      each input pixel a specified number of times along each dimension,
      to create a new NDF structure. Each block of output pixels (formed
      by duplicating a single input pixel) can optionally be masked, for
      instance to set selected pixels within each output block bad.
   }
   \sstusage{
      pixdupe in out expand
   }
   \sstparameters{
      \sstsubsection{
         EXPAND() = \_INTEGER (Read)
      }{
         Linear expansion factors to be used to create the new data
         array.  The number of factors should equal the number of
         dimensions in the input NDF.  If fewer are supplied the last
         value in the list of expansion factors is given to the
         remaining dimensions.  Thus if a uniform expansion is required
         in all dimensions, just one value need be entered.  If the net
         expansion is one, an error results.  The suggested default is
         the current value.
      }
      \sstsubsection{
         IMASK() = \_INTEGER (Read)
      }{
        Only used if a null (\texttt{{!}}) value is supplied for Parameter MASK.
        If accessed, the number of values supplied for this parameter should
        equal the number of pixel axes in the output NDF.  A mask array
        is then created which has bad values at every element except for
        the element with indices given by IMASK, which is set to the value
        \texttt{1.0}.  See Parameter MASK for a description of the use of the mask
        array. If a null value is supplied for IMASK, then no mask
        is used, and every output pixel in an output block is set to the
        value of the corresponding input pixel.  \texttt{[!]}
      }
      \sstsubsection{
         IN = NDF (Read)
      }{
         Input NDF structure to be expanded.
      }
      \sstsubsection{
         MASK = NDF (Read)
      }{
        An input NDF structure holding the mask to be used. If a null (\texttt{{!}})
        value is supplied, Parameter IMASK will be used to determine the
        mask. If supplied, the NDF Data array will be trimmed or padded
        (with bad values) to create an array in which the lengths of the
        pixel axes are equal to the values supplied for Parameter EXPAND.
        Each block of pixels in the output array (\emph{i.e.} the block of
        output pixels which are created from a single input pixel) are
        multiplied by this mask array before being stored in the output
        NDF.  \texttt{[!]}
      }
      \sstsubsection{
         OUT = NDF (Write)
      }{
         Output NDF structure.
      }
      \sstsubsection{
         TITLE = LITERAL (Read)
      }{
         \htmlref{Title}{apndf:title} for the output NDF structure.  A null value (\texttt{{!}})
         propagates the title from the input NDF to the output NDF.  \texttt{[!]} }
   }
   \sstexamples{
      \sstexamplesubsection{
         pixdupe aa bb 2
      }{
         This expands the NDF called aa duplicating pixels along each

         dimension, and stores the enlarged data in the NDF called bb.
         Thus if aa is two-dimensional, this command would result in a
         four-fold increase in the array components.
      }
      \sstexamplesubsection{
         pixdupe cosmos galaxy [2,1]
      }{
         This expands the NDF called cosmos by duplicating along the
         first axis, and stores the enlarged data in the NDF called
         galaxy.
      }
      \sstexamplesubsection{
         pixdupe cube1 cube2 [3,1,2] title="Reconfigured cube"
      }{
         This expands the NDF called cube1 by having three pixels for
         each pixel along the first axis and duplicating along the
         third axis, and stores the enlarged data in the NDF called
         cube2.  The title of cube2 is \texttt{"Reconfigured cube"}.
      }
   }
   \sstdiytopic{
      Related Applications
   }{
KAPPA: \htmlref{COMPADD}{COMPADD},
\htmlref{COMPAVE}{COMPAVE},
\htmlref{COMPICK}{COMPICK},
\htmlref{INTERLEAVE}{INTERLEAVE}.
   }
   \sstimplementationstatus{
      \sstitemlist{

         \sstitem
         This routine correctly processes the \htmlref{AXIS}{apndf:axis}, DATA, \htmlref{QUALITY}{apndf:quality},
         \htmlref{VARIANCE}{apndf:variance}, \htmlref{LABEL}{apndf:label}, \htmlref{TITLE}{apndf:title}, \htmlref{UNITS}{apndf:units}, \htmlref{WCS}{apndf:wcs}, and \htmlref{HISTORY}{apndf:history}, components of an NDF
         data structure, and propagates all \htmlref{extensions}{apndf:extensions}.

         \sstitem
         The AXIS centre, width and variance values in the output are
         formed by duplicating the corresponding input AXIS values.

         \sstitem
         All \htmlref{non-complex numeric data types}{ap:HDStypes} can be handled.

         \sstitem
         Any number of NDF dimensions is supported.
      }
   }
}

\sstroutine{
   PLUCK
}{
   Plucks slices from an NDF at arbitrary positions
}{
   \sstdescription{
      This application's function is to extract data at scientifically
      relevant points such as the spatial location of a source or
      wavelength of a spectral feature, rather than at data sampling
      points (for which \htmlref{NDFCOPY}{NDFCOPY} is appropriate).
      This is achieved by the extraction of interpolated slices from
      an \NDFref{NDF}.  A slice is located at a supplied set of
      co-ordinates in the \htmlref{current WCS Frame}{se:curframe}~
      for some but not all axes, and it possesses one fewer
      significant dimension per supplied co-ordinate.  The slices run
      parallel to pixel axes of the NDF.

      The interpolation uses one of a selection of resampling methods
      to effect the non-integer shifts along the fixed axes, applied to
      each output element along the retained axes (see the METHOD,
      PARAMS, and TOL parameters).

      Three routes are available for obtaining the fixed positions,
      selected using Parameter MODE:

      \sstitemlist{

         \sstitem
         from the parameter system (see Parameter POS);

         \sstitem
         from a specified positions list (see Parameter INCAT); or

         \sstitem
         from a simple text file containing a list of co-ordinates (see
         Parameter COIN).

      }
      In the first mode the application loops, asking for new extraction
      co-ordinates until it is told to quit or encounters an error.
      However there is no looping if the position has been supplied on
      the command line.

      Each extracted dataset is written to a new NDF, which however, may
      reside in a single container file (see the CONTAINER parameter).
   }
   \sstusage{
      pluck in axes out method [mode]
         $\left\{ {\begin{tabular}{l}
                   pos \\
                   coin=? \\
                   incat=?
                   \end{tabular} }
          \right.$
          \newline\latexhtml{\hspace*{15.85em}}{~~~~~~~~~~~~~~~~~~~~~~~~~~~~~~~}
          \makebox[0mm][c]{\small mode}
   }
   \sstparameters{
      \sstsubsection{
         AXES( ) = \_INTEGER (Read)
      }{
         The WCS axis or axes to remain in the output NDF.   The slice
         will therefore contain an array comprising all the elements
         along these axes.  The maximum number of axes is one fewer
         than the number of WCS axes in the NDF.

         Each axis can be specified using one of the following options.

         \ssthitemlist{

            \sstitem
            Its integer index within the current Frame of the
            input  NDF (in the range 1 to the number of axes in the
            current Frame).

            \sstitem
            Its \htmlattref{Symbol}{Symbol(axis)}~ string such as
            \texttt{"RA"} or \texttt{"VRAD"}.

            \sstitem
            A generic option where \texttt{"SPEC"} requests the
            spectral axis, \texttt{"TIME"} selects the time axis,
            \texttt{"SKYLON"} and \texttt{"SKYLAT"} picks the sky longitude
            and latitude axes  respectively.  Only those axis domains
            present are available as options.
         }

         A list of acceptable values is displayed if an illegal value is
         supplied.  If the axes of the current Frame are not parallel to
         the NDF pixel axes, then the pixel axis which is most nearly
         parallel to the specified current Frame axis will be used.
      }
      \sstsubsection{
         COIN = FILENAME (Read)
      }{
         Name of a text file containing the co-ordinates of slices to
         be plucked.  It is only accessed if Parameter MODE is given the
         value \texttt{"File"}.  Each line should contain the formatted axis
         values for a single position, in the current Frame of the NDF.
         Axis values can be separated by spaces, tabs or commas.  The
         file may contain comment lines with the first character
         \texttt{\#} or \texttt{!}.
      }
      \sstsubsection{
         CONTAINER = \_LOGICAL (Read)
      }{
         If \texttt{TRUE}, each slice extracted is written as an NDF component of
         the HDS container file specified by the OUT parameter.  The $n$th
         component will be named PLUCK\_$n$.  If set \texttt{FALSE}, each
         extraction is written to a separate file.  On-the-fly format
         conversion to foreign formats is not possible when
         CONTAINER=\texttt{TRUE}.  \texttt{[FALSE]}
      }
      \sstsubsection{
         DESCRIBE = \_LOGICAL (Read)
      }{
         If \texttt{TRUE}, a detailed description of the co-ordinate Frame in
         which the fixed co-ordinates are to be supplied is displayed
         before the positions themselves.  It is ignored if
         MODE=\texttt{"Catalogue"}.  \texttt{[}current value\texttt{{]}}
      }
      \sstsubsection{
         INCAT = FILENAME (Read)
      }{
         A catalogue containing a positions list giving the co-ordinates
         of the fixed positions, such as produced by applications
         \htmlref{CURSOR}{CURSOR}, \htmlref{LISTMAKE}{LISTMAKE},
         \emph{etc.}  It is only accessed if Parameter MODE is given
         the value \texttt{"Catalogue"}.  The catalogue should have a WCS Frame
         common with the NDF, so that the NDF and catalogue FrameSets
         can be aligned.
      }
      \sstsubsection{
         MODE = \htmlref{LITERAL}{se:parmenu} (Read)
      }{
         The mode in which the initial co-ordinates are to be obtained.
         The supplied string can be one of the following values.

         \ssthitemlist{

            \sstitem
            \texttt{"Interface"} --- positions are obtained using Parameter POS.

            \sstitem
            \texttt{"Catalogue"} --- positions are obtained from a positions list
              using Parameter INCAT.

            \sstitem
            \texttt{"File"} --- positions are obtained from a text file using
              Parameter COIN.

         }
         \texttt{[}current value\texttt{{]}}
      }
      \sstsubsection{
         IN = NDF (Read)
      }{
         The NDF structure containing the data to be extracted.  It
         must have at least two dimensions.
      }
      \sstsubsection{
         METHOD = LITERAL (Read)
      }{
         The method to use when sampling the input pixel values.  For
         details of these schemes, see the descriptions of routine
         \xref{AST\_RESAMPLEx}{sun210}{AST_RESAMPLE\$<X>\$} in
         \xref{SUN/210}{sun210}{}.  METHOD can take the following values.

         \ssthitemlist{

            \sstitem
            \texttt{"Linear"} --- When resampling, the output pixel values are
            calculated by linear interpolation in the input NDF among the
            two nearest pixel values along each axis chosen by AXES.  This
            method produces smoother output NDFs than the
            nearest-neighbour scheme, but is marginally slower.

            \sstitem
            \texttt{"Sinc"} --- Uses the ${\textrm{sinc}}({\pi}x)$ kernel, where
            $x$ is the pixel offset from the interpolation point and
            ${\textrm{sinc}}(z)=\sin(z)/z$.  Use of this scheme is not recommended.

            \sstitem
            \texttt{"SincSinc"} --- Uses the ${\textrm{sinc}}({\pi}x){\textrm{sinc}}(k{\pi}x)$
            kernel.  A valuable general-purpose scheme, intermediate in its visual
            effect on NDFs between the linear option and using the
            nearest neighbour.

            \sstitem
            \texttt{"SincCos"} --- Uses the  ${\textrm{sinc}}({\pi}x)\cos(k{\pi}x)$
            kernel.  Gives similar results to the \texttt{"SincSinc"} scheme.

            \sstitem
            \texttt{"SincGauss"} --- Uses the ${\textrm{sinc}}({\pi}x)e^{-kx^2}$
            kernel.  Good results can be obtained by matching the FWHM of the
            envelope function to the point-spread function of the
            input data (see Parameter PARAMS).

            \sstitem
            \texttt{"Somb"} --- Uses the  ${\textrm{somb}}({\pi}x)$ kernel, where
            $x$ is the pixel offset from the interpolation point, and
            ${\textrm{somb}}(z)=2*J_{1}(z)/z$. $J_1$ is the first-order Bessel
            function of the first kind.  This scheme is similar to the
            \texttt{"Sinc"} scheme.

            \sstitem
            \texttt{"SombCos"} --- Uses the ${\textrm{somb}}({\pi}x)\cos(k{\pi}x)$
            kernel.  This scheme is similar to the \texttt{"SincCos"} scheme.

            \sstitem
            \texttt{"BlockAve"}  --- Block averaging over all pixels in the
            surrounding $N$-dimensional cube.

         }
         All methods propagate variances from input to output, but the
         variance estimates produced by interpolation schemes need to be
         treated with care since the spatial smoothing produced by these
         methods introduces correlations variance estimates.  The
         initial default is \texttt{"SincSinc"}.  \texttt{[}current value\texttt{{]}}
      }
      \sstsubsection{
         OUT = NDF (Write)
      }{
         The name for the output NDF, or the name of the single
         container file if CONTAINER=\texttt{TRUE}.
      }
      \sstsubsection{
         PARAMS( 2 ) = \_DOUBLE (Read)
      }{
         An optional array which consists of additional parameters
         required by the Sinc, SincSinc, SincCos, SincGauss, Somb,
         SombCos, and Gauss methods.

         PARAMS(1) is required by all the above schemes.  It is used
         to specify how many pixels are to contribute to the
         interpolated result on either side of the interpolation in
         each dimension.  Typically, a value of \texttt{2} is appropriate
         and the minimum allowed value is \texttt{1} (\emph{i.e.} one
         pixel on each side).  A value of zero or fewer indicates that
         a suitable number of pixels should be calculated
         automatically.  \texttt{[0]}

         PARAMS(2) is required only by the SombCos, Gauss, SincSinc,
         SincCos, and SincGauss schemes.  For the SombCos, SincSinc, and
         SincCos schemes, it specifies the number of pixels at which the
         envelope of the function goes to zero.  The minimum value is
         \texttt{1.0}, and the run-time default value is \texttt{2.0}.  For the
         Gauss and SincGauss schemes, it specifies the full-width at half-maximum
         (FWHM) of the Gaussian envelope.  The minimum value is \texttt{0.1}, and
         the run-time default is \texttt{1.0}.  On astronomical images and
         spectra, good results are often obtained by approximately
         matching the FWHM of the envelope function, given by PARAMS(2),
         to the point-spread function of the input data.  []
      }
      \sstsubsection{
         POS( ) = LITERAL (Read)
      }{
         An the co-ordinates of the next slice to be extracted, in the
         current co-ordinate Frame of the NDF (supplying a colon \texttt{":"}
         will display details of the current co-ordinate Frame).  The
         position should be supplied as a list of
         \xref{formatted axis values}{sun210}{AST_UNFORMAT}
         separated by spaces or commas.  POS is only accessed if
         Parameter MODE is given the value \texttt{"Interface"}.  If the
         co-ordinates are supplied on the command line only one
         slice will be extracted; otherwise the application will ask
         for further positions which may be terminated by supplying the
         null value (\texttt{!}).
      }
      \sstsubsection{
         TITLE = LITERAL (Read)
      }{
         A Title for every output NDF structure.  A null value (\texttt{{!}})
         propagates the title from the input NDF to all output NDFs.
         \texttt{[!]}
      }
      \sstsubsection{
         TOL = \_DOUBLE (Read)
      }{
         The maximum tolerable geometrical distortion that may be
         introduced as a result of approximating non-linear Mappings
         by a set of piece-wise linear transforms.  Both
         algorithms approximate non-linear co-ordinate transformations
         in order to improve performance, and this parameter controls
         how inaccurate the resulting approximation is allowed to be,
         as a displacement in pixels of the input NDF.  A value of
         zero will ensure that no such approximation is done, at the
         expense of increasing execution time.  \texttt{[0.05]}
      }
   }
   \sstexamples{
      \sstexamplesubsection{
         pluck omc1 pos="5:35:13.7,-5:22:13.6" axes=FREQ
            method=sincgauss params=[3,5] out=omc1\_trap
      }{
         The NDF omc1 is a spectral-imaging cube with
         (Right ascension, declination, frequency) World Co-ordinate
         axes.  This example extracts a spectrum at
         RA=\ra{5;35;13.7}, Dec=\dec{-5;22;13.6}
         using the SincGauss interpolation method.
         Three pixels either side of the point are used to interpolate,
         the full-width half-maximum of the Gaussian is five pixels.
         The resultant spectrum called omc1\_trap, is still a cube, but
         its spatial dimensions each only have one element.
      }
      \sstexamplesubsection{
         pluck omc1 mode=cat incat=a axes=FREQ container out=omc1\_spectra
      }{
         This example reads the fixed positions from the
         positions list in file \texttt{a.FIT}.  The selected spectra are stored
         in an HDS container file called \texttt{omc1\_spectra.sdf}.
      }
      \sstexamplesubsection{
         pluck omc1 mode=cat incat=a axes=SPEC container out=omc1\_spectra
      }{
         As the previous example, plucking spectra, this time by
         selecting the generic spectral axis.
      }
      \sstexamplesubsection{
         pluck omc1 pos=3.45732E11 axes="RA,Dec" method=lin out=peakplane
      }{
         This example extracts a plane from omc1 at frequency
         3.45732E11 Hz using linear interpolation and stores it in NDF
         peakplane.
      }
   }
   \sstnotes{
      \sstitemlist{

         \sstitem
         In Interface or File modes all positions should be supplied in
         the current co-ordinate Frame of the NDF.  A description of the
         co-ordinate Frame being used is given if Parameter DESCRIBE is set
         to a \texttt{TRUE} value.  Application \htmlref{WCSFRAME}{WCSFRAME}
         can be used to change the current co-ordinate Frame of the
         NDF before running this application if required.

         \sstitem
         The output NDF has the same dimensionality as the input NDF,
         although the axes with fixed co-ordinates (those not specified
         by the AXES parameter) are degenerate, having bounds of 1:1.
         The retention of these insignificant axes enables the
         co-ordinates of where the slice originated to be recorded.
         Such fixed co-ordinates may be examined with say
         \htmlref{NDFTRACE}{NDFTRACE}.  NDFCOPY may be used to trim the
         degenerate axes if their presence
         prevents some old non-KAPPA tasks from operating.

         \sstitem
         In Catalogue or File modes the table file need only contain
         columns supplying the fixed positions.  In this case the
         co-ordinates along the retained axes are deemed to be independent,
         that is they do not affect the shifts required of the other axes.
         In practice this assumption only affects File mode, as catalogues
         made with CURSOR or LISTMAKE will contain WCS information.

         In Interface mode representaive co-ordinates along retained axes
         are the midpoints of the bounds of an array that would contain the
         resampled copy of the whole input array.
      }
   }
   \sstdiytopic{
      Related Applications
   }{
KAPPA: \htmlref{NDFCOPY}{NDFCOPY},
\htmlref{REGRID}{REGRID}.
   }
   \sstimplementationstatus{
      \ssthitemlist{

         \sstitem
         The \htmlref{LABEL}{apndf:label}, \htmlref{UNITS}{apndf:units}, and
         \htmlref{HISTORY}{apndf:history}~ components, and all extensions
         are propagated.  \htmlref{TITLE}{apndf:title} is controlled by the
         TITLE parameter.  DATA, \htmlref{VARIANCE}{apndf:variance},
         \htmlref{AXIS}{apndf:axis}, and \htmlref{WCS}{apndf:wcs} are
         propagated after appropriate modification.  The
         \htmlref{QUALITY}{apndf:quality} component is not propagated.

         \sstitem
         The processing of \htmlref{bad pixels}{se:masking} and automatic
         \htmlref{quality masking}{se:qualitymask} are supported.

         \sstitem
         All \htmlref{non-complex numeric data types}{ap:HDStypes} can be handled.

         \sstitem
         The minimum number of dimensions in the input NDF is two.

         \sstitem
         Processing a group of input NDFs is not supported unless
         CONTAINER=\texttt{TRUE} or when only one output NDF is created per
         input file.
      }
   }
}

\sstroutine{
   POW
}{
   Takes the specified power of each pixel of an NDF
}{
   \sstdescription{
      This routine copies the supplied input \NDFref{NDF}, raising each value in
      the DATA array to the specified power.  The \htmlref{VARIANCE}{apndf:variance}~ component, if
      present, is modified appropriately.  Negative data values will only
      generate good output pixels when the power is an integer.
   }
   \sstusage{
      pow in power out
   }
   \sstparameters{
      \sstsubsection{
         IN = NDF (Read)
      }{
         The input NDF structure.
      }
      \sstsubsection{
         OUT = NDF (Write)
      }{
         The output NDF structure.
      }
      \sstsubsection{
         POWER = \_DOUBLE (Read)
      }{
         The power.
      }
      \sstsubsection{
         TITLE = LITERAL (Read)
      }{
         A \htmlref{title}{apndf:title} for the output NDF.  A null value will cause the title
         of the NDF supplied for Parameter IN to be used instead.
         \texttt{[!]}
      }
   }
   \sstexamples{
      \sstexamplesubsection{
         pow m51 2 m51sq
      }{
         Square all values in the NDF called m51, and store the results in the
         NDF called m51sq.
      }
   }
   \sstdiytopic{
      Related Applications
   }{
KAPPA: \htmlref{ADD}{ADD},
\htmlref{CADD}{CADD},
\htmlref{CMULT}{CMULT},
\htmlref{CDIV}{CDIV},
\htmlref{CSUB}{CSUB},
\htmlref{DIV}{DIV},
\htmlref{MATHS}{MATHS},
\htmlref{MULT}{MULT},
\htmlref{SUB}{SUB}.
   }
   \sstimplementationstatus{
      \sstitemlist{

         \sstitem
         This routine correctly processes the \htmlref{AXIS}{apndf:axis}, DATA, \htmlref{QUALITY}{apndf:quality},
         \htmlref{LABEL}{apndf:label}, \htmlref{TITLE}{apndf:title}, \htmlref{UNITS}{apndf:units}, \htmlref{HISTORY}{apndf:history}, \htmlref{WCS}{apndf:wcs}, and \htmlref{VARIANCE}{apndf:variance}~ components of an NDF
         data structure and propagates all \htmlref{extensions}{apndf:extensions}.

         \sstitem
         Processing of \htmlref{bad pixels}{se:masking} and automatic \htmlref{quality masking}{se:qualitymask} are
         supported.

         \sstitem
         All \htmlref{non-complex numeric data types}{ap:HDStypes} can be handled.  Arithmetic
         is performed using single-precision floating point, or double
         precision, if appropriate, but the numeric type of the input pixels
         is preserved in the output NDF.
      }
   }
}

\sstroutine{
   PROFILE
}{
   Creates a one-dimensional profile through an \textit{n}-dimensional NDF
}{
   \sstdescription{
      This application samples an \textit{n}-dimensional \NDFref{NDF} at a set of positions,
      producing a one-dimensional output NDF containing the sample values.
      Nearest-neighbour interpolation is used.

      The samples can be placed at specified positions within the input NDF,
      or can be spaced evenly along a poly-line joining a set of vertices
      (see Parameter MODE).  The positions of the samples may be saved in an
      output positions list (see Parameter OUTCAT).
   }
   \sstusage{
      profile in out
        $\left\{ {\begin{tabular}{l}
                  start finish [nsamp] \\
                  incat=?
                  \end{tabular} }
        \right.$
        \newline\latexhtml{\hspace*{7.35em}}{~~~~~~~~~~~~~}
        \makebox[0mm][c]{\small mode}
   }
   \sstparameters{
      \sstsubsection{
         CATFRAME = LITERAL (Read)
      }{
         A string determining the \htmlref{co-ordinate Frame}{se:domains}~  in which positions are
         to be stored in the output catalogue associated with Parameter
         OUTCAT.  The string supplied for CATFRAME can be one of the
         following options.

         \ssthitemlist{

            \sstitem
            A \htmlref{Domain name}{se:domains}~ such as \htmlref{SKY, AXIS, PIXEL}{se:resdoms}.

            \sstitem
            An integer value giving the index of the required Frame.

            \sstitem
            An IRAS90 \emph{Sky Co-ordinate System} (SCS) values such as
            \texttt{"EQUAT(J2000)"} (see \xref{SUN/163}{sun163}{}).

         }
         If a null (\texttt{{!}}) value is supplied, the positions will be stored
         in the current Frame. \texttt{[!]}
      }
      \sstsubsection{
         CATEPOCH = \_DOUBLE (Read)
      }{
         The epoch at which the sky positions stored in the output
         catalogue were determined.  It will only be accessed if an epoch
         value is needed to qualify the co-ordinate Frame specified by
         COLFRAME.  If required, it should be given as a decimal years
         value, with or without decimal places (\texttt{"1996.8"}, for example).
         Such values are interpreted as a Besselian epoch if less than
         1984.0 and as a Julian epoch otherwise.
      }
      \sstsubsection{
         FINISH = LITERAL (Read)
      }{
         The co-ordinates of the last sample in the profile, in the
         current co-ordinate Frame of the NDF (supplying \texttt{":"}
         will display details of the required co-ordinate Frame).  The
         position should be supplied as a list of \xref{formatted axis
         values}{sun210}{AST_UNFORMAT} separated by spaces.  This
         parameter is only accessed if Parameter MODE is set to \texttt{
         "Curve"} and a null (\texttt{{!}}) value is given for INCAT.  If
         the last (top-right) pixel in the NDF has valid co-ordinates
         in the current co-ordinate Frame of the NDF, then these
         co-ordinates will be used as the suggested default. Otherwise
         there will be no suggested default.
      }
      \sstsubsection{
         GEODESIC = \_LOGICAL (Read)
      }{
         If \texttt{TRUE} then the line segments which form the profile will be
         geodesic curves within the current co-ordinate Frame of the NDF.
         Otherwise, the line segments are simple straight lines.  This
         parameter is only accessed if Parameter MODE is set to \texttt{"Curve"}.

         As an example, consider a profile consisting of a single line segment
         which starts at RA=0h DEC=$+$80d and finishes at RA=12h DEC=$+$80d.  If
         GEODESIC is \texttt{FALSE}, the line segment will be a line of constant
         declination, \emph{i.e.} the \texttt{"straight"} line from the position (0,80) to the
         position (12,~80), passing through (1,~80), (2,~80), \emph{etc}.  If GEODESIC
         is \texttt{TRUE}, then the line segment will be the curve of shortest
         distance on the celestial sphere between the start and end.  In this
         particaular case, this will be a great circle passing through the
         north celestial pole.  \texttt{[FALSE]}
      }
      \sstsubsection{
         IN = NDF (Read)
      }{
         Input NDF structure containing the data to be profiled.
      }
      \sstsubsection{
         INCAT = FILENAME (Read)
      }{
         A catalogue containing a set of vertices or sample positions defining
         the required profile.  The file should be in the format of a
         \emph{positions list} such as produced by applications CURSOR and LISTMAKE.  If a
         null value (\texttt{{!}}) is given then Parameters START and FINISH will be
         used to obtain the vertex positions.  If Parameter MODE is given the
         value \texttt{"Curve"}, then the Parameter INCAT is only accessed if a value
         is given for it on the command line (otherwise a null value is
         assumed).
      }
      \sstsubsection{
         MODE = \htmlref{LITERAL}{se:parmenu} (Read)
      }{
         The mode by which the sample positions are selected.  The alternatives
         are listed below.

         \ssthitemlist{

            \sstitem
            \texttt{"Curve"} --- The samples are placed evenly along a curve specified
            by a set of vertices obtained from the user.  The line segments
            joining these vertices may be linear or geodesic (see Parameter
            GEODESIC).  Multiple vertices may be supplied using a text file
            (see Parameter INCAT).  Alternatively, a single line segment can
            be specified using Parameters START and FINISH.  The number of
            samples to take along the curve is specified by Parameter NSAMP.

            \sstitem
            \texttt{"Points"} --- The positions at which samples should be taken are
            given explicitly by the user in a text file (see Parameter
            INCAT).  No other sample positions are used.

         }
         \texttt{["Curve"]}
      }
      \sstsubsection{
         NSAMP = \_INTEGER (Read)
      }{
         The number of samples required along the length of the profile.
         The first sample is at the first supplied vertex, and the last
         sample is at the last supplied vertex.  The sample positions are
         evenly spaced within the current co-ordinate Frame of the NDF.  If
         a null value is supplied, a default value is used equal to one
         more than the length of the profile in pixels.  This is only accessed
         if Parameter MODE is given the value \texttt{"Curve"}.  \texttt{[!]}
      }
      \sstsubsection{
         OUT = NDF (Write)
      }{
         The output NDF.  This will be one-dimensional with length specified
         by Parameter NSAMP.
      }
      \sstsubsection{
         OUTCAT = FILENAME (Write)
      }{
         An output positions list in which to store the sample positions.
         This is the name of a catalogue which can be used to communicate
         positions to subsequent applications.  It includes information
         describing the available WCS co-ordinate Frames as well as the
         positions themselves.  If a null value is supplied, no output
         positions list is produced.  See also Parameter CATFRAME.  \texttt{[!]}
      }
      \sstsubsection{
         START = LITERAL (Read)
      }{
         The co-ordinates of the first sample in the profile, in the current
         co-ordinate Frame of the NDF (supplying \texttt{":"} will display details of
         the required co-ordinate Frame).  The position should be supplied as
         a list of \xref{formatted axis values}{sun210}{AST_UNFORMAT}
         separated by spaces.  This parameter
         is only accessed if Parameter MODE is set to \texttt{"Curve"} and a null
         (\texttt{{!}}) value is given for INCAT.  If the first (bottom-left) pixel in
         the NDF has valid co-ordinates in the current co-ordinate Frame of
         the NDF, then these co-ordinates will be used as the suggested
         default.  Otherwise there will be no suggested default.
      }
   }
   \sstexamples{
      \sstexamplesubsection{
         profile my\_data prof "0 0" "100 100" 40 outcat=samps
      }{
         Creates a one-dimensional NDF called prof, holding a profile of the
         data values in the input NDF my\_data along a profile starting at
         pixel co-ordinates [0.0,~0.0] and ending at pixel co-ordinates
         [100.0,~100.0].  The profile consists of forty samples spread evenly
         (in the pixel co-ordinate Frame) between these two positions.
         This example assumes that the current co-ordinate Frame in the NDF
         my\_data represents pixel co-ordinates.  This can be ensured by
         issuing the command \texttt{"wcsframe my\_data pixel"} before running
         profile.  A FITS binary catalogue is created called \texttt{samps.FIT}
         containing the positions of all samples in the profile, together with
         information describing all the co-ordinate Frames in which the
         positions of the samples are known.  This file may be examined
         using application LISTSHOW.
      }
      \sstexamplesubsection{
         profile my\_data prof "15:32:47 23:40:08" "15:32:47 23:42"
      }{
         This example is the same as the last one except that it is
         assumed that the current co-ordinate Frame in the input NDF my\_data
         is an equatorial (RA/DEC) system.  It creates a one-dimensional
         profile starting at RA=15:32:47 DEC=23:40:08, and ending at the same
         RA and DEC=23:42:00.  The number of points in the profile is
         determined by the resolution of the data.
      }
      \sstexamplesubsection{
         profile allsky prof incat=prof\_path npoint=200 geodesic outcat=aa.fit
      }{
         This examples creates a profile of the NDF allsky through a set of
         points given in a FITS binary catalogue called \texttt{prof\_path.FIT}.  Such
         catalogues can be created (for example) using application CURSOR.
         Each line segment is a geodesic curve.  The profile is sampled at 200
         points.  The samples positions are written to the output positions
         list \texttt{aa.fit}.
      }
      \sstexamplesubsection{
         profile allsky2 prof2 mode=point incat=aa.fit
      }{
         This examples creates a profile of the NDF allsky2 containing
         samples at the positions given in the positions list \texttt{aa.fit}.  Thus,
         the profiles created by this example and the previous example
         will sample the two images allsky and allsky2 at the same
         positions and so can be compared directly.
      }
   }
   \sstnotes{
      \sstitemlist{

         \sstitem
         This application uses the conventions of the \CURSAref\ package
         for determining the formats of input and output positions list
         catalogues.  If a file type of .fit is given, then the catalogue is
         assumed to be a FITS binary table.  If a file type of .txt is given,
         then the catalogue is assumed to be stored in a text file in \emph{Small
         Text List} (STL) format.  If no file type is given, then \texttt{.fit} is
         assumed.
      }
   }
   \sstdiytopic{
      Related Applications
   }{
KAPPA: \htmlref{LINPLOT}{LINPLOT},
\htmlref{CURSOR}{CURSOR},
\htmlref{LISTMAKE}{LISTMAKE},
\htmlref{LISTSHOW}{LISTSHOW};
\CURSA: \xref{XCATVIEW}{sun190}{XVIEW}.
   }
   \sstimplementationstatus{
      \sstitemlist{

         \sstitem
         This routine correctly processes the DATA, \htmlref{VARIANCE}{apndf:variance}, \htmlref{WCS}{apndf:wcs}, \htmlref{LABEL}{apndf:label},
         \htmlref{TITLE}{apndf:title}, and \htmlref{UNITS}{apndf:units}~ components of the NDF.

         \sstitem
         All \htmlref{non-complex numeric data types}{ap:HDStypes} can be handled.  Only
         double-precision floating-point data can be processed directly.
         Other non-complex data types will undergo a type conversion before the profile is produced.
      }
   }
}
\sstroutine{
   PROVADD
}{
   Stores provenance information in an NDF
}{
   \sstdescription{
      This application modifies the provenance information stored in an
      NDF.  It records a second specified NDF as a direct parent of the
      first NDF.  If an NDF has more than one direct parent then this
      application should be run multiple times, once for each parent.
   }
   \sstusage{
      provadd ndf parent creator isroot moretext
   }
   \sstparameters{
      \sstsubsection{
         CREATOR = LITERAL (Read)
      }{
         A text identifier for the software that created the main NDF
         (usually the name of the calling application).  The format of the
         identifier is arbitrary, but the form \texttt{"PACKAGE:COMMAND"} is
         recommended.  If a null (\texttt{{!}}) value is supplied, no creator
         information is stored.  \texttt{[!]}
      }
      \sstsubsection{
         ISROOT = \_LOGICAL (Read)
      }{
         If \texttt{TRUE}, then the NDF given by Parameter PARENT will be treated
         as a root NDF.  That is, any provenance information within PARENT
         describing its own parents is ignored.  If \texttt{FALSE}, then any provenance
         information within PARENT is copied into the main NDF.  PARENT
         is then a root NDF only if it contains no provenance
         information.  \texttt{[FALSE]}
      }
      \sstsubsection{
         MORETEXT = \htmlref{GROUP}{se:groups} (Read)
      }{
         A group of ``keyword=value'' strings that give additional
         information about the parent NDF, and how it was used in the
         creation of the main NDF.  If supplied, this information is
         stored with the provenance in the main NDF.

         The supplied value should be either a comma-separated list of
         strings, or the name of a text file preceded by an up-arrow
         character \texttt{"$\wedge$"}, containing one or more comma-separated
         list of strings.  Each string is either a ``keyword=value'' setting, or
         the name of a text file preceded by an up-arrow character \texttt{"$\wedge$"}.
         Such text files should contain further comma-separated lists
         which will be read and interpreted in the same manner (any
         blank lines or lines beginning with \texttt{\#} are ignored).  Within a
         text file, newlines can be used as delimiters as well as
         commas.

         Each individual setting should be of the form:

            $<$keyword$>$=$<$value$>$

         where $<$keyword$>$ is either a simple name, or a dot-delimited
         hierarchy of names (\emph{e.g.} \texttt{"camera.settings.exp=1.0"}).  The
         $<$value$>$ string should not contain any commas.  \texttt{[!]}
      }
      \sstsubsection{
         NDF = NDF (Read and Write)
      }{
         The NDF which is to be modified.
      }
      \sstsubsection{
         PARENT = NDF (Read)
      }{
         An NDF that is to be recorded as a direct parent of the NDF
         given by Parameter NDF.
      }
   }
   \sstexamples{
      \sstexamplesubsection{
         provadd m51\_ff ff
      }{
         Records the fact that NDF ff was used in the creation of NDF m51\_ff.
      }
   }
   \sstnotes{
      Provenance information is stored in an NDF extension called
      PROVENANCE, and is propagated automatically by all \KAPPA\ applications.
   }
   \sstdiytopic{
      Related Applications
   }{
      KAPPA: \htmlref{PROVMOD}{PROVMOD}, \htmlref{PROVSHOW}{PROVSHOW},
      \htmlref{HISCOM}{HISCOM}.
   }
}

\sstroutine{
   PROVMOD
}{
   Modifies provenance information for an NDF
}{
   \sstdescription{
      This application modifies the provenance information stored in
      the PROVENANCE extension of an NDF.
   }
   \sstusage{
      provmod ndf ancestor path
   }
   \sstparameters{
      \sstsubsection{
         ANCESTOR = LITERAL (Read)
      }{
         Specifies the indices of one or more ancestors that are to be
         modified.  An index of zero refers to the supplied NDF itself.
         A positive index refers to one of the NDFs listed in the
         ANCESTORS table in the \htmlref{PROVENANCE extension}{apndf:provenance} of
         the NDF.  The maximum number of ancestors is limited to 100 unless \texttt{"ALL"} or
         \texttt{"$*$"} is specified.  The supplied parameter value can take any of
         the following forms.

         \ssthitemlist{

            \sstitem
            \texttt{"ALL"} or \texttt{"$*$"} ---  All ancestors.

            \sstitem
            \texttt{"xx,yy,zz"} --- A list of ancestor indices.

            \sstitem
            \texttt{"xx:yy"} ---  Ancestor indices between \emph{xx} and
            \emph{yy} inclusively.  When \emph{xx} is omitted, the range
            begins from 0; when \emph{yy} is omitted, the range ends with the
            maximum value it can take, that is the number of ancestors
            described in the PROVENANCE extension.

            \sstitem
            Any reasonable combination of above values separated by
            commas.  \texttt{["ALL"]}
         }
      }
      \sstsubsection{
         CREATOR = LITERAL (Read)
      }{
         If the supplied string includes no equals signs, then it is a
         new value for the \texttt{"CREATOR"} string read from each of the
         ancestors being modified. If the supplied string includes one or
         more equals signs, then it specifies one or more substitutions to
         be performed on the \texttt{"CREATOR"} string read from each of the
         ancestors being modified.  See
         \htmlref{``Substitution Syntax''}{substitution_syntax:provmod}
         below.  If null (\texttt{{!}}) is supplied, the CREATOR item is left
         unchanged.  \texttt{[!]}
      }
      \sstsubsection{
         DATE = LITERAL (Read)
      }{
         If the supplied string includes no equals signs, then it is a
         new value for the \texttt{"DATE"} string read from each of the
         ancestors being modified. If the supplied string includes one or
         more equals signs, then it specifies one or more substitutions to
         be performed on the \texttt{"DATE"} string read from each of the
         ancestors being modified.  See ``Substitution Syntax'' below.  If
         null (\texttt{{!}}) is supplied, the DATE item is left unchanged.
         \texttt{[!]}
      }
      \sstsubsection{
         MORETEXT = \htmlref{GROUP}{se:groups} (Read)
      }{
         This parameter is accessed only if a single ancestor is being
         modified (see Parameter ANCESTORS). It gives information to
         store in the MORE component of the ancestor (any existing
         information is first removed).  If a null (\texttt{{!}}) value is
         supplied, then existing MORE component is left unchanged.

         The supplied value should be either a comma-separated list of
         strings, or the name of a text file preceded by an up-arrow
         character \texttt{"$\wedge$"}, containing one or more comma-separated
         list of strings.  Each string is either a ``keyword=value'' setting, or
         the name of a text file preceded by an up-arrow character \texttt{"$\wedge$"}.
         Such text files should contain further comma-separated lists
         which will be read and interpreted in the same manner (any
         blank lines or lines beginning with \texttt{\#} are ignored).  Within a
         text file, newlines can be used as delimiters as well as
         commas.

         Each individual setting should be of the form:

            $<$keyword$>$=$<$value$>$

         where $<$keyword$>$ is either a simple name, or a dot-delimited
         hierarchy of names (\emph{e.g.} \texttt{"camera.settings.exp=1.0"}).  The
         $<$value$>$ string should not contain any commas.  \texttt{[!]}
      }
      \sstsubsection{
         NDF = NDF (Update)
      }{
         The NDF data structure.
      }
      \sstsubsection{
         PATH = LITERAL (Read)
      }{
         If the supplied string includes no equals signs, then it is a
         new value for the \texttt{"PATH"} string read from each of the
         ancestors being modified.  If the supplied string includes one or
         more equals signs, then it specifies one or more substitutions to
         be performed on the \texttt{"PATH"} string read from each of the
         ancestors being modified.  See ``Substitution Syntax'' below.  If
         null (\texttt{{!}}) is supplied, the PATH item is left unchanged.
         \texttt{[!]}
      }
   }
   \label{examples:provmod}
   \sstexamples{
      \sstexamplesubsection{
         provmod ff path=/home/dsb/real-file.sdf
      }{
         This modifies any ancestor within the NDF called ff by setting
         its PATH to \texttt{"/home/dsb/real-file.sdf"}.
      }
      \sstexamplesubsection{
         provmod ff ancestor=3 moretext="obsidss=acsis\_00026\_20080322T055855\_1"
      }{
         This modifies ancestor Number 3 by storing a value of
         \texttt{acsis\_00026\_20080322T055855\_1} for key \texttt{obsidss} within the
         additonal information for the ancestor.  Any existing additional
         information is removed.
      }
      \sstexamplesubsection{
         provmod ff path='(\_x)\$=\_y'
      }{
         This modifies any ancestor within the NDF called ff that has a
         path ending in \texttt{"\_x"} by replacing the final \texttt{"\_x"} with
         \texttt{"\_y"}.
      }
      \sstexamplesubsection{
         provmod ff path='(.$*$)\_(.$*$)=\$2=\$1'
      }{
         This modifies any ancestor within the NDF called ff that has a
         path consisting of two parts separated by an underscore by
         swapping the parts.  If there is more than one underscore in
         the ancestor path, then the final underscore is used (because
         the initial quantifier \texttt{".$*$"} is greedy).
      }
      \sstexamplesubsection{
         provmod ff path='(.$*$?)\_(.$*$)=\$2=\$1'
      }{
         This modifies any ancestor within the NDF called ff that has a
         path consisting of two parts separated by an underscore by
         swapping the parts.  If there is more than one underscore in the
         ancestor path, then the first underscore is used (because the
         initial quantifier \texttt{".$*$?"} is not greedy).
      }
   }
   \label{substitution_syntax:provmod}
   \sstdiytopic{
      Substitution Syntax
   }{
      The syntax for the CREATOR, DATE, and PATH parameter values is a
      minimal form of regular expression.  The following atoms are
      allowed.

% This conditional text is to avoid a space appearing before the backslash
% in the hypertext, such as " \W" instead of "\W".
      \latex{
         \sstitemlist{

         \sstitem
         \texttt{"[chars]"} --- Matches any of the characters within the brackets.

         \sstitem
         \texttt{"[$\wedge$chars]"} --- Matches any character that is not within the
                       brackets (ignoring the initial \texttt{"$\wedge$"} character).
         \sstitem
         \texttt{"."} --- Matches any single character.

         \sstitem
         \texttt{"$\backslash$d"} --- Matches a single digit.

         \sstitem
         \texttt{"$\backslash$D"} --- Matches anything but a single digit.

         \sstitem
         \texttt{"$\backslash$w"} --- Matches any alphanumeric character, and \texttt{"\_"}.

         \sstitem
         \texttt{"$\backslash$W"} --- Matches anything but alphanumeric characters, and \texttt{"\_"}.

         \sstitem
         \texttt{"$\backslash$s"} --- Matches white space.

         \sstitem
         \texttt{"$\backslash$S"} --- Matches anything but white space.
         }
      }
      \html{
         \sstitemlist{

         \sstitem
         \texttt{"[chars]"} --- Matches any of the characters within the brackets.

         \sstitem
         \texttt{"[$\wedge$chars]"} --- Matches any character that is not within the
                       brackets (ignoring the initial \texttt{"$\wedge$"} character).
         \sstitem
         \texttt{"."} --- Matches any single character.

         \sstitem
         \verb+"\d"+ --- Matches a single digit.

         \sstitem
         \verb+"\D"+ --- Matches anything but a single digit.

         \sstitem
         \verb+"\w"+ --- Matches any alphanumeric character, and \texttt{"\_"}.

         \sstitem
         \verb+"\W"+ --- Matches anything but alphanumeric characters, and \texttt{"\_"}.

         \sstitem
         \verb+"\s"+ --- Matches white space.

         \sstitem
         \verb+"\S"+ --- Matches anything but white space.
         }
      }

      Any other character that has no special significance within a
      regular expression matches itself.  Characters that have special
      significance can be matched by preceding them with a backslash
      ($\backslash$) in which case their special significance is ignored (note,
      this does not apply to the characters in the set \emph{dDsSwW}).

      Note, minus signs (\texttt{"-"}) within brackets have no special
      significance, so ranges of characters must be specified
      explicitly.

      The following quantifiers are allowed.
      \ssthitemlist{

         \sstitem
         \texttt{"*"} --- Matches zero or more of the preceding atom, choosing the
             largest possible number that gives a match.

         \sstitem
         \texttt{"*?"}--- Matches zero or more of the preceding atom, choosing the
            smallest possible number that gives a match.

         \sstitem
         \texttt{"+"} --- Matches one or more of the preceding atom, choosing the
             largest possible number that gives a match.

         \sstitem
         \texttt{"+?"}--- Matches one or more of the preceding atom, choosing the
             smallest possible number that gives a match.

         \sstitem
         \texttt{"?"} --- Matches zero or one of the preceding atom.

         \sstitem
         \texttt{"\{n\}"} --- Matches exactly $n$ occurrences of the preceding atom.
      }

      The following constraints are allowed.
      \ssthitemlist{

         \sstitem
         \texttt{"$\wedge$"} --- Matches the start of the test string.

         \sstitem
         \texttt{"\$"} --- Matches the end of the test string.
      }

      Multiple templates can be concatenated, using the \texttt{"$|$"} character to
      separate them.  The test string is compared against each one in
      turn until a match is found.

      A template should use parentheses to enclose the sub-strings that
      are to be replaced, and the set of corresponding replacement
      values should be appended to the end of the string, separated by
      \texttt{"="} characters.  The section of the test string that matches the
      first parenthesised section in the template string will be
      replaced by the first replacement string.  The section of the test
      string that matches the second parenthesised section in the
      template string will be replaced by the second replacement string,
      and so on.

      The replacement strings can include the tokens \texttt{"\$1"},\texttt{"\$2"},
      \emph{etc.} The section of the test string that matched the corresponding
      parenthesised section in the template is used in place of the
      token.

      See the \htmlref{``Examples''}{examples:provmod} section above for how to
      use these facilities.
   }
   \sstdiytopic{
      Related Applications
   }{
      KAPPA: \htmlref{PROVADD}{PROVADD}, \htmlref{PROVREM}{PROVREM}, \htmlref{PROVSHOW}{PROVSHOW}.
   }
}

\sstroutine{
   PROVREM
}{
   Removes selected provenance information from an NDF
}{
   \sstdescription{
      This application removes selected ancestors, either by hiding them,
      or deleting them from the provenance information stored in a given NDF.
      The `generation gap' caused by removing an ancestor is bridged by
      assigning all the direct parents of the removed ancestor to each of
      the direct children of the ancestor.

      The ancestors to be removed can be specified either by giving
      their indices (Parameter ANCESTOR), or by comparing each ancestor
      with a supplied pattern matching template (Parameter PATTERN).

      If an ancestor is hidden rather than deleted (see Parameter HIDE),
      the ancestor is retained within the NDF, but a flag is set telling
      later applications to ignore the ancestor (exactly how the flag is
      used will depend on the particular application).
   }
   \sstusage{
      provrem ndf pattern item
   }
   \sstparameters{
      \sstsubsection{
         ANCESTOR = LITERAL (Read)
      }{
         Specifies the indices of one or more ancestors that are to be
         removed.  If a null \texttt{{(!)}} value is supplied, the ancestors to be
         removed are instead determined using the PATTERN parameter.
         Each supplied index must be positive and refers to one of the
         NDFs listed in the ANCESTORS table in the PROVENANCE extension
         of the NDF (including any hidden ancestors).  Note, if ancestor
         indices are determined using the PROVSHOW command, then PROVSHOW
         should be run with the HIDE parameter set to \texttt{FALSE}; otherwise
         incorrect ancestor indices may be determined, resulting in the
         wrong ancestors being removed by PROVREM.

         The maximum number of ancestors that can be removed is limited to
         100 unless \texttt{"LL"}, \texttt{"*"} or \texttt{!} is specified.  The
         supplied parameter value can take any of the following forms.

         \ssthitemlist{

            \sstitem
            \texttt{"ALL"} or \texttt{"*"} ---  All ancestors.

            \sstitem
            \texttt{"xx,yy,zz"} --- A list of ancestor indices.

            \sstitem
            \texttt{"xx:yy"} ---  Ancestor indices between \emph{xx} and \emph{yy}
            inclusively.  When \emph{xx} is omitted, the range begins from 0; when
            \emph{yy} is omitted the range ends with the maximum value it can take,
            that is the number of ancestors described in the PROVENANCE extension.

            \sstitem
            Any reasonable combination of above values separated by
            commas.  \texttt{[!]}
         }
      }
      \sstsubsection{
         HIDE = \_LOGICAL (Read)
      }{
         If \texttt{TRUE}, then the ancestors are not deleted, but instead have a
         flag set indicating that they have been hidden.  All information
         about hidden ancestors is retained unchanged, and can be viewed
         using \htmlref{PROVSHOW}{PROVSHOW} if the HIDE parameter is set
         \texttt{FALSE} when running PROVSHOW.  \texttt{[FALSE]}
      }
      \sstsubsection{
         ITEM = \htmlref{LITERAL}{se:parmenu} (Read)
      }{
         Specifies the item of provenance information that is checked
         against the pattern matching template specified for Parameter
         PATTERN.  It can be \texttt{"PATH"}, \texttt{"CREATOR"} or \texttt{"DATE"}.
         \texttt{["PATH"]}
      }
      \sstsubsection{
         NDF = NDF (Update)
      }{
         The NDF data structure.
      }
      \sstsubsection{
         PATTERN = LITERAL (Read)
      }{
         Specifies a pattern matching template using the syntax
         described below in \htmlref{``Pattern Matching
         Syntax''}{pattern_matching_syntax:provrem}.  Each ancestor
         listed in the PROVENANCE extension of the NDF is compared with
         this template, and each ancestor that matches is removed.  The
         item of provenance information to be compared to the pattern is
         specified by Parameter ITEM.
      }
      \sstsubsection{
         REMOVE = \_LOGICAL (Read)
      }{
         If \texttt{TRUE}, then the ancestors specified by Parameter PATTERN or
         ANCESTORS are removed.  Otherwise, these ancestors are retained
         and all other ancestors are removed.  \texttt{[TRUE]}
      }
   }
   \sstexamples{
      \sstexamplesubsection{
         provrem ff ancestor=1
      }{
         This removes the first ancestor from the NDF called ff.
      }
      \sstexamplesubsection{
         provrem ff ancestor=all
      }{
         This erases all provenance information.
      }
      \sstexamplesubsection{
          provrem ff pattern='\_xb\$|\_yb\$' hide
       }{
          This hides, but does not permanently delete, all ancestors that
          have paths that end with \texttt{"\_xb"} or \texttt{"\_yb"}.  Note,
          provenance paths do not include a trailing \texttt{".sdf"} string.
      }
      \sstexamplesubsection{
         provrem ff pattern='\_ave'
      }{
         This removes all ancestors that have paths that contain the
         string \texttt{"\_ave"} anywhere.
      }
      \sstexamplesubsection{
         provrem ff pattern='\_ave' remove=no
      }{
         This removes all ancestors that have paths that do not contain
         the string \texttt{"\_ave"} anywhere.
      }
      \sstexamplesubsection{
         provrem ff pattern='\_d[$\wedge$/]*\$'
      }{
         This removes all ancestors that have file base-names that begin
         with \texttt{"\_d"} . The pattern matches \texttt{"\_d"} followed by any number of
         characters that are not \texttt{"/"}, followed by the end of the string.
      }
      \sstexamplesubsection{
         provrem ff pattern='$\wedge$m51|$\wedge$m31'
      }{
         This removes all ancestors that have paths that begin with
         \texttt{"m51"} or \texttt{"m31"}.
      }
   }
   \label{pattern_matching_syntax:provrem}
   \sstdiytopic{
      Pattern Matching Syntax
   }{
      The syntax for the PATTERN parameter value is a minimal form of
      regular expression. The following atoms are allowed.

% This conditional text is to avoid a space appearing before the backslash
% in the hypertext, such as " \W" instead of "\W".
      \latex{
         \sstitemlist{

         \sstitem
         \texttt{"[chars]"} --- Matches any of the characters within the brackets.

         \sstitem
         \texttt{"[$\wedge$chars]"} --- Matches any character that is not within the
                       brackets (ignoring the initial \texttt{"$\wedge$"} character).
         \sstitem
         \texttt{"."} --- Matches any single character.

         \sstitem
         \texttt{"$\backslash$d"} --- Matches a single digit.

         \sstitem
         \texttt{"$\backslash$D"} --- Matches anything but a single digit.

         \sstitem
         \texttt{"$\backslash$w"} --- Matches any alphanumeric character, and \texttt{"\_"}.

         \sstitem
         \texttt{"$\backslash$W"} --- Matches anything but alphanumeric characters, and \texttt{"\_"}.

         \sstitem
         \texttt{"$\backslash$s"} --- Matches white space.

         \sstitem
         \texttt{"$\backslash$S"} --- Matches anything but white space.
         }
      }
      % \html{
      %   \sstitemlist{

      %    \sstitem
      %    \texttt{"[chars]"} --- Matches any of the characters within the brackets.

      %    \sstitem
      %    \texttt{"[$\wedge$chars]"} --- Matches any character that is not within the
      %                  brackets (ignoring the initial \texttt{"$\wedge$"} character).
      %    \sstitem
      %    \texttt{"."} --- Matches any single character.

      %    \sstitem
      %    \verb+"\d"+ --- Matches a single digit.

      %    \sstitem
      %    \verb+"\D"+ --- Matches anything but a single digit.

      %    \sstitem
      %    \verb+"\w"+ --- Matches any alphanumeric character, and \texttt{"\_"}.

      %    \sstitem
      %    \verb+"\W"+ --- Matches anything but alphanumeric characters, and \texttt{"\_"}.

      %    \sstitem
      %    \verb+"\s"+ --- Matches white space.

      %    \sstitem
      %    \verb+"\S"+ --- Matches anything but white space.
      %    }
      % }

      Any other character that has no special significance within a
      regular expression matches itself.  Characters that have special
      significance can be matched by preceding them with a backslash
      ($\backslash$) in which case their special significance is ignored (note,
      this does not apply to the characters in the set dDsSwW).

      Note, minus signs (\texttt{"-"}) within brackets have no special
      significance, so ranges of characters must be specified
      explicitly.

      The following quantifiers are allowed.
      \ssthitemlist{

         \sstitem
         \texttt{"*"} --- Matches zero or more of the preceding atom, choosing the
             largest possible number that gives a match.

         \sstitem
         \texttt{"*?"}--- Matches zero or more of the preceding atom, choosing the
            smallest possible number that gives a match.

         \sstitem
         \texttt{"+"} --- Matches one or more of the preceding atom, choosing the
             largest possible number that gives a match.

         \sstitem
         \texttt{"+?"}--- Matches one or more of the preceding atom, choosing the
             smallest possible number that gives a match.

         \sstitem
         \texttt{"?"} --- Matches zero or one of the preceding atom.

         \sstitem
         \texttt{"\{n\}"} --- Matches exactly $n$ occurrences of the preceding atom.
      }

      The following constraints are allowed.
      \ssthitemlist{

         \sstitem
         \texttt{"$\wedge$"} --- Matches the start of the test string.

         \sstitem
         \texttt{"\$"} --- Matches the end of the test string.
      }

      Multiple templates can be concatenated, using the \texttt{"|"} character to
      separate them.  The test string is compared against each one in
      turn until a match is found.
   }
   \sstdiytopic{
      Related Applications
   }{
      KAPPA: \htmlref{PROVADD}{PROVADD}, \htmlref{PROVMOD}{PROVMOD}, \htmlref{PROVSHOW}{PROVSHOW}.
   }
}

\sstroutine{
   PROVSHOW
}{
   Displays provenance information for an NDF
}{
   \sstdescription{
      This application displays details of the NDFs that were used in
      the creation of the supplied NDF.  This information is read from the
      PROVENANCE extension within the NDF, and includes both immediate
      parent NDFs and older ancestor NDFs (\emph{i.e.} the parents of the
      parents, \emph{etc.}).

      Each displayed NDF (see Parameter SHOW) is described in a
      block of lines.  The first line holds an integer index for the NDF
      followed by the path to that NDF.  Note, this path is where the NDF
      was when the provenance information was recorded.  It is of course
      possible that the NDF may subsequently have been moved or deleted.

      The remaining lines in the NDF description are as follows.

      \sstitemlist{

         \sstitem
         \texttt{"Parents"} --- A comma-separated list of integers that
         are the indices of the immediate parents of the NDF.  These
         are the integers that are displayed on the first line of each
         NDF description.

         \sstitem
         \texttt{"Date"} --- The formatted UTC date and time at which the
         provenance information for the NDF was recorded.

         \sstitem
         \texttt{"Creator"} --- A string identifying the software that created
         the NDF.

         \sstitem
         \texttt{"More"} --- A summary of any extra information about the
         NDF stored with the provenance information.  In general this may
         be an arbitrary HDS structure and so full details cannot be
         given on a single line.  The \HDSTRACEref\ command can be used to
         examine the MORE field in detail.  To see full details of the
         NDF with \texttt{"ID"} value of \texttt{12} (say), enter (from a UNIX shell)
         \texttt{"hdstrace fred.more.provenance.ancestors'(12)'"}, where
         \texttt{fred} is the name of the NDF supplied for Parameter NDF.
         If the NDF has no extra information, this item will not be
         present.

         \sstitem
         \texttt{"History"} --- This is only displayed if Parameter HISTORY is
         set to a \texttt{TRUE} value.  It contains information copied from the
         HISTORY component of the ancestor NDF.  See Parameter HISTORY.
      }

      In addition, a text file can be created containing the paths for the
      direct parents of the supplied NDF.  See Parameter PARENTS.

   }
   \sstusage{
      provshow ndf [show]
   }
   \sstparameters{
      \sstsubsection{
         DOTFILE = FILENAME (Read)
      }{
         Name of a new text file in which to store a description of the
         provenance tree using the ``dot'' format. This file can be
         visualised using third-party tools such as Graphviz,
         ZGRViewer, OmniGraffle, \emph{etc.}
      }
      \sstsubsection{
         HIDE = \_LOGICAL (Read)
      }{
        If \texttt{TRUE}, then any ancestors which are flagged as `hidden'
        (for example, using \htmlref{PROVREM}{PROVREM}) are excluded from
        the display.  If \texttt{FALSE}, then all requested ancestors, whether
        hidden or not, are included in the display (but hidden ancestors will
        be highlighted as such).  Note, choosing to exclude hidden ancestors
        may change the index displayed for each ancestor.  The default is
        to display hidden ancestors if and only if history is being
        displayed (see Parameter HISTORY).  []
      }
      \sstsubsection{
         HISTORY = \_LOGICAL (Read)
      }{
        If \texttt{TRUE}, any history records stored with each ancestor are
        included in the displayed information.  Since the amount of
        history information displayed can be large, and thus swamp other
        information, the default is not to display history information.

        When an existing NDF is used in the creation of a new NDF, the
        provenance system will copy selected records from the HISTORY
        component of the existing NDF and store them with the provenance
        information in the new NDF.  The history records copied are those
        that describe operations performed on the existing NDF itself.
        Inherited history records that describe operations performed on
        ancestors of the existing NDF are not copied.  \texttt{[FALSE]}
      }
      \sstsubsection{
         INEXT = LITERAL (Read)
      }{
        Determines which ancestor to display next. Only used if
        Parameter SHOW is set to \texttt{"Tree"}. The user is re-prompted for
        a new value for this parameter after each NDF is displayed. The
        new value should be the integer identifier for one of the parents
        of the currently displayed NDF. Alternatively, the string \texttt{"up"}
        can be supplied, causing the previously displayed NDF to be
        displayed again.
      }
      \sstsubsection{
         NDF = NDF (Read)
      }{
         The NDF data structure.
      }
      \sstsubsection{
         PARENTS = FILENAME (Read)
      }{
         Name of a new text file in which to put the paths to the direct
         parents of the supplied NDF.  These are written one per line with
         no extra text.  If null, no file is created.  \texttt{[!]}
      }
      \sstsubsection{
         SHOW = LITERAL (Read)
      }{
         Determines which ancestors are displayed on the screen. It can
         take any of the following case-insensitive values (or any
         abbreviation).

         \ssthitemlist{

            \sstitem
            \texttt{"All"} --- Display all ancestors, including the supplied
                            NDF itself.

            \sstitem
            \texttt{"Roots"}  --- Display only the root ancestors (i.e.
                               ancestors that do not themselves have any
                               recorded parents). The supplied NDF itself
                               is not displayed.

            \sstitem
            \texttt{"Parents"} --- Display only the direct parents of the
                                supplied NDF. The supplied NDF itself is
                                not displayed.

            \sstitem
            \texttt{"Tree"} --- Display the top level NDF and then asks the
                                user which parent to display next (see
                                Parameter INEXT). The whole family tree can
                                be navigated in this way.
         }
         \texttt{["All"]}

      }
   }
   \sstexamples{
      \sstexamplesubsection{
         provshow m51
      }{
        This displays information about the NDF m51, and all its
        recorded ancestors.
      }
      \sstexamplesubsection{
         provshow m51 roots
      }{
        This displays information about the root ancestors of the NDF
        m51.
      }
      \sstexamplesubsection{
         provshow m51 parents
      }{
        This displays information about the direct parents of the NDF
        m51.
     }
   }

   \sstnotes{
      \sstitemlist{

         \sstitem

         An input NDF is included in the provenance of an output NDF only
         if the DATA component of the input NDF is mapped for read or
         update access by the application. In other words, input NDFs
         which are accessed only for their metadata (\emph{e.g.} WCS
         information) are not included in the output provenance of an
         application.

         \sstitem

         If a \KAPPA\ application uses one or more input NDFs to create an
         output NDF, the output NDF may or may not contain provenance
         information depending on two things: 1) whether any of the input
         NDFs already contain provenance information, and 2) the value of
         the \texttt{AUTOPROV} environment variable.  It is usually necessary to
         set the \texttt{AUTOPROV} variable to \texttt{"1"} in order to create output
         NDFs that contain provenance information.  The exception to this
         if you are supplied with NDFs from another source that already
         contain provenance.  If such NDFs are used as inputs to \KAPPA\
         applications, then the output NDFs will contain provenance even
         if the \texttt{AUTOPROV} variable is unset.  However, setting \texttt{AUTOPROV}
         to \texttt{"0"} will always prevent provenance information being
         stored in the output NDFs.

         \sstitem
         Some other packages, such as \CCDPACK, follow the same strategy
         for creating and propagating provenance information.
      }
   }
   \sstdiytopic{
      Related Applications
   }{
      KAPPA: \htmlref{PROVADD}{PROVADD}, \htmlref{HISLIST}{HISLIST}.
   }
}

\sstroutine{
   PSF
}{
   Determines the parameters of a model star profile by fitting star
   images in a two-dimensional NDF
}{
   \sstdescription{
      This application finds a set of parameters to describe a model
      Gaussian star image.  It can be used for profile-fitting stellar
      photometry, to evaluate correction terms to aperture
      photometry, or for filtering.

      The model has a S\'{e}rsic radial profile:
      {\Large
      \[   D =  A \exp^{-0.5\,(r/\sigma)^{\gamma}} \]
      }where $r$ is calculated from the true radial distance from the star
      centre allowing for image ellipticity, $\sigma$ is the Gaussian
      precision constant or profile width.  The application combines a
      number of star images you specify  and determines a mean
      seeing-disc size, radial fall-off parameter ($\gamma$), axis ratio,
      and orientation of a model star image.

      A table, giving details of the seeing and ellipticity of each
      star image used can be reported to an output text file.  This
      table indicates if any star could not be used.  Reasons for
      rejecting stars are too-many bad pixels present in the image,
      the star is too close to the edge of the data array, the
      `star' is a poor fit to model or it could not be located.

      An optional plot of the mean profile and the fitted function may
      be produced.  The two-dimensional point-spread function may be stored
      in an \NDFref{NDF} for later use, as may the one-dimensional fitted profile.
   }
   \sstusage{
      psf in incat [device] [out] [cut] [range] [isize] [poscols]
   }
   \sstparameters{
      \sstsubsection{
         AXES = \_LOGICAL (Read)
      }{
         \texttt{TRUE} if labelled and annotated axes are to be drawn around the
         plot.  The width of the margins left for the annotation may be
         controlled using Parameter MARGIN.  The appearance of the axes
         (colours, founts, \emph{etc.}) can be controlled using the Parameter
         STYLE.  The dynamic default is \texttt{TRUE} if CLEAR is \texttt{TRUE}, and
         \texttt{FALSE} otherwise. \texttt{[]}
      }
      \sstsubsection{
         CLEAR = \_LOGICAL (Read)
      }{
         If \texttt{TRUE} the current picture is cleared before the plot is
         drawn.  If CLEAR is \texttt{FALSE} not only is the existing plot retained,
         but also an attempt is made to align the new picture with the
         existing picture.  Thus you can generate a composite plot within
         a single set of axes, say using different colours or modes to
         distinguish data from different datasets.  \texttt{[TRUE]}
      }
      \sstsubsection{
         COFILE = FILENAME (Read)
      }{
         Name of a text file containing the co-ordinates of the stars
         to be used.  It is only accessed if Parameter INCAT is given a null (\texttt{{!}})
         value.  Each line should contain the
         \xref{formatted axis values}{sun210}{AST_UNFORMAT} for a
         single position, in the \htmlref{current Frame}{se:curframe}~ of
         the NDF.  Columns can be separated by spaces, tabs or commas.  The
         file may contain
         comment lines with the first character \# or \texttt{{!}}.  Other columns may
         be included in the file, in which case the columns holding the
         required co-ordinates should be specified using Parameter POSCOLS.
      }
      \sstsubsection{
         CUT = \_REAL (Read)
      }{
         This parameter controls the size of the output NDF.  If it is
         null, \texttt{{!}}, the dimension of the square NDF will be the size of
         the region used to calculate the radial profile, which usually
         is given by RANGE $*$ width in pixels $*$ AXISR, unless truncated.
         If CUT has a value it is the threshold which must be included
         in the PSF NDF, and it is given as the fraction of the peak
         amplitude of the PSF.  For example, if CUT=\texttt{{0.5}} the NDF would
         contain the point-spread function to half maximum.  CUT must
         be greater than 0 and less than 1.  The suggested default is
         0.0001.  \texttt{[!]}
      }
      \sstsubsection{
         DEVICE = \htmlref{DEVICE}{se:selgradev} (Read)
      }{
         The graphics workstation on which to produce a plot of the
         mean radial profile of the stars and the fitted function.  A
         null (\texttt{{!}}) name indicates that no plot is required.
         \texttt{[}current graphics device\texttt{{]}}
      }
      \sstsubsection{
         GAUSS = \_LOGICAL (Read)
      }{
         If \texttt{TRUE}, the $\gamma$ coefficient is fixed to be 2; in other words
         the best-fitting two-dimensional Gaussian is evaluated.  If
         \texttt{FALSE}, $\gamma$ is a free parameter of the fit, and the derived
         value is returned in Parameter GAMMA.  \texttt{[FALSE]}
      }
      \sstsubsection{
         IN = NDF (Read)
      }{
         The NDF containing the star images to be fitted.
      }
      \sstsubsection{
         INCAT = FILENAME (Read)
      }{
         A catalogue containing a positions list (such as produced by applications
         \htmlref{CURSOR}{CURSOR}, \htmlref{LISTMAKE}{LISTMAKE})  giving the star
         positions to use.  If a null (\texttt{{!}}) value is supplied Parameter
         COFILE will be used to get the star positions from a simple text file.
      }
      \sstsubsection{
         ISIZE = \_INTEGER (Read)
      }{
         The side of the square area to be used when forming the marginal
         profiles for a star image, given as a number of pixels.  It should
         be sufficiently large to contain the entire star image.  It should
         be an odd number and must lie in the range from 3 to 101.  \texttt{[15]}
      }
      \sstsubsection{
         LOGFILE = FILENAME (Read)
      }{
         Text file to contain the table of parameters for each star.  A
         null (\texttt{{!}}) name indicates that no log file is required.  \texttt{[!]}
      }
      \sstsubsection{
         MARGIN( 4 ) = \_REAL (Read)
      }{
         The widths of the margins to leave for axis annotation, given
         as fractions of the corresponding dimension of the current picture.
         Four values may be given, in the order: bottom, right, top, left.
         If fewer than four values are given, extra values are used equal to
         the first supplied value.  If these margins are too narrow, any axis
         annotation may be clipped.  If a null (\texttt{{!}}) value is supplied, the
         value used is \texttt{0.15} (for all edges) if either annotated axes or a
         key are produced, and zero otherwise.  \texttt{[}current value\texttt{{]}}
      }
      \sstsubsection{
         MARKER = \_INTEGER (Read)
      }{
         The \PGPLOT\  marker type to use for the data values in the plot.
         \texttt{[}current value\texttt{{]}}
      }
      \sstsubsection{
         MINOR = \_LOGICAL (Read)
      }{
         If MINOR is \texttt{TRUE} the horizontal axis of the plot is annotated
         with distance along the minor axis from the centre of the PSF.  If
         MINOR is \texttt{FALSE}, the distance along the major axis is used.  \texttt{[TRUE]}
      }
      \sstsubsection{
         NORM = \_LOGICAL (Read)
      }{
         If \texttt{TRUE}, the model PSF is normalized so that it has a peak value
         of unity.  Otherwise, its peak value is equal to the peak
         value of the fit to the first usable star, in the data units of the input
         NDF.  \texttt{[TRUE]}
      }
      \sstsubsection{
         OUT = NDF (Write)
      }{
         The NDF containing the fitted point-spread function evaluated
         at each pixel.  If null, \texttt{{!}}, is entered no output NDF will
         be created.  The dimensions of the array are controlled by
         Parameter CUT.  The pixel origin is chosen to align the model PSF
         with the first fitted star in pixel co-ordinates, thus allowing
         the NDF holding the model PSF to be compared directly with the
         input NDF.  A \htmlref{WCS}{apndf:wcs} component is stored in the output NDF holding a
         copy of the input WCS component.  An additional Frame with Domain
         name OFFSET is added, and is made the current Frame.  This Frame
         measures the distance from the PSF centre in the units in which
         the FWHM is reported.  These changes allows the NDF holding the
         model PSF to be compared directly with the input NDF.  \texttt{[!]}
      }
      \sstsubsection{
         POSCOLS = \_INTEGER (Read)
      }{
         Column positions of the co-ordinates (x then y) in an input record
         of the file specified by Parameter COFILE.  The columns must be
         different amongst themselves.  If there is duplication new values
         will be requested.  Only accessed if INCAT is given a null (\texttt{{!}})
         value.  If a null (\texttt{{!}}) value is supplied for POSCOLS, the values
         [1,2] will be used.  \texttt{[!]}
      }
      \sstsubsection{
         PROFOUT = NDF (Write)
      }{
         The NDF containing the one-dimensional fitted profile as displayed
         in the plot.  If null, \texttt{{!}}, is entered no output NDF will be created.
         The DATA component of this NDF holds the fitted PSF value at each
         radial bin.  The \htmlref{VARIANCE}{apndf:variance}~ component holds the square of the residuals
         between the fitted values and the binned values derived from the
         input NDF.  An \htmlref{AXIS}{apndf:axis}~ component is included in the NDF containing the
         radial distance as displayed in the plot.  \texttt{[!]}
      }
      \sstsubsection{
         RANGE = \_REAL (Read)
      }{
         The number of image profile widths out to which the radial
         star profile is to be fitted.  (There is an upper limit of 100
         pixels to the radius at which data are actually used.) \texttt{[4.0]}
      }
      \sstsubsection{
         STYLE = \htmlref{GROUP}{se:groups} (Read)
      }{
         A group of attribute settings describing the plotting style to use
         when drawing the annotated axes, data values, and the model profile.

         A comma-separated list of strings should be given in which each
         string is either an attribute setting, or the name of a text
         file preceded by an up-arrow character \texttt{"$\wedge$"}.  Such text files
         should contain further comma-separated lists which will be
         read and interpreted in the same manner.  Attribute settings
         are applied in the order in which they occur within the list,
         with later settings overriding any earlier settings given for
         the same attribute.

         Each individual attribute setting should be of the form:

            $<$name$>$=$<$value$>$

         where $<$name$>$ is the name of a plotting attribute, and $<$value$>$
         is the value to assign to the attribute.  Default values will be
         used for any unspecified attributes.  All attributes will be
         defaulted if a null value (\texttt{{!}})---the initial default---is supplied.
         To apply changes of style to only the current invocation, begin these
         attributes with a plus sign.  A mixture of persistent and temporary
         style changes is achieved by listing all the persistent attributes
         followed by a plus sign then the list of temporary attributes.

         See \slhyperref{Plotting Attributes}{Section~}{}{ap:plotting_attr}
         for a description of the available attributes.  Any unrecognised
         attributes are ignored (no error is reported).
         The appearance of the model curve is controlled by the attributes
         \htmlattref{Colour(Curves)}{Colour(element)},
         \htmlattref{Width(Curves)}{Width(element)}, \emph{etc.} (the synonym
         \att{Line} may be used in place of \att{Curves}).  The appearance of the
         markers representing the real data is controlled by \att{Colour(Markers)},
         \att{Width(Markers)},
         \emph{etc.} (the synonym \att{Symbols} may be used in place of \att{Markers}).
         \texttt{[}current value\texttt{{]}}
      }
      \sstsubsection{
         TITLE = LITERAL (Read)
      }{
         The title for the NDF to contain the fitted point-spread
         function.  If null, \texttt{{!}}, is entered the NDF will not contain a
         title.  \texttt{["KAPPA - PSF"]}
      }
      \sstsubsection{
         USEAXIS = GROUP (Read)
      }{
         USEAXIS is only accessed if the current \htmlref{co-ordinate Frame}{se:domains}~ of
         the NDF has more than two axes.  A group of two strings should be
         supplied specifying the two axes which are to be used when
         determining distances, reporting positions, \emph{etc}.  Each axis can be
         specified using one of the following options.

         \ssthitemlist{

            \sstitem
            Its integer index within the current Frame of the
            input  NDF (in the range 1 to the number of axes in the
            current Frame).

            \sstitem
            Its \htmlattref{Symbol}{Symbol(axis)}~ string such as
            \texttt{"RA"} or \texttt{"VRAD"}.

            \sstitem
            A generic option where \texttt{"SPEC"} requests the spectral axis,
            \texttt{"TIME"} selects the time axis, \texttt{"SKYLON"} and
            \texttt{"SKYLAT"} picks the sky longitude and latitude axes
            respectively.  Only those axis domains present are
            available as options.
         }

         A list of acceptable values is displayed if an illegal value is
         supplied.  If a null (\texttt{{!}}) value is supplied, the axes with the
         same indices as the two significant NDF pixel axes are used.  \texttt{[!]}
      }
      \sstsubsection{
         XLEFT = \_DOUBLE (Read)
      }{
         The axis value to place at the left hand end of the horizontal
         axis of the plot.  If a null (\texttt{{!}}) value is supplied, a suitable
         default value will be found and used.  The value supplied may be
         greater than or less than the value supplied for XRIGHT.  \texttt{[!]}
      }
      \sstsubsection{
         XRIGHT = \_DOUBLE (Read)
      }{
         The axis value to place at the right hand end of the horizontal
         axis of the plot.  If a null (\texttt{{!}}) value is supplied, a suitable
         default value will be found and used.  The value supplied may be
         greater than or less than the value supplied for XLEFT.  \texttt{[!]}
      }
      \sstsubsection{
         YBOT = \_DOUBLE (Read)
      }{
         The axis value to place at the bottom end of the vertical axis of
         the plot.  If a null (\texttt{{!}}) value is supplied, a suitable default
         value will be found and used.  The value supplied may be greater
         than or less than the value supplied for YTOP.  \texttt{[!]}
      }
      \sstsubsection{
         YTOP = \_DOUBLE (Read)
      }{
         The axis value to place at the top end of the vertical axis of
         the plot.  If a null (\texttt{{!}}) value is supplied, a suitable default
         value will be found and used.  The value supplied may be greater
         than or less than the value supplied for YBOT.  \texttt{[!]}
      }
   }
   \sstresparameters{
      \sstsubsection{
         AMP1 = \_REAL (Write)
      }{
         The fitted peak amplitude of the first usable star, in the data
         units of the input NDF.
      }
      \sstsubsection{
         AXISR = \_REAL (Write)
      }{
         The axis ratio of the star images: the ratio of the major
         axis length to that of the minor axis.
      }
      \sstsubsection{
         CENTRE = LITERAL (Write)
      }{
         The formatted co-ordinates of the first fitted star position,
         in the current Frame of the NDF.
      }
      \sstsubsection{
         FWHM = \_REAL (Write)
      }{
         The seeing-disc size: the full width at half maximum across the
         minor axis of the stars.  It is in units defined by the current
         Frame of the NDF.  For instance, a value in arcseconds will be
         reported if the current Frame is a SKY Frame, but pixels will be
         used if it is a PIXEL Frame.
      }
      \sstsubsection{
         GAMMA = \_REAL (Write)
      }{
         The radial fall-off parameter, $\gamma$, of the star images.  See
         the description for more details.  A $\gamma$ of two would be a
         Gaussian.
      }
      \sstsubsection{
         ORIENT = \_REAL (Write)
      }{
         The orientation of the major axis of the star images, in degrees.
         If the current Frame of the NDF is a SKY Frame, this will be a
         position angle (measured from north through east).  Otherwise, it
         will be measured from the positive direction of the first
         current Frame axis (\texttt{"X"}) towards the second current Frame axis
         (\texttt{"Y"}).
      }
      \sstsubsection{
         TOTAL = \_REAL (Write)
      }{
         The flux of the fitted function integrated to infinite radius.
         Its unit is the product of the data unit of the input NDF and
         the square of the radial unit, such as pixel or arcsec, for
         the current WCS Frame, when NORM=\texttt{FALSE}.  When NORM=\texttt{TRUE},
         TOTAL is just measured in the squared radial unit.  Therefore, for
         direct comparison of total flux, the same units must be used.
       }
   }
   \sstexamples{
      \sstexamplesubsection{
         psf ngc6405i starlist.FIT $\backslash$
      }{
         Derives the mean point-spread function for the stars images
         in the NDF called ngc6405i that are situated near the
         co-ordinates given in the positions list \texttt{starlist.FIT}.  A
         plot of the profile is drawn on the \htmlref{current graphics device}{se:devglobal}.
      }
      \sstexamplesubsection{
         psf ngc6405i starlist device=!
      }{
         As above but there is no graphical output, and the file type of
         the input positions list is defaulted.
      }
      \sstexamplesubsection{
         psf ngc6405i cofile=starlist.dat gauss $\backslash$
      }{
         As the first example, except the psf is fitted to a
         two-dimensional Gaussian, and the positions are given in a simple
         text file (\texttt{starlist.dat}) instead of a positions list.
      }
      \sstexamplesubsection{
         psf incat=starlist.FIT in=ngc6405i logfile=fit.log fwhm=(seeing) $\backslash$
      }{
         As the first example, but the results, including the fits to
         each star, are written to the text file \texttt{fit.log}.  The
         full-width half-maximum is written to the ICL variable SEEING
         rather than the parameter file.
      }
      \sstexamplesubsection{
         psf ngc6405i starlist isize=31 style="'title=Point spread function'"
      }{
         As the first example, but the area including a star image is
         31 pixels square, say because the seeing is poor or the pixels
         are smaller than normal.  The graph is titled \texttt{"Point spread
         function"}.
      }
   }
   \sstnotes{
      \sstitemlist{

         \sstitem
         Values for the FWHM seeing are given in arcseconds if the
         Current co-ordinate Frame of the NDF is a SKY Frame.

         \sstitem
         The stars used to determine the mean image parameters should
         be chosen to represent those whose magnitudes are to be found
         using a stellar photometry application, and to be sufficiently
         bright, uncrowded, and noise-free to allow an accurate fit to be
         made.

         \sstitem
         It is assumed that the image scale does not vary significantly
         across the image.

         \sstitem
         The method to calculate the fit is as follows.

         \ssthitemlist{

            \sstitem
               Marginal profiles of each star image are formed in four
               directions: at 0, 45, 90 and 135 degrees to the $x$ axis.  The
               profiles are cleaned via an iterative modal filter that
               removes contamination such as neighbouring stars; moving from
               the centre of the star, the filter prevents each data point
               from exceeding the maximum of the two previous data values.

            \sstitem
               A Gaussian curve and background is fitted to each profile
               iteratively refining the parameters until parameters differ
               by less than 0.1 per cent from the previous iteration.  If
               convergence is not met after fifteen iterations, each fit
               parameter is approximately the average of its last pair of
               values.  The initial background is the lower quartile.

               Using the resulting four Gaussian centres, a mean centre is
               found for each star.  Iterations cease when the mean centroid
               position shifts by less 0.001 from the previous iteration, or
               after three iterations if the nominal tolerance is not
               achieved.

           \sstitem
               The four Gaussian widths of all the stars are combined
               modally, using an amplitude-weighted average with rejection of
               erroneous data (using a maximum-likelihood function for a
               statistical model in which any of the centres has a constant
               probability of being corrupt).  From the average widths along
               the four profiles, the seeing-disc size, axis ratio and axis
               inclination are calculated.

           \sstitem
               The data surrounding each star is then binned into isophotal
               zones which are elliptical annuli centred on the star---the
               ellipse parameters being those just calculated.  The data in
               each zone is processed to remove erroneous points (using the
               aforementioned maximum-likelihood function) and to find an
               average value.  A Gaussian profile is fitted to these average
               values and the derived amplitude is used to normalise the
               values to an amplitude of unity.  The normalised values are put
               into bins together with the corresponding data from all other
               stars and these binned data represent a weighted average radial
               profile for the set of stars, with the image ellipticity
               removed.  Finally a radial profile is fitted to these data,
               giving the radial profile parameter gamma and a final
               re-estimate of the seeing-disc size.
         }

         \sstitem
         If a plot was requested the application stores two pictures in the
         \htmlref{graphics database}{se:agitate}~ in the following order:
         a FRAME of the specified size containing the title, annotated
         axes, and line plot; and a DATA picture, containing  just the data
         plot.  Note, the FRAME picture is only created  if annotated axes
         have been drawn, or if non-zero margins were specified using
         Parameter MARGIN.  The NDF associated with the plot is not stored
         by reference with the DATA picture.  On exit the current database
         picture for the chosen device reverts to the input picture.
      }
   }
   \sstdiytopic{
      Related Applications
   }{
\xref{PHOTOM}{sun45}{};
\xref{Starman}{sun141}{}.
   }
   \sstimplementationstatus{
      \sstitemlist{

         \sstitem
         This routine correctly processes the \htmlref{AXIS}{apndf:axis}, DATA,
         \htmlref{QUALITY}{apndf:quality}, \htmlref{LABEL}{apndf:label},
         \htmlref{WCS}{apndf:wcs}, and \htmlref{TITLE}{apndf:title} components
         of an NDF data structure.

         \sstitem
         Processing of \htmlref{bad pixels}{se:masking} and automatic
         \htmlref{quality masking}{se:qualitymask} are supported.

         \sstitem
         All \htmlref{non-complex numeric data types}{ap:HDStypes} can be handled.  The
         output point-spread-function NDF has the same type as the input NDF.
      }
   }
}




\sstroutine{
   QUALTOBAD
}{
   Set selected NDF pixels bad on the basis of Quality
}{
   \sstdescription{
      This routine produces a copy of an input \NDFref{NDF}~ in which selected pixels
      are set bad.  The selection is based on the values in the
      \htmlref{QUALITY component}{apndf:quality}~ of the input NDF; any pixel which
      holds a set of qualities satisfying the \htmlref{quality expression}{se:qnames}~ given
      for Parameter QEXP is set bad in the output NDF.  Named
      qualities can be associated with specified pixels using the
      SETQUAL task.
   }
   \sstusage{
      qualtobad in out qexp
   }
   \sstparameters{
      \sstsubsection{
         IN = NDF (Read)
      }{
         The input NDF.
      }
      \sstsubsection{
         OUT = NDF (Write)
      }{
         The output NDF.
      }
      \sstsubsection{
         QEXP = LITERAL (Read)
      }{
         The quality expression.
      }
      \sstsubsection{
         TITLE = LITERAL (Read)
      }{
         Title for the output NDF.  A null (\texttt{{!}}) value will cause the input
         title to be used.  \texttt{[!]}
      }
   }
   \sstexamples{
      \sstexamplesubsection{
         qualtobad m51$*$ $*$\_clean saturated.or.glitch
      }{
         This example copies all NDFs starting with the string \texttt{"m51"} to
         a set of corresponding output NDFs.  The name of each output
         NDF is formed by extending the name of the input NDF with the
         string \texttt{"\_clean"}.  Any pixels which hold either of the qualities
         \texttt{"saturated"} or \texttt{"glitch"} are set to the bad value in the output
         NDFs.
      }
   }
   \sstdiytopic{
      Related Applications
   }{
KAPPA: \htmlref{REMQUAL}{REMQUAL},
\htmlref{SETBB}{SETBB},
\htmlref{SETQUAL}{SETQUAL},
\htmlref{SHOWQUAL}{SHOWQUAL}.
   }
}

\sstroutine{
   REGIONMASK
}{
   Applies a mask to a region of an NDF
}{
   \sstdescription{
      This routine masks out a region of an NDF by setting pixels to
      the bad value, or to a specified constant value. The region to
      be masked is specified by a file (see Parameter REGION) that
      should contain a description of the region in a form readable
      by the Starlink AST library (see \xref{SUN/211}{sun211}{} or
      \xref{SUN/210}{sun210}{}). Such formats include AST's own native
      format and other formats that can be converted automatically to
      an AST Region (\emph{e.g.} IVOA MOC and STC-S regions).  AST
      Regions can be created, for instance, using the Starlink ATOOLS
      package (a high-level interface to the facilities of the AST
      library).
   }
   \sstusage{
      regionmask in region out
   }
   \sstparameters{
      \sstsubsection{
         CONST = LITERAL (Given)
      }{
         The constant numerical value to assign to the region, or the
         string \texttt{"Bad"}.  [\texttt{"Bad"}]
      }
      \sstsubsection{
         IN = NDF (Read)
      }{
         The name of the input NDF.
      }
      \sstsubsection{
         INSIDE = \_LOGICAL (Read)
      }{
         If a \texttt{TRUE} value is supplied, the constant value is assigned
         to the inside of the region. Otherwise, it is assigned to the
         outside.  \texttt{[TRUE]}
      }
      \sstsubsection{
         OUT = NDF (Write)
      }{
         The name of the output NDF.
      }
      \sstsubsection{
         REGION = FILENAME (Read)
      }{
         The name of the file containing a description of the Region.
         This can be a text file holding a dump of an AST Region (any
         sub-class of Region may be supplied---\emph{e.g.} Box, Polygon,
         CmpRegion, Prism, \emph{etc.}), or any file that can be converted
         automatically to an AST Region (for instance an IVOA MOC in text
         or FITS format, an IVOA STC-S region in text format). An NDF may
         also be supplied, in which case the rectangular boundary of the
	 NDF is used as the Region. If the axes spanned by the Region are
	 not the same as those of the current WCS Frame in the input NDF,
	 an attempt will be made to create an equivalent new Region that
	 does match the current WCS Frame. An error will be reported if
	 this is not possible.
      }
   }
   \sstexamples{
      \sstexamplesubsection{
         regionmask a1060 galaxies.txt a1060\_sky
      }{
         This copies input NDF a1060 to the output NDF a1060\_sky,
         setting pixels bad if they are contained within the Region
         specified in text file \texttt{"galaxies.txt"}.
      }
   }
   \sstdiytopic{
      Related Applications
   }{
      KAPPA: \htmlref{ARDMASK}{ARDMASK};
      ATOOLS: ASTBOX, ASTCMPREGION, ASTELLIPSE, ASTINTERVAL, ASTPOLYGON.
   }
   \sstimplementationstatus{
      \sstitemlist{

         \sstitem
         This routine correctly processes the \htmlref{WCS}{apndf:wcs}, \htmlref{AXIS}{apndf:axis}, DATA, \htmlref{QUALITY}{apndf:quality},
         \htmlref{LABEL}{apndf:label}, \htmlref{TITLE}{apndf:title}, \htmlref{UNITS}{apndf:units}, \htmlref{HISTORY}{apndf:history}, and \htmlref{VARIANCE}{apndf:variance}~ components of an NDF
         data structure and propagates all \htmlref{extensions}{apndf:extensions}.

         \sstitem
         Processing of \htmlref{bad pixels}{se:masking} and automatic \htmlref{quality masking}{se:qualitymask} are
         supported.

         \sstitem
         All \htmlref{non-complex numeric data types}{ap:HDStypes} can be handled.
      }
   }
}

\sstroutine{
   REGRID
}{
   Applies a geometrical transformation to an NDF
}{
   \sstdescription{
      This application uses a specified Mapping to re-grid the pixel
      positions in an \NDFref{NDF}. The specified Mapping should transform pixel
      co-ordinates in the input NDF into the corresponding pixel
      co-ordinates in the output NDF.

      By default, the bounds of the output pixel grid are chosen so that
      they just encompass all the transformed input data, but they can
      be set explicitly using Parameters LBOUND and UBOUND.

      Two algorithms are available for determining the output pixel
      values: resampling and rebinning (the method used is determined by
      the REBIN parameter).

      The Mapping to use can be supplied in several different ways (see
      Parameter MAPPING).
   }
   \sstusage{
      regrid in out [method]
   }
   \sstparameters{
      \sstsubsection{
         AXES() = \_INTEGER (Read)
      }{
         The indices of the pixel axes that are to be re-gridded. These
         should be in the range 1 to NDIM (the number of pixel axes in
         the NDF). Each value may appear at most once. The order of the
         supplied values is insignificant.  If a null (\texttt{{!}}) value is
         supplied, then all pixel axes are re-gridded.  Otherwise, only
         the specified pixel axes are regridded.  Note, it is not always
         possible to specify completely arbitrary combinations of pixel
         axes to be regridded.  For instance, if the current WCS Frame
         contains RA and Dec. axes, then it is not possible to regrid one
         of the corresponding pixel axes without the other.  An error will
         be reported in such cases.  \texttt{[!]}
      }
      \sstsubsection{
         CONSERVE = \_LOGICAL (Read)
      }{
         If set \texttt{TRUE}, then the output pixel values will be scaled in
         such a way as to preserve the total data value in a feature on
         the sky.  The scaling factor is the ratio of the output pixel
         size to the input pixel size.  This option can only be used if
         the Mapping is successfully approximated by one or more linear
         transformations.  Thus an error will be reported if it used
         when the TOL parameter is set to zero (which stops the use of
         linear approximations), or if the Mapping is too non-linear to
         be approximated by a piece-wise linear transformation.  The
         ratio of output to input pixel size is evaluated once for each
         panel of the piece-wise linear approximation to the Mapping,
         and is assumed to be constant for all output pixels in the
         panel.  This parameter is ignored if the NORM parameter is set
         \texttt{FALSE}.  \texttt{[TRUE]}
      }
      \sstsubsection{
         IN = NDF (Read)
      }{
         The NDF to be transformed.
      }
      \sstsubsection{
         LBOUND( ) = \_INTEGER (Read)
      }{
         The lower pixel-index bounds of the output NDF.  The number of
         values must be equal to the number of dimensions in the output
         NDF.  If a null value is supplied, default bounds will be used
         which are just low enough to fit in all the transformed pixels
         of the input NDF.  \texttt{[!]}
      }
      \sstsubsection{
         MAPPING = FILENAME (Read)
      }{
         The name of a file containing the Mapping to be used, or null (\texttt{{!}})
         if the input NDF is to be mapped into its own
         \htmlref{current Frame}{se:domains}.  If a file is supplied,
         the forward direction of the Mapping should transform pixel
         co-ordinates in the input NDF into the corresponding pixel
         co-ordinates in the output NDF. If only a subset of pixel
         axes are being re-gridded, then the inputs to the Mapping
         should correspond to the pixel axes specified via Parameter
         AXES.  The file may be one of the following.

         \ssthitemlist{

            \sstitem
            A text file containing a textual representation of the AST Mapping
            to use. Such files can be created by \htmlref{WCSADD}{WCSADD}.

            \sstitem
            A text file containing a textual representation of an \htmlref{AST
            FrameSet}{se:curframe}. If the FrameSet contains a Frame with
            \htmlref{Domain PIXEL}{se:resdoms}, then the Mapping used is the
            Mapping from the PIXEL Frame to the current Frame. If there is no
            PIXEL Frame in the FrameSet, then the Mapping used is the Mapping
            from the base Frame to the Current Frame.

            \sstitem
            A {\FITSref}~ file.  The Mapping used is the Mapping from the FITS
            pixel co-ordinates in which the centre of the bottom-left pixel
            is at co-ordinates (1,1), to the co-ordinate system represented
            by the primary WCS headers, CRVAL, CRPIX, \emph{etc}.

            \sstitem
            An NDF.  The Mapping used is the Mapping from the PIXEL Frame
            to the Current Frame of its WCS FrameSet.

         }
         If a null (\texttt{{!}}) value is supplied, the Mapping used is the Mapping
         from pixel co-ordinates in the input NDF to the current Frame in
         the input NDF. The output NDF will then have pixel co-ordinates
         which match the co-ordinates of the current Frame of the input
         NDF (apart from possible additional scalings as specified by the
         SCALE parameter).
      }
      \sstsubsection{
         METHOD = \htmlref{LITERAL}{se:parmenu} (Read)
      }{
         The method to use when sampling the input pixel values (if
         resampling), or dividing an input pixel value between a group
         of neighbouring output pixels (if rebinning).  For details
         of these schemes, see the descriptions of routines
         \xref{AST\_RESAMPLEx}{sun210}{AST_RESAMPLE\$<X>\$} and
         \xref{AST\_REBINSEQx}{sun210}{AST_REBINSEQ\$<X>\$} in
         \xref{SUN/210}{sun210}{}. METHOD can take the following values.

         \ssthitemlist{

            \sstitem
            \texttt{"Bilinear"} --- When resampling, the output pixel values are
            calculated by bi-linear interpolation among the four nearest pixels
            values in the input NDF.  When rebinning, the input pixel value
            is divided up bi-linearly between the four nearest output pixels.
            Produces smoother output NDFs than the nearest-neighbour scheme, but
            is marginally slower.

            \sstitem
            \texttt{"Nearest"} --- When resampling, the output pixel values are assigned
            the value of the single nearest input pixel.  When rebinning,
            the input pixel value is assigned completely to the single
            nearest output pixel.

            \sstitem
            \texttt{"Sinc"} --- Uses the ${\textrm{sinc}}({\pi}x)$ kernel, where $x$ is the pixel
            offset from the interpolation point (resampling) or transformed
            input pixel centre (rebinning), and ${\textrm{sinc}}(z)=\sin(z)/z$.  Use of this
            scheme is not recommended.

            \sstitem
            \texttt{"SincSinc"} --- Uses the ${\textrm{sinc}}({\pi}x){\textrm{sinc}}(k{\pi}x)$ kernel.  A
            valuable general-purpose scheme, intermediate in its visual effect
            on NDFs between the bi-linear and nearest-neighbour schemes.

            \sstitem
            \texttt{"SincCos"} --- Uses the ${\textrm{sinc}}({\pi}x)\cos(k{\pi}x)$ kernel.  Gives
            similar results to the \texttt{"Sincsinc"} scheme.

            \sstitem
            \texttt{"SincGauss"} --- Uses the ${\textrm{sinc}}({\pi}x)e^{-kx^2}$ kernel.  Good
            results can be obtained by matching the FWHM of the
            envelope function to the point-spread function of the
            input data (see Parameter PARAMS).

            \sstitem
            \texttt{"Somb"} --- Uses the ${\textrm{somb}}({\pi}x)$ kernel, where $x$ is the pixel
            offset from the interpolation point (resampling) or transformed
            input pixel centre (rebinning), and
            ${\textrm{somb}}(z)=2*J_{1}(z)/z$.  $J_1$ is the
            first-order Bessel function of the first kind.  This scheme is
            similar to the \texttt{"Sinc"} scheme.

            \sstitem
            \texttt{"SombCos"} --- Uses the ${\textrm{somb}}({\pi}x)\cos(k{\pi}x)$ kernel.  This
            scheme is similar to the \texttt{"SincCos"} scheme.

            \sstitem
            \texttt{"Gauss"} --- Uses the $e^{-kx^2}$ kernel.
            The FWHM of the Gaussian is given by Parameter PARAMS(2), and
            the point at which to truncate the Gaussian to zero is given by
            Parameter PARAMS(1).

            \sstitem
            \texttt{"BlockAve"} --- Block averaging over all pixels in the
            surrounding $N$-dimensional cube. This option is only available
            when resampling (\emph{i.e.} if REBIN is set to \texttt{FALSE}).

         }
         All methods propagate variances from input to output, but the
         variance estimates produced by these schemes other than
         nearest neighbour need to be treated with care since the spatial
         smoothing produced by these methods introduces
         correlations in the variance estimates.  Also, the degree of
         smoothing produced varies across the NDF. This is because a
         sample taken at a pixel centre will have no contributions from the
         neighbouring pixels, whereas a sample taken at the corner of a
         pixel will have equal contributions from all four neighbouring
         pixels, resulting in greater smoothing and lower noise.  This
         effect can produce complex Moir\'{e} patterns in the output
         variance estimates, resulting from the interference of the
         spatial frequencies in the sample positions and in the pixel-centre
         positions.  For these reasons, if you want to use the
         output variances, you are generally safer using nearest-neighbour
         interpolation.  The initial default is \texttt{"Nearest"}.
         \texttt{[}current value\texttt{{]}}
      }
      \sstsubsection{
         NORM = \_LOGICAL (Read)
      }{
         In general, each output pixel contains contributions from
         multiple input pixel values, and the number of input pixels
         contributing to each output pixel will vary from pixel to
         pixel.  If NORM is set \texttt{TRUE} (the default), then each output
         value is normalised by dividing it by the number of
         contributing input pixels, resulting in each output value being
         the weighted mean of the contributing input values.  However,
         if NORM is set \texttt{FALSE}, this normalisation is not applied.  See
         also Parameter CONSERVE.  \texttt{[TRUE]}
      }
      \sstsubsection{
         OUT = NDF (Write)
      }{
         The transformed NDF.
      }
      \sstsubsection{
         PARAMS( 2 ) = \_DOUBLE (Read)
      }{
         An optional array which consists of additional parameters
         required by the Sinc, SincSinc, SincCos, SincGauss, Somb,
         SombCos, and Gauss methods.

         PARAMS(1) is required by all the above schemes.  It is used to
         specify how many pixels are to contribute to the interpolated
         result on either side of the interpolation or binning point
         in each dimension.  Typically, a value of \texttt{2} is
         appropriate and the minimum allowed value is \texttt{1}
         (\emph{i.e.} one pixel on each side). A value of zero or
         fewer indicates that a suitable number of pixels should be
         calculated automatically.  \texttt{[0]}

         PARAMS(2) is required only by the Gauss, SincSinc, SincCos,
         and SincGauss schemes.  For the SombCos, SincSinc, and
         SincCos schemes, it specifies the number of pixels at which
         the envelope of the function goes to zero.  The minimum value
         is \texttt{1.0}, and the run-time default value is \texttt{2.0}.
         For the Gauss and SincGauss schemes, it specifies the
         full-width at half-maximum (FWHM) of the Gaussian envelope
         measured in output pixels.
         The minimum value is \texttt{0.1}, and the run-time default is
         \texttt{1.0}.  On astronomical images and spectra, good results
         are often obtained by approximately matching the FWHM of the
         envelope function, given by PARAMS(2), to the point-spread
         function of the input data.  \texttt{[]}
      }
      \sstsubsection{
         REBIN = \_LOGICAL (Read)
      }{
         Determines the algorithm used to calculate the output pixel
         values.  If a \texttt{TRUE} value is given, a rebinning algorithm is used.
         Otherwise, a resampling algorithm is used.  See the
         \htmlref{``Choice of Algorithm''}{choice:regrid} topic below.
         \texttt{[}current value\texttt{{]}}
      }
      \sstsubsection{
         SCALE( ) = \_DOUBLE (Read)
      }{
         Axis scaling factors which are used to modify the supplied Mapping.
         If the number of supplied values is fewer than the number of output
         axes associated with the Mapping, the final supplied value is
         duplicated for the missing axes.  In effect, transformed input
         co-ordinate axis values would be multiplied by these factors to
         obtain the corresponding output pixel co-ordinates.  If a null (\texttt{{!}})
         value is supplied for SCALE, then default values are used which
         depends on the value of Parameter MAPPING.  If a null value is
         supplied for MAPPING then the default scaling factors are chosen
         so that pixels retain their original size (very roughly) after
         transformation.  If as non-null value is supplied for MAPPING then
         the default scaling factor used is 1.0 for each axis
         (\emph{i.e.} no scaling).  \texttt{[!]}
      }
      \sstsubsection{
         TITLE = LITERAL (Read)
      }{
         A \htmlref{Title}{apndf:title} for the output NDF structure.
         A null value (\texttt{{!}}) propagates the title from the input NDF
         to the output NDF.  \texttt{[!]}
      }
      \sstsubsection{
         TOL = \_DOUBLE (Read)
      }{
         The maximum tolerable geometrical distortion which may be
         introduced as a result of approximating non-linear Mappings
         by a set of piece-wise linear transforms.  The resampling
         algorithm approximates non-linear co-ordinate transformations
         in order to improve performance, and this parameter controls
         how inaccurate the resulting approximation is allowed to be,
         as a displacement in pixels of the input NDF.  A value of
         zero will ensure that no such approximation is done, at the
         expense of increasing execution time.
         \texttt{[0.2]}
      }
      \sstsubsection{
         UBOUND( ) = \_INTEGER (Read)
      }{
         The upper pixel-index bounds of the output NDF.  The number of
         values must be equal to the number of dimensions of the output
         NDF.  If a null value is supplied, default bounds will be used
         which are just high enough to fit in all the transformed pixels
         of the input NDF.  \texttt{[!]}
      }
      \sstsubsection{
         WLIM = \_REAL (Read)
      }{
         This parameter is only used if REBIN is set \texttt{TRUE}.  It specifies the
         minimum number of good pixels which must contribute to an output pixel
         for the output pixel to be valid.  Note, fractional values are
         allowed.  A null (\texttt{{!}}) value causes a very small positive value to
         be used resulting in output pixels being set bad only if they
         receive no significant contribution from any input pixel.  \texttt{[!]}
      }
   }
   \sstexamples{
      \sstexamplesubsection{
         regrid sg28948 sg28948r mapping=rotate.ast
      }{
         Here sg28948 is resampled into a new co-ordinate system using
         the AST Mapping stored in a text file called \texttt{rotate.ast} (which
         may have been created using WCSADD for instance).
      }
      \sstexamplesubsection{
         regrid flat distorted mapping=!
      }{
         This transforms the NDF called flat into its current
         \htmlref{co-ordinate Frame}{se:domains}, writing the result to
         an NDF called distorted.  It uses nearest-neighbour resampling.
         If the units of the PIXEL and current co-ordinate Frames of
         flat are of similar size, then the pixel co-ordinates of
         distorted will be the same as the current co-ordinates of
         flat, but if there is a large scale discrepancy a scaling
         factor will be applied to give the output NDF a similar size
         to the input one.  The output NDF will be just large enough
         to hold the transformed copies of all the pixels from NDF flat.
      }
      \sstexamplesubsection{
         regrid flat distorted mapping=! scale=1 method=sinccos params=[0,3]
      }{
         As the previous example, but the additional scaling factor will
         not be applied even in the case of large size discrepancy,
         and a sinc$*$cos one-dimensional resampling kernel is used which
         rolls off at a distance of 3 pixels from the central one.
      }
      \sstexamplesubsection{
         regrid flat distorted mapping=! scale=0.2 method=blockave params=2
      }{
         In this case, an additional shrinking factor of 0.2 is being
         applied to the output NDF (\emph{i.e.} performed following the
         Mapping from pixel to current co-ordinates), and the resampling
         is being done using a block averaging scheme in which a
         cube extending two pixels either side of the central pixel
         is averaged over to produce the output value.  If the
         PIXEL-domain and current Frame pixels have (about) the same
         size, this will result in every pixel from the input NDF
         adding a contribution to one pixel of the output NDF.
      }
      \sstexamplesubsection{
         regrid a119 a119s mapping=! lbound=[1,-20] ubound=[256,172]
      }{
         This transforms the NDF called a119 into an NDF called a119s.
         It uses nearest-neighbour resampling.  The shape of a119s
         is forced to be (1:256,$-$20:172) regardless of the location
         of the transformed pixels of a119.
      }
   }
   \sstnotes{
      \sstitemlist{

         \sstitem
         If the input NDF contains a VARIANCE component, a VARIANCE
         component will be written to the output NDF.  It will be
         calculated on the assumption that errors on the input data
         values are statistically independent and that their variance
         estimates may simply be summed (with appropriate weighting
         factors) when several input pixels contribute to an output data
         value.  If this assumption is not valid, then the output error
         estimates may be biased.  In addition, note that the statistical
         errors on neighbouring output data values (as well as the
         estimates of those errors) may often be correlated, even if the
         above assumption about the input data is correct, because of
         the sub-pixel interpolation schemes employed.

         \sstitem
         This task is based on the \xref{AST\_RESAMPLEx}{sun210}{AST_RESAMPLE\$<X>\$}
         and \xref{AST\_REBINSEQx}{sun210}{AST_REBINSEQ\$<X>\$} routines described in
         \xref{SUN/210}{sun210}{}.
      }
   }
   \label{choice:regrid}
   \sstdiytopic{
      Choice of Algorithm
   }{
      The algorithm used to produce the output image is determined by
      the REBIN parameter, and is based either on resampling the output
      image or rebinning the corresponding input image.

      The resampling algorithm steps through every pixel in the output
      image, sampling the input image at the corresponding position and
      storing the sampled input value in the output pixel.  The method
      used for sampling the input image is determined by the METHOD
      parameter.  The rebinning algorithm steps through every pixel in
      the input image, dividing the input pixel value between a group
      of neighbouring output pixels, incrementing these output pixel
      values by their allocated share of the input pixel value, and
      finally normalising each output value by the total number of
      contributing input values.  The way in which the input sample is
      divided between the output pixels is determined by the METHOD
      parameter.

      Both algorithms produce an output in which the each pixel value is
      the weighted mean of the nearby input values, and so do not alter
      the mean pixel values associated with a source, even if the pixel
      size changes.  Thus the total data sum in a source will change if
      the input and output pixel sizes differ.  However, if the CONSERVE
      parameter is set \texttt{TRUE}, the output values are scaled by the ratio
      of the output to input pixel size, so that the total data sum in a
      source is preserved.

      A difference between resampling and rebinning is that resampling
      guarantees to fill the output image with good pixel values
      (assuming the input image is filled with good input pixel values),
      whereas holes can be left by the rebinning algorithm if the output
      image has smaller pixels than the input image.  Such holes occur
      at output pixels that receive no contributions from any input
      pixels, and will be filled with the value zero in the output
      image.  If this problem occurs, the solution is probably to change
      the width of the pixel spreading function by assigning a larger
      value to PARAMS(1) and/or PARAMS(2) (depending on the specific
      METHOD value being used).

      Both algorithms have the capability to introduce artefacts into
      the output image.  These have various causes described below.

      \sstitemlist{
         \sstitem
         Particularly sharp features in the input can cause rings around
         the corresponding features in the output image.  This can be
         minimised by suitable settings for the METHOD and PARAMS
         parameters.  In general such rings can be minimised by using a
         wider interpolation kernel (if resampling) or spreading function
         (if rebinning), at the cost of degraded resolution.

         \sstitem
         The approximation of the Mapping using a piece-wise linear
         transformation (controlled by Parameter TOL) can produce artefacts
         at the joints between the panels of the approximation.  These can
         occur when using the rebinning algorithm, or when using the
         resampling algorithm with CONSERVE set to \texttt{TRUE}.  They are caused
         by the discontinuities  between the adjacent panels of the
         approximation, and can be minimised by reducing the value assigned
         to the TOL parameter.
      }
   }
   \sstdiytopic{
      Related Applications
   }{
     KAPPA: \htmlref{FLIP}{FLIP},
     \htmlref{ROTATE}{ROTATE},
     \htmlref{SLIDE}{SLIDE},
     \htmlref{WCSADD}{WCSADD},
     \htmlref{WCSALIGN}{WCSALIGN};
     \xref{CCDPACK}{sun139}{}:
     \xref{TRANLIST}{sun139}{TRANLIST},
     \xref{TRANNDF}{sun139}{TRANNDF},
     \xref{WCSEDIT}{sun139}{WCSEDIT}.
   }
   \sstimplementationstatus{
      \sstitemlist{

         \sstitem
         The \htmlref{LABEL}{apndf:label}, \htmlref{UNITS}{apndf:units}, and
         \htmlref{HISTORY}{apndf:history} components, and all extensions are
         propagated.  \htmlref{TITLE}{apndf:title} is controlled by the TITLE
         parameter.  DATA, \htmlref{VARIANCE}{apndf:variance}, and
         \htmlref{WCS}{apndf:wcs} are propagated after appropriate modification.
         The \htmlref{QUALITY}{apndf:quality} component is also propagated if
         Nearest-Neighbour interpolation is being used (note, REBIN must
         be \texttt{FALSE}). The
         \htmlref{AXIS}{apndf:axis} component is not propagated.

         \sstitem
         Processing of \htmlref{bad pixels}{se:masking} and automatic
         \htmlref{quality masking}{se:qualitymask} are supported.

         \sstitem
         All \htmlref{non-complex numeric data types}{ap:HDStypes} can be handled.
         If REBIN is \texttt{TRUE}, the data type will be converted to one of
         \_INTEGER, \_DOUBLE or \_REAL for processing.

         \sstitem
         There can be an arbitrary number of NDF dimensions.
      }
   }
}

\sstroutine{
   REMQUAL
}{
   Removes specified quality definitions from an NDF
}{
   \sstdescription{
      This routine removes selected \htmlref{quality name}{se:qnames}~ definitions
      from an \NDFref{NDF}~ (see Task SETQUAL) and optionally clears the corresponding bit
      in the QUALITY array of the supplied NDF.  All quality names information may
      be removed by specifying a quality name of \texttt{"ANY"}.

      An error will be reported if an attempt is made to remove a quality
      name that has been flagged as ``read-only'' (\emph{e.g.} using the
      READONLY parameter of the SETQUAL application).
   }
   \sstusage{
      remqual ndf qnames
   }
   \sstparameters{
       \sstsubsection{
          CLEAR = \_LOGICAL (Read)
       }{
         If \texttt{TRUE}, the bits in the NDF's QUALITY array that correspond to
         the removed quality names will be cleared. If \texttt{FALSE}, no change
         will be made to the QUALITY array. \texttt{[FALSE]}
      }
      \sstsubsection{
         NDF = NDF (Update)
      }{
         The NDF to be modified.
      }
      \sstsubsection{
         QNAMES = LITERAL (Read)
      }{
         A group of up to 10 quality names to be removed from the input
         NDF.  The group may be supplied as a comma-separated list, or
         within a text file (in which case the name of the text file should
         be given, preceded by a \texttt{"$\wedge$"} character.) If more than 10 names are
         supplied, only the first 10 are used.  If any of the supplied
         quality names are not defined in the NDF, then warning
         messages are given but the application continues to remove any
         other specified quality names.  If the string ANY is specified,
         then all defined quality names are removed.  If no defined
         quality names remain, the structure used to store quality name
         information is deleted.  This feature can be used to get rid of
         corrupted quality name information.
      }
   }
   \sstdiytopic{
      Related Applications
   }{
KAPPA: \htmlref{QUALTOBAD}{QUALTOBAD},
\htmlref{SHOWQUAL}{SHOWQUAL},
\htmlref{SETQUAL}{SETQUAL}.
   }

   \sstexamples{
      \sstexamplesubsection{
         remqual "m51$*$" any
      }{
         This example will remove all defined quality names from all
         NDFs with names starting with the string \texttt{"m51"}.
      }
   }
}
\sstroutine{
   RESHAPE
}{
   Reshapes an NDF, treating its arrays as vectors
}{
   \sstdescription{
      This application reshapes an \NDFref{NDF} to create another NDF by copying
      array values.  The array components in the input NDF are treated
      as vectors.  Each output array is filled in order with values from
      the input vector, until it is full or the input vector is
      exhausted.  Output data and variance pixels not filled are set to
      the bad value; unfilled quality pixels are set to zero.  The
      filling is in Fortran order, namely the first dimension, followed
      by the second dimension,\ldots to the highest dimension.

      It is possible to form a vectorized NDF using Parameter VECTORIZE
      without having to specify the shape.
   }
   \sstusage{
      reshape in out shape=?
   }
   \sstparameters{
      \sstsubsection{
         IN = NDF (Read)
      }{
         The input NDF to be reshaped.
      }
      \sstsubsection{
         OUT = NDF (Read)
      }{
         The NDF after reshaping.
      }
      \sstsubsection{
         SHAPE( ) = \_INTEGER (Read)
      }{
         The shape of the output NDF.  For example, \texttt{[50,30,20] }would
         create 50 columns by 30 lines by 20 bands.  It is only
         accessed when VECTORIZE=\texttt{FALSE}.
      }
      \sstsubsection{
         TITLE = LITERAL (Read)
      }{
         \htmlref{Title}{apndf:title} for the output NDF structure.  A null value (\texttt{{!}})
         propagates the title from the base NDF to the output NDF.  \texttt{[!]}
      }
      \sstsubsection{
         VECTORIZE = \_LOGICAL (Read)
      }{
         If \texttt{TRUE}, the output NDF is the vectorized form of the input
         NDF.  If \texttt{FALSE}, Parameter SHAPE is used to specify the new
         shape.  \texttt{[FALSE]}
      }
   }
   \sstexamples{
      \sstexamplesubsection{
         reshape shear normal shape=[511,512]
      }{
         This reshapes the NDF called shear to form NDF normal, whose
         shape is 511$\times$512 pixels.  One example is where the original
         image has 512$\times$512 pixels but one pixel was omitted from each
         line during some data capture, causing the image to be sheared
         between lines.
      }
      \sstexamplesubsection{
         reshape cube cube1d vectorize
      }{
         This vectorizes the NDF called cube to form NDF cube1d.  This
         could be used for a task that only permits one-dimensional
         data.
      }
   }
   \sstdiytopic{
      Related Applications
   }{
KAPPA: \htmlref{CHAIN}{CHAIN},
\htmlref{PASTE}{PASTE},
\htmlref{RESHAPE}{RESHAPE}.
   }
   \sstimplementationstatus{
      \sstitemlist{

         \sstitem
         This routine correctly processes the DATA, \htmlref{QUALITY}{apndf:quality},
         \htmlref{VARIANCE}{apndf:variance}, \htmlref{LABEL}{apndf:label}, \htmlref{TITLE}{apndf:title}, \htmlref{UNITS}{apndf:units}, and \htmlref{HISTORY}{apndf:history}, components of an NDF
         data structure and propagates all \htmlref{extensions}{apndf:extensions}.  \htmlref{WCS}{apndf:wcs}, and \htmlref{AXIS}{apndf:axis}~
         information is lost.

         \sstitem
         All \htmlref{non-complex numeric data types}{ap:HDStypes} can be handled.

         \sstitem
         Any number of NDF dimensions is supported.
      }
   }
}

\sstroutine{
   RIFT
}{
   Adds a scalar to a section of an NDF data structure to correct
   rift-valley defects
}{
   \sstdescription{
      The routine adds a scalar (\emph{i.e.} constant) value to each pixel of
      an \NDFref{NDF's}  data array within a sub-section to produce a new NDF
      data structure.
   }
   \sstusage{
      rift in scalar out section
   }
   \sstparameters{
      \sstsubsection{
         IN = NDF (Read)
      }{
         Input NDF data structure, to which the value is to be added.
      }
      \sstsubsection{
         OUT = NDF (Write)
      }{
         Output NDF data structure.
      }
      \sstsubsection{
         SCALAR = \_DOUBLE (Read)
      }{
         The value to be added to the NDF's data array within the
         section.
      }
      \sstsubsection{
         SECTION = LITERAL (Read)
      }{
         The pixels to which a scalar is to be added.  This is defined
         as an NDF section, so that ranges can be defined along any
         axis, and be given as pixel indices or axis (data)
         co-ordinates.  So for example \texttt{"3,4,5"} would select the pixel
         at (3,4,5); \texttt{"3:5,"} would select all elements in columns 3 to
         5; \texttt{",4"} selects line 4.
         See \slhyperref{NDF Sections}{Section~}{}{se:ndfsect} for details.
      }
      \sstsubsection{
         TITLE = LITERAL (Read)
      }{
         The title for the output NDF.  A null value will cause
         the title of the NDF supplied for Parameter IN to be used
         instead.  \texttt{[!]}
      }
   }
   \sstexamples{
      \sstexamplesubsection{
         rift aa 10.7 bb "100:105" 20
      }{
         This adds 10 in the columns 100 to 105 in the data array of
         the NDF called aa and stores the result in the NDF called bb.
         In other respects bb is a copy of aa.
      }
      \sstexamplesubsection{
         rift cubein -100 cubeout ",,4"
      }{
         This adds $-$100 to all values in the fourth plane of the data
         array of the NDF called cubein and stores the result in the
         NDF called cubeout.  In other respects cubeout is a copy of
         cubeout.
      }
      \sstexamplesubsection{
         rift in=aa scalar=2 out=bb section="-10:5,200$\sim$9"
      }{
         This adds 2 to the rectangular section between columns $-$10 to
         5 and lines 196 to 204 of the data array of the NDF called aa
         and stores the result in the NDF called bb.  In other respects
         bb is a copy of aa.
      }
   }
   \sstnotes{
      For similar operations performed on a subset, use the appropriate
      application to process the relevant section and then run PASTE to
      paste the result back into the full array.
   }
   \sstdiytopic{
      Related Applications
   }{
KAPPA: \htmlref{CADD}{CADD},
\htmlref{CHPIX}{CHPIX},
\htmlref{GLITCH}{GLITCH},
\htmlref{PASTE}{PASTE},
\htmlref{SEGMENT}{SEGMENT},
\htmlref{ZAPLIN}{ZAPLIN};
\xref{FIGARO}{sun86}{}: \xref{CSET}{sun86}{CSET},
\xref{ICSET}{sun86}{ICSET},
\xref{NCSET}{sun86}{NCSET},
\xref{TIPPEX}{sun86}{TIPPEX}.
   }
   \sstimplementationstatus{
      \sstitemlist{

         \sstitem
         This routine correctly processes the \htmlref{AXIS}{apndf:axis}, DATA, \htmlref{QUALITY}{apndf:quality},
         \htmlref{VARIANCE}{apndf:variance}, \htmlref{LABEL}{apndf:label}, \htmlref{TITLE}{apndf:title}, \htmlref{UNITS}{apndf:units}, \htmlref{WCS}{apndf:wcs}, and \htmlref{HISTORY}{apndf:history}~ components of an NDF
         data structure and propagates all \htmlref{extensions}{apndf:extensions}.

         \sstitem
         Processing of \htmlref{bad pixels}{se:masking} and automatic \htmlref{quality masking}{se:qualitymask} are
         supported.

         \sstitem
         The \htmlref{bad-pixel flag}{setbad:badpixelflag}~ is set to \texttt{TRUE} if undefined values are
         created during the arithmetic.

         \sstitem
         All \htmlref{non-complex numeric data types}{ap:HDStypes} can be handled.
      }
   }
}
\sstroutine{
   ROTATE
}{
   Rotates a two-dimensional NDF about its centre through any angle
}{
   \sstdescription{
      This routine rotates an array stored in an \NDFref{NDF} data
      structure by an arbitrary angle.  The rotation angle can be chosen
      automatically to make north vertical in the output NDF (see Parameter
      ANGLE).  The origin of the rotation is around the point (0,~0) in pixel
      co-ordinates.  The output array dimensions just accommodate the rotated
      array.  Output pixels can  be generated from the input array by one of
      two methods: nearest-neighbour substitution or by bi-linear interpolation.
      The latter is slower, but gives better results.  Output pixels not
      corresponding to input pixels take the \htmlref{bad value}{se:masking}.

      The NDF may have two or three dimensions.  If it has three
      dimensions, then the rotation is applied in turn to each plane in
      the cube and the result written to the corresponding plane in the
      output cube.  The orientation of the rotation plane can be
      specified using the AXES parameter.

   }
   \sstusage{
      rotate in out angle
   }
   \sstresparameters{
      \sstsubsection{
         ANGLEUSED( ) = \_REAL (Write)
      }{
         An output parameter holding the rotation angle actually used, in
         degrees. This is useful if a null value is supplied for parameter
         ANGLE.
      }
   }
   \sstparameters{
      \sstsubsection{
         ANGLE = \_REAL (Read)
      }{
         Number of clockwise degrees by which the data array is to be
         rotated.  It must lie between -360 and 360 degrees.  The suggested
         default is the current value.  If a null (\texttt{{!}}) value is supplied,
         then the rotation angle is chosen to make north vertical at the
         centre of the image.  If the
         current \htmlref{co-ordinate Frame}{se:domains}~  in the input NDF is not a celestial co-ordinate
         frame, then the rotation angle is chosen to make the second axis
         of the \htmlref{current Frame}{se:curframe}~ vertical.
      }
      \sstsubsection{
         AXES(2) = \_INTEGER (Read)
      }{
         This parameter is only accessed if the NDF has exactly three
         significant pixel axes.  It should be set to the indices of the
         NDF pixel axes which span the plane in which rotation is to
         be applied.  All pixel planes parallel to the specified plane
         will be rotated independently of each other.  The dynamic
         default comprises the indices of the first two significant
         axes in the NDF.  Note that excluding the first significant
         axis may be very inefficient for large cubes; a prior
         reconfiguration with application PERMAXES that is compatible
         with the dynamic default for AXES, will often prove beneficial.
         \texttt{[]}
      }
      \sstsubsection{
         IN = NDF (Read)
      }{
         NDF structure containing the two- or three-dimensional array to
         be rotated.
      }
      \sstsubsection{
         NNMETH = \_LOGICAL (Read)
      }{
         If \texttt{TRUE}, the nearest-neighbour method will be used to evaluate
         the output data-array pixels.  This is only accessed when the
         rotation is not a multiple of 90 degrees.  \texttt{[FALSE]}
      }
      \sstsubsection{
         OUT = NDF (Write)
      }{
         Output NDF to contain the rotated arrays.
      }
      \sstsubsection{
         QUALITY = \_LOGICAL (Read)
      }{
         This parameter is only accessed when NNMETH is \texttt{FALSE} and ANGLE
         is not a multiple of 90 degrees.  Strictly, the quality values
         are undefined by the bi-linear interpolation and hence cannot
         be propagated.  However, QUALITY=\texttt{TRUE} offers an approximation
         to the quality array by propagating the nearest-neighbour quality
         to the output NDF.  \texttt{[FALSE]}
      }
      \sstsubsection{
         TITLE = LITERAL (Read)
      }{
         A \htmlref{title}{apndf:title} for the output NDF.  A null value
         will cause the title of the NDF supplied for Parameter IN to be
         used instead.  \texttt{[!]}
      }
      \sstsubsection{
         USEAXIS = \htmlref{GROUP}{se:groups} (Read)
      }{
         USEAXIS is only accessed if the current \htmlref{co-ordinate Frame}{se:domains}~ of
         the NDF has more than two axes.  A group of two strings should be
         supplied specifying the two axes which are to be used when
         determining the rotation angle needed to make north vertical.  Each
         axis can be specified using one of the following options.

         \ssthitemlist{

            \sstitem
            Its integer index within the current Frame of the
            input  NDF (in the range 1 to the number of axes in the
            current Frame).

            \sstitem
            Its \htmlattref{Symbol}{Symbol(axis)}~ string such as
            \texttt{"RA"} or \texttt{"VRAD"}.

            \sstitem
            A generic option where \texttt{"SPEC"} requests the spectral axis,
            \texttt{"TIME"} selects the time axis, \texttt{"SKYLON"} and
            \texttt{"SKYLAT"} picks the sky longitude and latitude axes
            respectively.  Only those axis domains present are
            available as options.
         }

         A list of acceptable values is displayed if an illegal value is
         supplied.  If a null (\texttt{{!}}) value is supplied, the axes
         with the same indices as the two used pixel axes within the NDF
         are used.  \texttt{[!]}
      }
      \sstsubsection{
         VARIANCE = \_LOGICAL (Read)
      }{
         A \texttt{TRUE} value causes variance values to be used as weights for
         the pixel values in bi-linear interpolation, and also causes
         output variances to be created.  This parameter is ignored if
         ANGLE is a multiple of 90 degrees or NNMETH=\texttt{TRUE}; in these cases
         the variance array is merely propagated.  If a null (\texttt{{!}}) value is
         supplied, the value used is \texttt{TRUE} if the input NDF has a VARIANCE
         component, and \texttt{FALSE} otherwise.  Note that following this operation
         the errors are no longer independent.  \texttt{[!]}
      }
   }
   \sstexamples{
      \sstexamplesubsection{
         rotate ns ew 90
      }{
         This rotates the array components in the NDF called ns by 90
         degrees clockwise around pixel co-ordinates [0,~0] and stores the
         result in the NDF called ew.  The former \textit{x} axis becomes the new
         \textit{y} axis, and the former \textit{y} axis becomes the new \textit{x} axis.
         The former \textit{y}-axis arrays are also reversed in the process.
      }
      \sstexamplesubsection{
         rotate m31 m31r angle=!
      }{
         This rotates the NDF called m31 so that north is vertical and
         stores the results in an NDF called m31r.  This assumes that the
         \htmlref{current WCS Frame}{se:curframe}~ in the input NDF is a
         celestial co-ordinate Frame.
      }
      \sstexamplesubsection{
         rotate angle=180 out=sn in=ns
      }{
         This rotates the array components in the NDF called ns by 180
         degrees clockwise around the pixel co-ordinates [0,~0], and stores
         the result in the NDF called sn.  The axis arrays are flipped in
         the output NDF.
      }
      \sstexamplesubsection{
         rotate f1 f1r 37.2 novariance
      }{
         This rotates the array components in the NDF called f1 by 37.2
         degrees clockwise around the pixel co-ordinates [0,~0], and stores
         the result in the NDF called f1r.  The original axis information
         is lost.  Bi-linear interpolation is used without variance
         information.  No quality or variance information is propagated.
      }
      \sstexamplesubsection{
         rotate f1 f1r 106 nnmeth title="Reoriented features map"
      }{
         This rotates the array components in the NDF called f1 by 106
         degrees clockwise around the pixel co-ordinates [0,~0], and stores
         the result in the NDF called f1r.  The original axis information
         is lost.  The resultant array components, all of which are
         propagated, are calculated by the nearest-neighbour method.  The
         title of the output NDF is \texttt{"Reoriented features map"}.
      }
      \sstexamplesubsection{
         rotate velmap rotvelmap 70
      }{
         This rotates the array components in the three-dimensional NDF
         called velmap by 70 degrees clockwise around the pixel
         co-ordinates [0,0], and stores the result in the NDF called
         rotvelmap.  The rotation is applied to the first two pixel axes
         repeated for all the planes in the cube's third pixel axis.
      }
      \sstexamplesubsection{
         rotate velmap rotvelmap 70 axes=[1,3]
      }{
         This as the previous example except that the rotation is
         applied in the plane given by the first and third pixel axes.
      }
   }
   \sstnotes{
      \sstitemlist{

         \sstitem
         Bad pixels are ignored in the bi-linear interpolation.  If all four pixels are bad, the result is bad. }
   }
   \sstdiytopic{
      Related Applications
   }{
KAPPA: \htmlref{FLIP}{FLIP},
\htmlref{REGRID}{REGRID};
\xref{FIGARO}{sun86}{}: \xref{IREVX}{sun86}{IREVX},
\xref{IREVY}{sun86}{IREVY},
\xref{IROT90}{sun86}{IROT90}.
   }
   \sstimplementationstatus{
      The propagation rules depend on Parameters ANGLE and NNMETH.

      \sstitemlist{

         \sstitem
         For rotations that are multiples of 90-degrees, \htmlref{VARIANCE}{apndf:variance},
         \htmlref{QUALITY}{apndf:quality}, \htmlref{AXIS}{apndf:axis}, \htmlref{HISTORY}{apndf:history}, \htmlref{LABEL}{apndf:label}~ \htmlref{WCS}{apndf:wcs}, and \htmlref{UNITS}{apndf:units}~ components of the input
         NDF are propagated to the output NDF.  The axis and WCS components
         are switched and flipped as appropriate.

         \sstitem
         For the nearest-neighbour method VARIANCE, QUALITY, HISTORY,
         LABEL, WCS and UNITS components of the input NDF are propagated to
         the output NDF.

         \sstitem
         For the linear-interpolation method HISTORY, LABEL, WCS and
         UNITS components of the input NDF are propagated to the output
         NDF.  In addition if Parameter VARIANCE is \texttt{TRUE}, variance
         information is derived from the input variance; and if Parameter
         QUALITY is \texttt{TRUE}, QUALITY is propagated using the nearest
         neighbour.

         \sstitem
         Processing of \htmlref{bad pixels}{se:masking} and automatic \htmlref{quality masking}{se:qualitymask} are
         supported.

         \sstitem
         All \htmlref{non-complex numeric types}{ap:HDStypes} are supported, though for linear
         interpolation the arithmetic is performed using single- or
         double-precision floating point as appropriate; and for 90 and
         270-degree rotations \_INTEGER is used for all integer types.
      }
   }
}
\sstroutine{
   SCATTER
}{
   Displays a scatter plot between data in two NDFs
}{
   \sstdescription{
      This application displays a two-dimensional plot in which the
      horizontal axis corresponds to the data value in the \NDFref{NDF} given
      by Parameter IN1, and the vertical axis corresponds to the data
      value in the NDF given by Parameter IN2.  Optionally, the variance,
      standard deviation or quality may be used instead of the data value for
      either axis (see Parameters COMP1 and COMP2).  A symbol is displayed
      at an appropriate position in the plot for each pixel which has a
      good value in both NDFs, and falls within the bounds specified by
      Parameters XLEFT, XRIGHT, YBOT, and YTOP.  The type of symbol may be
      specified using Parameter MARKER.

      The supplied arrays may be compressed prior to display (see Parameter
      COMPRESS).  This reduces the number of points in the scatter plot, and
      also reduces the noise in the data.

      The Pearson correlation coefficient of the displayed scatter plot is
      also calculated and displayed, and written to output Parameter CORR.

      A linear fit to the data can be calculated and displayed (see
      Parameter FIT).
   }
   \sstusage{
      scatter in1 in2 [comp1] [comp2] [device]
   }
   \sstparameters{
      \sstsubsection{
         AXES = \_LOGICAL (Read)
      }{
         \texttt{TRUE} if labelled and annotated axes are to be drawn around the
         plot.  The width of the margins left for the annotation may be
         controlled using Parameter MARGIN.  The appearance of the axes
         (colours, founts, \emph{etc.}) can be controlled using the Parameter
         STYLE.  The dynamic default is \texttt{TRUE} if CLEAR is \texttt{TRUE},
         and \texttt{FALSE} otherwise. \texttt{[]}
      }
      \sstsubsection{
         CLEAR = \_LOGICAL (Read)
      }{
         If \texttt{TRUE} the current picture is cleared before the plot is
         drawn.  If CLEAR is \texttt{FALSE} not only is the existing plot retained,
         but also an attempt is made to align the new picture with the
         existing picture.  Thus you can generate a composite plot within
         a single set of axes, say using different colours or modes to
         distinguish data from different datasets.  \texttt{[TRUE]}
      }
      \sstsubsection{
         COMP1 = \htmlref{LITERAL}{se:parmenu} (Read)
      }{
         The NDF array component to be displayed on the horizontal axis.
         It may be \texttt{"Data"}, \texttt{"Quality"}, \texttt{"Variance"}, or \texttt{"Error"} (where \texttt{"Error"}
         is an alternative to \texttt{"Variance"} and causes the square root of the
         variance values to be displayed).  If \texttt{"Quality"} is specified,
         then the quality values are treated as numerical values (in
         the range 0 to 255).  \texttt{["Data"]}
      }
      \sstsubsection{
         COMP2 = LITERAL (Read)
      }{
         The NDF array component to be displayed on the vertical axis.
         It may be \texttt{"Data"}, \texttt{"Quality"}, \texttt{"Variance"}, or \texttt{"Error"} (where \texttt{"Error"}
         is an alternative to \texttt{"Variance"} and causes the square root of the
         variance values to be displayed).  If \texttt{"Quality"} is specified,
         then the quality values are treated as numerical values (in
         the range 0 to 255).  \texttt{["Data"]}
      }
      \sstsubsection{
         COMPRESS() = \_INTEGER (Read)
      }{
         The compression factors to be used when compressing the supplied
         arrays prior to display.  If any of the supplied values are greater
         than 1, then the supplied arrays are compressed prior to display
         by replacing each box of input pixels by a single pixel equal to
         the mean of the pixels in the box.  The size of each box in pixels
         is given by the compression factors.  No compression occurs if all
         values supplied for this parameter are 1.  If the number of values
         supplied is smaller than the number of axes, the final value
         supplied is duplicated for the remaining axes.  \texttt{[1]}
      }
      \sstsubsection{
         DEVICE = \htmlref{DEVICE}{se:selgradev} (Read)
      }{
         The graphics workstation on which to produce the plot.  If a
         null value (\texttt{{!}}) is supplied no plot will be made.
         \texttt{[}\htmlref{current graphics device}{se:devglobal}\texttt{{]}}
      }
      \sstsubsection{
         FIT = \_LOGICAL (Read)
      }{
         If \texttt{TRUE}, then a linear fit to the scatter points is added to the
         plot. The slope and offset of this fit is displayed on the screen
         and written to output Parameters SLOPE, OFFSET, and RMS.  A symmetric
         linear-fit algorithm is used, which caters for the presence of
         noise in both $X$ and $Y$ values. Outliers are identified and
         ignored. Note, the fit is based on just those points that are
         visible in the scatter plot. Points outside the bounds of the
         plot are ignored. Points that are inside the plot are also ignored
         if their reflection through the best-fit line are outside the plot.
         This avoids biasing the fit if the plot bounds omit more points on
         one side of the line than the other. \texttt{[}current value\texttt{{]}}
      }
      \sstsubsection{
         IN1 = NDF (Read)
      }{
         The NDF to be displayed on the horizontal axis.
      }
      \sstsubsection{
         IN2 = NDF (Read)
      }{
         The NDF to be displayed on the vertical axis.
      }
      \sstsubsection{
         MARGIN( 4 ) = \_REAL (Read)
      }{
         The widths of the margins to leave for axis annotation, given
         as fractions of the corresponding dimension of the current picture.
         Four values may be given, in the order bottom, right, top, left.
         If fewer than four values are given, extra values are used equal to
         the first supplied value.  If these margins are too narrow any axis
         annotation may be clipped.  If a null (\texttt{{!}}) value is supplied, the
         value used is \texttt{0.15} (for all edges) if annotated axes are produced,
         and zero otherwise.  \texttt{[}current value\texttt{{]}}
      }
      \sstsubsection{
         MARKER = \_INTEGER (Read)
      }{
         Specifies the symbol with which each position should be marked in
         the plot.  It should be given as an integer \PGPLOT\  marker type.  For
         instance, \texttt{0} gives a box, \texttt{1} gives a dot, \texttt{2} gives a
         cross, \texttt{3} gives an asterisk, \texttt{7} gives a triangle.  The value
         must be larger than or equal to $-$31.  \texttt{[}current value\texttt{{]}}
      }
      \sstsubsection{
         PERC1( 2 ) = \_REAL (Read)
      }{
         The percentiles that define the default values for XLEFT and
         XRIGHT.  For example, \texttt{[5,95]} would result in the lowest and
         highest 5\% of the data value in IN1 being excluded from the plot
         if the default values are accepted for XLEFT and XRIGHT.
         \texttt{[}current value\texttt{{]}}
      }
      \sstsubsection{
         PERC2( 2 ) = \_REAL (Read)
      }{
         The percentiles that define the default values for YBOT and
         YTOP.  For example, \texttt{[5,95]} would result in the lowest and
         highest 5\% of the data value in IN2 being excluded from the plot
         if the default values are accepted for YBOT and YTOP.
         \texttt{[}current value\texttt{{]}}
      }
      \sstsubsection{
         STYLE = \htmlref{GROUP}{se:groups} (Read)
      }{
         A group of attribute settings describing the plotting style to use
         when drawing the annotated axes, and markers.

         A comma-separated list of strings should be given in which each
         string is either an attribute setting, or the name of a text
         file preceded by an up-arrow character \texttt{"$\wedge$"}.  Such text files
         should contain further comma-separated lists which will be
         read and interpreted in the same manner.  Attribute settings
         are applied in the order in which they occur within the list,
         with later settings overriding any earlier settings given for
         the same attribute.

         Each individual attribute setting should be of the form:

            $<$name$>$=$<$value$>$

         where $<$name$>$ is the name of a plotting attribute, and $<$value$>$
         is the value to assign to the attribute.  Default values will be
         used for any unspecified attributes.  All attributes will be
         defaulted if a null value (\texttt{{!}})---the initial default---is supplied.
         To apply changes of style to only the current invocation, begin these
         attributes with a plus sign.  A mixture of persistent and temporary
         style changes is achieved by listing all the persistent attributes
         followed by a plus sign then the list of temporary attributes.

         See \slhyperref{Plotting Attributes}{Section~}{}{ap:plotting_attr}
         for a description of the available attributes.  Any unrecognised
         attributes are ignored (no error is reported).

         The appearance of markers is controlled by
         \htmlattref{Colour(Markers)}{Colour(element)},
         \htmlattref{Width(Markers)}{Width(element)}, \emph{etc.} (the synonym
         \att{Symbols} may be used in place
         of \att{Markers}).  \texttt{[}current value\texttt{{]}}
      }
      \sstsubsection{
         XLEFT = \_DOUBLE (Read)
      }{
         The axis value to place at the left hand end of the horizontal
         axis.  If a null (\texttt{{!}}) value is suplied, the value used is determined
         by Parameter PERC1.  The value supplied may be greater than or less
         than the value supplied for XRIGHT.  \texttt{[!]}
      }
      \sstsubsection{
         XRIGHT = \_DOUBLE (Read)
      }{
         The axis value to place at the right hand end of the horizontal
         axis.  If a null (\texttt{{!}}) value is suplied, the value used is determined
         by Parameter PERC1.  The value supplied may be greater than or less
         than the value supplied for XLEFT.  \texttt{[!]}
      }
      \sstsubsection{
         YBOT = \_DOUBLE (Read)
      }{
         The axis value to place at the bottom end of the vertical axis.
         If a null (\texttt{{!}}) value is suplied, the value used is determined
         by Parameter PERC2.  The value supplied may be greater than or less
         than the value supplied for YTOP.  \texttt{[!]}
      }
      \sstsubsection{
         YTOP = \_DOUBLE (Read)
      }{
         The axis value to place at the top end of the vertical axis.
         If a null (\texttt{{!}}) value is suplied, the value used is determined
         by Parameter PERC2.  The value supplied may be greater than or less
         than the value supplied for YBOT.  \texttt{[!]}
      }
   }
   \sstresparameters{
      \sstsubsection{
         CORR = \_DOUBLE (Write)
      }{
         The Pearson correlation coefficient of the visible points in the
         scatter plot (points outside the plot are ignored). A value of
         zero is stored if the correlation coefficient cannot be calculated.
      }
      \sstsubsection{
         NPIX = \_INTEGER (Write)
      }{
         The number of pixels used to form the correlation coefficient.
      }
      \sstsubsection{
         OFFSET = \_DOUBLE (Write)
      }{
         An output parameter giving the offset in the linear fit:
         IN2 = SLOPE $*$ IN1 $+$ OFFSET. Only used if Parameter FIT is \texttt{TRUE}.
      }
      \sstsubsection{
         RMS = \_DOUBLE (Write)
      }{
         An output parameter giving the RMS residual of the data (excluding
         outliers) about the linear fit. Only used if Parameter FIT is \texttt{TRUE}.
      }
      \sstsubsection{
         SLOPE = \_DOUBLE (Write)
      }{
         An output parameter giving the slope of the linear fit:
         IN2 = SLOPE $*$ IN1 $+$ OFFSET. Only used if Parameter FIT is \texttt{TRUE}.
      }
   }
   \sstexamples{
      \sstexamplesubsection{
         scatter cl123a cl123b
      }{
         This displays a scatter plot of the data value in NDF cl123b
         against the data value in NDF cl123a, on the current graphics
         device.
      }
      \sstexamplesubsection{
         scatter cl123a cl123a pscol\_l comp2=error compress=3
      }{
         This displays a scatter plot of the error in NDF cl123a
         against the data value in the same NDF.  The graphics device used
         is pscol\_l.  The data are compressed by a factor of 3 on each
         axis before forming the plot.
      }
   }
   \sstnotes{
      \sstitemlist{

         \sstitem
         Any pixels that are bad (after any compression) in either array
         are excluded from the plot, and from the calculation of the
         default axis limits

         \sstitem
         The application stores two pictures in the
         \htmlref{graphics database}{se:agitate} in the following order: a
         FRAME picture containing the annotated axes
         and data plot, and a DATA picture containing just the data plot.
         Note, the FRAME picture is only created if annotated axes have been
         drawn, or if non-zero margins were specified using Parameter
         MARGIN.  The world co-ordinates in the DATA picture will correspond
         to data value in the two NDFs.
      }
   }
   \sstdiytopic{
      Related Applications
   }{
KAPPA: \htmlref{NORMALIZE}{NORMALIZE}.
   }
   \sstimplementationstatus{
      \sstitemlist{

         \sstitem
         Processing of \htmlref{bad pixels}{se:masking} and automatic \htmlref{quality masking}{se:qualitymask} are
         supported.

         \sstitem
         Only \_REAL data can be processed directly.  Other \htmlref{non-complex numeric data types}{ap:HDStypes} will undergo a type conversion before
         processing occurs.
      }
   }
}
\sstroutine{
   SEGMENT
}{
   Copies polygonal segments from one NDF into another
}{
   \sstdescription{
      This routine copies one or more polygonal segments from the first
      input \NDFref{NDF} (Parameter IN1), and pastes them into the second input NDF
      (Parameter IN2) at the same \htmlref{pixel co-ordinates}{se:pixgrd}.  The
      resulting mosaic is stored in the output NDF (see OUT).  Either input NDF may be
      supplied as null (\texttt{{!}}) in which case the corresponding areas of the
      output NDF are filled with \htmlref{bad values}{se:masking}.  For
      instance, supplying a null value for IN2 allows segments to be cut from
      IN1 and pasted on to a background of bad values.  Supplying a null value
      for IN1 allows `holes' to be cut out of IN2 and filled with bad values.

      Each polygonal segment is specified by giving the positions of its
      vertices.  This may be done using a graphics cursor, by supplying
      a positions list or text file containing the positions, or by
      supplying the positions in response to a parameter prompt.  The
      choice is made by Parameter MODE.

      This application may also be used to cut and paste cylinders with
      polygonal cross-sections from NDFs with more than two dimensions.  See
      the \htmlref{``Notes''}{notes:segment} section below for further details.
   }
   \sstusage{
      segment in1 in2 out
        $\left\{ {\begin{tabular}{l}
                  coords=? \\
                  incat1-incat20=? \\
                  poly1-poly20=?
                  \end{tabular} }
        \right.$
        \newline\latexhtml{\hspace*{9.8em}}{~~~~~~~~~~~~~~~~~~}
        \makebox[0mm][c]{\small mode}
   }
   \sstparameters{
      \sstsubsection{
         COORDS = LITERAL (Read)
      }{
         The co-ordinates of a single vertex for the current polygon.  If
         Parameter MODE is set to \texttt{"Interface"}, this parameter is accessed
         repeatedly to obtain the co-ordinates of all vertices in the
         polygon.  A null value should be given when the final vertex has
         been specified.  Each position should be supplied within the current
         \htmlref{co-ordinate Frame}{se:domains}~  of the output NDF (see Parameter OUT).  Supplying
         a colon \texttt{":"} will display details of the required co-ordinate Frame.
         No more than two \xref{formatted axis values}{sun210}{AST_UNFORMAT}
         (separated by a comma or space)
         may be supplied.  If the co-ordinate Frame being used has more than
         two axes, then the two axes to use must be specified using Parameter
         USEAXIS.
      }
      \sstsubsection{
         DEVICE = \htmlref{DEVICE}{se:selgradev} (Read)
      }{
         The name of the graphics device on which an image is displayed.
         Only used if Parameter MODE is given the value \texttt{"Cursor"}.  Any
         graphics specified by Parameter PLOT will be produced on this
         device.  This device must support \htmlref{cursor interaction}{se:interaction}.
         \texttt{[}\htmlref{current graphics device}{se:devglobal}\texttt{{]}}
      }
      \sstsubsection{
         IN1 = NDF (Read)
      }{
         The input NDF containing the data to be copied to the inside of
         the supplied polygonal segments.  If a null value is supplied,
         the inside of the polygonal segments will be filled with bad
         values.
      }
      \sstsubsection{
         IN2 = NDF (Read)
      }{
         The input NDF containing the data to be copied to the outside
         of the supplied polygonal segments.  If a null value is
         supplied, the outside of the polygonal segments will be filled
         with bad values.
      }
      \sstsubsection{
         INCAT1-INCAT20 = FILENAME (Read)
      }{
         If MODE is \texttt{"Catalogue"}, each of the Parameters INCAT1 to INCAT20
         are used to access catalogues containing the co-ordinates of the
         vertices of a single polygon.  Suitable catalogues may be created
         using \htmlref{CURSOR}{CURSOR}, \htmlref{LISTMAKE}{LISTMAKE},
         \emph{etc}.  If a value is assigned to INCAT1
         on the command line, you are not prompted for any of the remaining
         parameters in this group; additional polygon catalogues must also
         be supplied on the command line.  Otherwise, you are prompted
         for INCAT1, then INCAT2, \emph{etc.} until a null value is given or
         INCAT20 is reached.

         The positions in each catalogue are mapped into the pixel
         co-ordinate Frame of the output NDF by aligning the WCS
         information stored in the catalogue with the \htmlref{WCS}{apndf:wcs} information in
         the output NDF.  A message indicating the Frame in which the
         positions were aligned with the output NDF is displayed.
      }
      \sstsubsection{
         LOGFILE = FILENAME (Write)
      }{
         The name of a text file in which the co-ordinates of the
         polygon vertices are to be stored.  A null value (\texttt{{!}}) means that
         no file is created.  \texttt{[!]}
      }
      \sstsubsection{
         MARKER = \_INTEGER (Read)
      }{
         This parameter is only accessed if Parameter PLOT is set to
         \texttt{"Chain"} or \texttt{"Mark"}.  It specifies the type of marker with which
         each cursor position should be marked, and should be given as
         an integer \PGPLOT\  marker type.  For instance, \texttt{0} gives a box,
         \texttt{1} gives a dot, \texttt{2} gives a cross, \texttt{3} gives an asterisk,
         \texttt{7} gives a triangle.  The value must be larger than or equal to $-$31.
         \texttt{[}current value\texttt{{]}}
      }
      \sstsubsection{
         MODE = \htmlref{LITERAL}{se:parmenu} (Read)
      }{
         The mode in which the co-ordinates of each polygon vertex are to
         be obtained.  The supplied string can be one of the following
         selection.

         \ssthitemlist{

            \sstitem
            \texttt{"Interface"} --- positions are obtained using Parameter COORDS.
            These positions must be supplied in the current co-ordinate
            Frame of the output NDF (see Parameter OUT).

            \sstitem
            \texttt{"Cursor"} --- positions are obtained using the graphics cursor of
            the device specified by Parameter DEVICE.  The WCS information
            stored with the picture in the graphics database is used to
            map the supplied cursor positions into the pixel co-ordinate
            Frame of the output NDF.  A message is displayed indicating the
            co-ordinate Frame in which the picture and the output NDF were
            aligned.

            \sstitem
            \texttt{"Catalogue"} --- positions are obtained from positions lists
            using Parameters INCAT1 to INCAT20.  Each catalogue defines a
            single polygon.  The WCS information in each catalogue is used to
            map the positions in the catalogue into the pixel co-ordinate
            Frame of the output NDF.  A message is displayed for each catalogue
            indicating the co-ordinate Frame in which the catalogue and the
            output NDF were aligned.

            \sstitem
            \texttt{"File"} --- positions are obtained from text files using Parameters
            POLY1 to POLY20.  Each file defines a single polygon.  Each line
            in a file must contain two formatted axis values in the current
            co-ordinate Frame of the output NDF (see Parameter OUT),
            separated by white space or a comma.

         }
         \texttt{[}current value\texttt{{]}}
      }
      \sstsubsection{
         MAXPOLY = \_INTEGER (Read)
      }{
         The maximum number of polygons which can be used.  For
         instance, this can be set to \texttt{1} to ensure that no more than one
         polygon is used (this sort of thing can be useful when writing
         procedures or scripts).  A null value causes no limit to be
         imposed (unless MODE=\texttt{"File"} or \texttt{"Catalogue"} in which case a limit
         of 20 is imposed).  \texttt{[!]}
      }
      \sstsubsection{
         MINPOLY = \_INTEGER (Read)
      }{
         The minimum number of polygons which can be used.  For
         instance, this can be set to \texttt{2} to ensure that at least two
         polygons are used.  The supplied value must be fewer than or
         equal to the value given for MAXPOLY and must be greater than
         zero.  \texttt{[1]}
      }
      \sstsubsection{
         OUT = NDF (Write)
      }{
         The output NDF.  If only one input NDF is supplied (that is, if
         one of IN1 and IN2 is assigned a null value), then the output
         NDF has the same shape and size as the supplied input NDF.  Also,
         ancillary data such as WCS information is propagated from the
         supplied input NDF.  In particular, this means that the current
         co-ordinate Frame of the output NDF (in which vertex positions
         should be supplied if MODE is \texttt{"File"} or \texttt{"Interface"}) is
         inherited from the input NDF.  If two input NDFs are supplied,
         then the shape and size of the output NDF corresponds to the
         area of overlap between the two input NDFs (in pixel space), and
         the WCS information and \htmlref{current Frame}{se:curframe}~
         are inherited from the NDF associated with Parameter IN1.
      }
      \sstsubsection{
         PLOT = LITERAL (Read)
      }{
         The type of graphics to be used to mark the position of each
         selected vertex.  It is only used if Parameter MODE is given
         the value \texttt{"Cursor"}.  The appearance of these graphics (colour,
         size, \emph{etc.}) is controlled by the STYLE parameter.  PLOT can take
         any of the following values.

         \ssthitemlist{

            \sstitem
            \texttt{"None"} --- No graphics are produced.

            \sstitem
            \texttt{"Mark"} --- Each position is marked with a marker of type specified
            by Parameter MARKER.

            \sstitem
            \texttt{"Poly"} --- Causes each position to be joined by a line to the
            previous position.  Each polygon is closed by joining the last
            position to the first.

            \sstitem
            \texttt{"Chain"} --- This is a combination of \texttt{"Mark"} and \texttt{"Poly"}.  Each
            position is marked by a marker and joined by a line to the
            previous position.  Parameter MARKER is used to specify the marker
            to use.  \texttt{[}current value\texttt{{]}}
         }
      }
      \sstsubsection{
         POLY1-POLY20 = FILENAME (Read)
      }{
         If MODE is \texttt{"File"}, each of the Parameters POLY1 to POLY20 are
         used to access text files containing the co-ordinates of the
         vertices of a single polygon.  If a value is assigned to POLY1 on
         the command line, you are not prompted for any of the remaining
         parameters in this group; additional polygon files must also
         be supplied on the command line.  Otherwise, you are prompted
         for POLY1, then POLY2, \emph{etc.} until a null value is given or
         POLY20 is reached.

         Each position should be supplied within the current co-ordinate
         Frame of the output NDF (see Parameter OUT).  No more than two
         formatted axis values (separated by a comma or space) may be
         supplied on each line.  If the co-ordinate Frame being used has
         more than two axes, then the two axes to use must be specified using
         Parameter USEAXIS.
      }
      \sstsubsection{
         QUALITY = \_LOGICAL (Read)
      }{
         If a \texttt{TRUE} value is supplied for Parameter QUALITY then quality
         information is copied from the input NDFs to the output NDFs.
         Otherwise, the quality information is not copied.  This
         parameter is only accessed if all supplied input NDFs have
         defined \htmlref{QUALITY}{apndf:quality}~ components.  If any of
         the supplied input NDFs do not have defined QUALITY components,
         then no quality is copied.  Note, if a null input NDF is given then the
         corresponding output QUALITY values are set to zero.  \texttt{[TRUE]}
      }
      \sstsubsection{
         STYLE = \htmlref{GROUP}{se:groups} (Read)
      }{
         A group of attribute settings describing the style to use when
         drawing the graphics specified by Parameter PLOT.

         A comma-separated list of strings should be given in which each
         string is either an attribute setting, or the name of a text
         file preceded by an up-arrow character \texttt{"$\wedge$"}.  Such text files
         should contain further comma-separated lists which will be
         read and interpreted in the same manner.  Attribute settings
         are applied in the order in which they occur within the list,
         with later settings overriding any earlier settings given for
         the same attribute.

         Each individual attribute setting should be of the form:

            $<$name$>$=$<$value$>$

         where $<$name$>$ is the name of a plotting attribute, and $<$value$>$
         is the value to assign to the attribute.  Default values will be
         used for any unspecified attributes.  All attributes will be
         defaulted if a null value (\texttt{{!}})---the initial default---is supplied.
         To apply changes of style to only the current invocation, begin these
         attributes with a plus sign.  A mixture of persistent and temporary
         style changes is achieved by listing all the persistent attributes
         followed by a plus sign then the list of temporary attributes.

         See \slhyperref{Plotting Attributes}{Section~}{}{ap:plotting_attr}
         for a description of the available attributes.  Any unrecognised
         attributes are ignored (no error is reported).

         The appearance of the lines forming the edges of each polygon is
         controlled by the attributes \htmlattref{Colour(Curves)}{Colour(element)},
         \htmlattref{Width(Curves)}{Width(element)}, \emph{etc.}
         (either of the synonyms \att{Lines} and \att{Edges} may be used in place of
         \att{Curves}).  The appearance of the vertex markers is controlled by the
         attributes \att{Colour(Markers)},
         \htmlattref{Size(Markers)}{Size(element)}, \emph{etc.} (the synonyms
         \att{Vertices} may be used in place of \att{Markers}).  \texttt{[}current value\texttt{{]}}
      }
      \sstsubsection{
         USEAXIS = \htmlref{GROUP}{se:groups} (Read)
      }{
         USEAXIS is only accessed if the current \htmlref{co-ordinate Frame}{se:domains}~ of
         the output NDF has more than two axes.  A group of two strings should be
         supplied specifying the two axes spanning the plane in which the
         supplied polygons are defined.  Each axis can be specified
         using one of the following options.

         \ssthitemlist{

            \sstitem
            An integer index of an axis within the current Frame of the
            output NDF (in the range 1 to the number of axes in the current
            Frame).

            \sstitem
            An axis \htmlattref{Symbol}{Symbol(axis)}~ string such as
            \texttt{"RA"} or \texttt{"VRAD"}.

            \sstitem
            A generic option where \texttt{"SPEC"} requests the spectral
            axis, \texttt{"TIME"} selects the time axis, \texttt{"SKYLON"}
            and \texttt{"SKYLAT"} picks the sky longitude and latitude
            axes respectively.  Only those axis domains present are
            available as options.
         }

         A list of acceptable values is displayed
         if an illegal value is supplied.  If a null (\texttt{{!}}) value is supplied,
         the axes with the same indices as the first two significant NDF pixel
         axes are used.  \texttt{[!]}
      }
      \sstsubsection{
         VARIANCE = \_LOGICAL (Read)
      }{
         If a \texttt{TRUE} value is supplied for Parameter VARIANCE then
         variance information is copied from the input NDFs to the
         output NDFs.  Otherwise, the variance information is not
         copied.  This parameter is only accessed if all supplied input
         NDFs have defined \htmlref{VARIANCE}{apndf:variance}~ components.  If
         any of the supplied input NDFs do not have defined VARIANCE components, then no
         variances are copied.  Note, if a null input NDF is given then
         the corresponding output variance values are set bad.  \texttt{[TRUE]}
      }
   }
   \sstexamples{
      \sstexamplesubsection{
         segment in1=m51a in2=m51b out=m51\_comp incat1=coords mode=cat
      }{
         Copies a region of the NDF m51a to the corresponding position
         in the output NDF m51\_comp.  The region is defined by the list
         of vertex co-ordinates held in catalogue \texttt{coords.FIT}.  All
         pixels in the output NDF which fall outside this region are
         given the corresponding pixel values from NDF m51b.
      }
      \sstexamplesubsection{
         segment in1=m51a out=m51\_cut mode=cursor plot=poly accept
      }{
         Copies a region of the NDF m51a to the corresponding position
         in the output NDF m51\_cut.  The region is defined by selecting
         vertices using a graphics cursor.  The image m51a should
         previously have been displayed.  Each vertex is joined to the
         previous vertex by a line on the graphics device.  The
         ACCEPT keyword causes the suggested null default value for IN2
         to be accepted.  This means that all pixels outside the region
         identified using the cursor will be set bad in the output NDF.
      }
   }
   \label{notes:segment}
   \sstnotes{
      \sstitemlist{

         \sstitem
         Supplied positions are mapped into the pixel co-ordinate Frame
         of the output NDF before being used.  This means that the two input
         NDFs (if supplied) must be aligned in pixel space before using this
         application.

         \sstitem
         The routine can handle NDFs of arbitrary dimensionality.  If
         either input has three or more dimensions then all planes in the
         NDF pixel arrays are processed in the same way, that is the same
         polygonal regions are extracted from each plane and copied to the
         corresponding plane of the output NDF.  The plane containing the
         polygons must be defined using Parameter USEAXIS.  This plane is a
         plane within the current co-ordinate Frame of the output NDF (which
         is inherited from the first supplied input NDF).  This scheme will
         only work correctly if the selected plane in the current co-ordinate
         Frame is parallel to one of the planes of the pixel array.
      }
   }
   \sstdiytopic{
      Related Applications
   }{
KAPPA: \htmlref{ARDMASK}{ARDMASK},
\htmlref{ERRCLIP}{ERRCLIP},
\htmlref{FILLBAD}{FILLBAD},
\htmlref{FFCLEAN}{FFCLEAN},
\htmlref{PASTE}{PASTE},
\htmlref{REGIONMASK}{REGIONMASK},
\htmlref{SETMAGIC}{SETMAGIC},
\htmlref{THRESH}{THRESH}.
   }
   \sstimplementationstatus{
      \sstitemlist{

         \sstitem
         This routine will propagate \htmlref{VARIANCE}{apndf:variance}~ component values so long
         as all supplied input NDFs have defined VARIANCE components, and
         Parameter VARIANCE is not \texttt{FALSE}.

         \sstitem
         This routine will propagate \htmlref{QUALITY}{apndf:quality}~ component values so long
         as all supplied input NDFs have defined QUALITY components, and
         Parameter QUALITY is not \texttt{FALSE}.

         \sstitem
         The \htmlref{UNITS}{apndf:units}, \htmlref{AXIS}{apndf:axis}, \htmlref{LABEL}{apndf:label}, \htmlref{TITLE}{apndf:title}, \htmlref{WCS}{apndf:wcs}, and \htmlref{HISTORY}{apndf:history}~ components are
         propagated from the first supplied input NDF, together with all
         extensions.

         \sstitem
         All \htmlref{non-complex numeric types}{ap:HDStypes} are supported.  The following
         data types are processed directly: \_WORD, \_INTEGER, \_REAL,
         \_DOUBLE.
      }
   }
}
\sstroutine{
   SETAXIS
}{
   Sets values for an axis array component within an NDF data
   structure
}{
   \sstdescription{
      This routine modifies the values of an \htmlref{axis array
      component}{apndf:axis}~ or
      system within an \NDFref{NDF} data structure.  There are a number of
      options (see Parameter MODE).  They permit the deletion of the
      axis system, or an individual variance or width component; the
      replacement of one or more individual values; assignment of the
      whole array using Fortran-like mathematical expressions, or values
      in a text file, or to \htmlref{pixel co-ordinates}{se:pixgrd}, or by copying from
      another NDF.

      If an AXIS structure does not exist, a new one whose centres are
      pixel co-ordinates is created before any modification.
   }
   \sstusage{
      setaxis ndf dim mode [comp]
        $\left\{ {\begin{tabular}{l}
                  file=? \\
                  index=? newval=? \\
                  exprs=? \\
                  axisndf=? \\
                  \end{tabular} }
        \right.$
        \newline\latexhtml{\hspace*{13.9em}}{~~~~~~~~~~~~~~~~~~~~~~~~~~}
        \makebox[0mm][c]{\small mode}
   }
   \sstparameters{
      \sstsubsection{
         AXISNDF = NDF (Read)
      }{
         The Data values in this NDF are used as the axis centre values
         if Parameter MODE is set to \texttt{"NDF"}. The supplied NDF must be
         one dimensional and must be aligned in pixel co-ordinates with
         the NDF axis that is being modified.
      }
      \sstsubsection{
         COMP = \htmlref{LITERAL}{se:parmenu} (Read)
      }{
         The name of the NDF axis array component to be modified.  The
         choices are: \texttt{"Centre"}, \texttt{"Data"}, \texttt{"Error"}, \texttt{"Width"} or \texttt{"Variance"}.
         \texttt{"Data"} and \texttt{"Centre"} are synonyms and selects the axis centres.
         \texttt{"Variance"} is the variance of the axis centres, \emph{i.e.} measures
         the uncertainty of the axis-centre values.  \texttt{"Error"} is the
         alternative to \texttt{"Variance"} and causes the square of the
         supplied error values to be stored.  \texttt{"Width"} selects the axis
         width array.  \texttt{["Data"]}
      }
      \sstsubsection{
         DIM = \_INTEGER (Read)
      }{
         The axis dimension for which the array component is to be
         modified.  There are separate arrays for each NDF dimension.
         The value must lie between 1 and the number of dimensions of
         the NDF.  This defaults to \texttt{1} for a one-dimensional NDF.  DIM is
         not accessed when COMP=\texttt{"Centre"} and MODE=\texttt{"Delete"}.  The
         suggested default is the current value.  \texttt{[]}
      }
      \sstsubsection{
         EXPRS = LITERAL (Read)
      }{
         A Fortran-like arithmetic expression giving the value to be
         assigned to each element of the axis array specified by
         Parameter COMP.  The expression may just contain a constant
         for the axis widths or variances, but the axis-centre values
         must vary.  In the latter case and whenever a constant value
         is not required, there are two tokens available---INDEX and
         CENTRE---either or both of which may appear in the expression.
         INDEX represents the pixel index of the corresponding array
         element, and CENTRE represents the existing axis centres.
         Either the CENTRE or the INDEX token must appear in the
         expression when modifying the axis centres.  All of the
         standard Fortran-77 intrinsic functions are available for use
         in the expression, plus a few others (see
         \xref{SUN/61}{sun61}{} for details
         and an up-to-date list).

         Here are some examples.  Suppose the axis centres are being
         changed, then EXPRS=\texttt{"INDEX-0.5"} gives pixel co-ordinates,
         EXPRS=\texttt{"2.3 $*$ INDEX $+$ 10"} would give a linear axis at offset 10
         and an increment of 2.3 per pixel, EXPRS=\texttt{"LOG(INDEX$*$5.2)"}
         would give a logarithmic axis, and EXPRS=\texttt{"CENTRE$+$10"} would add
         ten to all the array centres.  If COMP=\texttt{"Width"}, EXPRS=0.96
         would set all the widths to 0.96, and EXPRS=\texttt{"SIND(INDEX-30)$+$2"}
         would assign the widths to two plus the sine of the pixel
         index with respect to index 30 measured in degrees.

         EXPRS is only accessed when MODE=\texttt{"Expression"}.
      }
      \sstsubsection{
         FILE = FILENAME (Read)
      }{
         Name of the text file containing the free-format axis data.
         This parameter is only accessed if MODE=\texttt{"File"}.  The
         suggested default is the current value.
      }
      \sstsubsection{
         INDEX = \_INTEGER (Read)
      }{
         The pixel index of the array element to change.  A null value
         (\texttt{{!}}) terminates the loop during multiple replacements.  This
         parameter is only accessed when MODE=\texttt{"Edit"}.  The suggested
         default is the current value.
      }
      \sstsubsection{
         LIKE = NDF (Read)
      }{
         A template NDF containing axis arrays.  These arrays will be
         copied into the NDF given by Parameter NDF.  All axes are copied.
         The other parameters are only accessed if a null (\texttt{{!}}) value is
         supplied for LIKE.  If the NDF being modified extends beyond the
         edges of the template NDF, then the template axis arrays will be
         extrapolated to cover the entire NDF.  This is done using linear
         extrapolation through the last two extreme axis values.  \texttt{[!]}
      }
      \sstsubsection{
         MODE = \htmlref{LITERAL}{se:parmenu} (Read)
      }{
         The mode of the modification.  It can be one of the following.

         \ssthitemlist{

            \sstitem
            \texttt{"Delete"} --- Deletes the array, unless COMP=\texttt{"Data"} or
                            \texttt{"Centre"} whereupon the whole axis structure
                            is deleted.

            \sstitem
            \texttt{"Edit"} --- Allows the modification of individual
                            elements within the array.

            \sstitem
            \texttt{"Expression"} --- Allows a mathematical expression to define
                            the array values.  See Parameter EXPRS.

            \sstitem
            \texttt{"File"} --- The array values are read in from a
                            free-format text file.

            \sstitem
            \texttt{"Linear\_WCS"} --- The axis centres are set to the least-squares
                            linear fit to the values of the selected axis
                            in the current co-ordinate Frame of the NDF.
                            This is useful for exporting to packages with
                            limited FITS WCS compatibility and when the
                            non-linearity is small.  \texttt{"Linear\_WCS"} is only
                            available when COMP=\texttt{"Data"} or \texttt{"Centre"}.

            \sstitem
            \texttt{"NDF"} --- The axis centres are set to the corresponding
                            Data values read from the NDF specified by
                            Parameter AXISNDF. This is only available when
                            COMP=\texttt{"Data"} or \texttt{"Centre"}.

            \sstitem
            \texttt{"Pixel"} --- The axis centres are set to pixel
                            co-ordinates.  This is only available when
                            COMP=\texttt{"Data"} or \texttt{"Centre"}.

            \sstitem
            \texttt{"WCS"} --- The axis centres are set to the values
                            of the selected axis in the current
                            \htmlref{co-ordinate Frame}{se:domains}~ of
                            the NDF.  This is only available when
                            COMP=\texttt{"Data"} or \texttt{"Centre"}.
         }
         The suggested default is the current value.
      }
      \sstsubsection{
         NDF = NDF (Read and Write)
      }{
         The NDF data structure in which an axis array component is to
         be modified.
      }
      \sstsubsection{
         NEWVAL = LITERAL (Read)
      }{
         Value to substitute in the array element.  The range of
         allowed values depends on the data type of the array being
         modified.  NEWVAL=\texttt{"Bad"} instructs that the bad value
         appropriate for the array data type be substituted.  Placing
         NEWVAL on the command line permits only one element to be
         replaced.  If there are multiple replacements, a null value
         (\texttt{{!}}) terminates the loop.  This parameter is only accessed when
         MODE=\texttt{"Edit"}.
      }
      \sstsubsection{
         TYPE = LITERAL (Read)
      }{
         The data type of the modified axis array.  TYPE can be either
         \texttt{"\_REAL"} or \texttt{"\_DOUBLE"}.  It is only accessed for MODE=\texttt{"File"},
         \texttt{"Expression"}, or \texttt{"Pixel"}.  If a null (\texttt{{!}}) value is supplied, the
         value used is the current data type of the array component if
         it exists, otherwise it is \texttt{"\_REAL"}.  \texttt{[!]}
      }
   }
   \sstexamples{
      \sstexamplesubsection{
         setaxis ff mode=delete
      }{
         This erases the axis structure from the NDF called ff.
      }
      \sstexamplesubsection{
         setaxis ff like=hh
      }{
         This creates axis structures in the NDF called ff by copying
         them from the NDF called hh, extrapolating them as necessary to
         cover ff.
      }
      \sstexamplesubsection{
         setaxis abell4 1 expr exprs="CENTRE $+$ 0.1 $*$ (INDEX-1)"
      }{
         This modifies the axis centres along the first axis in the NDF
         called abell4.  The new centre values are spaced by 0.1 more
         per element than previously.
      }
      \sstexamplesubsection{
         setaxis cube 3 expr error exprs="25.3$+$0.2$*$MOD(INDEX,8)"
      }{
         This modifies the axis errors along the third axis in the NDF
         called cube.  The new errors values are given by the
         expression \texttt{"25.3$+$0.2$*$MOD(INDEX,8)"}, in other words the noise
         has a constant term (25.3), and a cyclic ramp component of
         frequency 8 pixels.
      }
      \sstexamplesubsection{
         setaxis spectrum mode=file file=spaxis.dat
      }{
         This assigns the axis centres along the first axis in the
         one-dimensional NDF called spectrum.  The new centre values are
         read from the free-format text file called spaxis.dat.
      }
      \sstexamplesubsection{
         setaxis ndf=plate3 dim=2 mode=pixel
      }{
         This assigns pixel co-ordinates to the second axis's centres
         in the NDF called plate3.
      }
      \sstexamplesubsection{
         setaxis datafile 2 expression exprs="centre" type=\_real
      }{
         This modifies the data type of axis centres along the second
         dimension of the NDF called datafile to be \_REAL.
      }
      \sstexamplesubsection{
         setaxis cube 2 edit index=3 newval=129.916
      }{
         This assigns the value 129.916 to the axis centre at index 3
         along the second axis of the NDF called cube.
      }
      \sstexamplesubsection{
         setaxis comp=width ndf=cube dim=1 mode=edit index=-16 newval=1E-05
      }{
         This assigns the value 1.0E-05 to the axis width at index $-$16
         along the first axis of the NDF called cube.
      }
   }
   \sstnotes{
      \sstitemlist{

         \sstitem
         An end-of-file error results when MODE=\texttt{"File"} and the file
         does not contain sufficient values to assign to the whole array.
         In this case the axis array is unchanged.  A warning is given if
         there are more values in a file record than are needed to complete
         the axis array.

         \sstitem
         An invalid expression when MODE=\texttt{"Expression"} results in an
         error and the axis array is unchanged.

         \sstitem
         The chapter entitled \xref{``The Axis Coordinate System''}
         in \xref{SUN/33}{sun33}{the_axis_coordinate_system}~
         describes the NDF axis co-ordinates system and
         is recommended reading especially if you are using axis widths.

         \sstitem
         There is no check, apart from constraints on Parameter NEWVAL,
         that the variance is not negative and the widths are positive.
      }
   }
   \sstdiytopic{
      File Format
   }{
      The format is quite flexible.  The number of axis-array values
      that may appear on a line is variable; the values are separated
      by at least a space, comma, tab or carriage return.  A line can
      have up to 255 characters.  In addition a record may have
      trailing comments designated by a hash or exclamation mark.  Here
      is an example file, though a more regular format would be clearer
      for the human reader (say 10 values per line with commenting).

      \texttt{\begin{verse}
      \# Axis Centres along second dimension \\
      -3.4 -0.81 \\
      .1 3.3 4.52 5.6 9 10.5 12.  15.3   18.1  20.2 \\
      23 25.3 ! a comment \\
      26.8,27.5 29. 30.76  32.1 32.4567 \\
       35.2 37. \\
      <EOF> \\
      \end{verse}}

   }
   \sstdiytopic{
      Related Applications
   }{
KAPPA: \htmlref{AXCONV}{AXCONV},
\htmlref{AXLABEL}{AXLABEL},
\htmlref{AXUNITS}{AXUNITS};
\xref{FIGARO}{sun86}{}: \xref{LXSET}{sun86}{LXSET},
\xref{LYSET}{sun86}{LYSET}.
   }
   \sstimplementationstatus{
      Processing is in single- or double-precision floating point.
   }
}
\sstroutine{
   SETBAD
}{
   Sets new bad-pixel flag values for an NDF
}{
   \sstdescription{
      This application sets new logical values for the
      \htmlref{bad-pixel flags}{setbad:badpixelflag}~ associated with an
      \NDFref{NDF's}  data and/or variance arrays.  It may either be used
      to test whether \htmlref{bad pixels}{se:masking}~ are  actually
      present in these arrays and to set their bad-pixel flags  accordingly,
      or to set explicit \texttt{TRUE}~ or \texttt{FALSE}~ values for these flags.
   }
   \sstusage{
      setbad ndf [value]
   }
   \sstparameters{
      \sstsubsection{
         DATA = \_LOGICAL (Read)
      }{
         This parameter controls whether the NDF's data array is
         processed.  If a \texttt{TRUE} value is supplied (the default), then it
         will be processed.  Otherwise it will not be processed, so that
         the variance array (if present) may be considered on its own.
         The DATA and VARIANCE parameters should not both be set to
         \texttt{FALSE}.  \texttt{[TRUE]}
      }
      \sstsubsection{
         MODIFY = \_LOGICAL (Read)
      }{
         If a \texttt{TRUE} value is supplied for this parameter (the default),
         then the NDF's bad-pixel flags will be permanently modified if
         necessary.  If a \texttt{FALSE} value is supplied, then no modifications
         will be made.  This latter mode allows the routine to be used
         to check for the presence of bad pixels without changing the
         current state of an NDF's bad-pixel flags.  It also allows the
         routine to be used on NDFs for which write access is not
         available.  \texttt{[TRUE]}
      }
      \sstsubsection{
         NDF = NDF (Read and Write)
      }{
         The NDF in which bad pixels are to be checked for, and/or
         whose bad-pixel flags are to be modified. (Note that setting
         the MODIFY parameter to \texttt{FALSE} makes it possible to check for
         bad pixels without permanently modifying the NDF.)
      }
      \sstsubsection{
         VALUE = \_LOGICAL (Read)
      }{
         If a null (\texttt{{!}}) value is supplied for this parameter (the
         default), then the routine will check to see whether any bad
         pixels are present.  This will only involve testing the value
         of each pixel if the bad-pixel flag value is initially \texttt{TRUE},
         in which case it will be reset to \texttt{FALSE} if no bad pixels are
         found.  If the bad-pixel flag is initially \texttt{FALSE}, then it will
         remain unchanged.

         If a logical (\texttt{TRUE} or \texttt{FALSE}) value is supplied for
         this parameter, then it indicates the new bad-pixel flag value
         which is to be set.  Setting a \texttt{TRUE} value indicates to later
         applications that there may be bad pixels present in the NDF,
         for which checks must be made.  Conversely, setting a \texttt{FALSE}
         value indicates that there are definitely no bad pixels
         present, in which case later applications need not check for
         them and should interpret the pixel values in the NDF
         literally.

         The VALUE parameter is not used (a null value is assumed) if
         the MODIFY parameter is set to \texttt{FALSE} indicating that the NDF
         is not to be permanently modified.  \texttt{[!]}
      }
      \sstsubsection{
         VARIANCE = \_LOGICAL (Read)
      }{
         This parameter controls whether the NDF's variance array is
         processed.  If a \texttt{TRUE} value is supplied (the default), then it
         will be processed.  Otherwise it will not be processed, so that
         the data array may be considered on its own.  The DATA and
         VARIANCE parameters should not both be set to \texttt{FALSE}.
         \texttt{[TRUE]}
      }
   }
   \sstexamples{
      \sstexamplesubsection{
         setbad ngc1097
      }{
         Checks the data and variance arrays (if present) in the NDF
         called ngc1097 for the presence of bad pixels.  If the initial
         bad-pixel flag values indicate that bad pixels may be present,
         but none are found, then the bad-pixel flags will be reset to
         \texttt{FALSE}.  The action taken will be reported.
      }
      \sstexamplesubsection{
         setbad ndf=ngc1368 nomodify
      }{
         Performs the same checks as described above, this time on the
         NDF called ngc1368.  The presence or absence of bad pixels is
         reported, but the NDF is not modified.
      }
      \sstexamplesubsection{
         setbad myfile nodata
      }{
         Checks the variance array (if present) in the NDF called
         myfile for the presence of bad pixels, and modifies its
         bad-pixel flag accordingly.  Specifying \texttt{nodata} inhibits
         processing of the data array, whose bad-pixel flag is left
         unchanged.
      }
      \sstexamplesubsection{
         setbad halpha false
      }{
         Sets the bad-pixel flag for the NDF called halpha to \texttt{FALSE}.
         Any pixel values which might previously have been regarded as
         bad will subsequently be interpreted literally as valid
         pixels.
      }
      \sstexamplesubsection{
         setbad hbeta true
      }{
         Sets the bad-pixel flags for the NDF called hbeta to be \texttt{TRUE}.
         If any pixels have the special `bad' value, then they will
         subsequently be regarded as invalid pixels.  Note that if this
         is followed by a further command such as \texttt{"setbad hbeta"}, then
         an actual check will be made to see whether any pixels have
         this special value.  The bad-pixel flags will be returned to
         \texttt{FALSE} if they do not.
      }
   }
   \label{setbad:badpixelflag}
   \sstdiytopic{
      Bad-pixel Flag Values
   }{
      If a \htmlref{bad-pixel flag}{apndf:bpflag}~ is \texttt{TRUE}, it indicates
      that the associated NDF array may contain the special `bad' value and
      that affected pixels are to be regarded as invalid.  Subsequent applications
      will need to check for such pixels and, if found, take account of
      them.

      Conversely, if a bad-pixel flag value is \texttt{FALSE}, it indicates that
      there are no bad pixels present.  In this case, any special `bad'
      values appearing in the array are to be interpreted literally as
      valid pixel values.
   }
   \sstdiytopic{
      Quality Components
   }{
      Bad pixels may also be introduced into an NDF's data and variance
      arrays implicitly through the presence of an associated NDF
      \htmlref{QUALITY}{apndf:quality}~ component.  This application will
      not take account of such a component, nor will it modify it.

      However, if either of the NDF's data or variance arrays do not
      contain any bad pixels themselves, a check will be made to see
      whether a QUALITY component is present.  If it is (and its
      associated \htmlref{bad-bits mask}{se:qualitymask}~ is non-zero),
      then a warning message will be issued indicating that bad pixels may
      be introduced via this QUALITY component.  If required, these bad
      pixels may be eliminated either by setting the bad-bits mask to
      zero or by erasing the QUALITY component.
   }
   \sstdiytopic{
      Related Applications
   }{
KAPPA: \htmlref{NOMAGIC}{NOMAGIC},
\htmlref{SETMAGIC}{SETMAGIC}.
   }
}
\sstroutine{
   SETBB
}{
   Sets a new value for the quality bad-bits mask of an NDF
}{
   \sstdescription{
      This application sets a new value for the
      \htmlref{bad-bits mask}{se:qualitymask}~
      associated with the \htmlref{QUALITY}{apndf:quality}~ component
      of an \NDFref{NDF}.  This 8-bit mask is used to select which
      of the bits in the quality array should normally be used to
      generate \htmlref{`bad'}{se:masking}~ pixels when the NDF is
      accessed.

      Wherever a bit is set to 1 in the bad-bits mask, the
      corresponding bit will be extracted from the NDF's quality array
      value for each pixel (the other quality bits being ignored).  A
      pixel is then considered `bad' if any of the extracted quality
      bits is set to 1.  Effectively, the bad-bits mask therefore allows
      selective activation of any of the eight 1-bit masks which can be
      stored in the quality array.

      The bit mask can be given either numerically (in decimal, binary,
      octal or hexadecimal format), or as a set of quality names (see
      \htmlref{SETQUAL}{SETQUAL}).
   }
   \sstusage{
      setbb ndf bb
   }
   \sstparameters{
      \sstsubsection{
         AND = \_LOGICAL (Read)
      }{
         By default, the value supplied via the BB parameter will be
         used literally as the new bad-bits mask value.  However, if a
         \texttt{TRUE} value is given for the AND parameter, then a bit-wise
         `AND' will first be performed with the old value of the mask.
         This facility allows individual bits in within the mask to be
         cleared (\emph{i.e.} reset to zero) without affecting the current
         state of other bits (see the \texttt{"Examples"} section).

         The AND parameter is not used if a \texttt{TRUE} value is given for the
         OR parameter.  \texttt{[FALSE]}
      }
      \sstsubsection{
         BB = LITERAL (Read)
      }{
         The new integer value for the bad-bits mask.  This may either
         be specified in normal decimal notation, or may be given using
         binary, octal or hexadecimal notation by adding a \texttt{"B"},
         \texttt{"O"} or \texttt{"Z"} prefix (respectively) to the appropriate
         string of digits.  The value supplied should lie in the range
         0 to 255 decimal (or 8 bits of binary).

         If the AND and OR parameters are both \texttt{FALSE}, then the value
         supplied will be used directly as the new mask value.
         However, if either of these logical parameters is set to \texttt{TRUE},
         then an appropriate bit-wise `AND' or `OR' operation with the
         old mask value will first be performed.

         It may also be specified as a comma-separated list of
         \htmlref{quality names.}{se:qnames}~
         A quality name is a symbolic name that identifies a
         specific quality bit (quality names can be defined using
         SETQUAL, and displayed using SHOWQUAL).

         The default value suggested when prompting for this value is
         chosen so as to leave the original mask value unchanged.
      }
      \sstsubsection{
         NDF = NDF (Read and Write)
      }{
         The NDF whose bad-bits mask is to be modified.
      }
      \sstsubsection{
         OR = \_LOGICAL (Read)
      }{
         By default, the value supplied via the BB parameter will be
         used literally as the new bad-bits mask value.  However, if a
         \texttt{TRUE} value is given for the OR parameter, then a bit-wise `OR'
         will first be performed with the old value of the mask.  This
         facility allows individual bits in within the mask to be set
         to 1 without affecting the current state of other bits (see
         the \texttt{"Examples"} section).  \texttt{[FALSE]}
      }
   }
   \sstexamples{
      \sstexamplesubsection{
         setbb myframe 3
      }{
         Sets the bad-bits mask value for the QUALITY component of the
         NDF called myframe to the value 3.  This means that bits 1 and
         2 of the associated quality array will be used to generate bad
         pixels.
      }
      \sstexamplesubsection{
         setbb myframe "SKY,BACK"
      }{
         Sets the bad-bits mask value for the quality component of the
         NDF called myframe so that any pixel that is flagged with either
         of the two qualities \texttt{"SKY"} or \texttt{"BACK"} will be set bad.
         The NDF should contain information that associates each of these
         quality names with a specific bit in the quality array.  Such
         information can for instance be created using the SETQUAL command.
      }
      \sstexamplesubsection{
         setbb ndf=myframe bb=b11
      }{
         This example performs the same operation as above, but in this
         case the new mask value has been specified using binary
         notation.
      }
      \sstexamplesubsection{
         setbb xspec b10001000 or
      }{
         Causes the bad-bits mask value in the NDF called xspec to
         undergo a bit-wise `OR' operation with the binary value
         10001000.  This causes bits 4 and 8 to be set without changing
         the state of any other bits in the mask.
      }
      \sstexamplesubsection{
         setbb quasar ze7 and
      }{
         Causes the bad-bits mask value in the NDF called quasar to
         undergo a bit-wise `AND' operation with the hexadecimal value
         E7 (binary 11100111).  This causes bits 4 and 5 to be cleared
         (\emph{i.e.}  reset to zero) without changing the state of any other
         bits in the mask.
      }
   }
   \sstnotes{
      The bad-bits value will be disregarded if the NDF supplied does not have a QUALITY
      component present.  A warning message will be issued if this should occur.
   }
   \sstdiytopic{
      Related Applications
   }{
KAPPA: \htmlref{QUALTOBAD}{QUALTOBAD},
\htmlref{REMQUAL}{REMQUAL},
\htmlref{SETQUAL}{SETQUAL},
\htmlref{SHOWQUAL}{SHOWQUAL};
\xref{FIGARO}{sun86}{}: \xref{Q2BAD}{sun86}{Q2BAD}.
   }
}
\sstroutine{
   SETBOUND
}{
   Sets new bounds for an NDF
}{
   \sstdescription{
      This application sets new pixel-index bounds for an \NDFref{NDF},
      either trimming it to remove unwanted pixels, or padding it with
      \htmlref{bad pixels}{se:masking}~ to achieve the required shape.  The
      number of dimensions may also be altered.  The NDF is accessed in
      update mode and modified {\it in situ}, preserving existing pixel
      values which lie within the new bounds.
   }
   \sstusage{
      setbound ndf
   }
   \sstparameters{
      \sstsubsection{
         LIKE = NDF (Read)
      }{
         This parameter may be used to specify an NDF which is to be
         used as a shape template.  If such a template is supplied, then
         its bounds will be used to determine the new shape required
         for the NDF specified via the NDF parameter.  By default no
         template will be used and the new shape will be determined
         by means of a section specification applied to the NDF being
         modified (see the ``Examples'').  \texttt{[!]}
      }
      \sstsubsection{
         NDF = NDF (Read and Write)
      }{
         The NDF whose bounds are to be modified.  In normal use, an
         \htmlref{NDF section}{se:ndfsect}~ will be specified for this
         parameter (see the ``Examples'') and the routine
         will use the bounds of this section
         to determine the new bounds required for the base NDF from
         which the section is drawn.  The base NDF is then accessed in
         update mode and its bounds are modified {\it in situ\/} to make them
         equal to the bounds of the section specified.  If a section is
         not specified, then the NDF's shape will only be modified if a
         shape template is supplied via the LIKE parameter.
      }
   }
   \sstexamples{
      \sstexamplesubsection{
         setbound datafile(1:512,1:512)
      }{
         Sets the pixel-index bounds of the NDF called datafile to be
         (1:512,~1:512), either by trimming off unwanted pixels or by
         padding out with bad pixels, as necessary.
      }
      \sstexamplesubsection{
         setbound alpha(:7,56:)
      }{
         Modifies the NDF called alpha so that its first dimension has
         an upper bound of 7 and its second dimension has a lower bound
         of 56.  The lower bound of the first dimension and the upper
         bound of the second dimension remain unchanged.
      }
      \sstexamplesubsection{
         setbound ndf=kg74b(,5500.0$\sim$100.0)
      }{
         Sets new bounds for the NDF called kg74b.  The bounds of the
         first dimension are left unchanged, but those of the second
         dimension are changed so that this dimension has an extent of
         100.0 centred on 5500.0, using the physical units in which
         this second dimension is calibrated.
      }
      \sstexamplesubsection{
         setbound newspec like=oldspec
      }{
         Changes the bounds of the NDF newspec so that they are equal
         to the bounds of the NDF called oldspec.
      }
      \sstexamplesubsection{
         setbound xflux(:2048) like=xflux
      }{
         Extracts the section extending from the lower bound of the
         one-dimensional NDF called xflux up to pixel 2048, and then
         modifies the bounds of this section to be equal to the
         original bounds of xflux, replacing xflux with this new NDF.
         This leaves the final shape unchanged, but sets all pixels
         from 2049 onwards to be equal to the bad-pixel value.
      }
      \sstexamplesubsection{
         setbound whole(5:10,5:10) like=whole(0:15,0:15)
      }{
         Extracts the section (5:10,~5:10) from the base NDF called
         whole and then sets its bounds to be equal to those of the
         section whole(0:15,~0:15), replacing whole with this new NDF.
         The effect is to select a 6-pixel-square region from the
         original NDF and then to pad it with a 5-pixel-wide border of
         bad pixels.
      }
   }
   \sstnotes{
      This routine modifies the NDF {\it in situ\/}~ and will not release unused
      file space if the size of the NDF is reduced.  If recovery of
      unused file space is required, then the related application
      NDFCOPY should be used.  This will copy the selected region of an
      NDF to a new data structure from which any unused space will be
      eliminated.
   }
   \sstdiytopic{
      Related Applications
   }{
KAPPA: \htmlref{NDFCOPY}{NDFCOPY},
\htmlref{SETORIGIN}{SETORIGIN};
\xref{FIGARO}{sun86}{}: \xref{ISUBSET}{sun86}{ISUBSET}.
   }
}
\sstroutine{
   SETEXT
}{
   Manipulates the contents of a specified NDF extension
}{
   \sstdescription{
      This task enables the contents of a specified \NDFref{NDF} extension to be
      edited.  It can create a new extension or delete an existing one,
      can create new scalar components within an extension, or modify
      or display the values of existing scalar components within the
      extension.  The task operates on only one extension at a
      time, and must be closed down and restarted to work on a new
      extension.

      The task may operate in one of two modes, according to the
      LOOP parameter.  When LOOP=\texttt{FALSE} only a single option is
      executed at a time, making the task suitable for use from an
      \ICL\ procedure.  When LOOP=\texttt{TRUE} several
      options may be executed at once, making it easier to modify
      several extension components interactively in one go.
   }
   \sstusage{
      setext ndf xname option cname
        $\left\{ {\begin{tabular}{l}
                  ok \\
                  ctype=? shape=? ok \\
                  newname=? \\
                  xtype=?
                  \end{tabular} }
        \right.$
        \newline\latexhtml{\hspace*{14.95em}}{~~~~~~~~~~~~~~~~~~~~~~~~~~~}
        \makebox[0mm][c]{\small option}
   }
   \sstparameters{
      \sstsubsection{
         CNAME = LITERAL (Read)
      }{
         The name of component (residing within the extension) to be
         examined or modified.  It is only accessed when OPTION=\texttt{"Erase"},
         \texttt{"Get"}, \texttt{"Put"}, or \texttt{"Rename"}.
      }
      \sstsubsection{
         CTYPE = LITERAL (Read)
      }{
         The type of component (residing within the extension) to be
         created.  Allowed values are \texttt{"LITERAL"}, \texttt{"\_LOGICAL"}, \texttt{"\_DOUBLE"},
         \texttt{"\_REAL"}, \texttt{"\_INTEGER"}, \texttt{"\_CHAR"}, \texttt{"\_BYTE"}, \texttt{"\_UBYTE"}, \texttt{"\_UWORD"},
         \texttt{"\_WORD"}.  The length of the character type may be defined by
         appending the length, for example, \texttt{"\_CHAR$*$32"} is a
         32-character component.  \texttt{"LITERAL"} and \texttt{"\_CHAR"} generate
         80-character components.  CTYPE is only accessed when
         OPTION=\texttt{"Put"}.
      }
      \sstsubsection{
         CVALUE = LITERAL (Read)
      }{
         The value(s) for the component.  Each value is converted to the
         appropriate data type for the component.  CVALUE is only
         accessed when OPTION=\texttt{"Put"}.  Note that for an array of values
         the list must be enclosed in brackets, even in response to a
         prompt.  For convenience, if LOOP=\texttt{TRUE}, you are prompted for
         each string.
      }
      \sstsubsection{
         LOOP = \_LOGICAL (Read)
      }{
         LOOP=\texttt{FALSE} requests that only one operation be performed.
         This allows batch and non-interactive processing or use in
         procedures.  LOOP=\texttt{TRUE} makes SETEXT operate in a looping mode
         that allows several modifications and/or examinations to be
         made to the NDF for one activation.  Setting OPTION to \texttt{"Exit"}
         will end the looping.  \texttt{[TRUE]}
      }
      \sstsubsection{
         NDF = NDF (Update)
      }{
         The NDF to modify or examine.
      }
      \sstsubsection{
         NEWNAME = LITERAL (Read)
      }{
         The new name of a renamed extension component.  It is only
         accessed when OPTION=\texttt{"Rename"}.
      }
      \sstsubsection{
         OK = \_LOGICAL (Read)
      }{
         This parameter is used to seek confirmation before a component
         is erased or overwritten.  A \texttt{TRUE} value permits the operation.
         A \texttt{FALSE} value leaves the existing component unchanged.  This
         parameter is ignored when LOOP=\texttt{FALSE}.
      }
      \sstsubsection{
         OPTION = \htmlref{LITERAL}{se:parmenu} (Read)
      }{
         The operation to perform on the extension or a component
         therein.  The recognised options are listed below.
         \begin{aligndesc}
         \item [\texttt{"Delete"}] --- Delete an existing NDF extension.
         \item [\texttt{"Erase"}]  --- Erase a component within an NDF extension
         \item [\texttt{"Exit"}]   --- Exit from the task (when LOOP=\texttt{TRUE})
         \item [\texttt{"Get"}]    --- Display the value of a component within an NDF
                                  extension.  The component must exist.
         \item [\texttt{"Put"}]    --- Change the value of a component within an NDF
                         extension or create a new component.
         \item [\texttt{"Rename"}] --- Renames a component.  The component must exist.
         \item \texttt{"Select"}] --- Selects another extension.  If the extension
                         does not exist a new one is created.  This
                         option is not allowed when LOOP=\texttt{FALSE}.
         \end{aligndesc}

         The suggested default is the current value, except for the
         first option where there is no default.
      }
      \sstsubsection{
         SHAPE( ) = \_INTEGER (Read)
      }{
         The shape of the component.  Thus \texttt{3,2} would be a two-dimensional
         object with three elements along each of two lines.  \texttt{0} creates
         a scalar.  The suggested default is the shape of the object
         if it already exists, otherwise it is the current value.  It
         is only accessed when OPTION=\texttt{"Put"}.
      }
      \sstsubsection{
         XNAME = LITERAL (Read)
      }{
         The name of the extension to modify.
      }
      \sstsubsection{
         XTYPE = LITERAL (Read)
      }{
         The type of the extension to create.  The suggested default is
         the current value or \texttt{"EXT"} when there is no current value.
      }
   }
   \sstexamples{
      \sstexamplesubsection{
         setext hh50 fits delete noloop
      }{
         This deletes the \htmlref{FITS extension}{se:fitsairlock} in the NDF called hh50.
      }
      \sstexamplesubsection{
         setext myndf select xtype=mytype noloop
      }{
         This creates the extension MYEXT of data type MYTYPE in the
         NDF called myndf.
      }
      \sstexamplesubsection{
         setext xname=ccdpack ndf=abc erase cname=filter noloop
      }{
         This deletes the FILTER component of the CCDPACK extension in
         the NDF called abc.
      }
      \sstexamplesubsection{
         setext abc ccdpack put cname=filter cvalue=B ctype=\_char noloop
      }{
         This assigns the character value \texttt{"B"} to the FILTER component
         of the CCDPACK extension a the NDF called abc.
      }
      \sstexamplesubsection{
         setext virgo plate put cname=pitch shape=2 cvalue=[32,16] ctype=\_byte noloop
      }{
         This sets the byte two-element vector of component PITCH
         of the PLATE extension in the NDF called virgo.  The first
         element of PITCH is set to 32 and the second to 16.
      }
      \sstexamplesubsection{
         setext virgo plate rename cname=filter newname=waveband noloop
      }{
         This renames the FILTER component of the PLATE extension in
         the NDF called virgo to WAVEBAND.
      }
   }
   \sstnotes{
      \sstitemlist{

         \sstitem
         The \texttt{"Put"} option allows the creation of extension
         components with any of the primitive data types.

         \sstitem
         The task creates the extension automatically if it does not
         exist and only allows one extension to be modified at a time.
      }
   }
   \sstdiytopic{
      Related Applications
   }{
KAPPA: \htmlref{FITSIMP}{FITSIMP},
\htmlref{FITSLIST}{FITSLIST},
\htmlref{NDFTRACE}{NDFTRACE};
\xref{CCDPACK}{sun139}{}: \xref{CCDEDIT}{sun139}{CCDEDIT};
\xref{FIGARO}{sun86}{}: \xref{FITSKEYS}{sun86}{FITSKEYS};
\xref{HDSTRACE}{sun102}{};
\xref{IRAS90}{sun163}{}: \xref{IRASTRACE}{sun163}{IRASTRACE},
\xref{PREPARE}{sun163}{PREPARE}.
   }
}
\sstroutine{
   SETLABEL
}{
   Sets a new label for an NDF data structure
}{
   \sstdescription{
      This routine sets a new value for the \htmlref{LABEL component}{apndf:label}~ of an
      existing \NDFref{NDF} data structure.  The NDF is accessed in update mode
      and any pre-existing label is over-written with a new value.
      Alternatively, if a `null' value (\texttt{{!}}) is given for the LABEL
      parameter, then the NDF's label will be erased.
   }
   \sstusage{
      setlabel ndf label
   }
   \sstparameters{
      \sstsubsection{
         LABEL = LITERAL (Read)
      }{
         The value to be assigned to the NDF's LABEL component.  This
         should describe the type of quantity represented in the NDF's
         data array (\emph{e.g.} \texttt{"Surface Brightness"} or \texttt{"Flux Density"}).
         The value may later be used by other applications, for instance to
         label the axes of graphs where the NDF's data values are
         plotted.  The suggested default is the current value.
      }
      \sstsubsection{
         NDF = NDF (Read and Write)
      }{
         The NDF data structure whose label is to be modified.
      }
   }
   \sstexamples{
      \sstexamplesubsection{
         setlabel ngc1068 "Surface Brightness"
      }{
         Sets the LABEL component of the NDF structure ngc1068 to be
         \texttt{"Surface Brightness"}.
      }
      \sstexamplesubsection{
         setlabel ndf=datastruct label="Flux Density"
      }{
         Sets the LABEL component of the NDF structure datastruct to be
         \texttt{"Flux Density"}.
      }
      \sstexamplesubsection{
         setlabel raw\_data label=!
      }{
         By specifying a null value (\texttt{{!}}), this example erases any
         previous value of the LABEL component in the NDF structure
         raw\_data.
      }
   }
   \sstdiytopic{
      Related Applications
   }{
KAPPA: \htmlref{AXLABEL}{AXLABEL},
\htmlref{SETTITLE}{SETTITLE},
\htmlref{SETUNITS}{SETUNITS}.
   }
}
\sstroutine{
   SETMAGIC
}{
   Replaces all occurrences of a given value in an NDF array with
   the bad value
}{
   \sstdescription{
      This task flags all pixels that have a defined value in an
      \NDFref{NDF} with the standard \htmlref{bad}{se:masking}~ (`magic')
      value.  Other values are
      unchanged.  The number of replacements is reported.  SETMAGIC's
      applications include the import of data from software that has a
      different magic value.
   }
   \sstusage{
      setmagic in out repval [comp]
   }
   \sstparameters{
      \sstsubsection{
         COMP = \htmlref{LITERAL}{se:parmenu} (Read)
      }{
         The components whose values are to be flagged as bad.  It
         may be \texttt{"Data"}, \texttt{"Variance"}, \texttt{"Error"}, or \texttt{"All"}.
         The last of the options forces substitution of bad pixels in both
         the data and variance arrays.  This parameter is ignored if the
         data array is the only array component within the NDF.
         \texttt{["Data"]}
      }
      \sstsubsection{
         IN = NDF  (Read)
      }{
         Input NDF structure containing the data and/or variance array
         to have some of its elements flagged with the magic-value.
      }
      \sstsubsection{
         OUT = NDF (Write)
      }{
         Output NDF structure containing the data and/or variance array
         that is a copy of the input array, but with bad values flagging
         the replacement value.
      }
      \sstsubsection{
         REPVAL = \_DOUBLE (Read)
      }{
         The element value to be substituted with the bad value.  The
         same value is replaced in both the data and variance arrays
         when COMP=\texttt{"All"}.  It must lie within the minimum and maximum
         values of the data type of the array with higher precision.
         The replacement value is converted to data type of the array
         being converted before the search begins.  The suggested
         default is the current value.
      }
      \sstsubsection{
         TITLE = LITERAL (Read)
      }{
         \htmlref{Title}{apndf:title} for the output NDF structure.  A null value (\texttt{{!}})
         propagates the title from the input NDF to the output NDF.  \texttt{[!]}
      }
   }
   \sstexamples{
      \sstexamplesubsection{
         setmagic irasmap aitoff repval=-2000000
      }{
         This copies the NDF called irasmap to the NDF aitoff, except
         that any pixels with the IPAC blank value of \texttt{$-$2000000} are
         flagged with the standard bad value in aitoff.
      }
      \sstexamplesubsection{
         setmagic saturn saturnb 9999.0 comp=All
      }{
         This copies the NDF called saturn to the NDF saturnb, except
         that any elements in the data and variance arrays that have
         value 9999.0 are flagged with the standard bad value.
      }
   }
   \sstnotes{
      \sstitemlist{

         \sstitem
         The comparison for floating-point values tests that the
         difference between the replacement value and the element value is
         less than their mean times the precision of the data type.
      }
   }
   \sstdiytopic{
      Related Applications
   }{
KAPPA: \htmlref{CHPIX}{CHPIX},
\htmlref{FILLBAD}{FILLBAD},
\htmlref{GLITCH}{GLITCH},
\htmlref{NOMAGIC}{NOMAGIC},
\htmlref{SEGMENT}{SEGMENT},
\htmlref{SUBSTITUTE}{SUBSTITUTE},
\htmlref{ZAPLIN}{ZAPLIN};
\linebreak
\xref{FIGARO}{sun86}{}: \xref{GOODVAR}{sun86}{GOODVAR}.
   }
   \sstimplementationstatus{
      \ssthitemlist{

         \sstitem
         This routine correctly processes the \htmlref{AXIS}{apndf:axis}, DATA, \htmlref{QUALITY}{apndf:quality},
         \htmlref{VARIANCE}{apndf:variance}, \htmlref{LABEL}{apndf:label}, \htmlref{TITLE}{apndf:title}, \htmlref{UNITS}{apndf:units}, \htmlref{WCS}{apndf:wcs}, and \htmlref{HISTORY}{apndf:history}~ components of an NDF
         data structure and propagates all \htmlref{extensions}{apndf:extensions}.

         \sstitem
         Processing of \htmlref{bad pixels}{se:masking} and automatic \htmlref{quality masking}{se:qualitymask} are
         supported.

         \sstitem
         All \htmlref{non-complex numeric data types}{ap:HDStypes} can be handled.

         \sstitem
         Any number of NDF dimensions is supported.
      }
   }
}

\sstroutine{
   SETNORM
}{
   Sets a new value for one or all of an NDF's axis-normalisation
   flags
}{
   \sstdescription{
      This routine sets a new value for one or all the normalisation
      flags in an \NDFref{NDF} \htmlref{AXIS}{apndf:axis}~
      \latex{\goodbreak} data structure.  The NDF is accessed in
      update mode.  This flag determines how the NDF's data and
      variance arrays behave when the associated axis information is
      modified.

      If an AXIS structure does not exist, a new one whose centres are
      pixel co-ordinates is created.
   }
   \sstusage{
      setnorm ndf dim
   }
   \sstparameters{
      \sstsubsection{
         ANORM = \_LOGICAL (Read)
      }{
         The normalisation flag for the axis.  \texttt{TRUE} means that the
         data and variance values in the NDF are normalised to the
         pixel width values for the chosen axis so that the product
         of data value and width, and variance and the squared width
         are constant if the width is altered.

         A \texttt{FALSE} value means that the data and variance need not alter
         as the pixel widths are varied.  This is the default for an
         axis.  The suggested default is the current value.
      }
      \sstsubsection{
         DIM = \_INTEGER (Read)
      }{
         The axis dimension for which the normalisation flag is to be
         modified.  There are separate units for each NDF dimension.
         A value of \texttt{0} sets the normalisation flag for all the axes.
         The value must lie between 0 and the number of dimensions of
         the NDF.  This defaults to 1 for a one-dimensional NDF.  The
         suggested default is the current value.  \texttt{[]}
      }
      \sstsubsection{
         NDF = NDF (Read and Write)
      }{
         The NDF data structure in which an axis-normalisation flag
         is to be modified.
      }
   }
   \sstexamples{
      \sstexamplesubsection{
         setnorm hd23568 0 anorm
      }{
         This sets the normalisation flags along all axes of the
         NDF structure hd23568 to be true.
      }
      \sstexamplesubsection{
         setnorm ndf=spect noanorm
      }{
         This sets the normalisation flag of the one-dimensional NDF
         structure spect to be false.
      }
      \sstexamplesubsection{
         setnorm borg 3 anorm
      }{
         This sets the normalisation flag for the third dimension
         in the NDF structure borg.
      }
   }
   \sstdiytopic{
      Axis Normalisation
   }{
      In general, the axis-normalisation property is not needed.  An
      example where it is relevant is a spectrum in which data values
      representing energy per unit wavelength and each pixel has a
      known spread in wavelength.  The sum of each pixel's data value
      multiplied by its width gives the energy in a part of the
      spectrum.  A change to the axis width, say to allow for the
      redshift, necessitates a corresponding modification to the data
      value to retain this property.  In two dimensions an example is
      where the data measure flux per unit area of sky and the pixel
      widths are defined in terms of angular size.
   }
   \sstdiytopic{
      Related Applications
   }{
KAPPA: \htmlref{SETAXIS}{SETAXIS}.
   }
}
\sstroutine{
   SETORIGIN
}{
   Sets a new pixel origin for an NDF
}{
   \sstdescription{
      This application sets a new \htmlref{pixel origin}{apndf:origin}~ value
      for an \NDFref{NDF}~ data
      structure.  The NDF is accessed in update mode and the indices of
      the first pixel (the NDF's lower pixel-index bounds) are set to
      specified integer values, which may be positive or negative.  No
      other properties of the NDF are altered.  If required, a template
      NDF may be supplied and the new origin values will be derived
      from it.
   }
   \sstusage{
      setorigin ndf origin
   }
   \sstparameters{
      \sstsubsection{
         LIKE = NDF (Read)
      }{
         This parameter may be used to supply an NDF which is to be
         used as a template.  If such a template is supplied, then its
         origin (its lower pixel-index bounds) will be used as the new
         origin value for the NDF supplied via the NDF parameter.  By
         default, no template will be used and the new origin will be
         specified via the ORIGIN parameter.  \texttt{[!]}
      }
      \sstsubsection{
         NDF = NDF (Read and Write)
      }{
         The NDF data structure whose pixel origin is to be modified.
      }
      \sstsubsection{
         ORIGIN() = \_INTEGER (Read)
      }{
         A one-dimensional array specifying the new pixel origin values,
         one for each NDF dimension.
      }
   }
   \sstexamples{
      \sstexamplesubsection{
         setorigin image\_2d [1,1]
      }{
         Sets the indices of the first pixel in the two-dimensional image
         image\_2d to be (1,~1).  The image pixel values are unaltered.
      }
      \sstexamplesubsection{
         setorigin ndf=starfield
      }{
         A new pixel origin is set for the NDF structure called
         starfield.  SETORIGIN will prompt for the new origin values,
         supplying the existing values as defaults.
      }
      \sstexamplesubsection{
         setorigin ndf=cube origin=[-128,-128]
      }{
         Sets the pixel origin values for the first two dimensions of
         the three-dimensional NDF called cube to be ($-$128,~$-$128).  A value
         for the third dimension is not specified, so the origin of
         this dimension will remain unchanged.
      }
      \sstexamplesubsection{
         setorigin betapic like=alphapic
      }{
         Sets the pixel origin of the NDF called betapic to be equal to
         that of the NDF called alphapic.
      }
   }
   \sstnotes{
      If the number of new pixel origin values is fewer than the number
      of NDF dimensions, then the pixel origin of the extra dimensions
      will remain unchanged.  If the number of values exceeds the number
      of NDF dimensions, then the excess values will be ignored.
   }
   \sstdiytopic{
      Timing
   }{
      Setting a new pixel origin is a quick operation whose timing does
      not depend on the size of the NDF.
   }
   \sstdiytopic{
      Related Applications
   }{
KAPPA: \htmlref{SETBOUND}{SETBOUND}.
   }
}
\sstroutine{
   SETQUAL
}{
   Assigns a specified quality to selected pixels within an NDF
}{
   \sstdescription{
      This routine assigns (or optionally removes) the quality
      specified by Parameter QNAME to (or from) selected pixels in an
      \NDFref{NDF}.  For more information about using quality within
      {\KAPPA}~ see  \slhyperref{``Using Quality Names''}{Section~}{}{se:qnames}.

      The user can select the pixels to be operated on in one of three
      ways (see Parameter SELECT).

      \ssthitemlist{

         \sstitem
         By giving a `mask' NDF.  Pixels with bad values in the mask NDF
         will be selected from the corresponding input NDF.

         \sstitem
         By giving a list of \htmlref{pixel indices}{se:pixgrd}~ for
         the pixels that are to be selected.

         \sstitem
         By giving an ARD file  containing a description of the regions of
         the NDF that are to be selected.  The \htmlref{ARD
         system}{se:ardwork}~ (see \xref{SUN/183}{sun183}{}) uses a
         textual language to describe geometric regions of an array.  Text
         files containing ARD description suitable for use with this
         routine can be created interactively using the routine
         \htmlref{ARDGEN}{ARDGEN} or with \GAIAref.

      }
      The operation to be performed on the pixels is specified by
      Parameter FUNCTION.  The given quality may be assigned to or
      removed from pixels within the NDF.  The pixels operated on
      can either be those selected by the user (as described above),
      or those not selected.  The quality of all other pixels is left
      unchanged (unless the Parameter FUNCTION is given the value
      \texttt{"NS$+$HU"}  or \texttt{"NU$+$HS"}).  Thus for instance if
      pixel (1,~1) already held the
      quality specified by QNAME, and the quality was then assigned to
      pixel (2,~2) this would not cause the quality to be removed from
      pixel (1,~1).

      This routine can also be used to copy all quality information from
      one NDF to another (see Parameter LIKE).
   }
   \sstusage{
      setqual ndf qname comment mask
   }
   \sstparameters{
      \sstsubsection{
         ARDFILE = FILENAME (Read)
      }{
         The name of the ARD file containing a description of the parts
         of the NDF to be `selected'.  The ARD parameter is only prompted
         for if the SELECT parameter is given the value \texttt{"ARD"}.  The co-ordinate
         system in which positions within this file are given should be
         indicated by including suitable COFRAME or WCS statements within
         the file (see SUN/183), but will default to pixel co-ordinates
         in the absence of any such statements.  For instance, starting the
         file with a line containing the text \texttt{"COFRAME(SKY,System=FK5)"} would
         indicate that positions are specified in RA/DEC (FK5,J2000).  The
         statement \texttt{"COFRAME(PIXEL)"} indicates explicitly that positions are
         specified in pixel co-ordinates.
      }
      \sstsubsection{
         COMMENT = LITERAL (Read)
      }{
         A comment to store with the quality name.  This parameter is
         only prompted for if the NDF does not already contain a
         definition of the quality name.
      }
      \sstsubsection{
         FUNCTION = \htmlref{LITERAL}{se:parmenu} (Read)
      }{
         This parameter specifies what function is to be performed on
         the `selected' pixels specified using Parameters MASK, LIST or
         ARDFILE.  It can take any of the following values.

         \ssthitemlist{

            \sstitem
            \texttt{"HS"} --- Ensure that the quality specified by QNAME is held by
                 all the selected pixels.  The quality of all other
                 pixels is left unchanged.

            \sstitem
            \texttt{"HU"} --- Ensure that the quality specified by QNAME is held by all
                 the pixels that have not been selected.  The quality of
                 the selected pixels is left unchanged.

            \sstitem
            \texttt{"NS"} --- Ensure that the quality specified by QNAME is not held by
                 any of the selected pixels.  The quality of all other
                 pixels is left unchanged.

            \sstitem
            \texttt{"NU"} --- Ensure that the quality specified by QNAME is not held by
                 any of the pixels that have not been selected.  The
                 quality of the selected pixels is left unchanged.

            \sstitem
            \texttt{"HS$+$NU"} --- Ensure that the quality specified by QNAME is held by
                 all the selected pixels and not held by any of the other
                 pixels.

            \sstitem
            \texttt{"HU$+$NS"} --- Ensure that the quality specified by QNAME is held by
                 all the pixels that have not been selected and not held
                 by any of the selected pixels.
         }
         \texttt{["HS"]}
      }
      \sstsubsection{
         LIKE = NDF (Read)
      }{
         An existing NDF from which the QUALITY component and quality names are
         to be copied. These overwrite any corresponding information in the
         NDF given by Parameter NDF. If null (\texttt{{!}}), then the operation of
         this command is instead determined by Parameter SELECT. \texttt{[!]}
      }
      \sstsubsection{
         LIST = LITERAL (Read)
      }{
         A group of pixels positions within the input NDF listing the
         pixels that are to be `selected' (see Parameter FUNCTION).
         Each position should be giving as a list of pixel indices
         (\emph{e.g.} $X1$, $Y1$, $X2$, $Y2$,\dots  for a two dimensional NDF).  LIST is
         only prompted for if Parameter SELECT is given the value \texttt{"LIST"}.
      }
      \sstsubsection{
         MASK = NDF (Read)
      }{
         A mask NDF used to define the `selected' pixels within the
         input NDF (see Parameter FUNCTION).  The mask should be aligned
         pixel-for-pixel with the input NDF.  Pixels that are bad in
         the mask NDF are `selected'.  The quality of any pixels that
         lie outside the bounds of the mask NDF are left unaltered.  This
         parameter is only prompted for if the Parameter SELECT is given
         the value \texttt{"MASK"}.
      }
      \sstsubsection{
         NDF = NDF (Update)
      }{
         The NDF in which the quality information is to be stored.
      }
      \sstsubsection{
         QNAME = LITERAL (Read)
      }{
         The quality name.  If the supplied name is not already defined
         within the input NDF, then a definition of the name is
         added to the NDF.  The user is warned if the quality name is
         already defined within the NDF.
      }
      \sstsubsection{
         QVALUE = \_INTEGER (Read)
      }{
         If not null, then the whole QUALITY array is filled with the
         constant value given by QVALUE, which must be in the range 0 to
         255. No other changes are made to the NDF. {\tt [!]}
      }
      \sstsubsection{
         READONLY = \_LOGICAL (Read)
      }{
         If \texttt{TRUE}, then an error will be reported if any attempt is
         subsequently made to remove the quality name (\emph{e.g.} using
         REMQUAL).  \texttt{[FALSE]}
      }
      \sstsubsection{
         SELECT = LITERAL (Read)
      }{
         If Parameter LIKE is null, then this parameter determines how
         the pixels are selected, and can take the values \texttt{"Mask"},
         \texttt{"List"} or \texttt{"ARD"} (see Parameters MASK, LIST, and
         ARD).  \texttt{["Mask"]}
      }
      \sstsubsection{
         XNAME = LITERAL (Read)
      }{
         If an NDF already contains any quality name definitions then
         new quality names are put in the same extension as the old
         names.  If no previous quality names have been stored in the
         NDF then Parameter XNAME will be used to obtain the name of an
         NDF extension in which to store the new quality name.  The
         extension will be created if it does not already exist (see
         Parameter XTYPE).  \texttt{[QUALITY\_NAMES]}
      }
      \sstsubsection{
         XTYPE = LITERAL (Read)
      }{
         If a new NDF extension is created to hold quality names (see
         Parameter XNAME), then Parameter XTYPE is used to obtain the
         HDS data type for the created extension.  The run time default
         is to give the extension a type identical to its name.  \texttt{[]}
      }
   }
   \sstexamples{
      \sstexamplesubsection{
         setqual m51 saturated "Saturated pixels" m51\_cut
      }{
         This example ensures that the quality \texttt{"SATURATED"} is defined
         within the NDF m51.  The comment \texttt{"Saturated pixels"} is stored
         with the quality name if it did not already exist in the NDF.
         The quality SATURATED is then assigned to all pixels for which
         the corresponding pixel in NDF m51\_CUT is bad.  The quality of
         all other pixels is left unchanged.
      }
      \sstexamplesubsection{
         setqual "m51,cena" source\_a select=list list=$\wedge$source\_a.lis function=hs$+$nu
      }{
         This example ensures that pixels within the two NDFs m51 and
         cena which are included in the list of pixel indices held in
         text file \texttt{source\_a.lis}, have the quality \texttt{"SOURCE\_A"}, and also
         ensures that none of the pixels which were not included in
         \texttt{source\_a.lis} have the quality.
      }
      \sstexamplesubsection{
         setqual m51 source\_b select=ard ardfile=background.ard
      }{
         This example assigns the quality \texttt{"source\_b"} to pixels of the
         NDF m51 as described by an ARD description stored in the text
         file \texttt{"background.ard"}.  This text file could for instance have
         been created using routine ARDGEN.
      }
   }
   \sstnotes{
      \sstitemlist{

         \sstitem
         All the quality names which are currently defined within an
         NDF can be listed by application SHOWQUAL.  Quality name
         definitions can be removed from an NDF using application REMQUAL.
         If there is no room for any more quality names to be added to the
         NDF then REMQUAL can be used to remove a quality name in order to
         make room for the new quality names.
      }
   }
   \sstdiytopic{
      Related Applications
   }{
KAPPA: \htmlref{QUALTOBAD}{QUALTOBAD},
\htmlref{REMQUAL}{REMQUAL},
\htmlref{SHOWQUAL}{SHOWQUAL}.
   }
}
\sstroutine{
   SETSKY
}{
   Stores new  WCS information within an NDF
}{
   \sstdescription{
      This application adds \htmlref{WCS}{se:wcsuse}~ information describing
      a celestial sky co-ordinate system to a two-dimensional \NDFref{NDF}.  This
      information can be stored either in the form of a standard NDF
      \htmlref{WCS component}{apndf:wcs}, or
      in the form of an \emph{IRAS90 astrometry structure}~ (see Parameter
      IRAS90).

      The astrometry is determined either by you supplying explicit
      values for certain projection parameters, or by you providing the
      sky and corresponding image co-ordinates for a set of positions
      (see Parameter POSITIONS).  In the latter case, the projection
      parameters are determined automatically by searching through
      parameter space in order to minimise the sum of the squared
      residuals between the supplied \htmlref{pixel co-ordinates}{se:pixgrd}~ and
      the transformed sky co-ordinates.  You may force particular
      projection parameters to take certain values by assigning an
      explicit value to the corresponding application parameter listed
      below.  The individual residuals at each position can be written
      out to a logfile so that you can identify any aberrant points.
      The RMS residual (in pixels) implied by the best-fitting
      parameters is displayed.
   }
   \sstusage{
      setsky ndf positions coords epoch [projtype] [lon] [lat]
             [refcode] \latex{\goodbreak} [pixelsize] [orient] [tilt] [logfile]
   }
   \sstparameters{
      \sstsubsection{
         COORDS = LITERAL (Read)
      }{
         The sky co-ordinate system to use.  Valid values include
         \texttt{"Ecliptic"} (IAU 1980), \texttt{"Equatorial"} (FK4 and FK5), and
         \texttt{"Galactic"} (IAU 1958).  Ecliptic and equatorial co-ordinates
         are referred to the mean equinox of a given epoch.  This epoch
         is specified by appending it to the system name, in
         parentheses, for example, \texttt{"Equatorial(1994.5)"}.  The epoch may
         be preceded by a single character, \texttt{"B"} or \texttt{"J"}, indicating that
         the epoch is Besselian or Julian respectively.  If this letter
         is missing, a Besselian epoch is assumed if the epoch is less
         than 1984.0, and a Julian epoch is assumed otherwise.
      }
      \sstsubsection{
         EPOCH = \_DOUBLE (Read)
      }{
         The Julian epoch at which the observation was made (\emph{e.g.}
         \texttt{"1994.0"}).
      }
      \sstsubsection{
         IRAS90 = \_LOGICAL (Read)
      }{
         If a \texttt{TRUE} value is supplied, then the WCS information will be
         stored in the form of an IRAS90 astrometry structure.  This is the
         form used by the IRAS90 package (see \xref{SUN/163}{sun163}{}).  In this case, any
         existing IRAS90 astrometry structure will be over-written.  See
         the \htmlref{``Notes''}{notes:setsky} section below for warnings about using this form.

         If a \texttt{FALSE} value is supplied, then the WCS information will be
         stored in the form of a standard NDF WCS component which will be
         recognized, used and updated correctly by most other Starlink
         software.

         If a null value (\texttt{{!}}) is supplied, then a \texttt{TRUE} value will be used
         if the supplied NDF already has an IRAS90 extension.  Otherwise a
         \texttt{FALSE} value will be used.  \texttt{[!]}
      }
      \sstsubsection{
         LAT = LITERAL (Read)
      }{
         The latitude of the reference point, in the co-ordinate system
         specified by Parameter COORDS.  For example, if COORDS is
         \texttt{"Equatorial"}, LAT is the declination.  See
         \xref{SUN/163}{sun163}{SEC:COF}, Section
         4.7.2 for full details of the allowed syntax for specifying
         this position.  For convenience here are some examples how you
         may specify the declination -45 degrees, 12 arcminutes: \texttt{"-45 12 00"},
         \texttt{"-45 12"}, \texttt{"-45d 12m"}, \texttt{"-45.2d"}, \texttt{"-451200"}, \texttt{"-0.78888r"}.
         The last of these is a radians value.  A null value causes the
         latitude of the reference point to be estimated automatically
         from the data supplied for Parameter POSITIONS.  \texttt{[!]}
      }
      \sstsubsection{
         LOGFILE = FILENAME (Read)
      }{
         Name of the text file to log the final projection parameter
         values and the residual at each supplied position.  If null,
         there will be no logging.  This parameter is ignored if a null
         value is given to Parameter POSITIONS.  \texttt{[!]}
      }
      \sstsubsection{
         LON= LITERAL (Read)
      }{
         The longitude of the reference point, in the co-ordinate
         system specified by Parameter COORDS.  For example, if COORDS
         is \texttt{"Equatorial"}, LON is the right ascension.  See
         \xref{SUN/163}{sun163}{SEC:COF},
         Section 4.7.2 for full details of the allowed syntax for
         specifying this position.  For convenience here are some
         examples how you may specify the right ascension 11 hours, 34
         minutes, and 56.2 seconds: \texttt{"11 34 56.2"}, \texttt{"11h 34m 56.2s"}, \texttt{"11
         34.9366"}, \texttt{"11.58228"}, \texttt{"113456.2"}.  See Parameter LAT for
         examples of specifying a non-equatorial longitude.  A null
         value causes the longitude of the reference point to be
         estimated automatically from the data supplied for Parameter
         POSITIONS.  \texttt{[!]}
      }
      \sstsubsection{
         NDF = NDF (Read and Write)
      }{
         The NDF in which to store the WCS information.
      }
      \sstsubsection{
         ORIENT = LITERAL (Read)
      }{
         The position angle of the NDF's \textit{y} axis on the celestial
         sphere, measured from north through east.  North is defined as
         the direction of increasing sky latitude, and east is the
         direction of increasing sky longitude.  Values are constrained
         to the range 0 to two-pi radians.  A null value causes the
         position angle to be estimated automatically from the data
         supplied for Parameter POSITIONS.  \texttt{[!]}
      }
      \sstsubsection{
         PIXELREF( 2 ) = REAL (Read)
      }{
         The pixel co-ordinates of the reference pixel (x then y).
         This parameter is ignored unless REFCODE=\texttt{"Pixel"}.  Remember
         that the centre of a pixel at indices \textit{i},\textit{j} is
         $(i-0.5,j-0.5)$.  A
         null value causes the pixel co-ordinates of the reference
         point to be estimated automatically from the data supplied for
         Parameter POSITIONS.  \texttt{[!]}
      }
      \sstsubsection{
         PIXELSIZE( 2 ) = \_REAL (Read)
      }{
         The \textit{x} and \textit{y} pixel sizes at the reference position.  If only
         one value is given, the pixel is deemed to be square.  Values
         may be given in a variety of units (see Parameter LAT).  For
         example, 0.54 arcseconds could be specified as \texttt{"0.54s"} or
         \texttt{"0.009m"} or \texttt{"2.618E-6r"}.  A null value causes the pixel
         dimensions to be estimated automatically from the data
         supplied for Parameter POSITIONS.  \texttt{[!]}
      }
      \sstsubsection{
         POSITIONS = LITERAL (Read)
      }{
         A list of sky co-ordinates and corresponding image
         co-ordinates for the set of positions which are to be used to
         determine the astrometry.  If a null value is given then the
         astrometry is determined by the explicit values you supply for
         each of the other parameters.  Each position is defined by
         four values, the sky longitude (in the same format as for
         Parameter LON), the sky latitude (in the same format as for
         Parameter LAT), the image pixel \textit{x} co-ordinate and the image
         pixel \textit{y} co-ordinate (both decimal values).  These should be
         supplied (in the order stated) for each position.  These
         values are given in the form of a `group expression' (see
         SUN/150).  This means that values can be either typed in
         directly or supplied in a text file.  If typed in directly,
         the items in the list should be separated by commas, and you
         are re-prompted for further values if the last supplied value
         ends in a minus sign.  If conveyed in a text file, they should
         again be separated by commas, but can be split across lines.
         The name of the text file is given in response to the prompt,
         preceded by an `up arrow' symbol ($\wedge$).
      }
      \sstsubsection{
         PROJTYPE = \htmlref{LITERAL}{se:parmenu} (Read)
      }{
         The type of projection to use.  The options are:
            \texttt{"Aitoff"}         --- Aitoff equal-area,
            \texttt{"Gnomonic"}       --- Gnomonic (\emph{i.e.} tangent plane),
            \texttt{"Lambert"}        --- Lambert normal equivalent cylindrical,
            \texttt{"Orthographic"}   --- Orthographic.

         The following synonyms are also recognised:
             \texttt{"All\_sky"}      --- Aitoff,
             \texttt{"Cylindrical"}   --- Lambert,
             \texttt{"Tangent\_plane"} --- Gnomonic.

         See \xref{SUN/163}{sun163}{SEC:PROJ} for descriptions of these projections.  A null
         value causes the projection to be determined automatically
         from the data supplied for Parameter POSITIONS.  \texttt{[!]}
      }
      \sstsubsection{
         REFCODE = LITERAL (Read)
      }{
         The code for the reference pixel.  If it has value \texttt{"Pixel"}
         this requests that pixel co-ordinates for the reference point
         be obtained through Parameter PIXELREF.  The other options are
         locations specified by two characters, the first corresponding
         to the vertical position and the second the horizontal.  For
         the vertical, valid positions are \texttt{T}(op), \texttt{B}(ottom), or
         \texttt{C}(entre); and for the horizontal the options are \texttt{L}(eft),
         \texttt{R}(ight), or \texttt{C}(entre).  Thus REFCODE=\texttt{"CC"} means the reference
         position is at the centre of the NDF image, and \texttt{"BL"} specifies
         that the reference position is at the centre of the
         bottom-left pixel in the image.  A null value causes the pixel
         co-ordinates of the reference point to be estimated
         automatically from the data supplied for Parameter POSITIONS.
         \texttt{[!]}
      }
      \sstsubsection{
         TILT = LITERAL (Read)
      }{
         The angle through which the celestial sphere is to be rotated
         prior to doing the projection.  The axis of rotation is a
         radius passing through the reference point.  The rotation is
         in an anti-clockwise sense when looking from the reference
         point towards the centre of the celestial sphere.  In common
         circumstances this can be set to zero.  Values may be given in
         a variety of units (see Parameter LAT).  Values are
         constrained to the range 0 to two-pi radians.  A null value
         causes the latitude of the reference point to be estimated
         automatically from the data supplied for Parameter POSITIONS.
         \texttt{["0.0"]}
      }
   }
   \sstexamples{
      \sstexamplesubsection{
         setsky m51 $\wedge$stars.lis ecl(j1994.0) 1994.0 logfile=m51.log
      }{
         This creates a WCS component to a two-dimensional NDF called
         m51.  The values for Parameters PROJTYPE, LON, LAT, PIXELREF,
         PIXELSIZE, and ORIENT are determined automatically so
         that they minimised the sum of the squared residuals (in
         pixels) at each of the positions specified in the file
         \texttt{stars.lis}.  This file contains a line for each position, each
         line containing an ecliptic longitude and latitude, followed
         by a pair of image co-ordinates.  These values should be
         separated by commas.  The ecliptic co-ordinates were
         determined at Julian epoch 1994.0, and are referred to the
         mean equinox at Julian epoch 1994.0.  The determined parameter
         values together with the residual at each position are logged
         to file \texttt{m51.log}.
      }
      \sstexamplesubsection{
         setsky m51 $\wedge$stars.lis ecl(j1994.0) 1994.0 orient=0 projtype=orth
      }{
         This creates a WCS component within the two-dimensional NDF
         called m51.  The values for Parameters PROJTYPE, LON, LAT,
         PIXELREF, and PIXELSIZE are determined automatically as in
         the previous example.  In this example however, an Orthographic
         projection is forced, and the value zero is assigned to
         Parameter ORIENT, resulting in north being `upwards' in the image.
      }
      \sstexamplesubsection{
         setsky virgo "!" eq(j2000.0) 1989.3 gn "12 29" "$+$12 30" bl 1.1s 0.0d
      }{
         This creates a WCS component within the two-dimensional
         NDF called virgo.  It is a gnomonic projection in the
         equatorial system at Julian epoch 2000.0.  The bottom-left
         pixel of the image is located at right ascension 12 hours 29
         minutes, declination $+$12 degrees 30 minutes.  A pixel at that
         position is square and has angular size of 1.1 arcseconds.
         The image was observed at epoch 1989.3.  At the bottom-left of
         the image, north is at the top, parallel to the \textit{y}-axis of the
         image.
      }
      \sstexamplesubsection{
         setsky map "!" galactic(1950.0) 1993.8 aitoff 90 0 cc [0.5d,0.007r] 180.0d
      }{
          This creates a WCS component within the two-dimensional
          NDF called map.  It is an Aitoff projection in the galactic
          system at Besselian epoch 1950.0.  The centre of the image is
          located at galactic longitude 90 degrees, latitude 0 degrees.
          A pixel at that position is rectangular and has angular size
          of 0.5 degrees by 0.007 radians.  The image was made at epoch
          1993.8.  At the image centre, south is at the top and is
          parallel to the \textit{y}-axis of the image.
      }
      \sstexamplesubsection{
         setsky zodiac "!" ec 1983.4 or 10.3 -5.6 Pixel 20m 0.3d pixelref=[9.5,-11.2] IRAS90=YES
      }{
          This creates an IRAS90 astrometry extension within the
          two-dimensional NDF called zodiac.  It is an orthographic
          projection in the Ecliptic system at Besselian epoch 1950.0.
          The reference point at pixel co-ordinates (9.5,~$-$11.2)
          corresponds to ecliptic longitude 10.3 degrees, latitude
          $-$5.6 degrees.  A pixel at that position is square and has
          angular size of 20 arcminutes.  The image was observed at
          epoch 1983.4.  At the reference point the \textit{y}-axis of the image
          points to 0.3 degrees east of north.
      }
   }
   \label{notes:setsky}
   \sstnotes{
      \sstitemlist{

         \sstitem
         The \GAIA\ image display tool (\xref{SUN/214}{sun214}{}) provides various interactive
         tools for storing new WCS information within an NDF.

         \sstitem
         This application was written to supply the limited range of WCS
         functions required by the IRAS90 package.  For instance, it does not
         support the complete range or projections or sky co-ordinate systems
         which may be represented by the more general NDF WCS component.

         \sstitem
         If WCS information is stored in the form of an IRAS90 astrometry
         structure (see Parameter IRAS90), it will in general be invalidated
         by any subsequent \KAPPA\ commands which modify the transformation
         between sky and pixel co-ordinates.  For instance, if the image is
         moved using \htmlref{SLIDE}{SLIDE} (for example), then the IRAS90 astrometry
         structure will no longer correctly describe the sky co-ordinates
         associated with each pixel.  For this reason (amongst others) it is
         better to set Parameter IRAS90 to \texttt{FALSE}.
      }
   }
   \sstdiytopic{
      Related Applications
   }{
\xref{ASTROM}{sun5}{};
\xref{IRAS90}{sun163}{}: \xref{SKYALIGN}{sun163}{SKYALIGN},
\xref{SKYBOX}{sun163}{SKYBOX},
\xref{SKYGRID}{sun163}{SKYGRID},
\xref{SKYLINE}{sun163}{SKYLINE},
\xref{SKYMARK}{sun163}{SKYMARK},
\xref{SKYPOS}{sun163}{SKYPOS},
\xref{SKYWRITE}{sun163}{SKYWRITE}.
   }
}
\sstroutine{
   SETTITLE
}{
   Sets a new title for an NDF data structure
}{
   \sstdescription{
      This routine sets a new value for the \htmlref{TITLE component}{apndf:title}~ of
      an existing \NDFref{NDF} data structure.  The NDF is accessed in update mode
      and any pre-existing title is over-written with a new value.
      Alternatively, if a `null' value (\texttt{{!}}) is given for the TITLE
      parameter, then the NDF's title will be erased.
   }
   \sstusage{
      settitle ndf title
   }
   \sstparameters{
      \sstsubsection{
         NDF = NDF (Read and Write)
      }{
         The NDF data structure whose title is to be modified.
      }
      \sstsubsection{
         TITLE = LITERAL (Read)
      }{
         The value to be assigned to the NDF's TITLE component (\emph{e.g.}
         \texttt{"NGC1068 with a B filter"} or \texttt{"Ice band in HD123456"}).  This
         value may later be used by other applications as a heading for
         graphs and other forms of display where the NDF's data values
         are plotted.  The suggested default is the current value.
      }
   }
   \sstexamples{
      \sstexamplesubsection{
         settitle ngc1068 "NGC1068 with a B filter"
      }{
         Sets the TITLE component of the NDF structure ngc1068 to be
         \texttt{"NGC1068 with a B filter"}.
      }
      \sstexamplesubsection{
         settitle ndf=myspec title="Ice band, short integration"
      }{
         Sets the TITLE component of the NDF structure myspec to be
         \texttt{"Ice band, short integration"}.
      }
      \sstexamplesubsection{
         settitle dat123 title=!
      }{
         By specifying a null value (\texttt{{!}}), this example erases any
         previous value of the TITLE component in the NDF structure
         dat123.
      }
   }
   \sstdiytopic{
      Related Applications
   }{
KAPPA: \htmlref{SETLABEL}{SETLABEL},
\htmlref{SETUNITS}{SETUNITS}.
   }
}
\sstroutine{
   SETTYPE
}{
   Sets a new numeric type for the DATA and VARIANCE components of
   an NDF
}{
   \sstdescription{
      This application allows the \htmlref{numeric type}{ap:HDStypes}~ of
      the DATA and VARIANCE components of an \NDFref{NDF} to be
      changed.  The NDF is accessed in update mode and the values stored
      in these components are converted in situ to the new type.  No other
      attributes of the NDF are changed.
   }
   \sstusage{
      settype ndf type
   }
   \sstparameters{
      \sstsubsection{
         COMPLEX = \_LOGICAL (Read)
      }{
         If a \texttt{TRUE} value is given for this parameter, then the NDF's
         array components will be altered so that they hold complex
         values, an imaginary part containing zeros being created if
         necessary.  If a \texttt{FALSE} value is given, then the components will
         be altered so that they hold non-complex values, any imaginary
         part being deleted if necessary.  If a null (\texttt{{!}}) value is supplied,
         the value used is chosen so that no change is made to the current
         state.  \texttt{[!]}
      }
      \sstsubsection{
         DATA = \_LOGICAL (Read)
      }{
         If a \texttt{TRUE} value is given for this parameter, then the numeric
         type of the NDF's data array will be changed.  Otherwise, this
         component's type will remain unchanged.  \texttt{[TRUE]}
      }
      \sstsubsection{
         NDF = NDF (Read and Write)
      }{
         The NDF data structure whose array components are to have
         their numeric type changed.
      }
      \sstsubsection{
         TYPE = \htmlref{LITERAL}{se:parmenu} (Read)
      }{
         The new numeric type to which the NDF's array components are
         to be converted.  The value given should be one of the
         following: \_DOUBLE, \_REAL, \_INTEGER, \_WORD, \_UWORD, \_BYTE or
         \_UBYTE (note the leading underscore).  Existing pixel values
         stored in the NDF will not be lost, but will be converted to
         the new type.  Any values which cannot be represented using the
         new type will be replaced with the
         \htmlref{bad-pixel}{se:masking}~ value.
      }
      \sstsubsection{
         VARIANCE = \_LOGICAL (Read)
      }{
         If a \texttt{TRUE} value is given for this parameter, then the numeric
         type of the NDF's \htmlref{VARIANCE}{apndf:variance}~ array will
         be changed.  Otherwise, this component's type will remain
         unchanged.  \texttt{[TRUE]}
      }
   }
   \sstexamples{
      \sstexamplesubsection{
         settype rawdata \_real
      }{
         Converts the data and variance values held in the NDF data
         structure rawdata to have a numeric type of \_REAL (\emph{i.e.} to be
         stored as single-precision floating-point numbers).
      }
      \sstexamplesubsection{
         settype inst.run1 \_word novariance
      }{
         Converts the data array in the NDF structure inst.run1 to be
         stored as word (\emph{i.e.} Fortran INTEGER$*$2) values.  No change is
         made to the VARIANCE component.
      }
      \sstexamplesubsection{
         settype hd26571 \_double complex
      }{
         Causes the DATA and VARIANCE components of the NDF structure
         hd26571 to be altered so as to hold complex values using
         double precision numbers.  The existing pixel values are
         converted to this new type.
      }
   }
   \sstdiytopic{
      Timing
   }{
      The execution time is approximately proportional to the number of
      pixel values to be converted.
   }
   \sstdiytopic{
      Related Applications
   }{
\xref{FIGARO}{sun86}{}: \xref{RETYPE}{sun86}{RETYPE}.
   }
}
\sstroutine{
   SETUNITS
}{
   Sets a new units value for an NDF data structure
}{
   \sstdescription{
      This routine sets a new value for the \htmlref{UNITS component}{apndf:units}~ of an
      existing \NDFref{NDF} data structure.  The NDF is accessed in update mode
      and any pre-existing UNITS component is over-written with a new
      value.  Alternatively, if a `null' value (\texttt{{!}}) is given for the
      UNITS parameter, then the NDF's UNITS component will be erased.

      There is also an option to modify the pixel values within the NDF
      to reflect the change in units (see Parameter MODIFY).
   }
   \sstusage{
      setunits ndf units
   }
   \sstparameters{
      \sstsubsection{
         NDF = NDF (Read and Write)
      }{
         The NDF data structure whose UNITS component is to be
         modified.
      }
      \sstsubsection{
         MODIFY = \_LOGICAL (Read)
      }{
         If a \texttt{TRUE} value is supplied, then the pixel values in the DATA and
         \htmlref{VARIANCE}{apndf:variance}~ components of the NDF will be modified to reflect the
         change in units.  For this to be possible, both the original
         Units value in the NDF and the new Units value must both correspond
         to the format for units strings described in the FITS WCS standard
         (see \texttt{"Representations of world coordinates in FITS"}, Greisen \&
         Calabretta, 2002, A\&A---available at
         (\htmladdnormallink{http://www.aoc.nrao.edu/$\sim$egreisen/wcs\_AA.ps.gz}{http://www.aoc.nrao.edu/$\sim$egreisen/wcs\_AA.ps.gz})
         If either of the two units strings are not of this form, or if it is
         not possible to find a transformation between them (for instance,
         because they represent different quantities), an error is
         reported.  \texttt{[FALSE]}
      }
      \sstsubsection{
         UNITS = LITERAL (Read)
      }{
         The value to be assigned to the NDF's UNITS component (\emph{e.g.}
         \texttt{"J/(m$*$$*$2$*$Angstrom$*$s)"} or \texttt{"count/s"}).  This value may later be used
         by other applications for labelling graphs and other forms of
         display where the NDF's data values are shown.  The suggested
         default is the current value.
      }
   }
   \sstexamples{
      \sstexamplesubsection{
         setunits ngc1342 "count/s"
      }{
         Sets the UNITS component of the NDF structure ngc1342 to have
         the value \texttt{"count/s"}.  The pixel values are not changed.
      }
      \sstexamplesubsection{
         setunits ndf=spect units="J/(m$*$$*$2$*$Angstrom$*$s)"
      }{
         Sets the UNITS component of the NDF structure spect to have
         the value \texttt{"J/(m$*$$*$2$*$Angstrom$*$s)"}.  The pixel values are not changed.
      }
      \sstexamplesubsection{
         setunits datafile units=!
      }{
         By specifying a null value (\texttt{{!}}), this example erases any
         previous value of the UNITS component in the NDF structure
         datafile.  The pixel values are not changed.
      }
      \sstexamplesubsection{
         setunits ndf=spect units="MJy" modify
      }{
         Sets the UNITS component of the NDF structure spect to have
         the value \texttt{"MJy"}.  If possible, the pixel values are changed from
         their old units to the new units.  For instance, if the UNITS
         component of the NDF was original \texttt{"J/(m$*$$*$2$*$s$*$GHz)"}, the DATA
         values will be multiplied by 1.0E11, and the variance values by
         1.0E22.  However, if the original UNITS component was (say) \texttt{"K"}
         (Kelvin) then an error would be reported since there is no
         direct conversion from Kelvin to Megajansky.
      }
   }
   \sstdiytopic{
      Related Applications
   }{
KAPPA: \htmlref{AXUNITS}{AXUNITS},
\htmlref{SETLABEL}{SETLABEL},
\htmlref{SETTITLE}{SETTITLE}.
   }
}
\sstroutine{
   SETVAR
}{
   Sets new values for the VARIANCE component of an NDF data
   structure
}{
   \sstdescription{
      This routine sets new values for the
      \htmlref{VARIANCE}{apndf:variance}~ component of an \NDFref{NDF}
      data structure.  The new values can be copied from a specified
      component of a second NDF or can be generated from the supplied NDF's
      data array by means of a Fortran-like arithmetic expression.  Any
      previous variance information is over-written with the new values.
      Alternatively, if a `null' value (\texttt{{!}}) is given for the
      variance, then any pre-existing variance information is erased.
   }
   \sstusage{
      setvar ndf variance
   }
   \sstparameters{
      \sstsubsection{
         COMP = \htmlref{LITERAL}{se:parmenu} (Read)
      }{
         The name of an NDF array component within the NDF specified by
         Parameter FROM.  The values in this array component are used as
         the new variance values to be stored in the VARIANCE component
         of the NDF specified by Parameter NDF.  The supplied value must
         be one of \texttt{"Data"} or \texttt{"Variance"}.  \texttt{["Data"]}
      }
      \sstsubsection{
         FROM = NDF (Read)
      }{
         An NDF data structure containing the values to be used as the
         new variance values.  The NDF component from which to read the
         new variance values is specified by Parameter COMP.  If NDF is
         not contained completely within FROM, then the VARIANCE
         component of NDF will be padded with bad values.  If a null
         (\texttt{{!}}) value is supplied, the new variance values are
         determined by the expression given for Parameter VARIANCE.  \texttt{[!]}
      }
      \sstsubsection{
         NDF = NDF (Read and Write)
      }{
         The NDF data structure whose variance values are to be
         modified.
      }
      \sstsubsection{
         VARIANCE = LITERAL (Read)
      }{
         A Fortran-like arithmetic expression giving the variance value
         to be assigned to each pixel in terms of the variable DATA,
         which represents the value of the corresponding data array
         pixel.  For example, VARIANCE=\texttt{"DATA"} implies normal
         $\surd N$ error estimates, whereas VARIANCE=\texttt{"DATA + 50.7"}
         might be used if a sky background of 50.7 units had previously
         been subtracted.

         If a `null' value (\texttt{{!}}) is given for this parameter, then
         no new VARIANCE component will be created and any pre-existing
         variance values will be erased.
      }
   }
   \sstexamples{
      \sstexamplesubsection{
         setvar ngc4709 data
      }{
         This sets the VARIANCE component within the NDF structure
         ngc4709 to equal its corresponding data-array component.
      }
      \sstexamplesubsection{
         setvar ndf=arcspec "data - 0.31"
      }{
         This sets the VARIANCE component within the NDF structure
         arcspec to be its corresponding data-array component less a
         constant 0.31.
      }
      \sstexamplesubsection{
         setvar cube4 Variance=!
      }{
         This erases the values of the VARIANCE component within
         the NDF structure cube4, if it exists.
      }
   }
   \sstnotes{
      \sstitemlist{

         \sstitem
         All of the standard Fortran 77 intrinsic functions are
         available for use in the variance expression, plus a few others
         (see \xref{SUN/61}{sun61}{} for details and an up-to-date list).

         \sstitem
         Calculations are performed using real arithmetic (or double
         precision if appropriate) and are constrained to be non-negative.

         \sstitem
         The data type of the VARIANCE component is set to match that of
         the DATA component.
      }
   }
   \sstdiytopic{
      Related Applications
   }{
KAPPA: \htmlref{ERRCLIP}{ERRCLIP};
\xref{FIGARO}{sun86}{}: \xref{GOODVAR}{sun86}{GOODVAR}.
   }
}
\pagebreak
\sstroutine{
   SHADOW
}{
   Enhances edges in a two-dimensional NDF using a shadow effect
}{
   \sstdescription{
      This routine enhances a two-dimensional \NDFref{NDF} by creating a
      bas-relief or shadow effect, that causes features in an array to
      appear as though they have been illuminated from the side by some
      imaginary light source.  The enhancement is useful in locating
      edges and fine detail in an array.
   }
   \sstusage{
      shadow in out
   }
   \sstparameters{
      \sstsubsection{
         IN = NDF (Read)
      }{
         The two-dimensional NDF to be enhanced.
      }
      \sstsubsection{
         OUT = NDF (Write)
      }{
         The output NDF containing the enhanced image.
      }
      \sstsubsection{
         SHIFT( 2 ) = \_INTEGER (Read)
      }{
         The shift in \textit{x} and \textit{y} \htmlref{pixel
         indices}{se:pixgrd}~ to be used in the
         enhancement.  If the \textit{x} shift is positive, positive features
         in the original array will appear to be lit from the positive
         \textit{x} direction, \emph{i.e.} from the right.  Similarly, if the \textit{y} shift
         is positive, the light source will appear to be shining from
         the top of the array.  A one- or two-pixel shift is normally
         adequate.  \texttt{[1,1]}
      }
      \sstsubsection{
         TITLE = LITERAL (Read)
      }{
         The \htmlref{title}{apndf:title}~ for the output NDF.  A null value will cause
         the title of the NDF supplied for Parameter IN to be used
         instead.  \texttt{[!]}
      }
   }
   \sstexamples{
      \sstexamplesubsection{
         shadow horse horse\_bas
      }{
         This enhances the NDF called horse by making it appear to be
         illuminated from the top-right, and stores the result in the
         NDF called horse\_bas.
      }
      \sstexamplesubsection{
         shadow out=aash in=aa [-1,-1] title="Bas relief"
      }{
         This enhances the NDF called aa by making it appear to be
         illuminated from the bottom left, and stores the result in the
         NDF called aash, which has the title \texttt{"Bas relief"}.
      }
   }
   \sstdiytopic{
      Related Applications
   }{
KAPPA: \htmlref{LAPLACE}{LAPLACE},
\htmlref{MEDIAN}{MEDIAN};
\xref{FIGARO}{sun86}{}: \xref{ICONV3}{sun86}{ICONV3}.
   }
   \sstimplementationstatus{
      \sstitemlist{

         \sstitem
         This routine correctly processes the \htmlref{AXIS}{apndf:axis}, DATA, \htmlref{QUALITY}{apndf:quality},
         \htmlref{VARIANCE}{apndf:variance}, \htmlref{LABEL}{apndf:label}, \htmlref{TITLE}{apndf:title}, \htmlref{UNITS}{apndf:units}, \htmlref{WCS}{apndf:wcs}, and \htmlref{HISTORY}{apndf:history}~ components of an
         NDF data structure and propagates all \htmlref{extensions}{apndf:extensions}.

         \sstitem
         Processing of \htmlref{bad pixels}{se:masking} and automatic \htmlref{quality masking}{se:qualitymask} are
         supported.

         \sstitem
         All \htmlref{non-complex numeric data types}{ap:HDStypes} can be handled.

         \sstitem
         The output NDF will be trimmed compared with the input NDF
         by the shifts applied.
      }
   }
}



\sstroutine{
   SHOWQUAL
}{
   Display the quality names defined in an NDF
}{
   \sstdescription{
      This routine displays a list of all the \htmlref{quality
      names}{se:qnames}~ currently
      defined within a supplied \NDFref{NDF} (see task SETQUAL).  The descriptive
      comments which were stored with the quality names when they were
      originally defined are also displayed.  An option exists for also
      displaying the number of pixels which hold each quality.
   }
   \sstusage{
      showqual ndf [count]
   }
   \sstparameters{
      \sstsubsection{
         COUNT = \_LOGICAL (Read)
      }{
         If \texttt{TRUE}, then the number of pixels in each NDF which holds
         each defined quality is displayed.  These figures are shown
         in parentheses between the quality name and associated
         comment.  This option adds significantly to the run time.  \texttt{[NO]}
      }
      \sstsubsection{
         NDF = NDF (Read)
      }{
         The NDF whose quality names are to be listed.
      }
   }
   \sstresparameters{
      \sstsubsection{
         QNAMES( ) = LITERAL (Write)
      }{
         The quality names associated with each bit, starting from the
         lowest significant bit.  Unassigned bits have blank strings.
      }
   }
   \sstexamples{
      \sstexamplesubsection{
         showqual "m51,cena" yes
      }{
         This example displays all the quality names currently defined
         for the two NDFs m51 and cena together with the number of
         pixels holding each quality.
      }
   }
   \sstdiytopic{
      Related Applications
   }{
KAPPA: \htmlref{REMQUAL}{REMQUAL},
\htmlref{QUALTOBAD}{QUALTOBAD},
\htmlref{SETQUAL}{SETQUAL}.
   }
}

\sstroutine{
   SLIDE
}{
   Realigns an NDF using a translation
}{
   \sstdescription{
      The pixels of an \NDFref{NDF} are shifted by a given number of pixels along
      each pixel axis.  The shift need not be an integer number of pixels,
      and pixel interpolation will be performed if necessary using the
      scheme selected by Parameter METHOD.  The shifts to use are specified
      either by an absolute vector given by the ABS parameter or by the
      difference between a fiducial point and a standard object given by
      the FID and OBJ parameters respectively.  In each case the co-ordinates
      are specified in the NDF's pixel \htmlref{co-ordinate Frame}{se:domains}.
   }
   \sstusage{
      slide in out abs method
   }
   \sstparameters{
      \sstsubsection{
         ABS( ) = \_DOUBLE (Read)
      }{
         Absolute shifts in pixels.  The number of values supplied must
         match the number of pixel axes in the NDF.  It is only used if
         STYPE=\texttt{"Absolute"}.
      }
      \sstsubsection{
         FID( ) = \_DOUBLE (Read)
      }{
         Position of the fiducial point in pixel co-ordinates.  The number
         of values supplied must match the number of pixel axes in the NDF.
         It is only used if STYPE=\texttt{"Relative"}.

         An object centred at the pixel co-ordinates given by Parameter
         OBJ in the input NDF will be centred at the pixel co-ordinates
         given by Parameter FID in the output NDF.
      }
      \sstsubsection{
         IN = NDF (Read)
      }{
         The NDF to be translated.
      }
      \sstsubsection{
         METHOD = \htmlref{LITERAL}{se:parmenu} (Read)
      }{
         The interpolation method used to perform the translation.
         The following values are permitted:

         \ssthitemlist{

            \sstitem
            \texttt{"Nearest"}   --- Nearest-neighbour sampling.

            \sstitem
            \texttt{"Linear"}    --- Linear interpolation.

            \sstitem
            \texttt{"Sinc"}      --- Sum of surrounding pixels weighted using
                              a one-dimensional ${\textrm{sinc}}({\pi}x)$ kernel.

            \sstitem
            \texttt{"SincSinc"}  --- Sum of surrounding pixels weighted using
                              a one-dimensional
                              ${\textrm{sinc}}({\pi}x){\textrm{sinc}}(k{\pi}x)$ kernel.

            \sstitem
            \texttt{"SincCos"}   --- Sum of surrounding pixels weighted using
                              a one-dimensional \linebreak
                              ${\textrm{sinc}}({\pi}x)\cos(k{\pi}x)$ kernel.

            \sstitem
            \texttt{"SincGauss"} --- Sum of surrounding pixels weighted using
                              a one-dimensional
                              ${\textrm{sinc}}({\pi}x)e^{-kx^2}$ kernel.

            \sstitem
            \texttt{"BlockAve"}  --- Block averaging over all pixels in the
                              surrounding \textit{n}-dimensional cube.

         }
         In the above, ${\textrm{sinc}}(z)=\sin(z)/z$.  Some of these schemes will
         require additional parameters to be supplied via the PARAMS
         parameter.  A more-detailed discussion of these schemes is
         given in the \htmlref{``Sub-pixel Interpolation Schemes''}{subpixel:sqorst}
         section below.  The initial default is \texttt{"Linear"}.
         \texttt{[}current value\texttt{{]}}
      }
      \sstsubsection{
         OBJ = LITERAL (Read)
      }{
         Position of the standard object in pixel co-ordinates.  The number
         of values supplied must match the number of pixel axes in the NDF.
         It is only used if STYPE=\texttt{"Relative"}.

         An object centred at the pixel co-ordinates given by Parameter
         OBJ in the input NDF will be centred at the pixel co-ordinates
         given by Parameter FID in the output NDF.
      }
      \sstsubsection{
         OUT = NDF (Write)
      }{
         The translated NDF.
      }
      \sstsubsection{
         PARAMS( ) = \_DOUBLE (Read)
      }{
         Parameters required to control the resampling scheme.  One or
         more values may be required to specify the exact resampling
         behaviour, according to the value of the METHOD parameter.
         See the section on \htmlref{``Sub-pixel Interpolation
         Schemes''}{subpixel:sqorst}.
      }
      \sstsubsection{
         STYPE = LITERAL (Read)
      }{
         The sort of shift to be used.  The choice is \texttt{"Relative"} or
         \texttt{"Absolute"}.  \texttt{["Absolute"]}
      }
      \sstsubsection{
         TITLE = LITERAL (Read)
      }{
         Title for the output NDF.  A null (\texttt{{!}}) value will cause the input
         title to be used.  \texttt{[!]}
      }
   }
   \sstexamples{
      \sstexamplesubsection{
         slide m31 m31\_acc [3.2,2.3]
      }{
         The pixels in the NDF m31 are shifted by 3.2 pixels in \textit{x} and
         2.3 pixels in \textit{y}, and written to NDF m31\_acc.  Linear interpolation
         is used to produce the output data (and, if present, variance) array.
      }
      \sstexamplesubsection{
         slide m31 m31\_acc [3.2,2.3] nearest
      }{
         The same as the previous example except that nearest-neighbour
         resampling is used.  This will be somewhat faster, but may
         result in features shifted by up to half a pixel.
      }
      \sstexamplesubsection{
         slide speca specb stype=rel fid=11.2 obj=11.7
      }{
         The pixels in the NDF speca are shifted by 0.5 (\emph{i.e.} $11.7 - 11.2$)
         pixels and the output NDF is written as specb.
      }
      \sstexamplesubsection{
         slide speca specb stype=abs abs=0.5
      }{
         This does just the same as the previous example.
      }
   }
   \sstdiytopic{
      Sub-Pixel Interpolation Schemes
   }{
      When performing the translation the pixels are resampled from
      the input grid to the output grid by default using linear
      interpolation.  For many purposes this default scheme will
      be adequate, but for greater control over the resampling
      process the METHOD and PARAMS parameters can be used.  Detailed
      discussion of the use of these parameters can be found in the
      \xref{``Sub-pixel Interpolation Schemes''}{sun210}{AST_RESAMPLE\$<X>\$}~
      section of the AST\_RESAMPLE documentation.
   }
   \sstnotes{
      \sstitemlist{

         \sstitem
         If the NDF is shifted by a whole number of pixels along each
         axis, this application merely changes the pixel origin in the NDF.
         It can thus be compared to the \htmlref{SETORIGIN}{SETORIGIN} command.

         \sstitem
         Resampled axis centres that are beyond the bounds of the
         input NDF are given extrapolated values from the first (or last)
         pair of valid centres.
      }
   }
   \sstimplementationstatus{
      \sstitemlist{

         \sstitem
         The \htmlref{LABEL}{apndf:label}, \htmlref{UNITS}{apndf:units}, and \htmlref{HISTORY}{apndf:history}~ components, and all
         \htmlref{extensions}{apndf:extensions}~ are propagated.
         \htmlref{TITLE}{apndf:title}~ is controlled by the TITLE parameter.  DATA,
         \htmlref{VARIANCE}{apndf:variance}, \htmlref{AXIS}{apndf:axis}~ and \htmlref{WCS}{apndf:wcs} are propagated after appropriate modification.
         The \htmlref{QUALITY}{apndf:quality}~ component is also
         propagated if nearest-neighbour interpolation is being used.

         \sstitem
         Processing of \htmlref{bad pixels}{se:masking} and automatic \htmlref{quality masking}{se:qualitymask} are
         supported.

         \sstitem
         All \htmlref{non-complex numeric data types}{ap:HDStypes} can be handled.

         \sstitem
         There can be an arbitrary number of NDF dimensions.
      }
   }
   \sstdiytopic{
      Related Applications
   }{
KAPPA: \htmlref{REGRID}{REGRID},
\htmlref{SQORST}{SQORST},
\htmlref{WCSADD}{WCSADD}.
   }
}

\sstroutine{
   SQORST
}{
   Squashes or stretches an NDF
}{
   \sstdescription{
      An output \NDFref{NDF} is produced by squashing or stretching an
      input NDF along one or more of its dimensions.  The shape of the
      output NDF can be specified in one of two ways, according to
      the value of the MODE parameter; either a distortion factor
      is given for each dimension, or its lower and upper pixel
      bounds are given explicitly.
   }
   \sstusage{
      sqorst in out
        $\left\{ {\begin{tabular}{l}
                  factors \\
                  lbound=? ubound=? \\
                  pixscale=?
                  \end{tabular} }
        \right.$
        \newline\latexhtml{\hspace*{6.9em}}{~~~~~~~~~~}
        \makebox[0mm][c]{\small mode}
   }
   \sstparameters{
      \sstsubsection{
         AXIS = \_INTEGER (Read)
      }{
         Assigning a value to this parameter indicates that a single
         axis should be squashed or stretched.  If a null (\texttt{{!}})
         value is supplied for AXIS, a squash or stretch factor must
         be supplied for each axis in the manner indicated by the MODE
         parameter.  If a non-null value is supplied for AXIS, it
         should be the integer index of the axis to be squashed or
         stretched (the first axis has index 1).  In this case, only a
         single squash or stretch factor should be supplied, and all
         other axes will be left unchanged.  If MODE is set to
         \texttt{PixelScale"}, then the supplied value should be the index
         of a WCS axis.  Otherwise it should be the index of a pixel
         axis.  \texttt{[!]}
     }
      \sstsubsection{
         CENTRE = LITERAL (Read)
      }{
         Determines the centre about which the WCS co-ordinates are
         stretching or squashing. The following values are permitted.

         \sstitemlist{

            \sstitem
            \texttt{"Centre"} --- The WCS co-ordinates at the centre of the
                              output NDF are the same as those at the centre
                              of the input NDF.

            \sstitem
            \texttt{"Origin"} --- The WCS co-ordinates at the pixel origin of the
                              output NDF are the same as those at the pixel
                              origin of the input NDF.
            \sstitem
            \texttt{"Edge"}   --- The WCS co-ordinates at the edges of each
                              rebinned pixel in the output NDF is a multiple of
                              any new pixel width along each axis.

            \sstitem
            \texttt{"Middle"} --- The WCS co-ordinates at the centre of each
                              rebinned pixel of the output NDF is a multiple
                              of any new pixel width along each axis.

         }
         \texttt{["Centre"]}
      }
      \sstsubsection{
         CONSERVE = \_LOGICAL (Read)
      }{
         If set \texttt{TRUE}, then the output pixel values will be scaled in
         such a way as to preserve the total data value in a feature on
         the sky. The scaling factor is the ratio of the output pixel
         size to the input pixel size. This ratio is evaluated once for
         each panel of a piece-wise linear approximation to the Mapping,
         and is assumed to be constant for all output pixels in the
         panel. \texttt{[FALSE]}
     }
      \sstsubsection{
         FACTORS( ) = \_DOUBLE (Read)
      }{
         This parameter is used only if MODE=\texttt{"Factors"}.  It
         defines the factor by which each dimension will be distorted
         to produce the output NDF.  A factor greater than one is a
         stretch and less than one is a squash.  If no value has been
         supplied for Parameter AXIS, the number of values supplied
         for FACTORS must be the same as the number of pixel axes in
         the NDF.  If a non-null value has been supplied for Parameter
         AXIS, then only a single value should be supplied for FACTORS
         and that value will be used to distort the axis indicated by
         Parameter AXIS.
      }
      \sstsubsection{
         IN = NDF (Read)
      }{
         The NDF to be squashed or stretched.
      }
      \sstsubsection{
         LBOUND( ) = \_INTEGER (Read)
      }{
         This parameter is only used if MODE=\texttt{"Bounds"}.  It
         specifies the lower pixel-index values of the output NDF.  If
         no value has been supplied for Parameter AXIS, the number of
         values supplied for LBOUND must be the same as the number of
         pixel axes in the NDF.  If a non-null value has been supplied
         for Parameter AXIS, then only a single value should be
         supplied for LBOUND and the supplied value will be used as
         the new lower bounds on the axis indicated by Parameter AXIS.
         If null (\texttt{{!}}) is given, the lower pixel bounds of the
         input NDF will be used.
      }
      \sstsubsection{
         METHOD = \htmlref{LITERAL}{se:parmenu} (Read)
      }{
         The interpolation method used to perform the one-dimensional
         resampling operations which constitute the squash or stretch.
         The following values are permitted.

         \ssthitemlist{

            \sstitem
            \texttt{"Auto"}      --- Equivalent to \texttt{"BlockAve"} with an appropriate
                              PARAMS for squashes by a factor of 2 or more,
                              otherwise equivalent to \texttt{"Linear"}.

            \sstitem
            \texttt{"Nearest"}   --- Nearest-neighbour sampling.

            \sstitem
            \texttt{"Linear"}    --- Linear interpolation.

            \sstitem
            \texttt{"Sinc"}      --- Sum of surrounding pixels weighted using
                              a one-dimensional ${\textrm{sinc}}({\pi}x)$ kernel.

            \sstitem
            \texttt{"SincSinc"}  --- Sum of surrounding pixels weighted using
                              a one-dimensional
                              ${\textrm{sinc}}({\pi}x){\textrm{sinc}}(k{\pi}x)$ kernel.

            \sstitem
            \texttt{"SincCos"}   --- Sum of surrounding pixels weighted using
                              a one-dimensional \linebreak
                              ${\textrm{sinc}}({\pi}x)\cos(k{\pi}x)$ kernel.

            \sstitem
            \texttt{"SincGauss"} --- Sum of surrounding pixels weighted using
                              a one-dimensional
                              ${\textrm{sinc}}({\pi}x)e^{-kx^2}$ kernel.

            \sstitem
            \texttt{"BlockAve"}  --- Block averaging over surrounding pixels.

         }
         In the above, ${\textrm{sinc}}(z)=\sin(z)/z$.  Some of these schemes will
         require additional parameters to be supplied via the PARAMS
         parameter.  A more-detailed discussion of these schemes is
         given in the \texttt{"Sub-Pixel Interpolation Schemes"} section below.
         \texttt{["Auto"]}
      }
      \sstsubsection{
         MODE = LITERAL (Read)
      }{
         This determines how the shape of the output NDF is to be specified.
         The allowed values and their meanings are as follows.

         \ssthitemlist{

            \sstitem
            \texttt{"Factors"} --- the FACTORS parameter will be used to
                                determine the factor by which each dimension
                                should be multiplied.

            \sstitem
            \texttt{"Bounds"}  --- the LBOUND and UBOUND parameters will be used
                                to get the lower and upper pixel bounds of the
                                output NDF.

            \sstitem
            \texttt{"PixelScale"} --- the PIXSCALE parameter will be used to
                                   obtain the new pixel scale to use for
                                   each WCS axis.
         }
         \texttt{["Factors"]}
      }
      \sstsubsection{
         OUT = NDF (Write)
      }{
         The squashed or stretched NDF.
      }
      \sstsubsection{
         PARAMS( ) = \_DOUBLE (Read)
      }{
         Parameters required to control the resampling scheme.  One or
         more values may be required to specify the exact resampling
         behaviour, according to the value of the METHOD parameter.
         See the section on
         \htmlref{``Sub-pixel Interpolation Schemes''}{subpixel:sqorst}.
      }
      \sstsubsection{
         PIXSCALE = LITERAL (Read)
      }{
         The PIXSCALE parameter is only used if Parameter MODE is set to
         \texttt{"PixelScale"}. It should be supplied as a comma-separated
         list of the required new pixel scales.  In this context, a pixel
         scale for a WCS axis is the increment in WCS axis value caused
         by a movement of one pixel along the WCS axis, and are measured
         at the first pixel in the array.  Pixel scales for celestial axes
         should be given in arcseconds. An asterisk, \texttt{"$*$"}, can be used
         instead of a numerical value to indicate that an axis should
         retain its current scale. The suggested default values are the
         current pixel scales.  If no value has been supplied for
         Parameter AXIS, the number of values supplied for PIXSCALE
         must be the same as the number of WCS axes in the NDF.  If a
         non-null value has been supplied for Parameter AXIS, then only
         a single value should be supplied for PIXSCALE and that value
         will be used as the new pixel scale on the WCS axis indicated
         by Parameter AXIS.
      }
      \sstsubsection{
         TITLE = LITERAL (Read)
      }{
         Title for the output NDF.  A null (\texttt{{!}}) value causes the input
         title to be used.  \texttt{[!]}
      }
      \sstsubsection{
         UBOUND( ) = \_INTEGER (Read)
      }{
         This parameter is only used if MODE=\texttt{"Bounds"}.  The upper
         pixel index values of the output NDF. If no value has been
         supplied for Parameter AXIS, the number of values supplied
         for UBOUND must be the same as the number of pixel axes in
         the NDF.  If a non-null value has been supplied for Parameter
         AXIS, then only a single value should be supplied for UBOUND
         and the supplied value will be used as the new upper bounds
         on the axis indicated by Parameter AXIS.  If null (\texttt{{!}})
         is given, the upper pixel bounds of the input NDF will be
         used.
      }
   }
   \sstexamples{
      \sstexamplesubsection{
         sqorst block blocktall [1,2,1]
      }{
         The three-dimensional NDF called block is stretched by a factor
         of two along its second axis to produce an NDF called
         blocktall with twice as many pixels.  The same data block
         is represented, but each pixel in the output NDF corresponds
         to half a pixel in the input NDF.  The default resampling
         scheme, linear interpolation in the stretch direction, is used.
      }
      \sstexamplesubsection{
         sqorst block blocktall [1,2,1] method=sincsinc params=[2,2]
      }{
         The same operation as the previous example is performed,
         except that a Lanczos kernel is used for the interpolation.
      }
      \sstexamplesubsection{
         sqorst cygnus1 squish1 mode=bounds lbound=[1,1] ubound=[50,50]
      }{
         This turns the two-dimensional NDF cygnus1 into a new NDF squish1
         which has 50 pixels along each side.  The same region of sky
         is represented, but the input image is squashed along both
         axes to fit the specified dimensions.
      }
      \sstexamplesubsection{
         sqorst fred mode=pixelscale pixscale=5 axis=3
      }{
         This resamples a cube NDF called fred on to a velocity scale
         of 5 km/s per pixel along its third axis.
      }
      \sstexamplesubsection{
         sqorst in=\^tilecubes out=*aligned mode=pixelscale pixscale=0.5 axis=3 centre=Middle
      }{
         This resamples a list of spectral cubes stored in text file
         tilecubes, forming a new set of NDFs with the suffix \texttt{"aligned"}.
         As the cubes form a larger survey, not only are they compressed
         to a common velocity scale of 0.5 km/s per channel along the
         third axis, but they are also aligned so that channel centres
         are the same in all the cubes, being a multiple of 0.5 km/s.
      }

   }
   \sstnotes{
      If the input NDF contains a \htmlref{VARIANCE}{apndf:variance}~ component,
      a VARIANCE component will be written to the output NDF.  It will be
      calculated on the assumption that errors on the input data
      values are statistically independent and that their variance
      estimates may simply be summed (with appropriate weighting
      factors) when several input pixels contribute to an output data
      value.  If this assumption is not valid, then the output error
      estimates may be biased.  In addition, note that the statistical
      errors on neighbouring output data values (as well as the
      estimates of those errors) may often be correlated, even if the
      above assumption about the input data is correct, because of
      the sub-pixel interpolation schemes employed.
   }
   \label{subpixel:sqorst}
   \sstdiytopic{
      Sub-Pixel Interpolation Schemes
   }{
      When squashing or stretching an NDF, a separate one-dimensional
      resampling operation is performed for each of the dimensions
      in which a resize is being done.  By default (when METHOD=\texttt{"Auto"})
      this is done using linear interpolation, unless it is a
      squash of a factor of two or more, in which case a block-averaging
      scheme which averages over 1/FACTOR pixels.  For many
      purposes this default scheme will be adequate, but for greater
      control over the resampling process the METHOD and PARAMS
      parameters can be used.  Detailed discussion of the use of these
      parameters can be found in the
      \xref{``Sub-pixel Interpolation Schemes''}{sun210}{AST_RESAMPLE\$<X>\$}~
      section of the AST\_RESAMPLE documentation. By default, all
      interpolation schemes preserve flux density rather than total
      flux, but this may be changed using the CONSERVE parameter.
   }
   \sstimplementationstatus{
      \sstitemlist{

         \sstitem

         The \htmlref{LABEL}{apndf:label},
         \htmlref{UNITS}{apndf:units}, and
         \htmlref{HISTORY}{apndf:history}~ components, and all
         \htmlref{extensions}{apndf:extensions}~ are propagated.
         \htmlref{TITLE}{apndf:title}~ is controlled by the TITLE
         parameter.  DATA, \htmlref{VARIANCE}{apndf:variance},
         \htmlref{AXIS}{apndf:axis}~ and \htmlref{WCS}{apndf:wcs} are
         propagated after appropriate modification.  The
         \htmlref{QUALITY}{apndf:quality}~ component is also
         propagated if nearest-neighbour interpolation is being used.


         \sstitem
         Processing of \htmlref{bad pixels}{se:masking} and automatic \htmlref{quality masking}{se:qualitymask} are
         supported.

         \sstitem
         All \htmlref{non-complex numeric data types}{ap:HDStypes} can be handled.

         \sstitem
         There can be an arbitrary number of NDF dimensions.
      }
   }
   \sstdiytopic{
      Related Applications
   }{
KAPPA: \htmlref{REGRID}{REGRID},
\htmlref{SLIDE}{SLIDE},
\htmlref{WCSADD}{WCSADD}.
   }
}

\sstroutine{
   STATS
}{
   Computes simple statistics for an NDF's pixels
}{
   \sstdescription{
      This application computes and displays simple statistics for the
      pixels in an \NDFref{NDF's} data, quality or variance array.  The
      statistics available are:
      \ssthitemlist{

         \sstitem
         the pixel sum,

         \sstitem
         the pixel mean,

         \sstitem
         the pixel population standard deviation,

         \sstitem
         the pixel population skewness and excess kurtosis,

         \sstitem
         the value and position of the minimum- and maximum-valued
         pixels,

         \sstitem
         the total number of pixels in the NDF,

         \sstitem
         the number of pixels used in the statistics, and

         \sstitem
         the number of pixels omitted.

      }
      Iterative $\kappa$-sigma clipping may also be applied as an option
      (see Parameter CLIP).

      Order statistics (median and percentiles) may optionally be
      derived and displayed (see Parameters ORDER and PERCENTILES).
      Although this can be a relatively slow operation on large arrays,
      unlike \htmlref{HISTAT}{HISTAT} the reported order statistics are
      accurate, not approximations, irrespective of the distribution of values
      being analysed.
   }
   \sstusage{
      stats ndf [comp] [clip] [logfile]
   }
   \sstparameters{
      \sstsubsection{
         CLIP( ) = \_REAL (Read)
      }{
         An optional one-dimensional array of clipping levels to be
         applied, expressed as standard deviations.  If a null value is
         supplied for this parameter (the default), then no iterative
         clipping will take place and the statistics computed will
         include all the valid NDF pixels.

         If an array of clipping levels is given, then the routine will
         first compute statistics using all the available pixels.  It
         will then reject all those pixels whose values lie outside
         $\kappa$ standard deviations of the mean (where $\kappa$ is
         the first value
         supplied) and will then re-evaluate the statistics.  This
         rejection iteration is repeated in turn for each value in the
         CLIP array.  A maximum of five values may be supplied, all of
         which must be positive.  \texttt{[!]}
      }
      \sstsubsection{
         COMP = \htmlref{LITERAL}{se:parmenu} (Read)
      }{
         The name of the NDF array component for which statistics are
         required: \texttt{"Data"}, \texttt{"Error"}, \texttt{"Quality"} or
         \texttt{"Variance"} (where \texttt{"Error"} is the alternative
         to \texttt{"Variance"} and causes the square root of the variance
         values to be taken before computing the statistics).  If
         \texttt{"Quality"} is specified, then the quality values are treated
         as numerical values (in the range 0 to 255).  \texttt{["Data"]}
      }
      \sstsubsection{
         LOGFILE = FILENAME (Write)
      }{
         A text file into which the results should be logged.  If a null
         value is supplied (the default), then no logging of results
         will take place.  \texttt{[!]}
      }
      \sstsubsection{
         NDF = NDF (Read)
      }{
         The NDF data structure to be analysed.
      }
      \sstsubsection{
         ORDER = \_LOGICAL (Read)
      }{
        Whether or not to calculate order statistics.  If set \texttt{TRUE}
        the median and optionally percentiles are determined and
        reported.  \texttt{[FALSE]}
      }
      \sstsubsection{
         PERCENTILES( 100 ) = \_REAL (Read)
      }{
         A list of percentiles to be found.  None are computed if this
         parameter is null (\texttt{{!}}).  The percentiles must be in the range
         0.0 to 100.0.  This parameter is ignored unless ORDER is \texttt{TRUE.}
         \texttt{[!]}
      }
   }
   \sstresparameters{
      \sstsubsection{
         KURTOSIS = \_DOUBLE (Write)
      }{
         The population excess kurtosis of all the valid pixels in the
         NDF array.  This is the normal kurtosis minus 3, such that a
         Gaussian distribution of values would generate an excess
         kurtosis of 0.
      }
      \sstsubsection{
         MAXCOORD( ) = \_DOUBLE (Write)
      }{
         A one-dimensional array of values giving the \htmlref{WCS}{apndf:wcs}
         co-ordinates of the centre of the (first) maximum-valued pixel found
         in the NDF array.  The number of co-ordinates is equal to the number
         of NDF dimensions.
      }
      \sstsubsection{
         MAXIMUM = \_DOUBLE (Write)
      }{
         The maximum pixel value found in the NDF array.
      }
      \sstsubsection{
         MAXPOS( ) = \_INTEGER (Write)
      }{
         A one-dimensional array of \htmlref{pixel indices}{se:pixgrd}~
         identifying the (first)
         maximum-valued pixel found in the NDF array.  The number of
         indices is equal to the number of NDF dimensions.
      }
      \sstsubsection{
         MAXWCS = LITERAL (Write)
      }{
         The formatted WCS co-ordinates at the maximum pixel value.  The
         individual axis values are comma separated.
      }
      \sstsubsection{
         MEAN = \_DOUBLE (Write)
      }{
         The mean value of all the valid pixels in the NDF array.
      }
      \sstsubsection{
         MEDIAN = \_DOUBLE (Write)
      }{
        The median value of all the valid pixels in the NDF array when
        ORDER is \texttt{TRUE}.
      }
      \sstsubsection{
         MINCOORD( ) = \_DOUBLE (Write)
      }{
         A one-dimensional array of values giving the data co-ordinates of
         the centre of the (first) minimum-valued pixel found in the
         NDF array.  The number of co-ordinates is equal to the number of
         NDF dimensions.
      }
      \sstsubsection{
         MINIMUM = \_DOUBLE (Write)
      }{
         The minimum pixel value found in the NDF array.
      }
      \sstsubsection{
         MINPOS( ) = \_INTEGER (Write)
      }{
         A one-dimensional array of pixel indices identifying the (first)
         minimum-valued pixel found in the NDF array.  The number of
         indices is equal to the number of NDF dimensions.
      }
      \sstsubsection{
         MINWCS = LITERAL (Write)
      }{
         The formatted WCS co-ordinates at the minimum pixel value.  The
         individual axis values are comma separated.
      }
      \sstsubsection{
         NUMBAD = \_INTEGER (Write)
      }{
         The number of pixels which were either not valid or were
         rejected from the statistics during iterative $\kappa$-sigma
         clipping.
      }
      \sstsubsection{
         NUMGOOD = \_INTEGER (Write)
      }{
         The number of NDF pixels which actually contributed to the
         computed statistics.
      }
      \sstsubsection{
         NUMPIX = \_INTEGER (Write)
      }{
         The total number of pixels in the NDF (both good and bad).
      }
      \sstsubsection{
         PERVAL() = \_DOUBLE (Write)
      }{
         The values of the percentiles of the good pixels in the NDF
         array.  This parameter is only written when one or more
         percentiles have been requested.
      }
      \sstsubsection{
         SIGMA = \_DOUBLE (Write)
      }{
         The population standard deviation of the pixel values in the
         NDF array.
      }
      \sstsubsection{
         SKEWNESS = \_DOUBLE (Write)
      }{
         The population skewness of all the valid pixels in the NDF
         array.
      }
      \sstsubsection{
         TOTAL = \_DOUBLE (Write)
      }{
         The sum of the pixel values in the NDF array.
      }
   }
   \sstexamples{
      \sstexamplesubsection{
         stats image
      }{
         Computes and displays simple statistics for the data array in
         the NDF called image.
      }
      \sstexamplesubsection{
         stats image order percentiles=[25,75]
      }{
         As the previous example but it also reports the median, 25 and
         75 percentiles.
      }
      \sstexamplesubsection{
         stats ndf=spectrum variance
      }{
         Computes and displays simple statistics for the variance array
         in the NDF called spectrum.
      }
      \sstexamplesubsection{
         stats spectrum error
      }{
         Computes and displays statistics for the variance array in the
         NDF called spectrum, but takes the square root of the variance
         values before doing so.
      }
      \sstexamplesubsection{
         stats halley logfile=stats.dat
      }{
         Computes statistics for the data array in the NDF called
         halley, and writes the results to a logfile called \texttt{stats.dat}.
      }
      \sstexamplesubsection{
         stats ngc1333 clip=[3.0,2.8,2.5]
      }{
         Computes statistics for the data array in the NDF called
         ngc1333, applying three iterations of $\kappa$-sigma clipping.  The
         statistics are first calculated for all the valid pixels in
         the data array.  Those pixels with values lying more than 3.0
         standard deviations from the mean are then rejected, and the
         statistics are re-computed.  This process is then repeated
         twice more, rejecting pixel values lying more than 2.8 and 2.5
         standard deviations from the mean.  The final statistics are
         displayed.
      }
   }
   \sstimplementationstatus{
      \sstitemlist{

         \sstitem
         This routine correctly processes the \htmlref{AXIS}{apndf:axis}, DATA, \htmlref{VARIANCE}{apndf:variance},
         \htmlref{QUALITY}{apndf:quality}, \htmlref{TITLE}{apndf:title}, and \htmlref{HISTORY}{apndf:history}~ components of the NDF.

         \sstitem
         Processing of \htmlref{bad pixels}{se:masking} and automatic \htmlref{quality masking}{se:qualitymask} are
         supported.

         \sstitem
         All \htmlref{non-complex numeric data types}{ap:HDStypes} can be handled.  Arithmetic
         is performed using double-precision floating point.

         \sstitem
         Any number of NDF dimensions is supported.
      }
   }
   \sstdiytopic{
      Related Applications
   }{
KAPPA: \htmlref{HISTAT}{HISTAT},
\htmlref{NDFTRACE}{NDFTRACE};
\xref{FIGARO}{sun86}{}: \xref{ISTAT}{sun86}{ISTAT}.
   }
}
\sstroutine{
   SUB
}{
   Subtracts one NDF data structure from another
}{
   \sstdescription{
      The routine subtracts one \NDFref{NDF} data structure from another
      pixel-by-pixel to produce a new NDF.
   }
   \sstusage{
      sub in1 in2 out
   }
   \sstparameters{
      \sstsubsection{
         IN1 = NDF (Read)
      }{
         First NDF, from which the second NDF is to be subtracted.
      }
      \sstsubsection{
         IN2 = NDF (Read)
      }{
         Second NDF, to be subtracted from the first NDF.
      }
      \sstsubsection{
         OUT = NDF (Write)
      }{
         Output NDF to contain the difference of the two input NDFs.
      }
      \sstsubsection{
         TITLE = LITERAL (Read)
      }{
         The title for the output NDF.  A null value will cause
         the title of the NDF supplied for Parameter IN1 to be used
         instead.  \texttt{[!]}
      }
   }
   \sstexamples{
      \sstexamplesubsection{
         sub a b c
      }{
         This subtracts the NDF called b from the NDF called a, to make
         the NDF called c.  NDF c inherits its title from a.
      }
      \sstexamplesubsection{
         sub out=c in1=a in2=b title="Background subtracted"
      }{
         This subtracts the NDF called b from the NDF called a, to make
         the NDF called c.  NDF c has the title \texttt{"Background subtracted"}.
      }
   }
   \sstnotes{
      If the two input NDFs have different pixel-index bounds, then
      they will be trimmed to match before being subtracted.  An error
      will result if they have no pixels in common.
   }
   \sstdiytopic{
      Related Applications
   }{
KAPPA: \htmlref{ADD}{ADD},
\htmlref{CADD}{CADD},
\htmlref{CDIV}{CDIV},
\htmlref{CMULT}{CMULT},
\htmlref{CSUB}{CSUB},
\htmlref{DIV}{DIV},
\htmlref{MATHS}{MATHS},
\htmlref{MULT}{MULT}.
   }
   \sstimplementationstatus{
      \sstitemlist{

         \sstitem
         This routine correctly processes the \htmlref{AXIS}{apndf:axis}, DATA, \htmlref{QUALITY}{apndf:quality},
         \htmlref{LABEL}{apndf:label}, \htmlref{TITLE}{apndf:title}, \htmlref{HISTORY}{apndf:history}, \htmlref{WCS}{apndf:wcs}, and \htmlref{VARIANCE}{apndf:variance}~ components of an NDF
         data structure and propagates all \htmlref{extensions}{apndf:extensions}.

         \sstitem
         The \htmlref{UNITS}{apndf:units}~ component is propagated only if it has the same
         value in both input NDFs.

         \sstitem
         Processing of \htmlref{bad pixels}{se:masking} and automatic \htmlref{quality masking}{se:qualitymask} are supported.

         \sstitem
         All \htmlref{non-complex numeric data types}{ap:HDStypes} can be handled.

         \sstitem
         Huge NDFs are supported.

      }
   }
}
\sstroutine{
   SUBSTITUTE
}{
   Replaces all occurrences of a given value in an NDF array with
   another value
}{
   \sstdescription{
      This application changes all pixels that have a defined value in
      an \NDFref{NDF} with an alternate value. The number of replacements is
      reported. Two modes are available.

      \sstitemlist{

         \sstitem
         A single pair of old and new values can be supplied (see
         Parameters OLDVAL and NEWVAL). All occurrences of OLDVAL are
         replaced with NEWVAL. Other values are unchanged.

         \sstitem
         A look-up table containing corresponding old and new values can
         be supplied. By default, each input pixel that equals an old value
         is replaced by the corresponding new value. Alternatively, all
         input pixels can be changed to a new value by interpolation between
         the input values (see Parameters LUT and INTERP).
      }
   }
   \sstusage{
      substitute in out oldval newval [comp]
   }
   \sstparameters{
      \sstsubsection{
         COMP = \htmlref{LITERAL}{se:parmenu} (Read)
      }{
         The components whose values are to be substituted.  It may
         be \texttt{"Data"}, \texttt{"Error"}, \texttt{"Variance"}, or \texttt{"All"}.  The last of the
         options forces substitution in both the data and variance
         arrays.  This parameter is ignored if the data array is the
         only array component within the NDF.  \texttt{["Data"]}
      }
      \sstsubsection{
         IN = NDF  (Read)
      }{
         Input NDF structure containing the data and/or variance array
         to have some of its elements substituted.
      }
      \sstsubsection{
         INTERP = LITERAL (Read)
      }{
         Determines how the values in the file specified by Parameter
         LUT are used, from the following options.

         \sstitemlist{

            \sstitem
            \texttt{"None"}  -- Pixel values that equal an input value are replaced
            by the corresponding output value. Other values are left
            unchanged.

            \sstitem
            \texttt{"Nearest"}  -- Every pixel value is replaced by the output value
            corresponding to the nearest input value.

            \sstitem
            \texttt{"Linear"}  -- Every pixel value is replaced by an output value
            determined using linear interpolation between the input values.

         }
         If \texttt{"Nearest"} or \texttt{"Linear"} is used, pixel values that are outside
         the range of input value covered by the look-up table are set
         bad in the output. Additionally, an error is reported if the
         old data values are not montonic increasing. \texttt{["None"]}
      }
      \sstsubsection{
         LUT = FILENAME (Read)
      }{
         The name of a text file containing a look-up table of old and
         new data values. If null (\texttt{{!}}) is supplied for this parameter the
         old and new data values will instead be obtained using Parameters
         OLDVAL and NEWVAL. Lines starting with a hash (\#) or exclamation
         mark (\texttt{{!}}) are ignored, as are blank lines. Other lines
         should contain an old data
         value followed by the corresponding new data value. The way in
         which the values in this table are used is determined by
         Parameter INTERP. \texttt{[!]}
      }
      \sstsubsection{
         NEWVAL = \_DOUBLE (Read)
      }{
         The value to replace occurrences of OLDVAL.  It must lie
         within the minimum and maximum values of the data type of the
         array with higher precision.  The new value is converted to
         data type of the array being converted before the search
         begins.  The suggested default is the current value.  This
         parameter is only accessed if a null value is supplied for
         Parameter LUT.
      }
      \sstsubsection{
         OLDVAL = \_DOUBLE (Read)
      }{
         The element value to be replaced.  The same value is
         substituted in both the data and variance arrays when
         COMP=\texttt{"All"}.  It must lie within the minimum and maximum values
         of the data type of the array with higher precision.  The
         replacement value is converted to data type of the array being
         converted before the search begins.  The suggested default is
         the current value.  This parameter is only accessed if a null
         value is supplied for Parameter LUT.
      }
      \sstsubsection{
         OUT = NDF (Write)
      }{
         Output NDF structure containing the data and/or variance array
         that is a copy of the input array, but with replacement values
         substituted.
      }
      \sstsubsection{
         TITLE = LITERAL (Read)
      }{
         \htmlref{Title}{apndf:title} for the output NDF structure.  A null value (\texttt{{!}})
         propagates the title from the input NDF to the output NDF.  \texttt{[!]}
      }
      \sstsubsection{
         TYPE = LITERAL (Read)
      }{
         The numeric type for the output NDF. The value given should be
         one of the following: \_DOUBLE, \_REAL, \_INTEGER, \_INT64, \_WORD,
         \_UWORD, \_BYTE or \_UBYTE (note the leading underscore).  If a
         null (\texttt{{!}}) value is supplied, the output data type equals the
         input data type. \texttt{[!]}
      }
   }
   \sstexamples{
      \sstexamplesubsection{
         substitute aa bb 1 0
      }{
         This copies the NDF called aa to the NDF bb, except
         that any pixels with value 1 in aa are altered to have value
         0 in bb.
      }
      \sstexamplesubsection{
         substitute aa bb oldval=1 newval=0 comp=v
      }{
         As above except the substitution occurs to the variance
         values.
      }
      \sstexamplesubsection{
         substitute in=saturn out=saturn5 oldval=2.5 newval=5 comp=All
      }{
         This copies the NDF called saturn to the NDF saturn5, except
         that any elements in the data and variance arrays that have
         value 2.5 are altered to have value 5 in saturn5.
      }
   }
   \sstnotes{
      \sstitemlist{

         \sstitem
         The comparison for floating-point values tests that the
         difference between the replacement value and the element value is
         less than their mean times the precision of the data type.
      }
   }
   \sstdiytopic{
      Related Applications
   }{
KAPPA: \htmlref{CHPIX}{CHPIX},
\htmlref{FILLBAD}{FILLBAD},
\htmlref{GLITCH}{GLITCH},
\htmlref{NOMAGIC}{NOMAGIC},
\htmlref{SEGMENT}{SEGMENT},
\htmlref{SETMAGIC}{SETMAGIC},
\htmlref{ZAPLIN}{ZAPLIN};
\linebreak
\xref{FIGARO}{sun86}{}: \xref{GOODVAR}{sun86}{GOODVAR}.
   }
   \sstimplementationstatus{
      \sstitemlist{

         \sstitem
         This routine correctly processes the \htmlref{AXIS}{apndf:axis}, DATA, \htmlref{QUALITY}{apndf:quality},
         \htmlref{VARIANCE}{apndf:variance}, \htmlref{LABEL}{apndf:label}, \htmlref{TITLE}{apndf:title}, \htmlref{UNITS}{apndf:units}, \htmlref{WCS}{apndf:wcs}, and \htmlref{HISTORY}{apndf:history}~ components of an NDF
         data structure and propagates all \htmlref{extensions}{apndf:extensions}.

         \sstitem
         All \htmlref{non-complex numeric data types}{ap:HDStypes} can be handled.

         \sstitem
         Any number of NDF dimensions is supported.
      }
   }
}

\sstroutine{
   SURFIT
}{
   Fits a polynomial or bi-cubic spline surface to two-dimensional
   data array
}{
   \sstdescription{
      The background of a two-dimensional data array in the supplied \NDFref{NDF}
      structure is estimated by condensing the array into equally sized
      rectangular bins, fitting a spline or polynomial surface to the
      bin values, and finally evaluating the surface for each pixel in
      the data array.

      There is a selection of estimators by which representative
      values for each bin are determined.  There are several options to
      make the fit more accurate.  Values beyond upper and lower
      thresholds may be excluded from the binning.  Bad pixels are also
      excluded, so prior masking may help to find the background more
      rapidly.  $\kappa$-sigma clipping of the fitted bins is available
      so that the fit is not biased by anomalous bins, such as those
      entirely within an extended object.  If a given bin contains more
      than a prescribed fraction of bad pixels, it is excluded from the
      fit.

      The data array representing the background is evaluated at each
      pixel by one of two methods.  It is written to the output NDF
      structure.

      The raw binned data, the weights, the fitted binned data and the
      residuals to the fit may be written to a logfile.  This also
      keeps a record of the input parameters and the rms error of the
      fit.
   }
   \sstusage{
      surfit in out [fittype] [estimator] [bindim] [evaluate]
   }
   \sstparameters{
      \sstsubsection{
         BINDIM() = \_INTEGER (Read)
      }{
         The \textit{x}-\textit{y} dimensions of a bin used to estimate the local
         background.  If you supply only one value, it is used for
         both dimensions.  The minimum value is 2.  The maximum may be
         constrained by the number of polynomial terms, such that in
         each direction there are at least as many bins as terms.  If a
         null (\texttt{{!}}) value is supplied, the value used is such that 32 bins
         are created along each axis.  \texttt{[!]}
      }
      \sstsubsection{
         CLIP() = \_REAL (Read)
      }{
         Array of limits for progressive clipping of pixel values
         during the binning process in units of standard deviation.  A
         null value means only unclipped statistics are computed and
         presented.  Between one and five values may be supplied.  \texttt{[2,3]}
      }
      \sstsubsection{
         ESTIMATOR = \htmlref{LITERAL}{se:parmenu} (Read)
      }{
         The estimator for the bin.  It must be one of the following
         values: \texttt{"Mean"} for the mean value, \texttt{"Ksigma"} for the mean with
         $\kappa$-sigma clipping; \texttt{"Mode"} for the mode, and \texttt{"Median"} for
         the median.  \texttt{"Mode"} is only available when there are at least
         twelve pixels in a bin.  If a null (\texttt{{!}}) value is supplied, \texttt{"Median"}
         is used if there are fewer than 6 values in a bin, and \texttt{"Mode"} is
         used otherwise.  \texttt{[!]}
      }
      \sstsubsection{
         EVALUATE = LITERAL (Read)
      }{
         The method by which the resulting data array is to be
         evaluated from the surface-fit.  It must be either
         \texttt{"Interpolate"} where the values at the corners of the bins are
         derived first, and then the pixel values are found by linear
         interpolation within those bins; or \texttt{"All"} where the
         surface-fit is evaluated for every pixel.  The latter is
         slower, but can produce more-accurate results, unless the
         surface is well behaved.  The default is the current value,
         which is initially set to \texttt{"Interpolate"}.  \texttt{[]}
      }
      \sstsubsection{
         FITCLIP() = \_REAL (Read)
      }{
         Array of limits for progressive clipping of the binned array
         in units of the rms deviation of the fit.  A null value (\texttt{{!}})
         means no clipping of the binned array will take place.
         Between 1 and 5 values may be supplied.  The default is the
         current value, which is \texttt{{!}} initially.  \texttt{[]}
      }
      \sstsubsection{
         FITTYPE = LITERAL (Read)
      }{
         The type of fit.  It must be either \texttt{"Polynomial"} for a
         Chebyshev polynomial or \texttt{"Spline"} for a bi-cubic spline.  The
         default is the current value, which initially is \texttt{"Spline"}.  \texttt{[]}
      }
      \sstsubsection{
         GENVAR = \_LOGICAL (Read)
      }{
         If \texttt{TRUE}, a constant variance array is created in the output NDF
         assigned to the mean square surface-fit error.  \texttt{[FALSE]}
      }
      \sstsubsection{
         LOGFILE = FILENAME (Read)
      }{
         Name of the file to log the binned array and errors before and
         after fitting.  If null, there will be no logging.  \texttt{[!]}
      }
      \sstsubsection{
         IN = NDF (Read)
      }{
         NDF containing the two-dimensional data array to be fitted.
      }
      \sstsubsection{
         KNOTS( 2 ) = \_INTEGER (Read)
      }{
         The number of interior knots used for the bi-cubic-spline fit
         along the \textit{x} and \textit{y} axes.  These knots are equally spaced within
         the image.  Both values must be in the range 0 to 11.  If you
         supply a single value, it applies to both axes.  Thus \texttt{1}
         creates one interior knot, \texttt{[5,4]} gives 5 along the \textit{x} axis and
         4 along the \textit{y} direction.  Increasing this parameter's values
         increases the flexibility of the surface.  Normally, \texttt{4} is a
         reasonable value.  The upper limit of acceptable values will
         be reduced along each axis when its binned array dimension is
         fewer than 29.  KNOTS is only accessed when FITTYPE=\texttt{"Spline"}.
         The default is the current value, which is \texttt{4} initially.  \texttt{[]}
      }
      \sstsubsection{
         ORDER( 2 ) = \_INTEGER (Read)
      }{
         The orders of the fits along the \textit{x} and \textit{y} directions.  Both
         values must be in the range 0 to 14.  If you supply a single
         single value, it applies to both axes.  Thus \texttt{0} gives a
         constant, \texttt{[3,1]} gives a cubic along the \textit{x} direction and a
         linear fit along the \textit{y} axis.  Increasing this parameter's values
         increases the flexibility of the surface.  The upper limit of
         acceptable values will be reduced along each axis when its
         binned array dimension is fewer than 29.  ORDER is only
         accessed when FITTYPE=\texttt{"Polynomial"}.  The default is the current
         value, which is \texttt{4} initially.  \texttt{[]}
      }
      \sstsubsection{
         OUT = NDF (Write)
      }{
         NDF to contain the fitted two-dimensional data array.
      }
      \sstsubsection{
         THRHI = \_REAL (Read)
      }{
         Upper threshold above which values will be excluded from the
         analysis to derive representative values for the bins.  If it
         is null (\texttt{{!}}) there will be no upper threshold.  \texttt{[!]}
      }
      \sstsubsection{
         THRLO = \_REAL (Read)
      }{
         Lower threshold below which values will be excluded from the
         analysis to derive representative values for the bins.  If it
         is null (\texttt{{!}}) there will be no lower threshold.  \texttt{[!]}
      }
      \sstsubsection{
         TITLE = LITERAL (Read)
      }{
         The title for the output NDF.  A null value will cause
         the title of the NDF supplied for Parameter IN to be used
         instead.  \texttt{[!]}
      }
      \sstsubsection{
         WLIM = \_REAL (Read)
      }{
         The minimum fraction of good pixels in a bin that permits the
         bin to be included in the fit.  Here good pixels are ones that
         participated in the calculation of the bin's representative
         value.  So they exclude both \htmlref{bad pixels}{se:masking}~ and ones rejected
         during estimation (\emph{e.g.} ones beyond the thresholds or were
         clipped).  \texttt{[!]}
      }
   }
   \sstresparameters{
      \sstsubsection{
         RMS = \_REAL (Write)
      }{
         The RMS deviation of the fit from the original data (per pixel).
      }
   }
   \sstnotes{
      A polynomial surface fit is stored in a SURFACEFIT extension,
      component FIT of type POLYNOMIAL, variant CHEBYSHEV or BSPLINE.

      For further details of the CHEBYSHEV variant see
      \xref{SGP/38}{sgp38}{}.  The CHEBYSHEV variant includes the fitting
      variance for each coefficient.

      The BSPLINE variant structure is provisional.  It contain the
      spline coefficients in the two-dimensional DATA\_ARRAY component,
      the knots in XKNOTS and YKNOTS arrays, and a scaling factor to
      restore the original values in SCALE.  All of these components have
      type \_REAL.

      Also stored in the SURFACEFIT extension is the r.m.s. deviation
      to the fit (component RMS); and the co-ordinate system component
      COSYS, set to \texttt{"GRID"}.
   }
   \sstexamples{
      \sstexamplesubsection{
         surfit comaB comaB\_bg
      }{
         This calculates the surface fit to the two-dimensional NDF
         called comaB using the current defaults.  The evaluated fit is
         stored in the NDF called comaB\_bg.
      }
      \sstexamplesubsection{
         surfit comaB comaB\_bg poly median order=5 bindim=[24,30]
      }{
         As above except that 5th-order polynomial fit is chosen,
         the median is used to derive the representative value for each
         bin, and the binning size is 24 pixels along the first axis,
         and 32 pixels along the second.
      }
      \sstexamplesubsection{
         surfit comaB comaB\_bg fitclip=[2,3] logfile=comaB\_fit.lis
      }{
         As the first example except that the binned array is clipped at
         2 then 3 standard deviations to remove outliers before the
         final fit is computed.  The text file \texttt{comaB\_fit.lis} records a
         log of the surface fit.
      }
      \sstexamplesubsection{
         surfit comaB comaB\_bg estimator=ksigma clip=[2,2,3]
      }{
         As the first example except that the representative value of
         each bin is the mean after clipping twice at 2 then once at
         3 standard deviations.
      }
      \sstexamplesubsection{
         surfit in=irasorion out=sback evaluate=all fittype=s knots=7
      }{
         This calculates the surface fit to the two-dimensional NDF called
         irasorion.  The fit is evaluated at every pixel and the
         resulting array stored in the NDF called sback.  A spline with
         seven knots along each axis is used to fit the surface.
      }
   }
   \sstdiytopic{
      Related Applications
   }{
KAPPA: \htmlref{ARDMASK}{ARDMASK},
\htmlref{FITSURFACE}{FITSURFACE},
\htmlref{MAKESURFACE}{MAKESURFACE},
\htmlref{REGIONMASK}{REGIONMASK}.
   }
   \sstimplementationstatus{
      \sstitemlist{

         \sstitem
         This routine correctly processes the \htmlref{AXIS}{apndf:axis}, DATA, \htmlref{QUALITY}{apndf:quality},
         \htmlref{LABEL}{apndf:label}, \htmlref{TITLE}{apndf:title}, \htmlref{UNITS}{apndf:units}, \htmlref{WCS}{apndf:wcs}, and \htmlref{HISTORY}{apndf:history}~ components of the input NDF.
         Any input \htmlref{VARIANCE}{apndf:variance}~ is ignored.

         \sstitem
         Processing of \htmlref{bad pixels}{se:masking} and automatic \htmlref{quality masking}{se:qualitymask} are
         supported.

         \sstitem
         All \htmlref{non-complex numeric data types}{ap:HDStypes} can be handled.  Arithmetic
         is performed using single- or double-precision floating point for
         FITTYPE=\texttt{"Spline"} or \texttt{"Polynomial"} respectively.  The output
         NDF's DATA and VARIANCE components have type \_REAL (single-precision).
      }
   }
}




\pagebreak
\sstroutine{
   THRESH
}{
   Edits an NDF to replace values between or outside given limits
   with specified constant values
}{
   \sstdescription{
      This application creates an output \NDFref{NDF} by copying values from an
      input NDF, replacing all values within given data ranges by
      a user-specified constant or by the bad value. Upper and lower
      thresholds are supplied using Parameters THRLO and THRHI.

      If THRLO is less than or equal to THRHI, values between and
      including the two thresholds are copied from the input to output
      array.  Any values in the input array greater than the upper
      threshold will be set to the value of Parameter NEWHI, and anything
      less than the lower threshold will be set to the value of Parameter
      NEWLO, in the output data array.  Thus the output NDF is constrained
      to lie between the two bounds.

      If THRLO is greater than THRHI, values greater than or equal to
      THRLO are copied from the input to output array, together with
      values less than or equal to THRHI.  Any values between THRLO and
      THRHI will be set to the value of Parameter NEWLO in the output NDF.

      Each replacement value may be the \htmlref{bad-pixel}{se:masking}~ value
      for masking.
   }
   \sstusage{
      thresh in out thrlo thrhi newlo newhi [comp]
   }
   \sstparameters{
      \sstsubsection{
         COMP = \htmlref{LITERAL}{se:parmenu} (Read)
      }{
         The components whose values are to be constrained between
         thresholds.  The options are limited to the arrays within the
         supplied NDF.  In general the value may be \texttt{"Data"}, \texttt{"Quality"},
         \texttt{"Error"}, or \texttt{"Variance"}.  If \texttt{"Quality"} is specified, then the
         quality values are treated as numerical values in the range 0
         to 255.  \texttt{["Data"]}
      }
      \sstsubsection{
         IN = NDF  (Read)
      }{
         Input NDF structure containing the array to have thresholds
         applied.
      }
      \sstsubsection{
         NEWHI = LITERAL (Read)
      }{
         This gives the value to which all input array-element values
         greater than the upper threshold are set.  If this is set to
         \texttt{"Bad"}, the bad value is substituted.  Numerical values of
         NEWHI must lie in within the minimum and maximum values of the
         data type of the array being processed.  The suggested default
         is the upper threshold.  This parameter is ignored if THRLO is
         greater than THRHI.
      }
      \sstsubsection{
         NEWLO = LITERAL (Read)
      }{
         This gives the value to which all input array-element values
         less than the lower threshold are set.  If this is set to
         \texttt{"Bad"}, the bad value is substituted.  Numerical values of
         NEWLO must lie in within the minimum and maximum values of the
         data type of the array being processed.  The suggested default
         is the lower threshold.
      }
      \sstsubsection{
         OUT = NDF (Write)
      }{
         Output NDF structure containing the thresholded version of
         the array.
      }
      \sstsubsection{
         THRHI = \_DOUBLE (Read)
      }{
         The upper threshold value within the input array.  It must lie
         in within the minimum and maximum values of the data type of
         the array being processed.  The suggested default is the
         current value.
      }
      \sstsubsection{
         THRLO = \_DOUBLE (Read)
      }{
         The lower threshold value within the input array.  It must lie
         within the minimum and maximum values of the data type of
         the array being processed.  The suggested default is the
         current value.
      }
      \sstsubsection{
         TITLE = LITERAL (Read)
      }{
         \htmlref{Title}{apndf:title} for the output NDF structure.  A null value (\texttt{{!}})
         propagates the title from the input NDF to the output NDF.  \texttt{[!]} }
   }
   \sstresparameters{
      \sstsubsection{
         NUMHI = \_INTEGER (Write)
      }{
         The number of pixels whose values were thresholded as being
         greater than the THRHI threshold.
      }
      \sstsubsection{
         NUMLO = \_INTEGER (Write)
      }{
         The number of pixels whose values were thresholded as being
         less than the THRLO threshold.
      }
      \sstsubsection{
         NUMRANGE = \_INTEGER (Write)
      }{
         The number of pixels whose values were thresholded as being
         between the THRLO and THRHI thresholds, if THRLO is greater
         than THRHI.
      }
      \sstsubsection{
         NUMSAME = \_INTEGER (Write)
      }{
         The number of unchanged pixels.
      }
   }
   \sstexamples{
      \sstexamplesubsection{
         thresh zzcam zzcam2 100 500 0 0
      }{
         This copies the data array in the NDF called zzcam to the NDF
         called zzcam2.  Any data value less than 100 or greater than
         500 in zzcam is set to 0 in zzcam2.
      }
      \sstexamplesubsection{
         thresh zzcam zzcam2 500 100 0
      }{
         This copies the data array in the NDF called zzcam to the NDF
         called zzcam2.  Any data value less than 500 and greater than
         100 in zzcam is set to 0 in zzcam2.
      }
      \sstexamplesubsection{
         thresh zzcam zzcam2 100 500 0 0 comp=Variance
      }{
         As above except that the data array is copied unchanged and the
         thresholds apply to the variance array.
      }
      \sstexamplesubsection{
         thresh n253 n253cl thrlo=-0.5 thrhi=10.1 $\backslash$
      }{
         This copies the data array in the NDF called n253 to the NDF
         called n253cl.  Any data value less than $-$0.5 in n253 is set
         to $-$0.5 in n253cl, and any value greater than 10.1 in n253
         becomes 10.1 in n253cl.
      }
      \sstexamplesubsection{
         thresh pavo pavosky -0.02 0.02 bad bad
      }{
         All data values outside the range $-$0.02 to 0.02 in the NDF
         called pavo become bad in the NDF called pavosky.  All values
         within this range are copied from pavo to pavosky.
      }
   }
   \sstdiytopic{
      Related Applications
   }{
KAPPA: \htmlref{HISTEQ}{HISTEQ},
\htmlref{MATHS}{MATHS};
\xref{FIGARO}{sun86}{}: \xref{CLIP}{sun86}{CLIP},
\xref{IDIFF}{sun86}{IDIFF},
\xref{RESCALE}{sun86}{RESCALE}.
   }
   \sstimplementationstatus{
      \sstitemlist{

         \sstitem
         This routine correctly processes the \htmlref{AXIS}{apndf:axis}, DATA, \htmlref{QUALITY}{apndf:quality},
         \htmlref{VARIANCE}{apndf:variance}, \htmlref{LABEL}{apndf:label}, \htmlref{TITLE}{apndf:title}, \htmlref{UNITS}{apndf:units}, \htmlref{WCS}{apndf:wcs}, and \htmlref{HISTORY}{apndf:history}~ components of an NDF
         data structure and propagates all \htmlref{extensions}{apndf:extensions}.

         \sstitem
         Processing of \htmlref{bad pixels}{se:masking} and automatic \htmlref{quality masking}{se:qualitymask} are
         supported.

         \sstitem
         All \htmlref{non-complex numeric data types}{ap:HDStypes} can be handled.

         \sstitem
         Any number of NDF dimensions is supported.
      }
   }
}

\sstroutine{
   TRANDAT
}{
   Converts free-format text data into an NDF
}{
   \sstdescription{
      This application takes grid data contained in a free-format text
      file and stores them in the data array of an \NDFref{NDF}.  The data file
      could contain, for example, mapping data  or results from
      simulations which are to be converted into an image for analysis.

      There are two modes of operation which depend on whether the
      text file contains co-ordinate information, or solely data
      values (determined by Parameter AUTO).

      a) {\bf AUTO=FALSE}  ~~If the file contains co-ordinate information
      the format of the data is tabular; the positions and values are
      arranged in columns and a record may contain information for only
      a single point.  Where data points are duplicated only the last
      value appears in the NDF.  Comment lines can be given, and are
      indicated by a hash or exclamation mark in the first column.
      Here is an example file (the vertical ellipses indicate missing
      lines in the file):
\texttt{\begin{verse}
          \# Model 5, phi = 0.25,  eta = 1.7 \\
          1 -40.0   40.0   1121.9 \\
          2  0.0   30.0     56.3 \\
          3 100.0   20.0   2983.2 \\
          4 120.0   85.0    339.3 \\
          . ~. ~~. ~ .   \\
          . ~. ~~. ~ .   \\
          . ~. ~~. ~ .   \\
          <EOF>
\end{verse}}

      The records do not need to be ordered (but see the warning in the
      \htmlref{``Notes''}{notes:trandat}), as the application searches for the maximum and minimum
      co-ordinates in each dimension so that it can define the size of
      the output image.  Also, each record may contain other data
      fields (separated by one or more spaces), which need not be all
      the same data type.  In the example above only columns 2, 3 and 4
      are required.  There are parameters (POSCOLS, VALCOL) which
      select the co-ordinate and value columns.

      The distance between adjacent pixels (given by Parameter PSCALE)
      defaults to 1, and is in the same units as the read-in
      co-ordinates.  The pixel index of a data value is calculated
      using the expression

      \[   {\textrm{index}} = {\textrm{FLOOR}}( ( x - xoff ) / {\textrm{scale}} ) + 1 \]

      where \textit{x} is the supplied co-ordinate and $xoff$ is the value
      of the POFFSET parameter (which defaults to the minimum supplied
      co-ordinate along an axis), $scale$ is the value of Parameter PSCALE,
      and FLOOR is a function that returns the largest integer that is
      smaller (\emph{i.e.} more negative) than its argument.

      You are informed of the number of points found and the maximum
      and minimum co-ordinate values for each dimension.  There is no
      limit imposed by the application on the number of points or the
      maximum output array size, though there may be external
      constraints.  The derived array size is reported in case you have
      made a typing error in the text file.  If you realise that this
      has indeed occurred just abort (\texttt{{!!}}) when prompted for the output
      NDF.

      b) {\bf AUTO=TRUE}  ~~If the text file contains no co-ordinates, the
      format is quite flexible, however, the data are read into the data array
      in Fortran order, \emph{i.e.} the first dimension is the most rapidly
      varying, followed by the second dimension and so on.  The number
      of data values that may appear on a line is variable; data values
      are separated by at least a space, comma, tab or carriage return.
      A line can have up to 255 characters.  In addition a record may
      have trailing comments designated by a hash or exclamation mark.
      Here is an example file, though a more regular format would be
      clearer for the human reader.
\texttt{\begin{verse}
          \# test for the new TRANDAT \\
          23 45.3 ! a comment \\
          50.7,47.5 120. 46.67  47.89 42.4567 \\
          .1 23.3 45.2 43.2  56.0 30.9 29. 27. 26. 22.4 20. 18. -12. 8. \\
           9.2 11. \\
          <EOF>
\end{verse}}

      Notice that the shape of the NDF is defined by a parameter rather
      than explicitly in the file.
   }
   \sstusage{
      trandat freename out [poscols] [valcol] [pscale] [dtype] [title]
   }
   \sstparameters{
      \sstsubsection{
         AUTO = \_LOGICAL (Read)
      }{
         If \texttt{TRUE} the text file does not contain co-ordinate
         information.  \texttt{[FALSE]}
      }
      \sstsubsection{
         BAD = \_LOGICAL (Read)
      }{
         If \texttt{TRUE} the output NDF data array is initialised with the
         \htmlref{bad value}{se:masking}, otherwise it is filled with zeroes.  \texttt{[TRUE]}
      }
      \sstsubsection{
         DTYPE = LITERAL (Read)
      }{
         The HDS type of the data values within the text file, and
         the type of the data array in the output NDF.  The options
         are: \texttt{'\_REAL'}, \texttt{'\_DOUBLE'}, \texttt{'\_INTEGER'},
         \texttt{'\_BYTE'}, \texttt{'\_UBYTE'}, \texttt{'\_WORD'}, \texttt{'\_UWORD'}.
         (Note the leading underscore.) \texttt{['\_REAL']}
      }
      \sstsubsection{
         FREENAME = FILENAME (Read)
      }{
         Name of the text file containing the free-format data.
      }
      \sstsubsection{
         LBOUND( ) = \_INTEGER (Read)
      }{
         The lower bounds of the NDF to be created.  The number of
         values must match the number supplied to Parameter SHAPE.  It
         is only accessed in automatic mode.  If a null (\texttt{{!}}) value is
         supplied, the value used is 1 along each axis.  \texttt{[!]}
      }
      \sstsubsection{
         POFFSET() = \_REAL (Read)
      }{
         The supplied co-ordinates that correspond to the origin of
         floating point pixel co-ordinates.  It is only used in co-ordinate
         mode.  Its purpose is to permit an offset from some arbitrary units
         to pixels. If a null (\texttt{{!}}) value is supplied, the value used is
         the minimum supplied co-ordinate value for each dimension. \texttt{[!]}
      }
      \sstsubsection{
         POSCOLS() = \_INTEGER (Read)
      }{
         Column positions of the co-ordinates in an input record
         of the text file, starting from \textit{x} to higher dimensions.  It
         is only used in co-ordinate mode.  The columns must be
         different amongst themselves and also different from the
         column containing the values.  If there is duplication,
         new values for both POSCOLS and VALCOL will be requested.
         \texttt{[1,2]}
      }
      \sstsubsection{
         PSCALE() = \_REAL (Read)
      }{
         Pixel-to-pixel distance in co-ordinate units for each
         dimension.  It is only used in co-ordinate mode.  Its purpose
         is to permit linear scaling from some arbitrary units to
         pixels.  \texttt{[}1.0 in each co-ordinate dimension\texttt{{]}}
      }
      \sstsubsection{
         QUANTUM = \_INTEGER (Read)
      }{
         You can safely ignore this parameter.  It is used for
         fine-tuning performance in the co-ordinate mode.

         The application obtains work space to store the position-value
         data before they can be copied into the output NDF so that the
         array bounds can be computed.  Since the number of lines in
         the text file is unknown, the application obtains chunks of
         work space whose size is three times this parameter whenever
         it runs out of storage.  (Three because the parameter
         specifies the number of lines in the file rather than the
         number of data items.)  If you have a large number of points
         there are efficiency gains if you make this parameter either
         about 20--30 per cent or slightly greater than or equal to the
         number of lines your text file.  A value slightly less than
         the number of lines is inefficient as it creates nearly 50 per
         cent unused space.  A value that is too small can cause
         unnecessary unmapping, expansion and re-mapping of the work
         space.  For most purposes the default should give acceptable
         performance.  It must lie between 32 and 2097152.  \texttt{[2048]}
      }
      \sstsubsection{
         SHAPE( ) = \_INTEGER (Read)
      }{
         The shape of the NDF to be created.  For example, \texttt{[50,30,20]}
         would create 50 columns by 30 lines by 10 bands.  It is only
         accessed in automatic mode.
      }
      \sstsubsection{
         NDF = NDF (Write)
      }{
         Output NDF for the generated data array.
      }
      \sstsubsection{
         TITLE = LITERAL (Read)
      }{
         \htmlref{Title}{apndf:title}~ for the output NDF.  \texttt{["KAPPA - Trandat"]}
      }
      \sstsubsection{
         VALCOL = \_INTEGER (Read)
      }{
         Column position of the array values in an input record of
         the text file.  It is only used in co-ordinate mode.  The
         column position must be different from those specified for
         the co-ordinate columns.  If there is duplication, new values
         for both POSCOLS and VALCOL will be requested.  \texttt{[3]}
      }
   }
   \sstexamples{
      \sstexamplesubsection{
         trandat simdata.dat model
      }{
         Reads the text file \texttt{simdata.dat} and stores the data into the
         data array of a two-dimensional, \_REAL NDF called model.  The
         input file should have the co-ordinates and real values
         arranged in columns, with the \textit{x}-\textit{y} positions in columns 1 and 2
         respectively, and the real data in column 3.
      }
      \sstexamplesubsection{
         trandat freename=simdata out=model auto shape=[50,40,9]
      }{
         Reads the text file \texttt{simdata} and stores the data into the
         data array of a three-dimensional, \_REAL NDF called model.
         Its \textit{x} dimension is 50, \textit{y} is 40 and \textit{z} is 9.  The input file only
         contains real values and comments.
      }
      \sstexamplesubsection{
         trandat freename=simdata out=model auto shape=[50,40,9] dtype=\_i
      }{
         As the previous example except an \_INTEGER NDF is created, and
         the text file must contain integer data.
      }
      \sstexamplesubsection{
         trandat simdata.dat model [6,3,4] 2
      }{
         Reads the text file \texttt{simdata.dat} and stores the data into the
         data array of a three-dimensional, \_REAL NDF called model.  The
         input file should have the co-ordinates and real values
         arranged in columns, with the \textit{x}-\textit{y}-\textit{z} positions in columns 6, 3
         and 4 respectively, and the real data in column 2.
      }
      \sstexamplesubsection{
         trandat spectrum.dat lacertid noauto poscols=2 valcol=4 pscale=2.3
      }{
         Reads the text file \texttt{spectrum.dat} and stores the data into the
         data array of a one-dimensional, \_REAL NDF called lacertid.
         The input file should have the co-ordinate and real values
         arranged in columns, with its co-ordinates in columns 2, and
         the real data in column 4.  A one-pixel step in the NDF
         corresponds to 2.3 in units of the supplied co-ordinates.
      }
   }
   \label{notes:trandat}
   \sstnotes{
      \sstitemlist{

         \sstitem
         Bad data values may be represented by the string ``BAD'' (case
         insensitive) within the input text file.

         \sstitem
         All \htmlref{non-complex numeric data types}{ap:HDStypes} can be handled.  However,
         byte, unsigned byte, word and unsigned word require data
         conversion, and therefore involve additional processing.
         to a vector element (for \textit{n}-d generality).

         \sstitem
         {\bf WARNING:} In non-automatic mode it is strongly advisable for
         large output NDFs to place the data in Fortran order, \emph{i.e.} the
         first dimension is the most rapidly varying, followed by the
         second dimension and so on.  This gives optimum performance.  The
         meaning of `large' will depend on working-set quotas on your
         system, but a few megabytes gives an idea.  If you jump randomly
         backwards and forwards, or worse, have a text file in
         reverse-Fortran order, this can have disastrous performance
         consequences for you and other users.

         \sstitem
         In non-automatic mode, the co-ordinates for each dimension are
         stored in the NDF axis structure.  The first centre is at the
         minimum value found in the list of positions for the dimension
         plus half of the scale factor.  Subsequent centres are
         incremented by the scale factor.

         \sstitem
         The output NDF may have between one and seven dimensions.

         \sstitem
         In automatic mode, an error is reported if the shape does not
         use all the data points in the file.
      }
   }
   \sstdiytopic{
      Related Applications
   }{
\xref{CONVERT}{sun55}{}: \xref{ASCII2NDF}{sun55}{ASCII2NDF},
\xref{NDF2ASCII}{sun55}{NDF2ASCII};
\xref{FIGARO}{sun86}{}: \xref{ASCIN}{sun86}{ASCIN},
\xref{ASCOUT}{sun86}{ASCOUT}.
   }
}
\sstroutine{
   TRIG
}{
   Performs a trigonometric transformation on a NDF
}{
   \sstdescription{
      This routine copies the supplied input \NDFref{NDF}, performing a specified
      trigonometric operation (sine, tangent, \emph{etc.}) on each value in the
      DATA array.  The \htmlref{VARIANCE}{apndf:variance}~ component, if present, is modified
      appropriately.  Pixels for which the required value is undefined, or
      outside the numerical range of the NDFs data type, are set bad in
      the output.
   }
   \sstusage{
      trig in trigfunc out title
   }
   \sstparameters{
      \sstsubsection{
         IN = NDF (Read)
      }{
         The input NDF structure.
      }
      \sstsubsection{
         OUT = NDF (Write)
      }{
         The output NDF structure.
      }
      \sstsubsection{
         TRIGFUNC = LITERAL (Read)
      }{
           Trigonometrical function to be applied.  The options are as follows.

         \ssthitemlist{

            \sstitem
            \texttt{"ACOS"} ---  arc-cosine (radians)

            \sstitem
            \texttt{"ACOSD"} ---  arc-cosine (degrees)

            \sstitem
            \texttt{"ASIN"} ---  arc-sine (radians)

            \sstitem
            \texttt{"ASIND"} ---  arc-sine (degrees)

            \sstitem
            \texttt{"ATAN"} ---  arc-tangent (radians)

            \sstitem
            \texttt{"ATAND"} ---  arc-tangent (degrees)

            \sstitem
            \texttt{"COS"} ---  cosine (radians)

            \sstitem
            \texttt{"COSD"} ---  cosine (degrees)

            \sstitem
            \texttt{"SIN"} ---  sine (radians)

            \sstitem
            \texttt{"SIND"} ---  sine (degrees)

            \sstitem
            \texttt{"TAN"} ---  tangent (radians)

            \sstitem
            \texttt{"TAND"} ---  tangent (degrees)
         }
      }
      \sstsubsection{
         TITLE = LITERAL (Read)
      }{
         A \htmlref{title}{apndf:title} for the output NDF.  A null value will cause the title
         of the NDF supplied for Parameter IN to be used instead.
         \texttt{[!]}
      }
   }
   \sstexamples{
      \sstexamplesubsection{
         trig sindata asind data
      }{
         Take the arc-sine of the data values in the NDF called sindata, and
         store the results (in degrees) in the NDF called data.
      }
      \sstexamplesubsection{
         trig sindata asin data
      }{
         As above, but the output values are stored in radians.
      }
   }
   \sstdiytopic{
      Related Applications
   }{
KAPPA: \htmlref{ADD}{ADD},
\htmlref{CADD}{CADD},
\htmlref{CMULT}{CMULT},
\htmlref{CDIV}{CDIV},
\htmlref{CSUB}{CSUB},
\htmlref{DIV}{DIV},
\htmlref{MATHS}{MATHS},
\htmlref{MULT}{MULT},
\htmlref{SUB}{SUB}.
   }
   \sstimplementationstatus{
      \sstitemlist{

         \sstitem
         This routine correctly processes the \htmlref{AXIS}{apndf:axis}, DATA, \htmlref{QUALITY}{apndf:quality},
         \htmlref{LABEL}{apndf:label}, \htmlref{TITLE}{apndf:title}, \htmlref{UNITS}{apndf:units}, \htmlref{HISTORY}{apndf:history}, \htmlref{WCS}{apndf:wcs}, and \htmlref{VARIANCE}{apndf:variance}~ components of an NDF
         data structure and propagates all \htmlref{extensions}{apndf:extensions}.

         \sstitem
         Processing of \htmlref{bad pixels}{se:masking} and automatic \htmlref{quality masking}{se:qualitymask} are
         supported.

         \sstitem
         All \htmlref{non-complex numeric data types}{ap:HDStypes} can be handled.  Arithmetic
         is performed using single-precision floating point, or double
         precision, if appropriate, but the numeric type of the input pixels
         is preserved in the output NDF.
      }
   }
}

\sstroutine{
   VECPLOT
}{
   Plots a two-dimensional vector map
}{
   \sstdescription{
      This application plots vectors defined by the values contained
      within a pair of two-dimensional \NDFref{NDFs}, the first holding the
      magnitude of the vector quantity at each pixel, and the second
      holding the corresponding vector orientations.  It is assumed that
      the two NDFs are aligned in \htmlref{pixel co-ordinates}{se:pixgrd}.  The number of
      vectors in the plot is kept to a manageable value by only
      plotting vectors for pixels on a sparse regular matrix.  The
      increment (in pixels) between plotted vectors is given by
      Parameter STEP.  Zero orientation may be fixed at any position
      angle within the plot by specifying an appropriate value for
      Parameter ANGROT.  Each vector may be represented either by an
      arrow or by a simple line, as selected by Parameter ARROW.

      The plot is produced within the \htmlref{current graphics database picture}{se:agitate},
      and may be aligned with an existing DATA picture if the existing
      picture contains suitable \htmlref{co-ordinate Frame}{se:domains}~ information
      (see Parameter CLEAR).

      Annotated axes can be produced (see Parameter AXES), and the
      appearance of these can be controlled in detail using Parameter
      STYLE.  The axes show co-ordinates in the current co-ordinate Frame
      of NDF1.

      A key to the vector scale can be displayed to the right of the
      vector map (see Parameter KEY).  The appearance and position of this
      key may be controlled using Parameters KEYSTYLE and KEYPOS.
   }
   \sstusage{
      vecplot ndf1 ndf2 [comp] [step] [vscale] [arrow] [just] [device]
   }
   \sstparameters{
      \sstsubsection{
         ANGROT = \_REAL (Read)
      }{
         A rotation angle in degrees to be added to each vector
         orientation before plotting the vectors (see Parameter NDF2).
         It should be in the range 0--360.  \texttt{[0.0]}
      }
      \sstsubsection{
         ARROW = LITERAL (Read)
      }{
         Vectors are drawn as arrows, with the size of the arrow head
         specified by this parameter.  Simple lines can be drawn by setting
         the arrow head size to zero.  The value should be expressed as a
         fraction of the largest dimension of the vector map.  \texttt{[}current value\texttt{{]}}
      }
      \sstsubsection{
         AXES = \_LOGICAL (Read)
      }{
         \texttt{TRUE} if labelled and annotated axes are to be drawn around the
         vector map.  These display co-ordinates in the current co-ordinate
         Frame NDF1, which may be changed using application \htmlref{WCSFRAME}{WCSFRAME}
         (see also Parameter USEAXIS).  The width of the margins left for
         the annotation may be controlled using Parameter MARGIN.  The
         appearance of the axes (colours, founts, \emph{etc.}) can be controlled
         using the STYLE parameter.  \texttt{[TRUE]}
      }
      \sstsubsection{
         CLEAR = \_LOGICAL (Read)
      }{
         \texttt{TRUE} if the graphics device is to be cleared before displaying
         the vector map.  If you want the vector map to be drawn over
         the top of an existing DATA picture, then set CLEAR to \texttt{FALSE}.  The
         vector map will then be drawn in alignment with the displayed
         data.  If possible, alignment occurs within the current co-ordinate
         Frame of the NDF.  If this is not possible (for instance, if
         suitable \htmlref{WCS information}{se:wcsuse}~ was not stored with
         the existing DATA picture), then alignment is attempted in PIXEL
         co-ordinates.  If this is not possible, then alignment is
         attempted in \htmlref{GRID co-ordinates}{se:pixgrd}.  If this is
         not possible, then alignment is attempted in the first suitable
         Frame found in the NDF
         irrespective of its domain.  A message is displayed indicating the
         domain in which alignment occurred.  If there are no suitable Frames
         in the NDF then an error is reported.  \texttt{[TRUE]}
      }
      \sstsubsection{
         COMP = \htmlref{LITERAL}{se:parmenu} (Read)
      }{
         The component of NDF1 which is to be used to define the vector
         magnitudes.  It may be \texttt{"Data"}, \texttt{"Error"} or \texttt{"Variance"}.  The
         last two are not available if NDF1 does not contain a VARIANCE
         component.  The vector orientations are always defined by the
         \texttt{"Data"} component of NDF2.  \texttt{["Data"]}
      }
      \sstsubsection{
         DEVICE = \htmlref{DEVICE}{se:selgradev} (Read)
      }{
         The plotting device.  \texttt{[}Current graphics device\texttt{{]}}
      }
      \sstsubsection{
         FILL = \_LOGICAL (Read)
      }{
         The DATA picture containing the vector map is usually produced with
         the same shape as the data.  However, for maps with markedly different
         dimensions this default behaviour may not give the clearest result.
         When FILL is \texttt{TRUE}, the smaller dimension of the picture is expanded
         to produce the largest possible picture within the current picture.
         \texttt{[FALSE]}
      }
      \sstsubsection{
         JUST = LITERAL (Read)
      }{
         The justification for each vector; it can take any of the
         following values:

         \ssthitemlist{

            \sstitem
             \texttt{"Centre"} --- the vectors are drawn centred on the
             corresponding pixel,

            \sstitem
             \texttt{"Start"}  --- the vectors are drawn starting at the
             corresponding pixel, and

            \sstitem
             \texttt{"End"} --- the vectors are drawn ending at the corresponding
             pixel.

         }
         \texttt{["Centre"]}
      }
      \sstsubsection{
         KEY = \_LOGICAL (Read)
      }{
         \texttt{TRUE} if a key indicating the vector scale is to be produced.  \texttt{[TRUE]}
      }
      \sstsubsection{
         KEYPOS() = \_REAL (Read)
      }{
         Two values giving the position of the key.  The first value gives
         the gap between the right-hand edge of the vector map and the
         left-hand edge of the key (\texttt{0.0} for no gap, \texttt{1.0} for the
         largest gap).  The second value gives the vertical position of the
         top of the key (\texttt{1.0} for the highest position, \texttt{0.0} for
         the lowest).  If the second value is not given, the top of the key
         is placed level with the top of the vector map.  Both values should
         be in the range 0.0 to 1.0.  If a key is produced, then the
         right-hand margin specified by Parameter MARGIN is ignored.
         \texttt{[}current value\texttt{{]}}
      }
      \sstsubsection{
         KEYSTYLE = \htmlref{GROUP}{se:groups} (Read)
      }{
         A group of attribute settings describing the plotting style to use
         for the key (see Parameter KEY).

         A comma-separated list of strings should be given in which each
         string is either an attribute setting, or the name of a text
         file preceded by an up-arrow character \texttt{"$\wedge$"}.  Such text files
         should contain further comma-separated lists which will be
         read and interpreted in the same manner.  Attribute settings
         are applied in the order in which they occur within the list,
         with later settings overriding any earlier settings given for
         the same attribute.

         Each individual attribute setting should be of the form:

            $<$name$>$=$<$value$>$

         where $<$name$>$ is the name of a plotting attribute, and $<$value$>$
         is the value to assign to the attribute.  Default values will be
         used for any unspecified attributes.  All attributes will be
         defaulted if a null value (\texttt{{!}})---the initial default---is supplied.
         To apply changes of style to only the current invocation, begin these
         attributes with a plus sign.  A mixture of persistent and temporary
         style changes is achieved by listing all the persistent attributes
         followed by a plus sign then the list of temporary attributes.

         See \slhyperref{Plotting Attributes}{Section~}{}{ap:plotting_attr}
         for a description of the available attributes.  Any unrecognised
         attributes are ignored (no error is reported).

         The appearance of the text in the key is controlled using \att{String}
         attributes (\emph{e.g.} \att{Colour(Strings)},
         \htmlattref{Font(Strings)}{Font(element)}; the
         synonym \att{Text} can be used in place of \att{Strings}).  Note, the
         \htmlattref{Size}{Size(element)}~ attribute specifies the size of key
         text relative to the size of the numerical labels on the vector-map
         axes.  Thus a value of \texttt{2.0} for \att{Size} will result in text which is
         twice the size of the numerical axis labels. The appearance of the
         example vector is controlled using \att{Curve} attributes
         (\emph{e.g.} \htmlattref{Colour(Curves)}{Colour(element)}; the synonym \att{Vector} can be
         used in place of \att{Curves}).  The numerical scale value is
         formatted as an axis-1 value (using attributes \htmlattref{Format(1)}{Format(axis)},
          \htmlattref{Digits(1)}{Digits/Digits(axis)}, \emph{etc}; the synonym \att{Scale} can be used in
         place of the value 1).  The length of the example vector is formatted
         as an axis-2 value (using attribute \att{Format(2)}, \emph{etc}; the
         synonym \att{Vector} can be used in place of the value 2).  The
         vertical space between lines in the key can be controlled using
         attribute \htmlattref{TextLabGap}{TextLabGap(axis)}.  A value of \texttt{1.0} is used if no
         value is set for this attribute, and produces default vertical
         spacing.  Values larger than 1.0 increase the vertical space, and
         values less than 1.0 decrease the vertical space.  \texttt{[}current
         value\texttt{{]}}
      }
      \sstsubsection{
         KEYVEC = \_REAL (Read)
      }{
         Length of the vector to be displayed in the key, in data units.
         If a null (\texttt{{!}}) value is supplied, the value used is generated
         on the basis of the spread of vector lengths in the plot.  \texttt{[!]}
      }
      \sstsubsection{
         MARGIN( 4 ) = \_REAL (Read)
      }{
         The widths of the margins to leave around the vector map for axis
         annotation.  The widths should be given as fractions of the
         corresponding dimension of the current picture.  The actual margins
         used may be increased to preserve the aspect ratio of the DATA
         picture.  Four values may be given, in the order; bottom, right,
         top, left.  If fewer than four values are given, extra values are
         used equal to the first supplied value.  If these margins are too
         narrow any axis annotation may be clipped.  If a null (\texttt{{!}}) value
         is supplied, the value used is \texttt{0.15} (for all edges) if annotated
         axes are being produced, and zero otherwise.  See also Parameter
         KEYPOS.  \texttt{[}current value\texttt{{]}}
      }
      \sstsubsection{
         NDF1 = NDF (Read)
      }{
         NDF structure containing the two-dimensional image giving the
         vector magnitudes.
      }
      \sstsubsection{
         NDF2 = NDF (Read)
      }{
         NDF structure containing the two-dimensional image giving the
         vector orientations.  The values are considered to be in units
         of degrees unless the \htmlref{UNITS}{apndf:units}~ component of the NDF has the value
         \texttt{"Radians"} (case insensitive).  The positive \textit{y} pixel axis defines
         zero orientation, and rotation from the \textit{x} pixel axis to the
         \textit{y} pixel is considered positive.
      }
      \sstsubsection{
         STEP = \_INTEGER (Read)
      }{
         The number of pixels between adjacent displayed vectors (along
         both axes).  Increasing this value reduces the number of
         displayed vectors.  If a null (\texttt{{!}}) value is supplied, the value
         used gives about thirty vectors along the longest axis of the plot.  \texttt{[!]}
      }
      \sstsubsection{
         STYLE = \htmlref{GROUP}{se:groups} (Read)
      }{
         A group of attribute settings describing the plotting style to use
         for the vectors and annotated axes.

         A comma-separated list of strings should be given in which each
         string is either an attribute setting, or the name of a text
         file preceded by an up-arrow character \texttt{"$\wedge$"}.  Such text files
         should contain further comma-separated lists which will be
         read and interpreted in the same manner.  Attribute settings
         are applied in the order in which they occur within the list,
         with later settings overriding any earlier settings given for
         the same attribute.

         Each individual attribute setting should be of the form:

            $<$name$>$=$<$value$>$

         where $<$name$>$ is the name of a plotting attribute, and $<$value$>$
         is the value to assign to the attribute.  Default values will be
         used for any unspecified attributes.  All attributes will be
         defaulted if a null value (\texttt{{!}})---the initial default---is supplied.
         To apply changes of style to only the current invocation, begin these
         attributes with a plus sign.  A mixture of persistent and temporary
         style changes is achieved by listing all the persistent attributes
         followed by a plus sign then the list of temporary attributes.

         See \slhyperref{Plotting Attributes}{Section~}{}{ap:plotting_attr}
         for a description of the available attributes.  Any unrecognised
         attributes are ignored (no error is reported).

         The appearance of the vectors is controlled by the attributes
         \htmlattref{Colour(Curves)}{Colour(element)},
         \htmlattref{Width(Curves)}{Width(element)}, \emph{etc.} (the
         synonym \att{Vectors} may be
         used in place of \att{Curves}).  \texttt{[}current value\texttt{{]}}
      }
      \sstsubsection{
         VSCALE = \_REAL (Read)
      }{
         The scale to be used for the vectors.  The supplied value
         should give the data value corresponding to a vector length of
         one centimetre.  If a null (\texttt{{!}}) value is supplied, a default value
         is used.  \texttt{[!]}
      }
      \sstsubsection{
         USEAXIS = GROUP (Read)
      }{
         USEAXIS is only accessed if the current co-ordinate Frame of the
         NDF has more than two axes.  A group of two strings should be
         supplied specifying the two axes which are to be used when annotating
         and aligning the vector map.  Each axis can be specified
         using one of the following options.

         \ssthitemlist{

            \sstitem
            Its integer index within the current Frame of the
            input  NDF (in the range 1 to the number of axes in the
            current Frame).

            \sstitem
            Its \htmlattref{Symbol}{Symbol(axis)}~ string such as
            \texttt{"RA"} or \texttt{"VRAD"}.

            \sstitem
            A generic option where \texttt{"SPEC"} requests the spectral axis,
            \texttt{"TIME"} selects the time axis, \texttt{"SKYLON"} and
            \texttt{"SKYLAT"} picks the sky longitude and latitude axes
            respectively.  Only those axis domains present are
            available as options.
         }

         A list of acceptable values is displayed if an illegal value is
         supplied.  If a null (\texttt{{!}}) value is supplied, the axes with the same
         indices as the two significant NDF pixel axes are used.  \texttt{[!]}
      }
   }
   \sstexamples{
      \sstexamplesubsection{
         vecplot polint polang
      }{
         Produces a vector map on the \htmlref{current graphics device}{se:devglobal} with
         vector magnitude taken from the NDF called polint and vector
         orientation taken from NDF polang.  All other settings are
         defaulted, so for example about 20 vectors are displayed along
         the longest axis, and a key is plotted.
      }
      \sstexamplesubsection{
         vecplot polint polang angrot=23.4 clear=no
      }{
         Produces a vector map in which the primary axis of the vectors
         (as defined by the value zero in the NDF polang) is at the
         position angle 23.4 degrees (measured anti-clockwise from the
         positive \textit{y} axis) in the displayed map.  The map is drawn over the
         top of the previously drawn DATA picture, aligned in a
         suitable co-ordinate Frame.
      }
      \sstexamplesubsection{
         vecplot stack(,,2) stack(,,1) arrow=0.1 just=start nokey
      }{
         Produces a vector map in which the vectors are defined by two
         planes in the three-dimensional NDF called stack.  There is no
         need to copy the two planes into two separate NDFs before
         running VECPLOT.  Each vector is represented by an arrow,
         starting at the position of the corresponding pixel.  No key
         to the vector scale and justification is produced.
      }
   }
   \sstnotes{
      \sstitemlist{

         \sstitem
         If no \htmlattref{Title}{plotel:Title}~ is specified via the
         STYLE parameter, then the \htmlref{TITLE}{apndf:title}
         component in NDF1 is used as the default title for the
         annotated axes.  Should the NDF not have a TITLE component,
         then the default title is instead taken from current
         co-ordinate Frame in NDF1, unless this attribute has not
         been set explicitly, whereupon the name of NDF1 is used as
         the default title.

         \sstitem
         The application stores a number of pictures in the
         \htmlref{graphics database}{se:agitate}~ in the following order: a
         FRAME picture containing the
         annotated axes, vectors, and key; a KEY picture to store
         the key if present; and a DATA picture containing just the vectors.
         Note, the FRAME picture is only created if annotated axes or a key
         has been drawn, or if non-zero margins were specified using Parameter
         MARGIN.  The world co-ordinates in the DATA picture will be pixel
         co-ordinates.  A reference to NDF1, together with a copy of the WCS
         information in the NDF are stored in the DATA picture.  On exit the
         current database picture for the chosen device reverts to the
         input picture.
      }
   }
   \sstdiytopic{
      Related Applications
   }{
KAPPA: \htmlref{CALPOL}{CALPOL}.
   }
   \sstimplementationstatus{
      \sstitemlist{

         \sstitem
         Only real data can be processed directly.  Other \htmlref{non-complex numeric data types}{ap:HDStypes} will undergo a type conversion before the
         vector plot is drawn.

         \sstitem
         \htmlref{Bad pixels}{se:masking} and \htmlref{quality masking}{se:qualitymask} are supported.
      }
   }
}




\sstroutine{
   WCSADD
}{
   Creates a Mapping and optionally adds a new co-ordinate Frame into the
   WCS component of an NDF
}{
   \sstdescription{
      This application can be used to create a new
      \htmlref{AST Mapping}{se:curframe}~ and optionally use the Mapping to
      add a new \htmlref{co-ordinate Frame}{se:domains}~ into the
      \htmlref{WCS}{apndf:wcs}~ component of an \NDFref{NDF} (see Parameter
      NDF).  An output text file may also be created holding a textual
      representation of the Mapping for future use by other applications
      such as REGRID (see Parameter MAPOUT).  A number of different types
      of Mapping can be used (see Parameter MAPTYPE).

      When adding a new Frame to a WCS component, the Mapping is used to
      connect the new Frame to an existing one (called the \emph{basis} Frame:
      see Parameter FRAME).  The specific type of Frame to add is specified
      using Parameter FRMTYPE (the default is to simply copy the basis Frame).
      Optionally (see Parameter TRANSFER), attributes which have been
      assigned an explicit value in the basis Frame are transferred to
      the new Frame (but only if they are relevant to the type of the new
      Frame). The value of the \htmlattref{Domain}{Domain}~ attribute for
      the new Frame can be specified using Parameter DOMAIN.  Other attribute
      values for the new Frame may be specified using Parameter ATTRS.  The
      new Frame becomes the current co-ordinate Frame in the NDF (unless
      Parameter RETAIN is set \texttt{TRUE}).

      WCSADD will only generate Mappings with the same number of
      input and output axes; this number is determined by the number
      of axes in the basis Frame if an NDF is supplied, or by the
      NAXES parameter otherwise.
   }
   \sstusage{
      wcsadd ndf frame domain maptype
   }
   \sstparameters{
      \sstsubsection{
         ATTRS = \htmlref{GROUP}{se:groups} (Read)
      }{
         A group of attribute settings to be applied to the new Frame
         before adding it into the NDF.

         A comma-separated list of strings should be given in which each
         string is either an attribute setting, or the name of a text file
         preceded by an up-arrow character \texttt{"$\wedge$"}.  Such text files should
         contain further comma-separated lists which will be read and
         interpreted in the same manner.  Attribute settings are applied in
         the order in which they occur within the list, with later settings
         overriding any earlier settings given for the same attribute.

         Each individual attribute setting should be of the form:

            $<$name$>$=$<$value$>$

         where $<$name$>$ is the name of an attribute appropriate to the type
         of Frame specified by Parameter FRMTYPE (see \xref{SUN/210}{sun210}{} for a complete
         description of all attributes), and $<$value$>$ is the value to assign
         to the attribute.  Default values will be used for any unspecified
         attributes---these defaults are inherited from the basis Frame.
         Any unrecognised attributes are ignored (no error is reported).
      }
      \sstsubsection{
         CENTRE( 2 ) = \_DOUBLE (Read)
      }{
         The co-ordinates of the centre of a pincushion distortion.
         It is only used when MAPTYPE=\texttt{"PINCUSHION"}.  See also DISCO.
         \texttt{[0,0]}
      }
      \sstsubsection{
         DIAG( ) = \_DOUBLE (Read)
      }{
         The elements along the diagonal of the linear transformation
         matrix.  There will be as many of these as there are axes in the
         basis Frame.  Each effectively gives the factor by which
         co-ordinates on the corresponding axis should be multiplied.
         This parameter is only used when MAPTYPE=\texttt{"DIAG"}.
      }
      \sstsubsection{
         DISCO = \_DOUBLE (Read)
      }{
         The distortion coefficient of a pincushion distortion. Used
         in conjunction with the CENTRE parameter, this defines the
         forward transformation to be used as follows:

         \[    XX = X + D * (X - C1) * ( (X - C1)**2 + (Y - C2)**2 ) \]
         \[    YY = Y + D * (Y - C2) * ( (X - C1)**2 + (Y - C2)**2 ) \]

         where ($X$,$Y$) are the input co-ordinates, ($XX$,$YY$) the output
         co-ordinates, $D$ is DISCO, and $C1$ and $C2$ are the two elements of
         CENTRE.  DISCO is only used when MAPTYPE=\texttt{"PINCUSHION"}.
      }
      \sstsubsection{
         DOMAIN = LITERAL (Read)
      }{
         The value for the \att{Domain} attribute for the new Frame.  Care should be
         taken to ensure that domain names are used consistently.  This
         will usually mean avoiding any domain names that are already in
         use within the WCS component, particularly the
         \htmlref{standard domain names}{se:domains}~
         such as GRID, FRACTION, PIXEL, AXIS, and GRAPHICS.  The supplied value is
         stripped of spaces, and converted to upper case before being used.

         Note, if Parameter MAPTYPE is set to \texttt{"REFNDF"}, then the value
         supplied for Parameter DOMAIN indicates the Domain of the Frame
         within the reference NDF that is to be copied (see Parameter
         REFNDF).
      }
      \sstsubsection{
         EPOCH = \_DOUBLE (Read)
      }{
         If the basis Frame is specified using a `Sky Co-ordinate System'
         specification for a celestial co-ordinate system (see Parameter
         FRAME), then an epoch value is needed to qualify it.  This is the
         epoch at which the supplied sky positions were determined.  It should
         be given as a decimal-years value, with or without decimal places
         (\texttt{"1996.8"} for example).  Such values are interpreted as a Besselian
         epoch if less than 1984.0 and as a Julian epoch otherwise.  The
         suggested default is the value stored in the basis Frame.
      }
      \sstsubsection{
         FOREXP = LITERAL (Read)
      }{
         A group of expressions to be used for the forward co-ordinate
         transformations in a MathMap.  There must be at least as many
         expressions as the number of axes of the Mapping, but there
         may be more if intermediate expressions are to be used.  The
         expressions may be given directly in response to the prompt, or
         read from a text file, in which case the name of the file should
         be given, preceded by a \texttt{"$\wedge$"} character.  Individual expression
         should be separated by commas or, if they are supplied in a file,
         newlines (see \slhyperref{Specifying Groups of Objects}{Section~}{}{se:groups}).

         The syntax for each
         expression is Fortran-like; see the \texttt{"Examples"} section below, and
         \slhyperref{Using MathMaps}{Appendix~}{}{ap:MathMaps} for
         details.  FOREXP is only used when MAPTYPE=\texttt{"MATH"}.
      }
      \sstsubsection{
         FRAME = LITERAL (Read)
      }{
         A string specifying the basis Frame.  If a null value is supplied
         the current co-ordinate Frame in the NDF is used.  The string can
         be one of the following:

         \ssthitemlist{

            \sstitem
            A \htmlref{domain name}{se:domains}~ such as \htmlref{SKY, AXIS, PIXEL}{se:resdoms}.  The two
            \texttt{"pseudo-domains"} WORLD and DATA may be supplied and will be
            translated into PIXEL and AXIS respectively, so long as the WCS
            component of the NDF does not contain Frames with these domains.

            \sstitem
            An integer value giving the index of the required Frame within
            the WCS component.

            \sstitem
            An IRAS90 \emph{Sky Co-ordinate System} (SCS) values such as
            \texttt{"EQUAT(J2000)"} (see \xref{SUN/163}{sun163}{}).
         }
      }
      \sstsubsection{
         FRMTYPE = LITERAL (Read)
      }{
         The type of Frame to add to the NDF.  If a null (\texttt{{!}}) value is
         supplied, a copy of the basis Frame is used (as modified by
         Parameters ATTRS and DOMAIN).  The allowed values are as follows.

         \ssthitemlist{

            \sstitem
            \texttt{"FRAME"}      --- A simple Cartesian Frame (the number of axes is
                            equal to the number of outputs from the Mapping)

            \sstitem
            \texttt{"SKYFRAME"}   --- A two-dimensional Frame representing positions on
                            the celestial sphere.

            \sstitem
            \texttt{"SPECFRAME"}  --- A one-dimensional Frame representing positions
                            within an electromagnetic spectrum.

            \sstitem
            \texttt{"TIMEFRAME"}  --- A one-dimensional Frame representing
                            moments in time.
         }

         Note, if Parameter MAPTYPE is set to \texttt{"REFNDF"}, then Parameter
         FRMTYPE will not be used---the Frame used will instead always be
         a copy of the Frame from the reference NDF (as selected by
         Parameter DOMAIN). \texttt{[!]}
      }
      \sstsubsection{
         INVEXP = LITERAL (Read)
      }{
         The expressions to be used for the inverse co-ordinate
         transformations in a MathMap.  See FOREXP.  INVEXP is only used when
         MAPTYPE=\texttt{"MATH"}.
      }
      \sstsubsection{
         MAPIN = FILENAME (Read)
      }{
         The  name of a file containing an AST Mapping with which to
         connect the basis Frame to the new one.  The file may be a text
         file which contains the textual representation of an AST Mapping,
         or a {\FITSref}~ file which contains the Mapping as an AST object
         encoded in its headers, or an NDF.  If it is an NDF, the
         Mapping from its base (GRID-domain) to current Frame will be
         used.  Only used when MAPTYPE=\texttt{"FILE"}.
      }
      \sstsubsection{
         MAPOUT = FILENAME (Write)
      }{
         The name of a text file in which to store a textual representation
         of the Mapping.  This can be used, for instance, by the REGRID
         application.  If a null (\texttt{{!}}) value is supplied, no file is
         created.  \texttt{[!]}
      }
      \sstsubsection{
         MAPTYPE = \htmlref{LITERAL}{se:parmenu} (Read)
      }{
         The type of Mapping to be used to connect the new Frame to the
         basis Frame.  It must be one of the following strings, each
         of which require some additional parameters as indicated.

         \ssthitemlist{

            \sstitem
            \texttt{"DIAGONAL"}   --- A linear mapping with no translation
                            of off-diagonal coefficients (see Parameter DIAG)

            \sstitem
            \texttt{"FILE"}       --- A mapping defined by an AST Mapping supplied
                            in a separate file (see Parameter MAPIN)

            \sstitem
            \texttt{"LINEAR"}     --- A general linear mapping (see Parameter TR)

            \sstitem
            \texttt{"MATH"}       --- A general algebraically defined mapping
                            (see Parameters FOREXP, INVEXP, SIMPFI, SIMPIF)

            \sstitem
            \texttt{"PINCUSHION"} --- A pincushion/barrel distortion (see
                            Parameters DISCO and CENTRE)

            \sstitem
            \texttt{"REFNDF"}      --- The Mapping is obtained by aligning
                 the NDF with a second reference NDF (see Parameter REFNDF)

            \sstitem
            \texttt{"SHIFT"}      --- A translation (see Parameter SHIFT)

            \sstitem
            \texttt{"UNIT"}       --- A unit mapping

            \sstitem
            \texttt{"ZOOM"}       --- A uniform expansion/contraction (see Parameter ZOOM)

         }
         \texttt{["LINEAR"]}
      }
      \sstsubsection{
         NAXES = \_INTEGER (Read)
      }{
         The number of input and output axes which the Mapping will have.
         Only used if a null value is supplied for Parameter NDF.
      }
      \sstsubsection{
         NDF = NDF (Read and Write)
      }{
         The NDF in which to store a new co-ordinate Frame.  Supply a null (\texttt{{!}})
         value if you do not wish to add a Frame to an NDF (you can still
         use the MAPOUT parameter to write the Mapping to a text file).
      }
      \sstsubsection{
         REFNDF = NDF (Read)
      }{
         A reference NDF from which to obtain the Mapping and Frame. The
         NDFs specified by Parameters NDF and REFNDF are aligned in a
         suitable co-ordinate system (usually their current Frames---an
         error is reported if the two NDFs cannot be aligned). The
         Mapping from the basis Frame in ``NDF'' (specified by Parameter
         FRAME) to the required Frame in ``REFNDF'' (specified by Parameter
         DOMAIN) is then found and used. The Frame added into ``NDF'' is
         always a copy of the reference Frame---regardless of the setting
         of Parameter FRMTYPE. Parameter REFNDF is only used when
         Parameter MAPTYPE is set to \texttt{"REFNDF"}, in which case a value must
         also be supplied for Parameter NDF (an error will be reported
         otherwise).
      }
      \sstsubsection{
         RETAIN = \_LOGICAL (Read)
      }{
         Indicates whether the original current Frame should be retained
         within the WCS FrameSet of the modified NDF (see Parameter NDF).
         If \texttt{FALSE}, the newly added Frame is the current Frame on exit.
         Otherwise, the original current Frame is retained on exit. \texttt{[FALSE]}
      }
      \sstsubsection{
         SHIFT( ) = \_DOUBLE (Read)
      }{
         A vector giving the displacement represented by the translation.
         There must be one element for each axis.  Only used when
         MAPTYPE=\texttt{"SHIFT"}.
      }
      \sstsubsection{
         SIMPFI = \_LOGICAL (Read)
      }{
         The value of the Mapping's \att{SimpFI} attribute (whether it is
         legitimate to simplify the forward followed by the inverse
         transformation to a unit transformation).  This parameter is
         only used when MAPTYPE=\texttt{"MATH"}.  \texttt{[TRUE]}
      }
      \sstsubsection{
         SIMPIF = \_LOGICAL (Read)
      }{
         The value of the Mapping's \att{SimpIF} attribute (whether it is
         legitimate to simplify the inverse followed by the forward
         transformation to a unit transformation).  This parameter is
         only used when MAPTYPE=\texttt{"MATH"}.  \texttt{[TRUE]}
      }
      \sstsubsection{
         TR( ) = \_DOUBLE (Read)
      }{
         The values of this parameter are the coefficients of a linear
         transformation from the basis Frame specified by Parameter FRAME
         to the new Frame.  This parameter is only used when \texttt{MAPTYPE="LINEAR"}.
         For instance, if a feature has co-ordinates ($X$,$Y$,$Z$,\ldots) in
         the basis Frame, and co-ordinates ($U$,$V$,$W$,\dots) in the new Frame,
         then the following transformations would be used, depending on
         how many axes the two Frames have:

         \begin{itemize}

            \item one-dimensional:
             \[  U = TR(1) + TR(2)*X  \]

            \item two-dimensional:
             \[  U = TR(1) + TR(2)*X + TR(3)*Y \]
             \[  V = TR(4) + TR(5)*X + TR(6)*Y \]

            \item three-dimensional:
             \[  U = TR(1) + TR(2)*X + TR(3)*Y + TR(4)*Z \]
             \[  V = TR(5) + TR(6)*X + TR(7)*Y + TR(8)*Z \]
             \[  W = TR(9) + TR(10)*X + TR(11)*Y + TR(12)*Z \]

         \end{itemize}

         The correct number of values must be supplied (that is, $N*(N+1)$
         where $N$ is the number of axes in the new and old Frames).  If a
         null value (\texttt{{!}}) is given it is assumed that the new Frame and the
         basis Frame are connected using a unit mapping (\emph{i.e.} corresponding
         axis values are identical in the two Frames).  This parameter
         is only used when MAPTYPE=\texttt{"LINEAR"}.  \texttt{[!]}
      }
      \sstsubsection{
         TRANSFER = \_LOGICAL (Read)
      }{
         If \texttt{TRUE}, attributes which have explicitly set values in the basis
         Frame (specified by Parameter FRAME) are transferred to the new
         Frame (specified by Parameter FRMTYPE), if they are applicable
         to the new Frame. If \texttt{FALSE}, no attribute values are transferred.
         The dynamic default is \texttt{TRUE} if and only if the two Frames are
         of the same class and have the same value for their \att{Domain}
         attributes.  \texttt{[]}
      }
      \sstsubsection{
         ZOOM = \_DOUBLE (Read)
      }{
         The scaling factor for a ZoomMap; every co-ordinate will be
         multiplied by this factor in the forward transformation.
         ZOOM is only used when MAPTYPE=\texttt{"ZOOM"}.
      }
   }
   \sstexamples{
      \sstexamplesubsection{
         wcsadd speca axis frmtype=specframe maptype=unit \latex{\linebreak}
         attrs="'system=wave,unit=Angstrom'"
      }{
         This example assumes the NDF called speca has an Axis
         structure describing wavelength in {\AA}ngstroms.  It adds a
         corresponding \xref{SpecFrame}{sun210}{SpecFrame}~ into the
         WCS component of the NDF.  The SpecFrame is connected to the
         Frame describing the NDF Axis structure using a unit Mapping.
         Subsequently, \htmlref{WCSATTRIB}{WCSATTRIB} can be used to
         modify the SpecFrame so that it describes the spectral-axis
         value in some other system (frequency, velocities of various
         forms, energy, wave number, \emph{etc.}).
      }
      \sstexamplesubsection{
         wcsadd ngc5128 pixel old\_pixel unit
      }{
         This adds a new co-ordinate Frame into the WCS component of the
         NDF called ngc5128.  The new Frame is given the domain OLD\_PIXEL
         and is a copy of the existing PIXEL Frame.  This OLD\_PIXEL Frame
         will be retained through further processing and can be used as a
         record of the original pixel co-ordinate Frame.
      }
      \sstexamplesubsection{
         wcsadd my\_data dist-lum dist(au)-lum linear tr=[0,2.0626E5,0,0,0,1]
      }{
         This adds a new co-ordinate Frame into the WCS component of the
         NDF called my\_data.  The new Frame is given the domain DIST(AU)-LUM
         and is a copy of an existing Frame with domain DIST-LUM.  The first
         axis in the new Frame is derived from the first axis in the basis
         Frame but is in different units (AU instead of parsecs).  This
         change of units is achieved by multiplying the old Frame Axis 1
         values by 2.0626E5.  The values on the second axis are copied
         without change.  You could then use application WCSATTRIB to set
         the \htmlattref{Unit}{Unit(axis)}~ attribute for Axis 1 of the
         new Frame to \texttt{"AU"}.
      }
      \sstexamplesubsection{
         wcsadd my\_data dist-lum dist(au)-lum diag diag=[2.0626E5,1]
      }{
         This does exactly the same as the previous example.
      }
      \sstexamplesubsection{
         wcsadd ax322 ! shrunk zoom zoom=0.25 mapout=zoom.ast
      }{
         This adds a new Frame to the WCS component of ax322 which is a
         one-quarter-scale copy of its current co-ordinate Frame.  The
         Mapping is also stored in the text file \texttt{zoom.ast}.
      }
      \sstexamplesubsection{
         wcsadd cube grid slid shift shift=[0,0,1024]
      }{
         This adds a new Frame to the WCS component of the NDF cube
         which matches the GRID-domain co-ordinates in the first two
         axes, but is translated by 1024 pixels on the third axis.
      }
      \sstexamplesubsection{
         wcsadd plane pixel polar math simpif simpfi \latex{\linebreak}
            forexp="'r=sqrt(x$*$x+y$*$y),theta=atan2(y,x)'" \latex{\linebreak}
            invexp="'x=r$*$cos(theta),y=r$*$sin(theta)'"
      }{
         A new Frame is added which gives pixel positions in polar
         co-ordinates.  Fortran-like expressions are supplied which define
         both the forward and inverse transformations of the Mapping.  The
         symbols \textit{x} and \textit{y} are used to represent the two input Cartesian
         pixel co-ordinate axes, and the symbols \textit{r} and \textit{theta} are used to
         represent the output polar co-ordinates.  Note, the single quotes
         are needed when running from the UNIX shell in order to prevent
         the shell interpreting the parentheses and commas within the
         expressions.
      }
      \sstexamplesubsection{
         wcsadd plane pixel polar math simpif simpfi forexp=$\wedge$ft invexp=$\wedge$it
      }{
         As above, but the expressions defining the transformations are
         supplied in two text files called \texttt{ft} and \texttt{it}, instead of being
         supplied directly.  Each file could contain the two expression on
         two separate lines.
      }
      \sstexamplesubsection{
         wcsadd ndf=! naxes=2 mapout=pcd.ast maptype=pincushion disco=5.3e-10
      }{
         This constructs a pincushion-type distortion Mapping centred
         on the origin with a distortion coefficient of 5.3e-10,
         and writes out the Mapping as a text file called pcd.ast.
         This file could then be used by REGRID to resample
         the pixels of an NDF according to this transformation.
         No NDF is accessed.
      }
      \sstexamplesubsection{
         wcsadd qmosaic frame=grid domain=polanal maptype=refndf refndf=imosaic
      }{
         This adds a new co-ordinate Frame into the WCS component of the
         NDF called qmosaic. The new Frame has domain ``POLANAL'' and is
         copied from the NDF called imosaic (an error is reported if
         there is no such Frame with imosaic). The new co-ordinate Frame
         is attached to the base Frame (\emph{i.e.} GRID co-ordinates) within
         qmosaic using a Mapping that produces alignment between qmosaic
         and imosaic.
      }
   }
   \sstnotes{
      \sstitemlist{

         \sstitem
         The new Frame has the same number of axes as the basis Frame.

         \sstitem
         An error is reported if the transformation supplied using Parameter
         TR is singular.
      }
   }
   \sstdiytopic{
      Related Applications
   }{
KAPPA: \htmlref{NDFTRACE}{NDFTRACE},
\htmlref{REGRID}{REGRID},
\htmlref{WCSATTRIB}{WCSATTRIB},
\htmlref{WCSFRAME}{WCSFRAME},
\htmlref{WCSREMOVE}{WCSREMOVE};
\xref{CCDPACK}{sun139}{}: \xref{WCSEDIT}{sun139}{WCSEDIT}.
   }
}
\sstroutine{
   WCSALIGN
}{
   Aligns a group of NDFs using World Co-ordinate System information
}{
   \sstdescription{
      This application resamples or rebins a group of input \NDFref{NDFs},
      producing corresponding output NDFs which are aligned pixel-for-pixel
      with a specified reference NDF, or POLPACK catalogue (see Parameter
      REFCAT).

      If an input NDF has more pixel axes than the reference NDF, then
      the extra pixel axes are retained unchanged in the output NDF.
      Thus, for instance, if an input RA/Dec/velocity cube is aligned
      with a reference two-dimensional galactic-longitude/latitude image, the
      output NDF will be a galactic-longitude/latitude/velocity cube.

      The transformations needed to produce alignment are derived from the
      co-ordinate system information stored in the
      \htmlref{WCS}{apndf:wcs} components of the supplied NDFs.  For each
      input NDF, alignment is first attempted in the current
      \htmlref{co-ordinate Frame}{se:domains}~ of the reference NDF.  If
      this fails, alignment is attempted in the current co-ordinate Frame
      of the input NDF.  If this fails, alignment occurs in the pixel
      co-ordinate Frame.  A message indicating which Frame alignment was
      achieved in is displayed.

      Two algorithms are available for determining the output pixel
      values: resampling and rebinning (the method used is determined by
      the REBIN parameter).

      Two methods exist for determining the bounds of the output NDFs.
      First you can give values for Parameters LBND and UBND
      which are then used as the pixel index bounds for all output
      NDFs.  Second, if a null value is given for LBND or UBND,
      default values are generated separately for each output NDF so
      that the output NDF just encloses the entire area covered by the
      corresponding input NDF.  Using the first method will ensure that
      all output NDFs have the same pixel \htmlref{origin}{apndf:origin},
      and so the resulting
      NDFs can be directly compared.  However, this may result in the
      output NDFs being larger than necessary.  In general, the second
      method results in smaller NDFs being produced, in less time.
      However, the output NDFs will have differing pixel origins which
      need to be taken into account when comparing the aligned NDFs.
   }
   \sstusage{
      wcsalign in out lbnd ubnd ref
   }
   \sstparameters{
      \sstsubsection{
         ABORT = \_LOGICAL (Read)
      }{
         This controls what happens if an error occurs whilst processing
         one of the input NDFs.  If a \texttt{FALSE} value is supplied for ABORT,
         then the error message will be displayed, but the application will
         attempt to process any remaining input NDFs.  If a \texttt{TRUE} value is
         supplied for ABORT, then the error message will be displayed, and
         the application will abort.  \texttt{[FALSE]}
      }
      \sstsubsection{
         ACC = \_REAL (Read)
      }{
         The positional accuracy required, as a number of pixels.  For
         highly non-linear projections, a recursive algorithm is used in
         which successively smaller regions of the projection are fitted
         with a least-squares linear transformation. If such a transformation
         results in a maximum positional error greater than the value
         supplied for ACC (in pixels), then a smaller region is used. High
         accuracy is paid for by larger run times.  \texttt{[0.5]}
      }
      \sstsubsection{
         ALIGNREF = \_LOGICAL (Read)
      }{
         Determines the co-ordinate system in which each input NDF is
         aligned with the reference NDF. If \texttt{TRUE}, alignment is performed
         in the co-ordinate system described by the current Frame of the WCS
         FrameSet in the reference NDF. If \texttt{FALSE}, alignment is performed
         in the co-ordinate system specified by the following set of WCS
         attributes in the reference NDF: AlignSystem, AlignStdOfRest,
         AlignOffset, AlignSpecOffset, AlignSideBand, AlignTimeScale. The
         AST library provides fixed defaults for all these. So for
         instance, AlignSystem defaults to ICRS for celestial axes and
         Wavelength for spectral axes, meaning that celestial axes will
         be aligned in ICRS and spectral axes in wavelength, by default.
         Similarly, AlignStdOfRest defaults to Heliocentric, meaning that
         by default spectral axes will be aligned in the Heliocentric rest
         frame.

         As an example, if you are aligning two spectra which both use
         radio velocity as the current WCS, but which have different rest
         frequencies, then setting ALIGNREF to \texttt{TRUE} will cause alignment
         to be performed in radio velocity, meaning that the differences
         in rest frequency are ignored. That is, a channel with 10 Km/s
         in the input is mapping onto the channel with 10 km/s in the output.
         If ALIGNREF is \texttt{FALSE} (and no value has been set for the AlignSystem
         attribute in the reference WCS), then alignment will be performed
         in wavelength, meaning that the different rest frequencies cause
         an additional shift. That is, a channel with 10 Km/s in the input
         will be mapping onto which ever output channel has the same
         wavelength, taking into account the different rest frequencies.

         As another example, consider aligning two maps which both have
         (azimuth,elevation) axes. If ALIGNREF is \texttt{TRUE}, then any given
         (az,el) values in one image will be mapped onto the exact same
         (az,el) values in the other image, regardless of whether the
         two images were taken at the same time. But if ALIGNREF is \texttt{FALSE},
         then a given (az,el) value in one image will be mapped onto
         pixel that has the same ICRS co-ordinates in the other image
         (since AlignSystem default to ICRS for celestial axes). Thus any
         different in the observation time of the two images will result
         in an additional shift.

         As yet another example, consider aligning two spectra which are
         both in frequency with respect to the LSRK, but which refer to
         different points on the sky. If ALIGNREF is \texttt{TRUE}, then a given
         LSRK frequency in one spectrum will be mapped onto the exact same
         LSRK frequency in the other image, regardless of the different sky
         positions. But if ALIGNREF is \texttt{FALSE}, then a given input frequency
         will first be converted to Heliocentric frequency (the default
         value for AlignStdOfRest is \texttt{"Heliocentric"}), and will be mapped
         onto the output channel that has the same Heliocentric frequency.
         Thus the differecen in sky positions will result in an additional
         shift.   \texttt{[FALSE]}
      }
      \sstsubsection{
         CONSERVE = \_LOGICAL (Read)
      }{
         If set \texttt{TRUE}, then the output pixel values will be scaled in
         such a way as to preserve the total data value in a feature on
         the sky.  The scaling factor is the ratio of the output pixel
         size to the input pixel size.  This option can only be used if
         the Mapping is successfully approximated by one or more linear
         transformations.  Thus an error will be reported if it used
         when the ACC parameter is set to zero (which stops the use of
         linear approximations), or if the Mapping is too non-linear to
         be approximated by a piece-wise linear transformation.  The
         ratio of output to input pixel size is evaluated once for each
         panel of the piece-wise linear approximation to the Mapping,
         and is assumed to be constant for all output pixels in the
         panel.  The dynamic default is \texttt{TRUE} if rebinning, and
         \texttt{FALSE} if resampling (see Parameter REBIN).  \texttt{[]}
      }
      \sstsubsection{
         IN = NDF (Read)
      }{
         A group of input NDFs (of any dimensionality).  This should be
         given as a comma-separated list, in which each list element
         can be:

         \ssthitemlist{

            \sstitem
            an NDF name, optionally containing wild-cards and/or regular
            expressions (\texttt{"$*$"}, \texttt{"?"}, \texttt{"[a-z]"} \emph{etc.}).

            \sstitem
            the name of a text file, preceded by an up-arrow character \texttt{"$\wedge$"}.
            Each line in the text file should contain a comma-separated list
            of elements, each of which can in turn be an NDF name (with
            optional wild-cards, \emph{etc.}), or another file specification
            (preceded by an up-arrow).  Comments can be included in the file
            by commencing lines with a hash character \texttt{"\#"}.

         }
         If the value supplied for this parameter ends with a minus
         sign \texttt{"-"}, then you are re-prompted for further input until
         a value is given which does not end with a hyphen.  All the
         NDFs given in this way are concatenated into a single group.
      }
      \sstsubsection{
         INSITU = \_LOGICAL (Read)
      }{
         If INSITU is set to \texttt{TRUE}, then no output NDFs are created.
         Instead, the pixel origin of each input NDF is modified in order
         to align the input NDFs with the reference NDF (which is a much
         faster operation than a full resampling).  This can only be done
         if the mapping from input pixel co-ordinates to reference pixel
         co-ordinates is a simple integer pixel shift of origin.  If this is
         not the case an error will be reported when the input is processed
         (what happens then is controlled by the ABORT parameter).  Also,
         in-situ alignment is only possible if null values are supplied for
         LBND and UBND.  \texttt{[FALSE]}
      }
      \sstsubsection{
         LBND() = \_INTEGER (Read)
      }{
         An array of values giving the lower pixel-index bound on each axis
         for the output NDFs.  The number of values supplied should equal
         the number of axes in the reference NDF.  The given values are used
         for all output NDFs.  If a null value (\texttt{{!}}) is given for this parameter
         or for Parameter UBND, then separate default values are calculated for
         each output NDF which result in the output NDF just encompassing
         the corresponding input NDF.  The suggested defaults are the
         lower pixel-index bounds from the reference NDF, if supplied (see Parameter REF).
      }
      \sstsubsection{
         MAXPIX = \_INTEGER (Read)
      }{
         A value which specifies an initial scale size in pixels for the
         adaptive algorithm which approximates non-linear
         \htmlref{Mappings}{se:curframe}~ with
         piece-wise linear transformations.  If MAXPIX is larger than any
         dimension of the region of the output grid being used, a first
         attempt will be made to approximate the Mapping by a linear
         transformation over the entire output region.  If a smaller value
         is used, the output region will first be divided into subregions
         whose size does not exceed MAXPIX pixels in any dimension, and then
         attempts will be made at approximation.  \texttt{[1000]}
      }
      \label{method:wcsalign}
      \sstsubsection{
         METHOD = \htmlref{LITERAL}{se:parmenu} (Read)
      }{
         The method to use when sampling the input pixel values (if
         resampling), or dividing an input pixel value up between a group
         of neighbouring output pixels (if rebinning).  For details
         of these schemes, see the descriptions of routines
         \xref{AST\_RESAMPLEx}{sun210}{AST_RESAMPLE\$<X>\$} and
         \xref{AST\_REBINx}{sun210}{AST_REBIN\$<\$X\$>\$} in
         \xref{SUN/210}{sun210}{}.  METHOD can take the following values.

         \ssthitemlist{

            \sstitem
            \texttt{"Bilinear"} --- When resampling, the output pixel values are
            calculated by bi-linear interpolation among the four nearest pixels
            values in the input NDF.  When rebinning, the input pixel value
            is divided up bi-linearly between the four nearest output pixels.
            Produces smoother output NDFs than the nearest-neighbour scheme, but
            is marginally slower.

            \sstitem
            \texttt{"Nearest"} --- When resampling, the output pixel values are assigned
            the value  of the single nearest input pixel.  When rebinning,
            the input pixel value is assigned completely to the single
            nearest output pixel.

            \sstitem
            \texttt{"Sinc"} --- Uses the ${\textrm{sinc}}({\pi}x)$ kernel, where $x$ is the pixel
            offset from the interpolation point (resampling) or transformed
            input pixel centre (rebinning), and ${\textrm{sinc}}(z)=\sin(z)/z$.  Use of this
            scheme is not recommended.

            \sstitem
            \texttt{"SincSinc"} --- Uses the ${\textrm{sinc}}({\pi}x){\textrm{sinc}}(k{\pi}x)$ kernel.  A
            valuable general-purpose scheme, intermediate in its visual effect
            on NDFs between the bi-linear and nearest-neighbour schemes.

            \sstitem
            \texttt{"SincCos"} --- Uses the ${\textrm{sinc}}({\pi}x)\cos(k{\pi}x)$ kernel.  Gives
            similar results to the \texttt{"Sincsinc"} scheme.

            \sstitem
            \texttt{"SincGauss"} --- Uses the ${\textrm{sinc}}({\pi}x)e^{-kx^2}$ kernel.  Good
            results can be obtained by matching the FWHM of the
            envelope function to the point-spread function of the
            input data (see Parameter PARAMS).

            \sstitem
            \texttt{"Somb"} --- Uses the ${\textrm{somb}}({\pi}x)$ kernel, where $x$ is the pixel
            offset from the interpolation point (resampling) or transformed
            input pixel centre (rebinning), and
            ${\textrm{somb}}(z)=2*J_{1}(z)/z$.  $J_1$ is the
            first-order Bessel function of the first kind.  This scheme is
            similar to the \texttt{"Sinc"} scheme.

            \sstitem
            \texttt{"SombCos"} --- Uses the ${\textrm{somb}}({\pi}x)\cos(k{\pi}x)$ kernel.  This
            scheme is similar to the \texttt{"SincCos"} scheme.

            \sstitem
            \texttt{"Gauss"} --- Uses the $e^{-kx^2}$ kernel.  The FWHM of the
            Gaussian is given by Parameter PARAMS(2), and the point at which to
            truncate the Gaussian to zero is given by Parameter PARAMS(1).

         }
         All methods propagate variances from input to output, but the
         variance estimates produced by interpolation schemes other than
         nearest neighbour need to be treated with care since the spatial
         smoothing produced by these methods introduces
         correlations in the variance estimates.  Also, the degree of
         smoothing produced varies across the NDF. This is because a
         sample taken at a pixel centre will have no contributions from the
         neighbouring pixels, whereas a sample taken at the corner of a
         pixel will have equal contributions from all four neighbouring
         pixels, resulting in greater smoothing and lower noise.  This
         effect can produce complex Moir\'{e} patterns in the output
         variance estimates, resulting from the interference of the
         spatial frequencies in the sample positions and in the pixel
         centre positions.  For these reasons, if you want to use the
         output variances, you are generally safer using nearest-neighbour
         interpolation.  The initial default is \texttt{"SincSinc"}.
         \texttt{[}current value\texttt{{]}}
      }
      \sstsubsection{
         OUT = NDF (Write)
      }{
         A group of output NDFs corresponding one-for-one with the list
         of input NDFs given for Parameter IN.  This should be given as
         a comma-separated list, in which each list element can be:

         \ssthitemlist{

            \sstitem
            an NDF name. If the name contains an asterisk character \texttt{"$*$"},
            the name of the corresponding input NDF (without directory or
            file suffix) is substituted for the asterisk (for instance,
            \texttt{"$*$\_al"} causes the output NDF name to be formed by appending
            the string \texttt{"\_al"} to the corresponding input NDF name).  Input
            NDF names can also be edited by including original and replacement
            strings between vertical bars after the NDF name (for instance,
            $*$\_al$|$b4$|$B1$|$ causes any occurrence of the string \texttt{"B4"}
            in the input NDF name to be replaced by the string \texttt{"B1"} before
            appending the string \texttt{"\_al"} to the result).

            \sstitem
            the name of a text file, preceded by an up-arrow character \texttt{"$\wedge$"}.
            Each line in the text file should contain a comma-separated list
            of elements, each of which can in turn be an NDF name (with
            optional editing, \emph{etc.}), or another file specification
            (preceded by an up-arrow). Comments can be included in the file
            by commencing lines with a hash character \texttt{"\#"}.

         }
         If the value supplied for this parameter ends with a hyphen
         \texttt{"-"}, then you are re-prompted for further input until
         a value is given which does not end with hyphen.  All the
         NDFs given in this way are concatenated into a single group.

         This parameter is only accessed if the INSITU parameter is \texttt{FALSE}.
      }
      \sstsubsection{
         PARAMS( 2 ) = \_DOUBLE (Read)
      }{
         An optional array which consists of additional parameters
         required by the Sinc, SincSinc, SincCos, SincGauss, Somb,
         SombCos, and Gauss methods.

         PARAMS(1) is required by all the above schemes. It is used to
         specify how many pixels are to contribute to the interpolated
         result on either side of the interpolation or binning point
         in each dimension.  Typically, a value of \texttt{2} is
         appropriate and the minimum allowed value is \texttt{1}
         (\emph{i.e.} one pixel on each side).  A value of zero or
         fewer indicates that a suitable number of pixels should be
         calculated automatically.  \texttt{[0]}

         PARAMS(2) is required only by the Gauss, SombCos, SincSinc,
         SincCos, and SincGauss schemes.  For the SombCos, SincSinc
         and SincCos schemes, it specifies the number of pixels at
         which the envelope of the function goes to zero.  The minimum
         value is \texttt{1.0}, and the run-time default value is \texttt{
         2.0}.  For the Gauss and SincGauss schemes, it specifies the
         full-width at half-maximum (FWHM) of the Gaussian envelope
         measured in output pixels.
         The minimum value is \texttt{0.1}, and the run-time default is
         \texttt{1.0}.  On astronomical NDFs and spectra, good results
         are often obtained by approximately matching the FWHM of the
         envelope function, given by PARAMS(2), to the point-spread
         function of the input data.  \texttt{[]}
      }
      \sstsubsection{
         REBIN = \_LOGICAL (Read)
      }{
         Determines the algorithm used to calculate the output pixel
         values.  If a \texttt{TRUE} value is given, a rebinning algorithm is
         used.  Otherwise, a resampling algorithm is used.  See the
         \htmlref{``Choice of Algorithm''}{choice:wcsalign} topic below.
         The initial default is \texttt{FALSE}.  \texttt{[}current value\texttt{{]}}
      }
      \sstsubsection{
         REF = NDF (Read)
      }{
         The NDF to which all the input NDFs are to be aligned. If a null
         value is supplied for this parameter, the first NDF supplied for
         Parameter IN is used. This parameter is only used if no catalogue
         is supplied for Parameter REFCAT.
      }
      \sstsubsection{
         REFCAT = FILENAME (Read)
      }{
         A POLPACK catalogue defining the WCS to which all the input NDFs
         are to be aligned. If a null value is supplied for this parameter,
         the WCS will be obtained from an NDF using Parameter REF. \texttt{[!]}
      }
      \sstsubsection{
         UBND() = \_INTEGER (Read)
      }{
         An array of values giving the upper pixel-index bound on each axis
         for the output NDFs. The number of values supplied should equal
         the number of axes in the reference NDF. The given values are used
         for all output NDFs.  If a null value (\texttt{{!}}) is given for this parameter
         or for Parameter LBND, then separate default values are calculated for
         each output NDF which result in the output NDF just encompassing
         the corresponding input NDF.  The suggested defaults are the
         upper pixel-index bounds from the reference NDF, if supplied (see Parameter REF).
      }
      \sstsubsection{
         WLIM = \_REAL (Read)
      }{
         This parameter is only used if REBIN is set \texttt{TRUE}.  It specifies the
         minimum number of good pixels which must contribute to an output pixel
         for the output pixel to be valid.  Note, fractional values are
         allowed.  A null (\texttt{{!}}) value causes a very small positive value to
         be used resulting in output pixels being set bad only if they
         receive no significant contribution from any input pixel.  \texttt{[!]}
      }
   }
   \sstexamples{
      \sstexamplesubsection{
         wcsalign image1 image1\_al ref=image2 accept
      }{
         This example resamples the NDF called image1 so that it is aligned
         with the NDF call image2, putting the output in image1\_al.  The output
         image has the same pixel-index bounds as image2 and inherits WCS
         information from image2.
      }
      \sstexamplesubsection{
         wcsalign m51$*$ $*$\_al lbnd=! accept
      }{
         This example resamples all the NDFs with names starting with the
         string \texttt{"m51"} in the current directory so that
         they are aligned with the first input NDF. The output NDFs
         have the same names as the input NDFs, but extended with the
         string \texttt{"\_al"}. Each output NDF is just big enough to contain all
         the pixels in the corresponding input NDF.
      }
      \sstexamplesubsection{
         wcsalign $\wedge$in.lis $\wedge$out.lis lbnd=! accept
      }{
         This example is like the previous example, except that the names
         of the input NDFs are read from the text file \texttt{in.lis}, and the
         names of the corresponding output NDFs are read from text file
         \texttt{out.lis}.
      }
   }
   \label{choice:wcsalign}
   \sstdiytopic{
      Choice of Algorithm
   }{
      The algorithm used to produce the output image is determined by
      the REBIN parameter, and is based either on resampling the output
      image or rebinning the input image.

      The resampling algorithm steps through every pixel in the output
      image, sampling the input image at the corresponding position and
      storing the sampled input value in the output pixel.  The method
      used for sampling the input image is determined by the METHOD
      parameter.  The rebinning algorithm steps through every pixel in
      the input image, dividing the input pixel value between a group
      of neighbouring output pixels, incrementing these output pixel
      values by their allocated share of the input pixel value, and
      finally normalising each output value by the total number of
      contributing input values. The way in which the input sample is
      divided between the output pixels is determined by the METHOD
      parameter.

      Both algorithms produce an output in which the each pixel value is
      the weighted mean of the near-by input values, and so do not alter
      the mean pixel values associated with a source, even if the pixel
      size changes. Thus the total data sum in a source will change if
      the input and output pixel sizes differ.  However, if the CONSERVE
      parameter is set \texttt{TRUE}, the output values are scaled by the ratio
      of the output to input pixel size, so that the total data sum in a
      source is preserved.

      A difference between resampling and rebinning is that resampling
      guarantees to fill the output image with good pixel values
      (assuming the input image is filled with good input pixel values),
      whereas holes can be left by the rebinning algorithm if the output
      image has smaller pixels than the input image.  Such holes occur
      at output pixels which receive no contributions from any input
      pixels, and will be filled with the value zero in the output
      image.  If this problem occurs the solution is probably to change
      the width of the pixel spreading function by assigning a larger
      value to PARAMS(1) and/or PARAMS(2) (depending on the specific
      METHOD value being used).

      Both algorithms have the capability to introduce artefacts into the
      output image.  These have various causes described below.

      \sstitemlist{
         \sstitem
         Particularly sharp features in the input can cause rings around
         the corresponding features in the output image. This can be
         minimised by suitable settings for the METHOD and PARAMS
         parameters. In general such rings can be minimised by using a
         wider interpolation kernel (if resampling) or spreading function
         (if rebinning), at the cost of degraded resolution.

         \sstitem
         The approximation of the Mapping using a piece-wise linear
         transformation (controlled by Parameter ACC) can produce artefacts
         at the joints between the panels of the approximation.  They are
         caused by the discontinuities between the adjacent panels of the
         approximation, and can be minimised by reducing the value assigned
         to the ACC parameter.
      }
   }
   \sstnotes{
      \sstitemlist{

         \sstitem
         \htmlref{WCS}{apndf:wcs} information (including the current
         co-ordinate Frame) is propagated from the reference NDF to all
         output NDFs.

         \sstitem
         \htmlref{QUALITY}{apndf:quality} is propagated from input to output
         only if Parameter METHOD is set to \texttt{"Nearest"} and REBIN is
         set to \texttt{FALSE}.
      }
   }
   \sstdiytopic{
      Related Applications
   }{
     KAPPA: \htmlref{WCSFRAME}{WCSFRAME},
     \htmlref{REGRID}{REGRID};
     \xref{CCDPACK}{sun139}{}: \xref{TRANNDF}{sun139}{TRANNDF}.
   }
   \sstimplementationstatus{
      \sstitemlist{

         \sstitem
         This routine correctly processes the DATA, \htmlref{VARIANCE}{apndf:variance},
         \htmlref{LABEL}{apndf:label}, \htmlref{TITLE}{apndf:title},
         \htmlref{UNITS}{apndf:units}, \htmlref{WCS}{apndf:wcs}, and
         \htmlref{HISTORY}{apndf:history} components of the input NDFs (see the
         \htmlref{METHOD}{method:wcsalign} parameter for notes on the interpretation of output variances).

         \sstitem
         Processing of \htmlref{bad pixels}{se:masking} and automatic
         \htmlref{quality masking}{se:qualitymask} are supported.

         \sstitem
         All \htmlref{non-complex numeric data types}{ap:HDStypes} can be handled.
         If REBIN is \texttt{TRUE}, the data type will be converted to one of \_INTEGER,
         \_DOUBLE or \_REAL for processing.
      }
   }
}

\sstroutine{
   WCSATTRIB
}{
   Manages attribute values associated with the \htmlref{WCS}{apndf:wcs} component of an NDF
}{
   \sstdescription{
      This application can be used to manage the values of \htmlref{attributes}{ap:frmatt}
      associated with the current \htmlref{co-ordinate Frame}{se:domains}~  of an \NDFref{NDF} (title, axis
      labels, axis units, \emph{etc.}).

      Each attribute has a name, a value, and a state.  This application
      accesses all attribute values as character strings, converting to
      and from other data types as necessary.  The attribute state is a
      Boolean flag (\emph{i.e.} TRUE or FALSE) indicating whether or not a value
      has been assigned to the attribute.  If no value has been assigned to
      an attribute, then it adopts a default value until an explicit value
      is assigned to it.  An attribute value can be cleared, causing the
      attribute to revert to its default value.

      The operation performed by this application is controlled by
      Parameter MODE, and can:

      \sstitemlist{

         \sstitem
         display the value of an attribute;

         \sstitem
         set a new value for an attribute;

         \sstitem
         set new values for a list of attributes;

         \sstitem
         clear an attribute value; and

         \sstitem
         test the state of an attribute.

      }
      Note, the attributes of the PIXEL, GRID and AXIS Frames are managed
      internally by the NDF library.  They may be examined using this
      application, but an error is reported if any attempt is made to
      change them.  The exception to this is that the
      \htmlattref{Domain}{Domain}~ attribute may
      be changed, resulting in a copy of the Frame being added to the WCS
      component of the NDF with the new Domain name.  The AXIS Frame is
      derived from the AXIS structures within the NDF, so the AXLABEL
      and AXUNITS commands may be used to change the axis label or units
      string for the AXIS Frame.
   }
   \sstusage{
      wcsattrib ndf mode name newval
   }
   \sstparameters{
      \sstsubsection{
         MODE = \htmlref{LITERAL}{se:parmenu} (Read)
      }{
         The operation to be performed on the attribute specified by
         Parameter NAME.  It can be one of the following options.

         \ssthitemlist{

            \sstitem
            \texttt{"Clear"} --- Clears the current attribute value, causing it to
            revert to its default value.

            \sstitem
            \texttt{"Get"} --- The current value of the attribute is displayed on the
            screen and written to output Parameter VALUE.  If the attribute
            has not yet been assigned a value (or has been cleared), then the
            default value will be displayed.

            \sstitem
            \texttt{"MSet"} --- Assigns new values to multiple attributes. The
            attribute names and values are obtained using Parameter SETTING.

            \sstitem
            \texttt{"Set"} --- Assigns a new value, given by Parameter NEWVAL, to the
            attribute.

            \sstitem
            \texttt{"Test"} --- Displays \texttt{"TRUE"} if the attribute has been assigned a
            value, and \texttt{"FALSE"} otherwise (in which case the attribute will
            adopt its default value).  This flag is written to the output
            Parameter STATE.

         }
         The initial suggested default is \texttt{"Get"}.
      }
      \sstsubsection{
         NAME = LITERAL (Read)
      }{
         The attribute name.  It is not used if MODE is \texttt{"MSet"}.
      }
      \sstsubsection{
         NDF = NDF (Read and Write)
      }{
         The NDF to be modified.  When MODE=\texttt{"Get"}, the access is Read only.
      }
      \sstsubsection{
         NEWVAL = LITERAL (Read)
      }{
         The new value to assign to the attribute.  It is only used if
         MODE is \texttt{"Set"}.
      }
      \sstsubsection{
         REMAP = \_LOGICAL (Read)
      }{
         Only accessed if MODE is \texttt{ "Set"} or \texttt{ "Clear"}.  If REMAP is
         \texttt{ TRUE}, then the \htmlref{Mappings}{se:curframe}~ which connect
         the current Frame to the other Frames within the
         \htmlref{WCS FrameSet}{se:curframe}~ will be modified (if necessary)
         to maintain the FrameSet integrity.  For instance, if the current
         Frame of the NDF represents FK5 RA and DEC, and you change {\att System}
         from \texttt{ "FK5"} to \texttt{ "Galactic"}, then the Mappings which connect
         the SKY Frame to the other Frames (\emph{e.g.} PIXEL, AXIS) will be
         modified so that each pixel corresponds to the correct Galactic
         co-ordinates.  If REMAP is \texttt{FALSE}, then the Mappings will not be
         changed.  This can be useful if the FrameSet has incorrect attribute
         values for some reason, which need to be corrected without altering
         the Mappings to take account of the change.  \texttt{ [TRUE]}
      }
      \sstsubsection{
         SETTING = LITERAL (Read)
      }{
         This is only accessed if MODE is set to \texttt{"MSet"}.  It should hold
         a comma-separated list of \texttt{"<attribute>=<value>"} strings, where
         \texttt{<attribute>} is the name of an attribute and \texttt{<value>} is
         the value to assign to the attribute.
      }
   }
   \sstresparameters{
      \sstsubsection{
         STATE = \_LOGICAL (Write)
      }{
         On exit, this holds the state of the attribute on entry to this
         application.  It is not used if MODE is \texttt{"MSet"}.
      }
      \sstsubsection{
         VALUE = LITERAL (Write)
      }{
         On exit, this holds the value of the attribute on entry to this
         application.  It is not used if MODE is \texttt{"MSet"}.
      }
   }
   \sstexamples{
      \sstexamplesubsection{
         wcsattrib my\_spec set System freq
      }{
         This sets the \htmlattref{System}{System}~ attribute of the
         current co-ordinate Frame in the NDF called my\_Spec so that
         the Frame represents frequency (this assumes the current
         Frame is a \xref{SpecFrame}{sun210}{SpecFrame}).  The
         Mappings between the current Frame and the other Frames are
         modified to take account of the change of system.
      }
      \sstexamplesubsection{
         wcsattrib my\_spec mset setting='unit(1)=km/s,system(1)=vrad'
      }{
         This sets new values of "km/s" and "vrad" simultaneously for the
         Unit and System attributes for the first axis of the NDF called
         my\_spec.
      }
      \sstexamplesubsection{
         wcsattrib ngc5128 set title "Polarization map of Centaurus-A"
      }{
         This sets the \htmlattref{Title}{Title}~ attribute of the current co-ordinate Frame in
         the NDF called ngc5128 to the string \texttt{"Polarization map of Centaurus-A"}.
      }
      \sstexamplesubsection{
         wcsattrib my\_data set domain saved\_pixel
      }{
         This sets the \att{Domain} attribute of the current co-ordinate Frame in
         the NDF called my\_data to the string SAVED\_PIXEL.
      }
      \sstexamplesubsection{
         wcsattrib my\_data set format(1) "\%10.5G"
      }{
         This sets the \htmlattref{Format}{Format(axis)}~ attribute for Axis 1 in
         the current co-ordinate Frame in the NDF called my\_data, so
         that axis values are formatted as floating-point values using
         a minimum field width of ten characters, and displaying five
         significant figures.  An exponent is used if necessary.
      }
      \sstexamplesubsection{
         wcsattrib ngc5128 set format(2) bdms.2
      }{
         This sets the \htmlattref{Format}{Format(axis)}~ attribute for Axis 2 in
         the current co-ordinate Frame in the NDF called ngc5128, so that axis
         values are formatted as separate degrees, minutes and seconds
         fields, separated by blanks.  The seconds field has two decimal places.
         This assumes the current co-ordinate Frame in the NDF is a celestial
         co-ordinate Frame.
      }
      \sstexamplesubsection{
         wcsattrib my\_data get label(1)
      }{
         This displays the label associated with the first axis of the
         current co-ordinate Frame in the NDF called my\_data.  A default
         label is displayed if no value has been set for this attribute.
      }
      \sstexamplesubsection{
         wcsattrib my\_data test label(1)
      }{
         This displays \texttt{"TRUE"} if a value has been set for the
         \htmlattref{Label}{Label(axis)}~ attribute for the first axis of the
         current co-ordinate Frame in
         the NDF called my\_data, and \texttt{"FALSE"} otherwise.
      }
      \sstexamplesubsection{
         wcsattrib my\_data clear label(1)
      }{
         This clears the \att{Label} attribute for the first axis of the current
         co-ordinate Frame in the NDF called my\_data.  It reverts to its
         default value.
      }
      \sstexamplesubsection{
         wcsattrib my\_data set equinox J2000 remap=no
      }{
         This assumes that the \htmlattref{Equinox}{Equinox}~ attribute for the current
         co-ordinate Frame within NDF \texttt{"my\_data"} has been set to some
         incorrect value, which needs to be corrected to \texttt{"J2000"}.  The
         REMAP parameter is set \texttt{FALSE}, which prevents the inter-Frame
         Mappings from being altered to take account of the new Equinox
         value.  This means that each pixel in the NDF will retain its
         original RA and DEC values (but they will now be interpreted as
         J2000).  If REMAP had been left at its default value of \texttt{TRUE},
         then the RA and DEC associated with each pixel would have been
         modified in order to precess them from the original (incorrect)
         equinox to J2000.
      }
   }
   \sstnotes{
      \sstitemlist{

         \sstitem
         An error is reported if an attempt is made to set or clear the
         Base Frame in the WCS component.

         \sstitem
         The Domain names GRID, FRACTION, AXIS, and PIXEL are reserved for use by
         the NDF library and an error will be reported if an attempt is made
         to assign one of these values to any Frame.
      }
   }
   \sstdiytopic{
      Related Applications
   }{
KAPPA: \htmlref{AXLABEL}{AXLABEL},
\htmlref{AXUNITS}{AXUNITS},
\htmlref{NDFTRACE}{NDFTRACE},
\htmlref{WCSADD}{WCSADD},
\htmlref{WCSFRAME}{WCSFRAME},
\htmlref{WCSREMOVE}{WCSREMOVE},
\htmlref{WCSCOPY}{WCSCOPY}.
   }
}

\sstroutine{
   WCSCOPY
}{
   Copies WCS information from one NDF to another
}{
   \sstdescription{
      This application copies the \htmlref{WCS}{apndf:wcs}~ component from one
      \NDFref{NDF} to another, optionally modifying it to take account of a linear
      mapping between the \htmlref{pixel co-ordinates}{se:pixgrd}~ in the two
      NDFs.  It can be used, for instance, to rectify the loss of WCS information produced
      by older applications which do not propagate the WCS component.
   }
   \sstusage{
      wcscopy ndf like [tr] [confirm]
   }
   \sstparameters{
      \sstsubsection{
         CONFIRM = \_LOGICAL (Read)
      }{
         If \texttt{TRUE}, the user is asked for confirmation before replacing any
         existing WCS component within the input NDF.  No confirmation is
         required if there is no WCS component in the input NDF.  \texttt{[TRUE]}
      }
      \sstsubsection{
         LIKE = NDF (Read)
      }{
         The reference NDF data structure from which WCS information is to be
         copied.
      }
      \sstsubsection{
         NDF = NDF (Read and Write)
      }{
         The input NDF data structure in which the WCS information is to be
         stored.  Any existing WCS component is over-written (see Parameter
         CONFIRM).
      }
      \sstsubsection{
         OK = \_LOGICAL (Read)
      }{
         This parameter is used to get a confirmation that an existing
         WCS component within the input NDF can be over-written.
      }
      \sstsubsection{
         TR( ) = \_DOUBLE (Read)
      }{
         The values of this parameter are the coefficients of a linear
         transformation from pixel co-ordinates in the reference NDF given
         for Parameter LIKE, to pixel co-ordinates in the input NDF given
         for Parameter NDF.  For instance, if a feature has pixel co-ordinates
         (\textit{X},\textit{Y},\textit{Z},\dots) in the reference NDF, and pixel co-ordinates
         ($U$,$V$,$W$,\dots) in the input NDF, then the following transformations
         would be used, depending on how many axes each NDF has:

         \begin{itemize}

            \item one-dimensional:
             \[  U = TR(1) + TR(2)*X  \]

            \item two-dimensional:
             \[  U = TR(1) + TR(2)*X + TR(3)*Y \]
             \[  V = TR(4) + TR(5)*X + TR(6)*Y \]

            \item three-dimensional:
             \[  U = TR(1) + TR(2)*X + TR(3)*Y + TR(4)*Z \]
             \[  V = TR(5) + TR(6)*X + TR(7)*Y + TR(8)*Z \]
             \[  W = TR(9) + TR(10)*X + TR(11)*Y + TR(12)*Z \]

         \end{itemize}

         If a null value (\texttt{{!}}) is given it is assumed that the pixel co-ordinates
         of a given feature are identical in the two NDFs.  \texttt{[!]}
      }
   }
   \sstexamples{
      \sstexamplesubsection{
         wcscopy m51\_sim m51
      }{
         This copies the WCS component from the NDF called m51 to the NDF
         called m51\_sim, which may hold the results of a numerical simulation
         for instance.  It is assumed that the two NDFs are aligned (\emph{i.e.} the
         pixel co-ordinates of any feature are the same in both NDFs).
      }
      \sstexamplesubsection{
         wcscopy m51\_sqorst m51 [125,0.5,0.0,125,0.0,0.5]
      }{
        This example assumes that an application similar to SQORST has
        previously been used to change the size of a two-dimensional NDF
        called m51, producing a new NDF called m51\_sqorst.  It is assumed
        that this SQORST-like application does not propagate WCS and also
        resets the pixel origin to [1,~1].  In fact, this is what
        \htmlref{SQORST}{SQORST} actually did, prior to \KAPPA\ version 1.0.
        This example shows how WCSCOPY can be used to rectify this by copying
        the WCS component from the original NDF m51 to the squashed NDF m51\_sqorst,
        modifying it in the process to take account of both the
        squashing and the resetting of the pixel origin produced by
        SQORST.  To do this, you need to work out the
        transformation in pixel co-ordinates produced by SQORST, and specify
        this when running WCSCOPY using the TR parameter.  Let's assume the first
        axis of NDF m51 has pixel-index bounds of I1:I2 (these values
        can be found using NDFTRACE).  If the first axis in the squashed
        NDF m51\_sqorst spans\textit{M} pixels (where \textit{M} is the value assigned to
        SQORST Parameter XDIM), then it will have pixel-index bounds of
        1:\textit{M}.  Note, the lower bound is 1 since the pixel origin has been
        reset by SQORST.  The squashing factor for the first axis is
        then:
          \[
          FX = M/(I2 - I1 + 1)
          \]
         and the shift in the pixel origin is:
          \[
          SX = FX*( 1 - I1 )
          \]
         Likewise, if the bounds of the second axis in m51 are J1:J2, and
         SQORST Parameter YDIM is set to \textit{N}, then the squashing factor for
         the second axis is:
          \[
          FY = N/(J2 - J1 + 1)
          \]
         and the shift in the pixel origin is:
          \[
          SY = FY*( 1 - J1 )
          \]
         You would then use the following values for Parameter TR when
         running WCSCOPY:
          \[
          TR = [SX, FX, 0.0, SY, 0.0, FY]
          \]
         Note, the zero terms indicate that the axes are independent (\emph{i.e.}
         there is no rotation of the image).  The numerical values in the
         example are for an image with pixel-index bounds of 52:251 on both
         axes which was squashed by SQORST to produce an image with 100 pixels
         on each axis.
      }
   }
   \sstnotes{
      \sstitemlist{

         \sstitem
         An error is reported if the transformation supplied using Parameter
         TR is singular.

         \sstitem
         The pixel with pixel index \textit{I} spans a range of pixel co-ordinate
         from $(I - 1.0)$ to \textit{I}.

         \sstitem
         The pixel indices of the bottom-left pixel in an NDF is called
         the \emph{pixel origin} of the NDF, and can take any value.  The pixel
         origin can be examined using application NDFTRACE and set using
         application \htmlref{SETORIGIN}{SETORIGIN}.  WCSCOPY takes account of the
         pixel origins in the two NDFs when modifying the WCS component.  Thus, if
         a null value is given for Parameter TR, the supplied WCS component may still
         be modified if the two NDFs have different pixel origins.
      }
   }
   \sstdiytopic{
      Related Applications
   }{
KAPPA: \htmlref{NDFTRACE}{NDFTRACE},
\htmlref{WCSADD}{WCSADD},
\htmlref{WCSATTRIB}{WCSATTRIB},
\htmlref{WCSFRAME}{WCSFRAME},
\htmlref{WCSREMOVE}{WCSREMOVE}.
   }
}
\sstroutine{
   WCSFRAME
}{
   Changes the current co-ordinate Frame in the \htmlref{WCS}{apndf:wcs} component of an NDF
}{
   \sstdescription{
      This application displays the current \htmlref{co-ordinate Frame}{se:domains}~  associated
      with an \NDFref{NDF} and then allows the user to specify a new Frame.  The
      current co-ordinate Frame determines the co-ordinate system in
      which positions within the NDF will be expressed when communicating
      with the user.

      Having selected a new current co-ordinate Frame, its attributes
      (such the specific system it uses to represents points within its
      Domain, its units, \emph{etc.}) can be changed using \KAPPA\ command
      \htmlref{WCSATTRIB}{WCSATTRIB}.
   }
   \sstusage{
      wcsframe ndf frame epoch
   }
   \sstparameters{
      \sstsubsection{
         EPOCH = \_DOUBLE (Read)
      }{
         If a \emph{Sky Co-ordinate System} specification is supplied (using
         Parameter FRAME) for a celestial co-ordinate system, then an epoch
         value is needed to qualify it.  This is the epoch at which the
         supplied sky positions were determined.  It should be given as a
         decimal years value, with or without decimal places  (\texttt{"1996.8"} for
         example).  Such values are interpreted as a Besselian epoch if less
         than 1984.0 and as a Julian epoch otherwise.
      }
      \sstsubsection{
         FRAME = LITERAL (Read)
      }{
         A string specifying the new co-ordinate Frame.  If a null parameter
         value is supplied, then the current Frame is left unchanged.  The
         suggested default is the Domain (or index if the Domain is not
         set) of the current Frame.  The string can be one of the following:

         \ssthitemlist{

            \sstitem
            A \htmlref{domain name}{se:domains}~ such as SKY, SPECTRUM, AXIS, PIXEL.  The two
            `pseudo-domains' WORLD and DATA may be supplied and will be
            translated into PIXEL and AXIS respectively, so long as the WCS
            component of the NDF does not contain Frames with these domains.

            \sstitem
            An integer value giving the index of the required Frame within
            the WCS component.

            \sstitem
            An IRAS90 \emph{Sky Co-ordinate System} (SCS) values such as
            \texttt{"EQUAT(J2000)"} (see \xref{SUN/163}{sun163}{}).  Using an SCS value
            is equivalent to specifying \texttt{"SKY"} for this parameter and then setting
            the \htmlattref{System}{System}~ attribute (to \texttt{"FK5"},
            \texttt{"Galactic"}, \emph{etc.}) using \KAPPA\ command
            WCSATTRIB.  The specific system used to describe positions in other
            Domains (SPECTRUM, for instance) must be set using WCSATTRIB.
         }
      }
      \sstsubsection{
         NDF = NDF (Read and Write)
      }{
         The NDF data structure in which the current co-ordinate Frame is to
         be modified.
      }
   }
   \sstexamples{
      \sstexamplesubsection{
         wcsframe m51 pixel
      }{
         This chooses pixel co-ordinates for the current co-ordinate
         Frame in the NDF m51.
      }
      \sstexamplesubsection{
         wcsframe m51 sky
      }{
         This chooses celestial co-ordinates for the current co-ordinate
         Frame in the NDF m51 (if available).  The specific celestial
         co-ordinate system (FK5, Galactic, \emph{etc.}) will depend on the contents
         of the WCS component of the NDF, but may be changed by setting a
         new value for the \htmlattref{System}{System}~ attribute using the
         WCSATTRIB command.
      }
      \sstexamplesubsection{
         wcsframe m51 spectral
      }{
         This chooses spectral co-ordinates for the current co-ordinate
         Frame in the NDF m51 (if available).  The specific spectral
         co-ordinate system (wavelength, frequency, \emph{etc.}) will depend on the
         contents of the WCS component of the NDF, but may be changed by
         setting a new value for the \att{System} attribute using the WCSATTRIB
         command.
      }
      \sstexamplesubsection{
         wcsframe m51 equ(J2000) epoch=1998.2
      }{
         This chooses equatorial (RA/DEC) co-ordinates referred to the
         equinox at Julian epoch 2000.0 for the current co-ordinate
         Frame in the NDF m51.  The positions were determined at the
         Julian epoch 1998.2 (this is needed to correct positions for
         the fictitious proper motions which may be introduced when
         converting between different celestial co-ordinate systems).
      }
      \sstexamplesubsection{
         wcsframe m51 2
      }{
         This chooses the second co-ordinate Frame in the WCS component
         of the NDF.
      }
      \sstexamplesubsection{
         wcsframe m51 data
      }{
         This chooses a co-ordinate Frame with domain DATA if one exists,
         or the AXIS co-ordinate Frame otherwise.
      }
      \sstexamplesubsection{
         wcsframe m51 world
      }{
         This chooses a co-ordinate Frame with domain WORLD if one exists,
         or the PIXEL co-ordinate Frame otherwise.
      }
   }
   \sstnotes{
      \sstitemlist{

         \sstitem
         The current co-ordinate Frame in the supplied NDF is not displayed
         if a value is assigned to Parameter FRAME on the command line.

         \sstitem
         This routine may add a new co-ordinate Frame into the WCS component
         of the NDF.

         \sstitem
         The NDFTRACE command can be used to examine the co-ordinate
         Frames in the WCS component of an NDF.
      }
   }
   \sstdiytopic{
      Related Applications
   }{
KAPPA: \htmlref{NDFTRACE}{NDFTRACE},
\htmlref{WCSATTRIB}{WCSATTRIB},
\htmlref{WCSCOPY}{WCSCOPY},
\htmlref{WCSREMOVE}{WCSREMOVE}.
   }
}
\sstroutine{
   WCSMOSAIC
}{
   Tiles a group of NDFs using World Co-ordinate System information
}{
   \sstdescription{
      This application aligns and rebins a group of input NDFs into a
      single output NDF.  It differs from WCSALIGN in both the algorithm
      used, and in the requirements placed on the input NDFs.  WCSMOSAIC
      requires that the transformation from pixel to WCS co-ordinates
      be defined in each input NDF, but (unlike \htmlref{WCSALIGN}{WCSALIGN})
      the inverse transformation from WCS to pixel co-ordinates need not be
      defined.  For instance, this means that WCSMOSAIC can process data in
      which the WCS position of each input pixel is defined via a look-up
      table rather than an analytical expression.  Note however, that the WCS
      information in the reference NDF (see Parameter REF) must have a defined
      inverse transformation.

      The WCSMOSAIC algorithm proceeds as follows.  First, the output NDF
      is filled with zeros.  An associated array of weights (one weight
      for each output pixel) is created and is also filled with zeros.
      Each input NDF is then processed in turn.  For each pixel in the
      current input NDF, the corresponding transformed position in the
      output NDF is found (based on the \htmlref{WCS}{apndf:wcs}
      information in both NDFs).
      The input pixel value is then divided up between a small group of
      output pixels centred on this central output position.  The method
      used for choosing the fraction of the input pixel value assigned
      to each output pixel is determined by the METHOD and PARAMS
      parameters.  Each of the affected output pixel values is then
      incremented by its allocated fraction of the input pixel value.
      The corresponding weight values are incremented by the fractions
      used (that is, if 0.25 of an input pixel is assigned to an output
      pixel, the weight for the output pixel is incremented by 0.25).
      Once all pixels in the current input NDF have been rebinned into
      the output NDF in this way, the algorithm proceeds to rebin the
      next input NDF in the same way.  Once all input NDFs have been
      processed, output pixels which have a weight less than the value
      given by Parameter WLIM are set bad.  The output NDF may then
      optionally (see Parameter NORM) be normalised by dividing it
      by the weights array.  This normalisation of the output NDF takes
      account of any difference in the number of pixels contributing to
      each output pixel, and also removes artefacts which may be
      produced by aliasing between the input and output pixel grids.
      Thus each output pixel value is a weighted mean of the input pixel
      values from which it receives contributions.  This means that the
      units of the output NDF are the same as the input NDF.  In
      particular, any difference between the input and output pixel
      sizes is ignored, resulting in the total input data sum being
      preserved only if the input and output NDFs have equal pixel\
      sizes.  However, an option exists to scale the input values before
      use so that the total data sum in each input NDF is preserved even
      if the input and output pixel sizes differ (see Parameter
      CONSERVE).

      If the input NDFs contain variances, then these are propagated to
      the output.  Alternatively, output variances can be generated from
      the spread of input values contributing to each output pixel (see
      Parameter GENVAR). Any input variances can also be used to weight
      the input data (see Parameter VARIANCE).  By default, all input data
      are given equal weight. An additional weight for each NDF can be
      specified using Parameter WEIGHTS.

      The transformations needed to produce alignment are derived from
      the co-ordinate system information stored in the WCS components of
      the supplied NDFs.  For each input NDF, alignment is first
      attempted in the current \htmlref{co-ordinate Frame}{se:domains}~
      of the reference NDF. If this fails, alignment is attempted in the
      current co-ordinate Frame of the input NDF.  If this fails, alignment
      occurs in the pixel co-ordinate Frame.  A message indicating which Frame
      alignment was achieved in is displayed.
   }
   \sstusage{
      wcsmosaic in out lbnd ubnd ref
   }
   \sstparameters{
      \sstsubsection{
         ACC = \_REAL (Read)
      }{
         The positional accuracy required, as a number of pixels.  For
         highly non-linear projections, a recursive algorithm is used in
         which successively smaller regions of the projection are
         fitted with a least-squares linear transformation.  If such a
         transformation results in a maximum positional error greater
         than the value supplied for ACC (in pixels), then a smaller
         region is used.  High accuracy is paid for by longer run times.
         \texttt{[0.05]}
      }
      \sstsubsection{
         ALIGNREF = \_LOGICAL (Read)
      }{
         Determines the co-ordinate system in which each input NDF is
         aligned with the reference NDF. If \texttt{TRUE}, alignment is performed
         in the co-ordinate system described by the current Frame of the WCS
         FrameSet in the reference NDF. If \texttt{FALSE}, alignment is performed
         in the co-ordinate system specified by the following set of WCS
         attributes in the reference NDF: AlignSystem AlignStdOfRest,
         AlignOffset, AlignSpecOffset, AlignSideBand, AlignTimeScale. The
         AST library provides fixed defaults for all these. So for
         instance, AlignSystem defaults to ICRS for celestial axes and
         Wavelength for spectral axes, meaning that celestial axes will
         be aligned in ICRS and spectral axes in wavelength, by default.
         Similarly, AlignStdOfRest defaults to Heliocentric, meaning that
         by default spectral axes will be aligned in the Heliocentric rest
         frame.

         As an example, if you are mosaicing two spectra which both use
         radio velocity as the current WCS, but which have different rest
         frequencies, then setting ALIGNREF to \texttt{TRUE} will cause alignment
         to be performed in radio velocity, meaning that the differences
         in rest frequency are ignored. That is, a channel with 10 Km/s
         in the input is mapping onto the channel with 10 km/s in the output.
         If ALIGNREF is \texttt{FALSE} (and no value has been set for the AlignSystem
         attribute in the reference WCS), then alignment will be performed
         in wavelength, meaning that the different rest frequencies cause
         an additional shift. That is, a channel with 10 Km/s in the input
         will be mapping onto which ever output channel has the same
         wavelength, taking into account the different rest frequencies.

         As another example, consider mosaicing two maps which both have
         (azimuth,elevation) axes. If ALIGNREF is \texttt{TRUE}, then any given
         (az,el) values in one image will be mapped onto the exact same
         (az,el) values in the other image, regardless of whether the
         two images were taken at the same time. But if ALIGNREF is \texttt{FALSE},
         then a given (az,el) value in one image will be mapped onto
         pixel that has the same ICRS co-ordinates in the other image
         (since AlignSystem default to ICRS for celestial axes). Thus any
         different in the observation time of the two images will result
         in an additional shift.

         As yet another example, consider mosaicking two spectra which are
         both in frequency with respect to the LSRK, but which refer to
         different points on the sky. If ALIGNREF is \texttt{TRUE}, then a given
         LSRK frequency in one spectrum will be mapped onto the exact same
         LSRK frequency in the other image, regardless of the different sky
         positions. But if ALIGNREF is \texttt{FALSE}, then a given input frequency
         will first be converted to Heliocentric frequency (the default
         value for AlignStdOfRest is \texttt{Heliocentric"}), and will be mapped
         onto the output channel that has the same Heliocentric frequency.
         Thus the difference in sky positions will result in an additional
         shift.   \texttt{[FALSE]}
      }
      \sstsubsection{
         CONSERVE = \_LOGICAL (Read)
      }{
         If set \texttt{TRUE}, then the output pixel values will be scaled in
         such a way as to preserve the total data value in a feature on
         the sky.  The scaling factor is the ratio of the output pixel
         size to the input pixel size.  This option can only be used if
         the Mapping is successfully approximated by one or more linear
         transformations.  Thus an error will be reported if it used
         when the ACC parameter is set to zero (which stops the use of
         linear approximations), or if the Mapping is too non-linear to
         be approximated by a piece-wise linear transformation.  The
         ratio of output to input pixel size is evaluated once for each
         panel of the piece-wise linear approximation to the Mapping,
         and is assumed to be constant for all output pixels in the
         panel.  This parameter is ignored if the NORM parameter is set
         \texttt{FALSE}.  \texttt{[TRUE]}
      }
      \sstsubsection{
         GENVAR = \_LOGICAL (Read)
      }{
         If \texttt{TRUE}, output variances are generated based on the spread of input
         pixel values contributing to each output pixel.  Any input variances
         then have no effect on the output variances (although input variances
         will still be used to weight the input data if the VARIANCE parameter
         is set \texttt{TRUE}).  If GENVAR is set \texttt{FALSE}, the output variances
         are based on the variances in the input NDFs, so long as all input NDFs
         contain variances (otherwise the output NDF will not contain any
         variances).   If a null (\texttt{{!}}) value is supplied, then a value of
         \texttt{FALSE} is adopted if and only if all the input NDFs have variance
         components (\texttt{TRUE} is used otherwise).  \texttt{[FALSE]}
      }
      \sstsubsection{
         IN = NDF (Read)
      }{
         A group of input NDFs (of any dimensionality).  This should be
         given as a comma-separated list, in which each list element
         can be one of the following options.

         \ssthitemlist{

            \sstitem
            An NDF name, optionally containing wild-cards and/or regular
            expressions (\texttt{"$*$"}, \texttt{"?"}, \texttt{"[a-z]"} \emph{etc.}).

            \sstitem
            The name of a text file, preceded by an up-arrow character
            \texttt{"$\wedge$"}.  Each line in the text file should
            contain a comma-separated list of elements, each of which
            can in turn be an NDF name (with optional wild-cards,
            \emph{etc.}), or another file specification (preceded by an
            up-arrow).  Comments can be included in the file by
            commencing lines with a hash character \texttt{"\#"}.

         }
         If the value supplied for this parameter ends with a hyphen,
         then you are re-prompted for further input until a value is
         given which does not end with a hyphen.  All the NDFs given in
         this way are concatenated into a single group.
      }
      \sstsubsection{
         LBND() = \_INTEGER (Read)
      }{
         An array of values giving the lower pixel-index bound on each
         axis for the output NDF.  The suggested default values just
         encompass all the input data.  A null value (\texttt{{!}}) also
         results in these same defaults being used.  \texttt{[!]}
      }
      \sstsubsection{
         MAXPIX = \_INTEGER (Read)
      }{
         A value which specifies an initial scale size in pixels for the
         adaptive algorithm which approximates non-linear
         \htmlref{Mappings}{se:curframe}~ with piece-wise linear
         transformations.  If MAXPIX is larger than sny
         dimension of the region of the output grid being used, a
         first attempt will be made to approximate the Mapping by a
         linear transformation over the entire output region.  If a
         smaller value is used, the output region will first be divided
         into subregions whose size does not exceed MAXPIX pixels in any
         dimension, and then attempts will be made at approximation.
         \texttt{[1000]}
      }
      \label{method:wcsmosaic}
      \sstsubsection{
         METHOD = \htmlref{LITERAL}{se:parmenu} (Read)
      }{
         The method to use when dividing an input pixel value between a
         group of neighbouring output pixels.  For details on these
         schemes, see the description of \xref{AST\_REBINx}{sun210}{AST_REBIN\$<X>\$}
         in \xref{SUN/210}{sun210}{}.  METHOD can take the following
         values.

         \ssthitemlist{

            \sstitem
            \texttt{"Bilinear"} --- The input pixel value is divided bi-linearly
            between  the four nearest output pixels.  This produces smoother
            output NDFs than the nearest-neighbour scheme, but is
            marginally slower.

            \sstitem
            \texttt{"Nearest"} --- The input pixel value is assigned completely to
            the single nearest output pixel.

            \sstitem
            \texttt{"Sinc"} --- Uses the ${\textrm{sinc}}({\pi}x)$ kernel, where $x$ is
            the pixel offset from the transformed input pixel centre, and
            ${\textrm{sinc}}(z)=\sin(z)/z$.  Use of this scheme is not recommended.

            \sstitem
            \texttt{"SincSinc"} --- Uses the ${\textrm{sinc}}({\pi}x){\textrm{sinc}}(k{\pi}x)$
            kernel.  This is a valuable general-purpose scheme, intermediate in
            its visual effect on NDFs between the bilinear and nearest-neighbour
            schemes.

            \sstitem
            \texttt{"SincCos"} --- Uses the ${\textrm{sinc}}({\pi}x)\cos(k{\pi}x)$ kernel.
            It gives similar results to the \texttt{"Sincsinc"} scheme.

            \sstitem
            \texttt{"SincGauss"} --- Uses the ${\textrm{sinc}}({\pi}x)e^{-kx^2}$ kernel.
            Good results can be obtained by matching the FWHM of the
            envelope function to the point-spread function of the
            input data (see Parameter PARAMS).

            \sstitem
            \texttt{"Somb"} --- Uses the  ${\textrm{somb}}({\pi}x)$ kernel, where
            ${\textrm{somb}}(z)=2*J_{1}(z)/z$. $J_1$ is the first-order Bessel function
            of the first kind.  This scheme is similar to the \texttt{"Sinc"} scheme.

            \sstitem
            \texttt{"SombCos"} --- Uses the ${\textrm{somb}}({\pi}x)\cos(k{\pi}x)$ kernel.
            This scheme is similar to the \texttt{"SincCos"} scheme.

            \sstitem
            \texttt{"Gauss"} --- Uses the $e^{-kx^2}$ kernel.  The FWHM of the
            Gaussian is given by Parameter PARAMS(2), and the point at
            which to truncate the Gaussian to zero is given by Parameter
            PARAMS(1).

         }
         All methods propagate variances from input to output, but the
         variance estimates produced by schemes other than
         nearest neighbour need to be treated with care since the
         spatial smoothing produced by these methods introduces
         correlations in the variance estimates.  Also, the degree of
         smoothing produced varies across the NDF.  This is because a
         sample taken at a pixel centre will have no contributions from
         the neighbouring pixels, whereas a sample taken at the corner
         of a pixel will have equal contributions from all four
         neighbouring pixels, resulting in greater smoothing and lower
         noise.  This effect can produce complex Moir\'{e} patterns in the
         output variance estimates, resulting from the interference of
         the spatial frequencies in the sample positions and in the
         pixel-centre positions.  For these reasons, if you want to use
         the output variances, you are generally safer using
         nearest-neighbour interpolation.  The initial default is
         \texttt{"SincSinc"}.  \texttt{[}current value\texttt{{]}}
      }
      \sstsubsection{
         NORM = \_LOGICAL (Read)
      }{
         In general, each output pixel contains contributions from
         multiple input pixel values, and the number of input pixels
         contributing to each output pixel will vary from pixel to
         pixel.  If NORM is set \texttt{TRUE} (the default), then each
         output value is normalised by dividing it by the number of
         contributing input pixels, resulting in each output value being
         the weighted mean of the contibuting input values.  However, if
         NORM is set FALSE, this normalisation is not applied.  See also
         Parameter CONSERVE.  Setting NORM to \texttt{FALSE} and VARIANCE to
         \texttt{TRUE} results in an error being reported.  \texttt{[TRUE]}
      }
      \sstsubsection{
         OUT = NDF (Write)
      }{
         The output NDF.  If a null (\texttt{{!}}) value is supplied,
         WCSMOSAIC will terminate early without creating an output cube, but
         without reporting an error.  Note, the pixel bounds which the
         output cube would have had will still be written to output
         Parameters LBOUND and UBOUND, even if a null value is supplied
         for OUT.
      }
      \sstsubsection{
         PARAMS( 2 ) = \_DOUBLE (Read)
      }{
         An optional array which consists of additional parameters
         required by the Sinc, SincSinc, SincCos, SincGauss, Somb,
         SombCos, and Gauss methods.

         PARAMS(1) is required by all the above schemes.  It is used
         to specify how many output pixels on either side of the
         central output pixel are to receive contribution from the
         corresponding input pixel.  Typically, a value of \texttt{2} is
         appropriate and the minimum allowed value is \texttt{1}
         (\emph{i.e.} one pixel on each side).  A value of zero or
         fewer indicates that a suitable number of pixels should be
         calculated automatically.  \texttt{[0]}

         PARAMS(2) is required only by the Gauss, SombCos, SincSinc,
         SincCos, and SincGauss schemes. For the SombCos, SincSinc and
         SincCos schemes, it specifies the number of output pixels at
         which the envelope of the function goes to zero.  The minimum
         value is \texttt{1.0}, and the run-time default value is \texttt{
         2.0}.  For the Gauss and SincGauss schemes, it specifies the
         full-width at half-maximum (FWHM) of the Gaussian envelope
         measured in output pixels.
         The minimum value is \texttt{0.1}, and the run-time default is
         \texttt{1.0}.  \texttt{[]}
      }
      \sstsubsection{
         REF = NDF (Read)
      }{
         The NDF to which all the input NDFs are to be aligned.  If a
         null value is supplied for this parameter, the first NDF
         supplied for Parameter IN is used.  The WCS information in this
         NDF must have a defined inverse transformation (from WCS
         co-ordinates to pixel co-ordinates).  \texttt{[!]}
      }
      \sstsubsection{
         UBND() = \_INTEGER (Read)
      }{
         An array of values giving the upper pixel-index bound on each
         axis for the output NDF.  The suggested default values just
         encompass all the input data.  A null value (\texttt{{!}}) also results in
         these same defaults being used.  \texttt{[!]}
      }
      \sstsubsection{
         VARIANCE = \_LOGICAL (Read)
      }{
         If \texttt{TRUE}, then any input VARIANCE components in the input NDFs
         are used to weight the input data (the weight used for each
         data value is the reciprocal of the variance).  If \texttt{FALSE}, all
         input data is given equal weight. Note, some applications (such
         as \xref{CCDPACK:MAKEMOS}{sun139}{MAKEMOS}) use a parameter named
         USEVAR to determine both whether input variances are used to weights
         input data values, and also how to calculate output variances.
         However, WCSMOSAIC uses the VARIANCE parameter only for the first of these purposes
         (determining whether to weight the input data).  The second
         purpose (determining how to create output variances) is
         fulfilled by the GENVAR parameter.  \texttt{[FALSE]}
      }
      \sstsubsection{
         WEIGHTS = LITERAL (Read)
      }{
         An optional group of numerical weights, one for each of the input
         NDFs specified by Parameter IN. If VARIANCE is \texttt{TRUE}, the weight
         assigned to each input pixel is the value supplied in this group
         correspoinding to the appropriate input NDF, divided by the variance
         of the pixel value. An error is reported if the number of supplied
         weights does not equal the number of supplied input NDFs. \texttt{[!]}
      }
      \sstsubsection{
         WLIM = \_REAL (Read)
      }{
         This parameter specifies the minimum number of good pixels
         that must contribute to an output pixel for the output pixel
         to be valid.  Note, fractional values are allowed.  If a value
         less than 1.0E-10 is supplied, a value of \texttt{1.0E-10} is used.
         \texttt{[1.0E-10]}
      }
   }
   \sstresparameters{
      \sstsubsection{
         FLBND( ) = \_DOUBLE (Write)
      }{
         The lower bounds of the bounding box enclosing the output NDF
         in the \htmlref{current WCS Frame}{se:curframe}.  The number of elements
         in this parameter is equal to the number of axes in the current WCS
         Frame.  Celestial axis values will be in units of radians.
      }
      \sstsubsection{
         FUBND( ) = \_DOUBLE (Write)
      }{
         The upper bounds of the bounding box enclosing the output NDF
         in the current WCS Frame.  The number of elements in this parameter
         is equal to the number of axes in the current WCS Frame.  Celestial
         axis values will be in units of radians.
      }
      \sstsubsection{
         LBOUND() = \_INTEGER (Write)
      }{
         The lower pixel bounds of the output NDF.  Note, values will be
         written to this output parameter even if a null value is supplied
         for Parameter OUT.
      }
      \sstsubsection{
         UBOUND() = \_INTEGER (Write)
      }{
         The upper pixel bounds of the output NDF.  Note, values will be
         written to this output parameter even if a null value is supplied
         for Parameter OUT.
      }
   }
   \sstexamples{
      \sstexamplesubsection{
         wcsmosaic m51$*$ mosaic lbnd=! accept
      }{
         This example rebins all the NDFs with names starting with
         the string \texttt{"m51"} in the current directory so that they are
         aligned with the first input NDF, and combines them all into a
         single output NDF called mosaic.  The output NDF is just big
         enough to contain all the pixels in all the input NDFs.
      }
   }
   \sstnotes{
      \sstitemlist{

         \sstitem
         \htmlref{WCS}{apndf:wcs} information (including the current
         co-ordinate Frame) is propagated from the reference NDF to the
         output NDF.

         \sstitem
         \htmlref{QUALITY}{apndf:quality} is not propagated from input to output.

         \sstitem
         There are different facts reported, their verbosity depending
         on the current message-reporting level set by environment variable
         MSG\_FILTER.  If this is set to \texttt{QUIET}, no information will be
         displayed while the command is executing.  When the filtering
         level is at least as verbose as \texttt{NORMAL}, the interpolation method
         being used will be displayed.  If set to \texttt{VERBOSE}, the name of each
         input NDF will also be displayed as it is processed.

      }
   }
   \sstdiytopic{
      Related Applications
   }{
      KAPPA: \htmlref{WCSFRAME}{WCSFRAME},
      \htmlref{WCSALIGN}{WCSALIGN},
      \htmlref{REGRID}{REGRID};
      \xref{CCDPACK}{sun139}{}: \xref{TRANNDF}{sun139}{TRANNDF}.
   }
   \sstimplementationstatus{
      \sstitemlist{

         \sstitem
         This routine correctly processes the DATA, \htmlref{VARIANCE}{apndf:variance},
         \htmlref{LABEL}{apndf:label}, \htmlref{TITLE}{apndf:title},
         \htmlref{UNITS}{apndf:units}, \htmlref{WCS}{apndf:wcs}, and
         \htmlref{HISTORY}{apndf:history} components of the input NDFs (see the
         \htmlref{METHOD}{method:wcsmosaic} parameter for notes on the interpretation of output variances).

         \sstitem
         Processing of \htmlref{bad pixels}{se:masking} and automatic
         \htmlref{quality masking}{se:qualitymask} are supported.

         \sstitem
         All \htmlref{non-complex numeric data types}{ap:HDStypes} can be handled,
         but the data type will be converted to one of \_INTEGER,
         \_DOUBLE or \_REAL for processing.

      }
   }
}

\sstroutine{
   WCSREMOVE
}{
   Remove co-ordinate Frames from the WCS component of an NDF
}{
   \sstdescription{
      This application allows you to remove one or more \htmlref{co-ordinate Frames}{se:domains}~
      from the \htmlref{WCS}{apndf:wcs}~ component in an \NDFref{NDF}.  The indices of any remaining Frames
      are `shuffled down' to fill the gaps left by the removed Frames.
   }
   \sstusage{
      wcsremove ndf frames
   }
   \sstparameters{
      \sstsubsection{
         FRAMES() = LITERAL (Read
      }{
         Specifies the Frame(s) to be removed. It can be a list of indices
         (within the WCS component of the supplied NDF) or list of Domain
         names. If one or more Domain name are specified, any WCS Frames
         which have a matching Domain are removed. If a list of indices is
         supplied, any indices outside the range of the available Frames are
         ignored. Single Frames or a set of adjacent Frames may be specified,
         \emph{e.g.}. typing \texttt{[4,6-9,12,14-16] } will remove Frames
         4,6,7,8,9,12,14,15,16.
         (Note that the brackets are required to distinguish this array of
         characters from a single string including commas.  The brackets are
         unnecessary when there only one item.) If you wish to remove all
         the files enter the wildcard $*$. \texttt{5-}$*$ will remove from 5 to
         the last Frame.
      }
      \sstsubsection{
         NDF = NDF (Read and Write)
      }{
         The NDF data structure.
      }
   }
   \sstexamples{
      \sstexamplesubsection{
         wcsremove m51 "3-5"
      }{
         This removes Frames 3, 4 and 5 from the NDF called m51.  Any remaining
         Frames with indices higher than 5 will be re-numbered to fill the
         gaps left by the removed Frames (\emph{i.e.} the original Frame 6 will
         become Frame 3, \emph{etc.}).
      }
   }
   \sstnotes{
      \sstitemlist{

         \sstitem
         The Frames within the WCS component of an NDF may be examined
         using application NDFTRACE.
      }
   }
   \sstdiytopic{
      Related Applications
   }{
KAPPA: \htmlref{NDFTRACE}{NDFTRACE},
\htmlref{WCSADD}{WCSADD},
\htmlref{WCSATTRIB}{WCSATTRIB},
\htmlref{WCSCOPY}{WCSCOPY},
\htmlref{WCSFRAME}{WCSFRAME}.
   }
}
\sstroutine{
   WCSSHOW
}{
   Examines the internal structure of a WCS description.
}{
   \sstdescription{
      This application allows you to examine \htmlref{WCS}{se:wcsuse}~
      information (represented by an AST Object) stored in a
      specified \NDFref{NDF} or \HDSref\ object, or a catalogue.  The structure can be
      dumped to a text file, or a Graphical User Interface can be used
      to navigate through the structure (see Parameter LOGFILE).  A new
      FrameSet can also be stored in the \htmlref{WCS component}{apndf:wcs}~
      of an NDF (see Parameter NEWWCS).  This allows an NDF WCS component to
      be dumped to a text file, edited, and then restored to the NDF.

      The GUI main window contains the attribute values of the
      supplied \xref{AST Object}{sun210}{Object}.  Only those
      associated with the Object's class are displayed initially, but
      attributes of the Objects parent classes can be displayed by
      clicking one of the class button to the top left of the window.

      If the Object contains \htmlref{attributes}{ap:frmatt}~ which are
      themselves AST Objects (such as the Frames within a
      \htmlref{FrameSet}{se:curframe}), then  new windows can be created
      to examine these attributes by clicking over the attribute name.
   }
   \sstusage{
      wcsshow ndf object logfile newwcs full quiet
   }
   \sstparameters{
      \sstsubsection{
         CAT = FILENAME (Read)
      }{
         A catalogue containing a positions list such as produced by
         applications \htmlref{LISTMAKE}{LISTMAKE}, \htmlref{CURSOR}{CURSOR}.
         If supplied, the WCS Information in the catalogue is displayed.  If a
         null (\texttt{{!}}) is supplied, the WCS information in the NDF
         specified by Parameter NDF is displayed.  \texttt{[!]}
      }
      \sstsubsection{
         FULL = \_INTEGER (Read)
      }{
         This parameter is a three-state flag and takes values of \texttt{-1},
         \texttt{0}, or \texttt{+1}.  It controls the amount of information included in the output
         generated by this application.  If FULL is zero, then a modest
         amount of non-essential but useful information will be included
         in the output.  If FULL is negative, all non-essential information
         will be suppressed to minimise the amount of output, while if it is
         positive, the output will include the maximum amount of detailed
         information about the Object being examined.   \texttt{[}current value\texttt{{]}}
      }
      \sstsubsection{
         LOGFILE = FILENAME (Write)
      }{
         The name of the text file in which to store a dump of the
         specified AST Object.  If a null (\texttt{{!}}) value is supplied, no log
         file is created.  If a log file is given, the Tk browser window
         is not produced.  \texttt{[!]}
      }
      \sstsubsection{
         NDF = NDF (Read or Update)
      }{
         If an NDF is supplied, then its WCS FrameSet is displayed.  If a
         null (\texttt{{!}}) value is supplied, then the Parameter OBJECT is used to
         specify the AST Object to display. Update access is required to
         the NDF if a value is given for Parameter NEWWCS.  Otherwise, only
         read access is required.  Only accessed if a null (\texttt{{!}}) value is
         supplied for CAT.
      }
      \sstsubsection{
         NEWWCS = \htmlref{GROUP}{se:groups} (Read)
      }{
         A group expression giving a dump of an AST FrameSet which
         is to be stored as the WCS component in the NDF given by Parameter
         NDF.  The existing WCS component is unchanged if a null value is
         supplied.  The value supplied for this parameter is ignored if a
         null value is supplied for Parameter NDF.  The Base Frame in the
         FrameSet is assumed to be the GRID Frame.  If a value is given for
         this parameter, then the log file or Tk browser will display the
         new FrameSet (after being stored in the NDF and retrieved).  \texttt{[!]}
      }
      \sstsubsection{
         OBJECT = LITERAL (Read)
      }{
         The HDS object containing the AST Object to display.  Only
         accessed if Parameters NDF and CAT are null.  It must have an HDS
         type of WCS, must be scalar, and must contain a single one-dimensional array
         component with name DATA and type \_CHAR.
      }
      \sstsubsection{
         QUIET = \_LOGICAL (Read)
      }{
         If \texttt{TRUE}, then the structure of the AST Object is not displayed
         (using the Tk GUI).  Other functions are unaffected.  If a null (\texttt{{!}})
         value is supplied, the value used is \texttt{TRUE} if a non-null value is
         supplied for Parameter LOGFILE or Parameter NEWWCS, and \texttt{FALSE}
         otherwise.  \texttt{[!]}
      }
   }
   \sstexamples{
      \sstexamplesubsection{
         wcsshow m51
      }{
         Displays the WCS component of the NDF m51 in a Tk GUI.
      }
      \sstexamplesubsection{
         wcsshow m51 logfile=m51.ast
      }{
         Dumps the WCS component of the NDF m51 to text file \texttt{m51.ast}.
      }
      \sstexamplesubsection{
         wcsshow m51 newwcs=$\wedge$m51.ast
      }{
         Reads a FrameSet from the text file m51.ast and stores it in the
         WCS component of the NDF m51.  For instance, the text file \texttt{m51.ast}
         could be an edited version of the text file created in the
         previous example.
      }
      \sstexamplesubsection{
         wcsshow object="$\sim$/agi\_starprog.agi\_3800\_1.picture(4).more.ast\_plot"
      }{
         Displays the AST Plot stored in the \htmlref{AGI database}{se:agitidy}~ with X windows
         picture number 4.
      }
   }
}

\sstroutine{
   WCSSLIDE
}{
   Applies a translational correction to the WCS in an NDF
}{
   \sstdescription{
      This application modifies the \htmlref{WCS}{apndf:wcs} information
      in an NDF so that the WCS position of a given pixel is moved by
      specified amount along each WCS axis.  The shifts to use are
      specified either by an absolute offset vector given by the ABS
      parameter or by the difference between a fiducial point and a
      standard object given by the FID and OBJ parameters
      respectively.  In each case the co-ordinates are specified in
      the NDF's \htmlref{current WCS co-ordinate Frame}{se:curframe}.
   }
   \sstusage{
      wcsslide ndf abs
   }
   \sstparameters{
      \sstsubsection{
         ABS( ) = \_DOUBLE (Read)
      }{
         Absolute shift for each WCS axis.  The number of values supplied
         must match the number of WCS axes in the NDF.  It is only used if
         STYPE=\texttt{"Absolute"}.  Offsets for celestial longitude and
         latitude axes should be specified in arcseconds.  Offsets for all
         other types of axes should be given directly in the units of the
         axis.
      }
      \sstsubsection{
         FID = LITERAL (Read)
      }{
         A comma-separated list of \xref{formatted axis values}{sun210}{AST_UNFORMAT}
         giving the position of the fiducial point in WCS co-ordinates.  The
         number of values supplied must match the number of WCS axes in the
         NDF.  It is only used if STYPE=\texttt{"Relative"}.
      }
      \sstsubsection{
         NDF = NDF (Update)
      }{
         The NDF to be translated.
      }
      \sstsubsection{
         OBJ = LITERAL (Read)
      }{
         A comma-separated list of formatted axis values giving the position
         of the standard object in WCS co-ordinates.  The number of values
         supplied must match the number of WCS axes in the NDF.  It is only
         used if STYPE=\texttt{"Relative"}.
      }
      \sstsubsection{
         STYPE = LITERAL (Read)
      }{
         The sort of shift to be used.  The choice is \texttt{"Relative"} or
         \texttt{"Absolute"}.  \texttt{["Absolute"]}
      }
   }
   \sstexamples{
      \sstexamplesubsection{
         wcsslide m31 [32,23]
      }{
         The (RA,Dec) axes in the NDF m31 are shifted by 32~arcseconds
         in right ascension and 23~arcseconds in declination.
      }
      \sstexamplesubsection{
         wcsslide speca stype=rel fid=211.2 obj=211.7
      }{
         The spectral axis in the NDF speca (which measures frequency in
         GHz), is shifted by 0.5~GHz (\emph{i.e.} 211.7--211.2).
      }
      \sstexamplesubsection{
         wcsslide speca stype=abs abs=0.5
      }{
         This does just the same as the previous example.
      }
   }
   \sstnotes{
      \sstitemlist{

         \sstitem
         The correction is affected by translating pixel co-ordinates
         by a constant amount before projection them into WCS
         co-ordinates. Therefore, whilst the translation will be
         constant across the array in pixel co-ordinates, it may vary
         in WCS co-ordinates depending on the nature of the
         pixel$\rightarrow$WCS transformation. The size of the
         translation in pixel co-ordinates is chosen in order to
         produce the required shift in WCS co-ordinates at the OBJ
         position (if STYPE is \texttt{"Relative"}), or at the array
         centre (if STYPE is \texttt{"Absolute"}).
      }
   }
   \sstdiytopic{
      Related Applications
   }{
      KAPPA: \htmlref{SLIDE}{SLIDE}.
   }
   \sstimplementationstatus{
      \sstitemlist{

         \sstitem
         There can be an arbitrary number of NDF dimensions.
      }
   }
}

\sstroutine{
   WCSTRAN
}{
   Transform a position from one NDF co-ordinate Frame to another
}{
   \sstdescription{
      This application transforms a position from one \NDFref{NDF} \htmlref{co-ordinate Frame}{se:domains}~
      to another.  The input and output Frames may be chosen freely from the
      Frames available in the \htmlref{WCS component}{apndf:wcs}~ of the supplied NDF.  The
      transformed position is formatted for display and written to the screen
      and also to an output parameter.
   }
   \sstusage{
      wcstran ndf posin framein [frameout]
   }
   \sstparameters{
      \sstsubsection{
         EPOCHIN = \_DOUBLE (Read)
      }{
         If a `Sky Co-ordinate System' specification is supplied (using
         Parameter FRAMEIN) for a celestial co-ordinate system, then an epoch
         value is needed to qualify it.  This is the epoch at which the
         supplied sky position was determined.  It should be given as a
         decimal years value, with or without decimal places  (\texttt{"1996.8"} for
         example).  Such values are interpreted as a Besselian epoch if less
         than 1984.0 and as a Julian epoch otherwise.
      }
      \sstsubsection{
         EPOCHOUT = \_DOUBLE (Read)
      }{
         If a `Sky Co-ordinate System' specification is supplied (using
         Parameter FRAMEOUT) for a celestial co-ordinate system, then an epoch
         value is needed to qualify it.  This is the epoch at which the
         transformed sky position is required.  It should be given as a
         decimal years value, with or without decimal places  (\texttt{"1996.8"} for
         example).  Such values are interpreted as a Besselian epoch if less
         than 1984.0 and as a Julian epoch otherwise.
      }
      \sstsubsection{
         FRAMEIN = LITERAL (Read)
      }{
         A string specifying the co-ordinate Frame in which the input
         position is supplied (see Parameter POSIN).  If a null parameter
         value is supplied, then the current Frame in the NDF is used.  The
         string can be one of the following:

         \ssthitemlist{

            \sstitem
            A \htmlref{domain name}{se:domains}~ such as \htmlref{SKY, AXIS, PIXEL}{se:resdoms}.  The two
            `pseudo-domains' WORLD and DATA may be supplied and will be
            translated into PIXEL and AXIS respectively, so long as the WCS
            component of the NDF does not contain Frames with these domains.

            \sstitem
            An integer value giving the index of the required Frame within
            the WCS component.

            \sstitem
            An IRAS90 \emph{Sky Co-ordinate System} (SCS) values such as
            \texttt{"EQUAT(J2000)"} (see \xref{SUN/163}{sun163}{}).
         }
      }
      \sstsubsection{
         FRAMEOUT = LITERAL (Read)
      }{
         A string specifying the co-ordinate Frame in which the transformed
         position is required.  If a null parameter value is supplied, then
         the current Frame in the NDF is used.  The string can be one of the
         following:

         \ssthitemlist{

            \sstitem
            A domain name such as SKY, AXIS, PIXEL.  The two
            `pseudo-domains' WORLD and DATA may be supplied and will be
            translated into PIXEL and AXIS respectively, so long as the WCS
            component of the NDF does not contain Frames with these domains.

            \sstitem
            An integer value giving the index of the required Frame within
            the WCS component.

            \sstitem
            An IRAS90 \emph{Sky Co-ordinate System} (SCS) values such as
            \texttt{"EQUAT(J2000)"} (see SUN/163).
         }
      }
      \sstsubsection{
         NDF = NDF (Read and Write)
      }{
         The NDF data structure containing the required co-ordinate Frames.
      }
      \sstsubsection{
         POSIN = LITERAL (Read)
      }{
         The co-ordinates of the position to be transformed, in the
         co-ordinate Frame specified by Parameter FRAMEIN (supplying
         a colon \texttt{":"} will display details of the required co-ordinate Frame).
         The position should be supplied as a list of
         \xref{formatted axis values}{sun210}{AST_UNFORMAT}
         separated by spaces or commas.
      }
      \sstsubsection{
         QUIET = \_LOGICAL (Read)
      }{
         If \texttt{TRUE}, the transformed position is not written to the screen
         (it is still written to the output Parameter POSOUT).  \texttt{[FALSE]}
      }
      \sstsubsection{
         SKYDEG = \_INTEGER (Read)
      }{
         If greater than zero, the values for any celestial longitude or
         latitude axes are formatted as decimal degrees, irrespective of
         the Format attributes in the NDF WCS component. The supplied
         integer value indicates the number of decimal places required.
         If the SKYDEG value is less than or equal to zero, the formats
         specified by the Format attributes in the WCS component are
         honoured. \texttt{[0] }
      }
   }
   \sstresparameters{
      \sstsubsection{
         POSOUT = LITERAL (Write)
      }{
         The formatted co-ordinates of the transformed position, in the
         co-ordinate Frame specified by Parameter FRAMEOUT.  The position
         will be stored as a list of formatted axis values separated by
         spaces or commas.
      }
   }
   \sstexamples{
      \sstexamplesubsection{
         wcstran m51 "100.1 21.5" pixel
      }{
         This transforms the pixel position \texttt{"100.1 21.5"} into the current
         co-ordinate Frame of the NDF m51.  The results are displayed on
         the screen and written to the output Parameter POSOUT.
      }
      \sstexamplesubsection{
         wcstran m51 "1:00:00 -12:30" equ(B1950) pixel
      }{
         This transforms the RA/DEC position \texttt{"1:00:00 -12:30"} (referred
         to the J2000 equinox) into pixel co-ordinates within the NDF m51.
         The results are written to the output Parameter POSOUT.
      }
      \sstexamplesubsection{
         wcstran m51 "1:00:00 -12:30" equ(B1950) equ(j2000)
      }{
         This is like the previous example except that the position is
         transformed into RA/DEC referred to the B1950 equinox, instead of
         pixel co-ordinates.
      }
   }
   \sstnotes{
       \ssthitemlist{

          \sstitem
          The transformed position is not written to the screen when the
          message filter environment variable MSG\_FILTER is set to \texttt{QUIET}.
          The creation of the output Parameter POSOUT is unaffected
          by MSG\_FILTER.

      }
   }
   \sstdiytopic{
      Related Applications
   }{
KAPPA: \htmlref{LISTMAKE}{LISTMAKE},
\htmlref{LISTSHOW}{LISTSHOW},
\htmlref{NDFTRACE}{NDFTRACE},
\htmlref{WCSATTRIB}{WCSATTRIB},
\htmlref{WCSFRAME}{WCSFRAME}.
   }
}
\sstroutine{
   WIENER
}{
   Applies a Wiener filter to a one- or two-dimensional array
}{
   \sstdescription{
      This application filters the supplied one- or two-dimensional array
      using a Wiener filter.  It takes an array holding observed data
      and another holding a Point-Spread Function as input and produces
      an output restored array with potentially higher resolution and
      lower noise.  Generally superior results can be obtained using
      applications MEM2D or LUCY, but at the cost of much more
      processing time.

      The Wiener filter attempts to minimise the mean squared
      difference between the undegraded image and the restored image.
      To do this it needs to know the power spectrum of the undegraded
      image (\emph{i.e.} the power at each spatial frequency before the
      instrumental blurring and the addition of noise).  Obviously,
      this is not usually available, and instead the power spectrum of
      some other image must be used (the `model' image).  The idea is
      that a model image should be chosen for which there is some a
      priori reason for believing it to have a power spectrum similar
      to the undegraded image.  Many different suggestions have been
      made for the best way to make this choice and the literature
      should be consulted for a detailed discussion (for instance, see
      the paper \emph{Wiener Restoration of HST Images: Signal Models and
      Photometric Behavior} by I.C.~Busko in the proceedings of the
      first Annual Conference  on Astronomical Data Analysis Software
      and Systems, Tucson).  By default, this application uses a
      `white' model image, \emph{i.e.} one in which there is equal power at
      all spatial frequencies.  The default value for this constant
      power is the mean power per pixel in the input image.  There is
      also an option to use the power spectrum of a supplied model
      image.

      The filter also depends on a model of the noise in the supplied
      image.  This application assumes that the noise is 'white' and is
      constant across the image.  You can specify the noise power
      to use.  If a noise power of zero is supplied, then the Wiener
      filter just becomes a normal inverse filter which will tend to
      amplify noise in the supplied image.

      The filtering is done by multiplying the Fourier transform of the
      supplied image by the Fourier transform of the filter function.
      The output image is then created by taking the inverse Fourier
      transform of the product.  The Fourier transform of the filter
      function is given by:

         {\Large
         \[ \frac{H^{\ast}}{ \left | H \right |^{2} + \frac{P_{n}}{P_{g}}} \]
         }

      where $H$ is the Fourier transform of the supplied Point-Spread
      Function, $P_{n}$ is the noise power, $P_{g}$ is the power in the model
      image, and $H^{\ast}$ is the complex conjugate of $H$.  If the
      supplied model includes noise (as indicated by Parameter QUIET)
      then $P_{n}$ is subtracted from $P_{g}$ before evaluating the above
      expression.
   }
   \sstusage{
      wiener in psf out xcentre ycentre
   }
   \sstparameters{
      \sstsubsection{
         IN = NDF (Read)
      }{
         The input \NDFref{NDF} containing the observed data.  This image may
         contain \htmlref{bad values}{se:masking}, in which case the bad values will be
         replaced by zero before applying the filter.  The resulting
         filtered image is normalised by dividing each pixel value by
         the corresponding weight of the good input pixels.  These
         weights are found by filtering a mask image which holds the
         value one at every good input pixel, and zero at every bad
         input pixel.
      }
      \sstsubsection{
         MODEL = NDF (Read)
      }{
         An NDF containing an image to use as the model for the power
         spectrum of the restored image.  Any bad values in this image
         are replaced by the mean of the good values.  If a null value
         is supplied then the model power spectrum is taken to be
         uniform with a value specified by Parameter PMODEL.  \texttt{[!]}
      }
      \sstsubsection{
         OUT = NDF (Write)
      }{
         The restored output array.  An extension named WIENER is added
         to the output NDF to indicate that the image was created by
         this application (see Parameter QUIET).
      }
      \sstsubsection{
         PMODEL = \_REAL (Read)
      }{
         The mean power per pixel in the model image.  This parameter
         is only accessed if a null value is supplied for Parameter
         MODEL.  If a value is obtained for PMODEL then the model image
         is assumed to have the specified constant power at all spatial
         frequencies.  If a null (\texttt{{!}}) value is supplied, the value used is
         the mean power per pixel in the input image.  \texttt{[!]}
      }
      \sstsubsection{
         PNOISE = \_REAL (Read)
      }{
         The mean noise power per pixel in the observed data.  For
         Gaussian noise this is equal to the variance.  If a null (\texttt{{!}})
         value is supplied, the value used is an estimate of the noise
         variance based on the difference between adjacent pixel values in
         the observed data.  \texttt{[!]}
      }
      \sstsubsection{
         PSF = NDF (Read)
      }{
         An NDF holding an estimate of the Point-Spread Function (PSF)
         of the input array.  This could, for instance, be produced
         using the \KAPPA\ application PSF.  There should be no bad
         pixels in the PSF otherwise an error will be reported.  The
         PSF can be centred anywhere within the array, but the location
         of the centre must be specified using Parameters XCENTRE and
         YCENTRE.  The PSF is assumed to have the value zero outside
         the supplied NDF.
      }
      \sstsubsection{
         QUIET = \_LOGICAL (Read)
      }{
         This specifies whether or not the image given for Parameter
         MODEL (or the value given for Parameter PMODEL), includes
         noise.  If the model does not include any noise then a \texttt{TRUE}
         value should be supplied for QUIET.  If there is any noise in
         the model then QUIET should be supplied \texttt{FALSE}.  If a null (\texttt{{!}})
         value is supplied, the value used is \texttt{FALSE}, unless the image
         given for Parameter MODEL was created by a previous run of WIENER
         (as indicated by the presence of a WIENER extension in the NDF),
         in which case the run time default is \texttt{TRUE} (\emph{i.e.} the previous
         run of WIENER is assumed to have removed the noise).  \texttt{[!]}
      }
      \sstsubsection{
         THRESH = \_REAL (Read)
      }{
         The fraction of the PSF peak amplitude at which the extents of
         the PSF are determined.  These extents are used to derive the
         size of the margins that pad the supplied input array.  Lower
         values of THRESH will result in larger margins being used.
         THRESH must be positive and less than 0.5.  \texttt{[0.0625]}
      }
      \sstsubsection{
         TITLE = LITERAL (Read)
      }{
         A \htmlref{title}{apndf:title} for the output NDF.  A null (\texttt{{!}}) value means using the
         title of the input NDF.  \texttt{[!]}
      }
      \sstsubsection{
         WLIM = \_REAL (Read)
      }{
         If the input array contains bad values, then this parameter
         may be used to determine the minimum weight of good input
         values required to create a good output value.  It can be
         used, for example, to prevent output pixels from being
         generated in regions where there are relatively few good input
         values to contribute to the restored result.  It can also be
         used to `fill in' small areas (\emph{i.e.} smaller than the PSF) of
         bad pixels.

         The numerical value given for WLIM specifies the minimum total
         weight associated with the good pixels in a smoothing box
         required to generate a good output pixel (weights for each
         pixel are defined by the normalised PSF).  If this specified
         minimum weight is not present, then a bad output pixel will
         result, otherwise a smoothed output value will be calculated.
         The value of this parameter should lie between 0.0 and
         1.0.  WLIM=\texttt{0} causes a good output value to be created even if
         there is only one good input value, whereas WLIM=\texttt{1} causes a
         good output value to be created only if all input values are
         good.  \texttt{[0.001]}
      }
      \sstsubsection{
         XCENTRE = \_INTEGER (Read)
      }{
         The \textit{x} \htmlref{pixel index}{se:pixgrd}~ of the centre of the
         PSF within the supplied PSF array.  The suggested default is the
         middle pixel (rounded down if there are an even number of pixels per line).
      }
      \sstsubsection{
         YCENTRE = \_INTEGER (Read)
      }{
         The \textit{y} pixel index of the centre of the PSF within the supplied
         PSF array.  The suggested default is the middle line (rounded
         down if there are an even number of lines).
      }
   }
   \sstexamples{
      \sstexamplesubsection{
         wiener cenA star cenA\_hires 11 13
      }{
         This example deconvolves the array in the NDF called cenA,
         putting the resulting array in the NDF called cenA\_hires.
         The PSF is defined by the array in NDF star, and the centre
         of the PSF is at pixel (11,~13).
      }
      \sstexamplesubsection{
         wiener cenA star cenA\_hires 11 13 pnoise=0
      }{
         This example performs the same function as the previous
         example, except that the noise power is given as zero.  This
         causes the Wiener filter to reduce to a standard inverse
         filter, which will result in more high frequencies being
         present in the restored image.
      }
      \sstexamplesubsection{
         wiener cenA star cenA\_hires 11 13 model=theory quiet
      }{
         This example performs the same function as the first example,
         except that the power spectrum of the restored image is
         modelled on that of NDF theory, which may for instance
         contain a theoretical model of the object in NDF cenA,
         together with a simulated star field.  The Parameter QUIET is
         set to a \texttt{TRUE} value to indicate that the theoretical model
         contains no noise.
      }
   }
   \sstnotes{
      \sstitemlist{

         \sstitem
         The convolutions required by the Wiener filter are performed by
         the multiplication of Fourier transforms.  The supplied input
         array is extended by a margin along each edge to avoid problems
         of wrap-around between opposite edges of the array.  The width of
         this margin is about equal to the width of the significant part
         of the PSF (as determined by Parameter THRESH).  The application
         displays the width of these margins.  The margins are filled by
         replicating the edge pixels from the supplied input NDFs.
      }
   }
   \sstdiytopic{
      Related Applications
   }{
KAPPA: \htmlref{FOURIER}{FOURIER},
\htmlref{LUCY}{LUCY},
\htmlref{MEM2D}{MEM2D}.
   }
   \sstimplementationstatus{
      \sstitemlist{

         \sstitem
         This routine correctly processes the \htmlref{AXIS}{apndf:axis}, DATA, \htmlref{QUALITY}{apndf:quality},
         \htmlref{LABEL}{apndf:label}, \htmlref{TITLE}{apndf:title}, \htmlref{UNITS}{apndf:units}, \htmlref{WCS}{apndf:wcs}, and \htmlref{HISTORY}{apndf:history}~ components of the
         input NDF and propagates all \htmlref{extensions}{apndf:extensions}.

         \sstitem
         Processing of \htmlref{bad pixels}{se:masking} and automatic \htmlref{quality masking}{se:qualitymask} are
         supported.

         \sstitem
         All \htmlref{non-complex numeric data types}{ap:HDStypes} can be handled.  Arithmetic
         is performed using single-precision floating point.
      }
   }
}
\sstroutine{
   ZAPLIN
}{
   Replaces regions in a two-dimensional NDF by bad values or by linear
   interpolation
}{
   \sstdescription{
      This routine replaces selected areas within a two-dimensional input \NDFref{NDF}
      (specified by Parameter IN), either by filling the areas with bad
      values, or by linear interpolation between neighbouring data values
      (see Parameter ZAPTYPE).  Each area to be replaced can be either
      a range of pixel columns extending the full height of the image, a
      range of pixel lines extending the full width of the image, or a
      rectangular region with edges parallel to the pixel axes (see
      Parameter LINCOL).

      The bounds of the area to be replaced can be specified either by
      using a graphics cursor, or directly in response to parameter prompts,
      or by supplying a text file containing the bounds (see Parameter MODE).
      In the first two modes the application loops asking for new areas
      to zap, until told to quit or an error is encountered.  In the last
      mode processing stops when the end of file is found.  An output text
      file may be produced containing a description of the areas replaced
      (see Parameter COLOUT).  This file may be used to specify the
      regions to be replaced in a subsequent invocation of ZAPLIN.
   }
   \sstusage{
      zaplin in out [title]
        $\left\{ {\begin{tabular}{l}
                  lincol=? \\
                  columns=? lines=? \\
                  colin=?
                  \end{tabular} }
        \right.$
        \newline\latexhtml{\hspace*{10.92em}}{~~~~~~~~~~~~~~~~~~~~}
        \makebox[0mm][c]{\small mode}
   }
   \sstparameters{
      \sstsubsection{
         COLIN =  FILENAME (Read)
      }{
         The name of a text file containing the bounds of the areas to be
         replaced.  This parameter is only accessed if Parameter MODE is
         set to \texttt{"File"}.  Each record in the file must be either a blank
         line, a comment (indicated by a \texttt{"!"} or \texttt{"\#"} in column 1), or a
         definition of an area to be replaced, consisting of three or
         four space-separated fields.  If a range of columns is to be
         replaced, each of the first two fields should be a formatted
         value for the first axis of the current \htmlref{co-ordinate Frame}{se:domains}~  of the
         input NDF, and the third field should be the single character
         \texttt{"C"}.  If a range of lines is to be replaced, each of the first
         two fields should be a formatted value for the second axis of
         the current co-ordinate Frame, and the third field should be the
         single character \texttt{"L"}.  If a rectangular region is to be replaced,
         the first two fields should give the formatted values on axes 1
         and 2 at one corner of the box, and the second two fields should
         give the formatted values on axes 1 and 2 at the opposite corner
         of the box.
      }
      \sstsubsection{
         COLOUT =  FILENAME (Read)
      }{
         The name of an output text file in which to store descriptions of
         the areas replaced by the current invocation of this application.
         It has the same format as the input file accessed using Parameter
         COLIN, and so may be used as input on a subsequent invocation.  This
         parameter is not accessed if Parameter MODE is set to \texttt{"File"}.  If
         COLOUT is null (\texttt{{!}}), no file will be created.  \texttt{[!]}
      }
      \sstsubsection{
         COLUMNS = LITERAL (Read)
      }{
         A pair of \textit{x} values indicating the range of columns to be replaced.
         All columns between the supplied values will be replaced.  This
         parameter is only accessed if Parameter LINCOL is set to
         \texttt{"Columns"} or \texttt{"Region"}, and Parameter MODE is set to \texttt{"Interface"}.
         Each \textit{x} value should be given as a formatted value for Axis 1
         of the current co-ordinate Frame of the input NDF.  The two values
         should be separated by a comma, or by one or more spaces.
      }
      \sstsubsection{
         DEVICE = \htmlref{DEVICE}{se:selgradev} (Read)
      }{
         The graphics device to use if Parameter MODE is set to \texttt{"Cursor"}.
         \texttt{[}Current graphics device\texttt{{]}}
      }
      \sstsubsection{
         IN  =  NDF (Read)
      }{
         The input image.
      }
      \sstsubsection{
         LINCOL = LITERAL (Read)
      }{
         The type of area is to be replaced.  This parameter is only
         accessed if Parameter MODE is set to \texttt{"Cursor"} or \texttt{"Interface"}.
         The options are as follows.

         \ssthitemlist{

            \sstitem
            \texttt{"Lines"} --- Replaces lines of pixels between the \textit{y} values
            specified by Parameter LINES.  Each replaced line extends the
            full width of the image.

            \sstitem
            \texttt{"Columns"} --- Replaces columns of pixels between the \textit{x} values
            specified by Parameter COLUMNS.  Each replaced column extends the
            full height of the image.

            \sstitem
            \texttt{"Region"} --- Replaces the rectangular region of pixels within the
            \textit{x} and \textit{y} bounds specified by Parameters COLUMNS and LINES.  The
            edges of the box are parallel to the pixel axes.

         }
         If this parameter is specified on the command line, and Parameter
         MODE is set to \texttt{"Interface"}, only one area will be replaced;
         otherwise a series of areas will be replaced until a null (\texttt{{!}})
         value is supplied for this parameter.
      }
      \sstsubsection{
         LINES = LITERAL (Read)
      }{
         A pair of \textit{y} values indicating the range of lines to be replaced.
         All lines between the supplied values will be replaced.  This
         parameter is only accessed if Parameter LINCOL is set to
         \texttt{"Lines"} or \texttt{"Region"}, and Parameter MODE is set to \texttt{"Interface"}.
         Each \textit{y} value should be given as a formatted value for Axis 2
         of the current co-ordinate Frame of the input NDF.  The two values
         should be separated by a comma, or by one or more spaces.
      }
      \sstsubsection{
         MARKER = INTEGER (Read)
      }{
         This parameter is only accessed if Parameter PLOT is set to
         \texttt{"Mark"}.  It specifies the type of marker with which
         each cursor position should be marked, and should be given as
         an integer \PGPLOT\  marker type.  For instance, \texttt{0} gives a box,
         \texttt{1} gives a dot, \texttt{2} gives a cross, \texttt{3} gives an asterisk,
         \texttt{7} gives a triangle.  The value must be larger than or equal to $-$31.
         \texttt{[}current value\texttt{{]}}
      }
      \sstsubsection{
         MODE = \htmlref{LITERAL}{se:parmenu} (Read)
      }{
         The \htmlref{method}{se:interaction}~ used to obtain the bounds of the areas to be replaced.
         The supplied string can be one of the following options.

         \ssthitemlist{

            \sstitem
            \texttt{"Interface"} --- Bounds are obtained using Parameters COLUMNS
            and LINES.  The type of area to be replaced is specified using
            Parameter LINCOL.

            \sstitem
            \texttt{"Cursor"} --- Bounds are obtained using the graphics cursor of the
            device specified by Parameter DEVICE.  The type of area to be
            replaced is specified using Parameter LINCOL.  The WCS
            information stored with the picture in the
            \htmlref{graphics database}{se:agitate}~ is
            used to map the supplied cursor positions into the pixel
            co-ordinate Frame of the input NDF.  A message is displayed
            indicating the co-ordinate Frame in which the picture and the
            output NDF were aligned.  Graphics may be drawn over the image
            indicating the region to be replaced (see Parameter PLOT).

            \sstitem
            \texttt{"File"} --- The bounds and type of each area to be replaced are
            supplied in the text file specified by Parameter COLIN.

         }
         \texttt{[}current value\texttt{{]}}
      }
      \sstsubsection{
         NOISE = \_LOGICAL (Read)
      }{
         This parameter is only accessed if Parameter ZAPTYPE is set to
         \texttt{"Linear"}.  If a \texttt{TRUE} value is supplied, gaussian noise is added to
         each interpolated pixel value.  The variance of the noise is
         equal to the variance of the data value being replaced.  If the
         data variance is bad, no noise is added.  If the input NDF has
         no \htmlref{VARIANCE}{apndf:variance}~ component, variances equal to the absolute data
         value are used.  This facility is provided for cosmetic use.  \texttt{[FALSE]}
      }
      \sstsubsection{
         OUT  =  NDF (Write)
      }{
         The output image.
      }
      \sstsubsection{
         PLOT = LITERAL (Read)
      }{
         The type of graphics to be used to mark each cursor position.
         The appearance of these graphics (colour, size, \emph{etc.}) is
         controlled by the STYLE parameter.  PLOT can take any of the
         following values.

         \ssthitemlist{

            \sstitem
            \texttt{"Adapt"} --- Causes \texttt{"Box"} to be used if a region is being
            replaced, \texttt{"Vline"} is a range of columns is being replaced, and
            \texttt{"Hline"} if a range of lines is being replaced.

            \sstitem
            \texttt{"Box"} --- A rectangular box with edges parallel to the edges of
            the graphics device is drawn with the two specified positions at
            opposite corners.

            \sstitem
            \texttt{"Mark"} --- Each position is marked by the symbol specified
            by Parameter MARKER.

            \sstitem
            \texttt{"None"} --- No graphics are produced.

            \sstitem
            \texttt{"Vline"} --- A vertial line is drawn through each specified
            position, extending the entire height of the selected picture.

            \sstitem
            \texttt{"Hline"} --- A horizontal line is drawn through each specified
            position, extending the entire width of the selected picture.

         }
         \texttt{[}current value\texttt{{]}}
      }
      \sstsubsection{
         STYLE = \htmlref{GROUP}{se:groups} (Read)
      }{
         A group of attribute settings describing the style to use when
         drawing the graphics specified by Parameter PLOT.

         A comma-separated list of strings should be given in which each
         string is either an attribute setting, or the name of a text
         file preceded by an up-arrow character \texttt{"$\wedge$"}.  Such text files
         should contain further comma-separated lists which will be
         read and interpreted in the same manner.  Attribute settings
         are applied in the order in which they occur within the list,
         with later settings overriding any earlier settings given for
         the same attribute.

         Each individual attribute setting should be of the form:

            $<$name$>$=$<$value$>$

         where $<$name$>$ is the name of a plotting attribute, and $<$value$>$
         is the value to assign to the attribute.  Default values will be
         used for any unspecified attributes.  All attributes will be
         defaulted if a null value (\texttt{{!}})---the initial default---is supplied.
         To apply changes of style to only the current invocation, begin these
         attributes with a plus sign.  A mixture of persistent and temporary
         style changes is achieved by listing all the persistent attributes
         followed by a plus sign then the list of temporary attributes.

         See \slhyperref{Plotting Attributes}{Section~}{}{ap:plotting_attr}
         for a description of the available attributes.  Any unrecognised
         attributes are ignored (no error is reported).

         The appearance of vertical and horizontal lines is controlled by
         the attributes \latex{\goodbreak} \htmlattref{Colour(Curves)}{Colour(element)},
         \htmlattref{Width(Curves)}{Width(element)},
         \emph{etc.} (the synonym \att{Lines} may be used in place of
         \att{Curves}).  The appearance of boxes is controlled by the
         attributes \att{Colour(Border)},
         \htmlattref{Size(Border)}{Size(element)}, \emph{etc.}
         (the synonym \att{Box} may be used in place of Border).  The appearance
         of markers is controlled by attributes \att{Colour(Markers)},
         \att{Size(Markers)}, \emph{etc}.  \texttt{[}current value\texttt{{]}}
      }
      \sstsubsection{
         TITLE = LITERAL (Read)
      }{
         Title for the output image.  A null value (\texttt{{!}}) propagates the
         title from the input image to the output image.  \texttt{[!]}
      }
      \sstsubsection{
         USEAXIS = \htmlref{GROUP}{se:groups} (Read)
      }{
         USEAXIS is only accessed if the current co-ordinate Frame of the
         input NDF has more than two axes.  A group of two strings should be
         supplied specifying the two axes spanning the plane containing the
         areas to be replaced.  Each axis can be specified
         using one of the following options.

         \ssthitemlist{

            \sstitem
            Its integer index within the current Frame of the
            output NDF (in the range 1 to the number of axes in the
            current Frame).

            \sstitem
            Its \htmlattref{Symbol}{Symbol(axis)}~ string such as
            \texttt{"RA"} or \texttt{"VRAD"}.

            \sstitem
            A generic option where \texttt{"SPEC"} requests the spectral axis,
            \texttt{"TIME"} selects the time axis, \texttt{"SKYLON"} and
            \texttt{"SKYLAT"} picks the sky longitude and latitude axes
            respectively.  Only those axis domains present are
            available as options.
         }

         A list of acceptable values is displayed if an illegal value is
         supplied.  If a null (\texttt{{!}}) value is supplied, the axes with the
         same indices as the first two significant NDF pixel
         axes are used.  \texttt{[!]}
      }
      \sstsubsection{
         ZAPTYPE = LITERAL (Read)
      }{
         The method used to choose the replacement pixel values.  It
         should be one of the options below.

         \ssthitemlist{

            \sstitem
            \texttt{"Bad"} --- Replace the selected pixels by \htmlref{bad values}{se:masking}.

            \sstitem
            \texttt{"Linear"} --- Replace the selected pixels using linear
            interpolation.  If a range of lines is replaced, then the
            interpolation is performed vertically between the first
            non-bad pixels above and below the selected lines.  If a
            range of columns is replaced, then the interpolation is
            performed horizontally between the first non-bad pixels to
            the left and right of the selected columns.  If a rectangular
            region is replaced, then the interpolation is bi-linear
            between the nearest non-bad pixels on all four edges of
            the selected region.  If interpolation is not possible (for
            instance, if the selected pixels are at the edge of the array)
            then the pixels are replaced with bad values.  \texttt{["Linear"]}
         }
      }
   }
   \sstexamples{
      \sstexamplesubsection{
         zaplin out=cleaned colout=fudge.dat
      }{
         Assuming the current value of Parameter MODE is \texttt{"Cursor"}, this
         will copy the NDF associated with the last DATA picture to an
         NDF called cleaned, interactively replacing areas using the
         \htmlref{current graphics device}{se:devglobal}.  Linear interpolation is used to obtain
         the replacement values.  A record of the areas replaced will be
         stored in a text file named \texttt{fudge.dat}.
      }
      \sstexamplesubsection{
         zaplin grubby cleaned i lincol=r columns="188 190" lines="15 16"
      }{
         This replaces a region from pixel (188,~15) to (190,~16) within the
         NDF called grubby and stores the result in the NDF called
         cleaned.  The current co-ordinate Frame in the input NDF should
         be set to PIXEL first (using \htmlref{WCSFRAME}{WCSFRAME}).  The replacement is
         performed using linear interpolation.
      }
      \sstexamplesubsection{
         zaplin grubby(6,,) cleaned i lincol=r columns="188 190"
      }{
         This replaces columns 188 to 190 in the 6th y-z plane region
         within the NDF called grubby and stores the result in the NDF
         called cleaned.  The current co-ordinate Frame in the input NDF
         should be set to PIXEL first (using WCSFRAME).  The replacement is
         performed using linear interpolation.
      }
      \sstexamplesubsection{
         zaplin m42 m42c f colin=aaoccd1.dat zaptype=b
      }{
         This flags with bad values the regions in the NDF called m42
         defined in the text file called \texttt{aaoccd1.dat}, and stores the
         result in an NDF called m42c.
      }
      \sstexamplesubsection{
         zaplin m42 m42c f colin=aaoccd1.dat noise
      }{
         As above except that linear interpolation plus cosmetic noise
         are used to replace the areas to be cleaned rather than bad
         pixels.
      }
   }
   \sstnotes{
      \sstitemlist{

         \sstitem
         Bounds supplied in Interface and File mode are transformed into
         the PIXEL Frame of the input NDF before being used.

         \sstitem
         Complicated results arise if the axes of the current Frame of the
         input NDF are not parallel to the pixel axes.  In these cases it is
         usually better to switch to the PIXEL Frame (using \htmlref{WCSFRAME}{WCSFRAME}) prior to
         using ZAPLIN.  Roughly speaking, the range of pixel lines and/or
         columns which are replaced will include any which intersect the
         specified range on the current-Frame axis.

         \sstitem
         When using input files care should be taken to ensure that
         the co-ordinate system used in the file matches the current
         co-ordinate Frame of the input NDF.

         \sstitem
         If the input NDF is a \htmlref{section of an NDF}{se:ndfsect}~ with a higher
         dimensionality, the \texttt{"lines"} and \texttt{"columns"} are with respect to the
         two-dimensional section, and do not necessarily refer to the first
         and second dimensions of the NDF as a whole.  See the \texttt{"Examples"}.
      }
   }
   \sstdiytopic{
      Related Applications
   }{
KAPPA: \htmlref{ARDMASK}{ARDMASK},
\htmlref{CHPIX}{CHPIX},
\htmlref{FILLBAD}{FILLBAD},
\htmlref{GLITCH}{GLITCH},
\htmlref{NOMAGIC}{NOMAGIC},
\htmlref{REGIONMASK}{REGIONMASK},
\htmlref{SEGMENT}{SEGMENT},
\htmlref{SETMAGIC}{SETMAGIC};
\xref{FIGARO}{sun86}{}: \xref{CSET}{sun86}{CSET},
\xref{ICSET}{sun86}{ICSET},
\xref{NCSET}{sun86}{NCSET},
\xref{TIPPEX}{sun86}{TIPPEX}.
   }
   \sstimplementationstatus{
      \sstitemlist{

         \sstitem
         This routine correctly processes the \htmlref{AXIS}{apndf:axis}, DATA, \htmlref{QUALITY}{apndf:quality},
         \htmlref{VARIANCE}{apndf:variance},
\htmlref{LABEL}{apndf:label}, \htmlref{TITLE}{apndf:title}, \htmlref{UNITS}{apndf:units}, \htmlref{WCS}{apndf:wcs}, and \htmlref {HISTORY}{apndf:history}~ components of the
         input NDF and propagates all \htmlref{extensions}{apndf:extensions}.

         \sstitem
         Processing of \htmlref{bad pixels}{se:masking} and automatic \htmlref{quality masking}{se:qualitymask} are
         supported.

         \sstitem
         All \htmlref{non-complex numeric data types}{ap:HDStypes} can be handled.
      }
   }
}

\newpage
\section{\xlabel{ap_frmatt}Descriptions of Frame Attributes\label{ap:frmatt}}
Each co-ordinate Frame has several \emph{attributes} that determine
its properties.  This section lists the most important of these.  See
\xref{SUN/210}{sun210}{} for a complete description of the properties of
\xref{Frames}{sun210}{Frame}~ and their attributes.  The application
\htmlref{WCSATTRIB}{WCSATTRIB}
can be used to examine attribute values associated with the current Frame
of an NDF, and change the values of those that are not read-only.

\sstminitoc{}
\sstnomaintoc

\sstattribute{
   Digits/Digits(axis)
}{
   Number of digits of precision
}{
   \sstdescription{
      This attribute specifies how many digits of precision are
      required by default when a co-ordinate value is formatted for a
      \xref{Frame}{sun210}{Frame}~ axis (\emph{e.g.} when producing annotated
      plot axes).  Its value may be set either
      for a Frame as a whole, or (by subscripting the attribute name
      with the number of an axis) for each axis individually.  Any
      value set for an individual axis will override the value for
      the Frame as a whole.

      Note that the \att{Digits} value acts only as a means of determining a
      default value for the \htmlattref{Format}{Format(axis)}~ attribute.  Its
      effects are overridden if a \att{Format} string is set explicitly for an
      axis.

      The default \att{Digits} value for a Frame is \texttt{7}.  If a value fewer
      than 1 is supplied, then \texttt{1} is used instead.

      If the Frame is actually a \xref{SkyFrame}{sun210}{SkyFrame}~
      (\emph{e.g.}~ describes celestial longitude and latitude), then the
      \att{Digits} value specifies the total number of digits in the formatted
      axis value---that is, the sum of the hours (or degrees), minutes
      and seconds digits.

   }
   \sstattributetype{
      Integer
   }
   \sstexamples{
      \sstexamplesubsection{
         wcsattrib m51 set digits 4
      }{
         This sets the \att{Digits} attribute to the value 4 for all axes in
         the current Frame of the NDF \texttt{m51}.  This results in axis
         values being formatted with four digits of precision when they are
         displayed by any application, so long as no value has been set
         for the \att{Format} attribute.
      }
      \sstexamplesubsection{
         wcsattrib m51 set digits(1) 4
      }{
         This is like the previous example, except that the \att{Digits} value
         is only set for the first axis in the current Frame of the NDF.
      }
   }
}

\sstattribute{
   Domain
}{
   Co-ordinate system domain
}{
   \sstdescription{
      This attribute contains a string that identifies the physical
      domain of the co-ordinate system that a
      \xref{Frame}{sun210}{Frame}~ describes.

      The \att{Domain} attribute also controls how Frames align with each other.
      If the \att{Domain} value in a Frame is set, then only Frames with the
      same \att{Domain} value can be aligned with it.

      Some Frames are given standard \att{Domain} values when they are created
      (\emph{e.g.} GRID, FRACTION, PIXEL, AXIS, SKY, SPECTRUM, CURPIC, NDC,
      BASEPIC, GRAPHICS).
      Frames created by the user (for instance, using \htmlref{WCSADD}{WCSADD})
      can have any \att{Domain} value, but the standard \att{Domain} names should be
      avoided unless the standard meanings are appropriate for the Frame
      being created.

   }
   \sstattributetype{
      String
   }
   \sstexamples{
      \sstexamplesubsection{
         wcsattrib m51 set domain oldpixel
      }{
         If the current co-ordinate Frame in the NDF \texttt{m51} is the
         PIXEL Frame, then this command takes a `snap-shot' of the PIXEL
         Frame and stores it as a new Frame with Domain OLDPIXEL in the WCS
         component.  Subsequent changes to the PIXEL Frame (for instance,
         produced by applications that rotate, or move the contents of
         the NDF) will not effect the OLDPIXEL Frame, which will thus
         provide a `frozen' record of the original PIXEL Frame.
      }
   }
   \sstnotes{
      \sstitemlist{

         \sstitem
         All \att{Domain} values are converted to uppercase and white space
         is removed before use.
      }
   }
}

\sstattribute{
   DSBCentre
}{
   The central position of interest in a dual-sideband spectrum
   (DSBSpecFrames only)
}{
   \sstdescription{
      This attribute specifies the central position of interest in a
      \xref{dual-sideband spectrum}{sun210}{DSBSpecFrame}. Its sole use
      is to determine the local oscillator frequency (the frequency which
      marks the boundary between the lower and upper sidebands). See the
      description of the IF (intermediate frequency) attribute for
      details of how the local oscillator frequency is calculated. The
      sideband containing this central position is referred to as the
      \texttt{{"}}observed\texttt{{"}} sideband, and the other sideband as the
      \texttt{{"}}image\texttt{{"}} sideband.

      The value is accessed as a position in the spectral system specified
      by the \htmlattref{System}{System}~ attribute, but is stored internally as
      topocentric frequency. Thus, if the \att{System} attribute of the
      DSBSpecFrame is set to \texttt{{"}}VRAD\texttt{{"}}, the
      \htmlattref{Unit}{Unit(axis)}~ attribute set to \texttt{{"}}m/s\texttt{{"}}
      and the \htmlattref{StdOfRest}{StdOfRest}~ attribute set to
      \texttt{{"}}LSRK\texttt{{"}}, then values for the \att{DSBCentre} attribute
      should be supplied as radio velocity in units of
      \texttt{{"}}m/s\texttt{{"}} relative to the kinematic LSR (alternative
      units may be used by appending a suitable units string to the end
      of the value). This value is then converted to topocentric
      frequency and stored. If (say) the \att{Unit} attribute is subsequently
      changed to \texttt{{"}}km/s\texttt{{"}} before retrieving the current value
      of the \att{DSBCentre} attribute, the stored topocentric frequency will
      be converted back to LSRK radio velocity, this time in units of
      \texttt{{"}}km/s\texttt{{"}}, before being returned.

      The default value for this attribute is 30 GHz.
   }
   \sstattributetype{
      Floating point
   }
   \sstnotes{
      \sstitemlist{

         \sstitem
         The attributes which define the transformation to or from topocentric
         frequency should be assigned their correct values before accessing
         this attribute. These potentially include \att{System}, \att{Unit}, \att{StdOfRest},
         \htmlattref{ObsLon}{ObsLon}, \htmlattref{ObsLat}{ObsLat},
         \htmlattref{ObsAlt}{ObsAlt}, \htmlattref{Epoch}{Epoch},
         \htmlattref{RefRA}{RefRA}, \htmlattref{RefDec}{RefDec}, and
         \htmlattref{RestFreq}{RestFreq}.
      }
   }
}
\sstattribute{
   Epoch
}{
   Epoch of observation
}{
   \sstdescription{
      This attribute is used to qualify the co-ordinate system described by
      a \xref{Frame}{sun210}{Frame}, by giving the moment in time when the
      co-ordinates are known
      to be correct.  Often, this will be the date of observation.

      The \att{Epoch} value is important in cases where the co-ordinate
      system changes with time.  For instance, when considering celestial
      co-ordinate systems, possible reasons for change include:
      (i) changing aberration of light caused by the observer's
      velocity (\emph{e.g.} due to the Earth's motion around the Sun), (ii)
      changing gravitational deflection by the Sun due to changes in
      the observer's position with time, (iii) fictitious motion due
      to rotation of non-inertial co-ordinate systems (\emph{e.g.} the
      old FK4 system), and (iv) proper motion of the source itself
      (although this last effect is not handled by the
      \xref{SkyFrame}{sun210}{SkyFrame}~ class because it affects
      individual sources rather than the co-ordinate system as a
      whole).

      The \att{Epoch} attribute is stored as a Modified Julian Date, and is not
      usually changed.
   }
   \sstattributetype{
      Floating point
   }
   \sstnotes{
      \sstitemlist{

         \sstitem
         Care must be taken to distinguish the \att{Epoch} value, which
         relates to motion (or apparent motion) of the source, from the
         superficially similar \htmlattref{Equinox}{Equinox}~ value.  The
         latter is used to qualify a co-ordinate system which is
         itself in motion in a (notionally) predictable way as a
         result of being referred to a slowly moving reference plane
         (\emph{e.g.} the equator).

         \sstitem
         See the description of the \htmlattref{System}{System}~ attribute
         for details of which qualifying attributes apply to each celestial
         co-ordinate system.
      }
   }
   \label{epoch:input}
   \sstdiytopic{
      Input Formats
   }{
      The formats accepted when setting an \att{Epoch}~ value are listed
      below.  They are all case-insensitive and are generally tolerant
      of extra white space and alternative field delimiters.

      \sstitemlist{

         \sstitem
         Besselian Epoch: Expressed in decimal years, with or without
         decimal places (\texttt{"B1950"} or \texttt{"B1976.13"} for example).

         \sstitem
         Julian Epoch: Expressed in decimal years, with or without
         decimal places (\texttt{"J2000"} or \texttt{"J2100.9"} for example).

         \sstitem
         Year: Decimal years, with or without decimal places (\texttt{"1996.8"}
         for example).  Such values are interpreted as a Besselian epoch
         (see above) if less than 1984.0 and as a Julian epoch otherwise.

         \sstitem
         Julian Date: With or without decimal places (\texttt{"JD 2454321.9"} for
         example).

         \sstitem
         Modified Julian Date: With or without decimal places
         (\texttt{"MJD 54321.4"} for example).

         \sstitem
         Gregorian Calendar Date: With the month expressed either as an
         integer or a 3-character abbreviation, and with optional decimal
         places to represent a fraction of a day (\texttt{"1996-10-2"} or
         \texttt{"1996-Oct-2.6"} for example).  If no fractional part of a day is
         given, the time refers to the start of the day (zero hours).

         \sstitem
         Gregorian Date and Time: Any calendar date (as above) but with
         a fraction of a day expressed as hours, minutes and seconds
         (\texttt{"1996-Oct-2 12:13:56.985"} for example).
      }
   }
   \sstdiytopic{
      Output Format
   }{

      When enquiring \att{Epoch} values, the format used is the
      ``Year'' format described under \htmlref{``Input
      Formats''}{epoch:input}.  This is a value in decimal years, which
      will be a Besselian epoch if less than 1984.0, and a Julian epoch
      otherwise.
   }
}
\sstattribute{
   Equinox
}{
   Epoch of the mean equinox (SkyFrames only)
}{
   \sstdescription{
      This attribute is used to qualify those celestial co-ordinate
      systems described by a \xref{SkyFrame}{sun210}{SkyFrame}~ that are
      notionally based on
      the ecliptic (the plane of the Earth's orbit around the Sun)
      and/or the Earth's equator.

      Both of these planes are in motion and their positions are
      difficult to specify precisely.  In practice, therefore, a model
      ecliptic and/or equator are used instead.  These, together with
      the point on the sky that defines the co-ordinate origin (the
      intersection of the two planes termed the `mean equinox') move
      with time according to some models that remove the more-rapid
      fluctuations.  The SkyFrame class supports both the old FK4 and
      the current FK5 models.

      The position of a fixed source expressed in any of these
      co-ordinate systems will appear to change with time due to
      movement of the co-ordinate system itself (rather than motion of
      the source).  Such co-ordinate systems must therefore be
      qualified by a moment in time (the `epoch of the mean equinox'
      or `equinox' for short) which allows the position of the model
      co-ordinate system on the sky to be determined.  This is the r\^{o}le
      of the \att{Equinox} attribute.

      The \att{Equinox} attribute is stored as a Modified Julian Date, but
      when setting or getting its value you may use the same formats
      as for the \htmlattref{Epoch}{Epoch}~ attribute (q.v.).

   }
   \sstattributetype{
      Floating point
   }
   \sstnotes{
      \sstitemlist{

         \sstitem
         Care must be taken to distinguish the \att{Equinox} value, which
         relates to the definition of a time-dependent co-ordinate system
         (based on solar-system reference planes which are in motion),
         from the superficially similar \att{Epoch} value.
         The latter is used to qualify co-ordinate systems where the
         positions of sources change with time (or appear to do so)
         for a variety of other reasons, such as aberration of light
         caused by the observer's motion, \emph{etc.}

         \sstitem
         See the description of the \htmlattref{System}{System}~ attribute
         for details of which qualifying attributes apply to each celestial
         co-ordinate system.
      }
   }
}

\sstattribute{
   Format(axis)
}{
   Format specification for axis values
}{
   \sstdescription{
      This attribute specifies the format to be used when displaying
      co-ordinate values associated with a particular
      \xref{Frame}{sun210}{Frame}~ axis
      (\emph{i.e.} to convert values from binary to character form).

      If no \att{Format} value is set for a Frame axis, a default value is
      supplied instead.  This is based on the value of the
      \htmlattref{Digits}{Digits/Digits(axis)}, or
      \att{Digits(axis)} attribute and is chosen so that it displays the
      requested number of digits of precision.

      The interpretation of this string depends on whether or not the
      Frame is a \xref{SkyFrame}{sun210}{SkyFrame}.  If it is not, the
      string is interpreted as a format-specification string to be
      passed to the C ``\texttt{printf}'' function (\emph{e.g.} \texttt{
      "\%1.7G"}) in order to format a single co-ordinate value
      (supplied as a double-precision number).

      For SkyFrames, the syntax and default value of the \att{Format} string is
      re-defined to allow the formatting of sexagesimal values as appropriate
      for the particular celestial co-ordinate system being represented.  The
      syntax of SkyFrame Format strings is described (below) in the
      \htmlref{``SkyFrame Formats''}{SkyFrameFormats}~ section.

   }
   \sstattributetype{
      String
   }
   \label{SkyFrameFormats}
   \sstdiytopic{
      SkyFrame Formats
   }{
      The Format string supplied for a
      \xref{SkyFrame}{sun210}{SkyFrame}~ should contain zero or more
      of the following characters.  These may occur in any order, but
      the following is recommended for clarity.

      \sstitemlist{

         \sstitem
         \texttt{"$+$"}: Indicates that a plus sign should be prefixed to positive
         values.  By default, no plus sign is used.

         \sstitem
         \texttt{"z"}: Indicates that leading zeros should be prefixed to the
         value so that the first field is of constant width, as would be
         required in a fixed-width table (leading zeros are always
         prefixed to any fields that follow).  By default, no leading
         zeros are added.

         \sstitem
         \texttt{"i"}: Use the standard ISO field separator (a colon) between
         fields.  This is the default behaviour.

         \sstitem
         \texttt{"b"}: Use a blank to separate fields.

         \sstitem
         \texttt{"l"}: Use a letter (\texttt{"h"}/\texttt{"d"}, \texttt{"m"} or \texttt{"s"}
         as appropriate) to separate fields.

         \sstitem
         \texttt{"g"}: This is the same as \texttt{"l"}, except that the
         separator characters are displayed as small superscripts when drawn
         on a graphical device.

         \sstitem
         \texttt{"d"}: Include a degrees field.  Expressing the angle purely in
         degrees is also the default if none of \texttt{"h"}, \texttt{"m"},
         \texttt{"s"} or \texttt{"t"} are given.

         \sstitem
         \texttt{"h"}: Express the angle as a time and include an hours field
         (where 24 hours correspond to 360 degrees).  Expressing the angle
         purely in hours is also the default if \texttt{"t"} is given without
         either \texttt{"m"} or \texttt{"s"}.

         \sstitem
         \texttt{"m"}: Include a minutes field.  By default this is not included.

         \sstitem
         \texttt{"s"}: Include a seconds field.  By default this is not included.
         This request is ignored if \texttt{"d"} or \texttt{"h"} is given, unless a minutes
         field is also included.

         \sstitem
         \texttt{"t"}: Express the angle as a time (where 24 hours correspond to
         360 degrees).  This option is ignored if either \texttt{"d"} or \texttt{"h"} is
         given and is intended for use where the value is to be expressed
         purely in minutes and/or seconds of time (with no hours field).  If
         \texttt{"t"} is given without \texttt{"d"}, \texttt{"h"}, \texttt{"m"} or \texttt{"s"} being
         present, then it is equivalent to \texttt{"h"}.

         \sstitem
         \texttt{"."}: Indicates that decimal places are to be given for the
         final field in the formatted string (whichever field this
         is).  The \texttt{"."} should be followed immediately by an unsigned
         integer which gives the number of decimal places required.  By
         default, no decimal places are given.

      }
      All of the above format specifiers are case-insensitive.  If
      several characters make conflicting requests (\emph{e.g.} if both \texttt{"i"}
      and \texttt{"b"} appear), then the character occurring last takes
      precedence, except that \texttt{"d"} and \texttt{"h"} always override \texttt{"t"}.
   }
   \sstexamples{
      \sstexamplesubsection{
         wcsattrib my\_data set format(1) \%10.5G
      }{
         This sets the \att{Format} attribute for Axis 1 in the current
         co-ordinate Frame in the NDF called \texttt{my\_data}, so that axis values
         are formatted as floating-point values using a minimum field
         width of ten characters, and displaying five significant figures.  An
         exponent is used if necessary.
      }
      \sstexamplesubsection{
         wcsattrib ngc5128 set format(2) bdms.2
      }{
         This sets the \att{Format} attribute for Axis 2 in the current
         co-ordinate Frame in the NDF called \texttt{ngc5128}, so that axis
         values are formatted as separate degrees, minutes and seconds field,
         separated by blanks.  The seconds field has two decimal places.
         This assumes the current co-ordinate Frame in the NDF is a
         celestial co-ordinate Frame (\emph{i.e.} a SkyFrame).
      }
   }
   \sstnotes{
      \sstitemlist{

         \sstitem
         When specifying this attribute by name, it should be
         subscripted with the number of the Frame axis to which it
         applies.
      }
   }
}

\sstattribute{
   IF
}{
   The intermediate frequency in a dual-sideband spectrum (DSBSpecFrames only)
}{
   \sstdescription{
      This attribute specifies the (topocentric) intermediate frequency
      in a \xref{dual-sideband spectrum}{sun210}{DSBSpecFrame}. Its sole
      use is to determine the local oscillator (LO) frequency (the
      frequency which marks the boundary between the lower and upper
      sidebands). The LO frequency is equal to the sum of the centre
      frequency and the intermediate frequency. Here, the \texttt{{"}}centre
      frequency\texttt{{"}} is the topocentric frequency in Hz corresponding
      to the current value of the \htmlattref{DSBCentre}{DSBCentre}~
      attribute. The value of the
      \att{IF} attribute may be positive or negative: a positive value results
      in the LO frequency being above the central frequency, whilst a
      negative aatt{IF} value results in the LO frequency being below the
      central frequency. The sign of the IF attribute value determines
      the default value for the \htmlattref{SideBand}{SideBand}~ attribute.

      When setting a new value for this attribute, the units in which the
      frequency value is supplied may be indicated by appending a
      suitable string to the end of the formatted value. If the units are
      not specified, then the supplied value is assumed to be in units of
      GHz. For instance, the following strings all result in an IF of 4
      GHz being used: \texttt{{"}}4.0\texttt{{"}}, \texttt{{"}}4.0 GHz\texttt{{"}},
      \texttt{{"}}4.0E9 Hz\texttt{{"}}, \emph{etc.}

      When getting the value of this attribute, the returned value is
      always in units of GHz. The default value for this attribute is 4 GHz.
   }
   \sstattributetype{
      Floating point
   }
}

\sstattribute{
   ImagFreq
}{
   The image sideband equivalent of the rest frequency (DSBSpecFrames only)
}{
   \sstdescription{
      This is a read-only attribute of a \xref{dual-sideband
      spectrum}{sun210}{DSBSpecFrame} that gives the frequency
      corresponding to the rest frequency, but in the opposite sideband.

      The value is calculated by first transforming the rest frequency
      (given by the \htmlattref{RestFreq}{RestFreq}~ attribute) from the standard of rest of the
      source (given by the \htmlattref{SourceVel}{SourceVel} and
      \htmlattref{SourceVRF}{SourceVRF}~ attributes) to the
      standard of rest of the observer (\emph{i.e.} the topocentric standard of
      rest). The resulting topocentric frequency is assumed to be in the
      same sideband as the value given for the
      \htmlattref{DSBCentre}{DSBCentre}~ attribute (the
      \texttt{{"}}observed\texttt{{"}} sideband), and is transformed to the other
      sideband (the \texttt{{"}}image\texttt{{"}} sideband). The new frequency is
      converted back to the standard of rest of the source, and the
      resulting value is returned as the attribute value, in units of
      GHz.
   }
   \sstattributetype{
      Floating point, read-only
   }
}

\sstattribute{
   Label(axis)
}{
   Axis label
}{
   \sstdescription{
      This attribute specifies a label to be attached to each axis of
      a Frame when it is represented (\emph{e.g.}) in graphical output.

      If a \att{Label} value has not been set for a Frame axis, then a
      suitable default is supplied, depending on whether or not the
      Frame is a \xref{SkyFrame}{sun210}{SkyFrame}.

      The default for simple Frames is the string \texttt{"Axis <\texttt{{n}}>"}, where
      \texttt{<$\texttt{{n}}$>} is \texttt{1}, \texttt{2}, \emph{etc.} for each successive axis.

      The default labels for specialised Frames (SkyFrames,
      \xref{SpecFrames}{sun210}{SpecFrame}, \emph{etc.}) depend on the
      particular co-ordinate system represented by the Frame
      (\emph{e.g.} \texttt{"Right ascension"}, \texttt{"Galactic latitude"},
      \texttt{"Frequency"}, \texttt{"Wavelength in air"}, \emph{etc.}).

   }
   \sstattributetype{
      String
   }

   \sstexamples{
      \sstexamplesubsection{
         wcsattrib my\_data set label(2) "IRAS data (marked in white)"
      }{
         This sets the \att{Label} for Axis 2 in the current Frame in the NDF
         called \texttt{my\_data}, to the string \texttt{"IRAS data (marked in white)"}.
      }
   }
   \sstnotes{
      \sstitemlist{

         \sstitem
         Axis labels are intended purely for interpretation by human
         readers and not by software.

         \sstitem
         When specifying this attribute by name, it should be
         subscripted with the number of the Frame axis to which it
         applies.
      }
   }
}

\sstattribute{
   LTOffset
}{
   The offset from UTC to Local Time, in hours (TimeFrames only)
}{
   \sstdescription{
      This specifies the offset from UTC to Local Time, in hours, for a
      \xref{TimeFrame}{sun210}{TimeFrame}. Fractional hours can be
      supplied. It is positive for time zones east of Greenwich. AST uses
      the figure as given, without making any attempt to correct for
      daylight saving. The default value is zero.
   }
   \sstattributetype{
      Floating point
   }
}

\sstattribute{
   Naxes
}{
   Number of Frame axes
}{
   \sstdescription{
      This is a read-only attribute giving the number of axes in a
      Frame (\emph{i.e.} the number of dimensions of the
      co-ordinate space that the Frame describes).  This value is determined
      when the Frame is created.
   }
   \sstattributetype{
      Integer, read-only.
   }
   \sstexamples{
      \sstexamplesubsection{
         wcsattrib my\_data get naxes
      }{
         This displays the number of axes in the current Frame of the NDF
         called \texttt{my\_data}.
      }
   }
}

\clearpage
\sstattribute{
   ObsLat
}{
   The geodetic latitude of the observer (SpecFrames only)
}{
   \sstdescription{
      This attribute specifies the geodetic latitude of the observer, in
      degrees.  Together with the \htmlattref{ObsLon}{ObsLon},
      \htmlattref{Epoch}{Epoch}, \htmlattref{RefRA}{RefRA}, and
      \htmlattref{RefDec}{RefDec}~ attributes, it defines the Doppler
      shift introduced by the observers diurnal motion around the
      Earth's axis, which is needed when converting
      \xref{SpecFrames}{sun210}{SpecFrame}~ to or from the topocentric
      standard of rest.  The maximum velocity error, which can be
      caused by an incorrect value, is 0.5~km/s.  The default value for
      the attribute is zero.

      The value is stored internally in radians, but is converted to
      and from a degrees string for access.  Some example input
      formats are: \texttt{"22:19:23.2"}, \texttt{"22 19 23.2"}, \texttt{
        "22:19.387"}, \texttt{"22.32311"}, \texttt{"N22.32311"},
      \texttt{"-45.6"}, \texttt{"S45.6"}.  As indicated, the sign of the
      latitude can optionally be indicated using characters \texttt{"N"}
      and \texttt{"S"} in place of the usual \texttt{"$+$"} and \texttt{"-"}.
      When converting the stored value to a string, the format
      \texttt{{"}}[s]dd:mm:ss.s\texttt{{"}} is used, when \texttt{"[s]"} is
      \texttt{"N"} or \texttt{"S"}.
   }
   \sstattributetype{
      String
   }
}

\sstattribute{
   ObsLon
}{
   The geodetic longitude of the observer (\xref{SpecFrames}{sun210}{SpecFrame} only)
}{
   \sstdescription{
      This attribute specifies the geodetic (or equivalently,
      geocentric) longitude of the observer, in degrees, measured
      positive eastwards.  See also attribute
      \htmlattref{ObsLat}{ObsLat}.  The default value is zero.

      The value is stored internally in radians, but is converted to
      and from a degrees string for access.  Some example input
      formats are: \texttt{"155:19:23.2"}, \texttt{"155 19 23.2"}, \texttt{
      "155:19.387"}, \texttt{"155.32311"}, \texttt{"E155.32311"}, \texttt{
      "-204.67689"}, \texttt{"W204.67689"}.  As indicated, the sign of
      the longitude can optionally be indicated using characters \texttt{
      "E"} and \texttt{"W"} in place of the usual \texttt{"$+$"} and
      \texttt{"-"}.  When converting the stored value to a string, the
      format \texttt{{"}}[s]ddd:mm:ss.s\texttt{{"}} is used, when \texttt{"[s]"}
      is \texttt{"E"} or \texttt{"W"} and the numerical value is chosen to
      be less than 180 degrees.
   }
   \sstattributetype{
      String
   }
}

\sstattribute{
   RefDec
}{
    The declination of the reference point (\htmlref{SpecFrames}{system:SpecFrame}~ only)
}{
   \sstdescription{
      This attribute specifies the FK5 J2000.0 declination of a reference
      point on the sky.  See the description of attribute
      \htmlattref{RefRA}{RefRA}~ for details.
      This attribute has a default value of zero.
   }
   \sstattributetype{
      String
   }
}


\sstattribute{
   RefRA
}{
   The right ascension of the reference point (\htmlref{SpecFrames}{system:SpecFrame}~ only)
}{
   \sstdescription{
      This attribute, together with the \htmlattref{RefDec}{RefDec}~ attribute, specifies the FK5
      J2000.0 co-ordinates of a reference point on the sky.  For one-dimensional
      spectra, this should normally be the position of the source.  For
      spectral data with spatial coverage (spectral cubes, \emph{etc.}), this should
      be close to centre of the spatial coverage.  It is used to define the
      correction for Doppler shift to be applied when converting between
      different standards of rest.

      The \att{RefRA} and \att{RefDec} attributes are stored internally in radians,
      but are converted to and from a string for access.  The format
      \texttt{{"}}hh:mm:ss.ss\texttt{{"}} is used for \att{RefRA}, and
      \texttt{{"}}dd:mm:ss.s\texttt{{"}} is used for \att{RefDec}.

      \att{RefRA} has a default value of zero.
   }
   \sstattributetype{
      String
   }
}
\sstattribute{
   RestFreq
}{
   The rest frequency (\htmlref{SpecFrames}{system:SpecFrame}~ only)
}{
   \sstdescription{
     This attribute specifies the frequency corresponding to zero
     velocity.  It is used when converting between between
     velocity-based spectral co-ordinate systems and and other
     co-ordinate systems (such as frequency, wavelength, energy,
     \emph{etc.}).  The default value is 1.0E5 GHz.

     When setting a new value for this attribute, the new value can be
     supplied either directly as a frequency, or indirectly as a
     wavelength or energy, in which case the supplied value is
     converted to a frequency before being stored.  The nature of the
     supplied value is indicated by appending text to the end of the
     numerical value indicating the units in which the value is
     supplied.  If the units are not specified, then the supplied
     value is assumed to be a frequency in units of GHz.  If the
     supplied unit is a unit of frequency, the supplied value is
     assumed to be a frequency in the given units.  If the supplied
     unit is a unit of length, the supplied value is assumed to be a
     (vacuum) wavelength.  If the supplied unit is a unit of energy,
     the supplied value is assumed to be an energy.  For instance, the
     following strings all result in a rest frequency of around 1.4E14~Hz
     being used: \texttt{"1.4E5"}, \texttt{"1.4E14 Hz"}, \texttt{"1.4E14
     s$*$$*$-1"}, \texttt{"1.4E5 GHz"}, \texttt{"2.14E-6 m"}, \texttt{"21400
     Angstrom"}, \texttt{"9.28E-20 J"}, \texttt{"9.28E-13 erg"}, \texttt{
     "0.58 eV"}, \emph{etc.}

      When getting the value of this attribute, the returned value is
      always a frequency in units of GHz.
   }
   \sstattributetype{
      Floating point
   }
}
\sstattribute{
   SideBand
}{
   Indicates which sideband a dual sideband spectrum represents (DSBSpecFrames only)
}{
   \sstdescription{
      This attribute indicates whether a \xref{dual-sideband
      spectrum}{sun210}{DSBSpecFrame} currently represents its lower or
      upper sideband, or an offset from the local oscillator frequency.
      When querying the current value, the returned string is always one
      of \texttt{{"}}usb\texttt{{"}} (for upper sideband), \texttt{{"}}lsb\texttt{{"}}
      (for lower sideband), or \texttt{{"}}lo\texttt{{"}} (for offset from the
      local oscillator frequency). When setting a new value, any of the
      strings \texttt{{"}}lsb\texttt{{"}}, \texttt{{"}}usb\texttt{{"}},
      \texttt{{"}}observed\texttt{{"}}, \texttt{{"}}image\texttt{{"}} or
      \texttt{{"}}lo\texttt{{"}} may be supplied (case insensitive). The
      \texttt{{"}}observed\texttt{{"}} sideband is which ever sideband (upper or
      lower) contains the central spectral position given by attribute
      DSBCentre, and the \texttt{{"}}image\texttt{{"}} sideband is the other
      sideband. It is the sign of the \htmlattref{IF}{IF}~ attribute
      which determines if the observed sideband is the upper or lower
      sideband. The default value for \att{SideBand} is the observed sideband.
   }
   \sstattributetype{
      String
   }
}
\sstattribute{
   SourceVel
}{
   The source velocity (\htmlref{SpecFrames}{system:SpecFrame}~ only)
}{
   \sstdescription{

      This attribute (together with \htmlattref{SourceVRF}{SourceVRF},
      \htmlattref{RefRA}{RefRA}, and \htmlattref{RefDec}{RefDec})
      defines the `Source' standard of rest (see attribute
      \htmlattref{StdOfRest}{StdOfRest}).  This is a rest frame that
      is moving towards the position given by \att{RefRA} and {\att
      RefDec} at a velocity given by SourceVel (in km/s).  When
      setting a value for \htmlattref{SourceVel}{SourceVel}~ using
      \htmlref{WCSATTRIB}{WCSATTRIB}, the velocity should be supplied
      in the rest frame specified by the \att{SourceVRF} attribute.
      Likewise, when getting the value of \att{SourceVel}, it will be
      returned in the rest frame specified by the \att{SourceVRF}
      attribute.

      The default value is zero.

   }
   \sstattributetype{
      Floating point
   }
}

\pagebreak
\sstattribute{
   SourceVRF
}{
   Rest frame in which the source velocity is stored (\htmlref{SpecFrames}{system:SpecFrame}~ only)
}{
   \sstdescription{
      This attribute identifies the rest frame in which the source
      velocity is stored (the source velocity is accessed using attribute
      \htmlattref{SourceVel}{SourceVel}).  When setting a new value for the
      SourceVel attribute, the source velocity should be supplied in the
      rest frame indicated by this attribute.  Likewise, when getting the
      value of the \att{SourceVel} attribute, the velocity will be returned in
      this rest frame.

      If the value of \att{SourceVRF} is changed, the value stored for
      \att{SourceVel} will be converted from the old to the new rest frame.

      The values that can be supplied are the same as for the
      \htmlattref{StdOfRest}{StdOfRest}~ attribute (except that {\att
      SourceVRF} cannot be set to \texttt{"Source"}).  The default value
      is \texttt{"Helio"}.
   }
   \sstattributetype{
      String
   }
}

\sstattribute{
   StdOfRest
}{
   Standard of rest (SpecFrames only)
}{
   \sstdescription{
      This attribute identifies the standard of rest to which the spectral
      axis values of a \htmlref{SpecFrame}{system:SpecFrame}~ refer, and may take any of the values
      listed in the \htmlref{``Standards of Rest''}{stdofrest:StdOfRest}~ section (below).

      The default \att{StdOfRest} value is \texttt{"Helio"}.
   }
   \sstattributetype{
      String
   }
   \label{stdofrest:StdOfRest}
   \sstdiytopic{
      Standards of Rest
   }{
      The \xref{SpecFrame}{sun210}{SpecFrame}~ class supports the following
      \att{StdOfRest} values (all are case-insensitive).

      \sstitemlist{

         \sstitem
         \texttt{"Topocentric"}, \texttt{"Topocent"} or \texttt{"Topo"}: The observers rest-frame (assumed
         to be on the surface of the earth).  Spectra recorded in this standard of
         rest suffer a Doppler shift which varies over the course of a day
         because of the rotation of the observer around the axis of the earth.
         This standard of rest must be qualified using the \htmlattref{ObsLat}{ObsLat},
         \htmlattref{ObsLon}{ObsLon}, \htmlattref{Epoch}{Epoch},
         \htmlattref{RefRA}{RefRA}, and \htmlattref{RefDec}{RefDec}~ attributes.

         \sstitem
         \texttt{"Geocentric"}, \texttt{"Geocentr"} or \texttt{"Geo"}: The rest-frame of the earth centre.
         Spectra recorded in this standard of rest suffer a Doppler shift which
         varies over the course of a year because of the rotation of the earth
         around the Sun.  This standard of rest must be qualified using the
         \att{Epoch}, \att{RefRA}, and \att{RefDec} attributes.

         \sstitem
         \texttt{"Barycentric"}, \texttt{"Barycent"} or \texttt{"Bary"}: The rest-frame of the solar-system
         barycentre.  Spectra recorded in this standard of rest suffer a Doppler
         shift which depends both on the velocity of the Sun through the Local
         Standard of Rest, and on the movement of the planets through the solar
         system.  This standard of rest must be qualified using the
         \att{Epoch}, \att{RefRA}, and \att{RefDec} attributes.

         \sstitem
         \texttt{"Heliocentric"}, \texttt{"Heliocen"} or \texttt{"Helio"}: The rest-frame of the Sun.
         Spectra recorded in this standard of rest suffer a Doppler shift which
         depends on the velocity of the Sun through the Local Standard of Rest.
         This standard of rest must be qualified using the \att{RefRA} and
         \att{RefDec} attributes.

         \sstitem
         \texttt{"LSR"}, \texttt{"LSRK"}: The rest-frame of the kinematical Local Standard of
         Rest.  Spectra recorded in this standard of rest suffer a Doppler shift
         which depends on the velocity of the kinematical Local Standard of Rest
         through the galaxy.  This standard of rest must be qualified using the
         \att{RefRA} and \att{RefDec} attributes.

         \sstitem
         \texttt{"LSRD"}: The rest-frame of the dynamical Local Standard of Rest.  Spectra
         recorded in this standard of rest suffer a Doppler shift which depends
         on the velocity of the dynamical Local Standard of Rest through the
         galaxy.   This standard of rest must be qualified using the
         \att{RefRA} and \att{RefDec} attributes.

         \sstitem
         \texttt{"Galactic"}, \texttt{"Galactoc"} or \texttt{"Gal"}: The rest-frame of the galactic centre.
         Spectra recorded in this standard of rest suffer a Doppler shift which
         depends on the velocity of the galactic centre through the local group.
         This standard of rest must be qualified using the \att{RefRA} and
         \att{RefDec} attributes.

         \sstitem
         \texttt{"Local\_group"}, \texttt{"Localgrp"} or \texttt{"LG"}: The rest-frame of the local group.
         This standard of rest must be qualified using the \att{RefRA} and
         \att{RefDec} attributes.

         \sstitem
         \texttt{"Source"}, or \texttt{"src"}: The rest-frame of the source.  This standard of
         rest must be qualified using the \att{RefRA}, \att{RefDec}, and
         \htmlattref{SourceVel}{SourceVel}~ attributes.

      }
      Where more than one alternative \htmlattref{System}{System}~ value is shown above, the
      first of these will be returned when an enquiry is made.
   }
}
\sstattribute{
   Symbol(axis)
}{
   Axis symbol
}{
   \sstdescription{
      This attribute specifies a short-form symbol to be used to
      represent co-ordinate values for a particular axis of a
      Frame.  This might be used (\emph{e.g.}) in algebraic
      expressions where
      a full description of the axis would be inappropriate.  Examples
      include \texttt{"RA"} and \texttt{"Dec"} (for right ascension and declination).

      If a \att{Symbol} value has not been set for a Frame axis, then a
      suitable default is supplied.

      The default \att{Symbol} value supplied for simple Frames is the
      string ``$<$\texttt{Domain}$>$$<$\texttt{{n}}$>$'', where
      $<$\texttt{{n}}$>$ is \texttt{1}, \texttt{2}, \emph{etc.} for successive
      axes, and $<$Domain$>$ is the value of the Frame's \htmlattref{Domain}{Domain}~
      attribute (truncated if necessary so that the final string
      does not exceed 15 characters).  If no \att{Domain} value has been
      set, \texttt{"x"} is used as the $<$Domain$>$ value in constructing
      this default string.

      Specialised Frames (\xref{SkyFrame}{sun210}{SkyFrame},
      \xref{SpecFrame}{sun210}{SpecFrame},\emph{etc.}) re-define the
      default \att{Symbol} value to be appropriate for
      the particular co-ordinate system being represented.

   }
   \sstattributetype{
      String
   }
   \sstexamples{
      \sstexamplesubsection{
         wcsattrib my\_data set symbol(2)  AR
      }{
         This sets the \att{Symbol} for Axis 2 in the current Frame in the NDF
         called \texttt{my\_data}, to the string \texttt{"AR"}.
      }
   }
   \sstnotes{
      \sstitemlist{

         \sstitem
         When specifying this attribute by name, it should be
         subscripted with the number of the Frame axis to which it
         applies.
      }
   }
}
\sstattribute{
   System
}{
   Co-ordinate system used to describe positions within the domain
}{
   \sstdescription{
      In general it is possible for positions within a given physical
      domain to be described using one of several different co-ordinate
      systems.  For instance, the \xref{SkyFrame}{sun210}{SkyFrame}~ class can
      use galactic co-ordinates, equatorial co-ordinates \emph{etc.} to describe
      positions on the sky.  As another example, the
      \xref{SpecFrame}{sun210}{SpecFrame}~ class can use frequency, wavelength,
      velocity \emph{etc.} to describe a position within an electromagnetic
      spectrum.  The \att{System} attribute identifies the particular co-ordinate
      system represented by a \xref{Frame}{sun210}{Frame}.  Each class of Frame
      defines a set of acceptable values for this attribute, as listed
      below (all are case insensitive).  Where more than one alternative
      \att{System}~ value is shown, the first of will be returned when an
      enquiry is made.
   }
   \sstattributetype{
      String
   }
   \sstapplicability{
      \sstsubsection{
         Frame
      }{
         The \att{System} attribute for a basic Frame always equals \texttt{"Cartesian"},
         and may not be altered.
      }
      \label{system:SkyFrame}
      \sstsubsection{
         SkyFrame
      }{
         The \xref{SkyFrame class}{sun210}{SkyFrame}~ supports the following
         \att{System} values and associated celestial co-ordinate systems.

         \ssthitemlist{

            \sstitem
            \texttt{"FK4"}: The old FK4 (barycentric) equatorial
            co-ordinate system, which should be qualified by an
            \htmlattref{Equinox}{Equinox}~ value.  The underlying
            model on which this is based is non-inertial and rotates slowly
            with time, so for accurate work FK4 co-ordinate systems should
            also be qualified by an \htmlattref{Epoch}{Epoch}~ value.

            \sstitem
            \texttt{"FK4-NO-E"} or \texttt{"FK4\_NO\_E"}: The old FK4 (barycentric) equatorial
            system but without the \emph{E-terms of aberration} (\emph{e.g.} some radio
            catalogues).  This co-ordinate system should also be qualified by
            both an \att{Equinox} and an \att{Epoch} value.

            \sstitem
            \texttt{"FK5"} or \texttt{"EQUATORIAL"}: The modern FK5 (barycentric) equatorial
            co-ordinate system.  This should be qualified by an
            \htmlattref{Equinox}{Equinox}~ value.

            \sstitem
            \texttt{"GAPPT"}, \texttt{"GEOCENTRIC"} or \texttt{"APPARENT"}: The geocentric apparent
            equatorial co-ordinate system, which gives the apparent positions
            of sources relative to the true plane of the Earth's equator and
            the equinox (the co-ordinate origin) at a time specified by the
            qualifying \att{Epoch} value.  (Note that no \att{Equinox} is needed to
            qualify this co-ordinate system because no model `mean equinox'
            is involved.)  These co-ordinates give the apparent right
            ascension and declination of a source for a specified date of
            observation, and therefore form an approximate basis for
            pointing a telescope.  Note, however, that they are applicable to
            a fictitious observer at the Earth's centre, and therefore
            ignore such effects as atmospheric refraction and the (normally
            much smaller) aberration of light due to the rotational velocity
            of the Earth's surface.   Geocentric apparent co-ordinates are
            derived from the standard FK5 (J2000.0) barycentric co-ordinates
            by taking account of the gravitational deflection of light by
            the Sun (usually small), the aberration of light caused by the
            motion of the Earth's centre with respect to the barycentre
            (larger), and the precession and nutation of the Earth's spin
            axis (normally larger still).

            \sstitem
            \texttt{"ECLIPTIC"}: Ecliptic co-ordinates (IAU 1980), referred to the
            ecliptic and mean equinox specified by the qualifying \att{Equinox}
            value.

            \sstitem
            \texttt{"GALACTIC"}: Galactic co-ordinates (IAU 1958).

            \sstitem
            \texttt{"SUPERGALACTIC"}: De Vaucouleurs Supergalactic co-ordinates.

            \sstitem
            \texttt{"UNKNOWN"}: Any other general spherical co-ordinate system.  No
            \xref{Mapping}{sun210}{Mapping}~ can be created between a pair of
            SkyFrames if either of the SkyFrames has System set to \texttt{"Unknown"}.

         }
         Currently, the default System value is \texttt{"FK5"}.
      }
      \label{system:SpecFrame}
      \sstsubsection{
         SpecFrame
      }{
         The \xref{SpecFrame}{sun210}{SpecFrame}~ \xref{DSBSpecFrame}{sun210}{DSBSpecFrame}~
         classes supports the following \att{System} values and associated spectral co-ordinate
         systems (the default is \texttt{"WAVE"}---wavelength):

         \ssthitemlist{

            \sstitem
            \texttt{"FREQ"}: Frequency (Hz)

            \sstitem
            \texttt{"ENER"} or \texttt{"ENERGY"}: Energy (J)

            \sstitem
            \texttt{"WAVN"} or \texttt{"WAVENUM"}: Wave-number (1/m)

            \sstitem
            \texttt{"WAVE"} or \texttt{"WAVELEN"}: Vacuum wavelength (m)

            \sstitem
            \texttt{"AWAV"} or \texttt{"AIRWAVE"}: Wave-length in air (m)

            \sstitem
            \texttt{"VRAD"} or \texttt{"VRADIO"}: Radio velocity (m/s)

            \sstitem
            \texttt{"VOPT"} or \texttt{"VOPTICAL"}: Optical velocity (m/s)

            \sstitem
            \texttt{"ZOPT"} or \texttt{"REDSHIFT"}: Redshift (dimensionless)

            \sstitem
            \texttt{"BETA"}: Beta factor (dimensionless)

            \sstitem
            \texttt{"VELO"} or \texttt{"VREL"}: Relativistic velocity (m/s)

         }
         The default value for the \htmlattref{Unit}{Unit(axis)}~ attribute
         for each system is shown in parentheses.  Note, changes to
         the Unit attribute for a SpecFrame will result in the Mapping
         from pixel to spectral co-ordinates being modified in order
         to reflect the change in units.
      }

      \label{system:TimeFrame}
      \sstsubsection{
         TimeFrame
      }{
         The TimeFrame class supports the following System values and
         associated co-ordinate systems (the default is \texttt{{"}}MJD\texttt{{"}}):

         \sstitemlist{

            \sstitem
            \texttt{{"}}MJD\texttt{{"}}: Modified Julian Date (d)

            \sstitem
            \texttt{{"}}JD\texttt{{"}}: Julian Date (d)

            \sstitem
            \texttt{{"}}JEPOCH\texttt{{"}}: Julian epoch (yr)

            \sstitem
            \texttt{{"}}BEPOCH\texttt{{"}}: Besselian (yr)

         }

         The default value for the \att{Unit} attribute for each system is
         shown in parentheses. Strictly, these systems should not allow
         changes to be made to the units. For instance, the usual
         definition of \texttt{{"}}MJD\texttt{{"}} and \texttt{{"}}JD\texttt{{"}} include
         the statement that the values will be in units of days. However,
         AST does allow the use of other units with all the above
         supported systems (except BEPOCH), on the understanding that
         conversion to the \texttt{{"}}correct\texttt{{"}} units involves nothing
         more than a simple scaling (1 yr = 365.25 d, 1 d = 24 h, 1 h =
         60 min, 1 min = 60 s). Besselian epoch values are defined in
         terms of tropical years of 365.2422 days, rather than the usual
         Julian year of 365.25 days. Therefore, to avoid any confusion,
         the \att{Unit} attribute is automatically cleared to
         \texttt{{"}}yr\texttt{{"}} when a System value of BEPOCH System is
         selected, and an error is reported if any attempt is
         subsequently made to change the \att{Unit} attribute.
      }
   }
}
\sstattribute{
   TimeOrigin
}{
   The zero point for TimeFrame axis values (TimeFrames only)
}{
   \sstdescription{
      This specifies the origin from which all time values are measured
      within a \xref{TimeFrame}{sun210}{TimeFrame}.
      The default value (zero) results in the TimeFrame describing
      absolute time values in the system given by the
      \htmlattref{System}{System}~ attribute
      (\emph{e.g.} MJD, Julian epoch, \emph{etc}). If a TimeFrame is to be used to
      describe elapsed time since some origin, the TimeOrigin attribute
      should be set to hold the required origin value. The
      \att{TimeOrigin} value stored inside the TimeFrame structure is
      modified whenever \att{TimeFrame}
      attribute values are changed so that it refers to the original moment
      in time.
   }
   \sstattributetype{
      Floating point
   }
   \sstdiytopic{
      Input Formats
   }{
      The formats accepted when setting a \att{TimeOrigin} value are listed
      below. They are all case-insensitive and are generally tolerant
      of extra white space and alternative field delimiters:

      \sstitemlist{

         \sstitem
         Besselian Epoch: Expressed in decimal years, with or without
         decimal places (\texttt{{"}}B1950\texttt{{"}} or \texttt{{"}}B1976.13\texttt{{"}} for example).

         \sstitem
         Julian Epoch: Expressed in decimal years, with or without
         decimal places (\texttt{{"}}J2000\texttt{{"}} or \texttt{{"}}J2100.9\texttt{{"}} for example).

         \sstitem
         Units: An unqualified decimal value is interpreted as a value in
         the system specified by the TimeFrame's \att{System} attribute, in
         the units given by the TimeFrame's \htmlattref{Unit}{Unit(axis)}~
         attribute. Alternatively, an
         appropriate unit string can be appended to the end of the floating
         point value (\texttt{{"}}123.4 d\texttt{{"}} for example), in which case the supplied value
         is scaled into the units specified by the \att{Unit} attribute.

         \sstitem
         Julian Date: With or without decimal places (\texttt{{"}}JD 2454321.9\texttt{{"}} for
         example).

         \sstitem
         Modified Julian Date: With or without decimal places
         (\texttt{{"}}MJD 54321.4\texttt{{"}} for example).

         \sstitem
         Gregorian Calendar Date: With the month expressed either as an
         integer or a three-character abbreviation, and with optional decimal
         places to represent a fraction of a day (\texttt{{"}}1996-10-2\texttt{{"}} or
         \texttt{{"}}1996-Oct-2.6\texttt{{"}} for example). If no fractional part of a day is
         given, the time refers to the start of the day (zero hours).

         \sstitem
         Gregorian Date and Time: Any calendar date (as above) but with
         a fraction of a day expressed as hours, minutes and seconds
         (\texttt{{"}}1996-Oct-2 12:13:56.985\texttt{{"}} for example). The date and time can be
         separated by a space or by a \texttt{{"}}T\texttt{{"}} (as used by ISO8601 format).
      }
   }
   \sstdiytopic{
      Output Format
   }{
      When enquiring \att{TimeOrigin} values, the returned formatted floating
      point value represents a value in the TimeFrame's \att{System}, in the
      unit specified by the TimeFrame's \att{Unit} attribute.
   }
}
\sstattribute{
   TimeScale
}{
   Time scale (TimeFrames only)
}{
   \sstdescription{
      This attribute identifies the time scale to which the time axis
      values of a \xref{TimeFrame}{sun210}{TimeFrame} refer, and may take
      any of the values listed in the \texttt{{"}}Time Scales\texttt{{"}} section
      (below).

      The default \att{TimeScale} value depends on the current
      \htmlattref{System}{System} value; if the current \att{TimeFrame}
      system is \texttt{{"}}Besselian epoch\texttt{{"}} the default
      is \texttt{{"}}TT\texttt{{"}}, otherwise it is \texttt{{"}}TAI\texttt{{"}}.
      Note, if the \att{System} attribute is set so that the TimeFrame
      represents Besselian Epoch, then an error will be reported if an
      attempt is made to set the \att{TimeScale} to anything other than TT.

      Note, the supported time scales fall into two groups. The first
      group containing UT1, GMST, LAST and LMST define time in terms of
      the orientation of the earth. The second group (containing all the
      remaining time scales) define time in terms of an atomic process.
      Since the rate of rotation of the earth varies in an unpredictable
      way, conversion between two timescales in different groups relies
      on a value being supplied for the Dut1 attribute. This attribute
      specifies the difference between the UT1 and UTC time scales, in
      seconds, and defaults to zero. See the \xref{documentation for the Dut1
      attribute}{sun210}{Dut1} in SUN/210 for further details.
   }
   \sstattributetype{
      String
   }
   \sstdiytopic{
      Time Scales
   }{
      The TimeFrame class supports the following \att{TimeScale} values (all are
      case-insensitive):

      \sstitemlist{

         \sstitem
         \texttt{"TAI"} -- International Atomic Time

         \sstitem
         \texttt{"UTC"} -- Coordinated Universal Time

         \sstitem
         \texttt{"UT1"} -- Universal Time

         \sstitem
         \texttt{"GMST"} -- Greenwich Mean Sidereal Time

         \sstitem
         \texttt{"LAST"} -- Local Apparent Sidereal Time

         \sstitem
         \texttt{"LMST"} -- Local Mean Sidereal Time

         \sstitem
         \texttt{"TT"} -- Terrestrial Time

         \sstitem
         \texttt{"TDB"} -- Barycentric Dynamical Time

         \sstitem
         \texttt{"TCB"} -- Barycentric Coordinate Time

         \sstitem
         \texttt{"TCG"} -- Geocentric Coordinate Time

         \sstitem
         \texttt{"LT"} -- Local Time (the offset from UTC is given by
          attribute LTOffset)

      }
      An very informative description of these and other time scales is available at \newline
      \htmladdnormallink{\texttt{http://www.ucolick.org/$\sim$sla/leapsecs/timescales.htm}}
      {http://www.ucolick.org/$\sim$sla/leapsecs/timescales.htm}.
   }
   \sstdiytopic{
      UTC Warnings
   }{
      UTC should ideally be expressed using separate hours, minutes and
      seconds fields (or at least in seconds for a given date) if leap seconds
      are to be taken into account. Since the TimeFrame class represents
      each moment in time using a single floating point number (the axis value)
      there will be an ambiguity during a leap second. Thus an error of up to
      1 second can result when using AST to convert a UTC time to another
      time scale if the time occurs within a leap second. Leap seconds
      occur at most twice a year, and are introduced to take account of
      variation in the rotation of the earth. The most recent leap second
      occurred on 1st January 1999. Although in the vast majority of cases
      leap second ambiguities won't matter, there are potential problems in
      on-line data acquisition systems and in critical applications involving
      taking the difference between two times.
   }
}
\sstattribute{
   Title
}{
   Frame title
}{
   \sstdescription{
      This attribute holds a string that is used as a title in (\emph{e.g.})
      graphical output to describe the co-ordinate system that a
      Frame represents.  Examples might be \texttt{"Detector Co-ordinates"} or
      \texttt{"Galactic Co-ordinates"}.

      If a \att{Title} value has not been set for a Frame, then a suitable
      default is supplied.

      The default supplied by the Frame class is \texttt{"<n>-d co-ordinate
      system"}, where \texttt{<n>} is the number of Frame axes
      (\htmlattref{Naxes}{Naxes}~ attribute).

      Specialised Frames (\xref{SkyFrame}{sun210}{SkyFrame},
      \xref{SpecFrame}{sun210}{SpecFrame}, \emph{etc.}) re-define the
      default Title value to be appropriate to the particular co-ordinate
      system being represented.
   }
   \sstattributetype{
      String
   }
   \sstexamples{
      \sstexamplesubsection{
         wcsattrib my\_data set Title "My own data"
      }{
         This sets the \att{Title} for the current Frame in the NDF
         called \texttt{my\_data}, to the string \texttt{"My own data"}.
      }
   }
   \sstnotes{
      \sstitemlist{

         \sstitem
         A Frame's \att{Title} is intended purely for interpretation by human
         readers and not by software.
      }
   }
}

\sstattribute{
   Unit(axis)
}{
   Axis physical units
}{
   \sstdescription{
      This attribute contains a textual representation of the physical
      units used to represent co-ordinate values on a particular axis
      of a \xref{Frame}{sun210}{Frame}.

      Specialised Frames (\xref{SkyFrame}{sun210}{SkyFrame},
      \xref{SpecFrame}{sun210}{SpecFrame}, \emph{etc.}) re-define the
      default \att{Unit} values to be appropriate to the particular co-ordinate
      system being represented.

      For most classes, the \att{Unit} attribute is a purely descriptive
      comment intended for human readers and makes no difference to
      the operation of the software.  However, there are some classes
      that have \emph{active} \att{Unit} attributes.  Changing the \att{Unit}
      attribute for such classes will result in the
      \xref{Mappings}{sun210}{Mapping}~ within the WCS
      \xref{FrameSet}{sun210}{FrameSet}~ being modified in order
      to reflect the change in units.
      By default, only SpecFrames have an active \att{Unit} attribute.

      In general, the syntax of the \att{Unit} attribute should follow the
      recommendations made in the FITS standard (see the paper
      ``Representation of world coordinates in FITS'' by Greisen \&
      Calabretta (available at
\htmladdnormallink{\texttt{http://www.cv.nrao.edu/fits/documents/wcs/wcs.html}}
{http://www.cv.nrao.edu/fits/documents/wcs/wcs.html}).
   }
   \sstattributetype{
      String
   }
   \sstnotes{
      \sstitemlist{

         \sstitem
         When specifying this attribute by name, it should be
         subscripted with the number of the Frame axis to which it
         applies (unless the Frame has only one axis).
      }
   }
}
\sstmaintoc

\newpage
\section{\xlabel{ap_plotting_attr}Descriptions of Plotting Attributes\label{ap:plotting_attr}}
This appendix lists the available plotting attributes (see
\latexhtml{Section~\ref{se:style}}{\htmlref{Plotting Styles and Attributes}
{se:style}}). Note, the default value included in each description is the value
that will be used if no other default value is supplied (for instance within a
defaults file). In particular, the standard \KAPPA\ default style files
contained in \texttt{\$KAPPA\_DIR} include alternative default values for several
attributes which will be used in preference to those described below.

Individual applications may override the normal behaviour of
particular attributes, in which case the description of the application
will include details.

It is often useful to specify values for some of the attributes of the current
\xref{Frame}{sun210}{Frame}~ when giving a plotting style (title, axis labels,
\emph{etc.}).  These additional attributes are described
\slhyperref{here}{in Appendix~}{}{ap:frmatt}.

\sstminitoc{}
\sstnomaintoc

\sstattribute{
   Border
}{
   Draw a border around valid regions?
}{
   \sstdescription{
      This attribute controls the appearance of annotated axes by determining
      whether a border is drawn around regions corresponding to the
      valid co-ordinates.

      If the \att{Border} value of a plot is non-zero, then this border will
      be drawn as part of the axes.  Otherwise, the border is not drawn
      (although axis labels and tick marks will still appear, unless
      other relevant attributes indicate that they should
      not).  The default behaviour is to draw the border if tick marks
      and numerical labels will be drawn around the edges of the
      plotting area (see the \htmlref{\att{Labelling}}{Labelling}~ attribute),
      but to omit it otherwise.
   }
   \sstattributetype{
      Integer (boolean).
   }
}
\sstattribute{
   Colour(element)
}{
   Colour index for a plot element
}{
   \sstdescription{
      This attribute determines the colour index used when drawing each
      element of of a plot.  It takes a separate value for each graphical
      element so that, for instance, the setting \texttt{"Colour(title)=2"} causes
      the title to be drawn using colour index 2.  Standard X colour
      names, space-separated floating-point RGB triples, or
      \htmlref{HTML colour codes}{htmlcolour}
      (optionally using a \texttt{"@"} inplace of a \texttt{"\#"} to avoid the code
      being interpreted as a comment within a style file) can also be
      specified, and the nearest available colour will be used if
      the requested colour is not currently in the palette.

      The default behaviour is for all graphical elements to be drawn
      using Pen~1.
   }
   \sstattributetype{
      Integer.
   }
   \sstnotes{
      \sstitemlist{

         \sstitem
         For a list of the graphical elements available, see
         \slhyperref{here}{Section~}{}{se:styles}.

         \sstitem

         If no graphical element is specified, (\emph{e.g.} \att{Colour} instead
         of \att{Colour(title)}), then all graphical elements are affected.

      }
   }
}
\sstattribute{
   DrawAxes
}{
   Draw axes?
}{
   \sstdescription{
      This attribute controls the appearance of annotated axes by
      determining whether curves representing co-ordinate axes should be
      drawn.

      If drawn, these axis lines will pass through any tick marks
      associated with numerical labels drawn to mark values on the
      axes.  The location of these tick marks and labels (and hence the
      axis lines) is determined by the
      \htmlattref{LabelAt(axis)}{LabelAt(axis)}~ attribute.

      If the \att{DrawAxes} value is non-zero (the default), then
      axis lines will be drawn, otherwise they will be omitted.
   }
   \sstattributetype{
      Integer (boolean).
   }
   \sstnotes{
      \sstitemlist{

         \sstitem
         Axis lines are drawn independently of any co-ordinate grid
         lines (see the \htmlref{\att{Grid}}{Grid}~ attribute) so grid lines may be used to
         substitute for axis lines if required.

         \sstitem
         In some circumstances, numerical labels and tick marks are
         drawn around the edges of the plotting area (see the
         \htmlattref{Labelling}{Labelling}~ attribute).  In this case,
         the value of the \att{DrawAxes} attribute is ignored.
      }
   }
}
\sstattribute{
   DrawDSB
}{
   Annotate both sidebands if the spectral axis represents a dual
   sideband spectrum?
}{
   \sstdescription{
      This attribute controls the appearance of annotated axes by
      determining what to draw for a spectral axis representing a dual
      sideband spectrum. The sideband selected using the
      \xref{\att{SideBand}}{sun210}{SideBand} attribute will always be
      annotated on the edge of the Plot selected using the
      \htmlattref{Edge(axis)}{Edge(axis)} attribute. In addition, if the
      DrawDSB attribute is non-zero (the default) the other sideband will
      be annotated on the opposite edge of the Plot. If DrawDSB is zero,
      then the other sideband will not be annotated.
   }
   \sstattributetype{
      Integer (boolean).
   }
}
\sstattribute{
   DrawTitle
}{
   Draw a title?
}{
   \sstdescription{
      This attribute controls the appearance of annotated axes by determining
      whether a title is drawn.

      If the \att{DrawTitle}~ value is non-zero (the default), then
      the title will be drawn, otherwise it will be omitted.
   }
   \sstattributetype{
      Integer (boolean).
   }
   \sstnotes{
      \sstitemlist{

         \sstitem
         The text used for the title is obtained from the
         \htmlattref{Title}{plotel:Title}~ attribute.

         \sstitem
         The vertical placement of the title can be controlled using
         the \htmlattref{TitleGap}{TitleGap}~ attribute.
      }
   }
}
\sstattribute{
   Edge(axis)
}{
   Which edges to label
}{
   \sstdescription{

      This attribute controls the appearance of annotated axes by
      determining which edges are used for displaying numerical and
      descriptive axis labels.  It takes a separate value for each
      physical axis so that, for instance, the setting \texttt{Edge(2)=left}~
      specifies which edge to use to display labels for the second axis.

      The values \texttt{left}, \texttt{top}, \texttt{right} and \texttt{bottom} (or any
      abbreviation) can be supplied for this attribute.  The default is
      usually \texttt{bottom} for the first axis and \texttt{left} for the second
      axis.  However, if exterior labelling was requested (see the
      \htmlattref{Labelling}{Labelling}~ attribute) but cannot be produced
      using these default \att{Edge} values, then the default
      values will be swapped if this enables exterior labelling to be produced.
   }
   \sstattributetype{
      String.
   }
   \sstnotes{
      \sstitemlist{

         \sstitem
         In some circumstances, numerical labels will be drawn along
         internal grid lines instead of at the edges of the plotting
         area (see the \htmlattref{Labelling}{Labelling}~ attribute).
         In this case, the \att{Edge} attribute only affects the
         placement of the descriptive labels (these are drawn at the
         edges of the plotting area, rather than along the axis lines).
      }
   }
}
\sstattribute{
   FileInTitle
}{
   Include the NDF name as a second line in the title?
}{
   \sstdescription{
      This attribute controls the appearance of annotated axes by
      determining whether or not the title at the top of the plot will
      include a second line giving the NDF name. The default value of
      zero results in the title containing only the text given by the
      Title attribute. A non-zero value will result in a second line
      containing the NDF name being added to the title.
   }
   \sstattributetype{
      Integer (boolean).
   }
}
\sstattribute{
   Font(element)
}{
   Character fount for a plot element
}{
   \sstdescription{
      This attribute determines the character fount index used when
      drawing each element of a plot.  It takes a separate value for
      graphical element so that, for instance, the setting
      \texttt{Font(title)=2}~ causes the title to be drawn using fount number 2.

      The range of integer fount indices available and the appearance
      of the resulting text is determined by the \PGPLOT\  graphics package,
      but include:

      \begin{enumerate}
      \item A simple single-stroke fount (the default)
      \item A roman fount
      \item An italic fount
      \item A script fount
      \end{enumerate}

   }
   \sstattributetype{
      Integer.
   }
   \sstnotes{
      \sstitemlist{

         \sstitem
         For a list of the graphical elements available, see
         \slhyperref{here}{Section~}{}{se:styles}.

         \sstitem
         If no graphical element is specified, (\emph{e.g.} \att{Font} instead
         of \att{Font(title)}), then all graphical elements are affected.
      }
   }
}
\sstattribute{
   Gap(axis)
}{
   Interval between major axis values
}{
   \sstdescription{
      This attribute controls the appearance of annotated axes by
      determining the interval between the major axis values at which
      (for example) major tick marks are drawn.  It takes a separate value
      for each physical axis so that, for instance, the setting \texttt{
      Gap(2)=3.0}~ specifies the interval between major values along the
      second axis.

      The \att{Gap} value supplied will usually be rounded to the nearest
      `nice' value, suitable (\emph{e.g.}) for generating axis labels, before
      use.  To avoid this `nicing' you should set an explicit format for
      the axis using the \htmlattref{Format(axis)}{Format(axis)}~ or
      \htmlattref{Digits/Digits(axis)}{Digits/Digits(axis)}~
      attribute.  The default behaviour is for the
      application to generate its own \att{Gap} value when required, based on
      the range of axis values to be represented.
   }
   \sstattributetype{
      Floating point.
   }
   \sstnotes{
      \sstitemlist{

         \sstitem
         The \att{Gap} value should use the same units as are used internally
         for storing co-ordinate values on the corresponding axis.  For
         example, with a celestial co-ordinate system, the \att{Gap} value
         should be in radians, not hours or degrees.

         \sstitem
         If no axis is specified, (\emph{e.g.} \att{Gap} instead of \att{Gap(2)}),
         then both axes are affected.
      }
   }
}
\sstattribute{
   Grid
}{
   Draw grid lines?
}{
   \sstdescription{
      This attribute controls the appearance of annotated axes by
      determining whether grid lines (a grid of curves marking the
      `major' values on each axis) are drawn across the plotting area.

      If the \att{Grid} attribute is non-zero, then grid lines will be
      drawn.  Otherwise, short tick marks on the axes are used to mark
      the major axis values.  The default behaviour is to use tick
      marks if the entire plotting area is filled by valid
      co-ordinates, but to draw grid lines otherwise.
   }
   \sstattributetype{
      Integer (boolean).
   }
   \sstnotes{
      \sstitemlist{

         \sstitem
         The spacing between major axis values, which determines the
         spacing of grid lines, may be set using the
         \htmlattref{Gap(axis)}{Gap(axis)}~ attribute.
      }
   }
}
\sstattribute{
   LabelAt(axis)
}{
   Where to place numerical labels for a Plot
}{
   \sstdescription{
      This attribute controls the appearance of annotated axes by determining
      where numerical axis labels and associated tick marks are
      placed.  It takes a separate value for each physical axis
      so that, for instance, the setting \texttt{"LabelAt(2)=10.0"}~
      specifies where the numerical labels and tick marks for the
      second axis should be drawn.

      For each axis, the \att{LabelAt} value gives the value on the other
      axis at which numerical labels and tick marks should be placed.
      For example, in a celestial (right ascension, declination) co-ordinate
      system, \att{LabelAt(1)} gives a declination value that defines a line (of
      constant declination) along which the numerical right-ascension labels and their
      associated tick marks will be drawn.  Similarly, \att{LabelAt(2)} gives
      the right-ascension value at which the declination labels and ticks will be drawn.

      The default bahaviour is for the application to generate its own
      position for numerical labels and tick marks.
   }
   \sstattributetype{
      Floating point.
   }
   \sstnotes{
      \sstitemlist{

         \sstitem
         The \att{LabelAt} value should use the same units as are used
         internally for storing co-ordinate values on the appropriate
         axis.  For example, with a celestial co-ordinate system, the
         \att{LabelAt} value should be in radians, not hours or degrees.

         \sstitem
         Normally, the \att{LabelAt} value also determines where the lines
         representing co-ordinate axes will be drawn, so that the tick
         marks will lie on these lines (but also see the
         \htmlattref{DrawAxes}{DrawAxes}~ attribute).

         \sstitem
         In some circumstances, numerical labels and tick marks are
         drawn around the edges of the plotting area (see the
         \htmlattref{Labelling}{Labelling}~ attribute).  In this case,
         the value of the \att{LabelAt} attribute is ignored.
      }
   }
}
\sstattribute{
   LabelUnits(axis)
}{
   Include unit descriptions in axis labels?
}{
   \sstdescription{
      This attribute controls the appearance of annotated axes by determining
      whether the descriptive labels drawn for each axis
      should include a description of the units being used on the
      axis.  It takes a separate value for each physical axis
      so that, for instance, the setting \texttt{"LabelUnits(2)=1"}~
      specifies that a unit description should be included in the
      label for the second axis.

      If the \att{LabelUnits} value axis is non-zero, a unit
      description will be included in the descriptive label for that
      axis, otherwise it will be omitted.  The default behaviour is to
      include a unit description unless the
      \xref{Frame}{sun210}{Frame}~ being annotated
      is a \xref{SkyFrame}{sun210}{SkyFrame}~ (\emph{i.e.} a celestial
      co-ordinate system), in which case it is omitted.
   }
   \sstattributetype{
      Integer (boolean).
   }
   \sstnotes{
      \sstitemlist{

         \sstitem
         The text used for the unit description is obtained from
         \htmlattref{Unit(axis)}{Unit(axis)}~ attribute.

         \sstitem
         If no axis is specified, (\emph{e.g.} \att{LabelUnits} instead of
         \att{LabelUnits(2)}), then both axes are affected.
      }
   }
}
\sstattribute{
   LabelUp(axis)
}{
   Draw numerical axis labels upright?
}{
   \sstdescription{
      This attribute controls the appearance of annotated axes by determining
      whether the numerical labels for each axis should be
      drawn upright or not.  It takes a separate value for each
      physical axis so that, for instance, the setting
      \texttt{"LabelUp(2)=1"}~ specifies that numerical labels for the second
      axis should be drawn upright.

      If the \att{LabelUp} value axis is non-zero, it causes
      numerical labels for that axis to be plotted upright (\emph{i.e.} as
      normal, horizontal text), otherwise (the default) these labels
      rotate to follow the axis to which they apply.
   }
   \sstattributetype{
      Integer (boolean).
   }
   \sstnotes{
      \sstitemlist{

         \sstitem
         In some circumstances, numerical labels and tick marks are
         drawn around the edges of the plotting area (see the
         \htmlattref{Labelling}{Labelling}~ attribute).  In this case,
         the value of the \att{LabelUp} attribute is ignored.

         \sstitem
         If no axis is specified, (\emph{e.g.} \att{LabelUp} instead of
         \att{LabelUp(2)}), then both axes are affected.
      }
   }
}
\sstattribute{
   Labelling
}{
   Label and tick placement option
}{
   \sstdescription{
      This attribute controls the appearance of annotated axes by determining
      the strategy for placing numerical labels and tick marks.

      If the \att{Labelling} value is \texttt{"exterior"}~ (the default), then
      numerical labels and their associated tick marks are placed
      around the edges of the plotting area, if possible.  If this is
      not possible, or if the \att{Labelling}~ value is \texttt{"interior"}, then they
      are placed along grid lines inside the plotting area.
   }
   \sstattributetype{
      String.
   }
   \sstnotes{
      \sstitemlist{

         \sstitem
         The \att{LabelAt(axis)} attribute may be used to determine the exact
         placement of labels and tick marks that are drawn inside the
         plotting area.
      }
   }
}
\sstattribute{
   MajTickLen
}{
   Length of major tick marks
}{
   \sstdescription{
      This attribute controls the appearance of annotated axes by determining
      the length of the major tick marks drawn on the axes.

      The \att{MajTickLen}~ value should be given as a fraction of the
      minimum dimension of the plotting area.  Negative values cause
      major tick marks to be placed on the outside of the
      corresponding grid line or axis (but subject to any clipping
      imposed by \PGPLOT\ ), while positive
      values cause them to be placed on the inside.

      The default behaviour depends on whether a co-ordinate grid is
      drawn inside the plotting area (see the \htmlattref{Grid}{Grid}~
      attribute).  If so, the default \att{MajTickLen} value is zero (so that
      major ticks are not drawn), otherwise the default is $+0.015$.
   }
   \sstattributetype{
      Floating point.
   }
}
\sstattribute{
   MinTick(axis)
}{
   Density of minor tick marks
}{
   \sstdescription{
      This attribute controls the appearance of annotated axes by determining
      the density of minor tick marks which appear between the major
      axis values.  It takes a separate value for each
      physical axis so that, for instance, the setting
      \texttt{"MinTick(2)=2"}~ specifies the density of minor tick marks along
      the second axis.

      The value supplied should be the number of minor divisions
      required between each pair of major axis values, this being one
      more than the number of minor tick marks to be drawn.  By
      default, a value is chosen that depends on the gap between major
      axis values and the nature of the axis.
   }
   \sstattributetype{
      Integer.
   }
   \sstnotes{
      \sstitemlist{

         \sstitem
         If no axis is specified, (\emph{e.g.} \att{MinTick} instead of
         \att{MinTick(2)}), then both axes are affected.
      }
   }
}
\sstattribute{
   MinTickLen
}{
   Length of minor tick marks
}{
   \sstdescription{
      This attribute controls the appearance of annotated axes by determining
      the length of the minor tick marks drawn on the axes.

      The \att{MinTickLen}~ value should be given as a fraction of the
      minimum dimension of the plotting area.  Negative values cause
      minor tick marks to be placed on the outside of the
      corresponding grid line or axis (but subject to any clipping
      imposed by the underlying graphics system), while positive
      values cause them to be placed on the inside.

      The default value is $+0.007$.
   }
   \sstattributetype{
      Floating point.
   }
   \sstnotes{
      \sstitemlist{

         \sstitem
         The number of minor tick marks drawn is determined by the
         \att{MinTick(axis)} attribute.
      }
   }
}
\sstattribute{
   NumLab(axis)
}{
   Draw numerical axis labels?
}{
   \sstdescription{
      This attribute controls the appearance of annotated axes by
      determining whether labels should be drawn to represent the
      numerical values along each axis.  It takes a separate value for
      each physical axis so that, for instance, the setting \texttt{
      "NumLab(2)=1"}~ specifies that numerical labels should be drawn for
      the second axis.

      If the \att{NumLab} value for an axis is non-zero (the default),
      then numerical labels will be drawn for that axis, otherwise
      they will be omitted.
   }
   \sstattributetype{
      Integer (boolean).
   }
   \sstnotes{
      \sstitemlist{

         \sstitem
         The drawing of associated descriptive axis labels
         (describing the quantity being plotted along each axis) is
         controlled by the \htmlattref{TextLab(axis)}{TextLab(axis)}~ attribute.

         \sstitem
         If no axis is specified, (\emph{e.g.} \att{NumLab} instead of
         \att{NumLab(2)}), then both axes are affected.
      }
   }
}
\sstattribute{
   NumLabGap(axis)
}{
   Spacing of numerical labels
}{
   \sstdescription{
      This attribute controls the appearance of annotated axes by determining
      where numerical axis labels are placed relative to the axes they
      describe.  It takes a separate value for each physical axis so that,
      for instance, the setting \texttt{"NumLabGap(2)=-0.01"}~
      specifies where the numerical label for the second axis should
      be drawn.

      For each axis, the \att{NumLabGap} value gives the spacing between the
      axis line (or edge of the plotting area, if appropriate) and the
      nearest edge of the corresponding numerical axis
      labels.  Positive values cause the descriptive label to be placed
      on the opposite side of the line to the default tick marks,
      while negative values cause it to be placed on the same side.

      The \att{NumLabGap} value should be given as a fraction of the minimum
      dimension of the plotting area, the default value being $+0.01$.
   }
   \sstattributetype{
      Floating point.
   }
   \sstnotes{
      \sstitemlist{

         \sstitem
         If no axis is specified, (\emph{e.g.} \att{NumLabGap} instead of
         \att{NumLabGap(2)}), then both axes are affected.
      }
   }
}
\sstattribute{
   Size(element)
}{
   Character size for a plot element
}{
   \sstdescription{
      This attribute determines the character size used when drawing
      each element of graphical output.  It takes a
      separate value for each graphical element so that, for instance,
      the setting \texttt{"Size(title)=2.0"}~ causes the plot title to be drawn
      using twice the default character size.
   }
   \sstattributetype{
      Floating Point.
   }
   \sstnotes{
      \sstitemlist{

         \sstitem
         For a list of the graphical elements available, see
         \slhyperref{here}{Section~}{}{se:styles}.

         \sstitem
         If no graphical element is specified, (\emph{e.g.} \att{Size} instead
         of \att{Size(title)}), then all graphical elements are affected.
      }
   }
}
\sstattribute{
   Style(element)
}{
   Line style for a plot element
}{
   \sstdescription{
      This attribute determines the line style used when drawing each
      element of graphical output.  It takes a
      separate value for each graphical element so that, for instance,
      the setting \texttt{"Style(border)=2"}~ causes the border to be drawn
      using line Style~2 (which results in a dashed line).

      The range of integer line styles available and their appearance
      is determined by the \PGPLOT\  graphics system, but includes the
      following:

      \begin{enumerate}
      \item Full line (the default)
      \item Dashed
      \item Dot-dash-dot-dash
      \item Dotted
      \item Dash-dot-dot-dot
      \end{enumerate}

   }
   \sstattributetype{
      Integer.
   }
   \sstnotes{
      \sstitemlist{

         \sstitem
         For a list of the graphical elements available, see
         \slhyperref{here}{Section~}{}{se:styles}.

         \sstitem
         If no graphical element is specified, (\emph{e.g.} \att{Style} instead
         of \att{Style(border)}), then all graphical elements are affected.
      }
   }
}
\sstattribute{
   TextBackColour
}{
   The background colour to use when drawing text
}{
   \sstdescription{
      This attribute determines the background colour index to use when
      drawing textual strings---the foreground colour is determined by
      the \htmlattref{Colour(strings)}{Colour(element)}~ attribute.  Standard
      X colour names can also be specified, and the nearest colour will be used
      if the requested colour is not currently in the palette.

      If the value \texttt{"-1"} or \texttt{"clear"} is given, the background
      will be transparent.
   }
   \sstattributetype{
      Integer.
   }
}
\sstattribute{
   TextLab(axis)
}{
   Draw descriptive axis labels?
}{
   \sstdescription{
      This attribute controls the appearance of annotated axes by determining
      whether textual labels should be drawn to describe the quantity
      being represented on each axis.  It takes a separate
      value for each physical axis so that, for instance,
      the setting \texttt{"TextLab(2)=1"}~ specifies that descriptive labels
      should be drawn for the second axis.

      If the \att{TextLab} value of an axis is non-zero, then
      descriptive labels will be drawn for that axis, otherwise they
      will be omitted.  The default behaviour is to draw descriptive
      labels if tick marks and numerical labels are being drawn around
      the edges of the plotting area (see the
      \htmlattref{Labelling}{Labelling}~ attribute),
      but to omit them otherwise.
   }
   \sstattributetype{
      Integer (boolean).
   }
   \sstnotes{
      \sstitemlist{

         \sstitem
         The text used for the descriptive labels is derived from the
         \htmlattref{Label(axis)}{Label(axis)}~ attribute, together with its
         \htmlattref{Unit(axis)}{Unit(axis)}~ attribute if appropriate
         (see the \htmlattref{LabelUnits(axis)}{LabelUnits(axis)}~ attribute).

         \sstitem
         The drawing of numerical axis labels (which indicate values on the
         axis) is controlled by the \htmlattref{NumLab(axis)}{NumLab(axis)}~ attribute.

         \sstitem
         If no axis is specified, (\emph{e.g.} \att{TextLab} instead of
         \att{TextLab(2)}), then both axes are affected.
      }
   }
}
\sstattribute{
   TextLabGap(axis)
}{
   Spacing of descriptive axis labels
}{
   \sstdescription{
      This attribute controls the appearance of annotated axes by determining
      where descriptive axis labels are placed relative to the axes they
      describe.  It takes a separate value for each physical axis
      so that, for instance, the setting \texttt{"TextLabGap(2)=0.01"}~
      specifies where the descriptive label for the second axis should
      be drawn.

      For each axis, the \att{TextLabGap} value
      gives the spacing between the descriptive label and the edge of
      a box enclosing all other parts of the annotated grid (excluding
      other descriptive labels).  The gap is measured to the nearest
      edge of the label (\emph{i.e.} the top or the bottom).  Positive
      values cause the descriptive label to be placed outside the
      bounding box, while negative values cause it to be placed
      inside.

      The \att{TextLabGap} value should be given as a fraction of the minimum
      dimension of the plotting area, the default value being $+0.01$.
   }
   \sstattributetype{
      Floating point.
   }
   \sstnotes{
      \sstitemlist{

         \sstitem
         If drawn, descriptive labels are always placed at the edges of
         the plotting area, even although the corresponding numerical
         labels may be drawn along axis lines in the interior of the
         plotting area (see the \htmlattref{Labelling}{Labelling}~ attribute).

         \sstitem
         If no axis is specified, (\emph{e.g.} \att{TextLabGap} instead of
         \att{TextLabGap(2)}), then both axes are affected.
      }
   }
}
\sstattribute{
   TextMargin
}{
   The width of the margin to clear when drawing text
}{
   \sstdescription{
      This attribute determines the width of the margin that is cleared
      around the edges of each drawn text string, in units of the text
      height. That is, if TEXTMARGIN is set to 0.5, the margin will be
      half the height of the text. The default is zero.
   }
   \sstattributetype{
      Floating point.
   }
}
\sstattribute{
   TickAll
}{
   Draw tick marks on all edges?
}{
   \sstdescription{
      This attribute controls the appearance of annotated axes by determining
      whether tick marks should be drawn on all edges of the plot.

      If the \att{TickAll}~ value is non-zero (the default), then
      tick marks will be drawn on all edges of the plot.  Otherwise,
      they will be drawn only on those edges where the numerical and
      descriptive axis labels are drawn (see the
      \htmlattref{Edge(axis)}{Edge(axis)}~ attribute).
   }
   \sstattributetype{
      Integer (boolean).
   }
   \sstnotes{
      \sstitemlist{

         \sstitem
         In some circumstances, numerical labels and tick marks are
         drawn along grid lines inside the plotting area, rather than
         around its edges (see the \htmlattref{Labelling}{Labelling}~ attribute).
         In this case, the value of the \att{TickAll} attribute is ignored.
      }
   }
}
\sstattribute{
   TitleGap
}{
   Vertical spacing for the title
}{
   \sstdescription{
      This attribute controls the appearance of annotated axes by determining
      where the title is drawn.

      Its value gives the spacing between the bottom edge of the title
      and the top edge of a bounding box containing all the other parts
      of the annotated grid.  Positive values cause the title to be
      drawn outside the box, while negative values cause it to be drawn
      inside.

      The \att{TitleGap} value should be given as a fraction of the minimum
      dimension of the plotting area, the default value being $+0.05$.
   }
   \sstattributetype{
      Floating point.
   }
   \sstnotes{
      \sstitemlist{

         \sstitem
         The text used for the title is obtained from the
         \htmlattref{Title}{Title}~ attribute.
      }
   }
}
\sstattribute{
   Tol
}{
   Plotting tolerance
}{
   \sstdescription{
      This attribute specifies the plotting tolerance (or resolution) to
      be used for curves.  Smaller values will result in smoother and more
      accurate curves being drawn (particularly in the presence of
      discontinuities in the mapping between graphics co-ordinates and
      physical co-ordinates), but may slow down the plotting process.
      Conversely, larger values may speed up the plotting process in
      cases where high resolution is not required.

      The \att{Tol} value should be given as a fraction of the minimum
      dimension of the plotting area, and should lie in the range
      from 1.0E$-$7 to 1.0.  By default, a value of 0.001 is used.
   }
   \sstattributetype{
      Floating point.
   }
}
\sstattribute{
   Width(element)
}{
   Line width for a plot element
}{
   \sstdescription{
      This attribute determines the line width used when drawing each
      element of graphical output.  It takes a
      separate value for each graphical element so that, for instance,
      the setting \texttt{"Width(border)=2.0"}~ causes the border to be
      drawn using a line width of 2.0.  A value of 1.0 results in a
      line thickness that is approximately 0.0005 times the length of
      the diagonal of the entire display surface.
   }
   \sstattributetype{
      Floating point.
   }
   \sstnotes{
      \sstitemlist{

         \sstitem
         For a list of the graphical elements available, see
         \slhyperref{here}{Section~}{}{se:styles}.

         \sstitem
         If no graphical element is specified, (\emph{e.g.} \att{Width} instead
         of \att{Width(title)}), then all graphical elements are affected.

      }
   }
}
\sstmaintoc

\newpage
\section{\xlabel{ap_colset}Standard Named Colours\label{ap:colset}}
The standard set of named colours recognised by \KAPPA\ is
tabulated below together with their red, green, and blue relative
intensities.  It is the X-windows standard colour set so don't blame
\KAPPA\ if you think some of them are anomalous.  In addition to
those tabulated, there are grey levels at each percentage between
``Black'' and ``White''.  These are called ``Grey1'', ``Grey2'', \dots,
``Grey99''.  All the names containing ``Grey'' have synonyms spelt with
``Gray''.  \medskip

\begin{center}
\begin{tabular}{|l|l|l|l|l|l|l|l|}
\hline
\multicolumn{8}{|c|}{{\large Standard Colour Set}} \\ \hline
\multicolumn{1}{|c|}{Name} & \multicolumn{1}{|c|}{R} & \multicolumn{1}{c|}{G} &
\multicolumn{1}{c|}{B} & \multicolumn{1}{|c|}{Name} & \multicolumn{1}{c|}{R} &
\multicolumn{1}{c|}{G} & \multicolumn{1}{c|}{B}  \\ \hline
AliceBlue           & 0.941 & 0.973 & 1.000 & AntiqueWhite        & 0.980 & 0.922 & 0.843 \\
AntiqueWhite1       & 1.000 & 0.937 & 0.859 & AntiqueWhite2       & 0.933 & 0.875 & 0.800 \\
AntiqueWhite3       & 0.804 & 0.753 & 0.690 & AntiqueWhite4       & 0.545 & 0.514 & 0.471 \\
Aquamarine          & 0.498 & 1.000 & 0.831 & Aquamarine1         & 0.498 & 1.000 & 0.831 \\
Aquamarine2         & 0.463 & 0.933 & 0.776 & Aquamarine3         & 0.400 & 0.804 & 0.667 \\
Aquamarine4         & 0.271 & 0.545 & 0.455 & Azure               & 0.941 & 1.000 & 1.000 \\
Azure1              & 0.941 & 1.000 & 1.000 & Azure2              & 0.878 & 0.933 & 0.933 \\
Azure3              & 0.757 & 0.804 & 0.804 & Azure4              & 0.514 & 0.545 & 0.545 \\
Beige               & 0.961 & 0.961 & 0.863 & Bisque              & 1.000 & 0.894 & 0.769 \\
Bisque1             & 1.000 & 0.894 & 0.769 & Bisque2             & 0.933 & 0.835 & 0.718 \\
Bisque3             & 0.804 & 0.718 & 0.620 & Bisque4             & 0.545 & 0.490 & 0.420 \\
Black               & 0.000 & 0.000 & 0.000 & BlanchedAlmond      & 1.000 & 0.922 & 0.804 \\
Blue                & 0.000 & 0.000 & 1.000 & Blue1               & 0.000 & 0.000 & 1.000 \\
Blue2               & 0.000 & 0.000 & 0.933 & Blue3               & 0.000 & 0.000 & 0.804 \\
Blue4               & 0.000 & 0.000 & 0.545 & BlueViolet          & 0.541 & 0.169 & 0.886 \\
Brown               & 0.647 & 0.165 & 0.165 & Brown1              & 1.000 & 0.251 & 0.251 \\
Brown2              & 0.933 & 0.231 & 0.231 & Brown3              & 0.804 & 0.200 & 0.200 \\
Brown4              & 0.545 & 0.137 & 0.137 & Burlywood           & 0.871 & 0.722 & 0.529 \\
Burlywood1          & 1.000 & 0.827 & 0.608 & Burlywood2          & 0.933 & 0.773 & 0.569 \\
Burlywood3          & 0.804 & 0.667 & 0.490 & Burlywood4          & 0.545 & 0.451 & 0.333 \\
CadetBlue           & 0.373 & 0.620 & 0.627 & CadetBlue1          & 0.596 & 0.961 & 1.000 \\
CadetBlue2          & 0.557 & 0.898 & 0.933 & CadetBlue3          & 0.478 & 0.773 & 0.804 \\
CadetBlue4          & 0.325 & 0.525 & 0.545 & Chartreuse          & 0.498 & 1.000 & 0.000 \\
Chartreuse1         & 0.498 & 1.000 & 0.000 & Chartreuse2         & 0.463 & 0.933 & 0.000 \\
Chartreuse3         & 0.400 & 0.804 & 0.000 & Chartreuse4         & 0.271 & 0.545 & 0.000 \\
Chocolate           & 0.824 & 0.412 & 0.118 & Chocolate1          & 1.000 & 0.498 & 0.141 \\
Chocolate2          & 0.933 & 0.463 & 0.129 & Chocolate3          & 0.804 & 0.400 & 0.114 \\
Chocolate4          & 0.545 & 0.271 & 0.075 & Coral               & 1.000 & 0.498 & 0.314 \\
Coral1              & 1.000 & 0.447 & 0.337 & Coral2              & 0.933 & 0.416 & 0.314 \\
Coral3              & 0.804 & 0.357 & 0.271 & Coral4              & 0.545 & 0.243 & 0.184 \\
CornflowerBlue      & 0.392 & 0.584 & 0.929 & Cornsilk            & 1.000 & 0.973 & 0.863 \\
Cornsilk1           & 1.000 & 0.973 & 0.863 & Cornsilk2           & 0.933 & 0.910 & 0.804 \\
Cornsilk3           & 0.804 & 0.784 & 0.694 & Cornsilk4           & 0.545 & 0.533 & 0.471 \\
Cyan                & 0.000 & 1.000 & 1.000 & Cyan1               & 0.000 & 1.000 & 1.000 \\
Cyan2               & 0.000 & 0.933 & 0.933 & Cyan3               & 0.000 & 0.804 & 0.804 \\
Cyan4               & 0.000 & 0.545 & 0.545 & DarkGoldenrod       & 0.722 & 0.525 & 0.043 \\
DarkGoldenrod1      & 1.000 & 0.725 & 0.059 & DarkGoldenrod2      & 0.933 & 0.678 & 0.055 \\
\hline
\end{tabular}
\end{center}

\begin{center}
\begin{tabular}{|l|l|l|l|l|l|l|l|}
\hline
\multicolumn{8}{|c|}{{\large Standard Colour Set}} \\ \hline
\multicolumn{1}{|c|}{Name} & \multicolumn{1}{|c|}{R} & \multicolumn{1}{c|}{G} &
\multicolumn{1}{c|}{B} & \multicolumn{1}{|c|}{Name} & \multicolumn{1}{c|}{R} &
\multicolumn{1}{c|}{G} & \multicolumn{1}{c|}{B}  \\ \hline
DarkGoldenrod3      & 0.804 & 0.584 & 0.047 & DarkGoldenrod4      & 0.545 & 0.396 & 0.031 \\
DarkGreen           & 0.000 & 0.392 & 0.000 & DarkKhaki           & 0.741 & 0.718 & 0.420 \\
DarkOliveGreen      & 0.333 & 0.420 & 0.184 & DarkOliveGreen1     & 0.792 & 1.000 & 0.439 \\
DarkOliveGreen2     & 0.737 & 0.933 & 0.408 & DarkOliveGreen3     & 0.635 & 0.804 & 0.353 \\
DarkOliveGreen4     & 0.431 & 0.545 & 0.239 & DarkOrange          & 1.000 & 0.549 & 0.000 \\
DarkOrange1         & 1.000 & 0.498 & 0.000 & DarkOrange2         & 0.933 & 0.463 & 0.000 \\
DarkOrange3         & 0.804 & 0.400 & 0.000 & DarkOrange4         & 0.545 & 0.271 & 0.000 \\
DarkOrchid          & 0.600 & 0.196 & 0.800 & DarkOrchid1         & 0.749 & 0.243 & 1.000 \\
DarkOrchid2         & 0.698 & 0.227 & 0.933 & DarkOrchid3         & 0.604 & 0.196 & 0.804 \\
DarkOrchid4         & 0.408 & 0.133 & 0.545 & DarkSalmon          & 0.914 & 0.588 & 0.478 \\
DarkSeaGreen        & 0.561 & 0.737 & 0.561 & DarkSeaGreen1       & 0.757 & 1.000 & 0.757 \\
DarkSeaGreen2       & 0.706 & 0.933 & 0.706 & DarkSeaGreen3       & 0.608 & 0.804 & 0.608 \\
DarkSeaGreen4       & 0.412 & 0.545 & 0.412 & DarkSlateBlue       & 0.282 & 0.239 & 0.545 \\
DarkSlateGrey       & 0.184 & 0.310 & 0.310 & DarkSlateGrey1      & 0.592 & 1.000 & 1.000 \\
DarkSlateGrey2      & 0.553 & 0.933 & 0.933 & DarkSlateGrey3      & 0.475 & 0.804 & 0.804 \\
DarkSlateGrey4      & 0.322 & 0.545 & 0.545 & DarkTurquoise       & 0.000 & 0.808 & 0.820 \\
DarkViolet          & 0.580 & 0.000 & 0.827 & DeepPink            & 1.000 & 0.078 & 0.576 \\
DeepPink1           & 1.000 & 0.078 & 0.576 & DeepPink2           & 0.933 & 0.071 & 0.537 \\
DeepPink3           & 0.804 & 0.063 & 0.463 & DeepPink4           & 0.545 & 0.039 & 0.314 \\
DeepSkyBlue         & 0.000 & 0.749 & 1.000 & DeepSkyBlue1        & 0.000 & 0.749 & 1.000 \\
DeepSkyBlue2        & 0.000 & 0.698 & 0.933 & DeepSkyBlue3        & 0.000 & 0.604 & 0.804 \\
DeepSkyBlue4        & 0.000 & 0.408 & 0.545 & DimGrey             & 0.412 & 0.412 & 0.412 \\
DodgerBlue          & 0.118 & 0.565 & 1.000 & DodgerBlue1         & 0.118 & 0.565 & 1.000 \\
DodgerBlue2         & 0.110 & 0.525 & 0.933 & DodgerBlue3         & 0.094 & 0.455 & 0.804 \\
DodgerBlue4         & 0.063 & 0.306 & 0.545 & Firebrick           & 0.698 & 0.133 & 0.133 \\
Firebrick1          & 1.000 & 0.188 & 0.188 & Firebrick2          & 0.933 & 0.173 & 0.173 \\
Firebrick3          & 0.804 & 0.149 & 0.149 & Firebrick4          & 0.545 & 0.102 & 0.102 \\
FloralWhite         & 1.000 & 0.980 & 0.941 & ForestGreen         & 0.133 & 0.545 & 0.133 \\
Gainsboro           & 0.863 & 0.863 & 0.863 & GhostWhite          & 0.973 & 0.973 & 1.000 \\
Gold                & 1.000 & 0.843 & 0.000 & Gold1               & 1.000 & 0.843 & 0.000 \\
Gold2               & 0.933 & 0.788 & 0.000 & Gold3               & 0.804 & 0.678 & 0.000 \\
Gold4               & 0.545 & 0.459 & 0.000 & Goldenrod           & 0.855 & 0.647 & 0.125 \\
Goldenrod1          & 1.000 & 0.757 & 0.145 & Goldenrod2          & 0.933 & 0.706 & 0.133 \\
Goldenrod3          & 0.804 & 0.608 & 0.114 & Goldenrod4          & 0.545 & 0.412 & 0.078 \\
Green               & 0.000 & 1.000 & 0.000 & Green1              & 0.000 & 1.000 & 0.000 \\
Green2              & 0.000 & 0.933 & 0.000 & Green3              & 0.000 & 0.804 & 0.000 \\
Green4              & 0.000 & 0.545 & 0.000 & GreenYellow         & 0.678 & 1.000 & 0.184 \\
Grey                & 0.753 & 0.753 & 0.753 & Honeydew            & 0.941 & 1.000 & 0.941 \\
Honeydew1           & 0.941 & 1.000 & 0.941 & Honeydew2           & 0.878 & 0.933 & 0.878 \\
Honeydew3           & 0.757 & 0.804 & 0.757 & Honeydew4           & 0.514 & 0.545 & 0.514 \\
HotPink             & 1.000 & 0.412 & 0.706 & HotPink1            & 1.000 & 0.431 & 0.706 \\
HotPink2            & 0.933 & 0.416 & 0.655 & HotPink3            & 0.804 & 0.376 & 0.565 \\
HotPink4            & 0.545 & 0.227 & 0.384 & IndianRed           & 0.804 & 0.361 & 0.361 \\
IndianRed1          & 1.000 & 0.416 & 0.416 & IndianRed2          & 0.933 & 0.388 & 0.388 \\
IndianRed3          & 0.804 & 0.333 & 0.333 & IndianRed4          & 0.545 & 0.227 & 0.227 \\
Ivory               & 1.000 & 1.000 & 0.941 & Ivory2              & 0.933 & 0.933 & 0.878 \\
Ivory3              & 0.804 & 0.804 & 0.757 & Ivory4              & 0.545 & 0.545 & 0.514 \\
\hline
\end{tabular}
\end{center}

\begin{center}
\begin{tabular}{|l|l|l|l|l|l|l|l|}
\hline
\multicolumn{8}{|c|}{{\large Standard Colour Set}} \\ \hline
\multicolumn{1}{|c|}{Name} & \multicolumn{1}{|c|}{R} & \multicolumn{1}{c|}{G} &
\multicolumn{1}{c|}{B} & \multicolumn{1}{|c|}{Name} & \multicolumn{1}{c|}{R} &
\multicolumn{1}{c|}{G} & \multicolumn{1}{c|}{B}  \\ \hline
Khaki               & 0.941 & 0.902 & 0.549 & Khaki1              & 1.000 & 0.965 & 0.561 \\
Khaki2              & 0.933 & 0.902 & 0.522 & Khaki3              & 0.804 & 0.776 & 0.451 \\
Khaki4              & 0.545 & 0.525 & 0.306 & Lavender            & 0.902 & 0.902 & 0.980 \\
LavenderBlush       & 1.000 & 0.941 & 0.961 & LavenderBlush1      & 1.000 & 0.941 & 0.961 \\
LavenderBlush2      & 0.933 & 0.878 & 0.898 & LavenderBlush3      & 0.804 & 0.757 & 0.773 \\
LavenderBlush4      & 0.545 & 0.514 & 0.525 & LawnGreen           & 0.486 & 0.988 & 0.000 \\
LemonChiffon        & 1.000 & 0.980 & 0.804 & LemonChiffon1       & 1.000 & 0.980 & 0.804 \\
LemonChiffon2       & 0.933 & 0.914 & 0.749 & LemonChiffon3       & 0.804 & 0.788 & 0.647 \\
LemonChiffon4       & 0.545 & 0.537 & 0.439 & LightBlue           & 0.678 & 0.847 & 0.902 \\
LightBlue1          & 0.749 & 0.937 & 1.000 & LightBlue2          & 0.698 & 0.875 & 0.933 \\
LightBlue3          & 0.604 & 0.753 & 0.804 & LightBlue4          & 0.408 & 0.514 & 0.545 \\
LightCoral          & 0.941 & 0.502 & 0.502 & LightCyan           & 0.878 & 1.000 & 1.000 \\
LightCyan1          & 0.878 & 1.000 & 1.000 & LightCyan2          & 0.820 & 0.933 & 0.933 \\
LightCyan3          & 0.706 & 0.804 & 0.804 & LightCyan4          & 0.478 & 0.545 & 0.545 \\
LightGoldenrod      & 0.933 & 0.867 & 0.510 & LightGoldenrod1     & 1.000 & 0.925 & 0.545 \\
LightGoldenrod2     & 0.933 & 0.863 & 0.510 & LightGoldenrod3     & 0.804 & 0.745 & 0.439 \\
LightGoldenrod4     & 0.545 & 0.506 & 0.298 & LightGoldenrodYellow& 0.980 & 0.980 & 0.824 \\
LightGrey           & 0.827 & 0.827 & 0.827 & LightPink           & 1.000 & 0.714 & 0.757 \\
LightPink1          & 1.000 & 0.682 & 0.725 & LightPink2          & 0.933 & 0.635 & 0.678 \\
LightPink3          & 0.804 & 0.549 & 0.584 & LightPink4          & 0.545 & 0.373 & 0.396 \\
LightSalmon         & 1.000 & 0.627 & 0.478 & LightSalmon1        & 1.000 & 0.627 & 0.478 \\
LightSalmon2        & 0.933 & 0.584 & 0.447 & LightSalmon3        & 0.804 & 0.506 & 0.384 \\
LightSalmon4        & 0.545 & 0.341 & 0.259 & LightSeaGreen       & 0.125 & 0.698 & 0.667 \\
LightSkyBlue        & 0.529 & 0.808 & 0.980 & LightSkyBlue1       & 0.690 & 0.886 & 1.000 \\
LightSkyBlue2       & 0.643 & 0.827 & 0.933 & LightSkyBlue3       & 0.553 & 0.714 & 0.804 \\
LightSkyBlue4       & 0.376 & 0.482 & 0.545 & LightSlateBlue      & 0.518 & 0.439 & 1.000 \\
LightSlateGrey      & 0.467 & 0.533 & 0.600 & LightSteelBlue      & 0.690 & 0.769 & 0.871 \\
LightSteelBlue1     & 0.792 & 0.882 & 1.000 & LightSteelBlue2     & 0.737 & 0.824 & 0.933 \\
LightSteelBlue3     & 0.635 & 0.710 & 0.804 & LightSteelBlue4     & 0.431 & 0.482 & 0.545 \\
LightYellow         & 1.000 & 1.000 & 0.878 & LightYellow1        & 1.000 & 1.000 & 0.878 \\
LightYellow2        & 0.933 & 0.933 & 0.820 & LightYellow3        & 0.804 & 0.804 & 0.706 \\
LightYellow4        & 0.545 & 0.545 & 0.478 & LimeGreen           & 0.196 & 0.804 & 0.196 \\
Linen               & 0.980 & 0.941 & 0.902 & Magenta             & 1.000 & 0.000 & 1.000 \\
Magenta1            & 1.000 & 0.000 & 1.000 & Magenta2            & 0.933 & 0.000 & 0.933 \\
Magenta3            & 0.804 & 0.000 & 0.804 & Magenta4            & 0.545 & 0.000 & 0.545 \\
Maroon              & 0.690 & 0.188 & 0.376 & Maroon1             & 1.000 & 0.204 & 0.702 \\
Maroon2             & 0.933 & 0.188 & 0.655 & Maroon3             & 0.804 & 0.161 & 0.565 \\
Maroon4             & 0.545 & 0.110 & 0.384 & MediumAquamarine    & 0.400 & 0.804 & 0.667 \\
MediumBlue          & 0.000 & 0.000 & 0.804 & MediumOrchid        & 0.729 & 0.333 & 0.827 \\
MediumOrchid1       & 0.878 & 0.400 & 1.000 & MediumOrchid2       & 0.820 & 0.373 & 0.933 \\
MediumOrchid3       & 0.706 & 0.322 & 0.804 & MediumOrchid4       & 0.478 & 0.216 & 0.545 \\
MediumPurple        & 0.576 & 0.439 & 0.859 & MediumPurple1       & 0.671 & 0.510 & 1.000 \\
MediumPurple2       & 0.624 & 0.475 & 0.933 & MediumPurple3       & 0.537 & 0.408 & 0.804 \\
MediumPurple4       & 0.365 & 0.278 & 0.545 & MediumSeaGreen      & 0.235 & 0.702 & 0.443 \\
MediumSlateBlue     & 0.482 & 0.408 & 0.933 & MediumSpringGreen   & 0.000 & 0.980 & 0.604 \\
MediumTurquoise     & 0.282 & 0.820 & 0.800 & MediumVioletRed     & 0.780 & 0.082 & 0.522 \\
MidnightBlue        & 0.098 & 0.098 & 0.439 & MintCream           & 0.961 & 1.000 & 0.980 \\
\hline
\end{tabular}
\end{center}

\begin{center}
\begin{tabular}{|l|l|l|l|l|l|l|l|}
\hline
\multicolumn{8}{|c|}{{\large Standard Colour Set}} \\ \hline
\multicolumn{1}{|c|}{Name} & \multicolumn{1}{|c|}{R} & \multicolumn{1}{c|}{G} &
\multicolumn{1}{c|}{B} & \multicolumn{1}{|c|}{Name} & \multicolumn{1}{c|}{R} &
\multicolumn{1}{c|}{G} & \multicolumn{1}{c|}{B}  \\ \hline
MistyRose           & 1.000 & 0.894 & 0.882 & MistyRose1          & 1.000 & 0.894 & 0.882 \\
MistyRose2          & 0.933 & 0.835 & 0.824 & MistyRose3          & 0.804 & 0.718 & 0.710 \\
MistyRose4          & 0.545 & 0.490 & 0.482 & Moccasin            & 1.000 & 0.894 & 0.710 \\
NavajoWhite         & 1.000 & 0.871 & 0.678 & NavajoWhite1        & 1.000 & 0.871 & 0.678 \\
NavajoWhite2        & 0.933 & 0.812 & 0.631 & NavajoWhite3        & 0.804 & 0.702 & 0.545 \\
NavajoWhite4        & 0.545 & 0.475 & 0.369 & Navy                & 0.000 & 0.000 & 0.502 \\
NavyBlue            & 0.000 & 0.000 & 0.502 & OldLace             & 0.992 & 0.961 & 0.902 \\
OliveDrab           & 0.420 & 0.557 & 0.137 & OliveDrab1          & 0.753 & 1.000 & 0.243 \\
OliveDrab2          & 0.702 & 0.933 & 0.227 & OliveDrab3          & 0.604 & 0.804 & 0.196 \\
OliveDrab4          & 0.412 & 0.545 & 0.133 & Orange              & 1.000 & 0.647 & 0.000 \\
Orange1             & 1.000 & 0.647 & 0.000 & Orange2             & 0.933 & 0.604 & 0.000 \\
Orange3             & 0.804 & 0.522 & 0.000 & Orange4             & 0.545 & 0.353 & 0.000 \\
OrangeRed           & 1.000 & 0.271 & 0.000 & OrangeRed1          & 1.000 & 0.271 & 0.000 \\
OrangeRed2          & 0.933 & 0.251 & 0.000 & OrangeRed3          & 0.804 & 0.216 & 0.000 \\
OrangeRed4          & 0.545 & 0.145 & 0.000 & Orchid              & 0.855 & 0.439 & 0.839 \\
Orchid1             & 1.000 & 0.514 & 0.980 & Orchid2             & 0.933 & 0.478 & 0.914 \\
Orchid3             & 0.804 & 0.412 & 0.788 & Orchid4             & 0.545 & 0.278 & 0.537 \\
PaleGoldenrod       & 0.933 & 0.910 & 0.667 & PaleGreen           & 0.596 & 0.984 & 0.596 \\
PaleGreen1          & 0.604 & 1.000 & 0.604 & PaleGreen2          & 0.565 & 0.933 & 0.565 \\
PaleGreen3          & 0.486 & 0.804 & 0.486 & PaleGreen4          & 0.329 & 0.545 & 0.329 \\
PaleTurquoise       & 0.686 & 0.933 & 0.933 & PaleTurquoise1      & 0.733 & 1.000 & 1.000 \\
PaleTurquoise2      & 0.682 & 0.933 & 0.933 & PaleTurquoise3      & 0.588 & 0.804 & 0.804 \\
PaleTurquoise4      & 0.400 & 0.545 & 0.545 & PaleVioletRed       & 0.859 & 0.439 & 0.576 \\
PaleVioletRed1      & 1.000 & 0.510 & 0.671 & PaleVioletRed2      & 0.933 & 0.475 & 0.624 \\
PaleVioletRed3      & 0.804 & 0.408 & 0.537 & PaleVioletRed4      & 0.545 & 0.278 & 0.365 \\
PapayaWhip          & 1.000 & 0.937 & 0.835 & PeachPuff           & 1.000 & 0.855 & 0.725 \\
PeachPuff1          & 1.000 & 0.855 & 0.725 & PeachPuff2          & 0.933 & 0.796 & 0.678 \\
PeachPuff3          & 0.804 & 0.686 & 0.584 & PeachPuff4          & 0.545 & 0.467 & 0.396 \\
Peru                & 0.804 & 0.522 & 0.247 & Pink                & 1.000 & 0.753 & 0.796 \\
Pink1               & 1.000 & 0.710 & 0.773 & Pink2               & 0.933 & 0.663 & 0.722 \\
Pink3               & 0.804 & 0.569 & 0.620 & Pink4               & 0.545 & 0.388 & 0.424 \\
Plum                & 0.867 & 0.627 & 0.867 & Plum1               & 1.000 & 0.733 & 1.000 \\
Plum2               & 0.933 & 0.682 & 0.933 & Plum3               & 0.804 & 0.588 & 0.804 \\
Plum4               & 0.545 & 0.400 & 0.545 & PowderBlue          & 0.690 & 0.878 & 0.902 \\
Purple              & 0.627 & 0.125 & 0.941 & Purple1             & 0.608 & 0.188 & 1.000 \\
Purple2             & 0.569 & 0.173 & 0.933 & Purple3             & 0.490 & 0.149 & 0.804 \\
Purple4             & 0.333 & 0.102 & 0.545 & Red                 & 1.000 & 0.000 & 0.000 \\
Red1                & 1.000 & 0.000 & 0.000 & Red2                & 0.933 & 0.000 & 0.000 \\
Red3                & 0.804 & 0.000 & 0.000 & Red4                & 0.545 & 0.000 & 0.000 \\
RosyBrown           & 0.737 & 0.561 & 0.561 & RosyBrown1          & 1.000 & 0.757 & 0.757 \\
RosyBrown2          & 0.933 & 0.706 & 0.706 & RosyBrown3          & 0.804 & 0.608 & 0.608 \\
RosyBrown4          & 0.545 & 0.412 & 0.412 & RoyalBlue           & 0.255 & 0.412 & 0.882 \\
RoyalBlue1          & 0.282 & 0.463 & 1.000 & RoyalBlue2          & 0.263 & 0.431 & 0.933 \\
RoyalBlue3          & 0.227 & 0.373 & 0.804 & RoyalBlue4          & 0.153 & 0.251 & 0.545 \\
SaddleBrown         & 0.545 & 0.271 & 0.075 & Salmon              & 0.980 & 0.502 & 0.447 \\
Salmon1             & 1.000 & 0.549 & 0.412 & Salmon2             & 0.933 & 0.510 & 0.384 \\
Salmon3             & 0.804 & 0.439 & 0.329 & Salmon4             & 0.545 & 0.298 & 0.224 \\
\hline
\end{tabular}
\end{center}

\begin{center}
\begin{tabular}{|l|l|l|l|l|l|l|l|}
\hline
\multicolumn{8}{|c|}{{\large Standard Colour Set}} \\ \hline
\multicolumn{1}{|c|}{Name} & \multicolumn{1}{|c|}{R} & \multicolumn{1}{c|}{G} &
\multicolumn{1}{c|}{B} & \multicolumn{1}{|c|}{Name} & \multicolumn{1}{c|}{R} &
\multicolumn{1}{c|}{G} & \multicolumn{1}{c|}{B}  \\ \hline
SandyBrown          & 0.957 & 0.643 & 0.376 & SeaGreen            & 0.180 & 0.545 & 0.341 \\
SeaGreen1           & 0.329 & 1.000 & 0.624 & SeaGreen2           & 0.306 & 0.933 & 0.580 \\
SeaGreen3           & 0.263 & 0.804 & 0.502 & SeaGreen4           & 0.180 & 0.545 & 0.341 \\
Seashell            & 1.000 & 0.961 & 0.933 & Seashell1           & 1.000 & 0.961 & 0.933 \\
Seashell2           & 0.933 & 0.898 & 0.871 & Seashell3           & 0.804 & 0.773 & 0.749 \\
Seashell4           & 0.545 & 0.525 & 0.510 & Sienna              & 0.627 & 0.322 & 0.176 \\
Sienna1             & 1.000 & 0.510 & 0.278 & Sienna2             & 0.933 & 0.475 & 0.259 \\
Sienna3             & 0.804 & 0.408 & 0.224 & Sienna4             & 0.545 & 0.278 & 0.149 \\
SkyBlue             & 0.529 & 0.808 & 0.922 & SkyBlue1            & 0.529 & 0.808 & 1.000 \\
SkyBlue2            & 0.494 & 0.753 & 0.933 & SkyBlue3            & 0.424 & 0.651 & 0.804 \\
SkyBlue4            & 0.290 & 0.439 & 0.545 & SlateBlue           & 0.416 & 0.353 & 0.804 \\
SlateBlue1          & 0.514 & 0.435 & 1.000 & SlateBlue2          & 0.478 & 0.404 & 0.933 \\
SlateBlue3          & 0.412 & 0.349 & 0.804 & SlateBlue4          & 0.278 & 0.235 & 0.545 \\
SlateGrey           & 0.439 & 0.502 & 0.565 & SlateGrey1          & 0.776 & 0.886 & 1.000 \\
SlateGrey2          & 0.725 & 0.827 & 0.933 & SlateGrey3          & 0.624 & 0.714 & 0.804 \\
SlateGrey4          & 0.424 & 0.482 & 0.545 & Snow                & 1.000 & 0.980 & 0.980 \\
Snow1               & 1.000 & 0.980 & 0.980 & Snow2               & 0.933 & 0.914 & 0.914 \\
Snow3               & 0.804 & 0.788 & 0.788 & Snow4               & 0.545 & 0.537 & 0.537 \\
SpringGreen         & 0.000 & 1.000 & 0.498 & SpringGreen1        & 0.000 & 1.000 & 0.498 \\
SpringGreen2        & 0.000 & 0.933 & 0.463 & SpringGreen3        & 0.000 & 0.804 & 0.400 \\
SpringGreen4        & 0.000 & 0.545 & 0.271 & SteelBlue           & 0.275 & 0.510 & 0.706 \\
SteelBlue1          & 0.388 & 0.722 & 1.000 & SteelBlue2          & 0.361 & 0.675 & 0.933 \\
SteelBlue3          & 0.310 & 0.580 & 0.804 & SteelBlue4          & 0.212 & 0.392 & 0.545 \\
Tan                 & 0.824 & 0.706 & 0.549 & Tan1                & 1.000 & 0.647 & 0.310 \\
Tan2                & 0.933 & 0.604 & 0.286 & Tan3                & 0.804 & 0.522 & 0.247 \\
Tan4                & 0.545 & 0.353 & 0.169 & Thistle             & 0.847 & 0.749 & 0.847 \\
Thistle1            & 1.000 & 0.882 & 1.000 & Thistle2            & 0.933 & 0.824 & 0.933 \\
Thistle3            & 0.804 & 0.710 & 0.804 & Thistle4            & 0.545 & 0.482 & 0.545 \\
Tomato              & 1.000 & 0.388 & 0.278 & Tomato1             & 1.000 & 0.388 & 0.278 \\
Tomato2             & 0.933 & 0.361 & 0.259 & Tomato3             & 0.804 & 0.310 & 0.224 \\
Tomato4             & 0.545 & 0.212 & 0.149 & Turquoise           & 0.251 & 0.878 & 0.816 \\
Turquoise1          & 0.000 & 0.961 & 1.000 & Turquoise2          & 0.000 & 0.898 & 0.933 \\
Turquoise3          & 0.000 & 0.773 & 0.804 & Turquoise4          & 0.000 & 0.525 & 0.545 \\
Violet              & 0.933 & 0.510 & 0.933 & VioletRed           & 0.816 & 0.125 & 0.565 \\
VioletRed1          & 1.000 & 0.243 & 0.588 & VioletRed2          & 0.933 & 0.227 & 0.549 \\
VioletRed3          & 0.804 & 0.196 & 0.471 & VioletRed4          & 0.545 & 0.133 & 0.322 \\
Vory1               & 1.000 & 1.000 & 0.941 & Wheat               & 0.961 & 0.871 & 0.702 \\
Wheat1              & 1.000 & 0.906 & 0.729 & Wheat2              & 0.933 & 0.847 & 0.682 \\
Wheat3              & 0.804 & 0.729 & 0.588 & Wheat4              & 0.545 & 0.494 & 0.400 \\
White               & 1.000 & 1.000 & 1.000 & WhiteSmoke          & 0.961 & 0.961 & 0.961 \\
Yellow              & 1.000 & 1.000 & 0.000 & Yellow1             & 1.000 & 1.000 & 0.000 \\
Yellow2             & 0.933 & 0.933 & 0.000 & Yellow3             & 0.804 & 0.804 & 0.000 \\
Yellow4             & 0.545 & 0.545 & 0.000 & YellowGreen         & 0.604 & 0.804 & 0.196 \\
\hline
\end{tabular}
\end{center}

\newpage
\section{\xlabel{ap_MathMaps}Using MathMaps\label{ap:MathMaps}}
A MathMap is an \xref{AST}{}{sun210} Object which contains a recipe for
transforming positions from one \htmlref{co-ordinate Frame}{se:domains}~  to another and
(optionally) back again.  These transformations are specified using
arithmetic operations and mathematical functions similar to those
available in Fortran. \htmlref{WCSADD}{WCSADD} can be used to create a
MathMap and store it in a text file, and such text files can then be used
by applications such as \htmlref{REGRID}{REGRID} which require a mapping.

\subsection{Defining Transformation Functions}
   A MathMap's transformation functions are defined by a set of
   character strings holding Fortran-like expressions.  If the required
   Mapping has Nin input axes and Nout output axes, you would normally
   supply the Nout expressions for the forward transformation.  For
   instance, if Nout is 2 you might use:

   \begin{terminalv}
      'R = SQRT( X * X + Y * Y )'
      'THETA = ATAN2( Y, X )'
   \end{terminalv}

   which defines a transformation from Cartesian to polar
   co-ordinates.  Here, the variables that appear on the left of each
   expression (R and THETA) provide names for the output variables and
   those that appear on the right (X and Y) are references to input
   variables.

   To complement this, you must also supply expressions for the inverse
   transformation.  In this case, the number of
   expressions given would normally match the number of MathMap input
   co-ordinates, Nin.  If Nin is 2, you might use:

   \begin{terminalv}
      'X = R $*$ COS( THETA )'
      'Y = R $*$ SIN( THETA )'
   \end{terminalv}

   which expresses the transformation from polar to Cartesian
   co-ordinates.  Note that here the input variables (X and Y) are named on
   the left of each expression, and the output variables (R and THETA)
   are referenced on the right.

   Normally, you cannot refer to a variable on the right of an expression
   unless it is named on the left of an expression in the complementary
   set of functions.  Therefore both sets of functions (forward and
   inverse) must be formulated using the same consistent set of variable
   names.  This means that if you wish to leave one of the transformations
   undefined, you must supply dummy expressions which simply name each of
   the output (or input) variables.  For example, you might use:

   \begin{terminalv}
      'X'
      'Y'
   \end{terminalv}

   for the inverse transformation above, which serves to name the input
   variables but without defining an inverse transformation.

\subsection{Calculating Intermediate Values}
      It is sometimes useful to calculate intermediate values and then to
      use these in the final expressions for the output (or input)
      variables.  This may be done by supplying additional expressions for
      the forward (or inverse) transformation functions.  For instance, the
      following array of five expressions describes two-dimensional pin-cushion
      distortion:

      \begin{terminalv}
         'R = SQRT( XIN $*$ XIN $+$ YIN $*$ YIN )'
         'ROUT = R $*$ ( 1 $+$ 0.1 $*$ R $*$ R )'
         'THETA = ATAN2( YIN, XIN )',
         'XOUT = ROUT $*$ COS( THETA )'
         'YOUT = ROUT $*$ SIN( THETA )'
      \end{terminalv}

      Here, we first calculate three intermediate results ($R$, $ROUT$
      and $THETA$) and then use these to calculate the final results ($XOUT$
      and $YOUT$).  The MathMap knows that only the final two results
      constitute values for the output variables because its \att{Nout} attribute
      is set to 2.  You may define as many intermediate variables in this
      way as you choose.  Having defined a variable, you may then refer to it
      on the right of any subsequent expressions.

      Note that when defining the inverse transformation you may only refer
      to the output variables $XOUT$ and $YOUT$.  The intermediate variables $R$,
      $ROUT$ and $THETA$ (above) are private to the forward transformation and
      may not be referenced by the inverse transformation.  The inverse
      transformation may, however, define its own private intermediate
      variables.

\subsection{Expression Syntax}
      The expressions given for the forward and inverse transformations
      closely follow the syntax of Fortran (with some extensions for
      compatibility with the C language).  They may contain references to
      variables and literal constants, together with arithmetic, logical,
      relational and bitwise operators, and function invocations.  A set of
      symbolic constants is also available.  Each of these is described in
      detail below.  Parentheses may be used to override the normal order of
      evaluation.  There is no built-in limit to the length of expressions
      and they are insensitive to case or the presence of additional white
      space.

\subsection{Variables}
      Variable names must begin with an alphabetic character and may contain
      only alphabetic characters, digits, and the underscore character
      \texttt{{"\_"}}.  There is no built-in limit to the length of variable names.

\subsection{Literal Constants}
      Literal constants, such as \texttt{"0"}, \texttt{"1"}, \texttt{"0.007"} or \texttt{"2.505E-16"} may appear
      in expressions, with the decimal point and exponent being optional (a
      \texttt{"D"} may also be used as an exponent character).  A unary minus
      \texttt{"-"} may be used as a prefix.

\subsection{Arithmetic Precision}
      All arithmetic is floating point, performed in double precision.

\subsection{Propagation of Missing Data}
      Unless indicated otherwise, if any argument of a function or operator
      has a bad value (indicating missing data), then the result of
      that function or operation is also bad, so that such values are
      propagated automatically through all operations performed by MathMap
      transformations.  The special value used to flag bad values can be
      represented in expressions by the symbolic constant \texttt{{"}}$<$bad$>$\texttt{{"}}.

      A $<$bad$>$ result is also produced in response
      to any numerical error (such as division by zero or numerical
      overflow), or if an invalid argument value is provided to a function
      or operator.

\subsection{Arithmetic Operators}
      The following arithmetic operators are available:

      \begin{itemize}

         \item
         X1 $+$ X2: Sum of X1 and X2.

         \item
         X1 - X2: Difference of X1 and X2.

         \item
         X1 $*$ X2: Product of X1 and X2.

         \item
         X1 / X2: Ratio of X1 and X2.

         \item
         X1 $*$$*$ X2: X1 raised to the power of X2.

         \item
         $+$ X: Unary plus, has no effect on its argument.

         \item
         - X: Unary minus, negates its argument.
     \end{itemize}

\subsection{Logical Operators}
      Logical values are represented using zero to indicate .FALSE.  And
      non-zero to indicate .TRUE..  In addition, the bad value is taken to
      mean \texttt{"unknown"}.  The values returned by logical operators may therefore
      be 0, 1 or bad.  Where appropriate, \texttt{"tri-state"} logic is
      implemented.  For example, A.OR.B may evaluate to 1 if A is non-zero,
      even if B has the bad value.  This is because the result of the
      operation would not be affected by the value of B, so long as A is
      non-zero.

      The following logical operators are available:
      \begin{itemize}

         \item
         X1 .AND. X2: Logical AND between X1 and X2, returning 1 if both X1
         and X2 are non-zero, and 0 otherwise.  This operator implements
         tri-state logic. (The synonym \texttt{"\&\&"} is also provided for compatibility
         with C.)

         \item
         X1 .OR. X2: Logical OR between X1 and X2, returning 1 if either X1
         or X2 are non-zero, and 0 otherwise.  This operator implements
         tri-state logic. (The synonym \texttt{"$|$$|$"} is also provided for compatibility
         with C.)

         \item
         X1 .NEQV. X2: Logical exclusive OR (XOR) between X1 and X2,
         returning 1 if exactly one of X1 and X2 is non-zero, and 0
         otherwise.  Tri-state logic is not used with this operator. (The
         synonym \texttt{".XOR."} is also provided, although this is not standard
         Fortran.  In addition, the C-like synonym \texttt{"$\wedge$$\wedge$"} may be used, although
         this is also not standard.)

         \item
         X1 .EQV. X2: Tests whether the logical states of X1 and X2
         (\emph{i.e.} .TRUE./.FALSE.) are equal.  It is the negative of the exclusive OR
         (XOR) function.  Tri-state logic is not used with this operator.

         \item
         .NOT. X: Logical unary NOT operation, returning 1 if X is zero, and
         0 otherwise. (The synonym \texttt{{"}}!\texttt{{"}} is also provided for compatibility with
         C.)
     \end{itemize}

\subsection{Relational Operators}
      Relational operators return the logical result (0 or 1) of comparing
      the values of two floating-point values for equality or inequality.  The
      bad value may also be returned if either argument is $<$bad$>$.

      The following relational operators are available:
      \begin{itemize}

         \item
         X1 .EQ. X2: Tests whether X1 equals X2. (The synonym \texttt{"=="} is also
         provided for compatibility with C.)

         \item
         X1 .NE. X2: Tests whether X1 is unequal to X2. (The synonym \texttt{"!="} is
         also provided for compatibility with C.)

         \item
         X1 .GT. X2: Tests whether X1 is greater than X2. (The synonym \texttt{"$>$"} is
         also provided for compatibility with C.)

         \item
         X1 .GE. X2: Tests whether X1 is greater than or equal to X2. (The
         synonym \texttt{"$>$="} is also provided for compatibility with C.)

         \item
         X1 .LT. X2: Tests whether X1 is less than X2. (The synonym \texttt{"$<$"} is also
         provided for compatibility with C.)

         \item
         X1 .LE. X2: Tests whether X1 is less than or equal to X2. (The synonym
         \texttt{"$<$="} is also provided for compatibility with C.)

     \end{itemize}

      Note that relational operators cannot usefully be used to compare
      values with the $<$bad$>$ value (representing missing data), because the
      result is always $<$bad$>$.  The ISBAD() function should be used instead.

      Note, also, that because logical operators can operate on floating
      point values, care must be taken to use parentheses in some cases
      where they would not normally be required in Fortran.  For example,
      the expression:

      \begin{terminalv}
         .NOT.  A .EQ. B
      \end{terminalv}

      must be written:

      \begin{terminalv}
         .NOT. ( A .EQ. B )
      \end{terminalv}

      to prevent the .NOT.  Operator from associating with the variable A.

\subsection{Bitwise Operators}
      Bitwise operators are often useful when operating on raw data
      (\emph{e.g.} from instruments), so they are provided for use in MathMap
      expressions.  In this case, however, the values on which they operate
      are floating-point values rather than the more usual pure integers.  In
      order to produce results which match the pure integer case, the
      operands are regarded as fixed point binary numbers (\emph{i.e.} with the
      binary equivalent of a decimal point) with negative numbers
      represented using twos-complement notation.  For integer values, the
      resulting bit pattern corresponds to that of the equivalent signed
      integer (digits to the right of the point being zero).  Operations on
      the bits representing the fractional part are also possible, however.

      The following bitwise operators are available:
      \begin{itemize}

         \item
         X1 $>$$>$ X2: Rightward bit shift.  The integer value of X2 is taken
         (rounding towards zero) and the bits representing X1 are then
         shifted this number of places to the right (or to the left if the
         number of places is negative).  This is equivalent to dividing X1 by
         the corresponding power of 2.

         \item
         X1 $<$$<$ X2: Leftward bit shift.  The integer value of X2 is taken
         (rounding towards zero), and the bits representing X1 are then
         shifted this number of places to the left (or to the right if the
         number of places is negative).  This is equivalent to multiplying X1
         by the corresponding power of 2.

         \item
         X1 \& X2: Bitwise AND between the bits of X1 and those of X2
         (equivalent to a logical AND applied at each bit position in turn).

         \item
         X1 $|$ X2: Bitwise OR between the bits of X1 and those of X2
         (equivalent to a logical OR applied at each bit position in turn).

         \item
         X1 $\wedge$ X2: Bitwise exclusive OR (XOR) between the bits of X1 and
         those of X2 (equivalent to a logical XOR applied at each bit
         position in turn).

     \end{itemize}

      Note that no bit inversion operator is provided.  This is
      because inverting the bits of a twos-complement fixed point binary
      number is equivalent to simply negating it.  This differs from the
      pure integer case because bits to the right of the binary point are
      also inverted.  To invert only those bits to the left of the binary
      point, use a bitwise exclusive OR with the value -1 (\emph{i.e.} X$\wedge$-1).

\subsection{Functions}
      The following functions are available:

      \begin{itemize}

         \item
         ABS(X): Absolute value of X (sign removal), same as FABS(X).

         \item
         ACOS(X): Inverse cosine of X, in radians.

         \item
         ACOSD(X): Inverse cosine of X, in degrees.

         \item
         ACOSH(X): Inverse hyperbolic cosine of X.

         \item
         ACOTH(X): Inverse hyperbolic cotangent of X.

         \item
         ACSCH(X): Inverse hyperbolic cosecant of X.

         \item
         AINT(X): Integer part of X (round towards zero), same as INT(X).

         \item
         ASECH(X): Inverse hyperbolic secant of X.

         \item
         ASIN(X): Inverse sine of X, in radians.

         \item
         ASIND(X): Inverse sine of X, in degrees.

         \item
         ASINH(X): Inverse hyperbolic sine of X.

         \item
         ATAN(X): Inverse tangent of X, in radians.

         \item
         ATAND(X): Inverse tangent of X, in degrees.

         \item
         ATANH(X): Inverse hyperbolic tangent of X.

         \item
         ATAN2(X1, X2): Inverse tangent of X1/X2, in radians.

         \item
         ATAN2D(X1, X2): Inverse tangent of X1/X2, in degrees.

         \item
         CEIL(X): Smallest integer value not less then X (round towards
           plus infinity).

         \item
         COS(X): Cosine of X in radians.

         \item
         COSD(X): Cosine of X in degrees.

         \item
         COSH(X): Hyperbolic cosine of X.

         \item
         COTH(X): Hyperbolic cotangent of X.

         \item
         CSCH(X): Hyperbolic cosecant of X.

         \item
         DIM(X1, X2): Returns X1-X2 if X1 is greater than X2, otherwise 0.

         \item
         EXP(X): Exponential function of X.

         \item
         FABS(X): Absolute value of X (sign removal), same as ABS(X).

         \item
         FLOOR(X): Largest integer not greater than X (round towards
           minus infinity).

         \item
         FMOD(X1, X2): Remainder when X1 is divided by X2, same as
           MOD(X1, X2).

         \item
         GAUSS(X1, X2): Random sample from a Gaussian distribution with mean
           X1 and standard deviation X2.

         \item
         INT(X): Integer part of X (round towards zero), same as AINT(X).

         \item
         ISBAD(X): Returns 1 if X has the $<$bad$>$ value, otherwise 0.

         \item
         LOG(X): Natural logarithm of X.

         \item
         LOG10(X): Logarithm of X to base 10.

         \item
         MAX(X1, X2, ...): Maximum of two or more values.

         \item
         MIN(X1, X2, ...): Minimum of two or more values.

         \item
         MOD(X1, X2): Remainder when X1 is divided by X2, same as
           FMOD(X1, X2).

         \item
         NINT(X): Nearest integer to X (round to nearest).

         \item
         POISSON(X): Random integer-valued sample from a Poisson
           distribution with mean X.

         \item
         POW(X1, X2): X1 raised to the power of X2.

         \item
         RAND(X1, X2): Random sample from a uniform distribution in the
           range X1 to X2 inclusive.

         \item
         SECH(X): Hyperbolic secant of X.

         \item
         SIGN(X1, X2): Absolute value of X1 with the sign of X2
           (transfer of sign).

         \item
         SIN(X): Sine of X in radians.

         \item
         SINC(X): Sinc function of X [= SIN(X)/X].

         \item
         SIND(X): Sine of X in degrees.

         \item
         SINH(X): Hyperbolic sine of X.

         \item
         SQR(X): Square of X (= X$*$X).

         \item
         SQRT(X): Square root of X.

         \item
         TAN(X): Tangent of X in radians.

         \item
         TAND(X): Tangent of X in degrees.

         \item
         TANH(X): Hyperbolic tangent of X.
     \end{itemize}

\subsection{Symbolic Constants}

      The following symbolic constants are available (the enclosing
      \texttt{{"}}$<$$>$\texttt{{"}} brackets must be included):

      \begin{itemize}

         \item
         $<$bad$>$: The \htmlref{`bad'}{se:masking} value used to flag missing data.  Note
         that you cannot usefully compare values with this constant because the
         result is always $<$bad$>$.  The ISBAD() function should be used instead.

         \item
         $<$dig$>$: Number of decimal digits of precision available in a
         floating-point (double-precision) value.

         \item
         $<$e$>$: \htmlref{Base}{Base} of natural logarithms.

         \item
         $<$epsilon$>$: Smallest positive number such that 1.0$+$$<$epsilon$>$ is
         distinguishable from unity.

         \item
         $<$mant\_dig$>$: The number of base $<$radix$>$ digits stored in the
         mantissa of a floating-point (double-precision) value.

         \item
         $<$max$>$: Maximum representable floating-point (double-precision) value.

         \item
         $<$max\_10\_exp$>$: Maximum integer such that 10 raised to that power
         can be represented as a floating-point (double-precision) value.

         \item
         $<$max\_exp$>$: Maximum integer such that $<$radix$>$ raised to that
         power minus 1 can be represented as a floating-point (double-precision)
         value.

         \item
         $<$min$>$: Smallest positive number which can be represented as a
         normalised floating-point (double-precision) value.

         \item
         $<$min\_10\_exp$>$: Minimum negative integer such that 10 raised to that
         power can be represented as a normalised floating-point
         (double-precision) value.

         \item
         $<$min\_exp$>$: Minimum negative integer such that $<$radix$>$ raised to
         that power minus 1 can be represented as a normalised floating-point
         (double-precision) value.

         \item
         $<$pi$>$: Ratio of the circumference of a circle to its diameter.

         \item
         $<$radix$>$: The radix (number base) used to represent the mantissa of
         floating-point (double-precision) values.

         \item
         $<$rounds$>$: The mode used for rounding floating-point results after
         addition.  Possible values include: -1 (indeterminate), 0 (toward
         zero), 1 (to nearest), 2 (toward plus infinity) and 3 (toward minus
         infinity).  Other values indicate machine-dependent behaviour.
     \end{itemize}

\subsection{Evaluation Precedence and Associativity}

      Items appearing in expressions are evaluated in the following order
      (highest precedence first):

      \begin{itemize}

         \item
         Constants and variables

         \item
         Function arguments and parenthesised expressions

         \item
         Function invocations

         \item
         Unary $+$ - ! .not.

         \item
         $*$$*$

         \item
         $*$ /

         \item
         $+$ -

         \item
         $<$$<$ $>$$>$

         \item
         $<$ .lt. $<$= .le. $>$ .gt. $>$= .ge.

         \item
         == .eq. != .ne.

         \item
         \&

         \item
         $\wedge$

         \item
         $|$

         \item
         \&\& .and.

         \item
         $\wedge$$\wedge$

         \item
         $|$$|$ .or

         \item
         .eqv. .neqv. .xor.

     \end{itemize}

      All operators associate from left-to-right, except for unary $+$,
      unary -, !, .not.  And $*$$*$ which associate from right-to-left.

\newpage
\section{\xlabel{ap_NDFformat}Standard Components in an NDF
\label{ap:NDFformat}}

An NDF comprises a main data array plus a collection of objects drawn
from a set of standard items and extensions (see \xref{SUN/33}{sun33}{}).
Only the main data array must be present; all the other items are optional.

\texttt{example.sdf} is an NDF which contains all the standard NDF
components, except a WCS component and a \htmlref{FITS
extension}{se:fitsairlock}; it also has a FIGARO extension.  The
structure of the file (as revealed by \HDSTRACEref) is shown below.
The layout is

\begin{terminalv}
     NAME(dimensions)    <TYPE>     VALUE(S)
\end{terminalv}

and each level down the hierarchy is indented.  Note that scalar
objects have no dimensions

\begin{terminalv}

   EXAMPLE  <NDF>

      DATA_ARRAY     <ARRAY>         {structure}
         DATA(856)      <_REAL>         *,0.2284551,-2.040089,45.84504,56.47374,
                                        ... 746.2602,820.8976,570.0729,*,449.574
         ORIGIN(1)      <_INTEGER>      4
         BAD_PIXEL      <_LOGICAL>      FALSE

      TITLE          <_CHAR*30>      'HR6259 - AAT fibre data'
      LABEL          <_CHAR*20>      'Flux'
      UNITS          <_CHAR*20>      'Counts/s'
      QUALITY        <QUALITY>       {structure}
         BADBITS        <_UBYTE>        1
         QUALITY(856)   <_UBYTE>        1,0,0,0,0,1,0,0,0,0,0,0,0,0,0,0,0,0,
                                        ... 0,0,0,0,0,0,0,0,0,0,0,0,0,0,0,1,0

      VARIANCE       <ARRAY>         {structure}
         DATA(856)      <_REAL>         2.1,0.1713413,1.5301,34.38378,42.35531,
                                        ... 615.6732,427.5547,353.9127,337.1805
         ORIGIN(1)      <_INTEGER>      4
         BAD_PIXEL      <_LOGICAL>      FALSE

      AXIS(1)        <AXIS>          {structure}

      Contents of AXIS(1)
         DATA_ARRAY(856)  <_REAL>       3847.142,3847.672,3848.201,3848.731,
                                        ... 4298.309,4298.838,4299.368,4299.897
         LABEL          <_CHAR*20>      'Wavelength'
         UNITS          <_CHAR*20>      'Angstroms'

      HISTORY        <HISTORY>       {structure}
         CREATED        <_CHAR*30>      '1990-DEC-12 08:21:02.324'
         CURRENT_RECORD  <_INTEGER>     3
         RECORDS(10)    <HIST_REC>      {array of structures}

         Contents of RECORDS(1)
            TEXT           <_CHAR*40>      'Extracted spectrum from fibre data.'
            DATE           <_CHAR*25>      '1990-DEC-19 08:43:03.08'
            COMMAND        <_CHAR*30>      'FIGARO V2.4 FINDSP command'


      MORE           <EXT>           {structure}
         FIGARO         <EXT>           {structure}
            TIME           <_REAL>         1275
            SECZ           <_REAL>         2.13

   End of Trace.
\end{terminalv}

Of course, this is only an example format.  There are various ways of
representing some of the components.  These {\em variants\/} are
described in \xref{SGP/38}{sgp38}{}, but not all are currently supported.

The components are considered in detail below.  The names (in bold
typeface) are significant as they are used by the NDF access routines to
identify the components.

\begin{description}

\label{apndf:data}
\item[{\bf DATA}] --- the main data array (called DATA\_ARRAY for
historical reasons) is the only component which must be present in an
NDF.  In the case of \texttt{example.sdf}, is a one-dimensional array of
real type with 856 elements. It can have up to seven dimensions.  It
is particularly referenced via parameter names IN, OUT, and NDF.

If neither origin nor bad-pixel flag were present, the DATA component could
have been a one-dimensional array like this,

\begin{terminalv}
      DATA_ARRAY(856)  <_REAL>       *,0.2284551,-2.040089,45.84504,56.47374,
                                     ... 746.2602,820.8976,570.0729,*,449.574
\end{terminalv}

rather than the structure shown above.

\label{apndf:origin}
\item[{\bf ORIGIN}] --- an array of the indices of the first pixel along
each axis, defaulting to 1.  See \slhyperref{Pixel Indices, Pixel
Co-ordinates, and Grid Co-ordinates}{Section~}{}{se:pixgrd}~ for further
information and a graphic.  ORIGIN may be set with task
\htmlref{SETORIGIN}{SETORIGIN}, or after processing an \htmlref{NDF
section}{se:ndfsect}.

\label{apndf:bpflag}
\item[{\bf BAD\_PIXEL}] --- the bad-pixel flag indicating whether there
may be \htmlref{bad pixels}{se:masking} present.  See
\slhyperref{Bad-pixel Flag Values}{SETBAD}{}{setbad:badpixelflag} for further
information.  The flag may be set with task
\htmlref{SETBAD}{SETBAD}.

\label{apndf:title}
\item[{\bf TITLE}] --- the  character string \texttt{"HR6259 - AAT fibre
data"} describes the contents of the NDF.  The NDF's TITLE might be
used as the title of a graph \emph{etc.} It may be set with task
\htmlref{SETTITLE}{SETTITLE}.  Applications that create an NDF assign
a TITLE to the NDF via a parameter, called TITLE unless the
application generates several NDFs.

\label{apndf:label}
\item[{\bf LABEL}] --- the character string \texttt{"Flux"} describes the
quantity represented in the NDF's main data array.  The LABEL is
intended for use on the axis of graphs \emph{etc.}  It may be set
using the task \htmlref{SETLABEL}{SETLABEL}.

\label{apndf:units}
\item[{\bf UNITS}] --- this character string describes the physical units
of the quantity stored in the main data array, in this case,
\texttt{"Counts/s"}.  It may be set via the command
\htmlref{SETUNITS}{SETUNITS}.

\label{apndf:quality}
\item[{\bf QUALITY}] --- this component is used to indicate the quality
of each element in the main data array, for example whether each pixel
is vignetted or not.  The quality structure contains a quality array
and a BADBITS value, both of which {\em must\/} be of type
\htmlref{\_UBYTE}{ap:HDStypes}.  The quality array has the same shape
and size as the main data array and is used in conjunction with the
BADBITS value to decide the quality of a pixel in the main data array.
In \texttt{example.sdf} the BADBITS component has value 1.  This means
that a value of 1 in the quality array indicates a bad pixel in the
main data array, whereas any other value indicates that the associated
pixel is good.  (Note that the pixel is bad if the bit-wise comparison
QUALITY \texttt{{"}}AND\texttt{{"}} BADBITS is non-zero).  The meanings of the
QUALITY bits are arbitrary.  See the task \htmlref{SETBB}{SETBB}.
To enter new quality information, use the SETQUAL command.

\label{apndf:variance}\xlabel{apndf:variance}
\item[{\bf VARIANCE}] --- the variance array is the same shape and size
as the main data array and contains the errors associated with the
individual data values.  These are stored as {\em variance\/}
estimates for each pixel.  VARIANCE may be set with the
\htmlref{SETVAR}{SETVAR} command.

\label{apndf:axis}
\item[{\bf AXIS}] --- the AXIS structure may contain axis information for
any dimension of the NDF's main array.  In this case, the main data
array is only one-dimensional, therefore only the AXIS(1) structure is
present.  This structure contains the actual axis data array of pixel
centres, and also label and units information.  \KAPPA\ uses the
label and units for axis annotations.  Not shown in this example
are optional array components for storing pixel widths and the variance
of the axis centres.  All axes or none must be present.  Use
\htmlref{SETAXIS}{SETAXIS} to set the values of an AXIS array component;
\htmlref{AXLABEL}{AXLABEL} and \htmlref{AXUNITS}{AXUNITS} to set an
axis LABEL or \htmlref{UNITS}{apndf:units}~ component; and \htmlref{SETNORM}{SETNORM} to set
an axis normalisation flag.  For more information see
\slhyperref{`Co-ordinate Frames, Axes and Domains'}{Section~}{}{se:domains}.

\label{apndf:wcs}
\item[{\bf WCS}] --- this component provides an alternative, and superior,
method for storing co-ordinate information.  The AXIS structure described
above has the severe limitation that it can only describe co-ordinate
systems in which all axes are independent (\emph{i.e.} the value on
any axis can be changed without needing to change the values on any
other axes).  This means for instance that axes cannot be rotated (other
than by multiples of 90\dgs), and cannot be used to describe celestial
co-ordinates over a large field, or near a pole.  The WCS component
overcomes these restrictions.  It contains descriptions of an arbitrary
number of different \htmlref{co-ordinate Frames}{se:domains}, together
with the mappings required to convert positions between these Frames.  All
NDFs have four default Frames available; these are known as the GRID,
FRACTION, PIXEL, and AXIS Frames.  The PIXEL Frame corresponds to pixel
co-ordinates.  The AXIS Frame corresponds to the co-ordinate Frame implied
by the NDF AXIS structures.  The GRID Frame is similar to the PIXEL Frame,
the main difference being that it has a different origin; the centre of the
first (`bottom left') pixel is always $(1.0,1.0)$ in the GRID Frame regardless
of the pixel origin of the NDF.  The FRACTION Frame corresponds to normalised
PIXEL or GRID co-ordinates, scaled from zero to one. Other Frames can
be added to the WCS component in various ways, for instance, by importing
them from appropriate FITS headers (\htmlref{FITSIN}{FITSIN},
\htmlref{FITSDIN}{FITSDIN}), or using an appropriate application to
create them from scratch (\emph{e.g.} \htmlref{WCSADD}{WCSADD},
\htmlref{SETSKY}{SETSKY} or \GAIAref).  See \slhyperref{Using World
Co-ordinate Systems}{Section~}{}{se:wcsuse} for more information.

\label{apndf:history}
\item[{\bf HISTORY}] --- this component provides a record of the
processing history of the NDF.  Only the first of three records is
shown for \texttt{example.sdf}.  This indicates that the spectrum was
extracted from fibre data using the \FIGARO\ FINDSP command on
1990 December 19.  The history recording level is set by task
\htmlref{HISSET}{HISSET}.  This task also allows you to switch off
history recording or delete the history records.
\htmlref{HISLIST}{HISLIST} lists an NDF's history.  You can add
commentary with \htmlref{HISCOM}{HISCOM}.

\label{apndf:extensions}
\item[{\bf EXTENSIONs}] --- the purpose of extensions is to store
non-standard items.  These auxiliary data could be information about
the original observing setup, such as the airmass during the
observation or the temperature of the detector; they may be
calibration data or results produced during processing of the data
array, {\it{e.g.}}\ spectral-line fits. The extensions are located
within the MORE top-level component.  \texttt{example.sdf} began life as
a \FIGAROref\ file.  It was converted to an NDF using the command
\xref{DST2NDF}{sun55}{DST2NDF}\latex{ (see SUN/55)}.  It contains
values for the airmass and exposure time associated with the
observations.  These are stored in the FIGARO extension, and the
intention is that the \FIGARO\ applications which use these
values will know where to find them.  Task \htmlref{SETEXT}{SETEXT}
lets the contents of extensions be listed, created, deleted, renamed
and assigned new values.

One extension that is used by \KAPPA\ is the
\htmlref{FITS extension}{se:fitsairlock}.  This holds the FITS headers
as an array of 80-character elements, \emph{i.e.} one FITS card image
per array element.  You can extract the values of ancillary items from
the FITS extension to a non-standard extension via
\htmlref{FITSIMP}{FITSIMP}.  Use \htmlref{FITSEXP}{FITSEXP} to do the
reverse operation.  The extension can be listed via the command
\htmlref{FITSLIST}{FITSLIST}.  \htmlref{FITSEDIT}{FITSEDIT} allows you
to edit the headers prior to export of the dataset to another format
such as \FITSref\ or \IRAFref\@.

\label{apndf:provenance} \KAPPA\ uses a PROVENANCE extension to record
details of the ancestor NDFs used to generate an NDF.  This lets you
determine how the NDF before you, came to be.  For each ancestor NDF,
the provenance information includes its path at creation time, the
creation epoch, the software that generated the ancestor, and indices
to its parent NDFs (if any exist) within the PROVENANCE extension.
There is also a MORE component within PROVENANCE that can be filled
with arbitrary additional information.  Provenance recording is
controlled through the \texttt{AUTOPROV} environment variable: set it to \texttt{
1} to enable recording, and set it to any other value to disable
recording.  When \texttt{AUTOPROV} is undefined, then an output NDF will
have provenance only if at least one of the input NDFs has provenance.
You can examine provenance with
\htmlref{PROVSHOW}{PROVSHOW}, and modify it with
\htmlref{PROVADD}{PROVADD} and \htmlref{PROVMOD}{PROVMOD}.  Selected
provenance may be removed with \htmlref{PROVREM}{PROVREM}.

\end{description}

\newpage
\section{\xlabel{ap_IMAGEformat}IMAGE data format\label{ap:IMAGEformat}}

The IMAGE format as used by some commands in earlier versions of
\KAPPA\ is a simple \HDSref\ structure, comprising a floating-point
data array, a character title,
and the maximum and minimum data values.  It is variant of the
original Wright-Giddings IMAGE structure.  There are others is use
that contain more items.  An example structure is shown schematically
below using the \HDSTRACEref\latex{ (SUN/102)} notation; see
\latexhtml{Appendix~\ref{ap:NDFformat}}{\htmlref{the NDF format}{ap:NDFformat}}.
\begin{terminalv}
   HORSEHEAD  <IMAGE>

      DATA_ARRAY(384,512)  <_REAL>   100.5,102.6,110.1,109.8,105.3,107.6,
                                     ... 123.1,117.3,119,120.5,127.3,108.4
      TITLE          <_CHAR*72>      'KAPPA - Flip'
      DATA_MIN       <_REAL>         28.513
      DATA_MAX       <_REAL>         255.94

   End of Trace.
\end{terminalv}

The DATA\_ARRAY may have up to seven dimensions.
IMAGE structures are associated with parameters like INPIC and OUTPIC.
The TITLE object of new IMAGE structures takes the value of the
Parameter OTITLE.  DATA\_MIN and DATA\_MAX are now ignored.

The IMAGE format is not too dissimilar from a {\em primitive\/} NDF
with no extensions.  Indeed if it did not have DATA\_MAX and DATA\_MIN
it would be a {\it bona fide}\ NDF.  Thus applications that handle the
IMAGE format can follow the rules of \xref{SGP/38}{sgp38}{} and
process it like an NDF.  In effect this means that all extensions are
propagated to output files, and a quality array is propagated where
the processing does not invalidate its values.  IMAGE applications
also handle most {\em simple\/} NDFs correctly (those where the data
array is an array of numbers at the second level of the hierarchical
structure).  This similarity in formats enables NDF and IMAGE
applications to work in co-operation, and so the conversion within
\KAPPA\ was undertaken piecemeal over several years.
Note that the primitive
variant is no longer the norm for NDFs, since for example, it does not
support origin information.

\section{\xlabel{ap_HDStypes}Supported HDS Data Types\label{ap:HDStypes}}

\KAPPA\ applications can process \NDFref{NDFs}\ in one or more of the
following \HDSref\ data types.  The correspondence between Fortran types
and HDS data types is as follows:

\begin{center}
\begin{tabular}{|l|c|l|} \hline
{\bf HDS Type} & {\bf Number of bytes} & {\bf FORTRAN Type}\\ \hline
\_DOUBLE & 8 & DOUBLE PRECISION \\
\_INTEGER & 4 & INTEGER \\
\_INT64 & 8 & INTEGER*8 (non-standard) \\
\_REAL & 4 & REAL \\
\_UBYTE & 1 & BYTE \\
\_BYTE & 1 & BYTE \\
\_UWORD & 2 & INTEGER*2 \\
\_WORD & 2 & INTEGER*2\\
\hline
\end{tabular}
\end{center}

(\_UBYTE) and (\_UWORD) types are unsigned and so permit data ranges of
0--255 and 0--65535 respectively.

\newpage

\section{\xlabel{se_changes}Release Notes---V2.6\label{se:changes}}

\subsection{New Commands}
The following new applications have been added:

\begin{description}
   \item [\htmlref{COMPLEX}{COMPLEX}] \mbox{}
   Converts between representations of complex data, such as
   extracting or combining real and imaginary parts.
   \item [\htmlref{MOCGEN}{MOCGEN}] \mbox{}
   Creates a description of selected regions in an image using the IVOA's
   Multi-Order Coverage (MOC) scheme.
   \item [\htmlref{PIXBIN}{PIXBIN}] \mbox{}
   Places each input pixel into a bin defined by one or more index NDFs
   and creates an output NDF holding all the values in each bin.
\end{description}

\subsection{General Changes}
The following applications now support huge NDFs:
\htmlref{ADD}{ADD}, \htmlref{ADD}{CADD}, \htmlref{CDIV}{CDIV},
\htmlref{CMULT}{CMULT}, \htmlref{CSUB}{CSUB},
\htmlref{COLLAPSE}{COLLAPSE}, \htmlref{DIV}{DIV},
\htmlref{MFITTREND}{MFITTREND}, \htmlref{MSTATS}{MSTATS}, \htmlref{MULT}{MULT}
\htmlref{NDFCOPY}{NDFCOPY}, \htmlref{NDFTRACE}{NDFTRACE}, \htmlref{NUMB}{NUMB}
\htmlref{PERMAXES}{PERMAXES}, \htmlref{STATS}{STATS}, and
\htmlref{SUB}{SUB}.  Huge means more than 2,147,483,647 elements in an
array component.


\subsection{Modified Commands}
The following applications have been modified:
\begin{description}[style=nextline]

   \item [\htmlref{ARDMASK}{ARDMASK}]\mbox{}
      \begin{itemize}
         \item It can now process complex data.
      \end{itemize}

   \item [\htmlref{CHPIX}{CHPIX}]\mbox{}
      \begin{itemize}
        \item It is easier now to make several edits by supplying the
        NDF sections and fill values in a text file.   This is made
        possible new parameters called MODE and FILE.  The default
        for MODE preserves the historic interactive behaviour.
      \end{itemize}

   \item [\htmlref{DISPLAY}{DISPLAY}]\mbox{}
      \begin{itemize}
        \item Parameter KEYPOS(2) accepts negative values to instruct DISPLAY
        to plot the key aligned with the image.  Thus setting KEY=\texttt{TRUE}
        and KEYPOS=\texttt{[0,-1]} will draw the colour-table ramp so that it
        abuts the image with its vertical extent matching the image,
        as commonly needed for journal graphics.
      \end{itemize}

   \item [\htmlref{LINPLOT}{LINPLOT}]\mbox{}
      \begin{itemize}
        \item The XMAP parameter now has a new option called \texttt{"LRLinear"},
        which causes the $X$ axis to be annotated linearly increasing from
        left to right. This differs from the existing \texttt{"Linear"} option,
        which may sometimes produce annotated values that increase from
        right to left, depending on the nature of the WCS.
      \end{itemize}

   \item [\htmlref{LISTSHOW}{LISTSHOW}]\mbox{}
      \begin{itemize}
        \item New parameters called NDF and COMP allow the data, error,
        or variance values at the catalogued positions within a chosen
        NDF to be displayed.  These use the interpolation method specified
        by the new Parameters METHOD and PARAMS.  The pixel values are also
        written to a new output parameter called PIXVALS.
      \end{itemize}

   \item [\htmlref{LISTSHOW}{LISTSHOW}]\mbox{}
      \begin{itemize}
        \item New parameters called NDF and COMP allow the data, error,
        or variance values at the catalogued positions within a chosen
        NDF to be displayed.  These use the interpolation method specified
        by the new Parameters METHOD and PARAMS.  The pixel values are also
        written to a new output parameter called PIXVALS.
      \end{itemize}

   \item [\htmlref{MANIC}{MANIC}]\mbox{}
      \begin{itemize}
        \item It is now possible to collapse axes using a median instead
        of the default mean, controlled by a new parameter called ESTIMATOR.
      \end{itemize}

   \item [\htmlref{REMQUAL}{REMQUAL}]\mbox{}
      \begin{itemize}
        \item A new parameter called CLEAR allows the corresponding bit within
        the NDF's QUALITY array to be cleared.
      \end{itemize}

   \item [\htmlref{SCATTER}{SCATTER}]\mbox{}
      \begin{itemize}
        \item A new parameter called FIT allows a linear fit to the scatter
        plot to be calculated and displayed.
        \item The correlation coefficient is now calculated from the
        visible points alone. Any points outside the bounds of the
        plot are ignored. Previously, such points were not ignored.
        \item The reported number of points plotted now ignores any
        points outside the bounds of the plot.
      \end{itemize}

   \item [\htmlref{SUBSTITUTE}{SUBSTITUTE}]\mbox{}
      \begin{itemize}
        \item A new parameter called LUT allows multiple values to be
        changed simultaneously by supplying a table of old and new
        values in the form a text file. Interpolation between the values
        can be performed if required.
        \item A new parameter called TYPE allows the data type of the
        output NDF to be specified explicitly.
      \end{itemize}

\end{description}

\section{Notes from Previous Few Releases}

\subsection{Release Notes---V2.0}

\subsubsection{General Changes}

\begin{itemize}
   \item Now supports 64-bit integer data.
\end{itemize}

\subsubsection{New Commands}
The following new applications have been added:

\begin{description}
   \item [\htmlref{CONFIGECHO}{CONFIGECHO}] \mbox{}
   This is intended as a scripting tool.  It displays the value
   of a named entry in a group of configuration parameters.
   \item [\htmlref{NDFECHO}{NDFECHO}] \mbox{}
   This is intended as a scripting tool.  It expands a given group
   expression into a list of explicit NDF names, and displays a specified
   subset of the expanded names.
\end{description}

\subsubsection{Modified Commands}
The following applications have been modified:

\begin{description}
   \item [\htmlref{CHANMAP}{CHANMAP}]\mbox{}
      \begin{itemize}
        \item Four new estimators are available: FBAD, FGOOD, NBAD and
        NGOOD, which produce the fraction/count of good/bad pixel values.
      \end{itemize}
   \item [\htmlref{COLLAPSE}{COLLAPSE}]\mbox{}
      \begin{itemize}
        \item Four new estimators are available: FBAD, FGOOD, NBAD and
        NGOOD, which produce the fraction/count of good/bad pixel values.
      \end{itemize}
   \item [\htmlref{MSTATS}{MSTATS}]\mbox{}
      \begin{itemize}
        \item Four new estimators are available: FBAD, FGOOD, NBAD and
        NGOOD, which produce the fraction/count of good/bad pixel values.
      \end{itemize}
   \item [\htmlref{NORMALIZE}{NORMALIZE}]\mbox{}
      \begin{itemize}
        \item A new boolean parameter called LOOP permits normalisation
        against a single row or column when comparing two-dimensional NDFs.
      \end{itemize}
   \item [\htmlref{PARGET}{PARGET}]\mbox{}
      \begin{itemize}
        \item A new boolean parameter called VECTOR specifies the output
        format to use for vector-valued parameters.
      \end{itemize}
   \item [\htmlref{ROTATE}{ROTATE}]\mbox{}
      \begin{itemize}
        \item Now estimates north at the centre of the image rather than
        at the bottom left corner, and uses a more accurate method.
      \end{itemize}
   \item [\htmlref{WCSADD}{WCSADD}]\mbox{}
      \begin{itemize}
        \item The transfer of set attribute values from basis \xref{Frame}{sun210}{Frame}~
        to new Frame can now be controlled using a new boolean parameter
        called TRANSFER (previously, set attributes were always transferred). The
        new default is to transfer attributes only if the two Frames have
        the same class and \att{Domain}.
      \end{itemize}
   \item [\htmlref{WCSREMOVE}{WCSREMOVE}] \mbox{}
      \begin{itemize}
        \item The Frames to remove can now be specified by name as well as
              by index.
      \end{itemize}
\end{description}

\subsection{Release Notes---V2.1}

\subsubsection{New Commands}
The following new applications have been added:

\begin{description}
   \item [\htmlref{EXCLUDEBAD}{EXCLUDEBAD}] \mbox{}
   This will copy a two-dimensional NDF, excluding any rows or columns
   that contain too many bad values. Good rows or columns are shuffled
   down to lower indices to fill the gaps left by the excluded rows or
   columns, thus causing the output NDF to be smaller than the input NDF.
\end{description}

\subsubsection{Modified Commands}
The following applications have been modified:

\begin{description}
   \item [\htmlref{ARDPLOT}{ARDPLOT}]\mbox{}
      \begin{itemize}
        \item Can now display the outline of a Region even if no picture has
         been displayed previously on the graphics device. The size of
         the plot is controlled by the new SIZE parameter. Any existing
         picture can be ignored by setting the new CLEAR parameter to \texttt{TRUE}.
      \end{itemize}
   \item [\htmlref{BEAMFIT}{BEAMFIT}]\mbox{}
      \begin{itemize}
        \item There is now more control of the initial or fixed sizes
        and shapes of the beams.  Note that {\bf this has involved a
        change of the type and function of Parameter FIXFWHM}.
        FIXFWHM like other FIX- parameters is \_LOGICAL; it just
        constrains whether the FWHM values should be fixed.  A new
        parameter called FWHM allows you to set either initial values,
        or when FIXFWHM is also set \texttt{TRUE}, it sets fixed FWHM
        values.  The interpretation of FWHM values depends on a new
        CIRCULAR parameter, which constrains the fit to be circular
        thus there is no minor axis and orientation derived.  In
        combination it is possible to give a list of circular or
        elliptical FWHMs.
        \item The output parameters now store the statistics of every
        fitted beam, not just those of the primary beam.
      \end{itemize}
   \item [\htmlref{CENTROID}{CENTROID}]\mbox{}
      \begin{itemize}
        \item The centroid's formatted co-ordinates, such as right
        ascension and declination, are now normalised into the usual
        ranges.  This aplies both to the reported positions and the
        output parameters.
      \end{itemize}
   \item [\htmlref{COPYBAD}{COPYBAD}]\mbox{}
      \begin{itemize}
        \item  Now writes the number of good and bad pixels in the output
        NDF to output parameters NGOOD and NBAD.
        \item No longer sets the BAD\_PIXEL flag for the DATA and
        VARIANCE components.
      \end{itemize}
   \item [\htmlref{DISPLAY}{DISPLAY}]\mbox{}
      \begin{itemize}
        \item The MODE parameter can now be set to \texttt{"Current"} to force the
        current upper and lower limits to be re-used.
      \end{itemize}
   \item [\htmlref{ERASE}{ERASE}]\mbox{}
      \begin{itemize}
        \item Now has a parameter called REPORT that indicates if an error
        should be reported if the specified object does not exist.
      \end{itemize}
   \item [\htmlref{GDCLEAR}{GDCLEAR}]\mbox{}
      \begin{itemize}
        \item Will now remove any unused space from the graphics-database file,
        thus keeping its size to a minimum.
      \end{itemize}
   \item [\htmlref{HISTOGRAM}{HISTOGRAM}]\mbox{}
      \begin{itemize}
        \item The new WIDTH parameter offers the option to specify the bin
        width instead of the number of bins.
      \end{itemize}
   \item [\htmlref{MFITTREND}{MFITTREND}]\mbox{}
      \begin{itemize}
        \item Now has a parameter called PROPBAD, which controls whether to
        propagate bad input values to the returned fit.
      \end{itemize}
   \item [\htmlref{NDFECHO}{NDFECHO}]\mbox{}
      \begin{itemize}
        \item A new parameter called EXISTS has been added that allows
        the list of displayed NDF paths to be filtered by removing the paths
        for NDFs that do not exist.
      \end{itemize}
   \item [\htmlref{NORMALIZE}{NORMALIZE}]\mbox{}
      \begin{itemize}
        \item This will loop if the first NDF is one-dimensional and the
        second is two-dimensional, provided LOOP=\texttt{TRUE}.  It
        previously only worked if the dimensionalities were in the
        reverse sense.
      \end{itemize}
   \item [\htmlref{OUTSET}{OUTSET}]\mbox{}
      \begin{itemize}
        \item The USEAXIS parameter now works, needed when the supplied NDF
        has more than two axes.
      \end{itemize}
   \item [\htmlref{PROVADD}{PROVADD}]\mbox{}
      \begin{itemize}
        \item The inoperative parameter MORE has been removed.
      \end{itemize}
   \item [\htmlref{SCATTER}{SCATTER}]\mbox{}
      \begin{itemize}
        \item Now writes the number of pixels used to form the correlation
        coefficient to output parameter NPIX.
      \end{itemize}
   \item [\htmlref{SETQUAL}{SETQUAL}]\mbox{}
      \begin{itemize}
        \item It is now possible to copy all quality information from one
        NDF to another using a new parameter called LIKE.
      \end{itemize}
   \item [\htmlref{WCSALIGN}{WCSALIGN}]\mbox{}
      \begin{itemize}
        \item The Gaussian kernel may now be applied in resampling mode as
        well as rebinning mode.
      \end{itemize}
\end{description}

\subsection{Release Notes---V2.2}

\subsubsection{Documentation Changes}
\begin{itemize}
   \item SUN/95 has been upgraded to the new style of documentation.
   Some residual collateral damage to the typesetting is likely to be
   present.
   \item Most of the old release notes have been removed from SUN/95,
   with just the few most-recent sets of notes retained in a separate
   appendix.
   \item The detailed descriptions of plotting and AST attributes are now in
   appendices.
\end{itemize}

\subsubsection{Modified Commands}
The following applications have been modified:
\begin{description}[style=nextline]

   \item [\htmlref{COLLAPSE}{COLLAPSE}]\mbox{}
      \begin{itemize}
        \item  Fixed bug in the calculation of the variance for the Sum
        estimator.  Note that this applies to other collapsing commands
        such as \htmlref{MSTATS}{MSTATS}.
      \end{itemize}

   \item [\htmlref{CONFIGECHO}{CONFIGECHO}]\mbox{}
      \begin{itemize}
        \item  A new parameter called LOGFILE has been added that allows
        the list of displayed configuration parameters to be written to a
        text file.
      \end{itemize}

   \item [\htmlref{COPYBAD}{COPYBAD}]\mbox{}
      \begin{itemize}
        \item  Restore setting the BAD\_PIXEL flag for the DATA and VARIANCE
        components, only setting it false if no bad pixels were copied
        and none existed in the input NDF.
      \end{itemize}

   \item [\htmlref{SEGMENT}{SEGMENT}]\mbox{}
      \begin{itemize}
        \item  A bug that caused a crash for NDFs with degenerate axes
        has been fixed.
      \end{itemize}

   \item [\htmlref{SETQUAL}{SETQUAL}]\mbox{}
      \begin{itemize}
        \item  A new parameter QVALUE can be used to store a constant
        integer value in the range 0 to 255 in the QUALITY component for
        all pixels.
      \end{itemize}

   \item [\htmlref{WCSALIGN}{WCSALIGN}]\mbox{}
      \begin{itemize}
        \item  A new parameter ALIGNREF can be used to control the
        co-ordinate system in which the input NDFs are aligned.
      \end{itemize}

   \item [\htmlref{WCSMOSAIC}{WCSMOSAIC}]\mbox{}
      \begin{itemize}
        \item  A new parameter ALIGNREF can be used to control the
        co-ordinate system in which the input NDFs are aligned.
      \end{itemize}

\end{description}

\subsection{Release Notes---V2.3}

\subsubsection{New Commands}
The following new applications have been added:

\begin{description}
   \item [\htmlref{NDFCOMPARE}{NDFCOMPARE}] \mbox{}
   Compares two NDFs and reports whether they are equivalent, based
   on a range of different tests.
\end{description}

\subsubsection{Modified Commands}
The following applications have been modified:
\begin{description}[style=nextline]

   \item [\htmlref{BEAMFIT}{BEAMFIT}]\mbox{}
      \begin{itemize}
        \item Now works for HEALPix maps with its apparently non-square pixels.
        \item A long-standing issue of occasional nonsense WCS errors
        has been rectified by using a better-conditioned algorithm.
      \end{itemize}

   \item [\htmlref{FITSMOD}{FITSMOD}]\mbox{}
      \begin{itemize}
        \item A missing END header may be appended using the Write mode.
        Any associated value and/or comment are ignored.  The easiest way
        to append an END header is with the wrapper
        \htmlref{FITSWRITE}{FITSWRITE}.
      \end{itemize}

   \item [\htmlref{NORMALIZE}{NORMALIZE}]\mbox{}
      \begin{itemize}
        \item Now calculates and displays Pearson's coefficient of linear
        correlation on the remaining data at every iteration.
        \item New Parameter CORR added to hold the last displayed correlation
        coefficient.
        \item New Parameters OUTSLOPE, OUTOFFSET and OUTCORR added. These
        are one-dimensional NDFs in which the slopes, offsets and correlation
        coefficients respectively are stored when operating in looping mode (\emph{i.e.}
        LOOP=\texttt{TRUE}).
      \end{itemize}

   \item [\htmlref{ROTATE}{ROTATE}]\mbox{}
      \begin{itemize}
        \item Now writes out the rotation angle actually used to an output
         parameter (ANGLEUSED).
      \end{itemize}

   \item [\htmlref{SQORST}{SQORST}]\mbox{}
      \begin{itemize}
         \item Propagates UNITS as it used to in the IMAGE-format version.
      \end{itemize}

\end{description}



\subsection{Release Notes---V2.4}

\subsubsection{New Commands}
The following new application have been added:

\begin{description}
   \item [\htmlref{ALIGN2D}{ALIGN2D}] \mbox{}
   Aligns a pair of two-dimensional NDFs by minimising the residuals between them.
\end{description}

\subsubsection{Modified Commands}
The following applications have been modified:
\begin{description}[style=nextline]

   \item [\htmlref{APERADD}{APERADD}]\mbox{}
      \begin{itemize}
        \item Has a new parameter MASK, which can be used to save an NDF
              containing a mask showing which pixels were included in the
              aperture.
      \end{itemize}

   \item [\htmlref{COLLAPSE}{COLLAPSE}]\mbox{}
      \begin{itemize}
        \item A warning that suggested that WLIM should b lowered even
              when it had the minmum of zero no longer appears.
      \end{itemize}

   \item [\htmlref{LINPLOT}{LINPLOT}]\mbox{}
      \begin{itemize}
        \item Parameter TEMPSTYLE is withdrawn.  The \texttt{+} syntax
        should be used to set temporary style changes.
      \end{itemize}

      \item [\htmlref{LUCY}{LUCY}]\mbox{}
      \begin{itemize}
        \item A bug that prevented correct background removal when
               Parameter BACK was null was excised.
      \end{itemize}

   \item [\htmlref{MFITTREND}{MFITTREND}]\mbox{}
      \begin{itemize}
        \item Has a new FOREST parameter, which improves spectral-line
              masking in line forests using a smoothed mode rather than
              the mean and a better estimate of the baseline noise.
        \item A bug has been fixed preventing fits in the rare combination
              of neither variance nor bad values being present, and without
              masking of lines.  Bad variances are also now checked
              before spline fitting.
      \end{itemize}

      \item [\htmlref{NDFCOPY}{NDFCOPY}]\mbox{}
      \begin{itemize}
        \item A bug has been fixed that prevented excess WCS axes from
              being removed.
      \end{itemize}

      \item [\htmlref{NORMALIZE}{NORMALIZE}]\mbox{}
      \begin{itemize}
        \item Has two new parameters DRAWMARK and DRAWWIDTH that can be
              used to exclude central markers and width indicator from
              the plot.
      \end{itemize}

   \item [\htmlref{PROVSHOW}{PROVSHOW}]\mbox{}
      \begin{itemize}
        \item Has a new option SHOW=\texttt{"TREE"}, which allows the family tree to
              be stepped through in an interactive manner, with the user
              choosing which parent is to be displayed next.
      \end{itemize}

   \item [\htmlref{WCSADD}{WCSADD}]\mbox{}
      \begin{itemize}
        \item Has a new option MAPTYPE=\texttt{"REFNDF"}, which causes a copy of a
              co-ordinate Frame read from a reference NDF to be added
              into the modified NDF.
        \item New Parameter RETAIN allows control over whether or not the
              new Frame becomes the current Frame in the modified NDF on
              exit.
      \end{itemize}

\end{description}

\subsection{Release Notes---V2.5}

\subsubsection{General Changes}
\begin{itemize}
   \item A log of \KAPPA\ commands can now be written to a text file
   specified by the environment variable \texttt{KAPPA\_LOG}.  The log
   lists the application name and parameter values in separate headed
   lines.  Note that the format of the log may change to simple command
   lines that could be replayed in a script.
\end{itemize}

\subsubsection{Modified Commands}
The following applications have been modified:
\begin{description}[style=nextline]

   \item [\htmlref{ALIGN2D}{ALIGN2D}]\mbox{}
      \begin{itemize}
        \item Parameter TR may also include the scale and offset in
        its seventh and eight elements.
    \item The RMS residual between the aligned and the reference arrays is
        now written to an output parameter called ``RMS''.
      \end{itemize}

   \item [\htmlref{CONFIGECHO}{CONFIGECHO}]\mbox{}
      \begin{itemize}
        \item This now reports all elements in an array, not just the
        first element.
      \end{itemize}

   \item [\htmlref{CONTOUR}{CONTOUR}]\mbox{}
      \begin{itemize}
        \item The dynamic default for Parameter LABPOS is now \texttt{{!}}
        (\emph{i.e.} a null value), so no label is now drawn in
        \texttt{"Bounds"} or \texttt"{Good"} mode unless a value is supplied
        explicitly for LABPOS.
      \end{itemize}

   \item [\htmlref{DISPLAY}{DISPLAY}]\mbox{}
      \begin{itemize}
        \item Has a new Parameter PENRANGE, which can be used to restrict
        the range of pens (i.e. colour indices) used. The default
        is to use the full range of available pens.
        \item The vertical position of the key can now be controlled
        through Parameter KEYPOS.
      \end{itemize}

   \item [\htmlref{HISCOM}{HISCOM}]\mbox{}
      \begin{itemize}
        \item Has a new Parameter APPNAME, which can be used to change
        the application name stored in the new history record from the
        default of ``HISCOM''. Scripts that generate NDFs can use this
        facility to record the details of the invocation of the script
        in the form of a history record in the output NDF.
      \end{itemize}

   \item [\htmlref{MFITTREND}{MFITTREND}]\mbox{}
      \begin{itemize}
        \item The auto method uses the median rather than the mean to
         clip outliers. This permits better masking of strong and
         extended emission.
      \end{itemize}

   \item [\htmlref{SQORTST}{SQORST}]\mbox{}
      \begin{itemize}
        \item Permit an axis scale to be retained by using an asterisk
         in Parameter PIXSCALE.
      \end{itemize}

   \item [\htmlref{TRANDAT}{TRANDAT}]\mbox{}
      \begin{itemize}
        \item Will now recognise the string \texttt{"BAD"} (case
        insensitive) within the input text file and generate appropriate
        bad values in the output NDF.
      \end{itemize}

\end{description}

\section{Release Notes---V2.5-9}

\subsection{Modified Commands}
The following applications have been modified:
\begin{description}[style=nextline]

   \item [\htmlref{ALIGN2D}{ALIGN2D}]\mbox{}
      \begin{itemize}
        \item The residuals used to determine the fit are now weighted
        using the SNR of the data rather than the reciprocal of the variance.
        This will cause the final alignment to be determined more by the
        brighter sections of the map than previously.
      \end{itemize}

   \item [\htmlref{COLLAPSE}{COLLAPSE}]\mbox{}
      \begin{itemize}
        \item There is a new estimator option, \texttt{"FastMed"},
        that offers a substantially faster calculation of unweighted
        medians.
      \end{itemize}

   \item [\htmlref{SETAXIS}{SETAXIS}]\mbox{}
      \begin{itemize}
        \item There is a new mode, \texttt{"NDF"}.  It assigns the axis
        values using the data values in another NDF, specified by a
        parameter called AXISNDF.
      \end{itemize}

   \item [\htmlref{SQORST}{SQORST}]\mbox{}
      \begin{itemize}
        \item Has a new Parameter CENTRE, which specifies the centre
        about which the WCS co-ordinates are stretched or squashed.
      \end{itemize}

   \item [\htmlref{WCSALIGN}{WCSALIGN}]\mbox{}
      \begin{itemize}
        \item The input NDFs can now be aligned with a specified POLPACK
         catalogue---see new Parameter REFCAT.
      \end{itemize}

   \item [\htmlref{WCSMOSAIC}{WCSMOSAIC}]\mbox{}
      \begin{itemize}
        \item Has a new parameter called WEIGHTS that can be used to
         specify a weight for each of the input NDFs.
      \end{itemize}

\end{description}

\end{document}

