\documentstyle[11pt]{article} 
\pagestyle{myheadings}

%------------------------------------------------------------------------------
\newcommand{\stardoccategory}  {Starlink User Note}
\newcommand{\stardocinitials}  {SUN}
\newcommand{\stardocnumber}    {17.6}
\newcommand{\stardocauthors}
   {R.\,M.\,Prestage, H.\,Meyerdierks, J.\,F.\,Lightfoot}
\newcommand{\stardocdate}      {13 January 1995}
\newcommand{\stardoctitle}
   {SPECX --- A Millimetre Wave Spectral Reduction Package}
%------------------------------------------------------------------------------

\newcommand{\stardocname}{\stardocinitials /\stardocnumber}
\renewcommand{\_}{{\tt\char'137}}     % re-centres the underscore
\markright{\stardocname}
\setlength{\textwidth}{160mm}
\setlength{\textheight}{230mm}
\setlength{\topmargin}{-2mm}
\setlength{\oddsidemargin}{0mm}
\setlength{\evensidemargin}{0mm}
\setlength{\parindent}{0mm}
\setlength{\parskip}{\medskipamount}
\setlength{\unitlength}{1mm}

%------------------------------------------------------------------------------
% Add any \newcommand or \newenvironment commands here
%------------------------------------------------------------------------------

\begin{document}
\thispagestyle{empty}
DRAL / {\sc Rutherford Appleton Laboratory} \hfill {\bf \stardocname}\\
{\large Particle Physics \& Astronomy Research Council}\\
{\large Starlink Project\\}
{\large \stardoccategory\ \stardocnumber(tex)}
\begin{flushright}
\stardocauthors\\
\stardocdate
\end{flushright}
\vspace{-4mm}
\rule{\textwidth}{0.5mm}
\vspace{5mm}
\begin{center}
{\Large\bf \stardoctitle}
\end{center}
\vspace{5mm}

\begin{center}
{\Large\it Version 6.2 (VMS) and 6.6 (Unix)}
\end{center}
\vspace{5mm}

%------------------------------------------------------------------------------
%  Add this part if you want a table of contents
\setlength{\parskip}{0mm}
\tableofcontents
\setlength{\parskip}{\medskipamount}
\markright{\stardocname}
%------------------------------------------------------------------------------

\newpage

\section {Introduction}

SPECX is a general mm and sub-mm wavelength data reduction system
written by Rachael Padman at MRAO. At the time of writing the latest
formally released version on the VAX is 6.2. However, pre-release copies
of version 6.3 are widely used. Branching off from the 6.3 VMS version,
SPECX was ported to Unix by Horst Meyerdierks of the Starlink Project
and Rachael Padman. Remo Tilanus at JACH supplied the core routines to
read GSD files. The software version count on Unix increased rather
rapidly. 6.4 was the initial Unix port of 6.3, but it existed only in
beta-test versions. The first released version was 6.5, but it lacked
the commands to read GSD files. The current version on Unix is 6.6.

Although SPECX may be used to process spectra obtained from many
different instruments, Starlink users will find it particularly useful
for the reduction of data obtained with the James Clerk Maxwell
Telescope.  SPECX has it's own command line interpreter, and manipulates
spectra on a pop-down stack.  The package is designed to work on a
graphics terminal with dual-screen facility and VT emulation ({\em e.g.}
a Pericom MG100), but can be usefully run on other graphics terminals.
More and more, X displays are used where an {\tt xterm} client (or {\tt
DECterm} on VMS) takes the role of the alpha-numeric screen and a GWM
window that of the graphics screen.  Hardcopy may be output on a number
of printers for which GKS drivers are provided by Starlink. In the Unix
version these are mainly different PostScript printers.

Some of the major features of SPECX are:
\begin{itemize}
\item
The ability to process up to eight spectra (quadrants) simultaneously:
these may have the same or different centre frequencies, resolutions, 
{\em etc.} This allows users to operate on different filter-banks, or
dual polarisation data for example, in parallel. 
\item
The ability to automatically save the current status of the system 
after each command is executed. This means that if SPECX is
stopped (unintentionally or otherwise), when restarted it will
`remember' the previous state of the program.
\item
Commands for the listing and display of spectra on a graphics
terminal, with hardcopy on a variety of printers.
\item
Single and multiple scan arithmetic, scan averaging, {\em etc.}
\item
Commands to store and retrieve intermediate spectra in storage registers.
\item
Commands to perform the fitting and removal of polynomial, 
harmonic and gaussian baselines.
\item
Commands for filtering and editing spectra.
\item
Commands to determine important line parameters (peak intensity, width, 
{\em etc.}).
\item
The ability to perform Fourier transform and power spectrum calculations.
\item
Procedures for the calibration of data.
\item
The ability to assemble a number of reduced individual spectra into a 
map file, and contour or greyscale any plane or planes of the resulting cube.
\item
The ability to write indirect command files.
\end{itemize}

A thorough description of the package is given in the ``SPECX Users' Manual''
which has been distributed as a Starlink Miscellaneous User Document
(MUD). At the time of writing this MUD is available from Starlink for
version 6.2. An extended document for version 6.3 is available from
Rachael Padman at MRAO. No document has been prepared yet for the Unix
version, and the document on 6.3 is the closest approximation for Unix
users.

SPECX is intended primarily for the analysis of JCMT data and can read
such data directly from the GSD files produced at the
telescope.
In the VMS version there are also facilities for reading data from a
range of other important mm-wave telescopes, {\em e.g.} Onsala,
FCRAO, UMASS CO survey, IRAM-FITS, JCMT RxG, NRAO 8-beam (see 
the SPECX user manual for details). 
Spectra and maps can be output as FITS files,
Starlink NDFs (VMS version only), or ASCII files for reading into other
packages.


\section {The VMS version 6.2}
\subsection {Getting started}

SPECX is made available by typing:
\begin{verbatim}
      $ specxstart
\end{verbatim}
and started by typing:
\begin{verbatim}
      $ specx 
\end{verbatim}
The program will display the current version number and
a brief informational message, then the SPECX prompt ({\tt>>}) will appear.

\begin{verbatim}

                   -------------------------------------------
                           SPECX V6.2    7th-June-1992
                             Copyright (C) R.Padman
                   -------------------------------------------

        .
        .

      >> 
\end{verbatim}

Commands may then be entered to read and process data.
SPECX commands consist of one or more keywords, separated
by hyphens. Minimum matching is applied to each keyword
individually. A full list of all valid SPECX commands
may be found by entering {\tt SHOW-COMMANDS}, with a carriage return
in response to the resulting prompt. On-line help is available
using the command {\tt HELP}.

A DCL command may be executed by preceding it with a {\tt \$}, {\em e.g.}

\begin{verbatim}
      >> $dir
\end{verbatim}

The tab and blank characters, `,', `\verb+\+' and `;' are SPECX delimiters;
to pass these to DCL the entire string must be enclosed in single 
inverted commas, {\em e.g.}

\begin{verbatim}
      >> '$dir/size scan*.dat'
\end{verbatim}

Control-C may be used to interrupt commands in progress.
Control-C typed at the command level (when the
{\tt>>} prompt is displayed) will exit the program.

The EXIT command is the normal method of returning to DCL.


\subsection {SPECX files and directories}

SPECX uses and produces a variety of files. For graphics, 
intermediate plot files (called {\tt PLOT.}nnn) are produced in the directory 
with logical name {\tt SPECXDIR}. The {\tt SPECXSTART} command assigns 
{\tt SPECXDIR} to {\tt SYS\$LOGIN} so that, 
by default, the {\tt PLOT.}nnn files will appear in 
your login directory. This may be over-ridden if you define 
{\tt SPECXDIR} as a process logical name. For example, typing

\begin{verbatim}
      $ define SPECXDIR disk$scratch:[user.data]
\end{verbatim}

after the {\tt SPECXSTART} will cause plot files to appear in the named
directory.

SPECX uses Starlink PGPLOT for graphics  and thus will produce data  files for
plots sent to some graphics devices ({\em e.g.} {\tt CANON.DAT} for the
Canon laser printer, {\tt GKS\_72.PS} for a PostScript printer). These
are created in the current default directory, and are automatically
deleted when the plot finishes.

There is an option (defaulted true) in SPECX which  allows dumping of the
current status of the program after each  command is executed. This is saved in
the file {\tt SPECX.DMP} in the  current default directory. If this file
does not exist when SPECX  is started for the first time, it will be
created. If it is
subsequently deleted (or SPECX is started from a different  directory), the
package will restart with default initialisation; otherwise SPECX will be
re-started will all flags as previously  set, data files opened, and so on. 


\subsection {Examples}

New users should read the introductory chapters of the SPECX manual,
which give explanations and examples of basic techniques. Some sample
data and a demonstration command file are also available in the
{\tt [.EXAMPLE]} sub-directory of {\tt SYS\_SPECX}. These can be copied
to the user's own
file space and tried out. The demonstration file is {\tt DEMO.SPX},
which the user should print out and work through by typing the commands
in from the terminal. It demonstrates the reading of JCMT data,
the fitting of baselines and gaussians to spectra, and shows how to make
and display a datacube. The file can also be run as an indirect command
file by an experienced user to check that the package is working correctly.


\subsection{Linking user-supplied subroutines}

User-defined commands can be added to SPECX by making use of the
subroutines EXTRNL1 to EXTRNL10. For the Starlink version of SPECX 
the steps required are:

\begin{itemize}
\item
Compile your routine:
\begin{small}
\begin{verbatim}
$ specxstart
$ @sys_specx:specxdev ! to set up logical names for include files such as SPECX
$                     ! common blocks. These common blocks should always be
$                     ! the official ones referred to by the logical names, and
$                     ! not private copies.
$ fortran extrnl1

\end{verbatim}
\end{small}
\item
Set up the logical names required to link with the NDF library.
\begin{verbatim}
$ ndf_dev
\end{verbatim}

\item
Copy the file {\tt LINKPG\_NDF.COM} from {\tt SYS\_SPECX}. 
and edit it to link
in your own version of the appropriate routine. For example, the
line:
\begin{verbatim}
$ link /exe=specx_v6-2 -
\end{verbatim}
might be replaced by:
\begin{verbatim}
$ link /exe=specx_v6-2 extrnl1, -
\end{verbatim}
\item
Copy the CLD file {\tt SYS\_SPECX:SPECX62.CLD}, and 
edit it to run your own version of {\tt SPECX\_V6-2.EXE}. 
Redefine the SPECX command by typing:
\begin{verbatim}
$ set command specx62.cld
\end{verbatim}
\end{itemize}


\section {The Unix version 6.6}
\subsection {Getting started}

SPECX is made available by typing:
\begin{verbatim}
      % specxstart
\end{verbatim}
and started by typing:
\begin{verbatim}
      % specx 
\end{verbatim}
The program will display the current version number and
a brief informational message, then the SPECX prompt ({\tt>>}) will appear.

\begin{verbatim}
                   -------------------------------------------
                     Specx v6.6 (Unix) ---- 14 December 1994
                         Copyright (C)  R.Padman & PPARC
                      Acknowledgements to Horst Meyerdierks
                   -------------------------------------------

        .
        .

      >> 
\end{verbatim}

Commands may then be entered to read and process data.
SPECX commands consist of one or more keywords, separated
by hyphens. Minimum matching is applied to each keyword
individually. A full list of all valid SPECX commands
may be found by entering {\tt show-commands}, with a carriage return
in response to the resulting prompt. On-line help is available
using the command {\tt help}. However, the on-line help is mainly a copy
from the VMS version 6.3. There are some added topics on the Unix version.

A shell command may be executed by preceding it with a {\tt \$}, {\em e.g.}

\begin{verbatim}
      >> $ls
\end{verbatim}

The tab and blank characters, `,', `\verb+\+' and `;' are SPECX delimiters;
to pass these to the shell the entire string must be enclosed in single 
inverted commas, {\em e.g.}

\begin{verbatim}
      >> '$ls -l scan*.dat'
\end{verbatim}

{\em Control-C cannot be used to interrupt commands in progress.
Control-C typed at any time will abort the program. Open files may be
corrupted.}

The {\tt exit} command is the normal method of returning to the shell.
SPECX can also be suspended in the usual way with Control-Z and later
resumed with the shell command {\tt fg}.

In some situations SPECX has to be given an end-of-file character. On
VMS this was Control-Z. On Unix it is Control-D.


\subsection {SPECX files and directories}

SPECX uses and produces a variety of files. For graphics, 
intermediate plot files (called {\tt PLOT.}nnn) are produced in the
current working directory.

SPECX uses Starlink PGPLOT for graphics and thus will produce data files for
plots sent to some graphics devices ({\em e.g.} {\tt gks74.ps}.N for a
PostScript printer). These
are created in the current default directory. They are not submitted to
any printer and are not deleted. SPECX does not even report the exact
name of the file when it is created. The user has to keep score of these
files and send them to printers as desired.

There is an option (defaulted true) in SPECX which  allows dumping of the
current status of the program after each  command is executed. This is saved in
the file {\tt specx.dmp} in the  current default directory. If this file
does not exist when SPECX  is started for the first time, it will be
created. If it is
subsequently deleted (or SPECX is started from a different  directory), the
package will restart with default initialisation; otherwise SPECX will be
re-started will all flags as previously  set, data files opened, and so on. 


\subsection {Examples}

New users should read the introductory chapters of the SPECX manual,
which give explanations and examples of basic techniques. There is a
test procedure in {\tt \$SYS\_SPECX/test.spx} and the data files
necessary are in the same directory. However, this procedure is not
commented at all. User are advised to compare it with {\tt demo.spx},
which is a copy from the VMS version and will probably not work, but has
comments as to the meaning of the commands.
{\tt demo.spx} demonstrates the reading of JCMT data,
the fitting of baselines and gaussians to spectra, and shows how to make
and display a data cube.


\subsection{Linking user-supplied subroutines}

User-defined commands can be added to SPECX by making use of the
subroutines EXTRNL1 to EXTRNL10. There is currently little support for
this and the user is left to hack her way around the SPECX directory
tree.

It is relatively easy to modify the distributed source code. This is
mainly an option for a local but site-wide version with EXTRNLn
routines. Assuming that you are a Starlink site manager you would
proceed as follows
\begin{verbatim}
      % cd /star/specx                               # 1
      % setenv INSTALL /star
      % ./mk deinstall

      % cd /star                                     # 2
      % mv specx /star/specx /star/local/specx

      % cd /star/local/specx/external                # 3
      {Make your modifications in this directory}

      % cd /star/local/specx                         # 4
      % setenv INSTALL /star/local
      % ./mk build
      % ./mk clean
      % ./mk install

      % cd /star/local/etc                           # 5
      {Modify files login and cshrc to reflect that SPECX is in /star/local}
\end{verbatim}
\begin{itemize}
\item[1] De-install the existing package from {\tt /star/bin/specx} and
   {\tt /star/help/specx}. After this step the existing package will
   still be in its built form in {\tt /star/specx}.
\item[2] Move (or copy with {\tt cp -pr}) the built system to the local
   Starlink tree. You can use any directory.
\item[3] Generate your own version of the ``external'' library. You have
   to carefully inspect the {\tt makefile} and amend it. Also check {\tt
   makefile}s in sister directories like {\tt ../fitting} to see how
   include files are handled.
\item[4] Re-build and install SPECX. Installation goes into {\tt
   /star/local/bin/specx} and {\tt /star/\-local/help/specx}. You can use
   something else instead of {\tt /star/local}.
\item[5] Amend the Starlink login files to reflect the different place
   for the SPECX installation.
\end{itemize}
In order to re-build SPECX you need a number of Starlink libraries and
include files in their usual place. Non-Starlink sites may or may not
have been issued with these libraries. As indicators, you can check
whether you have {\tt /star/lib/libhds*}, {\tt /star/include/dat\_par}
and {\tt /star/figaro/lib/libfit.a}.


\section{Data formats in, and data migration to, the Unix version}

Unix SPECX introduced yet another version of data formats (4.1) for
spectra and for maps. It can read and write disk-FITS. From version 6.6
onwards it can also read GSD files.

Version 4.1 spectral format is based on Starlink's Hierarchical Data
Structures (HDS), a binary format that is portable between different
machine architectures. Thus you can take version 4.1 spectra from a Sun
workstation to an Alpha workstation and back, without noticing it.

Version 4.1 map format, too, is based on HDS. Thus maps are portable as
well.

Data currently held on a VMS file system can be carried across to Unix
as detailed below. Note that the way back may not be possible, or at
least difficult.
\begin{itemize}
\item {\bf GSD files:}
    Data in GSD format can be copied binary. One way is the
    Unix {\tt cp} command if both disks are mounted by the Unix machine.
    Another way is {\tt ftp} in binary mode. Then use {\tt
    READ-GSD-DATA} in the usual way.

    For going back from Unix to VMS similar problems may arise as with
    disk-FITS files (see below).

\item {\bf Disk-FITS files:}
    These must be written with IBM byte order (swap order
    on Digital machines). They can then be copied binary. One way is the
    Unix {\tt cp} command if both disks are mounted by the Unix machine.
    Another way is {\tt ftp} in binary mode.

    This works between Unix machines and from VMS to Unix. If disk-FITS
    files are moved from Unix to VMS, it may be necessary to make them
    files of fixed record length 2880 byte. {\tt cp} is thus unsuitable.
    {\tt ftp} in binary mode will create the right sort of file with the
    wrong record length of 512 byte. You can either rectify this with
    some VMS command, or use a more tolerant disk-FITS reader under VMS.

\item {\bf Spectral data version 2 and 3:}
    These are VMS binary formats. Take a
    binary copy to the destination machine. The VMS versions SPECX 6.2
    and 6.3 can use it straight away, of course. For SPECX 6.6 the file
    has to be imported (converted into a version 4.1 file) with the
    command {\tt CONVERT-VAX-FILE}. This keeps the existing file (say
    {\tt file.dat}) and creates an new file ({\tt file.sdf}).

\item {\bf Spectral data version 4.0 and 4.1:}
    These are portable binary formats
    based on HDS. Just take a binary copy to the destination machine
    (Unix {\tt cp} or {\tt ftp} in binary mode). The file can be used straight
    away. Optionally you can convert it to the native flavour of the HDS
    format on the destination machine. Use the command {\tt native} in the
    Kappa package. This should speed up data access somewhat.

    Version 4.0 or 4.1 spectral data cannot be imported back into SPECX
    6.2 or 6.3.

\item {\bf Map files version 2 and 3:}
    These are VMS binary formats. Take a binary
    copy to the destination machine. The VMS versions SPECX 6.2 and 6.3
    can use it straight away, of course. For SPECX 6.6 the file has to
    be imported (converted into a version 4.1 file) with the command
    {\tt CONVERT-VAX-MAP}. This keeps the existing file (say {\tt
    file.map}) and creates an new file ({\tt file\_map.sdf}).

\item {\bf Map files version 4.0:}
    These are ``local'' binary formats written only by
    the beta test version of SPECX 6.4. ``Local'' means they can only be
    read by the same type of machine that wrote them, either a Sun4
    workstation or an Alpha workstation. These files can be converted to
    format version 4.1 with {\tt CONVERT-VAX-MAP}. In spite of the name of the
    command, the file will be assumed to have been written by the same
    Unix machine that the command is run on. The existing file (say
    {\tt file.map}) is kept and a new file ({\tt file\_map.sdf}) created.

    Version 4.0 map files cannot be imported back into SPECX 6.2 or 6.3.

\item {\bf Map files version 4.1:}
    This is a portable binary format based on HDS.
    Just take a binary copy to the destination machine (Unix {\tt cp} or
    {\tt ftp} in binary mode). The file can be used straight away.
    Optionally you can convert it to the native flavour of the HDS
    format on the destination machine. Use the command {\tt native} in the
    Kappa package. This should speed up data access somewhat.

    Version 4.1 map files cannot be imported back into SPECX 6.2 or 6.3.
\end{itemize}

Dump files and the {\tt mapplane.tmp} files are not portable. Care has to
be taken when the same file space is used with different machines,
because SPECX may not run properly on one machine while there is a SPECX
dump in the working directory that was written by another machine.


\section{Acknowledgements}

Starlink would like to acknowledge Rachael Padman for making
the SPECX package available for use, assisting in its installation,
and providing the example data.

\end{document}
