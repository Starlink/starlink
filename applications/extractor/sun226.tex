\documentclass[twoside,11pt,nolof]{starlink}

% ? Specify used packages
% ? End of specify used packages

% -----------------------------------------------------------------------------
% ? Document identification
% Fixed part
\stardoccategory    {Starlink User Note}
\stardocinitials    {SUN}
\stardocsource      {sun\stardocnumber}

% Variable part - replace [xxx] as appropriate.
\stardocnumber      {226.13}
\stardocauthors   {A.J. Chipperfield\\
                                P.W. Draper}
\stardocdate        {19th September 2014}
\stardoctitle     {EXTRACTOR\\
                                An Astronomical Source Detection Program}
\stardocmanual      {User Manual}
\startitlepic{\includegraphics[scale=0.5]{sun226_fig}}
\stardocabstract  {\EXTRACTOR\ will detect sources in
an astronomical image and build a catalogue listing them. It is based on the
popular \SExtractor\ program, has very flexible configuration
facilities and can handle images and catalogues in a variety of formats.}
% ? End of document identification

% -----------------------------------------------------------------------------
% ? Document specific \providecommand or \newenvironment commands.

\providecommand{\sexversion}{2.19.5}

\providecommand{\EXTRACTOR}{\texttt{EXTRACTOR}}
\providecommand{\CONVERT}{\texttt{CONVERT}}
\providecommand{\SExtractor}{\texttt{SExtractor}}
\providecommand{\SExtractorURL}{http://terapix.iap.fr/soft/sextractor}
\providecommand{\DUMMIESURL}{SE_Handbook.4.pdf}
\providecommand{\IRAFURL}{http://iraf.noao.edu}
\providecommand{\FITSURL}{http://fits.gsfc.nasa.gov/}
\providecommand{\MUD}{mud165.pdf}
\providecommand{\dash}{--}

% ? End of document specific commands
% -----------------------------------------------------------------------------
%  Title Page.
%  ===========
\begin{document}
\scfrontmatter

\section{\xlabel{introduction}Introduction}
\EXTRACTOR\ is a program for automatically detecting objects on an
astronomical image and building a catalogue of their properties. It is
particularly suited for the reduction of large scale galaxy-survey
data, but also performs well on other astronomical images.

\EXTRACTOR\ is Emmanuel Bertin's \SExtractor\
(Source-Extractor) program\footnote{BERTIN E. and ARNOUTS S., 1996,
A\&AS 117,393} re-packaged for use in the
\xref{Starlink Software Environment}{sg4}{} \latex{(see SG/4)}.
This means that it uses the Starlink parameter system, accepts images in
\xref{NDF}{sun33}{} format \latex{(see SUN/33)} and uses the
\xref{AST}{sun210}{} library \latex{(SUN/210)} for astrometry.

A slightly modified version of \SExtractor\ \sexversion\ is also
included in the distribution of \EXTRACTOR.
You can find details of \SExtractor's operation, and how to configure
it, in the \htmladdnormallink{\SExtractor\ User's Guide}{\MUD} (which is
issued as a Starlink Miscellaneous User Document, MUD 165\footnote{Note this
document is still in draft, consequently you should read this in
conjunction with the \htmladdnormallink{A\&AS paper}{sexpaper.ps}
and the \htmladdnormallink{earlier SExtractor user document}{sex1_doc.ps}.})
\dash\ you configure \EXTRACTOR\
in much the same way, although with
\htmlref{some restrictions}{implementation_notes}
\latex{(see Section~\ref{implementation_notes})}. The \SExtractor\ home page
-- \htmladdnormallink{\texttt{\SExtractorURL}}{\SExtractorURL} --
offers more information, such as a link to the
\htmladdnormallink{\textit{SExtractor for Dummies}}{\DUMMIESURL}.

In the Starlink release both \EXTRACTOR\ and \SExtractor\ do not offer the
galaxy model fitting features that the full \SExtractor\ builds do. This has
been done to reduce the external package dependencies required, also it isn't
clear how useful these abilities would be without also creating an NDF version
of PSFEx.

This document describes how to run \EXTRACTOR, and includes
\htmlref{instructions for running \SExtractor}{running_sextractor}
from the Starlink distribution.

\section{\xlabel{running_extractor}Running \EXTRACTOR}

Like any Starlink program \EXTRACTOR\ may be run from the
Unix shell or from a user-interface such as
\xref{ICL}{sg5}{} \latex{(see SG/5)}.
A useful summary of the rich variety of methods of specifying parameter values
is given in \xref{SUN/95}{sun95}{se_param}. C-shell users should also
consult \xref{SC/4}{sc4}{}, the C-Shell CookBook.

The examples below assume you are running from the Unix shell but the commands
and parameters are exactly the same from ICL.

After normal Starlink startup, at the shell prompt, type:
\begin{terminalv}
% extractorsetup
\end{terminalv}
to initialise the package and then:
\begin{terminalv}
% extractor
\end{terminalv}
to run the program.  You will now be prompted for the values of parameters
CONFIG and IMAGE in turn:
\begin{terminalv}
CONFIG - Configuration file /'$EXTRACTOR_DIR/config/default.sex'/ >
IMAGE - Input image /'image'/ >
\end{terminalv}
where CONFIG is the name of the `preferences' file and IMAGE is the name of
the image file to be processed.

Suggested values are given between \texttt{//} in the prompt.  You can
accept the suggested values by just typing \verb!<RETURN>!.  If you
accept the suggested values shown above, \EXTRACTOR\ will process the
NDF \texttt{"image"} using the installed default configuration
files to produce a catalogue named \texttt{test.cat} (this name is
specified in the preferences file).

CONFIG has a `default' value of
\texttt{\$EXTRACTOR\_DIR/config/default.sex}.  This file has some
sensible defaults but you will probably want to copy it to your
own directories and modify it to your taste (as you would for the
native \SExtractor\ program).  Similarly with the other configuration
files.  The environment variable \texttt{EXTRACTOR\_DIR} is defined to
point to the directory containing the \EXTRACTOR\ program and the
\texttt{config} directory.

You can also provide parameter values on the command line.
Either positionally,
\begin{terminalv}
% extractor image $EXTRACTOR_DIR/config/default.sex
\end{terminalv}
or by keyword,
\begin{terminalv}
% extractor config=$EXTRACTOR_DIR/config/default.sex
\end{terminalv}
(In this case, you would be prompted for IMAGE but not for CONFIG.)

The additional parameters KEYWORDS, NAME and VALUE allow
configuration parameters specified in the preferences file to be overridden
without editing the file. For example:
\begin{terminalv}
% extractor keywords=true
NAME - Parameter name /!/ > catalog_name
VALUE - Parameter value /!/ > sky.cat
NAME - Parameter name /!/ > catalog_type
VALUE - Parameter value /!/ > ascii_skycat
NAME - Parameter name /!/ >
CONFIG - Configuration file /'$EXTRACTOR_DIR/config/default.sex'/ >
IMAGE - Input image /'image'/ >
\end{terminalv}
would change the name and type of the catalogue produced (for this
run only). The list of changes is terminated by replying with the NULL
response (\texttt{!}) to the prompt.
The
\htmladdnormallink{\SExtractor\ User's Guide}{\MUD}\latex{ (MUD/165)}
gives a full list of possible parameter names and values, but it is only
sensible to change a few in this way.

\subsection{Running \EXTRACTOR\ from scripts}
If you'd like to run \EXTRACTOR\ from a script, just say for instance
sampling different populations of objects at different thresholds, you
can do this using the NAME VALUE parameters (avoiding the need to have
multiple configuration files), but you need to adopt a slightly
different strategy to normal programs. Here's one example script:

\newpage
\begin{center}
\textbf{Example batch script}
\end{center}
\begin{terminalv}
#!/bin/csh

#  Initialize EXTRACTOR
extractorsetup

#  Extract all objects above 1 sigma
extractor keywords=true image=image config=default.sex <<EOF
catalog_name
thresh1.cat
detect_thresh
1.0
!
EOF

# Extract all objects above 2 sigma and measure on a different image.
extractor keywords image='"detect,measure"' config=default.sex <<EOF
catalog_name
thresh2.cat
detect_thresh
2.0
!
EOF
\end{terminalv}
Using the C-shell \verb+<<EOF+ mechanism allows you to send
information to the program as if you're typing it in.

\section{\xlabel{using_images_in_nonndf_formats}Using images in non-NDF formats}
If you want to read an image in one of the
\xref{formats handled by the
\CONVERT}{sun55}{the_default_conversion_commands} package,
just start up \CONVERT\ before running \EXTRACTOR\ and specify the image with
an appropriate extension. For example:

\begin{terminalv}
% convert

   CONVERT commands are now available

   Defaults for automatic NDF conversion are set.

   Type conhelp for help on CONVERT commands.
   Type "showme sun55" to browse the hypertext documentation.

% extractor
CONFIG - Configuration file /'$EXTRACTOR_DIR/config/default.sex'/ >
IMAGE - Input image /'image'/ > image.fit
\end{terminalv}
will use the FITS file \texttt{image.fit}, automatically converting it to a
temporary NDF file and then reading the temporary file.

\section{\xlabel{running_sextractor}Running \SExtractor \label{running_sextractor}}
A native, FITS-only, version of \SExtractor\ is also distributed with the
\EXTRACTOR\ package for convenience and use in the object detection toolbox in
the \xref{GAIA}{sun214}{} application. The executable image is installed in
\texttt{\$EXTRACTOR\_DIR} and is ran as described in the
\htmladdnormallink{\SExtractor\ User's Guide}{\MUD}\latex{(MUD/165)}.

To use this version either put \texttt{\$EXTRACTOR\_DIR} on your \texttt{PATH}
or define an alias:
\begin{terminalv}
% alias sex $EXTRACTOR_DIR/sex
\end{terminalv}
then type:
\begin{quote}
\texttt{\% sex \textit{image} [-c \textit{configuration-file}]}
\texttt{[-\textit{Parameter1 Value1}] [-\textit{Parameter2 Value2}] ...}
\end{quote}

Note that if the \texttt{-c} option is omitted, file
\texttt{default.sex} in the current working directory will be used
\dash\ if you want to use the installed default file, you will need to
specify it explicitly.

\section{\xlabel{using_gaia_and_extractor}Using GAIA and \EXTRACTOR}
The Graphical Astronomy and Image Analysis Tool (GAIA -
\xref{SUN/214}{sun214}{}), has an interactive toolbox facility that
uses the \EXTRACTOR\ (and \SExtractor) programs. This allows you to
interactively adjust the detection preferences and identifies the
objects that have been detected by drawing suitable ellipses over your
image (in fact GAIA produced the image used on the title page of this
document).

GAIA also displays the results catalogue so you can view the
measurements associated with each object and vice versa (\textit{i.e.}
you can select one of the displayed ellipses and view the associated
measurements, or select a row of measurements and view the associated
object). The catalogue interface also allows you to sort and select
objects on the basis of their measurements.

\section{\label{implementation_notes}\xlabel{implementation_notes}Implementation notes}

\EXTRACTOR, will read data in floating point format.  It will
also read NDFs that contain BAD pixels. These will appear as
``missing'' regions, so any objects that are intersected will be split
into parts.

Since \EXTRACTOR\ uses the AST (\xref{SUN/211}{sun211}{}) library to
perform astrometry, it can be used with images that contain DSS
calibrations, as well as FITS-WCS and AST native ones (native
\SExtractor\ will only use images that have FITS-WCS
calibrations). The GAIA display tool is a convenient way to add such
astrometrical calibrations.

Two current limitations are that no support is provided for the use of
NDF variance arrays (although these would probably map onto the
``weight'' image concept of \SExtractor) and that the propagation of
the input NDF to the check image is not done (so WCS calibration, for
instance, is lost).

NDF pixel coordinates (\textit{i.e.} ones that include the NDF origin)
can be obtained using the \texttt{X\_PIXEL} and \texttt{Y\_PIXEL}
parameters.

The \SExtractor\ program has a very large range of processing
options. Consequently many of these options have not been tried
extensively in the \EXTRACTOR\ incarnation, so some problems may
arise, when using non-default configurations.

A \SExtractor\ facility that has not been implemented is the piping of
catalogues to standard output.  The use of weight and flag images are
not supported.

Finally as noted earlier the galaxy fitting parameters are not available.

If you come across any problems with \EXTRACTOR\ please notify
Starlink Software Support
(\htmladdnormallink{starlink@jiscmail.ac.uk}{mailto:starlink@jiscmail.ac.uk}).

\section{Isophotal radii}

\SExtractor\ can measure the areas of objects at $8$ optimally selected
thresholds (such information is often used to estimate object
profiles). These thresholds are consequently different for each
object. Traditionally this has not been the case and isophotal
areas/radii have been measured at fixed magnitude intervals above the
detection threshold. Consequently, for convenience and compatibility
with existing data and analysis methods, the \EXTRACTOR\ and Starlink
\SExtractor\ programs have been extended to provide a flexible
configuration scheme that allows this traditional behaviour to be
recovered and naturally extended.

The configuration options that control the chosen isophotal thresholds
are:
\begin{itemize}
  \item \texttt{RAD\_TYPE} and
  \item \texttt{RAD\_THRESH}
\end{itemize}
\texttt{RAD\_TYPE} can be either \texttt{SB} or \texttt{INT},
which indicate that the levels will be defined in terms of surface
brightnesses or intensities (\textit{i.e}. magnitudes and data counts),
respectively.

The \texttt{RAD\_THRESH} option can have up to three qualifying
values, depending on the value of \texttt{RAD\_TYPE}. If
\texttt{RAD\_TYPE} is \texttt{SB} then you should enter a line
consisting of:
\begin{quote}
   \texttt{RAD\_THRESH\hspace{0.5in} step[,start,zp]}
\end{quote}
in your \texttt{default.sex} file. The value \texttt{step} being the
required interval between levels, \texttt{start} being the value used
as the first threshold and \texttt{zp} the data zero point, all in
magnitudes per square arc-second. If only one value is given then the
starting point is assumed to be the analysis threshold and the zero
point is derived from the photometric value (\texttt{MAG\_ZEROPOINT}).

If a \texttt{RAD\_THRESH} value is not given then a default step of
$0.75$ magnitudes per square arcsec is used.

The actual formula used to generate the thresholds is:
\begin{quote}
    $I_{i} = A * 10^{-0.4 * (start+step*i-zp)},\ i = 0, 15$
\end{quote}
where $I_{i}$ is the $i$th threshold, $A$ is the area of an image pixel
in arcseconds, $start$, $step$ and $zp$ are as described above.

If \texttt{RAD\_TYPE} is \texttt{INT} then the correct
\texttt{RAD\_THRESH} format is:
\begin{quote}
   \texttt{RAD\_THRESH \hspace{0.5in} step[,start]}
\end{quote}
\texttt{step} being the interval between thresholds in magnitudes
and \texttt{start} being the threshold used for the first level. If
\texttt{start} is not given then the analysis threshold is used.

The formula used to generate these thresholds is:
\begin{quote}
    $I_{i} = start * 10^{-0.4*step*i},\ i = 0, 15$
\end{quote}

If no \texttt{RAD\_THRESH} values are given this time then the
APM/PISA (see \xref{SUN/109}{sun109}{}) analysis thresholds are used:
\begin{quote}
    $I_{i} = start * 2^{(i+2)},\ i = 1, 15$
\end{quote}
This gives approx $0.75$ magnitude steps ($2.5*log(2)$). The first
threshold is set to the analysis threshold.

\textbf{Notes:} before any measurements will be made at least one
of the catalogue parameters \texttt{RAD0} through \texttt{RAD15} must
be present in the \texttt{default.param} file. The minimum starting
threshold that can be used is the analysis one ---  no information about
values below this is available.

For completeness the existing \texttt{ISO0}-\texttt{ISO7} areas in
\SExtractor\ are based on an \texttt{optimal} sampling of each object
profile. Under this scheme each threshold is:
\begin{quote}
    $I_{i} = start * (\frac{I_{p}}{start})^{i/8}, \ i = 0, 7$
\end{quote}
where $I_{p}$ = peak intensity.  So you get a range of levels for each
object spanning the range from its analysis threshold to just below
the peak intensity.

\section{\xlabel{references}References}
Bailey, J.A. : \xref{SG/5}{sg5}{} : ICL \dash\ The Interactive Command Language for ADAM.\\
Bertin, E. : MUD/165 : SExtractor User's Guide (DRAFT).\\
Bertin, E. and Arnouts, S., 1996, A\&AS 117,393\\
Currie, M.J. \& Berry D.S.: \xref{SUN/95}{sun95}{} : KAPPA \dash\ Kernel Application Package.\\
Currie, M.J. \textit{et al.} : \xref{SUN/55}{sun55}{} : CONVERT \dash\ A Format-conversion Package.\\
Davenhall, A.C. : \xref{SUN/190}{sun190}{} : CURSA \dash\ Catalogue and Table Manipulation Applications.\\
Draper, P.W. \& Gray, N.: \xref{SUN/214}{sun214}{} : GAIA \dash\ Graphical Astronomy and Image Analysis Tool.\\
Lawden, M.D.: \xref{SG/4}{sg4}{} : ADAM \dash\ The Starlink Software Environment.\\
Warren-Smith, R.F. : \xref{SSN/20}{ssn20}{} : Adding Format Conversion Facilities to the NDF Data Access Library.\\
Warren-Smith, R.F. : \xref{SUN/33}{sun33}{} : NDF \dash\ Routines for Accessing the Extensible N-Dimensional Data Format.\\

\newpage
\appendix
\section{\xlabel{specification_of_extractor}Specification of \EXTRACTOR}

\small
\sstroutine{
   EXTRACTOR
}{
   Extracts sources from astronomical images.
}{
   \sstdescription{
      \EXTRACTOR\ is a program for detecting and measuring the
      properties of all the sources on an astronomical image. It
      offers a large range of configuration options for controlling
      the way that objects are detected and the measurements that are
      made of them.

      The source measurements are written to a catalogue (which can be
      of several different formats), so that they can be analysed
      (possibly by a catalogue handling package like CURSA
      -- \xref{SUN/190}{sun190}{}).

      \EXTRACTOR, is based on the \SExtractor\ program which is
      described in the \SExtractor's User Guide
      (\htmladdnormallink{MUD/165}{\MUD}). Consult this about all the various
      options that are available and for the rationale behind the
      program.
   }
   \sstusage{
      extract image config [keywords] [name] [value]
   }
   \sstparameters{
      \sstsubsection{
         IMAGE = LITERAL (Read)
      }{
        The name of the image which contains the objects you wanted
        detected and parameterised. If you have initialised the
        CONVERT package (see \xref{SUN/55}{sun55}{}) then you may
        process foreign formats, such as FITS and IRAF.

        Using this parameter you may give two image files. The
        first image will be used for detection and parameterising
        and the second will be used to actually measure the data
        values. Using this method allows you to measure the same
        objects many images, or to use a high signal to noise image
        to determine the measurement regions on a low signal to noise
        image

        [global\_data\_file]
      }
      \sstsubsection{
         CONFIG = LITERAL (Read)
      }{
        The name of the file that contains the many program
        parameters (things like the threshold for object
        detection). This is initially a file named \texttt{default.sex}
        that can be found in the directory
        \texttt{\$EXTRACTOR\_DIR/config}. To
        modify the parameters used by this program, you must take a
        copy of this file and edit it. Guidance about the values
        that parameters can take may be found in this file as well
        as in the associated \SExtractor\ documentation (see MUD/165).

        The measurements made are determined by a list of parameters
        in the file \\
        \texttt{\$EXTRACTOR\_DIR/config/default.param}. Again if
        you want measurements that are not available by default, you
        must take a copy of this file and edit it. Remember to
        also change \texttt{default.sex} to use this file (otherwise you
        will continue to use the system-wide defaults).

        One off modifications of parameters can be made using the
        KEYWORDS, NAME and VALUE parameters.

        [\texttt{\$EXTRACTOR\_DIR/config/default.sex}]
      }
      \sstsubsection{
         KEYWORDS = \_LOGICAL (Read)
      }{
       Whether you want to enter a series of parameter names and
       values interactively. If TRUE then the parameters NAME and
       VALUE are used to cyclically prompt for program parameters
       and the values you want to use. To end the cycle respond
       with a null symbol (!)

       [FALSE]
      }
      \sstsubsection{
         NAME = LITERAL (Read)
      }{
       The name of a preferences parameter that you want to
       set interactively. Respond with \texttt{!} when you have no
       more to enter.
       [!]
      }
      \sstsubsection{
         VALUE = LITERAL (Read)
      }{
        The value of the parameter you have just specified using the
        NAME prompt.
        [!]
      }
   }
}
\normalsize

\end{document}
