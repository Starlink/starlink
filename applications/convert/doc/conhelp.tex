\sstroutine{
   CONHELP
}{
   Gives help about CONVERT
}{
   \sstdescription{
      Displays help about {\sc CONVERT}.  The help information has classified
      and alphabetical lists of commands, general information about
      {\sc CONVERT} and related material; it describes individual commands in
      detail.

      Here are some of the main options:
      \begin{tabbing}
      abc \= \kill
      {\tt CONHELP} \\
      \> No parameter is given so the introduction and the top-level
         help index is displayed.
      \\
      {\tt CONHELP application/topic} \\
         \>   This gives help about the specified application or topic. \\
      {\tt CONHELP application/topic subtopic} \\
      \> \begin{minipage}[t]{143mm}
            This lists help about a subtopic of the specified
            application or topic. The hierarchy of topics has a maximum
            of four levels.
      \end{minipage}
      \\
      {\tt CONHELP SUMMARY} \\
         \>   This shows a one-line summary of each application. \\
      \end{tabbing}

      Once in the help library, it can be navigated in the normal
      way.  CTRL/Z (on VMS) and CTRL/D (on UNIX) to exit from any level,
      and $<$CR$>$ to move up a level in the hierarchy.
   }
   \sstusage{
      CONHELP [TOPIC] [SUBTOPIC] [SUBSUBTOPIC] [SUBSUBSUBTOPIC]
   }
   \sstparameters{
      \sstsubsection{
         TOPIC = LITERAL (Read)
      }{
         Topic for which help is to be given. {\tt [" "]}
      }
      \sstsubsection{
         SUBTOPIC = LITERAL (Read)
      }{
         Subtopic for which help is to be given. {\tt [" "]}
      }
      \sstsubsection{
         SUBSUBTOPIC = LITERAL (Read)
      }{
         Subsubtopic for which help is to be given. {\tt [" "]}
      }
      \sstsubsection{
         SUBSUBSUBTOPIC = LITERAL (Read)
      }{
         Subsubsubtopic for which help is to be given. {\tt [" "]}
      }
   }
   \sstimplementationstatus{
      \sstitemlist{

         \sstitem
         Uses the portable help system.

         \sstitem
         The help libraries are slightly different for VMS and UNIX.
      }
   }
}
