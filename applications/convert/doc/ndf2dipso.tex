\newpage
\sstroutine{
   NDF2DIPSO
}{
   Converts an NDF to a DIPSO-format file
}{
   \sstdescription{
      The routine converts a 1-dimensional NDF data file into a
      DIPSO-format file.  The resultant file can be imported into DIPSO
      by its READ command.  See SUN/50. The rules for the conversion
      are listed in the Notes.
   }
   \sstusage{
      ndf2dipso in out
   }
   \sstparameters{
      \sstsubsection{
         IN = NDF (Read)
      }{
         Input NDF data structure.  A file extension must not be given
         after the name.  The suggested default is the current NDF if
         one exists, otherwise it is the current value.
      }
      \sstsubsection{
         OUT = FILENAME (Write)
      }{
         Output DIPSO file.  On VMS platforms a default file extension
         of {\tt ".DAT"} is appended when parameter OUT contains no file
         extension.
      }
   }
   \sstexamples{
      \sstexamplesubsection{
         ndf2dipso old new
      }{
         This converts the NDF called old (in file old.sdf) to the
         DIPSO file called new.
      }
      \sstexamplesubsection{
         ndf2dipso spectre spectre.dat
      }{
         This converts the NDF called spectre (in file spectre.sdf) to
         the DIPSO file spectre.dat.
      }
   }
   \sstnotes{
      \sstitemlist{

         \sstitem
         The NDF TITLE object is to the DIPSO file.

         \sstitem
         The NDF data array becomes the main array in the DIPSO file.
         Bad pixels found in the NDF result in `breaks' in the DIPSO file.

         \sstitem
         The axis centres becomes the $x$-axis array in the DIPSO file.

         \sstitem
         Most NDF components are not supported by the DIPSO format,
         and therefore anything but the data array, axis centres, and
         data title will not be copied.
      }
   }
   \sstimplementationstatus{
      \sstitemlist{

         \sstitem
         If the NDF data array exceeds the DIPSO limits of 28000
         elements or 1000 breaks in the data, the application will abort
         with an appropriate error message.

         \sstitem
         If the NDF does not have a title, {\tt "Data from NDF"} is written
         as the DIPSO file's title.

         \sstitem
         If the NDF does not have an axis, the application will abort
         with an appropriate error message.
      }
   }
}
