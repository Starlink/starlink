\newpage
\sstroutine{
   DIPSO2NDF
}{
   Converts a DIPSO file (as produced by the DIPSO WRITE command) to
   an NDF
}{
   \sstdescription{
      This application routine reads a DIPSO format file as produced by
      the DIPSO `WRITE' command.  The DIPSO TITLE is written to the NDF
      TITLE.  DIPSO records bad values by means of breaks in the data
      array.  The number and positions of these breaks are stored in
      the DIPSO file.  This application inserts bad pixels at these break
      positions.  The number of bad pixels inserted is based on the
      size of the gap in the wavelength scale.  At least one bad pixel
      is inserted at every break point.
   }
   \sstusage{
      DIPSO2NDF IN OUT
   }
   \sstparameters{
      \sstsubsection{
         IN = FILENAME (Read)
      }{
         Input DIPSO file.  File extension {\tt "}.DAT{\tt "} is assumed.
      }
      \sstsubsection{
         OUT = NDF (Write)
      }{
         Output NDF data structure.  A file extension must not be given
         after the name.  It becomes the new current NDF.
      }
   }
   \sstexamples{
      \sstexamplesubsection{
         DIPSO2NDF OLD NEW
      }{
         This converts the DIPSO file OLD.DAT file to the NDF file
         NEW.SDF.
      }
      \sstexamplesubsection{
         DIPSO2NDF SPECTRE SPECTRE
      }{
         This converts the DIPSO file SPECTRE.DAT to the NDF called
         SPECTRE in file SPECTRE.SDF.
      }
   }
   \sstimplementationstatus{
      \sstitemlist{

         \sstitem
         The output NDF has a primitive data array.

         \sstitem
         The input wavelength and flux data are always of Fortran REAL
         type, the output data arrays are of HDS type \_REAL.

         \sstitem
         The application assumes that the bad-pixel padding will not
         cause the number of elements in the data array to exceed twice
         the original number.
      }
   }
}
\end{document}
