\newpage
\sstroutine{
   NDF2BDF
}{
   Converts an NDF to a Starlink Interim BDF file
}{
   \sstdescription{
      This application converts an NDF (see SUN/33) to the Bulk Data
      Frame (BDF) format used by the INTERIM Environment (see SUN/4).
      Type conversion may be performed at the same time.
   }
   \sstusage{
      NDF2BDF IN OUT [TYPE] [DESCRIP]
   }
   \sstparameters{
      \sstsubsection{
         IN = NDF (Read)
      }{
         The NDF to be converted to a BDF.  The suggested default is
         the current NDF if one exists, otherwise it is the current
         value.
      }
      \sstsubsection{
         OUT = BDF (Write)
      }{
         The name of BDF converted from the NDF.  No file extension
         should be given, as the application will automatically give
         extension {\tt ".BDF"}.
      }
      \sstsubsection{
         TYPE = LITERAL (Read)
      }{
         The data type of the output BDF.  It must be one of the
         following Interim types: {\tt "SB"}, {\tt "SW"}, {\tt "R"},
         {\tt "SL"}, {\tt "DP"}, {\tt "UB"}, {\tt "UW"} corresponding
         to signed byte, signed word, real, signed
         longword, double precision, unsigned byte, unsigned word.
         See SUN/4 for further details.  The default is the type
         corresponding to that of the NDF.  {\tt []}
      }
      \sstsubsection{
         DESCRIP = \_LOGICAL (Read)
      }{
         If true the keyword and values in a FITS extension are copied
         to the BDF's descriptors with a number of exceptions listed
         in the Notes.  {\tt [FALSE]}
      }
      \sstsubsection{
         CONNECT = FILENAME (Write)
      }{
         The Interim connection file.  It is deleted when the
         application terminates.  {\tt [NDF2BDF.TMP]}
      }
      \sstsubsection{
         COMMAND = FILENAME (Write)
      }{
         The Interim command file.  It is deleted when the application
         terminates.  {\tt [USERCOM.TMP]}
      }
   }
   \sstexamples{
      \sstexamplesubsection{
         NDF2BDF NEW OLD
      }{
         This converts the NDF called NEW (in file NEW.SDF) to the
         BDF called OLD (in file OLD.BDF).  OLD's data array will have
         the same data type as that of NEW.  The FITS header within
         NEW is converted to descriptors within OLD.
      }
      \sstexamplesubsection{
         NDF2BDF NEW OLD DESCRIP
      }{
         This converts the NDF called NEW (in file NEW.SDF) to the
         BDF called OLD (in file OLD.BDF).  OLD's data array will have
         the same data type as that of NEW.  The FITS header within
         NEW is converted to descriptors within OLD, and are reported to 
         the user.
      }
      \sstexamplesubsection{
         NDF2BDF HORSE HORSE TYPE=R
      }{
         This converts the NDF called HORSE (in file HORSE.SDF) to the
         BDF also called HORSE (in file HORSE.BDF).  The BDF's data
         array will contain 4-byte floating-point numbers.  The FITS
         header within NEW is converted to descriptors within the HORSE
         BDF.
      }
   }
   \sstnotes{
      The details of the conversion are as follows:
      \sstitemlist{

         \sstitem
            the NDF main data array is written to the BDF data array.

         \sstitem
            QUALITY, and VARIANCE have no counterparts in the BDF, and
            so cannot be propagated.

         \sstitem
            HISTORY is not propagated.

         \sstitem
            UNITS is written to descriptor BUNITS.

         \sstitem
            The number of dimensions of the data array is written
            to the BDF descriptor NAXIS, and the actual dimensions to
            NAXIS1, NAXIS2 {\it etc.}\ as appropriate.

         \sstitem
            If the NDF contains any linear axis structures the
            information necessary to generate these structures is
            written to the BDF descriptors. For example, if a linear
            AXIS(1) structure exists in the input NDF the value of the
            first data point is stored in the BDF descriptor CRVAL1,
            and the incremental value between successive axis data is
            stored in CDELT1. If there is an axis label it is written to
            descriptor CRTYPE1, and axis unit is written to CTYPE1.
            (Similarly for AXIS(2) structures {\it etc.}) FITS does not have a
            standard method of storing axis widths and variances, so these
            NDF components will not be propagated.  Non-linear axis data
            arrays cannot be represented by CRVAL$n$ and CDELT$n$, and must be
            ignored.

         \sstitem
            If the input NDF contains TITLE and LABEL components these
            are stored in the BDF descriptors TITLE and LABEL.

         \sstitem
            If the input NDF contains a FITS extension, the FITS items
            may be written to the BDF descriptors, with the following
            exceptions:
            \begin{itemize}
               \item NAXIS, and NAXIS$n$ are derived from the dimensions of
               the NDF data array as described above, so these items
               are not copied from the NDF FITS extension.
               \item The TITLE, LABEL, and BUNITS descriptors are only copied
               if no TITLE, LABEL, and UNITS NDF components have already
               been copied into these descriptors.
               \item The CDELT$n$, CRVAL$n$, CTYPE$n$, and CRTYPE$n$
               descriptors in the FITS extension are only copied if the
               input NDF contained no linear axis structures.
               \item The standard order of the FITS keywords is preserved,
               thus NAXIS and NAXIS$n$ appear immediately after the second
               card image, which should be BITPIX.  No FITS comments are
               written following the values of the above exceptions for
               compatibility with certain INTERIM applications.
               FITS-header lines with blank keywords are not copied.
            \end{itemize}

         \sstitem
            Other extensions have no BDF counterparts and therefore are
            not propagated.

         \sstitem
            All character objects longer than 70 characters are
            truncated in a BDF descriptor.
      }
   }
   \sstimplementationstatus{
      \sstitemlist{

         \sstitem
         Primitive NDFs are created.

         \sstitem
         The value of bad pixels is not written to the descriptor BLANK.
      }
   }
}

