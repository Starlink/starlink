\newpage
\sstroutine{
   DST2NDF
}{
   Converts a Figaro version 2 DST file to an NDF
}{
   \sstdescription{
      This application converts a Figaro version 2 DST file to a
      version 3 file, {\it i.e.}\ to an NDF.  The rules for converting the
      various components of a DST are listed in the notes.  Since
      both are hierarchical formats most files can be be converted with
      little or no information lost.
   }
   \sstusage{
      DST2NDF IN OUT
   }
   \sstparameters{
      \sstsubsection{
         IN = Figaro file (Read)
      }{
         The file name of the version 2 file.  A file extension must
         not be given after the name, since {\tt ".DST"} is appended by the
         application.  The file name is limited to 80 characters.
      }
      \sstsubsection{
         OUT = NDF (Write)
      }{
         The file name of the output NDF file.  A file extension must
         not be given after the name, since {\tt ".SDF"} is appended by the
         application.  Since the NDF\_ library is not used, a section
         definition may not be given following the name.  The file
         name is limited to 80 characters.
      }
   }
   \sstexamples{
      \sstexamplesubsection{
         DST2NDF OLD NEW
      }{
         This converts the Figaro file OLD.DST file to the NDF file
         NEW.SDF.
      }
      \sstexamplesubsection{
         DST2NDF HORSE HORSE
      }{
         This converts the Figaro file HORSE.DST to the NDF called
         HORSE in file HORSE.SDF.
      }
   }
   \sstnotes{
      The rules for the conversion of the various components are as
      follows:
      \begin{center}
      \begin{tabular}{|lcl|p{47mm}|}
      \hline 
      \multicolumn{1}{|c}{Figaro file} & & \multicolumn{1}{c}{NDF} &
      \multicolumn{1}{|c|}{Comments} \\ \hline
      .Z.DATA   & $\Rightarrow$ & .DATA\_ARRAY & \\
      .Z.ERRORS & $\Rightarrow$ & .VARIANCE & after processing \\
      .Z.QUALITY & $\Rightarrow$ & .QUALITY.QUALITY & must be BYTE array
                                  (see Bad-pixel handling below) \\
      & $\Rightarrow$ & .QUALITY.BADBITS = 255 & \\
      .Z.LABEL  & $\Rightarrow$ & .LABEL & \\
      .Z.UNITS  & $\Rightarrow$ & .UNITS & \\
      .Z.IMAGINARY & $\Rightarrow$ & .DATA\_ARRAY.IMAGINARY\_DATA & \\
      .Z.MAGFLAG & $\Rightarrow$ & .MORE.FIGARO.MAGFLAG & \\
      .Z.RANGE  & $\Rightarrow$ & .MORE.FIGARO.RANGE & \\
      .Z.xxxx   & $\Rightarrow$ & .MORE.FIGARO.Z.xxxx & \\
      & & & \\
      .X.DATA   & $\Rightarrow$ & .AXIS(1).DATA\_ARRAY & \\ 
      .X.ERRORS & $\Rightarrow$ & .AXIS(1).VARIANCE & after processing \\
      .X.WIDTH  & $\Rightarrow$ & .AXIS(1).WIDTH & \\
      .X.LABEL  & $\Rightarrow$ & .AXIS(1).LABEL & \\
      .X.UNITS  & $\Rightarrow$ & .AXIS(1).UNITS & \\
      .X.LOG    & $\Rightarrow$ & .AXIS(1).MORE.FIGARO.LOG & \\
      .X.xxxx   & $\Rightarrow$ & .AXIS(1).MORE.FIGARO.xxxx & \\
      & & & (Similarly for .Y .T .U .V or .W structures which are
             renamed to AXIS(2), \ldots, AXIS(6) in the NDF.) \\ \hline
      \end{tabular}
      \end{center}

      \begin{center}
      \begin{tabular}{|lcl|p{43mm}|}
      \hline 
      \multicolumn{1}{|c}{Figaro file} & & \multicolumn{1}{c}{NDF} &
      \multicolumn{1}{|c|}{Comments} \\ \hline
      .OBS.OBJECT & $\Rightarrow$ & .TITLE & \\
      .OBS.SECZ & $\Rightarrow$ & .MORE.FIGARO.SECZ & \\
      .OBS.TIME & $\Rightarrow$ & .MORE.FIGARO.TIME & \\
      .OBS.xxxx & $\Rightarrow$ & .MORE.FIGARO.OBS.xxxx & \\
      & & & \\
      .FITS.xxxx& $\Rightarrow$ & .MORE.FITS.xxxx & into value part of
         the string \\
      .COMMENTS.xxxx  & $\Rightarrow$ & .MORE.FITS.xxxx & into comment part of
         the string \\
      .FITS.xxxx.DATA & $\Rightarrow$ & .MORE.FITS.xxxx & into value part of
         the string \\
      .FITS.xxxx.DESCRIPTION & $\Rightarrow$ & .MORE.FITS.xxxx & into comment
         part of the string \\
      & & & \\
      .MORE.xxxx& $\Rightarrow$ & .MORE.xxxx & \\
      & & & \\
      .TABLE    & $\Rightarrow$ & .MORE.FIGARO.TABLE & \\
      .xxxx     & $\Rightarrow$ & .MORE.FIGARO.xxxx & \\ \hline
      \end{tabular}
      \end{center}

      Axis arrays with dimensionality greater than one are not
      supported by the NDF.  Therefore, if the application encounters
      such an axis array, it processes the array using the following
      rules, rather than those given above.
      \begin{center}
      \begin{tabular}{|lcl|p{51mm}|}
      \hline 
      \multicolumn{1}{|c}{Figaro file} & & \multicolumn{1}{c}{NDF} &
      \multicolumn{1}{|c|}{Comments} \\ \hline
      .X.DATA   & $\Rightarrow$ & .AXIS(1).MORE.FIGARO.DATA &
            AXIS(1).DATA\_ARRAY is filled with pixel co-ordinates \\
      .X.ERRORS & $\Rightarrow$ & .AXIS(1).MORE.FIGARO.VARIANCE & after
            processing \\
      .X.WIDTH  & $\Rightarrow$ & .AXIS(1).MORE.FIGARO.WIDTH & \\ \hline
      \end{tabular}
      \end{center}
   }

   \sstdiytopic{
   Bad-pixel handling
   }{
The QUALITY array is only copied if the bad-pixel flag
(.Z.FLAGGED) is false or absent.  A simple NDF with the bad-pixel
flag set to false (meaning that there are no bad-pixels present)
is created when .Z.FLAGGED is absent or false.  Otherwise a
primitive NDF, where the data array is at the top level of the
data structure, is produced.
   }
   \sstimplementationstatus{
      \sstitemlist{

      \sstitem
      Note the data array in the NDF is of the primitive form.

      \sstitem
      The maximum number of dimensions is 7.
      }
   }
}
\end{document}
