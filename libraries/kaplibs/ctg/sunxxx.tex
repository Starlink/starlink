\documentstyle[11pt]{article}
\pagestyle{myheadings}

% -----------------------------------------------------------------------------
% ? Document identification
%------------------------------------------------------------------------------
\newcommand{\stardoccategory}  {Starlink User Note}
\newcommand{\stardocinitials}  {SUN}
\newcommand{\stardocsource}    {sunxxx}
\newcommand{\stardocnumber}    {1.0}
\newcommand{\stardocauthors}   {D.S. Berry}
\newcommand{\stardocdate}      {13th September 1999}
\newcommand{\stardoctitle}     {CTG \\ [1ex]
                                Routines for Accessing Groups of Catalogues}
\newcommand{\stardocversion}   {Version 1.0 (DRAFT)}
\newcommand{\stardocmanual}    {Programmer's Manual}
% ? End of document identification
% -----------------------------------------------------------------------------

\newcommand{\stardocname}{\stardocinitials /\stardocnumber}
\markright{\stardocname}
\setlength{\textwidth}{160mm}
\setlength{\textheight}{230mm}
\setlength{\topmargin}{-2mm}
\setlength{\oddsidemargin}{0mm}
\setlength{\evensidemargin}{0mm}
\setlength{\parindent}{0mm}
\setlength{\parskip}{\medskipamount}
\setlength{\unitlength}{1mm}

% -----------------------------------------------------------------------------
%  Hypertext definitions.
%  ======================
%  These are used by the LaTeX2HTML translator in conjunction with star2html.

%  Comment.sty: version 2.0, 19 June 1992
%  Selectively in/exclude pieces of text.
%
%  Author
%    Victor Eijkhout                                      <eijkhout@cs.utk.edu>
%    Department of Computer Science
%    University Tennessee at Knoxville
%    104 Ayres Hall
%    Knoxville, TN 37996
%    USA

%  Do not remove the %\begin{rawtex} and %\end{rawtex} lines (used by
%  star2html to signify raw TeX that latex2html cannot process).
%\begin{rawtex}
\makeatletter
\def\makeinnocent#1{\catcode`#1=12 }
\def\csarg#1#2{\expandafter#1\csname#2\endcsname}

\def\ThrowAwayComment#1{\begingroup
    \def\CurrentComment{#1}%
    \let\do\makeinnocent \dospecials
    \makeinnocent\^^L% and whatever other special cases
    \endlinechar`\^^M \catcode`\^^M=12 \xComment}
{\catcode`\^^M=12 \endlinechar=-1 %
 \gdef\xComment#1^^M{\def\test{#1}
      \csarg\ifx{PlainEnd\CurrentComment Test}\test
          \let\html@next\endgroup
      \else \csarg\ifx{LaLaEnd\CurrentComment Test}\test
            \edef\html@next{\endgroup\noexpand\end{\CurrentComment}}
      \else \let\html@next\xComment
      \fi \fi \html@next}
}
\makeatother

\def\includecomment
 #1{\expandafter\def\csname#1\endcsname{}%
    \expandafter\def\csname end#1\endcsname{}}
\def\excludecomment
 #1{\expandafter\def\csname#1\endcsname{\ThrowAwayComment{#1}}%
    {\escapechar=-1\relax
     \csarg\xdef{PlainEnd#1Test}{\string\\end#1}%
     \csarg\xdef{LaLaEnd#1Test}{\string\\end\string\{#1\string\}}%
    }}

%  Define environments that ignore their contents.
\excludecomment{comment}
\excludecomment{rawhtml}
\excludecomment{htmlonly}
%\end{rawtex}

%  Hypertext commands etc. This is a condensed version of the html.sty
%  file supplied with LaTeX2HTML by: Nikos Drakos <nikos@cbl.leeds.ac.uk> &
%  Jelle van Zeijl <jvzeijl@isou17.estec.esa.nl>. The LaTeX2HTML documentation
%  should be consulted about all commands (and the environments defined above)
%  except \xref and \xlabel which are Starlink specific.

\newcommand{\htmladdnormallinkfoot}[2]{#1\footnote{#2}}
\newcommand{\htmladdnormallink}[2]{#1}
\newcommand{\htmladdimg}[1]{}
\newenvironment{latexonly}{}{}
\newcommand{\hyperref}[4]{#2\ref{#4}#3}
\newcommand{\htmlref}[2]{#1}
\newcommand{\htmlimage}[1]{}
\newcommand{\htmladdtonavigation}[1]{}

%  Starlink cross-references and labels.
\newcommand{\xref}[3]{#1}
\newcommand{\xlabel}[1]{}

%  LaTeX2HTML symbol.
\newcommand{\latextohtml}{{\bf LaTeX}{2}{\tt{HTML}}}

%  Define command to re-centre underscore for Latex and leave as normal
%  for HTML (severe problems with \_ in tabbing environments and \_\_
%  generally otherwise).
\newcommand{\latex}[1]{#1}
\newcommand{\setunderscore}{\renewcommand{\_}{{\tt\symbol{95}}}}
\latex{\setunderscore}

%  Redefine the \tableofcontents command. This procrastination is necessary
%  to stop the automatic creation of a second table of contents page
%  by latex2html.
\newcommand{\latexonlytoc}[0]{\tableofcontents}

% -----------------------------------------------------------------------------
%  Debugging.
%  =========
%  Remove % on the following to debug links in the HTML version using Latex.

% \newcommand{\hotlink}[2]{\fbox{\begin{tabular}[t]{@{}c@{}}#1\\\hline{\footnotesize #2}\end{tabular}}}
% \renewcommand{\htmladdnormallinkfoot}[2]{\hotlink{#1}{#2}}
% \renewcommand{\htmladdnormallink}[2]{\hotlink{#1}{#2}}
% \renewcommand{\hyperref}[4]{\hotlink{#1}{\S\ref{#4}}}
% \renewcommand{\htmlref}[2]{\hotlink{#1}{\S\ref{#2}}}
% \renewcommand{\xref}[3]{\hotlink{#1}{#2 -- #3}}
% -----------------------------------------------------------------------------
% ? Document specific \newcommand or \newenvironment commands.
% ? End of document specific commands
% -----------------------------------------------------------------------------
%  Title Page.
%  ===========
\renewcommand{\thepage}{\roman{page}}
\begin{document}
\thispagestyle{empty}

%  Latex document header.
%  ======================
\begin{latexonly}
   CCLRC / {\sc Rutherford Appleton Laboratory} \hfill {\bf \stardocname}\\
   {\large Particle Physics \& Astronomy Research Council}\\
   {\large Starlink Project\\}
   {\large \stardoccategory\ \stardocnumber}
   \begin{flushright}
   \stardocauthors\\
   \stardocdate
   \end{flushright}
   \vspace{-4mm}
   \rule{\textwidth}{0.5mm}
   \vspace{5mm}
   \begin{center}
   {\Huge\bf  \stardoctitle \\ [2.5ex]}
   {\LARGE\bf \stardocversion \\ [4ex]}
   {\Huge\bf  \stardocmanual}
   \end{center}
   \vspace{5mm}

% ? Heading for abstract if used.
   \vspace{10mm}
   \begin{center}
      {\Large\bf Abstract}
   \end{center}
% ? End of heading for abstract.
\end{latexonly}

%  HTML documentation header.
%  ==========================
\begin{htmlonly}
   \xlabel{}
   \begin{rawhtml} <H1> \end{rawhtml}
      \stardoctitle\\
      \stardocversion\\
      \stardocmanual
   \begin{rawhtml} </H1> \end{rawhtml}

% ? Add picture here if required.
% ? End of picture

   \begin{rawhtml} <P> <I> \end{rawhtml}
   \stardoccategory \stardocnumber \\
   \stardocauthors \\
   \stardocdate
   \begin{rawhtml} </I> </P> <H3> \end{rawhtml}
      \htmladdnormallink{CCLRC}{http://www.cclrc.ac.uk} /
      \htmladdnormallink{Rutherford Appleton Laboratory}
                        {http://www.cclrc.ac.uk/ral} \\
      \htmladdnormallink{Particle Physics \& Astronomy Research Council}
                        {http://www.pparc.ac.uk} \\
   \begin{rawhtml} </H3> <H2> \end{rawhtml}
      \htmladdnormallink{Starlink Project}{http://star-www.rl.ac.uk/}
   \begin{rawhtml} </H2> \end{rawhtml}
   \htmladdnormallink{\htmladdimg{source.gif} Retrieve hardcopy}
      {http://star-www.rl.ac.uk/cgi-bin/hcserver?\stardocsource}\\

%  HTML document table of contents.
%  ================================
%  Add table of contents header and a navigation button to return to this
%  point in the document (this should always go before the abstract \section).
  \label{stardoccontents}
  \begin{rawhtml}
    <HR>
    <H2>Contents</H2>
  \end{rawhtml}
  \renewcommand{\latexonlytoc}[0]{}
  \htmladdtonavigation{\htmlref{\htmladdimg{contents_motif.gif}}
        {stardoccontents}}

% ? New section for abstract if used.
  \section{\xlabel{abstract}Abstract}


% ? End of new section for abstract
\end{htmlonly}

% -----------------------------------------------------------------------------
% ? Document Abstract. (if used)
%   ==================

This document describes the routines provided within the CTG subroutine 
library for accessing groups of catalogues.

% ? End of document
% -----------------------------------------------------------------------------
% ? Latex document Table of Contents (if used).
%  ===========================================
 \newpage
 \begin{latexonly}
   \setlength{\parskip}{0mm}
   \latexonlytoc
   \setlength{\parskip}{\medskipamount}
   \markright{\stardocname}
 \end{latexonly}
% ? End of Latex document table of contents
% -----------------------------------------------------------------------------
\newpage
\renewcommand{\thepage}{\arabic{page}}
\setcounter{page}{1}

\section {Introduction}

If an application prompts the user for an catalogue using the facilities
of the CAT\_ system (see SUN/181), the user may only reply with the name
of a single catalogue. Some applications require several input catalogues
and the need to type in every catalogue name explicitly each time the
program is run can become time consuming. The CTG package provides a
means of giving the user the ability to specify a list (or ``Group'') of
catalogues as a reply to a single prompt for an parameter.

The current version of CTG can process catalogues which are stored in
FITS, STL or HDS format.

Additional features provided by CTG which are not currently available in
CAT:

\begin{itemize}

\item The user can include shell meta-characters within the catalogue
specifications (eg \verb+~+, \verb+$HOME+, \verb+../+, etc).

\item When accessing an existing catalogue, CAT always assumes a file type 
of \verb+.FIT+ if none is given. However, if no file type is given, CTG
will use all files with the given base name and any known catalogue file
type. Known file types are currently .FIT .fit .FITS .fits .GSC .gsc .TXT
.txt .Txt .sdf.

\item CTG allows a prefered output file type to be set. All output catalogues
will be created with the preferred file type unless a specific file type
is included in the catalogue specification. The preferred file types are
specified by environment variable CAT\_FORMATS\_OUT which has the same
syntax as NDF\_FORMATS\_OUT (see SSN/20), except that the ``native'' format
specified by ''.'' is equivalent to ``.FIT'' in the case of CTG. The file
types included in CAT\_FORMATS\_OUT should be restricted to the known file
types listed in the previous paragraph (the associated format
descriptions are currently of no significance). If CAT\_FORMATS\_OUT is not
defined, a default value of ``*,FITS(.FIT )'' is used. This causes output
catalogues to be created with the same type as the corresponding input
catalogue, or .FIT if there is no corresponding input catalogue.

\item When an output catalogue is created, CTG will first delete any existing
file with the same name (CAT reports an error if there is an existing
file with the same name).

\item The error reported if a specified catalogue cannot be accessed is
much more concise than the several lines of reports typically produced by
CAT.

\end{itemize}


\section {Interaction Between CTG and GRP}
CTG uses the facilities of the GRP package and users of CTG should be familiar
with the content of SUN/150 which describes the GRP package. Groups
created by CTG routines should be deleted when no longer needed using
GRP\_DELET.

\section {General overview of the CTG\_ system}
As a broad outline, applications use the CTG\_ package as follows:

\begin{enumerate}

\item A call is made to CTG\_ASSOC which causes the user to be prompted for a
single parameter. This parameter can be of any type. The
user replies with a ``group expression'' (see SUN/150), which contains
the names of a group of {\em existing} catalogues to be used as inputs by
the application. For instance, the group expression may be 

\begin{verbatim}
     m51_b3s1_ds,m51_b3s2_ds,m51_b3s2?_ds,^files.lis
\end{verbatim}

This is a complicated example, probably more complicated than would be used in
practice, but it highlights the facilities of the GRP and CTG packages, e.g.
wild cards (``?'' or ``$*$''), lists of files, or indirection through 
a text file (``\verb+^+''). 

The CTG\_ASSOC routine produces a list of explicit catalogue names, which
are stored internally within the GRP system.

\item What happens next depends on the application, but a common example may
be the initiation of a DO loop to loop through the input catalogues
(CTG\_ASSOC returns the total number of catalogue names in the group).

\item To access a particular catalogue, the application calls routine
CTG\_CATAS supplying an index, $n$, within the group (i.e $n$ is an
integer in the range 1 to the group size returned by CTG\_ASSOC).
CTG\_CATAS returns an catalogue identifier to the $n$th catalogue in the
group. This identifier can then be used to access the catalogue in the
normal manner using the CAT\_ routines (SUN/181). The identifier should be
annulled when it is no longer needed using CAT\_TRLSE in the normal way.

\item Once the application has finished processing the group of catalogues,
it calls GRP\_DELET which deletes the group, releasing all resources
reserved by the group.

\item Routine CTG\_ASSOC can also be used to append a list of catalogue names 
obtained from the environment, to a previously defined group.

The routine CTG\_CREAT produces a group containing the names of
catalogues which are to be created by the application. The routine
CTG\_CATCR will create a new catalogue with a name given by a group
member, and returns an catalogue identifier to it. 

The names of output catalogues given by users usually relate to the input
catalogue names. When CTG\_CREAT is called, it creates a group of
catalogue names either by modifying all the names in a specified input
group using a ``modification element'' (see SUN/150), or by getting a
list of new names from the user.

\item Applications which produce a group of output catalogues could also
produce a text file holding the names of the output catalogues. Such a
file can be used as input to the next application, using the indirection
facility. A text file listing of all the catalogues in a group can be
produced by routine GRP\_LIST (or GRP\_LISTF).

\end{enumerate}

See the detailed descriptions of CTG\_ASSOC and CTG\_CREAT below for details 
of the processing of existing and new catalogue names.

\appendix
% Command for displaying routines in routine lists:
% =================================================

\newcommand{\noteroutine}[2]{{\small \bf #1} \\
                              \hspace*{3em} {\em #2} \\[1.5ex]}

\begin{htmlonly}
\renewcommand{\noteroutine}[3]{
\begin{description}
\item [{\small \bf {#1}{#2}}] 
{\em #3}
\end{description}
}

\end{htmlonly}

\section {Routine descriptions}

\subsection {Routine list}

\noteroutine{CTG\_ASSO1( PARAM, MODE, CI, FIELDS, STATUS )}{
      Obtain an identifier for a single existing catalogue using a specified
      parameter.}
\noteroutine{CTG\_ASSOC( PARAM, IGRP, SIZE, FLAG, STATUS )}{
      Store names of existing catalogues specified through the environment.}
\noteroutine{CTG\_CREA1( PARAM, CI, NAME, STATUS )}{
      Create a single new catalogue using a specified parameter.}
\noteroutine{CTG\_CREAT( PARAM, IGRP0, IGRP, SIZE, FLAG, STATUS )}{
      Obtain the names of a group of catalogue to be created from the
      environment.}
\noteroutine{CTG\_GTSUP( IGRP, I, FIELDS, STATUS )}{
      Get supplemental information for an catalogue.}
\noteroutine{CTG\_CATAS( IGRP, INDEX, MODE, CI, STATUS )}{
      Obtain an catalogue identifier for an existing catalogue.}
\noteroutine{CTG\_CATCR( IGRP, INDEX, CI, STATUS )}{
      Obtain an catalogue identifier for a new catalogue.}
\noteroutine{CTG\_PTSUP( IGRP, I, FIELDS, STATUS )}{
      Store suplemental information for an catalogue.}
\noteroutine{CTG\_SETSZ( IGRP, SIZE, STATUS )}{
      Reduces the size of an CTG group.}
\subsection {Full Routine Specifications}
\label {SEC:FULLSPEC}

% +
%  Name:
%     SST.TEX

%  Purpose:
%     Define LaTeX commands for laying out Starlink routine descriptions.

%  Language:
%     LaTeX

%  Type of Module:
%     LaTeX data file.

%  Description:
%     This file defines LaTeX commands which allow routine documentation
%     produced by the SST application PROLAT to be processed by LaTeX and
%     by LaTeX2HTML. The contents of this file should be included in the
%     source prior to any statements that make use of the SST commands.

%  Notes:
%     The commands defined in the style file html.sty provided with LaTeX2html
%     are used. These should either be made available by using the appropriate
%     sun.tex (with hypertext extensions) or by putting the file html.sty
%     on your TEXINPUTS path (and including the name as part of the
%     documentstyle declaration).

%  Authors:
%     RFWS: R.F. Warren-Smith (STARLINK)
%     PDRAPER: P.W. Draper (Starlink - Durham University)

%  History:
%     10-SEP-1990 (RFWS):
%        Original version.
%     10-SEP-1990 (RFWS):
%        Added the implementation status section.
%     12-SEP-1990 (RFWS):
%        Added support for the usage section and adjusted various spacings.
%     8-DEC-1994 (PDRAPER):
%        Added support for simplified formatting using LaTeX2html.
%     {enter_further_changes_here}

%  Bugs:
%     {note_any_bugs_here}

% -

%  Define length variables.
\newlength{\sstbannerlength}
\newlength{\sstcaptionlength}
\newlength{\sstexampleslength}
\newlength{\sstexampleswidth}

%  Define a \tt font of the required size.
\newfont{\ssttt}{cmtt10 scaled 1095}

%  Define a command to produce a routine header, including its name,
%  a purpose description and the rest of the routine's documentation.
\newcommand{\sstroutine}[3]{
   \goodbreak
   \rule{\textwidth}{0.5mm}
   \vspace{-7ex}
   \newline
   \settowidth{\sstbannerlength}{{\Large {\bf #1}}}
   \setlength{\sstcaptionlength}{\textwidth}
   \setlength{\sstexampleslength}{\textwidth}
   \addtolength{\sstbannerlength}{0.5em}
   \addtolength{\sstcaptionlength}{-2.0\sstbannerlength}
   \addtolength{\sstcaptionlength}{-5.0pt}
   \settowidth{\sstexampleswidth}{{\bf Examples:}}
   \addtolength{\sstexampleslength}{-\sstexampleswidth}
   \parbox[t]{\sstbannerlength}{\flushleft{\Large {\bf #1}}}
   \parbox[t]{\sstcaptionlength}{\center{\Large #2}}
   \parbox[t]{\sstbannerlength}{\flushright{\Large {\bf #1}}}
   \begin{description}
      #3
   \end{description}
}

%  Format the description section.
\newcommand{\sstdescription}[1]{\item[Description:] #1}

%  Format the usage section.
\newcommand{\sstusage}[1]{\item[Usage:] \mbox{} \\[1.3ex] {\ssttt #1}}


%  Format the invocation section.
\newcommand{\sstinvocation}[1]{\item[Invocation:]\hspace{0.4em}{\tt #1}}

%  Format the arguments section.
\newcommand{\sstarguments}[1]{
   \item[Arguments:] \mbox{} \\
   \vspace{-3.5ex}
   \begin{description}
      #1
   \end{description}
}

%  Format the returned value section (for a function).
\newcommand{\sstreturnedvalue}[1]{
   \item[Returned Value:] \mbox{} \\
   \vspace{-3.5ex}
   \begin{description}
      #1
   \end{description}
}

%  Format the parameters section (for an application).
\newcommand{\sstparameters}[1]{
   \item[Parameters:] \mbox{} \\
   \vspace{-3.5ex}
   \begin{description}
      #1
   \end{description}
}

%  Format the examples section.
\newcommand{\sstexamples}[1]{
   \item[Examples:] \mbox{} \\
   \vspace{-3.5ex}
   \begin{description}
      #1
   \end{description}
}

%  Define the format of a subsection in a normal section.
\newcommand{\sstsubsection}[1]{ \item[{#1}] \mbox{} \\}

%  Define the format of a subsection in the examples section.
\newcommand{\sstexamplesubsection}[2]{\sloppy
\item[\parbox{\sstexampleslength}{\ssttt #1}] \mbox{} \\ #2 }

%  Format the notes section.
\newcommand{\sstnotes}[1]{\item[Notes:] \mbox{} \\[1.3ex] #1}

%  Provide a general-purpose format for additional (DIY) sections.
\newcommand{\sstdiytopic}[2]{\item[{\hspace{-0.35em}#1\hspace{-0.35em}:}] \mbox{} \\[1.3ex] #2}

%  Format the implementation status section.
\newcommand{\sstimplementationstatus}[1]{
   \item[{Implementation Status:}] \mbox{} \\[1.3ex] #1}

%  Format the bugs section.
\newcommand{\sstbugs}[1]{\item[Bugs:] #1}

%  Format a list of items while in paragraph mode.
\newcommand{\sstitemlist}[1]{
  \mbox{} \\
  \vspace{-3.5ex}
  \begin{itemize}
     #1
  \end{itemize}
}

%  Define the format of an item.
\newcommand{\sstitem}{\item}

%  Now define html equivalents of those already set. These are used by
%  latex2html and are defined in the html.sty files.
\begin{htmlonly}

%  Re-define \ssttt.
   \newcommand{\ssttt}{\tt}

%  \sstroutine.
   \renewcommand{\sstroutine}[3]{
      \subsection{#1\xlabel{#1}-\label{#1}#2}
      \begin{description}
         #3
      \end{description}
   }

%  \sstdescription
   \renewcommand{\sstdescription}[1]{\item[Description:]
      \begin{description}
         #1
      \end{description}
   }

%  \sstusage
   \renewcommand{\sstusage}[1]{\item[Usage:]
      \begin{description}
         {\ssttt #1}
      \end{description}
   }

%  \sstinvocation
   \renewcommand{\sstinvocation}[1]{\item[Invocation:]
      \begin{description}
         {\ssttt #1}
      \end{description}
   }

%  \sstarguments
   \renewcommand{\sstarguments}[1]{
      \item[Arguments:]
      \begin{description}
         #1
      \end{description}
   }

%  \sstreturnedvalue
   \renewcommand{\sstreturnedvalue}[1]{
      \item[Returned Value:]
      \begin{description}
         #1
      \end{description}
   }

%  \sstparameters
   \renewcommand{\sstparameters}[1]{
      \item[Parameters:]
      \begin{description}
         #1
      \end{description}
   }

%  \sstexamples
   \renewcommand{\sstexamples}[1]{
      \item[Examples:]
      \begin{description}
         #1
      \end{description}
   }

%  \sstsubsection
   \renewcommand{\sstsubsection}[1]{\item[{#1}]}

%  \sstexamplesubsection
   \renewcommand{\sstexamplesubsection}[2]{\item[{\ssttt #1}] \\ #2}

%  \sstnotes
   \renewcommand{\sstnotes}[1]{\item[Notes:]
      \begin{description}
         #1
      \end{description}
   }

%  \sstdiytopic
   \renewcommand{\sstdiytopic}[2]{\item[{#1}]
      \begin{description}
         #2
      \end{description}
   }

%  \sstimplementationstatus
   \renewcommand{\sstimplementationstatus}[1]{\item[Implementation Status:]
      \begin{description}
         #1
      \end{description}
   }

%  \sstitemlist
   \newcommand{\sstitemlist}[1]{
      \begin{itemize}
         #1
      \end{itemize}
   }
\end{htmlonly}

%  End of "sst.tex" layout definitions.
% .
% @(#)sst.tex   1.4   95/06/06 11:46:41   95/06/06 11:49:58

% Routine descriptions:
% =====================
\small

\newpage
\sstroutine{
   CTG\_ASSO1
}{
   Obtain an identifier for a single existing catalogue using a
   specified parameter
}{
   \sstdescription{
      This routine is equivalent to CAT\_ASSOC except that it allows the
      catalogue to be specified using a GRP group expression (for instance,
      its name may be given within a text file, etc). The first catalogue in
      the group expression is returned. Any other names in the group
      expression are ignored. Supplemental information describing the
      separate fields in the catalogue specification are also returned.
   }
   \sstinvocation{
      CALL CTG\_ASSO1( PARAM, MODE, CI, FIELDS, STATUS )
   }
   \sstarguments{
      \sstsubsection{
         PARAM = CHARACTER $*$ ( $*$ ) (Given)
      }{
         Name of the ADAM parameter.
      }
      \sstsubsection{
         MODE = CHARACTER $*$ ( $*$ ) (Given)
      }{
         Type of catalogue access required: {\tt '}READ{\tt '}, {\tt '}UPDATE{\tt '} or {\tt '}WRITE{\tt '}.
      }
      \sstsubsection{
         CI = INTEGER (Returned)
      }{
         catalogue identifier.
      }
      \sstsubsection{
         FIELDS( 5 ) = CHARACTER $*$ ( $*$ ) (Given)
      }{
         Each element contains the following on exit:

            1 - FITS extension specification (eg {\tt "}\{3\}{\tt "}) if any
            2 - File type
            3 - Base file name
            4 - Directory path
            5 - Full catalogue specification
      }
      \sstsubsection{
         STATUS = INTEGER (Given and Returned)
      }{
         The global status.
      }
   }
}
\newpage
\sstroutine{
   CTG\_ASSOC
}{
   Store names of existing catalogues specified through the environment
}{
   \sstdescription{
      A group expression is obtained from the environment using the
      supplied parameter. The expression is parsed (using the facilities
      of the GRP routine GRP\_GROUP, see SUN/150) to produce a list of
      explicit names for existing catalogues which are appended to the
      end of the supplied group (a new group is created if none is
      supplied). If an error occurs while parsing the group expression,
      the user is re-prompted for a new group expression. CAT identifiers
      for particular members of the group can be obtained using
      CTG\_CATAS.
   }
   \sstinvocation{
      CALL CTG\_ASSOC( PARAM, IGRP, SIZE, FLAG, STATUS )
   }
   \sstarguments{
      \sstsubsection{
         PARAM = CHARACTER$*$($*$) (Given)
      }{
         The parameter with which to associate the group expression.
      }
      \sstsubsection{
         IGRP = INTEGER (Given and Returned)
      }{
         The identifier of the group in which the catalogue names are to be
         stored. A new group is created if the supplied value is GRP\_\_NOID.
         It should be deleted when no longer needed using GRP\_DELET.
      }
      \sstsubsection{
         SIZE = INTEGER (Returned)
      }{
         The total number of catalogue names in the returned group.
      }
      \sstsubsection{
         FLAG = LOGICAL (Returned)
      }{
         If the group expression was terminated by the GRP {\tt "}flag
         character{\tt "}, then FLAG is returned true. Otherwise it is
         returned false. Returned .FALSE. if an error occurs.
      }
      \sstsubsection{
         STATUS = INTEGER (Given and Returned)
      }{
         The global status.
      }
   }
   \sstnotes{
      \sstitemlist{

         \sstitem
         Any file names containing wildcards are expanded into a list of
         catalogue names. The supplied strings are intepreted by a shell
         (/bin/tcsh if it exists, otherwise /bin/csh, otherwise /bin/sh),
         and so may contain shell meta-characters (eg twiddle, \$HOME, even
         command substitution and pipes - but pipe characters {\tt "}$|${\tt "} need to be
         escaped using a backslash {\tt "}$\backslash${\tt "} to avoid them being interpreted as
         GRP editing characters).

         \sstitem
         All possible catalogues matching the supplied string
         are included in the returned group. For instance, if the string
         {\tt "}fred{\tt "} is supplied, then the returned group will contain references
         to all files with basename fred which any of the file types: .FIT
         .fit .FITS .fits .GSC .gsc .TXT .txt .Txt .sdf

         \sstitem
         Names of catalogues stored in FITS format may include an FITS
         extension number. For instance, {\tt "}/home/dsb/mydata.fit\{3\} refers to
         a catalogue stored in the third extension of the FITS file
         myadata.fit.

         \sstitem
         Catalogues stored in HDS format must be stored as the top level
         object within the .sdf file.

         \sstitem
         All matching files are opened in order to ensure that they are
         valid catalogues. The user is notified if there are no valid
         catalogues matching a supplied name, and they are asked to supply a
         replacement parameter value.

         \sstitem
         Each element in the returned group contains a full specification
         for a catalogue. Several other groups are created by this routine,
         and are associated with the returned group by means of a GRP
         {\tt "}owner-slave{\tt "} relationship. These supplemental groups are
         automatically deleted when the returned group is deleted using
         GRP\_DELET. The returned group should not be altered using GRP
         directly because corresponding changes may need to be made to the
         supplemental groups. Routines CTG\_SETSZ, CTG\_GTSUP and CTG\_PTSUP
         are provided to manipulate the entire chain of groups. The full
         chain (starting from the head) is as follows:

         \sstitem
            FITS extension numbers (if any)

         \sstitem
            File types

         \sstitem
            Base file names

         \sstitem
            Directory paths

         \sstitem
            Full catalogue specification (this is the returned group IGRP)

         \sstitem
         If an error is reported the group is returned unaltered. If no
         group is supplied, an empty group is returned.

         \sstitem
         A null value (!) can be given for the parameter to indicate that
         no more catalogues are to be specified. The corresponding error is
         annulled before returning unless no catalogues have been added to
         the group.

         \sstitem
         If the last character in the supplied group expression is a colon
         (:), a list of the catalogues represented by the group expression
         (minus the colon) is displayed, but none are actually added to the
         group. The user is then re-prompted for a new group expression.
      }
   }
}
\newpage
\sstroutine{
   CTG\_CATAS
}{
   Obtain a CAT identifier for an existing catalogue
}{
   \sstdescription{
      The routine returns a CAT identifier for an existing catalogue. The
      name of the catalogue is held at a given index within a given group.
      It is equivalent to CAT\_ASSOC.
   }
   \sstinvocation{
      CALL CTG\_CATAS( IGRP, INDEX, MODE, CI, STATUS )
   }
   \sstarguments{
      \sstsubsection{
         IGRP = INTEGER (Given)
      }{
         A GRP identifier for a group holding the names of catalogues. This
         will often be creted using CTG\_ASSOC, but groups created {\tt "}by
         hand{\tt "} using GRP directly (i.e. without the supplemental groups
         created by CTG\_ASSOC) can also be used.
      }
      \sstsubsection{
         INDEX = INTEGER (Given)
      }{
         The index within the group at which the name of the catalogue to be
         accessed is stored.
      }
      \sstsubsection{
         MODE = CHARACTER $*$ ( $*$ ) (Given)
      }{
         Type of catalogue access required: {\tt '}READ{\tt '}, or {\tt '}WRITE{\tt '}.
      }
      \sstsubsection{
         CI = INTEGER (Returned)
      }{
         catalogue identifier.
      }
      \sstsubsection{
         STATUS = INTEGER (Given and Returned)
      }{
         The global status.
      }
   }
   \sstnotes{
      \sstitemlist{

         \sstitem
         If this routine is called with STATUS set, then a value of
         CAT\_\_NOID will be returned for the CI argument, although no
         further processing will occur. The same value will also be
         returned if the routine should fail for any reason. The CAT\_\_NOID
         constant is defined in the include file CAT\_PAR.
      }
   }
}
\newpage
\sstroutine{
   CTG\_CATCR
}{
   Obtain a CAT identifier for a new catalogue
}{
   \sstdescription{
      The routine returns a CAT identifier for a new catalogue. The name
      of the new catalogue is held at a given index within a given group.
      It is equivalent to CAT\_CREAT, except that any existing catalogue
      with the specified name is first deleted (unless the catalogue
      specification includes a FITS extension specifier).
   }
   \sstinvocation{
      CALL CTG\_CATCR( IGRP, INDEX, CI, STATUS )
   }
   \sstarguments{
      \sstsubsection{
         IGRP = INTEGER (Given)
      }{
         A GRP identifier for a group holding the names of catalogues. This
         will often be created using CTG\_CREAT, but groups created {\tt "}by
         hand{\tt "} using GRP directly can also be used.
      }
      \sstsubsection{
         INDEX = INTEGER (Given)
      }{
         The index within the group at which the name of the catalogue to be
         created is stored.
      }
      \sstsubsection{
         CI = INTEGER (Returned)
      }{
         Catalogue identifier.
      }
      \sstsubsection{
         STATUS = INTEGER (Given and Returned)
      }{
         The global status.
      }
   }
}
\newpage
\sstroutine{
   CTG\_CREA1
}{
   Create a single new catalogue using a specified parameter
}{
   \sstdescription{
      This routine is equivalent to CAT\_CREAT except that it allows the
      catalogue to be specified using a GRP group expression (for
      instance, its name may be given within a text file, etc), and it
      also ensures that any existing catalogue with the same name is
      deleted before the new one is created (so long as no FITS extension
      number is included in the catalogue specification). The first
      catalogue in the group expression is returned. Any other names in
      the group expression are ignored. Any modification elements in the
      supplied group expression will be treated literally.
   }
   \sstinvocation{
      CALL CTG\_CREA1( PARAM, CI, NAME, STATUS )
   }
   \sstarguments{
      \sstsubsection{
         PARAM = CHARACTER $*$ ( $*$ ) (Given)
      }{
         Name of the ADAM parameter.
      }
      \sstsubsection{
         CI = INTEGER (Returned)
      }{
         Catalogue identifier.
      }
      \sstsubsection{
         NAME = CHARACTER $*$ ( $*$ ) (Returned)
      }{
         The file specification of the catalogue.
      }
      \sstsubsection{
         STATUS = INTEGER (Given and Returned)
      }{
         The global status.
      }
   }
}
\newpage
\sstroutine{
   CTG\_CREAT
}{
   Obtain the names of a group of catalogue to be created from the
   environment
}{
   \sstdescription{
      A group expression is obtained from the environment using the
      supplied parameter. The expression is parsed (using the
      facilities of the GRP routine GRP\_GROUP, see SUN/150) to produce
      a list of explicit catalogue names. These names are appended
      to the group identified by IGRP. The user is re-prompted if an
      error occurs while parsing the group expression. If IGRP has the
      value GRP\_\_NOID on entry, then a new group is created and IGRP is
      returned holding the new group identifier.

      If IGRP0 holds a valid group identifier on entry, then the group
      identified by IGRP0 is used as the basis for any modification
      element contained in the group expression obtained from the
      environment. If IGRP0 holds an invalid identifier (such as
      GRP\_\_NOID) on entry then modification elements are included
      literally in the output group.
   }
   \sstinvocation{
      CALL CTG\_CREAT( PARAM, IGRP0, IGRP, SIZE, FLAG, STATUS )
   }
   \sstarguments{
      \sstsubsection{
         PARAM = CHARACTER$*$($*$) (Given)
      }{
         The parameter with which to associate the group.
      }
      \sstsubsection{
         IGRP0 = INTEGER (Given)
      }{
         The GRP identifier for the group to be used as the basis for
         any modification elements. If a valid GRP identifier is
         supplied, and if the supplied group expression contains a
         modification element, then:

         \sstitemlist{

            \sstitem
            the basis token (an asterisk) is replaced by the file basename
            associated with the corresponding element of the basis group (the
            {\tt "}basis catalogue{\tt "}).

            \sstitem
            if no directory specification is included in the group expression,
            the directory specification associated with the basis catalogue is
            used.

         }
         The supplied group will often be created by CTG\_ASSOC, but
         groups created {\tt "}by hand{\tt "} using GRP directly can also be used
         (i.e. without the supplemental groups created by CTG). In
         this case, there are no defaults for directory path or file type,
         and the basis token ({\tt "}$*${\tt "}) in the group expression represents the
         full basis file specification supplied in IGRP0, not just the file
         basename.
      }
      \sstsubsection{
         IGRP = INTEGER (Given and Returned)
      }{
         The GRP identifier for the group to which the supplied
         files are to be appended.
      }
      \sstsubsection{
         SIZE = INTEGER (Returned)
      }{
         The total number of file names in the returned group.
      }
      \sstsubsection{
         FLAG = LOGICAL (Returned)
      }{
         If the group expression was terminated by the GRP {\tt "}flag{\tt "}
         character, then FLAG is returned .TRUE. Otherwise it is
         returned .FALSE. Returned .FALSE. if an error occurs.
      }
      \sstsubsection{
         STATUS = INTEGER (Given and Returned)
      }{
         The global status.
      }
   }
   \sstnotes{
      \sstitemlist{

         \sstitem
         Any FITS extensions specified in the group expression are ignored.

         \sstitem
         If an error is reported the group is returned unaltered.

         \sstitem
         A null value (!) can be given for the parameter to indicate
         that no more catalogues are to be specified. The corresponding error
         is annulled before returning unless no catalogues have been added to
         the group.

         \sstitem
         If no file type is supplied in the group expression, then the first
         file type listed in the current value of the CAT\_FORMATS\_OUT environment
         variable is used. If this is {\tt "}$*${\tt "} then the file type is copied from the
         corresponding input file if a modification element was used to specify
         the output file name (if the catalogue was not specified by a
         modification element, the second file type in CAT\_FORMATS\_OUT is
         used).

         \sstitem
         If the last character in the supplied group expression is
         a colon (:), a list of the catalogues represented by the group
         expression (minus the colon) is displayed, but none are
         actually added to the group. The user is then re-prompted for
         a new group expression.

         \sstitem
         The returned group has no associated groups holding supplemental
         information (unlike the group returned by CTG\_ASSOC).
      }
   }
}
\newpage
\sstroutine{
   CTG\_GTSUP
}{
   Get supplemental information for a catalogue
}{
   \sstdescription{
      Returns the supplemental information associated with a given entry
      in a CTG group.
   }
   \sstinvocation{
      CALL CTG\_GTSUP( IGRP, I, FIELDS, STATUS )
   }
   \sstarguments{
      \sstsubsection{
         IGRP = INTEGER (Given)
      }{
         The CTG group as returned by CTG\_ASSOC, etc. This should be the last
         group in a GRP owner-slave chain.
      }
      \sstsubsection{
         I = INTEGER (Given)
      }{
         The index of the required entry.
      }
      \sstsubsection{
         FIELDS( 5 ) = CHARACTER $*$ ( $*$ ) (Returned)
      }{
         The supplemental information associated with the entry specified
         by I. Each element of the returned array contains the following:

            1 - FITS extension specification (eg {\tt "}\{3\}{\tt "}) if any
            2 - File type
            3 - Base file name
            4 - Directory path
            5 - Full catalogue specification

         This information is obtained from a set of groups associated with
         the supplied group IGRP by means of a chain of GRP {\tt "}owner-slave{\tt "}
         relationships. If any of these groups do not exist, the correponding
         elements of the above array are returned blank. Note, element 5,
         the full catalogue specification, is obtained directly from the
         supplied group IGRP.
      }
      \sstsubsection{
         STATUS = INTEGER (Given and Returned)
      }{
         The global status.
      }
   }
}
\newpage
\sstroutine{
   CTG\_PTSUP
}{
   Store suplemental information for an catalogue
}{
   \sstdescription{
      Stores the supplied items of supplemental information for a given
      entry in a CTG group. The GRP groups needed to store this
      supplemental information are created if they do not already exist,
      and associated with the supplied group by means of a chain of GRP
      {\tt "}owner-slave{\tt "} relationships. They will be deleted automaticaly when
      the supplied group is deleted using GRP\_DELET.
   }
   \sstinvocation{
      CALL CTG\_PTSUP( IGRP, I, FIELDS, STATUS )
   }
   \sstarguments{
      \sstsubsection{
         IGRP = INTEGER (Given)
      }{
         The CTG group as returned by CTG\_ASSOC, etc. This should be the last
         group in a GRP owner-slave chain.
      }
      \sstsubsection{
         I = INTEGER (Given)
      }{
         The index of the required entry.
      }
      \sstsubsection{
         FIELDS( 5 ) = CHARACTER $*$ ( $*$ ) (Given)
      }{
         The supplemental information to be stored with the entry specified
         by I. Each element of the supplied array should contain the
         following:
            1 - FITS extension (eg {\tt "}\{3\}{\tt "}) if any
            2 - File type
            3 - Base file name
            4 - Directory path
            5 - Full catalogue specification
      }
      \sstsubsection{
         STATUS = INTEGER (Given and Returned)
      }{
         The global status.
      }
   }
}
\newpage
\sstroutine{
   CTG\_SETSZ
}{
   Reduces the size of a CTG group
}{
   \sstdescription{
      This routine should be used instead of GRP\_SETSZ to set the size of
      a group created by CTG. It sets the size of the supplied group, and
      also sets the size of each of the supplemental groups associated with
      the supplied group.
   }
   \sstinvocation{
      CALL CTG\_SETSZ( IGRP, SIZE, STATUS )
   }
   \sstarguments{
      \sstsubsection{
         IGRP = INTEGER (Given)
      }{
         The CTG group as returned by CTG\_ASSOC, etc. This should be the last
         group in a GRP owner-slave chain.
      }
      \sstsubsection{
         SIZE = INTEGER (Given)
      }{
         The new group size. Must be less than or equal to the size of the
         smallest group in the chain.
      }
      \sstsubsection{
         STATUS = INTEGER (Given and Returned)
      }{
         The global status.
      }
   }
}

\end{document}
