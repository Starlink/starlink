\documentclass[11pt,nolof]{starlink}

% -----------------------------------------------------------------------------
% ? Document identification
\stardoccategory    {Starlink User Note}
\stardocinitials    {SUN}
\stardocsource      {sun\stardocnumber}
\stardocnumber      {191.2}
\stardocauthors     {M. J. Bly \& P. W. Draper}
\stardocdate        {7 April 2008}
\stardoctitle     {SAE --- Starlink Applications Environment
                                special files}
\stardocversion     {v1.1}
\stardocmanual      {Programmer's Manual}
\stardocabstract    {%
The SAE package brings together some include files, a facility error file
and a shell script needed for development of software in the Starlink
environment, and for building the USSC.
}
% ? End of document identificationabstarct

% -----------------------------------------------------------------------------
% ? Document specific \providecommand or \newenvironment commands.
% ? End of document specific commands
% -----------------------------------------------------------------------------
%  Title Page.
%  ===========
\begin{document}
\scfrontmatter

\section{Introduction}

The SAE package brings together some include files, a facility error file
and a shell script needed for development of software in the Starlink
environment, and for building the USSC.

\section{Inventory}

SAE contains the following elements:

\begin{itemize}

\item \texttt{SAE\_PAR} --- Fortran include defining the \texttt{SAI\_\_OK},
\texttt{SAI\_\_WARN} and \texttt{SAI\_\_ERROR} parameters.

\item \texttt{sae\_par.h} --- C header file defining the \texttt{SAI\_\_OK},
\texttt{SAI\_\_WARN} and \texttt{SAI\_\_ERROR} parameters.

\item \texttt{fac\_210\_err} --- The facility error code file for the SAI
parameters.

\end{itemize}

\section{Use of SAE}

Use the following INCLUDE statement:

\begin{terminalv}
INCLUDE 'SAE_PAR'
\end{terminalv}

at the top of any Fortran routines, and when compiling any code, locate the
\texttt{SAE\_PAR} include file using the compiler \texttt{-I} flag, as in:
\begin{terminalv}
%  f77 -c -I$STARLINK_DIR/include routine.f
\end{terminalv}


\end{document}
