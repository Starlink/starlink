\documentclass[twoside,11pt,nolof]{starlink}

% ? Specify used packages
% ? End of specify used packages



% -----------------------------------------------------------------------------
% ? Document identification
% Fixed part
\stardoccategory    {Starlink System Note}
\stardocinitials    {SSN}
\stardocsource      {ssn\stardocnumber}
\stardoccopyright
{Copyright \copyright\ 2000 Council for the Central Laboratory of the Research Councils}

% Variable part - replace [xxx] as appropriate.
\stardocnumber      {29.5}
\stardocauthors     {A J Chipperfield}
\stardocdate        {20 August 2001}
\stardoctitle    {PCS\\[1ex]
                                The Parameter and Communication Subsystems}
\stardocabstract  {Application programs often need to obtain
parameter values from a variety of sources and to communicate with other
programs.
The Parameter and Communication Subsystems (PCS) are a set of closely-related
subroutine libraries which provide these facilities for many Starlink
applications and the associated user-interfaces.
\par
The PCS libraries will not generally be called directly by application programs,
but form a basic part of the Starlink Software Environment which is
described in
\xref{Starlink Guide SG/4}{sg4}{}.
Additional notes on using it under Unix are given in
\xref{Starlink User Note SUN/144}{sun144}{}.}
% ? End of document identification
% -----------------------------------------------------------------------------
% ? Document specific \providecommand or \newenvironment commands.
\providecommand{\ROEURL}{http://www.roe.ac.uk/}
\providecommand{\STARURL}{http://www.starlink.ac.uk/}
\providecommand{\RALURL}{http://www.clrc.ac.uk/}
\providecommand{\AAOURL}{http://www.aao.gov.au/}
\providecommand{\JACHURL}{http://www.jach.hawaii.edu/}
% ? End of document specific commands
% -----------------------------------------------------------------------------
%  Title Page.
%  ===========
\begin{document}
\scfrontmatter

\section{\xlabel{the_parameter_subsystem}The Parameter Subsystem}
The Starlink parameter system interface library,
\xref{PAR}{sun114}{}\latexonly{ (see SUN/114)},
provides an interface between application programs and an underlying system
for handling program parameters.

The current underlying parameter subsystem allows parameters to be obtained
from a number of different sources such as the user (via prompts) or values
generated by other applications.
Facilities are also provided for dynamic generation of default values and
value limit checking.
Programs may also save parameter values for use in later invocations or by
other programs.

The system is implemented by five libraries in the PCS package.
\begin{description}
\item[SUBPAR] The top level of the underlying parameter system. Many
of the basic PAR subroutines are almost straight-through calls to the
corresponding SUBPAR subroutines.
\item[PARSECON] Parsing of the interface files associated with SUBPAR.
The package also includes the interface file compiler, COMPIFL.
\item[LEX] A lexical analyser used by SUBPAR in analysing command lines
\textit{etc}.
\item[STRING] Fortran string manipulation subroutines. These subroutines are
also used by software items outside the parameter system, but the library
has no published interface.
\item[MISC] Miscellaneous routines which do not fit into other libraries.
Two handle terminal I/O for programs running directly from a Unix shell
and others provide a Fortran interface into the C library for platforms
(notably Linux) which do not include them as part of the system.
\end{description}

\section{\xlabel{the_communication_subsystems}The Communication Subsystems}
These libraries provide the system which is currently used to construct
programs capable of communicating with each other using the ADAM message
protocol. Messages can control the actions of programs or convey information.

\xref{SG/4}{sg4}{}
describes the simple use of the system (which is usually all that is needed for
data analysis programs) whilst
\xref{SUN/134}{sun134}{}
describes more complicated use in instrumentation control systems.
\begin{description}
\item[MSP] This library provides an inter-program communication system based
upon a system of message `queues'.
\item[SOCK] This library provides a Unix-socket-based message transport system
for the MSP system.
\item[AMS] This library implements the ADAM message protocol on top of MSP.
AMS is written in C but a Fortran interface, FAMS, is provided.
\xref{SUN/241}{sun241}{}
is the programmers manul for AMS.
\item[DTASK] This provides an application program structure which allows the
program to be run directly from the shell or to respond to a specified set of
control messages from other programs using the ADAM message protocol.
The application may consist of multiple `actions' which can be controlled
separately.
DTASK provides the main routine of the program and applications are written
as subroutines which are called by the DTASK layer after the communication and
parameter systems have been initialised. This is described for system
programmers in
\xref{SSN/77}{ssn77}{}
The package also includes shell scripts to link such applications with PCS and
the other Starlink subroutine libraries required.
\item[TASK] This library provides an interface between the application code
and the DTASK layer so that an application can find out some information
about its own status. It also enables programs to control other co-operating
programs using the ADAM message protocol and includes subroutines for encoding
and decoding data values in messages.
For more details, see
\xref{SUN/134}{sun134}{}.
\item[ATIMER] This library provides a system of millisecond interval timers
used by the message system and DTASK.
Each timer has an associated handler which is invoked when the timer expires.
The library is written in C but a Fortran interface, FATIMER, is provided.
\end{description}

\section{\xlabel{hdspar}HDSPAR}
This library is a slight anomaly and should probably be a separate item.
It provides a link between the parameter system and the
\xref{Hierarchical Data System (HDS)}{sun92}{},
enabling object names to be specified by program parameters.
Unlike other PCS subroutines, the HDSPAR routines, DAT\_ASSOC \textit{etc.}, are
expected to be called directly from application programs.

The HDSPAR library is described in
\xref{SUN/224}{sun224}{}.

\section{\xlabel{obsolete_components}Obsolete Components}

\begin{description}
\item[ADAM] This library sends and receives messages using MESSYS.
The message content is assembled/disassembled from/to its constituent parts.
\item[MESSYS] This level of the message system provides compatibility with
earlier systems, thus removing the necessity for wholesale re-writing of
higher-level libraries. It consists mainly of the corresponding calls to AMS
via the AMS Fortran interface. Many of the important parameters of the message
system are defined here.
\end{description}

In PCS V4.0 use of these two libraries has been completely replaced by
calls direct to the Fortran interface of AMS. However, include files from both
packages are still required so the components are still included in PCS but
the subroutine libraries are not built or installed.

\emph{This situation needs re-organisation}

\section{\xlabel{origins}Origins}
The PCS libraries have been developed over many years at several different
establishments. Initially developed as part of the ADAM environment for
instrument control at the Royal Observatories, notably
\htmladdnormallink{ROE}{\ROEURL},
the environment was adopted in 1986 by the
\htmladdnormallink{Starlink}{\STARURL}
project at the
\htmladdnormallink{Rutherford Appleton Lab}{\RALURL}
to support data analysis programs.
Since then further developments have been made by ROE with substantial support
and further developments from Starlink and additional support from
\htmladdnormallink{AAO}{\AAOURL}
and
\htmladdnormallink{JACH}{\JACHURL}.

\section{\xlabel{references}References}
\emph{Note}: Only the first author is listed here.

\begin{latexonly}
\begin {tabular}{lll}
Lawden, M.D. & \xref{SG/4}{sg4}{}
& ADAM -- The Starlink Software Environment.\\
Chipperfield, A.J. & \xref{SUN/144}{sun144}{}
& ADAM -- Unix Version.\\
Currie, M.J. & \xref{SUN114}{sun114}{}
& PAR -- Interface to the ADAM Parameter System.\\
Kelly, B.D. & \xref{SUN/134}{sun134}{}
& ADAM -- Guide to Writing Instrumentation Tasks.\\
Kelly, B.D. & \xref{SUN/241}{sun241}{}
& AMS -- The Unix ADAM Message System.\\
Chipperfield, A.J. & \xref{SSN/77}{ssn77}{}
& ADAM -- The Control Subsystem.\\
Warren-Smith, R.F. & \xref{SUN/92}{sun92}{}
& HDS -- Hierarchical Data System.
\end {tabular}
\end{latexonly}



\end{document}
