\documentclass[twoside,11pt,nolof]{starlink}


% -----------------------------------------------------------------------------
% ? Document identification
\stardoccategory    {Starlink User Note}
\stardocinitials    {SUN}
\stardocsource      {sun\stardocnumber}
\stardocnumber      {172.1}
\stardocauthors     {Clive G Page \\ University of Leicester}
\stardocdate        {13 October 1993}
\stardoctitle     {FTNCHEK\\[2ex]
                                A Fortran77 Source-code Checker}
\stardocabstract{
\texttt{FTNCHEK} is designed to detect a variety of errors in Fortran77
source-code, especially mistakes that tend to be missed by compilers
and linkers.
}
% ? End of document identification

% -----------------------------------------------------------------------------
% ? Document-specific \providecommand or \newenvironment commands.
% ? End of document-specific commands
% -----------------------------------------------------------------------------
%  Title Page.
%  ===========
\begin{document}
\scfrontmatter

\section{Introduction\xlabel{introduction}}

Probably the most valuable feature is that it performs careful checks on
subprogram interfaces. \texttt{FTNCHEK} examines the actual arguments in
each \texttt{CALL} statement (or function call) and compares them with the
corresponding formal arguments of the \texttt{SUBROUTINE} (or \texttt{FUNCTION}) subprogram. It warns of mismatches in the number of arguments
or of incompatibilities in the data type or dimensions of each one. It
also checks that where an argument is used before being set in the
subprogram that it has a defined value on entry, and correspondingly
that arguments are set on exit where the value is used later in the
calling program.

Before using \texttt{FTNCHEK} it is advisable to get the source-code to
compile without errors. For the most comprehensive checking all the
source-code should be submitted to \texttt{FTNCHEK} in the same run. This
gives rise to an obvious difficulty where a program links with
subprograms in a pre-compiled library.  \texttt{FTNCHEK} can cope with
this, if the source-code of the library is available, by producing a
\emph{project file} which stores the characteristics of each subprogram
interface for subsequent use. After project files have been produced for
each library, they can be used in later runs of \texttt{FTNCHEK} to ensure
that calls to library routines contain no obvious errors.

Depending on the options selected, \texttt{FTNCHEK} will also flag
variables that are never used, are not initialized, or which have no
explicitly declared data type; it can also perform consistency checks on
\texttt{COMMON} blocks, and warn of the existence of non-standard or (not
quite the same thing) non-portable Fortran syntax.

\subsection{New Features\xlabel{new_features}}

Version 2.6.1 of \texttt{FTNCHEK} has several enhancements in its error
checking. Some new options have been added to give greater flexibility,
although this has resulted in some changes to the default settings
of existing options.  The \texttt{BYTE} type statement is now accepted, and
a useful new option lists a call-tree of subprogram dependencies.

\section{Non-standard Syntax\xlabel{nonstandard_syntax}}

Since \texttt{FTNCHEK} is designed to help in writing portable code,
features that do not conform to the Fortran77 Standard are supported
only to a limited extent.  It will accept \texttt{INCLUDE}, \texttt{IMPLICIT
NONE}, \texttt{DO WHILE}, \texttt{END DO}, \texttt{ACCEPT}, \texttt{TYPE}, and \texttt{NAMELIST} statements, and data-type specifications of \texttt{LOGICAL*1},
\texttt{BYTE}, \texttt{INTEGER*1}, \texttt{INTEGER*2}, and \texttt{REAL*8}.  It also
permits symbolic names that include underscores, and symbolic names up
to 31 characters long. Case differences are ignored, and tabs are
treated as if there were tab-stops every eight columns.  Most of these
extensions will, however, be flagged if the \texttt{-f77} switch is used.

\texttt{FTNCHEK} does \textbf{not}, however, understand the use of VMS
intrinsic functions such as \texttt{\%VAL}, \texttt{\%LOC}, or \texttt{\%REF}. In
addition spurious warnings may be generated from the use of array names
which are identical to certain Fortran keywords (for example \texttt{DATA}
or \texttt{READ}).  Details of these and other minor limitations are given
in the author's own documentation (see section 5 below).  Sometimes a
few unnecessary warnings appear, but this is usually acceptable given
the comprehensive checking that is performed.

\section{Running \texttt{FTNCHEK}\xlabel{running_ftnchek}}

\begin{terminalv}
$ ftnchek -option -option ...  filename filename ...
\end{terminalv}

runs the program on one or more source files (the dots above represent
optional repetition). If no options (switches) are specified, the
results go to standard output (the screen).  The default action is to
show error messages, warnings, and informational messages, but not the
program listing, symbol tables, or call-tree.

Each option begins with the \texttt{-} character (and may be abbreviated to
a dash plus 3 characters).  On VMS systems either \texttt{/} or \texttt{-}
may precede an option and a space is unnecessary before a slash. Each
option remains in effect from the point it is encountered until it is
over-ridden by a later option; for example an output listing may be
enabled for some source files and not for others. The options are
outlined below; a summary can be obtained on-line by using the command:

\begin{terminalv}
$ ftnchek -help
\end{terminalv}

If a filename does not include an explicit extension, \texttt{FTNCHEK} first
looks for a project file with extension \texttt{.prj}, otherwise it looks
for a Fortran source file with the extension \texttt{.f} (on Unix) or \texttt{.for} (on VMS).  If more than one filename is specified the modules are
treated as if they were on a single file. If no filename is given \texttt{FTNCHEK} reads from the standard input.

The defaults for \texttt{FTNCHEK} may be altered (on Unix) by setting
environment variables or (on VMS) by defining logical names.  For
example, to enable the \texttt{-list} option by default use the appropriate
command below:

\begin{terminalv}
% setenv FTNCHEK_LIST YES
$ DEFINE FTNCHEK_LIST YES
\end{terminalv}

\section{Options\xlabel{options}}

\emph{These options are \textbf{off} by default, but may be switched
\textbf{on} by using the appropriate switch:}

\begin{quote}
\begin{description}

\item [--calltree] List the subprogram call hierarchy as a call-tree.

\item [--division] Warn wherever division is done (except division by a
constant).

\item [--f77] Warn of violations of the Fortran 77 standard.

\item [--library] Enter library mode: do not warn if subprograms in a file
are defined but never used.

\item [--list] List the source-code complete with line-numbers.

\item [--project] Create a project file (see details in main manual).

\item [--sixchar] Warn of names which are not unique in their
first six characters.

\item [--symtab] List the symbol table of each subprogram.

\end{description}
\end{quote}

\emph{These options are \textbf{on} by default, but may be switched \textbf{off} by negating the switch (e.g. specifying on Unix \texttt{-nodeclare} or
on VMS \texttt{/NODECLARE }):}

\begin{quote}
\begin{description}

\item [--declare] List all identifiers with no
explicitly declared data type (equivalent to using \texttt{IMPLICIT NONE}).

\item [--extern] List all external subprograms that are never
defined (this will flag use of library routines or non-standard
intrinsics).

\item [--hollerith] Warn about Hollerith constants under the \textbf{--port} option.

\item [--linebreak] Treat line-breaks in continued statements as space.

\item [--portability] Warn of non-portable usage.

\item [--pretty] Warn of unusual layout, e.g. comments between continuation
lines.

\item [--pure] Warn of functions which have side-effects.

\item [--truncation] Warn if truncation may occur, e.g. in
converting real to integer.

\item [--verbose] Output all messages; if switched off, omits some
inessential messages, \emph{e.g.} reporting the name of each
\texttt{INCLUDE} file.

\end{description}
\end{quote}

\emph{These options each take a numeric or string argument:}

\begin{quote}
\begin{description}

\item [--arrays=n] Specify checks for array arguments, \textit{n} range 0
to 3, default 3. (0=none, 1=rank, 2=size, 3=all).

\item [--columns=n] Set the maximum line length to \textit{n} columns (text
beyond this is ignored), range 0 to 132, default 72.

\item [--common=n] Specify checks for \texttt{COMMON} blocks, \textit{n}
range 0 to 3, default 3.  (0=none, 1=data-types, 2=array-storage,
3=all).

\item [--include=path] Define a directory to search for include files; a
cumulative option.

\item [--output=filename] Send output to the given file (the default
file extension is \texttt{.lis}). By default all output goes to \texttt{stdout} (the screen) and on Unix may be redirected.

\item [--usage=n] Specify checks for unused variables, \textit{n} range
0 to 3, default 3. (0=none, 1=used-before-set, 2=unused, 3=all).

\end{description}
\end{quote}

\section{Further Information\xlabel{further_information}}

This note just summarizes the facilities of \texttt{FTNCHEK}. Its author,
Dr Robert Moniot of Fordham University (USA), has provided a detailed
manual which is available as an ASCII file \texttt{ftnchek.doc} or in
PostScript form as \texttt{ftnchek.ps}.

\section{An Alternative\xlabel{an_alternative}}

Starlink users have access to a commercial product called \texttt{FORCHECK}
which is licenced on the STADAT node.  This has rather similar
functionality but both programs have their strengths and
weaknesses. An obvious advantage of \texttt{FTNCHEK} is that it is a
public-domain product which can be used freely on any Unix, VMS, or
MS-DOS system.

\end{document}
