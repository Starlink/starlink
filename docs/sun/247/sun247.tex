\documentclass[twoside,11pt]{article}

% ? Specify used packages
% \usepackage{graphicx}        %  Use this one for final production.
% \usepackage[draft]{graphicx} %  Use this one for drafting.
% ? End of specify used packages

\pagestyle{myheadings}

% -----------------------------------------------------------------------------
% ? Document identification
% Fixed part
\newcommand{\stardoccategory}  {Starlink User Note}
\newcommand{\stardocinitials}  {SUN}
\newcommand{\stardocsource}    {sun\stardocnumber}
\newcommand{\stardoccopyright}
{Copyright \copyright\ 2004 Council for the Central Laboratory of the
Research Councils}

% Variable part - replace [xxx] as appropriate.
\newcommand{\stardocnumber}    {247.1}
\newcommand{\stardocauthors}   {A.C.~Davenhall \& R.T.~Platon}
\newcommand{\stardocdate}      {13 August 2004}
\newcommand{\stardoctitle}     {XRT: % \\ [1ex]
                                ROSAT XRT Data Reduction}
\newcommand{\stardocversion}   {Version 1.1}
\newcommand{\stardocmanual}    {User's Manual}
\newcommand{\stardocabstract}  {
XRT is a package for reducing data acquired with the ROSAT XRT
instruments.  The XRT (X-Ray Telescope) was the principal scientific
payload of the ROSAT X-ray astronomy satellite.  The XRT had two
instruments: the PSPC (Position Sensitive Proportional Counter) and the
HRI (High Resolution Imager).  The XRT package operates on data produced
by these instruments and can be used to transform them into calibrated
images, spectra, time-series \emph{etc}.

XRT was created by taking the ROSAT XRT-specific functions in the ASTERIX
general X-ray astronomy data reduction system and re-packaging them as
stand-alone applications.

\begin{latexonly}
\vspace{2mm}
\end{latexonly}

\begin{center}
{\bf Who Should Read this Document?}
\end{center}

This document is aimed at people who plan to reduce data acquired with
the ROSAT XRT instruments (the PSPC and the HRI).

\begin{latexonly}
\vspace{2mm}
\end{latexonly}

\begin{center}
{\bf Disclaimer}
\end{center}

\begin{quote}
{\bf We believe that the current version of XRT does not work.  It is
made available purely on an `as is' basis and comes with no
recommendation, guarantee or warranty whatsoever.  Further, one of us
(ACD), though an author of this note, was not involved in developing the
XRT source code.}
\end{quote}}
% ? End of document identification
% -----------------------------------------------------------------------------

% +
%  Name:
%     sun.tex
%
%  Purpose:
%     Template for Starlink User Note (SUN) documents.
%     Refer to SUN/199
%
%  Authors:
%     AJC: A.J.Chipperfield (Starlink, RAL)
%     BLY: M.J.Bly (Starlink, RAL)
%     PWD: Peter W. Draper (Starlink, Durham University)
%
%  History:
%     17-JAN-1996 (AJC):
%        Original with hypertext macros, based on MDL plain originals.
%     16-JUN-1997 (BLY):
%        Adapted for LaTeX2e.
%        Added picture commands.
%     13-AUG-1998 (PWD):
%        Converted for use with LaTeX2HTML version 98.2 and
%        Star2HTML version 1.3.
%      1-FEB-2000 (AJC):
%        Add Copyright statement in LaTeX
%     {Add further history here}
%
% -

\newcommand{\stardocname}{\stardocinitials /\stardocnumber}
\markboth{\stardocname}{\stardocname}
\setlength{\textwidth}{160mm}
\setlength{\textheight}{230mm}
\setlength{\topmargin}{-2mm}
\setlength{\oddsidemargin}{0mm}
\setlength{\evensidemargin}{0mm}
\setlength{\parindent}{0mm}
\setlength{\parskip}{\medskipamount}
\setlength{\unitlength}{1mm}

% -----------------------------------------------------------------------------
%  Hypertext definitions.
%  ======================
%  These are used by the LaTeX2HTML translator in conjunction with star2html.

%  Comment.sty: version 2.0, 19 June 1992
%  Selectively in/exclude pieces of text.
%
%  Author
%    Victor Eijkhout                                      <eijkhout@cs.utk.edu>
%    Department of Computer Science
%    University Tennessee at Knoxville
%    104 Ayres Hall
%    Knoxville, TN 37996
%    USA

%  Do not remove the %begin{latexonly} and %end{latexonly} lines (used by
%  LaTeX2HTML to signify text it shouldn't process).
%begin{latexonly}
\makeatletter
\def\makeinnocent#1{\catcode`#1=12 }
\def\csarg#1#2{\expandafter#1\csname#2\endcsname}

\def\ThrowAwayComment#1{\begingroup
    \def\CurrentComment{#1}%
    \let\do\makeinnocent \dospecials
    \makeinnocent\^^L% and whatever other special cases
    \endlinechar`\^^M \catcode`\^^M=12 \xComment}
{\catcode`\^^M=12 \endlinechar=-1 %
 \gdef\xComment#1^^M{\def\test{#1}
      \csarg\ifx{PlainEnd\CurrentComment Test}\test
          \let\html@next\endgroup
      \else \csarg\ifx{LaLaEnd\CurrentComment Test}\test
            \edef\html@next{\endgroup\noexpand\end{\CurrentComment}}
      \else \let\html@next\xComment
      \fi \fi \html@next}
}
\makeatother

\def\includecomment
 #1{\expandafter\def\csname#1\endcsname{}%
    \expandafter\def\csname end#1\endcsname{}}
\def\excludecomment
 #1{\expandafter\def\csname#1\endcsname{\ThrowAwayComment{#1}}%
    {\escapechar=-1\relax
     \csarg\xdef{PlainEnd#1Test}{\string\\end#1}%
     \csarg\xdef{LaLaEnd#1Test}{\string\\end\string\{#1\string\}}%
    }}

%  Define environments that ignore their contents.
\excludecomment{comment}
\excludecomment{rawhtml}
\excludecomment{htmlonly}

%  Hypertext commands etc. This is a condensed version of the html.sty
%  file supplied with LaTeX2HTML by: Nikos Drakos <nikos@cbl.leeds.ac.uk> &
%  Jelle van Zeijl <jvzeijl@isou17.estec.esa.nl>. The LaTeX2HTML documentation
%  should be consulted about all commands (and the environments defined above)
%  except \xref and \xlabel which are Starlink specific.

\newcommand{\htmladdnormallinkfoot}[2]{#1\footnote{#2}}
\newcommand{\htmladdnormallink}[2]{#1}
\newcommand{\htmladdimg}[1]{}
\newcommand{\hyperref}[4]{#2\ref{#4}#3}
\newcommand{\htmlref}[2]{#1}
\newcommand{\htmlimage}[1]{}
\newcommand{\htmladdtonavigation}[1]{}

\newenvironment{latexonly}{}{}
\newcommand{\latex}[1]{#1}
\newcommand{\html}[1]{}
\newcommand{\latexhtml}[2]{#1}
\newcommand{\HTMLcode}[2][]{}

%  Starlink cross-references and labels.
\newcommand{\xref}[3]{#1}
\newcommand{\xlabel}[1]{}

%  LaTeX2HTML symbol.
\newcommand{\latextohtml}{\LaTeX2\texttt{HTML}}

%  Define command to re-centre underscore for Latex and leave as normal
%  for HTML (severe problems with \_ in tabbing environments and \_\_
%  generally otherwise).
\renewcommand{\_}{\texttt{\symbol{95}}}

% -----------------------------------------------------------------------------
%  Debugging.
%  =========
%  Remove % on the following to debug links in the HTML version using Latex.

% \newcommand{\hotlink}[2]{\fbox{\begin{tabular}[t]{@{}c@{}}#1\\\hline{\footnotesize #2}\end{tabular}}}
% \renewcommand{\htmladdnormallinkfoot}[2]{\hotlink{#1}{#2}}
% \renewcommand{\htmladdnormallink}[2]{\hotlink{#1}{#2}}
% \renewcommand{\hyperref}[4]{\hotlink{#1}{\S\ref{#4}}}
% \renewcommand{\htmlref}[2]{\hotlink{#1}{\S\ref{#2}}}
% \renewcommand{\xref}[3]{\hotlink{#1}{#2 -- #3}}
%end{latexonly}
% -----------------------------------------------------------------------------
% ? Document specific \newcommand or \newenvironment commands.
% ? End of document specific commands
% -----------------------------------------------------------------------------
%  Title Page.
%  ===========
\renewcommand{\thepage}{\roman{page}}
\begin{document}
\thispagestyle{empty}

%  Latex document header.
%  ======================
\begin{latexonly}
   CCLRC / \textsc{Rutherford Appleton Laboratory} \hfill \textbf{\stardocname}\\
   {\large Particle Physics \& Astronomy Research Council}\\
   {\large Starlink Project\\}
   {\large \stardoccategory\ \stardocnumber}
   \begin{flushright}
   \stardocauthors\\
   \stardocdate
   \end{flushright}
   \vspace{-4mm}
   \rule{\textwidth}{0.5mm}
   \vspace{5mm}
   \begin{center}
   {\Huge\textbf{\stardoctitle \\ [2.5ex]}}
   {\LARGE\textbf{\stardocversion \\ [4ex]}}
   {\Huge\textbf{\stardocmanual}}
   \end{center}
   \vspace{5mm}

% ? Add picture here if required for the LaTeX version.
%   e.g. \includegraphics[scale=0.3]{filename.ps}
% ? End of picture

% ? Heading for abstract if used.
   \vspace{10mm}
   \begin{center}
      {\Large\textbf{Abstract}}
   \end{center}
% ? End of heading for abstract.
\end{latexonly}

%  HTML documentation header.
%  ==========================
\begin{htmlonly}
   \xlabel{}
   \begin{rawhtml} <H1> \end{rawhtml}
      \stardoctitle\\
      \stardocversion\\
      \stardocmanual
   \begin{rawhtml} </H1> <HR> \end{rawhtml}

% ? Add picture here if required for the hypertext version.
%   e.g. \includegraphics[scale=0.7]{filename.ps}
% ? End of picture

   \begin{rawhtml} <P> <I> \end{rawhtml}
   \stardoccategory\ \stardocnumber \\
   \stardocauthors \\
   \stardocdate
   \begin{rawhtml} </I> </P> <H3> \end{rawhtml}
      \htmladdnormallink{CCLRC / Rutherford Appleton Laboratory}
                        {http://www.cclrc.ac.uk} \\
      \htmladdnormallink{Particle Physics \& Astronomy Research Council}
                        {http://www.pparc.ac.uk} \\
   \begin{rawhtml} </H3> <H2> \end{rawhtml}
      \htmladdnormallink{Starlink Project}{http://www.starlink.ac.uk/}
   \begin{rawhtml} </H2> \end{rawhtml}
   \htmladdnormallink{\htmladdimg{source.gif} Retrieve hardcopy}
      {http://www.starlink.ac.uk/cgi-bin/hcserver?\stardocsource}\\

%  HTML document table of contents.
%  ================================
%  Add table of contents header and a navigation button to return to this
%  point in the document (this should always go before the abstract \section).
  \label{stardoccontents}
  \begin{rawhtml}
    <HR>
    <H2>Contents</H2>
  \end{rawhtml}
  \htmladdtonavigation{\htmlref{\htmladdimg{contents_motif.gif}}
        {stardoccontents}}

% ? New section for abstract if used.
  \section{\xlabel{abstract}Abstract}
% ? End of new section for abstract
\end{htmlonly}

% -----------------------------------------------------------------------------
% ? Document Abstract. (if used)
%  ==================
\stardocabstract
% ? End of document abstract

% -----------------------------------------------------------------------------
% ? Latex Copyright Statement
%  =========================
\begin{latexonly}
\vspace*{\fill}
\stardoccopyright
\end{latexonly}
% ? End of Latex copyright statement

% -----------------------------------------------------------------------------
% ? Latex document Table of Contents (if used).
%  ===========================================
%  \newpage
%  \begin{latexonly}
%    \setlength{\parskip}{0mm}
%    \tableofcontents
%    \setlength{\parskip}{\medskipamount}
%    \markboth{\stardocname}{\stardocname}
%  \end{latexonly}
% ? End of Latex document table of contents
% -----------------------------------------------------------------------------

\newpage
\renewcommand{\thepage}{\arabic{page}}
\setcounter{page}{1}

\section{\xlabel{INTRO}\label{INTRO}Introduction}

XRT is a package for reducing data acquired with the ROSAT XRT instruments.
\htmladdnormallink{ROSAT}{http://ledas-www.star.le.ac.uk/rosat-goc/index.html}
({\it R\"{O}entgen-SATellit}\/) was a highly successful German/US/UK
X-ray astronomy satellite which was operational from 1990 to 1999.
During this period it amassed a considerable archive of observations which
are still regularly used.  ROSAT's principal scientific payload was the
XRT (X-Ray Telescope).  The XRT had two instruments: the PSPC (Position
Sensitive Proportional Counter) and the HRI (High Resolution Imager).

During the late 1980s and early 1990s the
\htmladdnormallink{ASTERIX}{http://www.sr.bham.ac.uk/asterix/}
software package was developed for the analysis of ROSAT (and other X-ray
astronomy) data.  ASTERIX was mostly written by the
\htmladdnormallink{Astrophysics and Space Research Group}
{http://www.sr.bham.ac.uk/}
at the University of Birmingham, but with some involvement by Starlink.
Starlink used to distribute ASTERIX but no longer does so, though it
can still be obtained from the Birmingham group (see below).

Much of the functionality of ASTERIX is included in other packages.
However, some of the functions specific to ROSAT data are not available
elsewhere.  These ROSAT-specific functions have been re-packaged as
stand-alone applications and released as the XRT package, which is briefly
documented here.


\section{\xlabel{FORMATS}\label{FORMATS}Data Formats and Obtaining Data}

Various archives of ROSAT data are maintained at astronomical data
centres around the world, and the data are stored in several different
formats.  XRT requires data in the so-called RDF (Rationalized Data
Format) format.  RDF files contain data in the FITS\footnote{FITS (Flexible
Image Transport System) is a standard data format in astronomy.  The
details are not germane here, though there is a
\xref{brief introduction}{sc5}{FITS} in \xref{SC/5.3}{sc5}{}: {\it The
2-D CCD Data Reduction Cookbook}, A.C.~Davenhall, G.J.~Privett and
M.B.~Taylor, 16 August 2001, Starlink.  For a full description of the
standard, and much other information, see URL
\htmladdnormallink{{\tt http://fits.gsfc.nasa.gov/fits\_home.html}}
{http://fits.gsfc.nasa.gov/fits_home.html}.} format.  Typically an RDF
dataset comprises four FITS files each containing a separate table.  These
files are:

\begin{center}
\begin{tabular}{ll}
File                 & Contents \\ \hline
{\tt events.tbl}     & Photon event table \\
{\tt eventrates.tbl} & Instrument count rates for dead time corrections \\
{\tt attitude.tbl}   & Satellite attitude information  \\
{\tt orbit.tbl}      & Satellite orbital information   \\
\end{tabular}
\end{center}

ROSAT data in the RDF format can be obtained from the
\htmladdnormallink{LEDAS}{http://ledas-www.star.le.ac.uk/}
service provided by the
\htmladdnormallink{X-ray and Observational Astronomy Group}
{http://www.star.le.ac.uk/}
in the Department of Physics and Astronomy at the University of Leicester.
LEDAS includes an archive of ROSAT data which can be searched to locate
observations of interest.  See URL:

\begin{quote}
\htmladdnormallink{{\tt http://ledas-www.star.le.ac.uk/rosat/}}
{http://ledas-www.star.le.ac.uk/rosat/}
\end{quote}

Note that in addition to the basic RDF data the LEDAS ROSAT archive also
includes reduced `data products' such as images.  These products have been
constructed from the raw data in a standardised fashion as part of the
LEDAS service.


\section{\xlabel{START}\label{START}Starting XRT}

XRT is run from the Unix shell.  To start it simply type:

\begin{quote}
{\tt xrt}
\end{quote}

A message similar to the following should appear:

\begin{verbatim}

                       XRT
    (extracted from Asterix Rosat XRT Package)

                   Version 1.1

      Type "showme sun247" to see the manual
\end{verbatim}

and the XRT commands are now available.  If XRT does not start correctly
then it is probably not installed at your site.  In the first instance
see your site manager for assistance.


\section{\xlabel{COMMANDS}\label{COMMANDS}Commands Available}

The commands available in XRT are listed in Table~\ref{XRTCOMMANDS}.
To run each command simply type its name, for example:

\begin{quote}
{\tt xrtsort}
\end{quote}

and answer the ensuing prompts.

\begin{table}[htbp]

\begin{center}
\begin{tabular}{ll}
Command  & Description \\ \hline
{\tt ast2qdp}  & Convert a one-dimensional dataset to QDP format. \\
{\tt ast2txt}  & Convert a one-dimensional dataset to text format. \\
{\tt ast2xsp}  & Convert spectrum from ASTERIX to XSPEC format. \\
{\tt radopt}   & Finds optimum signal-to-noise ratio radius for a given
 source. \\
{\tt showxrt}  & Show observation information stored in header file. \\
{\tt xpsscorr} & Converts PSS XRT source counts into a flux. \\
{\tt xrad}     & Calculates the PSF radius for a PSPC source. \\
{\tt xrtcorr}  & Corrects XRT data files for vignetting effects \emph{etc}. \\
{\tt xrthk}    & Produce list of times based on housekeeping parameters. \\
{\tt xrtindex} & Extracts header information from ROSAT FITS files. \\
{\tt xrtsort}  & Sorts ROSAT PSPC and HRI raw data into ASTERIX datasets. \\
{\tt xrtsub}   & Subtracts an XRT background from an XRT source file. \\
{\tt xspokes}  & Defines the ribs region on a PSPC image. \\
\end{tabular}
\end{center}

\begin{quote}
\caption{The XRT commands \label{XRTCOMMANDS} }
\end{quote}

\end{table}


\section{\xlabel{INFO}\label{INFO}Further Information}

The XRT commands are essentially identical to, and have the same names as,
the corresponding ASTERIX functions.  These ASTERIX functions are
described, in some detail, in SUN/98\footnote{SUN/98.6: {\it ASTERIX --
X-ray Data Processing System},  D.J.~Allan and R.J.~Vallance, 3 May 1995,
Birmingham.  Note that this document is not distributed by Starlink (see
text).}, the ASTERIX manual, which should be consulted for further details.
Note that though SUN/98 is numbered in Starlink's SUN series, like ASTERIX
itself, it is no longer distributed by Starlink.  Rather, both hyper-text
and printable versions are available from Birmingham's ASTERIX Web site
at URL:

\begin{quote}
\htmladdnormallink{{\tt http://www.sr.bham.ac.uk/asterix/}}
{http://www.sr.bham.ac.uk/asterix/}
\end{quote}


\section{\xlabel{PROBLEMS}\label{PROBLEMS}Reporting Problems}

Problems with XRT can be reported by sending an e-mail message to {\tt
starlink@jiscmail.ac.uk}.  Alternatively, in the unlikely event that you do not
have access to e-mail, you can write to: The Software Librarian, Starlink
Project, Rutherford Appleton Laboratory, Chilton, DIDCOT, Oxfordshire,
OX11 0QX, UK.


\section{\xlabel{ACK}\label{ACK}Acknowledgements}

XRT comprises the ROSAT-specific functions in ASTERIX converted to
stand-alone applications.  Numerous people have contributed to ASTERIX,
including {\it inter alia}, David~Allan, Bob~Vallance and Richard~Saxton.


\section{\xlabel{REV}\label{REV}Revision History}

\begin{enumerate}

  \item 1 November 2002: Version 0.  Original draft (ACD).

  \item 13 August 2004: Version 1.  First released version.  Tidied
   previous draft for release (ACD).

\end{enumerate}


%------------------------------------------------------------------------------

\typeout{  }
\typeout{*****************************************************}
\typeout{  }
\typeout{Reminder: run this document through Latex three times}
\typeout{to resolve the references.}
\typeout{  }
\typeout{*****************************************************}
\typeout{  }

\end{document}
