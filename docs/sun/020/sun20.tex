\documentstyle[11pt]{article}
\pagestyle{myheadings}

%------------------------------------------------------------------------------
\newcommand{\stardoccategory}  {Starlink User Note}
\newcommand{\stardocinitials}  {SUN}
\newcommand{\stardocnumber}    {20.4}
\newcommand{\stardocauthors}   {M J Murray, D J Stickland}
\newcommand{\stardocdate}      {30 June 1993}
\newcommand{\stardoctitle}     {ULDA/USSP --- Accessing the IUE Uniform Low-Dispersion Archive}
%------------------------------------------------------------------------------

\newcommand{\stardocname}{\stardocinitials /\stardocnumber}
\renewcommand{\_}{{\tt\char'137}}     % re-centres the underscore
\markright{\stardocname}
\setlength{\textwidth}{160mm}
\setlength{\textheight}{230mm}
\setlength{\topmargin}{-2mm}
\setlength{\oddsidemargin}{0mm}
\setlength{\evensidemargin}{0mm}
\setlength{\parindent}{0mm}
\setlength{\parskip}{\medskipamount}
\setlength{\unitlength}{1mm}

\begin{document}
\thispagestyle{empty}
SCIENCE \& ENGINEERING RESEARCH COUNCIL \hfill \stardocname\\
RUTHERFORD APPLETON LABORATORY\\
{\large\bf Starlink Project\\}
{\large\bf \stardoccategory\ \stardocnumber}
\begin{flushright}
\stardocauthors\\
\stardocdate
\end{flushright}
\vspace{-4mm}
\rule{\textwidth}{0.5mm}
\vspace{5mm}
\begin{center}
{\Large\bf \stardoctitle}
\end{center}
\vspace{5mm}

\setlength{\parskip}{0mm}
\tableofcontents
\setlength{\parskip}{\medskipamount}
\markright{\stardocname}

\newpage
\section{The ULDA/USSP Project}

The usefulness of the data collected by the IUE satellite
has been enormously enhanced by the maintenance of a data archive
from which interested astronomers can request any image (subject
to the six-month data retrieval rule). See SUN/58 for details.
Most UK astronomers now reduce their IUE data from the photometric
images available in the archive using IUEDR (SUN/37).   However,
%for quick-look or other non-critical purposes,
it is now possible to obtain reduced low-resolution spectra on-line.
These data have been reduced in a fairly
homogenous way using the IUE Project's Spectral Image Processing System
(IUESIPS),
%which gives an adequate product in most cases (and certainly
%for data acquired since {!whenever SIPS2 was implemented!}).
and have been made available in the form of a {\it Uniform Low-Dispersion
Archive} (ULDA).
This comprises absolutely-calibrated,  extracted, low-dispersion
spectra corrected for exposure times and with the appropriate  ITF correction
(flux in ergs cm$^{-2}$  s$^{-1}$ {\AA}$^{-1}$).
A full description of the data is given in the {\it ESA IUE Newsletter No.~30,}
July 1988,
and a comparison of the ULDA with IUEDR products is featured in
the {\it Starlink Bulletin, no.~4}.

In order to facilitate access to the ULDA, the {\it ULDA Software Support
Package} (USSP) has been developed.
The USSP enables astronomers to select interactively
any images contained in the ULDA and provides facilities for efficient
intercomputer data transfer.
The USSP enables the searching of the ULDA
data base for various types of images.
For example, images can be selected on object class, image number, camera,
aperture,  homogeneous ID,
sky position or any combination of these attributes.

Users in the UK will use the USSP on the Starlink database
Microvax (STADAT) which is described in section~2.

\subsection {Version 1}

The original release of the ULDA contained around 98.5\% of those
low-dispersion spectra taken up to the beginning of 1984 -- around
25,000 spectra.

\subsection {Version 2}

Version 2  contained 37,000 spectra and was virtually complete
up to the start of 1987.
This version included a correction made to the absolute calibration
of very short exposures
and the erroneous photometric calibration for some early
LWP spectra was rectified.

Information on who has selected spectra previously and for what purpose was
made available in this version.
The option to inspect this so called `usage data'  is available
as described below.

\subsection{Version 3}

Version 3 is virtually complete up to January,  1989 and contains
over  44,000 spectra.
LWP images obtained after December~21st, 1987 have been processed using
a new  ITF  as  well  as  a  new  Absolute  Calibration;
this  extends  the  calibrated  range  from 3200{\AA}
to  3300{\AA}.
Images processed with the  new  calibration have fluxes which are
systematically lower (by about 7\%).
LWP data taken before December 1987 have not been reprocessed in
accordance with the new calibration, as
such reprocessing was not considered worthwhile in view
of the imminence of the IUE final archive.
However, for various reasons, a number of pre-December~1987 LWP images
{\sl have\/} been reprocessed with the new calibration.
These are included in the ULDA and can be easily distinguished as the
wavelength range  in such data extends up to 3300{\AA}.

Several minor changes have been made to the USSP software.
The only one previous
users of the USSP will notice is that the unnecessary questions relating to
ASCII conversion of data have been eliminated.

\subsection{Version 4}

This release contains in excess of 54,000 spectra taken before the
end of 1991. In all other respects its operation is as for the previous
version.



\section{STADAT --- Starlink Database Microvax}

This is not a general-purpose machine, but is primarily intended for the
retrieval of data, and is available to any accredited Starlink user.
Access is via a communal account, with one account
per Starlink area.
Your local Site Manager will tell you the account name and password.

Log in via your own Starlink account by typing:
\begin{verbatim}
     $ SET HOST STADAT
\end{verbatim}
Access via the PAD is not permitted for security reasons.
Your local Site Manager will be responsible for the management
of your area account.
However, it is not intended that significant amounts of permanent
disk space will be allocated to any user.
In any case, all user files generated by the USSP are written
to a single directory intended specifically for this purpose, as described
below.

\section{Using the USSP}

Retrieval of the spectra is a two-stage process.
\begin{enumerate}
\item Firstly you must run the program QUEST in order to
select the spectra you want.
\item Subsequently you return to your own local computer (or your
directory on the microvax) and copy the spectra
using the program UNSPL.
This second step is not a simple copy, but involves `UNscrambling/SPLitting'
the files output by QUEST.
\end{enumerate}
All user-files generated by QUEST are written to the
directory  STADAT::ULDA\$HOST\_DIR,  a location assumed by
the program UNSPL when copying.
In order to distinguish users' files it is necessary that they each have
a USSP identifier.
Thus you must register yourself with an identifier the first time you
use the USSP and subsequently use the same identifier.
Registration is a very simple interactive process (see below).
Your identifier must be 3-8 letters long with the first two letters
being the abbreviation of your country.
For example if your Starlink username is  ABC your first choice
should be UKABC.
Any files you generate with QUEST will be
UKABC.{\it filetype} (where {\it filetype} may be SPE, AUX {\it etc.}).
For convenience, the USSP identifier UKABC is assumed throughout this document.

{\bf N.B.} EACH RUN OF THE PROGRAM  QUEST WILL AUTOMATICALLY DELETE
ANY EXISTING FILES BELONGING TO THE IDENTIFIER USED.
Thus after selecting spectra which you actually want,  you must downlink with
UNSPL before running QUEST again.
Obviously users must NOT share a USSP identifier as they may accidentally
delete each others files.

The files should be downlinked within a short space of time
as QUEST will automatically delete files older than a defined
period\footnote{Currently one day}.

\subsection{QUEST --- Selecting spectra}

This program is intended to be self-explanatory; the options are listed
at every stage; typing `?' will usually solicit help, and invalid user
responses will simply result in a message and another chance to answer.
Nevertheless, a sample session is shown  below.

To begin a session with QUEST simply type:
\begin{verbatim}
     $ QUEST
\end{verbatim}
%\subsubsection{USSP identifier}
You will be invited to give your USSP identifier or register one for yourself.
Registration simply involves typing {\tt R} in response to QUEST's initial
question, and then replying with your chosen username to the second.
You will be asked to confirm what you have typed, and in the event someone has
already chosen this name, or your choice is invalid you will be asked to think
of another one. Spaces are ignored and special characters should not be used.
Your  name and address will also be requested at this point -- a computer
address will suffice.
You can choose to look at all the usernames already allocated to
people in any country.

%\subsubsection{Motive}

The next question offers a selection of motives from which you should
indicate which corresponds most closely to your reason
for using the USSP.
The only purpose of this is for an intended compilation of ULDA-usage data.

The exception is the `Recovery' option which
increases the lifetime of previously selected files.
This is intended for the situation where the user is having problems copying
the selected files to a local computer.
However this is not likely to be necessary in the UK.

The main menu is now displayed:
\begin{verbatim}
       1      -      SEARCH FOR A SET OF SPECTRA
       2      -      SEARCH FOR SPECTRA - SHORT PROMPT
       3      -      DISPLAY WHAT WAS FOUND BY SEARCH
       4      -      SELECT SPECTRA FROM THOSE FOUND BY SEARCH (FOR DOWNLINK)
       5      -      SAVE DESCRIPTORS OF WHAT LAST SEARCH FOUND (FOR DOWNLINK)
       6      -      DISPLAY & SAVE DUBIOUS INFO.FOR WHAT LAST SEARCH FOUND
       ? [N]  -      HELP ME, N = ONE OF ABOVE ANSWERS
\end{verbatim}
A typical session would be a SEARCH (for a set of spectra) followed
by a DISPLAY (what was found by search)
followed by a SELECT (spectra $\ldots$ for downlink).
Optionally a SAVE DESCRIPTORS and/or a DISPLAY \& SAVE DUBIOUS INFO.
can be performed after a SEARCH but before a SELECT.

\subsubsection{Search}% criteria}
Options 1 and 2 in the main menu both initiate a session of prompting for search
criteria, the only difference is that the first option  includes examples
and is therefore appropriate for the novice user.

The selection criteria are divided into four classes as follows;
\begin{enumerate}
\item Camera, and/or image number and/or aperture details.
\item Sky position.
\item IUE Object class.
\item Homogeneous ID.
\end{enumerate}
Whenever you request a search you will be prompted with each of these;
if you do not want to restrict the search with one of these criteria,
you simply type $<$CR$>$ in response to the appropriate prompt.
You may specify more than one item in any search class. For
example you may want to search for several image numbers or object classes.
QUEST will continue to prompt for such items until you indicate the end of your
list by replying to the prompt just with a $<$CR$>$.
QUEST will search for those spectra which satisfy the four sets of
criteria and advise you of the number of spectra which have been found.

The four classes are described in more detail below:
\begin{description}
\item [Camera, image number and aperture details.]
There is a choice from the four IUE cameras SWP, SWR, LWR and
LWP; aperture can be L or S (corresponding to large and small respectively).
Image numbers are integers greater than or equal to 1000.
For example should you want to confine the search to images taken with the SWP
camera you should type SWP.
A single image can be chosen by completely specifying it at this point.
{\it e.g.} SWP4276L.
\item [Sky position.] If an RA, DEC, is specified QUEST will find
those spectra whose RA, DEC fall within a window centered at the given position.
The size of the window is determined from the accuracy to which the RA,DEC
is given as shown in the table below.
Co-ordinates for each must be entered as 1, 2, 4 or 6 digit numbers.
For example you may specify an R.A. of 5 hours as 5, but if you want to
enter minutes you must type 0500, or 050000.
Should you wish to search only on RA or DEC the other co-ordinate can be
given as `ALL' or `*'.
For example
`ALL $-12$' specifies any RA, and a DEC between $-8^o$ and $-16^o$ (that is
$-12^o\pm4^o$).
The search window sizes are:

\begin{center}
\begin{tabular}{|ll|ll|} \hline
\multicolumn{2}{|c|}{\bf R.A.}&\multicolumn{2}{c|}{\bf DEC.} \\
Accuracy used &Window size&Accuracy used &Window size\\ \hline
Hours only&$\pm$30 mins&Degrees Only&$\pm$4 Degs\\
Hours \& minutes&$\pm$10 mins&Degrees \& minutes &$\pm$25 mins\\
Hours, mins \& secs&$\pm$30 secs&Degrees, mins \& secs&$\pm$1 min\\ \hline
\end{tabular}
\end{center}
\item [Object class.] All images are allocated an object class. For example,
quasars are object class 85.
A full list of object classes will be displayed if a `?' is entered in
response to the object class prompt.
\item [Homogeneous ID.] An attempt has been made to standardise the
way in which names of objects have been  entered in the IUE log.
However this is not yet completely reliable.
For example if you enter a HD number at this point you will find that
set of spectra for which the object was specified in that way, but
you may fail to find those which may have been entered in
an alternative way. Specifying an RA,DEC is usually a better way to find
spectra of a particular object.
\end{description}
{\bf Examples:}

In the session shown below the user is looking for images of HD52973 taken
with the SWP camera with the large aperture. (The user's responses are the
text following the `$>$' prompt.)
\begin{footnotesize}
\begin{verbatim}
     $ QUEST
         ENTER   -    YOUR USER ID
                 -    "R"    TO GET AN ID
                 -    "D"    TO DISPLAY A COUNTRY'S ID'S
                 -    "E"    TO END

                                        N O T E:-
          A) DO NOT USE SOMEONE ELSE'S ID, YOU WILL WIPE OUT HIS FILES, AND HE YOURS
          B) YOUR FILE NAMES WILL BE YOUR ID + STANDARD QUALIFIERS (E.G.'.SPE')

     YOUR ID/?/R/D/E> UKABC

      PLEASE SELECT WHICH OF THE FOLLOWING IS YOUR REASON GETTING SPECTRA

              1  - RECOVERY, SENDS YOU -> SELECT, USE IF DOWNLINK FAILED
              2  - CURIOSITY
              3  - M.SC. PROJECT
              4  - PH.D. PROJECT
              5  - UNDERGRADUATE CLASS PROJECT
              6  - NORMAL RESEARCH

     1 TO 6 OR E(STOP QUEST) OR ? > 2
                       +++  OPENING UKABC.SPE     (STATUS=OLD)  +++
                       +++  OPENING UKABC.SPE     (STATUS=NEW)  +++

                                  M A I N     M E N U
              1      -      SEARCH FOR A SET OF SPECTRA
              2      -      SEARCH FOR SPECTRA - SHORT PROMPT
              3      -      DISPLAY WHAT WAS FOUND BY SEARCH
              4      -      SELECT SPECTRA FROM THOSE FOUND BY SEARCH (FOR DOWNLINK)
              5      -      SAVE DESCRIPTORS OF WHAT LAST SEARCH FOUND (FOR DOWNLINK)
              6      -      DISPLAY & SAVE DUBIOUS INFO.FOR WHAT LAST SEARCH FOUND
              ?      -      HELP ME, N = ONE OF ABOVE ANSWERS

      1 TO 6, ? (HELP) OR  E (END) > 2
                                     E X A M P L E:-
     "SWP4123L ,  3000"   = (SPECTRUM FOR SWP CAMERA, IMAGE NO.4123 & LARGE APERTURE)
        P L U S    (ALL SPECTRA FOR IMAGE NUMBER 3000, FOR ANY CAMERA OR APERTURE)

     CAMERA,IM.NO.& AP'S/<CR>/?/R/X > SWPL
     CAMERA,IM.NO.& AP'S/<CR>/?/R/X >

                                     E X A M P L E:-
                                     ----------
       "10 -12, ALL 2030"  =(10HRS+/-30MINS & -12DEGS+/-4DEGS) OR DEC=20.5+/-0.5DEG

     RA+DEC.PAIRS/<CR>/?/X >
                                      E X A M P L E:
            "35 , 53"    =  ALL SPECTRA FOR OBJECTS WITH IUE CLASSES 35 OR 53

     OBJECT CLASSES/<CR>/?/X >
                                      E X A M P L E:
                     "NGC136, "    =   ALL SPECTRA FOR BOTH OBJECTS

     HOMOGENEOUS ID'S/<CR>/?/X > HD52973
     HOMOGENEOUS ID'S/<CR>/?/X >

                                10   R E C O R D S      F O U N D

                                  M A I N     M E N U
              1      -      SEARCH FOR A SET OF SPECTRA
              2      -      SEARCH FOR SPECTRA - SHORT PROMPT
              3      -      DISPLAY WHAT WAS FOUND BY SEARCH
              4      -      SELECT SPECTRA FROM THOSE FOUND BY SEARCH (FOR DOWNLINK)
              5      -      SAVE DESCRIPTORS OF WHAT LAST SEARCH FOUND (FOR DOWNLINK)
              6      -      DISPLAY & SAVE DUBIOUS INFO.FOR WHAT LAST SEARCH FOUND
              ? [N]  -      HELP ME, N = ONE OF ABOVE ANSWERS

     1 TO 6, ? (HELP) OR  E (END) > 5

              SPECTRAL DESCRIPTORS ARE BEING SAVED ON UKABC.DES - PLEASE WAIT

                                  M A I N     M E N U
              1      -      SEARCH FOR A SET OF SPECTRA
              2      -      SEARCH FOR SPECTRA - SHORT PROMPT
              3      -      DISPLAY WHAT WAS FOUND BY SEARCH
              4      -      SELECT SPECTRA FROM THOSE FOUND BY SEARCH (FOR DOWNLINK)
              5      -      SAVE DESCRIPTORS OF WHAT LAST SEARCH FOUND (FOR DOWNLINK)
              6      -      DISPLAY & SAVE DUBIOUS INFO.FOR WHAT LAST SEARCH FOUND
              ? [N]  -      HELP ME, N = ONE OF ABOVE ANSWERS

     1 TO 6, ? (HELP) OR  E (END) > 4

     REF>CAM+IMAGE!   HOMOGENEOUS  !  R.A. !  DECL. !OBSERV.!EXP.T. !OB!USAGE/ !EXP.!
     NO.>+APERTURE!       ID       !HH:MMSS! DD:MMSS! DATE  !SECS.  !CL!DUBIOUS!CODE!
       1>SWP 7271L!HD52973         ! 7: 1 8! 20:3842!30NOV79! 4200.0!53! U     !    !
       2>SWP 6379L!HD52973         ! 7: 1 8! 20:3842! 4SEP79! 2399.0!53! U     !    !
       3>SWP 6383L!HD52973         ! 7: 1 8! 20:3842! 4SEP79! 4800.0!53! U  D  !    !
       4>SWP10478L!HD52973         ! 7: 1 8! 20:3842!26OCT80! 3600.0!53! U     !    !
       5>SWP10514L!HD52973         ! 7: 1 8! 20:3842!31OCT80! 3000.0!53!       !    !
       6>SWP 7442L!HD52973         ! 7: 1 8! 20:3859!20DEC79! 5400.0!53!       ! 301!
       7>SWP 7443L!HD52973         ! 7: 1 8! 20:3859!20DEC79! 9600.0!53! U     ! 503!
       8>SWP17927L!HD52973         ! 7: 1 8! 20:3842!12SEP82! 8820.0!53!       !    !
       9>SWP17939L!HD52973         ! 7: 1 8! 20:3842!13SEP82!10800.0!53!       !    !
      10>SWP17915L!HD52973         ! 7: 1 8! 20:3842!11SEP82! 7800.0!53! U     !    !
                           <<<<  E N D    O F    L I S T  >>>>
        GIVE REF.NOS.(E.G. 3,8) AND/OR RANGES(E.G.3-8) TO SELECT/END/OR LAST PAGE
     PREV.PAGE (-)/E/SELECT LIST > 1-4,7,10
     PREV.PAGE (-)/E/SELECT LIST >
      END ? (Y/N) Y
     WANT 'USAGE' DATA FOR SPECTRA SELECTED FILED (<CR>/N/?) > N

                     THE END OF A QUEST
\end{verbatim}
\end{footnotesize}
\normalsize
The appropriate responses for other possible selection criteria are as follows:

\footnotesize
\begin{center}
\begin{tabular}{|l|l|l|l|l|} \hline
\footnotesize
{}&\multicolumn{4}{c|}{\bf Responses to prompts} \\ \hline
{\bf Selection criteria}&{\bf Camera {\it etc.}}&{\bf RA,DEC}&{\bf Obj. Class}
&{\bf Hom. ID}\\ \hline
Image SWP2347L &SWP2347L&&&\\
SWP images of dwarf novae &SWP&&56&\\
LWR large aperture images of HD52973&LWR L&&&HD52973\\
Quasars in the Southern Hemisphere&&ALL 4,ALL 12,ALL 20&85&\\
RA=9hrs$\pm$0.5hrs and DEC= 12$\pm$0.5&&9 12 &&\\
RA=9hrs 20min$\pm$10mins &&0920 ALL &&\\
RA=9hrs 20min$\pm$30secs &&092000 ALL &&\\ \hline
\end{tabular}
\end{center}

\normalsize
\subsubsection{Display}
Having found a set of spectra you may wish to inspect them before selecting
some or all for downlinking.
The DISPLAY option displays the results of a search one screen-full at a time,
up to a maximum of 400 spectra.
This information can be written to a file (see following section).

\subsubsection {Save Descriptors}

The option is available to write the results of a search to a file.
If this is selected the details
of the search request
followed by the details of the search result -- as produced by a DISPLAY
command -- are written to the file UKABC.DES.
This file is useful if you prefer to examine the possibilities at length
rather than make an immediate decision.

\subsubsection {Display and Save Dubious Information}

Dubious comments -- if a D appears in the USAGE/DUBIOUS column for
a spectrum when
the results of a search are displayed,
this indicates a comment of some sort is associated with that particular image.
Possible comments are that the spectrum has an  uncertain exposure time, or
was taken in trailed mode {\it etc.}
This information can be examined by choosing
the DISPLAY \& SAVE DUBIOUS INFO. on QUEST's main menu.
Any information generated by this selection will be written not
only to the screen, but also to a file UKABC.AUX.

\subsubsection{Select}

When you choose to select spectra,
the list of the search results are displayed.
To select all the spectra, you can reply 1--400 immediately.
Alternatively a subset of spectra can be selected. For example the reply
2--5,9,37 will select only spectra number 2,3,4,5,9 and 37 in the list.
If spectra are specified twice ({\it e.g.} 2--10, 7,8) they will
only be written once in the output file.

QUEST terminates automatically after a SELECT.

\subsubsection {Usage data}

The USSP maintains a record of the users who select each spectrum.
Should any of the spectra chosen in a session have been selected previously
the current user will be invited to inspect the usage information just before
QUEST terminates.
In the sample session above the user responded negatively to this
invitation but a reply of Y or <CR> will result
in a list of who has previously selected the chosen spectra.
This information is also written to the file UKABC.AUX.


\subsection{UNSPL --- Downlinking and reformatting spectra}

The purpose of this program is two-fold:
\begin{enumerate}
\item To split up and reformat the  files written by QUEST into
one file per spectrum.

\item To copy the files from ULDA\$HOST\_DIR to your local
computer\footnote{In the first instance you may of course wish to write the files output by UNSPL
to your area on the database microvax.
Obviously such files will be no longer be in the QUEST output format.
Should you subsequently wish to copy such files to a different computer only
an ordinary COPY or TRANSFER command is needed.}.
\end{enumerate}

The program is invoked by typing `UNSPL' from your local computer whereupon
you will be prompted to reply to three questions.

Firstly, the user is invited to choose an output format.
The range of options will be listed if the user replies with a ?
Several options including MIDAS, ASCII and FITS format are provided as standard.
(Selecting the ASCII format results in a text file which contains one
wavelength,
flux pair and the associated epsilon on each line.)
An additional format has been provided for UK users.
Selecting `DIPSO' format will result in files which can be read with the DIPSO
`USSPRD' command (see below).
More output formats may be provided in the future.

Next the user will be asked if the spectra should be downlinked to which an
affirmative reply should be given.

Finally the user will be prompted for the USSP identifier used whilst
selecting the spectra -- UKABC in the example shown below.

UNSPL will proceed to copy and unscramble/split all files .SPE, .DES, .AUX
in the STADAT directory ULDA\$HOST\_DIR which have this identifier
as the filename.

For example if your USSP identifier is UKABC and you want your output files in
DIPSO format an UNSPL session might proceed as follows
\begin{verbatim}
      $ UNSPL

        UNSPL> OUTPUT SPECTRAL FILES FORMAT : DIPSO
        UNSPL> DOWNLINK DATA FROM HOST SITE [Y/N] ?Y
        UNSPL> USSP USER IDENTIFICATION : UKABC
        DOWNLINKING FILE STADAT::DISK$STARDATA:[JM.USSP.USER]UKABC.SPE
                               END-OF-FILE - RECORDS COPIED :   1
        DOWNLINKING FILE STADAT::DISK$STARDATA:[JM.USSP.USER]UKABC.DES
                               END-OF-FILE - RECORDS COPIED : 262
        DOWNLINKING FILE STADAT::DISK$STARDATA:[JM.USSP.USER]UKABC.AUX
                               END-OF-FILE - RECORDS COPIED :   5
                                   DIPSO FORMAT SWP17368L.ULD
                               END-OF-FILE - NUMBER OF SPECTRA :   1
\end{verbatim}
In this particular example only one spectrum SWP17368L had been selected
from those found by the search with QUEST.
UNSPL produced the file SWP17368L.ULD in the users directory.
In this case both the list of the full results of the search and the
auxiliary information  had been requested.
The files UKABC.DES and UKABC.AUX are text files which can be typed or printed.

\section{Spectral files format}

Each data-point has an error code called an epsilon associated with it.
These are generated by  IUESIPS and have the following meanings:

\begin{center}
\begin{tabular}{|rl|} \hline
Epsilon& Condition\\ \hline
100&No special conditions\\
$-$200&Extrapolated at upper end of ITF\\
$-$220&Microphonic noise\\
$-$250&Filtered bright spot\\
$-$300&Unfiltered bright spot\\
$-$800&Reseau in extracted spectral region\\
$-$1600&Saturated pixel or excessive ITF extrapolation\\
$-$3200&Pixel outside photometrically corrected region\\
&(Low dispersion LBLS files only) \\ \hline
\end{tabular}
\end{center}

If the DIPSO format was selected the spectrum could be examined using
the DIPSO USSPRD and USSPCLIP commands as follows: {\sl (The user's input is
shown in capitals, with explanations following the exclamation marks.)}
\begin{verbatim}
     $ DIPSO
       Welcome to DIPSO 2.2
     > USSPRD SWP17368L                ! read the image
     > PUSH                            ! push it on to DIPSO stack
     > DEV GRAPHPACK                   ! device assignment (GKS workstation name)
     > PM                              ! plot graph
     > QUIT
\end{verbatim}
In the above example data with associated error flags more negative than $-250$
were discarded on input.
However the user can specify a different discrimination level for rejecting
data.
The descriptions of the two relevant DIPSO commands are as follows:
{\sl( Parameters enclosed in square brackets are optional.)}
\begin{description}
\item{\bf USSPRD filename$[$.typ$]$ $[${\it epsmin}$]$} --
reads the file into DIPSO. The filename extension defaults to `.ULD'.
DIPSO rejects points on input if they are flagged with an epsilon less than or
equal to {\it epsmin} (which defaults to $-251$), leaving the spectrum in
the current arrays.
It is forbidden to set epsmin less than or equal $-1600$, since this would
result in totally unflagged, certainly bad, data being acquired.

If epsmin is given a value greater than zero, then ALL datapoints are
read into the next available stack entry, and the epsilon array into
the subsequent stack entry.   More subtle doctoring of the data is then
possible, using USSPCLIP {\it(q.v.).}   However, it is recommended that the
default epsmin be accepted unless you really know what you're doing,
and have good reasons to choose a different value.

\item{\bf USSPCLIP {\it epsmin} $n1\ [n2\ w1\ w2]$} --
clips points out of IUE USSP spectra which have `epsilons' less than
or equal to epsmin.
The data are expected to have been previously read into the stack
using the USSPRD command (with its epsmin parameter given a positive
value);
$n1$ is the stack entry of the flux data, and $n2$ (which defaults
to $n1+1$) that of the epsilon array.
The clipping is done over the wavelength range $w1<w<w2$ (default:
full wavelength range).

\end{description}

The reader  is referred to SUN/50 for a full description of DIPSO.

\end{document}

