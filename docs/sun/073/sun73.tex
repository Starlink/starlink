\documentclass[11pt,twoside]{article}
\pagestyle{myheadings}

% -----------------------------------------------------------------------------
% ? Document identification
\newcommand{\stardoccategory}  {Starlink User Note}
\newcommand{\stardocinitials}  {SUN}
\newcommand{\stardocsource}    {sun\stardocnumber}
\newcommand{\stardocnumber}    {73.1}
\newcommand{\stardocauthors}   {M D Lawden}
\newcommand{\stardocdate}      {21 February 1990}
\newcommand{\stardoctitle}     {FORCHECK\\[2ex]
                               A Fortran Verifier and Programming Aid}
% ? End of document identification

% -----------------------------------------------------------------------------

\newcommand{\stardocname}{\stardocinitials /\stardocnumber}
\markright{\stardocname}
\setlength{\textwidth}{160mm}
\setlength{\textheight}{230mm}
\setlength{\topmargin}{-2mm}
\setlength{\oddsidemargin}{0mm}
\setlength{\evensidemargin}{0mm}
\setlength{\parindent}{0mm}
\setlength{\parskip}{\medskipamount}
\setlength{\unitlength}{1mm}

% -----------------------------------------------------------------------------
%  Hypertext definitions.
%  ======================
%  These are used by the LaTeX2HTML translator in conjunction with star2html.

%  Comment.sty: version 2.0, 19 June 1992
%  Selectively in/exclude pieces of text.
%
%  Author
%    Victor Eijkhout                                      <eijkhout@cs.utk.edu>
%    Department of Computer Science
%    University Tennessee at Knoxville
%    104 Ayres Hall
%    Knoxville, TN 37996
%    USA

%  Do not remove the %begin{latexonly} and %end{latexonly} lines (used by 
%  star2html to signify raw TeX that latex2html cannot process).
%begin{latexonly}
\makeatletter
\def\makeinnocent#1{\catcode`#1=12 }
\def\csarg#1#2{\expandafter#1\csname#2\endcsname}

\def\ThrowAwayComment#1{\begingroup
    \def\CurrentComment{#1}%
    \let\do\makeinnocent \dospecials
    \makeinnocent\^^L% and whatever other special cases
    \endlinechar`\^^M \catcode`\^^M=12 \xComment}
{\catcode`\^^M=12 \endlinechar=-1 %
 \gdef\xComment#1^^M{\def\test{#1}
      \csarg\ifx{PlainEnd\CurrentComment Test}\test
          \let\html@next\endgroup
      \else \csarg\ifx{LaLaEnd\CurrentComment Test}\test
            \edef\html@next{\endgroup\noexpand\end{\CurrentComment}}
      \else \let\html@next\xComment
      \fi \fi \html@next}
}
\makeatother

\def\includecomment
 #1{\expandafter\def\csname#1\endcsname{}%
    \expandafter\def\csname end#1\endcsname{}}
\def\excludecomment
 #1{\expandafter\def\csname#1\endcsname{\ThrowAwayComment{#1}}%
    {\escapechar=-1\relax
     \csarg\xdef{PlainEnd#1Test}{\string\\end#1}%
     \csarg\xdef{LaLaEnd#1Test}{\string\\end\string\{#1\string\}}%
    }}

%  Define environments that ignore their contents.
\excludecomment{comment}
\excludecomment{rawhtml}
\excludecomment{htmlonly}

%  Hypertext commands etc. This is a condensed version of the html.sty
%  file supplied with LaTeX2HTML by: Nikos Drakos <nikos@cbl.leeds.ac.uk> &
%  Jelle van Zeijl <jvzeijl@isou17.estec.esa.nl>. The LaTeX2HTML documentation
%  should be consulted about all commands (and the environments defined above)
%  except \xref and \xlabel which are Starlink specific.

\newcommand{\htmladdnormallinkfoot}[2]{#1\footnote{#2}}
\newcommand{\htmladdnormallink}[2]{#1}
\newcommand{\htmladdimg}[1]{}
\newenvironment{latexonly}{}{}
\newcommand{\hyperref}[4]{#2\ref{#4}#3}
\newcommand{\htmlref}[2]{#1}
\newcommand{\htmlimage}[1]{}
\newcommand{\htmladdtonavigation}[1]{}

% Define commands for HTML-only or LaTeX-only text.
\newcommand{\html}[1]{}
\newcommand{\latex}[1]{#1}

% Use latex2html 98.2.
\newcommand{\latexhtml}[2]{#1}

%  Starlink cross-references and labels.
\newcommand{\xref}[3]{#1}
\newcommand{\xlabel}[1]{}

%  LaTeX2HTML symbol.
\newcommand{\latextohtml}{{\bf LaTeX}{2}{\tt{HTML}}}

%  Define command to re-centre underscore for Latex and leave as normal
%  for HTML (severe problems with \_ in tabbing environments and \_\_
%  generally otherwise).
\newcommand{\setunderscore}{\renewcommand{\_}{{\tt\symbol{95}}}}
\latex{\setunderscore}

% -----------------------------------------------------------------------------
%  Debugging.
%  =========
%  Remove % from the following to debug links in the HTML version using Latex.

% \newcommand{\hotlink}[2]{\fbox{\begin{tabular}[t]{@{}c@{}}#1\\\hline{\footnotesize #2}\end{tabular}}}
% \renewcommand{\htmladdnormallinkfoot}[2]{\hotlink{#1}{#2}}
% \renewcommand{\htmladdnormallink}[2]{\hotlink{#1}{#2}}
% \renewcommand{\hyperref}[4]{\hotlink{#1}{\S\ref{#4}}}
% \renewcommand{\htmlref}[2]{\hotlink{#1}{\S\ref{#2}}}
% \renewcommand{\xref}[3]{\hotlink{#1}{#2 -- #3}}
%end{latexonly}
% -----------------------------------------------------------------------------
% ? Document-specific \newcommand or \newenvironment commands.
% ? End of document-specific commands
% -----------------------------------------------------------------------------
%  Title Page.
%  ===========
\renewcommand{\thepage}{\roman{page}}
\begin{document}
\thispagestyle{empty}

%  Latex document header.
%  ======================
\begin{latexonly}
   CCLRC / {\sc Rutherford Appleton Laboratory} \hfill {\bf \stardocname}\\
   {\large Particle Physics \& Astronomy Research Council}\\
   {\large Starlink Project\\}
   {\large \stardoccategory\ \stardocnumber}
   \begin{flushright}
   \stardocauthors\\
   \stardocdate
   \end{flushright}
   \vspace{-4mm}
   \rule{\textwidth}{0.5mm}
   \vspace{5mm}
   \begin{center}
   {\Huge\bf  \stardoctitle}
   \end{center}
   \vspace{5mm}
\end{latexonly}

%  HTML documentation header.
%  ==========================
\begin{htmlonly}
   \xlabel{}
   \begin{rawhtml} <H1> \end{rawhtml}
      \stardoctitle
   \begin{rawhtml} </H1> \end{rawhtml}

% ? Add picture here if required.
% ? End of picture

   \begin{rawhtml} <P> <I> \end{rawhtml}
   \stardoccategory\ \stardocnumber \\
   \stardocauthors \\
   \stardocdate
   \begin{rawhtml} </I> </P> <H3> \end{rawhtml}
      \htmladdnormallink{CCLRC}{http://www.cclrc.ac.uk} /
      \htmladdnormallink{Rutherford Appleton Laboratory}
                        {http://www.cclrc.ac.uk/ral} \\
      \htmladdnormallink{Particle Physics \& Astronomy Research Council}
                        {http://www.pparc.ac.uk} \\
   \begin{rawhtml} </H3> <H2> \end{rawhtml}
      \htmladdnormallink{Starlink Project}{http://www.starlink.ac.uk/}
   \begin{rawhtml} </H2> \end{rawhtml}
   \htmladdnormallink{\htmladdimg{source.gif} Retrieve hardcopy}
      {http://www.starlink.ac.uk/cgi-bin/hcserver?\stardocsource}\\

%  HTML document table of contents. 
%  ================================
%  Add table of contents header and a navigation button to return to this 
%  point in the document (this should always go before the abstract \section). 
  \label{stardoccontents}
  \begin{rawhtml} 
    <HR>
    <H2>Contents</H2>
  \end{rawhtml}
  \htmladdtonavigation{\htmlref{\htmladdimg{contents_motif.gif}}
        {stardoccontents}}

\end{htmlonly}

% -----------------------------------------------------------------------------
% ? Latex document Table of Contents (if used).
%  ===========================================
\newpage
\begin{latexonly}
   \setlength{\parskip}{0mm}
   \tableofcontents
   \setlength{\parskip}{\medskipamount}
   \markright{\stardocname}
\end{latexonly}
% ? End of Latex document table of contents
% -----------------------------------------------------------------------------
\newpage
\renewcommand{\thepage}{\arabic{page}}
\setcounter{page}{1}

\section{Introduction\xlabel{introduction}}

FORCHECK is a Fortran verifier and programming aid which has been purchased
from Polyhedron software and installed on the Starlink Database computer
(STADAT) for the use of all Starlink users.
It was developed by Erik W.\ Kruyt at Leiden University.
It is only available on STADAT and is not installed on any other Starlink
nodes.

FORCHECK is a valuable tool for all serious Fortran programmers.
Its capabilities fall roughly into four areas:
\begin{description}
\item [Validation]:

It can assist in the development of portable programs by checking code for
conformance to the ANSI standard X3.9-1978.
This check is very comprehensive, and far more complete than that available
with most compilers.
However, it can also deal with non-standard code.
In fact, you can, by a single command, customise it to deal with
different dialects.

\item [Global checks]:

It addresses the problem of inter-module consistency which has long been a
major source of error and confusion in large Fortran programs.
It does this by checking that the number, type, and size of elements in both
sub-program argument lists and COMMON blocks are the same throughout a program.
It also identifies recursive calls and misuse of arguments (e.g.\ changes
to a dummy argument where the actual argument is a constant).
Ensuring inter-module consistency in Fortran need now be no more of a problem
than it is in Ada or Modula-2.

\item [Maintenance]:

It identifies `clutter' --- the unused variables, COMMON blocks, INCLUDE files,
and code fragments which accumulate in old programs, and which make maintenance
such a time-consuming and costly task.
Software maintenance typically costs far more than initial development, and
FORCHECK is one of the few software tools available which can help.
It can also be used in conjunction with SPAG
(see \xref{SUN/63}{sun63}{}), Polyhedron's
`spaghetti code unscrambler', to `re-condition' old Fortran programs, bringing
them close to modern standards at a fraction of the cost of re-writing.

\item [Documentation]:

It composes cross-reference charts for constants, variables, COMMONs,
sub-programs, INCLUDE files, and I/O.
In addition, it can produce a `call tree', which shows the calling structure
of the program in diagrammatic form.
Automatically composed documentation of this type is an invaluable addition
to the system documentation.
As often as not, it is the only reliable source of information about old
programs.
\end{description}

A major feature of FORCHECK is its library facility.
All global program information may be stored in libraries.
New or changed modules can be analysed in the context of the entire program
without analysing the whole program source anew.

FORCHECK supports all of ANSI standard Fortran and most commonly occuring
extensions.

\section{Use\xlabel{use}}

In order to learn how to use FORCHECK properly you must obtain a copy of the
User's Guide from your Site Manager.
However, if you just want to try it out, you can do so easily by first logging
into STADAT from your local node:
\begin{verbatim}
    $ SET HOST STADAT

             Starlink Database microVAX II

    Username: <enter the name used to access STADAT from your site>
    Password: <ask your Site Manager to tell you this>
\end{verbatim}
and then copying into your working directory a file containing the source of
one of your Fortran programs.
Then analyse the program by entering the command:
\begin{verbatim}
    $ FORCHK/LIST=LISTFILE <filename>
\end{verbatim}
You can then copy the file LISTFILE.LIS into your local directory and print it
out for examination, or just look at it interactively.
This will give you a good idea of the sort of output generated by FORCHECK
and you can then decide whether or not to investigate it further.

A reference card which summarises the facilities of the FORCHK command and
the meaning of the cross-reference codes used in the program analysis is
attached to this note.

\section{Compiler Emulations\xlabel{compiler_emulations}}

FORCHECK can emulate various compilers, as specified on page 2-11 of the
User's Guide.
The default (specified by system logical name FCK\$CNF) is the VAX Fortran
compiler.
If you wish to use another compiler emulator you must define a process
logical name called FCK\$CNF.
You can either use one of the configuration files stored in
DISK\$RLSTAR\_SYS:[FORCHECK], or you can copy one of them into your working
directory and edit it as desired.
For example, if you want FORCHECK to emulate the Ryan-McFarlands
Fortran-77 compiler, you can either:
\begin{verbatim}
    $ DEFINE FCK$CNF DISK$RLSTAR_SYS:[FORCHECK]FCKCNF.RM
\end{verbatim}
or:
\begin{verbatim}
    $ COPY DISK$RLSTAR_SYS:[FORCHECK]FCKCNF.RM *.*
    $ DEFINE FCK$CNF FCKCNF.RM
\end{verbatim}
before running FORCHECK.
If you just want to compare your program with the ANSI standard, enter:
\begin{verbatim}
    $ FORCHK/STANDARD/LIST=LISTFILE <filename>
\end{verbatim}
\end{document}
