\documentclass[twoside,nolof]{starlink}

% ? Specify used packages
\makeindex
% ? End of specify used packages

% *****************************************************************************
%
%    DO NOT EDIT THIS DOCUMENT IN ONE PIECE.  This document is a composite
%    document assembled automatically from fragments.  If you need to edit
%    this document please contact the editor (starlink@jiscmail.ac.uk) to obtain
%    the fragment you need to edit, or provide full details of the required
%    changes.
%
% *****************************************************************************

% -----------------------------------------------------------------------------
% ? Document identification
% Fixed part
\stardoccategory    {Starlink User Note}
\stardocinitials    {SUN}
\stardocsource      {sun\stardocnumber}

% Variable part - replace [xxx] as appropriate.
\stardocnumber      {1.24}
\stardocauthors     {ed. S.\, E.\, Rankin}
\stardocdate        {5 August 2003}
\stardoctitle       {STARLINK\linebreak Software Collection}
\stardocabstract  {%
The Starlink Software Collection is a set of software which is managed
and distributed by the Starlink Project.  Some of the software was
written by members of the Project, but some of it comes from outside
the Project.  This note describes the functions of the individual items
in the Collection and provides an overview of the software so that
readers can identify the items they need.

The software is classified into four main divisions:

\begin{itemize}

\item \textbf{Packages} -- are large collections of programs for people
who want to analyse, convert, and display data.  They are subdivided
into eleven classes to help you find what you want.

\item \textbf{Utilities} -- are small programs devoted to a specific
purpose.  For example, they help you prepare for observations, write
documents, and write programs.

\item \textbf{Subroutine Libraries} -- are for programmers writing
astronomical software.  They provide facilities such as astronomical
calculations, data management and graphics.

\item \textbf{Infrastructure} -- are items which are mainly of interest to
people writing programs within the Starlink Software Environment.  They
are included for completeness.

\end{itemize}

Each item is described in sufficient detail for you to decide whether or
not to investigate it further.
If you want to find out more about an item, follow the document references
given opposite the item name.
If you are using the hypertext version of this document, the most up-to-date
document references can be found by following the link from the software item
name.
}

% ? End of document identification
% -----------------------------------------------------------------------------
% ? Document specific \providecommand or \newenvironment commands.

\providecommand{\previousdocnumber}{1.23}
\providecommand{\previousdocdate}  {5 August 2003}
\providecommand{\previousdoc}      {SUN/\previousdocnumber , \previousdocdate}

\providecommand{\radec}{$[\alpha,\delta\,]$}

%  LaTeX2HTML symbol.
\providecommand{\latextohtml}{\LaTeX2\texttt{HTML}}

\setcounter{secnumdepth}{3}
\setcounter{tocdepth}{2}
% ? End of document specific commands
% -----------------------------------------------------------------------------
%  Title Page.
%  ===========
\begin{document}
\scfrontmatter

% ? Main text

\section{Introduction}

This document contains a survey of the complete Starlink Software
Collection.  The level of detail should enable you to decide whether or
not to investigate a specific software item further.  Full details of
individual items can be found in the document references.

Installation instructions are contained in
\xref{SUN/212}{sun212}{}.

The strategy guiding the development of Starlink software is described
in \xref{SGP/42}{sgp42}{}.  A major influence on this strategy is the
set of Starlink Software Strategy Groups.  Guidelines for members of
these groups is given in \xref{SGP/44}{sgp44}{}.

New users of Starlink facilities should read the Starlink User's Guide
(\xref{SUG}{sug}{}).

The World Wide Web is an important tool for informing users about the
Starlink Project and its software, and Starlink has published
comprehensive documentation in this medium.  Eventually, this will be
linked together as a single hyper-linked document by utilising the
\htmlref{HTX}{item_HTX} utilities (\xref{SUN/188}{sun188}{}).  One of
the greatest advantages of the Web is its potential for updating and
distributing documents more frequently and rapidly than is practicable
on paper.  For current information about the Starlink software and
documentation you should refer to the information on the
\htmladdnormallink{Starlink Web Site}{http://www.starlink.ac.uk/}
\latex{(\texttt{\htmladdnormallink{http://www.starlink.ac.uk/}{http://www.starlink.ac.uk/}})}.

This edition describes the Collection as available in the Autumn 2002
CD-ROM distribution of December 2002.

\newpage

\section {Changes since the last issue}

\subsection{New packages}

Two new items have been added to the Collection since the previous
edition of this document (\previousdoc) and are indicated as such in the
heading of the appropriate sections.

The new packages are:

\begin{itemize}

\item \htmlref{AUTOASTROM}{item_AUTOASTROM} Autoastrometry of mosaics.
The new document (SUN/242) \\
\textit{\xref{AUTOASTROM -- Autoastrometry for Mosaics}{sun242}{}\/} \\
describes the package. AUTOASTROM is not available under Tru64 Unix.

\item \htmlref{JCMTDR}{item_JCMTDR} Applications for reducing JCMT GSD
on-the-fly data. The documents SC/1 \textit{\xref{JCMTDR Cookbook}{sc1}{}\/}
and SUN/132
\textit{\xref{Applications for Reducing JCMT GSD Data 1.2-2 User's manual}{sun132}{}\/}
describe the package.

\item \htmlref{STARJAVA}{item_STARJAVA} Starlink Java Infrastructure
and Applications Set. The new document (SUN/251)
\textit{\xref{Getting Started with the Starlink Java Infrastructure
and Applications Set}{sun251}{}\/} describes the package and \\
\texttt{/star/starjava/javadocs/index.html} \\
describes the Starlink STARJAVA classes API. There are miscellaneous
documents in \texttt{/star/starjava/docs/} for individula package information and
third party application documentation.

The STARJAVA package contains the following applications and classes. Note,
SPLAT, TREEVIEW, JNIAST and JNIHDS have been moved to the STARJAVA
package, instead of being individual packages.
\newpage
\begin{description}
\item[APPLICATIONS/UTILITIES in STARJAVA]
\item[ FROG V0.1a] Display and analysis of time series data (new)
\item[ SOG V0.1] Son of GAIA (new)
\item[ SPLAT V2.0] Spectral Analysis Tool (update)
\item[ TABLECOPY V0.3] Copy tables from one format to another (new)
\item[ TOPCAT V0.3b] Tool for OPerations on Catalogues and Tables (new)
\item[ TREEVIEW V2.1-3] Hierarchical data viewer (update)

\item[CLASS LIBRARIES in STARJAVA]
\item[ ARRAY V0.2] N-dimensional array manipulation and \\
I/O (new)
\item[ ASTGUI V1.0] AST specific UI components (new)
\item[ AXIS V1.0b2] Third generation Apache SOAP (new)
\item[ COCO V1.0] Java UI for Coco (new)
\item[ DOM4J V0.1] Third party DOM access library \\
(org.dom4j.*) (new)
\item[ FITS V0.1] STARJAVA-specific FITS access (new)
\item[ HDS V0.1] Non-native HDS utility classes (new)
\item[ HDX V0.1] A flexible, extensible, data model for \\
astronomical images, tables and other metadata (new)
\item[ JAIUTIL V1.0] Utility classes for JAI (new)
\item[ JETTY V4.0.4] HTTP Server and Servlet Container \\
(new)
\item[ JNIAST V2.0-4] Java Native interface to AST (update)
\item[ JNIHDS V0.3] Java Native Interface to HDS (update)
\item[ JSKY V2.0] Java Components for Astronomy (new)
\item[ JUNIT V3.7] Third party unit testing framework \\
(junit.*) (new)
\item[ NDX V0.1] N-dimensional astronomical object manipulation and I/O (new)
\item[ PAL V0.1] Positional Astronomy Library (new)
\item[ RV V1.0] Java UI for RV (new)
\item[ TABLE V0.3] Generic table manipulation and I/O (new)
\item[ TAMFITS V0.93] Third party basic FITS access \\
(nom.tam.*) (new)
\item[ UTIL V0.1] Miscellaneous utillity classes (new)
\item[ VOTABLE V1.0] VOTable I/O (new)
\end{description}


The STARJAVA package can be considered to be independent of the standard USSC,
and in future will be distributed separately. All the STARJAVA applications
and classes are distributed under the GPL licence.

The SPLAT documentation (SUN/243) \textit{\xref{SPLAT - A Spectral Analysis
Tool}{sun243}{}\/} has been updated and can be found in \\
\texttt{/star/starjava/docs/splat/sun243.htx/sun243.html}. \\
It also appears in the DOCS package and can be seen with the command
\texttt{showme sun243}.

STARJAVA is not available under Tru64 Unix.
\end{itemize}

\subsection{Withdrawn Packages}

No packages have been withdrawn at this release.

\subsection{Package details}

Some packages listed in this document are not distributed on
Starlink CDs for various reasons including the lack of a Linux port.
These packages are mainly older packages which are not used much and
which have little or no support available.  These packages have amended
descriptions to indicate their availability.

\newpage

\section {Packages}

Starlink aims to provide \emph{maintainable}, \emph{portable}, and
\emph{extensible}\/ applications packages that work in harmony by
sharing a common infrastructure toolkit, standards, conventions, and
above all, a standard data format.  Individual packages are no longer
required to perform all functions, so they are easier to change and are
more adaptable to outside developments.  New functions and user
interfaces can be added as required.  A recent example of this
flexibility is the introduction of the ability to access `foreign data
formats' from all Starlink packages, because they use a common
infrastructure library.

Applications are unified by sharing the same basic data structure --
the \htmlref{NDF}{item_NDF} (extensible $n$-dimensional data format).
This contains an $n$-dimensional data array that can store most
astronomical data, such as spectra, images, and spectral-line data
cubes.  The NDF may also contain information like a title, axis labels
and units, error, and quality arrays.  There are also places in the
NDF, called \emph{extensions}, to store any ancillary data associated
with the data array, even other NDFs.

\newpage

\subsection{General Purpose}

\rule{\textwidth}{0.5mm}

\begin{description}
\item [\htmlref{AIPS}{item_AIPS}]
--- Image processing \hfill \emph{Not distributed}
\item [\htmlref{IDL}{item_IDL}]
--- Interactive data language \hfill \emph{Not distributed}
\item [\htmlref{IRAF}{item_IRAF}]
--- Image reduction and analysis facility
\item [\htmlref{MIDAS}{item_MIDAS}]
--- Munich image data analysis \hfill \emph{Not distributed}
\item [\htmlref{XANADU}{item_XANADU}]
--- GSFC software system
\item [\htmlref{STARJAVA}{item_STARJAVA}]
--- Starlink Java Infrastructure and Applications Set
\end{description}

\rule{\textwidth}{0.5mm}



\subsubsection{AIPS \hfill\xref{SUN/207}{sun207}{} MUD/101}
\label{item_AIPS}
\index{AIPS}

\textbf{Calibrate and edit radio-interferometric data, construct images
using Fourier synthesis techniques and display and analyse these
images.}

The Astronomical Image Processing System (AIPS) is in widespread use by
radioastronomers.  It has a richer set of \emph{general}\/ image
processing functions than any other astronomical software package.

It can:
\begin{itemize}
\item Deconvolve a point-spread function from an image.
\item Transform one image into the coordinate system of another.
\item Fourier transform an image into the complementary Fourier domain,
allowing you to edit the transformed image and transform it back again (useful
for removing stripes from IRAS images).
\item Enhance various features in an image using gradient filters, or even the
Sobel edge enhancement filter.
\item Display successive planes of a data cube as a continuous movie.
\item Display a spectral line data set, using hue to denote velocity and
intensity to denote integral brightness.
\end{itemize}



\subsubsection{IDL \hfill MUD/29,30,31,161}
\label{item_IDL}
\index{IDL}
\textbf{Explore and manipulate data using a comprehensive set of tools.}

The Interactive Data Language (IDL) is proprietary package, written by Dave
Stern of Research Systems Inc, Denver,
It is only available on some Starlink nodes.
At its simplest level it can be used as a very powerful arithmetic and
graphics utility for processing data arrays.

It provides:
\begin{itemize}
\item A programming language.
\item Easy graphics and image display.
\item Mathematical and transcendental functions.
\item Array and string manipulation.
\item Input/output.
\item Type conversion.
\item Other complex operations.
\end{itemize}
It can be used with minimal effort for many applications, or developed into
more complex procedures as required.



\subsubsection{IRAF \hfill \xref{SUN/179}{sun179}{}.MUD/104,105,154,156,157}
\index{IRAF}
\label{item_IRAF}
\textbf{Reduce and analyse data -- general purpose.}

The Image Reduction and Analysis Facility (IRAF) was developed by the
National Optical Astronomical Observatory (Kitt Peak),
and has been adopted by the Space Telescope Science Institute in Baltimore as
the principal data analysis environment for Hubble Space Telescope (HST) data.
(This data can also be reduced by
\htmlref{KAPPA}{item_KAPPA} and
\htmlref{FIGARO}{item_FIGARO}.)
It contains general image processing and graphics applications, plus programs
for reducing and analysing optical astronomical data.
It is as portable and device independent as possible; for example, it includes
its own programming language.
Although it runs on a variety of computers, in practice it usually runs
best on one particular type (Suns at present).

It is a very large system which is supported by extensive documentation.
Every Starlink node should have a copy of the User Handbook.

IRAF is one of several major software environments which are available to
Starlink users and upon which entire data-analysis campaigns can be based.
Other examples include
\htmlref{AIPS}{item_AIPS},
\htmlref{IDL}{item_IDL}, and
\htmlref{MIDAS}{item_MIDAS}, as well as Starlink's own large
collection of infrastructure tools and application packages, and the various
forms of the
\htmlref{FIGARO}{item_FIGARO} system.
Each has its own particular strengths and special capabilities;
\htmlref{IRAF}{item_IRAF} and
\htmlref{AIPS}{item_AIPS}
are the most comprehensive overall.

The excellent support provided by IRAF's home institute (with which Starlink
maintains close contact) means that comparatively little national UK Starlink
support is called for and provided; the same is true for the other overseas
environments mentioned above.

\htmlref{IRAF}{item_IRAF}
is a sound choice for many Starlink users doing data analysis, especially where
compatibility with overseas collaborators is a requirement.
The choice is harder for those wishing to develop major applications of their
own, who may be reluctant to adopt IRAF's non-industry-standard
\verb|SPP| programming language, or who feel uncomfortable with the limitations
of the `flat' IRAF data formats.
Those who need formal guarantees of future support should also be very wary
about committing themselves to any package which is not under UK control.
However, IRAF is an important weapon in the armoury of the average Starlink
data-analysis user, and likely to remain so for some years.

An important part of Starlink's present software plans is to enhance
`interoperability' with other environments, IRAF being of special importance.
Starlink applications can already share data with other packages via
FITS files, but in the case of IRAF this data-interchange capability will be
extended by enabling Starlink applications to read and write IRAF data files.
This will mean that a user can run IRAF in one window on their screen and a
Starlink package (e.g.\ CCDPACK) in another, processing the same datasets
with each.
A further capability, currently being investigated, would be to run
Starlink applications from the IRAF command-line just as if they were
native IRAF applications.


\subsubsection{MIDAS}
\index{MIDAS}
\label{item_MIDAS}

\textbf{Reduce and analyse optical astronomy data within a basic image
processing environment.}

The Munich Image Data Analysis System (MIDAS) was developed by the Image
Processing Group of the European Southern Observatory (ESO).
It consists of a monitor which controls the execution of individual tasks,
and a large set of image processing applications.
Since the results of reducing astronomical images are usually numbers
(not other images), a comprehensive and flexible table system forms
an essential part of the system.
New applications can be added easily, even by non-professional programmers.

Its main features are:
\begin{itemize}
\item Support for computers with VMS and Unix operating systems.
\item Device independent interfaces to peripherals by using special libraries,
e.g.\ AGL for graphics, ID for image display, and TW for terminals.
\item Support for display and hard-copy standards like X11 and PostScript.
\item Easy and flexible integration of user software, written in standard
Fortran and C.
\item Full support for data exchange in FITS format.
\item On-line help, history, and logging facilities.
\item Flexible command and control language with full flow control and
debugging facilities.
\item Extensive interface libraries in Fortran and C for access to the
MIDAS database.
\end{itemize}
It also contains extensive packages in the areas of spectral reduction
(including data in long slit and \'{e}chelle formats), CCD observations, crowded
field photometry, object search and classification, fitting and modeling of
data, astrometry, and statistical analysis.



\subsubsection{STARJAVA \hfill \xref{SUN/251}{sun251}{}}
\index{STARJAVA}
\label{item_STARJAVA}

Starlink is developing a set of Java data reduction and analysis tools
and Java data access classes. The new Java tools and classes are needed
to produce applications in the Virtual Observatory (VO) era, and to
complement the AstroGrid and other VO capabilities such as the International
Virtual Observatory Alliance Data Model.

You can view a section of a proposed International Virtual Observatory
Alliance Data Model, including HDX, NDX and Table data structures with
the WCS astrometry library at:

\url{http://www.starlink.ac.uk/java/java.htm}

The Starlink Java Software (STARJAVA) has had numerous changes and
additions. STARJAVA can now be considered as a separate entity from
the classic Starlink applications and libraries, although in this
release it is provided as a standard Starlink package.

The document (SUN/251)
\textit{\xref{Getting Started with the Starlink Java Infrastructure
and Applications Set}{sun251}{}\/}
describes the package and \\
\texttt{/star/starjava/javadocs/index.html} \\
describes the Starlink STARJAVA classes API. There are miscellaneous
documents in \\
\texttt{/star/starjava/docs/} \\
for individula package information and third party application documentation.

The STARJAVA package contains the following applications and classes. Note,
SPLAT, TREEVIEW, JNIAST and JNIHDS have been moved to the STARJAVA
package, instead of being individual packages.

APPLICATIONS/UTILITIES:

FROG - Display and analysis of time series data.

SOG -  Son of GAIA.

SPLAT - Spectral Analysis Tool.

TABLECOPY - Copy tables from one format to another.

TOPCAT - Tool for OPerations on Catalogues and Tables.

TREEVIEW - Hierarchical data viewer.

CLASS LIBRARIES:

ARRAY - N-dimensional array manipulation and I/O.

ASTGUI - AST specific UI components.

AXIS - Third generation Apache SOAP.

COCO - Java UI for Coco.

DOM4J - Third party DOM access library (org.dom4j.*).

FITS - STARJAVA-specific FITS access.

HDS - Non-native HDS utility classes.

HDX - A flexible, extensible, data model for astronomical
images, tables and other metadata.

JAIUTIL - Utility classes for JAI.

JETTY - HTTP Server and Servlet Container.

JNIAST - Java Native interface to AST.

JNIHDS - Java Native Interface to HDS.

JSKY - Java Components for Astronomy.

JUNIT - Third party unit testing framework (junit.*).

NDX - N-dimensional astronomical object manipulation and I/O.

PAL - Positional Astronomy Library.

RV - Java UI for RV.

TABLE - Generic table manipulation and I/O.

TAMFITS - Third party basic FITS access (nom.tam.*).

UTIL - Miscellaneous utillity classes.

VOTABLE - VOTable I/O.

The STARJAVA package can be considered to be independent of the standard USSC,
and in future will be distributed separately. All the STARJAVA applications
and classes are distributed under the GPL licence.

The SPLAT documentation (SUN/243) \\
\textit{\xref{SPLAT - A Spectral Analysis Tool}{sun243}{}\/} \\
has been updated and can be found in \\
\texttt{/star/starjava/docs/splat/sun243.htx/sun243.html}. \\
It also appears in the DOCS package and can be seen with the command
\texttt{showme sun243}.

STARJAVA is not available under Tru64 Unix.


\subsubsection{XANADU}
\label{item_XANADU}
\index{XANADU}

\textbf{Analyse spectra, timing, and images of X-ray astronomical data
obtained from multiple missions.}

It was obtained from GSFC.
Its home page is:

\url{http://heasarc.gsfc.nasa.gov/docs/xanadu/xanadu.html}.

Its principal components are:


\begin{description}

\item[XSPEC]\index{XSPEC}\index{XANADU!XSPEC} \mbox{}
\label{item_XSPEC}
\textbf{Fit X-ray spectra -- command-driven, interactive.}

It is detector-independent, so it can be used for any spectrometer.
It has been used to analyze data from HEAO-1 A2, Einstein Observatory, EXOSAT,
Ginga, ROSAT, BBXRT, ASCA, CGRO, and IUE.
It has also been used for simulations for XTE and AXAF.


\item[XIMAGE]\mbox{}
\index{XIMAGE}\index{XANADU!XIMAGE}
\label{item_XIMAGE}
\textbf{Display and analysis of multi-mission X-ray images.}

It is instrument-independent and analyses data from any X-ray imaging
detector, provided calibration files are available.  It supports
detailed analysis of EXOSAT CMA, Einstein HRI and IPC, ROSAT PSPC and
HRI, and ASCA GIS and SIS data.  It also supports some basic analysis
of optical, infrared, and radio images.  It has a built in data
simulation program that can simulate images of current and future X-ray
missions (\emph{e.g.}\/ SAX, AXAF, and XMM).

It can:
\begin{itemize}
\item Read images and event files.
\item Rebin images, smooth, and display.
\item Detect and remove sources, statistical analysis.
\item Correct for vignetting, exposure, background, and point spread function.
\item Mosaic images.
\item Contour plots and overlays.
\item Convert sky grids and pixel-coordinates.
\item Change equinox.
\item Slice x/y image.
\item Generate point spread functions.
\item Select circular, annular, and box regions.
\item Interface to saoimage.
\item Extract spectra and lightcurves from event data.
\item Simulate images.
\end{itemize}

Its display and graphic capabilities are based on the PGPLOT graphics
package, which supports most terminals and workstations.  The saoimage
package can also be spawned to display images and select regions.

While XIMAGE is a multimission package, it must first `know' about the
calibration information associated with a mission in order to be able to
make a detailed image analysis.  Some functions are mission-independent
(\emph{e.g.}\/ display), but others are not (\emph{e.g.}\/ source detect).
It will read images from an unknown mission, but beware of trying to make a
detailed analysis -- adding new missions usually requires either adding new
files and/or adding new calls and relinking XIMAGE.


\item[XRONOS]\mbox{}
\index{XRONOS}\index{XANADU!XRONOS}
\label{item_XRONOS}
\textbf{Analyse timing -- general purpose.}

Although designed mostly for X-ray astronomy, it is basically detector
and wavelength-independent.
It has been used to analyse data from the Einstein Observatory, EXOSAT, and
Ginga, as well as optical photometry and helioseismology data.
It includes programs for:
\begin{itemize}
\item Light curve(s).
\item Hardness ratio and colour-colour plotting.
\item Epoch folding.
\item Power spectrum.
\item Autocorrelation and cross-correlation.
\item Time skewness and statistical analysis.
\end{itemize}
It consists of a collection of programs, each dedicated to one task, which can
be run from the control environment provided by the XRONOS program.
This is characterised by three different user interfaces:
`question/answer', `partial question/answer', and `command driven.'
Command files are supported by the last two.
Its applications can be run in command-driven fashion within the EXOSAT
Database System.



XANADU also contains native versions of:

\begin{itemize}
\item \textbf{PGPLOT} \index{XANADU!PGPLOT} \index{PGPLOT}
--- High-level graphics.
\item \textbf{QDP} \index{XANADU!QDP} \index{QDP}
--- Data plotter.
\item \textbf{FITSIO} \index{XANADU!FITSIO} \index{FITSIO}
--- FITS I/O on disk.
\end{itemize}

Of these, PGPLOT and FITSIO are also distributed independently by Starlink.

\end{description}

\newpage

\subsection{Pipeline systems}

\rule{\textwidth}{0.5mm}

\begin{description}
\item [\htmlref{ORAC-DR}{item_ORAC-DR}]
--- UKIRT and JCMT instrument data reduction pipeline
\end{description}

\rule{\textwidth}{0.5mm}



\subsubsection{ORAC-DR  \hfill
\xref{SUN/230}{sun230}{},\xref{SUN/231}{sun231}{},\xref{SUN/232}{sun232}{},
\xref{SUN/246}{sun246}{}}
\label{item_ORAC-DR}
\index{ORAC-DR}

\textbf{General purpose data reduction pipeline}

ORAC-DR is the data reduction component of the ORAC system used
for telescope and instrument control at the United Kingdom Infrared
Telescope (UKIRT) and the James Clerk Maxwell Telescope (JCMT) on Hawaii.
It provides a general purpose automatic data reduction pipeline for the
CGS4, UFTI and IRCAM instruments on UKIRT and the SCUBA instrument on JCMT.

ORAC-DR recipes to drive the reduction of a set of observations made to
a known observation pattern at the telescopes.  The headers data of
each dataset is read to discover which data processing recipe to apply to
it.  The data reduction is performed by tasks from other Starlink packages
such as \htmlref{KAPPA}{item_KAPPA},
\htmlref{CCDPACK}{item_CCDPACK}, \htmlref{SURF}{item_SURF},
\htmlref{POLPACK}{item_POLPACK} and \htmlref{PHOTOM}{item_PHOTOM}.
It can also use \htmlref{GAIA}{item_GAIA} for image display.



\newpage

\subsection{Image Processing \& Photometry}

\rule{\textwidth}{0.5mm}

\begin{description}
\item [\htmlref{ATOOLS}{item_ATOOLS}]
--- Coodinate frame manipulation
\item [\htmlref{CCDPACK}{item_CCDPACK}]
--- CCD data reduction
\item [\htmlref{DAOPHOT}{item_DAOPHOT}]
--- Stellar photometry
\item [\htmlref{ESP}{item_ESP}]
--- Extended surface photometry
\item [\htmlref{EXTRACTOR}{item_EXTRACTOR}]
--- Automatic detection of objects on an astronomical image
\item [\htmlref{GAIA}{item_GAIA}]
--- Graphical image analysis
\item [\htmlref{KAPPA}{item_KAPPA}]
--- Image processing \& visualisation
\item [\htmlref{KAPRH}{item_KAPRH}]
--- Retired KAPPA tasks
\item [\htmlref{PHOTOM}{item_PHOTOM}]
--- Aperture photometry
\item [\htmlref{PISA}{item_PISA}]
--- Position, intensity, and shape analysis
\item [\htmlref{SAOIMAGE}{item_SAOIMAGE}]
--- Astronomical image display
\item [\htmlref{STARMAN}{item_STARMAN}]
--- Stellar photometry
\item [\htmlref{SX}{item_SX}]
--- Data visualisation
\end{description}

\rule{\textwidth}{0.5mm}





\subsubsection{ATOOLS}
\index{ATOOLS}
\label{item_ATOOLS}

\textbf{Manipulation of coordinate frame descriptions.}

ATOOLS is a new package of applications which manipulate descriptions
of coordinate frames, mappings, \emph{etc.}, in the form of
\htmlref{AST}{item_AST} Objects.  Each application within the ATOOLS
package corresponds closely to one of the functions within the AST
library. ATOOLS thus provides a high-level interface to the AST library.



\subsubsection{CCDPACK \hfill \xref{SUN/139}{sun139}{}}
\label{item_CCDPACK}
\index{CCDPACK}

\textbf{Reduce CCD-like data, and mosaic frames together.}

You can debias, remove dark current, pre-flash, flatfield, register,
resample, and normalize your data.

Perhaps the most important and useful feature of the package is its
ability to handle large numbers of data files automatically, so that
repetitive reduction procedures can be handled in a straightforward and
efficient way.  This uses a scheduling system that only requires
knowledge of the frame types (bias, flatfield, etc.) and important CCD
geometric features.  Using this information, it can decide how to
reduce your data and may then run the necessary programs.  The frame
types and detector characteristics can be obtained from FITS headers,
for certain telescopes/CCDs, so your job could be reduced to
identifying the telescope/detector used and the frames you want
reducing.

The automated reduction system can be controlled from an X-based GUI
that has been specifically designed to help novice and/or occasional
users of CCD data (although it is expected to appeal to the more
experienced as well).  Ease-of-use is achieved by limiting the options
to those of immediate concern by providing a selection of known
detectors and by having a context-sensitive help system.  It also aims
to be complete by allowing you to define the necessary geometric
characteristics of your data interactively (if they cannot be obtained
elsewhere).  An equivalent command-line interface is also available.

Its core is a suite of programs to process large amounts of data.
Consequently, \emph{all}\/ the routines process \emph{lists}\/ of data,
and also record progress using a log system.

As well as performing the usual instrumental corrections, you can
also remove defects and generate and propagate data errors.
Debiassing can be performed using only the bias strips as well as
using bias frames (combined to reduce noise levels).
Calibration data can be combined using many different techniques
(mean, median, trimmed mean, etc.), so you can pick a method that
makes the most efficient use of your data.

Data registration is based primarily on the \emph{linear}\/
transformation (allowing offsets, scalings, rotation, and shear),
although more general transformations can be used.

General linear transforms can easily be determined using an interactive
procedure to display and select image features.  Alternatively, if your
datasets are just shifted with respect to each other, you may be able
to register them by using a series of commands which locate all the
objects in all the frames, determine the object-object correspondence,
and then derive the transforms.  A graphical application is also
provided that allows you to select the objects to be used by
identifying image pairs that overlap and have some objects in common.

Data resampling uses registering transforms which it stores inside
your data, removing the need to remember them.  They may also be
applied to `rubber-sheet' the data into novel configurations.

Normalisation and combination (often called mosaicing) is provided in a
single \emph{comprehensive}\/ application, which is designed to deal
with very large datasets.  This uses robust methods to determine scale
and/or zero-point corrections.

The CCDPACK package my be run from the IRAF CL.


\subsubsection{DAOPHOT  \hfill \xref{SUN/42}{sun42}{} MUD/9,10}
\label{item_DAOPHOT}
\index{DAOPHOT}

\textbf{Stellar photometry of crowded fields.}

It does the following tasks:
\begin{itemize}
\item Find objects.
\item Aperture photometry.
\item Obtain the point spread function.
\item Profile-fitting photometry.
\end{itemize}
Profile-fitting in crowded regions is performed iteratively, which improves the
accuracy of the photometry.
It does not directly use an image display (which aids portability), although
three additional routines allow results to be displayed on an image device.
It uses image data in NDF format, which means it is interoperable with other
Starlink packages.



\subsubsection{ESP \hfill\xref{SUN/180}{sun180}{}}
\label{item_ESP}
\index{ESP}

\textbf{Photometry of galaxies and other extended objects.}

It can:
\begin{itemize}
\item Detect/identify flatfielding faults.
\item Remove cosmic ray events.
\item Median-filter images on a defined scale.
\item Determine whole-image statistics including median and modal count,
kurtosis, and skewness.
\item Determine the local background value on a number of different parts
of an image.
\item Perform galaxy profiling using intensity analysis.
\item Perform galaxy profiling using contour analysis.
\item Perform 2-d Gaussian profiling (GAUFIT).
\item Generate galaxy pie-slice cross-sections.
\item Display graphs showing the profiling/cross-section results.
\item Detect faint diffuse objects in an image.
\end{itemize}

It processes images in NDF format, so you can use it in
conjunction with packages like
\htmlref{KAPPA}{item_KAPPA},
\htmlref{CCDPACK}{item_CCDPACK},
\htmlref{FIGARO}{item_FIGARO},
\htmlref{PHOTOM}{item_PHOTOM},
\htmlref{JCMTDR}{item_JCMTDR}, and
\htmlref{PISA}{item_PISA}.
You can define areas to exclude or include in the analysis,
using keyword descriptions in text files.


\subsubsection{Extractor \hfill \xref{SUN/226}{sun226}{}}
\label{item_EXTRACTOR}
\index{EXTRACTOR}
\textbf{Locate and parameterize objects in a 2-d image.}

EXTRACTOR is a program for automatically detecting objects on an
astronomical image and building a catalogue of their properties. It is
particularly suited for the reduction of large scale galaxy-survey
data, but also performs well on other astronomical images.

EXTRACTOR is a development of Emmanuel Bertin's SExtractor
(Source-Extractor) program repackaged for use in the Starlink Software
Environment.  This means that it uses the Starlink parameter system,
accepts images in NDF format and uses the AST library for astrometry.



\subsubsection{GAIA \hfill \xref{SUN/214}{sun214}{}}
\label{item_GAIA}
\index{GAIA}
\textbf{A traditional image display tool (like
\htmlref{SAOIMAGE}{item_SAOIMAGE}), but which can integrate other
programs.}

It is derived from the
\htmladdnormallink{RTD (Real Time Display)}{http://archive.eso.org/skycat/}
tool, developed as part of the
\htmladdnormallink{VLT}{http://www.eso.org/vlt/} project at
\htmladdnormallink{ESO}{http://www.eso.org/}.
RTD is free software under the terms of the GNU copyright.

The current version is a preliminary release which is intended to assess the
impact of GAIA/RTD's enhancements for performing highly interactive
graphical image analysis.
It provides the following functions:
\begin{itemize}
\item Display images in FITS and \htmlref{NDF}{item_NDF} formats (it will
display many other formats using the `on-the-fly' data conversion ability of
NDF).
\item Pan, zoom, and manipulate data range and colour table.
\item Continuously display of cursor position and image data value.
\item Continuously display of the RA and DEC coordinates for suitable
images (\emph{i.e.}\/ FITS images with WCS headers -- Guide Star
Catalog images for instance).
\item Display many images (clones), each in its own window.
\item Annotate in colour, using text and line graphics.
\item Print the displayed image and annotations to a Post\-Script file.
\item Profile in real time (\emph{i.e.}\/ move a line around on the image and see the
image data values displayed as a profile change).
\item Real time pixel value table (the data values in a region about the
 cursor).
\item Aperture photometry --
 A highly interactive environment for controlling the positions, sizes, and
 orientations of circular and elliptical apertures.
 The sky estimates can be made from annuli of these apertures, or from related
 sky apertures.
 The measurements can be in either instrumental magnitudes or mean counts.
\item Patch images --
 Select arbitrarily shaped regions on an image and replace them with a surface
 fit to other regions, together with an artificial noise component.
 An ideal way to remove cosmetic defects from an image.
\item Interactive \htmlref{ARD}{item_ARD} regions --
 Calculate statistics, mask out and extract arbitrarily shaped parts of your
 images, define ARD regions for other programs to use.
\item Blink images --
 Animate displayed images (as quickly as your hardware/CPU combination
 allows), or cycle through them by hand.
\end{itemize}

The GAIA package may be run from the IRAF CL.


\subsubsection{KAPPA  \hfill \xref{SUN/95}{sun95}{}}
\label{item_KAPPA}
\index{KAPPA}
\textbf{General-purpose functions, particularly suitable for image
processing, data visualisation, and manipulating \htmlref{NDF}{item_NDF}
components.}

KAPPA is the backbone of Starlink's application packages.  Its
facilities integrate with the more specialised Starlink packages
described in this survey.  Thus, the functionality of KAPPA should not
be regarded in isolation.

It can process data in formats other than NDF, such as FITS and IRAF, by using
an `on-the-fly' conversion scheme.
Many commands can process data arrays of arbitrary dimension, and others
work on both spectra and images.
It operates from both the Unix C-shell, and the ICL command language.

KAPPA should not be perceived as a rival to
\htmlref{FIGARO}{item_FIGARO}.
Now that FIGARO is integrated with other Starlink packages, they should be
seen as complementary, with FIGARO concentrating on spectroscopy and KAPPA on
image processing.

In a wider context, KAPPA offers facilities which are not in
\htmlref{IRAF}{item_IRAF},
for instance: handling data errors, quality masking, a graphics database,
availability from the shell, as well as more $n$-dimensional applications,
widespread use of data axes, and a different style.
It also integrates with instrument packages developed at UK observatories.

With the automatic data format conversion, and the likelihood that KAPPA and
other Starlink packages will be available from within the IRAF command
language, you should be able to pick the best or relevant tools from both
systems to get the job done.

Currently, about 180 commands are available from both the Unix
C-shell and from the
\htmlref{ICL}{item_ICL} command language.
They provide the following facilities:

\begin{itemize}
\item Generate NDFs and text tables using FITS readers, and import and export
ancillary data through the NDF FITS extension.
\item Generate test data and create NDFs from text files.
\item Set and examine NDF components.
\item Define or calculate a sky co-ordinate system for use in
 conjunction with IRAS tools.
\item Arithmetic, including a powerful application that handles expressions.
\item Edit pixels and regions, including polygons and circles, re-flag
 bad pixels by value or median filtering, and paste arrays over others.
\item Mask regions, and pixels whose variances are too large.
\item Change configuration: flip, rotate, shift, subset, change dimensionality.
\item Mosaic images; normalise NDF pairs.
\item Compress and expand images.
\item Generalised resampling of NDFs using arbitrary transformations.
\item Filter: box, Gaussian, and median smoothing; very efficient
 Fourier transform, convolution.
\item Deconvolution: maximum-entropy, Lucy-Richardson, Wiener filter.
\item Two-dimensional-surface fit.
\item Statistics, including ordered statistics, histograming;
pixel-by-pixel statistics over a sequence of images.
\item Inspect image values.
\item Centroids of features, particularly stars; fit stellar Point Spread Functions (PSFs).
\item Enhance detail using histogram equalisation and Laplacian convolution,
 enhance edges via a shadow effect, threshold.
\item Calculate polarimetry images.
\end{itemize}

There are also many applications for data visualisation:

\begin{itemize}
\item A graphics database, AGI, to pass information about pictures
 between tasks; tools to create, label, select pictures, and obtain world and
 data co-ordinate information from them.
\item Image and greyscale plots with a selection of scaling modes and many
 options such as axes.
\item Create, select, save, and manipulate colour tables and
 palettes (for axes, annotation, coloured markers and borders).
\item Snapshot an image display to hardcopy.
\item Blink and control visibility of image-display planes.
\item Line graphics: contouring, including overlay; columnar and
 hidden-line plots of images; histogram; line plots of 1-d arrays
 and multiple-line plot of images; pie sections and slices through an image;
 vector plot of image; all of which offer some control of the appearance of
 plots.
\end{itemize}

KAPPA handles bad pixels, and processes quality, variance, and other
information stored within NDFs.
In order to achieve generality, it does not process non-standard extensions;
however, it does not lose non-standard ancillary data since it copies
extensions to any NDFs that it creates.

Although oriented towards image processing, many commands will work on NDFs
of arbitrary dimension, and others operate on both spectra and images.
Many applications handle all non-complex data types directly, for
efficient memory and disk usage.
Those that do not will usually undergo automatic data conversion to produce
the desired result.

Its graphics are device independent.
X-windows and overlays are supported.

The KAPPA package may also be run from the IRAF CL.


\subsubsection{KAPRPH  \hfill \xref{SUN/239}{sun239}{}}
\label{item_KAPRH}
\index{KAPRH}
\textbf{Retired KAPPA commands}

KAPRH is a package containing commands retired from the
\htmlref{KAPPA}{item_KAPPA} package (see \xref{SUN/95}{sun95}{}).
These commands have been removed from KAPPA
but are retained in KAPRH to satisfy any occasional need there may
be for them. KAPRH is documented in \xref{SUN/239}{sun239}{}.

KAPRH contains the following retired KAPPA commands:

\texttt{CONTOVER}, \texttt{GREYPLOT}, \texttt{INSPECT},
\texttt{MOSAIC}, \texttt{QUILT}, \texttt{SNAPSHOT}, \texttt{TURBOCONT}


as well as a help command: \texttt{KRHHELP}.




\subsubsection{PHOTOM \hfill \xref{SUN/45}{sun45}{}}
\index{PHOTOM} 
\label{item_PHOTOM}
\textbf{Aperture photometry.}

It has two basic operating modes:
\begin{itemize}
\item Use an interactive display to specify the positions for the
 measurements.
\item Obtain those positions from a file.
\end{itemize}
The aperture is circular or elliptical, and the size and shape can be varied
interactively on the display, or by entering values from the keyboard or
parameter system.
The background sky level can be sampled interactively by manually positioning
the aperture, or automatically from an annulus surrounding the object.

It is used by the Graphical Astronomy and Image Analysis tool
(\htmlref{GAIA}{item_GAIA})
which integrates the tasks of aperture photometry with an image display tool.
This allows the detailed inspection of objects and their environments, and
provides a highly interactive environment for placing, rotating, and
resizing apertures.



\subsubsection{PISA \hfill \xref{SUN/109}{sun109}{}}
\index{PISA} 
\label{item_PISA}
\textbf{Locate and parameterize objects in a 2-d image.}

The Position, Intensity, and Shape Analysis package (PISA) is applicable in
most areas of astronomy where direct imaging is required.
It is particularly intended for faint object detection and can do this
automatically, based on software used with the APM automatic plate
measuring machine.

Its core is a routine which performs image analysis on a 2-d data frame.
It searches for objects having a minimum number of connected pixels above a
given threshold, and extracts the image parameters (position, intensity, shape)
for each object.
The parameters can be determined using thresholding techniques, or an analytical
stellar profile can be used to fit the objects.
In crowded regions it can deblend overlapping sources.


\subsubsection{SAOIMAGE \hfill \xref{SUN/166}{sun166}{}. MUD/140}
\index{SAOIMAGE}
\label{item_SAOIMAGE}
\textbf{Display astronomical images.}

You can manipulate images in a number of ways, see the changes applied, and
when you are happy with the result, print it on a Postscript printer.
It uses X-window hardware.

It can:
\begin{itemize}
\item{Scale between limits.}
\item{Magnify, zoom, pan.}
\item{Do histogram equalisation.}
\item{Scale -- Log and square root.}
\item{Use false colour with built-in or user-specified colour tables.
Clicking on a colour bar pops up a graphical representation of the colour
table's RGB values.
These can be adjusted interactively, or a previously saved colour table can
be loaded.}
\item{Stretch the contrast.}
\item{Change gamma ($\gamma$) to give non-linear contrast, so that the display
assigns more shades of grey to the darker or lighter end of the scale.
This may suit the eye better that a linear scale.}
\item{Type text on an image.}
\end {itemize}

Other features:

\begin{itemize}
\item{Blink between images.}
\item{Regions -- A description of a region of an image can be saved.
This can be used to flag regions as bad, or to examine only the counts
within a certain region.
This is mainly useful in conjunction with the
\htmlref{IRAF}{item_IRAF}
 and PROS (X-ray) packages.}
\item{Tracking -- The pixel coordinates and values under the mouse pointer and
the region around it can be displayed and constantly updated.}
\item{\htmlref{IRAF}{item_IRAF} image display.}
\item{Hard copy.}
\end{itemize}


\subsubsection{STARMAN  \hfill\xref{SUN/141}{sun141}{}}
\label{item_STARMAN}
\index{STARMAN}
\textbf{Stellar photometry.}

\textit{STARMAN is not distributed on Starlink CDs and has not been
ported to Linux.  It is optionally available at UK Starlink sites.}

Starman's many image and table handling programs have other uses, but its main
applications are:
\begin{itemize}
\item Crowded-field photometry.
\item Aperture photometry.
\item Star finding.
\item CCD data reduction.
\end{itemize}

The package is a coherent whole, for use in the entire process of stellar
photometry from raw images to the final standard-system magnitudes and
their plotting as colour-magnitude and colour-colour diagrams.
Its functions are:

\begin{itemize}
\item Stellar Photometry
\begin{itemize}
\item Convert raw CCD images to calibrated ones.
\item `Dust-ring' flat-field dealer.
\item Find star positions
\item Determine stellar profiles
\item Crowded-field, and also simple, stellar photometry measurers.
\item Average photometry estimates from different images.
\item Plot colour-magnitude and colour-colour diagrams.
\item Add `fake' stars to an image.
\item Automatic aperture photometry for all bright well isolated stars.
\end{itemize}
\item Image Handling
\begin{itemize}
\item Image display.

\item Interactive image work.  \textit{Display:} zoom, pan, pixel
value, `slice', solid-body plots, colour, crowded-field stellar
photometry output, \emph{etc}.; \textit{Measurement:} interactive
aperture photometry, interactive position file making;
\textit{Manipulation:} maths, rotation, weeding, \emph{etc.};
\textit{`GUI'-like interaction}.

\item Extensive general programs: Maths, joining, cutting, reading, etc..
\end{itemize}
\item Table Handling
\begin{itemize}
\item Spreadsheet, Calculator.
\item Input, Output, Listing.
\item Graphical plot, Star chart.
\item Joining.
\item Extensive general programs: sort, weed, statistics, matching, position
transforming, etc..
\end{itemize}
\end{itemize}

\newpage   %% paging


\subsubsection{SX  \hfill
\xref{SC/2}{sc2}{},
\xref{SG/8}{sg8}{},
\xref{SUN/203}{sun203}{}}

\label{item_SX}
\index{SX}\index{DX}
\textbf{Enhancements to DX.}

DX (Data Explorer) is a scientific data visualisation and analysis package
marketed by IBM.
It can visualise and display many sorts of astronomical data.
It employs a data-flow driven client-server execution model, and provides a
comprehensive range of data manipulation, visualisation, and display functions.
Visualisations can be generated using a visual programming editor or a
text-based scripting language.

Starlink recommends DX for displaying 3-d scalar and vector data.
However, it is not available at all Starlink sites -- ask you Site Manager.

The SX enhancements have a number of purposes.
They:

\begin{itemize}
\item Fill a few minor omissions in the functionality of basic DX.
\item Provide easy-to-use functions to accomplish common tasks in a
 single step.
\item Allow standard Starlink NDF data structures, and data in other
 common astronomical formats, to be imported into DX.
\end{itemize}



\newpage

\subsection{Spectroscopy}

\rule{\textwidth}{0.5mm}

\begin{description}
\item [\htmlref{DATACUBE}{item_DATACUBE}]
--- IFU datacube analysis
\item [\htmlref{DIPSO}{item_DIPSO}]
--- Spectrum analysis and plotting
\item [\htmlref{ECHOMOP}{item_ECHOMOP}]
--- \'{E}chelle data reduction
\item [\htmlref{FIGARO}{item_FIGARO}]
--- General data reduction
\item [\htmlref{NDPROGS}{item_NDPROGS}]
--- $n$-dimensional data analysis
\end{description}

\rule{\textwidth}{0.5mm}




\subsubsection{DATACUBE  \hfill \xref{SUN/237}{sun237}{}, \xref{SC/16}{sc16}{}}
\index{DATACUBE}
\label{item_DATACUBE}

\textbf{IFS datacube manipulation and visualisation.}

DATACUBE is a new package which provides a set of tools for manipulating
and visualising IFS (Integral Field Spectrograph) datacubes.  Also known
as an IFU (Integral Field Unit), the IFS is relatively new in
astronomical instrumentation and is therefore rapidly developing.  The
DATACUBE package has been implemented mainly as a series of scripts
which drive existing Starlink packages to manipulate IFU data.  This
approach allows changes to be made to tasks quickly as data analysis
requirements evolve.  There are also a small number of IFU-specific
tasks implemented in Fortran.

IFU data analysis is discussed in detail in the cookbook
\textit{The IFU Data Product Cookbook}\/(\xref{SC/16}{sc16}{}).
\newpage   %% paging



\subsubsection{DIPSO  \hfill\xref{SUN/50}{sun50}{}}
\index{DIPSO}
\label{item_DIPSO}
\textbf{Analyse and visualise spectroscopic data.}

A powerful and versatile package, specifically tailored to the requirements of
modern astronomical research.

It can access a large number of spectra simultaneously.
This lets you perform the same operations repeatedly on successive spectra,
either manually or using scripts, ensuring uniformity in the data (especially
important for time-series spectroscopy).
It can plot spectroscopic data rapidly and conveniently, and combines
analysis and high-quality graphical output in a simple command-line driven
environment.

It began as a simple plotting package incorporating some basic
astronomical applications.
Indeed, if you just want to read in some data, plot them, and measure some
equivalent widths or fluxes, you can do this easily.
To make more complicated things possible, a number of extra functions and
parameters are provided.
A macro facility allows convenient execution of regularly used sequences of
commands, and a simple Fortran interface permits `personal' software to be
integrated with the program.
User programs can be added to the system and defined as new commands.

The following operations are available:
\begin{itemize}
\item Arithmetic.
\item Fit emission lines.
\item Measure equivalent widths.
\item Measure fluxes.
\item Fourier analysis.
\item Interstellar line analysis.
\item Model atmospheres.
\item Model nebular continua.
\item Fit polynomials.
\item Simple statistics.
\end{itemize}



\subsubsection{ECHOMOP  \hfill \xref{SUN/152}{sun152}{}}
\label{item_ECHOMOP}
\index{ECHOMOP}
\textbf{Reduce \'{e}chelle spectra data frames.}

Options range from full-scale automated reduction to step-by-step
order-by-order manually assisted processing.
It was written originally to reduce data from the University College London
\'{E}chelle Spectrograph (UCLES), but the algorithms are sufficiently flexible
to reduce spectra from many sources.

It can:
\begin{itemize}
\item Locate spectral orders
\item Remove cosmic rays.
\item Detect bad image rows/columns and saturated pixels.
\item Trace spectral orders.
\item Determine object channels.
\item Generate flat-field balance model.
\item Model scattered light.
\item Extract optimal spectrum.
\item Correct \'{e}chelle blaze.
\item Extract quick-look spectra.
\item Locate lines in arc spectra automatically.
\item Calibrate wavelengths of arc spectra.
\item Fit distortion for spectral orders.
\item Extract distorted orders.
\item Scrunch extracted spectral orders.
\item Produce distortion-free image.
\item Plot data used for a reduction.
\item Output products.
\end{itemize}


\subsubsection{FIGARO  \hfill \xref{SUN/86}{sun86}{}.MUD/12,13,14}
\index{FIGARO}
\label{item_FIGARO}
\textbf{Reduce and analyse astronomical data.}

FIGARO is a general-purpose data reduction package.  It contains
particularly extensive facilities for reducing spectroscopic  data,
but also has powerful facilities for manipulating direct images
and data cubes.  Starlink recommends it as the most complete
spectroscopic data reduction system in the Collection.

It can be used interoperably with other packages, most notably
\htmlref{KAPPA}{item_KAPPA}, because it supports the
\htmlref{NDF}{item_NDF} data format, and therefore all foreign
formats for which conversion utilities exist.  These include its old
DST format, FITS, and IRAF.  The use of NDF also means that you can
automatically record data processing history, and can operate on
sub-sets of spectra and images.  Another feature is the propagation of
error/data-quality arrays through the data reduction calculations.

It can:
\begin{itemize}
\item Analyse absorption lines interactively.
\item Do aperture photometry.
\item Calibrate B stars.
\item Calibrate flat fields.
\item Calibrate using flux calibration standards.
\item Calibrate wavelengths of spectra.
\item Correct S-distortion.
\item Extract spectra from images and images from data cubes, and insert
 spectra into images and images into data cubes.
\item Extract spectra from images taken using optical fibres.
\item Fit Gaussians to lines in a spectrum interactively.
\item Generate and apply a spectrum of extinction coefficients.
\item Input, output, and display data.
\item Look at the contents of data arrays, other than graphically.
\item Manipulate complex data structures (mainly connected with Fourier
 transforms).
\item Manipulate data arrays `by hand'.
\item Manipulate images and spectra (arithmetic and more complicated).
\item Process data taken using FIGS (the AAO's Fabry-Perot infra-red grating
spectrometer).
\item Process \'{e}chelle data -- in particular the UCL \'{e}chelle in use at
the AAO.
\end{itemize}

The programs from the old SPECDRE and TWODSPEC packages have been merged with
FIGARO and provide additional facilities:

\begin{itemize}

\item \textbf{Hyper-cubes:} In general, the data being processed are a
hyper-cube where each row or hyper-column is a spectrum.  A single
spectrum can be an appropriate section of the hyper-cube cut out `on
the fly' as the data are accessed.

\item \textbf{Coherent storage of fit results:} The results of line or
continuum fits are stored with the data.  When a hyper-cube is a
coherent set of spectra, fit results are also stored coherently.  For
example, in a 3-d data set, the 2-d map of line integrals is
immediately available to display routines.

\item \textbf{Bad values and variance:} Bad values (or quality
information) are recognised and ignored or propagated, as appropriate.
If variance information is preset, it is propagated or used in the
processing.

\item \textbf{Analysis of line profiles:} Analysis of calibrated
long-slit spectra, fitting of Gaussians, either manually or
automatically in batch,  handling of data with two spatial
dimensions, such as Taurus data.

\end{itemize}

The topics addressed by the applications are mainly:

\begin{itemize}

\item \textbf{ASCII I/O:} The data and errors of hyper-cubes can be
written to or read from printable/editable tables.  Bad values are
converted between the two formats.  Single spectra can be read, even if
the axis data are not linear or monotonic.

\item \textbf{Graphics:} Displays allow full control of the plot,
including font, colour, line styles, error bars, \emph{etc}.  Overlay on
previous plots according to their `world coordinates' is possible,
including overlays on grey/colour/line plots made using
\htmlref{KAPPA}{item_KAPPA}, \htmlref{PONGO}{item_PONGO},
\emph{etc}.

\item \textbf{Cube processing:} You can extract averaged hyper-planes
from hyper-cubes, assemble hyper-cubes from hyper-planes, or fill in a
hyper-cube from several given hyper-cubes.

\item \textbf{Arc line axis (wavelength) calibration:} While full user
interaction via graphics is granted, automatic arc line identification
is also possible.

\item \textbf{Re-sampling:} You can either re-sample all spectra in a
hyper-cube, or re-sample and average into one spectrum any number of
input spectra.  Information about the covariance between pixels can be
carried through to a line fit routine.

\item \textbf{Spectral fits:} You can fit polynomials, blended Gaussian or
triangle profiles.  Fit results are stored along with the data and
can be turned into fake data sets for later subtraction, division,
\emph{etc}.

\end{itemize}

The FIGARO programs may be run from the IRAF CL but the former SPECDRE
and TWODSPEC programs are not yet available under the IRAF CL.

\newpage   %% paging




\subsubsection{NDPROGS  \hfill\xref{SUN/19}{sun19}{}}
\index{NDPROGS}
\label{item_NDPROGS}
\textbf{Manipulate and display images of up to six dimensions.}

\textit{The NDPROGS package is no longer supported and will be withdrawn
if operating system or compiler changes cause it to fail.  It is not
distributed on Starlink CDs but may be installed at UK Starlink sites
as a legacy package.}

Primarily, it is designed to manipulate Taurus spectral line data
cubes, which are 3-d images in which two of the dimensions are spatial
and the third is spectral.  However, it contains no instrument-specific
features, and can therefore be used to analyse similar data.

The term `image' here simply means a regular data array, which might be
anything from a 1-d spectrum or profile to a 4-d array consisting of
several long-slit spectra with polarization vectors.  The upper limit
of six dimensions is imposed by the software which interfaces with HDS
and has no other significance.

The package can read FIGARO images, and images written by NDPROGS can
be read by any FIGARO program which can handle the number of dimensions
involved.  It duplicates the functions of standard FIGARO programs as
far as 1-d, 2-d, and 3-d images are concerned, but with new features.
It will also accept NDF format files.  Most NDPROGS routines now handle
data quality and error arrays, thus widening the scope and
accessibility of the package.




\newpage

\subsection{Time Series \& Polarimetry}

\rule{\textwidth}{0.5mm}

\begin{description}
\item [\htmlref{PERIOD}{item_PERIOD}]
--- Time series analysis
\item [\htmlref{POLMAP}{item_POLMAP}]
--- Linear spectropolarimetry
\item [\htmlref{POLPACK}{item_POLPACK}]
--- Dual beam imaging polarimetry
\item [\htmlref{TSP}{item_TSP}]
--- Time series \& polarimetry
\end{description}

\rule{\textwidth}{0.5mm}


\subsubsection{PERIOD  \hfill \xref{SUN/167}{sun167}{}. \xref{SSN/25}{ssn25}{}}
\index{PERIOD}
\label{item_PERIOD}
\textbf{Search for periodicities in data.}

It is menu-driven.
You can:
\begin{description}
\item [Read and write data] \hfill

\begin{itemize}
\item Read raw data from both Ascii tables and FITS format OGIP files.
\item Store calculated fits in a log file.
\item Output power spectra to Ascii files.
\item Output modelled/modified data.
\item Invoke the QDP plotting/fitting package (allows labelling).
\end{itemize}

\item [Examine data] \hfill

\begin{itemize}
\item Generate a power spectrum showing periodicities.
\item Fold the data on a given period.
\item Fit sine curves to the input data.
\item Set data points to unity to investigate spectral leakage.
\end{itemize}

\item [Manipulate data] \hfill

\begin{itemize}
\item Remove background trends in the data via polynomial fits.
\item Create test data.
\item Add noise to data.
\item Add or subtract known periods to/from data.
\end{itemize}

\end{description}
Many sophisticated techniques are used to search data for periodicities;
data points do not have to be equally spaced.
You don't need in-depth knowledge of the methods employed in order to use them.
They can be treated as black-boxes.
These include:
\begin{itemize}
\item Chi-squared analysis of a folded sine fit versus frequency.
\item Cleaned and discrete Fourier power spectra.
\item Phase dispersion minimization (PDM Stelling\-werf).
\item Lomb-Scargle normalized periodograms.
\item String-length (Dworetsky) versus frequency estimates.
\item Periods from the periodogram.
\item Significance estimates.
\end{itemize}
Significance estimates for periods derived by most methods are notoriously
unreliable.
In PERIOD, the Fisher randomisation method, one of the best, is employed.



\subsubsection{POLMAP  \hfill
\xref{SUN/204}{sun204}{}}
\index{POLMAP}
\label{item_POLMAP}
\textbf{Analyse linear spectropolarimetry data.}

A linear polarization spectrum is a set of Stokes vectors
$(I_{\lambda},Q_{\lambda},U_{\lambda})$.  Hence, linear
spectropolarimetric data is 4-d (6-d if the variance arrays of the
Stokes parameters are included) and cannot be manipulated using
standard spectral analysis packages such as DIPSO.

Its user interface is similar to \htmlref{DIPSO's}{item_DIPSO}, but
doesn't provide its wealth of spectral analysis routines.  It is
designed specifically for the spectropolarimetrist.  The manual
includes a simple step-by-step guide to the program, and some example
data analysis recipes.

It is complimentary to \htmlref{TSP}{item_TSP} (like FIGARO is to
DIPSO).  TSP runs under the Starlink software environment and can
handle time-series and polarimetric data.  It is biased towards data
reduction and can handle several different instruments.  POLMAP, on the
other hand, does no data reduction, but was designed with data analysis
in mind.  POLMAP can also display data that TSP cannot handle, and can
read and write TSP polarization spectrum format files.



\subsubsection{POLPACK  \hfill\xref{SUN/223}{sun223}{}}
\index{POLPACK}
\label{item_POLPACK}
\textbf{Imaging Polarimetry data reduction}

POLPACK is a package of applications for mapping the linear or circular
polarization of extended astronomical objects. Data from both single and
dual beam polarimeters can be processed.

POLPACK processes data in NDF format.  Other astronomical data
formats may also be processed using transparent on-the-fly data
conversion facilities provided by the NDF subroutine library, and the
CONVERT package.

The facilities provided by POLPACK include:

\begin{itemize}
\item  alignment of images on the sky.
\item  extraction of O and E images from a single frame.
\item  sky subtraction.
\item  calculation of Stokes parameters.
\item  binning of Stokes parameters.
\item  creation of catalogues of polarization vectors.
\item  graphical display of vector maps.
\end{itemize}

POLPACK does not provide facilities for performing instrumental
corrections such as flat-fielding, de-biassing, \emph{etc}. Such
corrections should be applied to the data before using POLPACK, so that
POLPACK can assume that pixel values are proportional to the combined
intensity of sky and object. Corrections can be made, however, to take
account of any differences in the exposure times between raw frames, and
any difference in the sensitivity of the two channels of a dual-beam
polarimeter. These corrections rely on redundancy in the supplied data,
and require a specific set of analyser positions to be used.




\subsubsection{TSP  \hfill
\xref{SUN/66}{sun66}{}}
\index{TSP}
\label{item_TSP}
\textbf{Reduce time-series and polarimetric data.}

Time-series and polarimetry facilities are missing from most existing
data reduction packages, which are usually oriented towards either
spectroscopy or image processing, or both.  TSP, however, can process the
following data:

\begin{itemize}
\item Spectropolarimetry obtained with the AAO spectropolarimeters
using wave-plate or Pockels cell modulators in conjunction with either
IPCS or CCD detectors.
\item Infrared spectropolarimetry obtained with the IRPOL polarimeter
module in conjunction with the CGS2 grating spectrometer and the UKT6
and UKT9 CVF systems at UKIRT.
\item Infrared imaging polarimetry obtained with the IRIS instrument at
the AAT and with similar instruments.
\item Time series imaging and polarimetry obtained with the AAO Faint
Object Polarimeter.
\item Time series polarimetry obtained with the Hatfield Polarimeter at
either UKIRT or AAT.
\item Time series polarimetry obtained with the University of Turku
UBVRI polarimeter.
\item Five-channel time series photometry obtained with the Hatfield
polarimeter at the AAT in its high speed photometry mode.
\item Time series infrared photometry data obtained with the AAO Infrared
Photometer Spectrometer (IRPS).
\item Time series optical photometry data obtained using the HSP3 high speed
photometry package at the AAT.
\end{itemize}


\newpage

\subsection{Database Management}

\rule{\textwidth}{0.5mm}


\begin{description}
\item [\htmlref{CATPAC}{item_CATPAC}]
--- Catalogue and table manipulation
\item [\htmlref{CURSA}{item_CURSA}]
--- Catalogue and table manipulation
\end{description}

\rule{\textwidth}{0.5mm}




\subsubsection{CATPAC  \hfill\xref{SUN/120}{sun120}{}}
\index{CATPAC}
\label{item_CATPAC}
\textbf{Manipulate catalogues and tables.}

\textit{CATPAC is no longer supported and is not distributed on Starlink
CDs.  CATPAC may remain as a legacy package at UK Starlink sites.}

It can input, process, and report tabular data, including astronomical
catalogues.  In particular, it can:

\begin{itemize}
\item Create and delete catalogues.
\item Report catalogues.
\item Manipulate information about catalogues.
\item Manipulate data in catalogues.
\end{itemize}

CATPAC was designed as a replacement for SCA but is being superseded by
\htmlref{CURSA}{item_CURSA} (see below).



\subsubsection{CURSA  \hfill\xref{SUN/190}{sun190}{}}
\index{CURSA}
\label{item_CURSA}
\textbf{Manipulate catalogues and tables.}

It can:
\begin{itemize}
\item Browse or examine catalogues.
\item Select subsets from a catalogue.
\item Sort catalogues.
\item Copy catalogues.
\item Pair two catalogues.
\end{itemize}

Subsets can be extracted from a catalogue in a format suitable for
plotting by other Starlink packages such as \htmlref{PONGO}{item_PONGO}.
It can access catalogues held in the following formats:

\begin{itemize}
\item FITS,
\item Small Text List (STL),
\item CHI/HDS format used by \htmlref{CATPAC}{item_CATPAC}.
\end{itemize}
Catalogues in the STL format are simple Ascii text files.



\newpage

\subsection{Specific Wavelengths}

\rule{\textwidth}{0.5mm}

\begin{description}
\item [\htmlref{ASTERIX}{item_ASTERIX}]
--- X-ray data processing
\item [\htmlref{SPECX}{item_SPECX}]
--- Millimetre-wave spectral reduction
\end{description}

\rule{\textwidth}{0.5mm}



\subsubsection{ASTERIX  \hfill \xref{SUN/98}{sun98}{}. MUD/4}
\index{ASTERIX}
\label{item_ASTERIX}
\textbf{Analyse astronomical data in the X-ray waveband.}

\emph{ASTERIX has been withdrawn from the Starlink Software
distribution and is no longer included on Starlink CDs (from Spring
2000) due to the end of funding for the ROSAT satellite
program.  It may remain as a legacy package at UK Starlink sites.}

Continued `best efforts' support may be provided by the Space Research
group at the University of Birmingham for the recent development
version.  This may be found on the \htmladdnormallink{Asterix
Home Page}{http://www.sr.bham.ac.uk/asterix/} (see
\url{http://www.sr.bham.ac.uk/asterix/}).  No support will be provided
for the Starlink version.  \normalfont

Many of its programs are general purpose and are capable of analysing
any data in the correct format.  It is instrument-independent, and
currently has interfaces to the EXOSAT and ROSAT instruments.

Its data are stored in \htmlref{HDS}{item_HDS} files, and are therefore
compatible with many other Starlink packages.  There are basically two
different types:


\begin{description}
\item [Binned data] -- (\emph{e.g.} time series, spectra, images) are
stored in \htmlref{NDF}{item_NDF} format.  Data errors (stored in the
form of variances) and quality are catered for.
\item [Event data] -- store information about a set of photon
`events'.  Each event will have a set of properties, \emph{e.g.}
X-position, Y-position, time, raw pulse height.
\end{description}


The input data are first processed by an instrument interface.
Event data are then processed and binned, and the binned data are processed.
Finally, graphical output is generated.

The commands may be classified as follows:

\begin{itemize}
\item Interface to particular instruments (EXOSAT, ROSAT, \emph{etc}).
\item Event dataset and binned dataset processing.
\item Data conversion and display.
\item Mathematical manipulations.
\item Time series analysis.
\item Image processing.
\item Spectral analysis.
\item Statistical analysis.
\item Data quality analysis.
\item HDS editor.
\item Source searching.
\item Graphical and textual display.
\end{itemize}



\subsubsection{SPECX  \hfill \xref{SUN/17}{sun17}{}. MUD/69,70}
\index{SPECX}
\label{item_SPECX}
\textbf{Reduce and display mm and sub-mm data.}

Although it can process spectra from many different instruments, it is
particularly applicable to JCMT data.
It can:
\begin{itemize}
\item Process up to eight spectra simultaneously.
\item Save current status of system after each command is executed.
\item List and display spectra on a graphics terminal, and print hardcopy
on many printers.
\item Do single and multiple scan arithmetic, scan averaging, etc.
\item Store and retrieve intermediate spectra in storage registers.
\item Fit and remove polynomial, harmonic, and Gaussian baselines.
\item Filter and edit spectra.
\item Determine important line parameters (peak intensity, width, etc).
\item Calculate Fourier transforms and power spectra.
\item Calibrate data.
\item Assemble a number of reduced individual spectra into a map file, and
contour any plane or planes of the resulting cube.
\item Execute macro-command sequences and indirect command files.
\end{itemize}
It uses its own data format, so it is not possible to access directly
reduced spectra from other packages.
However, it can import data from GSD-format data files, as produced by the
JCMT; write out maps and spectra in the file formats of
\htmlref{FIGARO}{item_FIGARO} and
\htmlref{KAPPA}{item_KAPPA};
and write spectra and maps to Ascii files for input into other packages.



\newpage

\subsection{Specific Instruments}

\rule{\textwidth}{0.5mm}

\begin{description}
\item [\htmlref{CGS4DR}{item_CGS4DR}]
--- CGS4 (UKIRT) data reduction
\item [\htmlref{FLUXES}{item_FLUXES}]
--- JCMT flux density calibration
\item [\htmlref{IRAS90}{item_IRAS90}]
--- IRAS data reduction
\item [\htmlref{IRCAMDR}{item_IRCAMDR}]
--- IRCAM (UKIRT) data reduction
\item [\htmlref{IRCAMPACK}{item_IRCAMPACK}]
--- IRCAM (UKIRT) data reduction
\item [\htmlref{IUEDR}{item_IUEDR}]
--- IUE data reduction
\item [\htmlref{JCMTDR}{item_JCMTDR}]
--- JCMT (UKT14) data reduction
\item [\htmlref{REPACK}{item_REPACK}]
--- ROSAT (WFC) data reduction
\item [\htmlref{SURF}{item_SURF}]
--- SCUBA data reduction
\item [\htmlref{WFCPACK}{item_WFCPACK}]
--- ROSAT (WFC) data reduction
\item [\htmlref{XRT}{item_XRT}]
--- XRT data tools
\end{description}

\rule{\textwidth}{0.5mm}





\subsubsection{CGS4DR  \hfill \xref{SUN/27}{sun27}{}}
\index{CGS4DR}
\label{item_CGS4DR}
\textbf{Reduce CGS4 data.}

\textit{Cgs4dr is not available on Linux.}

The fourth generation \emph{cooled grating spectrometer}, CGS4,
operates on UKIRT in the 1--5 $\mu$m region at resolutions in the range
$\lambda$/$\Delta\lambda$ $\sim$ 300--20000.
To reduce background noise, it is maintained on the telescope in vacuum at
cryogenic temperatures.
It achieved first light at UKIRT on 4 February 1991.
On 22 April 1995 a new InSb 256 $\times$ 256 array was commissioned.
The new array is much more sensitive than previous detectors, and on a good
night you can acquire and reduce $\sim$ 100 Mb of high quality data.

CGS4DR was designed and tested within the
\htmlref{FIGARO}{item_FIGARO} and
\htmlref{ADAM}{item_ADAM} environments.
The output data is readable by standard FIGARO applications, although
not all may handle the quality and error arrays correctly.

It can reduce spectrographic data automatically.
No system, however, will do everything you want, so some post-processing
may be needed.
The aim is to produce publishable quality spectra at the telescope via an
automated reduction paradigm.

It can:
\begin{itemize}
\item Allow a wide variety of data reduction configurations.
\item Interlace oversampled data frames.
\item Reduce known \textsc{bias}, \textsc{dark}, \textsc{flat}, \textsc{arc}, \textsc{object}
and \textsc{sky} frames.
\item Wavelength calibrate via suitable \textsc{arc} lines.
\item Flux calibrate via a suitable \textsc{standard}.
\item Remove the \textsc{sky}, residual sky OH-lines ($\lambda <$ 2.3 $\mu$m)
  and thermal emission ($\lambda \geq$ 2.3 $\mu$m) from data.
\item Add data into groups for improved signal-to-noise.
\item Extract and de-ripple a spectrum.
\item Maintain an index of reduced observations.
\item Maintain, if required, an archive of observations.
\item Plot data in a variety of ways.
\end{itemize}



\subsubsection{FLUXES  \hfill \xref{SUN/213}{sun213}{}}
\index{FLUXES}
\label{item_FLUXES}

\textbf{Calculate accurate topocentric positions of the planets, and also
integrated flux densities of five of them at a number of wavelengths, for the
\htmladdnormallink{JCMT telescope}{http://www.eaobservatory.org/jcmt/}
on Mauna Kea, Hawaii.}

It provides calibration information at the effective frequencies and
beam-sizes employed by the UKT14 and SCUBA receivers on this telescope.

The filter centre and widths are shown for each wavelength used.
The filter frequencies are the effective frequencies for 1mm of water vapour.

The calculated planet positions should normally be accurate to better than
$1''$ of arc.
The value for the Moon is less accurate than this, and for critical applications
requiring sub arc-second accuracy you should use a different method.

The version distributed by Starlink has been modified to employ the highly
accurate \htmlref{JPL}{item_JPL} ephemeris in all planetary position
calculations.
It also adopts the Starlink parameter interface, and a more robust file
reading system.

A script, FLUXNOW, runs the program for the current time and date.
This gives you current positions and flux levels for all the
planets.



\subsubsection{IRAS90  \hfill
\xref{SUN/82}{sun82}{},
\xref{161}{sun161}{},
\xref{163}{sun163}{},
\xref{165}{sun165}{}}
\index{IRAS90}
\label{item_IRAS90}
\textbf{Process IRAS data.}

\textit{IRAS90 is not distributed on Starlink CDs and is no longer supported
though it may remain as a legacy package on UK Starlink sites.  There
is no Linux port.}

The Infrared Astronomical Satellite (IRAS) flew in 1983.  It carried
three instruments: a main detector array, a low resolution spectrometer
(LRS), and a chopped photometric channel (CPC).  Most of the
observations were part of a whole sky survey, but some pointed
observations of specific objects were made.  Raw data was calibrated
and cleaned up to produce a data set called the Calibrated
Reconstructed Detector Data (CRDD).  This was the source of the final
data products, which include catalogues and images.

Several items of Starlink software can be used to process and examine
IRAS data.
Use
\htmlref{KAPPA}{item_KAPPA} to process and display images,
\htmlref{DIPSO}{item_DIPSO} to analyse LRS spectra,
\htmlref{CURSA}{item_CURSA} to access and report on the catalogues.
However, the most closely related software is
\htmlref{IRAS90}{item_IRAS90}.
This can process CRDD data to produce an image of a region of the sky, or
search for and examine an object at a given position.



\subsubsection{IRCAMDR  \hfill\xref{SUN/41}{sun41}{}}
\index{IRCAMDR}
\label{item_IRCAMDR}
\textbf{Reduce, analyse, and display data from IRCAM3.  It can also
handle any 2-d image in the standard Starlink NDF data format.}

\textit{IRCAMDR is not distributed on Starlink CDs and there is little
or no support.  There is no Linux port.  IRCAMDR may remain as a legacy
package at UK Starlink sites.  Reduction of IRCAM data is now best
performed using the \htmlref{ORAC-DR}{item_ORAC-DR} data reduction facility.}

IRCAM3 is a camera on the United Kingdom Infra-red Telescope (UKIRT).
In addition to handling IRCAM3 images of 256$\times$256 pixel size, it
can handle IRCAM1 and IRCAM2 images of 62$\times$58 pixels.  Almost all
the applications (with the exception of \textbf{med3d}) will work on NDF
images of any physical (pixel) dimensions, for example,
1024$\times$1024 CCD images can be processed (\textbf{med3d}
median-filters stacks of images up to 256$\times$256 in size at
present).

It can:
\begin{itemize}
\item Display images, line graphics, colour tables.
\item Reduce data, including automatic single-star photometry and bad/hot
pixel removal.
\item Register and mosaic images, and determine positions.
\item Extract and smooth images, do arithmetic, change dimensions.
\item Statistical analysis.
\item Photometry.
\item Polarimetry.
\end{itemize}



\subsubsection{IRCAMPACK  \hfill \xref{SUN/177}{sun177}{}}
\index{IRCAMPACK}
\label{item_IRCAMPACK}
\textbf{Process IRCAM data.}

IRCAM is an infrared camera on the United Kingdom Infra-red Telescope (UKIRT).

\htmlref{CCDPACK}{item_CCDPACK} and
\htmlref{KAPPA}{item_KAPPA}
will probably do most of the processing you require.
IRCAMPACK provides extra facilities.
It can:

\begin{itemize}
\item Interpret and process the specific header information stored within
IRCAM data files.
\item Normalise frames to unit total exposure time.
\item Provide a simple imaging polarimetry reduction procedure.
\end{itemize}



\subsubsection{IUEDR  \hfill
\xref{SG/3}{sg3}{},
\xref{7}{sg7}{}.
\xref{SUN/37}{sun37}{}.
\xref{MUD/45}{mud45}{},
46,47}
\index{IUEDR}
\label{item_IUEDR}
\textbf{Process data from the IUE \'{e}chelle spectrograph, starting
with the raw image and finishing with the fully calibrated spectrum.}

The International Ultraviolet Explore (IUE) is an ultraviolet telescope
in geosynchronous orbit.

It can:

\begin{itemize}

\item Examine the contents of IUE tapes to find what images are present.

\item Read RAW, GPHOT, and PHOT images from IUE tapes onto disk.
Provide them with default calibrations.  (Can also read extracted
spectra from IUESIPS.)

\item Display images for assessment of validity and quality.  Various
interactive operations can be performed, including bad-pixel marking,
image modification, and feature identification.

\item Extract spectra from IUE images.  This uses techniques that are
enhancements of those in the TRAK program.  Spectra exposed in either
resolution mode (HIRES or LORES) can be extracted from RAW, GPHOT, or
PHOT images (the latter being the newer-style Photometric images that
retain geometric distortion).  You can correct photometric LORES images
obtained with the SWP and LWR cameras for defects in the original ITF
calibration.

\item Produce line-by-line spectra, corresponding to the IUESIPS product.

\item Produce fully calibrated spectra.  This includes various forms of
wavelength correction, absolute calibration, and (for HIRES) ripple
correction.  There is also a semi-empirical correction for the HIRES
background (order-overlap) problem.

\item  Display graphs to aid spectrum extraction and calibration.
Various types of graphics terminals can be used.

\item Combine spectra from groups of \'{e}chelle orders (HIRES) or from
different apertures (LORES) by mapping and averaging them onto an
evenly-spaced wavelength grid.

\item Write spectra to Starlink NDF files, which can then be processed
by \htmlref{DIPSO}{item_DIPSO} and other Starlink software.  Output
includes individual extracted order and aperture spectra or combined
spectra.  Files can also be created in the same format as that written
by TRAK.

\end{itemize}



\subsubsection{JCMTDR  \hfill
\xref{SC/1}{sc1}{}.
\xref{SUN/132}{sun132}{}}
\index{JCMTDR}
\label{item_JCMTDR}
\textbf{Reduce continuum mapping data obtained with the UKT14
instrument on the JCMT.}

\textit{JCMTDR is now distributed on Starlink CDs. There is now a Linux port.}

The James Clerk Maxwell Telescope (JCMT) at Mauna Kea Observatory
observes in the millimetre-wave part of the spectrum.

The program can:

\begin{itemize}
\item  Transform JCMT map data into a tangent plane image centred on a given
position.
Optionally, rebin each input dataset individually before coadding them into
the result.
\item  Append an IRAS astrometry structure to a rebinned map, so that IRAS90
can be used on it to overlay coordinate grids, object positions, etc.
\item  Correct JCMT data for the effect of atmospheric extinction.
\item  Convert JCMT data from GSD format to FIGARO format.
\item  Convert a JCMT map file into a format suitable for further processing
(deconvolution and resampling) by DBMEM.
\item  Convert an image into a `time spectrum' with map pixels sorted in order
of increasing LST of observation, and vice versa.
\item  Deconvolve the chopped-beam from a dual beam map of a source.
\end{itemize}

Data files can be read in either `old FIGARO' or NDF formats;
\htmlref{NDF}{item_NDF} is recommended as this allows other Starlink
packages to process the data.  \htmlref{IRCAMPACK}{item_IRCAMPACK} can
also access the main data arrays, although other parts of the data
structure will be inaccessible.



\subsubsection{REPACK \hfill \xref{SUN/208}{sun208}{}}
\index{REPACK}
\label{item_REPACK}
\textbf{Handle ROSAT Wide Field Camera survey data.}

\textit{REPACK is not distributed on Starlink CDs though it may remain as
a legacy package at UK Starlink sites.  There is no linux Port.  With the
demise of the ROSAT satellite there is no support for REPACK.}

During the period from July 1990 to January 1991, the ROSAT Wide Field
Camera (WFC) performed the first `all sky survey' at extreme-ultraviolet
(EUV) wavelengths.  The whole sky, or $\approx 96$\% of it, was imaged
in two passbands, S1 and S2, covering the ranges 60--140\AA~and
110--200\AA~respectively.  An initial bright source catalogue (BSC) of
383 sources was produced by Pounds et.al.  A new list, the
2RE~Catalogue of 479 sources, has recently been published.  The survey
data (images and `raw' photon event files) are now in the public
domain.  REPACK fully exploits this data.

The data is available through the Leicester Data Archive Service (LEDAS).
Images were sorted from the event files and screened for `high background' and
`moon in WFC field-of-view' and named after the ecliptic latitude and
longitude of the region of sky that they covered.
They are $\approx 2.7^{\circ} \times 2.7^{\circ}$ in extent with a resolution
of $1 \times 1$ arcmin$^{2}$ per pixel.
The photon event files are overlaid on a similar ecliptic grid system.
Some 13,000 image pairs and 13,000 event files were produced by this scheme,
covering almost the whole sky.

Images can be retrieved in various formats: HDS, FITS, and GIF.
REPACK only operates on HDS format images.
Event files, available just in FITS format, can be sorted to ASTERIX (HDS)
datasets such as images and light-curves.



\subsubsection{SURF  \hfill \xref{SC/10}{sc10}{}, \xref{11}{sc11}{}.%
\xref{SUN/216}{sun216}{}}
\index{SURF}
\label{item_SURF}
\textbf{SCUBA data re-reduction facility}

SURF is a set of programs for reducing demodulated Submillimetre
Common-User Bolometer Array (aka SCUBA) data obtained from the James
Clerk Maxwell Telescope.


The facilities provided by SURF include:

\begin{itemize}
\item nod compensation,
\item flatfielding,
\item extinction correction,
\item single beam restoration,
\item sky noise removal,
\item despiking,
\item array overlay.
\end{itemize}



\subsubsection{WFCPACK \hfill
\xref{SUN/62}{sun62}{}}
\index{WFCPACK}
\label{item_WFCPACK}
\textbf{Produce ASTERIX (HDS) datasets from Wide Field Camera data collected
during the pointed phase of the ROSAT mission.}

\textit{WFCPACK is not distributed on Starlink CDs though it may remain as
a legacy package at UK Starlink sites.  There is no linux Port.  With the
demise of the ROSAT satellite there is no support for WFCPACK.}

A sort program generates datasets such as time series and images from
pre-processed event data supplied as part of the ROSAT WFC Observation
Datasets (RWODs).
An exposure program corrects them allowing for instrument characteristics.
In addition, a simple database manager allows an index of RWODs to be
maintained and searched.
The programs make use of the
\htmlref{ADAM}{item_ADAM}/\htmlref{ICL}{item_ICL}
environment, and a number of ICL
procedures are provided to perform some of the more commonly required
operations, such as extracting background subtracted lightcurves.



\subsubsection{XRT}
\index{XRT}
% \xref{SUN/NNN}{sunNNN}{}
\label{item_XRT}
\textbf{ROSAT XRT data access.}

The XRT is a package provides access to ROSAT XRT data stored in FITS files
for analysis by other Starlink Packages.  The tools are based on
the ASTERIX ROSAT XRT toolset from version 2.3-b1, but do not include the ASTERIX
analysis tools.

At present the documentation and Help facilities are copies of the ASTERIX files,
so that some of the tools used in examples may be no longer available.


\newpage

\subsection{Format Conversion}

\rule{\textwidth}{0.5mm}

\begin{description}
\item [\htmlref{CONVERT}{item_CONVERT}]
--- Data format conversion
\end{description}

\rule{\textwidth}{0.5mm}



\subsubsection{CONVERT  \hfill
\xref{SUN/55}{sun55}{}}
\index{CONVERT}
\label{item_CONVERT}

\textbf{Convert the standard Starlink $n$-d data format \htmlref{NDF}{item_NDF}
to and from other formats.}

A major advantage of this package is that it deals sensibly (as far as
possible) with header information, which can be lost during translation
by less robust software.

Currently, the Unix version can handle these formats:
\begin{itemize}
\item Text (Ascii).
\item \htmlref{FIGARO}{item_FIGARO} (version 2) DST file.
\item GASP image.
\item GIF image.
\item \htmlref{IRAF}{item_IRAF} image.
\item \htmlref{IRCAMPACK}{item_IRCAMPACK} data file.
\item FITS
 (including some IUE final archive products and ISO datasets).
\item PBMPLUS PGM (output only).
\item TIFF image (output only).
\item Sequential unformatted.
\end{itemize}
The following formats can only be handled by the VMS version:
\begin{itemize}
\item Starlink Interim BDF file.
\item \htmlref{DIPSO}{item_DIPSO} file.
\end{itemize}



\newpage

\subsection{Mathematics \& Statistics}

\rule{\textwidth}{0.5mm}

\begin{description}
\item [\htmlref{ASURV}{item_ASURV}]
--- Astronomical survival statistics
\item [\htmlref{MAPLE}{item_MAPLE}]
--- Mathematical manipulation \hfill \emph{Not distributed}
\end{description}

\rule{\textwidth}{0.5mm}


\subsubsection{ASURV \index{ASURV} \hfill \xref{SUN/13}{sun13}{}. MUD/5,6}
\label{item_ASURV}

\textbf{Analyse statistically astronomical data with upper limits.}

Observational astronomers frequently encounter the situation where they
observe a particular property of a previously defined sample of objects,
but fail to detect them all.
The data then contains `upper limits' as well as detections, preventing the
use of simple and familiar statistical techniques in the analysis.
However, a variety of other statistical methods exist to deal with these
problems which are collectively called `survival analysis' or the `analysis
of lifetime data' from their origin in actuarial and related fields.
The upper limits are called `censored' data points.
ASURV is a menu-driven stand-alone computer package to help astronomers use
some of these methods.

No statistical procedure can magically recover information that was
never measured at the telescope.  However, frequently there is
important information implicit in the failure to detect some objects
which can be partially recovered under reasonable assumptions.  ASURV
provides several two-sample tests, correlation tests, and linear
regressions -- each based on different models of where upper limits
truly lie -- so that you can judge the importance of the different
assumptions.

\subsubsection{MAPLE \index{MAPLE} \hfill%
\xref{SUN/107}{sun107}{}. \xref{SGP/47}{sgp47}{}. MUD/52,137-139,144,145}
\label{item_MAPLE}

\textbf{Interactive symbolic algebra computation.}

It can perform hundreds of algebraic functions for use at all mathematical
levels, and can provide solutions to many types of problem:
\begin{itemize}
\item Arithmetic with integers, fractions, and polynomials.
\item Power series.
\item Differentiation and integration of functions.
\item Systems of equations.
\item Differential equations.
\item Linear optimization.
\item Tensor manipulation.
\item Symbolic and numeric approximation.
\item Automatic generation of Fortran code and \LaTeX\ source for mathematical
expressions.
\end{itemize}
In addition, it can generate plots to illustrate graphically any
function, including user-defined functions.
You can also extend or redefine the numerous functions by writing MAPLE
programs in the built-in Pascal-like language to create specialized functions.

More general information about computer algebra software is available in
\xref{SGP/47}{sgp47}{}.

It is marketed by WATCOM, Waterloo, Ontario, CANADA.


\newpage

\subsection{Graphics}

\rule{\textwidth}{0.5mm}

\begin{description}
\item [\htmlref{MONGO}{item_MONGO}]
--- Interactive data plotting \hfill \emph{Not distributed}
\item [\htmlref{PONGO}{item_PONGO}]
--- Interactive data plotting
\item [\htmlref{SM}{item_SM}] --- Interactive data plotting
\end{description}

\rule{\textwidth}{0.5mm}


\subsubsection{MONGO \index{MONGO} \hfill \xref{SUN/64}{sun64}{}. MUD/54}
\label{item_MONGO}

\textbf{Plot data interactively.}

You can build up a complicated diagram, including graphics, text, axes, and
so on, and then create publishable quality output.
In general, it has been superseded by
\htmlref{PONGO}{item_PONGO}.

\subsubsection{PONGO \index{PONGO} \hfill \xref{SUN/137}{sun137}{}}
\label{item_PONGO}

\textbf{Plot data interactively.}

It is like
\htmlref{MONGO}{item_MONGO}, but uses
\htmlref{Starlink-PGPLOT}{item_STARLINK-PGPLOT} as the plotting package.
It is more powerful than MONGO and is Starlink-compliant, which means you
can use it in conjunction with
\htmlref{ICL}{item_ICL} and
\htmlref{AGI}{item_AGI}.
Features include:

\begin{itemize}
\item Read data from a text file using a command with the following
features:

\begin{itemize}
\item the ability to read character strings as well as numeric values.
\item support for comment lines and column headings.
\item error trapping during file input.
\item user specification of column delimiters, allowing \LaTeX\ and \TeX\
format tables to be read.
\item selective reading of data.
\end{itemize}

\item Use Fortran-like statements to perform mathematical manipulations on data.
% \item Perform mathematical manipulations on data with For\-tran-like statements.
\item Specialised extra data columns.
\item Interactive cursor functions.
\item Error ellipses.
\item Vector plots.
\item Simple statistical analyses.
\item Resample data.
\item Draw user-specified functions defined by Fortran-like statements.
\item Plot astronomical positional data in one of several geometries.
\item Read positions as Right Ascension and Declination.
\end{itemize}

The PONGO package may be run from the IRAF CL.

\subsubsection{SM \index{SM} \hfill MUD/159,160}
\label{item_SM}

\textbf{Draw graphs and plots interactively.}

It has some image handling capability, but works mostly with vectors.
It can:
\begin{itemize}
\item Generate a nice-looking plot with a minimal number of simple commands.
\item View the plot on screen then, with a simple set of commands,
generate hardcopy.
\item Build and save your own plot routines, which can then be invoked with
a single user-defined command.
\item Keep a history of your plot commands, which you can edit and define as
a plot routine for reuse.
\item Specify plot data from within the program, or read it from a simple file.
\end{itemize}



\newpage

\section{Utilities}

\subsection{Astronomical Utilities}

\rule{\textwidth}{0.5mm}

\begin{description}
\item [\htmlref{ASTROM}{item_ASTROM}]
--- Basic astrometry
\item [\htmlref{AUTOASTROM}{item_AUTOASTROM}]
--- Autoastrometry for Mosaics
\item [\htmlref{CHART}{item_CHART}]
--- Finding chart and stellar data system
\item [\htmlref{COCO}{item_COCO}]
--- Celestial coordinate converter
\item [\htmlref{ECHWIND}{item_ECHWIND}]
--- \'{E}chelle observation planning
\item [\htmlref{FINDCOORDS}{item_FINDCOORDS}]
--- Find coordinates of named objects
\item [\htmlref{HDSTOOLS}{item_HDSTOOLS}]
--- HDS file manipulation \hfill \emph{NEW}
\item [\htmlref{HDSTRACE}{item_HDSTRACE}]
--- HDS object lister
\item [\htmlref{NAOS}{item_NAOS}]
--- NAOMI guide star tool
\item [\htmlref{OBSERVE}{item_OBSERVE}]
--- Star observability checker
\item [\htmlref{RV}{item_RV}]
--- Radial components of observer's velocity
\item [\htmlref{SKYCAL}{item_SKYCAL}]
--- Interactive almanac and calculator
\end{description}

\rule{\textwidth}{0.5mm}



\subsubsection{ASTROM \index{ASTROM} \hfill \xref{SUN/5}{sun5}{}}
\label{item_ASTROM}

\textbf{Basic astrometry -- simple plate reduction.}

It is designed to allow the non-specialist
to get good results with a minimum of trouble and esoteric knowledge.
You supply a text file containing information about the exposure and the
positions of reference and unknown stars;  it performs the various
coordinate transformation and fitting operations required, displays a synopsis
report on your terminal, and prepares a detailed report for printing.

\subsubsection{AUTOASTROM \index{AUTOASTROM} \hfill \xref{SUN/251}{sun251}{}}
\label{item_AUTOASTROM}

We describe the autoastrom package, which provides an interface
to the ASTROM application. This creates a semi-automatic route
to doing astrometry on CCD images.

THIS IS BETA SOFTWARE. Some features work only partially or unreliably.
It does not have all the functionality of a production release. The
interface may well change.

Starlink's ASTROM application provides powerful astrometry facilities,
for analysing astronomical images; it is, however, rather cumbersome
to use. As described in SUN/5, ASTROM can:

\begin{itemize}
\item obtain the plate centre and plate scale of a CCD image, by comparing
      plate positions of objects with those in a reference catalogue, and
      fitting with both four- and six-component distortion models;

\item with enough data, obtain cubic distortion and plate tilt, using seven-
      to nine-component models;

\item if enough information is available, it will do the reductions in
      observed place, correcting for atmospheric refraction.
\end{itemize}

Autoastrom provides a shell around ASTROM so that, as well as the core
astrometric facilities of ASTROM, autoastrom will:

\begin{itemize}
\item work on a CCD image provided as an NDF, as long as it has at least
      rough astrometry (plate centre and scale);

\item automatically download appropriate reference \\
      catalogue information from catalogue servers \\
      supported by the SkyCat library;

\item insert the astrometric results into the original NDF, as a WCS component;

\item alternatively or additionally make the astrometry available as a set of
      FITS-WCS header cards.
\end{itemize}

\subsubsection{CHART \index{CHART} \hfill \xref{SUN/32}{sun32}{}}
\label{item_CHART}

\textbf{Plot star fields (positions, magnitudes, and other data) from
the CSI79, SAO, AGK3, PERTH70, and Dixon non-stellar objects
catalogues.}

\textit{CHART is not distributed on Starlink CDs but it may remain as a
legacy package at UK Starlink sites.  There is no Linux port an little
support available.  CHART will be superceeded by facilities in CURSA in
the near future.}

You can specify a series of search areas, place magnitude or total
number limits on the search, and chose which source catalogues to
include or exclude.  After the search is made, you may list the results
at a terminal or on a printer.  Positional information may be precessed
to a specified equinox.  In the case of astrometric data, proper
motions may be applied up to a specified epoch.

You can plot the results in the form of an overlay or finding chart.
Extra objects, \emph{i.e.}\/ with positions supplied by you, may be
added to the plot at this stage.  The plot may be made on any
\htmlref{GKS}{item_GKS} device, and a number of different plot options
are available, such as scale and area, an RA and Dec coordinate grid,
and various forms of error box.

You may also use the results as input to an astrometry program,
\emph{i.e.}\/ it can be used to select astrometric stars as positional
references.  In this case, you will be asked to supply \emph{x,y}\/
positions for the reference stars from a measuring machine, and can
then convert unknown \emph{x,y}\/ positions to RA, Dec or vice versa.
\htmlref{ASTROM}{item_ASTROM} performs the actual astrometry, from
within CHART if necessary.

\subsubsection{COCO \index{COCO} \hfill \xref{SUN/56}{sun56}{}}
\label{item_COCO}

\textbf{Convert star coordinates from one system to another.}

Both the improved IAU system, post-1976, and the old pre-1976 system are
supported.
It can transform between the following coordinate systems:
\begin{itemize}
\item mean \radec, old system, with E-terms (loosely FK4).
\item mean \radec, old system, no E-terms (some radio positions).
\item mean \radec, new system (loosely FK5).
\item geocentric apparent \radec, new system.
\item ecliptic coordinates $[\lambda,\beta\,]$, new system (mean of date).
\item galactic coordinates $[l^{II},b^{II}]$, IAU 1958 system.
\end{itemize}
The input/output arrangements are flexible to allow a variety of operating
styles: interactive, input from a file, report to a file, batch, etc.
Also, in addition to the report which is produced, the results are
available in raw form, to a fixed resolution, and free from extraneous
formatting; this file is intended to be read easily by other programs.

It offers control over report resolution, and has simple online help.
All input is free-format, and defaults are provided where this is meaningful.

In order to comply with the IAU 1976 recommendations, all position data
published from 1984 on should be given in the new system, using equinox
J2000.0.
However, for years to come, positions will frequently be given in the old
system, using equinox B1950.0.
Discriminating astronomers and astrophysicists are best advised to give both
B1950 and J2000 positions for sources mentioned in their publications; the
conversion can be done using COCO.

\subsubsection{ECHWIND \index{ECHWIND} \hfill \xref{SUN/53}{sun53}{}}
\label{item_ECHWIND}

\textbf{Plan observations with either the Utrecht \'{e}chelle spectrograph
(UES) or the UCL coud\'{e} \'{e}chelle spectrograph (UCLES).}

It can be run at your home institution, when preparing an observing
proposal, or before or during an observing run.

It lets you view interactively that part of the spectrum which will fall on
the detector for a given central wavelength.
This is achieved by drawing a complete \'{e}chellogram in an X-window, along
with a box representing the exact size and shape of the detector.
The positions of individual spectral lines can also be marked on the display.
You can then move the detector window around by moving a crosshair cursor to
the desired centre of the detector, and then clicking the left-hand mouse button
to set the detector position, or else by explicitly specifying the desired
central wavelength.
At each position the screen displays the minimum, central, and maximum
wavelengths and order numbers falling on the detector, as well as an estimate
of the length of the detector (in Angstroms).
The position of the box can be marked on the screen, so as to ease preparation
of observations where multiple overlapping wavelength ranges are required.
You can also save the current detector position in a text file.

\subsubsection{FINDCOORDS \index{FINDCOORDS} \hfill \xref{SUN/240}{sun240}{}}
\label{item_FINDCOORDS}

\textbf{Finding the Coordinates of a Named Object.}

FINDCOORDS  is a new utility for using the name of an astronomical object to
look up its equatorial coordinates.  You simply enter the name of the object
and its coordinates are displayed.  Type:

\begin{quote}
\begin{terminalv}
findcoords  object-name
\end{terminalv}
\end{quote}

For example:

\begin{quote}
\begin{terminalv}
findcoords  ngc3379
\end{terminalv}
\end{quote}

If the name is recognised then the equatorial coordinates of the object will
be displayed.  The Right Ascension is shown in sexagesimal hours and the
Declination in sexagesimal degrees; both are for equinox J2000.

FINDCOORDS works by submitting a remote query via the Internet to the
version of the SIMBAD name-resolver provided by ESO (the basic SIMBAD is
maintained by the Centre de Donnees astronomiques de Strasbourg, CDS).
Consequently, findcoords will only work on computers with a suitable
Internet connection.  Also, the name given must be recognised by SIMBAD,
though the latter's dictionary of names is very extensive.


\subsubsection{HDSTOOLS\index{HDSTOOLS} \hfill \xref{SUN/245}{sun245}{}}
\label{item_HDSTOOLS}

\textbf{HDS file editing and manipulating tools.}

HDSTOOLS is a suite of tools for editing and manipulating HDS files.
They are updated versions of the tools once available in the ASTERIX package.
The following tools are available:

\begin{itemize}

\item HCOPY    - Copy HDS data objects.
\item HCREATE  - Create an HDS data object of specified type and dimensions.
\item HDELETE  - Delete an HDS object.
\item HDIR     - Produce a simple summary of an HDS object.
\item HDISPLAY - Display the contents of a primitive HDS object.
\item HFILL    - Fill an HDS data object with a specified value.
\item HGET     - Return information about an object.
\item HHELP    - Obtain HELP on the HDSTOOLS package.
\item HMODIFY  - Modify the value of an HDS object.
\item HREAD    - Read a file into an HDS object.
\item HRENAME  - Rename an HDS data object.
\item HRESET   - Change state of a primitive HDS object to undefined.
\item HRESHAPE - Reshape an HDS object.
\item HRETYPE  - Change the type of an HDS structure object.
\item HTAB     - Display one or more vector objects in table form.
\item HWRITE   - Write an HDS object into formatted or unformatted file.

\end{itemize}

\subsubsection{HDSTRACE \index{HDSTRACE} \hfill \xref{SUN/102}{sun102}{}}
\label{item_HDSTRACE}

\textbf{List the contents of an HDS data structure.}

The Hierarchical Data System
(\htmlref{HDS}{item_HDS})
is one of the most popular features of the
Starlink environment as it enables associated data items to be stored together
in a single file, but in a structured fashion.

\subsubsection{NAOS \index{NAOS} \hfill \xref{SUN/235}{sun235}{}}
\label{item_NAOS}

\textbf{Finding Guide Stars for use with NAOMI on the WHT.}

The NAOS package may be used to find guide stars suitable for use with
the NAOMI adaptive optics system available on the William Hershell Telescope
(WHT) on La Palma.

NAOMI usually requires a relatively bright guide star located close to the
target object being observed.  You prepare a list of potential target objects
and NAOS remotely searches a version of the USNO PMM astrometric catalogue to
find suitable guide stars for these targets.  You would usually use NAOS at
your home institution prior to travelling to La Palma to observe with NAOMI.

\subsubsection{OBSERVE \index{OBSERVE} \hfill \xref{SUN/146}{sun146}{}}
\label{item_OBSERVE}

\textbf{Show the observability of a star through the year from a given geographical
location.}

It generates a plot that shows, for a complete year the:
\begin{itemize}
\item Star rising \& setting times.
\item Times the star is 30$^{\circ}$ above the horizon.
\item Times of astronomical twilight.
\item Phase, rising, and setting times of the Moon.
\item Distance of the Moon from the star.
\end{itemize}

\subsubsection{RV \index{RV} \hfill \xref{SUN/78}{sun78}{}}
\label{item_RV}

\textbf{List the components, in a given direction, of the observer's velocity on a
given date.}

This allows an observed radial velocity to be referred to an appropriate
standard of rest.
It also computes light time components to the Sun, allowing the times of
phenomena observed from a terrestrial observatory to be referred to a
heliocentric frame of reference.

The accuracies, of better than $0.01~{\rm km~s}^{-1}$ and $0.1~{\rm s}$, are
adequate for most classes of work, the notable exception being some pulsar
observations.

The input/output arrangements are flexible to allow a variety of operating
styles: interactive, input from a file, batch, etc.
All input is free-format.
The normal operating style is interactive: prompts appear on a terminal, data
are entered specifying the report required, and a listing is sent to disk for
later printing.

\subsubsection{SKYCAL\index{SKYCAL} \hfill%
\xref{SUN/187}{sun187}{}. MUD/158}
\label{item_SKYCAL}

\textbf{Almanac and calculator -- interactive.}

This consists of two programs:

\begin{itemize}
\item \textbf{SKYCALC} \index{SKYCAL!SKYCALC} -- an `Interactive Almanac.'
It helps you plan and execute observing runs, and allows easy calculation of
airmasses, twilight, lunar interference, coordinate transformations and such
-- nearly everything except the weather.
\item \textbf{SKYCALENDAR} \index{SKYCAL!SKYCALENDAR} \index{SKYCALENDAR} --
a `Nighttime Astronomical Calendar' to calculate and print astronomical
calendars.
\end{itemize}

Both programs have several preset Observatory locations to choose from.
You can also specify your own.



\newpage

\subsection{General Utilities}

\rule{\textwidth}{0.5mm}

\begin{description}
\item [\htmlref{DOCFIND}{item_DOCFIND}]
--- Starlink document search
\item [\htmlref{EMAIL}{item_EMAIL}]
--- E-mail address search
\item [\htmlref{FORUM}{item_FORUM}]
--- Computer conferencing
\item [\htmlref{JRE}{item_JRE}]
--- Java Runtime Environment
\item [\htmlref{NEWS}{item_NEWS}]
--- Starlink news system
\item [\htmlref{PERL}{item_PERL}]
--- Practical extraction and report language
\item [\htmlref{PERLMODS}{item_PERLMODS}]
--- Additional CPAN Modules for Perl
\item [\htmlref{PINE}{item_PINE}]
--- E-mail interface
\item [\htmlref{STARADMIN}{item_STARADMIN}]
--- User database maintainer
\item [\htmlref{STARPERL}{item_STARPERL}]
--- Additional Modules for Perl
\item [\htmlref{XDISPLAY}{item_XDISPLAY}]
--- X window display setup
\end{description}

\rule{\textwidth}{0.5mm}


\subsubsection{DOCFIND\index{DOCFIND} \hfill \xref{SUN/38}{sun38}{}}
\label{item_DOCFIND}

\textbf{Search Starlink document and keyword indexes for a specified keyword.}

You can display or print any document that is found.  Unfortunately, it
will not print \LaTeX\ documents in a satisfactory form.  Thus, its
main use is to find documents, not print them.  Stocks of Starlink
Notes and Papers are kept at every Starlink site, so you can get a
paper copy.

It has been superseded by the more powerful \texttt{showme} and
\texttt{findme} commands provided by the \htmlref{HTX}{item_HTX}
utility.

\subsubsection{EMAIL \index{EMAIL} \hfill \xref{SUN/182}{sun182}{}}
\label{item_EMAIL}

\textbf{Look up information needed to send e-mail.}

The program searches the Starlink user database and the on-line World E-mail
Guide for usernames and sitenames.

\subsubsection{FORUM \index{FORUM} \hfill%
\xref{SUN/205}{sun205}{}. \xref{SSN/33}{ssn33}{}}
\label{item_FORUM}

\textbf{Starlink's conferencing software.}

It is based on the World Wide Web and provides similar capabilities under Unix
to those provided by the DEC product VAXnotes under VMS.
Its main features are that it:
\begin{itemize}
\item Retains an archive of entries or notes, organized in a hierarchy.
\item Restricts access.
\end{itemize}
Users can both read existing notes and submit new ones.
One difference to VAXnotes is that while VAXnotes could maintain
an index of all Conferences, even when they were located on separate systems
at several Starlink sites, FORUM maintains an index only for each instance of
FORUM.


\subsubsection{JRE \index{JRE}}
\label{item_JRE}
% \xref{SUN/NNN}{sunNNN}{}

\textbf{Java Runtime Environment.}

JRE is a copy of the standard Java Runtime Environment which has been
wrapped to allow installation in the Starlink software.  The JRE is
included to provide a standard stable base on which to base Starlink
Java tools.


\subsubsection{NEWS \index{NEWS} \hfill \xref{SUN/195}{sun195}{}}
\label{item_NEWS}

\textbf{Starlink's News service.}

Just type \texttt{news} and follow the instructions.

\subsubsection{PERL \index{PERL} \hfill \xref{SUN/193}{sun193}{}}
\label{item_PERL}

\textbf{A language for manipulating text, files, and processes easily.}

It provides a more concise and readable way to do many jobs that were
formerly accomplished (with difficulty) by programming in the C
language or one of the shells.  There are several O'Reilly books on
Perl that will help you understand and use it.  \emph{Learning Perl}\/
and \emph{Programming perl}\/ are two of these books.  Chapter 39 of
\emph{Unix Power Tools}\/ gives an introduction to Perl.

\subsubsection{PERLMODS \index{PERLMODS} \hfill \xref{SUN/193}{sun193}{}}
\label{item_PERLMODS}

\textbf{Additional CPAN modules for the Perl package}

PERLMODS is a package of additional modules from CPAN (Comprehensive
Perl Archive Network) which are added to the standard
\htmlref{PERL}{item_PERL} installation to provide extra facilities
needed by the \htmlref{STARPERL}{item_STARPERL} modules and by the
\htmlref{SURF}{item_SURF} package.

\subsubsection{PINE \index{PINE} \hfill%
\xref{SUN/169}{sun169}{}, \xref{170}{sun170}{}. MUD/141}
\label{item_PINE}

\textbf{Read e-mail.}

Just type \texttt{pine} to start it.
A lot of on-line help is available.

\subsubsection{STARADMIN \index{STARADMIN} \hfill \xref{SSN/27}{ssn27}{}}
\label{item_STARADMIN}

\textbf{Build and maintain a database of Starlink users.}

Used by Starlink Site Managers to manage their user lists.

\subsubsection{STARPERL \index{STARPERL} \index{NDFPERL}%
\index{STARPERL!NDFPERL} \hfill \xref{SUN/222}{sun222}{}, \xref{SUN/228}{sun228}{}}
\label{item_STARPERL}

\textbf{Additional modules for the Perl package}

STARPERL is a package of additional Starlink modules which are added to
the standard \htmlref{PERL}{item_PERL} installation to provide
facilities such as NDF access (NDFPERL).

\subsubsection{XDISPLAY \index{XDISPLAY} \hfill \xref{SUN/129}{sun129}{}}
\label{item_XDISPLAY}

\textbf{Get output from an X-client to display on the screen
of your X-terminal.}

In other words, it provides an easy way to use remote X windows.



\newpage

\subsection{Document Preparation}

\rule{\textwidth}{0.5mm}

\begin{description}
\item [\htmlref{A2PS}{item_A2PS}]
--- Ascii to PostScript converter
\item [\htmlref{EMACS}{item_EMACS}]
--- Text editor
\item [\htmlref{GHOSTSCRIPT}{item_GHOSTSCRIPT}]
--- PostScript previewer
\item [\htmlref{HTX}{item_HTX}]
--- Hypertext utilities
\item [\htmlref{ISPELL}{item_ISPELL}]
--- Spelling checker
\item [\htmlref{JED}{item_JED}]
--- Text editor
\item [\htmlref{LATEX2HTML}{item_LATEX2HTML}]
--- \LaTeX\ to HTML converter
\item [\htmlref{NUTPU}{item_NUTPU}]
--- Text editor
\item [\htmlref{PSMERGE}{item_PSMERGE}]
--- PostScript file manipulator
\item [\htmlref{STAR2HTML}{item_STAR2HTML}]
--- Starlink document to HTML converter
\item [\htmlref{TEX}{item_TEX}]
--- Document preparation system
\end{description}

\rule{\textwidth}{0.5mm}


\subsubsection{A2PS \index{A2PS} \hfill \xref{SUN/184}{sun184}{}}
\label{item_A2PS}

\textbf{Convert Ascii format files to PostScript format for printing on
PostScript printers.}

This gives a nicer looking output.

\subsubsection{EMACS \index{EMACS} \hfill%
\xref{SUN/34}{sun34}{}, \xref{170}{sun170}{}. MUD/102}
\label{item_EMACS}

\textbf{A powerful text editor with many advanced customizable features.}

It is fairly easy to make it emulate
EDT\index{EDT}\index{EMACS!EDT} and
EVE\index{EVE}\index{EMACS!EVE}.
This makes it convenient for users familiar with these VAX/VMS editors.

\subsubsection{GHOSTSCRIPT \index{HTX} \hfill \xref{SUN/197}{sun197}{}}
\label{item_GHOSTSCRIPT}

\textbf{PostScript interpreter.}

This provides:
\begin{itemize}
\item An interpreter for the PostScript language.
\item A set of C procedures that implement the graphics capabilities that
appear as primitive operations in the PostScript language.
\end{itemize}

\subsubsection{HTX \index{HTX} \hfill \xref{SUN/188}{sun188}{}}
\label{item_HTX}

\textbf{Hypertext cross-reference utilities.}

These are used to maintain large sets of documents which use
HTML,\index{HTX!HTML} the Hypertext Markup Language.
They can also form the basis for hypertext help systems for other software.
There are three components:
\begin{itemize}
\item \textbf{hlink} -- a hypertext linker.\index{HTX!hlink}
This establishes hypertext links between documents which use HTML, and maintains
the integrity of these links when the documents are changed.
\item \textbf{findme} -- searches for documents by keyword and displays a list of
those found.\index{findme} \index{HTX!findme}
\item \textbf{showme} -- displays a specified part of a document using a web
browser.
\index{showme} \index{HTX!showme}
\end{itemize}

\subsubsection{ISPELL \index{ISPELL} \hfill \xref{SUN/189}{sun189}{}}
\label{item_ISPELL}

\textbf{Check spelling.}

A screen-oriented spell-checker that shows spelling errors in the context
of the original file, and suggests possible corrections when it can figure
them out.

\newpage

\subsubsection{JED\index{JED} \hfill%
\xref{SUN/168}{sun168}{},%
\xref{170}{sun170}{}}
\label{item_JED}

\textbf{Edit text.}

A text editor which provides a reasonably good emulation of
EDT\index{EDT}\index{JED!EDT},
together with a few extra facilities such as cut-and-paste of rectangular
regions, unlimited undo, horizontal scroll, and display of two or more windows.

\subsubsection{LATEX2HTML\index{LATEX2HTML} \hfill \xref{SUN/201}{sun201}{}. MUD/152}
\label{item_LATEX2HTML}

\textbf{Convert \LaTeX\ to HTML.}

\textit{LATEX2HTML is now included in the STAR2HTML package.}

Processes \LaTeX\ source files into a set of Hypertext Markup Language
(HTML) files, suitable for browsing with hypertext viewers and web browsers.

\subsubsection{NUTPU\index{NUTPU} \hfill \xref{SUN/192}{sun192}{}}
\label{item_NUTPU}

\textbf{Process text.}

A fully programmable text processing utility.
It can build new text processors and batch-oriented text manipulation
routines.
Its main value for Starlink users is its close emulation of
EDT\index{EDT}\index{NUTPU!EDT}.

\subsubsection{PSMERGE\index{PSMERGE} \hfill \xref{SUN/164}{sun164}{}}
\label{item_PSMERGE}

\textbf{Merge PostScript files.}

Merges one or more EPSFs into a single PostScript file.
An EPSF (Encapsulated PostScript File) is a PostScript file structured so that
it can be incorporated or included into another PostScript file (so that, for
example, a diagram created with a graphics application can be inserted into a
text document created with a word processor).

Input files can be individually rotated, scaled, and shifted.
Output files can be either EPSF or `normal' PostScript suitable for a
printer.
This allows:
\begin{itemize}
\item Pictures created by different applications to be overlayed.
\item Multiple pictures to be pasted onto a single page.
\item Pictures to be scaled before insertion into a \TeX\ document.
\end{itemize}

\subsubsection{STAR2HTML \index{STAR2HTML} \hfill \xref{SUN/199}{sun199}{}.}
\label{item_STAR2HTML}

\textbf{Convert Starlink documents to Hypertext.}

Processes Starlink document \LaTeX\ source files into a set of
Hypertext Markup Language (HTML) files, suitable for browsing with
hypertext viewers and web browsers.

This is a major component of Starlink's system for producing hypertext
documents, which also includes
\htmlref{TEX}{item_TEX},
\htmlref{HTX}{item_HTX},
\htmlref{GHOSTSCRIPT}{item_GHOSTSCRIPT},
the \texttt{dvips} processor, and \htmlref{PERL}{item_PERL}.

The processing to hypertext is performed by \latextohtml\ which
is bundled with STAR2HTML.  \latextohtml\ is not used directly becuase
Starlink documents use a series of additional macros to allow
inter-document cross referencing within the Starlink document
collection.

\subsubsection{TEX \index{TEX}\index{LATEX}\index{TEX!LATEX} \hfill
\xref{SUN/9}{sun9}{},\xref{12}{sun12}{},\xref{93}{sun93}{}.%
MUD/48-51,72-3,79,80,126,132-3,152.%
\xref{SSN/24}{ssn24}{}}
\label{item_TEX}

\textbf{Produce high quality text and diagrams by type-setting.}

\LaTeX\ is a document preparation system based on \TeX, but easier to use.
The current versions are \TeX\ v3.1415 and \LaTeX2e\ (Dec 1994).
The final output is usually produced by a laser printer.

Starlink recommends that you create your documents using \LaTeX2e\ with formats
based on skeleton files such as \texttt{/star/docs/sun.tex.}

\xref{SUN/12}{sun12}{}
is a cookbook showing how to generate common kinds of format using
\LaTeX; this is a quick way to learn how to use it.
\xref{SGP/28}{sgp28}{} and \xref{SGP/50}{sgp50}{}
give advice on how to produce documents in a suitable style for Starlink.



\newpage

\subsection{Programming Support}

\rule{\textwidth}{0.5mm}

\begin{description}
\item [\htmlref{FTNCHEK}{item_FTNCHEK}]
--- Fortran source code checker
\item [\htmlref{GCC}{item_GCC}]
--- C compiler
\item [\htmlref{GENERIC}{item_GENERIC}]
--- Generic Fortran subroutine compiler
\item [\htmlref{MESSGEN}{item_MESSGEN}]
--- Error message generator
\item [\htmlref{SPAG}{item_SPAG}]
--- Fortran code improver
\item [\htmlref{SPT}{item_SPT}]
--- Software porting tools
\item [\htmlref{STARX}{item_STARX}]
--- X library linker
\end{description}

\rule{\textwidth}{0.5mm}


\subsubsection{FTNCHEK \index{FTNCHEK} \hfill \xref{SUN/172}{sun172}{}. MUD/153}
\label{item_FTNCHEK}

\textbf{Detect errors in Fortran 77 source code, especially mistakes that tend to
be missed by compilers and linkers.}

Probably its most valuable feature is the careful checks it performs on
subprogram interfaces.
It examines the actual arguments in each \texttt{CALL} statement (or \texttt{FUNCTION}
call) and compares them with the corresponding formal arguments of the
\texttt{SUBROUTINE} (or \texttt{FUNCTION}) subprogram.
It warns of mismatches in the number of arguments, and incompatibilities in
the data type and dimensions of each one.
It also checks that arguments used before being set in a subprogram have
defined values on entry, and correspondingly that arguments are set on exit
when their values are used later in the calling program.

\subsubsection{GCC \index{GCC}\hfill \mbox{}}
\label{item_GCC}

\textbf{The GNU C compiler.}

\subsubsection{GENERIC \index{GENERIC} \hfill \xref{SUN/7}{sun7}{}}
\label{item_GENERIC}

\textbf{Support generic subroutines.}

Preprocesses a \textit{generic}\/ Fortran subroutine -- one written to cover
several different data types -- into one routine per data type, and
concatenates these routines into a file.
The file is then compiled by the Fortran compiler to produce an object module.

\subsubsection{MESSGEN\index{MESSGEN} \hfill \xref{SUN/185}{sun185}{}}
\label{item_MESSGEN}

\textbf{Associate error messages with status values returned by Starlink
infrastructure routines.}

This works in conjunction with the Starlink Error Message Service
(\htmlref{EMS}{item_EMS}).
It enables more informative error messages to be sent to the user.

\subsubsection{SPAG \index{SPAG} \hfill \xref{SUN/63}{sun63}{}. MUD/127}
\label{item_SPAG}

\textbf{Improve the structure of the source code of a poorly organised
Fortran program.}

\textit{SPAG is commercial software and is not distributed
on Starlink CDs.  Access is available to registered Starlink
users on the central service machine at RAL -- contact
\htmladdnormallink{Hiten Patel}{mailto:hiten@star.rl.ac.uk}
\latexonly{(\texttt{hiten@star.rl.ac.uk})} for access.}

`SPAG' stands for `Spaghetti Unscrambler.' It re-orders blocks of
statements in such a way that the structure of the code is improved,
while remaining logically equivalent to the original program.  The
result improves the readability and maintainability of badly-written
Fortran code.  It is marketed by Polyhedron Software.

\subsubsection{SPT \index{SPT} \hfill \xref{SUN/111}{sun111}{}}
\label{item_SPT}

\textbf{Software porting tools.}

A set of tools to convert code between operating systems
(principally between VMS and Unix, and between different flavours of Unix).
They:
\begin{itemize}
\item Convert file name specifications in Fortran \texttt{INCLUDE} statements and
`escape' backslash characters.
The output file on VMS is in Stream\_lf format so it can be read on Unix
systems using NFS.
\item Manage the creation and removal of soft links on Unix systems.
Generally, these will point to Fortran \texttt{INCLUDE} files, thereby providing a
machine-independent way of writing \texttt{INCLUDE} statements.
\end{itemize}

\subsubsection{STARX \index{STARX} \hfill \xref{SSN/21}{ssn21}{}}
\label{item_STARX}

\textbf{Link X libraries.}

A Starlink environment package that provides a system-independent method of
linking with X libraries.
This relieves other libraries and applications from having to worry about the
various complications involved.
It allows the X library linking strategy to be changed easily, without
changing any of the other Starlink environment components.


\newpage

\section{Subroutine Libraries}

\subsection{Astronomical \& Mathematical}

\rule{\textwidth}{0.5mm}

\begin{description}
\item [\htmlref{JPL}{item_JPL}]
--- Solar system ephemeris
\item [\htmlref{MEMSYS}{item_MEMSYS}]
--- Maximum entropy image reconstruction
\item [\htmlref{NAG}{item_NAG}]
--- Mathematics, statistics, and graphics
\item [\htmlref{PDA}{item_PDA}]
--- Public domain algorithms
\item [\htmlref{SLALIB}{item_SLALIB}]
--- Positional astronomy
\end{description}

\rule{\textwidth}{0.5mm}


\label{item_JPL}
\subsubsection{JPL \index{JPL} \hfill \xref{SUN/87}{sun87}{}}

\textbf{Provide extremely accurate position and velocity data for any of the nine
planets, the Moon, and the solar system and Earth-Moon barycentres (also
nutation components) for any given epoch in the range of the ephemeris
(1960 to 2025).}

For many years, the Jet Propulsion Laboratory has distributed definitive solar
system ephemeris tapes containing files of ephemeris data, plus Fortran
programs to read them.
The ephemerides come from direct numerical integration of the equations
of motion of the large bodies in the solar system, subject to fitting to a
variety of observations.
For distribution they are expressed in terms of Chebyshev polynomials.

The version distributed by Starlink is the DE200/LE200, the one most
consistent with the latest IAU resolutions.
Users of the old (VMS-only) version should note that this new version differs
in some important respects.
Existing applications will need to be changed before they can be linked
against the new library, and existing executable programs are not compatible
with the new format of ephemeris file.

\subsubsection{MEMSYS\index{MEMSYS} \hfill \xref{SUN/117}{sun117}{}. MUD/53}
\label{item_MEMSYS}

\textbf{Quantified maximum entropy image reconstruction.}

\textit{The MEMSYS libraries are commercial software and are licenced for
UK Starlink sites only.  MEMSYS is not distributed on Starlink CDs.  Access
restrictions apply at Starlink sites (see below).}

Two versions are available: MEMSYS3 \index{MEMSYS!MEMSYS3} and
MEMSYS5. \index{MEMSYS!MEMSYS5}

The algorithms used in both are similar, but the programming interfaces
are quite different.  New applications should be written using MEMSYS5,
which provides more functionality than MEMSYS3.  MEMSYS3 is retained
primarily for the benefit of people who want to maintain old
applications.

They find and use maximum entropy reconstructions of images, power spectra,
and other such additive distributions from arbitrary sets of data.
They were written by J.\ Skilling and S.\ Gull of Maximum Entropy Data
Consultants Ltd, and embody their `Classic' Maximum Entropy algorithm,
described in more detail in the MEMSYS3 user's manual.

The algorithms of both MEMSYS3 and MEMSYS5 differ greatly from their previous
MEM algorithms, in that they are based on a fully Bayesian analysis of the
image reconstruction problem.
They provide the following extra functions:
\begin {itemize}
\item Calculate automatically the most probable noise level in the input data.
\item Estimate errors in the output reconstruction.
\item Handle negative data values.
\item Handle limited non-linearities in the data.
\item Calculate Poisson or Gaussian statistics for input data noise.
\end {itemize}

MEMSYS deals with `data sets' and `images'.
A \emph{data set}\/ holds information corresponding to the available
experimental data, and an \emph{image}\/ holds information corresponding
to the `true' image from which the data was generated.
It handles problems where the relationship between data and image can
described as follows:

\begin {equation}
F_{k}=\sum_{j=1}^{M_{k}} (R_{kj}*f_{j})+n_{k}  \label {EQ:DATA}
\end {equation}

where $F_{k}$ is the $k$th data value, $M_{k}$ is the number of data values,
$f_{j}$ is the $j$th image value, $R_{kj}$ is the response of sample $F_{k}$ to
pixel $f_{j}$, and $n_{k}$ is the noise on sample $k$.

This is a linear relationship between data and image.
It can also cope with non-linear data, so long as the data $D_{k}$ can be
expressed as follows:

\begin {equation}
D_{k} = \Phi(F_{k})
\end {equation}
 where $\Phi$ is some known function with known derivative, and $F_{k}$ is
linearly dependent on the data.

The image and data set are described as 1-d arrays purely for ease
of access within the package.
In fact, they could be of any dimensionality.
The calling program needs to set up the correspondence between
$n$-d coordinates and the 1-d coordinate used above.
For instance, for a 2-d image with coordinates $(pix,lin)$ the
relationship may be

\begin {equation}
 j = pix + NPIX*(lin-1)
\end {equation}

where $NPIX$ is the number of pixels per line of the image.

A common use is to deconvolve a 2-d image, given a Point Spread Function (PSF).
In this case, the data is linear and the matrix $R_{kj}$ embodies the PSF.

Before using it you must sign a form (available from your Site Manager)
accepting certain conditions.


\subsubsection{NAG \index{NAG} \hfill \xref{SUN/28}{sun28}{},%
\xref{29}{sun29}{}. MUD/55,56,57,58,128}
\label{item_NAG}

\textbf{Mathematical subroutine library.}

\textit{The NAG libraries are commercial software and are licensed to UK
Starlink sites only.  The software is not distributed on Starlink CDs.}

The Numerical Algorithms Group (NAG) library is commercial software.
The current versions are Mark 18 for the Numerical library and Mark 4 for the
Graphics library.  Double precision versions of the Numerical Library are
provided and, where available, single precision versions are also provided.
The Graphics Library is in double precision form.

The routines cover a very wide area of numerical and statistical mathematics.
They are extensively documented in a reference manual, guide, and introductory
book.
These are available at every Starlink site, and copies of the NAG newsletters
are kept at RAL.

The standard library organises its routines into the following chapters:
{\small
\begin{itemize}
\item A02 -- Complex arithmetic.
\item C02 -- Zeros of polynomials.
\item C05 -- Roots of one or more transcendental equations.
\item C06 -- Summation of series.
\item D01 -- Quadrature.
\item D02 -- Ordinary differential equations.
\item D03 -- Partial differential equations.
\item D04 -- Numerical differentiation.
\item D05 -- Integral equations.
\item E01 -- Interpolation.
\item E02 -- Curve and surface fitting.
\item E04 -- Minimizing or maximizing a function.
\item F01 -- Matrix operations, including inversion.
\item F02 -- Eigenvalues and eigenvectors.
\item F03 -- Determinants.
\item F04 -- Simultaneous linear equations.
\item F05 -- Orthogonalisation.
\item F06 -- Linear algebra support routines.
\item F07 -- Linear equations (LAPACK).
\item F08 -- Least-squares and eigenvalue problems (LAPACK)
\item G01 -- Simple calculations and statistical data.
\item G02 -- Correlation and regression analysis.
\item G03 -- Multivariate methods.
\item G04 -- Analysis of variance.
\item G05 -- Random number generators.
\item G07 -- Univariate estimation.
\item G08 -- Nonparametric statistics.
\item G10 -- Smoothing in statistics
\item G11 -- Contingency table analysis.
\item G12 -- Survival analysis.
\item G13 -- Time series analysis.
\item H   -- Operations research.
\item M01 -- Sorting.
\item P01 -- Error trapping.
\item S   -- Approximations of special functions.
\item X01 -- Mathematical constants.
\item X02 -- Machine constants.
\item X03 -- Innerproducts.
\item X04 -- Input/output utilities.
\item X05 -- Date and time utilities.
\end{itemize}
}

\subsubsection{PDA \index{PDA} \hfill \xref{SUN/194}{sun194}{}}
\label{item_PDA}

\textbf{Mathematical subroutine library.}

This is intended to replace the
\htmlref{NAG}{item_NAG}
library in Starlink application code.
It is coded in Fortran and has a Fortran 77 binding.
The interface is mostly double precision, although the Fourier transform
part provides for both double and single precision and the routines from
DIERCKX exist only for single precision.

\subsubsection{SLALIB\index{SLALIB} \hfill \xref{SUN/67}{sun67}{}}
\label{item_SLALIB}

\textbf{Positional astronomy library.}

\emph{SLALIB}\/ makes accurate and reliable positional-astronomy
applications easier to write.

Most of the routines are concerned with astronomical position and time, but
some have wider trigonometrical, numerical, or general applications.
\htmlref{ASTROM}{item_ASTROM},
\htmlref{COCO}{item_COCO}, and
\htmlref{RV}{item_RV}
all make extensive use of SLALIB, as do several
telescope control systems around the world.
Current versions are written in Fortran~77 and run on VAX/VMS, several Unix
platforms, and PC.
A generic ANSI~C version is also available from its author.

Its main functions are:
\begin{itemize}
\item String decoding, sexagesimal conversions.
\item Angles, vectors, rotation matrices.
\item Calendars, timescales.
\item Precession, nutation.
\item Proper motion.
\item FK4/5, elliptic aberration.
\item Geocentric coordinates.
\item Apparent and observed place.
\item Azimuth, elevation.
\item Refraction, air mass.
\item Ecliptic, galactic and supergalactic coordinates.
\item Ephemerides.
\item Astrometry.
\item Numerical methods.
\end{itemize}

It gives programmers a basic set of positional-astronomy tools which are
accurate and easy to use.
It is:
\begin{itemize}
\item Readily available, including source code and documentation.
\item Supported and maintained.
\item Portable -- coded in standard languages and available for
multiple computers and operating systems.
\item Thoroughly commented, both for maintainability and to
assist those wishing to cannibalize the code.
\item Stable.
\item Trustworthy -- some care has gone into testing it, both by
comparison with published data and by checks for internal consistency.
\item Rigorous -- corners are not cut, even where the practical consequences
would, in most cases, be negligible.
\item Comprehensive, without including too many esoteric features required
only by specialists.
\item Practical -- almost all the routines have been written to satisfy real
needs encountered during the development of real-life applications.
\item Environment-independent -- the package is completely free of pauses,
stops, I/O, etc.
\item Self-contained -- it calls no other libraries.
\end{itemize}


\newpage

\subsection{Data Access \& Management}

\rule{\textwidth}{0.5mm}

\begin{description}
\item [\htmlref{ARD}{item_ARD}]
--- Region description language
\item [\htmlref{AST}{item_AST}]
--- World Coordinate System data handling
\item [\htmlref{ARY}{item_ARY}]
--- ARRAY data structure access
\item [\htmlref{CAT}{item_CAT}]
--- Catalogue access
\item [\htmlref{CFITSIO}{item_CFITSIO}]
--- Disk FITS I/O (C and Fortran)
\item [\htmlref{FIO}{item_FIO}]
--- Fortran I/O
\item [\htmlref{FITSIO}{item_FITSIO}]
--- Disk FITS I/O
\item [\htmlref{GRP}{item_GRP}]
--- Manipulation of groups of objects
\item [\htmlref{GSD}{item_GSD}]
--- Read Global Section Datafiles
\item [\htmlref{HDS}{item_HDS}]
--- Hierarchical data system
\item [\htmlref{IMG}{item_IMG}]
--- Simple image data access
\item [\htmlref{MAG}{item_MAG}]
--- Magnetic tape handling
\item [\htmlref{NBS}{item_NBS}]
--- Noticeboard system
\item [\htmlref{NDF}{item_NDF}]
--- NDF data structure access
\item [\htmlref{NDG}{item_NDG}]
--- Access to groups of NDFs
\item [\htmlref{PAR}{item_PAR}]
--- Parameter system
\item [\htmlref{PRIMDAT}{item_PRIMDAT}]
--- Primitive numerical data processing
\item [\htmlref{REF}{item_REF}]
--- HDS object referencing
\item [\htmlref{TRANSFORM}{item_TRANSFORM}]
--- Coordinate transformation
\end{description}

\rule{\textwidth}{0.5mm}

\newpage

\subsubsection{ARD\index{ARD} \hfill \xref{SUN/183}{sun183}{}}
\label{item_ARD}
\textbf{Describe regions of a data array.}

The Ascii Region Definition system provides a textual language for
describing regions within a data array, together with software for converting
a textual description into a pixel mask.

The textual language is based on a set of keywords identifying simple shapes
(boxes, circles, lines, etc.).
These can be combined using Boolean operators (AND, OR, NOT, etc) to create
more complex shapes.
Data arrays can be multi-dimensional.


\subsubsection{ARY \index{ARY} \hfill\xref{SUN/11}{sun11}{}}
\label{item_ARY}

\textbf{Access Starlink ARRAY data structures built using
\htmlref{HDS}{item_HDS}.}

Details of these structures and the design philosophy behind them can be
found in
\xref{SGP/38}{sgp38}{},
although familiarity with that document is not necessarily required in order
to use the routines.

ARY is an essential component of the
\htmlref{NDF}{item_NDF}\index{NDF!ARY}
system, which contains routines for accessing Starlink NDF structures.

The main reason for using it is to access ARRAY structures stored in NDF
extensions.
At present, it only supports the `primitive' and `simple' forms of the
ARRAY data structure.

It can:
\begin{itemize}
\item Access existing arrays.
\item Create and delete arrays.
\item Create and control identifiers.
\item Create placeholders.
\item Copy arrays.
\item Enquire and set array attributes.
\item Use message system routines.
\end{itemize}


\subsubsection{AST \index{AST} \hfill%
 \xref{SUN/210}{sun210}{},\xref{SUN/211}{sun211}{}}
\label{item_AST}

\textbf{A library for handling World Coordinate Systems in astronomy}

The AST library may be used for attaching world coordinate systems to
astronomical data, for retrieving and interpreting that information
and for generating graphical output based on it.

The \xref{programmer's manual}{sun210}{} should be of interest to anyone
writing astronomical applications which need to manipulate coordinate system
data, especially celestial coordinate systems.  AST is portable and
environment-independent.


\subsubsection{CAT\index{CAT} \hfill \xref{SUN/181}{sun181}{}}
\label{item_CAT}

\textbf{Create, read, and write astronomical catalogues and similar tabular
datasets.}

It supports the following formats:
\begin{itemize}
\item FITS tables (Ascii and binary).
\item STL (Small Text List; simple lists in text files).
\item CHI/HDS.
\item SCAR/ADC (on VAX/VMS only).
\end{itemize}


\subsubsection{CFITSIO \index{CFITSIO} \hfill%
 \xref{SUN/227}{sun227}{}. MUD/166,167}
\label{item_CFITSIO}

\textbf{Access FITS format files on disk.}

This new library provides both a C implementation of a FITS i/o library
and a Fortran wrapper for it.  Thus one can use it from C or Fortran as
required.

The CFITSIO library is written and actively maintained by William Pence
of HEASARC (High Energy Astrophysics Science Archive Research Center)
at the Goddard Space Flight Center, USA.  It is fully compliant with
the FITS Year 2000 directive.   This is the recommended library for access
to FITS files.


\subsubsection{FIO \index{FIO} \hfill \xref{SUN/143}{sun143}{}}
\label{item_FIO}

\textbf{Access sequential and direct-access data files in a machine-independent
way.}

It consists of stand alone
FIO\index{FIO!FIO} and
RIO\index{FIO!RIO}\index{RIO} routines, which can be used independently of
the Starlink software environment, plus routines to interface to the Starlink
parameter system.


\subsubsection{FITSIO \index{FITSIO} \hfill \xref{SUN/136}{sun136}{}. MUD/16,162}
\label{item_FITSIO}

\textbf{Access FITS format files on disk.}

This is a Fortran-only library for access to FITS files.  It is no longer
being developed and is not compliant with the FITS Year 2000 directive.

Users requiring a library for FITS access are recommended to use the
\htmlref{CFITSIO}{item_CFITSIO} library.


\subsubsection{GRP \index{GRP} \hfill \xref{SUN/150}{sun150}{}}
\label{item_GRP}

\textbf{Handle groups of objects which can be described by character strings.}

These objects can be things like file names, astronomical objects, numerical
values, etc.
It can store and retrieve such strings within groups, and pass groups of
strings between applications by means of a text file.
It can also create groups in which each element is a modified version of an
element in another group.


\subsubsection{GSD \index{GSD} \hfill\xref{SUN/229}{sun229}{}}
\label{item_GSD}

\textbf{Global Section Datafile (GSD) access library.}

This library provides read-only access to Global Section Data files (GSD
files) created at the James Clerk Maxwell Telescope. A description of GSD
itself is presented in addition to descriptions of the library routines.

Its main use is in \htmlref{SPECX}{item_SPECX} to read the GSD files
created by the heterodyne instruments used on the JCMT.


\subsubsection{HDS \index{HDS} \hfill%
\xref{SUN/31}{sun31}{},%
\xref{39}{sun39}{},%
\xref{92}{sun92}{},%
\xref{102}{sun102}{}.%
\xref{SGP/38}{sgp38}{}}
\label{item_HDS}

\textbf{Store and retrieve data flexibly.}

The Hierarchical Data System is the basic Starlink data system and is of great
importance to the Project.
Its most significant features are the following:
\begin{itemize}
\item It manipulates entities called \emph{data objects}.
 These can be scalars or arrays of up to seven dimensions.
\item A data object has a tree-like shape whose components comprise other
 objects. The only restriction is that the leaves must be primitive data
 types.  The degenerate form of an HDS tree is a single scalar or array of a
 primitive data type.  The non-degenerate form is called a structure, which
 comprises two or more components.
\item The primitive data types include the usual Fortran numeric, logical, and
 character types, together with signed and unsigned bytes and words.
\item You can tree-walk.
 Any branch or sub-branch of a tree can be treated as a data object in its
 own right.
\item Data objects are dynamic in the sense that programs can add, modify,
 or delete components to or from a tree.
\item Subsets of primitive arrays can be defined and manipulated as separate
 data objects, and arrays can be mapped as vectors.
\end{itemize}
More powerful data systems have been built on top of HDS in order to make
specific types of data object easier to handle.
These include
\htmlref{ARY}{item_ARY}
for handling arrays, and
\htmlref{NDF}{item_NDF}
for handling objects in the standard Starlink data format.

\newpage

\subsubsection{IMG \index{IMG} \hfill \xref{SUN/160}{sun160}{}}
\label{item_IMG}

\textbf{Easy access to astronomical image data and associated headers.}

It is of most interest to astronomers who write their own software and
require uncomplicated access to their image data.
In this context, an `image' may be an ordinary (2-d) image, but could also
be a spectrum (1-d) or `data cube' (3-d).

Because of Starlink's automatic format conversion facilities, this library
also gives programmers straightforward access to a wide range of astronomical
data formats, including FITS and IRAF files as well as Starlink NDF data.


\subsubsection{MAG \index{MAG} \hfill \xref{SUN/171}{sun171}{}}
\label{item_MAG}

\textbf{Handle magnetic tapes.}

Astronomical data on magnetic tapes can be in many formats.
The MAG library allows you to write programs to read this data on any
type of computer and tape drive.
It doesn't interpret the data format -- it deals in tape blocks, and knows
nothing about FITS, VMS BACKUP, tar, etc.

MAG lets you write portable software.
This is very useful since different computers can differ greatly in their
facilities for handling magnetic tapes.
For example, on VMS it is quite common to treat magnetic tapes as normal
Fortran sequential files that can be opened, read, written, and closed using
the Fortran statements OPEN, READ, WRITE, and CLOSE.
On Suns, this is not the case, and special routines are provided for
processing magnetic tapes.
These differences are hidden from the programmer by the MAG package.

It can:
\begin{itemize}
\item Allocate a tape drive and mount a tape.
\item Open and close tape devices.
\item Read and write data contained in tape blocks.
\item Move tapes relatively (skip files/blocks and rewind).
\item Position tapes absolutely.
\end{itemize}

Not all operations are available on all types of tape deck.
Also, the status values returned by different machines can differ.


\subsubsection{NBS \index{NBS} \hfill \xref{SUN/77}{sun77}{}}
\label{item_NBS}

\textbf{Share data between processes.}

NBS stands for \emph{Noticeboard System}.

A given process may own many noticeboards and may access ones owned by others.
Normally, the only process that writes to a noticeboard is its owner, but
other processes subvert this rule if necessary.

Noticeboards are identified by name, and each can contain a hierarchy of items.
Each item has a name, type, structure/primitive attribute, and, if primitive,
a maximum number of dimensions, maximum number of bytes, current shape, and
value.
The type and shape are not used by the routines, but their values can be put
and got, and they can be used when implementing higher-level routines on top
of the noticeboard routines.

Noticeboards are self-defining -- a process can find and access data from a
noticeboard without knowing anything about what it contains.


\subsubsection{NDF \index{NDF} \hfill \xref{SUN/33}{sun33}{}.%
\xref{SSN/20}{ssn20}{}}
\label{item_NDF}

\textbf{Access data stored in NDF format.}

The \emph{Extensible $n$-Dimensional Data Format}\/ is the standard
Starlink format for $n$-d arrays of numbers, such as spectra, images,
etc, and it forms the basis of most Starlink spectral and image-processing
applications.
A `stand-alone' version is available for programs which do not use the
Starlink software environment.

The main reason for using NDF is to simplify data exchange between applications.
In principle, it should enable you to process the same data with any Starlink
package.
In practice, previous attempts to define a standard data format for this
purpose have met with two serious obstacles.
Firstly, different software authors have interpreted the meaning of data
items differently, so that although several packages might be capable of
reading the same data files, they actually perform incompatible operations
on the data.
Secondly, many authors have found a pre-defined data format to be too
restrictive and have ignored it.

The NDF data structure has, therefore, had to satisfy two apparently
contradictory requirements:

\begin{itemize}
\item Its interpretation should be closely defined, so different (and usually
geographically separated) programmers can process it in consistent and
mutually compatible ways.
\item It should be adaptable, so it can hold data associated with many software
systems whose detailed requirements vary considerably.
\end{itemize}

This problem has been solved by introducing the concept of \emph{extensibility},
and by dividing the data structure into two parts -- a set of \emph{standard
components}\/ and a set of \emph{extensions}\/ -- each of which individually
satisfies one of the two requirements.
The structure consists of a central `core' of information, whose
interpretation is well-defined and for which general-purpose software can be
written with wide applicability, together with an arbitrary number of
extensions, which may be used by specialist software but whose interpretation
is not otherwise defined.
Those who wish to know more of the background to this philosophy can find a
detailed discussion in
\xref{SGP/38}{sgp38}{}.


\subsubsection{NDG\index{NDG} \hfill \xref{SUN/2}{sun2}{}}
\label{item_NDG}

\textbf{Routines for Accessing Groups of NDFs.}

The NDG package provides a means of giving the user the ability to
specify a list (or ``Group'') of NDFs as a reply to a single prompt for
an parameter. NDG can process NDFs which are stored as components within
an HDS container file, and can also process foreign data formats using the
\htmlref{CONVERT}{item_CONVERT} utilities that the \htmlref{NDF}{item_NDF}
library uses (as described in \xref{SSN/20}{ssn20}{}).


\subsubsection{PAR \index{PAR} \hfill \xref{SUN/114}{sun114}{}}
\label{item_PAR}

\textbf{Parameter system.}

A convenient way for programs to exchange information with the
outside world through I/O channels called \emph{parameters}.
These enable a user to control a program's behaviour.

It supports numeric, character, and logical parameters, and is
currently implemented only on top of the
\htmlref{ADAM}{item_ADAM}
parameter system.
Parameter values may be obtained, with or without a variety of constraints.
Results may be put into parameters to be passed to other applications.
Other facilities include setting a prompt string, and suggested defaults.

\newpage

\subsubsection{PRIMDAT\index{PRIMDAT} \hfill \xref{SUN/39}{sun39}{}}
\label{item_PRIMDAT}

\textbf{Process primitive data.}

Does arithmetic, mathematical operations, type conversion, and inter-comparison
of any of the primitive numerical data types supported by
\htmlref{HDS}{item_HDS}.
Provides:
\begin{itemize}
\item Processing facilities which are not normally available.
\item A uniform interface.
\item Improved portability and efficiency.
\item Facilities for processing bad data.
\item Numerical error handling.
\item A set of constants.
\end{itemize}

A distinction is drawn between three classes of primitive data, which
differ in their interpretation and in the algorithms best suited to
processing them:

\begin{itemize}

\item \textbf{Values} -- scalar data which may take one of the
defined \emph{bad values}\/ (sometimes called `magic'
values), whose presence signifies that the affected datum is
undefined.

\item \textbf{Vectorised arrays} -- 1-d arrays of \emph{values}\/ (or
arrays treated as 1-d) whose elements are  processed in the way
described above.

\item \textbf{Numbers} -- always interpreted literally as scalar
numerical quantities (\emph{i.e.}\ \emph{bad}\/ values are not
recognised on input and are not explicitly generated on output).

\end{itemize}


\subsubsection{REF \index{REF} \hfill \xref{SUN/31}{sun31}{}}
\label{item_REF}

\textbf{Handle references to HDS objects.}

Stores references to
\htmlref{HDS}{item_HDS} data objects in special HDS reference objects, and
lets you obtain locators to reference objects.
Helps make your software portable.

You can:
\begin{itemize}
\item Create a reference object and put a reference in it.
\item Obtain a locator to an object (possibly via a reference).
\item Obtain a locator to a referenced object.
\item Create an empty reference object.
\item Put a reference into a reference object.
\item Annul a locator which may have been obtained via a reference.
\end{itemize}

Its two main uses are:
\begin{itemize}
\item To maintain a catalogue of HDS objects.
\item To avoid duplicating a large dataset.
\end{itemize}

\subsubsection{TRANSFORM \index{TRANSFORM} \hfill \xref{SUN/61}{sun61}{}}
\label{item_TRANSFORM}

\textbf{Relate different coordinate systems.}

Provides a standard, flexible method for manipulating coordinate
transformations and for transferring information about them between programs.
It can handle coordinate systems with any number of dimensions, and can
efficiently process large (i.e.\ image-sized) arrays of coordinate data using
a variety of numerical precisions.

No specific support for astronomical coordinate transformations or map
projections is included at present, but routines for handling these should
appear in future.
The current system provides tools for creating a wide variety of coordinate
transformations, so it should be possible to construct some of the simpler
astronomical transformations explicitly, if required.

It can:
\begin{itemize}
\item Define linear and non-linear graphics coordinate systems.
\item Attach coordinate systems to datasets (e.g.\ relating image pixels to
 sky positions).
\item Describe distortion in images and spectra.
\item Store and apply instrument calibration functions.
\end{itemize}

It uses
\htmlref{HDS}{item_HDS}
to store information in standard data structures for interchange between
applications.
These may be used as building blocks when constructing larger
HDS datasets, and also when designing `extensions' to the standard Starlink
\htmlref{NDF}{item_NDF}
data structure.


\newpage

\subsection{Graphics}

\rule{\textwidth}{0.5mm}

\begin{description}
\item [\htmlref{AGI}{item_AGI}]
--- Applications graphics interface
\item [\htmlref{GKS}{item_GKS}]
--- Graphical kernel system
\item [\htmlref{GNS}{item_GNS}]
--- Graphical workstation name service
\item [\htmlref{GRAPHPAR}{item_GRAPHPAR}]
--- ADAM graphics
\item [\htmlref{GWM}{item_GWM}]
--- X graphics window manager
\item [\htmlref{IDI}{item_IDI}]
--- Image display interface
\item [\htmlref{NCAR}{item_NCAR}]
--- High-level graphics
\item [\htmlref{Native-PGPLOT}{item_NATIVE-PGPLOT}]
--- High-level graphics
\item [\htmlref{Starlink-PGPLOT}{item_STARLINK-PGPLOT}]
--- High-level graphics
\item [\htmlref{SGS}{item_SGS}]
--- Simple graphics system
\item [\htmlref{SNX}{item_SNX}]
--- Starlink extensions to NCAR
\end{description}

\rule{\textwidth}{0.5mm}


\subsubsection{AGI \index{AGI} \hfill \xref{SUN/48}{sun48}{}}
\label{item_AGI}

\textbf{Graphics database.}

Used in Starlink programs to store, between applications, information
about the various plots on a graphics device.  It records their
position and extent.  These can be recalled later either to overlay
plots, or to recreate the picture.

Obvious uses are to align pictures, or to obtain cursor coordinates
from an existing picture.  But it has other uses, \emph{e.g.}\ to slice
up the screen into areas for subsequent plotting using more complicated
patterns than just a regular grid; the areas can be different sizes or
overlap.


\subsubsection{GKS \index{GKS} \hfill \xref{SUN/83}{sun83}{},%
\xref{113}{sun113}{}. MUD/27. \xref{SSN/39}{ssn39}{}}
\label{item_GKS}

\textbf{Low-level graphics library.}

The Graphical Kernel System is an international standard.
The current version is 7.4 and this what you should use.

It has two major advantages over other graphics packages:
\begin{itemize}
\item It is an international standard, and thus portable.
\item No other package provides the ability to write such device independent
 graphics programs.
 Only with GKS can you write a program that will run, and produce good pictures,
 on all devices supported by the implementation, including devices not supported
 at the time the program was written.
 Nonetheless, programs can still fully exploit all the facilities offered by the
 hardware.
\end{itemize}
In addition, the Starlink implementation has several advantages of its own:
\begin{itemize}
\item It is likely to be well supported.

\item It is used on several thousand computers, of many different
types, throughout the UK academic community.  This means that the
software has received a large amount of testing under a wide variety of
conditions, and should be highly reliable.

\item It contains features which are not all found in any other single package:
\begin{itemize}
\item Area fill with several fill styles on \emph{all}\/ devices.
\item Cell array supported on dot matrix printers.
\item Multiple fonts.
\item Selective clearing of display surface.
\end{itemize}
\end{itemize}
The advantages of GKS also apply to graphics software which use it, such as
\htmlref{SGS}{item_SGS}.


\subsubsection{GNS \index{GNS} \hfill \xref{SUN/57}{sun57}{}}
\label{item_GNS}

\textbf{Support names for GKS graphics devices.}

Almost any graphics program requires you to identify a graphics device to use.
When it uses
\htmlref{GKS}{item_GKS},
they are identified by two integers:
a `workstation type' and a `connection identifier.'
(`Workstation' is GKS terminology for a graphics device of any description.)
You can't remember the workstation types of all Starlink's graphics devices,
so the \emph{Graphics Name Service}\/ has been provided to translate
`friendly' and easy-to-remember names into their GKS equivalents.

Most high level graphics packages, such as
\htmlref{SGS}{item_SGS} and
\htmlref{Starlink-PGPLOT}{item_STARLINK-PGPLOT},
call GNS to do the
required name translation when a workstation is opened, so unless you are
writing programs that open GKS workstations directly, or need to make
specialised device inquiries, you will not need to call GNS routines yourself.

It can:
\begin{itemize}
\item Translate workstation names to their GKS equivalents.
\item Generate a list of the names and types of available graphics devices.
\item Answer a variety of enquiries about the properties of a particular
graphics device; for example, its category (pen plotter, image display,
etc) or its VMS device name.
\end{itemize}
It is used internally in
\htmlref{IDI}{item_IDI} and
\htmlref{AGI}{item_AGI},
but this isn't normally apparent to the programmer.


\subsubsection{GRAPHPAR \index{GRAPHPAR} \hfill \xref{SUN/113}{sun113}{}}
\label{item_GRAPHPAR}

\textbf{Access the graphics libraries
\htmlref{GKS}{item_GKS},
\htmlref{SGS}{item_SGS}, and
\htmlref{Star\-link-PGPLOT}{item_STARLINK-PGPLOT}
from Starlink programs.}

These are the components of GKS, SGS and Starlink-Pgplot that provide the
interface to the respective libraries from ADAM programs.

\subsubsection{GWM \index{GWM} \hfill \xref{SUN/130}{sun130}{}}
\label{item_GWM}

\textbf{Create graphics windows on an X display that do not disappear when a
program terminates.}

X was designed to support GUIs and it has two properties that make an X window
fundamentally different from a traditional graphics device.
Firstly, windows `belong' to application programs and are deleted when the
application exits, unlike the picture on an image display which remains there
until replaced by something else.
Secondly, application programs are expected respond to `events' occurring in
their windows; things like key presses, mouse movements, etc.
One event type that all applications must handle is the `window expose' event
which occurs whenever part of a window that was invisible -- because another
application's window was obscuring it for example -- becomes visible.
The application is responsible for restoring the contents of the newly exposed
part of the window, but only an application designed as an X application can do
this; an application that is using X via a graphics package, such as
\htmlref{GKS}{item_GKS} or
\htmlref{IDI}{item_IDI},
but is otherwise a conventional application that knows nothing of X, cannot.

The X graphics window manager\footnote{Not to be confused with the Window
Manager which allows you to move windows around the screen, iconize them, etc.}
makes a window on an X display behave like a traditional graphics
device by making the lifetime of the window independent of any applications
program and by handling window expose events.
Applications still send plotting commands directly to the window; they don't
have to go via a `server' process, so there is no adverse impact on
performance.
All communication between the window manager and the application is via the X
server, and the graphics window manager does not have to run on the same
machine as the application.

\newpage

\subsubsection{IDI \index{IDI} \hfill \xref{SUN/65}{sun65}{}. MUD/28}
\label{item_IDI}

\textbf{An international standard for displaying astronomical data on an
image display.}

It is intended for programs that need to manipulate images to a greater extent
than can be done with
\htmlref{GKS}{item_GKS} and its offspring.
Thus, it does not supersede GKS, but offers features not available in GKS.
It is not as good as GKS for producing line plots or character annotation --
it does have routines to draw lines and plot text, but these are primitive and
offer you little control over how the result will appear in terms of character
size, style of line width, etc.

Its major strength is its ability to perform many types of interaction using the
mouse.
Like GKS, it can display an image, move the cursor, and rotate the look-up
table.
However, it can also scroll and zoom, blink the memories, and read back a
representation of the whole display, which can then be used to obtain hardcopy.
It allows these functions to be programmed in a device independent way, so
a program can use any device for which IDI has been implemented.

It is not possible, at present, to mix GKS and IDI calls on the same display
because the two packages use completely different display models.
However, it is possible to run the packages one after the other.
For example, IDI could follow GKS, and the display could be opened without
resetting it.
However, this is an undesirable approach since the results could be
unpredictable.
A better solution is to use
\htmlref{AGI}{item_AGI}
to mediate between them.

\subsubsection{NCAR \index{NCAR}\index{NCAR!SNX}\index{SNX} \hfill%
\xref{SUN/88}{sun88}{}. MUD/59}
\label{item_NCAR}

\textbf{High-level graphics utilities.}

This extensive suite was obtained from the
National Center for Atmospheric Research in Boulder, Colorado.
(An associated software item called
\htmlref{SNX}{item_SNX} contains some Starlink extensions to it.)

The routines are thoroughly documented and just a few simple calls will
produce graphs of excellent appearance.
They also very flexible, with dozens of plot details independently
controllable to give you exactly the result required.
The plots provided are:
\begin{itemize}
\item Annotated curves or families of curves.
\item Contours for 1-d and 2-d data, with labels.
\item Dashed lines with labels.
\item Continental, national, and US state boundaries in nine map projections.
\item Graph paper, axes, etc.
\item Halftone (greyscale) pictures from 2-d arrays.
\item Histograms.
\item Iso-value surfaces (with hidden lines removed) from 3-d arrays.
\item Software characters.
\item 3-d displays of a surface (with hidden lines removed) from 2-d arrays.
\item Vector flows of fields for which planar vector components are given on
 regular rectangular lattices.
\item Lines in three space.
\item 2-d velocity fields.
\end{itemize}


\subsubsection{Native-PGPLOT \index{PGPLOT!Native} \hfill %
\xref{SUN/15}{sun15}{}, \xref{113}{sun113}{}. MUD/61}
\label{item_NATIVE-PGPLOT}

\textbf{A high-level graphics package for plotting \emph{ x,y}\/ plots,
functions, histograms, bar charts, contour maps, and grey-scale
images.}

Complete diagrams can be produced with a minimal number of subroutine
calls, but control over colour, line-style, character font,
\emph{etc}.\ is available if required.  It was written with
astronomical applications in mind and has become a \textit{de facto}
standard for graphics in astronomy.

Native-PGPLOT is the original version from Tim Pearson at Caltech based
on the low-level graphics package known as GRPCKG.
\index{GRPCKG}\index{PGPLOT!GRPCKG} It has the same user interface as
\htmlref{Starlink-PGPLOT}{item_STARLINK-PGPLOT},  and non-ADAM
applications can be moved from one version to the other simply by
re-linking using the appropriate libraries.


\subsubsection{Starlink-PGPLOT\index{PGPLOT!Starlink} \hfill \xref{SUN/15}{sun15}{},%
\xref{113}{sun113}{}.MUD/61}

\label{item_STARLINK-PGPLOT}
\textbf{A high-level graphics package for plotting \emph{x,y}\/ plots,
functions, histograms, bar charts, contour maps, and grey-scale
images.}

Complete diagrams can be produced with a minimal number of subroutine calls,
but control over colour, line-style, character font, \emph{etc}.\ is
available if required.  It was written with astronomical applications in mind
and has become a \textit{de facto} standard for graphics in astronomy.

Starlink-PGPLOT is the version based on the Starlink
\htmlref{GKS}{item_GKS} library and has the same user interface as
\htmlref{Native-PGPLOT}{item_NATIVE-PGPLOT}.  Non-ADAM  applications can be
moved from one version to the other simply by re-linking with the appropriate
libraries.

This is the version supported by Starlink for its applications.


\subsubsection{SGS\index{SGS} \hfill\xref{SUN/85}{sun85}{},%
\xref{113}{sun113}{}}
\label{item_SGS}

\textbf{A friendly interface to a subset of GKS.}

SGS stands for `Simple Graphics System.'

Many of its features are low-level (e.g.\ draw a line), but there are some
routines of a slightly higher level (drawing arcs, formatted numbers, etc).
It does not include routines for high-level operations like drawing annotated
axes or complete graphs.
Many plotting programs can be written entirely with SGS or higher-level calls.
There will, however, be occasions when GKS routines are used as well; usually
because a specialised feature of GKS is needed.

It can:
\begin{itemize}
\item Control workstations and zones.
\item Plot lines, text, and markers.
\item Control character attributes.
\item Accept graphical input.
\item Do GKS inquires.
\end{itemize}


\subsubsection{SNX \index{SNX} \hfill \xref{SUN/90}{sun90}{}}
\label{item_SNX}
\textbf{Starlink extensions to
\htmlref{NCAR}{item_NCAR}.}

They provide more convenient access to certain features of NCAR without
sacrificing flexibility.
(Beginners may be daunted by the mass of features offered by NCAR, and unless
they take the extreme step of reading the manual they may give up before
realising what the package can do for them.)

The AUTOGRAPH part of NCAR, used in conjunction with SNX and
\htmlref{SGS}{item_SGS}, offers an alternative to
\htmlref{Starlink-PGPLOT}{item_STARLINK-PGPLOT} for graphing one
variable against another.  SNX enhances the power of AUTOGRAPH and
makes it easier for the beginner.

It can:
\begin{itemize}
\item Open AUTOGRAPH/SGS/GKS with a single call.
\item Plot a finished graph, with annotation, by means of a single call.
\item Include special characters in graph labels (Greek, subscripts,
mathematical symbols, etc.).
\item Plot character strings anywhere on the graph.
\item Manage the graphics device display surface via the SGS system of
zones while using AUTOGRAPH to do the plotting.
\item Align the AUTOGRAPH and SGS coordinate systems.
\item Save and restore the state of AUTOGRAPH so that multiple active
plots can be managed.
\item Perform cursor input.
\end{itemize}




\newpage

\subsection{Other}

\rule{\textwidth}{0.5mm}

\begin{description}
\item [\htmlref{CHR}{item_CHR}]
--- Character handling
\item [\htmlref{CNF}{item_CNF}]
--- Mixed C and Fortran programming
\item [\htmlref{EMS}{item_EMS}]
--- Error message service
\item [\htmlref{HLP}{item_HLP}]
--- Interactive help system
\item [\htmlref{KAPLIBS}{item_KAPLIBS}]
--- General purpose KAPPA routines
\item [\htmlref{MERS}{item_MERS}]
--- Message and error reporting
\item [\htmlref{ONE}{item_ONE}]
--- Odds and Ends library
\item [\htmlref{PCS}{item_PCS}]
--- Parameters and communications
\item [\htmlref{PSX}{item_PSX}]
--- POSIX interface
\end{description}

\rule{\textwidth}{0.5mm}



\subsubsection{CHR \index{CHR} \hfill \xref{SUN/40}{sun40}{}}
\label{item_CHR}
\textbf{Augment the limited character handling facilities of Fortran 77.}

Converts Fortran data types into text strings and the reverse,
does wild card matching, string sorting, paragraph reformatting and
justification.
It can be used for building text strings for interactive applications, or as a
basis for more complex text processing applications.

It can:
\begin{itemize}
\item Change case.
\item Compare and edit strings.
\item Encode and decode Fortran data types.
\item Make enquires about character strings.
\item Enhance portability.
\item Search for strings.
\end{itemize}


\subsubsection{CNF \index{CNF} \hfill\xref{SUN/209}{sun209}{}}
\label{item_CNF}
\textbf{Mix program segments written in Fortran and C in a portable manner.}

It provides:
\begin{itemize}
\item C macros to hide the different ways that computers pass information
between subprograms.
\item C functions to handle Fortran character strings and logical values.
\end{itemize}


\subsubsection{EMS \index{EMS} \hfill \xref{SSN/4}{ssn4}{}}
\label{item_EMS}
\textbf{Construct and store error messages for future delivery to a user via ERR.}

ERR is the Starlink Error Reporting System, which is part of
\htmlref{MERS}{item_MERS}.
It can be regarded as a simplified version of ERR without the binding to
any software environment (e.g.\ for message output or access to the
parameter and data systems).
It conforms to the error reporting conventions of MERS.\index{MERS!EMS}


\subsubsection{HLP \index{HLP}\index{HELP|see{HLP}} \hfill \xref{SUN/124}{sun124}{}}
\label{item_HLP}
\textbf{Retrieve named items from a hierarchical library of text.}

Functionally, it is very similar to the VAX/VMS Help system.
Major differences:
\begin{itemize}
\item Implemented in a portable way and not tied to the VAX.
\item Allows independent creation of multiple libraries which are bound
together at run-time and appear to the user as a single tree.
\end{itemize}
It is free-standing and does not call any other Starlink packages.


\subsubsection{KAPLIBS \index{KAPLIBS} \hfill \xref{SUN/238}{sun238}{}}
\label{item_KAPLIBS}
\textbf{General purpose KAPPA libraries}

KAPLIBS is a new package containing several Fortran subroutine
libraries that were previously distributed as part of the KAPPA package. These
libraries contain general purpose subroutines which may be of use to
anyone developing their own KAPPA-like applications.
\xref{SUN/238}{sun238}{} contains
brief documentation, but users of KAPLIBS will probably find it
useful to examine the source code of relevant KAPPA commands.


\subsubsection{MERS\index{MERS} \hfill \xref{SUN/104}{sun104}{}}
\label{item_MERS}
\textbf{Report messages and errors.}

There is a general need for programs to inform the user about:

\begin {itemize}
\item What they do -- for example, what is going on.
\item What results have been obtained -- final, or intermediate if this would
be helpful.
\item What errors have occurred -- why the program crashed, or to ask for more
sensible parameter values.
\end {itemize}

MERS comprises two libraries to help you do this:
\begin {itemize}
\item \textbf{MSG}\index{MSG}\index{MERS!MSG}
-- Message Reporting System, used for reporting non-error information.
\item \textbf{ERR}\index{ERR}\index{MERS!ERR}
-- Error Reporting System, used specifically for reporting error messages.
\end {itemize}

\newpage

\subsubsection{ONE\index{ONE} \hfill\xref{SUN/234}{sun234}{}}
\label{item_ONE}
\textbf{Odds and Ends library}

The ONE library is a small set of Fortran and C routines of a general nature
and usefulness but which are not appropriate to be included in any of the
more focussed Starlink libraries.

The routines are written in C but are callable direct from Fortran.


\subsubsection{PCS \index{PCS} \hfill\xref{SSN/29}{ssn29}{}}
\label{item_PCS}

\textbf{Parameter and communication system.}

Application programs often need to obtain parameter values from a variety of
sources and to communicate with other programs.
The Parameter and Communication Subsystems (PCS) are a set of closely-related
subroutine libraries which provide these facilities for many Starlink
applications and associated user-interfaces.

These libraries will not generally be called directly by application programs,
but form a basic part of
\htmlref{ADAM}{item_ADAM},
the Starlink Software Environment.


\subsubsection{PSX \index{PSX} \hfill \xref{SUN/121}{sun121}{}}
\label{item_PSX}
\textbf{Use the POSIX and X/OPEN libraries.}

Lets you use operating system facilities in a machine independent way.




\newpage

\section{Infrastructure}

\rule{\textwidth}{0.5mm}

\begin{description}
\item [\htmlref{ADAM}{item_ADAM}]
--- Starlink software environment
\item [\htmlref{EXPECT}{item_EXPECT}]
--- Interactive program control
\item [\htmlref{ICL}{item_ICL}]
--- Command language
\item [\htmlref{INIT}{item_INIT}]
--- Starlink initialisation files
\item [\htmlref{SAE}{item_SAE}]
--- Starlink applications environment base files
\item [\htmlref{STARTCL}{item_STARTCL}]
--- Starlink additions to TCL
\item [\htmlref{TCL}{item_TCL}]
--- Embeddable tool command language
\item [\htmlref{TK}{item_TK}]
--- X window toolkit for TCL
\end{description}

\rule{\textwidth}{0.5mm}

\subsubsection{ADAM
\index{ADAM} \hfill
\xref{SG/4}{sg4}{},
\xref{6}{sg6}{}.
\xref{SUN/101}{sun101}{},
\xref{113}{sun113}{},
\xref{115}{sun115}{},
\xref{134}{sun134}{},
\xref{144}{sun144}{}.
\xref{SGP/45}{sgp45}{}.
\xref{SSN/64}{ssn64}{}}
\label{item_ADAM}

\textbf{Starlink's main software environment.}

It has many aspects which, taken together, provide users with a
coordinated environment for analysing data and writing programs.  ADAM
software also controls many telescopes and instruments at
observatories.

In the minds of users doing data analysis, ADAM is usually identified
with its application programs, such as \htmlref{KAPPA}{item_KAPPA} and
\htmlref{CCDPACK}{item_CCDPACK}.  However, it is really just the kernel
of the system, in other words, the subroutine libraries.  Many of the
libraries associated with ADAM are also available as stand-alone
libraries, thereby making it easier to take code to non-Starlink
computers.  Some of the libraries used in ADAM programs are Starlink
libraries that predate ADAM, while others are commercial packages.  In
fact, it is impossible to have a perfect definition of what is or is
not an ADAM program.  The best one seems to be `an ADAM program is one
that uses the ADAM parameter system to interact with the user.' The
fact that different groups of users have different ideas of what
constitutes an ADAM program is not a failing of ADAM; it demonstrates
the different ways in which it can be used.

As mentioned above, ADAM applications are really distinct from the ADAM
kernel.  However, without application programs, ADAM would only be half
a system.  Most of the packages described in this note are ADAM
programs.  There are also software tools to aid in writing ADAM
programs, and data analysis tools to browse data files.

One of ADAM's most important features is that it provides a standard
method of data storage called \htmlref{NDF}{item_NDF} (extensible $n$-d
data format).  NDF files have standard components that all programs
recognise, while also allowing for any extensions that may become
necessary for particular applications.  Use of NDF means that data
files can be exchanged between different packages without the need to
convert file formats.  This is very useful, as having to change file
formats is very annoying and takes up extra disk space. NDF is layered
on \htmlref{HDS}{item_HDS}, the hierarchical data system.  By using HDS
(version 4.0 or later), it becomes possible to read and write data
files on different computer systems completely transparently, so no
explicit data conversion needs be done, even when moving from one
computer to another.

ADAM programs can generally be run either from the command language of
the host, or from the \htmlref{ICL}{item_ICL}\index{ICL} command
language.  ICL is a simple programming language in itself, somewhat
along the lines of BASIC.  It can be quite useful for doing simple
operations, but is not fast enough for any major processing.  Use of
ICL makes it easier to build sequences of operations into ICL
procedures.  It is generally quicker to run ADAM programs from ICL if
they are to be run several times, since the programs only have to be
activated once.  On the other hand, the ability to run ADAM programs
from the shell is very useful for debugging.

One of the most distinctive features of ADAM is the parameter system.
This provides a method whereby a program can interact with a user, or
with other programs.  For a single program, a series of write and read
statements could sometimes be used instead, but for anything more
complex this becomes unacceptable.  It is particularly difficult to
maintain a standard style, and \emph{ad hoc}\/ fixes to problems
abound.  Use of the parameter system provides a consistent user
interface, a means of storing the value of a parameter between
invocations of a program, and the facility to pass the values of
parameters from one program to another.

From the perspective of a programmer, ADAM provides many subroutine
libraries that perform common functions.  They provide such diverse
functions as character handling, access to hierarchical, relational,
and simple data, graphics, the parameter system, error and message
reporting, numerical calculations, etc.  Some of these libraries are
provided purely for use in ADAM programs, whereas others, such as the
NAG library and graphics libraries, have a much wider use.

Although any subroutine library can be used in an ADAM program, those
that have been written specifically with this in mind are much easier
to use.  ADAM libraries conform to the Starlink standard on the use of
inherited status, which greatly reduces the amount of code needed to
test for error conditions.  They also report any error conditions via
the error message service.  This lets the programmer get at the error
messages as well the error code, to see if the error can be handled
internally or whether to send the error report on to the user.

By using the facilities provided in a consistent manner, simple
programs can be easily linked together to perform complex operations.
In particular, programs from packages written to do completely
different things can process each other's data, thereby avoiding
unnecessary duplication of software, with the attendant waste of
effort, both in writing the program and in maintaining it.

Finally, it should not be forgotten that ADAM originated as a telescope
and instrument control system.  The way that ADAM programs are written
when used in these circumstances is rather different from when they are
used for data analysis.  Use of ADAM for instrument control is not the
province of this document.  However, it is interesting to note that
programs written for data acquisition tasks are rather like windows
programs, which have become common.


\subsubsection{EXPECT\index{EXPECT} \hfill \xref{SUN/200}{sun200}{}}
\label{item_EXPECT}

\textbf{Emulate a user sitting at a terminal.}

When combined with
\htmlref{TK}{item_TK}, it can put a GUI in front of an existing application
without modifying it.



\subsubsection{ICL \index{ICL} \hfill \xref{SG/4}{sg4}{},\xref{5}{sg5}{}}
\label{item_ICL}

\textbf{A programmable user interface to an astronomical data reduction or data
acquisition system.}

It is one of the main user interfaces for the ADAM software environment.
In some ways it is similar to a high level programming language such as
Fortran or Pascal, but it has some important differences:
\begin{itemize}
\item It is a \emph{command}\/ language.
One of its main uses is to let you type commands with few restrictions
on the format.
\item It is an \emph{interactive}\/ language.
It provides a complete environment for entering, editing, and debugging
programs, rather than relying on external editors, linkers, etc.
\item It is a \emph{programming language}\/ suitable for relatively simple and
straightforward programs.
\end{itemize}


\subsubsection{INIT \index{INIT}\hfill \mbox{}}
\label{item_INIT}

\textbf{A grouping of the Starlink login resource files and other associated
bits and pieces.}

The \texttt{.cshrc} and \texttt{.login} files are now handled like standard packages
and are edited by the install procedure to contain the correct root path for the
USSC set.
This allows easy installation on systems where the directory \texttt{/star}
does not exist.


\subsubsection{SAE  \hfill\xref{SUN/191}{sun191}{}}
\index{SAE}
\label{item_SAE}
\textbf{Starlink application environment.}

INIT gathers together the global \texttt{include} and \texttt{error}
files needed by the infrastructure and applications.  It is purely an
administrative grouping.



\subsubsection{STARTCL \hfill
\xref{SUN/186}{sun186}{}}
\index{STARTCL}
\label{item_STARTCL}
\textbf{Extend TCL and TK.}

A public domain utility called
\htmlref{TCL}{item_TCL}/\htmlref{TK}{item_TK}
is in wide use as a command language and user interface builder.
STARTCL comprises three extensions to TCL and the TK Toolkit that enable
TCL/TK applications to cooperate with Starlink applications.
These are:
\begin{itemize}
\item A TK widget that displays Starlink graphics by emulating the \textbf{gwm}
graphics window server and a \textbf{gwm} canvas item.
\item An interface to the
\htmlref{ADAM}{item_ADAM}
message system.
\item An interface to the ADAM notice board system.
\end{itemize}


\subsubsection{TCL  \hfill
\xref{SUN/186}{sun186}{},
\xref{200}{sun200}{}. MUD/164}
\index{TCL}
\label{item_TCL}
\textbf{Control and extend applications using a scripting language.}

TCL stands for \emph{Tool Command Language}.


\subsubsection{TK  \hfill
\xref{SUN/186}{sun186}{},
\xref{200}{sun200}{}.}
\index{TK}
MUD/164
\label{item_TK}
\textbf{Build GUIs.}

An X window toolkit for use with
\htmlref{TCL}{item_TCL} and
\htmlref{EXPECT}{item_EXPECT}.
It can build Motif-like user interfaces using TCL scripts instead of C code.




\newpage

\section{Acknowledgements}

The majority of the text for this document was plundered from existing
Starlink documents by Mike Lawden who compiled and edited editions up to
1997.

The current form of this document was developed by Martin Bly who was
editor from 1997 to 2002.

Most of the raw material came from the introductory sections of
appropriate Starlink User Notes, but material was also grabbed from
other sources, such as Starlink Bulletin articles and Starlink Panel
papers, where these provide extra insight into the significance and
functions of a software item.

As this note covers \emph{all}\/ Starlink's software, I acknowledge and
thank \emph{all}\/ the authors of Starlink documents for providing the
source material.  Thanks particularly to Mike Lawden for his work on
pervious editions.

\newpage

\printindex

% ? End of main text
\end{document}
