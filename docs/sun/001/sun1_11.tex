\documentstyle[11pt]{article} 
\pagestyle{myheadings}
\makeindex

%------------------------------------------------------------------------------
\newcommand{\stardoccategory}  {Starlink User Note}
\newcommand{\stardocinitials}  {SUN}
\newcommand{\stardocnumber}    {1.11}
\newcommand{\stardocauthors}   {M D Lawden}
\newcommand{\stardocdate}      {25 September 1991}
\newcommand{\stardoctitle}     {The Starlink Software Collection}
%------------------------------------------------------------------------------

\newcommand{\stardocname}{\stardocinitials /\stardocnumber}
\renewcommand{\_}{{\tt\char'137}}     % re-centres the underscore
\markright{\stardocname}
\setlength{\textwidth}{160mm}
\setlength{\textheight}{230mm}
\setlength{\topmargin}{-2mm}
\setlength{\oddsidemargin}{0mm}
\setlength{\evensidemargin}{0mm}
\setlength{\parindent}{0mm}
\setlength{\parskip}{\medskipamount}
\setlength{\unitlength}{1mm}

\begin{document}
\thispagestyle{empty}
SCIENCE \& ENGINEERING RESEARCH COUNCIL \hfill \stardocname\\
RUTHERFORD APPLETON LABORATORY\\
{\large\bf Starlink Project\\}
{\large\bf \stardoccategory\ \stardocnumber}
\begin{flushright}
\stardocauthors\\
\stardocdate
\end{flushright}
\vspace{-4mm}
\rule{\textwidth}{0.5mm}
\vspace{5mm}
\begin{center}
{\LARGE\bf \stardoctitle}
\end{center}
\vspace{5mm}

\begin{abstract}
The Starlink Software Collection is a set of software which is managed and
distributed by the Starlink Project.
This paper describes the functions of the individual items in the Collection,
and provides an overview of the software so that readers can identify the items
that they need.

The paper begins by considering Software Environments.
These provide facilities which enable you to use the computer more easily.
Users are provided with a language with which to control programs and specify
their parameters.
Programmers are provided with subroutines to access data and parameter values,
handle errors, control graphics, and so on.

There are four main sets of applications.
ADAM Applications run under the ADAM environment and can therefore be used
together and have compatible data formats.
Stand-alone Applications are independent applications that run within their
own environment.
In general they cannot be used together and may have incompatible data formats.
INTERIM Applications run under the original Starlink environment which has been
superseded by ADAM.
Foreign Applications have been acquired from non-Starlink organisations and are
distributed and supported from the Starlink sites where they are used.
They are not distributed or supported from the central node at RAL.

In addition to the major applications, there are programs called Utilities
which do specific tasks that facilitate the use of Starlink computers.
Astronomical Utilities perform tasks, such as data copying, format conversion,
or preparing for an observation, which are closely associated with astronomy.
General Utilities assist you in using a computer, but have no particular
connection with astronomy.
They deal with such things as graphics and programming support.

Finally, there are Subroutine Libraries.
These are for programmers writing astronomical software.
They provide facilities such as data management and graphics.

Each item is described in sufficient detail for the reader to decide whether or
not it is worth investigating further.
Where several programs are described together, the most important or most widely
used are described first.
Most items have a Starlink User Note associated with them that either describes
how to use the software or points to a user manual for that item.

The level of support provided for different items varies widely.
Some items are essentially unsupported and you would be unwise to base any
long term work on them.
The support level is indicated for each item.
\end{abstract}

\newpage

\setlength{\parskip}{0mm}
\tableofcontents
\setlength{\parskip}{\medskipamount}
\markright{\stardocname}

\newpage

\section {Introduction}

The {\em Starlink Software Collection} contains a wide variety of items ranging
from single programs to elaborate and powerful self-contained data reduction
and analysis systems.
The items have been collected from many sources and differ enormously
in their standard of programming and documentation.
Some items may not be installed at every Starlink site; these make up the
{\em Option Set}.
This note gives a broad overview of what software is available, what it does,
and where it is documented, in order to narrow the field of search for the
answer to your particular needs.
If an item you want to use is not installed at your site, ask your Site
Manager to install it for you.

Nearly every item of Starlink Software is described in a Starlink User Note
(SUN).
The level of description ranges from a brief outline, with references to other
documents, to a complete manual.
Most SUNs are produced using the \LaTeX\ document preparation system.
You should get paper copies of them from your Site Manager rather
than trying to study them on-line or printing your own versions.
The text of all SUNs is stored in a directory with the logical name DOCSDIR
and have file names of the form SUNnn, where `nn' is the number of the note.
Many items have supplementary paper documentation which is indexed in file
DOCSDIR:MUD.LIS.
A complete list of all Starlink Notes, including those issued since this note
was last revised, can be found in file DOCSDIR:DOCS.LIS.
Also, file DOCSDIR:ANALYSIS.LIS shows in a compact form the notes associated
with every item, and file DOCSDIR:NEWS.LIS shows the most recent software and
document releases.

The level of support available varies greatly from item to item.
Some items, such as ADAM, have high levels of support from within the Project;
others, such as CHART, have little or no support.
If you are thinking of using a particular item, please check the current
support level in file ADMINDIR:SUPPORT.LIS.
The levels of support at the time of writing are shown in Appendix A, and are
also shown at the right hand side of the list at the start of every
subsection, for example by the mark $\surd\surd$ in:
\begin{description}
\begin{description}
\item [MAPLE] --- Mathematical manipulation language (STADAT only) \hfill 
 $\surd\surd$
\end{description}
\end{description}
You should regard items with a support level indicated by crosses
(`$\times\times$' or `$\times$') as unreliable (the meanings of the support
marks are specified in Appendix A).
You would be unwise to base your work on such items.
In brief: the more $\surd$'s the better, and avoid items with $\times$'s.

The items are described in groups based on the following classification:
\begin{itemize}
\item Software environments.
\item Applications (ADAM, Stand-alone, INTERIM, Foreign).
\item Utilities (Astronomical, General).
\item Subroutine libraries.
\end{itemize}
A complete list of items, based on this classification, is given at the end.
Where several programs are described together in the same section, the most
important or most widely used are described first.

\newpage

\section {Software Environments}

Many Starlink software items run within some kind of standard framework called
a `software environment'.
Several such environments exist; each is an integrated collection of:
\begin{itemize}
\item Interfaces (between application programs and, respectively, the user and
the computer),
\item Conventions (programming and data storage),
\item Software tools,
\end{itemize}
intended to improve the quality and cost-effectiveness of application programs
and the productivity of programmers by:
\begin{itemize}
\item Providing the {\bf user} with a command language and other facilities
which will allow him to run application programs, one at a time or as a
sequence, easily, efficiently, and in a flexible manner matched to his own
requirements.
\item Providing the {\bf programmer} with standardised and easy to use
facilities, covering areas such as the user interface, access to data, graphics,
error reporting, and so on.
This saves endless duplication, and means the astronomer gets more application
software from a given amount of programming effort.
\item Eliminating dependencies on one specific {\bf computer} type or
configuration, thereby allowing the application to run on a wide and evolving
range of computers.
This is essential to allow astronomers to exploit new generations of
hardware.
\end{itemize}
In practice, a software environment is seen by the programmer mainly as a
set of subroutine libraries which handle such tasks as obtaining parameters,
accessing data from mass storage, etc, and is perceived by the user as a
consistent command language and as a means of storing data in a format which is
independent of any single application.

Some data reduction systems (AIPS, IRAF, MIDAS, IDL, DIPSO, FIGARO) incorporate
their own environments.
Others (ADAM, INTERIM) are general purpose environments.

\rule{\textwidth}{0.5mm}
\begin{description}
\begin{description}
\item [ADAM] --- Standard Starlink software environment \hfill $\surd\surd\surd$
\item [INTERIM/DSCL/RUNSTAR] --- Interim Starlink software environment \hfill
 $\times\times$
\end{description}
\end{description}
\rule{\textwidth}{0.5mm}

\begin{description}

\item [ADAM]\index{ADAM} \hfill [SUN/35, 94, 101, 115]; [SG/4]

This is Starlink's preferred software environment.
It has many aspects which, taken together, provide users with a coordinated
environment for analysing data and writing programs.
ADAM software also controls many of the telescopes and instruments at
observatories.

In the minds of users doing data analysis, ADAM is usually identified with its
application programs such as KAPPA and ASTERIX.
However, it is really just the kernel of the system, in other words, the
subroutine libraries.
Many of the libraries that are associated with ADAM are also available as
stand alone libraries, thereby making it easier to take code to non-Starlink
computers.
Some of the libraries that are commonly used in ADAM programs are Starlink
libraries that predate ADAM, while others are commercial packages.
In fact, it is impossible to have a perfect definition of what is and is not an
ADAM program.
The best one seems to be `an ADAM program is one that uses the ADAM parameter
system to interact with the user'.
The fact that different groups of users have different ideas of what
constitutes an ADAM program is not a failing of ADAM; it demonstrates the
different ways in which it can be used.

As mentioned above, ADAM applications are really distinct from the ADAM kernel.
However, without application programs, ADAM would only be half a system.
At present, there are applications that deal with image processing, image
analysis, spectroscopy, CCD stellar photometry and aperture photometry,
time series and polarimetry, catalogue access, X-ray data and infrared camera
data.
There are also software tools to aid in writing ADAM programs, and data analysis
tools to browse data files.
Packages dealing with CCD data reduction and interactive graphics are expected
to be released shortly, and an IRAS package is under development.

One of ADAM's most important features is that it provides a standard method of
data storage called NDF (extensible N-dimensional data format).
NDF files have standard components that all programs recognise, while also
allowing for any extensions that may become necessary for particular
applications.
Use of NDF by ADAM programs means that data files can be exchanged between
different packages without the need to convert file formats.
This is very useful as having to change file formats is very annoying and takes
up extra disk space. NDF is layered on HDS, the hierarchical data system.
By using HDS (version 4.0 or later), it becomes possible to read and write data
files on different computer systems (VAX/VMS, Sun Sparcstation or DECstation)
completely transparently, so no explicit data conversion needs to be done, even
when moving from one computer to another.

ADAM programs can generally be run either from the command language of the host
computer (DCL in the case of VMS, one of the shells in the case of Unix) or
from the ICL\index{ICL} command language.
ICL is a simple programming language in itself, somewhat along the lines of
BASIC.
It can be quite useful for doing simple operations, but is not fast enough for
any major processing.
Use of ICL makes it easier to build sequences of operations into ICL procedures.
It is generally quicker to run ADAM programs from ICL if they are to be run
several times since the programs only have to be activated once.
On the other hand, the ability to run ADAM programs from DCL (or the shell on
Unix systems) is very useful for debugging.

One of the most distinctive features of ADAM programs is the parameter system.
It provides a method whereby a program can interact with a user or with other
programs.
For a single program, a series of write and read statements could sometimes be
used instead, but for anything more complex, this becomes unacceptable.
It is particularly difficult to maintain a standard style, and ad hoc fixes to
problems abound.
Use of the parameter system provides a consistent user interface, a means of
storing the value of a parameter between invocations of a program, and the
facility to pass the values of parameters from one program to another.

From the perspective of a programmer, ADAM provides many subroutine libraries
that perform common functions.
They provide such diverse functions as character handling, access to
hierarchical, relational and simple data, graphics, the parameter system, error
and message reporting, numerical calculations, etc.
Some of these libraries are provided purely for use in ADAM programs, whereas
others, such as the NAG library and graphics libraries, have a much wider use.

Although any subroutine library can be used within an ADAM program, those that
have been written specifically with this in mind are much easier to use.
ADAM libraries conform to the Starlink standard on the use of inherited status,
which greatly reduces the amount of code needed to test for error conditions.
They also report any error conditions via the error message service.
This lets the programmer get at the text of any error messages as well as a
numerical error code, to see if the error can be handled internally or whether
to send the error report on to the user.

By using the facilites provided in a consistent manner, simple ADAM programs
can easily be linked together to perform complex operations.
In particular, programs from packages written to do completely different things
can process each other's data, thereby avoiding unnecessary duplication of
software, with the attendant waste of effort, both in writing the program and
in maintaining it.

Finally it should not be forgotten that ADAM has its origins as a telescope and
instrument control system.
The way that ADAM programs are written when used in these circumstances is
rather different to when they are used for data analysis.
Use of ADAM for instrument control is not the province of this document.
However, it is interesting to note that programs written for data acquisition
tasks are rather like windows programs, which are becoming much more common
nowadays.

\item [INTERIM/DSCL/RUNSTAR]\index{INTERIM}\index{DSCL}\index{RUNSTAR}
 \hfill [SUN/4, 74]

INTERIM, with its command languages DSCL and RUNSTAR, comprised the first
Starlink software environment.
It creates and accesses data stored in a special format called `Bulk Data
Frame' (BDF).
ASPIC was the major collection of applications based on it.
It is now regarded as obsolete as it has been replaced by ADAM.
However, it will survive for many years because of the large amount of
software that was written to use it, though support will be `best efforts'
only.

\end{description}

\newpage

\section {ADAM Applications}

All these applications run within the ADAM environment.
They can be used together and use compatible data formats.

\subsection{Image analysis \& photometry}

\rule{\textwidth}{0.5mm}
\begin{description}
\begin{description}
\item [KAPPA] --- Kernel applications \hfill $\surd\surd\surd$
\item [DAOPHOT] --- Stellar photometry \hfill $\surd\surd$
\item [PHOTOM] --- Aperture photometry \hfill $\surd\surd$
\item [PISA] --- Object finding and analysis \hfill $\surd$
\end{description}
\end{description}
\rule{\textwidth}{0.5mm}

\begin{description}

\item [KAPPA]\index{KAPPA} \hfill [SUN/95]

The Kernel Application Package runs under ADAM, using the NDF data format, and
provides {\em general-purpose applications}.
It is the backbone of the software reorganisation around the ADAM environment,
and its applications integrate with other packages such as PHOTOM, PISA, and
FIGARO.
It is usable as a monolith from the ADAM command language ICL, or as individual
applications from DCL.

It handles bad pixels, and processes quality and variance information within
NDF data files.
Although oriented to image processing, many applications will work on NDFs of
arbitrary dimension.
Its graphics are device independent.
Currently, KAPPA has about 140 commands and provides the following facilities
for data processing:
\begin{itemize}
\item Generation of NDFs and ASCII tables by up-to-date FITS readers.
\item Generation of test data, and NDF creation from ASCII files.
\item Setting NDF components.
\item Arithmetic, including a powerful application that handles expressions.
\item Editing pixels and regions, including polygons and circles, and
 re-flagging bad pixels by value or by median filtering.
\item Configuration changing: flip, rotate, shift, subset, dimensionality.
\item Image mosaicing; normalisation of NDF pairs.
\item Compression and expansion of images.
\item Filtering: box, Gaussian, and median smoothing; very efficient Fourier
 transform, maximum-entropy deconvolution.
\item Surface fitting.
\item Statistics, including ordered statistics, histogram, pixel-by-pixel
 statistics over a sequence of images.
\item Inspection of image values.
\item Centroiding of features, particularly stars; stellar PSF fitting.
\item Detail enhancement via histogram equalisation, Laplacian convolution,
 edge enhancement via a shadow effect, thresholding.
\end{itemize}

There are also many applications for data visualisation:
\begin{itemize}
\item Use of the graphics database, AGI, to pass information about pictures
 between applications.
 Facilities for the creation, labelling and selection of pictures, and
 obtaining world and data co-ordinate information from them.
\item Image and greyscale plots with a selection of scaling modes and many
 options such as axes.
\item Creation, selection, saving and manipulation of colour tables and 
 palettes (for axes, annotation, coloured markers and borders).
\item Snapshot of an image display to hardcopy.
\item Blinking and visibility of image-display planes.
\item Line graphics: contouring, including overlay; columnar and hidden-line
 plots of images; histogram; line plots of 1-d arrays, and multiple-line plots
 of images; slices through an image.
 There is some control of the appearance of plots.
\end{itemize}

\item [DAOPHOT]\index{DAOPHOT} \hfill [SUN/42]

A {\em stellar photometry} package written by Peter Stetson at the Dominion
Astrophysical Observatory, Victoria, B.C., Canada.
It performs the following tasks:
\begin{itemize}
\item Finding objects.
\item Aperture photometry.
\item Obtaining the point spread function.
\item Profile-fitting photometry.
\end{itemize}
Profile fitting in crowded regions is performed iteratively, which improves the
accuracy of the photometry.
It does not directly use an image display (which aids portability), although
three additional routines allow results to be displayed on an image device.
It uses image data in NDF format which is ADAM compatible.

\item [PHOTOM]\index{PHOTOM} \hfill [SUN/45]

Performs aperture photometry.
It has two basic modes of operation:
\begin{itemize}
\item Using an interactive display to specify the positions for the
 measurements.
\item Obtaining those positions from a file.
\end{itemize}
The aperture is circular or elliptical, and the size and shape can be varied
interactively on the display, or by entering values from the keyboard or
parameter system.
The background sky level can be sampled interactively by manually positioning
the aperture, or automatically from an annulus surrounding the object.

\item [PISA]\index{PISA} \hfill [SUN/109]

The Position, Intensity and Shape Analysis package locates and parameterises
objects in an image frame.
The core of the package is a routine which performs image analysis on a
2-dimensional data frame.
It searches for objects having a minimum number of connected pixels above a
given threshold, and extracts the image parameters (position, intensity, shape)
for each object.
The parameters can be determined using thresholding techniques, or an analytical
stellar profile can be used to fit the objects.
In crowded regions, deblending of overlapping sources can be performed.

The package derives from the APM IMAGES routine originally written by Mike
Irwin at the University of Cambridge to analyse output from the Automated
Photographic Measuring system.

\end{description}

\newpage

\subsection{Spectroscopy}

\rule{\textwidth}{0.5mm}
\begin{description}
\begin{description}
\item [FIGARO] --- General data analysis \hfill
 $\surd\surd$
\end{description}
\end{description}
\rule{\textwidth}{0.5mm}

\begin{description}

\item [FIGARO]\index{FIGARO} \hfill [SUN/86]

A {\em general data reduction} system written by Keith Shortridge at Caltech
and the AAO.
Although many people find it of greatest use in the reduction of
spectroscopic data, it also has powerful image and data cube manipulation
facilities and a photometry package.
In many ways it is a descendant of SPICA, possessing virtually all SPICA's
reduction facilities, but its different methods of operation and data storage
offer significant improvements in features and performance.
Starlink recommends FIGARO as the most complete spectroscopic data reduction
system in the Collection.
Examples of its facilities are:
\begin{itemize}
\item Analyse absorption lines interactively.
\item Aperture photometry.
\item Calibrate B stars.
\item Calibrate flat fields.
\item Calibrate using flux calibration standards.
\item Calibrate wavelengths of spectra.
\item Correct S-distortion.
\item Extract spectra from images and images from data cubes, and insert
 spectra into images and images into data cubes.
\item Extract spectra from images taken using optical fibres.
\item Fit Gaussians to lines in a spectrum interactively.
\item Generate and apply a spectrum of extinction coefficients.
\item Input, output, and display data.
\item Look at the contents of data arrays, other than graphically.
\item Manipulate complex data structures (mainly connected with Fourier
 transforms).
\item Manipulate data arrays `by hand'.
\item Manipulate images and spectra (arithmetic and more complicated).
\item Process data taken using FIGS (the AAO's Fabry-Perot Infra-Red Grating
Spectrometer).
\item Process echelle data --- in particular the UCL echelle in use at the AAO.
\end{itemize}
At present, a number of related packages are bundled with FIGARO.
In future, these may be released as separate items.
They include:
\begin{description}
\item [NDPROGS]\index{NDPROGS} \hfill [SUN/19, 27]

This manipulates and displays images of up to six dimensions.
Primarily, it is for manipulating TAURUS spectral line data cubes, which are
3-D images in which two of the dimensions are spatial and the third is spectral.
However, it contains no instrument-specific features and can therefore be used
to analyse similar data. 

The term `image' here simply means a regular data array, which might be
anything from a 1-D spectrum or profile to a 4-D array consisting of several
long-slit spectra with polarization vectors.
The upper limit of six dimensions is imposed by the software which interfaces
with HDS and has no other significance. 

\item [NFIGARO]\index{NFIGARO} \mbox{}

FIGARO is distributed all over the world.
Arrangements have been made for local modifications and enhancements to be
installed in a separate directory.
NFIGARO is the UK national (Starlink) implementation of this arrangement.

\item [TAUCAL]\index{TAUCAL} \mbox{}

This carries out phase calibration of the Taurus instrument on the William
Herschel Telescope.

The principal advantage of Taurus compared to a slit spectrograph is that
it is possible to obtain spectra simultaneously at every point covered by the 
detector by taking a series of images, each with a different value of the
etalon gap.
These images are stacked together to form a data cube.

The problem with such a cube is that the transmitted wavelength is not just a
function of etalon gap, but also of position on the detector.
Thus, the raw Taurus cube does not have separable spatial and spectral axes,
as should be the case for a calibrated spectral line cube.
It is necessary to rebin the cube, applying a shift to each spectrum so
that all spectra have the same wavelength calibration.
This shift is referred to as the {\em phase shift}, the magnitude of the phase
shift as a function of position in the image is referred to as the {\em phase
map}, and the process of determining the phase map and rebinning the data cube
is referred to as {\em phase calibration}.

A user manual is available from your Site Manager.

\item [TWODSPEC]\index{TWODSPEC} \hfill [SUN/16]

This reduces and analyses long-slit and optical-fibre array spectra.
A number of its functions are useful outside the area of spectroscopy.
The main application areas are:
\begin{itemize}

\item Line profile analysis --- LONGSLIT analyses calibrated long-slit spectra.
For example, it can fit Gaussians, either manually or automatically, in batch.
It can handle data with two spatial dimensions, such as TAURUS data.
\mbox{FIBDISP} provides further options useful for such data, although it is
primarily designed for fibre array data.
An extensive range of options is available, especially for output.

\item Two-dimensional arc calibration.

\item Geometrical distortion correction --- S-distortion and Line curvature.

\item Conversion between FIGARO and IRAF data formats.

\item Display programs.
\item Removing continua.
\end{itemize}

\end{description}

\end{description}

\newpage

\subsection{Specific wavelengths}

\rule{\textwidth}{0.5mm}
\begin{description}
\begin{description}
\item [ASTERIX] --- X-ray data analysis \hfill $\surd\surd$
\end{description}
\end{description}
\rule{\textwidth}{0.5mm}

\begin{description}

\item [ASTERIX]\index{ASTERIX} \hfill [SUN/98]

A collection of programs to analyse astronomical data in the X-ray waveband.
Many of them are general purpose and are capable of analysing any data in the
correct format.
It is instrument independent and currently has interfaces to the Exosat and
Rosat instruments. 

ASTERIX data are stored in HDS files and are therefore compatible with all ADAM
packages.
There are basically two different types: binned and event datasets.
{\em Binned data} (e.g.\ time series, spectra, images) are stored in files whose
structure is based on the Starlink standard NDF format (SGP/38). 
Data errors (stored in the form of variances) and quality are catered for.
{\em Event data} store information about a set of photon `events'.
Each event will have a set of properties, e.g.\ X position, Y position, time,
raw pulse height.

The input data is first processed by an instrument interface.
Event data is then processed and binned, and then the binned data is
processed.
Finally, graphical output is generated.

The commands may be classified as follows:

\begin{itemize}
\item Interface to a particular instrument (EXOSAT, ROSAT, etc).
\item Event dataset and binned dataset processing.
\item Data conversion and display.
\item Mathematical manipulations.
\item Time series analysis.
\item Image processing.
\item Spectral analysis.
\item Statistical analysis.
\item Data quality analysis.
\item HDS editor.
\item Source searching.
\item Graphical and textual display.
\end{itemize}

\end{description}

\newpage

\subsection{Specific instruments}

\rule{\textwidth}{0.5mm}
\begin{description}
\begin{description}
\item [IRCAM] --- UKIRT (Infra-red) \hfill $\times$
\end{description}
\end{description}
\rule{\textwidth}{0.5mm}

\begin{description}

\item [IRCAM]\index{IRCAM} \hfill [SUN/41]

A package to reduce, display, and analyse 2D images from the
{\em UKIRT infrared camera} (IRCAM).
Two user interfaces are available: command line and menu.
The image data reduction facilities available are:
\begin{itemize}
\item Mathematical and statistical operations.
\item Size changing and mosaicing.
\item Inspection.
\item Interpolation.
\item Smoothing.
\item Feature enhancement.
\item Bad pixel removal.
\item Polarimetry.
\item Median filtering of flat-fields.
\end{itemize}
The graphics and image display facilities available are:
\begin{itemize}
\item Image display of various types (PLOT, CONTOUR, NSIGMA, RANPLOT).
\item Display cursor position and value.
\item Colour control.
\item Line graphics such as 1D cuts/slices through images and contour maps.
\item Annotation.
\end{itemize}

\end{description}

\newpage

\subsection{Polarimetry}

\rule{\textwidth}{0.5mm}
\begin{description}
\begin{description}
\item [TSP] --- Time series and polarimetry analysis \hfill $\surd$
\end{description}
\end{description}
\rule{\textwidth}{0.5mm}

\begin{description}

\item [TSP]\index{TSP} \hfill [SUN/66]

A data reduction package to handle time series and polarimetric data.
These facilities are missing from most existing data reduction packages which
are usually oriented towards either spectroscopy or image processing or both.
Currently TSP can process the following data:
\begin{itemize}
\item Spectropolarimetry obtained with the AAO Pockels cell
 spectropolarimeter in conjunction with either IPCS or CCD detectors.
\item Time series polarimetry obtained with the Hatfield Polarimeter
 at either UKIRT or AAT.
\item Time series polarimetry obtained with the University of Turku
 UBVRI polarimeter.
\item Five channel time series photometry obtained with the Hatfield
 polarimeter at the AAT in its high speed photometry mode.
\item Time series infrared photometry obtained with the AAO Infrared
 Photometer Spectrometer (IRPS).
\item Time series optical photometry obtained using the HSP3 high speed
photometry package at the AAT.
\end{itemize}

\end{description}

\newpage

\subsection{Database management}

\rule{\textwidth}{0.5mm}
\begin{description}
\begin{description}
\item [SCAR] --- Catalogue data base system \hfill $\surd$
\item [CHIAPP] --- Catalogue/table data base system \hfill $\surd\surd$
\end{description}
\end{description}
\rule{\textwidth}{0.5mm}

\begin{description}

\item [SCAR]\index{SCAR} \hfill [SUN/70, 106]

The Starlink Catalogue Access and Reporting system is a relational database
management system.
It was designed principally for extracting information from astronomical
catalogues, but it can be used to process any data stored in relational form.
A large number of catalogues are available, including the IRAS catalogues.
For general database requirements, REXEC may be preferable.
SCAR can perform the following functions:
\begin{itemize}
\item Extract data from a catalogue using input criteria.
\item Manipulate it using various statistical and plotting routines.
\item Output data from a catalogue.
\item Put a new catalogue into the database.
\item Search catalogues and generate reports on what has been found.
 (You can get information much more quickly from an on-line catalogue than by
 going to a library and browsing through a book or microfiche.)
\item Sort, merge, join, and difference catalogues.
\item Plot sources in a gnomonic (tangent plane) or Aitoff (equal area)
 projection.
\item Analyse the fields of a catalogue by scatterplot and histogram.
\item Calculate new fields.
\end{itemize}
A distinctive feature of SCAR, compared to MIDAS, is the use of index files
which you can create and which contain pointers to rows in one or more
catalogues.
This allows a very compact and flexible method of accessing catalogues;
for example, a very large catalogue may be physically ordered by declination,
but you can create an index giving access to it ordered by flux.
Alternatively, you may select a subset of a catalogue and store just the
pointers to it, not the data.
Another feature is the use of text files, called description files, to describe
catalogue contents.
This allows the aficionado to manually edit the description file to produce very
complicated output products.

\item [CHIAPP]\index{CHIAPP} \hfill [SUN/120]

This is a set of programs to access and manipulate astronomical catalogues
using an ADAM command language (such as ICL).
It is based on the CHI subroutine library.
The following functions are available:
\begin{itemize}
\item Create/delete a catalogue.
\item Add/delete a catalogue parameter.
\item Get parameter names and information about a parameter.
\item Get field names and information about a field.
\item Get number of entries.
\end{itemize}
CHIAPP is being developed as a replacement for SCAR.

\end{description}

\newpage

\subsection{Utilities}

\rule{\textwidth}{0.5mm}
\begin{description}
\begin{description}
\item [CONVERT] --- Data format conversion (DIPSO/FIGARO/BDF to NDF) \hfill 
 $\surd\surd\surd$
\item [TRACE] --- HDS data file listing \hfill $\surd\surd\surd$
\item [SST] --- Simple Software Tools package \hfill $\surd\surd\surd$
\end{description}
\end{description}
\rule{\textwidth}{0.5mm}

\begin{description}

\item [CONVERT]\index{CONVERT} \hfill [SUN/55]

Converts data between the Starlink standard $n$-dimensional-data format
(NDF), described in SGP/38, and other formats.
Currently, it can handle three data formats:
\begin{itemize}
\item DIPSO.
\item Figaro version 2.
\item INTERIM (BDF).
\end{itemize}

\item [TRACE]\index{TRACE} \hfill [SUN/102]

One of the most popular features of the ADAM environment is the Hierarchical
Data System (HDS) which enables associated data items to be stored together in
a single file but in a structured fashion.
TRACE lists the contents of an HDS data structure on your terminal and,
optionally, to a file for printing or documentation.

\item [SST]\index{SST} \hfill [SUN/110]

The Simple Software Tools package helps produce software and documentation,
with particular emphasis on ADAM programming using Fortran~77. 

As its name suggests, SST is intended to perform fairly simple manipulations of
software, but it aims to tackle some of the commonly encountered problems which
are not catered for in the more sophisticated commercial software tools (such as
FORCHECK and VAXset) available on Starlink. 
In future, SST may also duplicate a few of the simpler facilities which
commercial products offer in order to avoid the cost (and generally much
higher overheads, such as disk space) which prevent the commercial products
being freely available on all Starlink machines. 

The main purpose of the first version is to extract information from subroutine
`prologues', and to format it to produce various forms of user documentation. 
A simple source-code and comment statistics tool is also included.

There are five applications:
\begin{itemize}
\item Convert `old-Style' ADAM/SSE prologues.
\item Produce \LaTeX\ documentation.
\item Produce Help libraries.
\item Produce STARLSE package definitions.
\item Produce source-code statistics.
\end{itemize}

\end{description}

\newpage

\section{Stand-alone Applications}

These are independent applications that often run within their own environment.
In general they cannot be used together and may have incompatible data formats.

\subsection{Image analysis \& photometry}

\rule{\textwidth}{0.5mm}
\begin{description}
\begin{description}
\item [RGASP] --- Galaxy surface photometry \hfill $\surd$
\end{description}
\end{description}
\rule{\textwidth}{0.5mm}

\begin{description}

\item [RGASP]\index{RGASP} \hfill [SUN/52]

A reduced version of GASP, a {\em Galaxy Surface Photometry} package.
It accepts raw pixel-data as input, and outputs a set of luminosity profiles and
shape parameters for each galaxy.
The design objectives were to:
\begin{itemize}
\item Minimise interactive effort.
\item Use robust algorithms which can accept `don't know' values in the data,
thereby eliminating the necessity to interpolate in regions where images
overlap.
\end{itemize}
The facilities provided include:
\begin{itemize}
\item Calculator.
\item Data preparation, compression, re-organisation, re-scaling, analysis, and
 display.
\item Diagnostics.
\item File loading, reading, and re-organisation.
\item Graphics (including colour).
\item Identification.
\item Parameter display.
\item Simulation.
\item Tape handling, reading and writing.
\end{itemize}
\end{description}

\newpage

\subsection{Spectroscopy}

\rule{\textwidth}{0.5mm}
\begin{description}
\begin{description}
\item [DIPSO] --- Spectral analysis and plotting \hfill $\surd$
\item [APIG] --- Absorption profiles in the interstellar gas \hfill $\surd$
\end{description}
\end{description}
\rule{\textwidth}{0.5mm}

\begin{description}

\item [DIPSO]\index{DIPSO} \hfill [SUN/50]

Historically, this is a simple plotting package incorporating some basic
astronomical applications.
If you just want to read in some data, plot them, and measure some equivalent
widths or fluxes, this can do it without much effort on your part.

To make more complicated things possible, a number of rather rudimentary
functions and parameters are provided.
A macro facility allows convenient execution of regularly used sequences of
commands, and a simple Fortran interface permits `personal' software to be
integrated.
User programs can be added to the system and defined as new commands.

The following operations are available:
\begin{itemize}
\item Arithmetic.
\item Emission line fitting.
\item Equivalent width measurement.
\item Flux measurement.
\item Fourier analysis.
\item Interstellar line analysis.
\item Model atmospheres.
\item Nebular continuum modelling.
\item Polynomial fitting.
\item Simple statistics.
\end{itemize}

\item [APIG]\index{APIG} \hfill [SUN/103]

This analyses the interstellar absorption lines detected in the spectra
of galactic and extragalactic sources.
When observed with sufficiently high resolution, these lines often show a
complex structure with multiple components formed in absorbing regions at
different velocities along the line of sight.
The main purpose of the program is to determine the column density of absorbers
along the line of sight from an interactive analysis of the equivalent widths
and profiles of the observed absorption lines.
In cases where the line profiles are resolved, the velocity structure of the
absorbing material is also determined.

It is not a reduction system for astronomical spectra.
It assumes that the raw spectra have already been reduced to produce wavelength
calibrated spectra normalised relative to the background continuum, and
that equivalent widths have been determined.

\end{description}

\newpage

\subsection{Specific wavelengths}

\rule{\textwidth}{0.5mm}
\begin{description}
\begin{description}
\item [SPECX] ---  Mm-wave data analysis \hfill $\surd$
\item [NOD2] --- Radio astronomy data analysis \hfill $\times$
\end{description}
\end{description}
\rule{\textwidth}{0.5mm}

\begin{description}

\item [SPECX]\index{SPECX} \hfill [SUN/17]

A mm- and sub-mm wavelength data reduction system written
by Rachael Padman at MRAO.
Although it can process spectra from many different instruments, it is
particularly useful for reducing JCMT data.
Its major features are:
\begin{itemize}
\item Process up to eight spectra simultaneously.
\item Save current status of system after each command is executed.
\item List and display spectra on a graphics terminal, with hardcopy on many
printers.
\item Single and multiple scan arithmetic, scan averaging, etc.
\item Store and retrieve intermediate spectra in storage registers.
\item Fit and remove polynomial, harmonic, and Gaussian baselines.
\item Filter and edit spectra.
\item Determine important line parameters (peak intensity, width, etc).
\item Calculate Fourier transforms and power spectra.
\item Calibrate data.
\item Assemble a number of reduced individual spectra into a map file, and
contour any plane or planes of the resulting cube.
\item Macro-command sequences and indirect command files.
\end{itemize}
SPECX uses it's own data format, so it is not possible to access directly
reduced spectra using other packages.
However, facilities are provided for importing data from GSD-format data files,
as produced by the JCMT, for writing out maps and spectra in the file formats
of FIGARO and KAPPA, and for writing spectra and maps out to Ascii files for
reading into other packages.

\item [NOD2]\index{NOD2} \hfill [SUN/46]

Manipulates radio single-dish observations.
It was derived from a subroutine library written at the Max Planck Institut
f\"{u}r Radioastronomie, Bonn.
It reduces data from the {\em James Clerk Maxwell Telescope} (JCMT).
It contains implementations of certain algorithms, essential to the reduction
of mm-wave single dish observations, which are not available elsewhere.
These include the `RESTOR' routine, which will reconstruct the equivalent
single-beam map from data taken in beam-switched mode, and `CONVERT', which
will create an RA-Dec map from Az-El data.
The current method for reducing data from the JCMT is thus to use NOD2 for the
immediate processing required to perform the `instrument-specific' part of the
data reduction process, and then to convert the resulting maps into an
appropriate format for analysis by another package (such as IRCAM or FIGARO).
The following facilities are available:
\begin{itemize}
\item Map making.
\item Calibration.
\item Extinction correction.
\item Baseline subtraction.
\item Correction of bad pixels.
\item Pointing correction.
\item Change of weight factor or effective integration time.
\item Map averaging.
\item Display and editing.
\item Post processing.
\item Image transport and conversion (to HDS/SDF format).
\end{itemize}

\end{description}

\newpage

\subsection{Specific instruments}

\rule{\textwidth}{0.5mm}
\begin{description}
\begin{description}
\item [IUEDR] --- IUE (Ultra-violet) \hfill $\surd$
\item [HXIS] --- SMM (X-ray) \hfill $\surd$
\end{description}
\end{description}
\rule{\textwidth}{0.5mm}

\begin{description}

\item [IUEDR]\index{IUEDR} \hfill [SUN/37]; [SG/3]

IUE stands for `International Ultraviolet Explorer'.
The IUE Data Reduction package processes data from the {\em IUE echelle
spectrograph}, starting with the raw image and finishing with a fully calibrated
spectrum.
The following facilities are provided:
\begin{itemize}
\item Tape analysis.
\item IUE image input.
\item Spectrum extraction.
\item Line-by-line spectrum creation.
\item Spectrum calibration.
\item Graphical and image display.
\item IUESIPS extracted spectra input.
\item Spectrum averaging.
\item Output products --- spectra which can be read by DIPSO.
\end{itemize}

\item [HXIS]\index{HXIS} \hfill [SUN/76]

This processes data from the Hard X-ray Imaging Spectrometer (HXIS) carried on
the Solar Maximum Mission (SMM).
The following functions are provided:
\begin{itemize}
\item Image display and manipulation, time-series plots, rudimentary spectral
 analysis.
\item Transfer data between a HIMSEL tape and disk.
\item Convert a production data tape to a HIMSEL image tape or to a disk image.
\item Access the HIMSEL tape catalogue.
\item Concatenate two disk image files.
\end{itemize}

\end{description}

\newpage

\subsection{Statistics}

\rule{\textwidth}{0.5mm}
\begin{description}
\begin{description}
\item [GENSTAT] --- General statistical analysis \hfill $\surd\surd$
\item [CLUSTAN] --- Cluster analysis (Restricted to 10 sites) \hfill
 $\surd\surd$
\item [ASURV] --- Statistical analysis of data with upper limits \hfill $\surd$
\end{description}
\end{description}
\rule{\textwidth}{0.5mm}

\begin{description}

\item [GENSTAT]\index{GENSTAT} \hfill [SUN/54]

Version 5 of a commercial package for statistical analysis, with all the
facilities of a general-purpose statistical package, marketed by NAG.
All the usual analyses are readily available using the standard commands;
in particular:
\begin{itemize}
\item Regression analysis.
\item Multivariate analysis.
\item Time series analysis.
\end{itemize}
However, it is not just a collection of pre-programmed commands for
selecting from fixed recipes of available analyses.
It has a flexible command language in which you can write your own programs to
cover the occasions when the standard analyses do not give exactly what you
want, or when you want to develop a new technique.

Extensive documentation is available, including a reference manual, an
introduction, a reference summary, and a procedure library manual.
There is also a set of newsletters.

\item [CLUSTAN]\index{CLUSTAN} \hfill [SUN/26]

This is a specialised multivariate cluster analysis package marketed by
Clustan Ltd of Edinburgh.

A way is needed to summarise, categorise, condense, and represent the large
amount of astronomical data being produced.
Multivariate techniques and cluster analysis are appropriate means of
performing an initial survey of the data in this respect for many applications
such as star-galaxy discrimination, prediction of spectral type from
photometry etc.
CLUSTAN serves these purposes.
\begin{itemize}
\item It is a fully integrated package of tried and tested procedures for
 Cluster analysis and other multivariate statistical methods.
\item It is an extremely comprehensive tool for solving classification
 problems.
\item It operates from within a high-level programming language which uses
 English words and keywords and so is easy to use.
\item It has comprehensive on-line help plus syntax checking and error
 correction.
\item It can be used interactively, or in batch
\item You can define your own similarity measures for cluster analysis.
\item Data used with SCAR are easily accessible.
\item Input data can be continuous, real or integer, mixed, multi-state
 attributes, or binary presence/absence values.
\item There is an in-built graphics capability for plotting dendrograms,
 cluster diagrams, minimum spanning trees etc.
\end{itemize}

\item [ASURV]\index{ASURV} \hfill [SUN/13]

A package for statistical analysis of astronomical data with upper
limits, written by Takashi Isobe and Eric Feigelson of the Department of
Astronomy, Pennsylvania State University.

Observational astronomers frequently encounter the situation where they
observe a particular property of a previously defined sample of objects,
but fail to detect all of the objects.
The data then contains `upper limits' as well as detections, preventing the
use of simple and familiar statistical techniques in the analysis.
However, a variety of other statistical methods exist to deal with these
problems which are collectively called `survival analysis' or the `analysis
of lifetime data' from their origin in actuarial and related fields.
The upper limits are called `censored' data points.
ASURV is a menu-driven stand-alone computer package designed to assist
astronomers in using some of these methods.

No statistical procedure can magically recover information that was never
measured at the telescope.
However, frequently there is important information implicit in the failure
to detect some objects which can be partially recovered under reasonable
assumptions.
ASURV provides several two-sample tests, correlation tests, and linear
regressions --- each based on different models of where upper limits truly lie
--- so that you can judge the importance of the different assumptions.

\end{description}

\newpage

\subsection{Database management}

\rule{\textwidth}{0.5mm}
\begin{description}
\begin{description}
\item [REXEC] --- Relational database management system \hfill $\surd$
\end{description}
\end{description}
\rule{\textwidth}{0.5mm}

\begin{description}

\item [REXEC]\index{REXEC} \hfill [SUN/97]

A relational database management system developed by the Scientific Databases
Section within the Central Computing Department at RAL.
It is much smaller and easier to use than SCAR, and is primarily intended as a
tool to aid the handling and processing of experimental data.
It offers a convenient way of manipulating structured files in a
`Fortran-friendly' environment, and avoids the complexity and all-embracing
commitment of a conventional dbms.

It can be regarded as an extension of the UNIX philosophy of standard tools
acting on files in a flexible and interchangeable manner.
Here we have a set of processes stemming from relational algebra which
interface uniformly to database files.
It may be regarded as a successor to G-EXEC, and retains its two fundamental
features: the {\em self-describing file} and the {\em process program}.

It is good for storing and manipulating large datasets, allowing extensive
mathematical processing of data.
It is not appropriate for transaction processing.

\end{description}

\newpage

\section{INTERIM Applications}

These run under the original Starlink environment called INTERIM.
This has now been superseded by ADAM.

\subsection{Specific Instruments}

\rule{\textwidth}{0.5mm}
\begin{description}
\begin{description}
\item [IRAS] --- IRAS (Infra-red) \hfill $\surd\surd$
\end{description}
\end{description}
\rule{\textwidth}{0.5mm}

\begin{description}

\item [IRAS]\index{IRAS}\index{IRASLRS} \hfill [SUN/14, 60, 80, 81, 82, 91]

{\em IRAS} is the Infra-Red Astronomical Satellite, and this is a package to
process {\em IRAS} data.
It is meant to be used in conjunction with ASPIC.
It will:
\begin{itemize}
\item Access spectra in the IRAS LRS catalogue.
\item Combine, compress, and expand images.
\item Produce contour maps.
\item Display descriptor items.
\item Display positions of IRAS sources and AO's on images.
\item Convert BDF files to FITS format.
\item Handle AO's, galactic plane maps, and sky plates.
\end{itemize}

\end{description}

\newpage

\subsection{General purpose}

\rule{\textwidth}{0.5mm}
\begin{description}
\begin{description}
\item [ASPIC] --- General data analysis \hfill $\times$
\end{description}
\end{description}
\rule{\textwidth}{0.5mm}

\begin{description}

\item [ASPIC]\index{ASPIC} \hfill [SUN/23, 24]; [SG/1]

The Astronomical Picture Processing system is a {\em general data reduction}
system which contains a wide range of functions for image processing, along
with a number of more specialized astronomical functions.
The data are stored in BDF format.
It is based on the INTERIM software environment and is controlled by a command
language called DSCL.
A special guide [SG/1] is available which introduces the system to new users.
The system contains functions for:
\begin{itemize}
\item Arithmetic operations.
\item Astrometry.
\item Data compression.
\item Electronography data reduction.
\item Filtering.
\item Fourier transforms.
\item Geometrical transformations.
\item Image display.
\item Period analysis.
\item Photometry.
\item Polarimetry.
\end{itemize}
and many others.
It can read tapes in FITS, VICAR, IPCS, and PDS formats.
Apart from the EDRSX component, it is no longer being actively developed
or supported and is expected to be withdrawn sometime in the future.
You should, therefore, use newer packages such as FIGARO and KAPPA rather than
ASPIC.

\end{description}

\newpage

\section{Foreign Applications}

These applications have been acquired from non-Starlink organisations.
They are not installed on the Project node at RAL and are not supported or
distributed as part of the normal Starlink Software distribution service.
Enquires should be directed to the sites specified below.
\begin{description}
\begin{description}
\item [AIPS] --- Jodrell Bank
\item [IRAF] --- Cardiff
\item [IDL] --- RAL (Jeff Payne)
\item [MIDAS] --- St Andrews
\end{description}
\end{description}

\subsection{Specific wavelengths}

\rule{\textwidth}{0.5mm}
\begin{description}
\begin{description}
\item [AIPS] --- Radio astronomy data analysis \hfill $\times$
\end{description}
\end{description}
\rule{\textwidth}{0.5mm}

\begin{description}

\item [AIPS]\index{AIPS} \mbox{}

The Astronomical Image Processing System provides radioastronomers with
a set of tools for calibrating and editing radio-interferometric data,
constructing images using Fourier synthesis techniques, and displaying and
analysing these images.
Its use is widespread in the radio astronomy community.
It has also been used for IRAS work.

AIPS has a richer set of {\em general} image processing functions than any
other astronomical software package.
For example, AIPS can:
\begin{itemize}
\item Deconvolve a point-spread function from an image.
\item Transform one image into the coordinate system of another.
\item Fourier transform an image into the complementary Fourier domain,
 allowing you to edit the transformed image and transform it back again
 (useful for removing stripes from IRAS images).
\item Enhance various features in an image using gradient filters, or even the
 Sobel edge enhancement filter.
\item Display successive planes of a data cube as a continuous movie.
\item Display a spectral line data set using hue to denote velocity and
 intensity to denote integral brightness.
\end{itemize}

\end{description}

\newpage

\subsection{Specific instruments}

\rule{\textwidth}{0.5mm}
\begin{description}
\begin{description}
\item [IRAF] --- HST (multi-wavelength) \hfill $\times$
\end{description}
\end{description}
\rule{\textwidth}{0.5mm}

\begin{description}

\item [IRAF]\index{IRAF} \mbox{}

The Image Reduction and Analysis Facility is a general purpose software system
for the reduction and analysis of scientific data.
It was developed by the National Optical Astronomical Observatory (Kitt Peak)
and has been adopted by the Space Telescope Science Institute in Baltimore as
the principal data analysis environment for Hubble Space Telescope (HST) data.
(HST data can also be reduced with KAPPA and FIGARO.)
It provides a good selection of programs for general image processing and
graphics applications, plus a large number of programs for reducing and
analysing optical astronomy data.
The system has been made as portable and device independent as possible; for
example, it includes its own programming language.
Although it runs on a variety of computers, in practice at any one time it runs
best on one particular type (Suns at present).

It is a very large system which is supported by extensive documentation.
Every Starlink node should have a copy of the IRAF User Handbook to which
you can refer for further information.
You can also make enquiries to the Site Manager at the Cardiff Starlink
site (CARDIF::OPER) as the Starlink HST SIG programmer works there.

\end{description}

\newpage

\subsection{General purpose}

\rule{\textwidth}{0.5mm}
\begin{description}
\begin{description}
\item [IDL] --- General data analysis (US Commercial package) \hfill 
 $\surd\surd$
\item [MIDAS] --- General data analysis (ESO) \hfill $\times$
\end{description}
\end{description}
\rule{\textwidth}{0.5mm}

\begin{description}

\item [IDL]\index{IDL} \mbox{}

A proprietary package, written by Dave Stern of Research Systems Inc,
Denver, which is available on some Starlink nodes (currently Armagh, Belfast,
Birmingham, Cambridge, Oxford, QMW, and RAL) and which provides a comprehensive
set of tools for exploring and manipulating data.
At its simplest level it can be used as a very powerful arithmetic and
graphics utility for handling data arrays.
It provides:
\begin{itemize}
\item A programming language.
\item Easy graphics, image plotting and display.
\item Mathematical and transcendental functions.
\item Array and string manipulation.
\item Input/output.
\item Type conversion.
\item Other complex operations.
\end{itemize}
It can be used with minimal effort for many applications, or developed into
more complex procedures as required.
A procedure library is available.

Until recently it was a line-oriented, command-driven system.
However, a portable version (with X-windows) to run on Suns, DECstations, and
VAXstations is now available, providing very fast and versatile graphics and
image display.

\item [MIDAS]\index{MIDAS} \mbox{}

The Munich Image Data Analysis System was developed by the Image Processing
Group of the European Southern Observatory (ESO).
It is a basic image processing environment for reducing and analysing optical
astronomical data.
New applications can be added easily, even by non-professional programmers.
It consists of a monitor which controls the execution of individual tasks, 
and a large set of image processing applications.
Since the results of the reduction of astronomical images are usually numbers
(not other images), a comprehensive and flexible table system forms an
essential part of MIDAS.
Its main features are:
\begin{itemize}
\item Support of computers with VMS and Unix operating systems.
\item Device independent interfaces to peripherals by use of special libraries,
 e.g.\ AGL for graphics, ID for image display, and TW for terminals.
\item Support of display and hard-copy standards like X11 and PostScript.
\item Easy and flexible integration of user software, written in standard
 Fortran and C, into the system.
\item Full support of data exchange in FITS format.
\item On-line help, history, and logging facilities.
\item Flexible command and control language with full flow control and
 debugging facilities.
\item Extensive interface libraries in Fortran and C for access to the MIDAS
 data base.
\end{itemize}
MIDAS offers extensive packages in the areas of spectral reduction (including
data in long slit and echelle formats), CCD observations, crowded field
photometry, object search and classification, fitting and modeling of data,
astrometry, and statistical analysis.

\end{description}

\newpage

\section {Astronomical Utilities}

These are programs or packages which carry out some specific task related to
astronomy, but which do not analyse data directly in order to produce
astronomical results.

\subsection{Archive access}

\rule{\textwidth}{0.5mm}
\begin{description}
\begin{description}
\item [IUEDEARCH] --- Access IUE archive data \hfill $\surd\surd$
\item [USSP] --- Access IUE uniform low-dispersion archive (STADAT only)
 \hfill $\surd\surd$
\end{description}
\end{description}
\rule{\textwidth}{0.5mm}

\begin{description}

\item [IUEDEARCH]\index{IUEDEARCH} \hfill [SUN/58]

The IUE data archive is supported and maintained by members of the UK IUE 
Project in the Space Science Department at RAL. 
The archive comprises some 1000 tapes containing about 70000 images, and a
catalogue giving details of the exposures.
New images are continually added to the archive and the catalogue is updated
every month.
All IUE data are available to the astronomical community about six months after
the observations were made and they appear in the catalogue a few months before
release.
The aim of the system is to provide UK astronomers with quick and painless 
access to IUE data.
It enables you to interrogate the catalogue and extract data from the
archive.
Two other software items, IUEDR and USSP,  are related specifically to IUE
data.

\item [USSP]\index{USSP} \hfill [SUN/20]

The usefulness of IUE data has been enhanced by a data archive from which you
can request any image (subject  to the six-month data retrieval rule).
Most UK astronomers now reduce their IUE data from photometric images using
IUEDR.
However, it is possible to obtain reduced low-resolution spectra on-line.   
These have been reduced in a fairly homogenous way using the IUE Project's
Spectral Image Processing System, and have been made available in the form of a
{\it Uniform Low-Dispersion Archive} (ULDA).  
This comprises absolutely-calibrated, extracted, low-dispersion spectra 
corrected for exposure times and with the appropriate ITF correction.
The {\it ULDA Software Support Package} (USSP) has been installed on the
Starlink database microVAX (STADAT) to facilitate access to the ULDA.
It enables you to select any data in the ULDA and transfer them to your own
computer.
You can search for various types of image, based on object class, image number,
camera, aperture, homogeneous ID, sky position, or any combination of these
attributes.
\end{description}

\newpage

\subsection{Data copying \& format conversion}

\rule{\textwidth}{0.5mm}
\begin{description}
\begin{description}
\item [STARCON] --- Data format conversion (BDF to HDS) \hfill $\surd\surd\surd$
\item [EDFITS] --- Copy FITS tapes \hfill $\surd\surd$
\item [WFC\_SORT] --- Data format conversion (ROSAT-WFC to ASTERIX (HDS))
 \hfill $\surd\surd$
\item [FORMCON] --- Data format conversion (IPCS/VICAR to BDF) \hfill
 $\times\times$
\end{description}
\end{description}
\rule{\textwidth}{0.5mm}

\begin{description}

\item [STARCON]\index{STARCON} \hfill [SUN/96]

Converts data between BDF format and HDS format (i.e.\ between INTERIM and ADAM
environment formats).

\item [EDFITS]\index{EDFITS} \hfill [SUN/43]

Copies and edits FITS tapes.
It is designed for original data tapes, so it has a slightly unusual style
which places a heavy emphasis on making as sure as possible that you
are not doing anything rash --- such as mounting an input tape with a write 
ring fitted --- and that you understand the consequences of the commands given.
It can produce usefully formatted header listings for a variety of flavours of
FITS tapes.

\item [WFC\_SORT]\index{WFC\_SORT} \hfill [SUN/62]

Produces ASTERIX (HDS) datasets from Wide Field Camera data generated during
the pointed phase of the Rosat mission.
A sort program generates the files (e.g.\ time series, images, etc) from
pre-processed event files supplied on the Rosat WFC Observation Datasets
(RWODs).
An exposure program allows the correction of a raw count dataset allowing for
instrument characteristics.
In addition, a simple database manager allows an index of RWODs to be
maintained and searched.

\item [FORMCON]\index{FORMCON} \hfill [SUN/3]

Converts data into BDF format from IPCS and VICAR formats.

\end{description}

\newpage

\subsection{Observation preparation and positional astronomy}

\rule{\textwidth}{0.5mm}
\begin{description}
\begin{description}
\item [COCO] --- Celestial coordinate conversion \hfill $\surd\surd\surd$
\item [CHART] --- Finding chart and stellar data system \hfill $\times$
\item [ASTROM] --- Basic astrometry \hfill $\surd\surd\surd$
\item [RV] --- Calculate radial components of observer's velocity \hfill 
 $\surd\surd\surd$
\item [RPS] --- Submit Rosat proposals \hfill $\surd$
\item [ECHWIND] --- Plan use of UCL echelle spectrograph \hfill $\surd$
\item [TPOINT] --- Telescope pointing analysis \hfill $\surd\surd\surd$
\item [APLATE] --- Aperture plate preparation \hfill $\surd$
\item [AATGS] --- Guide probe predictions for AAT \hfill $\surd\surd\surd$
\end{description}
\end{description}
\rule{\textwidth}{0.5mm}

\begin{description}

\item [COCO]\index{COCO} \hfill [SUN/56]

Converts star coordinates between the following six different coordinate
systems:
\begin{itemize}
\item Mean {\em RA,Dec}, old system, with E-terms (loosely FK4).
\item Mean {\em RA,Dec}, old system, no E-terms (some radio positions).
\item Mean {\em RA,Dec}, new system (loosely FK5).
\item Geocentric apparent {\em RA,Dec}, new system.
\item Ecliptic {\em longitude,latitude}, new system (mean of date).
\item Galactic {\em longitude,latitude}, IAU 1958 system.
\end{itemize}

\item [CHART]\index{CHART} \hfill [SUN/32]

A package for plotting star fields (positions, magnitudes and other
data) from the CSI79, SAO, AGK3, PERTH70, and the Dixon non-stellar objects
catalogues.
You can specify a series of search areas, and can place magnitude or total
number limits on the search, and chose which source catalogues to include
or exclude.
After the search is made, you may list the results at a terminal or on a
printer.
Positional information may be precessed to a specified equinox.
In the case of astrometric data, proper motions may be applied up to a specified
epoch.

You can plot the results in the form of an overlay or finding chart.
Extra objects, i.e.\ with positions supplied by you, may be added to the plot at
this stage.
The plot may be made on any GKS device, and a number of different plot options
are available, such as scale and area, an RA and Dec coordinate grid, and
various forms of error box.

You may also use the results as input to an astrometry program, i.e.\ it can
be used to select astrometric stars as positional references.
In this case, you will be asked to supply {\em x,y} positions for the reference
stars from a measuring machine, and can then convert unknown {\em x,y} positions
to RA, Dec or vice versa.
ASTROM performs the actual astrometry, from within CHART if necessary.

\item [ASTROM]\index{ASTROM} \hfill [SUN/5]

A simple plate reduction utility, designed to allow the non-specialist
user to get good results with a minimum of trouble and esoteric knowledge.
You supply a text file containing information about the exposure and the
positions of reference and unknown stars;  ASTROM performs the various
coordinate transformation and fitting operations required, displays a synopsis
report on the command terminal, and prepares a detailed report for later
printing.

\item [RV]\index{RV} \hfill [SUN/78]

Produces a report listing the components, in a given direction, of the
observer's velocity on a given date.
This allows an observed radial velocity to be referred to an appropriate
standard of rest.
It also computes light time components to the Sun, thus allowing the times of
phenomena observed from a terrestrial observatory to be referred to a
heliocentric frame of reference.

\item [RPS]\index{RPS} \hfill [SUN/18]

Generates and submits electronically observing proposals for the UK Rosat
Pointed Observation Programme. 
It lets you do the following:
\begin{itemize}
\item Create a new Rosat Proposal Form (RPF) by `filling it in' interactively.
\item Edit an existing RPF, for example to make minor changes to observation
 details.
\item Summarise the details of an RPF.
\item Produce a draft (line-printer) version of the RPF for checking.
\item Produce a final (laser printer) version of the RPF for submission to the
 UK Programme.
\item Produce a version of the RPF that can be transmitted over the network and
 submit this electronically to the Rosat UK Data Centre (UKDC).
\item Check when a target will be observable by Rosat and get an estimate of
 the survey exposure.
\end{itemize}

\item [ECHWIND]\index{ECHWIND} \hfill [SUN/53]

The UCL Coud\'{e} echelle spectrograph (UCLES) can, on the AAT, record spectra
of objects as faint as V = 16--17 with a resolving power
($\lambda/\Delta\lambda$) of 55,000--115,000, depending on the detector used.
ECHWIND can prepare an observing proposal or an observing run in which UCLES
will be used.
It allows you to view interactively that part of the spectrum which will fall
on the detector for a given central wavelength.
The positions of individual spectral lines can be marked on the display.
You can then move the detector window around until it is in the correct
position.
At each position the screen displays the minimum, central, and maximum
wavelengths falling on the detector, as well as the range of order numbers and
an estimate of the length of the detector in Angstroms.
The position of the detector can also be marked on the screen so that
optimal wavelength coverage is achieved when multiple wavelength settings are
required.

\item [TPOINT]\index{TPOINT} \hfill [SUN/100]

An interactive telescope pointing analysis system of interest to specialists.
It allows data from pointing tests to be input and fitted to various models.
The residuals from the fits can be displayed in a variety of graphical formats. 
If systematic errors are visible, the pointing model can be adjusted by adding
and removing terms.
It can deal with data from either equatorial or altazimuth mounts.
It has been applied to many telescopes, including optical, IR, mm and radio,
and is in routine use at the AAO, UKST, UKIRT, LPO, and several other
observatories.

\item [APLATE]\index{APLATE} \hfill [SUN/89]

An Aperture Plate preparation program which computes required hole positions
for a fibre optics aperture plate for the AAT.

\item [AATGS]\index{AATGS} \hfill [SUN/6]

The AAT Guide Star prediction program calculates guide star positions for the 
AAT.

\end{description}

\newpage

\section{General Utilities}

These are programs or packages which provide useful facilities to the user
of a computer, but are not specifically related to astronomy.

\subsection{Document preparation \& search}

\rule{\textwidth}{0.5mm}
\begin{description}
\begin{description}
\item [TEX] --- Document preparation (includes \LaTeX) \hfill $\surd$
\item [DOCFIND] --- Starlink document search \hfill $\surd$
\item [GEROFF] --- Document preparation \hfill $\times\times$
\end{description}
\end{description}
\rule{\textwidth}{0.5mm}

\begin{description}

\item [TEX]\index{TEX}\index{LATEX} \hfill [SUN/9, 12, 93]

\TeX\ is a type-setting program for producing high quality text and diagrams.
\LaTeX\ is a document preparation system based on \TeX\ but is easier to use.
The final output is usually generated by a laser printer.
Starlink recommends that documents be prepared using \LaTeX\ with formats based
on skeletons such as DOCSDIR:SUN.TEX.
SUN/12 is a cook-book showing how to generate common kinds of format using
\LaTeX; this is a quick way to learn how to use it.
SGP/28 gives advice on how to produce documents in a suitable style for
Starlink.
The document you are now reading was prepared using \LaTeX.

\item [DOCFIND]\index{DOCFIND} \hfill [SUN/38]

Helps you search the Starlink documentation index for a specified keyword.
You can ask for any document that is found in the search to be printed out.
Unfortunately, this program will not print out \LaTeX\ documents in a
satisfactory form.
However, stocks of all Starlink Notes and Papers are kept at every Starlink
site, so you can get a paper copy.

\item [GEROFF]\index{GEROFF} \hfill [SUN/2]

An obsolete text processing program which has been superseded by \LaTeX.
However, some Starlink documentation is still stored in this form, so it is
retained in the Collection.

\end{description}

\newpage

\subsection{Graphics}

\rule{\textwidth}{0.5mm}
\begin{description}
\begin{description}
\item [MONGO] --- Interactive plotting \hfill $\surd$
\item [IKONPAINT] --- Ikon to inkjet hard-copy \hfill $\surd$
\item [QDP] --- Quick and Dandy Plotter \hfill $\times$
\item [VSHC] --- Obtain hardcopy from the display of a VAXstation \hfill $\surd$
\item [HONEY] --- Hardcopy output from the Honeywell camera \hfill
 $\surd\surd\surd$
\end{description}
\end{description}
\rule{\textwidth}{0.5mm}

\begin{description}

\item [MONGO]\index{MONGO} \hfill [SUN/64]

An interactive program which makes it possible to build up a complicated
diagram, including graphics, text, axes, and so on, and then to create
publishable quality output.
It works on all GKS devices.
It provides the following functions:
\begin{itemize}
\item Selection of graphics device and setup of display area.
\item Input of numerical (x,y) data.
\item Construction of (x,y) graphs.
\item Text placement in five fonts, including Greek.
\item Assembly and execution of command procedures.
\item Setting of line weight, line type, text size, etc.
\item Basic moving and drawing.
\item Generation of hard copy.
\end{itemize}
It is only really suitable for producing publishable quality output.
It is too slow and expensive in computer resources for use in your data
reduction system.

\item [IKONPAINT]\index{IKONPAINT} \hfill [SUN/71]

This provides `push-button' colour hard-copy from the Ikon screen to an
inkjet printer.
The DEC {\it Companion Colour Printer} (LJ250/LJ252) and the Hewlett-Packard
{\it Paintjet} are both supported.
These printers normally take 8$\times$11 inch fan-fold paper, but single
transparency sheets can be hand-fed.

\item [QDP]\index{QDP} \hfill [SUN/128]

The {\em Quick and Dandy Plotter} program reads ASCII files containing
various plotting commands and data.
It then executes the commands and plots the data.
At this point a prompt appears and you can enter additional commands to:
\begin{itemize}
\item Display information about the interactive commands.
\item Override various defaults.
\item Override the commands found in the file.
\item Add/remove labels.
\item Plot data with various combinations of lines, markers, and error bars.
\item Change the appearance/style of the plot, e.g.\ converting all text into
 Roman Font.
\item Plot the data as a function of a different variable.
\item Change the number of panels in which the data is plotted.
\item Define models to calculate the `best fit' parameter values.
\item Generate hardcopy.
\end{itemize}
The interactive commands allow you to tailor the plot to your needs/taste
and to do some simple analyses of the data.
Commands can be placed in the file, in an indirect command, and/or in a command
array created by the calling program.
For example, if you have a set of commands that you commonly use, you can
place them in a file and then execute them.
Programmers can try out commands interactively to find a set that works
best with the type of data being plotted, and then make these commands the
default ones.

The software is highly portable and uses PGPLOT.

\item [VSHC]\index{VSHC} \hfill [SSN/65]

Prints on a Canon Laserprinter a picture selected from the screen of a
VAXstation.

\item [HONEY]\index{HONEY} \hfill [SUN/72]

Uses the Honeywell model 3000 colour graphic recorder system to produce
publishable quality photographic hardcopy from an IKON image display.
Full colour and black \& white images can be recorded on positive or negative
35mm film.
The Honeywell consists of a built-in high resolution flat-faced monochrome
video monitor, a red/green/blue colour filter mechanism and a 35mm camera.

Honeywell systems are installed at nine Starlink sites, namely Belfast
(locally funded), Birmingham, Cambridge, Durham, Leicester, Manchester,
RAL, ROE and UCL.

\end{description}

\newpage

\subsection{Device handling, data compression, \& examination}

\rule{\textwidth}{0.5mm}
\begin{description}
\begin{description}
\item [TPU] --- Magnetic tape handling \hfill $\surd\surd$
\item [LZCMP] --- File compression and de-compression \hfill $\surd\surd\surd$
\item [TAPECOPY] --- Copy magnetic tapes \hfill $\surd\surd\surd$
\item [CDCOPY] --- CDROM reading \hfill $\surd$
\item [XDISPLAY] --- Xwindows display setup \hfill $\surd\surd\surd$
\end{description}
\end{description}
\rule{\textwidth}{0.5mm}

\begin{description}

\item [TPU]\index{TPU} \hfill [SUN/68]

Manipulates, examines, and copies magnetic tapes of any format.
It was designed to give you maximum flexibility and not make any assumptions
about standard formats or procedures.
It sees a magnetic tape as a series of files or records of unknown number
and size.
It gives you maximum flexibility and does not assume some `normal' or `standard'
procedure to be followed when processing the tape.
The need for something like this surfaced when it was found that tapes with some
physical or logical defect were often impossible to duplicate because the
appropriate utility would abort as soon as such an error was encountered without
giving you any say in the matter.
Hence, the objective of TPU is to give you maximum control over operations,
especially when tape errors occur.

\item [LZCMP]\index{LZCMP} \hfill [SUN/25]

Makes files smaller using the Lempel-Ziv file compression algorithm.
Files in compressed form cannot be used in that form but must first be
`decompressed' into their original form.
This is done using this same program.
It is particularly useful for reducing the size of files before transferring
them over a network.
Typical size reductions are 30-40\%.

\item [TAPECOPY]\index{TAPECOPY} \hfill [SUN/47]

Duplicates magnetic tapes.
The contents of the original tape are copied to a disk file which is then
copied to a new tape to create an exact copy of the original.
It can therefore copy tapes on a system with only one tape drive.
It can also copy selected files from one tape to another.
TPU can also copy tapes, but requires two tape drives.

\item [CDCOPY]\index{CDCOPY} \hfill [SUN/69]

Compact discs are becoming a valuable medium for distributing astronomical data,
especially for large files such as source catalogues and images, because each
disc holds up to 600 Mbytes of data.
The DEC RRD40 readers which have been purchased for most Starlink sites are
physically capable of reading these discs, but the VMS device driver
cannot handle the ISO 9660 format which is now in general use.
There are rumours that DEC may rectify this eventually.
Until then, this program allows non-DEC CD-ROM discs to be read on VAX/VMS
systems. 
It should also handle the earlier {\it High Sierra} and ECMA formats. 

\item [XDISPLAY]\index{XDISPLAY} \hfill [SUN/129]

This provides a simple way for users of VT1x00 series Xwindows terminals to
set up a path for Xwindows output from programs such as XDVI, and graphics
packages using the XWINDOWS device.
It also works when remote logins are made from Vaxstations, SUNs and
Decstations to VAX machines.
X-terminals being used for DECwindows sessions do not require XDISPLAY, nor do 
DECwindows sessions on Vaxstations.

\end{description}

\newpage

\subsection{Mathematical}

\rule{\textwidth}{0.5mm}
\begin{description}
\begin{description}
\item [MAPLE] --- Mathematical manipulation language (STADAT only) \hfill
 $\surd\surd$
\end{description}
\end{description}
\rule{\textwidth}{0.5mm}

\begin{description}

\item [MAPLE]\index{MAPLE} \hfill [SUN/107]

A commercial package purchased from WATCOM, Waterloo, CANADA.
It is an interactive system for symbolic algebra computation.
It can perform hundreds of algebraic functions for use at all mathematical
levels, and can provide solutions for many types of problems:
\begin{itemize}
\item Arithmetic with integers, fractions, and polynomials.
\item Power series.
\item Differentiation and integration of functions.
\item Systems of equations.
\item Differential equations.
\item Linear optimization.
\item Tensor manipulation.
\item Symbolic and numeric approximation.
\item Automatic generation of Fortran code and \LaTeX\ source for mathematical
expressions.
\end{itemize}
In addition, plots can be generated to illustrate graphically any function,
including user-defined functions.

You can extend or redefine the numerous functions by writing programs in
the built-in Pascal-like language to create specialized functions.

\end{description}

\newpage

\subsection{Operational}

A number of programs exist in the Collection to help Site Managers manage their
nodes.
Only two of these are likely to be of interest to normal users.
These are:

\rule{\textwidth}{0.5mm}
\begin{description}
\begin{description}
\item [NETWORK] --- DECNET utilities \hfill $\surd\surd\surd$
\item [QUOTAS] --- Other users' disk quotas \hfill $\surd\surd\surd$
\end{description}
\end{description}
\rule{\textwidth}{0.5mm}

\begin{description}

\item [NETWORK]\index{NETWORK} \hfill [SUN/36]

A number of network utilities are available as part of DECNET (MAIL), JANET
(File transfer, Electronic mail, Remote login) and Coloured Books (PAD,
TRANSFER, CBS-MAIL, LIST) software.
In addition, this item provides the following extra DECNET commands:
\begin{itemize}
\item NETSHOW
--- Execute the equivalent of a DCL SHOW command on a remote DECNET node.
\item SHOWNET
--- List all the DECNET nodes and their current state, either reachable or
unreachable.
\item TALK
--- Display a message on another user's terminal (guaranteed to cause
 annoyance or to be lost off the top of an editing screen).
\end{itemize}
Information on how to use networks is given in SUN/36.

\item [QUOTAS]\index{QUOTAS} \hfill [SUN/49]

Displays the usage of disk quota on VMS disks.
It enables you to identify the `disk hogs' on your disks and, perhaps, pay them
a friendly visit.

\end{description}

\newpage

\subsection{Programming support}

\rule{\textwidth}{0.5mm}
\begin{description}
\begin{description}
\item [SPAG] --- Improve structure of Fortran source code (STADAT only) \hfill
 $\surd\surd$
\item [LIBMAINT] --- Library maintenance \hfill $\surd\surd$
\item [FORCHECK] --- Fortran verifier and programming aid (STADAT only) \hfill
 $\surd\surd$
\item [STARLSE] --- Starlink language sensitive editor \hfill $\surd\surd\surd$
\item [GENERIC] --- Compile generic Fortran routines \hfill $\surd\surd$
\item [TOOLPACK] --- Fortran Software Tools package \hfill $\surd\surd$
\item [LIBX] --- Library maintenance \hfill $\surd\surd\surd$
\end{description}
\end{description}
\rule{\textwidth}{0.5mm}

\begin{description}

\item [SPAG]\index{SPAG} \hfill [SUN/63]

SPAG stands for `Spaghetti Unscrambler'.
It re-orders blocks of Fortran statements in such a way that the structure of
the code is improved, whilst remaining logically equivalent to the original
program. 
The result improves the readability and maintainability of badly-written
Fortran programs.
On Starlink, it may be used to convert unstructured Fortran 77 into the
structured and indented VAX Fortran required by the Starlink Programming
Standard (SGP/16). 
It can also be used to update Fortran 66 code.
It is marketed by Polyhedron Software and is available only on the Starlink
Database machine, STADAT.

\item [LIBMAINT]\index{LIBMAINT} \hfill [SUN/99]

Simplifies the maintenance of software held as modules in a
text/object library pair, and of modules in a help library.
New libraries can be created and modules inserted, extracted, replaced,
examined, and printed with simple commands.
It ensures that corresponding modules within a text/object library pair do not
get out of step with one another, and it can optimise the disk space used by
libraries.

\item [FORCHECK]\index{FORCHECK} \hfill [SUN/73]

A Fortran verifier and programming aid installed on the Starlink central
facilities computer (STADAT).
It checks code for conformance to the ANSI standard X3.9--1978.
However, it can also deal with non-standard code and by default accepts VAX
Fortran.
It:
\begin{itemize}
\item Checks inter-module consistency by checking that the number, type, and
 size of elements in both sub-program argument lists and COMMON blocks are the
 same throughout a program.
\item Identifies recursive calls and misuse of arguments.
\item Identifies `clutter' --- the unused variables, COMMON blocks, INCLUDE
 files, and code fragments which accumulate in old programs, and which make
 maintenance such a time-consuming and costly task.
\item Composes cross-reference charts for constants, variables, COMMONs,
 INCLUDE files, sub-programs, and I/O.
\item Produces a `call tree', which shows the calling structure of the program
 in diagrammatic form.
\end{itemize}
Automatically composed documentation of this type is an invaluable addition
to system documentation.
As often as not, it is the only reliable source of information about old
programs.

\item [STARLSE]\index{STARLSE} \hfill [SUN/105]

This is a `Starlink Sensitive' editor based on the \mbox{VAX} Language
Sensitive Editor \mbox{LSE}.
It takes advantage of the extensibility of \mbox{LSE} to provide additional
features which assist you in writing portable Fortran~77 software with a
standard Starlink style. 
It is intended mainly for people writing \mbox{ADAM} applications and
subroutine libraries which are intended for distribution as part of the
Collection, although it may also be suitable for other software projects. 

If you normally use a standard screen editor (like EDT), you will probably find
that using an \mbox{LSE}-based editor for the first time will slow you down
considerably. 
All those new keys to remember!
However, once you get used to it, you will start to realise how much time
you used to spend doing simple things like moving the cursor, formatting
prologues, indenting lines of code and going to fetch essential
documentation --- all things which \mbox{STARLSE} can do far more
efficiently. 

One of the strengths of \mbox{LSE}-based editing is that it allows
you to pick and choose --- to use the features that you personally find
time-saving, while still being able to type directly over anything which you
find too fussy. 
In this respect, \mbox{STARLSE} can be regarded as a sort of interactive
`Manual of Style' which provides guidelines on layout and programming
standards when you need reminding of them, but lets you work unhindered once
you know what you are doing. 

Perhaps the most important component of \mbox{STARLSE} is a version of the
Fortran~77 programming language called \mbox{STARLINK\_FORTRAN}, which
defines the style of programming that the editor supports. 
This language is used by default for files of type \mbox{.FOR} and
\mbox{.GEN}. 

The language is based on the Starlink Application Programming Standard
document SGP/16, and contains only a small number of approved extensions to
the Fortran~77 standard. 
It is therefore much simpler and easier to use than the \mbox{VAX}~Fortran
language which comes with `native' \mbox{LSE} (in fact, nearly 80\% of
\mbox{VAX}~Fortran consists of extensions to the Fortran~77 standard!). 
In \mbox{STARLINK\_FORTRAN}, the number of available options and ambiguous
abbreviations is greatly reduced, and most of the common language constructs
can be produced simply by typing a short token, like `DO' or `IF', followed
by \mbox{ctrl-E}. 

The most important features of the language are:
\begin{itemize}
\item Templates and Prologues.
\item Subroutine definitions.
\item On-line Help.
\item Alias definitions.
\item ADAM programming constructs.
\item Symbolic constants, error codes and Include files.
\item Enumerated type codes.
\item Tokens and Menus.
\end{itemize}

\item [GENERIC]\index{GENERIC} \hfill [SUN/7]

Preprocesses a {\it generic} Fortran subroutine --- one written so as to apply
to several different data types --- into one routine per data type, and
concatenates these routines into a file.
The file is then compiled with the Fortran compiler to produce an
object module.

\item [TOOLPACK]\index{TOOLPACK} \hfill [SUN/75]

Starlink has purchased Release 2 of Toolpack/1 from NAG for its users.
It is a suite of software tools designed to support the Fortran programmer.
In this context, a `software tool' is a utility program to assist in the
various phases of constructing, analysing, testing, adapting or maintaining
a body of Fortran software.
Typically, the input to such a tool is your Fortran software.
The tool processes this and produces output that may have one or both of the
following forms:
\begin{itemize}
\item A report that gives an analysis of the input program, e.g.\ a summary of
the types of statements used --- this type of tool is called a static analyser.
\item A modified version of the input program --- in this case, the tool is
called a transformer.
An example is a formatter which improves the appearance of the code.
\end{itemize}
In some cases the input may be test data, documentation, or a report generated
by a previously applied tool.
Tools that assist directly in preparing documents are usually called
documentation generation aids.
These and other tools serving utility functions all have an important role
to play and so, even if they do not process a program directly, they are
still regarded as programming aids.

Further examples of the software tools provided include:
\begin{itemize}
\item A text editor with Fortran 77 oriented features.
\item A transforming tool that standardises the declarative part of a Fortran
program.
\item An instrumenter that modifies the program by inserting monitoring
and other control statements.
The instrumented program is then compiled and executed and data is gathered that
is used to generate reports.
Execution of an instrumented program is an example of dynamic analysis.
\end{itemize}
If you want to know more about TOOLPACK, read the `Introductory Guide'
available from your Site Manager.

\item [LIBX]\index{LIBX} \hfill [SUN/8]

Contains two tools for use with libraries.
The first outputs a list of all modules in a library and is useful for building
more elaborate utility procedures.
The second extracts preamble comments from all the modules in a Fortran
source library.

To some extent they duplicate facilities in LIBMAINT.
The latter is a very powerful system but, unavoidably, is vulnerable to
changes in the formats of the reports which the DEC Librarian utility produces.
The LIBX facilities, though limited in what they do, use only published
interfaces and should survive new VMS releases; also they are fast.

\end{description}

\newpage

\section {Subroutine Libraries}

The programs described in the previous sections can be used directly to carry
out some procedure.
This section describes items which are subroutine libraries, i.e. they are
provided for people who need to write their own programs in a high-level
language such as Fortran.

\subsection{Astronomical \& mathematical}

\rule{\textwidth}{0.5mm}
\begin{description}
\begin{description}
\item [NAG] --- Numerical mathematics \& statistics \hfill $\surd\surd$
\item [SLALIB] --- Mainly positional astronomy \hfill $\surd\surd\surd$
\item [JPL] --- Solar system ephemeris \hfill $\surd\surd$
\item [TRANSFORM] --- Coordinate transformation \hfill $\surd\surd\surd$
\item [MEMSYS] --- Maximum entropy image reconstruction \hfill $\surd\surd$
\end{description}
\end{description}
\rule{\textwidth}{0.5mm}

\begin{description}

\item [NAG]\index{NAG} \hfill [SUN/28, 29]

The Numerical Algorithms Group library is an extensive library of mathematical
subroutines.
It is commercial software.
The current version is Mark 14 for the standard library and Mark 3 for the
graphics library.
The double, single, and G\_floating precision versions of the standard library
are provided, together with the double and single precision versions of the
graphics library. 
The on-line help system is also available.

The 906 routines cover a very wide area of numerical and statistical
mathematics.
They are extensively documented in a reference manual, guide, and introductory
book.
These are available at every Starlink site, and copies of the NAG newsletters
are kept at RAL.

The standard library organises its routines into the following `chapters':
{\small
\begin{itemize}
\item A02 --- Complex arithmetic.
\item C02 --- Zeros of polynomials.
\item C05 --- Roots of one or more transcendental equations.
\item C06 --- Summation of series.
\item D01 --- Quadrature
\item D02 --- Ordinary differential equations.
\item D03 --- Partial differential equations.
\item D04 --- Numerical differentiation.
\item D05 --- Integral equations.
\item E01 --- Interpolation.
\item E02 --- Curve and surface fitting.
\item E04 --- Minimizing or maximizing a function.
\item F01 --- Matrix operations, including inversion.
\item F02 --- Eigenvalues and eigenvectors.
\item F03 --- Determinants.
\item F04 --- Simultaneous linear equations.
\item F05 --- Orthogonalisation.
\item F06 --- Linear algebra support routines.
\item G01 --- Simple calculations and statistical data.
\item G02 --- Correlation and regression analysis.
\item G03 --- Multivariate methods.
\item G04 --- Analysis of variance.
\item G05 --- Random number generators.
\item G07 --- Univariate estimation.
\item G08 --- Nonparametric statistics.
\item G11 --- Contingency table analysis.
\item G13 --- Time series analysis.
\item H --- Operations research.
\item M01 --- Sorting.
\item P01 --- Error trapping.
\item S --- Approximations of special functions.
\item X01 --- Mathematical constants.
\item X02 --- Machine constants.
\item X03 --- Innerproducts.
\item X04 --- Input/output utilities.
\item X05 --- Date and time utilities.
\end{itemize}
}

\item [SLALIB]\index{SLALIB} \hfill [SUN/67]

The {\it Subprogram Library A} is a collection of subroutines designed for use
in certain types of astronomical applications software.
Most routines are concerned with astronomical position and time, but some have
wider trigonometrical, numerical, or general applications, and a few are
essentially miscellaneous.
They are well supported.
The main functions provided are:
\begin{itemize}
\item String decoding.
\item Sexagesimal and FK4/5 conversions.
\item Angles, vectors, and rotation matrices.
\item Calendars and time.
\item Precession and nutation.
\item Proper motion.
\item Elliptic aberration.
\item Geocentric, ecliptic, galactic, and supergalactic coordinates.
\item Apparent and observed place.
\item Refraction and air mass.
\item Ephemerides.
\item Astrometry.
\item Numerical methods.
\item Real time.
\end{itemize}

\item [JPL]\index{JPL} \hfill [SUN/87]

The definitive DE200/LE200 solar system ephemeris package from JPL, comprising
files of ephemeris data for 1800--2050, plus Fortran programs
to read and interpolate them.
Ephemerides are provided for the Sun, Moon, and planets, the Solar System
Barycentre, and the Earth-Moon Barycentre.
The ephemeris is described in detail on pages S26-8 of the 1984 Astronomical
Almanac.

\item [TRANSFORM]\index{TRANSFORM} \hfill [SUN/61]

Enables programs to process information describing the relationships between
different coordinate systems. 
It provides a standard, flexible method for manipulating coordinate
transformations and for transferring information about them between programs.
It can handle coordinate systems with any number of dimensions and can
efficiently process large (i.e.\ image-sized) arrays of coordinate data using
a variety of numerical precisions. 
No specific support for astronomical coordinate transformations or map
projections is included at present, but routines for handling these will
appear in future.
The current system provides tools for creating a wide variety of coordinate
transformations, so it should be possible to construct some of the simpler
astronomical transformations explicitly, if required, on an interim basis. 

Some possible applications include:
\begin{itemize}
\item Defining linear and non-linear graphics coordinate systems.
\item Attaching coordinate systems to datasets (e.g.\ relating image pixels to
 sky positions).
\item Describing distortion in images and spectra.
\item Storing and applying instrumental calibration functions.
\end{itemize}

It uses HDS to store its information in standard data structures for
interchange between applications. 
These data structures may be used therefore as building blocks when
constructing larger HDS datasets, and also when designing `extensions' to
the standard Starlink NDF data structure (SGP/38). 

\item [MEMSYS]\index{MEMSYS} \hfill [SUN/117, 118]

Two versions of MEMSYS are available on Starlink: MEMSYS3 and MEMSYS5.
It is a `quantified maximum entropy image reconstruction package',
consisting of subroutines callable from Fortran programs.
It was written by J. Skilling and S. Gull of Maximum Entropy Data 
Consultants Ltd, and embodies their `Classic' Maximum Entropy 
algorithm, described in more detail in the MEMSYS3 users manual

The algorithm differs greatly from their previous MEM algorithms, in
that it is based on a fully Bayesian analysis of the image reconstruction
problem.
MEMSYS3 provides the following extra functionality:
\begin {itemize}
\item Automatic calculation of the most probable noise level in the input 
data.
\item Estimation of errors in the output reconstruction.
\item Provision for handling negative data values.
\item Provision for handling limited non-linearities in the data.
\item Poisson or Gaussian statistics for input data noise.
\end {itemize}

MEMSYS deals with `data sets' and `images'.
A {\em data set} holds information corresponding to the available experimental
data, and an {\em image} holds information corresponding to the `true' image
from which the data was generated.
It handles problems where the relationship between data and image can
described as follows: 

\begin {equation}
F_{k}=\sum_{j=1}^{M_{k}} (R_{kj}*f_{j})+n_{k}  \label {EQ:DATA}
\end {equation}

where $F_{k}$ is the $k$th data value, $M_{k}$ is the number of data values,
$f_{j}$ is the $j$th image value, $R_{kj}$ is the response of sample $F_{k}$ to
pixel $f_{j}$, and $n_{k}$ is the noise on sample $k$. 

This is a linear relationship between data and image.
It can also cope with non-linear data, so long as the data $D_{k}$ can be
expressed as follows:

\begin {equation}
D_{k} = \Phi(F_{k})
\end {equation}
 where $\Phi$ is some known function with known derivative, and $F_{k}$ is 
linearly dependent on the data. 

The image and data set are described as one dimensional arrays purely for ease 
of access within the package.
In fact, they could be of any dimensionality.
The program calling MEMSYS needs to set up the correspondence between
N-dimensional coordinates and the one dimensional coordinate used above.
For instance, for a two dimensional image with coordinates $(pix,lin)$ the
relationship may be

\begin {equation}
 j = pix + NPIX*(lin-1)
\end {equation}

where $NPIX$ is the number of pixels per line of the image.

A common example of the use of MEMSYS is to deconvolve a 2D image given a
Point Spread Function (PSF).
In this case the data is linear and the matrix $R_{kj}$ embodies the PSF.

Before you attempt to use MEMSYS you must sign a form (available from your
Site Manager) accepting certain conditions of use.

\end{description}

\newpage

\subsection{Data management}
\label{datman}

\rule{\textwidth}{0.5mm}
\begin{description}
\begin{description}
\item [NDF] --- Accessing extensible n-dimensional data format (NDF) objects
 \hfill $\surd\surd\surd$
\item [HDS] --- Hierarchical data system \hfill $\surd\surd\surd$
\item [PRIMDAT] --- Processing primitive numerical data \hfill $\surd\surd\surd$
\item [ARY] --- Accessing ARRAY data structures \hfill $\surd\surd\surd$
\item [REF] --- Handling references to HDS objects \hfill $\surd\surd\surd$
\item [CHI] --- Catalogue handling \hfill $\surd\surd$
\end{description}
\end{description}
\rule{\textwidth}{0.5mm}

\begin{description}

\item [NDF]\index{NDF} \hfill [SUN/33]

`NDF' stands for {\em Extensible N-Dimensional Data Format}. 
It is the standard Starlink format for storing data which represent
N-dimensional arrays of numbers, such as spectra, images, etc, and it
will therefore form the basis of most Starlink spectral and image-processing
applications. 
This item is a subroutine library for accessing data stored in this form from
programs written to run within the Starlink ADAM programming environment.
For programs which do not require ADAM facilities, a non-ADAM, or
`stand-alone', version of the library is also available.

The main reason for using NDF data structures as a standard method of
storing astronomical data on Starlink is to simplify the exchange of data
between separate applications packages. 
In principle, this should make it possible for you to process the same data
using any Starlink package. 
In practice, previous attempts to define a standard data format for this
purpose have met with two serious obstacles. 
First, different software authors have interpreted the meaning of data
items differently, so that although several software packages might be
capable of reading the same data files, the different packages actually
performed incompatible operations on the data. 
Secondly, many authors have found a pre-defined data format to be
too restrictive, and have simply chosen not to use it. 

The NDF data structure has, therefore, had to satisfy two apparently
contradictory requirements: 

\begin{enumerate}

\item Its interpretation should be closely defined, so that different
(usually geographically separated) programmers can write software which
processes it in consistent and mutually compatible ways. 

\item It should be very adaptable, so that it can be used to hold data
associated with a wide variety of software systems whose detailed
requirements may vary considerably. 

\end{enumerate}

The solution to this problem has been to introduce the concept of {\em
extensibility}, and to divide the NDF data structure into two parts --- a set
of {\em standard components\/} and a set of {\em extensions\/} --- each of which
individually satisfies one of these two requirements. 
The structure therefore consists of a central `core' of information, whose
interpretation is well-defined and for which general-purpose software can be
written with wide applicability, together with an arbitrary number of extensions
which may be used by specialist software but whose interpretation is not
otherwise defined. 
Those who wish to know more of the background to this philosophy can find a 
detailed discussion in SGP/38.

\item [HDS]\index{HDS} \hfill [SUN/92]

The Hierarchical Data System is a flexible system for storing and retrieving
data.
It is the basic data system in the ADAM software environment and in many other
Starlink software items, and is of great importance to the Starlink project.
Its most significant features are:
\begin{itemize}
\item It manipulates entities called {\em data objects}.
 These can be scalars or arrays of up to seven dimensions.
\item A data object has a tree-like shape whose components comprise other
 objects. The only restriction is that the leaves must be primitive data
 types.  The degenerate form of an HDS tree is a single scalar or array of a
 primitive data type.  The non-degenerate form is called a structure and
 comprises two or more components.
\item The primitive data types include the usual Fortran numeric, logical, and
 character types, together with signed and unsigned bytes and words.
\item Facilities are provided for walking around the trees.  Any branch or
 sub-branch of a tree can be treated as a data object in its own right.
\item Data objects are dynamic in the sense that programs can add, modify,
 or delete components to or from a tree.
\item Subsets of primitive arrays can be defined and manipulated as separate
 data objects, and arrays can be mapped as vectors.
\end{itemize}
More powerful data systems have been built on top of HDS in order to make
specific types of data object easier to handle.
These include ARY, for handling arrays, and NDF, for handling objects in
the standard Starlink data format.

\item [PRIMDAT]\index{PRIMDAT} \hfill [SUN/39]

A collection of Fortran functions and subroutines providing support for
`primitive data processing'. 
They do arithmetic, mathematical operations, type conversion, and
inter-comparison of any of the primitive numerical data types supported by HDS.
They provide:
\begin{itemize}
\item Processing facilities which are not normally available.
\item A uniform interface.
\item Improved portability and efficiency.
\item Facilities for processing bad data.
\item Handling of numerical errors.
\item A set of constants.
\end{itemize}
A distinction is drawn between three classes of primitive data, which differ
in their interpretation and in the algorithms best suited to processing them: 
\begin{itemize}
\item {\bf Values} (VAL facility) are scalar data which may take one
of the Starlink-defined {\em bad values} (sometimes called `magic'
values), whose presence signifies that the affected datum is undefined. 
\item {\bf Vectorised arrays} (VEC facility) are 1-dimensional arrays
of {\em values} (or arrays treated as 1-dimensional) whose elements are
processed in the way described above. 
\item {\bf Numbers} (NUM facility) are always interpreted literally
as scalar numerical quantities (i.e.\ {\em bad} values are not recognised on
input and are not explicitly generated on output). 
\end{itemize}

\item [ARY]\index{ARY} \hfill [SUN/11]

This is a set of routines for accessing Starlink ARRAY data structures built
using HDS.
Details of these structures and the design philosophy behind them can be
found in SGP/38, although familiarity with that document is not necessarily 
required in order to use the routines.

It constitutes an essential sub-component of the NDF system,
which contains routines for accessing Starlink NDF structures. 

The most likely reason for needing to use these routines directly at
present is to access ARRAY structures stored in NDF extensions.
At present, it only supports the `primitive' and `simple' forms of the
ARRAY data structure. 

The routines can be classified as follows:
\begin{itemize}
\item Access existing arrays.
\item Create and delete arrays.
\item Create and control identifiers.
\item Create placeholders.
\item Copy arrays.
\item Enquire and set array attributes.
\item Message system routines.
\item Miscellaneous.
\end{itemize}

\item [REF]\index{REF} \hfill [SUN/31]

These routines enable you to store references to HDS data objects in special HDS
reference objects and allow locators to reference objects to be obtained.
They will help make your software portable.

The following facilities are available:
\begin{itemize}
\item Create a reference object and put a reference in it.
\item Obtain a locator to an object (possibly via a reference).
\item Obtain a locator to a referenced object.
\item Create an empty reference object.
\item Put a reference into a reference object.
\item Annul a locator which may have been obtained via a reference.
\end{itemize}

The two main uses for this package are:
\begin{itemize}
\item To maintain a catalogue of HDS objects.
\item To avoid duplicating a large dataset.
\end{itemize}

\item [CHI]\index{CHI} \hfill [SUN/119]

The Catalogue Handling Interface is a set of routines for processing catalogues
and tables.

Until now, programs accessing catalogues (such as those in SCAR) have used ADC
or ad hoc Fortran routines.
However, ADC has a number of problems associated with it, and we plan to
replace it with a commercial database management system.
We also plan to access some catalogues using HDS and its derivatives.
Programs written using CHI will not be affected by these changes since they
will only affect the implementation of CHI and not the interface CHI presents
to the programmer.
Also, they will be able to access catalogues based on ADC, HDS, and any
commercial system used.
In contrast, SCAR applications can only access ADC-based catalogues. 
You should therefore use CHI applications rather than the SCAR alternatives.

CHI offers the following functions:
\begin{itemize}
\item Create and delete catalogues.
\item List available catalogues.
\item Read and write records in a catalogue.
\item Examine and manipulate catalogue parameters and fields.
\item Select records in catalogues based on various criteria.
\item Join or merge two catalogues.
\end{itemize}

\end{description}

\newpage

\subsection{Graphics}

Several graphics packages are available.
Starlink recommends GKS for low-level graphics, SGS for simple graphics, and
PGPLOT or NCAR/SNX for high-level graphics.

\rule{\textwidth}{0.5mm}
\begin{description}
\begin{description}
\item [PGPLOT] --- High-level graphics \hfill $\surd\surd\surd$
\item [GKS] --- Low-level graphics \hfill $\surd\surd\surd$
\item [SGS] --- Simple graphics \hfill $\surd\surd\surd$
\item [NCAR/SNX] --- High-level graphics \hfill $\surd$/$\surd\surd\surd$
\item [GNS] --- Graphics workstation name service \hfill $\surd\surd\surd$
\item [AGI] ---  Graphics database \hfill $\surd\surd$
\item [IDI] --- Image display interface \hfill $\surd\surd$
\item [GWM] --- X graphics window manager \hfill $\surd\surd\surd$
\end{description}
\end{description}
\rule{\textwidth}{0.5mm}

\begin{description}

\item [PGPLOT]\index{PGPLOT} \hfill [SUN/15]

A high-level graphics package for plotting {\em x,y} plots, functions,
histograms, bar charts, contour maps, and grey-scale images.
Complete diagrams can be produced with a minimal number of subroutine calls,
but control over colour, line-style, character font, etc is available if
required.
The package was written by Dr T J Pearson of the Caltech astronomy department
with astronomical applications in mind and has become a {\it de facto} standard
for graphics in astronomy world wide.

The package exists in two version: the original version which uses a low
level graphics package known as GRPCKG, also written at Caltech, and a version
developed by Starlink, in collaboration with Dr Pearson, which uses GKS.
The two versions have identical subroutine interfaces, and applications can be
moved from one version to the other simply by re-linking (if you have used the
shareable image, you can change by redefining a couple of logical names).
The GKS version is the one distributed and supported by Starlink.

A programmers manual can be obtained from your Site Manager.

\item [GKS]\index{GKS} \hfill [SUN/83, 113]

The Graphical Kernel System is a low-level graphics library which is an
international standard.
The current version is 7.2, and this is the one you should use.
An old version (6.2) is still used by some Starlink software but will eventually
be withdrawn.

It has two major advantages over other graphics packages:
\begin{itemize}                   
\item It is an international standard and real portability of graphics software
 is now possible.
\item No other package provides the ability to write such device independent
 graphics programs.
 Only with GKS can you write a program that will run, and produce good pictures,
 on all devices supported by the implementation, including devices not supported
 at the time the program was written.
 Nonetheless, programs can still fully exploit all the facilities offered by the
 hardware.
\end{itemize}
In addition, the Starlink implementation has several advantages of its own:
\begin{itemize}
\item A substantial support commitment has been made by the SERC.
The level of support available will far exceed that provided for software unique
to Starlink.
\item The implementation will be in use on several hundred computers, of many
different types, throughout the UK academic community.
This means that the software will receive a large amount of testing under a wide
variety of conditions and will quickly become a highly reliable package.
\item It contains features which are not all found in any other single package:
\begin{itemize}
\item Area fill with several fill styles on {\em all} devices.
\item Cell array supported on dot matrix printers.
\item Multiple fonts.
\item Selective clearing of display surface.
\end{itemize}
\end{itemize}
The advantages of GKS also apply to graphics software, such as SGS, that are
based on it.

\item [SGS]\index{SGS} \hfill [SUN/85, 113]

The Simple Graphics System provides a friendlier interface to a subset of GKS
facilities.
Many of its features are low-level (e.g.\ draw a line), but there are some
routines of a slightly higher level (drawing arcs, formatted numbers, etc). 
It does not include routines for high-level operations like drawing annotated
axes or complete graphs.
Many plotting programs can be written entirely with SGS or higher-level calls.
There will, however, be occasions when GKS routines are used as well; usually
because a specialised feature of GKS is needed.

The basic functions provided are:
\begin{itemize}
\item Control of workstations and zones.
\item Plotting lines, text, and markers.
\item Control of character attributes.
\item Graphical input.
\item GKS inquires.
\end{itemize}

\item [NCAR/SNX]\index{NCAR}\index{SNX} \hfill [SUN/88, 90]

NCAR is an extensive suite of high-level graphics utilities obtained from the
National Center for Atmospheric Research in Boulder, Colorado.
SNX contains some Starlink extensions to NCAR.
The plots provided are:
\begin{itemize}
\item Annotated curves or families of curves.
\item Contours for one- and two-dimensional data, with labels.
\item Dashed lines with labels.
\item Continental, national, and US state boundaries in nine map projections.
\item Graph paper, axes, etc.
\item Halftone (greyscale) pictures from two dimensional arrays.
\item Histograms.
\item Iso-value surfaces (with hidden lines removed) from three-dimensional
 arrays.
\item Software characters.                      
\item Three-dimensional displays of a surface (with hidden lines removed)
 from two-dimensional arrays.
\item Vector flows of fields for which planar vector components are given on
 regular rectangular lattices.
\item Lines in three space.
\item Two-dimensional velocity fields.
\end{itemize}

\item [GNS]\index{GNS} \hfill [SUN/57]

Almost any program that does graphics requires you to identify a graphics
device to use.
When the graphics package used by the program is GKS, graphics devices are
identified by two integers: a `workstation type' and a `connection identifier'.
(`Workstation' is GKS terminology for a graphics device of any description.)
No one can be expected to remember the workstation types of all the devices
supported by Starlink (more than 60 at present), so this package has been
provided to translate `friendly' and easy to remember names into their GKS
equivalents.

Most high level graphics packages, such as SGS and PGPLOT, call GNS to do the
required name translation when a workstation is opened, so unless you are
writing programs that open GKS workstations directly (by calling {\tt GOPWK}),
or need to make specialised device inquiries, you will not need to call
GNS routines yourself.

It provides three sorts of service --- routines:
\begin{itemize}
\item Translate workstation names to their GKS equivalents.
\item Generate a list of the names and types of available graphics devices.
\item Answer a variety of enquiries about the properties of a particular
graphics device; for example, its category (pen plotter, image display,
etc) or its VMS device name.
\end{itemize}
It is used in IDI and AGI, but in both cases its use is internal so its
presence is not normally apparent to the programmer.

\item [AGI]\index{AGI} \hfill [SUN/48]

A graphics database system.
It is intended for use within the ADAM environment to store, between
applications, information about the various plots on a graphics device.
When called upon, it will record in the database the position and extent of a
plot.
These can be recalled later either to overlay one plot with another, or to
recreate the picture.

Its obvious uses are to align one picture with another, or to obtain cursor
coordinates from an existing picture.
But it can be used in other ways as well, e.g.\ to slice up the graphics
screen into areas for subsequent plotting using more complicated patterns than
just a regular grid; the areas could be different sizes or overlapping.

\item [IDI]\index{IDI} \hfill [SUN/65]

This is an international standard for displaying astronomical data on an
image display.
It is intended for programs that need to manipulate images to a greater extent
than can be done with GKS and its offspring.
Thus, it does not supersede GKS, but offers features not available in GKS.
It is not as good as GKS for producing line plots or character annotation ---
it does have routines to draw lines and plot text, but these are primitive and
offer you little control over how the result will appear in terms of character
size, style of line width, etc.

Its major strength is its ability to perform many types of interaction using the
mouse.
Like GKS, it can display an image, move the cursor, and rotate the look-up
table.
However, it can also scroll and zoom, blink the memories, and read back a
representation of the whole display, which can then be used to obtain a
hardcopy.
It allows these functions to be programmed in a device independent way, so
a program can use any device for which IDI has been implemented.

It is not possible, at present, to mix GKS and IDI calls on the same display
since the two packages use completely different display models.
However, it is possible to run the packages one after the other.
For example, IDI could follow GKS and the display could be opened without
resetting it.
However, this is an undesirable approach since the results could be
unpredictable.
A better solution is to use AGI when its interface to IDI becomes available.

\item [GWM]\index{GWM} \hfill [SUN/130]

These routines help you create graphics windows on an X display that do not
disappear when a program terminates.

X was designed to support `graphical user interfaces' (GUIs), and there are two
properties of X that make an X window fundamentally different from a
traditional graphics device.
Firstly, windows `belong' to application programs and are deleted when the
application exits, unlike the picture on an image display which remains there
until replaced by something else. 
Secondly, application programs are expected respond to `events' occurring in
their windows; things like key presses, mouse movements, etc.
One event type that all applications must handle is the `window expose' event
which occurs whenever part of a window that was invisible --- because another
application's window was obscuring it for example --- becomes visible.
The application is responsible for restoring the contents of the newly exposed
part of the window, but only an application designed as an X application can do
this; an application that is using X via a graphics package, such as GKS or IDI,
but is otherwise a conventional application that knows nothing of X, cannot.

The X graphics window manager\footnote{Not to be confused with the Window
Manager which allows you to move windows around the screen, iconize them etc.}
makes a window on an X windows display behave like a traditional graphics
device by making the lifetime of the window independent of any applications
program, and by handling window expose events.
Applications still send plotting commands directly to the window; they don't
have to go via a `server' process, so there is no adverse impact on
performance.
All communication between the window manager and the application is via the X
server, and the graphics window manager does not have to run on the same 
machine as the application.

\end{description}

\newpage

\subsection{Other}

\rule{\textwidth}{0.5mm}
\begin{description}
\begin{description}
\item [TAPEIO] --- Magnetic tape handling \hfill $\surd$
\item [CHR] --- Character handling \hfill $\surd\surd\surd$
\item [EMS] --- Error message service \hfill $\surd\surd\surd$
\item [HELP] --- Interactive Help system \hfill $\surd\surd\surd$
\item [CNF] --- C and Fortran mixed programming \hfill $\surd\surd\surd$
\end{description}
\end{description}
\rule{\textwidth}{0.5mm}

\begin{description}

\item [TAPEIO]\index{TAPEIO} \hfill [SUN/21]

A set of routines to perform physical I/O operations on tapes.
They are useful for handling non-VAX-standard magnetic tapes, and provide an
easier interface for Fortran programmers than calling system services directly.
The following operations are available:
\begin{itemize}
\item Assign/deassign an I/O channel to a device.
\item Set the recording density.
\item Mount/dismount a tape.
\item Read/write a physical block and an end-of-file mark.
\item Skip forward or backward a given number of tape marks or blocks.
\item Reposition a tape to the BOT marker.
\item Test for a physical end-of-file or error condition.
\item Return message text for a status value.
\item Get information about a tape drive.
\end{itemize}

\item [CHR]\index{CHR} \hfill [SUN/40]

A set of character handling functions which will assist the production of
portable programs.
There are four classes of routine:
\begin{itemize}
\item Decode character strings into numbers of various types.
\item Encode and format numbers of various types into character strings.
\item Inquire about character strings.
\item String manipulation.
\end{itemize}

\item [EMS]\index{EMS} \hfill [SUN/104]

There is a general need for Starlink application programs to provide the
user with information about:

\begin {itemize}
\item What they do --- for example, during long operations it is helpful if the
user is kept informed about what is going on.

\item What results have been obtained --- for example, the notification of  the
final results from a procedure, or of some intermediate results that would help
the user respond to further prompts. 

\item What errors have occurred --- for example, errors which lead to the user
being prompted to provide more sensible input to a program, or fatal
errors which cause an application to stop.
\end {itemize}

EMS comprises two subroutine libraries to enable a program to do these jobs.
They are:

\begin {quote}
\begin {itemize}
\item [MSG]\index{MSG} --- Message Reporting System, used for reporting
non-error information. 
\item [ERR]\index{ERR} --- Error Reporting System, used for reporting error
messages.
\end {itemize}
\end {quote}
They can be used as part of the ADAM system, or in stand-alone programs.

\item [HELP]\index{HELP} \hfill [SUN/124]

A set of routines and utilities which allows a program to retrieve named
items from a hierarchically-arranged library of text.
Functionally, it is very similar to the VAX/VMS Help system.
The major differences are that the Starlink Help system:
\begin{itemize}
\item Is implemented in a portable way and is not tied to the VAX.
\item Allows independent creation of multiple libraries which are bound
together at run-time and appear to the user as a single tree.
\end{itemize}
It is written in a free-standing manner and does not call any other Starlink
packages.

\item [CNF]\index{CNF} \hfill [SGP/5]

This is a facility that lets a programmer mix program segments that are written
in FORTRAN and C in a portable manner.
It provides C macros to hide the different ways that computers pass information
between subprograms and C functions to handle FORTRAN character strings.

\end{description}

\newpage

\subsection{ARGS support}

\rule{\textwidth}{0.5mm}
\begin{description}
\begin{description}
\item [ARGSLIB] --- ARGS manipulation \hfill $\times\times$
\item [ARGSMAC] --- ARGS programming system \hfill $\times$
\end{description}
\end{description}
\rule{\textwidth}{0.5mm}

The use of the ARGS image display device is being phased out in favour of the
IKON device.
However, some old Starlink programs (those in ASPIC for example) still use the
ARGS and the items described below will be retained until the use of the
ARGS ceases.

\begin{description}

\item [ARGSLIB]\index{ARGSLIB} \hfill [SUN/10]

Controls the ARGS at a low level.

\item [ARGSMAC]\index{ARGSMAC} \hfill [SSN/12]

A set of tools for writing programs to be executed by the ARGS graphics
processor.
\end{description}

\appendix
\newpage

\section{Software Support Levels}

Support of a software item means:
\begin{itemize}
\item Providing up-to-date, accurate and helpful documentation.
\item Providing a help service to which users can turn if they have problems.
\item Accepting and responding to bug reports.
This can either mean making a correction to the software or indicating a better
way of using it.
\item Developing the item by providing new or enhanced facilities.
\end{itemize}
The hundred or so items in the Collection have widely differing levels of
support and have reached different stages in their life cycle.
Some, like ASPIC, are obsolete and unsupported and are likely to be withdrawn.
Others, like ADAM, are well supported and are being actively developed.
Some items, like NAG, are commercially supported.

The table on the next page shows the support level at the time of writing for
each item in the Collection.
The level often depends on a single person who may leave, thus support levels
may change.
Also, many new items are released every year and old items are withdrawn.
You should, therefore, check the current status of any item in which you are
interested by looking at file ADMINDIR:SUPPORT.LIS.

The codes for `Level' have the following meaning:
\begin{center}
\begin{tabular}{||r|l||}
\hline
Level & Meaning \\
\hline
\hline
$\surd\surd\surd$ & Unconditional support, by Starlink staff \\
                  & at short notice.  Compatibility with other \\
                  & software will be maintained. \\
\hline
$\surd\surd$      & High level of support, but with \\
                  & reservations based on geographical location, \\
                  & response time, or control factors. \\
                  & (This category includes commercial software.) \\
\hline
$\surd$           & Best efforts, by a named individual. \\
\hline
$\times$          & Little or no support available. \\
\hline
$\times\times$    & Obsolescent.\\
\hline
\end{tabular}
\end{center}
You should regard items with support levels indicated by crosses ($\times$ or
$\times\times$) as being unreliable, and you should avoid them if possible.
Items marked with a single tick ($\surd$) are potentially unreliable and
should be used with discretion.

There is a difference between the support level of a software item and its
importance.
It is possible for an item to be important but not supported because of
the unavailability of key people or due to a lack of resources.
Obviously, Starlink will do its best to support important items but this may
not, in fact, be possible.
Also, remember that support levels can change.

\newpage
\twocolumn
\small
\begin{center}
\begin{tabular}{||l|r||}
\hline
 Item & Level \\
\hline
\hline
AATGS     & $\surd\surd\surd$ \\
ADAM      & $\surd\surd\surd$ \\
AGI       & $\surd\surd$ \\
AIPS      & $\times$ \\
APIG      & $\surd$ \\
APLATE    & $\surd$ \\
ARGSLIB   & $\times\times$ \\
ARGSMAC   & $\times$\\
ARY       & $\surd\surd\surd$ \\
ASPIC     & $\times$ \\
ASTERIX   & $\surd\surd$ \\
ASTROM    & $\surd\surd\surd$ \\
ASURV     & $\surd$ \\
CDCOPY    & $\surd$ \\
CHART     & $\times$ \\
CHI       & $\surd\surd$ \\
CHIAPP    & $\surd\surd$ \\
CHR       & $\surd\surd\surd$ \\
CLUSTAN   & $\surd\surd$ \\
CNF       & $\surd\surd\surd$ \\
COCO      & $\surd\surd\surd$ \\
CONVERT   & $\surd\surd\surd$ \\
DAOPHOT   & $\surd\surd$ \\
DIPSO     & $\surd$ \\
DOCFIND   & $\surd$ \\
DSCL      & $\times\times$ \\
ECHWIND   & $\surd$ \\
EDFITS    & $\surd\surd$ \\
EMS       & $\surd\surd\surd$ \\
FIGARO    & $\surd\surd$ \\
FORCHECK  & $\surd\surd$ \\
FORMCON   & $\times\times$ \\
GENERIC   & $\surd\surd$ \\
GENSTAT   & $\surd\surd$ \\
GEROFF    & $\times\times$ \\
GKS       & $\surd\surd\surd$ \\
GKS6      & $\times\times$ \\
GNS       & $\surd\surd\surd$ \\
GWM       & $\surd\surd\surd$ \\
HDS       & $\surd\surd\surd$ \\
HELP      & $\surd\surd\surd$ \\
HONEY     & $\surd\surd\surd$ \\
HXIS      & $\surd$ \\
IDI       & $\surd\surd$ \\
IDL       & $\surd\surd$ \\
IKONPAINT & $\surd$ \\
INTERIM   & $\times\times$ \\
IRAF      & $\times$ \\
IRAS      & $\surd\surd$ \\
IRCAM     & $\times$ \\
IUEDEARCH & $\surd\surd$ \\
IUEDR     & $\surd$ \\
\hline
\end{tabular}
\end{center}

\newpage

\begin{center}
\begin{tabular}{||l|r||}
\hline
Item & Level \\
\hline
\hline
JPL       & $\surd\surd$ \\
KAPPA     & $\surd\surd\surd$ \\
LIBMAINT  & $\surd\surd$ \\
LIBX      & $\surd\surd\surd$ \\
LZCMP     & $\surd\surd\surd$ \\
MAPLE     & $\surd\surd$ \\
MEMSYS    & $\surd\surd$ \\
MIDAS     & $\times$ \\
MONGO     & $\surd$ \\
NAG       & $\surd\surd$ \\
NCAR      & $\surd$ \\
NDF       & $\surd\surd\surd$ \\
NETWORK   & $\surd\surd\surd$ \\
NEWSMAINT & $\surd\surd$ \\
NOCBS     & $\surd\surd\surd$ \\
NOD2      & $\times$ \\
PAD\_AUDIT & $\surd\surd\surd$ \\
PGPLOT    & $\surd\surd\surd$ \\
PHOTOM    & $\surd\surd$ \\
PISA      & $\surd$ \\
PRIMDAT   & $\surd\surd\surd$ \\
PSSMB     & $\surd$ \\
QDP       & $\times$ \\
QUOTAS    & $\surd\surd\surd$ \\
REF       & $\surd\surd\surd$ \\
REXEC     & $\surd$ \\
RGASP     & $\surd$ \\
RPS       & $\surd$ \\
RUNSTAR   & $\times\times$ \\
RV        & $\surd\surd\surd$ \\
SCAR      & $\surd$ \\
SGS       & $\surd\surd\surd$ \\
SLALIB    & $\surd\surd\surd$ \\
SNX       & $\surd\surd\surd$ \\
SPAG      & $\surd\surd$ \\
SPECX     & $\surd$ \\
SST       & $\surd\surd\surd$ \\
STARCON   & $\surd\surd\surd$ \\
STARLSE   & $\surd\surd\surd$ \\
TAPECOPY  & $\surd\surd\surd$ \\
TAPEIO    & $\surd$ \\
TEX       & $\surd$ \\
TOOLPACK  & $\surd\surd$ \\
TPOINT    & $\surd\surd\surd$ \\
TPU       & $\surd\surd$ \\
TRACE     & $\surd\surd\surd$ \\
TRANSFORM & $\surd\surd\surd$ \\
TSP       & $\surd$ \\
USSP      & $\surd\surd$ \\
VSHC      & $\surd$ \\
WFC\_SHORT & $\surd\surd$ \\
XDISPLAY  & $\surd\surd\surd$ \\
\hline
\end{tabular}
\end{center}

\newpage
\onecolumn
\normalsize

\section{Functional Classification}

The next five pages present a functional classification of the items in the
Collection in a format designed for quick reference.
A summary of the structure is shown below.

\vspace{5mm}

\small

\begin{quote}
\begin{description}
\item [ADAM APPLICATIONS] \mbox{}

\begin{description}
\item [Image Analysis \& Photometry] \mbox{}
\item [Spectroscopy] \mbox{}
\item [Specific Wavelengths] \mbox{}
\item [Specific Instruments] \mbox{}
\item [Polarimetry] \mbox{}
\item [Database Management] \mbox{}
\item [Utilities] \mbox{}
\end{description}

\item [STAND-ALONE APPLICATIONS] \mbox{}

\begin{description}
\item [Image Analysis \& Photometry] \mbox{}
\item [Spectroscopy] \mbox{}
\item [Specific Wavelengths] \mbox{}
\item [Specific Instruments] \mbox{}
\item [Statistics] \mbox{}
\item [Database Management] \mbox{}
\end{description}

\item [INTERIM APPLICATIONS] \mbox{}

\begin{description}
\item [Specific Instruments] \mbox{}
\item [General Purpose] \mbox{}
\end{description}

\item [FOREIGN APPLICATIONS] \mbox{}

\begin{description}
\item [Specific Wavelengths] \mbox{}
\item [Specific Instruments] \mbox{}
\item [General Purpose] \mbox{}
\end{description}

\item [ASTRONOMICAL UTILITIES] \mbox{}

\begin{description}
\item [Archive Access] \mbox{}
\item [Data Copying \& Format Conversion] \mbox{}
\item [Observation Preparation and Positional Astronomy] \mbox{}
\end{description}

\item [GENERAL UTILITIES] \mbox{}

\begin{description}
\item [Document Preparation \& Search] \mbox{}
\item [Graphics] \mbox{}
\item [Device Handling, Data Compression \& Examination] \mbox{}
\item [Mathematical] \mbox{}
\item [Operational] \mbox{}
\item [Programming Support] \mbox{}
\end{description}

\item [SUBROUTINE LIBRARIES] \mbox{}

\begin{description}
\item [ARGS Support] \mbox{}
\item [Astronomical \& Mathematical] \mbox{}
\item [Data Management] \mbox{}
\item [Graphics] \mbox{}
\item [Other] \mbox{}
\end{description}

\end{description}
\end{quote}

\newpage
\normalsize

\begin{center}
{\LARGE\bf THE STARLINK SOFTWARE COLLECTION}
\end{center}


\begin{center}
{\bf\Large ADAM APPLICATIONS}
\end{center}

\begin{description}

\item [Image Analysis \& Photometry] \mbox{}
\begin{description}
\item [KAPPA] --- Kernel applications
\item [DAOPHOT] --- Stellar photometry
\item [PHOTOM] --- Aperture photometry
\item [PISA] --- Object finding and analysis
\end{description}

\item [Spectroscopy] \mbox{}
\begin{description}
\item [FIGARO] --- General data analysis
\end{description}

\item [Specific Wavelengths] \mbox{}
\begin{description}
\item [ASTERIX] --- X-ray data analysis
\end{description}

\item [Specific Instruments] \mbox{}
\begin{description}
\item [IRCAM] --- UKIRT (Infra-red)
\end{description}

\item [Polarimetry] \mbox{}
\begin{description}
\item [TSP] --- Time series and polarimetry analysis
\end{description}

\item [Database Management] \mbox{}
\begin{description}
\item [SCAR] --- Catalogue data base system
\item [CHIAPP] --- Catalogue/table data base system
\end{description}

\item [Utilities] \mbox{}
\begin{description}
\item [CONVERT] --- Data format conversion (DIPSO/FIGARO/BDF to NDF)
\item [TRACE] --- HDS data file listing
\item [SST] --- Simple Software Tools package
\end{description}

\end{description}

\newpage

\begin{center}
{\bf\Large STAND-ALONE APPLICATIONS}
\end{center}

\begin{description}

\item [Image Analysis \& Photometry] \mbox{}
\begin{description}
\item [RGASP] --- Galaxy surface photometry
\end{description}

\item [Spectroscopy] \mbox{}
\begin{description}
\item [DIPSO] --- Spectral analysis and plotting
\item [APIG] --- Absorption profiles in the interstellar gas
\end{description}

\item [Specific Wavelengths] \mbox{}
\begin{description}
\item [SPECX] ---  Mm-wave data analysis
\item [NOD2] --- Radio astronomy data analysis
\end{description}

\item [Specific Instruments] \mbox{}
\begin{description}
\item [IUEDR] --- IUE (Ultra-violet)
\item [HXIS] --- SMM (X-ray)
\end{description}

\item [Statistics] \mbox{}
\begin{description}
\item [GENSTAT] --- General statistical analysis
\item [CLUSTAN] --- Cluster analysis (Restricted to 10 sites)
\item [ASURV] --- Statistical analysis of data with upper limits
\end{description}

\item [Database Management] \mbox{}
\begin{description}
\item [REXEC] --- Relational database management system
\end{description}

\end{description}

\begin{center}
{\bf\Large INTERIM APPLICATIONS}
\end{center}

\begin{description}

\item [Specific Instruments] \mbox{}
\begin{description}
\item [IRAS] --- IRAS (Infra-red)
\end{description}

\item [General Purpose] \mbox{}
\begin{description}
\item [ASPIC] --- General data analysis
\end{description}

\end{description}

\begin{center}
{\bf\Large FOREIGN APPLICATIONS}
\end{center}

\begin{description}

\item [Specific Wavelengths] \mbox{}
\begin{description}
\item [AIPS] --- Radio astronomy data analysis
\end{description}

\item [Specific Instruments] \mbox{}
\begin{description}
\item [IRAF] --- HST (multi-wavelength)
\end{description}

\item [General Purpose] \mbox{}
\begin{description}
\item [IDL] --- General data analysis (US Commercial package)
\item [MIDAS] --- General data analysis (ESO)
\end{description}

\end{description}

\newpage

\begin{center}
{\bf\Large ASTRONOMICAL UTILITIES}
\end{center}

\vspace{10mm}

\begin{description}

\item [Archive Access] \mbox{}
\begin{description}
\item [IUEDEARCH] --- Access IUE archive data
\item [USSP] --- Access IUE uniform low-dispersion archive (ULDA), (STADAT only)
\end{description}

\item [Data Copying \& Format Conversion] \mbox{}
\begin{description}
\item [STARCON] --- Data format conversion (BDF to HDS)
\item [EDFITS] --- Copy FITS tapes
\item [WFC\_SORT] --- Data format conversion (ROSAT-WFC to ASTERIX (HDS))
\item [FORMCON] --- Data format conversion (IPCS/VICAR to BDF)
\end{description}

\item [Observation Preparation and Positional Astronomy] \mbox{}
\begin{description}
\item [COCO] --- Celestial coordinate conversion
\item [CHART] --- Finding chart and stellar data system
\item [ASTROM] --- Basic astrometry
\item [RV] --- Calculate radial components of observer's velocity
\item [RPS] --- Submit Rosat proposals
\item [ECHWIND] --- Plan use of UCL echelle spectrograph
\item [TPOINT] --- Telescope pointing analysis
\item [APLATE] --- Aperture plate preparation
\item [AATGS] --- Guide probe predictions for AAT
\end{description}

\end{description}

\newpage

\begin{center}
{\bf\Large GENERAL UTILITIES}
\end{center}

\vspace{10mm}

\small

\begin{description}

\item [Document Preparation \& Search] \mbox{}
\begin{description}
\item [TEX] --- Document preparation (includes \LaTeX)
\item [DOCFIND] --- Starlink document search
\item [GEROFF] --- Document preparation
\end{description}

\item [Graphics] \mbox{}
\begin{description}
\item [MONGO] --- Interactive plotting
\item [IKONPAINT] --- Ikon to inkjet hard-copy
\item [QDP] --- Quick and Dandy Plotter
\item [VSHC] --- Obtain hardcopy from the display of a VAXstation
\item [HONEY] --- Hardcopy output from the Honeywell camera
\end{description}

\item [Device Handling, Data Compression \& Examination] \mbox{}
\begin{description}
\item [TPU] --- Magnetic tape handling
\item [LZCMP] --- File compression and de-compression
\item [TAPECOPY] --- Copy magnetic tapes
\item [CDCOPY] --- CDROM reading
\item [XDISPLAY] --- Xwindows display setup
\end{description}

\item [Mathematical] \mbox{}
\begin{description}
\item [MAPLE] --- Mathematical manipulation language, (STADAT only)
\end{description}

\item [Operational] \mbox{}
\begin{description}
\item [NETWORK] --- DECNET utilities
\item [QUOTAS] --- Other users' disk quotas
\item [NEWSMAINT] --- News maintenance
\item [NOCBS] --- Error reporting for CBS on satellites
\item [PAD\_AUDIT] --- PAD auditing
\item [PSSMB ] --- Postscript print symbiont
\end{description}

\item [Programming Support] \mbox{}
\begin{description}
\item [SPAG] --- Improve structure of Fortran source code, (STADAT only)
\item [LIBMAINT] --- Library maintenance
\item [FORCHECK] --- Fortran verifier and programming aid, (STADAT only)
\item [STARLSE] --- Starlink language sensitive editor
\item [GENERIC] --- Compile generic Fortran routines
\item [TOOLPACK] --- Fortran Software Tools package
\item [LIBX] --- Library maintenance
\end{description}

\end{description}

\newpage

\begin{center}
{\bf\Large SUBROUTINE LIBRARIES}
\end{center}
\normalsize

\begin{description}

\item [Astronomical \& Mathematical] \mbox{}
\begin{description}
\item [NAG] --- Numerical mathematics \& statistics
\item [SLALIB] --- Mainly positional astronomy
\item [JPL] --- Solar system ephemeris
\item [TRANSFORM] --- Coordinate transformation
\item [MEMSYS] --- Maximum entropy image reconstruction
\end{description}

\item [Data Management] \mbox{}
\begin{description}
\item [NDF] --- Accessing extensible n-dimensional data format (NDF) objects
\item [HDS] --- Hierarchical data system
\item [PRIMDAT] --- Processing primitive numerical data
\item [ARY] --- Accessing ARRAY data structures
\item [REF] --- Handling references to HDS objects
\item [CHI] --- Catalogue handling
\end{description}

\item [Graphics] \mbox{}
\begin{description}
\item [PGPLOT] --- High-level graphics
\item [GKS] --- Low-level graphics
\item [SGS] --- Simple graphics
\item [NCAR/SNX] --- High-level graphics
\item [GNS] --- Graphics workstation name service
\item [AGI] ---  Graphics database
\item [IDI] --- Image display interface
\item [GWM] --- X graphics window manager
\end{description}

\item [Other] \mbox{}
\begin{description}
\item [TAPEIO] --- Magnetic tape handling
\item [CHR] --- Character handling
\item [EMS] --- Error message service
\item [HELP] --- Interactive HELP system
\item [CNF] --- C and Fortran mixed programming
\end{description}

\item [ARGS Support] \mbox{}
\begin{description}
\item [ARGSLIB] --- ARGS manipulation
\item [ARGSMAC] --- ARGS programming system
\end{description}
\end{description}

\newpage

\include{sun1index}

\end{document}
