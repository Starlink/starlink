\documentclass[twoside,11pt]{article}

% ? Specify used packages
% \usepackage{graphicx}        %  Use this one for final production.
% \usepackage[draft]{graphicx} %  Use this one for drafting.
% ? End of specify used packages

\pagestyle{myheadings}

% -----------------------------------------------------------------------------
% ? Document identification
\newcommand{\stardoccategory}   {Starlink User Note}
\newcommand{\stardocinitials}   {SUN}
\newcommand{\stardocsource}     {sun\stardocnumber}
\newcommand{\stardocnumber}     {168.3}
\newcommand{\stardocauthors}    {B.\ K.\ McIlwrath\\C.\ G.\ Page}
\newcommand{\stardocversion}    {Version 2.0}
\newcommand{\stardocdate}       {27 November 1998}
\newcommand{\stardoctitle}      {EDT mode for JED\\[2ex]
                                An advanced Unix text editor}
\newcommand{\stardocmanual}    {User's Guide}
\newcommand{\stardocabstract}  {This note describes Starlink extended EDT
emulation for the JED editor. It provides a Unix text editor which can
utilise the advanced facilities of DEC VTn00, xterm and similar terminals.
JED in this mode provides a reasonably good emulation of the VAX/VMS editor
EDT in addition to many extra facilities.}
% ? End of document identification
% -----------------------------------------------------------------------------

% +
%  Name:
%     sun.tex
%
%  Purpose:
%     Template for Starlink User Note (SUN) documents.
%     Refer to SUN/199
%
%  Authors:
%     AJC: A.J.Chipperfield (Starlink, RAL)
%     BLY: M.J.Bly (Starlink, RAL)
%
%  History:
%     17-JAN-1996 (AJC):
%        Original with hypertext macros, based on MDL plain originals.
%     16-JUN-1997 (BLY):
%        Adapted for LaTeX2e.
%        Added picture commands.
%     {Add further history here}
%
% -

\newcommand{\stardocname}{\stardocinitials /\stardocnumber}
\markboth{\stardocname}{\stardocname}
\setlength{\textwidth}{160mm}
\setlength{\textheight}{230mm}
\setlength{\topmargin}{-2mm}
\setlength{\oddsidemargin}{0mm}
\setlength{\evensidemargin}{0mm}
\setlength{\parindent}{0mm}
\setlength{\parskip}{\medskipamount}
\setlength{\unitlength}{1mm}

% -----------------------------------------------------------------------------
%  Hypertext definitions.
%  ======================
%  These are used by the LaTeX2HTML translator in conjunction with star2html.

%  Comment.sty: version 2.0, 19 June 1992
%  Selectively in/exclude pieces of text.
%
%  Author
%    Victor Eijkhout                                      <eijkhout@cs.utk.edu>
%    Department of Computer Science
%    University Tennessee at Knoxville
%    104 Ayres Hall
%    Knoxville, TN 37996
%    USA

%  Do not remove the %begin{latexonly} and %end{latexonly} lines (used by
%  star2html to signify raw TeX that latex2html cannot process).
%begin{latexonly}
\makeatletter
\def\makeinnocent#1{\catcode`#1=12 }
\def\csarg#1#2{\expandafter#1\csname#2\endcsname}

\def\ThrowAwayComment#1{\begingroup
    \def\CurrentComment{#1}%
    \let\do\makeinnocent \dospecials
    \makeinnocent\^^L% and whatever other special cases
    \endlinechar`\^^M \catcode`\^^M=12 \xComment}
{\catcode`\^^M=12 \endlinechar=-1 %
 \gdef\xComment#1^^M{\def\test{#1}
      \csarg\ifx{PlainEnd\CurrentComment Test}\test
          \let\html@next\endgroup
      \else \csarg\ifx{LaLaEnd\CurrentComment Test}\test
            \edef\html@next{\endgroup\noexpand\end{\CurrentComment}}
      \else \let\html@next\xComment
      \fi \fi \html@next}
}
\makeatother

\def\includecomment
 #1{\expandafter\def\csname#1\endcsname{}%
    \expandafter\def\csname end#1\endcsname{}}
\def\excludecomment
 #1{\expandafter\def\csname#1\endcsname{\ThrowAwayComment{#1}}%
    {\escapechar=-1\relax
     \csarg\xdef{PlainEnd#1Test}{\string\\end#1}%
     \csarg\xdef{LaLaEnd#1Test}{\string\\end\string\{#1\string\}}%
    }}

%  Define environments that ignore their contents.
\excludecomment{comment}
\excludecomment{rawhtml}
\excludecomment{htmlonly}

%  Hypertext commands etc. This is a condensed version of the html.sty
%  file supplied with LaTeX2HTML by: Nikos Drakos <nikos@cbl.leeds.ac.uk> &
%  Jelle van Zeijl <jvzeijl@isou17.estec.esa.nl>. The LaTeX2HTML documentation
%  should be consulted about all commands (and the environments defined above)
%  except \xref and \xlabel which are Starlink specific.

\newcommand{\htmladdnormallinkfoot}[2]{#1\footnote{#2}}
\newcommand{\htmladdnormallink}[2]{#1}
\newcommand{\htmladdimg}[1]{}
\newenvironment{latexonly}{}{}
\newcommand{\hyperref}[4]{#2\ref{#4}#3}
\newcommand{\htmlref}[2]{#1}
\newcommand{\htmlimage}[1]{}
\newcommand{\htmladdtonavigation}[1]{}

% Define commands for HTML-only or LaTeX-only text.
\newcommand{\html}[1]{}
\newcommand{\latex}[1]{#1}

% Use latex2html 98.2.
\newcommand{\latexhtml}[2]{#1}

%  Starlink cross-references and labels.
\newcommand{\xref}[3]{#1}
\newcommand{\xlabel}[1]{}

%  LaTeX2HTML symbol.
\newcommand{\latextohtml}{{\bf LaTeX}{2}{\tt{HTML}}}

%  Define command to re-centre underscore for Latex and leave as normal
%  for HTML (severe problems with \_ in tabbing environments and \_\_
%  generally otherwise).
\newcommand{\setunderscore}{\renewcommand{\_}{{\tt\symbol{95}}}}
\latex{\setunderscore}

% -----------------------------------------------------------------------------
%  Debugging.
%  =========
%  Remove % from the following to debug links in the HTML version using Latex.

% \newcommand{\hotlink}[2]{\fbox{\begin{tabular}[t]{@{}c@{}}#1\\\hline{\footnotesize #2}\end{tabular}}}
% \renewcommand{\htmladdnormallinkfoot}[2]{\hotlink{#1}{#2}}
% \renewcommand{\htmladdnormallink}[2]{\hotlink{#1}{#2}}
% \renewcommand{\hyperref}[4]{\hotlink{#1}{\S\ref{#4}}}
% \renewcommand{\htmlref}[2]{\hotlink{#1}{\S\ref{#2}}}
% \renewcommand{\xref}[3]{\hotlink{#1}{#2 -- #3}}
%end{latexonly}
% -----------------------------------------------------------------------------
% ? Document-specific \newcommand or \newenvironment commands.
% ? End of document-specific commands
% -----------------------------------------------------------------------------
%  Title Page.
%  ===========
\renewcommand{\thepage}{\roman{page}}
\begin{document}
\thispagestyle{empty}

%  Latex document header.
%  ======================
\begin{latexonly}
   CCLRC / {\sc Rutherford Appleton Laboratory} \hfill {\bf \stardocname}\\
   {\large Particle Physics \& Astronomy Research Council}\\
   {\large Starlink Project\\}
   {\large \stardoccategory\ \stardocnumber}
   \begin{flushright}
   \stardocauthors\\
   \stardocdate
   \end{flushright}
   \vspace{-4mm}
   \rule{\textwidth}{0.5mm}
   \vspace{5mm}
   \begin{center}
   {\Huge\bf  \stardoctitle \\ [2.5ex]}
   {\LARGE\bf \stardocversion \\ [4ex]}
   {\Huge\bf  \stardocmanual}
   \end{center}
   \vspace{5mm}

% ? Add picture here if required for the LaTeX version.
%   e.g. \includegraphics[scale=0.3]{filename.ps}
% ? End of picture

% ? Heading for abstract if used.
   \vspace{10mm}
   \begin{center}
      {\Large\bf Abstract}
   \end{center}
% ? End of heading for abstract.
\end{latexonly}

%  HTML documentation header.
%  ==========================
\begin{htmlonly}
   \xlabel{}
   \begin{rawhtml} <H1> \end{rawhtml}
      \stardoctitle\\
      \stardocversion\\
      \stardocmanual
   \begin{rawhtml} </H1> \end{rawhtml}

% ? Add picture here if required for the hypertext version.
%   e.g. \includegraphics[scale=0.7]{filename.ps}
% ? End of picture

   \begin{rawhtml} <P> <I> \end{rawhtml}
   \stardoccategory\ \stardocnumber \\
   \stardocauthors \\
   \stardocdate
   \begin{rawhtml} </I> </P> <H3> \end{rawhtml}
      \htmladdnormallink{CCLRC}{http://www.cclrc.ac.uk} /
      \htmladdnormallink{Rutherford Appleton Laboratory}
                        {http://www.cclrc.ac.uk/ral} \\
      \htmladdnormallink{Particle Physics \& Astronomy Research Council}
                        {http://www.pparc.ac.uk} \\
   \begin{rawhtml} </H3> <H2> \end{rawhtml}
      \htmladdnormallink{Starlink Project}{http://www.starlink.ac.uk/}
   \begin{rawhtml} </H2> \end{rawhtml}
   \htmladdnormallink{\htmladdimg{source.gif} Retrieve hardcopy}
      {http://www.starlink.ac.uk/cgi-bin/hcserver?\stardocsource}\\

%  HTML document table of contents.
%  ================================
%  Add table of contents header and a navigation button to return to this
%  point in the document (this should always go before the abstract \section).
  \label{stardoccontents}
  \begin{rawhtml}
    <HR>
    <H2>Contents</H2>
  \end{rawhtml}
  \htmladdtonavigation{\htmlref{\htmladdimg{contents_motif.gif}}
        {stardoccontents}}

% ? New section for abstract if used.
  \section{\xlabel{abstract}Abstract}
% ? End of new section for abstract
\end{htmlonly}

% -----------------------------------------------------------------------------
% ? Document Abstract. (if used)
%  ==================
\stardocabstract
% ? End of document abstract
% -----------------------------------------------------------------------------
% ? Latex document Table of Contents (if used).
%  ===========================================
\newpage
\begin{latexonly}
   \setlength{\parskip}{0mm}
   \tableofcontents
   \setlength{\parskip}{\medskipamount}
   \markboth{\stardocname}{\stardocname}
\end{latexonly}
% ? End of Latex document table of contents
% -----------------------------------------------------------------------------
\cleardoublepage
\renewcommand{\thepage}{\arabic{page}}
\setcounter{page}{1}

\section{\xlabel{what_is_jed}What is JED?}
\label{what_is_jed}

JED is a small, fast and freely available text-editor for Unix systems
written by \htmladdnormallink
{John E. Davis}{http://space.mit.edu/\~{}davis/jed.html} of MIT (the author's
copyright notice is reproduced in Section~{\ref{jed-cr}}). Core JED
facilities include utilising the cursor positioning abilities of suitable
terminals, highlighting of selected text ranges, multiple buffers, keypad
control of the editor, a multiple level \textit{undo} facility
\textit{etc}. JED is also fully programmable using a high level editor
language called \textit{S--Lang} which allows it to be easily customised.
\textit{S--Lang} files included with the JED distribution enable it to
emulate several common editors including \textit{emacs} or the VAX/VMS
editor \textit{EDT}.  In addition language sensitive extensions are
supported for many programming languages. The accuracy of these JED
emulations depend, of course, on the detail in the associated
\textit{S--Lang} files. For example, as distributed, the JED emacs
emulation is claimed to be perhaps the best of any non-GNU emacs based
editor.

The EDT emulation provided in the standard JED distribution is
somewhat more basic. This document describes a Starlink supplied and
supported enhancement to JED in EDT mode. This was originally designed
and developed by Clive Page of Leicester University and has been
released and in use for some time (together with JED version 0.93) on
Starlink Sun and Alpha workstations. This re-release was developed by
Starlink and includes, in addition to a totally revised \textit{S--Lang}
file for the changed syntax used in more recent releases of
JED, support for PC--Linux systems (under X11 only) and modern Sun
keyboards. Also the new language sensitive editing capabilities
of JED are optionally supported. Bug reports and suggestions for improvement
to this version may be made to the current author (\texttt{bkm@star.rl.ac.uk}).

The current version of Starlink EDT code was developed for JED version 0.98
but should also work with most JED releases later than version 0.94.

\section{\xlabel{edt_function_keys}EDT function keys}
\label{edt_function_keys}

VMS/EDT (and hence JED in EDT mode) makes extensive use of the
\textit{applications keypad}--a special mode of the (mostly numeric)
group of keys on the far right of DEC VT style terminals. The next
key-group to the left of this one (between it and the main alphanumeric
group) includes the cursor keys which are used with EDT and also
several labelled function keys such as \texttt{Page Up, Insert, Help},
\textit{etc}.  These labelled function keys will also work as expected
in JED and mainly provide alternative keys for some functions already
on the applications keypad. The keyboard layout for the EDT keys on a
DEC VT keyboard is shown in Figure~\ref{fig:dec-kb}.

Within Starlink VT style keyboards are most likely to be found on Alpha
consoles or on X-terminals.
\begin{figure}[t]
\scriptsize
\begin{center}
\begin{tabular}{ccc} \hline
\multicolumn{1}{|c|}{}       & \multicolumn{2}{|c|}{}  \\
\multicolumn{1}{|c|}{Help}    & \multicolumn{2}{|c|}{Do} \\
\multicolumn{1}{|c|}{}       & \multicolumn{2}{|c|}{}  \\  \hline
         &              &        \\ \hline
\multicolumn{1}{|c|}{} & \multicolumn{1}{|c|}{} & \multicolumn{1}{|c|}{} \\
\multicolumn{1}{|c|}{{Find}}    &  \multicolumn{1}{|c|}{Insert} &
\multicolumn{1}{|c|}{Remove} \\
\multicolumn{1}{|c|}{} & \multicolumn{1}{|c|}{Here} & \multicolumn{1}{|c|}{} \\
\hline
\multicolumn{1}{|c|}{} & \multicolumn{1}{|c|}{} & \multicolumn{1}{|c|}{} \\
\multicolumn{1}{|c|}{Select}    &  \multicolumn{1}{|c|}{Prev} &
\multicolumn{1}{|c|}{Next} \\
\multicolumn{1}{|c|}{} & \multicolumn{1}{|c|}{} & \multicolumn{1}{|c|}{} \\
\hline
         &              &        \\ \cline{2-2}
        & \multicolumn{1}{|c|}{}  & \\
        & \multicolumn{1}{|c|}{UP}  & \\
        & \multicolumn{1}{|c|}{}  & \\ \hline
\multicolumn{1}{|c|}{} & \multicolumn{1}{|c|}{} & \multicolumn{1}{|c|}{} \\
\multicolumn{1}{|c|}{LEFT}    &  \multicolumn{1}{|c|}{DOWN} &
\multicolumn{1}{|c|}{RIGHT} \\
\multicolumn{1}{|c|}{} & \multicolumn{1}{|c|}{} & \multicolumn{1}{|c|}{} \\
\hline
\end{tabular}
\hspace{5mm}
\begin{tabular}{|c|c|c|c|} \hline
PF1      & PF2       & PF3     & PF4   \\
GOLD     & HELP      & FNDNXT  & DEL L \\
         &           & \textit{FIND}    & \textit{UND L} \\ \hline
7        & 8         & 9       & -     \\
PAGE     & SECT      & APPEND  & DEL W \\
\textit{COMMAND}  & \textit{FILL}      & \textit{REPLACE}  & \textit{UND W} \\ \hline
4        & 5         & 6       & ,     \\
ADVANCE  & BACKUP    & CUT     & DEL C \\
\textit{BOTTOM}   & \textit{TOP}       & \textit{PASTE}   & \textit{UND C} \\ \hline
1        & 2         & 3       &       \\
WORD     & EOL       & CHAR    & ENTER \\
\textit{CHNGCASE} & \textit{DEL EOL}   & \textit{SPECINS} &       \\ \cline{1-3}
\multicolumn{2}{|c|}{0}         &~.       &       \\
\multicolumn{2}{|c|}{LINE}      & SELECT  & \textit{SUBS}  \\
\multicolumn{2}{|c|}{\textit{OPEN LINE}} & \textit{RESET}   &       \\ \hline
\end{tabular}
\end{center}
\caption{DEC keyboard layout}
\label{fig:dec-kb}
\normalsize
\end{figure}

\subsection{\xlabel{function_keys_on_solaris_and_linux}
Function keys on Solaris and Linux}
\label{function_keys_on_solaris_and_linux}

\textbf{IMPORTANT} \textit{Note that on both Solaris and Linux systems the
X11 event codes generated by the applications keypad keys are \textbf{NOT}
correct for EDT by default! Section~{\ref{xmodmap}} explains how
to correct this. }

The Sun series 5 keyboards (provided on all Starlink Suns purchased in
the last few years) and
those used on PC--Linux systems are identical in the key groups
used for JED (the Sun keyboard has an extra function group on the left-hand
side and other minor differences). The EDT key key layout for these systems
is shown in Figure~\ref{fig:pc-kb}.
\begin{figure}[ht]
\scriptsize
\begin{center}
\begin{tabular}{ccc} \hline
\multicolumn{1}{|c|}{} & \multicolumn{1}{|c|}{} & \multicolumn{1}{|c|}{} \\
\multicolumn{1}{|c|}{{Insert}}    &  \multicolumn{1}{|c|}{Home} &
\multicolumn{1}{|c|}{Page} \\
\multicolumn{1}{|c|}{} & \multicolumn{1}{|c|}{} & \multicolumn{1}{|c|}{Up} \\
\hline
\multicolumn{1}{|c|}{} & \multicolumn{1}{|c|}{} & \multicolumn{1}{|c|}{} \\
\multicolumn{1}{|c|}{Delete}    &  \multicolumn{1}{|c|}{End} &
\multicolumn{1}{|c|}{Page} \\
\multicolumn{1}{|c|}{} & \multicolumn{1}{|c|}{} & \multicolumn{1}{|c|}{Down}
 \\ \hline
         &              &        \\ \cline{2-2}
        & \multicolumn{1}{|c|}{}  & \\
        & \multicolumn{1}{|c|}{UP}  & \\
        & \multicolumn{1}{|c|}{}  & \\ \hline
\multicolumn{1}{|c|}{} & \multicolumn{1}{|c|}{} & \multicolumn{1}{|c|}{} \\
\multicolumn{1}{|c|}{LEFT}    &  \multicolumn{1}{|c|}{DOWN} &
\multicolumn{1}{|c|}{RIGHT} \\
\multicolumn{1}{|c|}{} & \multicolumn{1}{|c|}{} & \multicolumn{1}{|c|}{} \\
\hline
\end{tabular}
\hspace{5mm}
\begin{tabular}{|c|c|c|c|} \hline
Num Lock & /         & *     & -   \\
GOLD     & HELP      & FNDNXT  & DEL L \\
         &           & \textit{FIND}    & \textit{UND L} \\ \hline
7        & 8         & 9       & +     \\
PAGE     & SECT      & APPEND  &  \\
\textit{COMMAND}  & \textit{FILL}      & \textit{REPLACE}  & DEL W \\ \cline{1-3}
4        & 5         & 6       &      \\
ADVANCE  & BACKUP    & CUT     & \textit{UND W} \\
\textit{BOTTOM}   & \textit{TOP}       & \textit{PASTE}   &  \\ \hline
1        & 2         & 3       &       \\
WORD     & EOL       & CHAR    & ENTER \\
\textit{CHNGCASE} & \textit{DEL EOL}   & \textit{SPECINS} &       \\ \cline{1-3}
\multicolumn{2}{|c|}{0}         &~.       &       \\
\multicolumn{2}{|c|}{LINE}      & SELECT  & \textit{SUBS}  \\
\multicolumn{2}{|c|}{\textit{OPEN LINE}} & \textit{RESET}   &       \\ \hline
\end{tabular}
\end{center}
\caption{Sun and Linux keyboard layout}
\label{fig:pc-kb}
\normalsize
\end{figure}

\subsection{\xlabel{keyboard_mappings_for_solaris_and_linux}
Keyboard mappings for Solaris and Linux keyboards}
\label{keyboard_mappings_for_solaris_and_linux}
\label{xmodmap}

JED running in a X--window terminal window expects to receive the key
strokes which would be generated on a Digital VT style terminal. This is a two
stage process -- the user types a key, this generates an X11
\textit{event} and the terminal emulator (for example, \texttt{xterm})
receives this and creates the correct key sequence to be sent to the
application (JED in this case).

The X11 event generated by a key is under the control
of the X-server. This has a set of defaults which may be changed
by a user using the program \texttt{xmodmap} (which, of course, may also be
invoked from a user's \texttt{.login} file). Note, however, that this will
affect all windows running under a particular server and care should
be taken to avoid unexpected behaviour of other programs which read
the keyboard.

For JED/EDT the default X11-events for the Solaris and Linux keyboards
are not correct and \texttt{xmodmap} must be used \underline{\textbf{before
starting the editor}}. If JED is accidently started with the wrong
keymap the key sequence \verb|^Ke| (\texttt{Ctrl+k} simultaneously followed
by the \texttt{e} key) will exit.
\label{jed-lib}

Sample \texttt{xmodmap} files suitable for JED/EDT are included in the
\textit{JED library directory}. This is \texttt{/usr/local/jed/lib} for
Solaris and Digital Unix or \texttt{/usr/lib/jed/lib} for Linux PCs.
These files are called \texttt{solaris-keymap} and
\texttt{linux-keymap} respectively. The appropriate file can be used as
in the following example or copied to a local file and edited as
required.

\begin{quote}
\begin{verbatim}
xmodmap /usr/local/jed/lib/solaris-keymap
\end{verbatim}
\end{quote}

This mapping may also prove useful in remote logins to VMS systems
where considerable functionality is bound to the applications keys.

The exact procedure to derive a key re-mapping file is described in
Appendix~{\ref{kb-remap}}. This allows any
key to be assigned to any function. For example the top row of VT keys
(\texttt{GOLD} \textit{etc.}) could be assigned  to a function key group --
say (\texttt{F9->F12}) -- rather than using the
``\texttt{Num Lock}'' row.

Note that the Sun and PC keyboards have only \textbf{one} key (labelled
\texttt{+}) where VT keyboards have two (labelled \texttt{-} and
\texttt{'}). This will default to the EDT
functions \texttt{DEL~W} and \texttt{UDEL~W} which delete and un-delete
words in the current direction. To change this to the character
functions \texttt{DEL/UDEL~C} this key can either be re-mapped to
\texttt{KP\_Separator} X11 \textit{event} in the appropriate
\texttt{xmodmap} file or
the system default file \texttt{jed.rc} (Section~{\ref{default-jedrc}})
contains JED commands which, if uncommented, will change the default.

\section{\xlabel{starting_jed}Starting JED}
\label{starting_jed}

\subsection{\xlabel{command_line_arguments}Command line arguments}
\label{command_line_arguments}
\label{cl-switches}

JED is normally just invoked by typing \texttt{jed <file name(s)>} to
the shell prompt. There are a number of optional switches which may
also be set. Some of the more useful are:
\begin{center}
\begin{tabular}{lll}
\texttt{-lsmode} & process following file names in language sensitive mode & (b)\\
\texttt{-tmode} & process following file names in \textit{Text} mode & (b)\\
\texttt{-g} \textit{n} & go to line number \textit{n} & (a)\\
\texttt{-s} \textit{string} & search for string & (a)\\
%\texttt{-t} & sets TAB to 0 so tabs show as {\verb+^I+}\\
\texttt{-2} & split window & (a)\\
\texttt{-ftabs} & \textit{Text mode only} - tabs stops set at columns 7,10,13\ldots & (e)\\
\end{tabular}
\end{center}

In the table above the position of switches is indicated by the letters
\textit{b} for before a file name, \textit{a} for after and \textit{e} for
either.

For example to load file \texttt{prog.f} into one window with the cursor at
line number 123, and then load \texttt{prog.lis} into the second window
positioned at the first occurrence of the string \textit{myfunc} you could
use:
\begin{quote}
\begin{verbatim}
% jed prog.f -g 123 -2 prog.lis -s myfunc
\end{verbatim}
\end{quote}

\subsection{\xlabel{jed_initialisation_file}JED initialisation file}
\label{jed_initialisation_file}

The JED emulation mode is normally set within the initialisation file
\texttt{.jedrc} in the user's home directory. For convenience Starlink
supplied versions of JED use a system-wide default file to start up in EDT
mode if this local initialisation file is \textbf{not} present.
Alternative JED emulation modes
or an number of options within Starlink JED (for example various
aspects of the editor status line, the editing defaults in
Section~{\ref{gold-m}} or whether JED \textit{blinks} between
parentheses to show a matching pair) may be changed by creating a
local file. The default Starlink EDT file is called
\label{default-jedrc}
\texttt{jed.rc} in the JED library directory (see
Section~{\ref{jed-lib}}) and this may be copied to the local file
\texttt{\$HOME/.jedrc} and edited.

This file differs from the \texttt{jed.rc} file in a standard JED
distribution (which starts JED in \textit{emacs} emulation mode). The
default JED file has been renamed to \texttt{jed.rc.dist}  -- also in
the JED library directory.

\subsection{\xlabel{backup_and_auto-save_files}Back-up and Auto-save files}
\label{backup_and_auto-save_files}

If a new file is created with the same name as one that already exists
(for example when exiting), JED will rename the original file by
appending a $\sim$ to its name so that the original contents are
preserved. In addition, while editing is in progress JED saves the
contents of each buffer in a disc file after every 300 changes (which may
be every 300 keystrokes if you are simply entering text). These auto-save
files, which have names of the form {\tt
\#}{\em filename}\texttt{\#}, are deleted when JED exits
normally, but may be useful in the event of a crash.

If, on startup, JED finds a auto-save file which is newer than the file being
edited it will suggest recovery. This is performed by entering
\texttt{recover\_file} to the \texttt{S--Lang>} prompt (obtained by
\texttt{GOLD 7} in EDT).

\section{\xlabel{starlink_edt_mode}Starlink EDT mode}
\label{starlink_edt_mode}

\subsection{\xlabel{edt_basics}EDT basics}
\label{edt_basics}

This section mainly covers the  differences between VMS/EDT and
JED/EDT\@. Firstly, however, here is a brief reminder about two important EDT
concepts:
\begin{itemize}
\item The editor has the concept of direction for some operations and
this is shown in the status bar. \textit{Advance} is forward along a
line and towards the end of the buffer while \textit{Backup} is towards the
beginning of the buffer.

\item The most important EDT key is the one designated as \texttt{GOLD} on
the applications keypad. This both functions as a repeat count
(\texttt{GOLD <repeat count> <function>}) and as a
\textit{shift} key where an alternate function (shown lower and in italics in
Figures~{\ref{fig:dec-kb}}~and~\ref{fig:pc-kb}) is selected by typing
\texttt{GOLD} followed by the key. In Starlink JED/EDT other editing
functions are available by typing \texttt{GOLD} followed by letter keys in
the main keyboard group. To cancel \texttt{GOLD} press the key again.

\end{itemize}

\subsection{\xlabel{status_line}Status Line}
\label{status_line}

Like VMS/EDT JED/EDT has a status line in inverse video at the foot of
each buffer (editing window). The format of this line is, however,
somewhat different for JED/EDT and shows the status flags, the JED
version number, the file or buffer name, the current line-number, the
editing position in the file, the current editing direction and the
time.  The status flags start as a set of hyphens: the meaning of other
symbols that may appear there is as follows:

\begin{center}
\begin{tabular}{cl}
\texttt{**}   & this buffer has been modified \\
\texttt{\%\%} & this buffer is read-only \\
\texttt{m}    & a mark has been set \\
\texttt{d}    & disc file has been updated since the buffer was created \\
\texttt{+}    & the undo facility is enabled for this buffer. \\
\end{tabular}
\end{center}

Several options in the JED startup file \texttt{.jedrc} control aspects of
this status line. These include whether the time is included
and the exact format of the display of the editing position within the buffer
(this can be one of percentage, line number or line number and column position).
See comments within the system default initialisation file
(Section~{\ref{default-jedrc}}) for more details.

\subsection{\xlabel{jed/edt_notes}JED/EDT notes}
\label{jed/edt_notes}

Features of the Starlink JED EDT emulation (henceforth often
just called EDT) include a reasonably good emulation of VMS/EDT on
Unix systems together with a few extra facilities such as
cut-and-paste of rectangular regions, unlimited undo, horizontal
scroll, and display of two or more windows.

A few differences from the VMS EDT editor are worth noting:-

\begin{itemize}

\item JED only emulates the keypad-and-screen mode of EDT: various GOLD--key
sequences have been defined (as described below) to make other functions
available.

\item The normal \textit{carriage return} key will auto-indent the next
line. The \texttt{ENTER} key on the \textit{applications keypad} will
produce a ``hard return'' to column 1.

\item Text being entered will automatically wrap at the column set by the
JED \textit{WRAP} variable. The \textit{S--Lang} command
\texttt{toggle\_wrap} will toggle the text wrap state for an individual
buffer. Typing long lines is possible and a \verb|$| character will
indicate when a line is too long for the screen.

\item The \textit{abort} key in JED is control/G which will abort the
current \textit{S--Lang} procedure or operation.

\item To exit from JED use \texttt{GOLD E} (because as usual with Unix
control/Z just suspends the process).  Alternatively use \texttt{GOLD Q}
(for quit), or press the DO key (or GOLD followed by the keypad 7 key)
to bring up the \texttt{S--Lang>} prompt and enter either \texttt{exit} or
\texttt{quit}.  These all do the same thing.  If any buffer has been
modified JED asks you whether to re-write it to disc.

\item The HELP key just shows a keypad diagram together with a brief
listing of the additional GOLD--key assignments.

\item The FILL function (\texttt{GOLD 8}) fills the current paragraph and does
\textbf{not} require a mark to be set.  See Section~{\ref{gold-n}} for
further details.

\item The function to enter special characters, SPECINS or \texttt{GOLD 3},
is slightly simpler than with EDT: it prompts for the decimal value of
the ASCII character required.

\item The EDT function \textit{substitute} (\texttt{SUBS = GOLD Enter}) is
by default bound to an \textit{S--Lang} function \texttt{replace\_cmd}. If an
more accurate emulation of the (rather less obvious) functionality in VMS EDT
is desired this can be rebound to \texttt{edt\_subs} in the \texttt{.jedrc}
file (see Section~{\ref{default-jedrc}}).

\end{itemize}

\subsection{\xlabel{additional_goldkey_sequences}Additional GOLD-key sequences}
\label{additional_goldkey_sequences}

\begin{center}
\begin{tabular}{|ll|}
\hline
Key-seq. & Effect \\
\hline
\texttt{GOLD A} & ASCII codes of the next 10 characters are shown\\
\texttt{GOLD B} & Buffer/window options\ldots (see section~\ref{buff-ops}) \\
\texttt{GOLD C} & Centres current line using the \texttt{WRAP} value \\
\texttt{GOLD D} & Date inserted at current position (and date/time shown in mini-buffer)\\
\texttt{GOLD E} & Exits - if any buffers modified asks whether to update disc file \\
\texttt{GOLD F} & Finds string at start of line (prompts for string) \\
\texttt{GOLD G} & Global search and replace (prompts for both strings) \\
\texttt{GOLD I} & Inserts file at current position (prompts for name) \\
\texttt{GOLD K} & Prompts for string: moves lines containing this to another
buffer \\
\texttt{GOLD L} & Goes to specified line-number (prompts for number) \\
\texttt{GOLD M} & Mode settings displayed and can be altered (see section~\ref{gold-m}) \\
\texttt{GOLD N} & Narrows (centres) current paragraph (see section~\ref{gold-n}) \\
\texttt{GOLD O} & Occurrences of string listed with line-numbers (prompts for string)\\
\texttt{GOLD Q} & Quits -- same as exit. \\
\texttt{GOLD R} & Rectangular cut/paste options\ldots (see section~\ref{gold-r}) \\
\texttt{GOLD S} & Spawns command to operating system (some restrictions) \\
\texttt{GOLD T} & Toggles terminal width 80/132 columns, adjusts \texttt{WRAP} value\\
\texttt{GOLD U} & Undo - restores buffer state undoing preceding changes one by one\\
\texttt{GOLD W} & Writes current buffer to file (prompts for name) \\
\texttt{GOLD X} & Exchanges position of mark and current position \\
\texttt{GOLD} \textit{space}   & Shows current position and mode settings \\
\texttt{GOLD /} & Toggle insert/overwrite text flag for buffer \\
\texttt{GOLD $\downarrow$}  & Selects the next window as the active one\\
\texttt{GOLD $\uparrow$  }  & Expands current window by one line \\
\texttt{GOLD $\rightarrow$} & Scrolls screen right by half its width \\
\texttt{GOLD $\leftarrow$ } & Scrolls screen back to the left by half its width \\
\texttt{GOLD GOLD}          & Toggles keypad application/numeric modes.\\
\hline
\end{tabular}
\end{center}


\subsection{\xlabel{gold_b_-_using_multiple_buffers_and_windows}%
GOLD B - Using Multiple Buffers and Windows}
\label{gold_b_-_using_multiple_buffers_and_windows}
\label{buff-ops}

JED can handle two or more windows at once, and display in them either
different parts of the same file or two or more different files.  You
can cut and paste from one window to another. The {\tt
GOLD B} key prompts for a number of options.

\begin{center}
\begin{tabular}{|ll|}
\hline
Key-sequence & Buffer/window action \\
\hline
\texttt{GOLD B 2} & Splits current window into two.\\
\texttt{GOLD B 1} & Restores single window (the current one).\\
\texttt{GOLD B G} & Gets a file into the current buffer (prompts for name).\\
\texttt{GOLD B L} & Lists all existing buffers.\\
\texttt{GOLD B S} & Selects a buffer by name (prompts for it).\\
\hline
\end{tabular}
\end{center}

Whenever JED prompts for a file-name you can enter a partial name and
press the TAB key to get JED to complete it.  If the name is not unique
then press the space-bar at least twice to cycle among them.

\texttt{GOLD} down-arrow selects the next window in turn as the active one,
while \texttt{GOLD} up-arrow expands the current window by one line.

\subsection{\xlabel{gold_u_-_undo}GOLD U - Undo}
\label{gold_u_-_undo}

A powerful undo facility is provided using \texttt{GOLD U} (also mapped to
F14 on a VT terminal).  JED keeps
track of buffer modifications and undoes the changes one by one when you
press the UNDO key.  This can be used to reverse complex changes such as
\emph{fill} or \emph{sort}.

\subsection{\xlabel{modes_and_customizing}Gold M - Modes and Customizing}
\label{modes_and_customizing}
\label{gold-m}

Several built-in JED variables control aspects of its editing
interface and behaviour. A useful subset of these may be changed
using the \texttt{GOLD M} command while the default initial values
of both these and less commonly changed variables may be set in a local
\texttt{.jedrc} file (Section~{\ref{default-jedrc}}). In addition
all variables may be altered interactively by commands to the
\texttt{S--Lang>} prompt (\texttt{GOLD 7}).

Note that \textit{S--Lang} requires space separated commands
and operators with a semi-colon terminator. As an example \texttt{WRAP=0}
is incorrect while the correct syntax is \texttt{WRAP = 0;}.

\texttt{WRAP\_INDENTS} is 0 by default.  If set to 1, when the editor
automatically wraps text it will indent.

\texttt{CASE\_SEARCH} is 1 by default, giving case-sensitive searches.
If set to 0 all searches become case-insensitive.

\texttt{TAB} determines how tab characters in the text are displayed on
the screen and also the action of the \texttt{TAB} key on the keyboard.
JED converts the tab characters into \texttt{TAB} spaces.
The default TAB value is 8, which corresponds to the
way most VT terminals are set up.

\texttt{WRAP} specifies the right-hand margin for text entry, filling, and
centring text. When you enter text which crosses the \texttt{WRAP} boundary
and press the space bar the excess words are moved to the following line.
The default \texttt{WRAP} value is 75.  To avoid the temporary disappearance
of text when you are typing continuously it is advisable to choose a
value slightly smaller than the physical screen width. The \texttt{WRAP}
value is adjusted when the screen-width is changed using \texttt{GOLD T}.

To \textbf{toggle} the \textit{wrap flag} for a buffer type
\texttt{toggle\_wrap} at the \texttt{S--Lang>} prompt (\texttt{GOLD 7}).

The tab characters in a file may be converted to the appropriate set of
spaces using the current value of \texttt{TAB} by selecting a range of text,
pressing the \texttt{GOLD 7 or DO} key, and then entering the
\texttt{untab} command.

\subsection{Gold N - Paragraph Fill and Centre\xlabel{paragraph_fill_and_centre}}
\label{gold-n}

Text formatting operations work differently from those in EDT\@. The
FILL command (\texttt{GOLD 8}) fills the current paragraph, and does
\textit{not}\/use the selected range.  A paragraph is defined as any set
of lines ending with a blank line (or a line starting with either
\texttt{\%} or \verb+\+ to assist in filling Latex files). If the first
line of the paragraph was inset from the left-hand margin the whole
paragraph is indented by the same amount.  The \texttt{WRAP} value
defines the maximum line-length.

To centre the current line use \texttt{GOLD C}, while \texttt{GOLD N} centres
the current paragraph, with margins set by the indentation of the first
line.  In both cases the centre is half the \texttt{WRAP} value.

\subsection{GOLD R - Rectangular Column Cut/Paste
\xlabel{rectangular_column_cutpaste}}
\label{gold-r}

The default JED highlighting is distracting and incorrect for rectangular
editing so typing \texttt{GOLD R} \textbf{before} the select key will
temporarily turn this off.

Then define the rectangle by pressing the key-pad dot (or the
\texttt{SELECT} key) at one corner and move the cursor to the opposite
corner.  Then \texttt{GOLD R C} will copy the rectangle to a buffer, or
\texttt{GOLD R D} delete it, \emph{i.e.},\/move it to a buffer.  A
subsequent \texttt{GOLD R I} will insert the saved rectangle at the
current point.  Other options are \texttt{GOLD R B} to blank rectangle
(overwrite with spaces), or \texttt{GOLD R S} to shift text over by
inserting a blank rectangle of the size specified.

\subsection{Sorting Lines\xlabel{sorting_lines}}

JED can sort a series of lines into ascending order using the ASCII
collating sequence.  First select the rectangle containing the sort keys
(press SELECT at one corner and move to the other corner), then press the
DO key and at the prompt enter the \texttt{sort} command. Sorting is done by
interpreted code and is rather slow for regions of more than a few
hundred lines.

\section{\xlabel{language_sensitive_editing}Language sensitive editing}
\label{language_sensitive_editing}

JED is distributed with a large collection of language sensitive editing
\textit{S--Lang} files in its library directory (see
Section~{\ref{jed-lib}}). These include conventional programming
languages such as C and FORTRAN together with other \textit{languages}
such as \TeX, and IDL\@. JED can select these automatically based on the
file extension. These files can alter key behaviour (for example in
the ``\textit{C mode}'' the editor automatically indents and
un-indents on the \verb+{+ and \verb+}+ characters respectively)
and can also be programmed to recognise language specific keywords and
constructs.

This document cannot describe all these variants of language sensitive
editing behaviour and, in addition, none of these were supported by the
previous Starlink release of JED\@. This release will also, by default,
edit all files in \textit{Text} mode. The current editing mode is
indicated in the buffer status line.

For more advanced users comments in the system \textit{jed.rc} file describe
how to re-enable JED's automatic selection of language sensitive editing.
This may also be invoked manually
using the \texttt{-lsmode} switch described
in Section~{\ref{cl-switches}}. As a final alternative typing a mode
selection command to the \texttt{S--Lang>}
prompt (for example, \texttt{fortran\_mode} or \texttt{text\_mode}) will also
change the editing mode of the current buffer.

Documentation on some language sensitive editing modes may be obtained by
typing the \textit{S--Lang} command \texttt{describe\_mode}.
Alternatively the \textit{S--Lang} source files should be consulted.
\newpage
\section{JED Copyright\xlabel{copyright}}
\label{jed-cr}
The JED author's copyright notice is reproduced here.
\begin{quote}
 Copyright (c) 1992, 1995 John E. Davis
 All rights reserved.

 This program is free software; you can redistribute it and/or modify it
 under the terms of the GNU General Public License as published by the Free
 Software Foundation; either version 2 of the License, or (at your option)
 any later version.

 This program is distributed in the hope that it will be useful, but WITHOUT
 ANY WARRANTY; without even the implied warranty of MERCHANTABILITY or
 FITNESS FOR A PARTICULAR PURPOSE\@.  See the GNU General Public License for
 more details.

 You should have received a copy of the GNU General Public License along
 with this program; if not, write to the Free Software Foundation, Inc., 675
 Mass Ave, Cambridge, MA 02139, USA.
\end{quote}

\newpage
\appendix
\section{\xlabel{x11_keyboard_mapping}X11 keyboard mapping}
\label{x11_keyboard_mapping}
\label{kb-remap}

On all X11 systems the \textit{X11 event} corresponding to a particular
keyboard key is under the control of the X-server software. On Solaris
and Linux systems the default key-mappings for the applications keyboard
are not appropriate for JED/EDT.

To derive a new mapping file:
\begin{itemize}

\item Find the X-11 key-code for all the desired keys using the \texttt{xev}
program. On most systems this is standard X11 software but on Solaris
it is located in directory \texttt{/usr/openwin/demo}.

For example the default \texttt{Num Lock} key under Linux produces the
following \texttt{xev} output.
\begin{verbatim}
KeyPress event, serial 18, synthetic NO, window 0x2400001,
    root 0x2a, subw 0x0, time 1681183101, (77,108), root:(83,132),
    state 0x0, keycode 77 (keysym 0xff7f, Num_Lock), same_screen YES,
    XLookupString gives 0 characters:  ""
\end{verbatim}

This shows that the X-server assigned keycode of this key is 77 and that the
current mapping for this key is to the keyboard event \textit{Num Lock}.

\item Generate an input file for \texttt{xmodmap} which includes the following
X-events -- KP\_F1 to KP\_F4 for \texttt{GOLD} and the other three keys in
this row, KP\_0 to KP\_9 for the numeric keys, KP\_Separator for the
\texttt{,} key,
KP\_Decimal for the \texttt{.} key and KP\_Add for the \texttt{+} key.

Part of an \texttt{xmodmap} input file (from RedHat Linux 5.1) is:
\begin{verbatim}
!
! 'Num Lock' row
!
keycode 77  = KP_F1
keycode 112 = KP_F2
keycode 63  = KP_F3
keycode 82  = KP_F4
\end{verbatim}

\end{itemize}

\end{document}
