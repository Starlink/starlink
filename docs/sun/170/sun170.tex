\documentclass[twoside,11pt,nolof]{starlink}

% -----------------------------------------------------------------------------
% ? Document identification
\stardoccategory    {Starlink User Note}
\stardocinitials    {SUN}
\stardocsource      {sun\stardocnumber}
\stardocnumber      {170.2}
\stardocauthors     {C A Clayton}
\stardocdate        {11 October 1997}
\stardoctitle       {Editors on Unix}
\stardocabstract{
The purpose of this document is to give new users advice on how to
choose which editor to use on Unix machines.
Under Unix the default editors are considered to be
unfriendly and many users prefer to use other more sophisticated
alternatives. However, many such alternatives exist; there is not one
single editor that everyone finds acceptable and hence
each user must decide for himself or herself which to adopt.
}

% ? End of document identification

% -----------------------------------------------------------------------------

% ? Document-specific \providecommand or \newenvironment commands.
% ? End of document-specific commands
% -----------------------------------------------------------------------------
%  Title Page.
%  ===========
\begin{document}
\scfrontmatter

\section{\xlabel{choice_of_editors}Choice of editors}

In this section I point out some pros and cons of each editor to help you
choose for yourself which you should adopt. When Starlink started the move to
Unix, we recommended that users use either \verb|vi| or \verb|emacs|, mainly
since these editors are very widely available.
We considered \verb|vi| adequate
for the average user with \verb|emacs| as an option for those who needed
something friendlier and could cope with the greater
sophistication. However, this advice was not popular with all users. Some felt
that \verb|vi| was just too unfriendly
and \verb|emacs| too complex. Our original recommendation still
stands, but I am also including here information on other editor options.

\subsection{\xlabel{vi}vi}

\subsubsection*{Pros}

\begin{itemize}
\item used by lots of people
\item will be available on every Unix machine you encounter, anywhere
\item works with any keyboard since it only uses standard \texttt{qwerty} keys
and not the keypad
\item no cost (in terms of software support and installation)
\item fast
\item available on non-Unix platforms (e.g. MS--DOS)
\end{itemize}

\subsubsection*{Cons}

\begin{itemize}
\item unfriendly
\item not as powerful as some of the other editors
\end{itemize}


\subsection{\xlabel{emacs}emacs}

\subsubsection*{Pros}

\begin{itemize}
\item very powerful editor
\item available for most Unix platforms but sites must build and
install it themselves
\item free and readily available
\item has an X-windows interface
\item works with any keyboard since it only uses standard \texttt{qwerty} keys
and not the keypad (unless you choose to configure it to do so)
\item available on non-Unix platforms (e.g. VMS and MS--DOS)
\end{itemize}

\subsubsection*{Cons}
\begin{itemize}
\item not necessarily available on every Unix machine you will meet
\item too complex for some users
\end{itemize}

\subsection{\xlabel{vms_tpu_emulator}VMS TPU emulator (nu/TPU)}

\subsubsection*{Pros}

\begin{itemize}
\item Very similar to the VAX/VMS TPU editor
\end{itemize}

\subsubsection*{Cons}

\begin{itemize}
\item costs money
\item you are \textbf{very} unlikely to find a TPU emulator on non-Starlink machines
\item you \textbf{cannot} take a copy of a TPU emulator with you to non-Starlink sites
\item emulation problems on some keyboards
\end{itemize}

\subsection{\xlabel{jed}jed (see
\xref{SUN/168}{sun168}{}) (and other public domain editors)}

\subsubsection*{Pros}

\begin{itemize}
\item can emulate the VMS editor EDT incompletely but quite usably
\item is free
\item you \textbf{can} take a copy of \verb|jed| with you when you leave Starlink
\item available on non-Unix platforms (e.g. VMS and MS--DOS)
\end{itemize}

\subsubsection*{Cons}
\begin{itemize}
\item is less ubiquitous than \verb|vi| or \verb|emacs|
\item EDT emulation is not perfect
\end{itemize}


\subsection{\xlabel{xwindows_based_text_editors}X--windows based text editors}

\subsubsection*{Pros}

\begin{itemize}
\item some users find this type of editor easy to use
\end{itemize}

\subsubsection*{Cons}
\begin{itemize}
\item use limited to an X--display and hence remote editing can
be a problem in some circumstances
\item limited functionality
\end{itemize}

\end{document}
