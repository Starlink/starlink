\documentstyle[11pt]{article}
\pagestyle{myheadings}

%------------------------------------------------------------------------------
\newcommand{\stardoccategory}  {Starlink User Note}
\newcommand{\stardocinitials}  {SUN}
\newcommand{\stardocnumber}    {173.1}
\newcommand{\stardocauthors}   {M D Lawden}
\newcommand{\stardocdate}      {18 August 1993}
\newcommand{\stardoctitle}     {LOOKUP --- Looking up commands in applications}
%------------------------------------------------------------------------------

\newcommand{\stardocname}{\stardocinitials /\stardocnumber}
\renewcommand{\_}{{\tt\char'137}}     % re-centres the underscore
\markright{\stardocname}
\setlength{\textwidth}{160mm}
\setlength{\textheight}{230mm}
\setlength{\topmargin}{-2mm}
\setlength{\oddsidemargin}{0mm}
\setlength{\evensidemargin}{0mm}
\setlength{\parindent}{0mm}
\setlength{\parskip}{\medskipamount}
\setlength{\unitlength}{1mm}

%------------------------------------------------------------------------------
% Add any \newcommand or \newenvironment commands here
%------------------------------------------------------------------------------

\begin{document}
\thispagestyle{empty}
SCIENCE \& ENGINEERING RESEARCH COUNCIL \hfill \stardocname\\
RUTHERFORD APPLETON LABORATORY\\
{\large\bf Starlink Project\\}
{\large\bf \stardoccategory\ \stardocnumber}
\begin{flushright}
\stardocauthors\\
\stardocdate
\end{flushright}
\vspace{-4mm}
\rule{\textwidth}{0.5mm}
\vspace{5mm}
\begin{center}
{\Large\bf \stardoctitle}
\end{center}
\vspace{5mm}

This is a command procedure written by Andy Adamson of the University of
Central Lancashire.
It was prepared for release by Mike Lawden who added commands for several more
packages to the stored list.

Given a command name, it searches a list for a description of the command.
It prepares a report in a file called LOOKUP.TMP in your current directory.
This includes all the lines found in the list which contain the command name
you specified (some commands may be common to several application packages).
It also stores the names of all the packages searched.

To run the procedure, type:
\begin{verbatim}
    @DOCSDIR:LOOKUP <command name>...
\end{verbatim}
where \verb+<command name>+ is the name of the command being searched for.
You can specify more than one command name if you separate them with commas.
For example:
\begin{verbatim}
    @DOCSDIR:LOOKUP BDF2NDF,NDF2BDF
\end{verbatim}
will display descriptions of the commands {\tt BDF2NDF} and {\tt NDF2BDF} in
the {\tt CONVERT} package.

At the end of the display you are asked if you want to keep LOOKUP.TMP.
The default action is to delete it, but you can save it and print it out if
you want to.

The procedure sets your terminal to a width of 132 characters to accommodate
the long lines of description.

\end{document}
