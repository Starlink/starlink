\documentstyle{article}
\pagestyle{myheadings}

%------------------------------------------------------------------------------
\newcommand{\stardoccategory}  {Starlink User Note}
\newcommand{\stardocinitials}  {SUN}
\newcommand{\stardocnumber}    {54.1}
\newcommand{\stardocauthors}   {M D Lawden}
\newcommand{\stardocdate}      {22 February 1990}
\newcommand{\stardoctitle}     {GENSTAT --- Statistical Data Analysis}
%------------------------------------------------------------------------------

\newcommand{\stardocname}{\stardocinitials /\stardocnumber}
\markright{\stardocname}
\setlength{\textwidth}{160mm}
\setlength{\textheight}{240mm}
\setlength{\topmargin}{-5mm}
\setlength{\oddsidemargin}{0mm}
\setlength{\evensidemargin}{0mm}
\setlength{\parindent}{0mm}
\setlength{\parskip}{\medskipamount}
\setlength{\unitlength}{1mm}

\begin{document}
\thispagestyle{empty}
SCIENCE \& ENGINEERING RESEARCH COUNCIL \hfill \stardocname\\
RUTHERFORD APPLETON LABORATORY\\
{\large\bf Starlink Project\\}
{\large\bf \stardoccategory\ \stardocnumber}
\begin{flushright}
\stardocauthors\\
\stardocdate
\end{flushright}
\vspace{-4mm}
\rule{\textwidth}{0.5mm}
\vspace{5mm}
\begin{center}
{\Large\bf \stardoctitle}
\end{center}
\vspace{5mm}

The GENSTAT system is designed for routine and exploratory statistical data
analysis.
It was developed by the statistics department of Rothamsted Experimental
Station, a research establishment of the U.K.\ Agriculture and Food
Research Council, where many of the standard methods of Statistical Analysis
were pioneered by the systems originators: Professor John Nelder, Dr.\ Frank
Yates, and Sir Ronald A Fisher.

The GENSTAT system offers the user all the benefits of a modern, interactive
system, plus the security of a system with a long history of successful use:
\begin{itemize}
\item A general, command driven system.
\item Continuously developed over the last 20 years.
\item Regular new releases.
\item Open-ended, providing all the tools you require to investigate and
construct new methods of statistical analysis.
\item Flexible, allowing you to deal with awkward or unusual data as well as
routine analyses.
\item Standard statistical methods are specified in easily understood
English language commands.
\end{itemize}

\section{Functions}

GENSTAT has a rich set of functions:
\begin{description}
\item [Reads]:

Genstat allows you to access data from character or binary computer files
recorded in practically any format.
You can type data at a terminal directly, read data files prepared by an
editor or a wordprocessor, or read data or results produced by other programs.
Very large data sets can be condensed during input, to restrict memory usage.

\item [Manipulates]:

Data, and results from analyses are stored in appropriate data structures,
such as single numbers, series of numbers, matrices, tables, structure values
and text.
You can carry out arithmetic on all numeric structures and many standard
functions are provided.
Missing values are appropriately handled throughout Genstat.
Text can also be freely manipulated, particularly with the help of an internal
editor.
Other structures are available for holding arithmetic expressions and model
formulae.
Groups of structures may be referred to by pointers.
All data structures are dynamically created and may be deleted at any time
to free memory for later use within the same Genstat job.

\item [Summarises]:

Simple statistics can be formed and displayed easily, either as tabular
summaries or as graphs or histograms.
Tables can be extended, contracted, or combined by powerful commands, and
their printed form closely controlled if necessary.
Graphical displays can be produced either in a form suitable for ordinary
terminals and line-printers, or in a high-resolution form for graphics
terminals and plotting devices.
High-resolution graphical displays are made available by the provision of an
interface to GKS.

\item [Prints]:

As with reading in Genstat, you can print output on your terminal screen or you
can write to files, either in character or binary form.
The output format may be specified (such as how many decimal places to print,
or whether to right-justify values),  but if no constraints are given,
Genstat will select a suitable layout.

\item [Analyses]:

Genstat provides a wide range of standard statistical techniques which are
broadly grouped under the headings of regression, analysis of designed
experiments, multivariate analysis and analysis of time series.
All analyses are easy to specify and produce clear and well designed reports.
Results of analyses can be stored in data structures for further analysis or
processing by Genstat.

\item [Programs]:

Novel or non-standard techniques can be programmed in Genstat's command
language, which provides all of the elements of a high-level structured
programming language.
Easily understood looping and alternative choice selection commands,
combined with the data structures and manipulation facilities described
above, provide a powerful environment for the specification of any
statistical or mathematical task.
Procedures allow general programs to be stored and reused when needed, and
a library of tested procedures is distributed with Genstat.

\item [Stores]:

Procedures, data, and results of analyses can all be stored permanently
for retrieval when needed.
Storage can be in character files, or in quick-access or hierarchically
arranged binary files.
Commands given during an interactive Genstat run can be copied to a file
for editing and used later.
The current state of a session can, at any stage, be stored and the session
resumed later, with data structures intact.

\item [Extends]:

Genstat is designed to be extended by its users as well as by its developers.
General programs in the Genstat language can be stored as procedures and
made available to all users at a site, simply by attaching a
procedure-library file.
The invocation of a procedure looks like any standard Genstat command.
\end{description}

\section{The standard statistical analyses}

Genstat provides the standard statistical analyses:

\begin{description}

\item [Regression]:

Within Genstat, regression means much more than fitting straight lines.
Certainly, lines can be fitted very simply, but Genstat is also
designed to make it easy to compare alternative lines, both by using
different sets of explanatory variables, and by using groupings of the
observations.
The easy specification of groups simplifies the analysis of unbalanced
experimental designs.
Missing values are excluded automatically and analysis can be restricted
to subsets of observations.
Stepwise, weighted, grouped and offset regression can be specified, and
individual parameters can be constrained to fixed values

Regression in Genstat includes generalised linear and nonlinear models,
all within the same framework provided for linear models.
Many standard generalised linear models are available, including log-linear
models for contingency tables, and variance component models.
The problems of non-additivity of model terms and non-normality of error
terms, encountered in many regression problems, can be overcome by the use
of models from this wide class.
Nonlinear models are fitted by maximum likelihood, using a Newton-Raphson
or a Gauss-Newton algorithm.
Full provision is made for exploring partial linearity, and a range of
standard nonlinear curves can be fitted with particular simplicity.

\item [Multivariate]:

Standard methods of multivariate analysis are available with single commands,
while many more specialised methods are provided as procedures in the
Procedure Library.
The standard methods include:
\begin{itemize}
\item Principal components analysis, to describe objects in a space of reduced
dimension;
\item Canonical variates analysis, to optimise separation of groups;
\item Factor rotation on the results of principal components or canonical
variates analyses, to aid interpretation of the results of the analyses;
\item Principal coordinates analysis, to generate coordinates that reproduce
given distances between objects;
\item Procrustes rotation, to compare two configurations of a set of objects.
\end{itemize}
Procedures in the library include correspondence analysis, and generalized
and multiple Procrustes analyses.

Genstat provides a wide range of methods of classifying objects into clusters.
Hierarchical classifications can be generated from a variety of similarity
measures with a choice of clustering criteria and can be displayed as
dendrograms.
Nearest neighbours, most typical members of clusters, and minimum spanning
trees can be obtained.
Non-hierarchical classification is also available with a choice of criteria.

\item[Time Series]:

Individual time series can be modelled by Box-Jenkins ARIMA or seasonal
ARIMA models.
Sample statistics, such as autocorrelations and partial autocorrelations, are
available to aid model selection.
Allowance is made for missing values, for restriction to sub-series of
observations, for transformations of series and for differencing to
approximate stationarity.
Parameters can be estimated by least squares or exact likelihood, and the
resulting models can be used to generate forcasts with confidence
intervals.

Regression models with autocorrelated errors are available as a special case
of transfer function models, which relate the behaviour of an output series to
one or more input series.
The processes of model selection, fitting, checking and forecasting are
catered for in the same framework as for univariate models.

Spectral analysis of time series can be handled by use of Fourier
transformation functions.
\end{description}

\section{Documentation}

The {\em Genstat 5 Reference Manual} contains a full description of the Genstat
system, illustrated throughout by worked examples.
The {\em Genstat 5 Reference Summary} is a pocket book which contains the
essential information about every Genstat command; it is equivalent to
Appendix 1 of the Reference Manual and the same information is also available
from within Genstat through the on-line help facility.
Loan copies of the {\em Manual} and {\em Summary} should be available from your
Site Manager.
There is also an introductory book called {\em Genstat 5: An Introduction}.
A copy of this should be available at every Starlink site for reference.
NAG also produce a Genstat {\em Newsletter} and a {\em Procedure Library
Manual}; if you want a copy of these, please ask your Site Manager or contact
Mike Lawden directly (username RLVAD::MDL).

\section{Use}

The currently installed version of Genstat is Genstat 5, Release 1.3.
Genstat is invoked by the command:
\begin{verbatim}
    $ GENSTAT    [parameter, parameter, ...]
\end{verbatim}
where {\tt parameter} is [KEYWORD=]file-spec or [S=]integer.
The parameters assign files and data space for the current job.

The ordered list of keywords is:
\begin{verbatim}
    IN, OUT, IN2, OUT2, IN3, OUT3, IN4, OUT4, IN5, OUT5
    BS1, BS2, BS3, BS4, BS5, BS6, UF1, UF2, UF3, UF4, PL1, PL2, PL3, S
\end{verbatim}
To exit from a Genstat session, enter the statement:
\begin{verbatim}
    > STOP
\end{verbatim}
\subsection{Parameters}

If keywords are used, the parameters may appear in any order.
If not, you must give the file-specs in the correct order: a parameter may be
omitted if it is not required but the ``," must remain.
The space after ``GENSTAT" must be present; thereafter spaces are ignored.
If more than one line is needed, end each line but the last with ``--".

IN \ldots IN5 and OUT \ldots OUT5 attach input and output files to channels
for normal (formatted) reading and writing.
BS1 \dots BS6 and UF1 \dots UF4 attach files for unformatted I/O, by backing
store commands, or READ/PRINT commands with option UNFORMATTED=YES.
PL1, PL2 and PL3 attach (user) Procedure Libraries.

\subsubsection{Data space}

By default, one data segment (65,536 data units) is allocated.
Some of this, including the first 18000 units, is used by the system.
This should be adequate for most jobs.
For large jobs, set S = n (where $0<n<4$) to reserve n segments.
As a rough guide, assume that each extra segment will hold 40,000 data values.

\subsubsection{Workfiles}

Default workfiles for backing store and unformatted I/O are automatically
created at the beginning of a run and deleted when the STOP command is met.
If the run fails, and STOP is not executed, these files (GNBSWF.DAT and
GNUFWF.DAT) will be left in your directory and should be deleted.

\subsubsection{Errors}

If you make a mistake in the command line, self-explanatory messages appear.
If a file cannot be attached, the value of a variable IOSTAT is given: this is
a FORTRAN Run-Time Error Number (see Table 18-1 in the DEC manual ``Programming
in Vax Fortran").
You may have misspelt the filename.
You will be invited to try again.

\subsubsection{Examples}

To run a Genstat program taking instructions from a file JOB.DAT, sending
primary output to a file RESULTS.LIS and secondary output to a file OTHER.DAT,
any of the following forms of the command may be used:
\begin{verbatim}
    $ GENSTAT  IN=JOB.DAT, OUT=RESULTS.LIS, OUT2=OTHER.DAT
    $ GENSTAT     JOB.DAT,     RESULTS.LIS, OUT2=OTHER.DAT
    $ GENSTAT     JOB.DAT,     RESULTS.LIS,,     OTHER.DAT
    $ GENSTAT  IN = JOB, -
             OUT = RESULTS.LIS, OUT2 = OTHER
\end{verbatim}
You can display the output at your terminal by using the form:
\begin{verbatim}
    $ GENSTAT  IN=EXAMPLE
\end{verbatim}
By default the file extension is taken as .DAT.

\subsection{Interactive use}

If you do not set either of the first two parameters, Genstat will run
interactively, thus:
\begin{verbatim}
    $ GENSTAT
\end{verbatim}
invokes an interactive run with instructions expected from the terminal
and output returned to the terminal (with prompt $>$).
\begin{verbatim}
    $ GENSTAT   , , DATA, PLOT
\end{verbatim}
starts an interactive run with secondary input and output.

Having entered Genstat, try typing:
\begin{verbatim}
    $ HELP
\end{verbatim}
This initiates a self-explanatory Help system.

No error in your program will cause the run to be abandoned; only
STOP (or CTRL/Y) will terminate the run.

\subsubsection{Action after errors}

If an error is detected a fault message will appear, followed by an
indication of the part of the command at which the fault is detected.
The statement containing the fault and subsequent statements on the
same line will be ignored.
Given:
\begin{verbatim}
    > SCAL W ; 3  &  X Y ; 2  &  Z ; 10
\end{verbatim}
a fault in the second statement will be indicated; only the first
statement will be executed.

\subsection{Severity codes}

A severity code is set on exit from the program.
On normal exit, \$SEVERITY will be:
\begin{verbatim}
    1  (success)  if there have been no diagnostic comments or faults,
    0  (warning)  if there have been diagnostic comments but no faults,
    2  (error)    if there have been diagnostic faults.
\end{verbatim}
Genstat Procedures expect to generate warnings and switch off the output of
diagnostic comments warnings while they are being executed.
\$SEVERITY may be set to 0 as a consequence.

\subsection{Error recovery}

If a severe error (i.e.\ one that is not trapped by Genstat) occurs,
standard system messages including a traceback appear in the log file,
followed by a request that output and log should be sent to NAG, Oxford.
Before sending the output, rerun the job with the DIAGNOSTIC option of the JOB
statement set to E to get a full dump when the job fails.
(To get a copy of the log when you run the job from a terminal, DEFINE
SYS\$ERROR file-spec before invoking Genstat.)
The current Genstat job is abandoned and the primary input stream
scanned for the next ENDJOB or JOB command.
When exit from the program occurs, \$SEVERITY will be set by the
system (to something other than 0, 1 or 2) and the system message that
appears on exit should be ignored.

\subsection{Procedure Libraries}

Sites may set up their own Local Procedure Library, and users may set
up their own procedure libraries as follows:
\begin{itemize}
\item Run a normal Genstat job which defines one or more procedures and write
 the required set of procedures to a file using the STORE directive with option
 PROCEDURE=YES (see Manual, Chapter 6.3.3).
\item If these procedures are to be the Local Procedure Library, the file
 should be placed in the Genstat directory by the site manager.
\item A user procedure library can be attached to a subsequent run either
 by using a keyword PLn when invoking Genstat or by using an OPEN directive
 with option FILETYPE set to PROCEDURELIBRARY.
\end{itemize}

\section{Graphics}

The version of Genstat received from NAG incorporates the REGIS graphical
interface and is ready to use interactively on DEC VT125/240/340 graphical
terminals.
Eventually, the Starlink release of Genstat will be adapted to the GKS
graphical interface.
However, as it may take some time for this to be done, the REGIS version
has been released in order to make this software available as soon as
possible.
Until the GKS adaptation is available, users will find that the graphical
facilities are restricted.

\section{Data space}

\begin{itemize}
\item The numerical space can hold the equivalent of 196608 real values.
 If more is needed, see HELP GENSTAT Extensions.
\item The character space can hold 102400 characters.
\item The space for Identifiers can take 8000 names.
\item The space holding the current record from each open input channel and
 a copy of all text structures being used via \#\# is 8192 characters long.
\end{itemize}
Parts of these data spaces are used by the system.
The compressed form of a complete command (comments removed, multiple
spaces reduced to one space) must not exceed 2048 characters.

\section{Input/output channels}
\begin{itemize}
\item Input records are assumed to be 80 characters long.
\item Output records are assumed to be 132 characters long, except that the
 (primary) output to a terminal will be of 80 character records.
\end{itemize}
The user can override these assumptions by using an OPEN statement with
an appropriate setting of the WIDTH parameter.
The maximum permitted record length is 200 characters.

\section{Not implemented yet}

The directive FUNCTION, the CURVE=SPLINE option of INTERPOLATE,
and the BREAK option of PRINT are not implemented for Release 1.

Qualified identifiers (e.g.\ YIELD\$[5,7,9] ) are not implemented everywhere,
and will not be for Release 1. But:
\begin{itemize}
\item The compiler accepts references to single values of such forms
 (e.g.\ X\$[W\$[I]] where I has only one value).
 They may be used in lists of values and as list multipliers.
\item Qualified identifiers are accepted by the CALCULATE command for
 one-dimensional structures and matrices (but not tables).
\end{itemize}

\section{Examples}

The example files listed below should be available in directory GEN5\$DIR.
You can also enter Genstat interactively and type HELP; parts of the Reference
Summary will be output to the screen at your request.
\begin{verbatim}
    Analysis of Designed Experiments                 ANOVA and AVCCOX
    Declarations of Structures                                 DECLARE
    Help                                                       HELP
    Graphics                                                   GRAPH
    Input streams                                              INPUT
    Job control and procedures                                 JCP
    Lineprinter Graphs                                         LPGRAPH
    Manipulation of data structures                            MANPLATE
    Multivariate Analysis                                      MLTVAR
    Print                                                      OUTPUT
    Randomise                                                  RANDOM
    Read                                                       READ
    Regression:
     Fitting curves     CURVES;  Generalised Linear Models     GLM
     Linear models      LINEAR;  Fitting models (Optimisation) NONLINER
     Sums of Squares and Products                              SSP
    Structure Storage and Retrieval                            STORE
    Time Series Analysis                           BOXJENK and SPECTRAL
\end{verbatim}
Input files are of type .DAT and the corresponding output files are of
type .GNX.
Because Starlink's version of Genstat has been adapted to use GKS for its
high-resolution graphical output, the output obtained by running the
examples may differ slightly from that stored in GEN5\$DIR.
In particular, this will affect the data space used and left, and some
details of the output from the DUMP directive (which occurs in the example
files OUTPUT and SSP).

\section{Notice Board}

Information on Errors, Warnings and Hints is stored on a notice board which
can be accessed by typing:
\begin{verbatim}
    $ HELP GENSTAT NB
\end{verbatim}
\end{document}
