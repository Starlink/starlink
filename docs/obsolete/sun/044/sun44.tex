\documentstyle{article}
\pagestyle{myheadings}

%------------------------------------------------------------------------------
\newcommand{\stardoccategory}  {Starlink User Note}
\newcommand{\stardocinitials}  {SUN}
\newcommand{\stardocnumber}    {44.1}
\newcommand{\stardocauthors}   {C A Clayton}
\newcommand{\stardocdate}      {28 July 1988}
\newcommand{\stardoctitle}     {VAXnotes --- An introduction}
%------------------------------------------------------------------------------

\newcommand{\stardocname}{\stardocinitials /\stardocnumber}
\markright{\stardocname}
\setlength{\textwidth}{160mm}
\setlength{\textheight}{240mm}
\setlength{\topmargin}{-5mm}
\setlength{\oddsidemargin}{0mm}
\setlength{\evensidemargin}{0mm}
\setlength{\parindent}{0mm}
\setlength{\parskip}{\medskipamount}
\setlength{\unitlength}{1mm}

\begin{document}
\thispagestyle{empty}
SCIENCE \& ENGINEERING RESEARCH COUNCIL \hfill \stardocname\\
RUTHERFORD APPLETON LABORATORY\\
{\large\bf Starlink Project\\}
{\large\bf \stardoccategory\ \stardocnumber}
\begin{flushright}
\stardocauthors\\
\stardocdate
\end{flushright}
\vspace{-4mm}
\rule{\textwidth}{0.5mm}
\vspace{5mm}
\begin{center}
{\Large\bf \stardoctitle}
\end{center}
\vspace{5mm}

This Starlink User Note has been produced in order to provide a basic
guide to the use of the VAXnotes conferencing facility.
At present, there are two sources of instruction on the use of VAXnotes
in addition to this document:

\begin{itemize}

\item The VAXnotes on-line help facility is effective once you
have already had some experience with VAXnotes. In addition to having
the usual lists of commands along with their qualifiers, there are
also brief guides to using VAXnotes.

\item Guide to VAXnotes and VAXnotes command dictionary. This DEC
guide resides in one of the small orange document binders. Unfortunately,
there is generally only one of these guides per site. In
addition this guide is almost too detailed and can put off the novice user.


\end{itemize}

This user note is intended to fill the gap between the above two sources
of information by providing a gentle introduction to the main features and
new terminology.
In addition, it describes VAXnotes from an astronomer's
point of view and details the implementation of VAXnotes by Starlink.
Throughout this document you will see entries such as (4.1.4).
This is to tell the reader to look at section 4.1.4 in the ``Guide to
VAXnotes'' manual for further information on the subject under discussion.

\section{What is VAXnotes?}

VAXnotes is a computer conferencing facility which allows users to create and
access on-line conferences. VAXnotes allows users to conduct
what may be regarded as meetings, with people in different geographical locations, so that participants
can join in a discussion from their own desks at a time of their own choice. It
also offers the advantage of keeping a detailed record of the proceedings of a
meeting which can be searched by a variety of criteria, such as name of
participant, subject or keyword.

A private meeting to discuss Starlink security, an electronic
bulletin board to announce astronomical seminars, or a public forum on
circumstellar dust might all be examples of VAXnotes {\bf Conferences}.

A conference is organized into any number of {\bf Topics}, each of which can
have any number of {\bf Replies}.  A topic and its associated replies are
called a {\bf Discussion}.  Topics are identified as n.0, where n is a
number assigned sequentially by the system.  Replies to a topic note
are also sequentially numbered, starting with .1.  If topic 4 has 3
replies, the replies are numbered 4.1, 4.2, and 4.3.  Topics and
replies are referred to collectively as {\bf Notes}.

The first time you invoke VAXnotes, a {\bf Notebook} is automatically
created for you.  In your Notebook, you store the names of conferences
you are interested in.  This figure shows how a typical Notebook might
be organized, and how notes are arranged in a conference:

\begin{verbatim}



    +--------------------------------+     +---------------+     +-------------+
    |             VAXnotes           |     | Your Notebook |   ->| Conference J|
    | +-------------+ +------------+ |     |               |  /  |             |
    | |     All     | |    Your    | | --->|  conference F | /   | Topic 1     |
    | |             | |  Notebook  | |/    |  conference J _/    |   Reply 1.1 |
    | |  Available  | |    ****    |_/     |  conference P |     |   Reply 1.2 |
    | |             | |(conferences| |     +---------------+     | Topic 2     |
    | | Conferences | |     of     | |                           |   Reply 2.1 |
    | |             | |  interest) | |                           |   Reply 2.2 |
    | +-------------+ +------------+ |                           |   Reply 2.3 |
    +---|----------------------------+                           |      .      |
        |     +---------------+                                  |      .      |
        |     | conference  A |                                  +-------------+
        |---> | conference  B |
              | conference  C |
              |     .         |
              |     .         |
              |     .         |
              | conference  Z |
              +---------------+

\end{verbatim}

In an active conference, you need some way of keeping up with the latest
conversation. VAXnotes maintains a record of the notes that you have or
have not read in a conference, and provides convenient ways for you to
access only the notes that are new to you. Until you read the note, it
is marked as {\bf unseen}. VAXnotes marks notes you have read as {\bf seen}.

\section {Basic VAXnotes Commands}

To invoke VAXnotes, type:

\begin{quote}\tt
\$ NOTES
\end{quote}

A sample conference exists at every site. You can use this sample
conference to get hands--on experience with VAXnotes. When you invoke
VAXnotes for the first time, an entry for this conference is automatically
added to your Notebook and given the entry name SAMPLE\_CONFERENCE.

You can get online HELP for VAXnotes commands and key definitions
at any time during a VAXnotes session.
To get help on VAXnotes commands, type:

\begin{quote}\tt
Notes> HELP
\end{quote}

If an error occurs while in VAXnotes and you need more information
about the cause of it, type SHOW ERROR.  The full text of the last
error message is displayed on the screen.

Users are strongly recommended to become familiar with the VAXnotes keypad.
Most of the keys are used for moving around within and between
discussions in a conference with great speed.
To see the key definitions for the VAXnotes keypad, press PF2.

Before you can read the notes in a conference, you must first add the
conference to your Notebook, then open the conference. The first step is
to find out what conferences are available and to which you are allowed access:

\begin{quote}\tt
Notes> DIRECTORY/CONFERENCES
\end{quote}

The name of each conference appears in capital letters.  The title briefly
describes the subject of the conference.

To see a listing of the conferences at a remote node type:

\begin{quote}\tt
Notes> DIRECTORY/CONFERENCE nodename
\end{quote}

To add a conference to your Notebook, use the ADD ENTRY command.  For
example, to add the SUPERNOVA conference to your Notebook, type:

\begin{quote}\tt
Notes> ADD ENTRY SUPERNOVA
\end{quote}

If the conference is located on a remote node (e.g. QUVAD), then
you must instead issue the command:

\begin{quote}\tt
Notes> ADD ENTRY QUVAD::SUPERNOVA
\end{quote}

To open a conference, use the OPEN command.  For example, to open the
SUPERNOVA conference, type:

\begin{quote}\tt
Notes> OPEN SUPERNOVA
\end{quote}

A conference is organized into discussions.  Each discussion begins with a
topic, each having any number of replies.  Topics are numbered consecutively,
starting with 1.  Replies are also numbered consecutively, by topic, starting
with .1.

To see a listing of the topics in a conference, type:

\begin{quote}\tt
Notes> DIRECTORY
\end{quote}

To include replies, as well as topics, in the listing, type:

\begin{quote}\tt
Notes> DIRECTORY *.*
\end{quote}

The DIRECTORY command has many command qualifiers (2.18). For example,
it is possible to get directory listing of notes by a given author,
those created before or after a certain date, or those which contain a given
character string in the title.

Use NEXT SCREEN (0 on the keypad) and PREVIOUS SCREEN (. on the keypad)
to scroll through the directory.

To read a specific topic or reply, type the note-ID number at the Notes$>$
prompt.  For example, to read the third reply to topic 6, type:

\begin{quote}\tt
Notes> 6.3
\end{quote}

Use the ENTER or RETURN key to page through the replies in a discussion.

In order to read only new notes use NEXT UNSEEN (, on the keypad).

VAXnotes provides three separate screen-oriented user interfaces for
editing, allowing users to specify use of EVE, EDT-style keypad
interface, or WPS-style Gold keypad interface on ANSI-compatible terminals.
A full line mode editor is also available for non-ANSI terminal devices.

To set up EVE, the VAXnotes editor, with the EDT-style keypad, type the
following command:

\begin{quote}\tt
Notes> SET PROFILE/EDITOR=EDT
\end{quote}

To add a reply to a discussion, use the REPLY command.  You must be reading
a reply (or the topic) in the discussion before you issue the REPLY command.
For example, to add a reply to the discussion on topic 10, type:

\begin{quote}\tt
     Notes> 10

     Notes> REPLY
\end{quote}

To start a discussion on a new topic, type:

\begin{quote}\tt
Notes> WRITE
\end{quote}

To print notes, use the PRINT command.  If you do not specify a range of
notes to print, VAXnotes assumes you want to print the note you are
currently reading.  For example, to print topic 3 and all replies to that
topic, type:

\begin{quote}\tt
Notes> PRINT 3.*
\end{quote}

To extract (or save) the note you are currently reading
into a file called SPECTRUM.TXT, type:

\begin{quote}\tt
Notes> EXTRACT SPECTRUM.TXT
\end{quote}

To leave a conference, use the CLOSE command.  You can also press CTRL/Z
in place of the CLOSE command.  For example, to close a conference:

\begin{quote}\tt
Notes> CTRL/Z
\end{quote}

To update the record of which notes you have or have not read type:

\begin{quote}\tt
Notes> UPDATE
\end {quote}

To end your VAXnotes session, use the EXIT command (or press CTRL/Z).
For example, to exit VAXnotes, type:

\begin{quote}\tt
Notes> EXIT
\end{quote}

The following table shows introductory commands and gives a brief
description of each command.

\begin{tabular}{ l l }
& \\
\hline
& \\
{\large \bf Command}  & {\large \bf Description} \\
& \\
\hline
& \\
ADD ENTRY                &Adds a conference to your Notebook.\\
CLOSE (or CTRL/Z)        &Leaves a conference and returns you to your Notebook.\\
DIRECTORY                &Lists notes in a conference.\\
DIRECTORY/CONFERENCES    &Lists available conferences.\\
DIRECTORY/NOTEBOOK       &Displays a listing of the entries in your Notebook.\\
EXIT (or CTRL/Z)         &Ends a VAXnotes session.\\
EXTRACT (or SAVE)        &Creates a text file containing the notes you specify.\\
HELP                     &Accesses on--line HELP about VAXnotes commands.\\
NOTES                    &Invokes VAXnotes from DCL level.\\
OPEN                     &Opens an existing conference in your Notebook.\\
PRINT                    &Prints notes.\\
READ                     &Displays a note.\\
REPLY                    &Enters a reply to a topic.\\
SHOW ERROR               &Displays more information after an error.\\
UPDATE                   &Updates the ``unseen'' count in the Notebook directory.\\
WRITE                    &Enters a new topic.\\

\end{tabular}

\section {Your Notebook}

Your VAXnotes Notebook defines your VAXnotes working environment. Your Notebook
contains your personal list of conferences of interest to you ({\bf entries})
and a {\bf profile} that tailors your VAXnotes sessions.

An entry is a conference that you have added to your Notebook. When you
make a conference an entry in your Notebook, VAXnotes keeps track
of which notes you have and have not seen in this conference.

A profile is automatically created for you when you first invoke VAXnotes,
and contains all the information that VAXnotes needs to operate. Your profile
is useful for simplifying tasks that you perform repeatedly and for
personalizing notes that you write. Your current profile settings
can be displayed by issuing the command:

\begin{quote}\tt

Notes> SHOW PROFILE

\end{quote}

In your profile you can tell VAXnotes:

\begin{itemize}

\item which editor to use by default with commands such as
SET PROFILE/EDITOR=EDT or SET PROFILE/EDITOR=EVE.

\item which qualifiers to use with PRINT with commands such as
SET PROFILE/PRINT = ``COPIES=2 /NOTIFY''.

\item what personal name you wish to appear with your username
with commands such as SET PROFILE/PERSONAL=``Dr. John Blout''

\item whether you wish to see a directory, the first unseen note,
or neither when you first access a conference. This can be done
using commands such as SET PROFILE/AUTOMATIC=DIRECTORY or
SET PROFILE/NOAUTOMATIC.

\end{itemize}

\section {Other VAXnotes commands and features}

The purpose of this section is to alert users to the existence
of some of the more advanced
VAXnotes commands and features, but not to describe them in any great
detail. The interested user is directed to the Guide to VAXnotes manual (to
which section references are given).

\begin{itemize}

\item {\bf Markers} (4.18) are user-defined names pointing to special
entries in a note file. Markers can be used whenever a user needs to have a
special reminder of things to do, or to recall notes that are frequently
referenced.

\item {\bf Keywords} (2.5) allow the user to group notes that are concerned
with a particular subject or do not have other attributes (such as title,
author or time on entry) in common. Keywords are useful for grouping notes
that may not have a keyword in the note text or title, which can be searched
for, but which do address the subject the keyword represents.

\item A {\bf Class} (4.3) consists of a group of conferences that are related
in some way. A class works like a tabbed divider in your notebook to keep
related conferences together. You can also think of a class as something like a
mail folder in MAIL, or a VAX/VMS subdirectory.  Unlike a mail folder, however,
a conference can be in more than one class at a time.  You can create as many
classes as you want in your Notebook.

\item It is possible to SEARCH (6.3) a conference for a user specified
string. For example, if you remember reading a note which mentioned
at some point in it the work of Hubble, you could search for the
string ``Hubble'' to locate the note.

\item Once in a while, you may wish to suspend VAXnotes temporarily
so that you can start a batch job, read a file, etc. VAXnotes
provides a SPAWN command (6.4) to let you create a subprocess.

\item Using the SET NOTE command (6.2) it is possible to put a conference
on a ``shopping list'', hide a note you wrote, and prohibit users from
replying to your note.

\item VAXnotes allows notes and replies to be created outside of VAXnotes
and then later imported into the conference (3.3). This is strongly recommended.
One advantage is that it is then possible to put the file through the spelling checker
(which most Starlink sites have) before including your note in a conference.

\item  VAXnotes allows both public and private conferences (5.2). In private
conferences a {\bf moderator} restricts access to a specific group
of users. An example of a private conference is the SYSTEMS\_MANAGEMENT
conference at RL780 where sensitive security matters are sometimes
discussed. Public conferences have no restrictions on who may participate.
Most conferences on Starlink are of this type.

\item VAXnotes uses VMSmail to allow users to send notes or messages to
other users from within VAXnotes (3.6). This is equivalent to `whispering'
or passing notes at a meeting.

\end{itemize}

\section {Starlink implementation of VAXnotes}

Every Starlink site has VAXnotes installed. However, due to licensing
restrictions, VAXnotes is generally only available on one CPU per site.
On Local Area VAXclusters, this is the boot node.

VAXnotes conferences can exist at any Starlink site. Such a
conference can be accessed from any other site. It is not necessary
to have an account on the node on which the conference you wish to
access is located.

The conference on RL780 entitled CONFERENCES contains a shopping list of
conferences available on Starlink which are of general interest to
the whole astronomical community (i.e.\ not site-specific) along
with a brief description of the nature of the conference. Each topic
in CONFERENCES is linked to the conference that it describes
so that it is possible to add the conference to your Notebook
by simply issuing the SELECT command (7 on the keypad) whilst
you are reading the associated topic.

Most sites have a conference called LOCAL\_CONFERENCES which functions as
described above but only contains details of conferences of local interest
(e.g. seminar lists, hardware information).



\section {VAXnotes for astronomers}

VAXnotes has the potential to be of great benefit to the Starlink community.
Below are just a few examples of how astronomers might use the facility.

\begin{itemize}

\item VAXnotes is the ideal tool for maintaining up-to-date seminar lists by
using a ``Bulletin Board'' style conference.

\item Conferences on highly topical subjects (e.g. new supernova in Local Group)
can be set up in  order to keep interested parties informed of the very latest
facts.

\item Conferences on topics such as \TeX / \LaTeX { } or a graphics package
allows users to exchange tips and present (and have answered) questions
concerning these packages that are not covered clearly by the manuals.

\item Local conferences on hardware purchases and the future plans of Starlink
can be set up to allow users to discuss these matters and give site managers
and Starlink management a better feel for the needs and wishes of their users.

\item Lists of catalogues available on tape which individual users have
acquired (or ordered) could be set up in order to avoid duplication and
minimise the delay in obtaining such data.

\item Each Special Interest Group (SIG) has an on-going conference to keep
members up-to-date on any progress that is made. The occasional meetings of
these groups are seldom sufficient to discuss all aspects of the area of
interest embraced by the SIG.

\end{itemize}

In short, any group of astronomers with a common interest who need to be aware
the activities of other members can do so, without the need for meetings
(costly in time and money), from their own offices. The possibilities
are enormous and the above examples are only a few personal suggestions of how
VAXnotes may be of use to you. If you feel that there is a need for a
conference at your site on a particular subject, then simply ask your site
manager to create and advertise one for you. Topics which are likely to be of
interest to the whole of Starlink should be set up at RLVAD. In this case you
should again approach your local manager who will contact the relevant person.


\section {VAXnotes Command Summary}

This section list all VAXnotes commands alphabetically, and gives a brief description of each command.
The basic commands listed in Section 2 are also included. This list is provided as a reference and
not all the commands below are discussed elsewhere in this user note.

\begin{tabular}{ l l }
& \\
\hline
& \\
{\large \bf Command}  & {\large \bf Description} \\
& \\
\hline
& \\
ADD ENTRY               &Adds a conference to your Notebook.\\
ADD KEYWORD		&Adds an existing keyword to a note.\\
ADD MARKER		&Adds a marker to the specified note.\\
ADD MEMBER		&Adds a new member to the conference.\\
ANSWER			&Is a synonym for Reply.\\
ATTACH			&Attaches to the parent or specified process.\\
BACK			&Displays the previous reply in the current discussion.\\
BACK NOTE		& Displays the previous note.\\
BACK REPLY 		& Displays the previous reply.\\
BACK TOPIC		& Displays the previous topic.\\
CLOSE			& Lets you leave a conference.\\
CREATE CONFERENCE	& Creates a new conference.\\
CREATE KEYWORD		& Creates a new keyword for the conference.\\
DELETE ENTRY		& Removes a conference from your Notebook.\\
DELETE KEYWORD		& Deletes a keyword.\\
DELETE MARKER		& Deletes a marker from a note.\\
DELETE MEMBER		& Deletes a member from a conference.\\
DELETE NOTE		& Deletes a note you entered in the conference.\\
DIRECTORY		& Lists topics and replies in a conference.\\
DIRECTORY/CONFERENCES	& Lists available conferences.\\
DIRECTORY/ENTRIES	& Lists entries in a specified class.\\
DIRECTORY/NOTEBOOK	& Is a synonym for Directory/entries.\\
EVE			& Invokes the EVE editor.\\
EXIT			& Ends a VAXnotes session.\\
EXTRACT			& Is a synonym for Save.\\
EXTRACT/BUFFER		& Is a synonym for Save/Buffer.\\
FORWARD			& Forwards a note by MAIL.\\
HELP			& Accesses on--line HELP about VAXnotes commands.\\
LAST			& Displays the last topic in the conference (if outside VAXnotes).\\
LAST			& Displays the last reply in the current discussion.\\
\end{tabular}

\begin{tabular}{ l l }
& \\
\hline
& \\
{\large \bf Command}  & {\large \bf Description} \\
& \\
\hline
& \\
MARK			& Is a synonym for Add/Marker.\\
MODIFY ENTRY		& Changes information about an entry.\\
MODIFY KEYWORD		& Modifies the name of a keyword.\\
MODIFY MEMBER		& Modifies information about a member.\\
NEXT			& Displays the next reply or, if none, the next unseen note.\\
NEXT NOTE		& Displays the next note.\\
NEXT REPLY		& Displays the next reply.\\
NEXT TOPIC		& Displays the next topic.\\
NEXT UNSEEN		& Displays the next note you have not yet seen.\\
NOTES			& Invokes VAXnotes from the DCL level.\\
OPEN			& Opens the specified conference in your Notebook.\\
PRINT			& Prints notes.\\
READ 			& Displays the specified note.\\
READ/MARKER		& Displays the note having the specified marker name.\\
REPLY			& Adds a reply to the current discussion.\\
SAVE			& Saves notes in the specified VAX/VMS text file.\\
SAVE/BUFFER		& Saves notes in the specified buffer.\\
SEARCH			& Searches a range of notes for notes containing a specified string.\\
SEND			& Sends a message to one or more users by MAIL.\\
SET CLASS		& Makes the specified class the default class.\\
SET CONFERENCE		& Sets characteristics for a conference.\\
SET MODERATOR		& Enables moderator privileges.\\
SET NOTE		& Sets attributes for the specified note.\\
SET PROFILE		& Creates or changes settings in your profile.\\
SHOW CLASSES		& Lists classes in your Notebook.\\
SHOW CONFERENCES	& Shows information about the current conference.\\
SHOW ENTRY		& Shows information about a specified conference in your Notebook.\\
SHOW ERROR		& Shows the full text of the last error.\\
SHOW KEYWORDS		& Shows information about current keywords.\\
SHOW MARKER		& Shows the note number associated with the specified marker.\\
SHOW MEMBER		& Shows information about all or specified members.\\
SHOW MODERATOR		& Shows the names of current moderators.\\
SHOW NOTE		& Shows information about the specified note.\\
SHOW PROFILE		& Shows information about current profile settings.\\
SHOW VERSION		& Shows information about the current version of VAXnotes.\\
SPAWN			& Creates a subprocess.\\
TOPIC			& Displays the topic for the current discussion.\\
TPU			& Executes the rest of the command line as a VAXTPU command.\\
UPDATE			& Determines if new notes have been entered into your Notebook.\\
WRITE			& Adds a new topic in the current conferences.\\
\end{tabular}

\end{document}
