\documentstyle[11pt]{article}
\pagestyle{myheadings}

%------------------------------------------------------------------------------
\newcommand{\stardoccategory}  {Starlink User Note}
\newcommand{\stardocinitials}  {SUN}
\newcommand{\stardocnumber}    {151.2}
\newcommand{\stardocauthors}   {C A Clayton\\S.J. Tappin}
\newcommand{\stardocdate}      {30 September 1992}
\newcommand{\stardoctitle}     {VMSBACKUP --- VMS backup reader for UNIX}
%------------------------------------------------------------------------------

\newcommand{\stardocname}{\stardocinitials /\stardocnumber}
\renewcommand{\_}{{\tt\char'137}}     % re-centres the underscore
\markright{\stardocname}
\setlength{\textwidth}{160mm}
\setlength{\textheight}{230mm}
\setlength{\topmargin}{-2mm}
\setlength{\oddsidemargin}{0mm}
\setlength{\evensidemargin}{0mm}
\setlength{\parindent}{0mm}
\setlength{\parskip}{\medskipamount}
\setlength{\unitlength}{1mm}

%------------------------------------------------------------------------------
% Add any \newcommand or \newenvironment commands here
%------------------------------------------------------------------------------

\begin{document}
\thispagestyle{empty}
SCIENCE \& ENGINEERING RESEARCH COUNCIL \hfill \stardocname\\
RUTHERFORD APPLETON LABORATORY\\
{\large\bf Starlink Project\\}
{\large\bf \stardoccategory\ \stardocnumber}
\begin{flushright}
\stardocauthors\\
\stardocdate
\end{flushright}
\vspace{-4mm}
\rule{\textwidth}{0.5mm}
\vspace{5mm}
\begin{center}
{\Large\bf \stardoctitle}
\end{center}
\vspace{5mm}

%------------------------------------------------------------------------------
%  Add this part if you want a table of contents
%  \setlength{\parskip}{0mm}
%  \tableofcontents
%  \setlength{\parskip}{\medskipamount}
%  \markright{\stardocname}
%------------------------------------------------------------------------------

\section{Introduction}

VMSBACKUP is a utility written by John Douglas Carey and Sven-Ove  Westberg
(with some local extensions) that allows VMS backup tapes to be read on a Unix
machine. VMSBACKUP is available on the Sun and DECstation platforms.

\section{Usage}

\begin{quote}
{\tt
     vmsbackup -\{tx\}[bcdevw][s setnumber][f tapefile] [ name ... ]
}
\end{quote}

Vmsbackup reads a VMS generated backup tape, converting  the files  to  Unix
format  and writing the files to disk.  The default operation of the program is
to go through an  entire tape,  extracting  every  file and writing it to disk.
This may be modified by the following options:

\begin{description}
\begin{description}

\item [b] --- Read files in binary mode. This affects only record formats
variable and cr-terminated stream. If \verb!-b! is specified then these files
are read without interpretation of the RMS record terminators into linefeeds
(equivalent to using binary mode in \verb!ftp!).

\item [c] --- Use  complete  filenames,  including   the   version number. A
colon and the octal version number will be appended to all filenames. A colon,
rather than a semicolon, is used since the Unix Shell uses the semicolon as the
line separator. Using a colon prevents the user from having to escape the
semicolon when referencing the filename.  This option is useful  only when
multiple versions of the same file are on a single tape or when a file of the
same name already  exists  in  the destination directory.  The default is to
ignore version numbers.

\item [d] --- Use the directory structure from  VMS,  the  default value is
off.

\item [e] --- Process all filename extensions.  Since this program is  mainly
intended to move source code and possibly data from a DEC system to a Unix
system, the default is  to  ignore  all  files  whose filename extension
specifies system dependent  data.   The  file  types which will be ignored,
unless the {\bf e} option is specified, are

\begin{itemize}
\item               [exe] -- VMS executable file
\item               [lib] -- VMS object library file
\item		    [obj] -- RSX object file
\item               [odl] -- RSX overlay description file
\item               [olb] -- RSX object library file
\item               [pmd] -- RSX post mortem dump file
\item               [stb] -- RSX task symbol table file
\item               [sys] -- RSX bootable system file
\item               [tsk] -- RSX executable task file
\item               [tlb] -- RSX text library file
\item               [hlb] -- RSX help library file
\end{itemize}

\item [f] --- Use the next argument in the  command  line  as  the tape device
to be used, rather than the default. The default is {\tt /dev/rmt0h} (raw
mode, high density, drive 0).  This must be a raw mode tape device.

\item [s] --- Saveset. Process only the given saveset number.

\item [t] --- Produce a table of contents (a directory listing) on the
standard output of the files on tape.

\item [v] --- Verbose output.  Normally vmsbackup  does  its  work silently.
The   verbose  option  will  cause  the filenames of the files being read from
tape to  disk to be output on the standard output.

\item [w] --- Vmsbackup prints the action to be taken followed  by file  name,
then  wait  for user confirmation. If a word beginning with `y'  is  given,
the  action  is done. Any other input means don't do it.

\item [x] --- Extract the named files from the tape.

\end{description}
\end{description}

The optional name argument specifies one or more filenames to  be searched for
specifically on the tape and only those files are to be processed.  The name
may contain the usual shell meta-characters \verb+*?![]+.

When a list of filenames is given, then the matching observes the following
rules:
\begin{itemize}
\item All strings are converted to lower case before comparing.
\item If the \verb!-d! option is \underline{not} specified, the argument
strings are compared with just the filename of the file in the save set (i.e.
the part after the last \verb!]!), if \verb!-d! is given then the directory is
included in the match (in VMS form).
\item If the \verb!-c! option is \underline{not} specified then the version
number is ignored in the match, if \verb!-c! is given then the version number
is included.
\item The wildcards \verb!*! and \verb!?! along with choice and range
constructs (e.g. \verb![a-z]! (all chars between `a' and `z') or \verb![135]!
(`1', `3' or `5')) may be used. BUT they must be enclosed in single quotes to
prevent interpretation by the shell.
\end{itemize}

\end{document}
