\documentstyle[11pt]{article} 
\pagestyle{myheadings}

%------------------------------------------------------------------------------
\newcommand{\stardoccategory}  {Starlink User Note}
\newcommand{\stardocinitials}  {SUN}
\newcommand{\stardocnumber}    {98.5}
\newcommand{\stardocauthors}   {R D Saxton}
\newcommand{\stardocdate}      {31 October 1991}
\newcommand{\stardoctitle}     {ASTERIX --- X-ray Data Processing System}
%------------------------------------------------------------------------------

\newcommand{\stardocname}{\stardocinitials /\stardocnumber}
\renewcommand{\_}{{\tt\char'137}}     % re-centres the underscore
\markright{\stardocname}
\setlength{\textwidth}{160mm}
\setlength{\textheight}{230mm}
\setlength{\topmargin}{-2mm}
\setlength{\oddsidemargin}{0mm}
\setlength{\evensidemargin}{0mm}
\setlength{\parindent}{0mm}
\setlength{\parskip}{\medskipamount}
\setlength{\unitlength}{1mm}

\begin{document}
\thispagestyle{empty}
SCIENCE \& ENGINEERING RESEARCH COUNCIL \hfill \stardocname\\
RUTHERFORD APPLETON LABORATORY\\
{\large\bf Starlink Project\\}
{\large\bf \stardoccategory\ \stardocnumber}
\begin{flushright}
\stardocauthors\\
Dept of Physics, University of Leicester\\
\stardocdate
\end{flushright}
\vspace{-4mm}
\rule{\textwidth}{0.5mm}
\vspace{5mm}
\begin{center}
{\Large\bf \stardoctitle}
\end{center}
\vspace{5mm}

\setlength{\parskip}{0mm}
\tableofcontents
\setlength{\parskip}{\medskipamount}
\markright{\stardocname}

\newpage

\section{Introduction }

This document is aimed at the beginner to the ASTERIX data analysis system. It
is intended to supplement the ASTERIX help facility by explaining some of the
general concepts which a user will encounter. It does not discuss the details
of any application, this information may be found from the on-line  help system
or from the Starlink miscellaneous user document `ASTHELP.HLP --- An ASTERIX
Help Guide'

ASTERIX is a collection of applications written to analyse astronomical data in
the X-ray waveband. Many of these applications are however general  purpose and
are capable of analysing any data which is received in the correct format. The
package has been designed to be instrument independent and currently has
interfaces to the Exosat and ROSAT instruments. 

\section{Getting started }

The applications run under the ADAM environment which in turn uses the ICL
command language. To load the necessary definitions for running ASTERIX
type the following:
\begin{verbatim}
    $ ASTSTART     - loads VMS/DCL definitions 
    $ ICL          - starts ICL
    ICL> ASTERIX   - loads ASTERIX definitions into ICL
\end{verbatim}             
Now simply type the name of an application:       
\begin{verbatim}
    ICL> INTERP
\end{verbatim}

\section{Running an application }

The recommended way of running ASTERIX programs is to enter the ICL command
language as described above and execute commands from there. However it is
possible to execute most ASTERIX commands from DCL in an identical manner.  The
advantages of running under ICL are that applications invoke much more quickly,
and that commands which invoke ICL procedures to perform composite tasks are
available. However there is an initial delay whilst the system is loaded, so
that when only one or two simple commands are to be issued, it may be quicker
to stick with DCL.

In either case, the ADAM parameter system is used to pass parameters in and out
of applications. Further information about ADAM and ICL can be found in
Starlink document SG/4: `ADAM --- The Starlink Software Environment'.

\subsection{Running under ICL }

The simplest method of running an application interactively is to just
type the application name:
\begin{verbatim}
    ICL> BINSUBSET
\end{verbatim}
ADAM will then prompt the user for values which it needs and will take default
values for other parameters. To force ALL parameters to be prompted for, type:
\begin{verbatim}
    ICL> BINSUBSET PROMPT
\end{verbatim}
To save having to respond to prompts, parameter values may be entered on the 
command line. Either by position:
\begin{verbatim}
    ICL> BINSUBSET Input_filename Output_filename
\end{verbatim}
or by keyword:
\begin{verbatim}
    ICL> BINSUBSET INP=Input_filename OUT=Output_filename
\end{verbatim}
The position and keyword associated with each parameter is specified in the
help entry for each application. It is best to specify the value of
each (non-defaulted) parameter on the
command line when an application is run in batch.

Many parameters suggest default values along with the prompt, and
responding with $<$RETURN$>$ will cause these values to be accepted.
To force the application to accept all such defaults automatically
type:
\begin{verbatim}
    ICL> BINSUBSET ACCEPT      or     ICL> BINSUBSET \
\end{verbatim}
(If no defaults have been set by the system, the application will abort.)
This may also be used in conjunction with parameters,  e.g.:
\begin{verbatim}
    ICL> BINSUBSET NXBIN=30 
\end{verbatim}
will result in NXBIN being set to 30 and defaults being taken for all
the other parameters.

\subsection{Running under DCL }

Commands may be executed from DCL with an identical format to the ICL commands
described above. This is useful if you want to perform a quick  set of commands
or if the monolith is out of action. Note, however,  that not all commands are
available under DCL and in particular commands invoking ICL procedures, and all
Exosat commands must be executed from ICL.

\subsection{Replying to prompts }

When answering a prompt the user may reply with the following special
characters:

\begin{itemize}
\item To get help information on a parameter, type:  `?'.
\item To abort an application while being prompted, type:  `!!'.
\item For certain parameters a reply of  `!' (null response) has 
a special meaning; this should be detailed in the help where it applies.
\item Where a default response is offered as part of the prompt line this can
be edited using the normal DCL command line editor by first pressing
the `TAB' key.
\item Applications running under DCL can be aborted by typing `$<$CNTRL$>$ C'.
Unfortunately this will not work within ICL, and wherever possible one
should abort an application by entering `!!' at a prompt as this provides a 
clean exit back to ICL. In desperate circumstances one can use `$<$CNTRL$>$ Y',
but this will also abort ICL.
\end{itemize}

\subsection{User input }

Numerical data may be input in any form and will be translated by the parameter
system into the format required, e.g.\ a double precision number may be input 
as an integer or vice-versa. Character data may be input directly, it only
requires quotes if being entered on the command line and only then if it
contains spaces.

Ranges should be input using the following format:
\begin{verbatim}
    Numeric ranges use a colon          > 53.6:107.8
    For more than one range use a space > 53.6:107.8 120.5:156.0 
    Two dimensional ranges use a comma  > 53.6:107.8,54.9:83.0
    The combination looks like          > 53.6:107.8 120.5:156.0,54.9:83.0
\end{verbatim}
The convention used in ASTERIX is that lower bounds are inclusive and upper 
bounds (except where a default full range is offered) are exclusive.
Note that these ranges are character strings and the above comment about
quotes therefore applies.

Filenames should be entered WITHOUT the extension otherwise they will not be
accepted.

\section{Help system }

The ASTERIX HELP system is invoked by typing ASTHELP. Information is
grouped by topic, with a system of unique first characters for easy access.
You are advised to consult the `Getting\_started' section if using ASTERIX
for the first time.

Help on a specific application may be obtained directly by typing:
\begin{verbatim}
    ASTHELP application_name
\end{verbatim}
This may be done from the DCL or ICL prompt. Each application entry has a 
brief description followed by more details of the parameters required, the
method or algorithm used, examples of use, and any other relevant information.
Note that shortform and macro (ICL procedure) commands are distinguished in 
the HELP text by suffixes {\bf  } $+$ and $*$ respectively. These suffixes 
are not typed when invoking the commands. 

For hardcopy help, the output from ASTHELP may be piped to a text file by using
the following syntax: 
\begin{verbatim}
    ASTHELP > text_filename  
\end{verbatim}
or 
\begin{verbatim}
    ASTHELP > text_filename  application_name
\end{verbatim}
This writes the resulting help entry into a text file which
may subsequently be printed.

The first time ASTHELP is accessed in an ICL session it will take a while
starting as the system has to spawn a DCL subprocess. Subsequent HELP
enquiries will start immediately.

\section{Datafiles }

ASTERIX data is stored in HDS files. There are basically two different types:
binned and event datasets. These are described in detail in ASTERIX document
PROG\_002. The structure of any HDS file can be easily inspected using the HDS
screen editor HSE. 

ADAM assumes a default file extension of .SDF --- this is therefore used for 
all ASTERIX HDS files.

\subsection{Binned datasets }

Binned data (e.g. time series, spectra, images) are stored in files whose
structure is based on the Starlink standard `NDF' format described in Starlink
document SGP/38. Data errors (stored in the form of variances)  and quality are
catered for, and header information is stored in a special ASTERIX structure,
and is most easily inspected using the HEADER  application. In  general,
applications will treat errors as being independent  and Gaussian, and will
ignore (or warn the user about) bad quality data.

\subsection{Event datasets }

These files store information pertaining to a set of photon `events'. Each 
event will have a set of properties  e.g. X position, Y position, time, raw
pulse height. These properties are stored in a set of arrays of equal length
(-- number of events), each of which resides inside  a named LIST-type
structure inside the file. This allows all information from a photon counting
instrument to be  retained, whilst binning almost invariably results in some
loss of information.

\section{ASTERIX Overview }

The basic structure of the ASTERIX system is shown in Figure \ref{fig:str}.
Instrument interface routines can generally produce data in either binned or
event formats, leading to two different paths through the system. Most
analysis applications are designed to operate on binned data.

\begin{figure}
\begin{verbatim}
    -----------                -----------              -----------
   |  EXOSAT   |              |   ROSAT   |            |   Other   |
   | interface |              | interface |            | instrument|
    -----------                -----------              -----------
         |                          |                        |
         +--------------------------+------------------------+ 
                                    |
         +--------------------------+
         |        Event dataset     |     
    -----------                     |
   |   Event   |                    |
   | processing|                    |
   | routines  |                    |
    -----------                     |
         |                          |
    -----------                     |      Binned dataset
   |   Event   |                    |
   |  binning  |                    |
    -----------                     |    
         |                          |
    ------------                    |
   | Instrument |                   |
   | corrections|                   |
    ------------                    |
         |                          |                    
         +--------------------------+
                                    |      Binned dataset
                               ----------
                              |  Kernel  | e.g. Time analysis,spectral
                              | analysis |      fitting,image processing
                              | routines |      routines.
                               ----------          
                                    |
                               -----------
                              | Graphical |
                              |  output   |
                               -----------
\end{verbatim}
\caption{Basic Structure of ASTERIX System}
\label{fig:str}
\end{figure}

\section{ASTERIX Applications   }

A list of available applications is appended to this document. A regularly 
updated list can be accessed from ASTHELP, which also contains detailed 
information about each program.

\section{Graphics }

ASTERIX contains graphics facilities based on the PGPLOT package.  Full
details are provided in the help system, in particular you are advised
to look at the subsection `Getting\_started' in the GRAPHICS topic of
ASTHELP. 

To display an ASTERIX binned dataset type:  
\begin{verbatim}
    ICL>  DRAW  filename  devicename
\end{verbatim}
To find out which devices are available type: 
\begin{verbatim}
    ICL>  DEVICES 
\end{verbatim}
This should produce a list of graphics devices which can be used
on your system. If it doesn't then either the GKS/GNS graphics system
or ASTERIX has not been installed properly --- contact the system manager.

In addition to the DRAW command ASTERIX contains a separate image
processing package, which can only be used from ICL. It gives the 
ability to interactively display an image and manipulate it using
cursor commands. Type `ASTHELP K' for an overview of the image
processing system.

\section{Documentation }

A list of ASTERIX documents may be obtained by typing:
\begin{verbatim}
    $ DOCS
\end{verbatim}
There are two classes of document; the user documentation called USERnnn
and documentation for programmers, PROG\_nnn. Some of the latter may
be useful for general users although it is primarily intended for people
developing software for ASTERIX.

\section{ICL system commands }

The ADAM user guide, SG/4, includes a description of the facilities
of ICL and gives further references.
A few useful commands are listed below.

To show your default directory when in ICL type:
\begin{verbatim}
    ICL> DEFAULT
\end{verbatim}
To change your default directory type:
\begin{verbatim}
    ICL> DEFAULT  new_directory_name
\end{verbatim}
To list the entire contents of an HDS file type:
\begin{verbatim}
    ICL> TRACE  filename
\end{verbatim}
To escape to DCL temporarily (by starting a DCL subprocess) type:
\begin{verbatim}
    ICL> SPAWN 
\end{verbatim}
and return to ICL with a DCL logout.

To invoke a single DCL command preface the command with a `\$':
\begin{verbatim}
    ICL> $Command       e.g. $DIRECTORY
\end{verbatim}

\section{Influence of the environment }

As mentioned above, ASTERIX runs under the ADAM environment. ADAM stores
the values of parameters (input values) used in each application in its own HDS 
file and passes them between applications using another HDS file called 
GLOBAL.SDF . These files are stored in a directory which is pointed to by
the logical 'ADAMUSER'. ASTERIX frequently uses global values as 
defaults when a parameter is prompted for in an application, to save the user
unnecessary typing. A listing of currently defined global values may be
obtained by typing GLOBAL from the ICL prompt. 

The programs in ASTERIX have been divided into several separate sections 
of executable code known as `shared image monoliths'.
When a command is entered in ICL, it checks to see if the monolith containing
that program has already been loaded. If it hasn't, then the monolith is 
loaded, which can take several seconds. Once the monolith is in, however,
all the programs contained in it are instantly available for execution.

\section{Relationship to other Starlink software }

Since ASTERIX binned datasets conform to the Starlink standard, such data can 
also be analysed using any Starlink software which also conforms.
Conversely ASTERIX may be used to analyse suitable standard format data
produced by other packages. Several different packages can be loaded
simultaneously with the ICL/ADAM environment, and their applications
used in conjunction. However one has to be careful in crossing between
packages, since different packages support different subsets of the
full data standard specified in SGP38. For example, ASTERIX suports data
QUALITY but not, at present, magic values; whilst the reverse is true for
some other software. At present, ASTERIX supports a larger subset of the 
standard than any other major package, so that one can expect only partial
support of ASTERIX components when using `foreign' software.

\section{Troubleshooting \& Bug reporting}

This section describes ways of overcoming problems which may be encountered.
\begin{itemize}
\item If in response to typing `ICL' you get the message `HDS variable
   file not found', it probably means that the logical SYS\$SCRATCH 
   has been set to a strange location. Reset it to a more reasonable 
   scratch location.
\item Similarly, if an application tells you that it is having problems 
   mapping an array (it may do this very indirectly) then check the
   space available on the device pointed to by AST\_SCRATCH. In general
   it is not a good idea to assign AST\_SCRATCH to a device with quotas
   set up. 

\item If a task won't start in ICL and a message along the lines of `DCL 
   subprocess not started' comes up. It may be that the logical name 
   job table is full. The solution is to ask your system manager to 
   increase the job table byte quota in your UAF. Another possible 
   explanation is that you have exceeded your subprocess quota. It 
   is recommended that the subprocess quota should be set to ten. 
   Advice on this and other quotas is available from 
   ADAM\_DOCS:ARN012.RNO.

\item The message `HDS file not found' is a rather common error message 
   which could mean almost anything. If it occurs on exiting from a
   task then it could mean that the GLOBAL.SDF file in ADAM\_USER which
   stores global parameters has become corrupted. The solution is to
   delete this file and re-issue the command ADAMSTART.

\item ASTERIX makes extensive use of the virtual mapping facility
   provided in VMS. This can lead to quotas of various sorts being
   exceeded. ADAM/ICL often produces a message along the lines of
   `Quota exceeded' without actually saying which quota it means. 
   Initially, try setting your process quotas to those given by the
   ADAM\_DOCS:ARN012.RNO document. If the error persists then a 
   good bet is to ask your system manager to increase 
   the sysgen parameter VIRTUALPAGECNT. Unfortunately this is a
   non-dynamic parameter and so requires a system reboot to take
   effect. A value of 60000 is the absolute maximum you will need;
   don't bother trying to increase it beyond this value.

   The system quota problem often causes the message:
     `ADAM\_ERR: lib\$\% error activating image !AS'.

   One possible system parameter which may cause this is PROCSECTCNT. 
   Most systems can get away with having this set at 32 or 64 but
   The Cambridge X-ray machine (XMV) has recently had to up this 
   parameter to 96 to avoid this problem.

\item If an application which should be working, refuses to do anything
   except display an `ADAMERR' message then try exiting ICL and then
   going back into it.  It is anticipated that future releases will
   have a more tidy way of dealing with this.

\end{itemize}

Please report any bugs to either Richard Saxton (LTVAD::RDS)
or Bob Vallance (BHVAD::RJV).

\section{Writing an ASTERIX application }

The document AST\_DOCS:PROG\_012.LIS gives a detailed description of how to
write an Asterix application.
This section gives a brief overview.

Initially you should start up Asterix in development mode by:
\begin{verbatim}
    $ ASTSTART DEV
\end{verbatim}
An example Asterix application is given in AST\_DOCS:TEMPLATE.FOR.
This simply finds the average (or sum) of up to 20 datafiles.
It is probably best to use this as a starting point for writing a
brand new application . There is a good basic guide for writing ADAM
programs available in SG/4.

There are many useful subroutines for Asterix 
development in the library ASTLIB. For instance most of the components
of the Asterix datafile are easily accessible by using the BDA routines.
Dynamic memory can be easily obtained by using the DYN routines. The
contents of ASTLIB can be accessed with the development aid DOCLIB,
invoked from DCL by the command `DOCLIB'. Subroutines in this library
are grouped into categories defined by prefix, a list of which 
is available in AST\_DOCS:PREFIX.LIS

\subsection{The interface file }

Each ADAM application must have an interface file. This is used to describe
the parameters used in an application. An example for the MEANDAT program
can be seen in AST\_DOCS:TEMPLATE.IFL
 
\subsection{Linking an application }

After the application has been compiled, link it with the following command.
\begin{verbatim}
    $ ALINK <appname>,ASTOPT/OPT
\end{verbatim}

\section{Writing an ICL procedure }

ICL procedures are actually fairly simple to write --- but very hard to
debug. A description of the ICL programming language is available
in ICLDIR:ICL.TEX.

Two sample procedures can be seen in AST\_DOCS:TEMPLATE.ICL. The first,
XRTBCKBOX, is an example of a procedure interacting with the image
processing routines to obtain information to control sorting of raw
XRT data. The second, XRTMSPEC, shows how to use the output file from 
PSS to generate spectra from all the sources in an XRT field.

\newpage

\appendix

\section{Example session }

In this example, raw EXOSAT data is sorted into a source spectrum and a
background spectrum, a background subtracted spectrum is produced and a
spectral fit performed. Commentary is prefaced by asterisks, and some
program output has been omitted for brevity.

\begin{quote}\small
\begin{verbatim}
$ ADAMSTART                             * Start ADAM
  ADAM version 1.6 available
$ ASTSTART                              * Start ASTERIX
$ SET DEF DISK$SCRATCH:[RDS]
$ ICL                                   * Start ICL

Interactive Command Language   -   Version 1.5

  - Type HELP package_name for help on specific Starlink packages
  -   or HELP PACKAGES for a list of all Starlink packages
  - Type HELP [command] for help on ICL and its commands

 Help key ASTERIX redefined

loading ASTERIX command definitions........

Loading IGLU version 2.0
----------------------------------------------------------------------

 ASTERIX Version 1.0            - type ASTHELP for more information
                                - type ASTERIX to redefine commands

ICL> SHOWEXO                                 * List the available data
SHOWEXO version 1.0-2
LIST - List file for output /'lp.lis'/ > ?
Name of the text file used to log the results
LIST - List file for output /'lp.lis'/ > 
RAWDIR - Raw data directory /'DISK$SCRATCH:[RDS]'/ > DISK$SCRATCH:[DEMO.ME]
 111 files on this directory.
EXPT - Experiment (MEA, MEX, GSPC, K, L) /'MEA'/ > 
DTYPE - Data type of files /'E4'/ > E5
 Observation 1      DISK$SCRATCH:[DEMO.ME]MEDATA_M001.E5;1
 Observation 2      DISK$SCRATCH:[DEMO.ME]MEDATA_M002.E5;1
 Observation 3      DISK$SCRATCH:[DEMO.ME]MEDATA_M003.E5;1
 Observation 4      DISK$SCRATCH:[DEMO.ME]MEDATA_M004.E5;1
4 observation(s) of E5 data with experiment code MEA
Output is on LP.LIS

ICL> TYP LP.LIS                       * View the resulting file
Creating DCL subprocess
 SHOWEXO version 1.0-2  run on directory DISK$SCRATCH:[DEMO.ME]
 Expt-code MEA      Data-type E5  Layer Argon 
OBS DAY   START    STOP   EXPOS     R.A.    DEC      TARGET  Track OFF DETECTOR
 DATA REC.SHAPE    TDWELL --PHA-- RECS STA- 
SEQ     hh:mm:ss hh:mm:ss ~secs hh:mm:ss hh:mm:ss                  SET abcdefgh
 TYPE T1 PH ID T2    secs   range       TUS  OBS OBS
  1 243 07:32:22 07:40:06   472                   ---slew---  sssss oo 11011111
 E5    1  64 2  1   1.000   4  67  708 Ar    1    
  2 243 07:40:14 07:41:42    96  17:30: - 33:01:  RAPID BURST ggggg oo 11011111
 E5    1  64 2  1   1.000   4  67  144 Ar    2    
  3 243 07:41:50 07:45:42   240                   ---slew---  sssss oo 11011111
 E5    1  64 2  1   1.000   4  67  360 Ar    3    
  4 243 07:46:46 11:46:46 14216  17:02: - 36:22:  GX-349+2    gTTTT oo 11011111
 E5    1  64 2  1   1.000   4  6721324 Ar    4    

ICL> EXOMESORT                        * Sort observations 1 and 3 to
                                      * produce a spectrum from the 
                                      * slew data.
EXOMESORT Version 1.0-2
RAWDIR - Raw data directory /'DISK$SCRATCH:[RDS]'/ > DISK$SCRATCH:[DEMO.ME]
EXPT - Experiment (MEA / MEX / GSPC) /'MEA'/ > 
DTYPE - Data type of files to sort /'E5'/ > 
OBSLIST - Observations wanted /'  1:  4'/ > 1 3

Base epoch chosen as follows (all times in seconds are relative to this):
RECORD    SHF-KEY      SCLOCK    REF-TIME      MJD                 DATE
    2   178788742     2207376  -683147264    46308.31416   1985-Aug-31 07:32:23

 OBSERVATION  DATA-STARTS   DATA-ENDS RECORDS   ATTITUDE    SAT_STAT
      1            -1.156     469.844     708    Slewing
      3           566.844     805.844     360    Slewing

    On TARGET  -45261.39   -1940.435

     1) Time: -1.156128 - 805.8439  (seconds)
     2) PHA channel: 4 - 67   (channel no.)

RANGES - Which range(s) do you wish to change (Time=1,Pha=2) /'1'/ > !
                                       * Accept the ranges shown
 Ident chan.    1       2
 Detectors      ABD     EFGH
 Offsets        Source  Source

Mode: coalligned

NIDBINS - How many detector ID channels in the output file /1/ > 
ICHNS1 - Enter detector Ident chans. wanted in output channel 1 > 1:2
 Input IDENT                    1   2
 Mapped to output               1   1

     DATA AXES
 -----------------
  1) Time
  2) Energy

BIN_AXES - Axes required in data array /'1'/ > 2
ENBIN - Width of bins in the energy axis (channels) /1/ > 
There will be 64 bins, each of 1 raw PHA channel(s) , starting from 4
OUT - Binned dataset Filename /@DET4T/ > SLEW
Sorting records    1 to   708 of file DISK$SCRATCH:[DEMO.ME]MEDATA_M001.E
Sorting records    1 to   360 of file DISK$SCRATCH:[DEMO.ME]MEDATA_M003.E
Normalising to counts/sec
EXOMESORT complete

ICL> EXOMESORT                          * Produce a source spectrum
EXOMESORT Version 1.0-2
RAWDIR - Raw data directory /'DISK$SCRATCH:[RDS]'/ > DISK$scratch:[DEMO.ME ]
EXPT - Experiment (MEA / MEX / GSPC) /'MEA'/ > 
DTYPE - Data type of files to sort /'E5'/ > 
OBSLIST - Observations wanted /'  1:  4'/ > 4

Base epoch chosen as follows (all times in seconds are relative to this):
RECORD    SHF-KEY      SCLOCK    REF-TIME      MJD                 DATE
  110   178789606     2207808  -668991488    46308.32416   1985-Aug-31 07:46:47

 OBSERVATION  DATA-STARTS   DATA-ENDS RECORDS   ATTITUDE    SAT_STAT
      4            -1.156   14405.844   21324    On TARGET

    On TARGET  2474.604   215951.7

     1) Time: 1474.604 - 14405.84  (seconds)
     2) PHA channel: 4 - 67   (channel no.)

RANGES - Which range(s) do you wish to change (Time=1,Pha=2) /'1'/ > 
TSTART - Minimum value of Time in seconds /1474.604/ > 2474.604
TSTOP - Maximum value of Time in seconds /14405.84/ > 
                                  * Select times when the detector
                                  * was monitoring the target
 Ident chan.    1       2
 Detectors      ABD     EFGH
 Offsets        Source  Source

Mode: coalligned

NIDBINS - How many detector ID channels in the output file /1/ > 
ICHNS1 - Enter detector Ident chans. wanted in output channel 1 > 1:2
 Input IDENT                    1   2
 Mapped to output               1   1

     DATA AXES
 -----------------
  1) Time
  2) Energy

BIN_AXES - Axes required in data array /'2'/ > 
ENBIN - Width of bins in the energy axis (channels) /1/ > 
There will be 64 bins, each of 1 raw PHA channel(s) , starting from 4
OUT - Binned dataset Filename /@SLEW/ > SOURCE
Sorting records 3664 to 21324 of file DISK$SCRATCH:[DEMO.ME]MEDATA_M004.E
Normalising to counts/sec
EXOMESORT complete

ICL> EXOSUBPH                         * Subtract the slew spectrum 
                                      * from the source spectrum.
EXOSUBPH Version 1.0-2
Options are SLEW,2SLEW,NOD,2NOD,SELECT
OPTION - Which subtraction option do you want /'NOD'/ > SLEW
option chosen :  SLEW
 For this option the following data files are needed

   File 1 : Source/pointing data
   File 2 : Background/slew data in the same
            configuration as the source

   ACTION: background data subtracted from source data

INP1 - Input dataset name 1 /@COMFULL/ > ?
Enter the name without the .SDF extension 
INP1 - Input dataset name 1 /@COMFULL/ > SOURCE
INP2 - Input dataset name 2 /@VC2/ > SLEW
OUT - Output dataset name /@VCSUB/ > SRCSUB

ICL> EXORESP                     * Write detector response matrix 
                                 * into the datafile
EXORESP Version 1.0-2
INP - Input dataset name /@SRCSUB/ > 
Calculating the response matrix for detector 1
Calculating the response matrix for detector 2
Calculating the response matrix for detector 4
Calculating the response matrix for detector 5
Calculating the response matrix for detector 6
Calculating the response matrix for detector 7
Calculating the response matrix for detector 8

ICL> DRAW SRCSUB VWS             * Plot the subtracted spectrum
DRAW Version 1.0-7
Dataset: DISK$<SCRATCH>:[RDS]SRCSUB

ICL> SMODEL SMOD3                * Define the spectral model to be
SMODEL Version 1.0-1             * fitted to the data
Input model file :- DISK$SCRATCH:[RDS]SMOD3.SDF;1

A composite model can be synthesised using + - * ( ) and
any of the following primitive models:

   AG     galactic interstellar absorption             multiplicative
   AB     intrinsic absorption                         multiplicative
   BR     thermal bremsstrahlung (cosmic)              additive      
   PL     power law                                    additive      
   BB     black body                                   additive      
   WN     Wien                                         additive      
   BH     hydrogen bremsstrahlung                      additive      
   BP     pseudo-bremsstrahlung                        additive      
   PC     cutoff power law                             additive      
   CS     Chapline-Stevens Comptonised brems.          additive      
   ST     Sunyaev-Titarchuk Comptonisation             additive      
   RZ     Raymond-Smith plasma (metals combined)       additive      
   LG     Gaussian line                                additive      
   LL     Lorentzian line                              additive      

SPEC - Model specification > (BR+BB)*AG

                       COMPONENT  1
                      **************
Name: thermal bremsstrahlung (cosmic)          Keyword: BR     [additive]
  PARAMETER  1 - em10                       (1e60cm**-3/(10kpc)**2)         
                 value,min,max =   1.0000       1.00000E-04    10.000    

                      * etc...


ICL> SFIT SRCSUB                       * Fit the model to the data
SFIT Version 1.0-1
Dataset :-
    File:  DISK$SCRATCH:[RDS]SRCSUB.SDF;1
Instrument response found
Input model file :- DISK$SCRATCH:[RDS]SMOD3.SDF;1
                                       * Model is picked up automatically
Initial chisq & chi-red :  5721.4       96.973    

it,chi-red,lambda,slope :   1        95.783          0      8.169    
it,chi-red,lambda,slope :   2        89.244         -1      12.60    
it,chi-red,lambda,slope :   3        84.625          0      290.5    

                        * etc...

+++ Minimum found +++

Calculating approximate parameter errors


       Parameter                  Value      Approx error (1 sigma)
----------------------------------------------------------------------
BR     em10                       9.4541       0.10477    
       temperature                3.7092       9.53175E-02
BB     norm                      5.75460E-02   1.31136E-03
       temperature                1.6606       6.08231E-03
AG     hydrogen column density    9.4290       0.13274    

Chi-squared =   777.4       with 59 degrees of freedom

** Error estimates entered in model spec. **

OP - Output results? > Y                  * Send results to the printer

\end{verbatim}
\end{quote}

\newpage

\section{Command List }

This is a list of commands available in ASTERIX. These are loosely
grouped by their function. Shortform and macro commands are shown
with suffixes of a cross and an asterix respectively. 
For full details on each command type  `ASTHELP Command'.

\small
\begin{verbatim}
 1) Instrument Interfaces
    ~~~~~~~~~~~~~~~~~~~~~
   a) EXOSAT commands:

 EXOLESORT         Sort Exosat LE raw data files into ASTERIX datasets.
 EXOMESORT         Sort Exosat ME raw data files into ASTERIX datasets.
 EXORESP           Write EXOSAT ME responce matrix into a dataset.
 EXOSUBPH          Subtract background for ME data.
 READEXO           Read Exosat FOT's onto disc.
 SCANEXO           Scan an Exosat tape and produces a summary.
 SHOWEXO           Produce a summary of observations on a disc directory.
 BOXEXOLE*         Draw a box on an Exosat LE image and sort the data within this region.
 BCKBOXEXOLE*      Draw a source and background box on an Exosat LE image and
                   sort the data within these regions.
 TIMEXOLE*         Plot a time series select a time window and sort LE data
                   between these values.
 TIMEXOME*         Same for the ME.

   b) ROSAT WFC commands:

 S2ADDOWNER        Add ownership flag to an S2 catalogue, using WFC sky division.
 S2ADDPFLAG        Add a processing flag to an S2 catalogue.
 S2CATLIST         List contents of an S2 format catalogue.
 S2CATMERGE        Merge two S2 format catalogues to create a third.
 S2FILTMERGE       Merge source search results from each filter into an S2 catalogue.
 S2GIDECODE        Unpack cross-reference data from an S2 catalogue into
                   SCAR global index.
 S2GIENCODE        Pack a SCAR global index into an S2 catalogue.
 WFCBACK           Produce background subtracted image given raw image.
 WFCSPEC           Produce a WFC spectral file.

   c) ROSAT XRT commands:

 XRTDISK           Read XRT FITS disk files
 PREPXRT           Reformat the output from XRTDISK 
 XSORT             Sort raw XRT data.
 XRTORB            Reformat an XRT orbit file
 XRTSUB            Subtract XRT files.
 XRTCORR           Correct XRT files.
 XRTRESP           Write XRT response into file.


 2) Event Processing Commands - these all work on EVENT datasets.
    ~~~~~~~~~~~~~~~~~~~~~~~~~
 EVBIN             Create a binned dataset from an event dataset.
 EVCSUBSET         Extract a circular or annular region from an event dataset.
                   Output is an event dataset.
 (EVLIST           Display the DATA_ARRAY component of all lists in an event dataset.)
 EVMERGE           Merge two or more event datasets.
 EVSUBSET          Linearly extract a subset from an event dataset.
 EVCSUBSET         Produce a circular subset.
 EVSIM             Generate simulated event dataset.
 EVPOLAR           Bin an event dataset into a polar binned dataset.


 3) Binned Dataset Commands
    ~~~~~~~~~~~~~~~~~~~~~~~
 AXFLIP            Reverse directions of any or all of dataset's axes.
 AXORDER           Re-order axes of a dataset.
 AXSHOW            Display axes.
 BINMERGE          Merge up to ten datasets.
 BINSUBSET         Subset a binned dataset.
 ENMAP             Produce an energy map for plotting.
 INTERP            Reconstitute bad pixels by spline interpolation.
 MEANDAT           Find the mean of several datafiles.
 POLYFIT           Fit 1-d polynomial(s) to a dataset.
 PROJECT           Project data along one axis.
 RATIO             Give the ratio of two bands on any axis in an n-d array.
 REBIN             Rebin a binned dataset.
 SETRANGES         Set ranges to be used by a subsequent program.
 SIGNIF            Change an input dataset to its weighted significance.
 SCATTERGRAM       Produce scatter plot of one array versus another.
 SMOOTH            Smooth an n-d datafile with a user-selectable mask.
 SYSERR            Add a constant percentage to the variance of each point.
 VALIDATE          Basic validation of binned dataset.


 4) Conversion Commands
    ~~~~~~~~~~~~~~~~~~~
 ASTCONV           Convert between old & new ASTERIX binned datasets.
 AST2XSP           Convert Asterix spectral files into XSPEC format.
 ASTQDP            Allow an Asterix file to be used within QDP.
 (EVBIN            Create a binned dataset from an event dataset.)
 EXPORT            Output one or more datasets to ascii file.
 IMPORT            Read an ascii text file into an ASTERIX binned dataset.
 TEXT2HDS          Convert columns in a text file into HDS arrays.


 5) Display Commands
    ~~~~~~~~~~~~~~~~
 BINLIST           Display a 1D binned dataset.
 EVLIST            Display the DATA_ARRAY component of all lists in an event dataset.
 HEADER            Display header & processing information in a dataset.
 HISTORY           Display history of a dataset.


 6) Math Commands
    ~~~~~~~~~~~~~
 ARITHMETIC        Perform basic arithmetic (+,-,/,*) on two data objects.
 ADD+              Invoke ARITHMETIC in + mode.
 SUBTRACT+         Invoke ARITHMETIC in - mode.
 MULTIPLY+         Invoke ARITHMETIC in * mode.
 DIVIDE+           Invoke ARITHMETIC in / mode.
 OPERATE           Perform operations e.g. LOG10  on a dataset.


 7) Time Series Analysis Commands
    ~~~~~~~~~~~~~~~~~~~~~~~~~~~~~
 ACF               Auto-corralation program.
 BARYCORR          Barycentric correction.
 CROSSCOR          Cross correlates two 1D series.
 CROSSPEC          Compute the cross spectrum of two 1D datasets.
 DIFDAT            Difference adjacent data points in an array.
 DYNAMICAL         Find power spectrum of successive segments of a time series.
 FOLDAOV           Period search ANOVA folding routine
 FOLDBIN           Fold a time-series into phase bins at a given period.
 FOLDLOTS          Epoch folding, period search algorithm.
 LOMBSCAR          Lomb-Scargle power spectral routine
 PHASE             Convert a time-series into a phase series.
 POWER             Find the power spectrum of a full unweighted 1D dataset.
 SINFIT            Find a periodogram of irregularly spaced 1D data.
 STREAMLINE        Strips bad quality data out of a file.
 TIMSIM            Simulate a time series.
 VARTEST           Test a time series for variability.


 8) Image Processing Commands 
    ~~~~~~~~~~~~~~~~~~~~~~~~~
    a) interactive

 ISTART            Start up image processing system.
 INEW              Load new image.
 ICLOSE            Close down image processing system.
 IDISPLAY          Display current image.
 IREDISPLAY+       Redisplay current image.
 IBOX              Define rectangular section of image.
 IWHOLE            Select whole image.
 IZOOM*            Zoom in on section of image.
 IUNZOOM*          Redisplay whole image.
 ISCALE            Rescale image.
 ICONTOUR          Contour current image.
 IGRID             Put grid over image in specified coords.
 ICOLOUR           Change colour table.
 IBLUR             Smooth an image.
 IBLURG+           Smooth with a gaussian.
 IBLURB+           Smooth with a box filter.
 IPLOT             Display current 1D plot.
 IREPLOT+          Redisplay current 1D plot.
 ILIMITS           Change axis limits on 1D plot.
 ISTYLE            Set plotting style of 1D plot.
 ITITLE            Change title of current image or plot.
 IZONES            Change zones on display surface.
 ICLEAR            Clear current plotting zone.
 IHARD             Does hard copy of current plot.
 ISTATS            Give basic statistics on pre-selected region of image.
 IBOXSTATS*        Select region and give statistics.
 ICIRCSTATS*       Get flux from a circular region.
 ICENTROID         Find the centroid in a given region.
 ICURRENT          Display the current position, image ...
 IPOSIT            Set the current position.
 ISURFACE          Displays an image as a 3-d surface.
 ISEP              Calculate the separation between two positions.
 IMARK             Mark sources on image.
 INOISE            Add gaussian noise to image.
 IREMOVE           Remove sources from an image.
 IZAP              Remove individual pixels from image.
 IPATCH            Patch bad quality pixels in image.
 IPEEK             Take a peek at selected bit of image.
 IPEEKV+           Look at variances.
 IPEEKQ+           Look at quality.
 IPGDEF            Change basic graphics properties.
 IHIST             Histogram pixels inside current box.
 IPEAKS            Find peaks in an image.
 IRADIAL           Produce radial plot.
 IPSF              Put PSF profile on 1D plot.
 IAZIMUTH          Produce azimuthal distribution.
 ISLICE            Take a 1-d slice from an image.
 IPREVIOUS         Go back to previous image.
 ISAVE             Save current image to file.
 ISAVE1D           Save current 1D data to file.

    b) non-interactive

 IMOSAIC           Merge several non-congruent images.
 IPOLAR            Produce a polar surface brightness profile of an image.
 IDTODMS*          Convert decimal degrees to dd:mm:ss.
 IDMSTOD*          Convert dd:mm:ss to decimal degrees.
 IDTOHMS*          Convert decimal degrees to hh:mm:ss.
 IHMSTOD*          Convert hh:mm:ss to decimal degrees.


 9) Parameter Commands
    ~~~~~~~~~~~~~~~~~~
 GLOBAL            Display the current values of all global parameters.


 10) Spectral analysis commands
     ~~~~~~~~~~~~~~~~~~~~~~~~~~
 FREEZE            Freeze parameter(s) in spectral model.
 IGNORE+           Allow spectral channels to be excluded from fitting.
 RESTORE+          Reinstate spectral channels for fitting.
 SDATA             Define datasets to be fitted.
 SERROR            Evaluate confidence region for spectral model parameters.
 SFIT              Fit spectral model to one or more datasets.
 SFLUX             Evaluate flux from defined model over energy band.
 SMODEL            Allow user definition of muti-component spectral model.
 SPLOT             Plot data, fits and residuals.
 THAW              Free spectral model parameter (after FREEZEing).


 11) Statistical analysis commands
     ~~~~~~~~~~~~~~~~~~~~~~~~~~~~~
 BINSUM            Integrate a dataset - use for cumulative distributions.
 COMPARE           Compare two datafiles or a file and a model.
 FREQUENCY         Produce a histogram of values in a data array.
 KSTAT             Calculate the Kendall K statistic for a 1-d dataset.
 STATISTIX         Find mean, standard deviation etc.. of a data array.


 12) Quality processing commands
     ~~~~~~~~~~~~~~~~~~~~~~~~~~~
 IGNORE+           Invoke QUALITY in IGNORE mode: sets temporary bad quality bit.
                   (Only availible in ICL.)
 QUALITY           General quality manipulation application.
 RESTORE+          Invoke QUALITY in RESTORE mode: clears temporary bad quality bit.
                   (Only availible in ICL.)
 SETQUAL+          Invoke QUALITY in SET mode: sets quality to specified value.
                   (Only availible in ICL.)
 MAGIC             Set magic values.
 CQUALITY          Perform quality manipulation on a circular region.
 CRESTORE          Set quality good in a circular region.
 CIGNORE           Set quality bad in a circular region.


 13) HDS editor commands
     ~~~~~~~~~~~~~~~~~~~
 HCOPY             Recursively copy an HDS object.
 HCREATE           Create an HDS object in a file.
 HDELETE           Delete an HDS object.
 HDIR              Display the components of an HDS file or structure.
 HDISPLAY          Display the contents of an HDS primitive.
 HFILL             Fill a primitive object with one value.
 HMODIFY           Change the value(s) in an existing object.
 HREAD             Read from an ASCII (or binary) file to an HDS object.
 HRENAME           Rename an HDS object.
 HRESET            Set values in HDS primitive to `undefined'.
 HRESHAPE          Alter size of dimensions in an HDS primitive array.
 HRETYPE           Change type of a structured object.
 HSE               Interactive HDS screen editor - see HSEHELP.
 HTAB              Simultaneously display several HDS primitive vectors.
 HWRITE            Write an HDS primitive to an ASCII (or binary) file.


 14) Source Searching Commands
     ~~~~~~~~~~~~~~~~~~~~~~~~~
 CREPSF            Create a PSF file.
 PSS               Search a binned dataset for sources.
 PSSPAR+           Invoke PSS in parameterise mode.
 SS_ANOT+          Anotate an image given source search results.
 SSCARIN           Export source search results to binary SCAR catalogue.
 SSDUMP            Dump source search results to an ascii file.
 SSMERGE           Merge two or more source results together.
 SSZAP             Set quality bad around sources found by PSS.
 BSUB              Background subtraction program.
 XPSSCORR          Exposure corrects XRT source fluxes produced by PSS.

 15) GRAFIX display commands
     ~~~~~~~~~~~~~~~~~~~~~~~
 DEVICES           Show the available graphics display devices.
 DOPEN+            Open device (variant of DEVICES).
 DCLOSE+           Close device (variant of DEVICES).

 DRAW              Display specified (or current) dataset.
 QDRAW+            Draw quick and simple plot (variant of DRAW).
 RDRAW+            Plot raw data only (variant of DRAW).

 MULTI             Create multi graph dataset from binned datasets.
 AMULTI+           Add another binned dataset to a multi-graph dataset.
 DMULTI+           Remove a dataset from a multi-graph dataset (variants of MULTI).
 DESIGN            Display specified plots within a multi graph dataset.
 LAYOUT            Specify number of plots in x & y in multi graph dataset.

 POSIT             Specify position of plot in NDC or cm.
 XLOG+             Set x-axis logarithmic (variant of AXES).
 YLOG+             Set y-axis logarithmic (variant of AXES).
 XYLOG+            Set both axes logarithmic (variant of AXES).
 AXES              Set plotting attributes for axes.
 XAXIS+            Set attributes for x-axis (variant of AXES).
 YAXIS+            Set attributes for y-axis (variant of AXES).

 LABEL             Set labels on plot.
 XLABEL+           Set x-label (variant of LABELS).
 YLABEL+           Set y-label (variant of LABELS).
 LEGEND            Add, delete or modify legend lines.
 ALEGEND+          Add a legend line (variant of LEGEND).
 DLEGEND+          Delete a legend line (variant of LEGEND).
 ANOTATE           Place text in graph (binned dataset) at specified posn.

 POLYLINE          Draw 1D image as polyline.
 MARKER            Draw 1D dataset with specified PGPLOT marker.
 STEPLINE          Draw 1D dataset as histogram.
 ERRORS            Draw error boxes of specified shape on 1D dataset.

 PIXEL             Draw 2D image as pixels.
 CONTOUR           Draw 2D dataset as contours.
 THREED            Draw 2D dataset as quasi 3D plot.

 COLTAB            Manipulate colour table for 2D plot.
 COLBAR+           Cause colour bar to be put on 2D plot.
 GREYSCALE+        Set colour table to greyscale.
 SKYGRID           Put coordinate grid on 2D plot.

 CURSOR            Put interactive cursor on screen.
 SHAPES            Put a shape onto a displayed image.
 ZOOM*             Interactively set plot limits.

 PGDEF             Set global PGPLOT default attributes.
\end{verbatim}
\normalsize
\end{document}
