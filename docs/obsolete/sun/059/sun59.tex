\documentstyle{article} 
\pagestyle{myheadings}

%------------------------------------------------------------------------------
\newcommand{\stardoccategory}  {Starlink User Note}
\newcommand{\stardocinitials}  {SUN}
\newcommand{\stardocnumber}    {59.3}
\newcommand{\stardocauthors}   {M D Lawden}
\newcommand{\stardocdate}      {7 June 1989}
\newcommand{\stardoctitle}     {Starlink Software Reorganisation}
%------------------------------------------------------------------------------

\newcommand{\stardocname}{\stardocinitials /\stardocnumber}
\markright{\stardocname}
\setlength{\textwidth}{160mm}
\setlength{\textheight}{240mm}
\setlength{\topmargin}{-5mm}
\setlength{\oddsidemargin}{0mm}
\setlength{\evensidemargin}{0mm}
\setlength{\parindent}{0mm}
\setlength{\parskip}{\medskipamount}
\setlength{\unitlength}{1mm}

\begin{document}
\thispagestyle{empty}
SCIENCE \& ENGINEERING RESEARCH COUNCIL \hfill \stardocname\\
RUTHERFORD APPLETON LABORATORY\\
{\large\bf Starlink Project\\}
{\large\bf \stardoccategory\ \stardocnumber}
\begin{flushright}
\stardocauthors\\
\stardocdate
\end{flushright}
\vspace{-4mm}
\rule{\textwidth}{0.5mm}
\vspace{5mm}
\begin{center}
{\Large\bf \stardoctitle}
\end{center}
\vspace{5mm}

\section{Introduction}

The Starlink Software Collection was last reorganised on 9th December 1981 and
the structure then implemented has served us well through the 80's.
However, over 380 software releases and about 66 new software items have been
issued since then and the size and complexity of the software has grown
enormously.
Some of the original design concepts have become outdated or unworkable and the
management of the Collection now tends to be difficult and error prone --- in
particular, it is not immediately obvious which files belong to which
software items and some items are not stored in the appropriate directory.
Thus, the time has come to reorganise the way in which the Collection is stored,
particularly the \mbox{[STARLINK]} directory.
The new design will make the Collection easier to manage and distribute and
should see us through the 90s.

The changes should be invisible to the software user, so you need not take any
action.
However, you may find that, unless you take some action, some Starlink software
will run less efficiently after the reorganisation than it did before, so you
should read the section {\em Recommended User Actions} to determine whether or
not you will be affected and if so, what you should do about it.

The time of installation at your site of the reorganised software will depend
on your Site Manager.
RAL expects to implement the changes in early May.
If this is successful, other sites should implement it a few days or weeks
later --- ask your Site Manager.

\section{Organisational Considerations}

Two important aspects of the organisation of the Starlink Software Collection
are:
\begin{itemize}
\item The Physical organisation.
\item The Logical organisation.
\end{itemize}
By `Physical organisation' I mean how the software files are organised on disk
into directories.
As you know, directories are organised in tree structures and at the top of
the tree there is a directory with a simple name like [ABC].
Let us call this a top-level directory.
Some of the files in [ABC] may be directory files such as XYZ.DIR.
This means that there is a directory called [ABC.XYZ] which is a sub-directory
of [ABC].
Now, when I say `top-level directory', I mean just those files that belong to
[ABC] (say), and not those files that belong to any sub-directory such as
[ABC.XYZ].
However, when I say `tree', I mean all the files in the tree-like directory
structure with [ABC] at its head.
Thus
\begin{verbatim}
    $ DIR [ABC]
\end{verbatim}
will show me all the files in the {\em top-level directory} [ABC], while
\begin{verbatim}
    $ DIR [ABC...]
\end{verbatim}
will show me all the files in the {\em tree} [ABC].

The physical organisation of the Collection comprises a set of trees.
I will refer to these trees by the name of the top-level directory, such as
[STARLINK], [ADAM], and so on.
When Starlink started we thought it would be a good idea to store the Collection
in a single tree called [STARLINK].
The main attraction was that this would enable us to transport the Collection to
another computer using a single command:
\begin{verbatim}
    $ BACKUP [STARLINK...] TAPE:STARLINK.BCK
\end{verbatim}
to write to tape a single `save-set' called STARLINK.BCK which would preserve
the directory structure and enable the [STARLINK] tree to be recreated on
another computer using the command:
\begin{verbatim}
    $ BACKUP TAPE:STARLINK.BCK [*...]
\end{verbatim}
This worked fine until the Collection grew so large that some sites had
difficulty in storing the tree on a disk.
The solution to this problem was to extract the files belonging to some of the
larger items from the [STARLINK] tree and store them in a separate tree which
could be stored on a different disk or even omitted all together.
Starlink also began to install some astronomical catalogues which took up a
lot of disk space and which were also stored in their own trees.
Thus, the most convenient way of storing the Collection now is as a set of
moderately sized trees.

At the logical level, Starlink software is considered to be made up of items.
An {\em item} is an identifiable piece of software which is convenient to
manage and refer to as a unit.
It is given an acronym (such as `PHOTOM') and a title (such as `An aperture
photometry routine').
The Starlink software index \mbox{(ADMINDIR:SSI.LIS)} lists the acronyms and
titles of all the items in the Collection, and this is the final court of appeal
when deciding which software is or is not in the Collection.
An item is stored on the computer as one or more files in one or more
directories.
When I refer to an `item' in this note, I mean all the files which belong to
it.

One of the current problems is that some subdirectories of the [STARLINK]
tree contain files belonging to many different items, so sometimes it is
necessary to identify files individually in the Starlink software index when
describing items.
It would be much easier to identify the files belonging to an item if they
were all stored in a single directory or tree and if this did not contain any
files belonging to other items; then we could refer to the item by the name of
this directory or tree.
Also, the management of the software would be simplified as directories and
trees could easily be moved, copied or deleted as an entity, without interfering
with other items.
This is one of the main things that has been done in the current reorganisation
of the Starlink software.

The {\em size} of an item is the number of blocks needed to store the directory
or tree which holds its files.
The distribution of sizes in the Collection is such that there are a few big
items and a lot of small items.
The most convenient way of storing these is to store the big items in trees by
themselves and to store the small items in subdirectories of a composite tree.
That way, all the small items can be transported in a single save-set, yet no
single tree is very large.
The composite tree can also be used to store documentation and administration
data.
This is the [STARLINK] tree.

One of the advantages of having the Collection stored in a number of separate
trees is that it makes it easier for a site to install a subset.
All it need do is to select the trees that contain the software it wants and
install those.
This leads to another consideration, which is that some items depend on other
items in order to work properly.
The commonest case is an item which needs the GKS shareable graphics library
to be installed.
The web of interrelationships between Starlink software items has never been
fully documented.
However, we know that the items which are needed by other items tend to be
fairly small in size --- typically subroutine libraries or shareable images ---
and would, because of their size, be found in the [STARLINK] tree.
Any such item which wasn't there because of its size could be stored there
because it is required by other items.
Thus the [STARLINK] tree satisfies three separate requirements:
\begin{itemize}
\item Somewhere to store lots of small items which can then be transported as
a single entity.
\item Somewhere to store items which may be needed by other items and which must
therefore be installed.
\item Somewhere to store Starlink documentation and administrative data.
\end{itemize}
We insist that the [STARLINK] tree be installed at every Starlink site
(or site running any Starlink software), as then we can be fairly confident
that any other Starlink item will work if we install its tree.
Also, all the basic information on Starlink is available in DOCSDIR and
ADMINDIR.

Another consideration which affects the software organisation is that some
software items are {\em proprietary} and are licensed to be used only at
specific sites.
As we have said that the [STARLINK] tree should always be installed at
Starlink sites and other sites running Starlink software, we must not store
proprietary items in this tree, even if they are small; we must store them in
their own separate trees..
There is one exception to this, and that is GKS which is required by many other
items.
Fortunately, the Starlink project is permitted to distribute this without
charge to sites which run Starlink software for astronomical, non-commercial
research.
GKS has, therefore, been stored in the [STARLINK] tree.

Another useful concept to use when managing the Collection is that of a
{\em standard set} of items.
This is a recommended minimal set of items that should be installed at any
site claiming to run `Starlink software'.
It is also useful as a standard response to the oft-repeated request: {\em
Please send me the Starlink software.}
In general, it would be wasteful to respond to every such request by sending
the whole Collection as this is very large.
Normally, what is required is a useful subset to which specific items can be
added as required.

Some items are used mainly by Site Managers to manage hardware or users.
Other items may represent a security risk.
It is convenient to set up another composite tree like [STARLINK] to hold these
items.
The software held in this tree is called {\em secure software}.

\section{The New Organisation}

The considerations described in the previous section have formed the basis
of the reorganisation of the Starlink Software Collection.
This will now be stored as a set of trees belonging to the following categories:
\begin{description}
\item [Standard set] :

 The [STARLINK] tree, together with other trees as specified in the Starlink
 software index --- currently ADAM, ADAMAPP, FIGARO, and KAPPA.
 Items stored in one of these trees are called {\em standard items}.
 The fact that an item is in the standard set does {\em not} imply that it is
 important or supported --- it may merely be small.

\item [Secure set] :

 This is a single tree containing items issued mainly for managing Starlink
 software or users, and items which could represent a security risk.
 Items stored in this set are called {\em secure items}.

\item [Option set] :

 Trees containing a single item and which do not belong to the standard set or
 the secure set.
 The tree name should be the same as the acronym used in the Starlink software
 index to describe the item it contains.
 Items stored in trees in the option set are called {\em option items}.
 The fact that an item is in the option set does {\em not} imply that it is
 unimportant or not supported.
 
\end{description}
The word `set' can refer to trees or items.
Thus, the `standard set of items' comprises the items contained in the `standard
set of trees'.
Two further categories are useful:

\begin{description}
\item [Local software] :

 The [STARLOCAL] tree whose structure should be based on \mbox{[STARLINK]} and
 whose contents is determined locally.
 There should be a local software index (LADMINDIR:SSI.LIS) which defines the
 items stored in this tree; these items are called {\em local items}.

\item [Data] :

 A set of trees containing optional data files used by various items --- mainly
 astronomical catalogues at the moment.
 The trees are specified in file RLVAD::LADMINDIR:SIZE.LIS.

\end{description}

The structure of the [STARLINK] tree has been simplified and rationalised.
Libraries and utilities formerly stored in LIBDIR, STARDIR, SYSTEMDIR, and
UTILITYDIR have been transferred into their own sub-trees.
Also, the directory STARDIR has been eliminated and the packages in
[STARLINK.PACK] have been transferred to their own trees.
Software is now stored in only three subtrees:
\begin{description}
\item [LIBDIR] --- contains subroutine libraries.
\item [SYSTEMDIR] --- contains general purpose utilities.
\item [UTILITYDIR] --- contains astronomical utilities.
\end{description}

\section{Summary of changes}

The following transfers and deletions have taken place:

\begin{description}

\item [ASSOCIATED items $\Longrightarrow$ STANDARD items] :

The following items have been moved into the [STARLINK] tree:

\begin{description}
\item [DSCL]
\item [GKS]
\item [PRIMDAT]
\end{description}
DSCL has been extracted from the [ASPIC] tree as it is a way of using ASPIC
rather than an ASPIC program.
GKS has been put into [STARLINK] because it is {\em always} required when
Starlink software is distributed and is used by many other items.
PRIMDAT is a fairly small subroutine library and is most naturally stored in
LIBDIR with the other libraries.

\item [STANDARD items $\Longrightarrow$ OPTION items] :

The following items have been taken out of the [STARLINK] tree and stored in
their own trees.
They are mostly refugees from the deleted [STARLINK.PACK] tree.
\begin{description}
\item [ADAMAPP]
\item [DIPSO]
\item [IRAS]
\item [IUEDR]
\item [JPL]
\item [KAPPA]
\item [MONGO]
\item [REXEC]
\end{description}

\item [STANDARD items $\Longrightarrow$ SECURE items] :

The following items have been transferred from the [STARLINK] tree to the secure
tree.
\begin{description}
\item [NETWORK]
\item [NEWSMAINT]
\item [UAFPROBE]
\end{description}

\item [DELETIONS] :

The following items have been withdrawn.
If you still need them, ask your Site Manager to install them locally:
\begin{description}
\item [ADC]
\item [FINGS]
\item [HIGR]
\item [OPWRIT]
\item [SPECHAR]
\end{description}
The withdrawn ADC is an obsolete stand-alone version of software which still
exists as part of SCAR (where it is called FACTS).
FINGS, HIGR, and SPECHAR have come to the end of their working lives as they
have been replaced by better software.
OPWRIT is no longer useful as the Optronics device it was designed to feed is
no longer available.

\item [RECLASSIFICATIONS] :

The following item is still installed, but is no longer considered to be part
of the Starlink Software Collection.
It is commercial software that is only installed on STADAT:
\begin{description}
\item [SPAG]
\end{description}

\end{description}

\section{Recommended User Actions}

All but two of the changes should be invisible to users.
This is because the relocations have been accounted for by altering the
definitions of logical names and global symbols in the Starlink command
procedures executed at startup and login time, and by making appropriate
changes to command procedures belonging to the items involved.
However, some of the changes may slow you down because some of the logical
names used have become search lists rather than single values.
In these cases you should change the logical name to a more specific one.
The table below shows which logical name should be replaced by which other
logical name when using a specified item.
{\em In the case of JPL and NEWSMAINT these changes are essential as the old
logical names will not work at all.}

Old documentation on these items will no longer be correct --- I suggest you
amend your existing copies of the indicated documents.
New versions of all these documents have been released, except for SUN/33 which
is very large and very obsolete.
\begin{table}[h]
\begin{center}
\begin{tabular}{|l|l|l|l|}
\hline
Software item & Replace & By & Document \\
\hline
{\bf ARGSLIB}   & LIBDIR            & ARGSLIB\_DIR   & --- \\
{\bf ARGSMAC}   & LIBDIR            & ARGSMAC\_DIR   & SSN/12 \\
{\bf FORMCON}   & STARDIR           & FORMCON\_DIR   & SUN/3 \\
{\bf GENERIC}   & UTILITYDIR/LIBDIR & GENERIC\_DIR   & SUN/7 \\
{\bf HINDLEGS}  & UTILITYDIR        & SYSTEMDIR      & SUN/22 \\
{\bf IDDRIVER}  & SYSTEMDIR         & IDDRIVER\_DIR  & SSN/26 \\
{\bf INTERIM}   & STARDIR           & INTERIM\_DIR   & SUN/4 \\
{\bf JPL}       & LIBDIR            & JPLDIR         & SUN/87 \\
{\bf LIBX}      & UTILITYDIR        & LIBX\_DIR      & SUN/8 \\
{\bf NEWSMAINT} & UTILITYDIR        & NEWSMAINT\_DIR & SSN/50 \\
{\bf SGS6}      & LIBDIR            & SGS6\_DIR      & SUN/33 \\
{\bf SLALIB}    & LIBDIR            & SLALIB\_DIR    & SUN/67 \\
{\bf TAPECOPY}  & UTILITYDIR        & TAPECOPY\_DIR  & SUN/47 \\
{\bf TAPEIO}    & LIBDIR            & TAPEIO\_DIR    & SUN/21 \\
{\bf TPU}       & UTILITYDIR        & TPU\_DIR       & SUN/68 \\
\hline
\end{tabular}
\end{center}
\end{table}

As an example, suppose you want to link your program with the SLALIB routines.
SUN/67.7 tells you to do this using the command:
\begin{verbatim}
    $ LINK progname,LIBDIR:SLALIB/LIB
\end{verbatim}
After the reorganisation, you should change this command to:
\begin{verbatim}
    $ LINK progname,SLALIB_DIR:SLALIB/LIB
\end{verbatim}
because the SLALIB library has been moved from LIBDIR to a subdirectory with
the logical name SLALIB\_DIR.
The use of the other items will involve similar changes in logical names.
You may need to revise some of your own command procedures which refer to these
items.

If you have any problems that you suspect have been caused by the
reorganisation, please send a mail message to RLVAD::STAR describing the
problem.

\section{Old and new organisation}

The two tables at the end of this paper show the old and new organisation of
the Starlink software.
They show which trees hold the files for a specific item.
Table 2 shows the category to which each item belongs.
\end{document}
