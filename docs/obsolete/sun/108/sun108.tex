\documentstyle{article}
\pagestyle{myheadings}

%------------------------------------------------------------------------------
\newcommand{\stardoccategory}  {Starlink User Note}
\newcommand{\stardocinitials}  {SUN}
\newcommand{\stardocnumber}    {108.3}
\newcommand{\stardocauthors}   {C A Clayton}
\newcommand{\stardocdate}      {19 June 1991}
\newcommand{\stardoctitle}     {Using the Starlink SI59 Exabyte drives}
%------------------------------------------------------------------------------

\newcommand{\stardocname}{\stardocinitials /\stardocnumber}
\markright{\stardocname}
\setlength{\textwidth}{160mm}
\setlength{\textheight}{240mm}
\setlength{\topmargin}{-5mm}
\setlength{\oddsidemargin}{0mm}
\setlength{\evensidemargin}{0mm}
\setlength{\parindent}{0mm}
\setlength{\parskip}{\medskipamount}
\setlength{\unitlength}{1mm}

\begin{document}
\thispagestyle{empty}
SCIENCE \& ENGINEERING RESEARCH COUNCIL \hfill \stardocname\\
RUTHERFORD APPLETON LABORATORY\\
{\large\bf Starlink Project\\}
{\large\bf \stardoccategory\ \stardocnumber}
\begin{flushright}
\stardocauthors\\
\stardocdate
\end{flushright}
\vspace{-4mm}
\rule{\textwidth}{0.5mm}
\vspace{5mm}
\begin{center}
{\Large\bf \stardoctitle}
\end{center}
\vspace{5mm}

\section{Introduction}

The Exabyte CTS is a high performance, high capacity 8mm Cartridge Tape
Subsystem. It uses helical scan technology, in the same way as a video
recorder,
to attain a very high areal recording density and data storage capacity.
It uses industry standard 8mm tape cartridges, which are capable of storing
up to 2.3Gbyte of formatted user data.

This document is designed to introduce both users and system managers
to the idiosyncrasies of the SI59 Exabyte cartridge drives. There are
many contradictory rumours concerning the functionality and operation
of these devices. This document aims to inform the reader which of these
rumours are factual and which are not.

\section{Using the Exabyte}

\subsection{Loading and unloading}

The Exabyte drive has two LEDs and an unload button on the front panel.

After powering up the drive, the amber and green LEDs remain lit for
up to 2 minutes whilst the drive performs self tests. When the diagnostics are
complete, both LEDs are turned off. During normal use, the green LED indicates
that a cartridge is loaded. The amber LED flashes intermittently when
the drive is reading/writing.

To load a cartridge, press the unload switch to open the door. Insert the
cartridge, label side up, cartridge lid facing towards the drive.
Gently close the door. The cartridge will load automatically.

Simply press the unload switch to rewind, unload and eject the cartridge.

It is recommended that users issue the command

\begin{quote}\tt

\$ DISMOUNT/NOUNLOAD device-name

\end{quote}

to unload a tape if they do not intend to remove the cartridge immediately,
since otherwise the drive door will open automatically and allow dust
particles to enter the drive.

To write-protect a cartridge, use a ballpoint pen to change the
position of the write protect tab such that the red tab is visible.
If the red tab is not visible, then the cartridge is write enabled.

If you experience difficulty initializing a blank Exabyte tape, try

\begin{quote}\tt

\$ INIT/OVERRIDE=(EXPIRATION,OWNER,ACCESSIBILITY) device-name label

\end{quote}

to stop the tape deck getting confused when trying to read the
(non-existent) previous label.

Initializing tapes can take up to two minutes and mounting them can take
over half a minute so be patient. Do not panic and hit CTRL/Y.

\subsection{Capacity}

Although the Exabyte is touted as holding up to 2.3Gbyte of data (about the
same as 15 full 6250bpi tapes), it is quite easy to use it very  wastefully.
As a test,
one user tried copying many 1 block files to the Exabyte  using COPY. He
could only get 316 files onto a cartridge. This is only  161Kb on a full
cartridge, a factor of 15,000 less than  advertised. The reason for this is
that the tape marks that separate  files are several inches long and take up as
much space as about  2Mbyte (4,000 blocks) of data. Also, on a labelled tape,
each file is  preceded by a header file (plus tape mark) and a trailer file
(plus a  tape mark). The upshot of this is that unless your files are more than
30,000 blocks in size you are wasting more than 50\% of the cartridge.
The simple test above was wasting 99.993\% of the capacity of the  cartridge.

\begin{itemize}

\item COPY is supported only on Exabytes with CS1000 (rev. level H)
or more recent
controllers. If you {\it must} use COPY, ask your system manager to
confirm that it is supported on your Exabyte. If it is not supported,
you may find that the drive lets you use COPY to write to the
tape but that you are then unable to read the data back! Always
use BACKUP if possible.

\item Do not use COPY to write Exabyte cartridges unless your files are
    {\bf much} bigger than 30,000 blocks.
    Use BACKUP instead. This is much more efficient as no matter how many
    files you are saving, BACKUP only writes a single saveset (plus
    header and trailer) on the cartridge. Writing multiple savesets is
    fine, but again they should be bigger than 30,000 blocks to use
    the cartridge really effectively.

\item If you want to write many small files to tape, or your BACKUP
    savesets are typically less than a few thousand blocks, then you
    will get more on a standard half-inch magnetic tape then on an
    Exabyte cartridge.

\item Sometimes, the smaller physical size of the Exabyte cartridge
    compared with even a 600ft tape will be of overriding importance.
    However, no matter what you do, you cannot get more than about 320
    files on a labelled cartridge or 950 files on an unlabelled
    cartridge.

\end{itemize}

\subsection {Tapes}

The drive determines the length of a tape by  spinning backwards and forwards
for about a minute (hence long INIT times) and watching the spool
rotation speed. From that, it
deduces which size of tape loaded and hence the capacity.
Hence, when sensing the tape length, it assumes
the American size corresponding to the rotation speed. For P5-90 type tapes,
the tape is
actually longer. For shorter P5 tapes, the American tapes are
longer than the equivalent British ones (due to the different video/TV
standards). This means that using the shorter P5 tapes can end in disaster  if
you fill the tape, since there is no way to write end-of-volume labels etc.
when you have genuinely run out of tape. The logical end-of-tape is determined by
counting blocks down the tape. With a P5-90 tape the drive will ignore the last section
of the tape, whereas with shorter P5 tapes it will suddenly
discover the tape is shorter than expected.
To avoid this potential problem, use only P5-90 tapes even if you are only
going to be using a small fraction of the tape.

Your site manager will supply you with Exabyte tapes.
At present, the only recommended tapes are Sony P5-90MP and the Exabyte
data tapes specifically designed for use with the drive. The latter
are more expensive but are said to be more reliable for very long term
archiving.

Another important question plagued by hearsay is the lifetime and re-usability
of the tapes. Some sources preach  that tapes should be discarded
after as little as 6 passes (reads or writes).
Fortunately, these rumours appear to be unfounded. Our tests have
shown
that tapes can cope with a very large number of read/write
passes  (at least 100) without any reading/writing problems. However,
if you do not look after your tape (e.g. you leave it knocking around at the
bottom of your briefcase without its case) then the tape's lifetime will be
dramatically reduced.
Exabyte's lab testing claim that tapes would not
even begin to `degrade' until after around 800-900 passes, but due to their
super ECC methods they were still usable well beyond 2000-3000 passes.
Most Starlink users are unlikely to get anywhere near this many passes on
their backup tapes!

\subsection{Using the Exabyte for long-term archive}

The Starlink Exabytes are not intended for use as long-term archiving devices.
They were neither intended for this role nor are they particularly suitable
because of the poorer stability of the media compared with half-- and
quarter--inch tapes and the Starlink Project does not recommend that they be
used for these applications. If you do insist on using the Starlink Exabytes
for long-term storage, then you should be aware of the following facts.

Accelerated life testing by the NML (National Media Lab, USA) showed problems
with the dimensional stability of tape substrates and problems with the
permanence of the magnetic characteristics of the ``pigment'' used.  The
problem with the substrate is that most modern tapes (DAT, 8mm, Exabyte) are
highly tensilized (stretched) in the longitudinal direction.  This adds
strength and stiffness to the tape.  Unfortunately, in an environment with high
temperature and/or humidity, the substrate tries to revert to its original
shape.  This causes distortion of the tape geometry that is especially damaging
to helical recording schemes (like Exabyte and DAT). Note that quarter--inch
cartridge (QIC) and half--inch tapes do not use tensilized tape and moreover
do not depend on helical recording.

The second problem is that most recent tapes use a metal particle (MP)
formulation. The metal particles are encapsulated in a ceramic material to
prevent them from rusting.  Nonetheless, the magnetic properties of MP tapes
seem to be very sensitive to their environment.  Either high temperatures and
high humidity, or contaminants in the air can cause a rapid decrease in the
magnetic properties of the media.  Again, both DAT and Exabytes use MP tape.
Apparently QIC tapes use an older magnetic layer which is much more stable than
either of these.

Given these problems, the people from NML think that DAT or Exabyte tapes
should last for two to three years in a typical office environment, and longer
in a computer room.   In a report by NML on ``Proper Archival Data Storage on
8mm Metal-Particle Media'' (6th December 1990) they claim that
ten year archival is entirely feasible under the proper conditions:

\begin{itemize}

\item Archival tapes should not be brand new.  Rather they should have
undergone somewhere between four and twenty passes through a drive before the
data are recorded on them.  This is because new tapes often contain
manufacturing debris that gets removed in the first few passes.


\item The operational and storage environments should be maintained at a
constant temperature and humidity to reduce media stress.  The preferred
environment is between 62 and 75 degrees Fahrenheit at 40 percent humidity.

\item Always return the tapes to their protective enclosures and store them on
edge rather than stacked flat.

\item Do not drop them and or put heavy weights on top of them.

\item Keep the operational and storage areas clean.

\item Clean your Exabyte drives regularly, using only the Exabyte cleaning
tapes.  In order to assure that things are going well, sample data sets should
be read back periodically and the error rates should be carefully monitored.

\end{itemize}

The Exabyte manufacturers themselves have concluded after their own extensive
tests that the only thing that they feel is crucial is that you allow a tape to
acclimatize to the environment for 24 hours before writing or reading it.  This
means that you should remove the plastic wrapper on a brand new tape at least a
day before you use it.



\subsection{General}

All of the Starlink SI59 Exabytes should be capable of fast file skip.
If you suspect that your drive is not, then see your site manager.
For example, they
can skip a BACKUP saveset containing 60,000 disc blocks in around 30 seconds.
File search operations are performed at 10 times read/write speed.
The rewind speed is around 75 times the read/write speed.
The drive operates as either a streaming tape device or as a start/stop
device, depending on the ability of the host to supply a high enough
data rate to sustain streaming (at least 246 Kb/sec).

There has been some debate over the value of the maximum block size  that is
supported by the SI59 Exabytes. The limit is, in fact, 64K -- the VMS block size
limit. The tape format is $8 \times 1$K blocks
per helical scan. Hence, there is a disadvantage in using a block size that is not
a multiple of 1K (the drive pads tape blocks to 1K with blanks), so you may as well
let BACKUP use 8K blocks (the default for BACKUP) since it is then one block per
scan. It is not thought that this gives a higher overhead on the data check.

A user may operate the Exabyte in fixed block mode or variable block mode.
Organisation of user blocks into fixed length blocks for recording on the tape
is preformed by the controller. Each fixed length data block contains bytes
from only one user block. Multiple fixed length data blocks are required
when the user block is greater than 1024 bytes. In the case of a data block
which contains less then 1024 user bytes, the remaining bytes are
``pad bytes''.

Exabytes all do a read-after-write and their own CRC checking. This is how
they are able to cope with bad patches of tape. Use of the /NOCRC is
recommended if efficiency is of paramount importance. However, it should be
noted that if you have a fault with your controller, then this may not be
apparent unless you enable software CRC.
Use of the /VERIFY qualifier in BACKUP should not be
necessary since the read-after-write ensures the integrity of the data
written but this is left up to the individual user's discretion.

\section{ Technical details}

Rumours concerning how regularly you should clean an Exabyte drive are legion.
The manual instructs us that the drive does not require routine maintenance
such as tape head/path cleaning.

However, the suppliers and maintainers recommend that the drive should be
cleaned but {\bf only} with the wet type cleaners supplied by SI. These cost
around \pounds 20 each and can be used 3 times. You should clean the drive once
per month or once per 30Gbyte.

This discrepancy in advice would appear to stem from the fact that the manufacturers of
the Exabyte originally believed that cleaning was not necessary, but have
recently revised their opinion and now recommend it.

Helical scan recorders write very narrow tracks at an acute angle to the edge
of tape in a diagonal pattern on the tape. In this way a track length is
created which is several times longer than the width of the tape.  Tracks can
be positioned by the geometry of the tape path  very precisely, thus
giving a very high number of tracks per inch.  The heads are mounted on a
rotating drum, giving an effective tape speed of 150 inches/sec, although the
actual tape movement is less than 0.5 inches/sec.

The data are written as diagonal stripes (tracks) that are around  three inches
long and have an angle of 5 degrees to the edge of the tape. Each track
consists of 8 fixed length data blocks and a servo zone.
The Exabyte writes fixed length data blocks.
Each data block
contains up to 1024 bytes of user data plus address and ECC information.
The ECC can correct a burst as long as 264 bytes in error and as many as 80
additional random errors in each block. The servo zone information is used
to ensure proper head-to-track alignment and hence allow read interchange of
cartridges between different drives.

The logical beginning of tape (LBOT) is composed of an erased length of tape,
followed by a series of tracks. The logical end of tape is determined by the
controller based on the number of tracks recorded on the tape beyond LBOT.

An end-of-file mark is a composite of an erased length of tape, followed by a
series of unique Analogue Tape Marks (ATM) tracks.
Data on labelled tapes are preceded by header and trailer
records, so each logical file is actually written as three physical files on
the tape, each with its own tape mark.

\section{Acknowledgements}

I would like to thank all of the Starlink site managers who contributed
to this document via their reports in the system managers' VAXnotes
conference and in particular to Peter Allan for his recommendations
to users.

\end{document}
