\documentstyle[11pt]{article}
\pagestyle{myheadings}

%------------------------------------------------------------------------------
\newcommand{\stardoccategory}  {Starlink User Note}
\newcommand{\stardocinitials}  {SUN}
\newcommand{\stardocnumber}    {154.1}
\newcommand{\stardocauthors}   {M.\ J.\ Bly}
\newcommand{\stardocdate}      {21 October 1992}
\newcommand{\stardoctitle}     {LTEX --- Additional \TeX\ and \LaTeX\ facilities}
%------------------------------------------------------------------------------

\newcommand{\stardocname}{\stardocinitials /\stardocnumber}
\renewcommand{\_}{{\tt\char'137}}     % re-centres the underscore
\markright{\stardocname}
\setlength{\textwidth}{160mm}
\setlength{\textheight}{230mm}
\setlength{\topmargin}{-2mm}
\setlength{\oddsidemargin}{0mm}
\setlength{\evensidemargin}{0mm}
\setlength{\parindent}{0mm}
\setlength{\parskip}{\medskipamount}
\setlength{\unitlength}{1mm}

%------------------------------------------------------------------------------
% Add any \newcommand or \newenvironment commands here
%------------------------------------------------------------------------------

\begin{document}
\thispagestyle{empty}
SCIENCE \& ENGINEERING RESEARCH COUNCIL \hfill \stardocname\\
RUTHERFORD APPLETON LABORATORY\\
{\large\bf Starlink Project\\}
{\large\bf \stardoccategory\ \stardocnumber}
\begin{flushright}
\stardocauthors\\
\stardocdate
\end{flushright}
\vspace{-4mm}
\rule{\textwidth}{0.5mm}
\vspace{5mm}
\begin{center}
{\Large\bf \stardoctitle}
\end{center}
\vspace{5mm}

\vfill

%------------------------------------------------------------------------------
%  Add this part if you want a table of contents
\setlength{\parskip}{0mm}
\tableofcontents
\setlength{\parskip}{\medskipamount}
\markright{\stardocname}
%------------------------------------------------------------------------------

\vfill

\begin{quote}\em
This software is an Option Set item within the Starlink Software Collection.
As such it may not be installed at your site. If you want LTEX, please contact
your Site Manager. LTEX requires that the Starlink release of \TeX\ be
installed.
\end{quote}

\vfill

\newpage

\section{Introduction}
\label{se:intro}

A mechanism has been devised for adding extra facilities to the Starlink release
of \TeX\ on VMS, without prejudicing the integrity of the standard
installation. Currently, the extras that have been added are:

\begin{itemize}

\item RNOTOTEX --- a converter for DEC RUNOFF files to \LaTeX

\item TEXLSE --- a language sensitive editor for \LaTeX

\item MTEX and MTLATEX --- the Springer-Verlag A\&A styles.

\end{itemize}

\section{Getting Started}
\label{se:gtg}

The additions to \TeX\ and \LaTeX\ have been added to as no to affect the
working to standard \TeX\ and \LaTeX . The TEX system search paths for fonts
and styles have been extended with the additions coming at the end of the
search path. {\em e.g.}, the {\tt tex\_inputs} search path now has {\tt
ltex\_inputs} on the end:

\begin{verbatim}
      "TEX_INPUTS" = "TEX_PLAIN_INPUTS" (LNM$SYSTEM_TABLE)
           = "TEX_LATEX_INPUTS"
           = "BIBTEX_INPUTS"
           = "LTEX_INPUTS"
\end{verbatim}

LTEX is an optional part of Starlink \TeX . If the logical name LTEX\_DIR does
not exist, LTEX is not installed at your site. You should therefore contact
your Site Manager to request it.

If LTEX is installed, the various facilities become available as part of \TeX\
after an {\tt @SSC:LOGIN}. You have to take action to start some, others are
automatically available.

\subsection{RNOTOTEX}
\label{se:rtt}

RNOTOTEX converts DSR (RUNOFF) files to \LaTeX\ format. To make RNOTOTEX
available, type:

\begin{verbatim}
      $ RNOTOTEX_SETUP
\end{verbatim}

Then RNOTOTEX is invoked:

\begin{verbatim}
      $ RNOTOTEX runoff_file latex_file
\end{verbatim}

where {\tt runoff\_file} is the input RUNOFF source file, and {\tt latex\_file}
is the name of the output \LaTeX\ file.

SUN/153 gives more details of the capabilities of the RNOTOTEX program.

{\em This facility needs to be initialised before use}

\subsection{TEXLSE}
\label{se:tlse}

TEXLSE uses the VAX Language Sensitive Editor with a section and environment
files to provide an editing session using TPU. You can call up any of the
supported \LaTeX\ styles and it provides easy access to the \LaTeX\ document
structures.

To start a TEXLSE session, type:

\begin{verbatim}
      $ EDLATEX filename.ext
\end{verbatim}

where \verb+.ext+ {\bf must} be either {\tt .TEX} or {\tt .STY}.

The Starlink User Note SUN/155 give full details of TEXLSE and the styles which
it provides (including the Springer-Verlag A\&A styles).

{\em This facility does not require initialisation.}

\subsection{MTEX and MTLATEX --- the Springer-Verlag A\&A styles}
\label{se:mt}

LTEX provides style and font files for formatting articles to be submitted to
the journal {\em Astronomy and Astrophysics}\/. MTEX and MTLATEX are special
processing options for \TeX\ and \LaTeX\ which are used when generating {\tt
.DVI} files from the A\&A styles. To use MTEX or MTLATEX, you must first type:

\begin{verbatim}
      $ MTSETUP
\end{verbatim}

This is only necessary once per login session, and adds the commands MTEX and
MTLATEX to your DCL tables. {\em However, you will need to re-issue \/{\tt \$
MTSETUP} for each sub-process you create if you wish to use MTEX or MTLATEX
from the sub-process}.

For the A\&A styles with \LaTeX , the \verb+\documentstyle+s to use are:

\begin{itemize}

\item \verb+\documentstyle{laa}+ for Computer Modern fonts

\item \verb+\documentstyle{laamt}+ for Monotype Times fonts

\end{itemize}

To use MTLATEX for \LaTeX ing a file with the \verb+\documentstyle{laamt}+, you
type:
\begin{verbatim}
      $ MTLATEX filename
\end{verbatim}


For the A\&A styles with \TeX , the \verb+\input+ files to use are:

\begin{itemize}

\item \verb+\input aa.cmm+ for Computer Modern fonts

\item \verb+\input aa.mtm+ for Monotype Times fonts

\end{itemize}

To use MTEX for \TeX ing a file with the \verb+\input aa.mtm+, you type:

\begin{verbatim}
      $ MTEX filename
\end{verbatim}


There are two demo files which show the use of the A\&A styles:

\begin{itemize}

\item {\tt ltex\_inputs:aa.dem} --- the \TeX\ demo

\item {\tt ltex\_latex\_inputs:laa.dem} --- the \LaTeX\ demo

\end{itemize}

{\em This facility needs to be initialised before use}

\end{document}
