\documentstyle{article}
\pagestyle{myheadings}

%------------------------------------------------------------------------------
\newcommand{\stardoccategory}  {Starlink User Note}
\newcommand{\stardocinitials}  {SUN}
\newcommand{\stardocnumber}    {8.2}
\newcommand{\stardocauthors}   {P T Wallace}
\newcommand{\stardocdate}      {10 May 1989}
\newcommand{\stardoctitle}     {LIBX --- Simple Library Tools}
%------------------------------------------------------------------------------

\newcommand{\stardocname}{\stardocinitials /\stardocnumber}
\markright{\stardocname}
\setlength{\textwidth}{160mm}
\setlength{\textheight}{240mm}
\setlength{\topmargin}{-5mm}
\setlength{\oddsidemargin}{0mm}
\setlength{\evensidemargin}{0mm}
\setlength{\parindent}{0mm}
\setlength{\parskip}{\medskipamount}
\setlength{\unitlength}{1mm}

\begin{document}
\thispagestyle{empty}
SCIENCE \& ENGINEERING RESEARCH COUNCIL \hfill \stardocname\\
RUTHERFORD APPLETON LABORATORY\\
{\large\bf Starlink Project\\}
{\large\bf \stardoccategory\ \stardocnumber}
\begin{flushright}
\stardocauthors\\
\stardocdate
\end{flushright}
\vspace{-4mm}
\rule{\textwidth}{0.5mm}
\vspace{5mm}
\begin{center}
{\Large\bf \stardoctitle}
\end{center}
\vspace{5mm}

\section{Introduction}

This document describes two very simple library tools:
\begin{description}
\begin{description}
\item [LIBIND] :

which outputs a list of all the modules in a library and is useful
for building more elaborate library utility procedures, and

\item [LIBPRE] :

which extracts preamble comments from all the modules
in a Fortran source library.
\end{description}
\end{description}
LIBIND and LIBPRE each consist of a command procedure which
runs a executable program.

To some extent they duplicate facilities also available in
LIBMAINT (see SUN/99).  LIBMAINT is a very powerful system
but, unavoidably, is vulnerable to changes in the formats
of the reports which the DEC Librarian utility produces.
The LIBX facilities, though limited in what they
do, use only published interfaces and should survive
VMS releases;  also they are fast.

\section{LIBIND}

The LIBIND procedure
outputs a list of all the modules in a library and is
useful for building other command procedures which manipulate
libraries.  The command is:
\begin{verbatim}
    $ @LIBX_DIR:LIBIND  library  output
\end{verbatim}
where the two arguments are the names of the library file and the
output file respectively.
The library file type defaults to .TLB,
and the output defaults to SYS\$OUTPUT.

As an example of what can be done with LIBIND, here is a command
procedure which takes a library of Fortran source modules and
unpacks it into individual files:
\begin{verbatim}
    $!  UNPACK  library_name
    $!
    $!  Unpack a text library into Fortran modules
    $!
    $!  The file type of library_name defaults to .TLB.
    $!
    $!  Create file UNPACK_SCRATCH.TMP containing list of modules
    $     P1 = F$PARSE(''P1'',".TLB",,,)
    $     DEFINE/USER_MODE LIBRARY 'P1'
    $     INDEX := UNPACK_SCRATCH.TMP
    $     DEFINE/USER_MODE OUTPUT 'INDEX'
    $     RUN LIBX_DIR:LIBIND
    $!  Extract the source
    $     OPEN FILE 'INDEX'
    $ NEXT:
    $     READ/END=DONE FILE STRING
    $     LIBR/EXTR='STRING'/OUT='STRING'.FOR/TEXT 'P1'
    $     GOTO NEXT
    $ DONE:
    $     CLOSE FILE
    $     DELETE 'INDEX';*
    $!
    $     EXIT
    $!
\end{verbatim}

\section{LIBPRE}

LIBPRE extracts the preamble comments from a text library containing
Fortran source.  The command is:
\begin{verbatim}
     $ @LIBX_DIR:LIBPRE  library_name  output_file
\end{verbatim}
Both parameters are mandatory, but their file types
default to .TLB and .LIS respectively.

The output file consists of selected records from each module
in the text library, with a form-feed character preceding
each module.
The selection is made as follows:

Lines of source are copied up to the first flagged line
(i.e.\ one beginning with
\verb%"*+", "*-", "C+", "C-", "c+" or "c-"%) and from there only if they
lie within a section included within two flagged lines.
Second and subsequent flagged sections are preceded by a blank line.

As an example, a file PROLOG.LIS containing the preamble
comments from the SLALIB library
can be obtained by executing the following DCL command:
\begin{verbatim}
    $ @LIBX_DIR:LIBPRE SLALIB_DIR:SLALIB PROLOG
\end{verbatim}
\end{document}
