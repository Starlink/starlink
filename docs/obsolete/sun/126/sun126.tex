\documentstyle[11pt]{article}
\pagestyle{myheadings}

%------------------------------------------------------------------------------
\newcommand{\stardoccategory}  {Starlink User Note}
\newcommand{\stardocinitials}  {SUN}
\newcommand{\stardocnumber}    {126.2}
\newcommand{\stardocauthors}   {Adrian Fish}
\newcommand{\stardocdate}      {6th January 1992}
\newcommand{\stardoctitle}     {STEVE --- User Guide and Reference Manual}
%------------------------------------------------------------------------------

\newcommand{\stardocname}{\stardocinitials /\stardocnumber}
\renewcommand{\_}{{\tt\char'137}}     % re-centres the underscore
\markright{\stardocname}
\setlength{\textwidth}{160mm}
\setlength{\textheight}{230mm}
\setlength{\topmargin}{-2mm}
\setlength{\oddsidemargin}{0mm}
\setlength{\evensidemargin}{0mm}
\setlength{\parindent}{0mm}
\setlength{\parskip}{\medskipamount}
\setlength{\unitlength}{1mm}

% here are some extra commands for this document
\newcommand{\gold}{\mbox{\fbox{\scriptsize GOLD}}}
\newcommand{\keyname}[1]{\mbox{\fbox{\scriptsize #1}}}
\newcommand{\STEve}{\mbox{STEVE}}

\begin{document}
\thispagestyle{empty}
SCIENCE \& ENGINEERING RESEARCH COUNCIL \hfill \stardocname\\
RUTHERFORD APPLETON LABORATORY\\
{\large\bf Starlink Project\\}
{\large\bf \stardoccategory\ \stardocnumber}
\begin{flushright}
\stardocauthors\\
\stardocdate
\end{flushright}
\vspace{-4mm}
\rule{\textwidth}{0.5mm}
\vspace{5mm}
\begin{center}
{\Large\bf \stardoctitle}
\end{center}
\vspace{5mm}

%------------------------------------------------------------------------------
%  Add this part if you want a table of contents
\setlength{\parskip}{0mm}
\tableofcontents
\setlength{\parskip}{\medskipamount}
\markright{\stardocname}
%------------------------------------------------------------------------------

\newpage

\section{Introduction}

\subsection{Preamble}
With VAX/VMS Version 4.4 there appeared a utility called the VAX Text
Processing Utility (TPU). A closer look revealed the presence of EVE, the
Extensible VAX Editor, a higher level interface to the more basic TPU. Invoking
the EVE interface by adding an appropriate  qualifier to the {\tt EDIT} command
(i.e. typing {\tt EDIT/TPU}), revealed a full screen editor with an unfamiliar
`status line' and an even more unfamiliar keypad. To make the pill easier to
swallow, DEC provided an EDT keypad emulation within EVE. With TPU (and EVE),
the VMS user is provided with a powerful and versatile tool for the
manipulation of text.

It was inevitable that many busy Starlink users would be reluctant to devote
more than a few moments of their valuable time trying out this new utility.
Despite the EDT keypad emulation EVE has often been ignored or looked upon as
something `foreign' by many users in the past. These days, more and more
Starlink users are discovering the huge advantages available to them through
the use of EVE. Even in it's vanilla DEC version, EVE is an extremely useful
and versatile editor and it is particularly easy for existing EDT users thanks
to the EDT keypad emulation (which is much better then the standard EVE
keypad!).

Now however, users should be aware that there is an extended version of the EVE
editor that has been tailored to the general Starlink user's requirements.
This extended editor is STarlink Eve or \STEve, and this document (along
with it's introductory companion SUN/125) describes this editor, and offers
additional help, advice and tips on general EVE usage. Once you've tried
\STEve, you'll forget all about using native EDT and wonder how you ever managed
before.

Of course, some of you will be wary of an `all singing, all dancing' editor,
but even some hardened Software Tools Editor dinosaurs at UCL Starlink now use
nothing but \STEve. There are one or two features in \STEve\ that it would be
very tempting to extend further, a prime example being the {\tt COMPILE}
command which could be adapted (with effort) to search for errors in the buffer
containing the code just compiled. But other DEC products available on Starlink
do this job much better. The {\tt COMPILE} command is provided in \STEve\ as a
useful convenience for quickly checking code you are currently modifying. It is
not intended for full blown programme development. All the additional features
in \STEve\ are provided as aids to efficient use of a user's time and machine
resources.

If you have bothered to read this far, you are obviously interested in trying
\STEve. In that case the brief, introductory SUN/125 tells you how to get
started straight away, but for the regular users of EVE and those of you who
are more thorough, read the following sections describing how to use
\STEve, and you will find much useful reference material in the appendices.
For the really particular and those of you who still think EVE was a biblical
celebrity, you may like to consult the `Guide to Text Processing' and
`Processing Text - EVE Reference' manuals in the VAX/VMS General User
Documentation (Volumes 5A and 5B).

So now it's up to you; try it, it really is simple to use and invaluable.
Remember that you don't have to learn all the key sequences and commands to
benefit from the new editor. You can get instant help anytime during an editing
session. All users are encouraged to follow the Tutorial Guide in this
document, if only to make sure you are aware of the most useful and commonly
used extra facilities offered.

\subsection{Notes for Users new to EVE}

Users who have never used the EVE editor, or who are new to Starlink should
start off with SUN/125. This is (deliberately) a two page document that aims to
get you off the ground with this editor. Users already familiar with EDT will
have no problem adapting to \STEve. New users are advised to become familiar
with the EDT keypad as quickly as possible, so they should keep a copy of the
appendix containing the VT200 Keypad Diagram handy. Both of these classes of
user will always benefit from having an experienced EVE user around to offer
quick and speedy advice.

\subsection{Notes for Users who already use EVE}

For those of you who already use EVE as your default editor, you will most
likely use the EDT keypad, and many of you will have customised your initial
setup in various ways. It is well worth checking through the reference
summaries in this document to ensure you are not modifying any attributes that
\STEve\ takes care of itself. Remember that using an initialization file
(either an {\tt EVE\$INIT.EVE} or a TPU command file) will make the editor
startup much slower and setting up things that are already done in \STEve\
simply wastes your time.

Users should also note that it is highly inadvisable to maintain your own
personal TPU section file. Not only are these files over a 1000 blocks in size,
they take a long time to startup, use up memory unnecessarily and can in some
instances prejudice system performance. It is much better for you and other
users to make use of the system installed \STEve\ section file, and if you
want to customize your editing environment further, place EVE and \STEve\
commands in an {\tt EVE\$INIT.EVE} command file.

\subsection{Features of STarlink Eve}

Here's a list of the main features that are available with \STEve\, in addition
to the standard EVE features (the standard EVE features being listed in Section
1.1 of the VMS EVE Reference Manual, VAX/VMS General User Documentation Volume
5B).

\begin{itemize}

\item Default enhanced EDT keypad emulation.
\item A constantly visible list of numbered buffers is maintained at the
       bottom of the screen.
\item Move between up to nine buffers in the buffer list with a single
      keystroke combination.
\item Mouse support for buffer list selection
\item Box-mode select, cut and paste with insert or overstrike.
\item Compile FORTRAN or C code from within the editor (will operate on
      the entire buffer or on a selected range e.g. a subroutine).
\item Run \LaTeX\ on the contents of the current buffer. Jumps to
      line where first error occurs.
\item Run the Starlink SPELL Checker from within the editor.
\item Automatically insert matching closing brackets, braces, dollars etc.
      (Very useful in \LaTeX).
\item Trim trailing spaces from buffers or automatically on exit.
\item Insert the current date and time into your buffer.
\item The Fill Paragraph command recognizes \LaTeX\ commands.
\end{itemize}

\section{Tutorial Guide}
\label{tutorial_guide}

\subsection{Getting Started}

First of all you will need a terminal. To fully utilise the facilities \STEve\
offers, a VT200 is ideal and the recommended choice, but a VT100 will do. You
cannot use EVE on a VT52 terminal. Ensure your terminal has been set to a VT200
(or VT100) terminal-type with an appropriate DCL command \footnote{ Generally
{\tt SET TERM/DEV=VT200} or {\tt SET TERM/DEV=VT100} is fine.} (this tutorial
will generally assume you are using a VT200 terminal). Then all you need
to do is type:

\begin{verbatim}
      $ STEVE <filename>
\end{verbatim}

%where {\tt $<$filename$>$} is a \LaTeX\ file for the purposes of this tutorial.
Once you are in you will see a buffer status line near the bottom of the screen
with appropriate buffer information superimposed; below this  your current
buffer name is listed with a number `1' next to it. The EDT keypad is available
immediately.

EVE and \STEve\ commands are entered from the keyboard after pressing the
\keyname{Do} key (or \gold\ \keyname{KP7} on VT100 terminals)\footnote{
Throughout this document you will see references to keys that are labelled on
your VT200 terminal in a standard way. KP refers to the numeric keypad to the
right of your keyboard. PF refers to the top row of four keys of the numeric
keypad, and F refers to the 18 or so programmable function keys that are
arranged in a line across the top of the keyboard. So in most cases, the GOLD
key is PF1 at the top of the numeric keypad. If you are at a VT100 terminal,
you will not have those clearly labelled extra set of keys between the numeric
keypad and the QWERTY keyboard, but the EDT keypad emulation provides identical
functions. Remember that if you are at a VT200 terminal but have it configured
as a VT100, then these extra keys will not perform as advertised. Just to
confuse matters, Pericom terminals have the PF and F nomenclature reversed, so
take care.};
at this point, commands can be entered at the {\tt Command:} prompt that
appears below the numbered buffer list. Note however that although all EVE and
\STEve\ commands can be entered on the command line, many commonly used
commands have been assigned to various quick and easy keystroke combinations.
When entering commands at the {\tt Command:} prompt, they may be abbreviated
to the smallest unique initial string. If an abbreviation is not unique, the
editor will display the possible choices and prompt for the command name again.

Press the \keyname{HELP} key and a keypad diagram will appear on the screen.
This is one of several ways to access help information in \STEve\ (see Section
\ref{getting_help} for further details on accessing on-line help information).

\subsection{A Typical Session}

One of the most useful features that \STEve\ provides is the numbered buffer
list. As already mentioned, \STEve\ maintains a list of user buffers just below
the buffer status line. If you have just entered the editor, your current
buffer name will be listed here with a number `1' next to it. To get an idea of
the facility, read in another file. To do this type \gold\ \keyname{G} and
enter a filename. The new file is displayed on the screen and another entry
will have appeared in the buffer list (with a number `2' next to it). Moving
between buffers is now very easy. Simply typing \gold\ \keyname{1} will move
you back to the first buffer, and typing \gold\ \keyname{2} will move back to
the second buffer. \STEve\ will maintain a numbered list of up to nine buffers
and allow you to move between them by typing \gold\ followed by the appropriate
buffer number.

If you are editing in a DECterm window or from a VT1200 terminal there is
additional mouse support for selecting buffers in the buffer list. Simply place
the mouse pointer over the buffer name you wish to move to and click Mouse
Button 1 (MB1).

It is often useful to display two different buffers at once on the screen.
\STEve\ provides quick keystrokes to achieve this. To split the screen into
two, type \gold\ \keyname{=}. To move between the buffers type \gold\
\keyname{$\bigtriangleup$} and \gold\ \keyname{$\bigtriangledown$}. To switch
back to one window type \gold\ \keyname{=} again.

There are a few enhancements provided by \STEve\ that are useful if you are
writing a \LaTeX\ file. You can invoke \LaTeX\ from within the editor by typing
\keyname{CTRL} \keyname{L}. The first time you invoke \LaTeX\ in an editing
session causes a subprocess to be spawned\footnote{Subsequent invocations use
the existing subprocess and so start up much quicker.}
and the current buffer contents are written to a temporary file
and processed by \LaTeX . In the event of an error during this processing,
\STEve\ will display the line number where it thinks the error is and the
cursor will move to that line number in the buffer containing the text being
processed. If the \LaTeX\ processing is completed successfully, the temporary
files are renamed to the filename associated with the buffer being \LaTeX ed.
You may turn the renaming feature off by typing
\gold\ \keyname{F19}. Similarly you can use the spelling checker from within
\STEve\ by typing \gold\ \keyname{S}. Again a subprocess is spawned the first
time Spell is invoked within an editing session, and control is passed to the
Spell program running in this subprocess until you quit and return to \STEve\
(and your corrected text).

The latest version of standard EVE introduces `box' mode cut and paste where
columns of text can be inserted and removed. In \STEve\ these facilites are
made more readily available to the user. The `box' cut and paste mode is
toggled on and off by the \keyname{F17} key. Pressing this key will make all
subsequent select, insert and remove operations use `box' mode. Pressing
\keyname{F17} again will revert back to normal cut and paste mode. Notice that
the buffer status line includes an appropriate extra field when in `box' mode.
If you are editing in a DECterm window or from a VT1200 terminal, placing the
mouse pointer over the `box' field and clicking Mouse Button 1 (MB1). Note also
that the other fields on the status line respond to a click on MB1, most
notably the `Insert' field which toggles between Insert and Overstrike mode if
you place the mouse pointer there and click MB1.

A useful feature for the \LaTeX\ user is the MATCHing facility. You can
instruct \STEve\ to insert pairs of braces, brackets or dollar signs. To set a
matching character, type \keyname{F18}; you will be prompted for a character to
match. For example, type \{. Now type \{ into your current buffer. You will see
that the \{ character is accompanied by a \} character and the cursor is placed
between the two. To stop this behaviour type \gold\ \keyname{F18} followed by
the appropriate character (see Appendix B for a list of valid matchable
characters).

Another useful key is \keyname{F20} which allows the \STEve\ user to toggle
between 80 and 132 column screen width. Associated with this are the keys
\gold\ \keyname{$\rhd$} and \gold\ \keyname{$\lhd$} wich allow you to move
`horizontally' within a buffer.

Finally (for this tutorial) there is the Compile command (abbreviated to
\keyname{CTRL} \keyname{F} which allows you to compile FORTRAN (or VAX C) code
from within \STEve. The compiler is chosen from the filetype extension of the
file you are editing. If errors occur during compilation they are displayed in
an error window. You can compile a selected range of code (for instance a
subroutine) if you prefer. Remember that this facility does not rename
temporary files on successful completion, and you must recompile your corrected
code when you leave the editor.

Check Section 3.1 for other useful keystroke combinations; it's also worth
checking out the complete command listings in the Appendices as there may be
commands or further information there you did not know about and that may prove
very useful to you.

\subsection{Exiting the Editor}

To exit from the editor, simply press key \keyname{F10} . Alternatively press
the \keyname{Do} key followed by {\tt EXIT} at the {\tt Command:} prompt. Both
these actions will write your file and exit from the editor. You can also use
the EDT equivalents \gold\ \keyname{KP7} or \keyname{CTRL} \keyname{Z} to enter
command mode, which is equivalent to pressing the \keyname{DO} key. If any of
your buffers have been modified prior to exiting, then you will be given the
choice of writing them or not.

To quit the editor without writing your buffer(s), type \gold\ \keyname{Q} or
\keyname{Do} followed  by {\tt QUIT}. If any of your buffers have been
modified at all, then you are prompted with a warning and you have the choice
to continue quitting or abort the quit command.

\subsection{Don't Stop Here \ldots}

Well, that's \STEve\ (or EVE, whichever you prefer). For those of
you still not entirely familiar with all the available commands and their
keystrokes, don't worry. Most users will need only a few of the many options
offered and this document is intended to show the reader what is available.

The next section covers many of the little niceties and features that often
users do not realise exist. It's well worth a read especially if you are new
to EVE editors.

\section{Useful Hints/Tips/Commands}

\subsection{Useful Pre-defined Key Definitions and Sequences}

Items in the list below that are specific \STEve\ modifications or additions
are shown in {\bf bold face}.

\begin{itemize}
\item Pressing the \keyname{DO} key twice repeats the previous command.

\item {\bf The {\tt FILL PARAGRAPH} command ( \gold\ \keyname{KP8} ) has been
      modified to recognise {\LaTeX} paragraphs. Will operate on a selected
      range of text if it is defined.}

\item {\bf When searching for strings, the \keyname{FIND} key will allow the
      search string to be terminated by FORWARD (\keyname{KP4}) or REVERSE
      (\keyname{KP5}) with the same effect as in EDT.}

\item \keyname{F14} toggles between INSERT and OVERSTRIKE editing modes

\item \keyname{CTRL} \keyname{V} allows the addition of control characters to
      the buffer.
      Useful for sending escape sequences to printers etc. Equivalent to
      the {\tt QUOTE} command.

\item {\bf \gold\ \keyname{S} invokes the Starlink Spell Checker from within
      the editor.
      Allows you to check the spelling on the text in the current buffer or a
      selected range of text.}

\item {\bf \keyname{CTRL} \keyname{L} invokes the \LaTeX\ compiler from within
      the editor.
      Operates on the current buffer. If an error is encountered, the cursor is
      moved to the line where the error occurred. On successful completion the
      temporary output files are renamed to the buffer's filename.}

\item {\bf \keyname{CTRL} \keyname{F} invokes either the FORTRAN or VAX C
       Compiler (depending on the buffer filename extension). The current
       buffer or selected range of code is compiled. Any errors are displayed
       in an error window and the cursor moved back to the code buffer. On
       successful compilation the error window is removed. Temporary output
      files are not renamed or saved.}

\item {\bf \gold\ \keyname{=} toggles between one and two windows on the screen.
      (c.f. {\tt TWO WINDOWS} and {\tt ONE WINDOW} commands). }

\item {\bf \gold\ \keyname{$\bigtriangleup$} and \gold\
       \keyname{$\bigtriangledown$} switch
       between two windows on the screen (c.f. {\tt OTHER WINDOW} command).}

\item {\bf \gold\ \keyname{W} writes the current buffer to disk using the
       existing filename.}

\item {\bf \gold\ \keyname{G} gets a file and creates a new buffer which
       becomes the current buffer.}

\item {\bf \gold\ \keyname{L} displays the current line length in the message
       window.}

\item {\bf \gold\ \keyname{C} displays the current column number in the
       message window.}

\item {\bf \gold\ \keyname{D} copies a current date string into the current
       buffer.}

\item {\bf \gold\ \keyname{T} shows the current time in the message window.}

\item {\bf \gold\ \keyname{/} and \gold\ \keyname{?} execute the WHAT LINE
      command.}

\item {\bf \keyname{CTRL} \keyname{A} \ sets the left margin to the current
      cursor position.}

\item {\bf \keyname{F17} toggles between Box insert mode and normal insert
       mode.}

\item {\bf \keyname{F18} matches a character (c.f. SET MATCHING command) and
       \gold\ \keyname{F18} unmatches a character (c.f. SET NOMATCHING
       command).}

\item {\bf \keyname{F20} toggles the screen width between 80 and 132 columns.}

\item {\bf \gold\ \keyname{M} moves to the MESSAGES buffer.}

\end{itemize}

\subsection{How to get Help}
\label{getting_help}

There are several ways to get help in \STEve. One of the more useful help
procedures is simply to press the \keyname{HELP} key to get a keypad diagram.
This takes a second or two. Once this is displayed, you are prompted to press
the key(s) you want help on. Once in help, follow the instructions at the
prompt.

You can access informational topics on general editor issues as well as
specific command and key help. Just press \keyname{DO} and enter {\tt HELP} at
the {\tt Command:} prompt. This displays a list of topics, and gives
instructions on how to get specific or informational help.

Don't forget you can get help on any or all keys by pressing
\gold\ \keyname{HELP} (c.f. {\tt HELP KEYS} command). This will prepare and
display a complete list of defined keys (a similar, but edited list appears in
Appendix \ref{key_defs} of this document). You are then prompted to press the
key(s) you want help on.

For instant help on any particular key, then press \gold\ \keyname{PF2} , and
you will be prompted to press the key you want help on. A brief description of
the particular key function or associated command is displayed in the message
window without affecting the current windows.

And finally Appendix \ref{deceve} of this document lists all the available
help topics in standard EVE.

\subsection{Customizing it Yourself}

There are several ways of further customizing your own editing environment by
the use of command and/or initialization files. These are the best ways to
customize the editing environment. There is also the ability for users to
create their own section files. This is discouraged as it results in users
having to keep large section files ($>$ 1200 blocks) in their filespace, and
the use of personal section files is much less efficient than using a common
system-wide section file.

For more details. refer to the descriptions of command and initialization file
usage (and default logical names you can assign to them) in the VMS General
User Volume 5B, EVE Reference Manual. Section 1.2.1.1 deals with command files
and section 1.2.1.5 deals with initialization files.

\newpage

\appendix

\section{\STEve\ Command Summary}

For your reference, here are the commands specific to \STEve,
including some of the standard EVE commands that have been modified in \STEve.
Further details are given in the \STEve\ command reference (Appendix
\ref{command_reference}).

\begin{tabbing}
cccccccccccccccccccccccccc \= cccccccccccccccccccccccccc \= ccccccccccccccccccccc \= \kill
 EDITING TEXT AND BOX OPERATIONS \\
 \\
    Box Cut       \>    Box Paste   \>   Box Select      \> Copy Date        \\
    Copy Time     \>    Set Box Select  \> Set Box Noselect \> Set Matching \\
    Set NoMatching \\
 \\
 CURSOR MOVEMENT AND SCROLLING \\
 \\
    Mess Down    \>     Mess Up  \\
 \\
 GENERAL-PURPOSE COMMANDS \\
 \\
    Column           \> Compile        \> LaTeX        \>   Length  \\
    Set Latex Nosave \> Set Latex Save \> Show Time    \>   Spell   \\
 \\
 FILE, BUFFER, AND WINDOW COMMANDS \\
 \\
    Map Buffer     \>  Set STE Windows \> Set Mapping  \>   Set NoSTE Windows \\
    Set NoMapping  \>  Set NoTrimming  \> Set Trimming \>   Shift Left  \\
    Shift Right    \>  Toggle Windows  \> Trim Buffer   \\
 \\
\end{tabbing}

\newpage

\section{\STEve\ Command Reference}
\label{command_reference}

This appendix contains a more detailed reference to all the \STEve\ commands.
Almost all of the following information is contained in the on-line \STEve\
help, accessible from within the editor.

\bigskip

\rule{\textwidth}{0.3mm}

{\Large {\bf BOX CUT} \hfill Cut a Box of Text}

\medskip
Remove the contents of the selected box range. The range is replaced in one of
two ways depending on the mode of the current buffer, i.e. whether it is in
INSERT mode or OVERSTRIKE mode. In the former, the gap is closed by moving any
text to the right-hand side of the range to the left. In the latter, the space
is filled with blanks. Identical to the standard EVE command except the Box
commands maintain a Box field on buffer status line. Care must be taken when
inserting text in box mode when you place the cursor beyond the end of a line
(see the EVE commands SET BOX PADDING and SET BOX NOPADDING).

\begin{description}
\item[Related Topics]:
\begin{tabbing}
ccccccccccccccccc\=cccccccccccccccccccc\=ccccccccccccccccccccccccccccc\=\kill
BOX PASTE      \> BOX SELECT   \> SET BOX NOSELECT  \>   SET BOX SELECT \\
\end{tabbing}
\item[Keys]:
           \begin{itemize}
           \item \gold\ \keyname{Remove}
           \item \keyname{Remove} if in Box Mode (toggled by \keyname{F17} ).
           \end{itemize}

\end{description}

\goodbreak

\rule{\textwidth}{0.3mm}

{\Large {\bf BOX PASTE} \hfill Paste a Box of Text}

\medskip
Insert the contents of the box paste buffer at the current editing position.
The contents of the paste buffer are inserted in one of two ways depending on
the mode of the current buffer, i.e. whether it is in INSERT mode or OVERSTRIKE
mode. In the former, the contents of the paste buffer are inserted and any text
that is within it's range is moved over to the right. In the latter, the
contents of the paste buffer overwrite any existing text that is in it's range.
Identical to the standard EVE command except the Box commands maintain a Box
field on buffer status line. Care must be taken when inserting text in box mode
when you place the cursor beyond the end of a line (see the EVE commands SET
BOX PADDING and SET BOX NOPADDING).

\begin{description}
\item[Related Topics]:
\begin{tabbing}
cccccccccccccccc\=ccccccccccccccccccc\=ccccccccccccccccccccccccccccc\=\kill
BOX CUT        \> BOX SELECT   \> SET BOX NOSELECT  \>  SET BOX SELECT \\
\end{tabbing}
\item[Keys]:
         \begin{itemize}
         \item \gold\ \keyname{Insert}
         \item \keyname{Insert} if in Box Mode (toggled by \keyname{F17} ).
         \end{itemize}
\end{description}

\goodbreak

\rule{\textwidth}{0.3mm}

{\Large {\bf BOX SELECT} \hfill Select a Box of Text}

\medskip
Start selection of a box region which is to be removed and/or inserted.
Identical to the standard EVE command except the Box commands maintain a Box
field on buffer status line.

\begin{description}
\item[Related Topics]:
\begin{tabbing}
ccccccccccccccccc\=ccccccccccccccccccc\=ccccccccccccccccccccccccccccc\=\kill
BOX CUT \>   BOX PASTE  \>   SET BOX NOSELECT  \>   SET BOX SELECT \\
\end{tabbing}
\item[Keys]:
         \begin{itemize}
         \item \gold\ \keyname{Select}
         \item \keyname{Select} if in Box Mode (toggled by \keyname{F17} ).
         \end{itemize}

\end{description}

\goodbreak

\rule{\textwidth}{0.3mm}

{\Large {\bf COLUMN} \hfill Display Current Column}

\medskip
  Displays in the message window the column number that the current cursor
  position occupies.

\begin{description}
\item[Related Topics]:
\begin{tabbing}
cccccccccccc\=cccccccccccccccccc\=ccccccccccccccccccccccccccccc\=\kill
 \> LENGTH \\
\end{tabbing}

\item[Keys]:
           \begin{itemize}
           \item \gold\ \keyname{C}
           \end{itemize}
\end{description}

\goodbreak

\rule{\textwidth}{0.3mm}

{\Large {\bf COMPILE} \hfill Compile a Buffer or Range}

\medskip
  \STEve\ offers single keystroke compilation of the FORTRAN or C source
  in the current buffer. Success/Failure is reported on the message
  line. Any compilation errors will appear in an extra Compiler
  window. If the compilation is successful, then the Compiler
  window will be deleted.

  COMPILE will process the currently selected region of code, or the
  whole buffer if no select is active. Will not compile a selected box
  of code.

  Users should note that this facility is used only for checking code. The
  editor will not preserve the compiled code but will delete the temporary
  files associated with the procedure. The edited code is written
  on exit (unless the user {\tt QUIT}s the editing session), and the user
  must then compile again.

\begin{description}
\item[Keys]:
          \begin{itemize}
          \item \keyname{CTRL}\ \keyname{F}
          \end{itemize}
\end{description}

\goodbreak

\rule{\textwidth}{0.3mm}

{\Large {\bf COPY DATE} \hfill Copy Date String to Buffer}

\medskip
  Copies today's date in the form `dd-Mmm-yy' at the current editing
  position. The current text entry mode (insert or overstrike) is used.

\begin{description}
\item[Related Topics]:
\begin{tabbing}
cccccccccccccccccccccc\=cccccccccccccccccc\=ccccccccccccccccccccccccccccc\=\kill
COPY TIME \> SHOW TIME  \\
\end{tabbing}

\item[Keys]:
           \begin{itemize}
           \item \gold\ \keyname{D}
           \end{itemize}
\end{description}

\goodbreak

\rule{\textwidth}{0.3mm}

{\Large {\bf COPY TIME} \hfill Copy Current Time String to Buffer}

\medskip
  Copies the current time in the form `hh:mm:ss' at the current editing
  position. The current text entry mode (insert or overstrike) is used.

\begin{description}
\item[Related Topics]:
\begin{tabbing}
cccccccccccccccccccccc\=cccccccccccccccccc\=ccccccccccccccccccccccccccccc\=\kill
COPY DATE \> SHOW TIME \\
\end{tabbing}

\item[Keys]:
          \begin{itemize}
          \item \STEve\ does not define a key for COPY TIME
          \end{itemize}
\end{description}
\goodbreak

\rule{\textwidth}{0.3mm}

{\Large {\bf MAP BUFFER} \hfill Move to Mapped Buffer}

\medskip
  Map one of the buffers listed in the buffer map on to the current
  window. Equivalent to \gold\ \keyname{$<n>$} (where n is one of the numeric
  keys (1 to 9) across the top of the QWERTY keyboard) used to select buffer n
  in the buffer map. Alternatively a mouse can be used to select a buffer for
  editing sessions carried out from a DECterm or VT1200 window. Simply place
  the mouse pointer over the desired buffer name and click Mouse Button 1
  (MB1).
  \begin{itemize}
  \item The buffer map is displayed by default and is just below the status
        line of the lowest window on the screen.
  \item If \gold\ \keyname{$<n>$} is typed and buffer n does not exist, the user
        is prompted for a file to get.
  \item Where appropriate, place mouse pointer over desired buffer name and
        click Mouse Button 1 (MB1).
  \end{itemize}

\begin{description}
\item[Related Topics]:
\begin{tabbing}
cccccccccccccccccccccccccccccc\=ccccccccccccccccccccccccc\=\kill
SET MAPPING  \> SET NOMAPPING \\
\end{tabbing}

\item[Keys]:
          \begin{itemize}
          \item \gold\ \keyname{$<n>$} where 1 $\leq$ n $\leq$ 9
          \end{itemize}

\end{description}

\goodbreak

\rule{\textwidth}{0.3mm}

{\Large {\bf LATEX} \hfill \LaTeX\ the Current Buffer}

\medskip
  The LATEX command invokes \LaTeX\ and processes the entire contents
  of the current buffer (ranges are not allowed). If an error occurs,
  then the cursor is placed in the LaTeX buffer at the line where the
  error occured (on a best efforts basis!). On success the temporary output
  files are renamed to the filename associated with the buffer containing the
  \LaTeX\ code. This action can be turned off be the command SET LATEX NOSAVE
  (\gold\ \keyname{F19}).

\begin{description}
\item[Related Topics]:
\begin{tabbing}
cccccccccccccccccccccccccccccccccccccc\=cccccccccccccccccccc\=\kill
SET LATEX NOSAVE    \> SET LATEX SAVE \\
\end{tabbing}

\item[Keys]:
          \begin{itemize}
          \item \keyname{CTRL}\ \keyname{L}
          \end{itemize}

\end{description}

\goodbreak

\rule{\textwidth}{0.3mm}

{\Large {\bf LENGTH} \hfill Display Current Line Length}

\medskip
  Displays the current line length.

\begin{description}
\item[Related Topics]:
\begin{tabbing}
cccccccccccccccccccc\=cccccccccccccccccc\=ccccccccccccccccccccccccccccc\=\kill
COLUMN \\
\end{tabbing}

\item[Keys]:
         \begin{itemize}
         \item \gold\ \keyname{L}
         \end{itemize}
\end{description}

\goodbreak

\rule{\textwidth}{0.3mm}

{\Large {\bf MESS DOWN} \hfill Scroll Message Buffer}

\medskip
  Scroll the message window down by one line. Useful when that TALK
  message whizzes off the message line before you can read it.

\begin{description}
\item[Related Topics]:
\begin{tabbing}
cccccccccccccccccccccc\=cccccccccccccccccc\=ccccccccccccccccccccccccccccc\=\kill
MESS UP \\
\end{tabbing}
\item[Keys]:
        \begin{itemize}
        \item \gold\ \keyname{Next}
        \item \gold\ \keyname{M} moves to the MESSAGES buffer
        \end{itemize}
\end{description}

\goodbreak

\rule{\textwidth}{0.3mm}

{\Large {\bf MESS UP} \hfill Scroll Message Buffer}

\medskip
  Scroll the message window up by one line. Useful when that TALK
  message whizzes off the message line before you can read it.

\begin{description}
\item[Related Topics]:
\begin{tabbing}
cccccccccccccccccccccc\=cccccccccccccccccc\=ccccccccccccccccccccccccccccc\=\kill
MESS DOWN \\
\end{tabbing}

\item[Keys]:
        \begin{itemize}
        \item \gold\ \keyname{Prev}
        \item \gold\ \keyname{M} moves to the MESSAGES buffer
        \end{itemize}
\end{description}

\goodbreak

\rule{\textwidth}{0.3mm}

{\Large {\bf SET BOX NOSELECT} \hfill Disable Box Select Mode}

\medskip
  Disables the tying of the box cut and paste commands to the normal
  cut and paste keys. This does not prevent the appropriate \gold -shifted
  keys performing box cut and paste operations.

\begin{description}
\item[Related Topics]:
\begin{tabbing}
cccccccccccccccc\=cccccccccccccccccc\=ccccccccccccccccccccccccc\=\kill
BOX CUT   \> BOX PASTE   \> BOX SELECT \> SET BOX SELECT \\
\end{tabbing}

\item[Keys]:
          \begin{itemize}
          \item \keyname{F17} toggles between Box Select mode and Default
                Select mode
          \item Mouse support available from a DECterm or VT1200 window.
                Place mouse pointer over `Box' filed and click Mouse
                Button 1 (MB1).
          \end{itemize}

\end{description}

\rule{\textwidth}{0.3mm}

{\Large {\bf SET BOX SELECT} \hfill Enable Box Select Mode}

\medskip
  Ties the box cut and paste commands to the normal cut and paste keys.
  Original key definitions are restored by the command SET BOX NOSELECT.

\begin{description}
\item[Related Topics]:
\begin{tabbing}
cccccccccccccccc\=cccccccccccccccccc\=ccccccccccccccccccccccccc\=\kill
BOX CUT   \> BOX PASTE  \> BOX SELECT \> SET BOX NOSELECT \\
\end{tabbing}

\item[Keys]:
          \begin{itemize}
          \item \keyname{F17} toggles between Box Select mode and Default
                Select mode
          \end{itemize}

\end{description}

\goodbreak

\rule{\textwidth}{0.3mm}

{\Large {\bf SET STE WINDOWS} \hfill Enable Extra STE Window Keys}

\medskip
  Defines keys to perform common buffer manipulations. All of
  these key definitions, where relevant, maintain the buffer map
  display.

\begin{description}
\item[Related Topics]:
\begin{tabbing}
ccccccccccccccccccccccccccccccc\=cccccccccccccccccccc\=cccccccccccccccccccccc\=\kill
SET NOSTE WINDOWS \> SHIFT LEFT  \> SHIFT RIGHT \> TOGGLE WINDOWS \\
\end{tabbing}

\item[Keys defined by this command]:
       \begin{itemize}
       \item \gold\ \keyname{G} prompts the user for a new file.
       \item \gold\ \keyname{W} writes the current buffer. The user is
             prompted for a filename if none already specified.
       \item \gold\ \keyname{Delete} deletes the current buffer (you are
             given a chance to change your mind if the buffer has been
             modified).
       \item \gold\ \keyname{=} toggles between one and two windows.
       \item \gold\ \keyname{$\bigtriangleup$} and \gold\
             \keyname{$\bigtriangledown$} switch to the previous and next
             windows respectively.
       \item \gold\ \keyname{$\lhd$} and \gold\ \keyname{$\rhd$} shift the
             current window left and right by 40 characters to allow easy
             reading of very long lines.
       \end{itemize}

\end{description}

\goodbreak

\rule{\textwidth}{0.3mm}

{\Large {\bf SET LATEX NOSAVE} \hfill Delete \LaTeX\ Temporary Files}

\medskip
  After this command is issued, all temporary output files from LaTeX
  processing done with the LATEX command (i.e. .AUX,
   .LIS and .DVI files) will not be kept.

\begin{description}
\item[Related Topics]:
\begin{tabbing}
cccccccccccccccccccccccccccccccccc\=cccccccccccccccccccc\=\kill
    SET LATEX SAVE  \>     LATEX \\
\end{tabbing}

\item[Keys]:
        \begin{itemize}
        \item \gold\ \keyname{F19}
        \end{itemize}

\end{description}

\goodbreak

\rule{\textwidth}{0.3mm}

{\Large {\bf SET LATEX SAVE} \hfill Save and Rename \LaTeX\ Files}

\medskip
  After this command is issued, all temporary output files from LaTeX
  processing done with the LATEX command (i.e. .AUX,
  .LIS and .DVI files) will be saved and renamed to the filename
  associated with the buffer containing the LaTeX code. However,
  if the buffer has no filename associated with it, these files will
  not be saved.

  The default action is to SAVE the LaTeX output files.

\begin{description}
\item[Related Topics]:
\begin{tabbing}
cccccccccccccccccccccccccccccccccc\=cccccccccccccccccccc\=\kill
    SET LATEX NOSAVE  \>   LATEX \\
\end{tabbing}

\item[Keys]:
       \begin{itemize}
       \item \keyname{F19}
       \end{itemize}

\end{description}

\goodbreak

\rule{\textwidth}{0.3mm}

{\Large {\bf SET MAPPING} \hfill Enable Mapped Buffer List}

\medskip
  Mapping in \STEve\ allows the editor to create and maintain a numbered
  list of user buffers. In addition the \gold\ \keyname{$<n>$} keys as
  described below are
  defined with this command to allow quick and easy movement between
  multiple buffers. Enabled by default. Mouse support is also provided for
  editing sessions performed from a DECterm or VT1200 window.
  \begin{itemize}
  \item  The buffer map is displayed by default and is just below the status
         line of the lowest window on the screen.
  \item  It consists of a list of active user buffers, each preceded
         by a number which can be used with the \gold\ \keyname{$<n>$} key
         combination or the MAP BUFFER command.
  \item  For DECterm and VT1200 sessions, placing the mouse pointer over
         the desired buffer name and clicking Mouse Button 1 (MB1), selects
         that buffer.
  \end{itemize}

\begin{description}
\item[Related Topics]:
\begin{tabbing}
cccccccccccccccccccccccccccccccccc\=cccccccccccccccccccc\=\kill
MAP BUFFER  \> SET NOMAPPING  \\
\end{tabbing}

\item[Keys]:
          \begin{itemize}
          \item \gold\ \keyname{$<n>$} where 1 $\leq$ n $\leq$ 9
          \end{itemize}

\end{description}

\goodbreak

\rule{\textwidth}{0.3mm}

{\Large {\bf SET MATCHING} \hfill Match and Insert Characters}

\medskip
  Defines a set of characters, all of which must be the opening member
  of a pair. When one of these characters is typed, the matching closing
  character is inserted as well and the cursor is positioned between the
  two characters.

  Currently the matchable character list contains the following:
\begin{verbatim}
               ( [ { " $     matched to    $ " } ] )
\end{verbatim}

\begin{description}
\item[Related Topics]:
\begin{tabbing}
cccccccccccccccccccccccccccccccccccccc\=cccccccccccccccccccc\=\kill
    SET NOMATCHING \\
\end{tabbing}

\item[Keys]:
       \begin{itemize}
       \item \keyname{F18}
       \end{itemize}

\end{description}

\goodbreak

\rule{\textwidth}{0.3mm}

{\Large {\bf SET NOSTE WINDOWS} \hfill Disable STE Window Keys}

\medskip
  Undefine keys that perform common buffer manipulations.

\begin{description}
\item[Related Topics]:
\begin{tabbing}
cccccccccccccccccccccccccccccccccccccc\=cccccccccccccccccccc\=\kill
    SET STE WINDOWS \\
\end{tabbing}

\item[Keys]:
          \begin{itemize}
          \item \STEve\ does not define a key for SET NOSTE WINDOWS
          \end{itemize}
\end{description}

\goodbreak

\rule{\textwidth}{0.3mm}

{\Large {\bf SET NOMAPPING} \hfill Disable Mapped Buffers}

\medskip
  Disables the display of the buffer map window and undefines the buffer
  map keys. The buffer map window is displayed by default.

\begin{description}
\item[Related Topics]:
\begin{tabbing}
cccccccccccccccccccccccccccccccccccccc\=cccccccccccccccccccc\=\kill
    MAP BUFFER \>   SET MAPPING \\
\end{tabbing}

\item[Keys]:
          \begin{itemize}
          \item \STEve\ does not define a key for SET NOMAPPING
          \end{itemize}
\end{description}

\goodbreak

\rule{\textwidth}{0.3mm}

{\Large {\bf SET NOMATCHING} \hfill Unmatch Characters}

\medskip
  Removes a character or a string of characters from the list of characters
  set by previous use of the SET MATCHING command.

  Currently the matchable character list contains the following:
\begin{verbatim}
               ( [ { " $     matched to    $ " } ] )
\end{verbatim}

\begin{description}
\item[Related Topics]:
\begin{tabbing}
cccccccccccccccccccccccccccccccccccccc\=cccccccccccccccccccc\=\kill
    SET MATCHING \\
\end{tabbing}

\item[Keys]:
       \begin{itemize}
       \item \gold\ \keyname{F18}
       \end{itemize}

\end{description}

\goodbreak

\rule{\textwidth}{0.3mm}

{\Large {\bf SET NOTRIMMING} \hfill Disable Buffer Auto-Trim}

\medskip
  Disables automatic trimming of trailing white space from buffers
  prior to writing them to disk. Trimming is disabled by default.

\begin{description}
\item[Related Topics]:
\begin{tabbing}
cccccccccccccccccccccccccccccccccc\=cccccccccccccccccccc\=\kill
    SET TRIMMING \>   TRIM BUFFER \\
\end{tabbing}

\item[Keys]:
          \begin{itemize}
          \item \STEve\ does not define a key for SET NOTRIMMING
          \end{itemize}
\end{description}

\goodbreak

\rule{\textwidth}{0.3mm}

{\Large {\bf SET TRIMMING} \hfill Enable Buffer Auto-Trim}

\medskip
  Enables automatic trimming of trailing white space from buffers
  prior to writing them to disk. This is disabled by default.

\begin{description}
\item[Related Topics]:
\begin{tabbing}
cccccccccccccccccccccccccccccccccc\=cccccccccccccccccccc\=\kill
    SET NOTRIMMING \>   TRIM BUFFER \\
\end{tabbing}

\item[Keys]:
          \begin{itemize}
          \item \STEve\ does not define a key for SET TRIMMING
          \end{itemize}
\end{description}

\goodbreak

\rule{\textwidth}{0.3mm}

{\Large {\bf SHIFT LEFT} \hfill Shift Window Left}

\medskip
  Shifts the current window 40 characters to the left and indicates
  where you are via a message in the message buffer.

\begin{description}
\item[Related Topics]:
\begin{tabbing}
ccccccccccccccccccccccccccccccccccccc\=cccccccccccccccccccc\=\kill
    SET NOSTE WINDOWS \>   SET STE WINDOWS\\
\end{tabbing}

\item[Keys]:
     \begin{itemize}
     \item \gold\ \keyname{$\lhd$}
     \end{itemize}

\end{description}

\goodbreak

\rule{\textwidth}{0.3mm}

{\Large {\bf SHIFT RIGHT} \hfill Shift Window Right}

\medskip
  Shifts the current window 40 characters to the right and indicates
  where you are via a message in the message buffer.

\begin{description}
\item[Related Topics]:
\begin{tabbing}
ccccccccccccccccccccccccccccccccccccc\=cccccccccccccccccccc\=\kill
    SET NOSTE WINDOWS \>   SET STE WINDOWS\\
\end{tabbing}

\item[Keys]:
     \begin{itemize}
     \item \gold\ \keyname{$\rhd$}
     \end{itemize}

\end{description}

\goodbreak

\rule{\textwidth}{0.3mm}

{\Large {\bf SHOW TIME} \hfill Display Current Time}

\medskip
  Displays the current time in the form `hh:mm:ss' in the message
  window.

\begin{description}
\item[Related Topics]:
\begin{tabbing}
cccccccccccccccccccccc\=cccccccccccccccccccccccc\=ccccccccccccccccccccccccccccc\=\kill
COPY DATE \> COPY TIME \\
\end{tabbing}

\item[Keys]:
    \begin{itemize}
    \item \gold\ \keyname{T}
    \end{itemize}
\end{description}

\goodbreak

\rule{\textwidth}{0.3mm}

{\Large {\bf TOGGLE WINDOWS} \hfill Switch Windows}

\medskip
  Switches between one and two windows.

\begin{description}
\item[Related Topics]:
\begin{tabbing}
cccccccccccccccccccccccccccccccccccccc\=cccccccccccccccccccc\=\kill
SET NOSTE WINDOWS \> SET STE WINDOWS\\
\end{tabbing}

\item[Keys]:
        \begin{itemize}
        \item \gold\ \keyname{=}
        \end{itemize}
\end{description}

\goodbreak

\rule{\textwidth}{0.3mm}

{\Large {\bf TRIM BUFFER} \hfill Trim Buffer of whitespace}

\medskip
  Trim trailing white space from the current buffer.

\begin{description}
\item[Related Topics]:
\begin{tabbing}
cccccccccccccccccccccccccccccccccccccc\=cccccccccccccccccccc\=\kill
SET NOTRIMMING \> SET TRIMMING \\
\end{tabbing}

\item[Keys]:
          \begin{itemize}
          \item \STEve\ does not define a key for TRIM BUFFER
          \end{itemize}
\end{description}

\newpage

\section{\STEve\ Key Definitions}
\label{key_defs}

Below is an edited list of key definitions including those more commonly used
and provided by default with the current \STEve\ implementation. Help on a key
is available from within the editor by typing \gold\ \keyname{PF2} followed by
the key you want help on. Alternatively a complete list of key definitions,
similar to the one below can be created within the editor by pressing \gold\
\keyname{HELP} , and more detailed help is available at that point by following
the prompt instructions.
\begin{small}
\begin{verbatim}
 Key         Function                Gold sequence    Function
                                      (GOLD = PF1)
 CTRL/A      Set left margin
 CTRL/B      Recall                  GOLD-/           What line
 CTRL/F      Compile                 GOLD-0           Repeat
 CTRL/K      Learn                   GOLD-1           Map buffer #1
 CTRL/L      Latex                   GOLD-2           Map buffer #2
 RETURN      Return                  GOLD-3           Map buffer #3
 CTRL/R      Remember                GOLD-4           Map buffer #4
 CTRL/V      Quote                   GOLD-5           Map buffer #5
 CTRL/W      Refresh                 GOLD-6           Map buffer #6
 CTRL/Z      Do                      GOLD-7           Map buffer #7
 FIND        Find                    GOLD-8           Map buffer #8
 INSERT_HERE Insert here             GOLD-9           Map buffer #9
 REMOVE      Remove                  GOLD-=           Toggle number of windows
 SELECT      Select                  GOLD-?           What line
 PREV_SCREEN Previous screen         GOLD-C           Column
 NEXT_SCREEN Next screen             GOLD-D           Copy date
 F10         Exit                    GOLD-G           Get file
 F11         Forward reverse         GOLD-L           Line length
 F12         Start of line           GOLD-M           Buffer messages
 F13         Delete previous word    GOLD-Q           Quit
 F14         Insert overstrike       GOLD-S           Spell
 F17         Toggle Box Mode         GOLD-T           Show time
 F18         Match Character         GOLD-W           Write file
 F19         Latex File Save         GOLD-X           Exit
 F20         Toggle Width 80/132     GOLD-DELETE      Delete current buffer
                                     GOLD-UP          Previous window
                                     GOLD-DOWN        Next window
                                     GOLD-RIGHT       Shift right
                                     GOLD-LEFT        Shift left
                                     GOLD-INSERT_HERE Box insert
                                     GOLD-REMOVE      Box cut
                                     GOLD-SELECT      Box select
                                     GOLD-PREV_SCREEN Mess up
                                     GOLD-NEXT_SCREEN Mess dwn
                                     GOLD-HELP        Help keys
                                     GOLD-F17         UnMatch Character
                                     GOLD-F18         Latex File Nosave
\end{verbatim}
\end{small}

\newpage

\section{VT200 Keypad Diagram}

A VT200 terminal is the recommended terminal for \STEve\ users. The keypad
diagram below shows the EDT keypad, the standard ANSI extension keys and the
\STEve\ function keys \keyname{F17} to \keyname{F20} . Users are advised not to
define these four function keys to allow for future \STEve\ enhancements. The
keys \keyname{F7} to \keyname{F9} (and their \gold\ shifted functions) are
reserved for users to program as they wish.


\begin{small}
\begin{verbatim}
              GOLD key functions are shown BELOW primary key functions.
                     _______________________    _______________________________
   To get help on   | HELP  |      Do       |  |BoxMode| Match |Lat Sav|Scr Wid|
   commands, type   |KeyDefs|               |  |       |UnMatch|Lat NoS|       |
   a command or ?   |_______|_______________|  |_______|_______|_______|_______|
   and press         _______________________    _______________________________
   RETURN.          | Find  |Ins Her|Remove |  | Gold  | HELP  |FndNxt | Del L |
                    |Wil Fin|Box Ins|Box Cut|  |  key  |Sho Key| Find  |Res Lin|
   To list all key  |_______|_______|_______|  |_______|_______|_______|_______|
   definitions,     |Select |Pre Scr|Nex Scr|  |MovByPa| Sect  |Append | Del W |
   type Keys and    |Box Sel|Mess Up|Mes Dwn|  |  Do   | Fill  |EDT Rep|Res Wor|
   press RETURN,    |_______|_______|_______|  |_______|_______|_______|_______|
   or press GOLD-           |Move up|          |Forward|Reverse|Remove | Del C |
   HELP.                    |Pre Win|          |Bottom |  Top  |Ins Her|Res Cha|
                     _______|_______|_______   |_______|_______|_______|_______|
   To show a key    |Mov Lef|Mov Dow|Mov Rig|  | Word  |  EOL  | Char  |       |
   definition, use  |Shi Lef|Nex Win|Shi Rig|  |ChngCas|Del EOL|SpecIns|Return |
   SHOW KEY.        |_______|_______|_______|  |_______|_______|_______| Subs  |
                                               |   EDT Line    |Select |       |
                     Use the DO key to enter   |   Open Line   | Reset |       |
                        advanced commands      |_______________|_______|_______|
\end{verbatim}
\end{small}

\newpage

\section{VT100 Keypad Diagram}

A VT100 terminal provides less functionality than a VT200, but the EDT keypad
is fully configured, and `arrow' keys are provided. Any other commands can be
entered on the command line, by pressing the \keyname{Do} synonym \gold\
\keyname{KP7} .

\begin{small}
\begin{verbatim}
              GOLD key functions are shown BELOW primary key functions.
        _______________________________         _______________________________
       |Move up|Mov Dow|Mov Lef|Mov Rig|       | Gold  | HELP  |FndNxt | Del L |
       |Pre Win|Nex Win|Shi Lef|Shi Rig|       |  key  |Sho Key| Find  |Res Lin|
       |_______|_______|_______|_______|       |_______|_______|_______|_______|
                                               |MovByPa| Sect  |Append | Del W |
       To get help on commands, type a         |  Do   | Fill  |EDT Rep|Res Wor|
       command or ? and press RETURN.          |_______|_______|_______|_______|
                                               |Forward|Reverse|Remove | Del C |
       To list all key definitions, type       |Bottom |  Top  |Ins Her|Res Cha|
       Keys and press RETURN, or press         |_______|_______|_______|_______|
       GOLD-HELP.                              | Word  |  EOL  | Char  |       |
                                               |ChngCas|Del EOL|SpecIns|Return |
       To show a key definition, use           |_______|_______|_______| Subs  |
       SHOW KEY.                               |   EDT Line    |Select |       |
                                               |   Open Line   | Reset |       |
       Synonyms for the DO key:                |_______________|_______|_______|
              GOLD-KP7
\end{verbatim}
\end{small}

\newpage

\section{Installation Instructions}
\subsection{Installation Instructions for System Managers}
\label{system_install}

This appendix includes instructions for the system manager on how to install
\STEve\ system-wide. These instructions are identical to those issued in the
relevant SSC when this software item was originally released, and is included
here for reference.

The following logical definitions are in {\tt SSC:STARTUP.COM}:
\begin{small}
\begin{verbatim}
      $!
      $! STEVE - Set up logicals for STEVE TPU editor
      $!
      $       DEFINE/SYSTEM STEVE_DIR STARDISK:[STARLINK.SYSTEM.STEVE]
      $       DEFINE/SYSTEM/EXEC STEVE_SECTION STEVE_DIR:STEVE.TPU$SECTION
      $       DEFINE/SYSTEM STEVE_HELP STEVE_DIR:STEVE.HLB
      $!
\end{verbatim}
\end{small}
and the following image installation should be done in
{\tt LSSC:STARTUP.COM}:
\begin{small}
\begin{verbatim}
      $!
      $! STEVE - install the STEVE section file
      $!
      $ MCR INSTALL
      ADD STEVE_SECTION  /SHARED/HEADER/OPEN
      EXIT
      $!
\end{verbatim}
\end{small}
The following symbol assignments should be done in {\tt SSC:LOGIN.COM}:
\begin{small}
\begin{verbatim}
      $!
      $! STEVE TPU Editor
      $!
      $         EVE    :== EDIT/TPU
      $         STEV*E :== EDIT/TPU/SECTION=STEVE_SECTION
\end{verbatim}
\end{small}
Note that DEC recommend defining the {\tt EVE} symbol anyway, and this may
already have been set up on your node.

\subsection{Installation Instructions for Users}

The \STEve\ editor is an optional Starlink Software Item. At some nodes
therefore, other system-wide editors may be provided. This appendix shows how
any user can access this valuable editor at nodes where \STEve\ is not
installed system-wide.

To check if the \STEve\ editor is installed on your system, enter the DCL
command:

\begin{verbatim}
      $ SHOW LOGICAL STEVE_SECTION
\end{verbatim}

If this results in the following text:

\begin{verbatim}
      "STEVE_SECTION" = "STEVE_DIR:STEVE.TPU$SECTION" (LNM$SYSTEM_TABLE)
\end{verbatim}

then \STEve\ is installed on your system and you need not read this appendix
any further. New users should read SUN/125 (intended for the absolute
beginner), and Section \ref{tutorial_guide} in this document for information on
how to use the editor.

If this is not the case, or there is no translation for the logical name
{\tt TPU\$SECTION}, then you should follow the instructions below.

To access \STEve\ you must create a symbol and three logical names, usually by
placing them in your {\tt LOGIN.COM} file. The symbol {\tt EVE} may already be
set up on your node, and you can check this with the DCL command  {\tt \$ SHOW
SYMBOL EVE}. As an example the following  lines should be added to your {\tt
LOGIN.COM} file:

\begin{verbatim}
      $ STEV*E :== EDIT/TPU/SECTION=STEVE_SECTION
      $ DEFINE STEVE_DIR STARDISK:[STARLINK.SYSTEM.STEVE]
      $ DEFINE STEVE_SECTION STEVE_DIR:STEVE.TPU$SECTION
      $ DEFINE STEVE_HELP STEVE_DIR:STEVE.HLB
\end{verbatim}

A user may modify the editing environment further by creating a file in their
{\tt SYS\$LOGIN} directory, named {\tt EVE\$INIT.EVE} in which any \STEve\ and
EVE commands may be placed that a user wishes executed each time the editor is
invoked. Customizing the initial editing environment in this way will make
editor startup slower. The user is advised to check whether the standard
\STEve\ editor does not already have the commands and functions set up already.

{\bf Note:} {\em For optimum use of \STEve, the section file should be
installed system-wide. Users should ask their local system manager to install
\STEve\ as described in Appendix \ref{system_install}}.

\newpage
\section{Standard EVE Reference}
\label{deceve}
For completeness, here's a list of all the original EVE commands, some of which
have been modified slightly in \STEve.  This list is intended to help you
remember an EVE command without having to dig out a copy of the Guide to Text
Processing on VAX/VMS. This command list is complete for EVE V2.6 (as supplied
with VAX/VMS 5.4).

\begin{tabbing}
cccccccccccccccccccccccccc \= ccccccccccccccccccccccccccccc \= ccccccccccccccccccccc \= \kill

 EDITING TEXT \\
 \\
    Change Mode           \>  Erase Word       \>    Restore Character   \\
    Copy                  \>  Insert Here      \>    Restore Line        \\
    Cut                   \>  Insert Mode      \>    Restore Selection   \\
    Delete                \>  Overstrike Mode  \>    Restore Sentence    \\
    Erase Character       \>  Paste            \>    Restore Word        \\
    Erase Line            \>  Quote            \>    Select              \\
    Erase Previous Word   \>  Remove           \>    Select All          \\
    Erase Start Of Line   \>  Restore          \>    Store Text          \\
 \\
 BOX OPERATIONS\\
 \\
    Box Copy           \> Box Paste             \> Set Box Nopad         \\
    Box Cut            \> Box Paste Insert      \> Set Box Noselect      \\
    Box Cut Insert     \> Box Paste Overstrike  \> Set Box Pad           \\
    Box Cut Overstrike \> Box Select            \> Set Box Select        \\
                       \> Restore Box Selection \>                       \\
 \\
 SEARCHES\\
\\
    Find             \>  Set Find Case Exact    \> Set Wildcard Ultrix   \\
    Find Next        \>  Set Find Case Noexact  \> Set Wildcard VMS      \\
    Find Selected    \>  Set Find Nowhitespace  \> Show Wildcards        \\
    Replace          \>  Set Find Whitespace    \> Wildcard Find         \\
    Spell            \>                         \>                       \\
 \\
 CURSOR MOVEMENT AND SCROLLING \\
cccccccccccccccccccc \= cccccccccccccccccc \= ccccccccccccccccccccc\= \kill
 \\
    Bottom          \> Mark         \> Move Right      \> Set Cursor Free    \\
    Change Direction\> Move By Line \> Move Up         \> Set Scroll Margins \\
    End Of Line     \> Move By Page \> Next Screen     \> Start Of Line      \\
    Forward         \> Move By Word \> Previous Screen \> Top                \\
    Go To           \> Move Down    \> Reverse         \> What Line          \\
    Line            \> Move Left    \> Set Cursor Bound\>                    \\
 \\
 GENERAL-PURPOSE COMMANDS \\
 \\
cccccccccccc\=cccccccccccc\=cccccccccccc\=cccccccccccccc\=ccccccccccccc\=ccccccccccccc\= \kill
    Attach \>  Do   \> Help  \>  Recall  \>   Reset   \>  Show        \\
    DCL    \>  Exit \> Quit  \>  Repeat  \>   Return  \>  Spawn       \\
\end{tabbing}

\begin{tabbing}
ccccccccccccccccccccccc \= cccccccccccccccccccccccc \= cccccccccccccccccc \= \kill
 FILES AND BUFFERS \\
 \\
    Buffer         \>  Open Selected      \>  Set Journaling           \\
    Delete Buffer  \>  Previous Buffer    \>  Set Journaling All       \\
    Get File       \>  Recover Buffer     \>  Set Nojournaling         \\
    Include File   \>  Recover Buffer All \>  Set Nojournaling All     \\
    New            \>  Save File          \>  Show Buffers             \\
    Next Buffer    \>  Save File As       \>  Show System Buffers      \\
    Open           \>  Set Buffer         \>  Write File               \\
 \\
 WINDOWS AND DISPLAY\\
 \\
    Delete Window  \>  One Window      \>  Set Width   \>   Shrink Window  \\
    Enlarge Window \>  Previous Window \>  Shift Left  \>   Split Window   \\
    Next Window    \>  Refresh         \>  Shift Right \>   Two Windows    \\
 \\
 FORMATTING AND CASE CHANGES \\
 \\
    Capitalize Word  \>   Insert Page Break \>   Set Paragraph Indent      \\
    Center Line      \>   Lowercase Word    \>   Set Right Margin          \\
    Convert Tabs     \>   Paginate          \>   Set Tabs                  \\
    Fill             \>   Set Left Margin   \>   Set Wrap                  \\
    Fill Paragraph   \>   Set Nowrap        \>   Tab                       \\
    Fill Range       \>                     \>   Uppercase Word            \\
 \\
cccccccccccccccccccccccccccccccccc \= cccccccccccccccccccccccccc \= ccccccccccccccccccccc \= \kill
 KEY DEFINITIONS \\
 \\
    Define Key                \> Set Gold Key       \> Set Keypad VT100    \\
    Learn                     \> Set Keypad EDT     \> Set Keypad WPS      \\
    Remember                  \> Set Keypad NoEDT   \> Set Nogold Key      \\
    Set Func Key DECwindows   \> Set Keypad NoWPS   \> Show Key            \\
    Set Func Key NoDECwindows \> Set Keypad Numeric \> Undefine Key        \\
 \\
 CUSTOMIZING \\
ccccccccccccccccccccccccccccccccccc \= ccccc \= ccccccccccccccccccccccccccccccc \= \kill
    @                         \> \>   Set Noclipboard                     \\
    Define Menu Entry         \> \>   Set Nodefault Command File          \\
    Extend All                \> \>   Set Nodefault Section File          \\
    Extend EVE                \> \>   Set Noexit Attribute Check          \\
    Extend This               \> \>   Set Nopending Delete                \\
    Save Attributes           \> \>   Set Nosection File Prompting        \\
    Save Extended EVE         \> \>   Set Pending Delete                  \\
    Save System Attributes    \> \>   Set Section File Prompting          \\
    Set Clipboard             \> \>   Show Defaults Buffer                \\
    Set Default Command File  \> \>   Show Summary                        \\
    Set Default Section File  \> \>   TPU                                 \\
    Set Exit Attribute Check  \> \>   Undefine Menu Entry                 \\
 \\
 INFORMATIONAL TOPICS \\
ccccccccccccccccccccccccccccc \= ccccccccccccccccccccccccc \= ccccccccccccccccccccc \= \kill
 \\
    Abbreviating         \>  Gold Keys            \> Pending Delete        \\
    About                \>  Initialization Files \> Position Cursor       \\
    Attributes           \>  Journal Files        \> Prompts And Responses \\
    Canceling Commands   \>  Keypad (diagram)     \> Quick Copy            \\
    Choices Buffer       \>  Keys (list)          \> Ranges And Boxes      \\
    Command Files        \>  List Of Topics       \> Ruler Keys            \\
    Control Keys         \>  Mail Editing         \> Scroll Bars           \\
    DECwindows Differences\> Menus                \> Section Files         \\
    Defaults             \>  Message Buffer       \> Status Line           \\
    Dialog Boxes         \>  Mouse                \> Typing Keys           \\
    Editing Command Lines\>  Names For Keys       \> Windows               \\
    EDT Conversion       \>  New Features         \> WPS Differences       \\
    EDT Differences      \>  New User             \>                       \\
\end{tabbing}

\end{document}
