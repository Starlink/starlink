\documentstyle[11pt]{article}
\pagestyle{myheadings}

%------------------------------------------------------------------------------
\newcommand{\stardoccategory}  {Starlink User Note}
\newcommand{\stardocinitials}  {SUN}
\newcommand{\stardocnumber}    {100.10}
\newcommand{\stardocauthors}   {P.\,T.\,Wallace}
\newcommand{\stardocdate}      {9th December 1994}
\newcommand{\stardoctitle}     {TPOINT --- Telescope Pointing Analysis
                                System (v4.4)}
%------------------------------------------------------------------------------

\newcommand{\stardocname}{\stardocinitials /\stardocnumber}
\renewcommand{\_}{{\tt\char'137}}     % re-centres the underscore
\markright{\stardocname}
\setlength{\textwidth}{160mm}
\setlength{\textheight}{230mm}
\setlength{\topmargin}{-5mm} % European A4           } comment out as
%\setlength{\topmargin}{-18mm} % American quarto      } appropriate
\setlength{\oddsidemargin}{0mm}
\setlength{\evensidemargin}{0mm}
\setlength{\parindent}{0mm}
\setlength{\parskip}{\medskipamount}
\setlength{\unitlength}{1mm}

\begin{document}
\thispagestyle{empty}
DRAL / {\sc Rutherford Appleton Laboratory} \hfill {\bf \stardocname}\\
{\large Particle Physics \& Astronomy Research Council}\\
{\large Starlink Project}\\
{\large \stardoccategory\ \stardocnumber}
\begin{flushright}
\stardocauthors\\
\stardocdate
\end{flushright}
\vspace{-4mm}
\rule{\textwidth}{0.5mm}
\vspace{3mm}
\begin{center}
{\Large\bf \stardoctitle}
\end{center}
\vspace{2mm}

\setlength{\parskip}{0mm}
\tableofcontents
\setlength{\parskip}{\medskipamount}

\markright{\stardocname}

%------------------------------------------------------------------------------

\newenvironment{tabs}{\goodbreak\begin{tabbing}}{\end{tabbing}}
\newenvironment{cmnds}{\begin{tabs}
XXX \= XX \= XXXXXXXXXXXXXXXXXXXX \= \kill}{\end{tabs}}
\newenvironment{cmd}{\begin{tabbing}
XXXXXX \= XX \= XXXXXXXXXXXX \= \kill}{\end{tabbing}}

\newcommand{\fstring}[1]{\hbox{\hspace{0.05em}$'${\tt#1}\hspace{0.05em}$'$}}

\renewcommand{\_}{{\tt\char'137}}

\newpage

\section{INTRODUCTION}
TPOINT is an interactive telescope pointing analysis system.  It
allows data from pointing tests to be input and fitted to various
models.  The residuals from the fits can be displayed in a
variety of graphical formats.  If systematic errors are visible,
the pointing model can be adjusted by adding and removing terms.

The system is available for PC/MS-DOS, VAX/VMS, DECstation/Ultrix
and Sun SPARCstation/SunOS.  The present document concentrates on the VAX
and Unix versions.  For information on running the PC version please
refer to the file READ.ME.

TPOINT can deal with data from either equatorial
or altazimuth mounts, and could easily be extended to alt/alt and
other designs.  It has been applied to many telescopes, including
optical, IR, mm and radio, and is in routine use at the
AAO, UKST, UKIRT, LPO and several other observatories.

Many observatories have their own programs for fitting models
to pointing data, and in a numerical sense TPOINT cannot be
expected to produce
results that are different or better.  However, TPOINT is unusual
(a) in that the telescope model can be changed during the interactive
session, and (b) in the variety of its graphical displays.
These two features allow rapid exploration of the pointing possibilities
of any given telescope and may lead to a better model than the one
previously used; this has already happened in several cases.

The general approach taken by TPOINT is
that as far as possible the telescope model should
describe real effects (geometrical misalignments,
well understood flexures, {\it etc.}), and empirical functions
should be used only to mop up any remaining systematic
errors.  There is a school of thought which advocates
using empirical functions (for example polynomials or
spherical harmonics) for the whole job.  However, the
TPOINT approach has some advantages:
\begin{itemize}
\item Simple geometrical misalignments -- for example a miscentred
instrument on a mount which is located at a Nasmyth focus but is
not coincident with the elevation axis -- might require very
complicated empirical functions but are simple to deal with
analytically.
\item Direct manipulation of certain geometrical
terms while
the telescope is in operation can be very useful.  An important
application of this technique
is where a star is switched from one instrument
aperture to another simply by changing the collimation
parameters.  Another example is
where the
polar axis of a wide field telescope is routinely raised and
lowered as a function of declination,
to minimize the field rotation effects of differential
refraction; the polar axis elevation parameter in the pointing
model can simply be changed by the same amount
and accurate pointing is maintained.
\item A realistic model of a telescope frequently exposes
mechanical deficiencies which can then be diagnosed and cured.
\item A realistic model is likely to require fewer terms, and
the number of stars observed in a pointing test can be
correspondingly smaller.
\item Physically-based models are less likely to
misbehave when extrapolating outside the area covered by
the available test data.
\end{itemize}
In any case, the models available with TPOINT allow
considerable empiricism to be employed when called for.  A good example
of this is the AAT model.  Though firmly based on well-understood
geometrical misalignments and plausible tube flexure corrections,
there are also complicated harmonics describing
what are believed to be flexures in the horseshoe and centre
section.  These flexure models are encapsulated in two
terms called HFX and HFD, which are always left at their
nominal sizes and are not fitted when individual
pointing tests are reduced.  HFX and HFD were determined by
empirical fits to a very large sample (well over 1000 stars)
formed by superimposing the residuals from tests carried out
over several years.

\subsection{System Overview}
TPOINT is a monolithic program which accepts simple commands
from a terminal.  There is a rudimentary `command procedure'
capability, and a log file is produced.  Pointing test data
are input from ordinary ASCII files (as produced by a text
editor) in flexible formats.  On the VAX and the Unix platforms,
graphical output is via the SGS/GKS package.  On the PC a simplified
SGS-like interface to the Microsoft graphics library is used.
A small star catalogue is provided, and some sample data.  There
is a HELP system.

On all platforms, the command which starts the TPOINT system is this:
\begin{cmnds}
\> \> {\tt tpoint} {\it logfile} {\it proclib} {\it starcat}
\end{cmnds}
The three parameters -- the log file, the initial
procedure library, and the initial star catalogue --
are all optional and are normally allowed to default,
respectively, to the following files:
\begin{tabbing}
xxx \= xxxxxxxxxxx \= xxxxxxxxxxxxxxxx \= xxxxxxxxxxxxxxxxxxxx \= \kill
\> \> PC \> VAX \> Unix \\ \\
\> {\it logfile} \>
\verb|NUL| \>
\verb|LOG.LIS| \>
\verb|tpoint.log| \\
\> {\it proclib} \>
\verb|PROCS.DAT| \>
\verb|TPDIR:PROCS.DAT| \>
\verb|/star/etc/tpoint/procs| \\
\> {\it starcat} \>
\verb|STARS.DAT| \>
\verb|TPDIR:STARS.DAT| \>
\verb|/star/etc/tpoint.stars|
\end{tabbing}
On termination (via the END command) the specified log
file is PURGEd but not printed;  this is left to the user.  The
log contains a full record of the session, and includes extra
information ({\it e.g.}\ correlations between pointing terms) too
extensive to be displayed on the terminal.

Each TPOINT command line consists of one or more fields
separated by spaces.  The first field, of up to 6 characters,
is the {\bf command name}, and specifies the action to be performed.
The subsequent fields, if any, are the {\bf arguments}.
Where the arguments are numeric, free-format decoding is
performed, and numbers may be entered in a wide variety of
formats.
The precise way the arguments are interpreted depends on the
command in question.

TPOINT deals with all forms of input -- both commands and the
various sorts of data file -- in a standardized way which
allows for comments and blank lines to be used and provides
consistent handling of lowercase information.
Details of these conventions are given in the section
{\it Syntax}, later.

To abort a command, type CTRL/C.  The command will terminate
cleanly as soon as it can, and control will be returned to the
terminal ready for the next command.
To exit normally from TPOINT, use the END command.

A full list of all TPOINT commands, in alphabetical order,
is given later.  A quick reference list can be found at the
end.

Sequences of TPOINT commands can be executed from a library.  A
procedure library file is a sequential file of up to 80-character
records, searched from the start each time.  Standard TPOINT
syntax applies to such libraries, which may thus contain
blank lines and comments (beginning with \fstring{!}) to
enhance readability.  In using procedures
from a library, the following commands are involved:
\begin{tabs}
XXXXXX \= XXXXXXXXXXXXX \= \kill
\> {\tt CALL} {\it x} \> calls procedure {\it x}\, from the library \\
\> {\tt PROC} {\it x} \> specifies the start of procedure {\it x}\\
\> {\tt RETURN} \> returns from and ends a procedure
\end{tabs}
Of these commands,
CALL alone is permissible from the terminal; PROC and RETURN are
used only in library procedures.
The standard procedure library file is automatically loaded when
TPOINT is started unless a different one is specified on the
command line as described above.
After loading the library, TPOINT automatically executes a
\hbox{`CALL INIT'};
the INIT procedure supplied in the standard library
does nothing, but one in a private version can contain any required
preliminary operations -- for instance preparations for plotting on a
favourite graphics device.
All procedure libraries should contain an INIT procedure, otherwise an
error message will occur during TPOINT setup.
A private library can be loaded at any time during the TPOINT
session by using the INPRO command.
The INIT procedure is not called under these circumstances.
An INPRO command without arguments restores the original library.
By default, procedure commands are not echoed on the terminal;
they can be made to appear by means of the ECHO~ON command.

\subsection{Getting Started}
The best way to learn how TPOINT works is to reduce some
real data, using one of the sample files.  Proceed as follows.
First start the system:
\begin{cmnds}
\> \> {\tt tpoint}
\end{cmnds}
There will be various announcements, followed by a \fstring{*} prompt.
Except on the PC, select a suitable plotting device:
\begin{cmnds}
\> \> {\tt PLTON} {\it device}
\end{cmnds}
(Consult your system manager for local SGS device names.)
Read in the sample UKST data:
\begin{cmnds}
\> \> {\tt INDAT TPDIR:UKST} \> \hspace{10em}(VAX) \\
\> or \> {\tt INDAT UKST} \> \hspace{10em}(PC) \\
\> or \> {\tt INDAT /star/bin/examples/tpoint/ukst.dat} \> \hspace{10em}(Unix)
\end{cmnds}
The observations will be listed on the terminal.  Specify the
standard geometrical model for an equatorial mount, and then
fit the model to the data:
\begin{cmnds}
\> \> {\tt USE IH ID NP CH ME MA} \\ \\
\> \> {\tt FIT}
\end{cmnds}
The values of the coefficients are reported, and the RMS and
population standard deviation.  Plot the residuals:
\begin{cmnds}
\> \> {\tt CALL E9}
\end{cmnds}
The axis labelling will be described later.  In this particular case,
conspicuous systematic effects are present in the dD (declination
residual) versus H (hour angle) plot.
The term FO (fork flexure) is a standard one for fork mounted
telescopes and produces this characteristic
pattern of residuals.  Include the
term in the model and fit again:
\begin{cmnds}
\> \> {\tt USE FO} \\ \\
\> \> {\tt FIT}
\end{cmnds}
Note that the RMS and population standard deviation are much improved.
Plot the residuals again:
\begin{cmnds}
\> \> {\tt CALL E9}
\end{cmnds}
There are no obvious remaining systematic effects.

This illustrates the general strategy for modelling a telescope
from scratch.  Begin by suitably preparing the TPOINT system
and inputting the data.
Then create a preliminary model consisting of the basic set
of geometrical terms, and perform a fit:
\begin{cmnds}
\> \> {\tt INDAT} {\it file} \\ \\
\> \> {\tt USE IH ID NP CH MA ME} \> (for an equatorial) \\
\> or \> {\tt USE IA IE NPAE CA AW AN} \> (for an altazimuth) \\ \\
\> \> {\tt FIT}
\end{cmnds}
(Procedures for setting up the basic equatorial
and altaz models are also provided in the
standard library: use CALL~EQUAT or CALL~ALTAZ.)
The fit can be repeated until the solution settles down.
Make a note of the population
standard deviation, which is an indication of the general quality
of the telescope before any modelling of flexure {\it etc.} The
next step is to display the residuals graphically.  A useful
way to start is procedure E9 or A9 in the standard library:
\begin{cmnds}
\> \> {\tt PLTON} {\it device} \> to prepare the plotting device \\ \\
\> \> {\tt CALL E9} \> 9 favourite plots for an equatorial \\
\> or \> {\tt CALL A9} \> 9 favourite plots for an altazimuth
\end{cmnds}
(The standard library also includes procedures E6 and A6 for the six most
useful plots, to reduce plotting time on slow terminals.)

By means of the USE and FIT
commands, try out extra terms to reduce any systematic errors.  The
simple Hooke's-law tube flexure term TF is frequently required;
for fork mounted equatorials try the fork flexure term FO,
and for mounts with a declination axis supported at one end
only (the English cross-axis and the German mountings are
examples) try the declination axis flexure term DAF.
When adding new terms, pay attention to the standard deviation of
the new coefficient, the effect on other terms, and whether
the population standard deviation has been reduced.  Remove
new terms if they are reported as being indistinguishable
from existing ones, or if they lead to the appearance of
messages warning that the fit is ill-conditioned.  With the
exception of the six standard geometrical terms, it is wise
to reject any terms which are not at least 2~sigma in
size.
For each FIT, a table of the correlations between every
pair of terms is output to the LOG file, for inspection after the
TPOINT session;  values close to unity show terms which are
not easily distinguished given the current data.

More advanced information on modelling strategy is given in a later
section.

The current settings of various internal parameters can
be displayed as follows:
\begin{cmnds}
\> \> {\tt SHOW}
\end{cmnds}
The list produced by SHOW includes the command names required
to change the parameters, and is a useful guide to some of
the facilities available within TPOINT.

\subsection{Syntax}
All TPOINT commands, procedure library records, star
catalogue records, model data records, and pointing data
records, are subjected on input to a preliminary vetting and
conditioning, as follows:
\begin{itemize}
\item Comments -- records which are blank, or which
      have \fstring{!} as the first non-space character -- are
      logged and displayed if appropriate but are
      otherwise ignored.
\item Non-printing characters (TABs for instance) are
      replaced by single spaces.
\item Leading blanks are eliminated as appropriate.
\item Except within {\it string arguments}, lowercase
      characters are converted to uppercase.
\end{itemize}
{\it String arguments}\, are groups of characters delimited
by pairs of either \verb|"| or $'$ characters to show that
they must stay lowercase.
When
a procedure library file is being input, string argument
delimiters are left intact (to allow library commands to
have lowercase arguments).
For TPOINT commands, string argument
delimiters are interpreted by the command routine itself.  When
star catalogue files, pointing data files, and model data
files are input, string argument delimiters are removed.

\section{MODELLING}
The pointing model is a sequence of terms selected from an
internal repertoire.  As well as explicitly formulated terms
(geometrical effects for example) there is a generic type
covering a wide range of polynomials and harmonics.

Terms can be {\bf added} to the model by means of the USE command.
For example, to add the terms for polar axis misalignment:
\begin{cmnds}
\> \> {\tt USE ME MA}
\end{cmnds}
The USE command is also used to re-enable fitting after a
FIX command.

Terms can be {\bf removed} from the model by means of the LOSE
command.
For example, to drop the term PDH2:
\begin{cmnds}
\> \> {\tt LOSE PDH2}
\end{cmnds}

The value of the coefficient of a single term can be {\bf specified} by
means of the command:
\begin{cmnds}
\> \> {\it coeff val}
\end{cmnds}
where {\it coeff}\, is the name of the term and {\it val}\, the value in
arcseconds.  Thus to set the HA index error to $+20.7$~arcsec:
\begin{cmnds}
\> \> {\tt IH 20.7}
\end{cmnds}
You would normally only do this for `FIXed' coefficients -- ones
excluded from fitting.

The current value of the coefficient of a single term can
be {\bf inquired} by entering a command consisting simply of the
name of the term.
For example, to display the current value of the declination
index error, simply use the command:
\begin{cmnds}
\> \> {\tt ID}
\end{cmnds}

The current model is {\bf fitted} to the observations by means of
the FIT command.  Individual terms can be frozen at a
particular value by means of the FIX command, or reinstated
in the fitting by means of the USE command.

\subsection{The Pointing Terms}
The pointing terms fall into five categories -- equatorial,
altazimuth, special (to a telescope or type of telescope), polynomial
and harmonic.  The model for any particular telescope may be a
mixture of several of these types.

Some terms are functionally identical but have different names
for historical, convenience or efficiency reasons.  Most of them can
be expressed using generic terms (polynomials and harmonics), though
doing so will lead to slightly slower execution of the FIT and UNFIT
commands.
To find out what a given
term means in a geometrical sense, refer to the Fortran source for
the tpt\_PTERMS and tpt\_PTERML modules.

The following terms are most useful when modelling an {\bf equatorial}
telescope:
\begin{tabs}
XXXXXX \= XXXXXXXX \= \kill
\> IH   \> index error in HA \\
\> ID   \> index error in Dec \\
\> NP   \> nonperpendicularity of HA and Dec axes \\
\> CH   \> nonperpendicularity of Dec and pointing axes \\
\> ME   \> polar axis elevation error \\
\> MA   \> polar axis error east-west \\
\> FO   \> fork flexure \\
\> DAF  \> flexure of cantilevered Dec axis \\
\> HCES \> HA centring error (sin component) \\
\> HCEC \> HA centring error (cos component) \\
\> DCES \> Dec centring error (sin component) \\
\> DCEC \> Dec centring error (cos component) \\
\> DNP  \> dynamic nonperpendicularity \\
\> X2HS \> sin($2h$) term EW \\
\> X2HC \> cos($2h$) term EW \\
\> D2HS \> sin($2h$) effect in Dec \\
\> D2HC \> cos($2h$) effect in Dec \\
\> D2DS \> sin($2\delta$) term in Dec \\
\> D2DC \> cos($2\delta$) term in Dec \\
\> D4HS \> sin($4h$) effect in Dec \\
\> D4HC \> cos($4h$) effect in Dec \\
\> EHS  \> sin($h$) effect in polar axis elevation \\
\> EHC  \> cos($h$) effect in polar axis elevation \\
\> AUX1H \> HA change supplied through auxiliary reading 1 \\
\> AUX1X \> EW change supplied through auxiliary reading 1 \\
\> AUX1D \> Dec change supplied through auxiliary reading 1
\end{tabs}
The following terms apply to {\bf altazimuth} telescopes;
some also describe effects
in zenith distance which apply equally well to equatorial
telescopes:
\begin{tabs}
XXXXXX \= XXXXXXXX \= \kill
\> TF   \> tube flexure -- sin $\zeta$ law \\
\> TX   \> tube flexure -- tan $\zeta$ law \\
\> IE   \> index error in elevation \\
\> IA   \> index error in azimuth \\
\> CA   \> nonperpendicularity of elevation and pointing axes \\
\> AN   \> NS misalignment of azimuth axis \\
\> AW   \> EW misalignment of azimuth axis \\
\> NPAE \> nonperpendicularity of azimuth and elevation axes \\
\> ACES \> Az centring error (sin component) \\
\> ACEC \> Az centring error (cos component) \\
\> ECES \> El centring error (sin component) \\
\> ECEC \> El centring error (cos component) \\
\> NRX  \> Horizontal displacement of Nasmyth rotator \\
\> NRY  \> Vertical displacement of Nasmyth rotator \\
\> AUX1A \> Az change supplied through auxiliary reading 1 \\
\> AUX1S \> LR change supplied through auxiliary reading 1 \\
\> AUX1E \> El change supplied through auxiliary reading 1
\end{tabs}
The following terms are {\bf special} to the AAT:
\begin{tabs}
XXXXXX \= XXXXXXXX \= \kill
\> ZH   \> AAT HA Z-gear effect in HA \\
\> ZE   \> AAT HA Z-gear effect in polar axis elevation \\
\> HF   \> AAT main east-west horseshoe flexure \\
\> HGES \> $36^{m}$ gear error in HA -- sin \\
\> HGEC \> $36^{m}$ gear error in HA -- cos \\
\> DGES \> $9^{\circ}$ gear error in Dec -- sin \\
\> DGEC \> $9^{\circ}$ gear error in Dec -- cos \\
\> TFP  \> AAT tube flexure -- non-Hooke's-law term \\
\> HFX  \> AAT residual horseshoe east-west \\
\> HFD  \> AAT residual horseshoe north-south \\
\> CD4A \> AAT coud\a'e 4 collimation error A component \\
\> CD4B \> AAT coud\a'e 4 collimation error B component \\
\> CD5A \> AAT coud\a'e 5 collimation error A component \\
\> CD5B \> AAT coud\a'e 5 collimation error B component
\end{tabs}
{\bf Polynomial} terms have names of the form:
\begin{tabs}
XXXXXX \= \kill
\> P{\it rc}[{\it i}[{\it c}[{\it i}]]]
\end{tabs}
One example would be a term PXH2D1, which would model an effect
which produced an east-west shift on the sky (X) proportional
to $h^2\delta$ (H2D1).

The initial \fstring{P} identifies this term as a polynomial.

The {\it r}\, field describes the result, and is one of the following:
\begin{tabs}
XXXXXX \= XXXX \= \kill
\> H \> result is in hour angle \\
\> X \> result is east-west on the sky \\
\> D \> result is in declination \\
\> U \> result tilts polar axis up/down \\
\> L \> result tilts polar axis left/right \\
\> P \> result changes HA/Dec nonperpendicularity \\
\> A \> result is in azimuth \\
\> S \> result is left-right on the sky \\
\> Z \> result is in zenith distance \\
\> N \> result tilts azimuth axis north/south \\
\> W \> result tilts azimuth axis east/west \\
\> V \> result changes Az/El nonperpendicularity
\end{tabs}
The two {\it ci}\, fields indicate the independent variables and
their powers.  Each {\it i}\, is in the range 0--9;
each {\it c}\, can be any of:
\begin{tabs}
XXXXXX \= XXXX \= \kill
\> H \> hour angle \\
\> D \> declination \\
\> A \> azimuth \\
\> Z \> zenith distance \\
\> Q \> parallactic angle
\end{tabs}
Trailing {\it ci}\, or {\it i}\, fields, if omitted, default
to unity.  For example, a change in HA proportional to HA
({\it i.e.}\ a scale change, such as might be necessary
with a roller-driven encoder) can be produced by using
the term PHH.

The same independent variable can be specified twice, allowing
powers up to 18 to be used.

{\bf Harmonic} terms have names of the forms:
\begin{tabs}
XXXXXX \= \kill
\> H{\it rfc}[{\it i}][{\it fc}[{\it i}]] \\
or \> H{\it rfcii}
\end{tabs}
One example would be a term HXSH2, which would model an effect
which produced an east-west shift on the sky (X) proportional
to sine (S) of $h^2$ (H2).

The initial \fstring{H} identifies this term as a harmonic.

The {\it r}\, and {\it c}\, fields describe the result and an independent
variable respectively, and are as for polynomials, above.

Each {\it fci}, {\it fcii}\, and {\it fciii}\, field indicates a function
of an integer multiple of an independent variable {\it c}.
The {\it f}\, field is either S for sine or C for cosine.
Each {\it i}\, is in the range 0--9,
allowing frequencies from zero to 999 cycles per revolution in the
case of simple one-coordinate harmonics, and 0 to 9 cycles per revolution
in the case of compound two-coordinate harmonics.
An omitted {\it i}\, defaults to unity, as does an
omitted trailing {\it fci}.

\subsection{Order of Terms}
To match different styles of telescope control system,
the model can either start with raw telescope positions and
predict the corresponding true\footnote{Ones corrected for everything
up to and including diurnal aberration and refraction -- the
formal term is {\it observed}\, but the word {\it true}\, has been
adopted here as marginally less
likely to cause confusion.} position, or alternatively can
start with true star positions and predict the required
raw telescope position.  The selection is made with the
ADJUST command.  The sign convention for the coefficients is
the same for both options, so that the two options should
deliver coefficients that are substantially the same.  When
the ADJUST command is used to change the option, the order
of the terms in the model is reversed so that more or less
the same coefficient values will be produced if a FIT is
carried out (following a RESET to cancel the previous set
of corrections).

Normally, each term in the model is {\it chained}\, to the previous
one, the input position for each term being the output position
from the previous term.  Sometimes this may not be appropriate,
and a set of terms can instead share one input position and
their adjustments be added in as a group;  in this case the
terms are said to be {\it parallel}.  The commands CHAIN and PARAL
select which option applies to a given term, the first in a
parallel group being marked `chained' and the rest marked
`parallel' to indicate they share the former's starting
point.

In typical cases the coefficients will not be much altered by changing
the order of terms, or switching terms between chained and
parallel.  Significant changes are likely only if some
coefficients are large, or if high order polynomials or
harmonics are employed.  It is nevertheless reassuring
if the model being fitted matches the way the
pointing corrections are implemented in the telescope control
system itself, in direction, order, formulation, and chaining.

\subsection{Sizes of Coefficients}
The mathematical expressions used by TPOINT are rather simple and
designed to give results of adequate accuracy when not too close
to the pole of the mounting and {\it when the
pointing errors of the telescope are small}.  Flexures and misalignments
of up to a few arcminutes are unlikely to give problems;  they will not
interact significantly with one another, and will be calculated by
TPOINT to sufficient accuracy.  However, terms with larger
coefficients may not be so well-behaved, especially near a pole.

A particular source of problems in practice is collimation error, the
non-perpendicularity of the chosen pointing axis with respect to
the declination or elevation axis of the mounting.  If the measuring
device -- eyepiece, autoguider probe, CCD, TV camera {\it etc.}\ --
is offset too far, there may be unmodelled pointing errors due to
such effects as optical distortion.  The best
plan is to ensure that the measuring device is offset no more than a few
minutes of arc.  If this is impossible it is advisable to pre-process
pointing observations to remove the bulk of the offset before the
TPOINT run.  This might involve rigorous calculation of the
tangent-plane projection and optical distortions.

\subsection{Saving Models}
Commands are provided for writing the current model to a file, and
for replacing the current model with one read from such a file:
\begin{cmnds}
\> \> OUTMOD {\it file} \> write model to a file \\
\> \> INMOD {\it file} \> read model from a file
\end{cmnds}
where {\it file}\, is the name of the
file containing the model.  As well as providing long-term
storage for models, the file may also be read by programs
external to the TPOINT system.  In addition to the
model information required by INMOD, the
file contains the current caption and
the basic statistics which relate the model to the current
data list.  The INMOD command ignores these latter
items.

The file consists of variable length formatted character
records in `list' format ({\it i.e.}\ the records do not contain
printer vertical format characters).  The order of records
within the file is:

\begin{tabs}
XXXXXX \= \kill
\> caption record \\
\> method and statistics record \\
\> term records (one per term in the model) \\
\> end record
\end{tabs}

\goodbreak
Here is an example file:

\begin{tabs}
XXXXXX \= \kill
\> \verb*|AAT  f/15  1979/06/11| \\
\> \verb*|T   49   1.1639| \\
\> \verb*|  IH       +175.2234| \\
\> \verb*| =ZH         +3.5100| \\
\> \verb*|  ID        +20.7048| \\
\> \verb*|&=HFX        +1.0000| \\
\> \verb*|&=HFD        +1.0000| \\
\> \verb*|& HF        -22.4514| \\
\> \verb*|& X2HC       -3.0225| \\
\> \verb*|& NP         +2.9841| \\
\> \verb*|& CH        -19.1780| \\
\> \verb*| =ZE         +0.7000| \\
\> \verb*|& ME        +57.9758| \\
\> \verb*|& MA         +3.2308| \\
\> \verb*|  TF         +8.0261| \\
\> \verb*|& TFP        -2.1594| \\
\> \verb*|END| \\
\end{tabs}

The {\bf caption} record can be up to 80 characters long.

The {\bf method and statistics} record can be read with a format
specification of (A1,I5,F9.4).  The three fields are:

\begin{tabs}
XXXXXX \= \kill
\> method:  T or S \\
\> number of active observations \\
\> sky RMS (arcsec)
\end{tabs}

A method of \fstring{T} means that the model corrects the telescope
readings, whereas \fstring{S} means that the model is applied
to star positions to predict the required telescope setting.

Each {\bf term} record can be read with a format specification of
(2A1,A8,F10.4).  The fields are:

\begin{tabs}
XXXXXX \= \kill
\> chained/parallel flag \\
\> fixed/floating flag \\
\> term name \\
\> coefficient value
\end{tabs}

The chained/parallel flag is either a space (for `chained'
terms - ones which are computed from the position as
affected by all previous terms) or an ampersand (for
`parallel' terms - members of a group all computed
from the same starting position).  The fixed/floating
flag is either a space (for coefficients that are to be
fitted) or an equals sign (for coefficients of fixed
value).  The term name and coefficient value have their
normal meanings.

The {\bf end} record consists of the string \fstring{END}.  Reading programs
should also interpret end-of-file as equivalent to an end
record.

Note:  population standard deviation can be obtained as
follows.  Count the number of floating coefficients (ones
not marked \fstring{=}) and call it {\it n}.  With sky RMS $\rho$
and number of active observations $o$,
PSD $\sigma=\sqrt{\rho^{2}o/(o-n)}$.

\section{GRAPHICS}
The TPOINT graphics commands
are as follows:
\begin{cmnds}
\> \> CAPT   \> enable/disable plotting of caption \\
\> \> CLRNG  \> select zone clearing option \\
\> \> FRAME  \> enable/disable plotting of graph frame \\
\> \> G      \> plot residuals in one coordinate against another \\
\> \> GAM    \> look for changing polar/azimuth axis misalignment \\
\> \> GCLR   \> erase the whole display surface \\
\> \> GDIST  \> plot error distributions \\
\> \> GHYST  \> look for hysteresis \\
\> \> GMAP   \> plot error vectors on Cartesian Cylindrical projection \\
\> \> GSCAT  \> plot residuals as a scatter diagram \\
\> \> GSMAP  \> plot error vectors on orthographic projection \\
\> \> MARKH  \> specify marker size \\
\> \> PENS   \> specify pens for lines and text \\
\> \> PLTOFF \> switch back to previous graphics workstation \\
\> \> PLTON  \> switch to a new graphics workstation \\
\> \> PLTZ   \> specify plotting zone \\
\> \> TXF    \> specify text font \\
\> \> TXP    \> specify text precision
\end{cmnds}

A PLTON command is an essential prelude to plotting on the VAX
and Unix platforms but has no action on the PC.
The various graph plotting commands are
then available -- G, GAM, GDIST, GHYST, GMAP, GSCAT and GSMAP.
PLTOFF will be needed if switching back and forth between
multiple graphics devices is to be performed (not applicable to the PC).
PLTZ will be useful for displaying several
graphs at one, or keeping graph and text separate.

The remaining commands -- CAPT, CLRNG, FRAME, GCLR, MARKH,
PENS, TXF, TXP -- are used either for reducing plotting
time, or for achieving special effects for demonstration
and publication purposes.  They can be ignored by novice users.

TPOINT uses SGS/GKS for all its plotting, and any graphics device
supported by the local GKS can be selected by means of the PLTON command.
On the PC, all plotting goes to the screen using the best available graphics
mode, and there is no need to use the PLTON command.
On the other platforms, before any plotting can take place a PLTON command
must be executed, specifying the SGS name of the device to be used.
A PLTON command with no argument will produce a list of the devices
known to the site's SGS/GNS/GKS system.
If the appropriate name tables have not been set up (see SUN/57) the list
will be empty, and it will be necessary to specify the GKS
{\it workstation type}\, directly.
Further PLTON commands will switch the plotting successively to different
devices, while PLTOFF commands will revert to the previous device.
A common pattern is for the initial PLTON to specify a screen or window;
whenever a hardcopy version of a plot is subsequently required a further
PLTON is issued specifying the hardcopy device;
a PLTOFF then reverts to the screen or window.

Multiple plotting zones on the one display surface are
supported, which allows several plots to be displayed
at once.  The current plotting zone is specified
by the PLTZ command:
\begin{cmnds}
\> \> PLTZ \> use full display surface \\
\> or \> PLTZ name \> use named region \\
\> or \> PLTZ x1 x2 y1 y2 \> use numerically specified region
\end{cmnds}
The predefined zone names specify regular subdivisions
of the display surface:

\hspace{5em}
\begin{center}
\thicklines
\begin{picture}(107,81)

\put(0,0){\line(1,0){40}}
\put(0,30){\line(1,0){40}}
\put(0,51){\line(1,0){40}}
\put(0,81){\line(1,0){40}}

\put(67,0){\line(1,0){40}}
\put(67,30){\line(1,0){40}}
\put(67,51){\line(1,0){40}}
\put(67,81){\line(1,0){40}}

\put(0,0){\line(0,1){30}}
\put(40,0){\line(0,1){30}}
\put(67,0){\line(0,1){30}}
\put(107,0){\line(0,1){30}}

\put(0,51){\line(0,1){30}}
\put(40,51){\line(0,1){30}}
\put(67,51){\line(0,1){30}}
\put(107,51){\line(0,1){30}}

\put(0,66){\line(1,0){40}}
\put(0,66){\makebox(40,15){T}}
\put(0,51){\makebox(40,15){B}}

\put(87,51){\line(0,1){30}}
\put(67,51){\makebox(20,30){L}}
\put(87,51){\makebox(20,30){R}}

\put(0,15){\line(1,0){40}}
\put(20,0){\line(0,1){30}}
\put(0,15){\makebox(20,15){TLQ}}
\put(20,15){\makebox(20,15){TRQ}}
\put(0,0){\makebox(20,15){BLQ}}
\put(20,0){\makebox(20,15){BRQ}}

\put(67,10){\line(1,0){40}}
\put(67,20){\line(1,0){40}}
\put(80.33,0){\line(0,1){30}}
\put(93.67,0){\line(0,1){30}}
\put(67,20){\makebox(13.33,10){TL}}
\put(80.33,20){\makebox(13.33,10){TC}}
\put(93.67,20){\makebox(13.33,10){TR}}
\put(67,10){\makebox(13.33,10){CL}}
\put(80.33,10){\makebox(13.33,10){CC}}
\put(93.67,10){\makebox(13.33,10){CR}}
\put(67,0){\makebox(13.33,10){BL}}
\put(80.33,0){\makebox(13.33,10){BC}}
\put(93.67,0){\makebox(13.33,10){BR}}

\end{picture}
\end{center}
\vspace{2ex}

If numeric parameters are supplied they describe directly the
region of the display surface which is to contain the plotting
zone.  The numbers are the two X extremes and the two Y extremes
respectively, in units where the corresponding dimension of the
display surface is unity.  All four numbers are required.
Irrespective of the shape of the zone and how it was
specified,
plotting will actually occur in
a zone of the standard aspect ratio centred within this area.

The CLRNG command controls what zone or display
surface clearing occurs before each plot is drawn.  In conjunction
with the GCLR command, CLRNG can be used to speed up
plotting in cases where multiple zones are in use and
clearing of individual zones is a slow operation
on the graphics device concerned.
On VT100-compatible terminals, the text scroll region can
be specified by means of the VT command.  On VT100-compatible terminals with
a separate graphics plane, the VT and PLTZ commands can
be used in combination to confine the text and plots to separate
areas of the screen.

\subsection{Using the G...\ Commands}
The commands in TPOINT which plot graphs all have names
beginning with G.  The most important one for
deciding what terms to add to the model is called simply {\bf G},
and plots one
component of the residuals against one coordinate.  It has the
following syntax:
\begin{cmnds}
\> \> {\tt G} {\it ydata xdata} [{\it scale}]
\end{cmnds}
The argument {\it ydata}\, specifies which component of the
residuals is to be plotted:
\begin{tabs}
XXXXXX \= XXXX \= \kill
\> H \> in HA \\
\> X \> EW on the sky \\
\> D \> in Dec \\
\> P \> corresponding to HA/Dec nonperpendicularity \\
\> A \> in azimuth \\
\> S \> horizontally on the sky \\
\> Z \> in zenith distance \\
\> V \> corresponding to Az/El nonperpendicularity \\
\> R \> total residual
\end{tabs}
The argument {\it xdata}\, specifies what coordinate to plot against:
\begin{tabs}
XXXXXX \= XXXX \= \kill
\> H \> hour angle \\
\> D \> declination \\
\> A \> azimuth \\
\> Z \> zenith distance \\
\> Q \> parallactic angle \\
\> N \> observation order
\end{tabs}
The optional argument {\it scale}\, indicates the vertical plotting
scale: it is the absolute value of the largest residual to be
plotted.  In default, a scale is chosen automatically to suit
the data.

The general strategy is to plot different components of the residuals
against the various coordinates.  Terms can then be
added to the model (USE followed by FIT) to target
any systematic effects that are seen.  Some plots are more likely
than others to have a mechanical interpretation, and attention
should be confined to these, at least to start with.  For example,
an apparent relationship between polar axis elevation and azimuth,
on the evidence of a G~U~A plot, should be treated with great
scepticism (especially on an altazimuth mount), whereas systematic
residuals on G~N~A and G~W~A plots (for an altazimuth) or on
G~U~H and G~L~H plots (for an equatorial) are much more credible.

Because files of pointing observations are normally written
in time order, G~{\it ydata}\,~N plots are useful for exposing shifts
or drifts that have
happened during the pointing run.  Other x-axis meanings can
be contrived by appropriately sorting the observation file prior
to input.

The other plots have a variety of presentational and diagnostic roles.
{\bf GSCAT} plots residuals as a scatter diagram, like a shooting
target;  as well as being a good way of presenting the overall pointing
performance, it exposes errant observations and abnormal
distributions.
{\bf GSMAP} displays the residuals as error vectors on an
orthographic projection (a distant view of the celestial sphere
looking down on the zenith from above).  The vectors (lines
projecting from the square symbols which mark the stars)
are the great-circle continuations of the pointing
residuals.
{\bf GMAP} plots error vectors on a Cartesian Cylindrical projection,
either HA/Dec or Az/El.
{\bf GDIST} plots histograms showing various sorts of
error distribution, and is useful for testing whether
the residuals are reasonably normal.
{\bf GAM} looks for changing polar/azimuth axis misalignment.
Each residual is interpreted as being due to a misalignment
of the specified ``roll'' axis -- either polar or azimuth
as specified.  A histogram is plotted to show if the
residuals do, in fact, favour one particular direction.
The mean direction is calculated and reported, and then the
component of each residual in the direction of this misalignment
is plotted.
{\bf GHYST} looks for hysteresis, by drawing the error vector
at a position which indicates the direction from which the
telescope has come, assuming that the original data file contains
all the observations and in the correct order, and that any
hysteresis is related to the direction of the large scale
movements in cylindrical coordinates (either HA/Dec or
Az/El).  A numerical estimate of the hysteresis is made, by
summing the residuals oriented relative to the telescope
movements.

Many of the plotting commands have arguments, for
specifying scales and selecting HA/Dec and Az/El
options {\it etc.}  Full details are given in the section
{\it Commands}, later.

Finally, the command {\bf GCLR} erases the whole display surface.

\section{POINTING DATA}
At present, only one command is provided within TPOINT for reading
a set of pointing observations: INDAT.  Several different
formats are supported; all are converted into a standard
internal form, called the {\bf data list}.  The command is:
\begin{cmnds}
\> \> {\tt INDAT} {\it file}
\end{cmnds}
where {\it file}\, is the name of the
file of pointing observations.

The new observations may either replace or be appended to any
previous ones.  This is controlled by a flag armed by the
APPEND command.  The flag is reset
once the new data have
been appended.  In the append case the caption is reset and
the one supplied in the data file is ignored.  The new latitude
replaces the previous one;  it would, however, be very unusual
to combine the data from several telescopes.

The MASK and UNMASK commands allow specified subsets of
the observations to be suppressed temporarily.  They are useful
for verifying the independence of the solution in these different
subsets.  The most common application is where observations from only
a specified region of the sky are included.  Another use of
MASK and UNMASK is
to suppress observations whose total residual, on the sky, is
greater than some specified amount;
however, this is a dubious practice
unless there are {\it a priori}\, reasons for suspecting the data are
invalid -- telescope malfunction or star misidentification for
example.
MASK and UNMASK can also remove and re-instate individual
observations or sequences of observations.  Genuinely erroneous
observations (for example misidentified stars) should be removed
outside TPOINT simply by using a text editor.

\subsection{Reference Frames}
In the internal form produced by the INDAT command, each observation
consists of the raw mechanical HA/Dec of the telescope
and what is called in TPOINT the `true' HA/Dec of the star
({\it i.e.}\ as affected by both diurnal aberration and refraction).

The raw telescope HA/Dec normally means either the mechanical
HA/Dec directly read from dials or encoders, or, in the case of
non-equatorial mounts, the HA/Dec obtained by applying the
appropriate canonical transformation to the actual readings.
Az/El data, for example, are transformed by the standard
text-book Az/El to HA/Dec rotation, using the longitude
and geodetic latitude of the observatory as affected by polar motion.
For telescope control systems which can only conveniently log
mean places, an option is provided to convert this back to
`true' HA/Dec, assuming that all the telescope pointing
corrections proper have been disabled but that refraction and
diurnal aberration (as well as precession/nutation and annual
aberration) have been included.

In order to avoid having to log anything which changes
rapidly, the equatorial coordinates accepted by INDAT
are in the form of RA/Dec rather than HA/Dec.  The telescope HA
is deduced simply by subtracting the given RA from the given
sidereal time.  This applies to all but one of the data formats,
where star and telescope azimuths and elevations are supplied
directly and must all correspond to one moment in time.

The star `true' HA/Dec positions are obtained by allowing
as necessary for proper motion, precession, nutation,
annual aberration, and light deflection to obtain geocentric apparent
place, followed by corrections for diurnal aberration and
atmospheric refraction.  It is also possible to supply the star
data as `true' RA/Dec by (i)~including
an optional record which disables the diurnal
aberration corrections and (ii)~omitting the temperature, pressure,
{\it etc.}\ to disable the refraction corrections.  This would be done
in cases where the telescope control system which generated the
pointing data has already applied these corrections (either
correctly, or at least in a way which will be consistent with
any subsequent implementation of the pointing model obtained
from TPOINT).

The modelling procedures available elsewhere in this package have
the job of expressing the relationship between these two sets of
coordinates.  It is thus the responsibility of the
telescope control system
itself to perform accurate mean to apparent place transformations,
and to allow for Earth polar motion, diurnal aberration, refraction,
{\it etc.}, before applying the pointing model as determined by using
this package.

\subsection{INDAT Data Formats}
The INDAT
input file consists of free-format records up to 120 characters
in length, as follows:
\begin{tabs}
XXXXXX \= \kill
\> CAPTION record \\
\> OPTION records (optional) \\
\> RUN PARAMETERS record \\
\> OBSERVATION records \\
\> \fstring{END} or end-of-file
\end{tabs}

Blank lines and comments (which begin with a \fstring{!} character)
can be included anywhere.

\goodbreak
CAPTION record:
\begin{tabs}
XXXXXX \= \kill
\> First 80 characters used, trailing blanks eliminated
\end{tabs}

\goodbreak
OPTION records (optional):
\begin{tabs}
XXXXXX \= XXXXXXXXXXXXX \= \kill
\> \fstring{: NODA} \> disables correction for diurnal aberration \\
\> \fstring{: ALLSKY} \> disables horizon/zenith/pole checks \\
\> \fstring{: {\it equinox}} \>
             required if telescope positions are mean places
\end{tabs}
The correction for diurnal aberration (maximum 0.3~arcsec)
should be disabled if the telescope control system itself
omits it.  The ALLSKY option is useful in cases where, for one
reason or another, the observations are scattered all over the
celestial sphere.
Where the telescope positions are mean places, only one
{\it equinox}\, (probably \fstring{B1950} or \fstring{J2000})
may be specified, and applies to the whole file.

\goodbreak
RUN PARAMETERS record:
\begin{tabs}
XXXXXX \= \kill
\> latitude ($^\circ$,$'$,$''$) \\
\> UTC date (y,m,d) \\
\> temperature ($^\circ$C) \\
\> pressure (mB) \\
\> height above sea level (metre, default determined from pressure) \\
\> relative humidity (range 0--1, default 0.5) \\
\> observing wavelength (micrometre, default 0.55) \\
\> tropospheric lapse rate ($^\circ$K/metre, default 0.0065)
\end{tabs}
The latitude is mandatory.  If the UTC date is omitted, the
star and telescope data must not be mean places.  If the
refraction data (temperature and so on) are omitted, no
corrections for refraction are made.
Optical refraction is computed for wavelengths below 100~micrometres,
and radio refraction for wavelengths longer than this figure.
In the radio case, it is important to specify the humidity accurately.

\goodbreak
OBSERVATION records -- there are four different ways of expressing
each observation, as follows:

\hspace{2em} Observation record format 1:
\begin{tabs}
XXXXXX \= \kill
\> star geocentric apparent RA/Dec (h,m,s,$^\circ$,$'$,$''$) \\
\> telescope raw RA/Dec (h,m,s,$^\circ$,$'$,$''$) \\
\> local apparent sidereal time (h,m) \\
\> [up to 2 auxiliary readings]
\end{tabs}
\hspace{2em} Observation record format 2:
\begin{tabs}
XXXXXX \= \kill
\> star mean RA/Dec (h,m,s,$^\circ$,$'$,$''$,
                     $\mu_\alpha$,$\mu_\delta$,equinox) \\
\> telescope raw RA/Dec (h,m,s,$^\circ$,$'$,$''$) \\
\> local apparent sidereal time (h,m) \\
\> [up to 2 auxiliary readings]
\end{tabs}
\hspace{2em} Observation record format 3:
\begin{tabs}
XXXXXX \= \kill
\> \fstring{*} \\
\> identifier from star catalogue \\
\> telescope raw RA/Dec (h,m,s,$^\circ$,$'$,$''$) \\
\> local apparent sidereal time (h,m) \\
\> [up to 2 auxiliary readings]
\end{tabs}
\hspace{2em} Observation record format 4:
\begin{tabs}
XXXXXX \= \kill
\> star `true' Az/El ($^\circ$,$^\circ$) \\
\> telescope raw Az/El ($^\circ$,$^\circ$) \\
\> [up to 2 auxiliary readings]
\end{tabs}
Note that format 4 requires exact
times to be extracted from the telescope control
system, whereas formats 1-3 use instead the telescope RA/Dec,
which is unlikely to vary rapidly during tracking of a star
and is therefore easier to get right.
Moreover, precisely
consistent Az/El and time information may simply be
unavailable if special provision has not
been made in the control software.  However, in the case of
formats 1-3, if sidereal
time of high precision is available, this information can, and
should, be supplied, by appending decimals to the sidereal
time minutes.

Following the normal convention, the azimuths in format 4
are reckoned from north through east (though internally
within the TPOINT software they go from south through east).

Two approaches are available for reading `foreign' data formats.
The first is to implement a local TPOINT command which accepts the
foreign format and converts it directly into the internal form.
The second approach, which does not involve
modifications to TPOINT, is to write
a freestanding program which translates the foreign format into
one of the INDAT formats.  Such translation
programs are apt to be laborious
to write, and frequently involve applying positional astronomy
adjustments that will all be undone again when INDAT reads the
file.  Nevertheless, this approach is probably the
best in most cases.

\subsection{Star Catalogues}
A small star catalogue is supplied with the TPOINT system
and is loaded automatically when the system is initialized.  The
standard catalogue is a compilation of several
lists of stars used for pointing tests on various telescopes.
The first part
of the catalogue contains the selection of Perth70 stars used
on the AAT;  all the stars are magnitude 7-8 and lie
approximately on a $20^\circ$ grid, skewed
to give uniform coverage in both HA and Dec.

A private star catalogue can be substituted, either at startup time
via the command line or during a TPOINT session
via the INST command.  The original catalogue can be restored at
any time by means of an INST command with no arguments.

A star catalogue is a sequential file of up to 80-character
records.
Here is an example record:
\begin{tabs}
XXXXXX \= \kill
\> \verb|073716  00 00 51.781  +20 23 22.02  +0.0051 -0.146  1950.0|
\end{tabs}

The first six characters of each record are an
alphanumeric star identifier.  The remainder of the record
is the mean RA/Dec (hours, minutes, seconds, degrees,
arcminutes, arcseconds), RA/Dec proper motions (seconds and
arcseconds per year) and equinox (optionally preceded by
\fstring{B} or \fstring{J} to distinguish between the FK4 and FK5 systems).
Free-format input decoding is used.
The file is terminated by \fstring{END} or end-of-file.  See the
INCLUDE file STARS for the current maximum
allowed number of records.

The standard TPOINT syntax rules apply to such catalogue files,
which may thus contain blank lines and comments (beginning with
\fstring{!}) to enhance readability.  The INST command merely
reads the records;  actual decoding of the
information in the catalogue is deferred until a star is to be
looked up, and errors will not come to light until then.

\subsection{Saving and Combining Processed Data}
When the modelling possibilities of any one set of pointing data
have been exhausted it will be necessary to combine the results
of several runs if further systematic effects are to emerge from
the noise.  In general, the different runs cannot simply be
concatenated (which is done with the APPEND command)
because some of the terms will have
changed -- the collimation corrections in particular.  The UNFIT
command allows this difficulty to be overcome.

UNFIT applies a pointing model in reverse,
taking the current adjusted positions and predicting
the observations that produce those adjusted positions
when the model is applied.  Thus
a canonical model can be
applied to a set of pointing residuals to yield new
`observations' which can be combined for further
analysis.  Two modes of operation are provided:
\begin{cmnds}
\> \> {\tt UNFIT} \> to apply the current model \\
\> or \'\> {\tt UNFIT N} \> to apply a null model (coeffs all zero)
\end{cmnds}
If the star to telescope modelling option has been chosen
(via the ADJUST command), the star positions are set and the
telescope positions left alone.  If the telescope to star
option has been chosen, the reverse is true.

To understand UNFIT, think first of what the FIT command
does.  Assume that the modelling option `telescope to star' has
been chosen (the default).  Under these circumstances,
the FIT command takes the raw telescope positions and
computes what coefficient values in the current model
will give the smallest residuals, then applies the model
to the `raw telescope' positions to give the `adjusted
telescope' positions.  The differences between these
`adjusted telescope' positions and the `star true'
positions are the pointing residuals.
UNFIT leaves these pointing residuals at their current
values, but applies the pointing model in reverse to
generate fictitious `observations' (in this case raw
telescope positions) which are consistent with both
the current model and the current residuals.

If the current residuals and coefficient values are both
simply the result of fitting, UNFIT will not have any
significant effect.  However, if one or more of the
coefficients is set to a different value, UNFIT will
produce a changed set of observations which, if
fitted with the same model, will reproduce the changed
coefficient values.  This is how
different pointing runs can be combined.
First, you fit each run to the full model in the normal way,
to determine the residuals;  then
you set all the coefficients, one by one, to whatever
standard values are thought suitable;  then you
invoke the UNFIT command, to make the residuals and the
model match;  and finally you write out
the amended data by using the OUTDAT command:
\begin{cmnds}
\> \> {\tt OUTDAT} {\it file}
\end{cmnds}
where {\it file}\, is the name of the file to be written.
Two or more
amended runs can then be combined (by using APPEND
followed by INDAT), and will match
as long as the other coefficients were consistent
from run to run.  If the combined file is fitted with
the normal model, the arbitrary coefficients will have
approximately the values that were chosen for them, and
the other coefficients will be intermediate between the
the values produced by each of the two runs separately.

Very often, a null model (all coefficients zero) is required,
and here the UNFIT~N command is useful.
UNFIT N has essentially the
same effect as setting all the coefficients to zero by
hand and then executing an UNFIT command, except that it leaves
the original coefficient values intact.  UNFIT N is
useful for recording pure residuals and eliminating the
need for fitting with the original model when new
terms are being investigated.

Before using OUTDAT you should set an appropriate
caption with SETCAP.

\subsection{Sample Data}
Four examples of authentic pointing data are provided.  The file
UKST contains data from the UK Schmidt Telescope (1.2 metre aperture,
fork equatorial).  AAT15 contains data from the Anglo-Australian
Telescope (3.9 metre, horseshoe equatorial) in the f/15 Cassegrain
configuration.  HALE contains data from the Palomar 5~metre telescope
(horseshoe equatorial) in the f/17 IR configuration.  MMT contains
data from the central reference telescope of the Multiple Mirror
Telescope (4.5 metre equivalent aperture, altazimuth).  The UKST
file refers to the standard star catalogue;  the other three files
use explicit star positions and are therefore self contained.  The
filenames are as follows:
\begin{tabbing}
xxx \= xxxxxxxxxxxxx \= xxxxxxxx \= xxxxxxxxxxxxx \= \kill
\> \> PC \> VAX \> Unix \\ \\
\> UK Schmidt
\> {\tt UKST}
\> {\tt TPDIR:UKST}
\> {\tt /star/bin/examples/tpoint/ukst.dat} \\
\> AAT
\> {\tt AAT15}
\> {\tt TPDIR:AAT15}
\> {\tt /star/bin/examples/tpoint/aat15.dat} \\
\> Palomar
\> {\tt HALE}
\> {\tt TPDIR:HALE}
\> {\tt /star/bin/examples/tpoint/hale.dat} \\
\> MMT
\> {\tt MMT}
\> {\tt TPDIR:MMT}
\> {\tt /star/bin/examples/tpoint/mmt.dat}
\end{tabbing}

The recommended analysis procedure, described in the earlier section
{\it Getting Started}, begins as follows:
\begin{cmnds}
\> \> {\tt PLTON} {\it device} \\ \\
\> \> {\tt INDAT} {\it file} \\ \\
\> \> {\tt CALL EQUAT} \> (equatorial) \\
\> or \> {\tt CALL ALTAZ} \> (altazimuth) \\ \\
\> \> {\tt FIT} \\ \\
\> \> {\tt CALL E6} \> (equatorial) \\ \\
\> or \> {\tt CALL A6} \> (altazimuth)
\end{cmnds}
It is then necessary to start trying obvious extra terms via
the USE command -- tube flexure TF, fork flexure FO, {\it etc.}, as
appropriate.  Example models (which in the case of the UKST and AAT
are the ones actually in service) are supplied as procedures, yielding
the results shown in the following table:

\vspace{2ex}
\hspace{4.2em}
\begin{tabular}{|l|l|c|c|} \hline
datafile & procedure & $\sigma/''$ & terms \\ \hline
UKST & UKST & 6.17 &  7 \\
AAT15 & AAT  & 1.26 & 10 fitted + 4 fixed \\
HALE & HALE & 2.39 & 10 \\
MMT & MMT  & 0.73 & 11 \\ \hline
\end{tabular}
\vspace{2ex}

To reproduce, for example, the AAT result using a PC you would simply enter:
\begin{cmnds}
\> \> {\tt INDAT AAT15} \\ \\
\> \> {\tt CALL AAT}
\end{cmnds}

It is important to understand that these standard deviations are
{\it a posteriori}\, and may
not be realized in practice.  They do not take into account long
term stability, the effectiveness of any pre-observing calibration
procedures, the correctness of the implementation of the model
within the telescope control software, or the ability of the
user interface to give the astronomer access to the available
pointing performance.

\subsection{Simulated Pointing Tests}
When developing the pointing model of part of a telescope
control system, refer to the source code
for the tpt\_PTERMS module for information on the
mathematical form of TPOINT's
various pointing terms.  The operational implementation
of the terms may well involve techniques which differ in some way
from TPOINT's -- perhaps by being
more elaborate and general.  Any
substantial inconsistencies between TPOINT and the telescope
software will manifest themselves as pointing errors, and telescope
users will not experience the levels of performance
which the TPOINT results appear to offer.

In order to check that the model as implemented in the telescope
system is consistent with TPOINT, a ``dummy'' pointing test
is recommended.  This is like a real pointing test except that
the telescope is left where it is after each blind setting
rather than being guided onto the star.   When the dummy pointing test
is reduced with TPOINT, a perfect fit should be
reported, and the coefficients should exactly match
the values used by the control software.  Without a test of this
sort, errors of scaling and (especially) sign may go unnoticed.

Note that the dummy pointing test just described
is a check on the software alone, with the telescope itself
playing no part.  Consequently, it is usually best
to arrange that the control software can operate in
a ``simulation'' mode, capable of generating a dummy pointing test
offline.  It is {\bf strongly} recommended that dummy pointing tests
be carried out from time to time, to verify that any
revisions of the telescope software and TPOINT have not
led to incompatibilities.

\section{ADVANCED MODELLING}

TPOINT allows a straightforward model for a telescope to be generated
easily and without requiring great cunning or insight, but also
provides the tools more experienced and determined users will need when
conducting more exhaustive analyses.

As already described, the general strategy for modelling a telescope
from scratch is to input a data file with INDAT, to set up
a basic geometrical model using CALL~EQUAT or CALL~ALTAZ, and
to fit using the FIT command, repeating as necessary
until the coefficients
settle down.  With the aid of the G...\ commands, plots of residuals
can then be made and inspected for signs of systematic error.
Plausible flexure terms can be tried (USE~TF~{\it etc.}), and
any others that the particular telescope design suggests.

When adding new terms, pay attention to the standard deviation of
the new coefficient, the effect on other terms, and whether
the population standard deviation has gone down; a new term
will almost always reduce the RMS but population $\sigma$ is a better
indication of whether the improvement is real.  If an attempt
is made to use two terms which are more or less indistinguishable
(for example CH and PXH0, which are identical), the condition
will be reported;  if this happens, get rid of one of them
via the LOSE command.  In such ill-conditioned cases, the
singular value decomposition (SVD) methods used by FIT will normally
return a meaningful model, with the ``true'' value of a
duplicated term being shared out between the two (or more)
coefficients involved.  Two common cases are (i)~if two
inseparable terms are present from the start, they each receive
an equal share of the value, and (ii)~where a new term is added
to an already fitted model, and that term is inadequately
distinguished from something already present, then the new
term will be left close to zero.  This behaviour can be
controlled by means of the FITTOL command, which allows the
SVD ill-conditioned criterion to be set.  A command FITTOL~0 will switch
off the detection of ill-conditioning, whereupon FIT will return a
conventional least squares fit, and any highly correlated terms will
head off towards cancelling infinities.  A FITTOL argument
of between $10^{-3}$ and $10^{-2}$ will give a useful degree of control
over ill-conditioning;  the value is set to $10^{-2}$
initially.  With FITTOL arguments much bigger than
$10^{-2}$, there is increasing danger that respectable
terms will be excluded from full fitting.  However, whenever FIT
decides to take action over ill-conditioning, it reports that
this has occurred.

Do not try to extract more information from a given sample of
data than the size of the sample justifies.  A preliminary
assessment of a telescope's capabilities can be made using
perhaps 20-30 stars.  Routine fitting of a typical operational telescope
model, consisting of six geometrical terms, two or three
flexures, and perhaps a few scale adjustments, centring
corrections, {\it etc.}, calls for 50-100 stars, well-distributed over
the whole observable sky.
A campaign to expose high-frequency flexures
and bumps should not be embarked upon without a sample of 500-1500
stars, obtained by combining multiple tests by means of the UNFIT
and APPEND facilities.  Even with
this large number of stars, fitting a full set of polynomials or
harmonics up to the required degree in one go
is impractical, and new terms must be
tried out in groups of no more than about 20.

When first analysing a new telescope, it is not uncommon to see
obvious systematic residuals on the simple graphs of error in
one coordinate plotted against that or another one -- plotting
error in zenith distance versus against zenith distance (G~Z~Z) for
example.  However, once these straightforward effects
have been treated, systematic errors can still hide in what looks
like noise.  Some of
these can be exposed by using more specialized plots -- zenith
distance errors against parallactic angle (G~Z~Q), errors
in HA/Dec nonperpendicularity plotted against HA (G~P~H),
and so on.  However,
certain effects which depend simultaneously on two coordinates may
prove more stubborn;
given a data set of adequate size,
the solution is to use the MASK and UNMASK commands to select bands
in one coordinate, allowing the residuals for that one band to be
plotted.

At various stages, the GHYST command should be tried, to assess
whether the current pointing accuracy seems to be limited by
hysteresis.

When pursuing low-level flexures and irregularities, difficult
questions arise.  To what extent is the model physically meaningful?
Has adding empirical terms affected the supposedly well-understood
underlying model?  Is it worth trying to keep a grip on
physical reality, or wouldn't it be just as good to fit a
massive set of polynomials or harmonics?

An approach that seems to preserve the considerable advantages
of having a model firmly based on mechanical reality, yet which
allows all the power of empirical modelling to be unleashed, is
to construct a model which includes both
physical and empirical terms, but at no stage to
leave both sets free to float at once during fitting.  The
following procedure is one way to do this:
\begin{enumerate}
\item Fit many pointing tests with the best available model;
reject any obviously bad runs; UNFIT the rest onto a
model with a fixed set of typical coefficient values;  OUTDAT to a file;
use APPEND to concatenate all these files;  INDAT the
result.
\item FIX everything except maybe one or two zero points.
\item USE a selection of empirical terms.
\item FIT; reject any disruptive or poorly-determined terms and FIT again.
Repeat until only well-determined terms (say $2\sigma$ or better) remain.
\item Plot the residuals to see what extra terms might help.
\item Repeat from step 3 until further improvements seem unlikely.
\item FIX everything, un-fix the original model from step 1
by means of USE, and FIT.
\item Repeat from step 2 until neither the physical model nor the
empirical terms are changing significantly.
\item Repeat from step 1 inserting into to the standard model the
newly determined empirical terms {\bf FIXed at the values just
determined}.  Depending on how the terms will be coded
operationally, decide whether to flag them CHAINed or PARALlel.
\item Once everything has stabilized, code the empirical model
into one or more encapsulated local terms, using TRMLOC and
PTERML.  Include these terms in the standard model, with
their coefficients FIXed at unity.
\end{enumerate}

Such advanced modelling will require repeated fitting of large
datasets.  This will take time and be prone to mistakes.  A good way
to proceed is to pre-program the
key steps using a private procedure library.
Such a library can be edited during the TPOINT run by means of the
SPAWN command, and then re-input with the INPRO command.  A
private library is also an option when running in batch mode;
instead of specifying TPOINT commands in the batch control
file, the latter can simply start TPOINT, read the procedure
library with INPRO, and execute the required library routine by
means of an appropriate CALL.

\section{MAINTENANCE}
\subsection{Directories and Files}
The TPOINT development system resides on the VAX and consists of
a main directory and a subdirectory.  The
main directory contains the development system, and includes
separate Fortran modules, command procedures to build the libraries,
and so on.  The subdirectory is called [.RELEASE] and contains just
those files required to run the system, plus source in text libraries,
and the \LaTeX\ source for the Starlink User Note you are
currently reading.

The executable system requires the SGS, GKS, GNS, GWM, PSX, CNF,
CHR and EMS libraries.  To relink TPOINT also requires the SLALIB
and HELP libraries.

\goodbreak
The following files are in the {\bf development} directory:
\begin{tabs}
XXXXXX \= XXXXXXXXXXXXXXX \= \kill
\> READ.ME \> general information \\
\> TPOINT.NEWS \> news item on latest release \\
\> *.DAT \> copied to the runnable system (see below) \\
\> SUN100.TEX \> copied to the runnable system (see below) \\
\> *.FOR \> TPOINT Fortran source modules common to all platforms \\
\> *.IND \> platform-independent TPOINT Fortran source modules \\
\> *.VAX \> VAX/VMS-dependent TPOINT Fortran source modules \\
\> *.PCM \> PC/Microsoft-dependent TPOINT Fortran source modules \\
\> *.ASM \> PC/Microsoft Assembler TPOINT source modules \\
\> *.SUN4 \> Sun SPARCstation/SunOS-dependent TPOINT Fortran source modules \\
\> *.HLP \> HELP text \\
\> CREATE.COM \> DCL procedure to create the TPOINT system from source \\
\> LTP.COM \> DCL procedure to link TPOINT \\
\> LTP.OPT \> For linking TPOINT on VAX \\
\> PUT.COM \> DCL procedure to update a module in the libraries \\
\> TPOINT.COM \> DCL procedure for starting TPOINT \\
\> TPOINT \> C-shell script for starting TPOINT \\
\> MK \> \verb|mk| script for building Unix versions \\
\> MAKEFILE \> \verb|makefile| for building Unix versions \\
\> TPOINT.OLB \> object library \\
\> PCTOVAX.COM \> tailor PC/VAX release for VAX \\
\> *.BAT \> for PC \\
\> LTP.LB \> for PC \\
\> LTP.LNK \> for PC \\
\> VAX\_TO\_UNIX.USH \> C-shell script to produce a Unix archive file
\end{tabs}

\goodbreak
The following files are in the {\bf runnable system} directory:
\begin{tabs}
XXXXXX \= XXXXXXXXXXXXXXX \= \kill
\> AAT15.DAT \> sample data -- Anglo-Australian Telescope \\
\> UKST.DAT \> sample data -- UK Schmidt \\
\> HALE.DAT \> sample data -- Palomar 5 metre \\
\> MMT.DAT \> sample data -- Multiple Mirror Telescope \\
\> TPOINT.TLB \> Fortran source modules \\
\> TPOINT.COM \> command procedure to run the TPOINT program \\
\> PROCS.DAT \> library of TPOINT procedures \\
\> STARS.DAT \> star catalogue \\
\> TPOINT.SHL \> HELP library -- root \\
\> CMNDS.SHL \> HELP library -- ``Commands'' tree \\
\> MODEL.SHL \> HELP library -- ``Model'' tree \\
\> LOCAL.SHL \> HELP library -- ``Local\_Enhancements'' tree \\
\> TPOINT.EXE \> the TPOINT executable program \\
\> SUN100.TEX \> \LaTeX\ source for SUN/100
\end{tabs}

See the file READ.ME for information about recent changes and
instructions for setting up the SGS graphics system.

SUN100.TEX contains alternative page layout commands to suit
either European A4 or American Quarto paper sizes.  The file supplied
is set up for A4;  to produce the Quarto version simply make a private copy
of SUN100.TEX, comment out the A4 line, and remove the comment character
from the Quarto line.

\subsection{Compiling and Linking}
The procedure for modifying the TPOINT system is as
follows:
\begin{enumerate}
\item {\tt SET DEFAULT} to the development directory.
\item Modify the appropriate Fortran source module(s).
\item Use {\tt @PUT} {\it module}\, to compile each module
      and update the source and object libraries.
\item Relink with {\tt @LTP}.
\item Update the HELP source, and use
      {\tt @HELPDIR:HLIB *.HLP [.RELEASE]*.SHL} to
      create the help libraries.
\item Update the documentation.
\end{enumerate}

After a series of changes it is wise to execute the CRE
procedure in order to recreate the whole system from source.

\subsection{New Commands}
Special versions of TPOINT containing extra commands (for example
to read observation data in local formats) can easily be generated.
The essential details are as follows:

A command is implemented as an argument-less subroutine,
conventionally of the same name as the command, prefixed with
tpt\_.  Code to recognize
the command name and to execute the appropriate subroutine must be
added to the tpt\_CMDLOC module.  This is straightforward, and the
tpt\_CMDLOC implementation supplied with TPOINT contains example coding.
Information flow in and out of the routine is by COMMON blocks,
defined through a set of INCLUDE files.  These files should be
studied to see what information is available -- command line,
observations, model tables, {\it etc.}  Prior to the command subroutine
being called, the variable IOKF (defined in INCLUDE file CMD)
will have been set to \fstring{?}, and the subroutine should either reset
IOKF to a space on successful completion or left alone to indicate
failure.  The flag ABORT (also defined in CMD.INC) should be
examined during lengthy applications;  if ABORT becomes .TRUE.\ at
any time, the application should be brought to a speedy but clean
finish.  This may involve making sure that tables are internally
consistent and properly endmarked, and that the graphics system
has been left in a tidy state.  The best plan is to take one of
the existing applications and to use it as a template for the new
one.

n.b.  Locally-implemented commands have to make assumptions about
the location and meaning of the information in the COMMON blocks,
and may also call TPOINT internal routines.  The stability of
these ``unpublished'' interfaces cannot formally be guaranteed.

\subsection{Local Pointing Terms}
Where it is found that the existing library of pointing terms is
not enough to eliminate some observed or predicted systematic
effect, new pointing terms can be added to the system.  This
can be done without altering any of the standard TPOINT modules,
minimizing the work of re-implementing locally-defined terms each
time a new release of the system occurs.  The modules set aside
for local terms are tpt\_TRMLOC, which defines their names, and tpt\_PTERML,
which defines their form.  (The module tpt\_TRMLOC
can also perform any special
initialization actions required by a locally-implemented tpt\_PTERML module.)
The supplied versions of tpt\_TRMLOC and tpt\_PTERML
contain examples, and they are both consistent with their
counterparts tpt\_TRMSTD and tpt\_PTERMS, which perform the same role for
TPOINT's standard set of terms.  It is permissible to have
local terms which are identical with standard ones in form and/or
name; in the case of identical names, the locally defined term
takes precedence over the standard one.

To introduce a new pointing term, the following changes to tpt\_TRMLOC
and tpt\_PTERML are required:

Changes to tpt\_TRMLOC:
\begin{enumerate}
\item Update the value given in the NCOEFF parameter statement.
\item Add the name of the new coefficient to the CONAME data
      statement.
\end{enumerate}

Changes to tpt\_PTERML:
\begin{enumerate}
\setcounter{enumi}{2}
\item Add a further destination to the large computed GO TO near
      the start of the module.
\item Insert, at the appropriate place, a CONTINUE with the label
      chosen in the previous step, followed by the expressions which
      calculate the pointing change per unit coefficient.  Use
      existing terms (in tpt\_PTERML and tpt\_PTERMS) as a guide.
\end{enumerate}

After these changes, follow the compiling and linking instructions
given in the previous section.

Some implementors of telescope control systems have incorporated
tpt\_PTERML into their operational software, thereby ensuring that
the model encountered by users
is the same as the one that generated
the impressive RMS pointing performance quoted for the telescope
(a weak link at many observatories).
To facilitate this practice, tpt\_PTERML uses neither COMMON nor INCLUDE
files, and carries a command and
status argument which effectively does nothing in
the standard version of the TPOINT system.
TPOINT calls tpt\_PTERML with a command of zero,
telling private versions that normal TPOINT operation is required.
The standard version of
tpt\_PTERML ignores the command and always returns a
zero status, indicating success.  A private version of
tpt\_PTERML should return a status which is zero or positive, to indicate
success.  The routines within TPOINT
which use tpt\_PTERML check the status and, if less than zero,
flag the observation inactive, excluding it from further use until
re-activated by means of the UNMASK command.  The two TPOINT commands
which do this are FIT and OUTMOD.

\section{HISTORY AND ACKNOWLEDGMENTS}
TPOINT grew out of the work John~Straede and I did at the
AAT between 1974 and 1980 using Interdata~70 computers.
Most of the early least-squares software
was written by John~Straede, culminating in
the very successful IBORL2 program. I then developed
a new program based on the IBORL2 fitting algorithm,
but with improved
interactive model building capabilities, extra graphics options,
{\it etc.}; this program was the direct ancestor
of TPOINT.  In the early 1980s the program was ported from
the Interdata to the VAX and then refined and expanded, exploiting the
Starlink SGS/GKS and SLALIB packages as they became available.
TPOINT underwent a major refurbishment in 1986/87 to prepare
it for release through Starlink.  During the period 1990-92 a
version which could be run on both VAX/VMS and PC/MS-DOS was
developed, followed by further revisions to include Unix platforms.

TPOINT would probably not exist had
Peter~Gillingham, Joe~Wampler and Don~Morton not encouraged
continual refinement of the AAT pointing.
Recent development of TPOINT has been greatly assisted by
the many valuable comments and suggestions I
have received from Steve~Lee, Robert~Laing, Ken~Elliott,
Hilton~Lewis, Bob~Kibrick, and Fang~Yanling.  I have worked
closely with Russell~Owen of the Apache Point 3.5~metre project
on facilities for local additions to, and use of, TPOINT code,
and on extensions to the modelling and graphical tools.

\pagebreak
\section{COMMAND REFERENCE}
This section contains a description of every TPOINT
command, in alphabetical order.  A quick reference list
is given at the end of the document.

%--------------------------------------------------------------------

\goodbreak
\rule{\textwidth}{0.3mm}
{\Large {\bf \it coeff} \hfill Set/Inquire Coefficient Value \hfill
                                                      {\bf \it coeff}}
\begin{description}
\item [FUNCTION]:

The value of any of the coefficients in the current model
may be set or inquired by using its name as the command.

\item [COMMAND]:

\begin{cmd}
\> \> {\it coeff} {\it value}
\end{cmd}

where {\it coeff}\, is the name of the coefficient, and {\it value}\,
is the value.

\item [NOTES]:

If no argument is supplied, the coefficient is unaltered -- merely
logged.

In most cases {\it value}\, is in arcseconds; a few terms have
dimensionless coefficients.

\end{description}

%--------------------------------------------------------------------

\goodbreak
\rule{\textwidth}{0.3mm}
{\Large {\bf ADJUST} \hfill Select Model Direction \hfill {\bf ADJUST}}
\begin{description}
\item [FUNCTION]:

ADJUST allows one of two methods of applying the
pointing model to be selected -- whether the telescope positions
are to be adjusted to fit the star positions or {\it vice versa}.

\item [COMMAND]:

\begin{cmd}
\> \> ADJUST T \> to adjust the telescope positions \\
\> or \> ADJUST S \> to adjust the star positions
\end{cmd}

\item [NOTES]:

If no argument is supplied, the method is unaltered -- merely
logged.

If the ADJUST command results in the direction being changed,
the order of the terms in the pointing model is reversed to match.

A given set of data will not produce precisely the same
coefficients if the model is reversed.  The differences will
be small unless one or more of the coefficients is very
large -- hundreds of arcseconds or more.  Under these
circumstances there is in any case a danger that TPOINT's
simple formulation of the pointing terms does not match
exactly the way the model is implemented in the control
software of the telescope concerned.  Large collimation
corrections, in particular, may require more rigorous
geometry than TPOINT can conveniently offer, and detailed
knowledge of field distortions.

The results of any previous FIT are not changed when ADJUST is
used and, if a fresh FIT is to be carried out, the RESET command
should be used first in order to eliminate the previous corrections.

\end{description}

%--------------------------------------------------------------------

\goodbreak
\rule{\textwidth}{0.3mm}
{\Large {\bf APPEND} \hfill Prepare to Append Data \hfill {\bf APPEND}}
\begin{description}
\item [FUNCTION]:

APPEND prepares to concatenate the next set of pointing
observations with the ones already input.

\item [COMMAND]:

\begin{cmd}
\> \> APPEND \> next set of observations will be appended \\
\> or \> APPEND OFF \> cancel previous APPEND request
\end{cmd}

\item [NOTES]:

The new data will be appended when the next INDAT command is
executed.

At the time an APPEND request is made, a set of pointing
observations must already be present in the data list.

\end{description}

%--------------------------------------------------------------------

\goodbreak
\rule{\textwidth}{0.3mm}
{\Large {\bf CALL} \hfill Call a Library Procedure \hfill {\bf CALL}}
\begin{description}
\item [FUNCTION]:

CALL calls a TPOINT library procedure.

\item [COMMAND]:

\begin{cmd}
\> \> CALL {\it procedure}
\end{cmd}

where {\it procedure}\, is the name of the procedure to be
called (1--10 characters).

\end{description}

%--------------------------------------------------------------------

\goodbreak
\rule{\textwidth}{0.3mm}
{\Large {\bf CAPT} \hfill Enable/Disable Captions \hfill {\bf CAPT}}
\begin{description}
\item [FUNCTION]:

CAPT enables or disables plotting of graph
captions.

\item [COMMAND]:

\begin{cmd}
\> \> CAPT ON \> enable plotting of captions \\
\> or \> CAPT OFF \> disable plotting of captions
\end{cmd}

\item [NOTES]:

The argument defaults to ON.

The motivation for disabling caption plotting may
be to save time, or to avoid having the same caption repeated
on a multi-plot display.

\end{description}

%--------------------------------------------------------------------

\goodbreak
\rule{\textwidth}{0.3mm}
{\Large {\bf CHAIN} \hfill Apply Terms Sequentially \hfill {\bf CHAIN}}
\begin{description}
\item [FUNCTION]:

CHAIN is used to indicate which terms
are to be evaluated in sequence to their immediate predecessor in the
model, each such term then being a function of the position
as adjusted by the preceding term.

\item [COMMAND]:

\begin{cmd}
\> \> CHAIN {\it coeff1 coeff2 etc.}
\end{cmd}

where the arguments are the names of the coefficients to be
CHAINed to their predecessors.

\item [NOTES]:

If no arguments are supplied, all the terms in the model
are CHAINed.

If any argument is unrecognized, the entire command is
rejected.

The converse command is PARAL, which allows more than one
term to be calculated from a common starting position.

\end{description}

%--------------------------------------------------------------------

\goodbreak
\rule{\textwidth}{0.3mm}
{\Large {\bf CLIST} \hfill List Coefficients \hfill {\bf CLIST}}
\begin{description}
\item [FUNCTION]:

CLIST lists the names of the current coefficients,
their values, and whether they are fixed or floating.

\end{description}

%--------------------------------------------------------------------

\goodbreak
\rule{\textwidth}{0.3mm}
{\Large {\bf CLRNG} \hfill Enable/Disable Clearing \hfill {\bf CLRNG}}
\begin{description}
\item [FUNCTION]:

CLRNG specifies which mode of
automatic clearing of the display surface or plotting
zone is to occur whenever a new graph is drawn.

\item [COMMAND]:

\begin{cmd}
\> \> CLRNG \> enable screen clearing \\
\> or \> CLRNG ON \> enable zone clearing \\
\> or \> CLRNG OFF \> disable clearing
\end{cmd}

\item [NOTE]:

Clearing can be disabled to speed up multi-plot displays,
following an initial GCLR to fully erase the workstation.
The current zone can be cleared, preserving existing graphs
elsewhere on the screen, or the whole screen can be cleared
each time, to save time on slow devices when only one graph,
of reduced size, is being plotted

\end{description}

%--------------------------------------------------------------------

\goodbreak
\rule{\textwidth}{0.3mm}
{\Large {\bf ECHO} \hfill Enable/Disable Command Echoing \hfill {\bf ECHO}}
\begin{description}
\item [FUNCTION]:

ECHO enables or disables the echoing of commands to the
screen during execution of library procedures.

\item [COMMAND]:

\begin{cmd}
\> \> ECHO ON \> enable command echoing \\
\> or \> ECHO OFF \> disable command echoing
\end{cmd}

\item [NOTES]:

The argument defaults to ON.  However, echoing is disabled when
TPOINT is started.

Echoing is useful for debugging, but can clutter the screen on
some graphics terminals.

\end{description}

%--------------------------------------------------------------------

\goodbreak
\rule{\textwidth}{0.3mm}
{\Large {\bf END} \hfill Terminate the Session \hfill {\bf END}}
\begin{description}
\item [FUNCTION]:

END terminates TPOINT.

\item [NOTE]:

Additionally, QUIT, QU and CTRL/Z are all interpreted as
END commands.

\end{description}

%--------------------------------------------------------------------

\goodbreak
\rule{\textwidth}{0.3mm}
{\Large {\bf FIT} \hfill Fit Model to Data \hfill {\bf FIT}}
\begin{description}
\item [FUNCTION]:

FIT computes the coefficient values for the current
pointing model which fit the observations best, and updates
either the adjusted telescope coordinates or the adjusted star
coordinates depending on which of these two options is in force
(see the ADJUST command).

\item [COMMAND]:

\begin{cmd}
\> \> FIT \> to perform a normal fit \\
\> or \> FIT N \> to apply the current model without fitting
\end{cmd}

\item [NOTES]:

If two terms in the model are too highly correlated to be reliably
distinguished, this is reported by FIT so that one of
them can be eliminated from the model.  The FITTOL command provides
control over the treatment of such ill-conditioned cases, allowing
the acceptance criterion used by the ``singular value decomposition''
fitting method used in FIT to be specified.  If this criterion is
set to zero, by means of a FITTOL~0 command, the fit is simply a
conventional least-squares approximation, and any highly correlated
terms will be awarded meaningless, large and mutually cancelling
values, accompanied as a rule by a poor overall RMS result.  With
a FITTOL parameter of (for example) 0.01, such correlated terms
are kept under control and the model is sound.

If all the coefficients have been FIXed, the result of the FIT
command is simply to recompute the residuals, and in this case
FIT and FIT~N are equivalent.

\end{description}

%--------------------------------------------------------------------

\goodbreak
\rule{\textwidth}{0.3mm}
{\Large {\bf FITTOL} \hfill Set ill-conditioning tolerance \hfill {\bf FIT}}
\begin{description}
\item [FUNCTION]:


FITTOL specifies and reports the acceptance criterion
for the singular value decomposition algorithm used in the FIT
command.

\item [COMMAND]:

\begin{cmd}
\> \> FITTOL {\it v}
\end{cmd}

where {\it v}\, is the tolerance used by the FIT command to decide
whether action is to be taken to keep a poorly determined
solution under control.

\item [NOTES]:

A tolerance value {\it v}\, between zero and 0.01 is recommended.
The initial value is 0.01.

If no argument is supplied, the current tolerance is
reported but not changed.

\end{description}

%--------------------------------------------------------------------

\goodbreak
\rule{\textwidth}{0.3mm}
{\Large {\bf FIX} \hfill Fix the Specified Terms \hfill {\bf FIX}}
\begin{description}
\item [FUNCTION]:

FIX excludes one or more terms from the fit
but not from the model.

\item [COMMAND]:

\begin{cmd}
\> \> FIX {\it coeff1 coeff2 etc.}
\end{cmd}

where the arguments are the names of the coefficients to
be frozen at their current values.

\item [NOTES]:

If no arguments are supplied, the whole model is fixed.

If any argument is unrecognized, the entire command is rejected.

\end{description}

%--------------------------------------------------------------------

\goodbreak
\rule{\textwidth}{0.3mm}
{\Large {\bf FRAME} \hfill Enable/Disable Frames \hfill {\bf FRAME}}
\begin{description}
\item [FUNCTION]:

FRAME enables or disables the plotting of graph
frames.  A frame is a rectangular box drawn just inside the
edge of the plotting zone.

\item [COMMAND]:

\begin{cmd}
\> \> FRAME ON \> enable plotting of frames \\
\> or \> FRAME OFF \> disable plotting of frames
\end{cmd}

\item [NOTE]:

If no argument is supplied, the default is frames enabled.

\end{description}

%--------------------------------------------------------------------

\goodbreak
\rule{\textwidth}{0.3mm}
{\Large {\bf G} \hfill Plot Residuals \hfill {\bf G}}
\begin{description}
\item [FUNCTION]:

G plots the pointing residuals as the component in
one coordinate against that or another coordinate.

\item [COMMAND]:

\begin{cmd}
\> \> G {\it ydata xdata scale}
\end{cmd}

where {\it ydata}\, is one of the following:

\begin{tabs}
XXXXXX \= XXXXXX \= \kill
\> H \> pointing error in HA \\
\> X \> pointing error EW on the sky \\
\> D \> pointing error in Dec \\
\> P \> pointing error corresponding to HA/Dec nonperpendicularity \\
\> A \> pointing error in azimuth \\
\> S \> pointing error horizontally on the sky \\
\> Z \> pointing error in zenith distance \\
\> V \> pointing error corresponding to Az/El nonperpendicularity \\
\> R \> total pointing error
\end{tabs}

and {\it xdata}\, is one of the following:

\begin{tabs}
XXXXXX \= XXXXXX \= \kill
\> H \> hour angle \\
\> D \> declination \\
\> A \> azimuth \\
\> Z \> zenith distance \\
\> Q \> parallactic angle \\
\> N \> observation order
\end{tabs}

The optional argument {\it scale}\, indicates the vertical plotting
scale: it is the absolute value of the largest residual to be
plotted.  In default, a scale is chosen autoamtically to suit
the data.

\item [NOTE]:

The residuals are in the sense telescope minus true.

\end{description}

%--------------------------------------------------------------------

\goodbreak
\rule{\textwidth}{0.3mm}
{\Large {\bf GAM} \hfill Look For Axis Misalignment
\hfill {\bf GAM}}
\begin{description}
\item [FUNCTION]:

GAM plots the pointing residuals interpreted as a misalignment
of either the polar or the azimuth axis.  Two superimposed graphs
are plotted, showing respectively (i)~the orientation and (ii) the
amount of misalignment of the polar or azimuth axis that would
produce each of the residuals.  Graph (i)~consists of the orientations
calculated for each observation plotted in histogram form.  Graph (ii)
shows the component of pointing residual for each observation in the
direction of the mean orientation.  The x-axis of the two graphs is
the hour angle or azimuth.

\item [COMMAND]:

\begin{cmd}
\> \> GAM {\it type scale}
\end{cmd}

The {\it type}\, is E or Q for equatorial and A or Z
for Az/El.  The equatorial
plots have hour angle as their x-axis and look for misalignment of a
polar axis.  The Az/El plots have azimuth as their x-axis and look
for misalignment of an azimuth axis.  The Q and Z options
suppress an informational message which is normally sent to
the command terminal, and are intended for use where the
command terminal is of a type which does not have independent
alphanumeric and graphic planes.  This argument defaults to E.

The {\it scale}\, indicates the plotting scale for the
amount-of-misalignment plot, and is the absolute value of the
largest residual to be plotted.  In default, a scale is chosen
automatically to suit the data.

\item [NOTE]:

When constructing the histogram and computing the mean orientation,
each individual orientation is weighted by the size of the
supposed axis misalignment.

\end{description}

%--------------------------------------------------------------------

\goodbreak
\rule{\textwidth}{0.3mm}
{\Large {\bf GCLR} \hfill Clear the Display Surface \hfill {\bf GCLR}}
\begin{description}
\item [FUNCTION]:

GCLR clears the entire display surface of the
current plotting device.

\end{description}

%--------------------------------------------------------------------

\goodbreak
\rule{\textwidth}{0.3mm}
{\Large {\bf GDIST} \hfill Plot Error Distributions \hfill {\bf GDIST}}
\begin{description}
\item [FUNCTION]:

GDIST plots distributions of the pointing residuals.

\item [NOTE]:

The histograms labelled X, D, S and Z are the components east-west,
north-south, Left-Right and Up-Down; the histogram labelled R is
for the total pointing errors.

\end{description}

%--------------------------------------------------------------------

\goodbreak
\rule{\textwidth}{0.3mm}
{\Large {\bf GHYST} \hfill Hysteresis Plot \hfill {\bf GHYST}}
\begin{description}
\item [FUNCTION]:

GHYST attempts to expose hysteresis by assuming that the
direction (and perhaps distance) from the current observation
to the previous observation determine the direction (and perhaps
size) of the effect.  The format is a polar plot, with north
at the top if the equatorial option has been selected, or up at
the top if the altazimuth option has been selected.  Each observation
(except the first) is plotted as a square marker, from the centre of
which comes a line indicating the pointing residual.  The position
of the marker shows in what direction the telescope had to move in
order to travel from the previous star, and how far (on a
logarithmic scale).

\item [COMMAND]:

\begin{cmd}
\> \> GHYST {\it type scale}
\end{cmd}

The {\it type}\, is E or Q for equatorial and A or Z
for Az/El.  The equatorial
plot has north at the top, whereas the Az/El plot has up
at the top.  The Q and Z options
suppress an informational message which is normally sent to
the command terminal, and are intended for use where the
command terminal is of a type which does not have independent
alphanumeric and graphic planes.  This argument defaults to E.

The {\it scale}\, is the residual in arcsec to
give a vector of nominal maximum length.  This defaults
to the largest actual residual providing it is reasonable.

\item [NOTES]:

A numerical estimate of the hysteresis is made and reported.  It
is obtained by rotating each pointing residual by the orientation
of the preceding telescope movement, accumulating vectorially,
and finally dividing by the number of active observations.

Note that various important but dubious assumptions are being made:
\begin{itemize}
\item The order of observations in the data list is the same
as in reality.
\item The telescope travelled directly from one star to the
next, along a straight line in Cartesian Cylindrical coordinates.
\item Any hysteresis comes from the large scale movements
between observations and is unaffected by small scale
adjustments during acquisition.
\end{itemize}

Residuals are in the sense telescope minus true.

For equatorial mounts, there may be hysteresis in both HA/Dec
and Az/El, and it is worth trying both.  Altazimuth mounts
are not likely to display any HA/Dec effects, and the (default)
E option is not appropriate for this case.

Because most observing sequences tend to favour particular
telescope movement directions the graph produced by GHYST is
cluttered, with many points bunched together.  The appearance
may be improved by using the MARKH command to reduce the size
of the markers.

\end{description}

%--------------------------------------------------------------------

\goodbreak
\rule{\textwidth}{0.3mm}
{\Large {\bf GMAP} \hfill Cartesian Cylindrical Plot \hfill {\bf GMAP}}
\begin{description}
\item [FUNCTION]:

GMAP draws a map of the pointing residuals as
error vectors on a Cartesian Cylindrical projection.

\item [COMMAND]:

\begin{cmd}
\> \> GMAP {\it type scale}
\end{cmd}

The {\it type}\, is E for equatorial and A for Az/El.
This argument defaults to E.

The {\it scale}\, is the residual in arcsec to
give a vector of nominal maximum length.  This defaults
to the largest actual residual providing it is reasonable.

\end{description}

%--------------------------------------------------------------------

\goodbreak
\rule{\textwidth}{0.3mm}
{\Large {\bf GSCAT} \hfill Scatter Plot \hfill {\bf GSCAT}}
\begin{description}
\item [FUNCTION]:

GSCAT plots the pointing residuals as a scatter
diagram.  The plot resembles a view of the telescope field, with
points showing where each star would have appeared had
the telescope been set blind, using the current pointing model.
The field can be one in which north-south is at a fixed orientation, or
one in which the vertical is fixed.

\item [COMMAND]:

\begin{cmd}
\> \> GSCAT {\it type radius}
\end{cmd}

The {\it type}\, is E for equatorial and A for Az/El.  The equatorial
plot is relative to east-west and north-south axes,
whereas the Az/El plot
is relative to left-right and up-down axes.  This argument
defaults to E.

The {\it radius}\, is that of the scatter graph in arcsec and
must lie in the range 0.1--9999.9.  It defaults to a suitable scale to
display the data concerned.

\end{description}

%--------------------------------------------------------------------

\goodbreak
\rule{\textwidth}{0.3mm}
{\Large {\bf GSMAP} \hfill Orthographic Map \hfill {\bf GSMAP}}
\begin{description}
\item [FUNCTION]:

GSMAP draws a map of the pointing residuals as
error vectors on an orthographic projection.

\item [COMMAND]:

\begin{cmd}
\> \> GSMAP {\it scale}
\end{cmd}

where {\it scale}\, is the plotting scale, the residual in arcsec to
give a vector of nominal maximum length.

\item [NOTE]:

The {\it scale}\, defaults to the largest actual residual providing
it is reasonable.

\end{description}

%--------------------------------------------------------------------

\goodbreak
\rule{\textwidth}{0.3mm}
{\Large {\bf HELP} \hfill Enter HELP Session \hfill {\bf HELP}}
\begin{description}
\item [FUNCTION]:

HELP enters a HELP session.

\item [COMMAND]:

\begin{cmd}
\> \> HELP {\it topic}
\end{cmd}

where {\it topic}\, is the HELP topic to be displayed.

\item [NOTES]:

If no {\it topic}\, is specified, the top level HELP topics are
displayed.

The TPOINT online HELP library resembles an ordinary VAX/VMS HELP
library, and may be explored in the normal way.  To end the
HELP session, either keep replying RETURN until the * prompt
reappears, or use CTRL/Z.

\end{description}

%--------------------------------------------------------------------

\goodbreak
\rule{\textwidth}{0.3mm}
{\Large {\bf INDAT} \hfill Read File of Observations \hfill {\bf INDAT}}
\begin{description}
\item [FUNCTION]:

INDAT reads a file of telescope pointing
observations and converts it into the internal form required
for analysis.

\item [COMMAND]:

\begin{cmd}
\> \> INDAT {\it file}
\end{cmd}

where {\it file}\, is the name of the file of pointing observations.

\item [NOTES]:

The internal form generated by INDAT is called the {\it data list}.

Several input formats are supported.  Details are given
in the section {\it Pointing Data}, earlier.

\end{description}

%--------------------------------------------------------------------

\goodbreak
\rule{\textwidth}{0.3mm}
{\Large {\bf INMOD} \hfill Read Model from File \hfill {\bf INMOD}}
\begin{description}
\item [FUNCTION]:

INMOD reads a pointing model from a file.

\item [COMMAND]:

\begin{cmd}
\> \> INMOD {\it file}
\end{cmd}

where {\it file}\, is the name of the file containing the model.

\item [NOTES]:

Model files can be written by the OUTMOD command.

Any existing model is superseded by the one input.

In addition to the model itself, the file contains a
caption, a number of observations, and an RMS;  these latter
items are not used by INMOD.

\end{description}

%--------------------------------------------------------------------

\goodbreak
\rule{\textwidth}{0.3mm}
{\Large {\bf INPRO} \hfill Input a Procedure Library \hfill {\bf INPRO}}
\begin{description}
\item [FUNCTION]:

INPRO inputs a procedure library, replacing any
previous one.

\item [COMMAND]:

\begin{cmd}
\> \> INPRO {\it file}
\end{cmd}

where {\it file}\, is the name of the procedure file to be input.

\item [NOTES]:

If a filename is not specified, the procedure library which was input
at the start of the TPOINT session is re-read.

This command is not permitted within procedures.

A useful sample procedure library
is supplied with the TPOINT
system and is loaded automatically when the system is initialized.
The command \mbox{INPRO} with no arguments will reload it.

The initialization procedure INIT, called when
TPOINT is first started, is {\bf not} called when
INPRO is used.

\end{description}

%--------------------------------------------------------------------

\goodbreak
\rule{\textwidth}{0.3mm}
{\Large {\bf INST} \hfill Input a Star Catalogue \hfill {\bf INST}}
\begin{description}
\item [FUNCTION]:

INST inputs a star catalogue, replacing any previous one.

\item [COMMAND]:

\begin{cmd}
\> \> INST {\it file}
\end{cmd}

where {\it file}\, is the name of the catalogue file to
be input.

\item [NOTE]:

If a filename is not specified, the star catalogue which was input
at the start of the TPOINT session is re-read.

A small star catalogue is supplied with the TPOINT
system and is loaded automatically when the system is initialized.
The command \mbox{INST} with no arguments will reload it.

\end{description}

%--------------------------------------------------------------------

\goodbreak
\rule{\textwidth}{0.3mm}
{\Large {\bf LOSE} \hfill Remove terms from Model \hfill {\bf LOSE}}
\begin{description}
\item [FUNCTION]:

LOSE removes one or more terms from the
pointing model.

\item [COMMAND]:

\begin{cmd}
\> \> LOSE {\it coeff1 coeff2 etc.}
\end{cmd}

where {\it coeff1 coeff2 etc.}\ are the names of the terms
to be eliminated.

\item [NOTES]:

If no arguments are supplied, the whole model is discarded.

If any argument is unrecognized, the entire command is rejected.

\end{description}

%--------------------------------------------------------------------

\goodbreak
\rule{\textwidth}{0.3mm}
{\Large {\bf MARKH} \hfill Specify Marker Height \hfill {\bf MARKH}}
\begin{description}
\item [FUNCTION]:

The MARKH command specifies the marker height for plotting.

\item [COMMAND]:

\begin{cmd}
\> \> MARKH {\it h}
\end{cmd}

where {\it h}\, is the marker height in plotting units.

\item [NOTES]:

The default is 0.2.

The markers controlled in this way are the square and asterisk
symbols used by the G... commands.

The normal use of MARKH is to reduce the marker size
where the number of observations is large and the default
size gives a cluttered result.

\end{description}

%--------------------------------------------------------------------

\goodbreak
\rule{\textwidth}{0.3mm}
{\Large {\bf MASK} \hfill Flag Selected Observations Inactive \hfill
                                                         {\bf MASK}}
\begin{description}
\item [FUNCTION]:

MASK flags selected pointing observations inactive.  The selection
is made by area of sky, total residual, or sequence number.

\item [COMMAND]:

\begin{cmd}
\> \> MASK {\it quantity  condition  value } \\
\> or \> MASK {\it n1  n2 }
\end{cmd}

where the arguments are as follows:
\begin{tabbing}
XXXXXX \= XXXXXXXXX \= \kill
\> {\it quantity} \> H (HA), D (Dec), A (Az), Z (ZD), \\
\>                \> R (radial error) or N (observation number) \\
\> {\it condition} \> L (less than) or G (greater than) \\
\> {\it value} \> cutoff value (degrees for H, D, A or Z; \\
\>             \> arcseconds for R; number for N) \\ \\
\' or \> {\it n1} \> first observation number \\
      \> {\it n2} \> last observation number (defaults to {\it n1}\,)
\end{tabbing}

\item [NOTES]:

If the arguments are omitted, all observations are MASKed.

The inverse command is UNMASK.

Azimuths lie within the range 0 to 360 degrees, reckoned
from north through east.

\goodbreak
Observation numbers:
\begin{itemize}
\item start at 1 and refer to all observations, whether or not
      currently active;
\item can be given in either order;
\item can be less than 1 or greater than the current number of
      observations without ill effect.
\end{itemize}

\item [EXAMPLES]:
\begin{tabbing}
XXX \= XXXXXXXXXXXXXXX \= \kill
\> {\tt MASK H L 60} \> makes inactive all observations where \\
\>                   \> hour angle is less than 60 degrees (4 hours) \\ \\
\> {\tt MASK Z G 20} \> makes inactive all observations where zenith \\
\>                   \> distance is greater than 20 degrees (elevation \\
\>                   \> less than 70 degrees) \\ \\
\> {\tt MASK N G 20} \> makes inactive all but the first 20 observations \\ \\
\> {\tt MASK 3}      \> makes inactive the 3rd observation \\ \\
\> {\tt MASK 1 25}   \> makes inactive the first 25 observations \\ \\
\> {\tt MASK}        \> makes all the observations inactive
\end{tabbing}

\end{description}

%--------------------------------------------------------------------

\goodbreak
\rule{\textwidth}{0.3mm}
{\Large {\bf OUTDAT} \hfill Write file for INDAT \hfill {\bf OUTDAT}}
\begin{description}
\item [FUNCTION]:

OUTDAT outputs the current pointing data as a file
that can be read in by INDAT.

\item [COMMAND]:

\begin{cmd}
\> \> OUTDAT {\it file}
\end{cmd}

where {\it file}\, is the name of the file to be written.

\item [NOTES]:

Only active observations (those
not rendered inactive via the MASK command) are output.

The file which is written does not contain true star right
ascensions.  Zero sidereal time and dummy right ascensions
equal to minus the hour angle are used instead.

\end{description}

%--------------------------------------------------------------------

\goodbreak
\rule{\textwidth}{0.3mm}
{\Large {\bf OUTMOD} \hfill Write model to a file \hfill {\bf OUTMOD}}
\begin{description}
\item [FUNCTION]:

OUTMOD outputs the current model to a file.

\item [COMMAND]:

\begin{cmd}
\> \> OUTMOD {\it file}
\end{cmd}

where {\it file}\, is the name of the file to be written.

\item [NOTES]:

The file can be read by means of the INMOD command.

As well as writing the model information itself, OUTMOD
stores the caption, number of stars, and RMS for
the current set of observations.

\end{description}

%--------------------------------------------------------------------

\goodbreak
\rule{\textwidth}{0.3mm}
{\Large {\bf PARAL} \hfill Apply Terms in Parallel \hfill {\bf PARAL}}
\begin{description}
\item [FUNCTION]:
PARAL is used to indicate which terms are
to be evaluated in parallel to the immediately preceding term in the
model.  This allows several consecutive terms to use the
same position as their starting point, rather than each
successively adjusting the position before it is used by
the next term.

\item [COMMAND]:

\begin{cmd}
\> \> PARAL {\it coeff1 coeff2 etc.}
\end{cmd}

where the arguments are the names of the terms to
share starting points with their predecessors.

\item [NOTES]:

If no arguments are supplied, all the terms in the model
use the same starting point -- the initial, unadjusted
position.

If any argument is unrecognized, the entire command is
rejected.

The starting position for a term named in this command will
be the adjusted position produced by the last term which is
`chained' rather than `parallel'.  Thus the first term in
a group of terms sharing a common starting point will, in
fact, be `chained'.

The converse command is CHAIN.

\end{description}

%--------------------------------------------------------------------

\goodbreak
\rule{\textwidth}{0.3mm}
{\Large {\bf PENS} \hfill Specify Pens \hfill {\bf PENS}}
\begin{description}
\item [FUNCTION]:

PENS specifies and reports the pens used for
plotting on the current workstation.

\item [COMMAND]:

\begin{cmd}
\> \> PENS {\it f c a l p o}
\end{cmd}

where the arguments are the SGS pen numbers for:

\begin{tabs}
XXXXXX \= XXXXXX \= \kill
\> f \> frame \\
\> c \> caption \\
\> a \> axes \\
\> l \> labels \\
\> p \> points which are valid \\
\> o \> points which are off-scale
\end{tabs}

\item [NOTES]:

Arguments default to the existing value.  If any argument
is invalid none of the pens are changed.

The actual colours and line styles are dependent on the
implementation of the GKS graphics package being used.
Note that although coloured pens may be used to produce
both coloured lines and text, pens giving dotted or broad
lines may not produce analogous effects on text.

\end{description}

%--------------------------------------------------------------------

\goodbreak
\rule{\textwidth}{0.3mm}
{\Large {\bf PERFCT} \hfill Create Ideal Observations \hfill {\bf PERFCT}}
\begin{description}
\item [FUNCTION]:

PERFCT creates a list of artificial error-free
pointing observations.

\item [COMMAND]:

\begin{cmd}
\> \> PERFCT $\phi$
\end{cmd}

where $\phi$ is the telescope latitude to which the artificial
observations will correspond, given as degrees, arcminutes,
arcseconds.

\item [NOTE]:

The arcminutes and/or arcseconds may be omitted.  If the
entire argument is omitted
the latitude found in any existing data list
is used.  If the data list is empty, latitude zero is
used.

\end{description}

%--------------------------------------------------------------------

\goodbreak
\rule{\textwidth}{0.3mm}
{\Large {\bf PLTOFF} \hfill Close Graphics Device \hfill {\bf PLTOFF}}
\begin{description}
\item [FUNCTION]:

PLTOFF stops plotting on the current graphics
workstation and reverts to the previous one (if any).

\end{description}

%--------------------------------------------------------------------

\goodbreak
\rule{\textwidth}{0.3mm}
{\Large {\bf PLTON} \hfill Open Graphics Device \hfill {\bf PLTON}}
\begin{description}
\item [FUNCTION]:

PLTON stops plotting on the current graphics
workstation (if any) and prepares to plot on a new one.

\item [COMMAND]:

\begin{cmd}
\> \> PLTON {\it device}
\end{cmd}

where {\it device}\, is the SGS workstation name.

\item [NOTES]:

To obtain a list of the SGS workstation names supported on
your machine, type PLTON with no argument.

A GKS workstation
type is acceptable in lieu of an SGS name.

\end{description}

%--------------------------------------------------------------------

\goodbreak
\rule{\textwidth}{0.3mm}
{\Large {\bf PLTZ} \hfill Select a Plotting Zone \hfill {\bf PLTZ}}
\begin{description}
\item [FUNCTION]:

PLTZ selects a plotting zone.

\item [COMMAND]:

\begin{cmd}
\> \> PLTZ \> use full display surface \\
\> or \> PLTZ {\it name} \> use named region \\
\> or \> PLTZ {\it x1 x2 y1 y2} \> use numerically specified region
\end{cmd}

where {\it name}\, is the zone name (see below) and {\it x1 x2 y1 y2}\,
specify the zone extent directly.

\item [NOTES]:

The zone {\it name}\, refers to a range of regular subdivisions of
the display surface.  For $3 \times 3$
partitioning, the names are TL, TC, TR,
CL, CC, CR, BL, BC, BR (T=top, C=centre, B=bottom, L=left, R=right).
For $2 \times 2$ the names are TLQ, TRQ, BLQ,
BRQ.  For $2 \times 1$ the names
are T, B, L, R.

If numeric parameters are supplied (all four are required) they
specify the X range and Y range
in a coordinate system where the
display surface is a unit square with its origin at
the bottom left-hand corner.

\end{description}

%--------------------------------------------------------------------

\goodbreak
\rule{\textwidth}{0.3mm}
{\Large {\bf REPLEN} \hfill Specify Report Length \hfill {\bf REPLEN}}
\begin{description}
\item [FUNCTION]:

REPLEN selects the report length option.

\item [COMMAND]:

\begin{cmd}
\> \> REPLEN {\it option}
\end{cmd}

where {\it option}\, is either S for short or L for long.

\item [NOTES]:

The default option is L.

REPLEN affects the volume of output from commands such as
INDAT and SLIST.

\end{description}

%--------------------------------------------------------------------

\goodbreak
\rule{\textwidth}{0.3mm}
{\Large {\bf RESET} \hfill Zero the Coefficients \hfill {\bf RESET}}
\begin{description}
\item [FUNCTION]:

RESET sets all pointing coefficients to zero and
removes all pointing corrections from the data list.

\end{description}

%--------------------------------------------------------------------

\goodbreak
\rule{\textwidth}{0.3mm}
{\Large {\bf RETURN} \hfill Return from a Procedure \hfill {\bf RETURN}}
\begin{description}
\item [FUNCTION]:

RETURN returns from a library procedure.

\item [NOTE]:

A RETURN command issued from the command device results in an
error message.

\end{description}

%--------------------------------------------------------------------

\goodbreak
\rule{\textwidth}{0.3mm}
{\Large {\bf SETCAP} \hfill Set the Caption \hfill {\bf SETCAP}}
\begin{description}
\item [FUNCTION]:

SETCAP sets the caption string.

\item [COMMAND]:

\begin{cmd}
\> \> SETCAP {\it string}
\end{cmd}

where {\it string}\, is the required caption, and consists of
either the whole of the command line following the command
name but with leading and trailing blanks eliminated or,
if the first non-space character following the command name
is a $'$ or $''$ delimiter, the string between that delimiter and
the next appearance of the same one (unless the end of the
command is reached first).

\item [NOTES]:

If no argument is supplied the caption is set to a single space.

If the caption contains lowercase characters, enclose it in
single or double quotes.

\end{description}

%--------------------------------------------------------------------

\goodbreak
\rule{\textwidth}{0.3mm}
{\Large {\bf SHOW} \hfill Display Parameters \hfill {\bf SHOW}}
\begin{description}
\item [FUNCTION]:

SHOW displays and logs the current values of various TPOINT
internal parameters.

\item [NOTE]:

The output from SHOW is a useful guide to some of the features
of TPOINT, and includes the names of the commands used for changing
the parameters listed.

\end{description}

%--------------------------------------------------------------------

\goodbreak
\rule{\textwidth}{0.3mm}
{\Large {\bf SLIST} \hfill List the Observations \hfill {\bf SLIST}}
\begin{description}
\item [FUNCTION]:

SLIST lists the pointing observations and the current
residuals.

\item [NOTES]:

The SLIST output includes the residuals in various directions
rather than simply in radius.

If just the RMS residuals are needed the command \fstring{REPLEN S} may
be used to suppress reporting the individual observations.

The azimuths are reckoned from north through east.

\end{description}

%--------------------------------------------------------------------

\goodbreak
\rule{\textwidth}{0.3mm}
{\Large {\bf SPAWN} \hfill Execute a Shell Command \hfill {\bf SPAWN}}
\begin{description}
\item [FUNCTION]:

SPAWN executes one or more shell commands (DCL on VAX/VMS).

\item [COMMAND]:

\begin{cmd}
\> \> SPAWN {\it string}
\end{cmd}

where {\it string}\, is the required shell command, and consists of
either the whole of the TPOINT command line following the \fstring{SPAWN}
name but with leading and trailing blanks eliminated or,
if the first non-space character following the command name
is a $'$ or $''$ delimiter, the string between that delimiter and
the next appearance of the same one (unless the end of the
command is reached first).

\item [NOTE]:

If no argument is supplied a new shell is spawned,
allowing a sequence
of shell commands then to be entered.  To return to TPOINT, issue
a ``logout'' command.

\end{description}

%--------------------------------------------------------------------

\goodbreak
\rule{\textwidth}{0.3mm}
{\Large {\bf TXF} \hfill Specify Text Font \hfill {\bf TXF}}
\begin{description}
\item [FUNCTION]:

TXF specifies and reports the fonts for text
plotting on the current workstation.

\item [COMMAND]:

\begin{cmd}
\> \> TXF {\it c l}
\end{cmd}

where the arguments are the SGS font numbers for:

\begin{tabs}
XXXXXX \= XXXXXX \= \kill
\> c \> caption \\
\> l \> labels
\end{tabs}

\item [NOTES]:

Arguments default to the existing value.  If any
argument is invalid (the font numbers must be greater
than zero) none of the fonts are changed.

The actual fonts depend on the implementation of the
GKS graphics kernel being used.  See local GKS documentation
for details.

\end{description}

%--------------------------------------------------------------------

\goodbreak
\rule{\textwidth}{0.3mm}
{\Large {\bf TXP} \hfill Specify Text Precision \hfill {\bf TXP}}
\begin{description}
\item [FUNCTION]:

TXP specifies and reports the text precision
for the current workstation.

\item [COMMAND]:

\begin{cmd}
\> \> TXP {\it n}
\end{cmd}

where {\it n}\, is the SGS precision:

\begin{tabs}
XXXXXX \= XXXXXX \= \kill
\> {\it n}\, = 0 \> STRING -- string size and orientation may not be correct \\
\> {\it n}\, = 1 \> CHAR -- character size and orientation may not be correct \\
\> {\it n}\, = 2 \> STROKE -- accurate as possible
\end{tabs}

\item [NOTES]:

Precision 0 typically uses hardware character generation and is
fast.  Precision 2 typically uses software character generation
and is accurate.

If no argument is supplied, the current precision is reported
but not changed.

\end{description}

%--------------------------------------------------------------------

\goodbreak
\rule{\textwidth}{0.3mm}
{\Large {\bf UNFIT} \hfill Apply Model in Reverse \hfill {\bf UNFIT}}
\begin{description}
\item [FUNCTION]:

UNFIT applies a pointing model in reverse,
taking the current adjusted positions and predicting
the observations that produce those adjusted positions
when the model is applied.

\item [COMMAND]:

\begin{cmd}
\> \> UNFIT \> to apply the current model \\
\> or \> UNFIT N \> to apply a null model (coeffs all zero)
\end{cmd}

\item [NOTES]:

If the star to telescope modelling option has been chosen,
(via the ADJUST command) the star positions are set and the
telescope positions left alone.  If the telescope to star
option has been chosen, the reverse is true.

A common use for the UNFIT command
is where a canonical model is
applied to a set of pointing residuals to yield
`observations' which can be combined for further
analysis.

For further details, see the section {\it Combining and Saving
Processed Data}, earlier.

\end{description}

%--------------------------------------------------------------------

\goodbreak
\rule{\textwidth}{0.3mm}
{\Large {\bf UNMASK} \hfill Flag Selected Observations Active \hfill
                                                         {\bf UNMASK}}
\begin{description}
\item [FUNCTION]:

UNMASK flags selected pointing observations active.  The selection
is made by area of sky, total residual, or sequence number.

\item [COMMAND]:

\begin{cmd}
\> \> UNMASK {\it quantity  condition  value } \\
\> or \> UNMASK {\it n1  n2 }
\end{cmd}

where the arguments are as follows:
\begin{tabbing}
XXXXXX \= XXXXXXXXX \= \kill
\> {\it quantity}  \> H (HA), D (Dec), A (Az), Z (ZD), \\
\>                 \> R (radial error) or N (observation number) \\
\> {\it condition} \> L (less than) or G (greater than) \\
\> {\it value}     \> cutoff value (degrees for H, D, A or Z; \\
\>                 \> arcseconds for R; number for N) \\ \\
\' or \> {\it n1}  \> first observation number \\
      \> {\it n2}  \> last observation number (defaults to {\it n1}\,)
\end{tabbing}

\item [NOTES]:

If the arguments are omitted, all observations are UNMASKed.

The inverse command is MASK.

Azimuths lie within the range 0 to 360 degrees, reckoned
from north through east.

\goodbreak
Observation numbers:
\begin{itemize}
\item start at 1 and refer to all observations, whether or not
      currently active;
\item can be given in either order;
\item can be less than 1 or greater than the current number of
      observations without ill effect.
\end{itemize}

\item [EXAMPLES]:
\begin{tabbing}
XXX \= XXXXXXXXXXXXXXX \= \kill
\> {\tt UNMASK H L 60} \> makes active all observations where \\
\>                     \> hour angle is less than 60 degrees (4 hours) \\ \\
\> {\tt UNMASK Z G 20} \> makes active all observations where zenith \\
\>                     \> distance is greater than 20 degrees (elevation \\
\>                     \> less than 70 degrees) \\ \\
\> {\tt UNMASK N G 20} \> makes active all but the first 20 observations \\ \\
\> {\tt UNMASK 3}      \> makes active the 3rd observation \\ \\
\> {\tt UNMASK 1 25}   \> makes active the first 25 observations \\ \\
\> {\tt UNMASK}        \> makes all the observations active
\end{tabbing}

\end{description}

%--------------------------------------------------------------------

\goodbreak
\rule{\textwidth}{0.3mm}
{\Large {\bf USE} \hfill Include Terms in Model \hfill {\bf USE}}
\begin{description}
\item [FUNCTION]:

USE includes one or more terms in the
pointing model, and flags it or them to be fitted.

\item [COMMAND]:

\begin{cmd}
\> \> USE {\it coeff1 coeff2 etc.}
\end{cmd}

where the arguments are the names of the terms to be
included in the model.

\item [NOTES]:

Any term already in the model is simply flagged to
be fitted; in this case USE cancels a prior FIX.

If no arguments are supplied, all terms currently in
the model are flagged to be fitted.

If any argument is unrecognized, the entire command is rejected.

\end{description}

%--------------------------------------------------------------------

\goodbreak
\rule{\textwidth}{0.3mm}
{\Large {\bf VT} \hfill Declare terminal VT100 \hfill {\bf VT}}
\begin{description}
\item [FUNCTION]:

VT declares whether the terminal is VT100-compatible and, if so,
specifies the scrolling region.

\item [COMMAND]:

\begin{cmd}
\> \> VT {\it top bottom} \> to declare VT100 and specify scrolling region \\
\> or \> VT {\it N} \> to declare non-VT100
\end{cmd}

where the scrolling region extends from line
{\it top}\, to line {\it bottom}\,
inclusive.  On a VT100 ({\it etc.}) the top line is line 1, and the bottom
line is line 24.  The specified region must be at least two lines
in extent.  The \fstring{N} option declares the screen to be
non-VT100.

\item [NOTES]:

Bottom defaults to 24 and top defaults to 1.  Thus if only the
first argument is given the scrolling region extends down to the
bottom of the screen, and if no arguments are given the whole
screen is used.

The screen is cleared each time this command is used, unless the
`N' option is specified.  The VT command without arguments is thus
a useful way of de-cluttering the screen on terminals which have
independent but superimposed graphics and alpha planes.

The effect of using this command to declare that the screen is
VT100-compatible cannot be predicted on
non-VT100-compatible terminals.  It is up to the user
to know whether the terminal is VT100-compatible.

\end{description}
\rule{\textwidth}{0.3mm}

%--------------------------------------------------------------------

\newpage
\vspace*{-10ex}
\vbox{
\section{COMMAND SUMMARY}
\vspace*{-1ex}
\begin{small}
\begin{tabs}
XXXXXXXXXXXXXXX \= XXXXXXXXXXXXXXXXXXXXXXXXXX \= \kill
COMMAND \> FUNCTION \> DEFAULT \\
{\it coeff value} \> set and report coefficient value \>
                        report only \\
{\tt ADJUST T {\rm or} S} \> select model direction \>
                        report only \\
{\tt APPEND ON {\rm or} OFF} \> next observations will [not] be appended \>
                        will be appended \\
{\tt CALL} {\it proc} \> call library procedure {\it proc} \> - \\
{\tt CAPT ON {\rm or} OFF} \> enable/disable plotting of captions \>
                        captions enabled \\
{\tt CHAIN {\it c1 c2 etc.}} \> selected terms are applied
                        sequentially \> whole model chained \\
{\tt CLIST} \> list current coefficients \> - \\
{\tt CLRNG ON {\rm or} OFF} \>  enable/disable screen/zone clearing \>
                        clearing enabled \\
{\tt ECHO ON {\rm or} OFF} \> enable/disable procedure command echo \>
                        echo on \\
{\tt END} \> exit \> - \\
{\tt FIT N} \> fit [or apply model without fitting] \>
                        fit \\
{\tt FITTOL} \> specify and report ill-conditioning tolerance \>
                        report only \\
{\tt FIX} {\it c1 c2 etc.} \> exclude selected terms from fit \>
                        fix all terms \\
{\tt FRAME ON {\rm or} OFF} \> enable/disable plotting of frames \>
                        frames enabled \\
{\tt G} {\it y x s} \> plot residuals \> autoscale \\
{\tt GAM} {\it t s} \> look for axis misaligment \> equatorial, autoscale \\
{\tt GCLR} \> clear display surface \> - \\
{\tt GDIST} \> plot distributions of residuals \> - \\
{\tt GHYST} {\it t s} \> look for hysteresis \> equatorial, autoscale \\
{\tt GMAP} {\it t s} \> Cartesian Cylindrical plot \> equatorial, autoscale \\
{\tt GSCAT} {\it t r} \> scatter plot \> equatorial, autoscale \\
{\tt GSMAP} {\it s} \> orthographic map \> autoscale \\
{\tt HELP} {\it topic} \> enter HELP session \> enter at top level \\
{\tt INDAT} {\it file} \> read file of observations \> - \\
{\tt INMOD} {\it file} \> read model from a file \> - \\
{\tt INPRO} {\it file} \> read library file \> original library \\
{\tt INST} {\it file} \> read star catalogue \> original catalogue \\
{\tt LOSE} {\it c1 c2 etc.} \> remove selected terms from model \>
                        discard all terms \\
{\tt MARKH} {\it h} \> set marker height \> 0.2 \\
{\tt MASK} {\it q c v}\, or {\it n1 n2} \>
                        flag selected observations inactive \>
                        mask all observations \\
{\tt OUTDAT} {\it file} \> write file of observations \> - \\
{\tt OUTMOD} {\it file} \> write model to a file \> - \\
{\tt PARAL {\it c1 c2 etc.}} \> terms
                        {\it c1 c2 etc.}\ are applied in parallel \>
                        whole model parallel \\
{\tt PENS} {\it f c a l p o} \> specify and report pens \>
                        report only \\
{\tt PERFCT} {\it $^\circ$ $'$ $''$} \> create ideal observations \>
                        $\phi$ from data list \\
{\tt PLTOFF} \> revert to previous graphics device \> - \\
{\tt PLTON} {\it device} \> open new graphics device \>
                        list of devices displayed \\
{\tt PLTZ} {\it z} \> select plotting zone \>
                        use whole display surface \\
{\tt PROC} {\it x} \> start of library procedure {\it x} \> - \\
{\tt REPLEN S {\rm or} L} \> specify report length option \>
                      full length reports \\
{\tt RESET} \> zero coefficients and cancel corrections \> - \\
{\tt RETURN} \> return from a procedure \> - \\
{\tt SETCAP} {\it string} \> specify caption \> caption blank \\
{\tt SHOW} \> display system parameters \> - \\
{\tt SLIST} \> list the observations \> - \\
{\tt SPAWN} {\it string} \> execute shell command \> spawn new shell \\
{\tt TXF} {\it c l} \> specify text fonts \> report only \\
{\tt TXP} {\it n} \> specify text precision \> report only \\
{\tt UNFIT N} \> apply pointing model in reverse \>
                        current model applied \\
{\tt UNMASK} {\it q c\, v} or {\it n1 n2} \>
                        flag selected observations active \>
                        unmask all observations \\
{\tt USE} {\it c1 c2 etc.} \> include selected terms in model \>
                                unfix all terms \\
{\tt VT {\it t b} {\rm or} N} \> declare whether terminal VT100 \>
                        VT100, whole screen \\
{\tt CTRL/C} \> abort current command \>
\end{tabs}
\end{small}
}
\end{document}
