\documentstyle{article}
\pagestyle{myheadings}

%------------------------------------------------------------------------------
\newcommand{\stardoccategory}  {Starlink User Note}
\newcommand{\stardocinitials}  {SUN}
\newcommand{\stardocnumber}    {70.11}
\newcommand{\stardocauthors}   {H J Walker, J H Fairclough, M D Lawden}
\newcommand{\stardocdate}      {1 Oct 1990}
\newcommand{\stardoctitle}     {SCAR --- Starlink Catalogue Access and Reporting}
%------------------------------------------------------------------------------

\newcommand{\stardocname}{\stardocinitials /\stardocnumber}
\markright{\stardocname}
\setlength{\textwidth}{160mm}
\setlength{\textheight}{240mm}
\setlength{\topmargin}{-5mm}
\setlength{\oddsidemargin}{0mm}
\setlength{\evensidemargin}{0mm}
\setlength{\parindent}{0mm}
\setlength{\parskip}{\medskipamount}
\setlength{\unitlength}{1mm}

\begin{document}
\thispagestyle{empty}
SCIENCE \& ENGINEERING RESEARCH COUNCIL \hfill \stardocname\\
RUTHERFORD APPLETON LABORATORY\\
{\large\bf Starlink Project\\}
{\large\bf \stardoccategory\ \stardocnumber}
\begin{flushright}
\stardocauthors\\
\stardocdate
\end{flushright}
\vspace{-4mm}
\rule{\textwidth}{0.5mm}
\vspace{5mm}
\begin{center}
{\Large\bf \stardoctitle}
\end{center}
\vspace{5mm}

\setlength{\parskip}{0mm}
\tableofcontents
\setlength{\parskip}{\medskipamount}
\markright{\stardocname}
\newpage

\section {How to Use this Document}

The SCAR Beginners Guide (SUN/106) and the associated demonstration should allow
beginners to use SCAR for straight forward catalogue tasks. Problems can
often be resolved by using the online help facility, in particular replying
with a '?' to any prompt will give further help about that parameter. This
document gives complete information about the CAR routines, parameters and
the QCAR language. The appropriate section of this document should be read
by any user
experiencing a problem when running a CAR routine. It will also be useful for
advanced users who want to
do more complex catalogue tasks. For convienience the entries in the CAR
routines section and in the parameters section are alphabetically ordered.

\section {Introduction}

The Starlink Catalogue Access and Reporting system (SCAR) is a relational
database management system for astronomical catalogues.
Whilst it is primarily designed for the processing of astronomical catalogue
data, such as the IRAS catalogue, it can be used for processing any category of
relational data.
SCAR has the capability to extract data from the requested catalogue using input
criteria, to manipulate it using various statistical and plotting routines,
to output data from the catalogue, and to assimilate new catalogues.
This new version of the SUN document refers specifically to SCAR version 5.3,
which is run in the ADAM environment (SUN/94).

The SCAR system comprises four main areas:
\begin{itemize}
\item The SCAR routines.
\item The FACTS system. SCAR uses the Flexible Astronomical Catalogue Transport
system (FACTS) format for the storage of astronomical catalogues.
FACTS was formerly called ADC (Astronomical Data Catalogue), but this has been
changed as it clashes with the acronym for the Astronomical Data Center, in
the United States.
Throughout this document ADC and FACTS are synonymous.
\item The online database
\item The offline database
\end{itemize}
First time users should consult SUN/106 which covers getting started and
introduces users to the facilities available in SCAR.
Section 3 describes, in alphabetical order, the CAR routines to access and
manipulate the catalogues available on-line. This is followed by the parameter
definitions
and information about the QCAR language.
The FACTS sytem is described in section 4.
The online database is resident on the Starlink database microvax (STADAT)
at the RAL Starlink node; this is described in section 5.

The offline database consists of approximately 800 files of catalogue data
obtained from the Stellar Data Centre, in Strasbourg.
These files are kept on tapes at the RAL Starlink node.
Rutherford Appleton Laboratory is a distributor of the catalogue data.
The offline database is described in SUN/79.

New versions of SCAR will appear from time to time.
All software changes are documented in SCARDOC1.TEX in SCAR\_DOC\_DIR.

Many people have contributed to the SCAR system, including S Wright,
J Fairclough, A Hobson, J Day, E Gershuny, B Stewart, J Cooke, S Leggett,
D Kelly, C Davenhall, C Singh, D. Giaretta, A. Wood and it's maintenance
is the responsibility of IPMAF.


\section {CAR routines, Parameters and QCAR Language}

This section details the CAR routines, so you can see the various options
available.
To understand more about the input fields, or to do complex queries you may
need to refer to the following section.
There are details of individual parameters given in a separate section.
SCAR takes as default the current directory, so if you
wish to access your own catalogues in a different directory, you can either
move directories
or define a logical to point at the correct directory
for both the data and DSCF files:
\begin{verbatim}
    ICL> DCL DEFINE/JOB MYCAT DISK$STARDATA:[ME.DIRECTORY]CAT.DAT
    ICL> DCL DEFINE/JOB DSCFMYCAT DISK$STARDATA:[ME.DIRECTORY]DSCFCAT.DAT
\end{verbatim}


\subsection {CAR Routines Summary}
CAR procedures are of four types:
\begin{enumerate}
\item Procedures that perform database management functions, {\em eg.} extract
sources from the catalogue.
\item Procedures that manipulate and display catalogue data.
\item Procedures that perform statistics.
\item Procedures that create and convert description files, {\em eg.} to change
coordinate systems.
\end{enumerate}
Procedures can be identified on the help menu by the topic names in UPPERCASE in
the CAR\_HELP library.
You run the program by prefixing the program name with the package name,
eg.\ CAR\_SEARCH.
Detailed specifications for all these programs are given below.
\subsubsection {Database Management Functions}
You can do the following to a data catalogue (omitting the CAR\_):
\begin{description}
\begin{description}
\item [SEARCH] it to select subsets.
\item [SORT] it to reorder the catalogue.
\item [JOIN] it with another catalogue to find objects in common.
\item [DIFFERence] it with another catalogue to find objects not in common.
\item [MERGE] it with another catalogue to add them together.
\item [SPLIT] it into two catalogues.
\item [WITHIN] a polygon, select objects from it.
\item [CONVERT] it to another format, make indexes, new fields etc.
\item [EDIT] it to add and delete records and correct values.
\end{description}
\end{description}
\subsubsection {Display Catalogue Data}
In order to study the contents of a catalogue you can:
\begin{description}
\begin{description}
\item [REPORT] it to list all or some of its contents.
\item [LISTOUT] the values of a given field.
\item [PRINT] it to a default file.
\item [CALCulate] new fields for it.
\item [RECALCulate] to update existing fields.
\item [CONVERT] it to another format.
\item [WRAP] producing a printable version of a catalogue with records longer
than 132 characters.
\item [CHART] objects in it by plotting on a finding chart.
\item [AITOFF] plot all the objects on an all-sky plot.
\item [IMAGEPLOT] objects in it for COSMOS style processing.
\end{description}
\end{description}
\subsubsection {Perform Statistics}
In order to generate statistics on the contents of a catalogue you can use:
\begin{description}
\begin{description}
\item [LITTLEBIG] to find either the largest or smallest numbers in the
  catalogue.
\item [SAMPLE] to select every Nth object from the catalogue.
\item [HISTOGRAM] to plot a histogram of a given field.
\item [SCATTER] to plot two fields, and perform regression analysis.
\item [CORRELATE] to correlate (non-parametrically) two fields.
\item [LINCOR] to compute the Pearson product-moment linear correlation
  coefficient.
\end{description}
\end{description}
\subsubsection {Create and Process Description Files}
\begin{description}
\begin{description}
\item [EXTAPE] -- Examines the contents of an ASCII tape and dumps it to disk.
\item [LISTIN] -- Reads a simple VMS file into SCAR.
\item [FORM1] -- Creates a description file in screen mode.
\item [FORM2] -- Creates a description file in prompt mode.
\item [ASPIC] -- Creates a description file for an ASPIC/IAM catalogue.
\item [ASCII] -- Converts a description file for BINARY data to one for ASCII
 data.
\item [BINARY] -- Converts a description file for ASCII data to one for BINARY
 data.
\item [POLYGON] -- Creates a description file of polygon vertices.
\end{description}
\end{description}
\subsubsection {Small Tools}
\begin{description}
\begin{description}
\item [SETUP] -- Defines default values for certain options.
\item [CATSIZE] -- Finds the number of objects in a catalogue.
\item [COUNTREC] -- Counts the number of objects in a sequential file.
\item [COSMAGCAL] -- Calculates magnitudes from COSMOS magnitudes.
\item [HARDCOPY] -- Produces hardcopy of output produced by graphics programs.
\item [MOUNT] -- Allocates a deck, mounts a foreign tape and assigns it a
logical name.
\item [DISMOUNT] -- Dismounts a tape, deallocates a deck and deassigns the
logical name.
\item [DSCFHELP] -- Inserts a description file into a help library.
\item [DEBUG] -- Switches on the VMS debugger (for Programmers only).
\item [GETPAR] -- Get a parameter value when in the ICL environment.
\end{description}
\end{description}


\subsection {CAR Routines}

\subsubsection{CAR\_AITOFF}

Plots catalogue objects in the all-sky Aitoff projection

A complete sky plot is made, in equal area projection, with the sources
plotted either as points or small symbols.
Aitoff plots can be in any one of the four coordinate systems:
equatorial, ecliptic, galactic and supergalactic.
Your selected catalogue must contain the necessary fields to make the plot.
If they are absent, generate them first with CALC.

Equatorial fields are RA and DEC, galactic are GLONG and GLAT,
supergalactic are SGLONG and SGLAT, and ecliptic are ELONG
and ELAT.

You can control the Y axis scale, (YSCALE), aspect ratio (AR or
XSCALE/YSCALE) and origin location (XSHIFT and YSHIFT).
Values of YSCALE=1, AR=1, XSHIFT=0 and YSHIFT=0 will give you a plot
that fills the device screen.
All the options are repeated when the plot is redrawn, except the input
catalogue.
The symbol required should be typed in, in response to the marker query.

\begin{verbatim}
    Invocation: ICL> CAR_AITOFF
      Parameter list:
       inputs          = Name(s) of input catalogue(s), e.g. [IRP1,IRP2]
       coordinates     = Coordinate system code 1=Equ,2=Gal,3=Supgal,4=Ecl
       title           = Name or short description for file or plot
       yscale          = Y axis scale factor
       ar              = The aspect ratio of X/Y
       xshift          = X axis origin
       yshift          = Y axis origin
       marker          = Marker character one of .,+,o,x
       device          = Output device (HELP lists devices)
       redraw          = Redraw the diagram? - yes or no
\end{verbatim}

\subsubsection{CAR\_ASCII}

Converts a description file to specify ascii data (formatted)

The ASCII command converts a description file for a catalogue  of
unformatted  data  to a description file for a catalogue of formatted
data.
You supply as input the name of the description file for the
binary data (e.g: DSCFIRPS) and as output the name of a description
file for the ascii data (e.g: DSCFIRPSF).
You will be asked if you want spaces between the fields.
ASCII is the inverse of BINARY.

\begin{verbatim}
    Invocation: ICL> CAR_ASCII
      Parameter list:
       input           = Name of input catalogue
       output          = Name of output catalogue
       spaces          = Do you want spaces between fields?
\end{verbatim}

\subsubsection{CAR\_ASPIC}

Creates a description file for an ASPIC/IAM catalogue

ASPIC creates an ADC description file suitable for use with a
catalogue created from the output of the IAM version.
You are prompted for the name of the output catalogue.

A standard catalogue for holding data from the ASPIC/IAM version
has the following fields;
\begin{verbatim}
    UXCEN        Unweighted X centroid                pixels
    UYCEN            "      Y    "                      "
    UMAJAX       Unweighted semi-major axis             "
    UMINAX           "      semi-minor  "               "
    UTHETA           "      orientation               degrees
    IXCEN        Intensity weighted X centroid        pixels
    IYCEN            "        "     Y    "              "
    IMAJAX       Intensity weighted semi-major axis     "
    IMINAX           "        "     semi-minor  "       "
    ITHETA           "        "     orientation       degrees
    AREA         Image area                           pixels
    MAG          Total magnitude                      magnitudes
    MAXINT       Maximum intensity in image
    XMIN         X minimum pixel in image             pixels
    XMAX         X maximum   "   "    "                 "
    YMIN         Y minimum   "   "    "                 "
    YMAX         Y maximum   "   "    "                 "
    CLASS        Often used as a classification parameter
    SPARE1       Unused, user definable
    SPARE2         "      "       "
 \end{verbatim}
 All fields are of type REAL, except `CLASS', which is INTEGER.\\
 Notes\\
 (1)  ellipticity = 1.0 -- (b/a)\\
      where   a = semi-major axis and b = semi-minor axis.\\
 (2)  This set of fields is different to the set used for COSMOS data.
\begin{verbatim}
    Invocation: ICL> CAR_ASPIC
      Parameter list:
       output          = Name of output catalogue
       title           = Name or short description for file or plot
\end{verbatim}

\subsubsection{CAR\_BINARY}

Converts description of ascii data to a description of binary data (unformatted)

The BINARY command converts a description file for a catalogue of ascii
(formatted) data to a description file for a catalogue of unformatted data.
You supply as input the name of description file
for the formatted data (e.g: DSCFIRPSF) and as output the name for
the description file of unformatted data (e.g: DSCFIRPS).
BINARY is the inverse of ASCII.

\begin{description}
\item [Binary and nullvalues] --
The routine changes null values according to the format and null format,
which means they are changed to the binary internal format null value,
ie.\ several blanks can be changed to a zero.
You should check that BINARY has produced the null values you want before using
CONVERT.
\end{description}

\begin{verbatim}
    Invocation: ICL> CAR_BINARY
      Parameter list:
       input           = Name of input catalogue
       output          = Name of output catalogue
       isam            = ISAM (Y) or Direct access (N) file?
\end{verbatim}

\subsubsection{CAR\_CALC}

Allows you to generate and display new fields for a catalogue

The OUTPUT catalogue contains the keys and pointers to the INPUT
plus new columns for each new field you define.
The output file always overwrites any existing file of the same name.
You are really creating an index to the ``real'' catalogue, with values for
your new columns.
After opening files CALC prompts you with KEYWORD:
possible answers to this are P, to define a new parameter (e.g: EQUINOX),
F, to define a new field, and E to signal that you have no more fields
or parameters to define.
The new field to be calculated is input as a QCAR expression.
You can `define' expressions that are just fields in the input catalogue.
You can use this to display the values of the fields in your expressions.

CALC calculates the new fields for the first few records
of the catalogue and then prompts you with SCROLL.
If you are satisfied with the results, you must jump to the end with the J
command.
If you are not satisfied type E, and CALC exits without calculating the
rest of the fields.
If you type +, the next page of results are shown.
You can type P to print any page.
The information on the screen is written to a file with the extension `.CAR'
and the same name as the output.

When you have finished your calculations you can generate a REPORT of
your final OUTPUT (you will see all the data) or CONVERT it back to a
simple catalogue --- that is add the new columns to your catalogue.

{\bf Units:} When adding or subtracting a constant in a QCAR expression (in
DEFINE below) you must generally specify the constant in the units of the
operand of the operator, which you can find by looking up the field(s) concerned
in the HELP information for the catalogue.
However, when it is a Radian quantity (eg RA, dec), you must present the
constant in the appropriate form e.g. HHMMSS and SDDMMSS.
If you set the BATCH parameter to 'YES' CALC will only add one new field or
parameter.

\begin{verbatim}
    Invocation: ICL> CAR_CALC
      Parameter list:
       input           = Name of input catalogue
       output          = Name of output catalogue
       title           = Name or short description for file or plot
       keyword         = Single letter keyword for command
                        (P for Parameter, F for Field, E to end definition loop)
       name            = Desired field or parameter name
       unit            = Unit of the field
       nullval         = Null value of Field or default value of Parameter
       define          = QCAR expression for evaluating field or parameter
       scroll          = + next, - last, 0 refresh, J jump, P print, H help, E end
       more            = + for more output, 0 to refresh screen, H help, P print
       batch           = Batch processing Y/N?
\end{verbatim}

\subsubsection{CAR\_CATSIZE}

Get the number of objects in a catalogue

\begin{verbatim}
    Invocation: ICL> CAR_CATSIZE
      Parameter list:
       input           = Name of input catalogue
\end{verbatim}

\subsubsection{CAR\_CHART}

Plot catalogue objects in a finding chart

CHART first asks you for the STYLE, two styles are possible:
\begin{description}
\item [Astrometric plot] -- A finding chart in tangent plane projection showing
the positions, with error boxes or ellipses, of the plotted sources.
\item [Photometric plot] -- A finding chart in tangent plane projection, showing
the sources as special symbols which indicate the type of source.
\end{description}
Tangent plane projections are performed in the equatorial coordinate system
(FK4 and FK5 conversions are permitted).

The style of the plot is controlled by many parameters.
Most of these are relatively stable, so values are set up for them in a FACTs
type file.
Rather than prompting you, CHART reads the files for their values.
These are set up for the most popular styles of plot --- all you have to know
is the name of each plate parameter set, they are:
\begin{description}
\begin{description}
\item [PSS] -- Palomar Sky Survey 2 degrees by 2 degrees in GNOMONIC projection
\item [PSS/BIG] -- Palomar Sky Survey 4 degrees by 4 degrees in GNOMONIC
  projection
\item [ESOB] -- ESO R plate 2 degrees by 2 degrees in GNOMONIC projection
\item [ESOB/BIG] -- ESO R plate 4 degrees by 4 degrees in GNOMONIC projection
\item [SRC] -- SRC J plate 2 degrees by 2 degrees in GNOMONIC projection
\item [SRC/BIG] -- SRC J plate 4 degrees by 4 degrees in GNOMONIC projection
\end{description}
\end{description}
You can store the parameters of more than one style, so it is easy to switch
plot style.
You can set up your own symbol table to control the size and shape of the
symbols plotted for each object (see STYLE in parameter subsection).
The symbol control file (SCAR\_DAT\_PATH:SYMB.DAT) defines the fields for
constructing error box/ellipse, colour codes and symbols sizes.
Plotting symbols have been set up for the IRPS, AS85, MLNS and CSIS
catalogues.
All other catalogues have a default symbol.
A key of the symbol codes is produced with each hard copy plot, and this shows
a list of each catalogue and the symbol(s) corresponding to it.
The catalogue symbol key produced after CHART is run, is only useful when a
photometric plot is requested, in the astrometric option the CODE
column may not be set.

Positioning an overlay on an optical plate is assisted if your
finding chart has background stars plotted.
You can request this by answering YES to the STARS question.
If you do want background stars, you are asked for the limiting magnitude
(MAG), 8th magnitude seems to give enough stars without overcrowding.
The background stars are taken from the RGO Astrometry Catalogue (AS85).

You are then asked for the names of the INPUT catalogues containing
the objects to be plotted on the finding chart.
There is no need to SEARCH and MERGE subsets first.
All the objects in the catalogues, within the chart dimensions, will be
selected from the catalogue and plotted on the chart.

When you opt to number the objects plotted, a summary listing is produced
of the sources on each plate to aid identification.
To aid identification of plotted objects CHART produces
a listing of the objects plotted in CHART.CAR.
You can make identification of objects easier by answering YES to the
questions about whether you want to number the INPUT catalogue
objects (NUMBER\_INPUT) or the backgound stars (NUMBER\_STARS).
Normally one answers YES to the former and NO to the latter
so that labels do not clutter up the chart.
You then have to nominate the DEVICE for plotting, typing HELP
will give you a list of devices.

When producing finding charts, several plates may be specified at once, and the
program will extract the sources covered by each plate and produce plots for
each.
You have to enter the positions of the chart centres.
If you reply TT to the CENTRES prompt, you will be asked for the
RA, DEC, Equinox, and Title of each chart.
You can create a list of plate centres and input it instead of entering field
centres interactively; this should be a FACTs style file with NAME, RA, DEC as
fields and EQUINOX as a field or parameter.
Input the file name in response to the CENTRES prompt and
the catalogue of that name will be opened and the centre positions
read from there.
Up to 30 centres are possible.
You are asked for START\_RECORD from which to begin taking centres.

\begin{verbatim}
    Invocation: ICL> CAR_CHART
      Parameter list:
       style           = Style of plot (1=Astrometric, 2=Photometric)
       plate           = Name of parameter set in PLATE catalogue
                         (eg PSS, PSS/BIG, ESOB, ESOB/BIG, SRC, SRC/BIG)
       stars           = Plot background stars? (YES/NO)
                        (Background stars taken from the RGO Astrometry catalogue)
       mag             = The limiting magnitude for background stars
                        (typically 8 gives you enough stars without overcrowding)
       inputs          = Name(s) of input catalogue(s), e.g. [IRP1,IRP2]
                         (separate names of input catalogues by a comma
                          if there is no catalogue to plot, other than the
                          finding stars, hit <RETURN>)
       number_input    = Number the INPUT catalogue objects? (YES/NO)
       number_stars    = Number background stars? (YES/NO)
       device          = Output device (HELP lists devices)
       centres         = Name of centres file (type TT for terminal input)
       start_record    = Record to start taking centres from
       ra              = RA of centre in double quotes (E to end)
       dec             = DEC of centre in double quotes (E to end)
       equinox         = Equinox for the centre
       title           = Name or short description for file or plot
       errbox          = Do you want an error box?
\end{verbatim}

\subsubsection{CAR\_CONVERT}

Convert a catalogue from one description to another

CONVERT is a general purpose program that you can use for
a wide variety of functions:
\begin{itemize}
\item Converting a catalogue from one medium to another.
The description files for both INPUT and OUTPUT catalogues must already exist.
\item Converting a catalogue from one format to another.
This is the most common function of CONVERT.
You can convert all the fields of a catalogue to binary for faster processing,
and can convert a binary file back to ascii for printing or file transfer.
\item Selecting fields of a catalogue (i.e. passing across columns of a table).
\item Creating an index to a catalogue.
If you want fast access to a catalogue on fields other than the keyfield, you
can create an index to that catalogue and then SORT the index on the field you
have passed from the catalogue to the index.
\item Converting an index to a catalogue.
If you have produced an index from a SEARCH, JOIN or MERGE operation, you can
CONVERT the index back to an ordinary catalogue, containing just the data you
want.
\item Deriving new quantities from the basic catalogue fields.
You can define new fields by specifying a functional definition of a new field
in terms of fields in the input catalogue.
As the astrometric functions (see QCAR) are built into CONVERT, you can derive
new coordinates.
\end{itemize}
Answer YES to the INDEX query, if you want to output an index.
If you have input an ordinary catalogue an index will be made.
If you input an index, the indexed information will be retained and the indexed
data inserted. If the answer to the INDEX query was NO, and the input is an
index, the index information will be removed and the index data inserted.
If you answer NO to the OPTIONS query, only the first five of the above
functions are possible, if you answer YES, all are possible.

The last three queries below (field, generic, groups) only appear, when you use
SELECT, if your terminal is not configured as VT100-like, so the full-screen
menu cannot be implemented.
The array required is input through GROUPS, e.g. IRPS fluxes are F(1) to
F(4), FQUALs are Q(1) to Q(4).
If you answer 'YES' to the SELECT and BATCH parameters the fields given by the
SELFIELDS parameter will be selected.

\begin{description}
\item [Units of position] --
When you CONVERT a catalogue from ascii to binary format, the
fields which have units declared as ANGLE or TIME are converted to RADIANS.
The description of the TIME or ANGLE field at the beginning of the comment on
the field is used to decode it.
When you CONVERT a catalogue from binary to ascii format, the
fields which have units declared as RADIANS are converted to ANGLE or TIME.
The description of the TIME or ANGLE field at the beginning of the comment on
the field is used to convert the RADIAN value.
\item [Null values] --
CONVERT does not move null values from the input to the output catalogue.
You can use this to change the null value of a field.
\item [Global index] --
When you convert a global index the fieldnames are automatically
prefixed with the catalogue mnemonic.
As 17 characters are allowed for a fieldname, it is possible that there may not
be enough room to prefix the name, especially if one of the catalogues is the
output of a previous JOIN/MERGE and CONVERT operation.
After converting a global index, edit the description file to reset any
fieldnames that have overflowed.
\end{description}

\begin{verbatim}
    Invocation: ICL> CAR_CONVERT
      Parameter list:
       input           = Name of input catalogue
                         (description file must exist)
       output          = Name of output catalogue
       options         = Do you want to specify options?
       ascii           = YES for ascii or NO for binary output
       spaces          = Do you want spaces between fields?
       index           = Ouput an Index (Y) or Master (N) catalogue?
       title           = Name or short description for file or plot
                         (used in heading reports, etc.)
       select          = NO=>all fields output, YES=>prompted for names
       batch           = Batch processing Y/N?
       selfields       = Names of selected fields eg. [RA,DEC]
    (  field           = YES or NO to select field                               )
    (  generic         = YES or NO to select generic                             )
    (  groups          = Array section e.g: A(n:m) or B(i:j,n:m)                 )
\end{verbatim}

\subsubsection{CAR\_CORRELATE}

Correlate (non-parametrically) numeric fields of a catalogue

CORRELATE computes non-parameteric correlation coefficients
between numeric fields in a catalogue, (see also LINCOR and SCATTER
which calculate the Pearson regression coefficient).
Two correlation coefficients are computed for each pair of fields,
Kendall's Tau and Spearman's Rho.
These coefficients measure the strength of any monotonic relation
between the two fields.
They both take values of +1.0 for perfect correlation, --1.0 for perfect
anti-correlation and 0.0 for no correlation, though in general the
values of the two coefficients are not equal.
These coefficients are further discussed in the NAG manual (Chapter G02)
and a more introductory discussion is given by Sprent (1981).

When CORRELATE is run you are prompted for the name of the catalogue
to be analysed.
You are given a choice of analysing all the numeric fields, or choosing
certain fields.
CORRELATE will only access fields which are of type INTEGER, REAL
or DOUBLE PRECISION.
The results are written to a file called $<$catalogue name$>$.LIS
in the current directory.
This file is suitable for printing.

Fields in SCAR catalogues can contain ``missing values" where a
measurement was not available for a particular field for a given
object.
In such cases the value of the field for the object is
set to the ``null value" defined for the field in the description
file, and these values are excluded from the correlations.
It is the responsibility of the user to ensure that the null values
chosen for the fields to be correlated do not overlap with
the range of genuine data values.
If such an overlap does occur genuine data will be excluded from
the correlation.
Because REAL and DOUBLE PRECISION values cannot be tested for equality,
a range of values within \underline{+}1.0E--13 of the null value will
be treated as missing values.

If a field to be correlated contains a missing value for a
particular object all the fields for that object will be excluded
from the correlation (this is casewise treatment of missing values
in the terminology of the NAG manual).
This approach has the disadvantage that some good data are discarded,
but it ensures that all the coefficients are based on the same number
of data points.
It can also lead to a considerable increase in computational efficiency.
Further details are given in the NAG manual (Chapter G02).

If there are more than a few hundred entries in the catalogue the
time required to compute the coefficients is not negligible.
The application is not suitable for large catalogues containing
of the order of 100,000 objects.

\begin{verbatim}
    Invocation: ICL> CAR_CORRELATE
      Parameter list:
       input           = Name of input catalogue
       select_numeric  = NO=>all fields output, YES=>prompted for names
       name            = Desired field or parameter name
       batch           = Batch processing Y/N?
       selfields       = Names of selected fields eg. [RA,DEC]

     Reference: Sprent P. (1981) in "Quick Statistics", p207. Penguin.
\end{verbatim}

\subsubsection{CAR\_COSMAGCAL}

Compute calibrated magnitude from COSMOS IAM magnitude

COSMAGCAL computes a ``calibrated" magnitude from the standard
COSMOS IAM parameters for step size and plate scale (STPSIZ and
PLTSCL) and the fields for COSMOS magnitude and sky intensity
(COSMAG and ISKY) contained in a Haggis catalogue.
It is assumed the step size is in microns and the plate scale is
in arcsec/mm.
You are prompted for the name of the catalogue.
If the catalogue is a standard COSMOS IAM catalogue, the parameters
containing STPSIZ and PLTSCL are obtained automatically from
the catalogue.
If these parameters cannot be found, you will be prompted for them.
You are prompted for the names of the fields containing
the COSMOS magnitude (COSMAG) and the sky intensity (ISKY) and
for the field to hold the ``calibrated" magnitude.
CONVERT must be run after this routine.

The ``calibrated" magnitude is computed using the formula
\begin{verbatim}
    calibrated magnitude = COSMAG/100.0 + 2.5 * LOG10 (ISKY)
                            - 5.0 * LOG10 (STPSIZ*PLTSCL/1000.0)
\end{verbatim}
If the ``calibrated" magnitude cannot be computed (e.g. because
of an attempt to take the logarthim of a negative number) the
null value for the output field is written to the catalogue.
\begin{verbatim}
    Invocation: ICL> CAR_COSMAGCAL
      Parameter list:
       input           = Name of input catalogue
       stpsiz          = Step (or pixel) size for the dataset
       pltscl          = Scale of plate from which data were measured (arcsec/mm)
       cosmag          = Field containing the COSMOS magnitude
       isky            = Field containing the sky intensity
       calmag          = Field to contain the calibrated magnitude
\end{verbatim}

\subsubsection{CAR\_COUNTREC}

Get the number of objects in a sequential file

COUNTREC does not require a description file for the
input catalogue, and therefore can be used to evaluate
NRECORDS prior to creating a catalogue description.
Note that CATSIZE does require a description file, and can be
used for quickly finding the size of a direct access file.

\begin{verbatim}
    Invocation: ICL> CAR_COUNTREC
      Parameter list:
       input           = Name of input catalogue
\end{verbatim}

\subsubsection{CAR\_DEBUG}

Switch on the VMS debugger --- for PROGRAMMERS only

This command can be used when testing SCAR, which should have been linked
with the command {\tt SCAR\_MLINK SCAR "/DEBUG"}.
\begin{verbatim}
    Invocation: ICL> CAR_DEBUG
\end{verbatim}

\subsubsection{CAR\_DIFFER}

Select the objects in one catalogue and not in another

The DIFFER routine is very similar to JOIN routine.
DIFFER selects all the objects in the first catalogue you specify in
your input list that do NOT have a matching partner in the second
catalogue you specify.
The prompts are the same as in JOIN except the order of the
catalogues in your INPUT list is crucial.
Objects in the first catalogue that fail the MATCH criterion are
written to your OUTPUT catalogue.
In SIMPLE mode your output catalogue is a local index to the first
catalogue.\\
For example
\begin{verbatim}
    INPUT=[CATX,IRPS] OUTPUT=CATY
\end{verbatim}
Selects all the CATX records not in the IRPS and puts them in CATY
whereas
\begin{verbatim}
    INPUT=[IRPS,CATX] OUTPUT=IRPS1
\end{verbatim}
Selects all the IRPS records not in the CATX and puts them in IRPS1.

The MATCH criterion must be expressed as a QCAR expression.\\
{\em Example:}  To DIFFER the IRPS and AIPS, the MATCH condition is
\begin{verbatim}
    IRPS__NAME.EQ.AIPS__NAME.AND.IRPS__DEC.EQ.AIPS__DEC.END.
\end{verbatim}
{\bf Units:} In a MATCH condition, when adding or subtracting a constant or
comparing it with an arithmetic expression, you must generally specify the
constant in the same units as the operand.
However, when the operand is a Radian quantity (eg RA, dec), you must present
the constant in the appropriate form for example HHMMSS for RA
and SDDMMSS for Dec.

The last three queries below (field, generic, groups) only appear, when you use
SELECT, if your terminal is not configured as VT100-like, so the full-screen
menu cannot be implemented.
The array required is input through GROUPS, e.g. IRPS fluxes are F(1) to
F(4), FQUALs are Q(1) to Q(4).
If you answer 'YES' to the SELECT and BATCH parameters the fields given by the
SELFIELDS parameter will be selected.

\begin{verbatim}
    Invocation: ICL> CAR_DIFFER
      Parameter list:
       inputs          = Name(s) of input catalogue(s), e.g. [IRP1,IRP2]
       output          = Name of output catalogue
    (  newout          = Create a new o/p file (Y), or else use existing one (N)?)
       title           = Name or short description for file or plot
       simple          = YES to make default output, NO to specify it
       match           = QCAR style match criterion
       satisfied       = Answer YES or NO to verify input criterion
       options         = Do you want to specify options?
       ascii           = YES for ascii or NO for binary output
       spaces          = Do you want spaces between fields?
       index           = Ouput an Index (Y) or Master (N) catalogue?
       select          = NO=>all fields output, YES=>prompted for names
       batch           = Batch processing Y/N?
       selfields       = Names of selected fields eg. [RA,DEC]
    (  field           = YES or NO to select field                               )
    (  generic         = YES or NO to select generic                             )
    (  groups          = Array section e.g: A(n:m) or B(i:j,n:m)                 )
\end{verbatim}

\subsubsection{CAR\_DISMOUNT}
This command dismounts the tape, deallocates the deck and deassigns
the group logical names TAPE0 or TAPE1 or TAPE2.
The tape deck name can be given as a parameter on the command line or is
prompted for.
\begin{verbatim}
    Invocation: ICL> CAR_DISMOUNT
\end{verbatim}

\subsubsection{CAR\_DSCFHELP}

Insert a description file in a VMS help library

DSCFHELP inserts a description file into a VMS help library.
You can only use it if you have write access to the help library.
You can create your own help library by:
\begin{verbatim}
    ICL> DCL LIB/CREATE/HELP/LOG MYHELP.HLB
    ICL> DCL DEFINE/JOB MYHELP DISK$USER:[XXX]MYHELP.HLB
    ICL> DCL DEFINE/JOB HLP$LIBRARY MYHELP
    ICL> CAR_DSCFHELP INPUT=MYCAT LIB=MYHELP
\end{verbatim}
This should allow you to get help on your own catalogues.
When you are creating a lot of files you may find it useful to
keep track of them in a help library.
\begin{verbatim}
    Invocation: ICL> CAR_DSCFHELP
      Parameter list:
       input           = Name of input catalogue
       lib             = Name of catalogue VMS help library
\end{verbatim}

\subsubsection{CAR\_EDIT}

Create a catalogue or edit an existing catalogue

The EDIT routine provides an editing facility like the editors
provided by most operating systems.
You can create a data file according to an existing description.
You can modify the contents of a data file; if you do this remember
that the new file replaces the old unedited file, and DELETES it.

The main reasons for a CAR editor is to permit:
\begin{itemize}
\item The controlled allocation of values into a fixed format file so you
can create FACTs catalogues of your own without ``punching errors".
\item The editing of unformatted files.
\item The correction of files without rewriting the whole file, so you
can amend large binary files quickly.
\end{itemize}
The VAX/EDT editor does not allow the above three functions.
In addition the features of EDIT allow you to browse through a
catalogue and save selected records for future analysis.
The EDIT routine works like a line editor.
It does not buffer the whole file, only the current record, so large
files can edited.
There is a delete function provided for KEYED access (ISAM) files,
but for other access modes it is not supported (yet).
However you can simulate a delete operation by skipping records and not
copying them from the input file to the edited file.

You INPUT the name of the file you wish to EDIT, a description file for
this data file must already exist.
You can create the description file using FORM1 or FORM2.
EDIT finds the description file and tries to find a data file
corresponding to the description for the INPUT.
If the data file is not found EDIT switches to CREATE mode, and you
insert the data values.
If EDIT finds the data file, it asks you how you want to ACCESS it.
Options are READ, WRITE or UPDATE.
If you select READ no output file is opened.
If you select WRITE a new output data file is opened.
If you select UPDATE you write back to the INPUT file.
If you are updating a file, you are advised to make a copy of it and the
description file before you start (use DCL COPY etc).

You have a `main buffer' for holding your current record and you have a
`memory buffer' for saving a record.
You change the entire `main buffer' record by positioning (P),
searching (S), inserting (I), unloading (--) the memory buffer,
or just by loading the next record (N).
You can replace (R) a field value in the main buffer.
A format statement is given for information as to the range and precision
in which the data is stored, in insert (I) and replace (R), but you can
add fields in any format.
Records are written to the output file by the action of a write (W),
copy (C), append (A), or rewrite/update (U) command.
After the write action, the buffer is loaded automatically with the next record.
In CREATE mode this next record is a null record.
Records can be deleted from the file you are writing to using
the delete command (D).
Only KEYED access files permit this.
When writing a new file you can read from either the input
file or the output file, selection is made using the file command (F).

The diagrams below summarise the data flow.
Note you cannot use some commands in some modes.
\begin{center}
\begin{picture}(135,30)
\thicklines
\put (30,17){\line (1,0){20}}
\put (50,17){\line (0,1){11}}
\put (30,17){\line (0,1){3}}
\put (30,20){\line (5,1){10}}
\put (40,22){\line (1,6){1}}
\put (41,28){\line (1,0){9}}
\put (70,0){\framebox(20,5){MEMORY}}
\put (70,20){\framebox(20,5){MAIN}}
\put (110,20){\framebox(20,5){OUTPUT}}
\put (50,22.5){\vector (1,0){20}}
\put (90,22.5){\vector (1,0){20}}
\put (79,20){\vector (0,-1){15}}
\put (81,5){\vector (0,1){15}}
\put (41,19){VDU}
\put (57.5,23.5){I,R}
\put (92,23.5){W,C,A,U}
\put (76,12.5){+}
\put (82,12.5){--}
\put (0,27){{\bf CREATE} mode:}
\end{picture}
\end{center}

\begin{center}
\begin{picture}(135,46)
\thicklines
\put (70,0){\framebox(20,5){MEMORY}}
\put (70,20){\framebox(20,5){MAIN}}
\put (70,40){\framebox(20,5){INPUT}}
\put (80,40){\vector (0,-1){15}}
\put (79,20){\vector (0,-1){15}}
\put (81,5){\vector (0,1){15}}
\put (76,12.5){+}
\put (82,12.5){--}
\put (81,32.5){P,N,S}
\put (0,40){{\bf READ} mode:}
\end{picture}
\end{center}

\begin{center}
\begin{picture}(135,46)
\thicklines
\put (30,17){\line (1,0){20}}
\put (50,17){\line (0,1){11}}
\put (30,17){\line (0,1){3}}
\put (30,20){\line (5,1){10}}
\put (40,22){\line (1,6){1}}
\put (41,28){\line (1,0){9}}
\put (70,0){\framebox(20,5){MEMORY}}
\put (70,20){\framebox(20,5){MAIN}}
\put (70,40){\framebox(20,5){INPUT}}
\put (110,20){\framebox(20,5){OUTPUT}}
\put (50,22.5){\vector (1,0){20}}
\put (90,22.5){\vector (1,0){20}}
\put (80,40){\vector (0,-1){15}}
\put (79,20){\vector (0,-1){15}}
\put (81,5){\vector (0,1){15}}
\put (41,19){VDU}
\put (57.5,23.5){I,R}
\put (92,23.5){W,C,A,U}
\put (76,12.5){+}
\put (82,12.5){--}
\put (81,32.5){P,N,S}
\put (0,40){{\bf WRITE} mode:}
\end{picture}
\end{center}

\begin{center}
\begin{picture}(135,30)
\thicklines
\put (30,17){\line (1,0){20}}
\put (50,17){\line (0,1){11}}
\put (30,17){\line (0,1){3}}
\put (30,20){\line (5,1){10}}
\put (40,22){\line (1,6){1}}
\put (41,28){\line (1,0){9}}
\put (70,0){\framebox(20,5){MEMORY}}
\put (70,20){\framebox(20,5){MAIN}}
\put (110,18){\framebox (20,9){}}
\put (112.5,19.6){\shortstack{OUTPUT\\INPUT}}
\put (50,22.5){\vector (1,0){20}}
\put (90,22.5){\vector (1,0){20}}
\put (110,22.5){\vector (-1,0){20}}
\put (79,20){\vector (0,-1){15}}
\put (81,5){\vector (0,1){15}}
\put (41,19){VDU}
\put (58,23.5){R}
\put (92,23.5){W,C,A,U}
\put (76,12.5){+}
\put (82,12.5){--}
\put (0,27){{\bf UPDATE} mode:}
\end{picture}
\end{center}
The commands available from EDIT are as follows:
\begin{description}
\item [+]: Copy the main buffer to the memory buffer
\item [--]: Copy the memory buffer to the main buffer
\item [A]: Append main buffer to the OUTPUT file
\item [B]: Backspace to previous record (not allowed when searching)
\item [C]: Copy records from the INPUT file to the OUTPUT file (only allowed
    in WRITE mode)
\item [D]: Delete record from OUTPUT file (only KEYED access files allow this)
\item [E]: End editing
\item [F]: Change file displayed (allowed in WRITE mode, so that INPUT and
   OUTPUT files can be seen)
\item [H]: Get HELP on commands
\item [I]: Insert a record to the main buffer. You will be prompted for the
   values of each field in turn, which you can add in any format
\item [L]: List the contents of the main buffer and its record number
\item [N]: Next record meeting search criterion is read into the main buffer
\item [P]: Position the displayed file by absolute record number. The record
   is loaded into the main buffer
\item [Q]: Quit editor (in WRITE mode this deletes the OUTPUT file)
\item [R]: Replace a field value in the main buffer with a new value
\item [S]: Search the input file, using a QCAR expression. The next
   record satisfying the selection criteria is loaded into the main buffer
\item [U]: Rewrite the main buffer to the OUTPUT file
\item [W]: Write the main buffer to the OUTPUT file
\item [X]: Cancel the search
\end{description}
\begin{verbatim}
    Invocation: ICL> CAR_EDIT
      Parameter list:
       input           = Name of input catalogue
       access          = File access type - one of READ, WRITE, UPDATE
       select          = NO=>all fields output, YES=>prompted for names
       command         = Edit command. Type H for list of commands
       number          = Number of records to copy (0=rest)
       record          = Absolute record number
       name            = Desired field or parameter name
       value           = Value of field
       answer          = Answer should be YES or NO
       range           = Key criterion - finish with .END.
       query           = Criterion - finish with .END.
       satisfied       = Answer YES or NO to verify input criterion
       more            = + for more output, 0 to refresh screen, H help, P print
       devtype         = Device type. -1 uses terminal setting, 0 scrolls
\end{verbatim}

\subsubsection{CAR\_EXTAPE}

Dump a catalogue from a foreign tape to disk file


The EXTAPE command allows you to examine the contents of an ascii tape
interactively and dump it to a sequential file.
The tape is normally mounted with the MOUNT command, but this is not essential.
You can specify the logical record length after viewing the data, and then
dump the data from tape to a file with a specified logical record length.
\begin{verbatim}
    Invocation: ICL> CAR_EXTAPE
      Parameter list:
       tape_device     = Tape device name
       marks           = Number of tape marks
       action          = Dump file (D)/Reject file (R)/View more (V)
       file_name       = Specification of file to contain data
       lrec            = Logical record length of the data
       nrec            = The exact number of records or an upper limit
\end{verbatim}

\subsubsection{CAR\_FORM1}

Create a description file in screen mode

CAR\_FORM1 is a utility to aid in building description files. It will place
user supplied information in the correct position in the description file.
The obligatory parameters come first. When you have supplied the appropriate
values for these parameters there is an opportunity to insert optional
parameters and the fields. Note that the RECORDSIZE parameter requires no
value as this is calculated and inserted for you. Use keyword 'P' for
parameter, 'F' for field and 'E' for endfield. After specifying 'E' for
endfield CAR\_FORM1 requests a comment for each parameter or field and
possible ADCNOTES and CATNOTES. During CAR\_FORM1 the bottom screen line
gives information about the input required.

This routine allows you to visualise the description file as you are
making it.
For each field of the description file you are prompted for input.
\begin{description}
\item [Entering data] --
For each field to be entered, a prompt appears along the bottom line of the
screen.
This tells you the name, format (ie.\ `F',`A',`I',`L',`E' or `D') and maximum
length of the description file field to be entered.
To enter data, type in the value required and hit $<$RETURN$>$.
The program will right justify numeric input data, move onto the next field
and await input.
If no input is required for a description file field, $<$RETURN$>$ will move
onto the next field.
A check is also made to ensure that the VALUE is within the length as
specified in the prompt line, if not the entry is rejected.
The RECORDSIZE parameter is automatically calculated for you.
\item [Mistakes] --
If you make a mistake entering any field on the current line, move back to the
previous field by typing: $\backslash$ $<$RETURN$>$.
The program will jump back to the previous field and allow the value to be
re-entered.
This may only be done up to the beginning of the current line.
You are advised to check the current line before pressing $<$RETURN$>$ on the
`VALUE' field.
Errors that are missed can be corrected later with VMS/EDT.
\item [Filing] --
When the routine is finished, the description will be filed under the name
provided in the `VALUE' field of the `FILENAME' record, prefixed with `DSCF'.
The file type will be `.DAT'.
The routine writes all input to a file in the local directory, JOURNAL.DAT.
In the event of a system break or accidental exit from the routine, the
description file can be recovered by editing this file.
\end{description}
If at any point you need help type `?' and $<$RETURN$>$, then type
CAR\_FORM1 in response to the CAR\_HELP menu.
After the help session is finished the program will return to the point
at which you left off.

 It is recommended that long CATNOTES and ADCNOTES are added later
with the VMS editor.
HELP modules within CATNOTES and ADCNOTES can be added by placing the
module number, at least 3, in the third column of the description file.
Remember that the record length for help modules follows the formula:
MAX\_RECORDLENGTH = (80-2*Help\_Key\_No).

\begin{verbatim}
    Invocation: ICL> CAR_FORM1
\end{verbatim}

\subsubsection{CAR\_FORM2}

Create a catalogue description from scratch

The FORM2 routine allows you to create a description file
from scratch.
In contrast to the screen based FORM1, which performs the same
function, FORM2 operates in prompt mode.
The main advantage of FORM2 over FORM1 is that it permits description
files to be included.
When you create a description file from scratch you must first define
the mandatory parameters TITLE, MEDIUM, ACCESSMODE.
If the ACCESSMODE is DIRECT you must also define NRECORDS and KEYFIELD
parameters, if you know their values.
For the output file name, give the suffix for the description file, DSCF
will be automatically added.
You cannot enter CATNOTES or ADCNOTES before the ENDFIELD definition,
and you cannot enter PARAMETER, FIELD, or GENERIC after.
You can carry over the existing set of FIELD or GENERIC definitions using
the I keyword.

Summary of keyword commands:
\begin{description}
\item [A]: ADC -- ADCNOTES line start
\item [C]: ADC -- CATNOTES line start
\item [D]: Display output switch -- on/off
\item [E]: ADC -- ENDFIELD marker
\item [F]: ADC -- FIELD definition start
\item [G]: ADC -- GENERIC definition start
\item [H]: Display the keyword options
\item [I]: Include description file
\item [L]: Default LENGTH -- on/off
\item [N]: Set FORMAT2=FORMAT1 -- on/off
\item [P]: ADC -- PARAMETER definition start
\item [R]: Display current RECORDSIZE
\item [S]: Default START -- on/off
\item [Z]: End input
\end{description}

\begin{verbatim}
    Invocation: ICL> CAR_FORM2
      Parameter list:
       output          = Name of output catalogue
       keyword         = Single letter keyword for command
                         (type H to review the possibilities)
       name            = Desired field or parameter name
       start           = Position of start of field [bytes]
       length          = Total number of bytes allocated to the field
       format1         = F, A, I, L, E, D, C, R allowed
       scale           = Scale factor for units (or exponent)
       unit            = Unit of the field
       format2         = F, A, I, L, E, D allowed - no unformatted types
       nullval         = Null value of Field or value of Parameter
       comment         = Lexical definition of the field/parameter
       express         = Functional definition of the field/parameter
                        (`!' is inserted if there is a functional definition,
                        otherwise it is not. If you put in HHMMSS without a
                        functional definition, put `!' at the end of the COMMENT)
       note            = Text of ADCnote or CATnote
       include         = Name of catalogue
       select          = NO=>all fields output, YES=>prompted for names
       save            = YES to save definition, NO to discard it
       answer          = Answer should be YES or NO
       batch           = Batch processing Y/N?
       selfields       = Names of selected fields eg. [RA,DEC]
\end{verbatim}

\subsubsection{CAR\_GETPAR}

Get a parameter value when in the ICL environment.

CAR\_GETPAR INPUT=IRPS PARAM=MEDIUM prints a message to the screen and if in ICL
the variable is set. This can be accessed by

\begin{verbatim}

                 GETPAR CAR_GETPAR RESULT (X)
                 =(X)

    Invocation: ICL> CAR_GETPAR
      Parameter list:
       input           = Name of input catalogue
       param           = Name of parameter.

\end{verbatim}

\subsubsection{CAR\_HARDCOPY}

You can use this procedure to produce your output after a run of the graphics
routines HISTOGRAM, SCATTER, AITOFF, CHART.

There is one mandatory parameter and two other optional parameters, as
described below:
\begin{description}
\begin{description}
\item [P1] -- Either AITOFF,CHART,HISTOGRAM or SCATTER: depending on the program.
\item [P2] -- Either VERSATEC, CANON, ZETA or PRINTRONIX: hardcopy device.
\item [P3] -- With CHART input the number of field centres.
\item [P4] -- Type YES if you want a points listing of objects.
\end{description}
\end{description}
Any parameters that you omit will be prompted for.
Examples:
\begin{verbatim}
     ICL> CAR_HARDCOPY HISTOGRAM VERSATEC
     ICL> CAR_HARDCOPY CHART CANON

     Invocation: ICL> CAR_HARDCOPY
\end{verbatim}

\subsubsection{CAR\_HISTOGRAM}

Plot a histogram of a field

Two styles of histogram are possible, bar chart and continuous line.
You will be prompted for the name of the INPUT catalogue and the
name of the field, or QCAR expression, to be plotted along the XAXIS.
The field values are read in from the catalogue and the mean, mode,
standard deviation, maximum and minumum values are presented to you.
You should note XMAX, XMIN and MODESZ for designing the
axis scales of your plot.
You can use constants to convert units for the X axis, for example
GLONG*57.295779 to give an axis in degrees, and you can label the
axis at the same time, so a reply to XAXIS could be:
\begin{verbatim}
    GLONG!GLONG*57.295779 \DEGREES|
\end{verbatim}
The reporting package will write the statistical data to HIST.CAR.
You can identify any statistic by looking up HIST in the catalogues
help.
It also appends the actual counts for each bin to this file.
The plotting loop starts, by asking you for a TITLE for the diagram.

In the bar chart style of histogram points are allotted to bins
of user specified size, and the height of the bar measures the
number of points in the bin.
In the continuous line style points are plotted and connected by a
line.
The answer to BARCHART sets the style.
The BINSIZE sets the X dimension of a bin.
The answer to SHAPE decides whether the plot will use the data to
determine the limits or whether you want to set your own with the
parameters, XMIN, NBINS, YMIN and YMAX.

You are asked to identify an output DEVICE for displaying the
histogram.
The diagram will then be drawn on the selected device.
On an interactive device a cursor will be drawn.
You can move the cursor arround (by tracker ball, joystick or the arrow
keys) to aid reading positions off the X and Y scales.
Type ``1'' to to read the cursor position, anything else exits
interactive mode.

When you have finished examining the diagram hit the select button
(e.g: any alpha key on a TEK terminal) and the screen is cleared.
You then have the option of filing the bin counts (STORE).
If you answer ``YES'' to the ``REDRAW'' prompt, you will start the
drawing loop again.
You may specify a hardcopy device second time around, or use HARDCOPY.
Graphical output will be sent to the file indicated in the list of
devices and will be in your current directory.
Statistical data will be sent to HIST.CAR.

{\bf Units:} When adding or subtracting a constant in a QCAR expression (in
DEFINE below) you must generally specify the constant in the units of the
operand of the operator, which you can find by looking up the field(s) concerned
in the HELP information for the catalogue.
However, when it is a Radian quantity (eg RA, dec), you must present the
constant in the appropriate form e.g. HHMMSS and SDDMMSS.
\begin{verbatim}
    Invocation: ICL> CAR_HISTOGRAM
      Parameter list:
       input           = Name of input catalogue
       xaxis           = Field or QCAR expression for X axis
                         (You can use constants to convert units
                         separate expressions with `!', if the label is longer
                         than 17 characters you will be prompted for a new label)
       xunits          = Units for the X axis
       xlabel          = Label for X axis
       more            = + for more output, 0 to refresh screen, H help, P print
       ready           = Reply should be yes or no
       title           = Name or short description for file or plot
       barchart        = Do you want barchart or line graph? - yes or no
       binsize         = Bin size for histogram
       shape           = Zero for default shape and limits, non-zero to set
       xmin            = Lower limit for X axis
       nbins           = The number of bins determines the X axis range
       ymin            = Lower limit for Y axis
       ymax            = Upper limit for Y axis
       device          = Output device (HELP lists devices)
       store           = Do you want to store the bin counts in HIST.CAR?
       redraw          = Redraw the diagram? - yes or no
\end{verbatim}

\subsubsection{CAR\_IMAGEPLOT}

Produce a finding chart plot for COSMOS objects

This routine produces a finding chart type of plot for catalogued
objects.
The routine is designed with image data, in particular COSMOS, in mind
and is largely based on the DOTPLOT and IGPLOT HAGGIS routines.
You can specify whether the plot is to be an overlay or not.
A map or overlay is produced, with the objects represented either as
dots, or by some symbol which gives information on the object (usually
the image size and orientation), or concentric symbols (giving image
contour information).
There is a maximum number of objects that can be plotted for each of
these modes : 1~000~000, 10~000, 1~000 respectively.
The input catalogue can be either an index or master catalogue.
You select which fields are to be plotted as X and Y, and which
fields are to be used for drawing the symbols.
The X and Y fields must have the same units, or be recognised time and
angle coordinates.
If precessed coordinates are required, the precession must be done first
using CALC.
You can superimpose data from more than catalogue, or from
different fields of the same catalogue, on to the plot.
This means for example that you can create indexes to catalogues
and plot the different classes of object on the same plot, distinguished
by symbol and/or line type.

You specify labels for the X and Y axes, and a title for the plot.
These labels can be up to 40 characters long.
The time and date are written at the bottom of the plot.
If the plot is an overlay it is annotated further, and there is also an
option to display certain `housekeeping' parameter information on the plot.
Housekeeping information is written down the right hand side of the
plot, for up to 3 catalogues if the plot is an overlay, or 4 otherwise.
The catalogue title, title comments, and the number of records in the
catalogue are obtained from the description file and displayed.
A description of the plot mode and line type is also given, and you
can give further information of up to 40 characters.
The device is selected from a list of devices which support GKS7.2 (SUN/83).

A square grid can be superimposed on the plot unless right ascension
and declination are being plotted as an overlay.
The direction of `North' and `East' is indicated on the plot.
The following information is displayed on the top right of the plot: the
plot centre, equinox and scale, and, if the X,Y are right ascension
and declination, the RA range (at the dec centre) and the dec range
(at the RA centre).

You are given the option of automatic scaling (using the smallest
and largest values of the X and Y fields of the first input
catalogue), or of specifying the X and Y ranges.
Note: auto-scaling by the first catalogue means that data may be lost
when subsequent catalogues are overlaid.

If an overlay is not required, the plot will be scaled to fit the
device, with the X and Y scales forced to be the same, as a map is
being produced.
If an overlay is being plotted X and Y will be in metres, and a warning
will be given if objects have been lost off the display surface.
You can use the HARDCOPY routine to produce hardcopy output.

\begin{description}
\item{Plotting information}
 \begin{itemize}
\item The units of the fields are restricted.
The units are obtained from the catalogue description file, and must
be as follows -
\begin{verbatim}
      X,Y fields : TIME, ANGLE, RADIAN, 0.1 MICRON, MICRON, MM
    image extent : TIME, ANGLE, RADIAN, 0.1 MICRON, MICRON, MM
     orientation : ANGLE, RADIAN
            area : PIXELS
      pixel size : ARCSEC, 0.1 MICRON, MICRON, MM
\end{verbatim}
\item The units of area measure must be `pixels'.
The routine decodes this into an angular or linear extent by obtaining
the value of a parameter in the description file which gives the pixel
side in units of arcsec, 0.1 micron, micron or mm.
The name of this parameter (given by the parameter `STPNAM') is defaulted
to STPSIZ.
If the program fails to obtain the value from the description file, you are
prompted for the value in units of 0.1 micron.
\item The parameter `CROSS' determines whether a central cross is to be
drawn on the plot, the value is defaulted to `NO'.
\item The plotted points can be labelled with the value of a specified
catalogue field.
The value is written out according to the null format of the field, as given
in the description file.
The description file can be edited to change this format.
\item There is a choice of 5 line types, if you are not plotting dots,
crosses or gunsights.
These are: solid, dash, dot, dash-dot, dash-dot-dot.
\item If {\bf points} are selected, a point is drawn at each X,Y value with
a marker chosen from the following GKS markers:
\begin{verbatim}
    1=dot `.'  2=plus sign `+'  3=asterisk `*'  4=circle `o'  5=cross `x'
\end{verbatim}
\item If {\bf symbol} is selected, a symbol is drawn at each X,Y chosen from:
\begin{verbatim}
    1: ellipse  2: rectangle  3: circle  4: cross  5: gunsight
\end{verbatim}
In each case the size of the symbol gives the image extent, and in
mode 1, information on the image orientation is also given.
If the X,Y are celestial, i.e. right ascension and declination, then,
unless the plot is an overlay, the units of the image extent must also be
celestial.
Similarly, if the X,Y are not celestial then, for every case, the image
extent must also be non-celestial, i.e. real units.\\
{\bf Ellipse}: You are prompted for the name of the fields which give
the value of : semi-major axis, semi-minor axis, and orientation.
The units of these fields, as given in the catalogue description
file, are restricted.\\
{\bf Rectangle}: You are prompted for the name of the fields which
give the value of minimum X value, maximum X value, minimum Y
value, maximum Y value.
These fields must have units of time, angle, radian, 0.1 micron, micron,
or mm.\\
{\bf Circle}: You are prompted for the name of the field which
gives the value of the image area, in pixels.
The circle is drawn such that the area of the circle equals this area.\\
{\bf Cross}: You are prompted for the name of the field which gives
the value of the image area, in pixels.
The cross is drawn such that the area of the square with sides equal to
the length of each bar of the cross would equal this area.\\
{\bf Gunsight}: You are prompted for the name of the field which gives
the value of the image area, in pixels.
The gunsight is drawn such that the area of the square with sides equal
to the extent of each bar of the gunsight would equal this area.
\item If {\bf concentric} symbols are selected, nine concentric ellipses
are drawn at each X,Y.
You are prompted for the names of the fields which give the value of
semi-major axis, semi-minor axis, orientation, image area, and areal
profiles.
The last must be a generic field in the catalogue, consisting of an array
of 8 elements.
If the X,Y are celestial, i.e. right ascension and declination, then,
unless the plot is an overlay, the units of the image extent must also be
celestial.
Similarly, if the X,Y are not celestial then, for every case, the image
extent must also be non-celestial, i.e. real units.
The units of these fields, as given in the catalogue description file, are
restricted.
 \end{itemize}
\item{Coordinate information}
 \begin{itemize}
\item It is not possible to mix celestial and non-celestial X,Y fields,
that is to superimpose catalogues where the X,Y are right ascension
and declination in one case and not in the other.
\item The program will attempt to obtain the equinox and epoch of the
catalogue positions from the description file; if it fails you
will be prompted for the values, which should be given as a real
value in years.
\item The overlay can be drawn in 2 orientations:
\begin{verbatim}
    a) North to the top and East to the right, i.e. declination
       increasing to the top and right ascension to the right, plate
       emulsion down, or
    b) North to the top and East to the left, i.e.declination increasing
       to the top and right ascension to the left, plate emulsion up.

       In case (a) the value of the parameter `DIRECT' is `Y',
       in (b) `N'. The value is defaulted to `Y'.
\end{verbatim}
\item If the X,Y are right ascension and declination, you will be
prompted for the desired values of the plot centre's right ascension
and declination.
In this case a tangent plane projection is performed with the celestial
sphere touching the plane at the plot centre.
You will be prompted for the desired values of the plot scale in
arcsec/mm.
\item If the X,Y are right ascension and declination, the plot will have
the centre and scale as specified for the first catalogue.
If the equinox differs from the previous catalogue value an error message
will be displayed and that input will be terminated.
If the epoch differs a warning message will be given.
\item If the X,Y fields are not right ascension and declination the routine
will attempt to obtain the value of the parameter ORIENT from the
catalogue description file.
If it fails, you will be prompted for the value, which is an integer
from 1 to 8 describing the following orientations for X increasing to the
right and Y to the top:
\begin{verbatim}
         top    : N E S W N W S E
         right  : W N E S E N W S
         orient : 1 2 3 4 5 6 7 8
\end{verbatim}
\item If the X,Y are not right ascension and declination, the routine
will obtain the right ascension and declination centre for each
catalogue plotted, and also the SCALE, either from the description file
using the parameters RACEN and DECEN, or, if this fails, from you.
\item If the X,Y are not right ascension and declination, you will
be prevented from superimposing catalogues that do not have the same
centre and scale.
If the equinox differs from the previous catalogue value an error message
will be displayed and that input will be terminated.
If the epoch differs a warning message will be given.
\end{itemize}
\end{description}
   \begin{verbatim}
    Invocation: ICL> CAR_IMAGEPLOT
      Parameter list:
       input           = Name of input catalogue
       imode           = 1: points; 2: one scaled symbol; 3: concentric ellipses
       x               = Name of field to be plotted as X axis
       y               = Name of field to be plotted as Y axis
       marker_code     = Marker number, one of: 1=.  2=+  3=* 4=o  5=x
       symbol          = 1=ellipse 2=rectangle 3=circle 4=cross 5=gunsight
       majax           = Name of field containing major axis of each image
       minax           = Name of field containing minor axis of each image
       theta           = Field containing orientation of each image
       device          = Output device (HELP lists devices)
       xmin_field      = Name of field containing X minimum extent for each image
       xmax_field      = Name of field containing X maximum extent for each image
       ymin_field      = Name of field containing Y minimum extent for each image
       ymax_field      = Name of field containing Y maximum extent for each image
       area            = Name of field containing area of each image
       areal           = Name of field containing areal profiles for each image
       linetype        = 1=solid 2=dash 3=dot 4=dash-dot 5=dash-dot-dot
       label_objects   = Is a label to be appended to each object plotted?
       flabel          = Name of field to be used to label objects
       overlay         = Is the plot to be scaled as an overlay?
       direct          = Direct (Y) or reverse (N) overlay
       cross           = Plot a cross at the plate centre?
       racen           = RA Centre of plot or catalogue
       decen           = Dec Centre of plot or catalogue
       plot_scale      = Plot or catalogue scale in arcsec/mm
       orient          = Plate orientation on XY measurement
       cat_equinox     = Equinox of current catalogue, real number of years
       cat_epoch       = Epoch of current catalogue, real number of years
       axes_type       = XY axes type - one of TIME, ANGLE, RADIAN, REAL
       auto            = Autoscale the plot by minimum and maximum XY
       min             = Minimum X or Y value to plotted
       max             = Maximum X or Y value to plotted
       grid            = Is a rectangular grid to be drawn over the plot?
       xlabel          = Label for X axis
       ylabel          = Label for Y axis
       title           = Name or short description for file or plot
       stpsiz          = Step (or pixel) size for the dataset
       stpnam          = Name of catalogue parameter giving step (or pixel) size
       plot_more       = Terminate plotting or plot another catalogue?
       hkinfo          = Annotate the plot with COSMOS housekeeping info
       info            = Extra information supplied by user for annotation
       again           = Try again?
       mxmod1          = Maximum number of objects permitted for mode 1 plot
       mxmod2          = Maximum number of objects permitted for mode 2 plot
       mxmod3          = Maximum number of objects permitted for mode 3 plot
\end{verbatim}

\subsubsection{CAR\_JOIN}

Join two catalogues on a common field

The JOIN routine allows you to join two catalogues on a common field
or fields.
To speed things up, you should put the smaller catalogue first in the list.
This process can also be used for associating two astronomical objects by
position or name.
JOIN requires that the two input catalogues be sorted in the same order,
on a field which has the same units in both catalogues.
For positional associations this order must be the latitude coordinate.
In the equatorial coordinate system, this field is of course DEC.
(The IRPS catalogue has been sorted by DEC.)
The JOIN is effected by a MATCH specification, which is a QCAR expression
that relates the keyfields of the two catalogues to be joined.
As the keyfields may have the same fieldnames, the fieldnames in a MATCH
specification should be prefixed by the characters of the catalogue
name, followed by double underscore, eg. IRPS\_\,\_DEC.
For example when you are associating two catalogues you would primarily
demand that the positions of the sources are nearly the same.
The QCAR function GREAT\_CIRCLE can be used for this.
This calculates the great circle distance between two positions, the result
being expressed in arcsecs.
The function has four arguments, LONG1, LAT1, LONG2, LAT2, so it can be
used in any coordinate system.

{\bf Units:} In a MATCH condition, when adding or subtracting a constant or
comparing it with an arithmetic expression, you must generally specify the
constant in the same units as the operand.
However, when the operand is a Radian quantity (eg RA, dec), you must present
the constant in the appropriate form for example HHMMSS for RA
and SDDMMSS for Dec.

There are 2 classes of relation which may be specified: those which express an
equality between corresponding fields:

{\em Example 1:}  To JOIN the IRPS and AIPS, the MATCH condition is
\begin{verbatim}
    IRPS__NAME.EQ.AIPS__NAME.AND.IRPS__DEC.EQ.AIPS__DEC.END.
\end{verbatim}
and those which express ``nearness", that is, equality to within a certain
tolerance:

{\em Example 2:} To associate every IRPS object within 90 arcsecs of
a RCBG object, the MATCH condition is
\begin{verbatim}
    GREAT_CIRCLE(RCBG__RA,RCBG__DEC,IRPS__RA,IRPS__DEC).LE.90.END.
\end{verbatim}
The two catalogues in this example are both sorted on DEC.

{\em Example 3:} If you have a list of galaxies and you want to look for
nearby, but not coincident, IRAS sources, the match condition would be:
\begin{verbatim}
    GREAT_CIRCLE(RCBG__RA,RCBG__DEC,IRPS__RA,IRPS__DEC).LT.360.AND.
    GREAT_CIRCLE(RCBG__RA,RCBG__DEC,IRPS__RA,IRPS__DEC).GT.120.END.
\end{verbatim}
The order of the criteria in this expression matters --- the first relational
expression containing both of the keys is used as the `coarse' test.

{\em Example 4:} If you wanted to associate IRAS sources with a list of
galaxies,  but exclude accidental associations with stars, then the criterion:
\begin{verbatim}
    GREAT_CIRCLE(RCBG__RA,RCBG__DEC,IRPS__RA,IRPS__DEC).LT.180.AND.
    IRPS__FLUX(2)/IRPS__FLUX(3).LT.3.AND.IRPS__Q(3).EQ.3.END.
\end{verbatim}
would effectively do this.

Ordinary stars have flux ratios at 25:60 microns of about 10, and only the
brightest stars have high quality fluxes at the longer wavelengths.
You can see from the above example that you can include relational expressions,
other than those which just relate keyfields, to `tighten' the join criterion.

{\em Example 5:} If you wanted to associate IRAS sources with your own
catalogue TEST which contains a field PERR for the maximum error of the
position, then the criterion:
\begin{verbatim}
    GREAT_CIRCLE(TEST__RA,TEST__DEC,IRPS__RA,IRPS__DEC).LT.PERR.END.
\end{verbatim}
would effectively do this. But note that PERR should be specified in radians
and the SCOPE parameter for CAR\_JOIN requires a non zero value. In this example
the SCOPE should be the maximum possible value for PERR.

The last three queries below (field, generic, groups) only appear, when you use
SELECT, if your terminal is not configured as VT100-like, so the full-screen
menu cannot be implemented.
The array required is input through GROUPS, e.g. IRPS fluxes are F(1) to
F(4), FQUALs are Q(1) to Q(4).
If you answer 'YES' to the SELECT and BATCH parameters the fields given by the
SELFIELDS parameter will be selected.

\begin{verbatim}
    Invocation: ICL> CAR_JOIN
      Parameter list:
       inputs          = Name(s) of input catalogue(s), e.g. [IRP1,IRP2]
       output          = Name of output catalogue
    (  newout          = Create a new o/p file (Y), or else use existing one (N)?)
       title           = Name or short description for file or plot
       simple          = YES to make default output, NO to specify it
       match           = QCAR style match criterion
       satisfied       = Answer YES or NO to verify input criterion
       options         = Do you want to specify options?
       ascii           = YES for ascii or NO for binary output
       spaces          = Do you want spaces between fields?
       index           = Output an Index (Y) or Master (N) catalogue?
       select          = NO=>all fields output, YES=>prompted for names
       batch           = Batch processing Y/N?
       selfields       = Names of selected fields eg. [RA,DEC]
       scope           = Maximum size (in radians) of the catalogue error circle
    (  field           = YES or NO to select field                               )
    (  generic         = YES or NO to select generic                             )
    (  groups          = Array section e.g: A(n:m) or B(i:j,n:m)                 )
\end{verbatim}


\subsubsection{CAR\_LINCOR}

Compute the Pearson product-moment linear correlation coefficients

LINCOR computes the Pearson product-moment linear correlation
coefficients between numeric fields in a catalogue.
The linear correlation coefficient measures the strength of any
linear relation between the two fields for which it is computed.
The coefficient takes values ranging from +1.0 to --1.0; +1.0
implies perfect correlation, --1.0 perfect anti-correlation
and 0.0 no correlation.
The coefficient is discussed further in the NAG manual (Chapter G02)
and most statistics books, see for example Bevington (1969).
When LINCOR is run you are first prompted for the name
of the catalogue which is to be analysed.
You are then asked whether you wish to analyse all the numeric fields in
the catalogue or whether you wish to select the fields to be correlated.
If you choose to select the fields to be correlated you are prompted for
the names of the fields which you require.
LINCOR will only access fields which are of type INTEGER, REAL or DOUBLE
PRECISION.
The results are written to a file called $<$catalogue name$>$.LIS which
is suitable for printing.
Fields in SCAR catalogues can contain ``missing values" where a
measurement was not available for a particular field for a given
object.
In such cases the value of the field for the object is set to the ``null
value" defined for the field in the description file.
Such values are excluded from the correlations.
It is your responsibility to ensure that the null values
chosen for the fields that are to be correlated do not overlap with
the range of genuine data values.
If such an overlap does occur genuine data will be excluded from the
correlation, because REAL and DOUBLE PRECISION values cannot be tested for
equality and so a range of values within \underline{+}1.0E--13 of the null
value will be treated as missing values.
If a field to be correlated contains a missing value for a
particular object all the fields for that object will be excluded
from the correlation (this is casewise treatment of missing values
in the terminology of the NAG manual).
This approach has the disadvantage that some good data are discarded, but
it ensures that all the coefficients are based on the same number of data
points.
It can also lead to a considerable increase in computational efficiency.
Further details are given in the NAG manual (Chapter G02).
\begin{verbatim}
    Invocation: ICL> CAR_LINCOR
      Parameter list:
       input           = Name of input catalogue
       select_numeric  = NO=>all fields output, YES=>prompted for names
       name            = Desired field or parameter name
       batch           = Batch processing Y/N?
       selfields       = Names of selected fields eg. [RA,DEC]

     Reference: Bevington P.R. (1969) in
     "Data reduction and error analysis for the physical sciences".
      p119. McGraw-Hill, New York.
\end{verbatim}

\subsubsection{CAR\_LISTOUT}

Write out values of specified fields of a catalogue

LIST writes out values of fields specified by you, from a
catalogue specified by you.
The output takes the form of columns, and the field name input is
terminated, either by you or when the required output width goes
over a maximum value.
A header and running record number are optional.
The output can be directed either to the screen, or to a specified
formatted file, or both.
The default maximum width for a file is 132, and for the screen 75,
characters.
If both file and screen are selected the default width is 75 characters.
There is a default column spacing of 2 characters.
The column width is set by the field length, or if a header is required,
by the length of the longest character string, field value, name or units.
Angular fields are converted to character strings as specified in
the comments section of the description file, unless the parameter
`ANGLE' is set to true, when the value will be output in radians.
The output format of all other fields is controlled by the null
format given in the catalogue description file.  The format can be
changed by editing the description file.
\begin{verbatim}
    Invocation: ICL> CAR_LISTOUT
      Parameter list:
       input           = Name of input catalogue
       listmode        = Output to Screen, File or Both
       swidth          = SWIDTH must be between 32 and 75
       width           = WIDTH must be between 32 and 160
       header          = Do you require a page header?
       count           = Include a running record count in the output?
       printfile       = Name of VMS file to which output is directed
       space           = Number of blank spaces between columns
       angle           = Angular coordinates to be specified in radians?
       name            = Desired field or parameter name
\end{verbatim}

\subsubsection{CAR\_LISTIN}

Read a user file into the SCAR system

LISTIN reads a simple VMS file containing columns of numbers into a
catalogue. The input data file and a description file for the output
must exist prior to running LISTIN.
The input file consists of columns of numbers or characters which are to
be copied into the catalogue.
It may be created with an editor or by another program.
Columns are identified by the name of the field.
Columns need to be separated by one or more blanks.
Fields with embedded blanks are illegal.
Occasionally there may be a field in the input file which can not
to be copied to the output catalogue.
This field should be given the name `OMIT'.
Consider the following input file:
\begin{verbatim}
    NAME        OMIT      MAG      T
    NGC4868    1245.8    13.14     6
    NGC4866    1367.4    11.84    -5
    NGC4880    1134.5    12.57     3
    .          .         .         .
    .          .         .         .
\end{verbatim}
The first, third and fourth columns will be copied to fields named `NAME'
`MAG' and `T' respectively.
The second column will not be copied.
Occasionally it may be required to change the name of the omitted
from the standard `OMIT'.
This might be because for some reason the catalogue has a field called
`OMIT' to which fields are to be written.
This end may be achieved by giving a value for the parameter OMTFLD other
than the default `OMIT'.
\begin{verbatim}
    Invocation: ICL> CAR_LISTIN
      Parameter list:
       infile          = Name of input file
       omtfld          = Name of columns in input file not to be copied
                         (default OMIT)
       output          = Name of output catalogue
\end{verbatim}

\subsubsection{CAR\_LITTLEBIG}

Extracts largest or smallest numbers from a catalogue

LITTLEBIG selects a given number of objects with the largest or smallest
values for a particular field.
You are prompted for the name of the input catalogue,
which may be either a master or an index catalogue.
A name for the output catalogue, to hold the selected objects, is next
requested.
If a description file for the output catalogue did not previously exist
one is created, based on the description file of the input
catalogue and you are asked if this operation is to be done
in simple or complex mode.
(In simple mode the description file is created automatically, in complex
mode you control what it contains.)
Once the description file for the output catalogue of selected
objects has been created, you are asked if a second output
catalogue of rejected objects is to be created.
If it is, you are prompted for the name of that catalogue.
Again, if no description file exists, one is created, either in simple or
complex mode.
The routine inquires whether the largest or smallest values for this field
are to be obtained (the response here may be abbreviated L or S), and the
number of objects to be selected.
The routine prompts for the maximum number of objects which may be
selected.
The purpose of this option is to allow procedures to be implemented which
limit the maximum permitted size of the output catalogue.
The selection then proceeds and the requested catalogues are generated.
The output catalogues are not sorted on the selected field; you may wish
to do this before reporting.

The three queries below (field, generic, groups) only appear, when you use
SELECT, if your terminal is not configured as VT100-like, so the full-screen
menu cannot be implemented.
The array required is input through GROUPS, e.g. IRPS fluxes are F(1) to
F(4), FQUALs are Q(1) to Q(4).
If you answer 'YES' to the SELECT and BATCH parameters the fields given by the
SELFIELDS parameter will be selected.

\begin{verbatim}
    Invocation: ICL> CAR_LITTLEBIG
      Parameter list:
       input           = Name of input catalogue
       output          = Name of output catalogue
    (  newout          = Create a new o/p file (Y), or else use existing one (N)?)
       title           = Name or short description for file or plot
       reject          = Do you want a REJECTS catalogue?
       rejects         = Name of output catalogue of rejects
       simple          = YES to make default output, NO to specify it
       options         = Do you want to specify options?
       ascii           = YES for ascii or NO for binary output
       spaces          = Do you want spaces between fields?
       index           = Ouput an Index (Y) or Master (N) catalogue?
       select          = NO=>all fields output, YES=>prompted for names
       batch           = Batch processing Y/N?
       selfields       = Names of selected fields eg. [RA,DEC]
    (  field           = YES or NO to select field                               )
    (  generic         = YES or NO to select generic                             )
    (  groups          = Array section e.g: A(n:m) or B(i:j,n:m)                 )
       name            = Desired field or parameter name
       xtrema          = Select largest or smallest values in the field?
       numsel          = Actual number of objects to be selected
       maxsel          = Maximum no. of objects permitted for selection
\end{verbatim}

\subsubsection{CAR\_MERGE}

Merge two or more catalogues

The MERGE routine adds two or more catalogues together.
The resulting catalogue is sorted on fields, specified by you, which
have the same units in each of the input catalogues.
If there is no common keyfield, the keyfield NUMBER (record number) should
be specified for the output file.
Currently the maximum number of input files permitted is nine.
In SIMPLE mode the default output is a global index to the input files.
In COMPLEX mode you can specify exactly the kind of output you want.
Note that each input local index implies two files and each global index
implies at least two.

The last three queries below (field, generic, groups) only appear, when you use
SELECT, if your terminal is not configured as VT100-like, so the full-screen
menu cannot be implemented.
The array required is input through GROUPS, e.g. IRPS fluxes are F(1) to
F(4), FQUALs are Q(1) to Q(4).
If you answer 'YES' to the SELECT and BATCH parameters the fields given by the
SELFIELDS parameter will be selected.

\begin{verbatim}
    Invocation: ICL> CAR_MERGE
      Parameter list:
       inputs          = Name(s) of input catalogue(s), e.g. [IRP1,IRP2]
       output          = Name of output catalogue
    (  newout          = Create a new o/p file (Y), or else use existing one (N)?)
       title           = Name or short description for file or plot
       simple          = YES to make default output, NO to specify it
       keys            = Key fields e.g: DEC or [DEC,RA]
       ascend          = Sort values in ascending order? e.g: Y or [Y,Y]
       options         = Do you want to specify options?
       ascii           = YES for ascii or NO for binary output
       spaces          = Do you want spaces between fields?
       index           = Ouput an Index (Y) or Master (N) catalogue?
       select          = NO=>all fields output, YES=>prompted for names
       batch           = Batch processing Y/N?
       selfields       = Names of selected fields eg. [RA,DEC]
    (  field           = YES or NO to select field                               )
    (  generic         = YES or NO to select generic                             )
    (  groups          = Array section e.g.: A(n:m) or B(i:j,n:m)                )
\end{verbatim}

\subsubsection{CAR\_MOUNT}

Allocate a tape drive, and mount tape

This command allocates a deck, mounts a foreign tape and assigns it
a group logical name `TAPE0', `TAPE1' or `TAPE2', (group logical names are
required for accessing tapes with the SSE).
This command should always be used before mounting a tape to be used by a CAR
command.
The dataset should be referred to as `TAPE0', `TAPE1', or `TAPE2' when you are
asked for INPUT or OUTPUT.
The description files for the dataset on tape must exist with a name
`DSCFTAPE0' etc.
The tape should be dismounted and the deck deallocated with the CAR\_DISMOUNT
command.
\begin{verbatim}
    Invocation: ICL> CAR_MOUNT
\end{verbatim}

\subsubsection{CAR\_OLIST}

Write out values of specified fields of a catalogue in the specified
order

OLIST writes out values of fields specified by you, from a
catalogue specified by you, in an order specified by you.
The output takes the form of columns, and the field name input is
terminated, either by you or when the required output width goes
over a maximum value.
A header and running record number are optional.
The output can be directed either to the screen, or to a specified
formatted file, or both.
The default maximum width for a file is 132, and for the screen 75,
characters.
If both file and screen are selected the default width is 75 characters.
There is a default column spacing of 2 characters.
The column width is set by the field length, or if a header is required,
by the length of the longest character string, field value, name or units.
Angular fields are converted to character strings as specified in
the comments section of the description file, unless the parameter
`ANGLE' is set to true, when the value will be output in radians.
The output format of all other fields is controlled by the null
format given in the catalogue description file.  The format can be
changed by editing the description file.
\begin{verbatim}
    Invocation: ICL> CAR_OLIST
      Parameter list:
       input           = Name of input catalogue
       listmode        = Output to Screen, File or Both
       swidth          = SWIDTH must be between 32 and 75
       width           = WIDTH must be between 32 and 160
       header          = Do you require a page header?
       count           = Include a running record count in the output?
       printfile       = Name of VMS file to which output is directed
       space           = Number of blank spaces between columns
       angle           = Angular coordinates to be specified in radians?
       name            = Desired field or parameter name
\end{verbatim}

\subsubsection{CAR\_POLYGON}

Generate description file for polygon vertices

One way of selecting objects is to specify the corners of a box.
The POLYGON routine enables the search area to be of any polygon shape.
POLYGON creates a description file for a catalogue which
defines a set of fields containing the positions of the corners of the
polygon.
This description is used by the WITHIN routine to select objects
inside or outside the polygon.
The polygon does not need to be ``closed'', i.e. by respecifying the first
corner at the end.
The catalogue created will contain the following fields:-
\begin{verbatim}
     Name    Units    Type     Description
     XPOLY            R*4      X coordinates of polygon vertices
     YPOLY            R*4      Y coordinates of polygon vertices
\end{verbatim}
The units are usually radians.
POLYGON asks for the name of data file holding the XPOLY and YPOLY values
(e.g. Z) and creates the description file (e.g. DSCFZ.DAT).
You use EDIT to put in the values of the polygon corner positions.
The number of corners is taken from NRECORDS.
\begin{verbatim}
    Invocation: ICL> CAR_POLYGON
      Parameter list:
       output          = Name of output catalogue
       title           = Name or short description for file or plot
       xunits          = Units for the X axis
       yunits          = Units for the Y axis
\end{verbatim}

\subsubsection{CAR\_PRINT}

Produce a catalogue report to a file with minimal prompting

The PRINT routine allows you to produce printable reports of sections of
catalogues with a minimum of prompts.
This makes it ideal for use in batch processing.
It allows you to select the start and end records as well as the fields to be
included in the report.
The width and header options are similar to those provided in REPORT.
The routine defaults are no header, width of 132 characters, all fields,
and the complete catalogue in the report.
The default output file for the report is $<$catalogue name$>$.CAR.

\begin{verbatim}
    Invocation: ICL> CAR_PRINT
      Parameter list:
       input           = Name of input catalogue
       output          = Name of the file in which the report will be placed
       selfields       = List of the fields to be included in the report
       srecord         = Start record for the report
       erecord         = End record for the report
       width           = WIDTH must be between 32 and 160
       header          = Do you require a page header?
\end{verbatim}

\subsubsection{CAR\_RECALC}

Update the values of existing fields in a catalogue

RECALC allows you to generate new values from fields in a catalogue
and to write the computed values back into existing fields in the
catalogue.
A QCAR-style arithmetic expression is used to compute the new values, and
the new values overwrite the previously existing values in the destination
field.
The catalogue can be a master catalogue, a local index, or a global index;
however for a global index only fields in the master being indexed can be
updated, not fields held in the index itself.

You are prompted for the name of the catalogue to be processed.
Since the values in the catalogue are to be updated you must have write
access to the catalogue files.
For index catalogues you must also have write access to the catalogues
being indexed, i.e. the master catalogue of a local index and the two or more
masters of a global index.
You are prompted for the field for which the new values are to be
computed and the QCAR-style arithmetic expression to be used in calculating
the new values.
You complete the specification of the updated field by supplying new
units and null values. RECALC uses the format of the null values you supply
to be the null format so be sure to express the null value to the correct
accuracy padding with 0's if necesseary.
Finally you are asked if all the values supplied are correct.
If they are not you are allowed to re-enter them.
The prompting for fields to be updated and their new functional definitions
continues whilst you have new fields to update.
To terminate the process you enter ``\#" in response to the prompt
to enter a new field name.
Once details for all the fields to be changed have been supplied, they are
updated for all the records in the catalogue.

If you supply a blank NULLVAL, the existing null format will be retained.
If you supply a non-blank NULLVAL, that value will be used to determine a
new null format string.
This is, of course, necessary if the expression you give results in a range
of values that cannot be represented using the original null format.

Note that if the values of the expression you supply fall out of range of the
internal format, RECALC will report an arithmetic error when trying to put
the field in the file.
In this case your catalogue data file has been corrupted.
You should regenerate it and use CALC, or ensure that the new file you
generate has a suitable format for the range of data you expect.
\begin{verbatim}
    Invocation: ICL> CAR_RECALC
      Parameter list:
       input           = Name of input catalogue
       name            = Desired field or parameter name
       define          = QCAR expression for evaluating field or parameter
                         (remember to finish QCAR expression with .END.)
       satisfied       = Answer YES or NO to verify input criterion
       nullval         = Null value of Field or value of Parameter
       unit            = Unit of the field
\end{verbatim}

\subsubsection{CAR\_REPORT}

Produce a catalogue report, to view it on the screen or as printout

The REPORT routine allows you to view a catalogue online and produce
printable reports for further study.
It allows you to treat a catalogue like any book, as you can flick from
page to page to find what you want.
You are encouraged to set up DCL symbols and templates for viewing your
favourite catalogue, so you do not have to type in all the start up
parameters each time (see How to make a template, section 4.6.1).

REPORT measures the current screen width and then chooses a
suitable style for output.
If there are few enough fields, it outputs a table, if not, it will output
the data in forms style.
If you are using a VT100-type terminal, you can give the command DCL
SET TERMINAL/WIDTH=width to control the report width.
The maximum width setting is 160 (suitable for some line printers) and the
minimum setting 32.
REPORT measures the current screen depth.
You can control the depth of the report page by the DCL SET
TERMINAL/PAGE=depth command.
The maximum depth setting is 60, the default is 24 on most terminals.
If you prefer a scrolling display, then do DCL SET TERMINAL/UNKNOWN before
starting REPORT.

The last three queries below (field, generic, groups) only appear, when you use
SELECT, if your terminal is not configured as VT100-like, so the full-screen
menu cannot be implemented.
The array required is input through GROUPS, e.g. IRPS fluxes are F(1) to
F(4), FQUALs are Q(1) to Q(4).
If you answer 'YES' to the SELECT and BATCH parameters the fields given by the
SELFIELDS parameter will be selected.

\begin{verbatim}
    Invocation: ICL> CAR_REPORT
      Parameter list:
       input           = Name of input catalogue
       mode            = 0=file, 1=screen & file, 2=screen, 3=screen & request file
                         (0 - print only,  1 - screen and print (as seen)
                          2 - screen only, 3 - screen and print to requested file)
       width           = WIDTH must be between 32 and 160
       header          = Do you require a page header?
       select          = NO=>all fields output, YES=>prompted for names
       batch           = Batch processing Y/N?
       selfields       = Names of selected fields eg. [RA,DEC]
       scroll          = + next, - last, 0 refresh, J jump, P print, H help, E end
       more            = + for more output, 0 to refresh screen, H help, P print
       record          = The absolute number of the next record to be displayed
    (  field           = YES or NO to select field                               )
    (  generic         = YES or NO to select generic                             )
    (  groups          = Array section e.g.: A(n:m) or B(i:j,n:m)                )
\end{verbatim}

\subsubsection{CAR\_SAMPLE}

Select every Nth object in a catalogue

SAMPLE extracts every Nth object from a catalogue and writes the selected
objects to a second catalogue.
Optionally the rejected objects may be written to another catalogue.
SAMPLE might typically be used to extract a small random sample from a large
dataset.
The small sample might be examined interactively to determine suitable
processing procedures, which would be applied to the entire dataset in batch.

Firstly you are prompted for the name of the input catalogue, which may
be either a master or an index catalogue.
The output catalogue to hold the selected objects is also requested.
If a description file for the output catalogue did not previously exist,
one is created, based on the description file of the input catalogue and
you are asked if this operation is to be done in simple or complex mode.
(In simple mode the description file is created automatically, in complex
mode you control what it contains.)
Once the description file for the output catalogue of selected objects has
been created, you are asked if a second output catalogue (of rejected
objects) is to be created.
If it is, you are prompted for the name of the catalogue and again if
no description file exists, one is created.
The mode used to create the description file for this catalogue (simple or
complex) is the same as that used for the first output catalogue.
Finally you are asked to give the frequency with which objects are to
be selected from the input catalogue, ie. to select every tenth, twentieth,
or hundredth object, etc.
The number of selected objects is displayed and the requested catalogues
are generated.
In most cases it is necessary for SAMPLE to read through the entire catalogue
in order to select the required objects.
However in some cases it is sufficient just to read the records that are
to be selected.
In particular the entire catalogue must be read if either of the following
conditions apply:-
\begin{itemize}
\item The access mode of the catalogue is sequential.
\item The rejected objects are to be written to a second output
     catalogue as well as the selected objects being written to
     the first output catalogue.
\end{itemize}
However if individual records in the catalogue can be read directly (the
access mode is `KEYED' or `DIRECT') and no catalogue of rejected objects
is to be produced then the objects to be selected can be read and copied
directly.
In this case even large catalogues can be processed interactively without
undue delay.
For example, if every 100th object were being selected from a 200,000 object
(direct access) catalogue it would only be necessary to read the 2000 objects
to be selected, which would take a negligible amount of elapsed time.

The three queries below (field, generic, groups) only appear, when you use
SELECT, if your terminal is not configured as VT100-like, so the full-screen
menu cannot be implemented.
The array required is input through GROUPS, e.g. IRPS fluxes are F(1) to
F(4), FQUALs are Q(1) to Q(4).
If you answer 'YES' to the SELECT and BATCH parameters the fields given by the
SELFIELDS parameter will be selected.

\begin{verbatim}
    Invocation: ICL> CAR_SAMPLE
      Parameter list:
       input           = Name of input catalogue
       output          = Name of output catalogue
    (  newout          = Create a new o/p file (Y), or else use existing one (N)?)
       title           = Name or short description for file or plot
       reject          = Do you want a REJECTS catalogue?
       rejects         = Name of output catalogue of rejects
       simple          = YES to make default output, NO to specify it
       options         = Do you want to specify options?
       ascii           = YES for ascii or NO for binary output
       spaces          = Do you want spaces between fields?
       index           = Ouput an Index (Y) or Master (N) catalogue?
       select          = NO=>all fields output, YES=>prompted for names
       batch           = Batch processing Y/N?
       selfields       = Names of selected fields eg. [RA,DEC]
    (  field           = YES or NO to select field                               )
    (  generic         = YES or NO to select generic                             )
    (  groups          = Array section e.g: A(n:m) or B(i:j,n:m)                 )
       freq            = Every Nth object will be selected
\end{verbatim}

\subsubsection{CAR\_SCATTER}

Make a scatterplot of two fields and perform regression analysis

Linear regression is performed on the two specified fields and a scatter
diagram with the line of best fit is plotted.
You may specify any of the diagram parameters and also draw up to two of
your own lines on the diagram.
You will be prompted for the name of the input catalogue and fieldnames, or
QCAR expressions, for the X axis and Y axis.
If the field name or expression is too long to be fitted on the diagram, you
are asked for an alternative string, for one or both of XLABEL and YLABEL.
If the units of an expression cannot be determined, you will be asked for one
or both of XUNITS and YUNITS.
Points in the scatter diagram may also be labelled, using the value of a
field, LABEL, for which you are asked.
The data is read in and the regression calculations are done using standard
NAG routines (G02CAF).
You should note the X and Y minima and maxima, for deciding on the range of
the diagram's X and Y axes (XMIN, XMAX, YMIN and YMAX).
The drawing loop then begins; you can redraw the diagram as many times as you
like, on any of the devices available.
You are asked for the diagram title and a MARKER character for points to mark
each object.
If the regression analysis produced results, you are asked if you want to plot
the regression line.
You can also plot your own lines of gradient M1,M2 and intercept C1,C2 if
you answer YES to the questions LINE1 and LINE2.
The output device is requested, and the plotting is done.
On interactive devices you can label points in the scatterplot using the
mouse, joystick, or arrow keys to move the cursor: note that `shift'ing the
arrow keys speeds up cursor movement, and hitting $<$RETURN$>$ ends the
interactive mode.
The drawing loop ends with the REDRAW prompt.
Statistical data will be sent to SCAT.CAR.
See the procedure HARDCOPY for obtaining hardcopy after the run.
\begin{verbatim}
    Invocation: ICL> CAR_SCATTER
      Parameter list:
       input           = Name of input catalogue
       xaxis           = Field or QCAR expression for X axis
                        (You can use constants to convert units, and can put your
                         own label for the expression in front, separating them
                         by `!'. e.g.
                         FIR_COLOUR!FLUX(3)/FLUX(4),GLONG!GLONG*57.295779 deg
                         If label longer than 17 characters you will be prompted)
       xunits          = Units for the X axis
       xlabel          = Label for X axis
       xerr_lim        = Maximum number of arithmetic errors in X
       yaxis           = Field or QCAR expression for Y axis
       yunits          = Units for the Y axis
       ylabel          = Label for Y axis
       yerr_lim        = Maximum number of arithmetic errors in Y
       label           = Field for labelling points in interactive mode
       ready           = Reply should be yes or no
       more            = + for more output, 0 to refresh screen, H help, P print
       title           = Name or short description for file or plot
       marker          = Marker character one of .,+,o,x
       xmin            = Lower limit for X axis
       xmax            = Upper limit for X axis
       ymin            = Lower limit for Y axis
       ymax            = Upper limit for Y axis
       line0           = Plot regression line?
       line1           = Plot user line of gradient M1 and intercept C1
       m1              = Gradient of line1
       c1              = Intercept of line1
       line2           = Plot user line of gradient M2 and intercept C2
       m2              = Gradient of line2
       c2              = Intercept of line2
       shape           = Zero for default shape and limits, non-zero to set
       device          = Output device (HELP lists devices)
       redraw          = Redraw the diagram? - yes or no
\end{verbatim}

\subsubsection{CAR\_SEARCH}

Select objects from a catalogue meeting user defined criteria

SEARCH selects objects from the input catalogue, which meet certain criteria
specified in the QCAR query language.
A search of the catalogue is optimised by originally sorting the catalogue
on a popular field.
This field is called the primary KEYFIELD, and in astronomical catalogues
this is one of the position fields, usually declination.
If you know the keyfield value you want, location of the data is much faster.
For example, if you want to find someone's telephone number you don't start
at the beginning of the directory and then turn over each page until you
reach the page you want.
You make use of the fact that the telephone directory is sorted in
alphabetical order and you flick through the pages until you find the name
you want.
If you do not restrict a keyfield then every record in the catalogue will be
checked for your query.
Sometimes you may want to do this, for example when you are selecting objects
from the whole sky.

Some catalogues are sorted into declination  zones  and  then  within
that zone they are sorted on right ascension.
When you specify the catalogue you wish to search, you will be informed what
the keyfield is.
There is more than one keyfield, and they have a seniority order.
The position of a record is dictated by its primary keyfield value and then
by its secondary keyfield value.
In CAR up to 9 secondary keys can be defined: KEYFIELD\_1 to KEYFIELD\_9.
You can use all the keyfields to speed up the search.

SCAR does not realise that 24hr RA is the same as 0hr RA, and so queries
spanning the boundary must be split into two parts.

When you have found roughly the objects you want, using the RANGE of the
primary keyfield, it is common to use further criteria to restrict the
database you are making, through the QUERY option.
Such criteria are applied to all the records in the RANGE.
The SEARCH program will not select those records which have null values for
fields referenced in the criterion expressions.

The RANGE specification is a QCAR expression, with some restrictions; only
key fieldnames and constants may be used, and the only logical operator
allowed is .AND..
If you have no range answer .END. to the query, and the entire catalogue will
be searched, which can take a long time (for example, the IRPS takes 25000
records per CPU minute).\\
The QUERY specification is a QCAR expression, defining the objects you want
from the restricted set selected by RANGE.
This expression can contain any of the fields of the catalogue, or arithmetic
expressions including them.
In the evaluation sequence .AND. comes before .OR..\\
{\em Example 1}: The keyfield of the IRPS is declination.
In order to select records from +103045 to +123045 in DEC, and to restrict
the RA from 10h30m to 11h20m, you use;
\begin{verbatim}
    RANGE Key criterion - finish with .END.> DEC>+103545.AND.DEC<+123045.END.
    QUERY - Criterion - finish with .END.> RA>103000.AND.RA<112000.END.
\end{verbatim}
{\em Example 2}: The keyfields of the CSIS are DECBAND and RA.
This means the catalogue is sorted in declination bands, and within each dec
band the objects are sorted by right ascension.
In order to select everything from DEC +80 to +82 and between RA 12h and 13h;
\begin{verbatim}
    RANGE Key criterion - finish with .END.> DECBAND>80.AND.DECBAND<82.AND.
    RANGE Key criterion - finish with .END.> RA>120000.AND.RA<130000.END.
    QUERY - Criterion - finish with .END.> .END.
\end{verbatim}
{\em Example 3}: If you wish to ignore the fact that the catalogue is sorted
then every entry in the catalogue will be examined. For example the keyfield
of the GCVS is DEC but to find a variable star with NAME SGRMV.
(N.B. the constellation name comes first);
\begin{verbatim}
    RANGE Key criterion - finish with .END.> .END.
    QUERY - Criterion - finish with .END.> NAME.EQ."SGRMV".END.
\end{verbatim}
{\em Example 4}: Suppose you have specified a DEC range for the IRPS, and you
want to look for sources with RA around 24h, you use;
\begin{verbatim}
    RANGE Key criterion - finish with .END.> DEC>+200000.AND.DEC<+250000.END.
    QUERY - Criterion - finish with .END.> RA>23300.AND.RA{240000.AND.
    QUERY - Criterion - finish with .END.> RA}000000.AND.RA<003000.END.
\end{verbatim}
{\em Example 5}: Suppose you want to have flux density restrictions as well as
the RA restriction, you use;
\begin{verbatim}
    RANGE Key criterion - finish with .END.> DEC>+103545.AND.DEC<+123045.END.
    QUERY - Criterion - finish with .END.> RA>103000.AND.RA<112000.AND.
    QUERY - Criterion - finish with .END.> (FLUX(3).GT.10.OR.FLUX(4).GT.30).END.
\end{verbatim}
{\em Example 6}: Suppose you want sources with a certain colour rather than
flux densities, in your restricted RA, you use;
\begin{verbatim}
    RANGE Key criterion - finish with .END.> DEC>+103545.AND.DEC<+123045.END.
    QUERY - Criterion - finish with .END.> RA>103000.AND.RA<112000.AND.
    QUERY - Criterion - finish with .END.> Q(3).EQ.3.AND.Q(4).EQ.3.AND.
    QUERY - Criterion - finish with .END.> FLUX(4)/FLUX(3).GT.3.END.
\end{verbatim}
{\em Example 7}: Suppose you want to find a particular IRAS source, you may
do this using the name parameter;
\begin{verbatim}
    RANGE Key criterion - finish with .END.> .END.
    QUERY - Criterion - finish with .END.> NAME.EQ."03456+0123".END.
\end{verbatim}
If the DEC range selected is suitable, you could find a number of sources;
\begin{verbatim}
    RANGE Key criterion - finish with .END.> DEC>-210000.AND.DEC<+020000.END.
    QUERY - Criterion - finish with .END.> NAME.EQ."03456+0123".OR.
    QUERY - Criterion - finish with .END.> NAME.EQ."13456-2040".END.
\end{verbatim}

The last three queries below (field, generic, groups) only appear, when you use
SELECT, if your terminal is not configured as VT100-like, so the full-screen
menu cannot be implemented.
The array required is input through GROUPS, e.g. IRPS fluxes are F(1) to
F(4), FQUALs are Q(1) to Q(4).
If you answer 'YES' to the SELECT and BATCH parameters the fields given by the
SELFIELDS parameter will be selected.

{\bf Units:} When adding or subtracting a constant, or
comparing it with an arithmetic expression, you must generally specify the
constant in the same units as the operand.
However, when it is a Radian quantity (eg RA, dec), you must present
the constant in the appropriate form for example HHMMSS for RA
and SDDMMSS for Dec.
\begin{verbatim}
    Invocation: ICL> CAR_SEARCH
      Parameter list:
       input           = Name of input catalogue
       output          = Name of output catalogue
    (  newout          = Create a new o/p file (Y), or else use existing one (N)?)
       title           = Name or short description for file or plot
       simple          = YES to make default output, NO to specify it
       range           = Give the selection criterion for key field
                         finish with .END.
                         (e.g. DEC>+103000.AND.DEC<+110000.END.)
       query           = Give the selection criterion - finish with .END.
                         (e.g. RA>110000.AND.RA<120000.AND.U(4).GT.0.END.)
       satisfied       = Answer YES or NO to verify input criterion
       options         = Do you want to specify options?
       ascii           = YES for ascii or NO for binary output
       spaces          = Do you want spaces between fields?
       index           = Ouput an Index (Y) or Master (N) catalogue?
       select          = NO=>all fields output, YES=>prompted for names
       batch           = Batch processing Y/N?
       selfields       = Names of selected fields eg. [RA,DEC]
    (  field           = YES or NO to select field                               )
    (  generic         = YES or NO to select generic                             )
    (  groups          = Array section e.g: A(n:m) or B(i:j,n:m)                 )
\end{verbatim}

\subsubsection{CAR\_SETUP}

Defines default values for some commonly used flags

You can use SETUP to define default values for the NEWOUT, MODE, SELECT,
SIMPLE, and HEADER parameters.
You only need to run this routine once.
After the default values have been defined by you, they will be HIDDEN, so you
will no longer see those prompts, and can forget about them.
You will not be prompted for those values by ANY SCAR routine.
The values are stored in ADAM\_USER:GLOBAL.SDF, where they will remain until
the file is deleted or edited.
After deleting GLOBAL.SDF, ADAMSTART must be rerun to create a new GLOBAL.SDF,
and the parameters will be prompted for again.
When you want to run a routine and use different values for some (or all) of
the defaulted parameters, you type in the new values on the command line.

{\em Example} After running SETUP to set MODE=2 which directs output to the
terminal, you type
\begin{verbatim}
    ICL> CAR_REPORT FRED
\end{verbatim}
which writes the catalogue report to the terminal only.\\
If you want to write the data to a file FRED.CAR instead, you type
\begin{verbatim}
    ICL> CAR_REPORT FRED MODE=0
\end{verbatim}
\begin{verbatim}
    Invocation: ICL> CAR_SETUP
      Parameter list:
       newout          = Create a new o/p file (Y), or else use existing one (N)?
       mode            = 0=file, 1=screen & file, 2=screen, 3=screen & request file
                         (0 - print only,  1 - screen and print (as seen)
                          2 - screen only, 3 - screen and print to requested file)
       header          = Do you require a page header?
       select          = NO=>all fields output, YES=>prompted for names
       simple          = YES to make default output, NO to specify it
\end{verbatim}

\subsubsection{CAR\_SORT}

Sort a catalogue on one or more fields

The SORT routine is a utility for reordering catalogues.
Up to ten keys can be specified, a primary key and 9 secondary keys.
The order of each key can be ascending or descending.
The program employs the VAX/VMS sort software, and is extremely efficient.
Whilst catalogues can be sorted directly using VAX/VMS sort, the SORT
routine also the handles the production of description files for the sorted
data, and instructs the sorter in the location of the fields which will
become keys.

When you sort a catalogue on declination you must be aware that the normal
text format for DEC strings SDDMMSS is such that the physical order (sources
from one pole to another) does not correspond to the ASCII order (+ comes
before -- in the ASCII collating sequence).
Because of this, when you are sorting on DEC, before using a catalogue with
JOIN, you must ensure that the units of DEC are numeric --- viz: RADIAN.
SORT must be run before JOIN.

There two ways of sorting a file in VMS, file sort and record sort.
File sort passes the whole file to the sorter and gets back a  file sorted
according  to  the  key  specification.
This is the DCL SORT command packaged for CAR.
In a file sort use TAG if your filespace is limited.
In a TAG sort the sorter keeps only the keyfield values and record pointers
in its workspace.
TAG is slower for small files but faster for larger (100~000 records plus)
files.
If you opt to change the record format then a record sort is used.
Record sort uses the VAX/VMS sort record interface.
Records are passed individually to the sorter, and when all the records have
been passed, the sort takes place.
Records are then got back from the sorter and written to the output file.
Record sort encapsulates a CONVERT operation followed by a simple SORT.
You should only need to use it when your filespace is limited as there is no
intermediate file.
Record sort is substantially slower than File sort as it uses FORTRAN I/O
to read and write the input and output files.
SORT uses two workfiles, SORTWORK0 and SORTWORK1, if it needs to buffer
information on disk, when sorting a large file.
You should assign these files to a disk where there is adequate unused
capacity, e.g. DCL ASSIGN DISK\$IRAS: SORTWORK0.

When a KEYED access (ISAM) file is sorted, it is only rewritten according
to the new key specifications.
Note that KEYED access files permit 4 keys, which must be of INTEGER or
CHARACTER data type.
The radian quantities can be stored as integer if the units scale factor is
--8 for RA, and --9 for DEC, longitude and and latitude.
A further restriction on KEYED accessmode files is that the primary key
values should be unique, so for astronomical catalogues make NAME the
primary key, and position fields the alternate keys.

The last three queries below (field, generic, groups) only appear, when you use
SELECT, if your terminal is not configured as VT100-like, so the full-screen
menu cannot be implemented.
The array required is input through GROUPS, e.g. IRPS fluxes are F(1) to
F(4), FQUALs are Q(1) to Q(4).
If you answer 'YES' to the SELECT and BATCH parameters the fields given by the
SELFIELDS parameter will be selected.

\begin{verbatim}
    Invocation: ICL> CAR_SORT
      Parameter list:
       input           = Name of input catalogue
       output          = Name of output catalogue
       simple          = YES to make default output, NO to specify it
       keys            = Key fields e.g: DEC or [DEC,RA]
       ascend          = Sort values in ascending order? e.g: Y or [Y,Y]
       tag             = YES => use less filespace but more time (usually say NO)
       options         = Do you want to specify options?
       ascii           = YES for ascii or NO for binary output
       spaces          = Do you want spaces between fields?
       index           = Ouput an Index (Y) or Master (N) catalogue?
       title           = Name or short description for file or plot
       select          = NO=>all fields output, YES=>prompted for names
       batch           = Batch processing Y/N?
       selfields       = Names of selected fields eg. [RA,DEC]
    (  field           = YES or NO to select field                               )
    (  generic         = YES or NO to select generic                             )
    (  groups          = Array section e.g: A(n:m) or B(i:j,n:m)                 )
\end{verbatim}

\subsubsection{CAR\_SPLIT}

Split a catalogue into two catalogues according to a criterion

SPLIT is a variant of SEARCH that allows you to split a catalogue into
two catalogues according to a selection criterion.
The first catalogue, OUTPUT, contains the objects matching the selection
criterion.
The second catalogue, REJECTS, contains the objects not fitting the selection
criterion.
The entire input catalogue is scanned.

The last three queries below (field, generic, groups) only appear, when you use
SELECT, if your terminal is not configured as VT100-like, so the full-screen
menu cannot be implemented.
The array required is input through GROUPS, e.g. IRPS fluxes are F(1) to
F(4), FQUALs are Q(1) to Q(4).
If you answer 'YES' to the SELECT and BATCH parameters the fields given by the
SELFIELDS parameter will be selected.

\begin{verbatim}
    Invocation: ICL> CAR_SPLIT
      Parameter list:
       input           = Name of input catalogue
       output          = Name of output catalogue
    (  newout          = Create a new o/p file (Y), or else use existing one (N)?)
       title           = Name or short description for file or plot
       rejects         = Name of output catalogue of rejects
       simple          = YES to make default output, NO to specify it
       query           = Criterion - finish with .END.
                         (e.g. DEC>000000.END.)
       satisfied       = Answer YES or NO to verify input criterion
       options         = Do you want to specify options?
       ascii           = YES for ascii or NO for binary output
       spaces          = Do you want spaces between fields?
       index           = Ouput an Index (Y) or Master (N) catalogue?
       select          = NO=>all fields output, YES=>prompted for names
       batch           = Batch processing Y/N?
       selfields       = Names of selected fields eg. [RA,DEC]
    (  field           = YES or NO to select field                               )
    (  generic         = YES or NO to select generic                             )
    (  groups          = Array section e.g: A(n:m) or B(i:j,n:m)                 )
\end{verbatim}

\subsubsection{CAR\_WITHIN}

Select points inside or outside a polygon

WITHIN selects objects from a catalogue that lie either inside or outside
a specified polygon.
The coordinates of the corners defining the polygon are themselves held as
a catalogue, made using POLYGON and EDIT.
You are prompted for the name of the catalogue holding the polygon,
followed by the names of the fields that hold the X and Y coords of the
polygon corners, and the name of the input catalogue, which may be either
a master or an index catalogue.
The name for output catalogue to hold the selected objects is requested.
If a description file for the output catalogue did not previously exist, one
is created based on the description file of the input catalogue and you are
asked if this operation is to be done in simple or complex mode.
Once the description file for the output catalogue of selected objects has
been created, you are asked if a second output catalogue of rejected
objects is to be created.
If it is, you are prompted for the name of the catalogue and again if
no description file exists, one is created.
The mode used to create the description file for this catalogue is the same
as that used for the first output catalogue.
You are prompted for the names of the fields holding the X and Y
coordinates of the data.
Finally you are asked to choose whether points inside or outside the
specified polygon are to be selected.
The permitted responses here may be abbreviated to `I' or `O'.

The three queries below (field, generic, groups) only appear, when you use
SELECT, if your terminal is not configured as VT100-like, so the full-screen
menu cannot be implemented.
The array required is input through GROUPS, e.g. IRPS fluxes are F(1) to
F(4), FQUALs are Q(1) to Q(4).
If you answer 'YES' to the SELECT and BATCH parameters the fields given by the
SELFIELDS parameter will be selected.

\begin{verbatim}
    Invocation: ICL> CAR_WITHIN
      Parameter list:
       input           = Name of input catalogue
       output          = Name of output catalogue
    (  newout          = Create a new o/p file (Y), or else use existing one (N)?)
       title           = Name or short description for file or plot
       reject          = Do you want a REJECTS catalogue?
       rejects         = Name of output catalogue of rejects
       simple          = YES to make default output, NO to specify it
       options         = Do you want to specify options?
       ascii           = YES for ascii or NO for binary output
       spaces          = Do you want spaces between fields?
       index           = Ouput an Index (Y) or Master (N) catalogue?
       select          = NO=>all fields output, YES=>prompted for names
       batch           = Batch processing Y/N?
       selfields       = Names of selected fields eg. [RA,DEC]
    (  field           = YES or NO to select field                               )
    (  generic         = YES or NO to select generic                             )
    (  groups          = Array section e.g: A(n:m) or B(i:j,n:m)                 )
       polygon         = Name of the cat. holding the polygon coords
       polyx           = Field for polygon X coord
       polyy           = Field for polygon Y coord
       datax           = Name of field holding X coord for selection
       datay           = Name of field holding Y coord for selection
       where           = Select points inside or outside polygon?
\end{verbatim}

\subsubsection{CAR\_WRAP}

Convert an ascii file for listing on a printer

The WRAP command writes an output file with a 132 character ``wrap".
It can be used for producing a printable version of a catalogue with records
longer than 132 characters.
The input and output filenames are prompted for by the program.
You can use this routine to print out formatted catalogues of a catalogue
when your printer has NOWRAP set.
WRAP may be used after you have converted your file to ascii.
Note that the output is not an FACTS format catalogue.
\begin{verbatim}
    Invocation: ICL> CAR_WRAP
      Parameter list:
       input           = Name of input catalogue
       output          = Name of output catalogue
\end{verbatim}

\subsection{Summary of parameter definitions}
\begin{description}
\begin{itemize}
\item{\bf ACCESS} File access type - one of READ, WRITE, UPDATE
\item{\bf ACTION} Dump file(D)/Reject file(R)/View more(V)
\item{\bf ANGLE} Angular coordinates will be displayed as radians (YES) or
converted to character string as specified in the comments field (NO).
\item{\bf AR} The aspect ratio of X/Y. 1.33 is the usual value for the CANON
 or screen.
\item{\bf ASCEND} To sort the values in ascending order, a list should be
given, specifying whether the keyfields in the KEYS list are
ascending (YES) or descending (NO).
If you have more than one key to specify enclose the list in square brackets.
\begin{verbatim}
    Examples:        primary key (e.g. DEC) ascends : YES or Y
    primary and secondary (e.g. [DEC,RA]) descending: [NO,NO] or [N,N]
\end{verbatim}
\item{\bf ASCII} YES for ascii or NO for binary output.
Answer YES if you want an ascii output file, NO if you want a binary file.
If you answer YES to this prompt, you will be asked if you want spaces
between each field so the ascii file is easy to read.
You can print ascii file when it has been made.
\item{\bf BARCHART} Do you want to plot barchart(Y) or line graph(N)?
\item{\bf BATCH} If you answer Y then all parameters can be specified on the
command line.
You should supply all parameter values in the command line.
\item{\bf BINSIZE} Give the bin size for the histogram (in HISTOGRAM).
A guide  to sensible values can be obtained from the statistical data.
\item{\bf C1} When plotting your own LINE1 C1 is the intercept and M1 the
gradient.
\item{\bf C2} When plotting your own LINE2 C2 is the intercept and M2 the
gradient.
\item{\bf CALMAG} Field to contain the calibrated magnitude.
\item{\bf CENTRES} Name of centres file for plotting finding charts.
If you are going to enter your plate centres interactively, type TT,
otherwise give the name of the file which contains your plate centres.
This file should be a FACTs style file with an EQUINOX defined as a a
parameter or a field and NAME, RA and DEC defined as fields in RADIANs.
It could be another catalogue for example.
The maximum number of centres allowed is 30.
If your catalogue contains more than 30 objects, only the first 30 will be
used.
You can use the START\_RECORD parameter to set the first centre position to
be used.
\item{\bf COMMAND} This is the EDIT command.
You can type H for a list of commands.
\begin{verbatim}
    +   Copies the main buffer to the memory buffer
    -   Copies the memory buffer to the main buffer
    A - Append the main buffer to the OUTPUT file
    B - Backspace to previous record (not allowed when searching)
    C - Copies a record from the INPUT file to the OUTPUT file (in WRITE
        mode)
    D - Deletes a record from the OUTPUT file. This can only be used with
        KEYED access files.
    E - EXIT - end editing
    F - Changes between the INPUT and OUTPUT file, for displaying records
        in WRITE mode
    H - Get help on commands
    I - Insert a record. You will be prompted for the values  of each
        field in turn. You can add them in any format - the format
        displayed is for information only, showing the range and
        precision in which the data is stored.
    L - List the contents of the main buffer and its record number
    N - Loads the next record meeting the search criterion into the main
        buffer
    P - Position the INPUT file by record number. The record is loaded
        into the main buffer.
    Q - QUIT - this option leaves the editor, and deletes any data files
        created during the edit, except the description files
    R - Replace a field value with a new value. You can replace them in
        any format - the format displayed is for information only, showing
        the range and precision in which the data is stored.
    S - Position the INPUT file by QCAR expression.  EDIT searches forward
        from the current record to find the next occurrence of the VALUE of
        FIELD, and the record is loaded into the main buffer.
    U - Rewrites the main buffer to the OUTPUT file
    W - Write to the OUTPUT file. In CREATE and UPDATE mode the current
        contents of the main buffer are appended to the output file and the
        buffer is reinitialised. In EDIT mode you will prompted with NUMBER,
        the number of records you wish to write to the output file. NUMBER=1
        is the current record. NUMBER>1 copies the current record and
        successive records from the INPUT file to the OUTPUT file and then
        loads the next record into the main buffer.
    X - Cancels the search.
\end{verbatim}
\item {\bf COMMENT} Comment on the field or expression defining field.
\item{\bf COORDINATES} Give coordinate system code ---
1: equatorial, 2: galactic, 3: supergalactic, 4: ecliptic.
Equatorial fields are RA and DEC, galactic are GLONG and GLAT,
supergalactic are SGLONG and SGLAT, and ecliptic are ELONG
and ELAT.
\item{\bf  COSMAG} Field containing the COSMOS magnitude.
\item{\bf DEC} Give DEC of centre in double quotes i.e. not +610000 as usual
but {\tt "+61 00 00"} (E to end for CHART).
Enter the DEC of your plate centre in any format using spaces and commas to
separate hours minutes and seconds.
If you have more than one number, enclose the string in double quotes.
Type E when you have no more centres.
\item{\bf COUNT} Include a running record count in the output (Y/N)?
\item {\bf DEFINE} QCAR expression for evaluating field or parameter
\item{\bf DEVICE} An uptodate list of options for the DEVICE parameter can be
found by replying HELP to this prompt.
\item{\bf DEVTYPE} Device type. -1 uses terminal setting, 0 scrolls.
DEVTYPE allows the application programmer to force VIO
to override the actual device type with that set by
user.  This can be used to force "scrolling" mode
by giving the value 0 (UNKNOWN).  If you give the value -1
then actual terminal setting will be used
\item{\bf DIRECT} Is the plot (in CHART) to be a direct or reverse overlay?
The options are --- Yes: E to right, emulsion down; No: E to left, emulsion up.
\item{\bf EQUINOX} Give the equinox for the centre, and it will be precessed
from the catalog equinox to this value.
\item {\bf ERRBOX} Do you want an error box?
\item{\bf EXPRESS} Give the functional definition of the field/parameter.
In the functional definition of the FIELD `!' is inserted if there is a
functional definition, used in FORM2.
If you put in HHMMSS etc without a functional definition then put `!' at
the end of the COMMENT.
\item{\bf FIELD} A field is a single parameter, which may be defined using
an arithmetrical expression.
\item{\bf FILE\_NAME} Specification of file to contain data.
\item{\bf FORMAT1} Format of the field: Iw, A, Dw.d, Fw.d, Ew.d, Lw, C*, I*,
R*.
\item{\bf FORMAT2} Format of the null value: Iw, A, Dw.d, Fw.d, Ew.d, Lw.
\item{\bf FREQ} Select every Nth object, give N.
\item{\bf GENERIC} A generic is an array of similar fields, e.g. IRPS fluxes
are stored as F(4).
\item{\bf GROUPS} Give array section e.g: A(n:m) or B(i:j,n:m).
Note that this parameter is used for selecting fields when not in screen mode.
This is needed to specify subsets of, for example, the IRPS fluxes, via
F(1:2), flux qualities via Q(2:3).
You can specify up to 2 array sections.
If you give more than one section you must enclose the input string in
square brackets.
\begin{verbatim}
    Examples are:
        1) A(1:2) selects A(1) and A(2)
        2) A(2) selects A(1) and A(2)
        3) [A(1:2),A(10:12)] selects A(1), A(2), A(10), A(11), A(12)
        4) B(1:1,1:2) selects B(1,1), B(1,2)
\end{verbatim}
\item{\bf HEADER} Do you require a page header (Y/N)?
\item {\bf INCLUDE} Name of a catalogue description file to be included.
\item{\bf INDEX}   Output an Index (Y) or Master (N) catalogue.
Answer YES if you want to output an index.
If you have input a master catalogue, a local index will be made.
If you have input an index, the index information will be retained and the
indexed data inserted.
Answer NO if you do not want to output an index.
If your input is an index, the index information will be removed and the
indexed data inserted.
A Master catalogue is self contained and contains no pointers to other
catalogues.
If you have answered Yes to the ascii prompt, you would normally generate a
Master catalogue, as an ascii catalogue is something you may want to archive
or export.
An Index catalogue contains pointers to another catalogue where the actual
data is held.
You would normally select fields when generating an Index catalogue (answer
YES to the SELECT prompt).
When you select fields for an Index, you normally select the keyfields of the
input catalogue.
While Master catalogues have the advantages of being self contained, they have
the disadvantage of taking a longer time to write than Index catalogues and
are more expensive in the use of disk space, because they contain more data.
The disadvantage of Index catalogue is, that if the Master catalogue it
depends on changes, the Index may become invalid.
SCAR commands (unless explicitly indicated otherwise) will accept an Index
catalogue as input, and process it like a Master catalogue.
\item{\bf INFILE} Give the name of the input file (for LISTIN), to be
reformatted as an ADC catalogue.
\item{\bf INPUT} Give the name of the input catalogue.
The data file INPUT must have a corresponding description file
with filename DSCFINPUT in existence which describes the physical
format of the INPUT data.
The search procedure for the INPUT file is $[\;]$INPUT.DAT, INPUT,
SCAR\_DAT\_PATH:INPUT.DAT.
The first item in the list is a file in the default directory.
The second item is a JOB logical name INPUT defined by you.
The third item causes SCAR to look in the central data file directories.
The search path for description files is similar.
The INPUT filename may be 1 to 35 characters long.
If you use the JOIN routine (the filename length + the longest
fieldname + 2) should be less than or equal to 17 characters, as the INPUT
filename is used to label fields globally.
Fieldnames have a maximum length 17 characters.
It is therefore recommended that the length of the INPUT filename be 7
characters or less.

Where an application expects a list of input catalogs, enclose the
list in square brackets $[\;]$.
\begin{verbatim}
     CAR_SEARCH INPUT=IRPS
     CAR_JOIN INPUT=[IRPS,CATX]
     CAR_MERGE INPUT=[A,B,C,D,E]
\end{verbatim}
\item{\bf INPUTS} This is a variation of the INPUT parameter, to allow
more than one catalogue to be input.
When there is more than one catalogue, the list should be enclosed in
square brackets.
The number of catalogues permitted depends on the routine, and the order
of the catalogues is significant in DIFFER.
\begin{verbatim}
    Examples:
       CAR_JOIN INPUTS=[IRPS,CATX]
       CAR_MERGE INPUTS=[A,B,C,D,E]

    Permitted number of catalogues
       CAR_AITOFF       1 - 5
       CAR_CHART        1 - 5  (4 if background stars selected)
       CAR_DIFFER       2
       CAR_JOIN         2
       CAR_MERGE        2 - 9
\end{verbatim}
\item{\bf ISAM} The choice is between an ISAM file and a direct access file.
If you have a database that needs frequent updating, answer YES.
The reply sets the ACCESSMODE parameter in the description file to KEYED or
DIRECT.

A typical arrangement would be, for ACCESSMODE=KEYED:
\begin{verbatim}
    KEYFIELD      NAME
    KEYFIELD_1    RA
    KEYFIELD_2    DEC
\end{verbatim}
Any keyfield can contain duplicate values.
Up to 4 keys are permitted.
The data types must be INTEGER or CHARACTER, so for RADIAN quantities a scale
factor of --8 is used so they can be stored as I*4.
\item{\bf ISKY} Field containing the sky intensity.
\item{\bf KEYS} Give the key fields e.g: DEC or [DEC,RA], as a list of the
fieldnames common to each of the input catalogues, and which will be used
as keyfields in the output catalogue.
If there is more than one keyfield in the list then enclose the list in
square brackets.
\begin{verbatim}
    Examples:              order by declination: DEC
     order by declination, then right ascension: [DEC,RA]
    order by name, declination, right ascension: [NAME,DEC,RA]
\end{verbatim}
In SEQUENTIAL and DIRECT type files, the order of the keys in the list
determines the sort order; the first keyfield is the primary key, the
second the secondary key etc.
Records with identical first key values will be sorted in order of the next
key in the list.
Up to 10 fields can be specified for SEQUENTIAL and DIRECT access modes.

In KEYED type files the first key is the primary key and must have unique
values for each record.
The subsequent keys are alternate keys, which may have duplicate values.
Up to 4 keys are allowed for KEYED accessmodes.
In KEYED access files it is usually best to make a NAME field the primary
key, and position fields alternate keys.
\item{\bf KEYWORD} This has a different parameter list in CALC and FORM2,
that for CALC being restricted to P for parameter, F for field, E to end,
and H for help information.
FORM2 has several other options.
\item{\bf LABEL} The field for labelling points in interactive mode in CHART.
The input value of the LABEL is plotted beside any point to identify it.
\item{\bf LENGTH} Total number of bytes allocated to the field.
\item{\bf LINE0} Plot the regression line in SCATTER?
Answer YES if you want the line representing the Y on X correlation to be
drawn, otherwise NO.
\item{\bf LINE1} Plot your own line (Y/N)? When plotting your own LINE1 C1
is the intercept and M1 the gradient.
\item{\bf LINE2} Plot your own line (Y/N)? When plotting your own LINE2 C2
is the intercept and M2 the gradient.

\item {\bf LIB} Name of catalogue VMS help library.
\item{\bf LISTMODE} Output to Screen, File or Both (S,F or B)
\item{\bf LREC} Logical record length of the data
\item{\bf MAG} Give the limiting magnitude for plotting background stars.
Only stars brighter than this limit will be plotted from the RGO Astrometry
catalogue, AS85.
The value depends on where you are in the sky, but typically 8 gives you
enough stars on your plot to position it without overcrowding it.
\item{\bf M1} When plotting your own LINE1 C1
is the intercept and M1 the gradient.
\item{\bf M2} When plotting your own LINE2 C2
is the intercept and M2 the gradient.
\item {\bf MARKER} Marker character one of .,+,o,x
\item{\bf MARKS} Number of tape marks.
\item{\bf MATCH}
This requires a QCAR logical expression that relates the primary keyfields
of the two catalogues to be JOINed, or DIFFERenced.
As the keyfields may have the same fieldnames, the fieldnames in a MATCH
specification should be prefixed by the catalogue name (no longer limited to
the first 4 characters), followed by double underscore.  e.g:  IRPS\_\,\_DEC.

To associate sources in one catalogue with those in another the QCAR function
GREAT\_CIRCLE can be used.
This calculates the great circle distance between two positions.
The result of the evaluation is expressed in arc seconds.
The function has four arguments, LONG1, LAT1, LONG2, LAT2.
It can therefore be used in any coordinate system.

Where you wish to JOIN two catalogues on a common field to within a specified
tolerance the function DIFF is used.
This function evaluates the absolute difference between two values.
It has two arguments, the names of the two fields.\\
{\em Example 1}: To JOIN the IRPS and AIPS, the MATCH condition is:
\begin{verbatim}
    IRPS__NAME.EQ.AIPS__NAME.AND.IRPS__DEC.EQ.AIPS__DEC.END.
\end{verbatim}
  The two catalogues IRPS and  AIPS  are  sorted  by  DEC.\\
{\em Example 2}: To associate every IRPS source within 90 arcsec of a RCBG
source the match condition is:
\begin{verbatim}
    GREAT_CIRCLE(RCBG__RA,RCBG__DEC,IRPS__RA,IRPS__DEC).LE.90.END.
\end{verbatim}
If you have a list of galaxies and you want to look for nearby, but
not coincident IRAS sources, then the match condition would be:
\begin{verbatim}
    GREAT_CIRCLE(RCBG__RA,RCBG__DEC,IRPS__RA,IRPS__DEC).LT.360.AND.
    GREAT_CIRCLE(RCBG__RA,RCBG__DEC,IRPS__RA,IRPS__DEC).GT.120.END.
\end{verbatim}
The order of the criteria in this expression matters --- the first
relational expression containing both of the keys is used to as the
``coarse" test.

If you wanted to associate IRAS sources with a list of galaxies, but
exclude accidental associations with stars, then the criterion:
\begin{verbatim}
  GREAT_CIRCLE(RCBG__RA,RCBG__DEC,IRPS__RA,IRPS__DEC).LT.180.AND.
  IRPS__FLUX(2)/IRPS__FLUX(3).LT.3.AND.IRPS__Q(3).EQ.3.END.
\end{verbatim}
would effectively do this.
Ordinary stars have flux ratios at 25:60 microns of about 10 and only the
brightest stars have high quality fluxes at the longer wavelengths.
You can see from the above example that you can include relational
expressions, other than those which just relate keyfields, to ``tighten"
the join criterion.
\item {\bf MAXSEL} Maximum number of objects permitted for selection.
\item{\bf MODE} There are 4 possible modes.
In mode 3, only those sources you request are loaded into the file, so when
you jump to the end no more sources are stored.
\begin{verbatim}
    Possible reporting modes are:
        0 - file                        2 - screen
        1 - screen & default file       3 - screen & request file
\end{verbatim}
\item{\bf MORE} When displaying on the screen and the complete record fills
more than one page '+' for more output, '0' to refresh screen, 'H' help,
'P' print.
\item {\bf NAME} Desired field or parameter name. The name of the field or
parameter to be created or acted upon.
\item{\bf NBINS} Give the number of bins you want in the histogram, which
determines the X axis range.
The upper limit of the X axis is determined by the number of bins needed,
the bin size and the lower X axis limit.
\item{\bf NEWOUT} Asks if you want to create a new version of an existing
output file.
If set to YES (TRUE) SCAR will ignore existing DSCF files with that name, and
will create a new version, prompting you for input.
\item{\bf NREC} The exact number of records or an upper limit
\item{\bf NOTE} Text of CATnote or ADCnote line.
\item{\bf NULLVAL} Null value of a Field or the value of a Parameter
\item{\bf NUMBER} Number of records to copy (0=rest).
\item{\bf NUMBER\_INPUT} Number the INPUT catalogue objects on a plot?
Answer YES if you want the objects from the INPUT catalogues numbered
and an accompanying listing produced in CHART.CAR.
\item{\bf NUMBER\_STARS} Number the background stars on a plot?
Answer YES if you want the stars on  the  plate numbered and an accompanying
listing produced in CHART.CAR.
Normally you would answer NO to this question since the background stars are
just used to align the chart.
\item {\bf NUMSEL} The actual number of objects to be selected.
\item{\bf OMTFLD} The name of columns in input file not to be copied. This
parameter allows you to change the name of columns that you want to omit.
Normally you would use the default name which is OMIT.
\item{\bf OPTIONS} Do you want to specify options?
If you opt to specify the output then you are asked a series of questions
about the output style.
If you prefer not to answer these questions and use the defaults, they are:
\begin{verbatim}
    Fields copied in original format
    Master catalogue created (i.e not an index)
    All relevant fields selected
\end{verbatim}
\item{\bf OVERLAY} Is the plot (in CHART) to be scaled as an overlay?
If yes the plot will overlay on top of a plate or chart.
\item{\bf PARAM}  The name of parameter whose value is required.
\item{\bf PLATE} Give name of parameter set in PLATE catalogue, used by
CHART.
\begin{verbatim}
  o PSS - Palomar Sky Survey 2 degrees by 2 degrees in gnomonic
    projection.
  o PSS/BIG - Palomar Sky Survey 4 degrees by 4 degrees in
    gnomonic projection.
  o ESOB - ESO R plate 2 degrees by 2 degrees in gnomonic
    projection.
  o ESOB/BIG - ESO R plate 4 degrees by 4 degrees in gnomonic
    projection.
  o SRC - SRC J plate 2 degrees by 2 degrees in gnomonic
    projection.
  o SRC/BIG - SRC J plate 4 degrees by 4 degrees in gnomonic
    projection.
\end{verbatim}
For TEK screens, select PSS, ESOB or SRC.
For CANON\_L select the /BIG styles.

The parameter definitions for the finding charts you can make
are stored in a file with logical name PLATE. If you wish to produce
your own style plots other than the ones set up, then use CAR\_EDIT
to modify this file and define your private parameter file as PLATE.
\begin{verbatim}
  o PLATE - Name of plot parameter (e.g: PSS)
  o NRECS - The maximum number of objects to plot
  o PAREA - The side of the square of the tangent plane
            plot in arcminutes.
  o SCALE - This is the scale of the tangent plane plot in
            arc seconds per millimeter.
  o SYMBOL- Type of symbol used for magnitude key. This sets the
            type of symbol for the magnitude key - CROSS or CIRCLE.
  o KEY   - Magnitude key (YES/NO). This determines whether the
            magnitude key is plotted.
  o GRID  - LAT and LONG grid (YES/NO). This determines whether
            you want a grid on the plot.
  o RATIO - Aspect ratio of plot.  Set to 1 for a versatec
            device, 1.3 for a raster device.
  o DIRECT- DIRECT for a positive, REVERSE for a negative.
  o CROSS - Controls whether you have cross at centre (TRUE/FALSE).
  o MARKER- Special markers for projection (YES/NO).  If NO
            then all objects are plotted with a dot "." and the
            symbol control file is ignored. You would set MARKER
            to NO if you were plotting many 1000's of objects.
  o EPOCH - The date the plate was made. The default for all
            plates is 1950. The positions are proper motion
            corrected from the catalogue epoch to the plate epoch. If
            you want accurate star positions then you should find out
            the date at which the plate was made and set this
            parameter.
  o EQUINOX-The equinox for the coordinate system applied to
            the plate. If you have a GRID, then the lines will be
            drawn for this equinox. Catalogue positions will be
            precessed from the catalogue equinox to this equinox.
            Likewise your field centre of will be precessed from the
            equinox you specify with it to this equinox. The default
            value is 1950.
\end{verbatim}
\item{\bf PLTSCL} Scale of plate from which data were measured (arcsec/mm)
\item{\bf PRINTFILE} Name of VMS file to which output is directed.
\item{\bf QUERY} The QUERY specification is a QCAR expression defining the
objects you want to select which have been extracted by RANGE. This
expression can contain any of the fields in the catalogue, or arithmetic
expressions of them, and it must be terminated by .END..

Suppose you have specified a DEC range. You can restrict the RA range to
10 hours 30 minutes to 11 hours 20 minutes by:
\begin{verbatim}
    RA>103000.AND.RA<112000.END.
\end{verbatim}
Suppose you want to further restrict your selection to those objects
with a 100 microns flux density greater than 30 Jansky or a 60 micron flux
density greater than 10 Jansky:
\begin{verbatim}
    RA>103000.AND.RA<112000.AND.(FLUX(3).GT.10.OR.FLUX(4).GT.30).END.
\end{verbatim}
Suppose you want instead to extract everything in the range with a
certain far infrared colour, but use only those objects which have
high quality fluxes:
\begin{verbatim}
    RA>103000.AND.RA<112000.AND.
    Q(3).EQ.3.AND.Q(4).EQ.3.AND.FLUX(4)/FLUX(3).GT.3.END.
\end{verbatim}
Suppose you want to extract information about a source with an IRAS name,
you can speed up the search by restricting the DEC range, and then using:
\begin{verbatim}
    NAME.EQ."03456+0123".END.
\end{verbatim}
\item{\bf RA} Give RA of a plate centre in double quotes (E to end for CHART).
Enter the RA of your plate centre in any format, using spaces and commas to
separate hours minutes and seconds.
If you have more than one number, enclose the string in double quotes.
Type E when you have no more centres.
\item{\bf RANGE} The RANGE specification is a QCAR expression with some
restrictions; only key fieldnames and constants may be used, and the only
logical operator permitted is .AND.. The expression must be terminated by
.END..

The primary keyfield of the IRPS is DEC. In order to select all records
from +103045 to +123045 you would specify:
\begin{verbatim}
    DEC>+103045.AND.DEC<+123045.END.
\end{verbatim}
The CSIS is sorted in declination bands and within each dec band it is
sorted by right ascension. The keyfields are DECBAND and RA. In order to
select everything from DEC +80 to +82 and between RA 12 hours and
13 hours you would specify:
\begin{verbatim}
    DECBAND>80.AND.DECBAND<82.AND.RA>120000.AND.RA<130000.END.
\end{verbatim}
In some circumstances you will not want to specify a range in which case every
entry in the catalogue will be considered.
\begin{verbatim}
    .END.
\end{verbatim}
\item {\bf READY} The reply should be yes to continue or no to respecify.
\item{\bf RECORD} The absolute record number of the next record to be displayed
\item {\bf REDRAW} Redraw the diagram? - yes or no. All the options are
repeated when the plot is redrawn except the input catalogue.
\item{\bf REJECT} Do you want to create a catalogue to contain the records
that are REJECTed (Y/N)?
\item{bf REJECTS} The name of the output catalogue for the rejected records.
\item{\bf SATISFIED} After a RANGE and QUERY, or a MATCH, you are asked if
you are satisfied with what you have typed in.
This gives you an opportunity to check (on the preceding lines) for
spelling mistakes and logical errors.
If you answer YES the computer will endeavour to execute the request, with
very little checking of its own.
\item{\bf SAVE} When asked about saving the FIELD or PARAMETER definition,
answer YES to save it, NO to discard it.
If you save it then it is written to the description file, otherwise it is
discarded.
\item{\bf SCALE} Scale factor for units (or exponent). Value=Value*10**EXPONENT.
\item{\bf SCOPE} The scope refers to the scope of possible matches in the
second catalogue for each entry in the first catalogue. In most joins the
join criteria is expression relational operator constant or constant
relational operator expression in which case the scope is defined by
the constant and the default 0 should be taken as the value of the
scope parameter for example
\begin{verbatim}
          great_circle(t__ra,t__dec,s__ra,s__dec).lt.300.end.
\end{verbatim}
 When the join criteria is expression relational
operator expression the scope parameter is required for example
\begin{verbatim}
          great_circle(t__ra,t__dec,s__ra,s__dec).lt.err.end.
\end{verbatim}
where err is a field that contains the maximum error in position
for this entry. SCOPE should be the maximum value of err over all
enteries. If the scope is too large the join will be correct but
slow, too small and the join result may be incomplete.
\item{\bf SCROLL} When displaying a catalogue on the screen use '+' to go onto
the next page, '-' to bring up the previous page, '0' to refresh the current
page, 'J' to jump to a specified record (you will be prompted for the record
number), 'P' print the current page , 'H' help, 'E' end. CAR\_CALC displays the
values of the new fields that have just been created, 'J' at this point causes
the calculation for the new fields to be calculated for all the records in the
catalogue.
\item{\bf SELECT} If your terminal is not configured as VT100-like, the full
screen edit mode cannot be invoked, so the SELECT option will appear.
If you want to select certain fields for output (not the default set) you will
be subsequently asked for the field/generic/group names.
\item {\bf SELECT\_NUMERIC} Reply NO to select all the numeric fields, YES
to select the fields.
\item{\bf SELFIELDS} Used in conjunction with batch. SELFIELDS allows you to
provide a list of fields you wish to select. Example [DEC,RA,FLUX(1)]
\item{\bf SHAPE} For SHAPE=0 the grid window fills the available area using
the limits of the data computed.
SHAPE$>$0 specifies a shape determined by the range you specify for X and Y,
this is the aspect ratio.
The value --1.61803 gives the golden rectangle.
For further information, see the GRID/SHAPE parameter in the NCAR manual.
\item{\bf SIMPLE} YES to create default output, NO to specify the fields.
The default output depends on the database process.
The default output is usually of the index type, to speed the processing and
conserve disk space.
\begin{verbatim}
    Options for subroutine default output style:
        SEARCH      Local Index
        DIFFER      Local Index
        JOIN        Global Index
        MERGE       Global Index
\end{verbatim}
If you find that you need to convert every output file you make, using the
options mode (SIMPLE=NO) with each database operation will save you that
step.
\item{\bf SPACE} The number of blank spaces between columns.
\item {\bf SPACES}  Do you want spaces between fields? Putting spaces between
the fields when creating an ascii catalogue makes the resulting catalogue
data file easier to look at.
\item{\bf STARS} Should the background stars be plotted?
If background are to be plotted then you should enter YES, otherwise
the answer is NO.
The stars are taken from the RGO Astrometry catalogue, AS85.
If you want to align an overlay on a photographic plate you will need
background stars.
\item{\bf START} The position of start of the field in bytes.
\item{\bf START\_RECORD} If you are reading centers from a file START\_RECORD
allows to specify where to start taking centres from'
\item{\bf STORE} Do you want to store the bin counts in HIST.CAR (Y/N)?
\item{\bf STPSIZ} Step (or pixel) size for the dataset.
\item{\bf STYLE} This is used to determine what kind of plot CHART will
produce (Astrometric or Photometric), and to control the symbols it uses.
It determines whether the object markers will be related to position and
positional errors, or magnitudes/fluxes.
The symbol control file (logical name SYMB) defines the fields for
constructing error box/ellipse, colour codes and symbols sizes.
Plotting symbols have been set up for the IRPS, AS85, MLNS and CSIS
catalogues.
All other catalogues get a default symbol.
You can replace the default SYMBol file by setting up your own SYMBol
control file.
Information is available in SCAR\_DAT\_DIR:DSCFSYMB.DAT.
A description of the SYMB file contents is given below.

1) Astrometric Symbols --- in the symbol file you can specify the fields to
be used for dimensioning the error box/ellipse and orienting it.
These are:
\begin{verbatim}
    o BOX    - RECTANGLE/ELLIPSE are permitted
    o MAJOR  - Longer side of box (QCAR expression - Unit: ARCSEC)
    o MINOR  - Shorter side of box (QCAR expression - Unit: ARCSEC)
    o POSANG - Position Angle Expression (In Degrees East of North)
\end{verbatim}
For BOX=RECTANGLE, MAJOR is the length of longer side, MINOR the length of
the shorter side.
If your position errors are DELX and DELY in arc seconds you would specify
2*DELX and 2*DELY.
For BOX=ELLIPSE, MAJOR is 2*semi-major axis and MINOR is 2*semi-minor axis.

2) Photometric Symbols --- there is a vocabulary of 32 symbols for plotting
sources.
Each symbol can be varied in size and shape to represent the brightness and
colour (found from the ratios of fluxes) of the object.
The catalogue considered here is IRPS, with flux densities (in Janskys) at
4 wavelengths.
The symbol size is a QCAR expression defined to calculate the flux in each
of the wavebands.
The unit of the expression should be magnitude.
To compare the brightness of sources at different wavelengths the
wavelength*flux density (nu*Snu) expression is derived (in units of 10**--15
Watts per square meter).
Zero point relations (cf: Allen, Astrophysical quantities p:197) have been
used for MV and MB.
\begin{verbatim}
    Band       Log10(nu*Snu)
    V          7.32-0.4*MV
    B          7.47-0.4*MB
    12mic      2.40+log10(S12)
    25mic      2.08+log10(S25)
    60mic      1.70+log10(S60)
    100mic     1.48+log10(S100)
\end{verbatim}
These expressions are normalised to the MB scale by subtracting 7.47 and
multiplying by --2.5.
To set a minimum size for legibility the CHART program uses a cutoff in
magnitude at 11.5.
This gives a minimum symbol size of about 2mm, which means a cutoff in
scaling at about 10 Janskies for IRAS sources.
For each source the largest flux is used to scale the symbol size.
Up to 4 wavebands are permitted by the algorithm.
A colour code formula is defined by a set of Boolean flags, i.e. T/F or 1/0.
You can specify conditions for each catalogue to fulfil.
As an example: the IRAS catalogue (which is allowed to have four condition
expressions) has four bands in which flux qualities are given, Q(1:4).
Each Q can have values 1 ,2, 3.
A valid flux measurement is Q(n)$>$1 so a condition, COND(1), is set
Q(1).GT.1 etc.
Suppose for a source the 4 condition expressions give T,T,T,F (the Boolean
flag for that source), this is assembled to give 1110 (=14) for symbol
number OFFSET+14.
Since the IRPS has OFFSET set to 0, the source would be plotted with the
shape of symbol 14.
Symbols are allocated contiguously for each catalogue starting from
OFFSET+1.
\item{bf SWIDTH} Required width of screen listing must be between 32 and 75.
\item{\bf TAPE\_DEVICE} Tape device name.
\item{\bf TITLE} The name or short description for file or plot is given.
For database programs, up to 50 characters are allowed.
For HISTOGRAM, SCATTER and AITOFF, up to 20 characters is acceptable.
In CHART strings are truncated to 10 characters.
\item{\bf UNIT} Requires the units of the field.
If the units of an expression are unknown or cannot be derived, you will
be prompted for them.
They are used for labelling the field.
\item{\bf VALUE} Value of a field.
\item{\bf WIDTH} The WIDTH must be between 32 and 160 characters.
If you have selected MODE=0 (in REPORT) you will be prompted for the width
of the report.
Obviously you will want to set it to the width of your output device
(e.g. 132 for a line printer, although some line printers will take 160).
In any other mode, the width is taken from the setting of the screen width
on your terminal.
\item{\bf XAXIS} Give field or QCAR expression for X axis.
Field is a simple field name (e.g. DEC).
The QCAR expression follows the normal QCAR syntax rules.
When providing an expression you can prefix the expression with the XLABEL,
and use `!' as the delimiter; for example VELOCITY!V.
The string VELOCITY will be used to label the X axis.
\item {\bf XERR\_LIM} Maximum number of arithmetic errors in X.
\item{\bf XLABEL} Give a label for the X axis.
If the expression is too long for the axis label (when XAXIS is invoked),
the prompt XLABEL will appear.
Type in a shorter axis label (less than 17 characters) and press return.
\item{\bf XMAX} Give the upper limit for the X axis.
\item{\bf XMIN} Give the lower limit for the X axis.
\item{\bf XSCALE} Give the X axis scale factor.
XSCALE=1 will file the make the chart fit the screen in the X direction.
\item{\bf XSHIFT} Give the X axis origin.
XSHIFT is measured in world coordinates,  which are usually defined so that
0,0 is the bottom left hand corner, and the shortest side of the screen
measures 1 in that direction.
For example the RH extremity of a TEK screen is about 1.4.
\item{\bf XTREMA} Select LARGEst or SMALLest values in the field?
\item{\bf XUNITS} Units for the X axis'
\item{\bf YAXIS} Give field or QCAR expression for Y axis.
Field is a simple field name (e.g. DEC).
The QCAR expression follows the normal QCAR syntax rules.
When providing an expression you can prefix the expression with the YLABEL,
and use `!' as the delimiter; for example VELOCITY!V.
The string VELOCITY will be used to label the Y axis.
\item{\bf YERR\_LIM} Maximum number of arithmetic errors in Y.
\item{\bf YLABEL} Give a label for the Y axis.
If the expression is too long for the axis label (when YAXIS is invoked),
the prompt YLABEL will appear.
Type in a shorter axis label (less than 17 characters) and press return.
\item{\bf YMAX} Give the upper limit for the Y axis.
\item{\bf YMIN} Give the lower limit for the Y axis.
\item{\bf YSCALE} Give the Y axis scale factor.
YSCALE=1 will file the make the chart fit the screen in the Y direction.
\item{\bf YSHIFT} Give the Y axis origin.
YSHIFT is measured in world coordinates which are usually defined so that
0,0 is the bottom left hand corner, and the shortest side of the screen
measures 1 in that direction.
For example the RH extremity of a TEK screen is about 1.4.
\item{\bf YUNITS} Units for the Y axis.
\end{itemize}
\end{description}

\subsection {QCAR Language}

QCAR is the language for expressing what you want CAR to do to fields of a
catalogue.
You use it for stating queries, calculating new fields, etc.
If you have a smattering of Fortran then you will find it easy to pick up.
\subsubsection {Introduction}
A QCAR statement or expression is a Fortran style expression.
There are  three types:
\begin{itemize}
\item Arithmetic
\item Relational
\item Logical
\end{itemize}
An {\em arithmetic expression} is precisely like a Fortran arithmetic
expression, using constants, variables, arithmetic operators, and function
references.
Differences from Fortran are, on the whole, very slight; the only substantial
ones are:
\begin{itemize}
\item Text constants must be enclosed in {\tt " "} rather than {\tt ' '}.
\item The index of a generic must be a literal integer constant, not
a variable or an expression.
\item A constant of any type will be interpreted in a totally non-Fortran-like
way if the context shows that it refers to a positional quantity (see Radian
constants.
\end{itemize}
A {\em relational expression} is simply a statement comparing two expressions,
i.e.\ using any of the relational operators.

A {\em logical expression} is either a relational expression, or a combination
of relational expressions connected by any of the logical operators
(.NOT., .AND., .OR., .XOR.).
\subsubsection {Radian constants}
{\bf Syntax:} In most online catalogues, positional quantities (RA, DEC, LAT,
LONG, etc.) are represented in radians, but when such quantities are to be
displayed to the user, more approachable units are generally desirable:
DEGREES, ARCMINS, ARCSECS, or some combination of these for ANGLE quantities;
HOURS, MINS, SECS, or some combination for TIME quantities.

CAR provides two mechanisms for dealing with the problem of representing radian
quantities to the outside world: one for catalogue fields and the other for
constants within QCAR expressions.
The former is provided by means of the UNIT and COMMENT fields for the item in
the Description file.
For constants,  however, it is simplest for the user to have a single mandatory
(canonical) form for each of the two classes of positional quantities:
\begin{itemize}
\item SDDMMSS  for DEC-like or ANGLE quantities
\item HHMMSS   for RA-like or TIME quantities
\end{itemize}
{\em Definition:} Constants in either of these forms are termed ``Radian
constants".
\begin{verbatim}
    Examples:    +344503  for  +34deg 45arcmin 03arcsec
                  021041  for   2hrs 10min 41sec
\end{verbatim}
The leading sign for the ANGLE quantity is taken as + if omitted.

You may, if you wish, enclose these Radian constants in {\tt "  "} marks.
If you do, you may also embed as many blanks as you want in the string.
The software will simply remove them, so you can't use them instead of zeros.
There must always be 6 actual digits; if there are more, the software will
truncate it to 6 digits without warning.

If there is a decimal point in the constant, then it, and any digits following
it, will be ignored.
This applies equally to a constant which is enclosed in {\tt " "} marks
(i.e.\ looking like a string constant) and to one looking like a simple number.
There must be 6 actual digits preceding the point; if there are more, the first
6 will be used without any warning that the constant has been truncated.

Not-quite-canonical forms that will work, because of the 6-digit rule, are:
\begin{verbatim}
          344503.7            taken as 344503
          0344503.7           taken as 034450
          "344503.7"          taken as 344503
          "  34  4  5 03 "    taken as 344503
\end{verbatim}
but not:
\begin{verbatim}
          "  34  4  5  3 "
\end{verbatim}
which will provoke an explanatory error message.

{\bf Semantics:} A QCAR Radian constant can only be recognized by context.
The rules are quite simple, but you must be careful because you will not be
warned by the CAR program that a constant has (or has not) been treated as a
Radian constant.

{\em Definition:} Radian quantity.
In this document, a Radian quantity is defined as one of the following:
\begin{itemize}
\item a catalogue field with UNIT of RADIAN, ANGLE or TIME
\item a function whose result is an RA or DEC
\item the sum or difference of 2 Radian quantities
\end{itemize}
If a constant and a Radian quantity are to be added, subtracted, or compared,
the constant will be interpreted as a Radian constant and must be in one of the
canonical forms described above.
If you multiply, divide, or exponentiate a Radian quantity by a constant, or
vice versa, the constant is {\em not} a Radian constant.
However, multiplying or dividing a Radian quantity by a constant gives a
Radian quantity as its result.

It is very important to realize that the result of an astrometric function like
GLONG\_EG50 is {\em not} a Radian quantity by this definition, even though the
internal unit is radians.
If you add, subtract, or compare a constant with it, that constant will be
interpreted as DEGREES and will be interpreted as an ordinary integer, real, or
double precision number.
The same applies if you add, subtract, or compare a constant with the result of
ASIN, ACOS or ATAN.
And if you add, subtract, or compare a constant with the result of
GREAT\_CIRCLE, the constant will be interpreted in ARCSECS.

The table in indicates which function results are Radian quantities by
this definition.
In a relational expression involving DEC and a constant, the constant is a
Radian constant in SDDMMSS form, with or without {\tt " "} marks.
\begin{verbatim}
    DEC.GE.-010000
    DEC.GE."-010000"

    RA-060000.GE.003000
    060000.LE.RA-003000
\end{verbatim}
These are identical to each other.
All the constants are Radian constants in HHMMSS form.
\begin{verbatim}
    (RA-060000)*(DEC-100000).LT.4.0
\end{verbatim}
060000, 100000 are RA-like, DEC-like Radian constants respectively.
The product of the two bracketed expressions is not a Radian quantity, so the
constant on the rhs is not a Radian constant.
\begin{verbatim}
    RA_FK4PRE(EPOCH,1900,RA,DEC)-RA.GT.000010
\end{verbatim}
Both the quantities on the lhs are RA-like Radian quantities, and hence so is
their difference.
So the constant on the rhs is an RA-like Radian constant.
\begin{verbatim}
    RA_FK4PRE(EPOCH,1900,RA,DEC)-RA.GT.DEC+000010
\end{verbatim}
Both the quantities on the lhs are RA-like Radian quantities, and hence so is
their difference.
But the Radian constant on the rhs is, by context, DEC-like.
\begin{verbatim}
    VAR*DEC.NE.-000100
\end{verbatim}
The lhs is not a Radian quantity, so the rhs will be taken to mean precisely
--100.

The context always determines unambiguously whether a constant is to be
interpreted as a Radian constant.
If you are in doubt, try entering a constant in a form which can't be a Radian
constant.
If it is wrong in the context, you will be told so.
\subsubsection {Arithmetic Expressions}
Arithmetic expressions consist of variables, constants, function references and
arithmetic operators.
\begin{description}
\item [Variables] ---
These must be fieldnames or generic element names defined in the catalogue.
You must familiarise yourself with the vocabulary for the catalogue you want to
use by looking in the CAT HELP library for that catalogue.
Variable names are defined with some consistency eg.\  RA, DEC, GLONG, GLAT,
ELONG, ELAT, so guessing may get you some way.
When catalogues are joined and merged the fieldnames from each catalogue are
prefixed with the first four letters of the catalogue name and a double
underscore to distinguish them, eg.\  IRPS\_\_NAME.
If a CAR program responds with a syntax error or illegal constant message, the
reason is usually that a fieldname in the expression does not exist in the data
dictionary (description file) for that catalogue.
\item [Constants] ---
\begin{tabbing}
....\=Character \=eg. \=1234\kill
\>Integers\>eg.\>1, 4, 100 , -678  (up to 2147483648 ).\\
\>Reals\>eg.\>2.5, 345.678, 2.e17, 1.5E22, 1E12 (up to 1.7E38, 15 sf).\\
\>Logical\>eg.\>.TRUE., .FALSE. (only !).\\
\>Character\>eg.\>``ABCDED".
\end{tabbing}
Strings must be enclosed in double quotes ({\tt "}), except in the special case
of Radian constants (see 4.5.2 above).
You can define the units of a constant by enclosing them with a backslash and
vertical line; eg.\  $\backslash$KM/SEC$|$.
QCAR can combine the units of the variables and constants of simple arithmetic
expressions.
This is intended to allow the correct labelling of axes when an expression is
processed by, for example, the HISTOGRAM routine.
\item [Arithmetic Operators] ---
In order of evaluation:
\begin{tabbing}
xxxxx\=xxxx\=xxx\=x\kill
\>//\>=\>String concatenation (Character data only).\\
\>**\>=\>Exponentiation ( x**y means x to the power of y).\\
\>/\>=\>Division.  As in Fortran, integer division occurs if every variable\\
\>\>\>and constant in both divisor and dividend is integer.\\
\>*\>=\>Multiplication.\\
\>--\>=\>Subtraction.\\
\>+\>=\>Addition.
\end{tabbing}
The // operator can only be used with character data types and the other
operators can only be used with the other data types.\\
Examples:
\begin{quote}
The infrared colour of an IRPS source:  FLUX(3)/FLUX(4).
Flux(3) and Flux(4) are fields in the IRPS catalogue.\\
The distance of a galaxy:  C*Z*0.01/h.
C and h may be parameters of a catalogue.
Z is a field.
0.01 is a constant.
\end{quote}
\item [Functions] ---
Functions can be used in expressions.
Arguments may be constants, fieldnames or arithmetic expressions,
including function references.

Simple Functions --- These are taken from the VAX Fortran library
(except for UCASE, LCASE):
\begin{quote}
\begin{tabbing}
xxx\=STRINGxx\=Absolute differencexxxxxx\=Numberxxxxxxxxxxxx\=..\kill
\>Name\>Comment\>Arguments\>Result\\
\>SQRT\>Square root\>Number\>Number\\
\>LOG\>Natural Logarithm\>Number\>Number\\
\>LOG10\>Common Logarithm\>Number\>Number\\
\>EXP\>Exponential\>Number\>Number\\
\>SIN\>Sine\>Angle\>Number\\
\>COS\>Cosine\>Angle\>Number\\
\>TAN\>Tangent\>Angle\>Number\\
\>ASIN\>Arcsine\>Number\>Angle (radians)\\
\>ACOS\>Arccosine\>Number\>Angle (radians)\\
\>ATAN\>Arctangent\>Number\>Angle (radians)\\
\>SINH\>Hyperbolic sine\>Angle\>Number\\
\>COSH\>Hyperbolic cosine\>Angle\>Number\\
\>TANH\>Hyperbolic tangent\>Angle\>Number\\
\>ABS\>Absolute value\>Number\>Positive number\\
\>STRING\>String search\>String,Substring\>.TRUE. or .FALSE.\\
\>DIFF\>Absolute difference\>Number, Number\>Number\\
\>INT\>Truncated integer\>Number\>Integer number\\
\>NINT\>Nearest integer\>Number\>Integer number\\
\>MIN\>Minimum\>Number\>Number\\
\>MAX\>Maximum\>Number\>Number\\
\>UCASE\>Upper case\>String\>String\\
\>LCASE\>Lower case\>String\>String\\
\end{tabbing}
\end{quote}
\item [Astrometric Functions] ---
These are adapted from the SLALIB library and are used in the Coordinate
Conversion Program COCO; see SUN/67 for a full definition of the arguments.
You can `COCO' a catalogue interactively (see program CALC).
The diagram below summarises the conversions that you can do.
Proper motion correction routines (not shown) are also available.
\begin{figure}[htbp]
\begin{center}
\begin{picture}(125,48)
\thicklines
\put (10,0){\framebox(40,5){SGLONG,SGLAT}}
\put (10,20){\framebox(40,5){GLONG,GLAT}}
\put (10,40){\framebox(40,5){RA,DEC,FK4,1950}}
\put (80,0){\framebox(40,5){ELONG,ELAT,EPOCH}}
\put (80,20){\framebox(40,5){RA,DEC,FK5,2000}}
\put (80,40){\framebox(40,5){RA,DEC,FK4,EPOCH}}
\put (29,5){\vector(0,1){15}}
\put (29,25){\vector(0,1){15}}
\put (31,20){\vector(0,-1){15}}
\put (31,40){\vector(0,-1){15}}
\put (99,5){\vector(0,1){15}}
\put (101,20){\vector(0,-1){15}}
\put (100,25){\vector(0,1){15}}
\put (100,40){\vector(0,-1){15}}
\put (50,21.5){\vector(1,0){30}}
\put (80,23.5){\vector(-1,0){30}}
\put (50,42.5){\vector(1,0){30}}
\put (80,42,5){\vector(-1,0){30}}
\put (50,41){\vector(2,-1){31}}
\put (80,24){\vector(-2,1){31}}
\put (51,44){FK4PRE ($<$1984)}
\put (19,36){GE50}
\put (32,27){EG50}
\put (13,16){GALSUP}
\put (32,7){SUPGAL}
\put (101,32){FK5PRE ($>$1984)}
\put (65,18){GALEQ}
\put (52,25){EQGAL}
\put (85,16){ECLEQ}
\put (102,7){EQECL}
\put (42,35){FK54Z}
\put (76.5,28){FK45Z}
\end{picture}
\end{center}
\end{figure}
The function names are constructed as: NAME\_SLASUFFIX.
NAME is the argument of the SLA subroutine to be found,
SLASUFFIX is the 2--5 character suffix to the prefix SLA\_.
Exceptions are those SLA routines which have FK4 or FK5 as the first argument.
To use these functions, your input catalogue must contain the EQUINOX parameter
whose value should be B1950 or J2000.
FK4 is the coordinate system for the B1950 equinox and FK5 is the coordinate
system for the J2000 equinox.
NOTE: B=Bessel, J=Julian.
Units of TIME and ANGLE are converted to RADIANS if necessary.
Results in RADIANS are converted back to TIME or ANGLE if necessary.
All calculations are double precision.
All EPOCHs are given years, unless an alternative unit follows it in square
brackets (MJD = Modified Julian Date for example).
\begin{verbatim}
  Function      Arguments                   Equinox    Result is Radian Quantity
  GLONG_EG50    RA, DEC                      B1950          No (degrees)
  GLAT_EG50     RA, DEC                      B1950          No (   "   )
  RA_GE50       GLONG, GLAT                  B1950         Yes
  DEC_GE50      GLONG, GLAT                  B1950         Yes
  RA_FK4PRE     EPOCH1, EPOCH2, RA, DEC      B1950         Yes
  DEC_FK4PRE    EPOCH1, EPOCH2, RA, DEC      B1950         Yes
  RA_FK5PRE     EPOCH1, EPOCH2, RA, DEC      J2000         Yes
  DEC_FK5PRE    EPOCH1, EPOCH2, RA, DEC      J2000         Yes
  GLONG_EQGAL   RA, DEC                      J2000          No (degrees)
  GLAT_EQGAL    RA, DEC                      J2000          No (   "   )
  RA_GALEQ      GLONG, GLAT                  J2000         Yes
  DEC_GALEQ     GLONG, GLAT                  J2000         Yes
  RA_FK425      RA, DEC, PMR, PMD, PX, PV    B1950         Yes
  DEC_FK425     RA, DEC, PMR, PMD, PX, PV    B1950         Yes
  RA_FK524      RA, DEC, PMR, PMD, PX, PV    J2000         Yes
  DEC_FK524     RA, DEC, PMR, PMD, PX, PV    J2000         Yes
  ELONG_EQECL   RA, DEC, EPOCH[MJD]          J2000          No (degrees)
  ELAT_EQECL    RA, DEC, EPOCH[MJD]          J2000          No (   "   )
  RA_ECLEQ      ELONG, ELAT, EPOCH[MJD]      J2000         Yes
  DEC_ECLEQ     ELONG, ELAT, EPOCH[MJD]      J2000         Yes

  RA_PM         RA, DEC, PMR, PMD, PX, PV, EPOCH1, EPOCH2  Yes
  DEC_PM        RA, DEC, PMR, PMD, PX, PV, EPOCH1, EPOCH2  Yes

  SGLONG_GALSUP GLONG, GLAT                                 No (degrees)
  SGLAT_GALSUP  GLONG, GLAT                                 No (   "   )
  GLONG_SUPGAL  SGLONG, SGLAT                               No (   "   )
  GLAT_SUPGAL   SGLONG, SGLAT                               No (   "   )

  GREAT_CIRCLE  LONG1, LAT1, LONG2, LAT2                    No (arcsec)

  GREAT_CIRCLE  LONG1, LAT1, LONG2, LAT2                    No (arcsec)

  CSI1EXP       IDENT                                       Not Applicable
  CSI2EXP       IDENT                                       Not Applicable
  CSI3EXP       IDENT                                       Not Applicable
    :
    :
  CSI10EXP      IDENT                                       Not Applicable

\end{verbatim}
If you wish to reference galactic latitude in a catalogue that does not contain
this field, but only RA and DEC in equinox B1950, then you express the latitude
as GLAT\_EG50(RA, DEC).
This means you want the GLAT derived from the existing RA and DEC.
Eg.\ in query: GLAT\_EG50(RA, DEC)$>$80, you would be selecting on GLAT
when in fact the only coordinates in the catalogue are RA and DEC at B1950.
Eg.\ in a field definition (somewhat compressed):
\begin{verbatim}
    F GLAT  2   9   F9.3  0 ANGLE  +90.000  DEGREE!GLAT_EG50(RA,DEC)
\end{verbatim}
Here you may be defining the GLAT field to be created by CONVERT or SEARCH from
the RA and DEC coordinates in the catalogue.
If you are regularly searching a catalogue using a position function, it is
highly recommended that you make your own index of that quantity, sort it in
ascending order, and use it for accessing the catalogue.
The GIRL catalogue is an example.

When you are precessing coordinates, be aware that the fieldnames are only
parsed with respect to the input catalogue(s).
This means that the epoch you are precessing to must be given as a constant,
even though you have declared what your output epoch is using in the EPOCH
parameter.
Eg.\ suppose your input catalogue has coordinates RA and DEC in Equinox B1950 at
epoch 1950.0 and you wish to precess them to epoch 1900.0.
You would put functional definitions for the precessed coordinates as follows:
\begin{verbatim}
    Fieldname     Unit     Comment
    RA1900        RADIAN   HHMMSS.S!RA_FK4PRE(EPOCH,1900,RA,DEC)
    DEC1900       RADIAN   SDDMMSS.S!DEC_FK4PRE(EPOCH,1900,RA,DEC)
\end{verbatim}
Note that the fieldnames are different from the input catalogue -- this is because
if the fieldnames and units of a field in the input catalogue are the same as
that in the output catalogue, the value of the field is simply moved and the
functional definition is ignored.

The GREAT\_CIRCLE function is not in the SLA library, so further explanation is
required here.
GREAT\_CIRCLE computes the great circle distance between two positions on the
celestial sphere.
The longitude and latitude of the two positions (in the same coordinate system)
are (LONG1, LAT1), (LONG2, LAT2).
The units of the arguments and result are RADIANS.
This function is used for associating catalogues, ie.\ JOINing catalogues by
position agreement between the objects in it; see program JOIN (3.24).

{\em Usage:} Arithmetic expressions are used in CAR routines, for example:
\begin{itemize}
\item SCATTER to derive the values and labels on the X and Y axes
\item CALC to generate the definition of a new field
\end{itemize}

The CSI?EXP functions are used for expanding the compressed information in
the CSI79 catalogue. CSI1EXP expands the first byte of the IDENT field, CSI2EXP
the second and so on. Futher details of the expansion can be found in the
appendix.

\end{description}
\subsubsection {Relational Expressions}
Relational expressions compare two arithmetic expressions.
The standard Fortran relational operators are allowed:
\begin{verbatim}
    .LT.(<), .LE.({), .EQ.(=), .GE.(}), .GT.(>), .NE.(x#)
\end{verbatim}
Alternative symbols for each operator are shown in parentheses.
The result of a relational expression is a logical value, either: .TRUE.
or .FALSE., eg.\
\begin{verbatim}
    RA.LT."001000"
    FLUX(3)/FLUX(4)>2
\end{verbatim}
\subsubsection {Logical Expressions}
Logical expressions sum the truth values of relational expressions or
arithmetic expressions which have a logical result.
The logical operators allowed are:
\begin{quote}
{\tt .NOT., .AND., .OR., .XOR.} (in order of evaluation)
\end{quote}
The result of a logical expression is a logical value:
\begin{quote}
{\tt .TRUE. or .FALSE.}
\end{quote}

{\em Usage:} Logical expressions are used, for example, in the following
CAR routines:
\begin{itemize}
\item SEARCH to specify the RANGE and QUERY conditions
(KEY\_CRITERION and CRITERION parameters in Description file)
\item JOIN to specify the MATCH condition (JOIN\_CRITERION
parameter in Description files)
\item DIFFER, to specify the MATCH condition (JOIN\_CRITERION
parameter in Description files)
\end{itemize}
{\em Examples:}
\begin{verbatim}
    DEC.GT."+450000".AND.DEC.LT."+463045"
    LOG10(FLUX(3)).GT.1
    FLUX(3).GT.1.AND.FLUX(4).GT.10.AND.Q(3).EQ.3.AND.Q(4).EQ.3
    FLUX(3)/FLUX(4).GT.0.1
    TYPE.EQ."S".AND.VELOCITY.GE.500.AND.10.0**DIAMETER/10.0.LT.2
    STRING(NAME,"MCG")  (function has logical result so no relational operator required)
    .NOT.NAME.EQ."NGC4151"
    GREAT_CIRCLE(CATX__RA,CATX__DEC,IRPS__RA,IRPS__DEC)<120.AND.
    GREAT_CIRCLE(CATX__RA,CATX__DEC,IRPS__RA,IRPS__DEC)>60
\end{verbatim}

\section {BACKGROUND --- FACTS and the catalogues}

FACTs is the Flexible Astronomical Catalogue Transport system.
It not only resembles the Flexible Image Transport System (FITS) in name, but
also in concept.
Each catalogue possesses a file that describes its content and format in a
similar way that header files are used to describe images.
Hopefully FACTs will become a standard for astronomical catalogue data, just as
FITS has become a standard in image processing.
\subsection {Format}
The multitude of astronomical catalogues published demands that any system
hoping to become a standard for transferring data between computer systems does
not go against the status quo.
Most distributors of catalogue data (eg.\ CDS, NSSDC) provide it as formatted
files in ASCII or EBCDIC code on magnetic tape.
This is the format that FACTS accepts as the de facto standard.
Nine track tapes written at 1600 bpi density are to be preferred.
\subsection {Data Structure}
FACTS demands that astronomical catalogues fit the relational model.
A relation is a two dimensional table consisting of horizontal rows and vertical
columns.
There must be some combination of columns that uniquely identifies each row;
such a combination is called a key.
Before use, a relation must be `normalised', that is reorganised to obey the
following rules:
\begin{itemize}
\item For all rows, each column must take a simple value, ie.\ a value without
repetition.
\item In every row, each column must depend on every part of the key.
\item In every row, all columns must depend directly on the key without any
`transitive' dependencies through other columns.
\end{itemize}
The last rule means that no other column dependencies other than on the key are
allowed.
Most astronomical catalogues obey these rules.
The most often used key is one of the position coordinates or the name.

FACTS terminology is described in the rest of this section.
A CATALOGUE is a table of data.
The rows are called RECORDs and the columns are called FIELDs.
The records should all contain the same number of bytes.
Groups of contiguous fields that can be described in the same way are called
GENERICs.
As well as information in the table, there are always other quantities called
PARAMETERs which are the same for every record in the catalog.
The definitions of the fields, parameters and generics are contained in a file
quite separate from the catalogue called a DESCRIPTION FILE.
\subsection {Description File}
A Description File has two parts.
The first part consists of parameter definitions, field definitions and generic
definitions.
The end of the first part is marked by the ENDFIELD record.
The first part itself is a table, and has its own description file called the
system description file.
This allows descriptions files to be processed as catalogues\footnote
{DSCFDSCF is the name of the system description file in the CAR system.
For example try:
\begin{quote}
   {\tt SCAR$>$ CAR\_REPORT INPUT=DSCFDSCF MODE=2 SELECT?=N HEADER?=Y}
\end{quote}
You will see the system description file describing itself.
Don't try to go beyond the ENDFIELD record!
Look in appendix A for examples of description files, or print out
the files DSCFDSCF and DSCFIRPS if you have access to SCAR.}.
The self-describing aspect of the description file system ensures a high degree
of functionality in the software.
Following the ENDFIELD record is documentation on the catalogue in the
CATNOTEs and ADCNOTEs.
Each definition starts with a KEYWORD to identify it as a PARAMETER,
FIELD, GENERIC, ENDFIELD, CATNOTE or ADCNOTE definition.
\subsubsection {PARAMETER}
The KEYWORD for a PARAMETER definition is `P'.
PARAMETERs vary from catalogue to catalogue, but for a particular catalogue they
are constants\footnote
{!!! Warning:
In this document the term `PARAMETER' is used in two different senses:
\begin{itemize}
\item FACTS PARAMETERs -- which are used to describe a catalogue.
\item Program parameters -- which are values required by a program for its task.
\end{itemize}
To avoid ambiguity when the word parameter is used, UPPERCASE means FACTS
PARAMETERs and lowercase means program parameters.}.
Some PARAMETERs carry general information, like TITLE and AUTHOR.
Others carry information about the physical properties of the catalogue, like
MEDIUM.
Others carry global attributes of the objects in the catalogue, like EPOCH and
EQUINOX.

PARAMETER definitions must contain:
\begin{itemize}
\item {\bf FIELDNAME (or NAME)} -- the name of the PARAMETER.
\item {\bf UNIT} -- the unit of the PARAMETER.
\item {\bf FORMAT2 (or NULLFORMAT)} -- the Fortran format for reading the
PARAMETER value.
\item {\bf VALUE (or NULLVALUE)} -- the value of the PARAMETER.
\item {\bf COMMENT} -- the lexical definition of the quantity (optional).
\item {\bf EXPRESSION} -- the functional definition of the quantity (optional).
\end{itemize}
FIELDNAMEs used in the CAR system are:
\begin{description}
\item [TITLE] -
This contains the name of the catalogue and is mandatory.
FORMAT2 should be A.
VALUE should be a mnemonic ($<$7 characters in the worst case, but some
routines allow up to 50 characters).
COMMENT should be the ordinary title.
\item [MEDIUM] -
This is the storage medium of the data and is mandatory.
FORMAT2 should be A.
VALUE should be TAPE or DISK.
\item [ACCESSMODE] -
This is the mode of getting records from and putting them into the file.
FORMAT2 should be A.
VALUE should be SEQUENTIAL, DIRECT or KEYED; the latter is for access to
Indexed Sequential Accessmode Files.
ACCESSMODE is mandatory when MEDIUM is DISK.
\item [RECORDSIZE] -
This is the length of the record in bytes.
UNIT should be BYTE.
FORMAT2 should be Iw.
VALUE is the length of the record in bytes; this should be an integral
number of longwords (4 bytes) when ACCESSMODE is DIRECT.
\item [KEYFIELD] -
This is the sort field of the catalogue.
FORMAT2 should be A.
VALUE is the FIELDNAME of the KEYFIELD.
COMMENT should start with DESCENDING, otherwise ASCENDING order is assumed.
KEYFIELD is mandatory when MEDIUM=DISK and ACCESSMODE=DIRECT.
Secondary KEYFIELDs should be named KEYFIELD\_1 to KEYFIELD\_9.
There is an important difference between the meaning of KEYFIELD in the
relational model and the meaning used above.
In the relational model, an objects identity comes from its keys.
In an astronomical catalogue, this is either the NAME of the object or its
position fields: RA and DEC etc.
The catalogue may not actually be sorted on these fields, but they are still the
keyfields of the catalogue as far as the relational model is concerned.
In the `strong' definition of KEYFIELD, the KEYFIELD(s) identify the object and
determine its location in the catalogue.
In the `weak' definition of KEYFIELD, the KEYFIELD(s) just determine a record's
location in the catalogue.
\item [NRECORDS] -
This is the number of records in the catalogue.
FORMAT2 should be Iw.
VALUE is the number of data records in the catalogue.
NRECORDS is mandatory when MEDIUM=DISK  and  ACCESSMODE=DIRECT.
\item [BLOCKSIZE] -
This is the number of bytes in a block of data when the MEDIUM is TAPE.
FORMAT2 is Iw.
VALUE is the number of bytes in the block.
BLOCKSIZE is  mandatory when MEDIUM=TAPE\footnote
{The value of BLOCKSIZE is only used by CAR when writing a tape file.
On reading, the actual blocksize overrides the PARAMETER value.}.
\item [FILENUMBER] -
This is mandatory when MEDIUM is TAPE.
FORMAT2 is I.
VALUE is the number of the file on the tape.
\item [EQUINOX] -
This defines the time reference frame used for position measurements.
FORMAT2 should be A.
VALUE should be of the style: B1950, J2000.
\item [EPOCH] -
This is the date of the position measurements in the EQUINOX time frame.
UNIT should be YEAR.
FORMAT2 should be Fw.d or Iw.
VALUE should be the date.
\item [AUTHOR] -
This is the first name in the reference, whether a person or an institution.
VALUE should be the author's name, followed by his initials,
(eg.\ EINSTEIN, A.).
COMMENT should contain a full reference with standard abbreviations.
\item [CLASS] -
This identifies the type of catalogue.
FORMAT2 should be A.
VALUE follows the CDS system: ASTROMETRY, PHOTOMETRY, SPECTROSCOPY,
X-IDENTIFICATION, COMBINED, MISCELLANEOUS, NON-STELLAR.
\item [WAVELENGTH] -
This summarises the part of the electromagnetic spectrum from which the data
originates.
FORMAT2 should be A.
VALUE should be one of RADIO, INFRA-RED, OPTICAL, UV, X-RAY, GAMMA-RAY.
\item [OBJECT] -
This describes the type of entity catalogued.
FORMAT2 should be A.
VALUE should be specified in the singular.
Examples are STAR, STAR/G (qualifiers may be used), GALAXY.
The underscore character should be used instead of spaces,
eg.\ GLOBULAR\_CLUSTER.
If there is more than one type of object, use GENERAL.
\item [START\_RECORD] -
This is the number of the record in the catalogue where the table described
starts.
FORMAT2 is Iw.
The default VALUE of this parameter is 1.
Catalogues may contain header records before the data described in the
description file.
This practice goes against the relational model and is not encouraged.
\item [LOCAL\_INDEX] -
This indicates that the catalogue contains pointers to one other catalogue.
FORMAT2 is A.
VALUE is the mnemonic/filename of the catalogue that is indexed.
\item [GLOBAL\_INDEX] -
This indicates that the catalogue contains pointers to at least one other
catalogue.
FORMAT2 should be A.
VALUE should be blank.
COMMENT should hold a list of catalogue names; eg.\ IRPS,CATX,AS85.
\item [CRITERION] -
This specifies a logical expression that applies to each record in the
catalogue.
FORMAT2 should be L1.
VALUE should be T or F.
COMMENT must contain a logical expression that constrains any of the fields in
the catalogue.
\item [KEY\_CRITERION] -
This specifies a logical expression that applies to each record in the
catalogue.
FORMAT2 should be L1.
VALUE must be T.
COMMENT must contain a logical expression that constrains the keyfield(s) in the
catalogue.
\item [JOIN\_CRITERION] -
This specifies a logical expression that applies to each record in a catalogue
which has been produced by joining two catalogues on a common field.
FORMAT2 should be L1.
VALUE must be T.
COMMENT must contain a logical expression relating at least two of the fields in
the catalogues that have been joined.
\item [CONTINUATION] -
This is a pseudo PARAMETER to allow extension of the comment field of the
preceding field or parameter.
FORMAT2 should be A.
VALUE should be blank.
COMMENT should continue the comment field of the previous FIELD, GENERIC or
PARAMETER\footnote
{In CAR, long comments and expressions can be accommodated by using up to 9
CONTINUATION parameters.}.
\end{description}
\paragraph {Minimum Parameter Sets}
TITLE, MEDIUM, ACCESSMODE and RECORDSIZE are always required.
The table below summarises the other PARAMETERs that are required according to
the values of MEDIUM and ACCESSMODE.
\begin{center}
\begin{tabular}{||c|c|c||} \hline
{\em ACCESS MODE} & \multicolumn{2}{c||}{{\em MEDIUM}} \\ \cline{2-3}
& {\em Disk} & {\em Tape} \\ \hline
{\em Sequential} & -- & BLOCKSIZE \\
& & FILENUMBER \\ \hline
{\em Direct} & NRECORDS & -- \\
& KEYFIELD & \\ \hline
\end{tabular}
\end{center}
\subsubsection {FIELD}
The KEYWORD for a FIELD definition is `F'.
FIELD definitions must contain:
\begin{itemize}
\item {\bf FIELDNAME (or NAME)} --  the name of the quantity.
\item {\bf START} --  the position of the first character.
\item {\bf LENGTH} --  the width of the field.
\item {\bf FORMAT (or FORMAT1)} --  the Fortran format of the data in the field.
\item {\bf EXPONENT (or SCALE)} --  the scale factor for the units
(eg.\ kilo = 3).
\item {\bf UNIT} --  The unit of the quantity.
\item {\bf NULLFORMAT (or FORMAT2)} -- the Fortran format for externally
representing the field in the UNIT of the quantity and for reading the
NULLVALUE of the quantity.
\item {\bf NULLVALUE (or VALUE)} -- the value that appears in the field when
there is no data.
\item {\bf COMMENT} --  the lexical definition of the quantity (optional).
\item {\bf EXPRESSION} -- the functional definition of the quantity (optional).
\end{itemize}
Further:
\begin{itemize}
\item FORMAT specifiers for formatted fields are:
Fw.d, A[w], Iw[.w], Lw, Ew.d, Dw.d. (Mne\-monic: FAILED !)
\item FORMAT specifiers for unformatted fields are:
I*1, L*1, I*2, I*4, R*4, R*8 and C*n where n is an integer.
\item In scaling: INTERNAL\_VALUE = EXTERNAL\_VALUE*10**EXPONENT UNIT.
This facility should be used so that scaling prefixes are not used in the
UNIT specification (eg.\ m,k,p,M).
\end{itemize}
Any software using FACTS should allow conversion of a quantity from one data
type to another.
To perform the conversion from a numeric data type to the character data type
(and vice-versa), the null format specifier is used.
This means that when data from an unformatted file is obtained as a string (for
representation to the outside world), that string is specified by the null
format.
If a value has a non zero EXPONENT, the null format should specify the scaled
value of the field to avoid the loss of significant figures.
\subsubsection {GENERIC}
The KEYWORD for a GENERIC definition is `G'.
GENERICS are arrays of similar FIELDs.
The FIELDNAME field in the description file must specify the dimensions of the
array.
FIELDNAMEs can be given according to the rules of Fortran 77:
\begin{verbatim}
    ARRAY([lb:]ub [,[lb:]ub]..........)
\end{verbatim}
with up to seven dimensions permitted, (if a string is enclosed by $[\;]$ then
it is optional).
The START of a generic field is the first byte of the array.
The LENGTH, FORMAT, EXPONENT, UNIT, NULLFORMAT, NULLVALUE describe the length,
format, exponent, unit, nullformat, nullvalue of each element of the array.
Tables, spectra and images can be stored as generic fields.
Examples are FLUX(4), LRS(200), MAP(100,100), W(50:100)\footnote
{The total number of elements allowed in a GENERIC is a FACTS parameter.
In the current version of CAR this is 1000 elements.
In practice, the actual limit will be somewhat less as element space is occupied
by FIELDs and PARAMETERs and of course other GENERICs.}.
\subsubsection {ENDFIELD}
The end of PARAMETER, FIELD and GENERIC definitions in a description file
is marked by the ENDFIELD record.
This can also be used to define the terminator record of a catalogue.
The KEYWORD for an ENDFIELD definition is `E'\footnote
{In the CAR system, only the E Keyword is used; the rest of the record may be
blank.}.
\subsubsection {CATNOTE}
Catnotes are used to hold documentation supplied with the catalogue\footnote
{CATNOTEs and ADCNOTEs are put in the CATALOGUES HELP library; see the
routine DSCFHELP.}.
\subsubsection {ADCNOTE}
Adcnotes are used to comment on the implementation of the catalogue in the
database.
\subsection {The NULLVALUE Field in a Description File}
The meaning of this FIELD depends on the KEYWORD.
A NULLVALUE really means VALUE for a PARAMETER and NULLVALUE for a FIELD or
GENERIC.
One of the problems encountered in handling astronomical data is that of null
values in catalogue FIELDS.
Frequently, the absence of data in a numeric field is indicated as a string of
blanks.
When the field is read in numeric format, a zero value is obtained.
This can have unfortunate results if the field is something like magnitude!
When creating a catalogue, astronomers are urged to ensure that this confusion
will never occur by using a more suitable string than a blank one\footnote
{The CAR programs handle null data specifically; see the section on each
program for details.}.
\subsection {Comments}
The COMMENT FIELD in a description file contains a definition of the FIELD
being described.
This definition may be lexical (for example `HHMMSSS') or functional where an
EXPRESSION is used to define it in terms of FIELDs in the CATALOGUE or some
other catalogue.
Where both kinds of definition are used, they should be separated by the `!'
character, with the lexical definition first and the functional definition
second, ie.
\begin{quote}
	Lexical Definition!Functional Definition\\
	eg.\ {\tt FIR Colour!FLUX(4)/FLUX(3)}
\end{quote}
Functional definitions should be QCAR expressions\footnote
{In the CAR system, functional definitions are used to derive new FIELDs.}
\subsection {Equivalencing}
FACTS permits the same piece of data to be defined in more than one way.
Moreover it is possible to have different definitions of a given field in the
same header and to overlap field definitions.
For example, the field NAME may be conventionally defined as a field with format
A10.
However, it can also be defined as a generic: NAME(10) with element formats A1.
The latter definition is useful for searching; often it is required to search on
a particular character in a text string (eg.\ the fifth character could be
referred to as NAME(5)).
\subsection {Reserved FIELDNAMEs}
\begin{description}
\item[NAME] ---
Most objects have some identity tag.
This identity tag is given the standard FIELDNAME `NAME'.
\item[NUMBER] ---
This field is used to number output records.
If this FIELDNAME is declared in an output file, it will take the value of the
current record number as it is output, and any previous value will be
overwritten.
\item[ASSOCIATION] ---
This field may be used to label objects that are output from a JOIN operation.
The original keyfield structure of a global index can always be recovered by
sorting it by ASSOCIATION.
\item[POSITION] ---
These fields are almost universal in astronomical catalogues, and it is highly
desirable to standardise their description:
\item[FIELDNAMEs] ---
The names to be used for position are:
\begin{description}
\item [RA] for RIGHT ASCENSION.
\item [DEC] for DECLINATION.
\item [ELONG] for ECLIPTIC LONGITUDE.
\item [ELAT] for ECLIPTIC LATITUDE.
\item [GLONG] for GALACTIC LONGITUDE.
\item [GLAT] for GALACTIC LATITUDE.
\item [SGLONG] for SUPERGALACTIC LONGITUDE.
\item [SGLAT] for SUPERGALACTIC LATITUDE.
\end{description}
\item[UNITS] ---
The units to be used for position are:
\begin{description}
\item [TIME] includes all varieties of hours, minutes and seconds.
\item [ANGLE] includes all varieties of sign, degrees, arc minutes and arc seconds.
\item [RADIAN] should be used for both TIME and ANGLE when the value of the position
field is unformatted (binary).
\end{description}
\item[FORMAT] ---
When the units are TIME or ANGLE and the field contains more than one number
(eg.\ HOURS, MINUTES and SECONDS), the format specifier should be of type A.
If it contains a single number, the format specifier can be of any type.
\item[COMMENT] ---
This is for position coordinates.
The first characters of the comment should contain a description of the position
value when a string is required to describe it in units of TIME or ANGLE.
Suggested formats for the string are listed below:
\begin{verbatim}
        HOUR               DEGREE
        MINUTE             ARCMIN
        SECOND             ARCSEC
        HHMMSS.SSS (*)     SDDMMSS.SSS (*)
        HHMMSS.SS  (*)     SDDMMSS.SS
        HHMMSS.S   (*)     SDDMMSS.S   (*)
        HHMMSSS            SDDMMSS     (*)
        HHMM.MM            SDDMM.M     (*)
        HHMMSS     (*)
\end{verbatim}
(Formats marked with a * are permitted to have spaces to separate the S, DD,
HH, MM, SS subfields\footnote
{The CAR software automatically handles the above formats.
The string must be terminated with a `!' character to separate it from the
remaining comment, and to mark the end of the string.}.)

When the UNIT of position is RADIAN, the COMMENT field should contain the one
of the above specifiers to allow the position to be externally represented in a
TIME or ANGLE format with the precision with which it is measured.
\item [EQUINOX/EPOCH] ---
These parameters apply to the reserved position fieldnames.
If there are other position fieldnames, the fieldname should be suffixed with
the Equinox or Epoch as appropriate.
\end{description}
\subsection {Index Catalogues}
The ability to create indexes to catalogues and process them as catalogues
themselves is one the essential features of the CAR system.
The purpose of an index is twofold:
\begin{enumerate}
\item An index can be used to store a subset of data from the main catalogue.
Only the numbers of the records in the main catalogue need be stored, so user
filestore does not get clogged up with data duplicating the main catalogue.
When you read an index, the data from the main catalogue is automatically
accessed.
You can see this when you are using SEARCH followed by REPORT.
You will see a field called POINTER which is the number of the record in the
main catalogue.
\item Fields other than the pointer can be included in an index so that the index
can be resorted on one of these fields to give fast access on that field.
The GIRL catalogue is an example; this stands for `Galactic IRAS Local index'.
The index was created by making a description file DSCFGIRL containing the three
fields POINTER, GLONG, and GLAT.
The IRPS catalogue was then CONVERTed to GIRL, and GIRL was then resorted by
GLAT.
If you want to search the IRPS on galactic latitude, SEARCH the GIRL, not the
IRPS.
\end{enumerate}
Indexes are generated by the database programs in `simple mode'.
The use of indexes saves you the chore of choosing which fields you want
transferred from the input catalogue to the output catalogue.
There are two kinds of indexes, LOCAL and GLOBAL.
\subsubsection {Local Indexes}
Local indexes index one catalogue.
The description file must contain:
\begin{itemize}
\item A {\bf PARAMETER} with fieldname LOCAL\_INDEX, format A and null value the
filename of the catalogue to be indexed.
\item A {\bf FIELD} with fieldname POINTER.
The format may be any numeric format, but I*4 or I10 are to be preferred.
\end{itemize}
When an index is produced, the value of the current indexed catalogue record
number is automatically inserted as POINTER.
Other fields can be defined as usual, referring directly or indirectly to
fields in the indexed catalogue.
If you make a DIRECT access index, make the KEYFIELD of the index POINTER,
unless you are actually moving the keyfield from the indexed catalogues.
You can later resort the index using program SORT.
\subsubsection {Global Indexes}
A Global Index indexes more than one catalogue.
The description file must contain:
\begin{itemize}
\item A {\bf PARAMETER} called GLOBAL\_INDEX, format A and value a blank string.
The comment field contains a list of catalogue acronyms, separated by commas.
This list contains all the possible values of the CATALOGUE field.
\item A {\bf FIELD} called CATALOGUE.
The format should be A or C*8 and the length 8.
This field contains the filename of the indexed catalogue.
\item A {\bf FIELD} called POINTER.
The format can be any numeric format, but obviously I10 or I*4 are to be
preferred.
\end{itemize}
If you are moving fields with the same name from different catalogues and you
have a single field with that name in the index, each index record will contain
the value from the CATALOGUE referenced by that index record.
You can keep fields with the same name apart by creating separate fields in your
index with the fieldnames consisting of the first four letters of the
CATALOGUE value, followed by the two characters $\_\,\_$, then the fieldname
from the indexed catalogue (eg.\ IRPS$\_\,\_$FLUX(4)).

Using Global Indexes you can create your own catalogues which reference many
others, but do not clog up filestore.
You could even make your own CSI!

If you are using keyfields in a Global Index, the strong definition of keyfield
applies, ie.\ the sort field(s) must identify the object.
The keyfield could be ASSOCIATION or a pair of longitude and latitude
coordinates, eg.\ DEC and RA.
It is not sufficient to just use DEC or RA as the keyfield as the DEC and the
RA alone do not uniquely identify the object.


\section {BACKGROUND --- Online database}

\subsection {Catalogues}
\subsubsection {Physical Characteristics}
Catalogues and description files are on DISK media and are organised as DIRECT
and SEQUENTIAL access files respectively.
Each catalogue consists of fixed length records; this means that it
looks like a table.
Catalogues typed in with the editor can be processed even though each record
contains only what you type in.
The maximum record length is an FACTS parameter; in version 5.3 of SCAR this is
1000 bytes.
The catalogue name should be not more than 8 characters long for
some of the CAR routines.

The description file for a catalogue is named by putting DSCF in front of the
CATALOG name.
Description files can be typed, printed or edited.
The recordsize is 132 bytes; this size is judged the minimum necessary to
contain a complete definition of a FIELD in a single relation.
It is the maximum size at which most lineprinters and modern terminals can
operate.

For both catalogues and description files the name may be logical or literal.
Filetypes are .DAT.
A single description file can describe a set of catalogues if they have names in
which the first five characters are the same but the rest differ, eg.\ FILE01,
FILE02, etc.
\subsubsection {Precision}
On the VAX, the range of values permitted is:
\begin{verbatim}
        D  ; R*8 : 0.29E-38 to 1.7E38 ; 16 significant digits.
        F,E; R*4 : 0.29E-38 to 1.7E38 ; 7 significant digits.
        I  ; I*4 : -2,147,483,648 to 2,177,483,647.
           ; I*2 : -32768 to 32767.
           ; I*1 : -128 to 127.
\end{verbatim}
\subsubsection {Acronyms}
Catalogues are referred to by, usually, a 4 or 5 character acronym
loosely related to the title of the catalogue, although there is no restriction
on how long or short the name must be.
The description filename for each catalogue consists of DSCF suffixed with the
acronym.
Eg.\ DSCFIRPS is the description file for the IRAS catalogue.
The description file for the formatted (printable) version of each catalogue
has an F suffixed to the description file filename of the online (unformatted
catalogue).
Eg.\ DSCFIRPSF is the description file for the formatted IRAS catalogue.
The help information on the data in the catalogues themselves is contained in
the CATALOGUES help library, (type CAT\_HELP).
\section {References}

\begin{description}
\item [SUN/35]: ICL --- Interactive Command Language Introduction
\item [SUN/64]: MONGO --- Interactive Plotting Program
\item [SUN/67]: SLALIB --- A Library of Subroutines
\item [SUN/79]: CDS --- Centre de Donnees Stellaire Catalogues
\item [SUN/83]: GKS --- Graphical Kernal System (7.2)
\item [SUN/90]: SNX --- Starlink extensions to the NCAR graphics utilities
\item [SUN/94]: ADAM --- Starlink Software Environment
\item [Scardoc1]: SCAR --- Release Notes
\item [Scardoc2]: SCAR --- Installation Notes
\item [Scardoc3]: SCAR --- Application Programmer Manual
\item [Scardoc4]: SCAR --- System Programmer Manual
\item [Scardoc5]: SCAR --- Adam Development Tools
\end{description}
The SCAR documentation can be found in SCAR\_DOC\_DIR.
\section {Online catalogues on RAL database microVAX}

New catalogues are continually being added to list of new catalogues available
on the RAL database microvax (STADAT). Use CAT\_HELP for the current situation.

\section {CSI expansion functions}
The CSI79 catalogue contains 80 bits of association data. Each bit set signifies
an association with a catalogue.
CSI1EXP to CSI10EXP are a series of new functions for expanding this coded
association data. The function CSI1EXP expands
the first byte, CSI2EXP the second and so on. In the vast majority of cases
the full catalogue identification is given. Where this is not possible
an easily interpreted code letter is given. The form is
*XXXXXXXX*
\begin{verbatim}
   where if X = A catalogue indentifier is 'BOSS'
                B .........................'CCFS'
                C..........................'YZ'
                D .........................'CPC'
                E .........................'HZ'
                F .........................'AGK'
                G .........................'HD'
   CSI2EXP
                H .........................'JSK'
                I .........................'FK4'
                J .........................'N30
                K .........................'YBS'
                L .........................'GCRV'
                M .........................'IDS'
                N .........................'ADS'
                O .........................'SAO'
   CSI3EXP
                P .........................'UBV'
                Q .........................'GCVS'
                R .........................'GCTP'
                S .........................'USNP'
                T .........................'BAY'
                U .........................'UVBY'
                V .........................'KDY'
                W .........................'A+B'
   CSI4EXP
                X .........................'GEN'
                Y .........................'Cel'
                Z .........................'IRC'
                a .........................'LS'
                b .........................'CLB'
                c .........................'CLA'
                d .........................'IC'
                e .........................'NGC'
   CSI5EXP
                f .........................'JP11'
                g .........................'RY'
                h .........................'FE'
                i .........................'UM'
                j .........................'GL'
                k .........................'BE*'
                l .........................'SB'
                m .........................'U+F'
   CSI6EXP
                n .........................'S+J'
                o .........................'MSS'
                p .........................'AMAS'
                q .........................'HGAM'
   CSI8EXP
                r .........................'MMAG'
                s .........................'YZO'
                t .........................'DM2'

\end{verbatim}

\end{document}
