\documentstyle{article}
\pagestyle{myheadings}

%------------------------------------------------------------------------------
\newcommand{\stardoccategory}  {Starlink User Note}
\newcommand{\stardocinitials}  {SUN}
\newcommand{\stardocnumber}    {3.6}
\newcommand{\stardocauthors}   {M D Lawden}
\newcommand{\stardocdate}      {10 May 1989}
\newcommand{\stardoctitle}     {FORMCON --- Data Format Conversion}
%------------------------------------------------------------------------------

\newcommand{\stardocname}{\stardocinitials /\stardocnumber}
\markright{\stardocname}
\setlength{\textwidth}{160mm}
\setlength{\textheight}{240mm}
\setlength{\topmargin}{-5mm}
\setlength{\oddsidemargin}{0mm}
\setlength{\evensidemargin}{0mm}
\setlength{\parindent}{0mm}
\setlength{\parskip}{\medskipamount}
\setlength{\unitlength}{1mm}

\begin{document}
\thispagestyle{empty}
SCIENCE \& ENGINEERING RESEARCH COUNCIL \hfill \stardocname\\
RUTHERFORD APPLETON LABORATORY\\
{\large\bf Starlink Project\\}
{\large\bf \stardoccategory\ \stardocnumber}
\begin{flushright}
\stardocauthors\\
\stardocdate
\end{flushright}
\vspace{-4mm}
\rule{\textwidth}{0.5mm}
\vspace{5mm}
\begin{center}
{\Large\bf \stardoctitle}
\end{center}
\begin{center}
{\Large\bf (IPCSIN, VICARIN)}
\end{center}
\vspace{5mm}

\section {INTRODUCTION}

Two data format conversion utilities are available:
\begin{description}
\begin{description}
\item[IPCSIN]: Convert IPCS data to Starlink format.
\item[VICARIN]: Convert VICAR data to Starlink format.
\end{description}
\end{description}
The Starlink format referred to is that used by the INTERIM Starlink
environment described in SUN/4 and used by programs such as those in the ASPIC
package.
The input data is assumed to be held on magnetic tape.
Both these programs are Starlink Application Programs and should be used within
the INTERIM environment.
This means that they are executed by the RUNSTAR command and use a connection
file.
The original programs and documentation were written by Dave Pearce at RAL.
\section {IPCSIN}
This utility will convert raw IPCS data held on magnetic tape to Starlink
{\em IMAGE-type frames} on disc.
To use the program, allocate a tape deck:
\begin{verbatim}
        $ ALLOC MT TAPE
\end{verbatim}
load the IPCS tape on the allocated deck and mount it:
\begin{verbatim}
        $ MOUNT/FOR TAPE
\end{verbatim}
then, run the program (for program parameters, see below):
\begin{verbatim}
        $ RUNSTAR FORMCON_DIR:IPCSIN
\end{verbatim}
For each IPCS data file selected, the following information is displayed on
your terminal before the OUTPUT prompt:
\begin{verbatim}
        RUN number,  FRAME SIZE
        Comment record #1
        Comment record #2
\end{verbatim}
After responding to the OUTPUT prompt, the program will process the input data
file and the last comment record will be displayed on your terminal.
When you have processed all the files you require, rewind and remove the tape
and deallocate the tape deck:
\begin{verbatim}
        $ DISMOUNT TAPE
        $ DEALLOC  TAPE
\end{verbatim}
\subsection {Program Parameters}
\begin{description}
\item[INPUT]:
specifies the name of the device on which the IPCS tape is loaded.
If you followed the conventions shown above, the value would be TAPE.
The following example illustrates the use of this parameter on the RUNSTAR
command line:
\begin{verbatim}
        $ RUNSTAR FORMCON_DIR:IPCSIN/INPUT=TAPE
\end{verbatim}
If you do not specify the parameter on the RUNSTAR command line, you will be
prompted for it:
\begin{verbatim}
        INPUT:=
\end{verbatim}
In the above example, your response would be:
\begin{verbatim}
        INPUT:= TAPE
\end{verbatim}
\item[FILES]:
determines which data files on the input tape are to be processed.
They are numbered consecutively, starting from 1.
You may specify any file sequence, either as single entities or as sets of
adjacent files, but the values must be increasing, eg:
\begin{verbatim}
        FILES:= 2,3,7-10,13,20-22
\end{verbatim}
will specify the file sequence: 2,3,7,8,9,10,13,20,21,22.
\item[COREDUMP]:
raw data tapes produced by IPCS systems normally have a `coredump'
file and a `dummy run' file preceding the actual image data files.
IPCSIN will automatically position the tape at the first data file by skipping
the two preceding files.
However, some installations have written utilities to produce copies of raw
IPCS data tapes without these coredump and dummy run files (ie.\ the first data
file is, in fact, the first physical file on the new tape).
In this situation you must specify COREDUMP=NO on the RUNSTAR command line to
ensure correct tape positioning, eg:
\begin{verbatim}
        $ RUNSTAR FORMCON_DIR:IPCSIN/COREDUMP=NO
\end{verbatim}
As the default for this parameter is COREDUMP=YES, you will not be prompted by
the system if you omit it from the RUNSTAR command line.
\item[OUTPUT]:
specify an output disc file for each tape file processed.
This file will be of type `.BDF'.
If only one tape file is to be processed, this parameter can be specified
on the RUNSTAR command line:
\begin{verbatim}
        $ RUNSTAR FORMCON_DIR:IPCSIN/INPUT=TAPE/FILES=3/OUTPUT=NGC1365
\end{verbatim}
However, for multi-file processing you will be prompted each time a new output
file name is required:
\begin{verbatim}
        OUTPUT:= M87ALPHA
\end{verbatim}
This prompt is repeated until all the input files requested have been processed.
\end{description}
\subsection {Example}
The following example illustrates the use of IPCSIN:
\begin{quote}
\begin{verbatim}
$ ALLOC MT TAPE
$ MOUNT/FOR TAPE
$ RUNSTAR FORMCON_DIR:IPCSIN/INPUT=TAPE
FILES:= 4,7-8

RUN   =          4   FRAME SIZE =       500     390
M83 FIELD 1
NARROW HALPHA FILTER
OUTPUT:= DISK$USER1:[IMAGES]M83HA
BAD SHADOWING IN CORNER, STOP AND LOOK

RUN   =          7   FRAME SIZE =       500     390
FIELD 1
S II FILTER
OUTPUT:= DISK$USER1:[IMAGES]M83S
THROUGH INCREASING CLOUD

RUN   =          8   FRAME SIZE =       500     390
DOME LIGHT
THROUGH 6400 CONT FILTER
OUTPUT:= [JR]DOME

$ DISMOUNT TAPE
$ DEALLOC  TAPE
\end{verbatim}
\end{quote}
\subsection {Descriptors}
The following descriptor items are constructed from the input file's header
information and written to the frame:
\begin{quote}
\begin{verbatim}
NAXIS:    Number of axes in frame (set to 2)
NAXIS1:   Size of 1st axis (record length/2)
NAXIS2:   Size of 2nd axis (number of records)

COMMENT:  1st comment record
COMMENT:  2nd comment record
COMMENT:  Last comment record
\end{verbatim}
\end{quote}
This version of IPCSIN does not process the {\em parameter} record, which
contains information about integration time, co-ordinates, etc.
This situation will be corrected when descriptor names have been formalised for
these data, provided programming effort is available.
\subsection {Run-time Errors}
The following error conditions may occur:
\begin{description}
\item[UNABLE TO ALLOCATE TAPE UNIT]:
Normally caused by incorrect device name specification for the INPUT parameter.
Can also occur when the MOUNT operation was not performed for the device.
\item[ERROR ENCOUNTERED ON REWIND]:
The INPUT device is not online.
\item[ERROR ENCOUNTERED ON FILE-SKIP]:
This can be caused if the INPUT device is switched offline during a file-skip
operation.
\item[ERROR WHILE READING TAPE]:
A number of factors can be responsible for this condition, such as
hardware I/O error, (parity, etc.), or the INPUT device being switched offline
during processing.
\item[NULL FILE-SPECIFICATION --- NO DEFAULTS]:
A null reply was given to the FILES parameter.
No default exists.
\item[INVALID FILE-SPECIFICATION]:
A file name was not in the required format.
\item[FILE SEQUENCE ERROR]:
The present tape-position is beyond the file to be processed.
The FILES list should form an increasing sequence.
\item[UNEXPECTED END-OF-TAPE ENCOUNTERED]:
A run number of -1 was detected in the file's header block.
This normally signifies the end of data files.
\item[UNABLE TO CONNECT TO BULK DATA FRAME --- RESPECIFY]:
This error is detected during allocation of the output frame.
It is caused by your supplying a `null' frame specification or, under this
environment, an incorrect VMS filename.
As the situation is recoverable, you will be reprompted with OUTPUT.
This error can also occur if the file specified for output already exists but is
protected against deletion.
\item[FATAL ERROR ENCOUNTERED DURING FRAME ALLOCATION]:
This error is detected during allocation of the output frame.
It is irrecoverable and the program will be aborted.
This normally occurs when processing large images which exceed your VMS resource
quotas.
\end{description}
\section {VICARIN}
This utility will convert standard VICAR format image data held on magnetic tape
to Starlink {\em IMAGE-type frames} on disc.
VICAR image files processed by this utility must satisfy these conventions:
\begin{itemize}
\item Each file must contain a standard VICAR system label specifying the
dimensions of the image.
\item All VICAR label information is encoded as EBCDIC characters.
\item If the image files are blocked, the blocks are assumed to be of fixed
length with no logical records spanning block boundaries, ie.\ each physical
block contains an integral number of logical records.
The maximum block size allowed is 32768 bytes.
\end{itemize}
To use the program, allocate a tape deck:
\begin{verbatim}
        $ ALLOC MT TAPE
\end{verbatim}
load the VICAR tape on the allocated deck and mount it:
\begin{verbatim}
        $ MOUNT/FOR TAPE
\end{verbatim}
then, run the program (for program parameters, see below):
\begin{verbatim}
        $ RUNSTAR FORMCON_DIR:VICARIN
\end{verbatim}
For each VICAR data file selected, the following information is displayed
on your terminal before the OUTPUT prompt:
\begin{verbatim}
        TAPE FILE number
        First VICAR label block
\end{verbatim}
When you have processed all the files you require, rewind and remove the tape
and deallocate the tape deck:
\begin{verbatim}
        $ DISMOUNT TAPE
        $ DEALLOC  TAPE
\end{verbatim}
\subsection {Program Parameters}
\begin{description}
\item[INPUT]:
specifies the name of the device on which the VICAR  tape is loaded.
If you followed the conventions shown above, the value would be TAPE.
The following example illustrates the use of this parameter on the RUNSTAR
command line:
\begin{verbatim}
        $ RUNSTAR FORMCON_DIR:VICARIN/INPUT=TAPE
\end{verbatim}
If you do not specify the parameter on the RUNSTAR command line, you will be
prompted for it:
\begin{verbatim}
        INPUT:=
\end{verbatim}
In the above example, your response would be:
\begin{verbatim}
        INPUT:= TAPE
\end{verbatim}
\item[FILES]:
determines which data files on the input tape are to be processed.
They are numbered consecutively, starting from 1.
You may specify any file sequence, either as single entities or as sets of
adjacent files, but the values must be increasing, eg:
\begin{verbatim}
        FILES:= 2,3,7-10,13,20-22
\end{verbatim}
will specify the file sequence: 2,3,7,8,9,10,13,20,21,22.
\item[LABELLED]:
by default, it is assumed that the input tape is {\em non-labelled} ie.\ each
image data file is separated from any adjacent files by a single tape mark.
If the tape is {\em labelled}, specify LABELLED=YES on the RUNSTAR command line
to ensure correct tape positioning:
\begin{verbatim}
        $ RUNSTAR FORMCON_DIR:VICARIN/LABELLED=YES
\end{verbatim}
As the default for this parameter is LABELLED=NO, you will not be prompted by
the system if you omit it from the RUNSTAR command line.
\item[OUTPUT]:
specify an output disc file for each tape file processed.
This file will be of type `.BDF'.
If only one tape file is to be processed, this parameter can be specified on the
RUNSTAR command line:
\begin{verbatim}
        $ RUNSTAR  FORMCON_DIR:VICARIN/INPUT=TAPE/FILES=3/OUTPUT=GANYMEDE
\end{verbatim}
However, for multi-file processing you will be prompted each time a new output
file name is required:
\begin{verbatim}
        OUTPUT:= CALLISTO
\end{verbatim}
This prompt is repeated until all the input files requested have been processed.
\item[FORMAT]:
pixel values within a VICAR image are most commonly represented as 8-bit
unsigned integers, ie.\ one byte corresponds to one pixel.
However, data in other formats do exist, such as 16-bit signed integer or 32-bit
floating point.
The setting of this parameter determines the data format of each VICAR image
file.
The possible character-type values that can be assigned to this parameter are
shown in the following table:
\begin{verbatim}
  PARAMETER
   VALUE               MEANING           BYTES/PIXEL

     SB             Signed Byte               1
     SW             Signed Word               2
     SL             Signed Longword           4
     R              Real                      4
     DP             Double Precision          8
     UB             Unsigned Byte             1
     UW             Unsigned Word             2
\end{verbatim}
VICARIN assigns an initial value of `UB' to this parameter and this will stay in
effect until you change it.
This can be accomplished after the first VICAR label block is displayed on the
terminal.
You will be prompted; if the current setting is correct for the image file
being processed, hit {\em return} and the image data will be interpreted
accordingly.
However, if the current setting is incorrect it can be superseded by the
appropriate value which then becomes the new current  setting.
For example, to change from unsigned byte to signed word:
\begin{verbatim}
        FORMAT/UB/:= SW
\end{verbatim}
The current setting is displayed as a {\em run-time default} --- see reference
(1).
\item[OVERRIDE]:
can be used to override VICARIN's interpretation of the image size
fields on the VICAR system label.
By convention, the number of video lines in the image is assumed to be held as
a 4-character field starting at byte 33.
The number of bytes in each line is assumed to be held in the same form starting
at byte 37 (both values are right justified).
If this convention does not hold true for any image data files on the tape,
specify OVERRIDE=YES on the RUNSTAR command line.
You will then be asked to supply the image size explicitly through use of the
NLINES and NSAMPLES parameters described below.
VICARIN's interpretation of these fields on the VICAR system label are displayed
to you as run-time defaults.
If they are correct, hit {\em return} and these will be taken as the true
image size.
The default for this parameter is  OVERRIDE=NO, therefore you will not be
prompted by the system if you omit it from the RUNSTAR command line.
\item[NLINES]:
only used if OVERRIDE=YES has been specified on the RUNSTAR command line.
If VICARIN's interpretation of this field on the VICAR system label is
incorrect, you must override it with the appropriate value.
\item[NSAMPLES]:
only used if OVERRIDE=YES has been specified on the RUNSTAR command line.
This value is the number of bytes in each video line of the image.
\end{description}
\subsection {Examples}
The following examples illustrate the use of VICARIN:
\begin{quote}
\begin{verbatim}
$ ALLOC MT TAPE
$ MOUNT/FOR TAPE
$ RUNSTAR FORMCON_DIR:VICARIN/FILES=1-2,4
INPUT:= TAPE

FILE   =    1 - FIRST VICAR LABEL BLOCK FOLLOWS:
                           1   1 495 495                               C
              IR1145   5 NOV1978                                       C
*BOXAV2  SLP  15 SSP  15 LIP   5 SIP   5 RSCALE   1.0000BYTE TO BYTE  RC
*ROTAT180                                                            1HL
FORMAT/UB/:=
OUTPUT:= [.IMAGES]IRMETEO

FILE   =    2 - FIRST VICAR LABEL BLOCK FOLLOWS:
                           1   1 495 495                               C
             VIS1145   5 NOV1978                                       C
*BOXAV2  SLP  15 SSP  20 LIP   5 SIP  10 RSCALE   1.0000BYTE TO BYTE  RC
*ROTAT180                                                            1HL
FORMAT/UB/:=
OUTPUT:= [.IMAGES]VISMETEO

FILE   =    4 - FIRST VICAR LABEL BLOCK FOLLOWS:
77                   500    1000 5001000 I 2                          SC
VGR-1   FDS 15311.06   PICNO 0101J1-036   SCET 79.028 13:26:11         C
NA CAMERA  EXP 00240.0 MSEC  FILT 2( BLUE )  LO GAIN  SCAN RATE  1:1   C
ERT 79.028 22:23:28   1/ 1 FULL    RES   VIDICON TEMP  -81.00 DEG C    C
IN/012450/10 OUT/******/**                 DSS #**   BIT SNR    9.711  C
FORMAT/UB/:= SW
OUTPUT:= [.IMAGES]REDSPOT
\end{verbatim}
\end{quote}
The last file is a 500x500 image in signed word format, ie.\ each pixel is
represented as a 2-byte signed integer.
The system label follows the conventions outlined in reference (2).

The next example shows the use of the OVERRIDE parameter when processing a
10000x10000 image file:
\begin{quote}
\begin{verbatim}
$ RUNSTAR FORMCON_DIR:VICARIN/INPUT=TAPE/FILES=23/OVERRIDE=YES

FILE   =   23 - FIRST VICAR LABEL BLOCK FOLLOWS:
77                 10000   10000         L 1                          SC
              WV1145   5 NOV1978                                       C
*ZOOM                                                                 HL
NLINES/0/:= 10000
NSAMPLES/0/:= 10000
FORMAT/UB/:=
OUTPUT:= [.IMAGES]WVMETEO

$ DISMOUNT TAPE
$ DEALLOC  TAPE
\end{verbatim}
\end{quote}
The above system label also follows the conventions outlined in reference (2),
ie.\ the number of video lines is held in bytes 17 to 24 and the number of samples
in bytes 25 to 32.
For images of dimensions less than 10000, these values will also be held in
those fields conventionally processed by this utility, namely 33 to 36 and 37 to
40.
However, for larger images, these last two fields are left blank which VICARIN
will interpret as zero.
Therefore, you must supply the image dimensions explicitly.
\subsection {Descriptors}
The following descriptor items are constructed from the input file's VICAR
system label and written to the frame:
\begin{quote}
\begin{verbatim}
NAXIS:    Number of axes in frame (set to 2)
NAXIS1:   Size of 1st axis (number of pixels in each line)
NAXIS2:   Size of 2nd axis (number of video lines)
\end{verbatim}
\end{quote}
Each 72-byte VICAR label is written to the frame as `HISTORY' descriptor items.
\subsection {Run-time Errors}
The following error conditions may occur:
\begin{description}
\item[UNABLE TO ALLOCATE TAPE UNIT]:
Normally caused by incorrect device name specification for the INPUT parameter.
Can also occur when the MOUNT operation was not performed for the device.
\item[ERROR ENCOUNTERED ON REWIND]:
The INPUT device is not online.
\item[ERROR ENCOUNTERED ON FILE-SKIP]:
This can be caused if the INPUT device is switched offline during a file-skip
operation.
\item[ERROR WHILE READING TAPE]:
A number of factors can be responsible for this condition, such as hardware I/O
error, (parity, etc.), or the INPUT device being switched offline during
processing.
\item[NULL FILE-SPECIFICATION --- NO DEFAULTS]:
A null reply was given to the FILES parameter.
No default exists.
\item[INVALID FILE-SPECIFICATION]:
A file name was not in the required format.
\item[FILE SEQUENCE ERROR]:
The present tape-position is beyond the file to be processed.
The FILES list should form an increasing sequence.
\item[UNEXPECTED END-OF-TAPE ENCOUNTERED]:
A run number of -1 was detected in the file's header block.
This normally signifies the end of data files.
\item[UNABLE TO CONNECT TO BULK DATA FRAME --- RESPECIFY]:
This error is detected during allocation of the output frame.
It is caused by your supplying a `null' frame specification or, under this
environment, an incorrect VMS filename.
As the situation is recoverable, you will be reprompted with OUTPUT.
This error can also occur if the file specified for output already exists but is
protected against deletion.
\item[FATAL ERROR ENCOUNTERED DURING FRAME ALLOCATION]:
This error is detected during allocation of the output frame.
It is irrecoverable and the program will be aborted.
This normally occurs when processing large images which exceed your VMS resource
quotas.
\item[IBM FLOATING POINT FORMAT NOT SUPPORTED]:
This version of VICARIN is unable to convert binary image data held in IBM
floating point format --- this facility may be added in the future.
\end{description}
\section {REFERENCES}
(1) SUN/4: `INTERIM --- Starlink software environment'\\
(2) J B Seidman, [1977]: `New VICAR Label Standard', Jet Propulsion Laboratory.
\end{document}
