\documentstyle[11pt]{article}
\pagestyle{myheadings}

%------------------------------------------------------------------------------
\newcommand{\stardoccategory}  {Starlink User Note}
\newcommand{\stardocinitials}  {SUN}
\newcommand{\stardocnumber}    {26.2}
\newcommand{\stardocauthors}   {Malcolm J. Currie \& Quentin A. Parker}
\newcommand{\stardocdate}      {1993 April 26}
\newcommand{\stardoctitle}     {CLUSTAN --- A Cluster-Analysis Package}
%------------------------------------------------------------------------------

\newcommand{\stardocname}{\stardocinitials /\stardocnumber}
\renewcommand{\_}{{\tt\char'137}}     % re-centres the underscore
\markright{\stardocname}
\setlength{\textwidth}{160mm}
\setlength{\textheight}{230mm}
\setlength{\topmargin}{-2mm}
\setlength{\oddsidemargin}{0mm}
\setlength{\evensidemargin}{0mm}
\setlength{\parindent}{0mm}
\setlength{\parskip}{\medskipamount}
\setlength{\unitlength}{1mm}

%------------------------------------------------------------------------------
% an environment for references
\newenvironment{refs}{\goodbreak
                      \vspace{3ex}
                      \begin{list}{}{\setlength{\topsep}{0mm}
                                     \setlength{\partopsep}{0mm}
                                     \setlength{\itemsep}{0mm}
                                     \setlength{\parsep}{0mm}
                                     \setlength{\leftmargin}{1.5em}
                                     \setlength{\itemindent}{-\leftmargin}
                                     \setlength{\labelsep}{0mm}
                                     \setlength{\labelwidth}{0mm}}
                    }{\end{list}}

%------------------------------------------------------------------------------

\begin{document}
\thispagestyle{empty}
SCIENCE \& ENGINEERING RESEARCH COUNCIL \hfill \stardocname\\
RUTHERFORD APPLETON LABORATORY\\
{\large\bf Starlink Project\\}
{\large\bf \stardoccategory\ \stardocnumber}
\begin{flushright}
\stardocauthors\\
\stardocdate
\end{flushright}
\vspace{-4mm}
\rule{\textwidth}{0.5mm}
\vspace{5mm}
\begin{center}
{\Large\bf \stardoctitle}
\end{center}
\vspace{5mm}

%------------------------------------------------------------------------------
%  Add this part if you want a table of contents
%  \setlength{\parskip}{0mm}
%  \tableofcontents
%  \setlength{\parskip}{\medskipamount}
%  \markright{\stardocname}
%------------------------------------------------------------------------------

\section{Introduction}
Astronomers have not, in the past, been among the most imaginative and
swift group when it came to realising the potentials afforded by
$20^{\rm th}$ century progress in statistics. However, this is now
changing and an increasing number of astronomical papers are
turning to modern statistical analysis techniques to assist in data exploration
and description ({\it e.g.}\ see Murtagh \& Heck 1987, for a list of
references).

A very convincing case highlighting the need for a proper statistical
methodology in astronomy has been given by Heck {\it et al.}\ (1985),
and details of the applications of such a methodology have been given by
Murtagh (1986). Heck {\it et al.} stressed that the most important
statistical methods for modern astronomical problems are probably the
multivariate methods such as principal-components analysis (PCA) and
cluster analysis. Astronomers are now realising that ways are urgently
needed to adequately handle, condense and interpret the huge and growing
quantities of data currently being produced from our telescopes,
instruments, satellites and measuring machines.

To assist UK astronomers working at the forefront in these areas
Starlink has purchased an integrated commercial package {\small
CLUSTAN}, which performs sophisticated Cluster Analysis and Multivariate
statistical analysis techniques. It is available at the following sites:
Birmingham, Cambridge, Durham, Leicester, Manchester, Preston, Queen
Mary and Westfield College, RAL, ROE, and Southampton.  Users at other
Starlink sites can access {\small CLUSTAN} on STADAT.  They will need to
ask their site manager to obtain a manual for their site from the
Starlink Software Librarian.

\section{What is Cluster Analysis?}
Cluster analysis is a modern statistical means of separating
observed samples into homogeneous classes, which may be quite distinct
or overlapping, to produce a working taxonomy or classification scheme.
It is an exploratory method for helping to solve classification problems by
generating an hypothesis about the category structure which may be
inherent in the data. Simply put, the aim is to sort out a sample of
objects or cases under consideration into groups such that the degree of
association is high between members of the same group and low between
members of different groups. The way in which this is carried out
depends on the particular choice of clustering option (of which there
are a great many).

From the beginning it is important that the philosophy behind cluster
analysis be understood (see the book by Anderberg for an excellent
introduction to this subject). It is not magic! The idea is not to
collect the data, put it through a clustering analysis
and then accept what comes out as `the answer'.
The context of the data must be known and properly laid out in terms of
the properties, parameters and attributes of each object. The choice of
variables is crucial and has the greatest influence on the results.
A proper research objective should exist and an understanding of the
principles of clustering analysis is required so that the most
appropriate strategy for analysis can be adopted.

Having said this, cluster analysis is a tool of discovery which can reveal
associations and structure in data which may not have been previously perceived
but are nevertheless meaningful, sensible and useful when found. More
specifically it can:
\begin{itemize}
\item help in developing a classification scheme against which new
objects can then be tested,
\item identify peculiar objects and outliers in $M$-dimensional space,
\item predict the future behaviour of population types,
\item formulate hypotheses concerning the origin of the sample,
\item measure the different effects of treatments on classes within the
population,
\item  suggest ideas and insights to the analyst concerning the structure
of the data.
\end{itemize}
Prior to a clustering analysis another important preliminary analysis can be
performed. This is a principal-components analysis (PCA) to create a number of
variables $m^{\prime}$, which contain the most power, from a larger original
set of variables $m$. Hence the dimensionality of the problem is reduced.
This is especially important where a large number of variables are associated
with each object. More specifically, performing a PCA can:
\begin{itemize}
\item enhance feature selection and help in choosing the most useful variables,
\item help in the visualisation of multi-dimensional space data,
\item assist in identification of underlying variables,
\item enable early identification of groups of objects or of outliers,
\item reduce the dimensionality of a clustering problem to a more-manageable
size by eliminating those variables which contribute nothing or little to the
inherent properties of a data set.
\end{itemize}

In addition to Murtagh \& Heck and the {\it Clustan User Manual}
there are a number of classic textbooks for cluster analysis, e.g.: Anderberg
(1973), Evritt (1980), Gordon (1981), Hartigan (1975) and Kendall (1980)
that should help you to appreciate and understand this topic.

\section{Examples of application of these techniques in Astronomy}

Below we give a few examples of areas in which clustering and\/ or multivariate
techniques have already been applied.

\begin{itemize}
\item star-galaxy discrimination;
\item prediction of spectral type from photometry;
\item high-redshift quasar searches;
\item galaxy morphological classification;
\item galaxy clustering in 3-D;
\item automation of interferogram analysis;
\item stellar light curves;
\item spectral classification;
\item lunar geology;
\item asteroid studies;
\item gamma and X-ray studies.
\end{itemize}

A more comprehensive list of published papers in this area is given by
Murtagh \& Heck (1987). It is not difficult to see that the range and
applicability of these multivariate techniques is limited largely by the
imagination of the astronomer.

\section{CLUSTAN}
Full documentation for {\small CLUSTAN} is provided in the {\it Clustan
User Manual} by David Wishart.  Your site manager should have a
reference copy and at least one for loan.  It is easily recognisable by
the Pleiades photograph on the cover.

In this document we just describe the Starlink-specific implementation.

\subsection{Summary}
{\small CLUSTAN} is a fully modular, completely integrated system which
enables great flexibility in its use. It has a simple to use `Command
Language' based on English keywords together with on-line help, review
and command-correction facilities. {\small CLUSTAN} can equally well be
run interactively or in batch and has self-explanatory error reporting.
Many thousands of objects can be hierarchically clustered whilst
hundreds of variables can be specified. Applications are only limited by
the size of the data matrix that can be held in computer memory.

\subsection{Getting Started}
To define the {\small CLUSTAN} command enter
\begin{verbatim}
    $ CLUSTANSTART
\end{verbatim}

To start {\small CLUSTAN} enter
\begin{verbatim}
    $ CLUSTAN
\end{verbatim}
You will be presented with two lists of graphics devices available at
your site.  The first covers non-hardcopy devices  {\it i.e.}\ mostly
terminals and image displays whilst the second comprises plotters. After each
list you will be prompted to select a choice from the list {\em even
if you do not intend to create any graphics}.  This is because of
the way {\small CLUSTAN} is designed, in particular how you select a graphics
device within {\small CLUSTAN}.  {\small CLUSTAN} has four graphics devices
described:
VDU, PLOTTER, CAMA and CAMB ({\it c.f.}\ the {\tt SELECT GRAPHICS DEVICE}
command in the user manual).  CAMA and CAMB are not available in the
Starlink version.  The device chosen from the first list corresponds
to VDU, and the second corresponds to PLOTTER.

{\small CLUSTAN} only permits a single graphics device to be used during a
{\small CLUSTAN} job.  Therefore, if you require graphical output you must still
specify to which of the two nominated devices you wish to plot.  It is
expected that future versions will permit any name to be given as the
parameter to the {\tt SELECT GRAPHICS DEVICE} command and so avoid
unnecessary typing.

You are then ready to use {\small CLUSTAN} applications as described in the
manual.

\subsection{Batch mode}
{\small CLUSTAN} has a batch mode so that you can prepare a series of commands
in a file.  Therefore, you can execute the same commands for a series
datasets with the minimum of effort.  To obtain batch mode you prepare
a text file of {\small CLUSTAN} commands {\em preceded by the names of the
VDU and PLOTTER graphics devices required}.  For example, if you type the
following into {\tt MYFILE.DAT}
\begin{verbatim}
    GRAPHON
    CANON_L
    COMMENT Read data from the terminal
    READ DATA, TITLE "TEST DATA", CASES 3 VARIABLES CONTINUOUS 1-4
    358   421   578   772
    272    33   115   670
    300   213   431   692

    COMMENT Cluster the data using Ward's method with the squared
    COMMENT Euclidean similarity measure.
    CLUSTER, METHOD WARDS, MEASURE SEUCLID
    COMMENT  Select plotting to the 'VDU' device alias the Graphon.
    SELECT GRAPHICS DEVICE VDU
    COMMENT  Plot the dendrogram.
    PLINK
    STOP
\end{verbatim}
and then enter
\begin{verbatim}
    $ CLUSTAN MYFILE.DAT
\end{verbatim}
{\small CLUSTAN} will read {\tt MYFILE.DAT} instead of obeying commands from the
terminal.

Should you require a permanent record of {\small CLUSTAN}'s analysis you can
redirect the output from the terminal to a file.  So for example:-
\begin{verbatim}
    $ CLUSTAN/OUTPUT=RESULTS.LIS  MYFILE.DAT
\end{verbatim}
will write the processing results to {\tt RESULTS.LIS} for the commands
given in {\tt MYFILE.DAT}.  You do not have to supply an input command
file to store the results in a file---just enter
\begin{verbatim}
    $ CLUSTAN/OUTPUT=RESULTS.LIS
\end{verbatim}
\subsection{Handling SCAR Catalogues in CLUSTAN}
It is intended to provide a {\tt READ SCAR} procedure in the near future
that should make the task of getting existing SCAR data sets into
{\small CLUSTAN} much easier.  Also it should be possible to write
classifications back into a SCAR file. In the meantime you will need to
do something equivalent to the following example to read your data into
{\small CLUSTAN} suitably modified for your files.

Given the following description file (truncated on the right-hand side
for clarity) you want to analyse fields VALUE1,
VALUE2, VALUE3 and VALUE4.  In {\small CLUSTAN} terminology these are
the {\em variables} whose names are limited to eight characters.
\begin{verbatim}
 P TITLE                                               A17    TEST
 P FILENAME                                            A17    TEST
 P MEDIUM                                              A17    DISK
 P ACCESSMODE                                          A17    SEQUENTIAL
 P RECORDSIZE                               BYTE       I5        51
 F NAME                  1     5 A5       0            A5
 F RA                    8     8 A8       0 TIME       A8
 F DEC                  19     9 A9       0 ANGLE      A9
 F VALUE1               31    10 F10.5    0            F10.5     0.00000
 F VALUE2               41    10 F10.5    0            F10.5     0.00000
 F VALUE3               51     4 I4       0            I4        0
 F VALUE4               56     6 I6       0            I6        0
 E
 C No catnotes
 A No ADC notes
\end{verbatim}
The corresponding (ASCII) catalogue is
\begin{verbatim}
STAR1  12 00 00   +45 00 00   1058.63122 421.34    578    372
STAR2  12 30 00   +44 30 00    271.78    -33.76   -115  21670
STAR3  12 35 00   +43 45 00    300.16      3.456781431    692
\end{verbatim}
To read the file you first must assign a {\small CLUSTAN} internal
file to the actual file.  Then read in the data, specifying the number
of records or {\em cases} and the format.  Note you can only read
formatted catalogues. {\small CLUSTAN} does not process integer data
other than binary masks.  Therefore, the format specifier for integers must be
{\tt Fx.0}. The {\tt READ LABELS} commands are optional.  Note the
limitation to {\em exactly} eight characters for LABELS, left justified and
padded out with blanks if necessary (see page 132 of the manual).

\begin{verbatim}
ASSIGN File=TRIAD  Specification='DISK$SCRATCH:[ABC]TEST.DAT'
READ DATA, Infile=TRIAD, Title "TEST DATA", Cases 3 Variables Continuous 1-4,
    Format=(30X, 2F10.5, F4.0, F6.0)
READ LABELS Variables
VALUE1  VALUE2  VALUE3  VALUE4
READ LABELS Cases
STAR1   STAR2   STAR3
\end{verbatim}
Now you are ready to perform cluster analysis on your data.

If you do not want to omit any field from the analysis you can omit
the format qualifier from the {\tt READ DATA} command and read in
free format, {\em provided each field is separated by at least one space
and is numeric}. If you are only interested in analysing certain numeric
fields in the catalogue you can generate an ASCII output file containing
just the required fields by using the SCAR application CAR\_LIST.
Alternatively, you can create a new SCAR catalogue just containing the
fields of interest using CAR\_CONVERT.

{\small CLUSTAN} supports missing values in a limited way.  See pp.128--129
of the Manual for details.

\subsection{Graphics}
The Starlink version of {\small CLUSTAN} uses GKS graphics.  However, it
does not access the graphics database. The HEIGHT qualifier in
procedures PLINK (for plotting dendrograms or tree-diagrams) and SCATTER
(for plotting cluster diagrams) refers to the height of characters as
fractions of the plot height rather than in absolute units. GKS error
messages are routed to channel 22.
\subsection{Help}
{\small CLUSTAN} has an on-line help system, which looks similar to the
familiar VMS help library.  However, the effect of entering {\tt CTRL/Z} is
not the same. In VMS you just exit the help system; in {\small CLUSTAN} you
exit {\small CLUSTAN} itself, perhaps losing many calculations. Hence you must
press {\tt <CR>} repeatedly until you reach the {\small CLUSTAN} command prompt.
\subsection{Error reporting}
{\small CLUSTAN} has comprehensive error reporting which is intended to be
self explanatory. Furthermore an interactive error-correction facility is also
available for Command-line errors ({\it e.g.}\ due to the omission
of essential parameters, incorrect values or parameter conflicts).
\subsection{Examples}
{\small CLUSTAN} provides a number of examples in Chapter 10 of the manual to
help you become familiar with typical cluster-analysis operations,
and {\small CLUSTAN} procedures.
The source files can be found in CLUSTAN\_DIR as follows:
\begin{center}
\begin{tabular}{|c|l|}
\hline
Name & File \\ \hline
A--G & TESTAG32.DAT \\
H & SPSS32.DAT \\
I & FISHER.DAT \\
J & PLOT32.DAT \\ \hline
\end{tabular}
\end{center}
In addition there are two further examples: {\tt CLUSTER32.DAT} and
{\tt CLASS32.DAT}.  The latter uses the output from the former.
If you wish to run an example you will need to copy one of the above
files to your directory and then edit appropriate VDU and PLOTTER device
names into the first two lines of that file.

Any bugs when using {\small CLUSTAN} should be reported in the normal way.
Starlink will pass on bug reports to the authors of the software.

\section{Quotas} {\small CLUSTAN} has been linked with large buffers so
that most problems can be solved without re-compilation and re-linking.
This has the side effect that a moderate page file limit (PGFLQUO) is
required, {\it viz.} 27000.  This will permit storage of up to 10Mb of
catalogue data.  If you require yet more capacity, you should consult
your site manager.

\section{Acknowledgement}
Thanks to Clive Davenhall who wrote the GKS interface.
\section{References}
\begin{refs}
\item Anderberg, M.R., 1973, {\it Cluster Analysis for Applications},
  Academic Press, New York.
\item Evritt, B., 1980, {\it Cluster Analysis}, Heinemann Educational Books,
  London.
\item Gordon, A.D., 1981, {\it Classification}, Chapman and Hall, London.
\item Hartigan, J.A., 1975, {\it Clustering Algorithms}, Wiley, New York.
\item Heck, A., Murtagh, F., and Ponz, D., 1985, {\it The Messenger},
  No. 41, 22.
\item Kendall, M., 1980, {\it Multivariate Analysis}, Charles Griffin \&
  Co.Ltd., London.
\item Murtagh, F., 1986, {\it Data Analysis and Astronomy}, Plenum Press,
  New York.
\item Murtagh, F., and Heck, A., 1987, {\it Multivariate Data Analysis},
  Astrophysics and Space Science Library, D.Reidel, Dordrecht, Holland.
\end{refs}
\end{document}
