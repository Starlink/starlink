\documentstyle[11pt]{article}
\pagestyle{myheadings}

%------------------------------------------------------------------------------
\newcommand{\stardoccategory}  {Starlink User Note}
\newcommand{\stardocinitials}  {SUN}
\newcommand{\stardocnumber}    {128.2}
\newcommand{\stardocauthors}   {M D Lawden}
\newcommand{\stardocdate}      {16 July 1993}
\newcommand{\stardoctitle}     {QDP --- Quick and Dandy Plotter}
%------------------------------------------------------------------------------

\newcommand{\stardocname}{\stardocinitials /\stardocnumber}
\renewcommand{\_}{{\tt\char'137}}     % re-centres the underscore
\markright{\stardocname}
\setlength{\textwidth}{160mm}
\setlength{\textheight}{230mm}
\setlength{\topmargin}{-2mm}
\setlength{\oddsidemargin}{0mm}
\setlength{\evensidemargin}{0mm}
\setlength{\parindent}{0mm}
\setlength{\parskip}{\medskipamount}
\setlength{\unitlength}{1mm}

\begin{document}
\thispagestyle{empty}
SCIENCE \& ENGINEERING RESEARCH COUNCIL \hfill \stardocname\\
RUTHERFORD APPLETON LABORATORY\\
{\large\bf Starlink Project\\}
{\large\bf \stardoccategory\ \stardocnumber}
\begin{flushright}
\stardocauthors\\
\stardocdate
\end{flushright}
\vspace{-4mm}
\rule{\textwidth}{0.5mm}
\vspace{5mm}
\begin{center}
{\Large\bf \stardoctitle}
\end{center}
\vspace{5mm}

This note presents a brief introduction to QDP.
The main user document is MUD/62 (`The QDP/PLT User's Guide') which should be
available at all Starlink sites.
Ask your Site Manager for a copy.

The text below has been extracted from Chapter 1 of the full guide.

\section{Overview}

The {\em Quick and Dandy Plotter (QDP)} program reads ASCII files containing
various plotting commands and data.
It then calls the {\em PLT}\ subroutine which executes the commands
and plots the data.
At this point the ``\verb@PLT>@'' prompt appears and the user can enter
additional PLT commands to:
\begin{itemize}
\item Display information about the interactive commands
{\em via} {\tt HElp},

\item Override various PLT defaults,

\item Override the PLT commands found in the QDP file,

\item Add/remove labels,

\item Plot data with various combinations of lines, markers, and error bars,

\item Change the appearance/style of the plot, for example converting
all text into Roman Font,

\item Plot the data as a function of a different $x$ variable,

\item Change the number of panels in which the data is plotted,

\item Define models to calculate the `best fit' parameter values,

\item Generate a hardcopy.
\end{itemize}

Thus, the interactive PLT commands allow you to both tailor the plot to your
needs/taste and to do some simple analyses of the data.
PLT commands can be placed in the QDP file, in an indirect command, and/or in
a command array created by the calling program.
For example, if you have a set of commands that you commonly use, you can
place them in a file and then have PLT execute the commands that it finds in
that file.
Since exactly the same command syntax is used in all places, it is not
necessary to learn a special programming language to write software that uses
PLT.
Programmers can try out PLT commands interactively to find a set that works
best with the type of data being plotted, and then make these commands the
default values.

The PLT software is highly portable.
It uses the PGPLOT Graphics Subroutine Library written by T J Pearson at the
California Institute of Technology.

\section{Running the program}

Before attempting to run the program, you must first initialise the
environment by entering the command:
\begin{verbatim}
   $ QDPSTART
\end{verbatim}
You can then start the program with the command:
\begin{verbatim}
   $ QDP
\end{verbatim}

\end{document}
