\documentstyle{article}
\pagestyle{myheadings}

%------------------------------------------------------------------------------
\newcommand{\stardoccategory}  {Starlink User Note}
\newcommand{\stardocinitials}  {SUN}
\newcommand{\stardocnumber}    {46.1}
\newcommand{\stardocauthors}   {R M Prestage}
\newcommand{\stardocdate}      {28 September 1988}
\newcommand{\stardoctitle}     {NOD2 --- A mm-wave Continuum Data Reduction
                                Package}
%------------------------------------------------------------------------------

\newcommand{\stardocname}{\stardocinitials /\stardocnumber}
\markright{\stardocname}
\setlength{\textwidth}{160mm}
\setlength{\textheight}{240mm}
\setlength{\topmargin}{-5mm}
\setlength{\oddsidemargin}{0mm}
\setlength{\evensidemargin}{0mm}
\setlength{\parindent}{0mm}
\setlength{\parskip}{\medskipamount}
\setlength{\unitlength}{1mm}

\begin{document}
\thispagestyle{empty}
SCIENCE \& ENGINEERING RESEARCH COUNCIL \hfill \stardocname\\
RUTHERFORD APPLETON LABORATORY\\
{\large\bf Starlink Project\\}
{\large\bf \stardoccategory\ \stardocnumber}
\begin{flushright}
\stardocauthors\\
\stardocdate
\end{flushright}
\vspace{-4mm}
\rule{\textwidth}{0.5mm}
\vspace{5mm}
\begin{center}
{\Large\bf \stardoctitle}
\end{center}
\vspace{5mm}

\section{Introduction}

NOD2 (Haslam, 1974) is a subroutine library for the general
manipulation of radio single-dish observations. The package was
developed at the Max Planck Institut f\"{u}r Radioastronomie, Bonn. The
library also forms the basis of the data reduction
package developed by NRAO, Tucson, for the 12-m telescope (Salter, 1985).
A large part of this package, with some extensions, has been adopted for
use with the James Clerk Maxwell Telescope.

Although the term `NOD2' should in principle be reserved to described
the subroutine library, it is also commonly used to refer to the NRAO
data reduction system. In this Note, the term NOD2 is used to refer to
the implementation of the package released by Starlink for the reduction
of JCMT data.

\section {Starting Up NOD2}

The data reduction package consists of a set of stand-alone
analysis programs that are either implemented as DCL foreign
commands, or as DCL global symbols. To invoke the package, simply
type:
\begin{verbatim}
    $ NOD2
\end{verbatim}
at DCL level. This will cause a command file to be run to set up the
DCL command table and the global symbols. NOD2 commands may then
be freely interspersed with normal DCL commands. For most NOD2 commands
it is sufficient to simply give the command name, and then answer the
resulting prompts. For example, type:
\begin{verbatim}
    $ MAKMAP
\end{verbatim}
to run the program to convert from raw data to a NOD2 map file.

\section {Uses and Deficiencies}

Many parts of the NOD2 system have not been fully implemented for Starlink
use. These include, for example, facilities for making hard copies of
images directly using the NOD2 system. The package has been made available
despite this however, because it contains implementations of certain
algorithms, essential to the reduction of mm-wave single dish observations,
which are not available elsewhere. These include the `RESTOR' routine,
which will reconstruct the equivalent single-beam map from data taken
in beam-switched mode (Emersion {\it et al.}, 1979), and
`CONVERT', which will create an R.A.-dec
map from Az-El data. The current method for reducing data from the JCMT
is thus to use NOD2 for the immediate processing required to
perform the `instrument-specific' part of the data reduction process,
and then to convert the resulting maps into another package (such as the
IRCAM reduction package, or FIGARO) for subsequent analysis. It is
hoped that in the long term, the capabilities of the NOD2 system will
be integrated into a more complete Starlink package.

\section {Additional Information}

Some online help about the package can be obtained by typing:
\begin{verbatim}
    $ HELP NOD2
\end{verbatim}
after the {\tt NOD2} command has been given to invoke the system.
Unfortunately, a lot of this information is out of date. More
information on the JCMT implementation of the package is available
in the document `A NOD2 Primer', written by G\"{o}ran Sandell, at
the Joint Astronomy Centre, Hilo, Hawaii. This is available on-line
in the file {\tt util\$disk:[nod2]primer.doc}. This primer should
be consulted by anyone seriously using the package. Finally, the
document `Continuum Mapping with the 12-M Telescope' by C.J. Salter
contains some more general information about reducing mm data.
This document may be available at your site; consult your site manager
for details.

\section {Acknowledgements}

Starlink would like to acknowledge Glyn Haslam for allowing the NOD2
subroutine library to be released by Starlink. Data reduced using this
package should quote the reference Haslam (1974), given below. We
would also like to thank Darrel Emerson for making the NRAO Tucson
package available to us.

\section {References}

Emerson, D.T., Klein, U. \& Haslam, C.G.T., 1979. Astron. Astrophys.,
{\bf 76}, 92.

Haslam, C.G.T., 1974. Astron. Astrophys. Suppl., {\bf 15}, 333.

Salter, C.J., 1985. `Continuum Mapping with the 12-M Telescope',
NRAO, Tucson.

\end{document}


