\documentstyle{article}
\pagestyle{myheadings}

%------------------------------------------------------------------------------
\newcommand{\stardoccategory}  {Starlink User Note}
\newcommand{\stardocinitials}  {SUN}
\newcommand{\stardocnumber}    {49.1}
\newcommand{\stardocauthors}   {A.~P.~Lotts}
\newcommand{\stardocdate}      {10~October~1989}
\newcommand{\stardoctitle}     {QUOTAS -- A utility to display VMS disk quotas}
%------------------------------------------------------------------------------

\newcommand{\stardocname}{\stardocinitials /\stardocnumber}
\markright{\stardocname}
\setlength{\textwidth}{160mm}
\setlength{\textheight}{240mm}
\setlength{\topmargin}{-5mm}
\setlength{\oddsidemargin}{0mm}
\setlength{\evensidemargin}{0mm}
\setlength{\parindent}{0mm}
\setlength{\parskip}{\medskipamount}
\setlength{\unitlength}{1mm}

%------------------------------------------------------------------------------
% Add any \newcommand or \newenvironment commands here
%------------------------------------------------------------------------------

\begin{document}
\thispagestyle{empty}
SCIENCE \& ENGINEERING RESEARCH COUNCIL \hfill \stardocname\\
RUTHERFORD APPLETON LABORATORY\\
{\large\bf Starlink Project\\}
{\large\bf \stardoccategory\ \stardocnumber}
\begin{flushright}
\stardocauthors\\
\stardocdate
\end{flushright}
\vspace{-4mm}
\rule{\textwidth}{0.5mm}
\vspace{5mm}
\begin{center}
{\Large\bf \stardoctitle}
\end{center}
\vspace{5mm}

%------------------------------------------------------------------------------
%  Add this part if you want a table of contents
%  \setlength{\parskip}{0mm}
%  \tableofcontents
%  \setlength{\parskip}{\medskipamount}
%  \markright{\stardocname}
%------------------------------------------------------------------------------

\section{Introduction}

QUOTAS is a utility to display the usage of disk quota on VMS disks.
On a VMS system the manager usually allocates individual users a quota
of disk space on one or more VMS disk volumes.
Normally a user can only examine their own usage on a particular volume.

Under some circumstances it is useful for users to be able to examine the
usage of all the users\footnote{
 A user in this document actually refers to a VMS User Identification code
 or UIC.
The information about usage is maintained using UICs but Quotas attempts to
translate these into Identifiers which are usually usernames.
See the VMS manuals for further explanations of UICs and Identifiers}
with quotas on a particular disk -- e.g. when a
disk that is used for scratch space is full, who is using all the space?
This utility provides a solution.
A single command will display an ordered list of the top users of a
disk.

Qualifiers are available to change the selection criteria, number of entries
displayed and the display format.

\section{Use}

The basic command {\bf QUOTAS} will display some information about the
current disk and then list the top users in descending order. By default,
it will use a slow but accurate method of obtaining the information.

The command has an optional parameter and several optional qualifiers.
The parameter and qualifiers are parsed using standard VMS techniques.

\begin{quote}
\begin{description}

\item{\bf format}
\vspace{3mm}

{\bf QUOTAS}~~~~~{\em [device]}
\vspace{2mm}

\item{\bf parameter}
\vspace{3mm}

{\bf device}\\
Specifies the disk to examine.
Any logical name (which eventually translates to a specific device), explicit
device name, or volume label is acceptable.
The default is SYS\$DISK, the current disk of the process.
\vspace{3mm}

\item{\bf qualifiers}
\vspace{4mm}

{\bf /ABOVE}{\em =usage limit}\\
 Only entries with a usage greater than /ABOVE will be considered for display.
 Default value is 1000 blocks.
\vspace{4mm}

{\bf /TOP}{\em =maximum to display}\\
The maximum number of entries to be displayed.
Default value is 10.
\vspace{4mm}

{\bf /FAST}\\
{\bf /NOFAST(default)}\\
Controls the technique to be used to obtain the quota information.

If /FAST is specified, a faster technique will be used.
Note however, this technique is unsupported by VMS.
Although approximately twice as fast, the results may be inaccurate.
\vspace{2mm}

{\bf /BRIEF}\\
{\bf /NOBRIEF(default)}\\
An abbreviated output display is used if the logical switch /BRIEF is specified.
\vspace{4mm}

{\bf /OUTPUT}{\em =file-spec}\\
Re-directs the output to a file or device.
\end{description}
\end{quote}

An example command using all the qualifiers could be
\begin{quote}
\begin{verbatim}
$ QUOTAS /BRIEF/TOP=20 /ABOVE=5000 /OUTPUT=NOW.LIS DISK$SCRATCH/FAST
\end{verbatim}
\end{quote}

New features may be added when VMS 5.2 is released.

\section{General Information}

The disk quota usage information kept by VMS is updated whenever a file is
created, modified or deleted. For various reasons, the values displayed
by this utility (or by the VMS command SHOW QUOTA) may disagree with the
space shown by the VMS DIRECTORY command.

\begin{itemize}

\item The DIRECTORY command should be used with the qualifier /SIZE=ALL
rather than just /SIZE.

\item The file system uses (and charges against the quota) one block for
every file on the disk.
This is not in the directory structure and so is not included in the output
from a DIRECTORY command.

\item A user may own files in other directories.

\item If large numbers of files are placed into the top directory of a
directory tree, it can itself become quite large.

\item Finally, there is a problem in VMS that appears to be related to
the way VMS caches quotas in memory.
In a VAXcluster, this cache appears not to be propagated correctly and so
even the supported methods of displaying quotas may produce incorrect results.
\end{itemize}

\section{Acknowledgments}

This utility is based on similar utilities used CALTECH and Cambridge.
Many thanks to the authors for freely providing the source files.


\end{document}
