\documentstyle{article}
\pagestyle{myheadings}

%------------------------------------------------------------------------------
\newcommand{\stardoccategory}  {Starlink User Note}
\newcommand{\stardocinitials}  {SUN}
\newcommand{\stardocnumber}    {72.2}
\newcommand{\stardocauthors}   {C.A. Clayton \& T.N. Wilkins}
\newcommand{\stardocdate}      {14 June 1991}
\newcommand{\stardoctitle}     {HONEY --- The Honeywell Camera}
%------------------------------------------------------------------------------

\newcommand{\stardocname}{\stardocinitials /\stardocnumber}
\markright{\stardocname}
\setlength{\textwidth}{160mm}
\setlength{\textheight}{240mm}
\setlength{\topmargin}{-5mm}
\setlength{\oddsidemargin}{0mm}
\setlength{\evensidemargin}{0mm}
\setlength{\parindent}{0mm}
\setlength{\parskip}{\medskipamount}
\setlength{\unitlength}{1mm}

\begin{document}
\thispagestyle{empty}
SCIENCE \& ENGINEERING RESEARCH COUNCIL \hfill \stardocname\\
RUTHERFORD APPLETON LABORATORY\\
{\large\bf Starlink Project\\}
{\large\bf \stardoccategory\ \stardocnumber}
\begin{flushright}
\stardocauthors\\
\stardocdate
\end{flushright}
\vspace{-4mm}
\rule{\textwidth}{0.5mm}
\vspace{5mm}
\begin{center}
{\Large\bf \stardoctitle}
\end{center}
\vspace{5mm}

\section{Introduction}

The Honeywell model 3000 colour graphic recorder system (hereafter referred to
simply as Honeywell) has been bought by Starlink for producing publishable
quality photographic hardcopy from the IKON image displays. Full colour and
black \& white images can be recorded on positive or negative 35mm film.
The Honeywell consists of a built-in high resolution flat-faced monochrome
video monitor, a red/green/blue colour filter mechanism and a 35mm camera.

The device works on the direct video signals from the IKON. This means that
changing the brightness or contrast on the IKON monitor will not affect any
photographs that you take. The video signals from the IKON consist of separate
red, green and blue signals. When you take a picture, the Honeywell takes the
red, green and blue signals in turn and displays three pictures consecutively
on its internal monitor. It takes an exposure through each of three filters
(red, green and blue) onto the film in the camera. This builds up the complete
colour picture on the film.

Honeywell systems are installed at nine Starlink sites, namely Belfast
(locally funded), Birmingham, Cambridge, Durham, Leicester, Manchester,
Rutherford, ROE and UCL.

\section{Using the Honeywell}

Once the Honeywell has been set up by your system manager, it is very simple
to use. Users should not touch any of the front panel control buttons, except
as described in Section 2.2. Your system may also be set up so that
you can control the Honeywell entirely via the VAX. Ask your
system manager if your system has this option. If so, then you
should read Section 2.3 entitled ``Using the Honeywell via the VAX''.

\subsection{Getting started}

At your site, the 35mm camera may not be permanently attached to the main
Honeywell unit
for security reasons. It is worth pointing out that the 35mm camera back has
been modified and is completely useless as a conventional camera. Once
you have obtained the camera back from your site manager (or from wherever it
is stored), attach the camera to the main body {\bf with the power off}.

Power the unit on via the green button (bottom left). The unit will now
warm-up for 4 minutes. During this period it is not possible to take
exposures. The message window indicates how much longer you must wait. For example,
the message WT3 indicates that you must wait for another
3 minutes. At the end of the warm-up sequence, an AUTO-CAL (automatic
brightness calibration) procedure will commence during which time CAL will
briefly
appear in the message window.
After the warm-up sequence the number showing in the display is the frame count
for that particular film setting.
It is worth noting that it is possible to skip
the warm up procedure if you do not intend to take any photos but are simply
checking parameters etc. This can be done by holding down RESET and pressing
SHIFT. Do NOT do this if you are intending to photograph the IKON or you
will get poor quality results.

To load a film, set the On/Off/AEL switch on the 35mm
camera back to the ``On'' position. Open the back cover of the 35mm camera back
by sliding the back cover lock release down. Place a roll of 35mm film into the
cavity at the left of the camera back. Lay the leading end of the film flat
along the back of the camera so that the holes in the film fit over the
sprocket at the other side. The leading edge of the film should extend over to
the roller. It does not have to wrap around the roller. Close the back
cover of the camera and the film will automatically advance to the first frame.

When the entire roll of film has been exposed, the End of Film indicator LED
will light. Push down the film rewind button on the bottom on the 35mm
camera body. Flip up the film rewind crank and turn it clockwise to
rewind the film into the film magazine.  When it is completely rewound,
the needle in the film advance indicator will stop turning and the rewind crank
will turn freely. Open the back cover of the 35mm camera back by sliding the
backing cover release down and remove the film.

\subsection{Using the Honeywell via the front panel}

The Honeywell can remember the settings for 8 different films in non-volatile
memory. There should be some indication on or near your Honeywell as to which
films are supported by your unit. This may take the form of a poster with brief
operating instructions. The four buttons in a vertical line on the left each
have 2 green LEDs and each button corresponds to 2 settings, as indicated. For
example, the top button is for settings 1 \& 2. If the left LED is lit then
film setting 1 is selected and if the right LED is lit, then setting 2 is
selected.

To select a film setting, hold down SHIFT and press the appropriate
button once for the left and twice for the right parameter. For example, if you
wish to select film setting 4, hold down SHIFT, and press the button marked 3-4
twice. The right LED on this button will then light.

Do NOT change any of the settings on either the camera or the main body {\bf under
any circumstances} without first consulting your site manager. The C/BW
(colour/black \& white) switch MUST be on C regardless of whether you
are using colour film or not. This is because the connection to the IKON
display is via RGB only and there is no input for the B/W mode
The P/N (positive/negative)
switch is disabled when in C mode and will be fixed on P.
The M/A (manual/automatic film wind-on switch) must be on A.
However, it can be set on M if superposition is required but exposure
times must obviously be shortened. Superposition could in theory be used to give
256$^3$ colours although this has not yet been tried.
The COMP (composite) must always be selected and NOT red, blue and
green. The camera shutter should be on B.
%and aperture set to f/8. The latter is almost impossible to check but should be
%correct.

After the warm-up sequence the number showing in the display is the frame count
for that particular film setting. This is useful if you have more than one
camera back, each  with a different film type in since the Honeywell can
remember how many exposures have been made on each film. If you have put in a
new film and wish to zero the frame counter, simply press the MINUS key until
the display reads 0.

You are now ready to take an exposure. Display the required image on the
IKON  attached to the Honeywell and press either the EXP button or the
footswitch if you have one. The camera will take three exposures of the image
through a red, blue and a green filter onto the same piece of film. This takes
about 10 seconds. When you hear the camera automatically wind on you may
change the display on the IKON ready for the next exposure.

\subsection{Using the Honeywell via the VAX}

Your system may have been set up so that your Honeywell can be controlled
entirely via the VAX using the program HONEY. Ask you system manager if this
is the case at your site and also the names of the films that it supports.

You must first startup HONEY and load the parameter set for the film
you wish to use from VAX disk into the Honeywell via the RESTORE command. The
parameter set files names that correspond to each type of film are listed in
Table 1. The following is an example session with HONEY.

\begin{quote}
\begin{verbatim}
$ HONEY
\end{verbatim}
\end{quote}
(Details of the currently selected parameter set is listed)

\begin{quote}
\begin{verbatim}
HONEY> RESTORE
  File to restore from: ED100
\end{verbatim}
\end{quote}

(The Ektachrome Daylight 100 parameter set will now be loaded and listed)

\begin{quote}
\begin{verbatim}
HONEY> RESET
\end{verbatim}
\end{quote}

(Resets film counter to zero - this step is optional)

You are now ready to take an exposure. Display the required image on the
IKON attached to the Honeywell and type EXPOSE at the HONEY prompt.

\begin{quote}
\begin{verbatim}
HONEY> EXPOSE
\end{verbatim}
\end{quote}

(Mechanical sounds emerge from Honeywell as picture is taken)

The
camera will take three exposures of the image through a red, blue and a green
filter onto the same piece of film. This takes about 10 seconds. When you hear
the camera automatically wind on you may change the display on the IKON
ready for the next exposure.

More HONEY instructions can be found in Appendix A. Users should only use
RESTORE, RESET, EXPOSE, BRACKET, CANCEL, FRAME, HELP, PAGE, SAVE \&
SHOW. The other
commands are for system managers to aid them in setting up the Honeywell
film settings.

Use CTRL-Z to exit from the HONEY program.

\section{Setting up the Honeywell (for system managers only)}

\subsection{Introduction}

The Honeywell should be connected to the IKON monitor via separate R,G,B \&
Sync cables to the BNC connectors on the rear panels of each device. The four
BNC ``out'' connectors on the Honeywell should be terminated at 75 $\Omega$.

Site managers are not expected to attempt the image calibration procedure
described in the manual. This  includes edge blanking adjustments, image size
and centring adjustments and should only be carried out by service engineers.
However, site managers are expected to set film parameters for the films used
at their site into the eight ``parameter sets''. This procedure is
described in the manual 2-8 through 2-10 but that
description contains errors. A corrected version is presented here for
completeness.

The four parameters to be set are brightness, contrast, ratio and exposure.
Brightness controls the illumination of the CRT when the video input level is
black. Contrast controls the illumination of the CRT (relative to the black
illumination) when the video input level is white. Ratio controls the exposure
factor for each colour relative to the other two. Overall tint can be
controlled via ratios. Exposure controls the total
exposure factor.

The parameters for each type of film are given in Table 1. Please note that
these values have NOT all been tested by the authors. These are values that have
been used in anger at Starlink sites, have given satisfactory results and
have been passed on to the authors by site managers for inclusion in this
document. Please send any corrections or additions to this table to Chris
Clayton at RLVAD::CAC.

\begin{table}[htb]
\begin{center}
\begin{tabular}{|l|l|c|c|c|c|c|c|c|c|c|c|}
\hline
{FILM TYPE} & {VAX file} &
\multicolumn{3}{|c|}{{BRIGHTNESS}} &
\multicolumn{3}{|c|}{{CONTRAST}} &
\multicolumn{3}{|c|}{{RATIO}} & {EXPOSE} \\
\cline {3-11}
& & {\tt R} & {\tt G} & {\tt B} & {\tt R} & {\tt G} &
{\tt B}& {\tt R} &  {\tt G} & {\tt B} & \\
\hline
%{\bf From Manual} &&&&&&&&&&& \\
%\hline
%{808} & {\tt 8x10} & {\tt 015} & {\tt 015} & {\tt 015} & {\tt 620} &
% {\tt 465} & {\tt 360} & {\tt 600} & {\tt 600} & {\tt 500} & {\tt 500} \\

% {809} & {\tt 8x10} & {\tt015} & {\tt 015} & {\tt 015} & {\tt 425} &
% {\tt 330} & {\tt 400} & {\tt 500} & {\tt 750} & {\tt 650} & {\tt 500} \\
%
% {891} & {\tt 8x10} & {\tt 015} & {\tt 015} & {\tt 015} & {\tt 425} &
% {\tt 330} & {\tt 400} & {\tt 500} & {\tt 750} & {\tt 650} & {\tt 500} \\
%
% {6118} & {\tt 8x10} & {\tt 015} & {\tt 015} & {\tt 015} & {\tt 620} &
% {\tt 465} & {\tt 360} & {\tt 600} & {\tt 600} & {\tt 230} & {\tt 500} \\
%
% {508 (558)} & {\tt 4x5} & {\tt 015} & {\tt 015} & {\tt 015} & {\tt 620} &
% {\tt 465} & {\tt 360} & {\tt 600} & {\tt 600} & {\tt 500} & {\tt 800} \\
%
% {509 (559)} & {\tt 4x5} & {\tt 015} & {\tt 015} & {\tt 015} & {\tt 425} &
% {\tt 330} & {\tt 400} & {\tt 500} & {\tt 750} & {\tt 650} & {\tt 800} \\
%
% {591} & {\tt 4x5} & {\tt 015} & {\tt 015} & {\tt 015} & {\tt 425} &
% {\tt 330} & {\tt 400} & {\tt 500} & {\tt 750} & {\tt 650} & {\tt 800} \\
%
% {6118} & {\tt 4x5} & {\tt 015} & {\tt 015} & {\tt 015} & {\tt 620} &
% {\tt 465} & {\tt 360} & {\tt 600} & {\tt 600} & {\tt 230} & {\tt 800} \\
%
% {708*} & {\tt SX70} & {\tt 015} & {\tt 015} & {\tt 015} & {\tt 210} &
% {\tt 305} & {\tt 285} & {\tt 300} & {\tt 750} & {\tt 400} & {\tt 300} \\
%
{EKT \scriptsize TUNGSTEN ASA 160} & {\tt ET160} & {\tt 015} & {\tt 015} &
{\tt 015} & {\tt 620} & {\tt 465} & {\tt 360} & {\tt 600} & {\tt 600} &
{\tt 230} & {\tt 120} \\

{EKT \scriptsize DAYLIGHT ASA 64} & {\tt ED64} & {\tt 015} & {\tt 015} &
{\tt 015} & {\tt 425} & {\tt 330} & {\tt 400} & {\tt 500} & {\tt 750} &
{\tt 650} & {\tt 200} \\

% {7240 \scriptsize TUNGSTEN ASA 125} & {\tt 16mm} & {\tt 015} & {\tt 015} &
% {\tt 015} & {\tt 620} & {\tt 465}
% & {\tt 360} & {\tt 600} & {\tt 600} &
% {\tt 230} & {\tt 350} \\
% \hline
% {\bf From David Malin (AAO)} &&&&&&&&&&& \\
% \hline
% {EKT-6117 \scriptsize ASA 64} & {\tt 4x5} & {\tt 010} & {\tt 010} & {\tt 010} &
% {\tt 400} & {\tt 300} & {\tt 400} & {\tt 600} & {\tt 750} & {\tt 650} &
% {\tt 800} \\
%
% {Polaroid Colour (559)} & {\tt 4x5} & {\tt 010} & {\tt 010} & {\tt 010} &
% {\tt 640} & {\tt 500} & {\tt 380} & {\tt 400} & {\tt 400} & {\tt 800} &
% {\tt 800} \\
%
% {Polaroid Colour} & {\tt SX70} & {\tt 025} & {\tt 025} & {\tt 025} &
% {\tt 200} & {\tt 300} & {\tt 280} & {\tt 350} & {\tt 750} & {\tt 600} &
% {\tt 275} \\
%
{EKT \scriptsize DAYLIGHT ASA 100} & {\tt ED100} & {\tt 015} & {\tt 015} &
{\tt 015} & {\tt 425} & {\tt 330} & {\tt 400} & {\tt 500} & {\tt 750} &
{\tt 650} & {\tt 160} \\

{EKT \scriptsize DAYLIGHT ASA 400} & {\tt ED400} & {\tt 015} & {\tt 015} &
{\tt 015} & {\tt 425} & {\tt 330} & {\tt 400} & {\tt 500} & {\tt 750} &
{\tt 650} & {\tt 40} \\

% {(B\&W) Polaroid 55 PN} & {\tt 4x5} & -- & {\tt 035} & -- & -- &
% {\tt 275} & -- & --& -- & -- & {\tt 950} \\
%
% {(B\&W) Polaroid 52} & {\tt 4x5} & -- & {\tt 035} & -- & -- & {\tt 275} &
% -- & -- & -- & -- & {\tt 350} \\
%
% {(B\&W) Polaroid 57} & {\tt 4x5} & -- & {\tt 010} & -- & -- & {\tt 350} &
%  -- & -- & -- & -- & {\tt 020} \\
%
%{(B\&W) Kodak Plus-X} & {\tt 35mm} & -- & {\tt 035} & -- & -- & {\tt 275} &
%   -- & -- & -- & -- & {\tt 150} \\
% \hline
% {\bf From IoA, Cambridge} &&&&&&&&&&& \\
% \hline
%{Polachrome} & {\tt 35mm} & {\tt 015} & {\tt 015} &
%{\tt 015} & {\tt 620} & {\tt 465} & {\tt 360} & {\tt 600} & {\tt 600} &
%{\tt 230} & {\tt 200} \\
%{Ektachrome 400 (day)} & {\tt 35mm} & {\tt 015} & {\tt 015} &
%{\tt 015} & {\tt 620} & {\tt 465} & {\tt 360} & {\tt 600} & {\tt 600} &
%{\tt 230} & {\tt 50} \\
{FP4 (B \& W)} & {\tt FP4} & {\tt 015}  & {\tt 015} & {\tt 015}
& {\tt 600} & {\tt 450} & {\tt 450} & {\tt 600} & {\tt 600} & {\tt 600} &
{\tt 180} \\
%{(B\&W) Polapan} & {\tt 35mm} & -- & {\tt 015} & -- & -- & {\tt 200} &
%   {\tt 500} & {\tt 500} & {\tt 200} & {\tt 200} & {\tt 250} \\
%{(B\&W) 2415 (Tech. pan)} & {\tt 35mm} & -- & {\tt 015} & -- & -- & {\tt 200} &
%   -- & -- & -- & -- & {\tt 300} \\
\hline
\end {tabular}
\end {center}
\end {table}

\subsection {\bf Brightness parameter adjustment}

When setting the brightness parameters, it is necessary to flick between two
modes -- brightness parameter mode (BRT) and brightness meter mode (MTR). The
meter mode is used to monitor the brightness of the CRT screen whilst the
parameter mode is used to adjust the brightness of the CRT. Please note that it
is vital that you do NOT adjust the brightness of the CRT (as described below)
whilst in the meter mode. If you do, then the auto-calibration routine which
takes place when one changes from MTR to BRT mode and will interpret the
change in brightness as an error and will compensate by changing the brightness
drive to the CRT. The end result will be new parameter and meter numbers
i.e. it manifests itself to the user as ``drift'' in the meter readings.

\begin{enumerate}

\item Select desired parameter set (film) as described in Section 2.2

\item Select RED video by pressing ``R'' switch on the front panel.

\item Select the Brightness Parameter function by holding down the SHIFT
switch and depressing the BRT-MTR switch once, such that the LED on
the left hand side of the BRT-MTR switch is illuminated.

\item Display a black screen on the IKON using HONEYWELL\_DIR:BLWH.EXE

\item Set the readout in the message window to approximately 365 using the
``+'' and ``-''  switches on the front panel.
In an unadjusted Honeywell,
an intial brightness parameter number of 365 will yield a brightness meter
reading of 015 with a black screen on the IKON. Hence, you should use 365
as a starting point on the brightness parameter side.

\item Hold down the SHIFT switch and press the BRT-MTR switch once such
that the LED on the right hand side of the BRT-MTR switch is
illuminated.  Release the SHIFT switch and then press the BRT-MTR switch
several times until the reading in the message window is constant.
Always press the meter switch several times to obtain an average reading.
Do not take the first reading as this is usual low. It will normally
take three presses to reach the final meter value. This is especially important
in the brightness meter mode with a black video display. A reading of
015 represents a very small light sensor output voltage and even a very
slight change in light level caused by ambient or residual light could change
that number.
Note this number.  If the reading is 015 go to step 9.

\item Should the reading not be 015 then the number set in step 5 must
be changed to a higher value if the reading is less than 015 and to a
lower value if the reading is greater than 015.

\item
NEVER use the ``+'' and ``-'' buttons when in brightness meter mode.
For example, if the brightness meter reads 005 when it should read 015, shift
over to the brightness parameter mode and increase the parameter number. Then switch
back to the brightness meter and note the new meter value -- It should be higher.
Do this by holding down the SHIFT switch and pressing the BRT-MTR switch
once such that the LED on the left hand side of the BRT-MTR switch is
illuminated.
Increase or decrease the parameter number from 300 in the message
window by, say 30, to 330 or 270 as appropriate, and repeat this
section until 015 is read consistently.

\item Hold in the SHIFT switch and press the BRT-MTR switch once such that
the left hand LED on the BRT-MTR is illuminated. Note the number for
future reference.

\item Select the GREEN video by pressing the G switch on the panel and repeat
steps 3 through 9.

\item Select the BLUE video by pressing the B switch on the panel and repeat
steps 3 through 9.

\end{enumerate}

\subsection {\bf Contrast parameter adjustment procedure}

Please note that the contrast meter and parameter modes may be used
interchangeably to increment the contrast drive voltage.
This is
NOT the case when adjusting the brightness parameter. The difference is
because the contrast meter (along with providing a readout of the light sensor
output voltage) parallels the function of the contrast parameter mode
in that it {\bf directly} controls the drive voltage to the video amplifier.
The brightness parameter setting, on the other hand, only {\bf indirectly}
controls the brightness drive.

\begin{enumerate}

\item Select Red video by pressing the RED switch on the front panel.

\item Select the Contrast Meter function by pressing and holding the
SHIFT switch while simultaneously pressing the CNT-MTR switch twice
such that the right hand LED on the CNT-MTR switch lights.

\item Display a white screen on the IKON using HONEYWELL\_DIR:BLWH.EXE.

\item Using the ``+'' or ``-'' switch, set the display in the message window to
the value listed in Table 1 under `CONTRAST - R' for the type
of film to be used.

\item Select green video by pressing the GRN switch on the front panel.

\item Using the ``+'' or ``-'' switch, set the display in the message window
to the value listed in Table 1 under `CONTRAST - G' for the
type of film to be used.

\item Select blue video by pressing the BLU switch on the front panel.

\item Using the ``+'' or ``-'' switch, set the display to the value listed in
Table 1 under `CONTRAST - B' for the type of film to be
used.

\end {enumerate}

\subsection {\bf Ratio parameter adjustment procedure}

\begin {enumerate}

\item Select Red video by pressing the RED switch on the front panel.

\item Select the Ratio Parameter by pressing and holding the SHIFT switch
while simultaneously pressing the RATIO button once, such that it
lights.

\item Using the ``+'' or ``-'' switch, set the display in the message window to
the value listed in Table 1 under `RATIO - R' for the type of film
to be used.

\item Select Green video by pressing the GRN switch on the front panel.

\item Using the ``+'' or ``-'' switch, set the display in the message window
to the value listed in Table 1 under `RATIO - G' for the type
of film to be used.

\item Select Blue video by pressing the BLU switch on the front panel.

\item Using the ``+'' or ``-'' switch, set the display in the message window to
the value listed in Table 1 under `RATIO - B' for the type of film
to be used.

\end{enumerate}

\subsection {\bf Exposure parameter adjustment procedure}

\begin{enumerate}

\item Select the Exposure Parameter by pressing and holding the SHIFT switch
while simultaneously pressing the EXP-NO switch once, such that the
left hand LED on the EXP-NO switch lights.

\item Using the ``+'' or ``-'' switch, set the display in the message window to
the value listed in Table 1 under EXPOSE for the type of film to be
used.

\item Select the Exposure Parameter by pressing and holding the SHIFT switch
while simultaneously pressing the EXP-NO. switch twice, such that the
right hand LED on the EXP-NO. switch lights.

\item Using the ``+'' or ``-'' switch, set the display in the message window to
1. This sets up single exposures. Multiple exposures are also possible by
selecting a higher number.

\end{enumerate}


The Honeywell has one important defect that Site managers should be aware
of. When the main unit is powered off, parameter settings are stored in
non-volatile memory. However, the lifetime of this non-volatile memory
is only about two weeks. Hence it is necessary to power-on your unit
about once per week in order to revitalize the non-volatile memory. Failure
to do this will result in the loss of your parameter setting and you will have
to restore them either by hand or using the HONEY program described below,
and also in Appendix A.

\subsection{Setting up your system for use with the HONEY control program}

To allow users to control the Honeywell entirely from the VAX as described in
Section 2.3, system managers must first set up the appropriate hardware and
software as indicated below.

Connection of the Honeywell to the VAX is via an RS232 cable. The Honeywell end should be male
and have pins 4, 5 \& 20 soldered together. Pins 2, 3 \& 7 should
run straight through to pins 2,3 \& 7 on the end to be connected
to a terminal server port.
The port should also have the following characteristics set:

\begin{itemize}

\item SPEED 1200

\item AUTOBAUD DISABLED

\item AUTOPROMPT DISABLED

\item CHARACTER SIZE 8

\item ACCESS REMOTE

\end{itemize}

The port must be set up as an application port using
a command procedure such as the example from Starlink Project VAXcluster
shown below. A number of logical names must also be defined.
You will have to change the
node name, port name and device name for your own site and place
these lines in a startup command procedure that is run at boot time.

\begin{quote}
\begin{verbatim}
$	RUN SYS$SYSTEM:LATCP
CREATE PORT/APPLICATION/NOLOG				LTA106:
SET PORT/APPLICATION/NODE=RLTS10/PORT=PORT_12/NOLOG	LTA106:
$	SET TERM/PERM/NOBROADCAST/NOTYPE_AHEAD 		LTA106:
$ SET PROTECTION=(W:RW) /DEVICE				LTA106:
$ DEFINE/SYSTEM HONEYWELL LTA106:
$ DEFINE/SYSTEM HONEYWELL_DIR STARDISK:[STARLINK.UTILITY.HONEY]
$ DEFINE/SYSTEM HONEY_S_DIR HONEYWELL_DIR
$ DEFINE/SYSTEM HONEYHELP HONEYWELL_DIR:HONEY.HLB
\end{verbatim}
\end{quote}

HONEY can be run via the symbol

\begin{quote}
\begin{verbatim}
$ HONEY
\end{verbatim}
\end{quote}

which is set up in a login command procedure with the entry

\begin{quote}
\begin{verbatim}
$ HONEY :== $ HONEYWELL_DIR:HONEY
\end{verbatim}
\end{quote}

Once system managers have set up their Honeywell cameras as described earlier in
Section 3, they should use the SAVE option in HONEY to save these settings
into VAX files in the system directory HONEY\_S\_DIR. These settings can then
easily be reloaded into the Honeywell using the RESTORE option in HONEY.
Indeed, RESTORE should always be used before taking photographs to ensure that
the parameter settings currently in the Honeywell
are correct. For example, the non-volatile memory
may have failed or an inquisitive user may have disrupted the settings.
The settings for each type of film should be saved into
the standard VAX file names listed in Table 1. An example showing the use
of RESTORE is shown in Section 2.3. All the HONEY commands are described
in more detail in Appendix~A

\section {Geometric adjustment of Honeywell (for system managers only)}

In order to facilitate the geometrical adjustment of the Honeywell, a
graticule that sits inside the CRT bevel has been supplied to all sites, marked
with a cross--hatch which matches the pattern produced by the program
HONEYWELL\_DIR:GRAT.EXE. This enables the geometry to be set up quite accurately and
also ensures that the overall size is right and that all the information
lies within the visible region of a mounted slide. Managers should not attempt
to tweak the Honeywell themselves. Instead, they should use it to diagnose
poor geometry and supply the graticule to service engineers when they arrive to
tune the camera.

By special permission from the service company the Starlink Project Manager
is empowered to carry out the basic geometry adjustments himself. If you feel
your unit would benefit from adjustment, ask your site manager to inform
the Project Manager so that the adjustments can be carried out should the
opportunity arise.

All the Honeywell cameras around Starlink should now be set up with
good geometry. Software does exist to illustrate deviations from perfect
geometry by analysing the deviations of the true positions of dots in an array
from their expected positions. However, this procedure requires specialist equipment
in order to determine the dot positions from photographs of the dot array
and hence is not described here. Furthermore, it is hoped
that the Honeywells are stable and that it will not become necessary to
use this software again.

For testing purposes, it is necessary to have an image displayed when not
exposing. This can be done in two ways. The crude method
is to hold down SHIFT and press BRT-MTR once. This will display the image on
the IKON for
about 20 seconds. The image can be left up on the CRT permanently on some
machines. Unlock the
cross-head screws on the back and remove the cover. On the right of the machine
are three PCBs. The middle one has at its top-right corner a switch marked S2.
Flip this switch and the image will stay on. The RGB buttons can select each
video channel. Alternatively, you can
remove the top plate just in front of the CRT and look directly onto the screen.
This plate is held on by two small allen screws but be {\bf very, very} careful
when taking them out since they are very easy to drop into the main unit.
Don't forget to flip switch S2 back again when you have finished. Keep clear of
the mains input and EHT regions. Starlink accepts no responsibility for
injuries sustained by users or staff who have gained access to the interior of
a machine as described.

The position of the interlace switch on the rear on the main unit did not
appear to affect the unit tested. This should be left on X1 since this
does not attempt any line smoothing.

\section{Testcard programs for IKON}

There are a number of test programs in HONEYWELL\_DIR which can be used to
aid system managers set-up and check their Honeywell camera. All
are self-explanatory when run, and use GKS device names. Each is briefly
described below:

\begin{description}
\item [BLWH] -- Produces black and white screens as required for
the parameter setting procedures described in Sections 3.2 \&
3.3.
\item[GRID] -- Produces a white grid on a black background.
\item[DOTS] -- Produces an array of dots.
\item[GRAT] -- Produces a grid identical to the overlay graticule
distributed to site managers.
\item[PATTRN] -- Interactive program to produce grids, dot patterns and
circles.
\item[TESTCARD] -- Colourful demonstration to check colour fidelity.
\item[TRIANGLES] -- Colourful demonstration to check colour fidelity.
\end{description}

\appendix

\section{HONEY - A Honeywell control program}

\subsection{Introduction}

The program HONEY gives a more user-friendly interface to the Honeywell
recorder than using it via the front panel. It also allows the user to
save device settings to disc and restore them at a later date. The
restore facility uses the directory search path, default directory,
HONEY\_U\_DIR, HONEY\_S\_DIR so standard settings can easily be
provided. The directory HONEY\_S\_DIR is designed as a ``system''
directory, and HONEY\_U\_DIR as a ``user'' directory. Alternatively
the user can specify the directory in the file name.

\subsection{Basic Use}

On running HONEY, the program will
tell you the current setup of the device, and return with a prompt.
For help information enter a the command ``HELP'', and you will get
a list of commands. The commands may be abbreviated provided that they
are still unique. The colour names for BRIGHTNESS, CONTRAST and RATIO
can also be abbreviated. Thus the user could type
\begin{quote}
\begin{verbatim}
HONEY> BRIGHTNESS GREEN 36
\end{verbatim}
\end{quote}
for example, but would get the same result from
\begin{quote}
\begin{verbatim}
HONEY> BRI G 36
\end{verbatim}
\end{quote}

The parameter set used may be from 1 to 8, but the program warns if
7 or 8 is accessed, so as to allow these two to be reserved for testing,
etc.

During an exposure, apart from CANCEL, only those commands which do not
communicate with the device (SAVE, SHOW, HELP) can be used. SHOW normally
communicates with the device to obtain the film position, but does not
if an exposure is in progress. With the BRACKET command, no commands may
be entered until the sequence has finished.

After a command to change the setting, the program will show the setting
of that parameter (or set of parameters), to allow the user to confirm
that the parameter is changed. Please note that some of the commands
such as {\bf MONOCHROME, POSITIVE \& NEGATIVE} should not be used with the
Starlink Honeywells but are included for completeness.

\subsection{Summary of Commands}

\newcounter{dummy}
\begin{list}{dummy}{\usecounter{dummy}}

\item[\bf BLUE] - Select blue mode
\item[\bf BRACKET {\em n1} {\em n2} {\em n3}...] Take a series of
exposures with exposure time (factor) set to {\em n1} {\em n2} {\em n3}
etc. The previous settings are then restored.
\item[\bf BRIGHTNESS {\em (colour) n}] - Set brightness to {\em n}
\item[\bf CANCEL] - Cancel exposure
\item[\bf COLOUR] - Select colour mode
\item[\bf COMPOSITE] - Select composite mode
\item[\bf CONTRAST {\em (colour) n}] - Set contrast to {\em n}
\item[\bf EXPOSE] - Make exposure
\item[\bf FRAME {\em n}] - Set frame counter to {\em n}
\item[\bf GREEN] - Select green mode
\item[\bf HELP] - Give help information
\item[\bf MONOCHROME] - Select black and white mode
\item[\bf MULTIPLE {\em n}] - Set number of multiple exposures to {\em
n}.
\item[\bf NEGATIVE] Select negative video
\item[\bf PAGE {\em n}] - Select parameter set {\em n} ({\em n} is in
the range 1 to 8)
\item[\bf POSITIVE] Select positive video
\item[\bf RATIO {\em colour n}] - Set exposure ratio  to {\em n}
(colour only)
\item[\bf RED] - Select red mode
\item[\bf RESET] - Reset frame counter
\item[\bf RESTORE] - Restore parameter set from a file created by the
SAVE option
\item[\bf SAVE] - Save parameter set to a file (the file name is
prompted for).
\item[\bf SHOW] - Show current settings
\item[\bf SINGLE] - Set exposures to single per command (this is the
same as MULTIPLE 1)
\item[\bf TIME {\em n}] - Set exposure time (factor) to {\em n}
\item[\bf CTRL-Z] - Leave program
\end{list}

For BRIGHTNESS, CONTRAST and RATIO the colour (e.g. RED) must be
specified if in colour mode, but omitted in monochrome mode (but note
that RATIO only applies to colour mode). Where {\em n} is given as a
parameter, it is an integer from 0 to 999, except for PAGE, where it is
from 1 to 8.

Note that only one of each of the following settings may apply at
any one time (most are fairly obvious):
BLUE, GREEN, RED, and COMPOSITE\\
COLOUR and MONOCHROME\\
NEGATIVE and POSITIVE\\
Also note that BLUE, GREEN, RED and COMPOSITE apply only to colour
mode, and NEGATIVE and POSITIVE only to black and white mode.

When using the BRACKET option for several images, it may be found
useful to recall the command using up arrow or control b, just
repeating the command with different images.

Note that, with the MULTIPLE command, problems can arise if the number
of exposures set is too large (around 20). This seems to be due to some
limit in the VMS buffering, and will crash the program. If this occurs
enter the command
\begin{quote}
\begin{verbatim}
$ SET HOST/DTE HONEYWELL
\end{verbatim}
\end{quote}
to allow the Honeywell to write to the terminal. If desired the exposure
can be cancelled by entering the command ``CT'' (in capitals). Enter
control backslash to end this connection.

\end {document}
