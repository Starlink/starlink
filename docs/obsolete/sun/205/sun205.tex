\documentclass[twoside,11pt,nolof,noabs]{starlink}

% -----------------------------------------------------------------------------
% ? Document identification
\stardoccategory    {Starlink User Note}
\stardocinitials    {SUN}
\stardocsource      {sun\stardocnumber}
\stardocnumber      {205.1}
\stardocauthors   {D.\,J.\,Rawlinson\\
                                J.\,C.\,Sherman}
\stardocdate        {12th  December 1995}
\stardoctitle       {FORUM --- Starlink Conferencing Software}
% ? End of document identification
% -----------------------------------------------------------------------------
% ? Document-specific \providecommand or \newenvironment commands.
% ? End of document-specific commands
% -----------------------------------------------------------------------------
%  Title Page.
%  ===========
\begin{document}
\scfrontmatter

\section{Introduction}

This \emph{User Note} is a brief introduction to FORUM for users.
A companion document, \xref{SSN/33}{ssn33}{}, intended for Site Managers, describes
how to install FORUM.

FORUM was first developed by Richard West, Leicester University, and
then re-implemented by Adrian Fish, Starlink Site Manager at UCL, to
tailor it to Starlink's requirements.

\section{What is FORUM?}

FORUM is Starlink's conferencing software.  It is based on the World
Wide Web (WWW) and provides similar capabilities under Unix to those
provided by the DEC product VAXnotes under VMS.  Its main features are
that it retains an archive of entries or notes, organized in a
hierarchy, and it can restrict access.  Users can both read existing
notes and submit new ones.    The hierarchy of entries, from the top
down, is:  ``Conference'', ``Topic'', ``Reply'' and ``Riposte''.  One
difference compared with VAXnotes is that while VAXnotes could maintain
an index of all Conferences, even when the Conferences were located on
separate systems at several Starlink sites, FORUM maintains an index
only for each instance of FORUM.

The best way to find out more about FORUM is to run it.  FORUM has also
been described in a \emph{Starlink Bulletin} article, December 1995.


\section{Using Starlink FORUM}

To run FORUM simply start up your
usual WWW browser, for example, Netscape or Mosaic, and either open the URL
\begin{quote}
\url{http://rlsaxps.bnsc.rl.ac.uk/Forum}
\end{quote}
or click on FORUM from the Starlink home page.

You will be presented with a list of Conferences and will usually find
that access is denied to some of them (access restrictions are described
in \S \ref{sec:access}).   If you find that access is denied to \emph{all} Conferences, then it may be that your system is not running the
correct identity checking software.   Alternatively, it may be that you
are using a WWW proxy (if a proxy is used then the machine calling
FORUM will be the proxy server and not yours, with obvious difficulties
for identity checking).  If you find you cannot access \emph{any} FORUM
Conferences, please contact your Site Manager.

Once connected, you should immediately enter your user profile
information. This is required only once but can be updated subsequently
(following a change in e-mail address etc.). The information will be
used to provide shortcuts when submitting notes.  Just click on the
[User Profile] button on the FORUM home page.  At the same time you can
set a flag which specifies whether you would like to see the next
unread note when entering a Conference (the default is an index of
notes).  When all unread notes are exhausted, FORUM will revert to
displaying an index.  Alternatively, don't set the flag and select
notes manually.

The user profile also allows you to specify the address of your WWW
server and its port number.  This information is required if you intend
to   upload  existing files to FORUM.  Such files can be plain text or
can include images, HTML, PostScript or \TeX/\LaTeX~DVI.  If your FORUM
submissions are typed in while running FORUM (this is always the case
for Ripostes), the server address and port number can be left blank.

FORUM has a short [Help] page and new users can experiment using the
\texttt{Test} Conference.  For example, try out new techniques such as
uploading GIF files or embedding HTML in notes.

\subsection{[Upload]}

The [Upload] function is accessed via the [Add Reply] function.  It
uploads existing files and, as noted previously, makes use of user
profile information.   The file to be uploaded must be placed in your
\emph{public\_html} sub-directory before running FORUM and FORUM relies
on the usual WWW syntax to access this directory (for example,
\verb+~jcs+).   The file to be uploaded will be copied to FORUM so you
are free to delete or rename your own copy after using [Upload].
Because of this copying, if you are uploading an HTML file then any
pointers to other HTML pages must use absolute addresses, rather than
relative addresses.  Please refer to your Site Manager if you are
unfamiliar with the use of \emph{public\_html}.

\subsection{[Extract Unseen]} This function extracts all unseen notes
in a Conference (the Conference in use at the time) to a Web page.
This page can be read directly or, using the usual Netscape or Mosaic
functions, output to a printer or file.

The [Extract Unseen] function is especially useful in conjunction with
the \texttt{Lynx} character-cell based WWW Browser, which can be run in
batch mode.    For example, a batch command

\begin{terminalv}
% lynx -dump http://rlsaxps.bnsc.rl.ac.uk/Forum/extractunseen/Conference >out
\end{terminalv}

where `Conference' is the case-sensitive Conference name, will extract
all unseen notes and redirect them to `out'.   In this way, all unseen
notes can be printed or e-mailed, once a day.

\section{Conferences Available}

There are several Conferences which are restricted to Site Managers or
other Starlink staff and, at the time of writing, three which are
available to all users.   Please send any requests for further
Conferences, or help, to Dave Rawlinson (djr@star.rl.ac.uk).

The three generally available Conferences at the time of writing are:
\texttt{Test, Software Links,} and \texttt{Questions and Answers}.  The
\texttt{Test} Conference has already been mentioned.
The \texttt{Software Links} Conference is intended for users to announce
any public domain software package they have in use at their site and
believe would be generally useful to UK astronomers.   Such software
can be made available by placing a link to it in the note.
The \texttt{Questions and Answers} Conference is a forum for all Starlink users
to raise questions and discuss problems on any aspect of Starlink.  It is
hoped that in time it will become a repository of information that
users can search if they come across a problem.

\section{\label{sec:access}Access Restrictions}

Access restrictions are important for some Site Manager Conferences,
which include, for example, material on Unix security and
commercial-in-confidence matters.  FORUM restricts access via an
authentication daemon called \texttt{pidentd}, and provides user-based
access control, Conference by Conference.  Identity checking is also an
essential part of determining unseen notes, which must be done user by
user.

Experience has shown that completely open Conferences can be abused.
We have therefore set up FORUM to retain a record of who has accessed
the system, even when access to a Conference is open to all users.
This requires that all users must be running the identity checking
daemon \texttt{pidentd} (or an equivalent) on their node.  We hope this
will deter abuse but  a penalty of identity checking is that it causes
a delay of a few seconds per page.

\end{document}
