\documentstyle{article} 
\pagestyle{myheadings}

%------------------------------------------------------------------------------
\newcommand{\stardoccategory}  {ADAM Release Note}
\newcommand{\stardocinitials}  {ARN}
\newcommand{\stardocnumber}    {19.1}
\newcommand{\stardocauthors}   {A J Chipperfield}
\newcommand{\stardocdate}      {31 January 1991}
\newcommand{\stardoctitle}     {ADAM --- Release 1.9}
%------------------------------------------------------------------------------

\newcommand{\stardocname}{\stardocinitials /\stardocnumber}
\markright{\stardocname}
\setlength{\textwidth}{160mm}
\setlength{\textheight}{240mm}
\setlength{\topmargin}{-5mm}
\setlength{\oddsidemargin}{0mm}
\setlength{\evensidemargin}{0mm}
\setlength{\parindent}{0mm}
\setlength{\parskip}{\medskipamount}
\setlength{\unitlength}{1mm}

%------------------------------------------------------------------------------
% Add any \newcommand or \newenvironment commands here
%------------------------------------------------------------------------------

\font\tt=CMTT10 scaled 1095
\renewcommand{\_}{{\tt\char'137}}

\begin{document}
\thispagestyle{empty}
SCIENCE \& ENGINEERING RESEARCH COUNCIL \hfill \stardocname\\
RUTHERFORD APPLETON LABORATORY\\
{\large\bf Starlink Project\\}
{\large\bf \stardoccategory\ \stardocnumber}
\begin{flushright}
\stardocauthors\\
\stardocdate
\end{flushright}
\vspace{-4mm}
\rule{\textwidth}{0.5mm}
\vspace{5mm}
\begin{center}
{\Large\bf \stardoctitle}
\end{center}
\vspace{5mm}
%------------------------------------------------------------------------------
%  Add this part if you want a table of contents
  \setlength{\parskip}{0mm}
  \tableofcontents
  \setlength{\parskip}{\medskipamount}
  \markright{\stardocname}
%------------------------------------------------------------------------------

\section{SUMMARY}
This is a `complete' release of ADAM to update the system with the latest 
versions of the Starlink graphics and other libraries and to provide some 
enhancements to ICL and SMS.
There are minor enhancements to other facilities and some bugs are fixed in 
ICL, the parameter system and the TASK library.

The release also begins the process of making Starlink subroutine
libraries available automatically to application developers.

IT WILL BE NECESSARY TO MODIFY ANY PRIVATE OR SITE-SPECIFIC VERSIONS OF
SYSLOGNAM.COM

It will not be necessary to re-link tasks for this release unless the 
alterations to the TASK or DTASK libraries are required.
(See Sections \ref{task} and \ref{dtask}.)
The TRACE utility has been re-linked to incorporate the improved DTASK
routines.

The following important Starlink Software Changes are incorporated:
\begin{itemize}
\item SGS
\item HDS -- (see also Section \ref{hds})
\item ERR, MSG and EMS
\item NDF, ARY
\end{itemize}
For details, refer to the latest relevant SSC number prior to that of this 
release.

The full system requires about 42000 blocks of disk storage and includes a
mini-system which can be extracted and put up separately. The mini-system
requires about 14000 blocks and allows tasks to be run and  developed.

\section{INSTALLATION}
Full installation instructions are given in SSN/44 and the Starlink Software 
Change Notice.

\section{NEW FEATURES IN THIS RELEASE}

\subsection{ICL}
\begin{itemize}
\item ICL Version 1.5-4 is released - the login banner has been changed
appropriately.\\ (LOGIN.ICL, ADAMLOGIN.ICL)
\item A new command, SET NOAUTOLOAD/AUTOLOAD is implemented. It prevents ICL
from automatically loading tasks which can cause trouble, particularly for
D-tasks following crashes.
The default is AUTOLOAD.\\
(ICLDEF.PAS, ICLMAIN.PAS, ICLPROC.PAS, ICLADAM.PAS).
\item The help library has been updated to include the new commands.\\
(ICLHELP.HLP/.HLB)
\end{itemize}

\subsection{SGS}
\label{sgs}
\begin{itemize}
\item The include files for error status values have been renamed to the
standard form.\\ (SGSERR becomes SGS\_ERR)
\item The graphics shared image includes the latest release of SGS - errors from
SGS are reported in standard fashion and standard status values are returned.
This means that the ADAM-special version of the SGS\_\$ERR subroutine is no
longer required -- it has been removed from the library.
Three new SGS status values are defined.\\
(SGS\_\-ERR\-.MSG/.FOR/.OBJ, 
\item The prologues of the environment-level subroutines have been improved
to provide better automatic documentation.\\
(SGS\_\-ANNUL, ASSOC, CANCL, DEACT )
\end{itemize}

\subsection{GKS}
\label{gks}
\begin{itemize}
\item The include files for error status values have been renamed to the
standard form.\\ (GKSERR becomes GKS\_ERR)
\item The prologues of the environment-level subroutines have been improved
to provide better automatic documentation.\\
(GKS\_\-ANNUL, ASSOC, CANCL, DEACT, GSTAT)
\end{itemize}

\subsection{HDS}
\label{hds}
The ADAMSHARE shared image is built with the latest version of HDS available
at RAL. It contains major internal revisions but should be functionally the
same as the released version, apart from an additional subroutine DAT\_WHERE.
It has been modified in line with SSC487, to correct the file-extend error 
when copying small files.

{\em This version has not yet been released}, therefore any attempt to rebuild
the system must take this into account, possibly removing DAT\_WHERE from the
transfer vector.

\subsection{DTASK}
\label{dtask}
\begin{itemize}
\item DTASK\_GET and DTASK\_SET, which implement the GET and SET context 
messages, have been modified.
DTASK\_SET has been made to work for monoliths in the same way as DTASK\_GET.
At the same time, improvements have been made to check that the format of
the parameter name is appropriate for the type of task ({\em i.e.}\ in the form 
{\em task:parameter} for a monolith).
A new SUBPAR routine is provided to determine whether or not the task is a
monolith (see Section \ref{subpar}).\\
(Modified: DTASK\_GET, DTASK\_SET)
\item The task ending messages from A-tasks ending with STATUS set when run 
from DCL, have been improved.
The command-line parameter string will now be output in the form:
\begin{quote}
Command parameters /{\em parameters}\//.
\end{quote}
No message will be output if there are no command-line parameters.\\
(Modified: DTASK\_DCLTASK)
\end{itemize}

\subsection{The Parameter System (SUBPAR)}
\label{subpar}
\begin{itemize}
\item A new subroutine, SUBPAR\_MLITH, is provided to return the value of the 
MONOLITH flag from the SUBPAR common block, SUBPARPTR.
(New routine: SUBPAR\_MLITH)
\item SUBPAR\_CMDLINE has been modified to increase the maximum number of 
parameters allowed on the command line from 32 to 50.
(Modified: SUBPAR\_CMDLINE)
\end{itemize}

\subsection{MSG, ERR and EMS}
Numerous enhancements have been made to these systems.
New subroutines have been provided and the behaviour of others, particularly
the token setting routines, has been changed.
ERR\_LOAD and ERR\_ELOAD have additional arguments and
the way in which they and ERR\_REP treat the STATUS argument has also changed.
These changes could result in different behaviour by tasks.
For full details, see EMS release notes and SUN/104.2.

\subsection{Starlink Libraries}
\label{starshrlib}
A start has been made on allowing application developers to use the main
Starlink subroutine libraries without the need to issue additional commands
or include additional specifications in their link commands.
To this end:
\begin{enumerate}
\item The `analysis' application link commands ALINK, MLINK and ICLMLINK have 
been modified to include STAR\_LINK\_ADAM/OPT\footnote{.
The scheme is not implemented for the non-shareable link commands, 
ANOSHR and MNOSHR.}.
The released version of SYSLOGNAM.COM defines a logical name STAR\_LINK\_ADAM
to be ADAM\-\_EXE:\-STAR\-\_LINK\-\_ADAM.

\item A link options file, ADAM\_EXE:STAR\_LINK\_ADAM.OPT, has been created.
Currently it includes the option:
\begin{quote}
STAR\_LINK\_ADAM/LIB
\end{quote}

\item A shareable image library, ADAM\_EXE:STAR\_LINK\_ADAM.OLB has been 
created.
Currently it contains the shareable images for NDF and ARY.

\item A procedure defined by the logical name STAR\_DEV\_ADAM is obeyed by 
the ADAMDEV procedure.
The released version of SYSLOGNAM.COM defines the logical name to be
ADAM\_DIR:\-STAR\_DEV\_ADAM.
The supplied procedure will obey NDF\_DEV, ARY\_DEV and PRM\_DEV if those 
symbols are defined.
\end{enumerate}

Notes:
\begin{itemize}

\item The additional library search should not cause any significant overhead
in link times and the shareable images themselves need only be available if 
routines within them are required by the task being linked.
However, those sites which are not interested in having this facility, or want
alternative libraries available, may wish to modify SYSLOGNAM.COM
to re-define STAR\_LINK\_ADAM and/or STAR\_DEV\_ADAM.

A null version of the link options file is provided in 
ADAM\_EXE:STAR\_LINK\_ADAM0.OPT and a null version of the procedure in
ADAM\_DIR:STAR\_DEV\_ADAM0.COM.
These may be used instead to cause effectively no action.
(For details of how to modify SYSLOGNAM.COM, see SSN/44.)

\item Existing link commands should continue to work, albeit with some
redundant library searching.

\item Users developing libraries which are included in 
STAR\_LINK\_ADAM.OLB will also need to prevent the normal operation.
This is easily done by defining a logical name STAR\_LINK\_ADAM to take 
precedence over the SYSTEM logical name and pointing to 
ADAM\_EXE:\-STAR\-\_LINK\_ADAM0 or to a
private library containing just those shareable images which they do want.

\item A procedure to create STAR\_LINK\_ADAM.OLB is held in SHARE\_DIR.
It may be used as a template for private libraries.

\item For the moment, the STAR\_LINK\_ADAM files and STAR\_DEV\_ADAM.COM
are part of the ADAM release.
They will probably be reorganised into the Starlink directories in the near 
future so that they may be maintained separately from ADAM releases.
\end{itemize}

(Modified: ALINK.COM, MLINK\-.COM, ICLMLINK.COM,
ADAM\-DEV\-.COM, SYS\-LOG\-NAM\-.COM, SYS\-LOG\-NAM\-.MINI.
 New files: STAR\-\_LINK\-\_ADAM\-.OPT/\-.OLB/\-.COM, STAR\-\_DEV\-\_ADAM\-.COM,
STAR\-\_LINK\-\_ADAM0\-.OPT, STAR\-\_DEV\-\_ADAM0\-.COM)

\subsection{Other Procedures Etc.}
\begin{description}

\item[SYSLOGNAM.COM/.MINI]
MODIFIED VERSIONS OF THESE FILES WILL NEED UPDATING.
\begin{itemize}
\item Logical names STAR\_DEV\_ADAM and STAR\_LINK\_ADAM are defined as 
indicated in Section \ref{starshrlib}. 
THEY MAY NEED CHANGING AT SPECIFIC SITES.
\item SYS\$DISK has been added to the `current directory' part of the 
definition of ADAM\_EXE.
\end{itemize}

\item[ADAMSTART.COM] is modified to:
\begin{itemize}
\item Display the latest ADAM version number.
\item Switch off verification -- restoring to its initial state on exit.
\item Symbol ADAM\_DEV is defined equivalent to ADAMDEV for compatibility
with other Starlink packages.
\end{itemize}

\item[ADAMDEV.COM] is modified to:
\begin{itemize}
\item Obey STAR\_DEV\_ADAM to make logical names for Starlink libraries 
available for application development (see Section \ref{starshrlib}).
\item Switch off verification -- restoring to its initial state on exit.
\end{itemize}

\item[ADAMSHARE.COM] is modified to:
\begin{itemize}
\item Increase the minor id of the shareable image.
\item Use the changed names of the object modules for SGS and GKS error values 
(see Sections \ref{sgs} and \ref{gks}).
\item Declare an additional common block for EMS.
\end{itemize}

\item[ADAMSHARE.MAR]
Additional subroutines for SUBPAR, MSG, ERR and EMS are added to the end of 
the transfer vector.

\item[ADAMGRAPH7.COM]
The names of the object modules for SGS and GKS error values have been changed.
SGS\_\$ERR is no longer included.

\item[LINKNOSHR.OPT]
The names of the object modules for SGS and GKS error values have been 
changed. SGS\_\$ERR is no longer included.

\item[APPLOG.COM, LOGICAL.COM]
The names of the include files for SGS and GKS error values have been changed.

\item[ALINK.COM, MLINK.COM and ICLMLINK.COM] Include STAR\_LINK\_ADAM link
options file (see Section \ref{starshrlib}).
\end{description}

\subsection{Documentation}
\begin{itemize}

\item SSN/45 and ARN/19 describe ADAM release 1.9.

\item The first steps are taken to withdraw all the documents in the ADAM
documentation set and replace relevant ones with Starlink documents.
\begin{itemize}
\item The APP series is no longer distributed.
\item AED/3, AED/7, AED/8 AND AED/12 are withdrawn. They have been superceded
by SG/4 and SUN/115.
\item APN/2 (GKS) and APN/3 (SGS) are withdrawn. They are replaced by SUN/113.
\end{itemize}

\item The following applicable Starlink documents have been released or
updated.
\begin{itemize}
\item SSN/44 ADAM -- Installation Guide.
\item SUN/104 MSG and ERR -- Message and Error Reporting Systems.
\item SUN/113 SGS and GKS -- ADAM Programmer's Guide.
\item SUN/115 ADAM -- Interface Module Reference Manual
\item SSN/4 EMS -- Error Message Service.
\end{itemize}

\item The summaries, ADAM\_DOCS:0CONTENTS.LIS, FULLDOCS.LIS and NEWDOCS.LIS,
have been updated. 
\end{itemize}


\section{BUGS FIXED}

\subsection{ICL}
\begin{itemize}
\item The CHECKTASK command has been modified.
If task does not exist but is present in ICL's internal tables
then the entry in the tables is now removed. 
This prevents confusion when a task crashes. 
Checktask used to find the task wasn't there, but issuing LOADW
would return with an error `Task already loaded'.\\
(ICLADAM.PAS)
\item A bug in the ELEMENT function has been fixed.
The last element in a list was being truncated
by one character if it wasn't terminated by the delimiter.\\
(ICLFUNC.PAS)
\end{itemize}

\subsection{Parameter System (SUBPAR)}

A number bugs which prevented correct interpretation of object names in some
cases, particularly where the top level object name was omitted or was an 
array, have been corrected.\\
(SUBPAR\_\-SPLIT, CREAT, DELET, HDS\-OPEN)

\subsection{STRING}
A bug which caused strings to be split up incorrectly if they contained
terminators within brackets ({\em e.g.}\ OBJECT(1,1) where `,' is normally
a terminator), has been fixed.
This affected parsing of the interface file.\\
(STRING\_\-ARRCHAR)

\subsection{TASK}
\label{task}
A bug which has prevented TASK\_ASKPARAM from working properly since a new
format for the parameter request message was introduced in ADAM V1.8, has
been corrected.\\
(TASK\_ASKPARAM)


\section{CLEANUP}
The STRING library is being considered for removal. It seems appropriate that
its general purpose functions should be transferred to portable code in the
CHR library and other more specific functions be incorporated into other
ADAM system packages.
\end{document}
