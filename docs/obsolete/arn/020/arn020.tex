\documentstyle{article} 
\pagestyle{myheadings}

%------------------------------------------------------------------------------
\newcommand{\stardoccategory}  {ADAM Release Note}
\newcommand{\stardocinitials}  {ARN}
\newcommand{\stardocnumber}    {20.0}
\newcommand{\stardocauthors}   {A J Chipperfield}
\newcommand{\stardocdate}      {22nd October 1991}
\newcommand{\stardoctitle}     {ADAM --- Release 2.0}
%------------------------------------------------------------------------------

\newcommand{\stardocname}{\stardocinitials /\stardocnumber}
\markright{\stardocname}
\setlength{\textwidth}{160mm}
\setlength{\textheight}{230mm}
\setlength{\topmargin}{-2mm}
\setlength{\oddsidemargin}{0mm}
\setlength{\evensidemargin}{0mm}
\setlength{\parindent}{0mm}
\setlength{\parskip}{\medskipamount}
\setlength{\unitlength}{1mm}

%------------------------------------------------------------------------------
% Add any \newcommand or \newenvironment commands here
%------------------------------------------------------------------------------

\font\tt=CMTT10 scaled 1095
\renewcommand{\_}{{\tt\char'137}}

\begin{document}
\thispagestyle{empty}
SCIENCE \& ENGINEERING RESEARCH COUNCIL \hfill \stardocname\\
RUTHERFORD APPLETON LABORATORY\\
{\large\bf Starlink Project\\}
{\large\bf \stardoccategory\ \stardocnumber}
\begin{flushright}
\stardocauthors\\
\stardocdate
\end{flushright}
\vspace{-4mm}
\rule{\textwidth}{0.5mm}
\vspace{5mm}
\begin{center}
{\Large\bf \stardoctitle}
\end{center}
\vspace{5mm}
%------------------------------------------------------------------------------
%  Add this part if you want a table of contents
  \setlength{\parskip}{0mm}
  \tableofcontents
  \setlength{\parskip}{\medskipamount}
  \markright{\stardocname}
%------------------------------------------------------------------------------

\section{SUMMARY}
This is a `complete' release of ADAM. 
Its main features are Version 2 of the inter-task message system (based on the
Message System Primitive Routines, MSP) and Version 2 of the task fixed-part
which introduces the concept of an Instrumentation Task to supercede D-tasks
and CD-tasks and goes some way to rationalizing the different task types.

HDS Version 4 is used -- this will produce files which cannot be read by earlier
versions of HDS (or ADAM).

The parameter system code has been extensively modified, mainly in the
interests of portability. 
EMS and the Starlink error conventions are introduced, notably in HDS and the
parameter system -- this will result in more helpful error messages but may
also cause extra warning messages if tasks have not obeyed the conventions
(see Section \ref{ems}).

There are minor enhancements and bug fixes to other facilities.

IT WILL BE NECESSARY TO MODIFY ANY PRIVATE OR SITE-SPECIFIC VERSIONS OF
SYSLOGNAM.COM

Although it should not be strictly necessary to re-link tasks at this release,
it has been required in some cases.
Re-linking is recommended anyway so that full benefit may be obtained from the 
improved error reporting, and future changes to the fixed-part will be
incorporated automatically into existing tasks.
(See Sections \ref{dtask} and \ref{task}.)

The full system requires about 39000 blocks of disk storage and includes a
mini-system which can be extracted and put up separately. The mini-system
requires about 13000 blocks and allows tasks to be run and  developed.

\section{INSTALLATION}
Full installation instructions are given in SSN/44 and the Starlink Software 
Change Notice.

\section{NEW FEATURES IN THIS RELEASE}

\subsection{DEPENDENCIES}
There are some additional Starlink Software items required to be installed
before tasks may be run.
They are:
\begin{itemize}
\item HDS -- (see Section \ref{hds})
\item EMS -- (see Section \ref{ems})
\item CHR -- (see Section \ref{chr})
\end{itemize}


\subsection{DTASK}
\label{dtask}
\begin{itemize}
\item Extensive changes to the DTASK library have been made to implement the 
Version 2 task fixed-part based on William Lupton's ADAMSYS system and further
developed at ROE.

The new fixed part is described in SUN/134.
\item The DTASK library is now linked into tasks using a shareable image,
DTASK\_IMAGE\_ADAM.
\end{itemize}

\subsection{TASK}
\label{task}
\begin{itemize}
\item Extensive changes to the TASK library have been made to implement the 
Version 2 task fixed-part.
See SUN/134.
\item The TASK library is now linked into tasks using the shareable image
DTASK\_IMAGE\_ADAM. References to the object library must be removed from
link commands.
\end{itemize}

\subsection{The Message System}
The Version 2 Message System based on MSP is now used.
Version 1 is not distributed.
Some changes have been made to support the new fixed part and rationalize the
system as follows:
\begin{itemize}
\item A `TRIGGER' facility has been added. (MESSYS\_ERR and MESSYS\_REPLY)
\item A new routine is provided to signal reschedule messages to the mainline
code. (MESSYS\_RESMSG)
\item A new routine is provided to signal an ASTINT event to the mainline
code.( MESSYS\_ASTMSG)
\item Multiple (up to four) different networks may be used. They are accessed
by using \verb%^^%, \verb%!!% or \verb%##% in place of the \verb%::% which 
causes MESSYS to route messages via ADAMNET. 
The alternatives access processes ADAMNET\_2, ADAMNET\_3 and ADAMNET\_4 
respectively.
\item MESSYSMSG.* is renamed MESSYS\_ERR.*
\end{itemize}

\subsection{The Parameter System (SUBPAR)}
\label{subpar}
Extensive changes have been made to the parameter system, mainly in the
interests of portability and improving error reporting to conform with SUN/104
and error reporting from HDS.
Exceptions to this at the moment are the the SUBPAR\_DEF routines.
\begin{itemize}
\item Calls to VMS run time library routines have been replaced as far as
possible - mainly by CHR routines.
\item Operations which are by their nature system-dependent have been
separated into a new library SUBPARZ which will need to be implemented for
different machines.
\item A further library SUBPARO has been created to hold routines which appear
only to be needed because they are referred to in the ADAMSHARE transfer
vector.
\item Type conversion for data read from or written to HDS objects is now
handled by HDS itself.
This will allow some conversions which failed before but could also give
slightly different results.
\item Better error reports are generated - where the user can be expected to 
respond, the messages will be prefixed with `SUBPAR:' 
For more obscure problems, the complete subroutine name will be given.
\item Interface files in their compiled form of are now run-length encoded.
This results in smaller files and faster startup times.
Existing compiled interface files will still be accepted but should be
re-compiled to gain the benefits.
\item SUBERR.* is renamed SUBPAR\_ERR.*
\item Additional error status values are defined to avoid SUBPAR setting status
values defined by other packages, and in preparation for future developments.
\end{itemize}

\subsection{The Parameter System (PARSECON)}
\label{parsecon}
\begin{itemize}
\item Calls to VMS run time library routines have been replaced as far as
possible - mainly by CHR routines.
\item The new, run-length encoded form of compiled interface file will be
produced by COMPIFL.
\item Error reports are prefixed with `PARSECON:'.
\item PARSERR.* is renamed PARSECON\_ERR.*
\item An additional status PARSE\_\_BIGVEC has been added to PARSECON\_ERR.
\item Stricter typing is imposed on values given in the interface file
(for DEFAULTS {\em etc.}). This may result in failures which did not occur
before, but is easily corrected.
\end{itemize}

\subsection{LEX}
Error reports are now generated. (LEX\_PARSE)

\subsection{MSG, ERR and EMS}
\label{ems}
\begin{itemize}
\item Numerous enhancements have been made to these systems.
New subroutines have been provided and the behaviour of others, particularly
the token setting routines, has been changed.
ERR\_LOAD and ERR\_ELOAD have additional arguments and
the way in which they and ERR\_REP treat the STATUS argument has also changed.
For full details, see EMS release notes and SUN/104.2.
\item EMS is now used extensively by ADAM system routines (notably HDS and
the parameter system.
This will result in much-improved error reporting {\em but could result in
extra warning messages if tasks have not obeyed the Starlink error 
conventions.}
In particular, a warning  message will be generated if ERR/EMS\_REP is 
called with STATUS OK or if some other facility routines are called with
bad status but no error reported.

If such message appear to come from the ADAM code, they should be reported
as bugs.
\item The EMG\_IMAGE\_ADAM (EMS) shareable image released as part of the 
Starlink EMS software item is now used by ADAM. 
Updates to EMS will therefore be included automatically in ADAM tasks. 
However, MSG and ERR still form part of the ADAMSHARE shareable image as they 
require the parameter system.
\end{itemize}

\subsection{HDS}
\begin{itemize}
\item The HDS\_IMAGE\_ADAM shareable image released as part of the Starlink HDS
software item is now used. Updates will therefore be included automatically
in ADAM tasks but files produced cannot be read by earlier versions of HDS
such as that incorporated into the ADAMSHARE shareable image for ADAM V1.
\item The DATPAR routines (DAT\_ASSOC) {\em etc.} still form part of the
ADAMSHARE shareable image as they require the parameter system.
\item The DAT\_ERx routines have been removed from the DATPAR library as they
are now included in the HDS kernel.
\item HDS now produces error reports and does checks which may result in 
changed behaviour of tasks which have not obeyed the Starlink error conventions
(See SUN/104).
\end{itemize}

\subsection{CHR}
\label{chr}
The CHR\_IMAGE\_ADAM shareable image released as part of the Starlink CHR
software item is now used. Updates will therefore be included automatically
in ADAM tasks.

\subsection{Other Procedures Etc.}
\begin{description}

\item[SYSLOGNAM.COM/.MINI]
The logical name DTASK\_IMAGE\_ADAM is now defined.

MODIFIED VERSIONS OF THESE FILES WILL NEED UPDATING.

\item[ADAMSTART.COM] is modified to:
\begin{itemize}
\item Display the latest ADAM version number.
\item Define the symbol LISTLOG to run the ICL log listing program.
\end{itemize}

\item[ADAMDEV.COM and SYSDEV.COM] Numerous changes have been made to:
\begin{itemize}
\item Use F\_DIR instead of D\_DIR.
\item Define symbols for the new task linking procedures.
\item Remove definition of DCAMLINK and CL\_MLINK.
\item Reflect the use by ADAM of separate Starlink software releases of HDS,
MSG/ERR/EMS and CHR.
\end{itemize}

\item[APPLOG.COM and LOGICAL.COM] Numerous changes have been made, mainly to 
reflect the use by ADAM of separate Starlink software releases of HDS, 
MSG/ERR/EMS and CHR.

\item[ADAMSHARE]
ADAMSHARE.COM and ADAMSHARE.MAR are altered to reflect:
\begin{itemize}
\item An increased minor id
\item Shareable images for HDS, EMS, MSP and CHR
\item The name changes to various directories and object (error message)
files.
\item VAXCRTL/LIB (shareable now included with EMS).
\end{itemize}

\item[LINKNOSHR.OPT]
The options file used for linking tasks `non-shareable' has been changed in
line with ADAMSHARE.COM

\item[Task Linking Procedures]
\label{links}
All task linking is now done by a common procedure, TASK\_LINK, called
via procedures with the familiar names and invocations:
\begin{itemize}
\item \begin{description}
\item[ALINK] Links an A-task (or C-task)
\item[MLINK] Links an A-task monolith.
\item[ICLMLINK] Links an ICL shareable monolith.
\item[ILINK] Links an I(Instrumentation)-task
\item[DLINK] Links an old-style D-task
\item[CDLINK] Links an old-style CD-task
\end{description}
For example:
\begin{quote} \begin{verbatim}
$ ALINK TEST /DEBUG
\end{verbatim} \end{quote}
Will link the A-task TEST in debug mode.
\item The `NOSHR' procedures are withdrawn.
To link a task with the ADAM object libraries rather than shared images, 
add a third parameter `*' to the appropriate link procedure invocation.
For example:
\begin{quote} \begin{verbatim}
$ ALINK TEST /NODEBUG *
\end{verbatim} \end{quote}
Notes:
\begin{enumerate}
\item The second parameter must be present. /NODEBUG may be used if nothing
else is required.
\item Object libraries will not be used for HDS, EMS or CHR.
\end{enumerate}
\item DCAMLINK and CL\_MLINK have been withdrawn.
\end{itemize}
\end{description}

\subsection{HDS}
\label{hds}
ADAM tasks are now linked with the ADAM version of the HDS shareable image
which forms part of the standard Starlink HDS release.

\subsection{ICL}
\begin{itemize}
\item ICL Version 1.5-6 is released - the login banner has been changed
appropriately. (LOGIN.ICL, ADAMLOGIN.ICL)
\item Modifications were required because of the new error reporting strategy
in HDS. (ICLMAIN, ICLEXT)
\item The Guide has been updated and released as SG/4. It has been removed
from the ADAM directories. (ICL.TEX)
\item See also Section \ref{sms} -- change to ICL\_DIR:SMSLINK and Section 
\ref{links} (ICLMLINK, MAIN.FOR)
\end{itemize}

\subsection{SMS}
\label{sms}
\begin{itemize}
\item The length of an expression allowed in the `from adamcl' option of a 
switch entry has been increased from 32 to 80 characters.
\item The procedure for linking ICLSMS has been modified to include the option
ISD\_MAX=200 to avoid a very large image being produced.
\end{itemize}

\subsection{Directory Structure}
\begin{itemize}
\item The AFIX and DFIX directories are deleted.
The DTASK library and task linking procedures {\em etc.} are now held in
DTASK\_DIR.
Those files in DTASK\_DIR which are required for task linking are copied into
a directory, FIX\_DIR, in the mini-release portion of the structure.
The link procedures are accessed via a logical name F\_DIR.
On obeying ADAM\_DEV, F\_DIR is set to FIX\_DIR; on obeying SYSDEV, F\_DIR is
set to DTASK\_DIR.
\item The directory containing MESSYS is now pointed at by logical name 
MESSYS\_DIR rather than MES\_DIR.
\end{itemize}

\subsection{Documentation}
\label{docs}
\begin{itemize}
\item SSN/45 and ARN/20 describe ADAM release 2.0
\item The following applicable Starlink documents have been released:
\begin{itemize}
\item SUN/134 ADAM -- Guide to Writing Instrumentation Tasks.
\item SUN/77 NBS -- The Noticeboard System (As Version Feb. 1990)
\item SG/5 ICL -- Interactive Command Language for ADAM (Revised version of
ICL -- The New ADAM Command Language Users Guide)
\end{itemize}
\item The following applicable Starlink documents have been updated.
\begin{itemize}
\item SSN/44 ADAM -- Installation Guide.
\item SUN/104 MSG and ERR -- Message and Error Reporting Systems.
\item SSN/4 EMS -- Error Message Service.
\item SUN/92 HDS -- Hierarchical Data System.
\end{itemize}
\item The following documents have been withdrawn:
\begin{itemize}
\item AED/1 How to write a D-task
\item AED/2 Fast monitoring and screen updates
\item AED/6 C Tasks with Actions -- C(d) Tasks
\item AED/9 Fast monitoring (noticeboard) and data transfer between tasks
\item AED/11 How to write a C-task
\item SSN/27 HDS -- Hierarchical Data System Design and Implementation
\item ICL -- The New ADAM Command Language Users Guide
\item NBS -- The Noticeboard System V2.3.1 (Version in NBS\_DIR)
\item MSP -- Message System primitive Routines (Version in MSP\_DIR)
\end{itemize}

\item The summaries, ADAM\_DOCS:0CONTENTS.LIS, FULLDOCS.LIS and NEWDOCS.LIS,
have been updated. 
\end{itemize}


\section{BUGS FIXED}

\subsection{MSP}
\begin{itemize}
\item A bug which caused ASTs to be occasionally re-enabled when they
shouldn't have been, has been fixed. (MSP\_SEND\_MESSAGE)
\item A bug which caused a failure to re-enable ASTs under the `deleted queue'
error condition, has been fixed. (MSP\_SEND\_MESSAGE)
\item A bug which caused problems with short task names, has been fixed.
(MSPFOR.FOR)
\end{itemize}

\subsection{LEX}
Some unused declarations have been removed. (LEX\_CMDSET)

\subsection{ANT}
In the event of a failure to forward a net GSOC start message, NET\_OUT\_TYPE
is now set to C\_NET\_GSOC\_END\_OUT. (ANT\_FORWARD\_START\_IN)

\subsection{PAR}
References to SAI\_PAR have been replace by SAE\_PAR.
(PAR\_GET0.GEN)

\subsection{ADAM\_DEV}
NOLOG is added to the DEFINE commands for ADAM\$\_INITIALIZED and
ADAM\$\_INITDONE.

\subsection{UFACE}
The maximum size of the executable image name is increase from 63 to 120
characters for ICL and SMS. (UFACE\_LOAD, UFACE\_LOADD)

\subsection{MAG/MIO}
A bug which caused corruption of the byte count returned on reading and writing
blocks greater than 32K bytes in length, has been corrected.
(MIO\_BREAD, MIO\_BWRIT)

\section{CLEANUP}
\begin{description}
\item[TRACE] The obsolete version of TRACE, which was temporarily retained in 
the ADAM directories after TRACE was issued as a separate Starlink software 
item, has been removed.
\item[MBLOGNAM.COM]
This procedure which was used for inter-job communication using the Version 1
message system, has been deleted. {\em This may require changes to system setup
procedures.}
\item[LIB\_DIR] The following sub-directories of LIB\_DIR have been removed.
\begin{itemize}
\item MESSYSV2 and NEWSHARE -- they are no longer relevant
\item BDKTEST -- an early architecture trial
\end{itemize}
\item[Documents] Source of documents which are now issued as Starlink 
documents has been removed from ICLDIR, NBS\_DIR and MSP\_DIR.
They may now be found in the Starlink DOCSDIR directory.
\end{description}

\section{PROPOSED DEVELOPMENTS}
\subsection{Parameter System}
It is intended that another ADAM release will be made in the near future to
include a number of improvements to the parameter system, notably:
\begin{itemize}
\item The ability to specify MIN or MAX as a parameter value, in which case
the system will take a specified minimum or maximum value.
\item The ability to respond to a parameter prompt with `\verb%\%', in which 
case the suggested value will automatically be taken for the current, and any 
remaining, parameters for which a prompt would be issued.
\item The ability to specify the null value `!' on the command line.
\item PAR routines with the functionality of the AIF routines which have been
used by several packages but never officially released for use.
\item Automatic creation of the GLOBAL parameter file if it does not exist.
\end{itemize}

\subsection{STRING}
The STRING library is being considered for removal. It seems appropriate that
its general purpose functions should be transferred to portable code in the
CHR library and other more specific functions be incorporated into other
ADAM system packages.
\end{document}
