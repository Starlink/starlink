\documentstyle[11pt]{article} 
\pagestyle{myheadings}

%------------------------------------------------------------------------------
\newcommand{\stardoccategory}  {ADAM Release Note}
\newcommand{\stardocinitials}  {ARN}
\newcommand{\stardocnumber}    {20-1.1}
\newcommand{\stardocauthors}   {A J Chipperfield}
\newcommand{\stardocdate}      {3 June 1992}
\newcommand{\stardoctitle}     {ADAM --- Release 2.0-1}
%------------------------------------------------------------------------------

\newcommand{\stardocname}{\stardocinitials /\stardocnumber}
\markright{\stardocname}
\setlength{\textwidth}{160mm}
\setlength{\textheight}{230mm}
\setlength{\topmargin}{-2mm}
\setlength{\oddsidemargin}{0mm}
\setlength{\evensidemargin}{0mm}
\setlength{\parindent}{0mm}
\setlength{\parskip}{\medskipamount}
\setlength{\unitlength}{1mm}

%------------------------------------------------------------------------------
% Add any \newcommand or \newenvironment commands here
%------------------------------------------------------------------------------

\font\tt=cmtt10 scaled 1095
\renewcommand{\_}{{\tt\char'137}}

\begin{document}
\thispagestyle{empty}
SCIENCE \& ENGINEERING RESEARCH COUNCIL \hfill \stardocname\\
RUTHERFORD APPLETON LABORATORY\\
{\large\bf Starlink Project\\}
{\large\bf \stardoccategory\ \stardocnumber}
\begin{flushright}
\stardocauthors\\
\stardocdate
\end{flushright}
\vspace{-4mm}
\rule{\textwidth}{0.5mm}
\vspace{5mm}
\begin{center}
{\Large\bf \stardoctitle}
\end{center}
\vspace{20mm}
\begin{center}
{\Large\bf Summary}
\end{center}
This is a `partial' release of ADAM -- its main features are:
\begin{itemize}
\item The use of separately released shareable images for the
graphics libraries -- thus allowing easier task linking and automatic inclusion
of updates into ADAM tasks.
\begin{itemize}
\item PGPLOT, SGS and GKS libraries will be searched automatically by ALINK,
MLINK and ICLMLINK but not by ILINK, DLINK or CDLINK.
\item ARGSLIB is no longer available as part of ADAM. Any tasks using it must
be re-linked in the next few months.
\end{itemize}
\item Improvements to error checking, reporting and recovery in interface file
parsing.
\item NBS available for application development when only the
mini-system is installed.
\item Change in the method of using the PAR\_PAR include file.
\item Bugs fixed in the parameter system, message system, ICL and DTASK
\end{itemize}

\newpage
%------------------------------------------------------------------------------
%  Add this part if you want a table of contents
  \setlength{\parskip}{0mm}
  \tableofcontents
  \setlength{\parskip}{\medskipamount}
  \markright{\stardocname}
%------------------------------------------------------------------------------

\section{INSTALLATION}
Full installation instructions are given in SSN/44 and the Starlink Software 
Change Notice.

The full system requires about 38000 blocks of disk storage and includes a
mini-system which can be extracted and put up separately. The mini-system
requires about 12000 blocks and allows tasks to be run and  developed.

\section{NEW FEATURES IN THIS RELEASE}

\subsection{Graphics Shareable Images}
\subsubsection{SGS, GKS and GNS}
\label{graphs}
ADAM tasks which use SGS, GKS or GNS will now link with the relevant
{\em lib}\_IMAGE\_ADAM shareable images, which are part of the separate
Starlink Software Items. These also now include the {\em lib}PAR routines
which connect the graphics libraries to the ADAM parameter system.
The SGS and GKS directories have been removed from the ADAM release.

This means that the separate items must be installed if they are used.


\subsubsection{Linking `Analysis' Tasks}
On being re-linked, `Analysis' tasks will link directly to the separate 
shareable images.
On standard systems, ALINK, MLINK and ICLMLINK will do this automatically via
the options file and shareable image library, STAR\_LINK\_ADAM;
{\em if you use a non-standard system, it must be altered accordingly if you 
are using graphics.} The advantage of this scheme is that the shareable image
library will be updated when the separate software items are updated and will
not depend on an ADAM release.

\subsubsection{Linkng `Instrumentation' Tasks}
For I-, D- and CD-tasks, the graphics libraries must now be
included separately in the linker inputs, {\em e.g.}:
\begin{quote} \begin{verbatim}
$ ILINK task,SGS_LINK_ADAM/OPT
\end{verbatim} \end{quote}
or
\begin{quote} \begin{verbatim}
$ DLINK task,STAR_LINK_ADAM/OPT
\end{verbatim} \end{quote}

Logical names required for application development will be defined at standard
sites via the procedure pointed at by logical name STAR\_DEV\_ADAM when 
ADAM\_DEV (ADAMDEV) is issued.

\subsection{SGS Test Program}
The source of the SGS test program SGSX1 has been moved to
[ADAM.ADAMEXE.SOURCE]. The build and copy procedures have been altered 
accordingly.

\subsection{ADAMGRAPH7}
The ADAMGRAPH7 shareable image now contains only DIAGRAM and dummy transfer
vectors which will allow existing linked tasks to continue to work.\\
(ADAMGRAPH7.COM/.MAR)

\subsubsection{ARGS}
\label{args}
The ARGS library is no longer available automatically to ADAM tasks. Existing
linked tasks which use it will continue to work but must be re-linked, 
specifying the ARGS library explicitly.
\begin{quote} \begin{verbatim}
$ ALINK task,ARGSLIB_DIR:ARGS/INCLUDE=(BDTAT$DATA)/LIBR
\end{verbatim} \end{quote}
It is proposed to remove ARGSLIB completely from the ADAM system in a few
months time.
Tasks will then cease to work unless they have been re-linked.\\
(ADAMGRAPH7.COM/.MAR)

\subsection{The Parameter System (PARSECON)}
\label{parsecon}
\begin{itemize}
\item The library is now held in CMS and has been completely rebuilt from CMS
using MMS.
\item PARSECON now conforms to the error reporting conventions in SUN/104 and 
uses EMS throughout for error reporting.\\
(Numerous routines changed)
\item PARSECON now checks, at {\em endinterface}, that the parameter {\em
positions} form a contiguous set. An additional error status, PARSECON\_\_NCPOS,
is defined and will be set in the event of a failure.\\
(PARSECON\_FACEND, \_SETFACE, \_SETPOS, \_SETFACE, PARSECON\_ERR,\\
PARSECON\-\_CMN, PARSECON4\_CMN,)
\item PARSECON will now recover sensibly from a missing {\em endinterface}
field.\\
(PARSECON\_TABINIT)
\item Case will be retained for default values other than quoted strings, and 
for help library specifiers.
This is required for the Unix implementation.
(PARSECON\_ARRCHAR, \_TOKTYP, \_READIFL)
\item Some modifications to the allowable characters in names have been made
to allow \verb%~% and \verb%/% in names, again for Unix purposes.\\
(PARSECON\_ARRCHAR, \_DECVAL)
\item The names of the routines PARSECON\_{\em x}COORDS have been changed to 
PARSECON\-\_{\em x}CRDS for portability reasons.\\
(Also changed PARSECON\_READIFL)
\item PARSECON\_ERFL has been changed to use WRITE with a FORMAT, and the
FORMAT includes an initial 1X for VMS. This is removed automatically in the 
process of building the Unix release, thus ensuring that no initial space is
printed.
\item An additional argument for SUBPAR\_CREATIFC specifies the STATUS ('NEW'
or 'UNKNOWN') to be specified in the OPEN statement for the .IFC file.
This is required because some Unix flavours report an error if an attempt is
made to create a file which already exists.\\
(Also changed COMPIFL)
\item COMPIFL has been modified for portability and improved error reporting.
Versions for VMS and Unix differ as follows:
\begin{center}
\begin{tabular}{lcc}
& {\bf VMS} & {\bf Unix} \\
File extensions & Upper case & Lower case \\
OPEN STATUS & \verb%'NEW'% & \verb%'UNKNOWN'% \\
Prompt FORMAT & \verb%( 1X, A, $ )% & \verb%( A, $ )%
\end{tabular}
\end{center}
\end{itemize}

\subsection{The Parameter System (SUBPAR)}
\label{subp}
\begin{itemize}
\item The library is now held in CMS and has been completely rebuilt from CMS 
using MMS.
\item The SUBPAR\_DEF1x routines are now non-generic and store values in the 
type supplied. Values are converted to the type of the parameter when they are
used.
If there are fewer than six values they will be stored in the parameter system
common blocks.
If new dynamic default values are supplied in the same type and number as
previously, the same common block storage will be re-used; otherwise an HDS
component of the task's parameter file will be created with the required
dimensionality and size and the value(s) will be stored there.\\
(SUBPAR\_DEF1x, \_DEACT)
\item SUBPAR\_DEF1x routines will now assume a scalar default if NVAL of 0 is
given. Previously specifying NVAL = 0 would have had unpredictable effects.
This allows a scalar HDS object to be created if a call to PAR/SUBPAR\_DEF0x
requires it.\\
(SUBPAR\_DEF0x, SUBPAR\_DEF1x)
\item HDS is now used to do type conversion when N-dimensional dynamic defaults
are set -- this may result in changes to the values stored. 
In particular the integer part, rather than the nearest integer, is taken in 
REAL to INTEGER conversion.\\
(SUBPAR\_DEFNx)
\item When a primitive object in the GLOBAL file is used, by association, as 
the value for a parameter, the relevant component in the task's parameter file
is created by copying the GLOBAL component using DAT\_COPY.
This procedure has been adopted to avoid the need for mapping, with its
requirement for non-portable features and complications with type \_CHAR --
it may result in a parameter file component with a type different from the
declared type of the parameter but HDS will take care of any required
conversions.\\
(SUBPAR\_HDSASS)
\item When object names are set repeatedly as dynamic defaults, the same common
block storage for the name will now be re-used.\\
(SUBPAR\_DATDEF)
\item New routine SUBPAR\_PFER has been added to the SUBPARZ library to
produce a system dependent message in the even of being unable to open the
parameter file.\\
(SUBPAR\_ACTIV, \_ACTDCL, \_ACTSHR)
\item New routine SUBPAR\_ADMUS has been added to the SUBPARZ library to return
the string represented by ADAM\_USER:.\\
(SUBPAR\_TSKNM, \_DEACT, \_HDSASS, \_VALASS)
\item The parameter system will now create the GLOBAL parameter file if it does
not exist.\\ (SUBPAR\_DEACT)
\item SUBPAR no longer sets PARSECON error values. An additional error value,
SUBPAR\-\_\_NOACT, is defined.\\
(SUBPAR\_DATDEF, \_FINDACT, \_FINDKEY, SUBPAR\_ERR)
\item Unused variables {\em etc.} have been cleaned up.\\
(SUBPAR\_TYPE, \_SELHELP, \_LOADIFC, \_SHRNM)
\end{itemize}

\subsection{The Parameter System (PAR)}
\label{par}
There has been a change to the use of the PAR\_PAR include file. 
Formerly this was included automatically if SAE\_PAR was included in the code 
but the approved method now is to include PAR\_PAR explicitly.
\begin{quote} \begin{verbatim}
INCLUDE 'SAE_PAR'
INCLUDE 'PAR_PAR'
\end{verbatim} \end{quote}
This makes PAR consistent with other Starlink libraries and is necessary on 
Unix implementations.
In order that existing code will continue to work on VMS and to enable common
code to be used for VMS or Unix, a scheme has been set up similar to that used 
for DAT\_PAR. On VMS, SAE\_PAR will \verb%INCLUDE 'PAR_FIX'% which actually
defines the PAR symbolic constants; PAR\_PAR will point to an empty file.
Then, whether or not the code has an explicit \verb%INCLUDE 'PAR_PAR'%, it will
correctly include the requisite definitions.

\subsection{NBS}
\label{nbs}
Elements of NBS which are required for application development, together with
the NBS shareable image have been brought into a newly created directory,
[ADAM.RELEASE.NBS]. This allows NBS applications to be developed when only the
mini-system is installed.

System logical names NBS\_DIR and NBS\_SHARE are defined by SYSLOGNAM.COM at
system startup and process logical names NBS\_PAR and NBS\_ERR are defined by
logging in for ADAM application development (ADAMDEV.COM).

\subsection{MESSYS}
The values MESSYS\_\_MXPATH and MESSYS\_\_MXTRANS, which specify the maximum
number of concurrently active paths and transactions, have been increased from
16 to 32.

\subsection{Other Procedures etc.}
\begin{description}

\item[ADAMSTART.COM] Changes have been made to:
\begin{itemize}
\item Display the latest ADAM version number.
\item Set JOB logical name ADAM\$\_VERSION to `ADAM\_V2.0-1'.
This will be updated with each new release and can be used by procedures
to determine which version of ADAM is in use.
\end{itemize}
\item[SYSLOGNAM.COM (SYSLOGNAM.MINI)] Now define NBS\_DIR 
and NBS\_SHR. 
{\em If you have modified versions of these files, they should be
updated in line with the new versions.}
\item[ADAMDEV.COM (APPLOG.COM)] Changes have been made to:
\begin{itemize}
\item Define PAR\_FIX (see Section \ref{par}).
\item Define  NBS\_PAR and NBS\_ERR.
\item Reflect the renamed files for SUBPAR\_PAR and LEX\_ERR.
\item Define SGS and GKS logical names only via STAR\_DEV\_ADAM.
\end{itemize}

\item[SYSDEV.COM (LOGICAL.COM, DIR.COM)] Changes have been made to:
\begin{itemize} 
\item Set logical names for SUBPAR, PARSECON and  LEX via procedures
LOGICAL\-.COM in the facility directory.
\item Remove definitions for SGSPAR and GKSPAR which are now completely within
the separate SGS/GKS Starlink software item.
\end{itemize}

\item[STAR\_LINK\_ADAM, STAR\_DEV\_ADAM] These are not actually part of the 
ADAM release, but for correct operation of the revised ADAM graphics system,
SGS, GKS and GNS are added to these files. PGPLOT is also added.

\item[ADAMSHARE]
ADAMSHARE.COM and ADAMSHARE.MAR are altered to reflect:
\begin{itemize}
\item An increased minor id
\item The name change to LEX\_ERR
\item An additional common block for PARSECON.
\item Renamed PARSECON\_xCOORDS routines.
\end{itemize}

\item[LINKNOSHR.OPT]
The options file used for linking tasks `non-shareable' has been changed in
line with ADAMSHARE.COM and ADAMGRAPH7.COM.

\item[GOD.PRC]
The ADAMCL login procedure has been altered to remove references to packages
which no longer work.
\end{description}

\subsection{ICL}
\begin{itemize}
\item ICL Version 1.5-7 is released - the login banner has been changed
appropriately.\\
(LOGIN.ICL, ADAMLOGIN.ICL)
\item NBS\_SHR.EXE is removed from ICLDIR (see Section \ref{nbs}).
\item DEVLOGNAM.COM no longer defines NBS logical names.
\item AAOLIB has been updated and the corresponding text library added.
\end{itemize}

\subsection{SMS}
Now uses the routine ADAM\_ACKNOW, rather than ADAM\_REPLY to reply to
parameter requests.\\
(SMS\_ASKPARAM, \_GETUSRLIN)

\subsection{Directory Structure}
\begin{itemize}
\item The SGS and GKS directories are deleted from [ADAM.LIB] (see Section 
\ref{graphs}).
\item A new directory, NBS, has been created in [ADAM.RELEASE] (see Section
\ref{nbs}).
\end{itemize}

\subsection{Documentation}
\label{docs}
\begin{itemize}
\item SSN/45 and ARN/21 describe ADAM release 2.0-1.
\item The ADAM\_CHANGES help library introduced with V2.0 has been updated 
with the information contained in this document. The information for V2.0 is
retained in the library.
The source files are kept in DOCS\_DIR and the library in ADAM\_DOCS.
\item The following applicable Starlink document has been released since the
last ADAM release:
\begin{itemize}
\item SUN/144 ADAM -- Unix Version.
\end{itemize}
\item The following applicable Starlink documents have been updated since the
last ADAM release.
\begin{itemize}
\item SSN/44 ADAM -- Installation Guide.
\item SSN/64 ADAM -- VAX Organization of Applications Packages.
\item SUN/134 ADAM -- Guide to Writing Instrumentation Tasks.
\end{itemize}
\item The summaries, 0CONTENTS.LIS, FULLDOCS.LIS and NEWDOCS.LIS in
ADAM\-\_DOCS, have been updated. 
\item Most of the early release notes have been removed from ADAM\_DOCS.
They are retained in [ADAM.DOCS].
\end{itemize}


\section{BUGS FIXED}

\subsection{SUBPAR}
\begin{itemize}
\item The parameter system will now correctly handle a non-integer numeric 
character string being `put' as the value for an INTEGER parameter.
The integer part will be used. (This is actually a change from the behaviour
before Version 2.0, when the nearest integer was used.) 
This fixes a problem detected in the IRCAM SETQUAD command.\\
(SUBPAR\_PUT0C)
\item The error context will now be released correctly after a name has been
obtained as a parameter value.\\
(SUBPAR\_GETNAME)
\end{itemize}

\subsection{PARSECON}
\begin{itemize}
\item A bug which caused overwriting of the common blocks when non-contiguous
parameter position numbers were specified, has been corrected.\\
(For details see Section \ref{parsecon})
\item A spurious error message on COMPIFL completion has been prevented.\\
(COMPIFL)
\end{itemize}

\subsection{DTASK}
\begin{itemize}
\item Correctly handle return status from CACT.\\
(DTASK\_APPLIC\_CD)
\item ACTCODE has been added to the action details stored.\\
(DTASK\_ASSOC, \_ADDLST)
\end{itemize}

\subsection{ICL}
\begin{itemize}
\item ICL now reports if it fails to start up MESSYS.\\
(ICL.PAS)
\item Spurious error reports occurring on ({\em variable}\/) constructs, have 
been removed.\\
(ICLEXT, ICLDEF)
\end{itemize}

\subsection{MESSYS}
`Delete the acknowledgement queue' now uses ISTAT not STATUS.\\
(MESSYS\_CALL\_OUT)

\subsection{MSP}
TASK\_LIST is now correctly initialised.\\
(MSP\_INIT)

\subsection{AZUSS}
A bug in AZ\$SNDAST which caused the AST quota to be exceeded in the target
process has been corrected.\\
(AZ\$SNDAST)

\section{PROPOSED DEVELOPMENTS}
\subsection{Parameter System}
It is intended that another ADAM release will be made in the near future to
include a number of improvements to the parameter system, notably:
\begin{itemize}
\item The ability to specify MIN or MAX as a parameter value, in which case
the system will take a specified minimum or maximum value.
\item The ability to respond to a parameter prompt with `\verb%\%', in which 
case the suggested value will automatically be taken for the current, and any 
remaining, unset parameters.
\item The ability to specify the null value `!' on the command line.
\item PAR routines with the functionality of the AIF routines which have been
used by several packages but never officially released for use.
\item The use of a portable help system.
\end{itemize}

\subsection{STRING}
The STRING library is being considered for removal. It seems appropriate that
its general purpose functions should be transferred to portable code in the
CHR library and other more specific functions be incorporated into other
ADAM system packages.
\end{document}
