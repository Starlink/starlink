\documentstyle{article}
\pagestyle{myheadings}

%------------------------------------------------------------------------------
\newcommand{\stardoccategory}  {ADAM Programmer's Note}
\newcommand{\stardocinitials}  {APN}
\newcommand{\stardocnumber}    {3.3}
\newcommand{\stardocauthors}   {A J Chipperfield}
\newcommand{\stardocdate}      {27 February 1990}
\newcommand{\stardoctitle}     {SGS --- ADAM Programmer's Guide}
%------------------------------------------------------------------------------

\newcommand{\stardocname}{\stardocinitials /\stardocnumber}
\markright{\stardocname}
\setlength{\textwidth}{160mm}
\setlength{\textheight}{240mm}
\setlength{\topmargin}{-5mm}
\setlength{\oddsidemargin}{0mm}
\setlength{\evensidemargin}{0mm}
\setlength{\parindent}{0mm}
\setlength{\parskip}{\medskipamount}
\setlength{\unitlength}{1mm}

%+                              M A N . S T Y
%
%  Module name:
%    MAN.STY
%
%  Function:
%    Starlink definitions for \LaTeX\ macros used in MAN output
%
%  Description:
%    As much as possible of the output from the MAN automatic manual generator
%    uses calls to user-alterable macros rather than direct calls to built-in
%    \LaTeX\ macros. This file is a version of the MAN default definitions for
%    these macros modified for Starlink preferences.
%
%  Language:
%    \LaTeX
%
%  Support:
%    William Lupton, {AAO}
%    Alan Chipperfield (RAL)
%-
%  History:
%    16-Nov-88 - WFL - Add definitions to permit hyphenation to work on
%		 words containing special characters and in teletype fonts.
%    27-Feb-89 - AJC - Redefine \manroutine
%                      Added \manheadstyle
%                      Switch order of argument descriptors
%    07-Mar-89 - AJC - Narrower box for parameter description
%                      Remove Intro section and other unused bits
%
% permit hyphenation when in teletype font (support 9,10,11,12 point only -
% could extend), define lccodes for special characters so that the hyphen-
% ation algorithm is not switched off. Define underscore character to be
% explicit underscore rather than lots of kerns etc.

\hyphenchar\nintt=`-\hyphenchar\tentt=`-\hyphenchar\elvtt=`-\hyphenchar\twltt=`-

\lccode`_=`_\lccode`$=`$

\renewcommand{\_}{{\tt\char'137}}
%+                      M A N _ S U M M A R Y
%
%  Section name:
%    MAN_SUMMARY
%
%  Function:
%    Macros used in the .TEX_SUMMARY file
%
%  Description:
%    There is a command to introduce a new section (mansection) and a list-like
%    environment (mansectionroutines) that handles the list of routines in the
%    current section. In addition a mansectionitem command can be used instead
%    of the item command to introduce a new routine in the current section.
%-

\newcommand {\mansection}[2]{\subsection{#1 --- #2}}

\newenvironment {mansectionroutines}{\begin{description}\begin{description}}%
{\end{description}\end{description}}

\newcommand {\mansectionitem}[1]{\item [#1:] \mbox{}}

%+                      M A N _ D E S C R
%
%  Section name:
%    MAN_DESCR
%
%  Function:
%    Macros used in the .TEX_DESCR file
%
%  Description:
%    There is a command to introduce a new routine (manroutine) and a list-like
%    environment (manroutinedescription) that handles the list of paragraphs
%    describing the current routine. In addition a manroutineitem command can
%    be used instead of the item command to introduce a new paragraph for the
%    current routine.
%
%    Two-column tables (the ones that can occur anywhere and which are
%    triggered by "=>" as the second token on a line) are bracketed by a
%    new environment (mantwocolumntable). Other sorts of table are introduced
%    by relevant  environments (manparametertable, manfunctiontable and
%    manvaluetable). The definitions of these environments call various other
%    user-alterable commands, thus allowing considerable user control over such
%    tables... (to be filled in when the commands have been written)
%-

\newcommand {\manrule}{\rule{\textwidth}{0.5mm}}

%\newcommand {\manroutine}[2]{\subsection{#1 --- #2}}
\newlength{\speccaption}
\newlength{\specname}
\newcommand{\manroutine}[2]{\goodbreak
                          \rule{\textwidth}{0.5mm}  % draw thick line
                          \settowidth{\specname}{{\Large {\bf #1}}}
                        % left and right box width is text width plus gap
                          \addtolength{\specname}{4ex}
                        % caption width is width of page less the two names
                        % less than empirical fudge factor
                          \setlength{\speccaption}{\textwidth}
                          \addtolength{\speccaption}{-2.0\specname}
                          \addtolength{\speccaption}{-4.45pt}
                        % move text up the page because \flushleft environ-
                        % ment creates a paragraph
                          \vspace{-7mm}
                          \newline
                          \parbox[t]{\specname}{\flushleft{\Large {\bf #1}}}
                          \parbox[t]{\speccaption}{\flushleft{\Large #2}}
                          \parbox[t]{\specname}{\flushright{\Large {\bf #1}}}
                          }

\newenvironment {manroutinedescription}{\begin{description}}{\end{description}}

\newcommand {\manroutineitem}[2]{\item [#1:] #2\mbox{}}


% parameter tables

\newcommand {\manparametercols}{lllp{75mm}}

\newcommand {\manparameterorder}[3]{#2 & #3 & #1 &}

\newcommand {\manparametertop}{}

\newcommand {\manparameterblank}{\gdef\manparameterzhl{}\gdef\manparameterzss{}}

\newcommand {\manparameterbottom}{}

\newenvironment {manparametertable}{\gdef\manparameterzss{}%
\gdef\manparameterzhl{}\hspace*{\fill}\vspace*{-\partopsep}\begin{trivlist}%
\item[]\begin{tabular}{\manparametercols}\manparametertop}{\manparameterbottom%
\end{tabular}\end{trivlist}}

\newcommand {\manparameterentry}[3]{\manparameterzss\gdef\manparameterzss{\\}%
\gdef\manparameterzhl{\hline}\manparameterorder{#1}{#2}{#3}}


% list environments

\newenvironment {manenumerate}{\begin{enumerate}}{\end{enumerate}}

\newcommand {\manenumerateitem}[1]{\item [#1]}

\newenvironment {manitemize}{\begin{itemize}}{\end{itemize}}

\newcommand {\manitemizeitem}{\item}

\newenvironment {mandescription}{\begin{description}\begin{description}}%
{\end{description}\end{description}}

\newcommand {\mandescriptionitem}[1]{\item [#1]}

\newcommand {\mantt}{\tt}

% manheadstyle for Starlink
\newcommand {\manheadstyle}{}

\catcode`\$=12 \catcode`\_=12
\font\tt=CMTT10 scaled 1095

\begin{document}
\thispagestyle{empty}
SCIENCE \& ENGINEERING RESEARCH COUNCIL \hfill \stardocname\\
RUTHERFORD APPLETON LABORATORY\\
{\large\bf Starlink Project\\}
{\large\bf \stardoccategory\ \stardocnumber}
\begin{flushright}
\stardocauthors\\
\stardocdate
\end{flushright}
\vspace{-4mm}
\rule{\textwidth}{0.5mm}
\vspace{5mm}
\begin{center}
{\Large\bf \stardoctitle}
\end{center}
\vspace{5mm}

%------------------------------------------------------------------------------
%  Add this part if you want a table of contents
%  \setlength{\parskip}{0mm}
%  \tableofcontents
%  \setlength{\parskip}{\medskipamount}
%  \markright{\stardocname}
%------------------------------------------------------------------------------

\section{INTRODUCTION}
This guide describes the use of the Simple Graphics System (SGS)
in ADAM application programs.
It is expected that the reader is familiar with programming for
ADAM and with SGS.

\section{APPLICATION}
First ask yourself `Should I be using SGS?'.
The answer to this question should probably be `No' because higher-level
graphics packages  NCAR (see SUN/88) or DIAGRAM (see SUN/54) should be
adequate for most application programs wanting to do graphics.
In addition, AGI, the Applications Graphics Interface (see SUN/48), now
provides a more versatile way of handling graphical information.
There are some applications, however, for which the high level packages are
not appropriate and SGS or even GKS should be considered.

SGS (The Simple Graphics System) in its stand-alone use is fully described
in SUN/85.
This guide will therefore only deal with issues relating to use of SGS in
ADAM application programs.

All SGS routines may be used in ADAM applications with the exception of the
following routines, which must {\em never} be called in ADAM programs :
\begin{itemize}
\item SGS\_INIT
\item SGS\_OPEN
\item SGS\_OPNWK
\item SGS\_CLOSE
\end{itemize}
The functions of SGS\_INIT, SGS\_OPEN and SGS\_OPNWK are performed by the
Environment routine SGS\_ASSOC.
The Environment routines SGS\_ANNUL and SGS\_CANCL dispose of resources used in
SGS programs.
SGS\_CLOSE will be called by the environment-level SGS de-activation routine
SGS\_DEACT.

\section{THE SGS PARAMETER ROUTINES}
As with all other packages in the Software Environment, the only access to
objects outside of application programs is via Program Parameters.
SGS has three environment-level subroutines (the SGS `parameter' or SGSPAR
routines) which provide the necessary interaction with the outside world.
They are :

\begin{center}
\begin{tabular}{||l|l||} \hline
Subroutine & Function \\ \hline
SGS\_ASSOC  & Associate a graphics workstation with a parameter and open it.\\
SGS\_ANNUL  & Close a graphics workstation without cancelling the parameter.\\
SGS\_CANCL  & Close a graphics workstation and cancel the parameter.\\
SGS\_DEACT  & De-activate ADAM SGS.\\ \hline
\end{tabular}
\end{center}

Here is a skeletal example of a program using SGS :
\begin{quote}
\begin{verbatim}
      SUBROUTINE SGSTEST( STATUS )
      ..
      INTEGER STATUS
      INTEGER ZONE

      ..
*    Obtain Zone on a graphics workstation
      CALL SGS_ASSOC('DEVICE', 'WRITE', ZONE, STATUS)
      IF (STATUS .EQ. SAI__OK) THEN

         ..

*       Perform graphics operations on the zone

         ..

*       Release Zone
         CALL SGS_ANNUL(ZONE, STATUS)
      ENDIF

      ..

      CALL SGS_DEACT (STATUS)

\end{verbatim}
\end{quote}

SGS\_ASSOC should be the first SGS routine to be called in the application.
It obtains (via the parameter system) the name of the graphics workstation
to be used, creates an initial SGS zone on the workstation, and returns an
SGS zone identifier which can be used in subsequent SGS subroutine calls.

The first argument of SGS\_ASSOC is a Program Parameter (which should be defined
to be a graphics device parameter in the Interface Module for the application).
The parameter should be given a name as a value and the name should be a
workstation name as defined by GNS (see SUN/57).

The second argument is the access mode required. This can be one of :
\begin{description}
\item[READ]   The application is only going to `read' from the workstation
({\em i.e.}\ perform cursor or similar operations).
Screens will not be cleared when vdu workstations are opened.
\item[WRITE]  The application is going to `write' on the workstation,
{\em i.e.}\ actually do some graphics.
Workstations are completely initialized when they are opened.
\item[UPDATE] The application will modify the graphics on the workstation.
Screens will not be cleared when vdu workstations are opened.
\end{description}

Note that the facility to prevent screen clearing is specific to RAL GKS and
is not implemented for all workstations.

The third argument is the SGS zone identifier returned to the application.

The fourth argument is the usual status value. It follows the ADAM error
strategy as described in SG/4.

When the application has finished using the workstation, it should be closed
using SGS\_CANCL unless it is required to keep the parameter active
(to update a global parameter for example), in which case SGS\_ANNUL
should be used.

When the application has finished using SGS, it should be de-activated by
calling SGS_DEACT.

\section{INTERFACE FILE}
An simple interface file for the above example would be:
\begin{quote}
\begin{verbatim}
## SGSTEST

interface SGSTEST

   parameter	DEVICE
      ptype		'DEVICE'
      position		1
      type		'GRAPHICS'
      access		'read'
      vpath		'prompt'
      help		'name of workstation to be used'
      default           PERICOM_MG
   endparameter

endinterface
\end{verbatim}
\end{quote}

\section{ERROR HANDLING}
The SGS parameter routines conform to the Starlink error reporting and
inherited status strategy described in SUN/104.
Error reports will generally be of the form:
\begin{quote}
\begin{verbatim}
ROUTINE_NAME: Text
\end{verbatim}
\end{quote}
but where there routine name is not useful, it will be omitted.

If it is required to test for particular status values, symbolic names should
be used.
They can be defined by including, in the application,  the statements:
\begin{quote}
\begin{verbatim}
INCLUDE 'SAE_PAR'   ! To define SAI__OK etc.
INCLUDE 'SGS_ERR'   ! To define the SGS__ error values
\end{verbatim}
\end{quote}
The names are listed in Section \ref{errs}.

Note that a substitute SGS error handling routine is linked with ADAM
applications.
This will result in conforming error reports and ADAM status values being
returned, even from SGS stand-alone routines.

\section{COMPILING AND LINKING}
The normal ADAM compiling and linking procedures will make both the SGSPAR
and SGS stand-alone routines available.
For example:
\begin{quote}
\begin{verbatim}
$ ADAMSTART
$ ADAMDEV
$ ALINK program
\end{verbatim}
\end{quote}

\section{ADDITIONAL MATERIAL}
\begin{itemize}
\item Starlink Guide 4 : {\it ADAM --- The Starlink Software Environment}
\item Starlink User Note 57 : {\it GNS --- Graphics Workstation Name Service}
\item Starlink User Note 85 : {\it SGS --- Simple Graphics System}
\item Starlink User Note 48 : {\it AGI --- Applications Graphics Interface}
\item Starlink User Note 54 : {\it DIAGRAM --- A High Level Graphics Package}
\item Starlink User Note 88 : {\it NCAR --- Graphics Utilities}
\item Starlink User Note 104 : {\it MSG and ERR --- Message and Error Reporting
Systems}
\end{itemize}

\section{SGS STATUS VALUES}
\label{errs}
The following status values may be returned by the SGS parameter
routines.

\begin{tabular}{ll}
SGS\_\_BADWK     & Bad Workstation name \\
SGS\_\_BADWT     & Bad Workstation type \\
SGS\_\_GKSER     & GKS Error \\
SGS\_\_GNSER     & Error returned by GNS routine \\
SGS\_\_ILLAC     & Illegal access mode \\
SGS\_\_INQER     & Error returned by GKS inquiry routine \\
SGS\_\_INVZN     & Invalid zone id \\
SGS\_\_ISACT     & Parameter currently active \\
SGS\_\_PENER     & SGS pen does not exist \\
SGS\_\_SGSER     & Unspecified SGS error \\
SGS\_\_SURFN     & Display surface is not empty \\
SGS\_\_SVCRP     & Save data is corrupt \\
SGS\_\_TOOZD     & No more available zone descriptors \\
SGS\_\_UNKPA     & Parameter not found \\
SGS\_\_UNOPN     & Workstation could not be opened \\
SGS\_\_WRKEX     & Too many Workstations \\
SGS\_\_ZEREX     & Zero Extent \\
SGS\_\_ZNOUT     & New zone not inside current zone \\
SGS\_\_ZNZEX     & Too many Zones \\
SGS\_\_ZONNB     & Specified zone is not a base zone \\
SGS\_\_ZONNF     & Specified zone does not exist \\
SGS\_\_ZONTB     & Requested zone bigger than current zone
\end{tabular}





\newpage
\section{ROUTINE DESCRIPTIONS}
\manroutine {{\manheadstyle{SGS\_ANNUL}}}{ Close a graphics workstation without %
cancelling the parameter}
\begin{manroutinedescription}
\manroutineitem {Description }{}
     De-activate and close the graphics workstation whose base zone
     ({\mantt{ZONE}}) was obtained using {\mantt{SGS\_ASSOC}}, and annul
     the zone identifier.
     Do not cancel the associated parameter.
\manroutineitem {Invocation }{}
     {\mantt{CALL}} {\mantt{SGS\_ANNUL}} ( {\mantt{ZONE}}, {\mantt{STATUS}} )
\manroutineitem {Arguments }{}
\begin{manparametertable}
\manparameterentry {Modified}{{\mantt{ZONE}} }{{\mantt{INTEGER}} }
           A variable containing the base zone identifier.
\manparameterentry {Modified}{{\mantt{STATUS}} }{{\mantt{INTEGER}} }
           Variable to contain the status. This routine is executed
           regardless of the import value of status.
           If its input value is not {\mantt{SAI\_\_OK}} then it is left %
unchanged
           by this routine, even if it fails to complete.   If its
           input value is {\mantt{SAI\_\_OK}} and this routine fails, then the
           value is changed to an appropriate error number.
\end{manparametertable}
\manroutineitem {Authors }{}
     Sid Wright. ({\mantt{UCL}}::{\mantt{SLW}})
\end{manroutinedescription}
\manroutine {{\manheadstyle{SGS\_ASSOC}}}{ Associate a graphics workstation with a %
parameter, and open it.}
\begin{manroutinedescription}
\manroutineitem {Description }{}
     Associate a graphics workstation with the specified Graphics
     Device Parameter and return an {\mantt{SGS}} zone identifier to reference
     the base zone of the workstation.
\manroutineitem {Invocation }{}
     {\mantt{CALL}} {\mantt{SGS\_ASSOC}} ( {\mantt{PNAME}}, {\mantt{MODE}}; {%
\mantt{ZONE}}, {\mantt{STATUS}})
\manroutineitem {Arguments }{}
\begin{manparametertable}
\manparameterentry {Given}{{\mantt{PNAME}} }{{\mantt{CHARACTER*}}({\mantt{*}}) }

           Expression specifying the name of a graphics parameter.
\manparameterentry {Given}{{\mantt{MODE}} }{{\mantt{CHARACTER*}}({\mantt{*}}) }
           Expression specifying the access mode.
           If it is '{\mantt{READ}}' or '{\mantt{UPDATE}}', a vdu workstation %
screen
           will not be cleared when the workstation is opened.
           (Note that this facility is not implemented on all
workstations.)
           If it is '{\mantt{WRITE}}', the screen will be cleared as usual.
\manparameterentry {Returned}{{\mantt{ZONE}} }{{\mantt{INTEGER}} }
           A Variable to contain the Zone identifier.
\manparameterentry {Modified}{{\mantt{STATUS}} }{{\mantt{INTEGER}} }
           Variable holding the status value.   If this variable
           is not {\mantt{SAI\_\_OK}} on input, then the routine will return
           without action.   If the routine fails to complete,
           this variable will be set to an appropriate error
           number.
\end{manparametertable}
\manroutineitem {Authors }{}
     Sid Wright.  ({\mantt{UCL}}::{\mantt{SLW}})
     A J Chipperfield.  ({\mantt{RAL}}::{\mantt{AJC}})
\end{manroutinedescription}
\manroutine {{\manheadstyle{SGS\_CANCL}}}{ Close a graphics workstation and cancel %
the parameter.}
\begin{manroutinedescription}
\manroutineitem {Description }{}
     De-activate and close the graphics workstation associated with
     the specified graphics device parameter and cancel the parameter.
     The workstation must have been opened using {\mantt{SGS\_ASSOC}}.
\manroutineitem {Invocation }{}
     {\mantt{CALL}} {\mantt{SGS\_CANCL}} ( {\mantt{PNAME}}; {\mantt{STATUS}} )
\manroutineitem {Arguments }{}
\begin{manparametertable}
\manparameterentry {Given}{{\mantt{PNAME}} }{{\mantt{CHARACTER*}}({\mantt{*}}) }

           Expression specifying the name of a Graphics Device
           Parameter.
\manparameterentry {Modified}{{\mantt{STATUS}} }{{\mantt{INTEGER}} }
           Variable holding the status value.
           The routine will attempt to execute regardless of the input
           value of this variable.
           If its input value is not {\mantt{SAI\_\_OK}} then it is left %
unchanged
           by this routine, even if it fails to complete.   If its
           input value is {\mantt{SAI\_\_OK}} and this routine fails, then the
           value is changed to an appropriate error number.
\end{manparametertable}
\manroutineitem {Authors }{}
     Sid Wright.  ({\mantt{UCL}}::{\mantt{SLW}})
\end{manroutinedescription}
\manroutine {{\manheadstyle{SGS\_DEACT}}}{ De-activate {\mantt{ADAM}} {\mantt{SGS}}.}
\begin{manroutinedescription}
\manroutineitem {Description }{}
     De-activate {\mantt{ADAM}} {\mantt{SGS}} after use by an application.
\manroutineitem {Invocation }{}
     {\mantt{CALL}} {\mantt{SGS\_DEACT}} ( {\mantt{STATUS}} )
\manroutineitem {Arguments }{}
\begin{manparametertable}
\manparameterentry {Modified}{{\mantt{STATUS}} }{{\mantt{INTEGER}} }
           Variable to contain the status. This routine is executed
           regardless of the import value of status.
           If its input value is not {\mantt{SAI\_\_OK}} then it is left %
unchanged
           by this routine, even if it fails to complete.   If its
           input value is {\mantt{SAI\_\_OK}} and this routine fails, then the
           value is changed to an appropriate error number.
\end{manparametertable}
\manroutineitem {Authors }{}
     Sid Wright.  ({\mantt{UCL}}::{\mantt{SLW}})
\end{manroutinedescription}

\manrule

\end{document}
