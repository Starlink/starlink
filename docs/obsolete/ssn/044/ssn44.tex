\documentstyle[11pt]{article}
\pagestyle{myheadings}

%------------------------------------------------------------------------------
\newcommand{\stardoccategory}  {Starlink System Note}
\newcommand{\stardocinitials}  {SSN}
\newcommand{\stardocnumber}    {44.11}
\newcommand{\stardocauthors}   {A J Chipperfield}
\newcommand{\stardocdate}      {19 January 1994}
\newcommand{\stardoctitle}     {ADAM --- Installation Guide}
%------------------------------------------------------------------------------

\newcommand{\stardocname}{\stardocinitials /\stardocnumber}
\renewcommand{\_}{{\tt\char'137}}     % re-centres the underscore
\markright{\stardocname}
\setlength{\textwidth}{160mm}
\setlength{\textheight}{230mm}
\setlength{\topmargin}{-2mm}
\setlength{\oddsidemargin}{0mm}
\setlength{\evensidemargin}{0mm}
\setlength{\parindent}{0mm}
\setlength{\parskip}{\medskipamount}
\setlength{\unitlength}{1mm}

%------------------------------------------------------------------------------
% Add any \newcommand or \newenvironment commands here
%------------------------------------------------------------------------------

\begin{document}
\thispagestyle{empty}
SCIENCE \& ENGINEERING RESEARCH COUNCIL \hfill \stardocname\\
RUTHERFORD APPLETON LABORATORY\\
{\large\bf Starlink Project\\}
{\large\bf \stardoccategory\ \stardocnumber}
\begin{flushright}
\stardocauthors\\
\stardocdate
\end{flushright}
\vspace{-4mm}
\rule{\textwidth}{0.5mm}
\vspace{5mm}
\begin{center}
{\Huge\bf \stardoctitle}
\end{center}
\vspace{20mm}

%------------------------------------------------------------------------------
%  Package Description
\begin{center}
{\Large\bf Description}
\end{center}


This guide gives instructions on how to install a Starlink release of ADAM.
Some guidance is also given for those who wish to set up non-standard
installations.

Changes in this version of the document are listed in Appendix \ref{changes}.
\markright{\stardocname}
\newpage
\markright{\stardocname}

%------------------------------------------------------------------------------
\small
%  Add this part if you want a table of contents
  \setlength{\parskip}{0mm}
  \tableofcontents
  \setlength{\parskip}{\medskipamount}
  \markright{\stardocname}
\normalsize
%------------------------------------------------------------------------------
\newpage
\section{INTRODUCTION}
\subsection{Use of the Guide}
This guide gives instructions on how to install a Starlink release
of ADAM. It is assumed that certain other items of
Starlink software have been installed.
These requirements are detailed later.

It will be helpful to have read chapters 2 and 6 of SG/4 \cite{sg4}
before continuing with the installation.
Obviously it is not possible to try the examples in that document until
after installation is completed.

Sections of this guide marked with *** should be checked on each new release.
Other sections can usually be omitted, and existing modified files re-used, if
ADAM has already been installed.
To facilitate the re-use of modified files, it is recommended that they be
saved outside the ADAM directory tree.

{\em Where existing directories are used, ensure that they are empty before
performing backups.}

Release notes \cite{ssn45} will give details of changes for each release and
specify the disk space required.
They will also indicate if changes have been made to the released versions
of the files which may be modified locally.

\subsection{Full and Mini-Systems}
ADAM may be installed as a {\em full} system, which includes all system source
and object libraries {\em etc.}, or as a {\em mini} system consisting only of
the files necessary for running the system and developing application programs.

The full system will be distributed as a backup save set on magnetic tape and
the mini-system may be easily selected from it.
Initially upon release, the mini-system will be available as a disk save set
obtainable over the network.
It is expected that most sites will only install the mini-system as it requires
a quarter of the disk space of the full system.

\subsection{Complete and Partial Releases}
Releases may take one of two forms.
A {\em complete} release will contain all the files necessary for installing
either the full or mini-system but, if only a small number of changes to the
last release are required, a {\em partial} release will be issued.
This will consist of the changed files and a DCL procedure to perform the
update for either a full or mini-system.

\subsection{Version Numbers}
\label{vers}
Each release, whether complete or partial, will have an {\em ADAM version
number}.
The ADAM version number is tied to a Starlink Software Change and can therefore
be used to check that changes are up to date.
Releases which contain new functionality will have a version number like `2.1'
and subsequent releases containing only bug fixes or other minor changes will
be numbered `2.1-1', `2.1-2' {\em etc.}
A version number of the form 2.1 does not imply a complete release. This is a
change from earlier practice.

Items within ADAM, such as ICL, may have a different version number which will
not necessarily change with each ADAM release.
The ADAM version number will be displayed when \mbox{\$ ADAMSTART} is obeyed.
This implies that a new version of ADAMSTART.COM will be issued with every ADAM
release.

\section{REQUIRED STARLINK SOFTWARE}
\label{reqs}
The following separate items of Starlink software must be installed before any
ADAM application can be used:
\small \begin{quote}
HDS \\
EMS \\
CHR \\
HELP

\end{quote} \normalsize
Appropriate logical names and symbols must be defined as specified for each
item.
If the releases are not up to date, problems may arise.

Other Starlink software items, PAR, NDF, SGS, GKS {\em etc.}, need only be
installed if the applications to be run require them.

The following additional items of Starlink software must be installed in order
to provide everything necessary for ADAM system development:
\small \begin{quote}
LIBMAINT \\
GENERIC \\
TEX
\end{quote} \normalsize

\section{CREATING THE FILE STRUCTURE}
\subsection{Creation of ADAM Directories}
\label{dirs}
\begin{enumerate}
\item Create a directory to contain the ADAM system.
At Starlink sites, assign system logical name ADAMDISK to an appropriate disk
and create ADAMDISK:[ADAM] for this purpose.
Any site which follows the Starlink scheme will find subsequent installation
easier.
\item Assign process logical name ADAM\_SYS to the directory just created.
\item Create a directory to hold miscellaneous ADAM applications.
At Starlink sites this will be:
\small \begin{quote}
\begin{verbatim}
STARDISK:[ADAMAPP].
\end{verbatim}
\end{quote} \normalsize
\item Later in the installation, it may be found necessary to create directories
to contain site specific parts of ADAM and/or local applications software.
These are given logical names LADAM\_DIR and LADAM\_APPLIC.\\
At Starlink sites the directories should be:
\small \begin{quote}
\begin{verbatim}
LSTARDISK:[STARLOCAL.ADAM] and
LSTARDISK:[STARLOCAL.PACK.ADAMAPP].
\end{verbatim}
\end{quote} \normalsize
\end{enumerate}

\subsection{*** Installation of the Files}
\subsubsection{Mini-release}
This will normally be done by obeying the instructions in the applicable
Starlink Software Change notice after copying a disk save set.

To install the mini-release from tape, create subdirectory [.RELEASE]
of the ADAM top level directory created above and assign logical name ADAM\_DIR
to it. If an existing directory is used, ensure that it is empty. Then obey:
\small \begin{quote}
\begin{verbatim}
$ SET DEF ADAM_DIR
$ BACKUP mt:save_set/SELECT=([ADAM.RELEASE...]*.*;*) [...]*.*;*
\end{verbatim}
\end{quote} \normalsize
where \verb+mt:save_set+ is the appropriate unit and save set name.

\subsubsection{Full Release}
To install the full system from tape, ensure that directory ADAM\_SYS is empty
then obey:
\small \begin{quote}
\begin{verbatim}
$ SET DEF ADAM_SYS
$ BACKUP mt:save_set/SELECT=([ADAM...])*.*;* [...]*.*;*
\end{verbatim}
\end{quote} \normalsize
where: \verb+mt:save_set+ is the appropriate unit and save set name.

\subsubsection{Partial Release}
This will be done by obeying the instructions in the applicable Starlink
Software Change Notice.
It is assumed that the latest Full or Mini-release of ADAM has already been
installed and that the logical name ADAM\_SYS has been defined accordingly
(see Section \ref{nomodsys} item 4).

\section{SYSTEM LOGICAL NAMES}
\subsection{Introduction}
Certain logical names must be defined before ADAM can be run.
This is done by obeying the DCL procedure ADAM\_DIR:\-SYS\-LOGNAM.\-COM.
The version supplied is suitable for a standard Starlink site
with the full release installed; it may be modified, as described below,
for particular site requirements.
Release notes will indicate if SYS\-LOG\-NAM\-.COM has changed since the last
release and the installation should proceed accordingly.
\subsection{*** Selecting SYSLOGNAM.COM}
\subsubsection{If SYSLOGNAM.COM has not changed}
\label{nomodsys}
If the release notes do not indicate that the supplied version of
SYS\-LOGNAM\-.COM has changed, do the first one of the following which applies:
\begin{enumerate}
\item For a partial release, take no action.
\item If a customized version of SYSLOGNAM.COM has been saved from the previous
installation, substitute that version.
\item If the full release is installed, take no action.
\item If the mini-release is installed, obey:
\small \begin{quote}
\begin{verbatim}
$ COPY ADAM_DIR:SYSLOGNAM.MINI ADAM_DIR:SYSLOGNAM.COM
\end{verbatim}
\end{quote} \normalsize
Note that the difference between SYSLOGNAM.COM and SYSLOGNAM.MINI is that
in the former, system logical name ADAM\_SYS is defined.
ADAM\_SYS should not be defined unless the full release is installed as it may
be used to flag that the full system is installed if partial updates are
released later.
\end{enumerate}
Now, if you are happy that the version of SYS\-LOGNAM\-.COM that you have is
suitable for your site, continue with the installation; otherwise, particularly
if this is the first time you have installed ADAM, proceed to Section
\ref{modsys}.

\subsubsection{If SYSLOGNAM.COM has changed}
\label{modsys}
If this is your first installation of ADAM, or the release notes indicate that
SYS\-LOGNAM\-.COM has changed, check procedure ADAM\_DIR:\-SYS\-LOG\-NAM.\-COM
to see if any of the following items need changing:
\begin{itemize}
\item The definition of ADAM\_SYS (see Section \ref{dirs} and \ref{nomodsys}
item 4).
\item The definition of ADAM\_DIR and ADAM\_APPLIC (see Section \ref{dirs}).
\item The definition of LADAM\_DIR and LADAM\_APPLIC (see Section \ref{dirs}).
These two logical names are commented out of the released version.
\item The definitions of ADAM\_EXE and ADAM\_HELP may be altered for example
to include LADAM\-\_APP\-LIC. They are the search lists used to locate
executable images and ADAMCL help files respectively.
\item The definitions of STAR\_LINK\_ADAM and/or STAR\_DEV\_ADAM (see Section
\ref{starlibs}).
\item The definition of logical names MSP\_SHR or AZUSS.
\item The /SYSTEM qualifiers --
It is recommended that the logical names are set up as SYSTEM logical names
but they could be GROUP or even JOB logical names if necessary.
\item The definition of ICL global login command files ICL\-\_LOGIN\-\_SYS
{\em etc.} (see Section \ref{adampacks}).
\item The instructions to INSTALL shared images ADAMSHARE,
DTASK\-\_IMAGE\-\_ADAM, ADAMGRAPH7, NBS\-\_SHR, MSP\-\_SHR or AZUSS.
Note that shared image AZUSS must be INSTALLed; the other shared
images need not be INSTALLed.
\item The graphics logical names (For non-Starlink sites).
\item The definition of DEVDATASET (see Section \ref{devdataset}).
\end{itemize}

If changes are required, modify ADAM\_DIR:SYSLOGNAM.COM and ensure that
changes are noted in the prologue.

Arrange that ADAMDISK:[ADAM.RELEASE]SYSLOGNAM.COM, or the equivalent at your
site, is obeyed at system startup.
Ensure that system logical name ADAMDISK is defined first if it is used.
Note that ADAM\_DIR is defined within SYSLOGNAM.COM and therefore cannot be used
to point to the directory containing SYSLOGNAM.COM at system startup.
For Starlink sites, a suitable version of SSC:STARTUP.COM will be issued.

Save any revised SYSLOGNAM.COM elsewhere (LADAM\_DIR is suggested) for
possible use with subsequent releases of ADAM.


\section{GLOBAL LOGIN COMMAND FILES}
\subsection{ICL}
\label{adampacks}
A complete description of the way in which Starlink uses the login command
files feature of the Interactive Command Language (see \cite{icl}) is given in
\cite{ssn64}.
This section summarizes the role played by SYS\-LOGNAM\-.COM and indicates
the ways in which it might be modified at non-Starlink sites.

SYSLOGNAM.COM defines four system logical names associated with the global
login command file automatically obeyed by ICL when it is activated.
They are:
\begin{description}
\item[ICL\_LOGIN\_SYS] This defines the ICL `system' login command file.
The standard equivalence name of the {\em system} logical name is
ICLDIR:\-LOGIN\-.ICL.
This command file will warn users that ADAM tasks cannot be run because
the DCL procedure ADAM\_DIR:\-ADAM\-START\-.COM has not been obeyed.

When ADAM\_DIR:\-ADAM\-START\-.COM is obeyed, it defines a {\em process}
logical name ICL\-\_LOGIN\-\_SYS (with equivalence name ADAMLOGIN) to override
the {\em system} logical name and result in an ADAM specific ICL login command
file being obeyed.
\item[ADAMLOGIN] is the equivalence name of the {\em process} logical name
ICL\_LOGIN\_SYS {\em after} ADAM\-START has been obeyed; its standard
equivalence name is ICL\-DIR:\-ADAM\-LOGIN\-.ICL.
This command file will attempt to `LOAD' a command file with the logical name
ADAM\-\_PACKAGES.
\item[ADAM\_PACKAGES] This defines an ICL command file to be `LOADed' by
ICL\-DIR:\-ADAM\-LOGIN\-.ICL; its standard equivalence name will be
SSC:\-ADAM\-\_PACK\-AGES\-.ICL.
Note that this is a command file which is not part of the ADAM
release; it will define the initial commands and messages associated with
Starlink {\em standard}\/ and {\em option}\/ ADAM application  packages.
It will also attempt to `LOAD' a command file specified by the logical name
LADAM\-\_PACKAGES.
\item[LADAM\_PACKAGES] This specifies a command file which will be `LOADed'
if SSC:\-ADAM\-\_PACK\-AGES\-.ICL is `LOADed'; its standard equivalence name
will be LSSC:\-ADAM\-\_PACK\-AGES\-.ICL which defines the initial commands and
messages associated with {\em local}\/ ADAM application packages.
\end{description}

If this arrangement is not appropriate at your site, a number of options are
available.
\begin{enumerate}
\item If you wish to use a different ICL login command file {\em before}
ADAMSTART is obeyed, alter SYSLOGNAM\-.COM to define ICL\-\_LOGIN\-\_SYS
appropriately.
\item If you wish to use a different ICL login command file {\em after}
ADAM\-START is obeyed, alter SYS\-LOGNAM\-.COM to define ADAM\-LOGIN
appropriately.
\item If you do not want the Starlink ADAM\-\_PACK\-AGES command file to be
obeyed, alter SYS\-LOG\-NAM.COM to define ADAM\-\_PACK\-AGES appropriately for
your site.
A dummy command file DUMMY\-.ICL, provided in ICLDIR, may be of use here.
\end{enumerate}

Notes:
\begin{enumerate}
\item Alternative versions of the ICL login command files should be held in
LADAM\_DIR created in the way described above.
They should include at least the version heading of the released versions.
\item For instructions on modifying SYSLOGNAM.COM, see Section \ref{modsys}.
\end{enumerate}

ICL will also obey a command file pointed at by the logical name
ICL\_LOGIN\_LOCAL.
This is intended for local site use and can also be set in SYSLOGNAM.COM.
It is commented out of the supplied version.

\section{STARLINK SUBROUTINE LIBRARIES}
\label{starlibs}
SYSLOGNAM.COM defines two logical names required to make important Starlink
subroutine libraries available automatically for application development.
\begin{description}
\item[STAR\_LINK\_ADAM] This defines a link options file which will be used
with the task link commands, ALINK, ILINK {\em etc.}
The released version of SYSLOGNAM.COM points to a standard Starlink options
file which in turn searches a shareable image library defined by the same
logical name. The shareable images themselves need only be available if routines within
them are required by the task being linked.
\item[STAR\_DEV\_ADAM] This defines a procedure to be obeyed by
the ADAM application development startup procedure, ADAM\_DEV.
The released version points to the standard Starlink procedure which will
define logical names required for application development using the selected
Starlink libraries.
This is done by calling the selected {\em lib}\_DEV procedures if the
associated symbol is defined.
\end{description}

Those sites which are not interested in having this facility, or want
alternative libraries available, may wish to modify SYSLOGNAM.COM to re-define
STAR\_LINK\_ADAM and/or STAR\-\_DEV\-\_ADAM.
The STAR\-\_LINK\-\_ADAM options file should pick up the PAR library as a
minimum.
A minimal options file is provided in ADAM\_EXE:\-STAR\_LINK\_ADAM0\-.OPT
and a null version of STAR\-\_DEV\-\_ADAM is provided in
ADAM\-\_DIR:\-STAR\-\_DEV\-\_ADAM0\-.OPT. These may be used instead of the
standard.

For instructions on modifying SYSLOGNAM.COM, see Section \ref{modsys}.

\section{GLOBAL SYMBOLS}
Put the definition :
\small \begin{quote}
\begin{verbatim}
$ ADAMSTART:==@ADAM_DIR:ADAMSTART
\end{verbatim}
\end{quote} \normalsize
into your system-wide DCL login procedure. An appropriate
version of SSC:LOGIN.COM will be issued for Starlink sites.
(If you do not have a system-wide login procedure,
all users of ADAM should include the appropriate definition in their
own login command file.)

\section{*** PREPARING FOR USE}
If ADAM shared images are already `INSTALLed', they must be de-INSTALLed
using the INSTALL Utility command DELETE at this point.

Obey the appropriate SYSLOGNAM procedure now to set up the logical names
required to continue installation and to INSTALL the shared images

\section{*** VERIFICATION}
\subsection{Quotas Needed to Run ADAM}
\label{quotas}
Typical process quotas required for running ADAM are as follows:
\small \begin{quote}
\begin{verbatim}
CPU limit:                      Infinite  Direct I/O limit:       18
Buffered I/O byte count quota:     20480  Buffered I/O limit:     18
Timer queue entry quota:               9  Open file quota:        75
Paging file quota:                 40000  Subprocess quota:       10
Default page fault cluster            32  AST limit:              23
Enqueue quota:                        30  Shared file limit:       0
Max detached processes:                0  Max active jobs:         0
JTQUOTA                             3072
\end{verbatim}
\end{quote} \normalsize

\subsection{Verification - Running ADAM}
You should now be able to run ADAM.
Note that doing so will create a sub-directory [.ADAM] of the user's
SYS\$LOGIN directory if one does not already exist so it may be more
convenient to log in to some non-system username for the test.

Log in to a suitable username and set an appropriate default directory.
(Note that the logging in step is necessary unless the symbol ADAM\-START
has already been defined.)
Then type :
\small \begin{quote}
\begin{verbatim}
$ ADAMSTART
\end{verbatim}
\end{quote} \normalsize
to set up the required directory, files and symbols, and to display the ADAM
version number.

Now type:
\small \begin{quote}
\begin{verbatim}
$ ICL
\end{verbatim}
\end{quote} \normalsize
to start up the command language interpreter.
Any messages caused by the ICL system, local or private login command files
should appear followed by the prompt:
\small \begin{quote}
\begin{verbatim}
ICL>
\end{verbatim}
\end{quote} \normalsize
The system is then ready to receive commands. For details of what can be
done, see \cite{icl}.

As examples, try the following sessions. It is assumed that the standard
Starlink login command files are in use; otherwise it may be necessary to
DEFINE all the ICL tasks before they can be used.
It is also assumed that the Starlink software item TRACE is installed.

xxxxxxxxx is a subprocess name invented by ADAM (different in each case) and
yyyy is the suggested value (again different in each case and usually the last
value used by this user.)

If nothing is shown after the prompt for a parameter, it indicates that
\verb%<return>% was typed and the suggested value used.
\small \begin{quote}
\begin{verbatim}
ICL> TRACE
Loading TRACE_DIR:TRACE into xxxxxxxxx

OBJECT - Object to be examined   /@yyyy/ > DEVDATASET
\end{verbatim}
\end{quote} \normalsize
The contents of the device dataset will be displayed.
\small \begin{quote}
\begin{verbatim}
ICL>
\end{verbatim}
\end{quote} \normalsize
If any of the tape devices in DEVDATASET can be used, the
installation can be tested as follows:
\small \begin{quote}
\begin{verbatim}
ICL> TAPEMOUNT
Loading TAPEMOUNT into xxxxxxxxx
ACCESS MODE   /'READ'/ >
TAPEDRIVE   /@yyyy/ > mua0
%MOUNT-I-OPRQST, Please mount device _xxxxx$MUA0
%MOUNT-I-MOUNTED, mounted on _xxxxx$MUA0
ICL> TAPEDISM
Loading TAPEDISM into xxxxxxxxx
TAPEDRIVE   /@mua0/ >
ICL>
\end{verbatim}
\end{quote} \normalsize

To check the graphics installation, try:
\small \begin{quote}
\begin{verbatim}
ICL> DEFINE SGSX1 SGSX1
Loading SGSX1 into xxxxxxxxx
ACMODE - Access mode to device /'WRITE'/ > WRITE
V_DEVICE   /@XWINDOWS/ >
CANCL - Do you wish to cancel the device parameter? > Y
\end{verbatim}
\end{quote} \normalsize

The word STARLINK will be drawn in box on the device whose GNS \cite{sun57}
name is specified, and ICL will prompt for the next command.

To terminate the ICL session, type EXIT in response to the prompt.
\small \begin{quote}
\begin{verbatim}
ICL> EXIT
$
\end{verbatim}
\end{quote} \normalsize

\subsection{Verification - Program Development}
A reasonably complete test that program development is possible can be
carried out by building a monolith.
After ADAMSTART, execute the DCL command ADAM\_DEV. Then copy the source
of the test monolith into your working directory as follows:
\small \begin{quote}
\begin{verbatim}
$ ADAMSTART
$ ADAM_DEV
$ COPY ADAM_TEST:TESTADAM.* *
\end{verbatim}
\end{quote} \normalsize
This will have copied the source and interface file for a monolith
called TESTADAM containing the single A-task TESTRUN.
\small \begin{quote}
\begin{verbatim}
$ FORTRAN TESTADAM
$ MLINK TESTADAM
$ ICL
ICL> DEFINE TESTRUN TESTADAM
ICL> TESTRUN
TESTRUN successfully executed
ICL> EXIT
\end{verbatim}
\end{quote} \normalsize

\section{ADAM DEVICE DATA STRUCTURES}
\label{devdataset}
\subsection{Introduction}
ADAM obtains information about the tape drives on your VAX system from data
structures contained in an HDS container file which has the logical name
DEVDATASET assigned to it.
A SYSTEM DEVDATASET is set up at each site but users can create
and use their own versions if required, by defining a GROUP or JOB logical
name DEVDATASET to override the SYSTEM one.

The contents of a device dataset may be inspected using the TRACE facility.
as shown above.

\subsection{Creating ADAM Device Dataset Structures}
The version of DEVDATASET provided in ADAM\_DIR:DEVICE7.SDF may suffice for
your installation. If it does not, create a suitable one as described
below, copy it to LADAM\_DIR and alter the definition of DEVDATASET in
SYSLOGNAM.COM to point to the new file.

Note that the following procedure will create a sub-directory [.ADAM] of the
user's SYS\$LOGIN directory if one does not already exist so it may be more
convenient to create the DEVDATASET in a temporary file not owned by the
manager, and then copy it into LADAM\_DIR.

The command TAPECREATE used in the procedure will create the
container file if it does not exist and create (or alter) the required
structures within it.
It has a parameter DEVDATASET, which defaults to `DEVDATASET', defining
the container file to be created or updated.

For each tape drive on the system a magnetic tape dataset is created
using the TAPECREATE command, {\em e.g.} :
\small \begin{quote}
\begin{verbatim}
TAPECREATE MTA0 _MTA0:
\end{verbatim}
\end{quote} \normalsize
The first argument is the name by which the device will be known to users of
ADAM. In principle it can be any valid name but it is probably best to use
the VAX device name.

The second argument must be the full VAX device name. This can be a logical
name if so desired.

The procedure is usually carried out using an ICL command file.
The command file used to create the supplied DEVDATASET is
supplied in ADAM\-\_DIR:\-DEV\-ICE7.ICL.
Copy this to your working directory and edit it as required.

Supposing that the necessary TAPECREATE commands have been set up
in a file named DEVICES.ICL, perform the following procedure.
\small \begin{quote}
\begin{verbatim}
$ ADAMSTART
\end{verbatim}
\end{quote} \normalsize
Now define logical name DEVDATASET to point to the required container file.
For example:
\small \begin{quote}
\begin{verbatim}
$ DEFINE/JOB DEVDATASET LADAM_DIR:DEVICES
\end{verbatim}
\end{quote} \normalsize
Now start ICL and obey the commands in DEVICES.ICL.
\small \begin{quote}
\begin{verbatim}
$ ICL
ICL> LOAD DEVICES
\end{verbatim}
\end{quote} \normalsize

You should save DEVICES.ICL to enable DEVDATASET to be rapidly re-created
in the event of problems or if it is required by a new release.

\begin{thebibliography}{9}
\bibitem{sg4} M D Lawden, {\it ADAM --- The Starlink Software Environment},
Starlink Guide 4.
\bibitem{ssn45} A J Chipperfield, {\it ADAM --- Release n.m}, Starlink System
Note 45.
\bibitem{icl} J A Bailey, {\it ICL --- The Interactive Command Language for
ADAM - Users Guide}, Starlink Guide 5.
\bibitem{ssn64} A J Chipperfield, {\it ADAM --- Organization of Applications
Packages}, Starlink System Note 64.
\bibitem{sun57} D L Terrett, {\it GNS --- Graphics Workstation Name Service},
Starlink User Note 57.
\end{thebibliography}

\newpage
\appendix
\section{Document Changes}
\label{changes}
The changes made to this document since the previous version are as follows:
\begin{description}
\item[Section \ref{vers}] There is no longer a connection between version
numbering and the type of release (complete or partial).
\item[Section \ref{reqs}] The list of required separate Starlink software items
has been modified to include HELP, PSX and PAR.
\item[Section \ref{starlibs}] The STAR\_LINK\_ADAM options file must now pick
up the PAR library as a minimum.
\end{description}
\end{document}
