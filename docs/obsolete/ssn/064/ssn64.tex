\documentstyle[11pt]{article} 
\pagestyle{myheadings}

%------------------------------------------------------------------------------
\newcommand{\stardoccategory}  {Starlink System Note}
\newcommand{\stardocinitials}  {SSN}
\newcommand{\stardocnumber}    {64.2}
\newcommand{\stardocauthors}   {A J Chipperfield}
\newcommand{\stardocdate}      {3 April 1992}
\newcommand{\stardoctitle}     {ADAM --- VAX Organization of Applications
Packages}
%------------------------------------------------------------------------------

\newcommand{\stardocname}{\stardocinitials /\stardocnumber}
\markright{\stardocname}
\setlength{\textwidth}{160mm}
\setlength{\textheight}{230mm}
\setlength{\topmargin}{-2mm}
\setlength{\oddsidemargin}{0mm}
\setlength{\evensidemargin}{0mm}
\setlength{\parindent}{0mm}
\setlength{\parskip}{\medskipamount}
\setlength{\unitlength}{1mm}

%------------------------------------------------------------------------------
% Add any \newcommand or \newenvironment commands here
\renewcommand{\_}{{\tt\char'137}}
%------------------------------------------------------------------------------

\begin{document}
\thispagestyle{empty}
SCIENCE \& ENGINEERING RESEARCH COUNCIL \hfill \stardocname\\
RUTHERFORD APPLETON LABORATORY\\
{\large\bf Starlink Project\\}
{\large\bf \stardoccategory\ \stardocnumber}
\begin{flushright}
\stardocauthors\\
\stardocdate
\end{flushright}
\vspace{-4mm}
\rule{\textwidth}{0.5mm}
\vspace{5mm}
\begin{center}
{\Large\bf \stardoctitle}
\end{center}
\vspace{20mm}
%------------------------------------------------------------------------------
%  Package Description
\begin{center}
{\Large\bf Description}
\end{center}
This document describes the way in which Starlink uses the facilities
of the Interactive Command Language (ICL) for setting up easy access to ADAM 
applications packages whilst keeping the installation of packages separate 
from the release of ADAM.

Changes in this version of the document are listed in Appendix \ref{changes}.
\markright{\stardocname}
\newpage
%------------------------------------------------------------------------------
  Add this part if you want a table of contents
  \setlength{\parskip}{0mm}
  \tableofcontents
  \setlength{\parskip}{\medskipamount}
  \markright{\stardocname}
%------------------------------------------------------------------------------
\newpage
\section{Introduction}
This document describes the way in which Starlink uses the facilities
of the Interactive Command Language (ICL) \cite{icl} for setting up easy 
access to ADAM applications packages whilst keeping the installation of 
packages separate from the release of ADAM.

The term {\em package} in this document may be taken to mean a related group of 
ADAM applications. Packages may be described as {\em standard}, {\em option}
or {\em local}\/ (see SUN/59 \cite{sun59} for an explanation of these terms).

The scheme will be implemented gradually and depends, to some extent, on the
co-operation of package support programmers.
When a new package is being developed, the way in which it will fit into the
scheme should be discussed with the Starlink Software Librarian and the Head 
of Applications.

\section{Action at ICL Startup}
\subsection{ICL Login Command Files}
ICL uses logical names to identify files containing commands which it will 
obey automatically before taking input from any file specified as a parameter 
of the ICL command, or prompting for input.
The logical names, in the order they are accessed, are:

\begin{tabular}{ll}
ICL\_LOGIN\_SYS   & Intended for `system' login commands\\
ICL\_LOGIN\_LOCAL & Intended for local site login commands\\
ICL\_LOGIN      & Intended for user's login commands
\end{tabular}

Two `system' login command files will form part of the ADAM release.
\begin{description}
\item[LOGIN.ICL] This command file will be obeyed unless \$ ADAMSTART has been
obeyed.
It will display the ICL version number, warn users that ADAM\-START has not 
been obeyed (therefore ADAM tasks may not be run) and invite the use of the 
HELP command.
\item[ADAMLOGIN.ICL] This command file will be obeyed if \$ ADAM\-START has been
obeyed.
It will display the ICL version number then
obey (by means of a LOAD command) the ICL command file defined by logical name
ADAM\-\_PACKAGES.
Finally it will invite the use of the HELP command.
\end{description}
A system logical name ADAM\_PACKAGES will be defined as 
SSC:\-ADAM\-\_PACK\-AGES.
This will also be used to locate the ADAM\-\_PACKAGES help library, described
later.

Similarly, a system logical name LADAM\_PACKAGES will be defined as
LSSC:\-ADAM\-\_PACK\-AGES.


\subsection{Notes}
\begin{enumerate}
\item The {\em system} logical names associated with this scheme are defined by
ADAM's SYS\-LOG\-NAM procedure which is obeyed at VMS system startup.
(For details, see SSN/44 \cite{ssn44}.)
\item The switch between login command files is achieved by defining
ICL\-\_LOGIN\-\_SYS as a system logical name and having ADAM\-START define a 
process logical name to override it. The process logical name is actually
defined in terms of another {\em system} logical name, ADAM\-LOGIN.
\item Both the system login command files will be held in ICLDIR and will be
controlled by ADAM support staff.
\item ICL\-\_LOGIN\-\_LOCAL and ICL\-\_LOGIN will not be defined in the
standard ADAM installation.
\end{enumerate}

\section{The Package}
Each package will be stored in a directory tree with a system logical name
{\em package}\_DIR pointing to the top level of the tree.
For the purposes of this scheme, the top-level directory should contain a
{\em Package Help Library} and a {\em Package Definition Command File}.
In most cases the package tasks will all be linked into a single monolith.
Packages could however be defined to consist of any mixture of task types 
and ICL procedures.
They could, in fact, include elements from other packages.

Because a package could well consist of applications written by a number of
different authors, the term {\em package administrator} is used for the
person with overall responsibility for the package.

\subsection{The Package Help Library}
Each package should provide a {\em Package Help Library}.
This is a VMS help library the top-level topic of which gives a brief 
description of the package.
Subtopics giving further information may be arranged at lower levels at the 
package administrator's discretion.
A subtopic for each command in the package would be expected.

\subsection{The Package Definition Command File}
Each package should contain a {\em Package Definition Command File}.
This is a file containing ICL commands to:
\begin{enumerate}
\item define the commands which the user will use to run the package,
\item specify the source(s) of help information for the package,
\item display information about the package.
\end{enumerate}
 
For example, the Package Definition Command File for KAPPA would be
KAPPA\_DIR:\-KAPPA\-.ICL and could be something like:
\begin{quote}
\begin{verbatim}
{ KAPPA - Package Definition Command File}
{ History:
{   26-Sep-1989 Original (RAL::CUR)

DEFINE ADD KAPPA_DIR:KAPPA
DEFHELP ADD KAPPA_DIR:KAPPA

DEFINE APERADD KAPPA_DIR:KAPPA
DEFHELP APERADD KAPPA_DIR:KAPPA

... etc...

PRINT


PRINT  " --    Initialised for KAPPA    --"
PRINT  " -- Version 0.5, 1989 September --"
PRINT
\end{verbatim}
\end{quote}

The Package Definition Command File is controlled by the package administrator.

\section{The ADAM\_PACKAGES Help Library}
A VMS help library, SSC:\-ADAM\-\_PACKAGES\-.HLB will be maintained by the 
Starlink Software Librarian.
The top-level topic (PACKAGES) will give a list, with a single line
description, of all the {\em standard} and {\em option} packages.
The second level will be subtopics for the individual packages.
Each subtopic will give a brief description of the package and describe how
to start using it.

A similar help library should be maintained in LSSC by the Site Manager to
describe any {\em local} packages in the same way.

\section{The Starlink ADAM\_PACKAGES Command Files}
\label{packs}
\subsection{SSC:ADAM\_PACKAGES.ICL}
This is the ICL command file LOADed by ADAM\-LOGIN\-.ICL at ICL startup.
The command file will be controlled by the Starlink Software Librarian and
will:
\begin{enumerate}
\item For each {\em standard}\/ or installed {\em option}\/ package, define 
the {\em Package Startup Command} which, if issued by the user, will cause ICL
to obey the Package Definition Command File.
\item For each non-installed {\em option}\/ package, it will define a dummy 
Package Startup Command which, if issued, will inform the user politely that 
the package is not available at the site.
\item For each {\em standard} or {\em option} package, define, for the ICL
help system, the relevant entry in the ADAM\-\_PACKAGES help library.
\end{enumerate}

For example, for KAPPA (a {\em standard} package), it will contain:
\begin{quote}
\begin{verbatim}
{  Definitions for KAPPA }
DEFHELP KAPPA ADAM_PACKAGES PACKAGES KAPPA
DEFSTRING KAPPA LOAD KAPPA_DIR:KAPPA
\end{verbatim}
\end{quote}

For an {\em option} package, it will contain a DEFHELP command and then check
for the presence an appropriate file to determine whether or not the package 
is installed. For example:
\begin{quote}
\begin{verbatim}
{ Definitions for  ASTERIX  }
DEFHELP ASTERIX ADAM_PACKAGES PACKAGES ASTERIX
IF FILE_EXISTS("AST_ROOT:[EXE]STARTUP.ICL")
   DEFSTRING ASTERIX LOAD AST_ROOT:[EXE]STARTUP
ELSE
   DEFSTRING ASTERIX NOTINSTALLED ASTERIX
ENDIF
\end{verbatim}
\end{quote}
Where NOTINSTALLED is a procedure which politely tells the user that the
package is not available.

Following these definitions, SSC:\-ADAM\-\_PACKAGES\-.ICL will check for the
presence of the command file defined by logical name LADAM\-\_PACKAGES, and
LOAD it if it is found.

\subsection{LSSC:ADAM\_PACKAGES.ICL}
This is the ICL command file defined by the Starlink standard logical name
LADAM\-\_PACK\-AGES and LOADed by SSC:\-ADAM\-\_PACKAGES\-.ICL -- it will define
both the Package Startup Command and the source of introductory help on the 
package (possibly an entry in LSSC:\-ADAM\-\_PACKAGES\-.HLB) for {\em local} 
packages.

LSSC:\-ADAM\-\_PACKAGES\-.ICL will be controlled by the local Site Manager and 
may also contain any other site specific login commands.
A template version will be issued by Starlink and the Site Manager can edit 
this file according to what is installed at his site before installing it in 
LSSC.

\subsection{Notes}
\begin{enumerate}
\item The analogy between this system and the DCL login command files 
SSC:\-LOGIN\-.COM and LSSC:\-LOGIN\-.COM can easily be seen.
Some packages may be defined in both systems so that they may be run 
from ICL or directly from DCL.
\item If the ADAM\-\_PACKAGES command file is not obeyed ({\em i.e.}
 ADAM\-START has not been obeyed), the LADAM\-\_PACKAGES command file will not
be obeyed either.
If a local login file is required for non-ADAM use of ICL, the logical name
ICL\-\_LOGIN\-\_LOCAL should be defined.
\item If a `package' consists of only a single command with the same name
as the package (for example PHOTOM \cite{photom}), the definition of that 
command and any associated DEFHELP command may be placed directly in the 
relevant ADAM\-\_PACKAGES command file.
\end{enumerate}

\section{Summary}
\begin{enumerate}
\item The ADAM release includes two `system' ICL login command files,
one for non-ADAM users of ICL and the other for ADAM users.
\item System logical names ADAM\-\_PACKAGES and LADAM\-\_PACKAGES are defined
to be SSC:\-ADAM\-\_PACKAGES and LSSC:\-ADAM\-\_PACKAGES respectively.
\item The Starlink Software Librarian provides the ADAM\-\_PACKAGES command file
and help library which together define introductory help information and
Package Startup Commands for all the Starlink {\em standard} and
{\em option} packages.
\item If there are any {\em local} packages, the local Site Manager provides 
the LADAM\-\_PACKAGES command file which defines the startup commands and the
source of introductory help for them.
\item Package administrators provide a `Package Definition Command 
File' and a `Package Help Library'.
\item When ICL is started up for ADAM use on Starlink machines, it obeys
ICLDIR:\-ADAM\-LOGIN\-.ICL which LOADs the ADAM\-\_PACKAGES command file which,
in turn, LOADs the LADAM\-\_PACKAGES command file.
\end{enumerate}

\section{The Effect}
\label{effect}
The following example session shows the sort of effect it is hoped to achieve
by means of this scheme.
\small
\begin{quote}
\begin{verbatim}
$ ICL

  Interactive Command Language   -   Version 1.5-6

  - Type HELP [command] for help on ICL and its commands

  WARNING - $ ADAMSTART has not been obeyed;
           therefore ADAM tasks will not run

ICL> EXIT
$ ADAMSTART
 ADAM version 2.0 available
$ ICL

  Interactive Command Language   -   Version 1.5-6

  - Type HELP package_name for help on specific Starlink packages
  -   or HELP PACKAGES for a list of all Starlink packages
  - Type HELP [command] for help on ICL and its commands

ICL> HELP PACKAGES 
  
PACKAGES 
  
   The following ADAM applications packages are available from Starlink: 
  
   Standard Packages: 
    CONVERT   -  Data format conversion. 
    FIGARO    -  General data-reduction. 
    KAPPA     -  Image processing. 
    SPECDRE   -  Spectroscopy Data Reduction 
    SST       -  Simple Software Tools 
    TSP       -  Time series and polarimetry data analysis. 
  
   Option Packages: 
    ASTERIX   -  X-ray data analysis. 
    CCDPACK   -  CCD data reduction 
    DAOPHOT   -  Stellar photometry. 
    PHOTOM    -  Aperture photometry. 
    PISA      -  Position, Intensity and Shape Analysis 
    SCAR      -  Catalogue access and reporting. 
  
  
  Additional information available: 
  
  ASTERIX    CCDPACK    CONVERT    DAOPHOT    FIGARO     KAPPA 
  MISCELLANEOUS         PHOTOM     PISA       SCAR       SPECDRE    SST 
  TSP 
  
PACKAGES Subtopic? KAPPA 
  
PACKAGES 
  
  KAPPA 
  
     KAPPA---the Kernel APplication PAckage---currently  comprises  applications 
     for  general  image  processing,  many  of which will function with data of 
     dimensionality other than two; data visualisation, with flexible control of 
     the  location and size of pictures; and the manipulation of NDF components. 
     Some of KAPPA's main features are fast  algorithms;  the  handling  of  bad 
     pixels  and  quality,  processing  of data errors, a help system, extensive 
     error  reporting,  the  utilisation  of  the  AGI  graphics  database,  and 
     graphics-device  independence  via  GKS  and  IDI.   Many applications work 
     generically, that is  handle  all  non-complex  data  types  directly,  for 
     efficiency.  KAPPA is fully described in SUN/95. 
  
     To make the commands of KAPPA available, type: 
  
         ICL> KAPPA 
  
PACKAGES Subtopic? 
Topic? 
ICL> HELP KAPPA 
  
PACKAGES 
  
  KAPPA 
  
     KAPPA---the Kernel APplication PAckage---currently  comprises  applications 
     for  general  image  processing,  many  of which will function with data of 
     dimensionality other than two; data visualisation, with flexible control of 
     the  location and size of pictures; and the manipulation of NDF components. 
     Some of KAPPA's main features are fast  algorithms;  the  handling  of  bad 
     pixels  and  quality,  processing  of data errors, a help system, extensive 
     error  reporting,  the  utilisation  of  the  AGI  graphics  database,  and 
     graphics-device  independence  via  GKS  and  IDI.   Many applications work 
     generically, that is  handle  all  non-complex  data  types  directly,  for 
     efficiency.  KAPPA is fully described in SUN/95. 
  
     To make the commands of KAPPA available, type: 
  
         ICL> KAPPA 
  
Topic? 
ICL> KAPPA 
 Help key KAPPA redefined 
 
 --    Initialised for KAPPA    -- 
 --   Version 0.8, 1991 August  -- 
 
 Type HELP KAPPA or KAPHELP for KAPPA help 
  
ICL> HELP KAPPA 
  
HELP 
 
   This is the KAPPA online help system. It invokes the VMS HELP Facility 
   to display information about a KAPPA command or topic.  If you need 
   assistance using this help library, enter "Using_help" in response to 
   the "Topic?" prompt.  If you need more information about getting help 
   about KAPPA from the ICL level, then enter "Command-line_help". 
 
Topic? ADD 
 
ADD 
  
  Add two NDF data structures. 
  
  Description: 
  
     The routine adds two NDF data structures pixel-by-pixel to produce 
     a new NDF. 
  
  Additional information available: 
  
  Parameters Notes      Authors    History 
  
ADD Subtopic? PARAMETERS
  
ADD 
  
  Parameters 
  
    For information on individual parameters, select from the list below: 
  
    Additional information available: 
  
    IN1        IN2        OUT        TITLE 
  
ADD Parameters Subtopic? IN1 
  
ADD 
  
  Parameters 
  
    IN1 
  
      IN1 = NDF (Read) 
         First NDF to be added. 
  
ADD Parameters Subtopic? 
ADD Subtopic? 
Topic? 
ICL> ASTERIX

      ASTERIX  is not installed at this site.
   If you really want it, contact your Site Manager.

ICL> EXIT 
$
\end{verbatim}
\end{quote}
\normalsize

\section{Problems}
\label{probs}
\begin{enumerate}
\item In some cases it would be nicer to come out of HELP than to be prompted
for another HELP topic.
For example when HELP package is issued before the package is started.
\item At the moment, when the source of help on a package is switched from
the PACKAGES help library to the package's help library, ICL produces a
warning message.
It should be possible to have a SET command which will switch this message off
when it is not required.
\end{enumerate}

\begin{thebibliography}{9}
\bibitem{icl} J A Bailey, {\it ICL --- The New ADAM Command Language - Users
Guide}, Joint Astronomy Centre.
\bibitem{sun59} M D Lawden, {\it Starlink Software Reorganization}, Starlink
User Note 59.
\bibitem{kappa} M J Currie, {\it KAPPA --- Kernel Application Package},
Starlink User Note 95.
\bibitem{ssn44} A J Chipperfield, {\it ADAM --- Installation Guide}, Starlink
System Note 44.
\bibitem{photom} N Eaton, {\it PHOTOM --- An aperture photometry routine},
Starlink User Note 45.
\end{thebibliography}

\newpage
\appendix
\section{Document Changes}
\label{changes}
The changes made to this document since the previous version are as follows:
\begin{description}
\item[Layout] The format of the document has changed.
\item[Section \ref{packs}] ADAM\_PACKAGES now defines both {\em standard}\/ and
{\em option}\/ packages. LADAM\-\_PACK\-AGES is reserved for {\em local}\/
packages.
\item[Section \ref{effect}] The example has been brought up to date.
\item[Section \ref{probs}] two of the original problems have been overcome:
\begin{itemize}
\item Site Managers are no longer required to edit the local
LADAM\-\_PACK\-AGES command file for installed {\em option}\/ packages.
\item Multiple keys are now allowed in the ICL HELP command, {\em e.g.}:
\begin{quote} \begin{verbatim}
ICL> HELP ADD PARAMETERS IN1
\end{verbatim} \end{quote}
is accepted.
\end{itemize}
\end{description}

\end{document}
