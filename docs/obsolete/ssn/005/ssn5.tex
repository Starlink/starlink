\documentstyle{article}
\pagestyle{myheadings}

%------------------------------------------------------------------------------
\newcommand{\stardoccategory}  {Starlink System Note}
\newcommand{\stardocinitials}  {SSN}
\newcommand{\stardocnumber}    {5.1}
\newcommand{\stardocauthors}   {A C Davenhall}
\newcommand{\stardocdate}      {13 May 1990}
\newcommand{\stardoctitle}     {APIG --- Installation Guide \footnote{Document version number 1.
This document applies to APIG version 1.0.}}
%------------------------------------------------------------------------------

\newcommand{\stardocname}{\stardocinitials /\stardocnumber}
\renewcommand{\_}{{\tt\char'137}}     % re-centres the underscore
\markright{\stardocname}
\setlength{\textwidth}{160mm}
\setlength{\textheight}{240mm}
\setlength{\topmargin}{-5mm}
\setlength{\oddsidemargin}{0mm}
\setlength{\evensidemargin}{0mm}
\setlength{\parindent}{0mm}
\setlength{\parskip}{\medskipamount}
\setlength{\unitlength}{1mm}

%------------------------------------------------------------------------------
% Add any \newcommand or \newenvironment commands here
%------------------------------------------------------------------------------

\begin{document}
\thispagestyle{empty}
SCIENCE \& ENGINEERING RESEARCH COUNCIL \hfill \stardocname\\
RUTHERFORD APPLETON LABORATORY\\
{\large\bf Starlink Project\\}
{\large\bf \stardoccategory\ \stardocnumber}
\begin{flushright}
\stardocauthors\\
\stardocdate
\end{flushright}
\vspace{-4mm}
\rule{\textwidth}{0.5mm}
\vspace{5mm}
\begin{center}
{\Large\bf \stardoctitle}
\end{center}
\vspace{5mm}

%------------------------------------------------------------------------------
%  Add this part if you want a table of contents
\setlength{\parskip}{0mm}
\tableofcontents
\setlength{\parskip}{\medskipamount}
\markright{\stardocname}
%------------------------------------------------------------------------------



\section{Readership}

You should read this document if you are a system manager of other
person intending to install the APIG software for analysing high
resolution spectra of interstellar absorption lines on a VAX running
VMS.

\section{Introduction}

APIG is a Starlink associated software item. It is distributed by the
Starlink project from the Rutherford Appleton Laboratory and this note
describes how to install it. It is assumed that you will be installing
it on either a Starlink node or at least a VAX where Starlink software
is available.

If you are attempting to install it on a VAX which does not have any
other Starlink software then some additional items are required (see
`Starlink software required' below). Note that some of these items are
not available in the public domain and a license may be required. If you
are attempting to install it on anything except a VAX running the VMS
operating system then forget it!

\section{Starlink software required}

APIG requires the following items of either Starlink software or
proprietary software distributed with the Starlink software collection:

\begin{enumerate}

  \item RAL GKS 7 (see SUN 83, SSN 39).

  \item The NCAR graphics system (see SUN 88).

  \item The CHR library of character handling subroutines (see SUN 40).

\end{enumerate}

Note that the various command procedures used by APIG assume the
standard Starlink logical names, locations and filenames for these
items. The CHR and NCAR libraries should only be required if the APIG
execution modules have to be regenerated. However, the execution
modules supplied with the release assume that a RAL GKS shareable
image is available in the standard Starlink form.

\section{Directory structure}

A `root' directory or subdirectory to hold APIG should be created at
some convenient point in your directory structure. This directory or
subdirectory should be called APIG. The save set containing APIG can
then be re-constructed in this directory. The following directory
structure will be created;

\nopagebreak
\begin{verbatim}

  APIG --+-- .COM     Command procedures.
         |
         +-- .DATA    Data files.
         |
         +-- .DOCS    Documentation.
         |
         +-- .HELP    On-line help library.
         |
         +-- .RUN     Execution modules.
         |
         +-- .SOURCE -+-- .INCLUDE  Include files.
                      |
                      +-- .LIBS     Libraries.
                      |
                      +-- .MPROG    Main programs.

\end{verbatim}

\section{Logical names and symbols}

A number of logical names and symbols have to be set up in order to make
APIG available to users. You should carry out the following steps:

\begin{enumerate}

  \item First define the following logical name in the system logical
   name table:

   \begin{verse}

DEFINE/SYSTEM  APIGDIR (APIG\_root) -- set up in LSSC:STARTUP.COM

   \end{verse}

   where (APIG\_root) is wherever you have chosen to place the APIG root
   directory.

  \item Arrange for users to run command procedure

   \begin{verse}

APIG*SETUP :== APIGDIR:SETUPAPIG.COM  -- set up in SSC:LOGIN.COM

   \end{verse}

   prior to their attempting to use APIG.


   The logical names and symbols set up by SETUPAPIG.COM are as
   follows:

   \begin{description}

     \item[APIGDOCS] (logical name) Directory containing documentation.

     \item[APIGDATA] (logical name) Directory containing data files.

     \item[START\_APIG] (symbol) Runs a command procedure to set up for
      using APIG.

     \item[RUN\_APIG] (symbol) A command procedure to use APIG in its
      simplest form.

     \item[PROGAM\_APIG] (symbol) A command procedure to set up further
      logical names and symbols only necessary for maintenance and
      development of APIG.

   \end{description}
\end{enumerate}

APIG should now be available for use (but see `customising the graphics
devices' below, if you have any graphics devices other than the usual
Starlink ones).

\section{Check that APIG is working}

As a quick check that APIG is working type

\begin{verse}

START\_APIG

\end{verse}

followed by

\begin{verse}

CONTROL\_APIG

\end{verse}

and a prompt should appear. If it does all is ok and you can type
control\_Y to break out. If it does not something is awry.

\section{Customising the graphics devices}

APIG has two text files which describe the various GKS workstations
that it knows about:

\begin{description}

  \item[APIGDATA:MONITORS.DAT] for interactive devices.

  \item[APIGDATA:PLOTTERS.DAT] for hard-copy devices.

\end{description}

These files contain most of the standard Starlink graphics devices
current when they were created. Devices are available to APIG {\em
only} if they are included in one of these files. Thus you may wish to
edit them to reflect the devices available at your site. Each line of
the files refers to a separate device. The various columns of the
files are (reading left to right):

\begin{enumerate}

  \item The RAL GKS workstation type number for the device.

  \item The GKS connection identifier to be used (usually = 0).

  \item A brief description of the device which will be presented to
   the user. It should be enclosed in single quotes.

\end{enumerate}

\section{Resources required by users}

Users should not require any special resources, quotas or privileges
to use APIG, just a few hundred blocks of file space for data files,
log files, plotter files etc.

\section{Further information}

Further information on the use of APIG can be found in the APIG user
guide. The Latex source for this document is available in file

\begin{center}

APIGDOCS:USERGUIDE.TEX

\end{center}

\section{Relinking}

If you need to relink APIG the following procedure should be followed

\begin{enumerate}

  \item Issue the command

   \begin{verse}

PROGRAM\_APIG

   \end{verse}

   which sets up the logical names and symbols necessary for program
   development and maintenance, including those for relinking.

  \item To generate new execution modules type

   \begin{verse}

LINK\_APIG CONTROL

LINK\_APIG APIG

   \end{verse}

  \item Check that the new execution modules are working satisfactorily
   (see above).

  \item To purge the old execution modules

   \begin{verse}

PURGE APIGRUN:*.EXE

   \end{verse}

\end{enumerate}

\vspace{15 ex}

\begin{verse}

A C Davenhall.

Royal Observatory,  \\
Blackford Hill,     \\
Edinburgh.          \\
EH9 3HJ.            \\
UK.

ACD @ UK.AC.ROE.STAR (Janet)

\end{verse}

\typeout{  }
\typeout{***********************************************}
\typeout{  }
\typeout{Reminder: run this document through Latex twice}
\typeout{to resolve the entries in the table of contents.}
\typeout{  }
\typeout{***********************************************}
\typeout{  }

\end{document}


\end{document}
