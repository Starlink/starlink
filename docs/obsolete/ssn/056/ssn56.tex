\documentstyle{article}
\pagestyle{myheadings}
\markright{SSN/56.1}
\setlength{\textwidth}{160mm}
\setlength{\textheight}{240mm}
\setlength{\topmargin}{-5mm}
\setlength{\oddsidemargin}{0mm}
\setlength{\evensidemargin}{0mm}
\setlength{\parindent}{0mm}
\setlength{\parskip}{\medskipamount}
\setlength{\unitlength}{1mm}

\begin{document}
\thispagestyle{empty}
SCIENCE \& ENGINEERING RESEARCH COUNCIL \hfill SSN/56.1\\
RUTHERFORD APPLETON LABORATORY\\
{\large\bf Starlink Project\\}
{\large\bf Starlink System Note 56.1}
\begin{flushright}
M D Lawden\\
17th February 1988
\end{flushright}
\vspace{-4mm}
\rule{\textwidth}{0.5mm}
\vspace{5mm}
\begin{center}
{\Large\bf Updating Starlink Software over a Network}
\end{center}
\vspace{5mm}
The best way to keep a copy of the Starlink Software Collection up to date
is to receive and implement the Starlink Software Change (SSC) notices
issued by the Starlink Software Librarian from Rutherford Appleton Laboratory.
About 50 SSC's are issued every year.
For you to receive updates in this way, it must be possible for RAL to send you
a text file and for you to read a BACKUP save set from the Starlink computer
RLVAD at RAL.

You must supply the following information to the Starlink Software Librarian.
\begin{itemize}
\item The name of a contact at your site who is responsible for reading and
implementing the SSC's issued by Starlink.
\item A full network address to be used by Starlink to send you SSC's.
This should include an appropriate username.
\item The full postal address, telephone and telex numbers of your site.
\end{itemize}
The corresponding RAL details are as follows:
\begin{itemize}
\item Starlink Software Librarian (currently Mr M D Lawden or Mr S E Black)
\item DECNET --- RLVAD::STAR\\
JANET --- STAR@UK.AC.RL.STAR
\item Starlink Software Librarian, Building R68, Rutherford Appleton Laboratory,
Chilton, DIDCOT, Oxon, OX11 0QX, UK.
Tel: Abingdon (0235) 21900 X5114.
Tx: 83159 RUTHLB G
\end{itemize}
The updating process is as follows:
\begin{itemize}
\item The Starlink Software Librarian will send you an SSC each time one is
released.
This is a text file which tells you how to implement the update.
It will tell you what new documentation is available, and will include notes
about the release (what has changed, verification tests, \ldots).
It will also tell you which information summaries have been updated.
It will be sent by electronic mail to the username and site you have specified.
\item You must copy the backup save set mentioned in the SSC.
This is stored at RAL in directory STARDISK:[STAR.TEMP].
The file name will be SSCnnn.BCK where `nnn' is the release number.
This save set will only be stored at RAL for one month before being deleted
without notice.
This is necessary because of disk space limitations.

{\bf YOU MUST COPY THE SAVE SET WITHIN A MONTH OF RELEASE!}

If you don't copy the save set in time, it may be necessary to send you a
complete copy of the Starlink Software and this could take a long time.
\item The copying method depends on whether you use the TRANSFER command on
JANET, or the COPY command on DECNET.
The JANET command is:
\begin{verbatim}
    $ TRANSFER/CODE=FAST/USERNAME=NETUSR,NETUSR -
        UK.AC.RL.STAR::STARDISK:[STAR.TEMP]SSCnnn.BCK  *.*
\end{verbatim}
The DECNET command is:
\begin{verbatim}
    $ COPY RLVAD::STARDISK:[STAR.TEMP]SSCnnn.BCK  *.*
\end{verbatim}
\item Starlink recommends that you set up a username STAR to be used for
managing the Starlink software.
Set up a directory [STAR.TEMP] to be used for storing the release files and
for carrying out the updating procedures.
\item Assuming you have copied SSCnnn.BCK into [STAR.TEMP], your next step is
to decompress it and recreate the directory structure holding the update.
Do this as follows:
\begin{verbatim}
    $ SET DEF [STAR.TEMP]
    $ LZDCM SSCnnn.BCK
    $ BACKUP SSCnnn.BCK/SAVE [*...]
\end{verbatim}
You can qualify the BACKUP command with /LOG if you want to see what is going
on, and with /VERIFY if you are a perfectionist; in practice, these qualifiers
just slow you down.
This will create a directory structure [STAR.TEMP.SSCnnn...] containing the
files being released.
\item Follow the instructions in the SSC note.
Normally, this will be a quick and simple operation; in most cases all you do is
give the directory [STAR.TEMP.SSCnnn] the logical name SSCTEMP and execute the
command procedure SSCTEMP:SSCnnn.COM.
\end{itemize}
You should also be aware of the following general notes:
\begin{itemize}
\item Most releases are less than 1000 blocks in size.
It has proved possible to copy releases occupying many thousand blocks without
difficulty at rates of up to 4000 blocks an hour.
However, if you want a release to be sent to you on tape, send a message to
STAR at RAL.
This will be much slower than a network release.
\item We will normally send you paper copies of any documentation released since
this may not be easily accessible on-line.
\item Be sure to implement the SSC's in the order they are released, even if
you are not interested in a particular release.
SSC's update documentation and index files which may have nothing to do with the
software being released.
\item If it is inconvenient to implement updates when they are released,
store the backup save sets until you are ready to use them.
Remember, they are only stored at RAL for a month and cannot be recovered after
that.
\item The initial installation of the Starlink Software Collection is
described in SSN/15.
The updating process described above assumes that your Starlink software has
been implemented as described there.
\item The preparation and release of Starlink Software Change notices is
described in SSN/41.
\item A description of how Starlink sites are run can be found in SGP/25.
\end{itemize}
\end{document}
