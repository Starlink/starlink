\documentstyle{article}
\pagestyle{myheadings}

%------------------------------------------------------------------------------
\newcommand{\stardoccategory}  {Starlink System Note}
\newcommand{\stardocinitials}  {SSN}
\newcommand{\stardocnumber}    {50.3}
\newcommand{\stardocauthors}   {P N Daly}
\newcommand{\stardocdate}      {28 August 1989}
\newcommand{\stardoctitle}     {NEWSMAINT --- News Maintenance Procedure}
%------------------------------------------------------------------------------

\newcommand{\stardocname}{\stardocinitials /\stardocnumber}
\markright{\stardocname}
\setlength{\textwidth}{160mm}
\setlength{\textheight}{240mm}
\setlength{\topmargin}{-5mm}
\setlength{\oddsidemargin}{0mm}
\setlength{\evensidemargin}{0mm}
\setlength{\parindent}{0mm}
\setlength{\parskip}{\medskipamount}
\setlength{\unitlength}{1mm}

\begin{document}
\thispagestyle{empty}
SCIENCE \& ENGINEERING RESEARCH COUNCIL \hfill \stardocname\\
RUTHERFORD APPLETON LABORATORY\\
{\large\bf Starlink Project\\}
{\large\bf \stardoccategory\ \stardocnumber}
\begin{flushright}
\stardocauthors\\
\stardocdate
\end{flushright}
\vspace{-4mm}
\rule{\textwidth}{0.5mm}
\vspace{5mm}
\begin{center}
{\Large\bf \stardoctitle}
\end{center}
\vspace{5mm}

\section{Function}

This item of software is a command procedure which is designed to help Starlink
Site Managers maintain their news libraries. It has been written under
Version 5 of the VMS operating system and is incompatible with earlier versions.

It offers more options than earlier versions of NEWSMAINT and works on any
library. The use of the logical names HLP\$LIBRARY  (and its subsidiaries
HLP\$LIBRARY\_1, HLP\$LIBRARY\_2 etc, see VMS General User Volume 4, DCL
Dictionary for more information) allows these libraries to be picked up by
the HELP command.

Default values are offered in bold type to the user, so a video terminal is
required. The procedure does not discriminate between upper and lower case
characters.

\section{Use}

The procedure may be invoked in one of four ways:
\begin{itemize}
\item by typing:
\begin{verbatim}
    $ NEWSMAINT
\end{verbatim}
in which case you will be presented with a menu of options. The procedure
defaults to a library called sys\$manager:sysnews.hlb. (which can be shortened
to sysnews).
Most options have default values indicated by bold type (and usually within
parentheses).
After selecting an option, you will be prompted for a parameter in the form of
a file, news item name, or cutoff date etc.
\item by typing:
\begin{verbatim}
    $ NEWSMAINT [Library]
\end{verbatim}
in which case you will be prompted for a parameter. The library may be
specified as sysnews, oldnews, jobsnews, faultsnews, localnews or by
a full file specification. System managers should edit the procedure to
reflect any other libraries they may have at their site.
\item by typing:
\begin{verbatim}
    $ NEWSMAINT [Library] [Option]
\end{verbatim}
in which case the procedure will either perform the task right away or
prompt for further input.
\item by typing:
\begin{verbatim}
    $ NEWSMAINT [Library] [Option] [Parameter]
\end{verbatim}
in which case the procedure will perform the task required straightaway.
\end{itemize}
In all four cases, the menu will be displayed on completion of the specified
task.
To exit from the procedure, type ``Q" after the ``Option :" prompt, or press
RETURN.

\section{Options}

The options available within the news maintenance procedure are as follows:
\begin{verbatim}
       [A]    Add       -      Add a new news item
       [B]    Before    -      List items added before a given date
       [C]    Change    -      Change item from one library to another
       [D]    Delete    -      Delete a news item
       [E]    Extract   -      Extract a news item to a file
       [H]    History   -      List creation dates of all items
       [L]    Library   -      Select a default library
       [M]    Modify    -      Modify an existing news item
       [P]    Print     -      Print a news item
       [Q]    Quit      -      Exit from NEWSMAINT
       [S]    Since     -      List items added since a given date
       [V]    View      -      View an item
\end{verbatim}

\section {Error checking}

The following checks are performed before the insertion or replacement of news
items:
\begin{itemize}
\item A message is output on your terminal whenever a line of text is found
which is more than 75 characters long.
Longer lines are automatically wrapped by the librarian, thus producing an
unindented layout, and this check has been included in order to give warning
that this will happen.
\item A message is output on your terminal whenever a numeral is found in
column 1 of a line of text.
This check has been included in order to highlight the position of numerals
which are used by the librarian to determine the {\em help level} of inserted
text, and also to enable you to find any numerals which have accidentally been
placed in column 1 and could therefore cause the insertion or replacement of
the news item to fail.
\end{itemize}
The procedure then summarises the findings produced by these two checks and
prompts you for permission to insert the news item.
You may attempt to insert text, whether or not it contains errors.

\section{The temporary file NEWS.TMP}

The procedure creates  a number of temporary files named ``NEWS.TMP" in the
current default directory.
Whenever either item addition or item modification have been performed, you will
be asked whether or not you wish to delete all files of this name.
If you do not wish to do so, a purge will be performed, leaving the last
{\em two} versions of the file intact.

If the {\bf Add} option is selected and no filename is supplied, the latest
version of NEWS.TMP, if such a file exists, will be loaded into the editor.

If the {\bf Before} or {\bf Since} option is selected, you may specify the
cutoff date either in days (eg 30, the default) or in the standard system
time format (eg 30-AUG-1989).
\end{document}

END OF SSN50.TEX ============================================================
