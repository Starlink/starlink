\documentstyle[11pt]{article}
\pagestyle{myheadings}

%------------------------------------------------------------------------------
\newcommand{\stardoccategory}  {Starlink System Note}
\newcommand{\stardocinitials}  {SSN}
\newcommand{\stardocnumber}    {27.1}
\newcommand{\stardocauthors}   {Adrian Fish}
\newcommand{\stardocdate}      {13 October 1995}
\newcommand{\stardoctitle}     {\staradmin\ --- Starlink User Database
                                Maintainer}
%------------------------------------------------------------------------------

\newcommand{\stardocname}{\stardocinitials /\stardocnumber}
\renewcommand{\_}{{\tt\char'137}}     % re-centres the underscore
\markright{\stardocname}
\setlength{\textwidth}{160mm}
\setlength{\textheight}{230mm}
\setlength{\topmargin}{-2mm}
\setlength{\oddsidemargin}{0mm}
\setlength{\evensidemargin}{0mm}
\setlength{\parindent}{0mm}
\setlength{\parskip}{\medskipamount}
\setlength{\unitlength}{1mm}

%------------------------------------------------------------------------------
% Add any \newcommand or \newenvironment commands here
\newcommand{\staradmin}{{\tt staradmin}}

%------------------------------------------------------------------------------

%\special{!userdict begin /bop-hook{gsave 200 30 translate
%65 rotate /Times-Roman findfont 216 scalefont setfont
%0 0 moveto 0.93 setgray (DRAFT) show grestore}def end}

\begin{document}
\thispagestyle{empty}
CCLRC / {\sc Rutherford Appleton Laboratory} \hfill {\bf \stardocname}\\
{\large Particle Physics \& Astronomy Research Council}\\
{\large Starlink Project\\}
{\large \stardoccategory\ \stardocnumber}
\begin{flushright}
\stardocauthors\\
\stardocdate
\end{flushright}
\vspace{-4mm}
\rule{\textwidth}{0.5mm}
\vspace{5mm}
\begin{center}
{\Large\bf \stardoctitle}
\end{center}
\vspace{5mm}

%------------------------------------------------------------------------------
%  Add this part if you want a table of contents
  \setlength{\parskip}{0mm}
%\begin{small}
  \tableofcontents
%\end{small}
  \setlength{\parskip}{\medskipamount}
  \markright{\stardocname}
%------------------------------------------------------------------------------

\newpage

\section{What \staradmin\ can do for you}

The reason I've written this stuff is to get round the tedious and
error prone regular updating of the local user summary lists, namely
(in  {\tt /star/local/admin}) {\tt usernames.lis}, {\tt users.lis} and
{\tt people.adr}.  At UCL our 200 or so entries used to take ages to
check. No longer. Here's what I've done.

The subject of this SSN is a utility called \staradmin. This utility
allows the system administrator to build and maintain a Starlink User
Database (UDB). The principal source of information for each user is a
text file, named after their username. The content of each file is a
list consisting of one keyword followed by the relevant user data per
line. These user database files reside in a single directory. The
\staradmin\ program is used to manipulate these user data files and
automatically generate user summary lists.

When a new user is added to the system, \staradmin\ is used to create a
new entry in the UDB directory. The user summary lists can be rebuilt
immediately in a very short time. Entries in the UDB can be edited and
manipulated with the \staradmin\ program.

\staradmin\ also provides a facility for cross-checking the UDB with the
{\tt passwd} database and displays any discrepancies \footnote{This
doubles up as a useful tool for getting your {\tt usernames.lis} into
shape prior to building your UDB.}.

The actions of the \staradmin\ program are controlled by the configuration
file {\tt staradmin.conf}. The system administrator has a wide range of
options available in this file for customising the \staradmin\ environment
to the needs of the local site.

As well as the {\tt usernames.lis} file, \staradmin\ will generate several
other files including a statistical summary of your users by category and
site, mail lists, mail aliases based on real name, the RGO `astropersons'
list for your site and an address list for sticky labels. Appendix
\ref{summarylists} has a complete list of summary files \staradmin\ can
generate.

There are two levels to the UDB --- {\tt BASE} and {\tt EXTENDED}. The
{\tt BASE} level UDB can be built from just a valid, up-to-date {\tt
usernames.lis}.  The {\tt EXTENDED} level can be built from an additional,
correctly formatted {\tt users.lis}\footnote{Refer to Appendix
\ref{ifusers} for full details on how to squeeze an existing {\tt
users.lis} into the UDB.}. The advantage of the {\tt EXTENDED} UDB option
is the ability to auto-generate a {\tt users.lis} and a local phone book.
The system administrator may choose the {\tt EXTENDED} option even if they
have no {\tt users.lis}, and they are then free to add the additional user
data by hand at a later date.

All fields in the {\tt BASE} and {\tt EXTENDED} UDB can be populated with
the data that can be found on a correctly filled out Starlink Application
Form.

{\large\bf NOTE:} This document is mainly concerned with the initial
customisation of the \staradmin\ enviroment and the building of a UDB. The
majority of the text covers items that only need to be performed
\underline{once} (so don't be put off). When the UDB has been built, then
the administrator only need concern themself with the \staradmin\ program
(and its {\tt man} pages for reference).

\staradmin\ requires at least Perl Version 5.001m.

%\newpage

\section{Copying and unpacking the \staradmin\ kit}
\label{sec:unpack}

The current version of the kit is available via anonymous FTP from the Starlink
FTP server at RAL:

\begin{verbatim}
      Site:  starlink-ftp.rl.ac.uk
      File:  pub/ussc/base/axp/staradmin-X.Y.tar.Z  (approx. 160KB)
\end{verbatim}

where {\tt X.Y} is the latest \staradmin\ version number.

On your local system, log on as {\tt star} and create a directory to hold
the kit.  The recommended place is under the {\tt /star/local} tree in a
directory called {\tt staradmin}.  Create and move to this directory:

\begin{verbatim}
      % mkdir /star/local/staradmin
      % cd /star/local/staradmin
\end{verbatim}

Now copy over the kit using anonymous FTP, {\em e.g.}:

\begin{verbatim}
      % ftp starlink-ftp.rl.ac.uk
              .....
        Name (starlink-ftp.rl.ac.uk:afish): anonymous
        Password: <user@yoursite>
        ftp> cd pub/system
        ftp> binary
        ftp> get staradmin-X.Y.tar.Z
              .....
        ftp> quit
\end{verbatim}

Unpack the kit.  This is a two stage process; unpacking the distribution
kit and then unpacking the source archive:

\begin{verbatim}
      % zcat staradmin-X.Y.tar.Z | tar xvf -
      % setenv SYSTEM <alpha_OSF1 OR sun4_Solaris>
      % ./mk build
\end{verbatim}

Now you must follow the Pre-installation Tasks as described in Section
\ref{preinstall}.  The distribution kit is no longer needed and may be
deleted.

\section{Pre-installation Tasks}
\label{preinstall}

Before you actually install the kit, some decisions must be made and
certain files should be customised for your site.
Follow the instructions in the following subsections.

\subsection{{\tt BASE} or {\tt EXTENDED} UDB?}

At this point you should consider which level of UDB you want to implement
at your site. If you have only a {\tt usernames.lis}, and you have a large
number of users, you may want to stay with the {\tt BASE} UDB which only
requires a {\tt usernames.lis}. Choosing the {\tt BASE} option however
means that certain summary files cannot be generated.

If you have a {\tt users.lis} and would like to be able to build a {\tt
users.lis} automatically, you should choose the {\tt EXTENDED} UDB. For
sites with existing {\tt users.lis} you should refer to Appendix
\ref{ifusers} on how best to get your {\tt users.lis} data into the UDB.
Sites that do not have a {\tt users.lis} but would like to take advantage
of the additional fields in the {\tt EXTENDED} UDB can select this. The
additional data must be added by hand.

An example {\tt users.lis} file is provided ({\tt users.eg.UCL}) in the
kit so you can see what you would be getting. Use the format of this
example file as a template if you are modifying your own {\tt users.lis}.

{\large\bf ACTION:} Decide which level you are going to use. If you choose
the {\tt EXTENDED} level and want to include the contents of an existing
{\tt users.lis} in the UDB, follow the instructions in Appendix
\ref{ifusers} \underline{carefully}.

\subsection{Saving your current user summary lists}

{\large\bf ACTION:} Before you do {\it anything} with this kit, it is {\bf
highly recommended} that you save copies of your existing user summary
lists, especially {\tt usernames.lis} (also save {\tt users.lis} and {\tt
people.adr} if you have them). Copy them to the \staradmin\ directory with
the extension {\tt .SAV}.

\begin{verbatim}
      % cd /star/local/staradmin
      % cp /star/local/admin/usernames.lis  usernames.lis.SAV
    ( % cp /star/local/admin/users.lis  users.lis.SAV         )
    ( % cp /star/local/admin/people.adr  people.adr.SAV       )
\end{verbatim}

{\large\bf ACTION:} In the \staradmin\ directory, make another copy of
{\tt usernames.lis} and point the environment variable {\tt
ADMIN\_USERNAMES\_SRC} at it. This copy will be used to build your UDB and
any changes that need to be made to update {\tt usernames.lis} should be
done to this copy.

\begin{verbatim}
      % cp /star/local/admin/usernames.lis  usernames.SRC
      % setenv ADMIN_USERNAMES_SRC /star/local/staradmin/usernames.SRC
\end{verbatim}

{\large\bf ACTION:} If you have selected the {\tt EXTENDED} UDB, you will
also have to point another environment variable, {\tt ADMIN\_USERS\_SRC}
at a copy of {\tt users.lis}. If you haven't already done so, this copy of
{\tt users.lis} should have been edited to the UCL format as described in
Appendix \ref{ifusers}. If you do not have a {\tt users.lis} (in any
format), then point this environment variable to an empty file.

\begin{verbatim}
      % cp <UCL format users.lis> users.SRC
or:
      % touch users.SRC

      % setenv ADMIN_USERS_SRC  /star/local/staradmin/users.SRC
\end{verbatim}


\subsection{Editing {\tt staradmin.conf}}

The distribution copy of {\tt staradmin.conf} contains default entries for
all keywords. By convention, all files and scripts are stored under the
{\tt /star/local} tree. The default location for this file is {\tt
/star/local/etc/staradmin.conf}.

\begin{quote}
  {\bf *** It is highly recommended that you use the defaults locations for
           files referred to in {\tt staradmin.conf} *** }
\end{quote}

{\large\bf ACTION:} There are \underline{three} keywords you must edit at
this point:

\begin{table}[ht]
\centering
\begin{tabular}{rl}
{\tt UDBLEVEL}   & Select {\tt BASE} or {\tt EXTENDED} \\
{\tt MYLOCATION} & Your site's 3-letter location code \\
{\tt MYLOC}      & Your site's 1-letter location code \\
{\tt EMAIL}      & Your site's default email address \\
\end{tabular}
\end{table}

Read the comments in the {\tt staradmin.conf} file to get an idea of all
the options available. Most sites will find that the default settings will
be best.

\staradmin\ looks for the configuration file in the default
location. If you want \staradmin\ to read an alternative configuration file,
use the environment variable {\tt ADMIN\_CONFIG} to point at the alternative
location.

\subsection{Editing {\tt killusers}}
\label{killusers}

{\tt staradmin} provides an option for cross-checking the UDB (or under certain
circumstances {\tt usernames.lis}) against the {\tt passwd} database, and
displaying any discrepancies.

Invariably there will be many system accounts in your {\tt passwd} database
(e.g. {\tt root, ftp, usenet} etc.), that do not (and should not) appear in
your UDB (or {\tt usernames.lis}).  The {\tt killusers} file provides a way of
masking out such accounts, so that only `real' users are cross-checked.

The {\tt killusers} file format is very simple. It can consist of a list of
usernames to be ignored, one per line, and it can have special lines that
include another file of the same format.

The distribution kit contains an example {\tt killusers} file that contains two
entries that {\tt :include:} other files. These files are {\tt sysusers} and
{\tt starusers}. The idea is that system accounts are placed in {\tt sysusers}
and special Starlink accounts are placed in {\tt starusers}. The sort of
accounts appropriate for {\tt starusers} are {\tt star}, visitor accounts,
programmer accounts etc. Typically accounts that appear in {\tt starusers} will
also appear in the file {\tt usernames.misc} and will be included at the top of
{\tt usernames.lis} (See Section \ref{usernames.misc}).

There are two versions of {\tt sysusers} provided in the kit. One contains
typical system accounts for Digital Unix ({\tt sysusers.alpha\_OSF1}), the
other for Solaris ({\tt sysusers.sun4\_Solaris}).  The build procedure in
Section \ref{sec:unpack} will have extracted the appropriate version for
your system.

If you need to use or see the other version, extract it from the source
archive:

\begin{verbatim}
      % cd /star/local/staradmin
      % tar xf staradmin_source.tar sysusers.alpha_OSF1
or
      % tar xf staradmin_source.tar sysusers.sun4_Solaris
\end{verbatim}

Note that some sites may have an account or two that do not appear in
the example files. It's much easier to wait until you run the
cross-checker for the first time in Section \ref{xcheck_first} to
complete the customisation of these files for your site, rather than
trawl through the {\tt passwd} database by hand at this stage. The
example files provide a good starting point.

The location of the {\tt killusers} file is defined in {\tt staradmin.conf}
and will be moved there during the installation of \staradmin. The default
location is {\tt /star/local/etc/killusers}. The example {\tt killusers}
expects both {\tt sysusers} and {\tt starusers} to be in the
{\tt /star/local/etc} directory also.

\subsection{Editing {\tt location\_local}}

The Starlink distributed file {\tt /star/admin/location} contains a long
list of unique location codes followed by a location postal address. These
location codes appear in the fifth field of your {\tt usernames.lis}.

In order to automate the generation of postal address lists (and sticky
labels) the optional file {\tt people.adr} can be produced. The postal
address associated with the location code for each user is placed after
each user's real name in {\tt people.adr}.

The file {\tt location\_local} is intended for system administrators to
override a location code entry found in {\tt /star/admin/location}. This
allows internal mail addresses to be substituted for full postal addresses
for local users.

The contents of {\tt /star/admin/location} are read in first, followed by
{\tt location\_local}. If a particular location code appears in both
files, the one in {\tt location\_local} overrides the one in {\tt
/star/admin/location}. If {\tt location\_local} contains more than one
occurence of a particular location code, the last one read will be the one
used.

{\large\bf ACTION:} The distribution example of this file contains the
copy of {\tt location\_local} used at UCL. You should remove the UCL local
entries from the end of this file and add local internal mail addresses
for your site at the end. Note that postal addresses in {\tt location} and
{\tt location\_local} are split onto separate lines at semi-colon
separators when the file is processed\footnote{Unless there are no
semi-colons at all in the address part, then the line is split at comma
separators. This fallback method is included to provide compatibility with
the old {\tt /star/admin/location} format.}.

For this first release of \staradmin, {\tt location\_local} contains a
fair few addresses for other sites. This is because the file {\tt
/star/admin/location} does not yet contain full postal addresses for any
location codes.

The location of {\tt location\_local} is defined in {\tt staradmin.conf}.
The default location is \newline {\tt /star/local/admin/location\_local}.

\subsection{Editing {\tt Aliases.header}}

The example {\tt Aliases.header} file shows how UCL manages its mailing
lists. Other sites may already do this in a different way. For sites who
have not centralised their local mailing lists this may be an appropriate
opportunity to do this.

On Unix machines running {\tt sendmail} and NIS, there are two places that
the mail system looks up mail aliases. There is the {\tt aliases} file in
the appropriate {\tt sendmail} directory, and there is the NIS distributed
map {\tt aliases} (source file {\tt mail.aliases}; on Digital Unix this
file is in {\tt /var/yp/src}). The \staradmin\ utility can be used to
centralise the management of local mail lists and mail aliases that are
distributed by NIS.

The example {\tt Aliases.header} file contains {\tt :include:} references
to files containing mail lists. In the distribution example from UCL,
there are {\tt :include:} references for lists specifically generated by
\staradmin, plus local entries that are maintained manually. For
convenience, all these lists are kept in a single directory ({\tt
/star/local/mail} is the default).

When \staradmin\ is used to update user summary lists, two operations are
performed that apply to the mail lists directory. First the
auto-generation of mailing lists from the UDB is done, then aliases based
on a users real name are generated. These real name aliases are appended
to the file {\tt mail.aliases} which has a copy of {\tt Aliases.header}
(minus comments) at the top.

The system administrator is free to name and specify which if any mail
lists are generated by \staradmin. They are configured through the file
{\tt starmail.conf}. See Section \ref{starmailconf} for details.

For convenience, at UCL we soft-link this file ({\tt
/star/local/mail/mail.aliases}) to the NIS source directory (e.g. {\tt
/var/yp/src}). It is then a simple matter to rebuild the NIS {\tt aliases}
map. The NIS aliases map can be rebuilt automatically by an appropriate
{\tt cron} entry (see Section \ref{cronalias}).

{\large\bf ACTION:} If you wish to follow the UCL example, then the
distribution examples will provide you with sufficient information to do
this for you site. Otherwise you may tell \staradmin\ to place the
auto-generated lists elsewhere via {\tt staradmin.conf}, or even switch
off the generation of mail-associated lists altogether (by commenting out
the relevant keywords in {\tt staradmin.conf}).

The location of the {\tt Aliases.header} file is defined in {\tt
staradmin.conf}. The default location is \newline {\tt
/star/local/mail/Aliases.header}.

\subsection{Editing {\tt starmail.conf}}
\label{starmailconf}

The example {\tt starmail.conf} file shows the UCL mailing list files and
the UDB information used to generate them. The format of the file is
explained fully in the example file.

{\large\bf ACTION:} You can configure as many mail lists as you like, and
call them what you like. Don't forget to modify/add the appropriate
entries in {\tt Aliases.header} so they are incorporated correctly in {\tt
mail.aliases}.

The default location for this file is {\tt /star/local/mail/starmail.conf}.
This can be overridden by defining the environment variable {\tt
ADMIN\_MAILCONF} to point to an alternative location.

\subsection{Editing {\tt usernames.misc}}
\label{usernames.misc}

This file is used to describe any miscellaneous user accounts you have set
up at your site that are not attributed to a single user, or are set up
for special purposes. The contents of this file (minus comments, but
including any blank lines) is included near the top of the auto-generated
{\tt usernames.lis}. The sort of usernames that end up here are the {\tt
star} account, visitor accounts, Mailclub accounts etc. Such accounts
should probably appear in your local {\tt killusers} file (or included
file, e.g. {\tt starusers}; see Section \ref{killusers}).

{\large\bf ACTION:} The distribution example has the UCL data for this
file. Edit this appropriately for your site by including the relevant
chunk from your original {\tt usernames.lis} file.

The location of {\tt usernames.misc} is defined in {\tt staradmin.conf}.
The default location is \newline {\tt /star/local/admin/usernames.misc}.

\subsection{Editing {\tt nextaccnum}}

This file is used to contain the number of the next Starlink Application
Form for your site. It must contain a four-digit number {\em e.g.}, {\tt
0348}. This number in {\tt nextaccnum} is read when adding a new user to
the UDB. If the user is successfully added, the number in {\tt nextaccnum}
is incremented by one.

{\large\bf ACTION:} Edit the distribution example and add the appropriate
number for your site.

The location of {\tt nextaccnum} is defined in {\tt staradmin.conf}.
The default location for this file is \newline
{\tt /star/local/admin/nextaccnum}.

\section{Installing the \staradmin\ kit}

Once you are happy with the modifications you have made to the example
distribution files, you can install the \staradmin\ kit.

{\large\bf ACTION:} You should be logged in as {\tt star}. Ensure the
environment variable {\tt INSTALL} is set to {\tt /star/local} ({\em not}
{\tt /star}) and {\tt SYSTEM} is set for your OS.

\begin{verbatim}
      % setenv SYSTEM <alpha_OSF1 OR sun4_Solaris>
      % setenv INSTALL /star/local
      % cd /star/local/staradmin
      % ./mk install
\end{verbatim}

Amongst the executable scripts and the configuration files and support
files already discussed, two {\tt man} pages ({\tt staradmin(8)} and {\tt
buildudb(8)}) are installed into {\tt /star/local/man/man8}).

\section{Building a UDB}

There are three stages to building a UDB:

\begin{enumerate}

\item Ensuring your {\tt usernames.\-lis} is up-to-date (and confirming
the format of {\tt users.lis} if you have selected the {\tt EXTENDED}
UDB).

\item Running a check on the UDB source files at {\tt ADMIN\_USERNAMES\_SRC}
and {\tt ADMIN\_USERS\_SRC}.

\item Building the UDB itself.

\end{enumerate}

\subsection{Checking your {\tt usernames.lis} against the {\tt passwd} database}
\label{xcheck_first}

{\bf N.B.} You must edit {\tt staradmin.conf} so that the keyword {\tt XCHECK}
has the value {\tt USERNAMES}. Once this is done you can cross-check a copy of
your original {\tt usernames.lis} with the {\tt passwd} database to  ensure
your {\tt usernames.lis} is up-to-date. It is assumed that your {\tt
usernames.lis} file is of the accepted format (you would have heard from Mike
Lawden by now if it was not!). The copy of {\tt usernames.lis} pointed to by
the environment variable {\tt ADMIN\_USERNAMES\_SRC} is used.

{\large\bf ACTION:} Run \staradmin\ in cross-check mode:

\begin{verbatim}
    % staradmin -x
\end{verbatim}

Typical output would look something like this:

\begin{small}
\begin{verbatim}
      Cross-checking usernames.lis against passwd file ...
      Reading usernames.lis ...
      No. users found: 194

      -----------------------------------------------------------
      + Usernames in user database, but not in passwd database

      ug_jwm          "James Manfield"
      ug_gajh         "Gaitee Hussein"
      ug_sg           "Sakura Gooneratne"

\end{verbatim}
\end{small}

There may be users listed under other headings as there are additional
{\tt passwd} health checks built into \staradmin.

{\large\bf ACTION:} You may also see some usernames appearing in this list that
are not appropriate to your {\tt usernames.lis} file. These should be placed in
the {\tt killusers} file or the relevant {\tt :include:} files you have chosen.
Section \ref{killusers} gives full details.

By default, the {\tt passwd} database is checked for entries with empty
password fields, whether the home directory specified exists (check can be
switched off), whether the shell is valid i.e. in {\tt /etc/shells}
\footnote{You can point to an alternative list of shells by pointing the
environment variable {\tt ETC\_SHELLS} at another file.}
(check can be switched off) and if logins have been disabled for the user
(check can be switched off).

{\large\bf ACTION:} There may also be a couple of users that appear in the {\tt
passwd} database but have not yet been put into {\tt usernames.lis}. Do that
right away.

Once you have updated your {\tt killusers} files and {\tt usernames.lis} if
necessary, run \staradmin\ in cross-check mode again until you are happy that
your {\tt passwd} database and {\tt usernames.lis} are up-to-date.

{\large\bf ACTION:} Before moving onto the next stage, you must change the
value of {\tt XCHECK}  in {\tt staradmin.conf} from {\tt USERNAMES} to {\tt
UDB}. This means that once your UDB has been built, further cross-checks
against the {\tt passwd} database will compare the contents of the UDB and not
{\tt usernames.lis}.

Now go to Section \ref{checkusers}.

\subsection{Checking {\tt ADMIN\_USERNAMES\_SRC} and {\tt ADMIN\_USERS\_SRC}}
\label{checkusers}

If you have chosen the {\tt EXTENDED} UDB option, then you should follow the
instructions in this section to check your edited {\tt usernames.lis} and
{\tt users.lis}. The program {\tt udbsrcchk} is used for this. It looks for
the files pointed to by {\tt ADMIN\_USERNAMES\_SRC} and {\tt
ADMIN\_USERS\_SRC}.

{\large\bf ACTION:} Run the UDB source file checking program:

\begin{verbatim}
      % udbsrcchk
\end{verbatim}

If everything is OK you'll get something that looks like the following:

\begin{verbatim}
      Checking UDB source files ...
      UDBLEVEL is set to EXTENDED
      Using usernames.SRC and
      users.SRC to check/generate new User Database
      No. users found in usernames.SRC: 193
      No. users found in users.SRC (Section 1): 193
      No. users found in users.SRC (Section 2): 193
\end{verbatim}

Otherwise read the warning and error messages carefully and re-edit the source
files and re-run the source file checker until the messages go away. Note that
the checker program isn't foolproof and it could get confused if your formats
are way, way out.

Once you're happy that your source files are OK, it's time to build your UDB.
Go to Section \ref{buildudb}.

\subsection{Building a UDB with {\tt buildudb}}
\label{buildudb}

To ease the job of the system administrator and assist the take-up of this kit,
there is a tool called {\tt buildudb} that generates your UDB automatically
from the contents of your UDB source files.

You should ensure that your {\tt passwd} database and {\tt usernames.lis} file
has been thoroughly checked and is up-to-date (Section \ref{xcheck_first}).

{\large\bf ACTION:} Before {\tt buildudb} will actually build a UDB, you must
set the following environment variable (no value required). This is a safety
catch to stop you inadvertently overwriting your UDB.

\begin{verbatim}
      % setenv PP_ENABLE
\end{verbatim}

{\large\bf ACTION:} To build your UDB, type the following and answer the
prompts with a `{\tt y}'. You are given two chances to bail out. If the target
UDB directory (defined in {\tt staradmin.conf}) does not exist you will be
asked if you want it created.

\begin{verbatim}
      % buildudb
\end{verbatim}

The utility will now extract user information from your UDB source files
{\tt ADMIN\_USERNAMES\_SRC} and {\tt ADMIN\_USERS\_SRC}
and populate your UDB directory.

\subsection{Generating the user summary lists}

{\large\bf ACTION:} Once you have built your UDB and are happy with the
settings in {\tt staradmin.conf}, you may run the update option to
automatically build your  user summary lists.

\begin{verbatim}
      % staradmin -u
\end{verbatim}

\section{Using \staradmin}

There is a {\tt man} page {\tt staradmin(8)} containing the information below.
For instant help, specify the {\tt '-h'} option:

\begin{verbatim}
      % staradmin -h
           staradmin V1.0  10-Oct-95   Adrian Fish (afish@star.ucl.ac.uk)
     Usage:
        staradmin <opts>
                -a <username>  - add user database entry
                -d <username>  - delete user database entry
                -m <username>  - modify user database entry
                -v <username>  - view user database entry
                -u             - update all user lists
                -x             - cross-check user database against passwd file
\end{verbatim}

\subsection{Adding a user to the UDB}
\label{addinguser}

The syntax for adding a user to the UDB is:

\begin{verbatim}
      % staradmin -a <username>
\end{verbatim}

\staradmin\ checks whether the username specified already exists. If it does
the program returns an error message.

Otherwise the administrator is prompted to  enter data for the user. Typing
{\tt '?'} at any time provides brief help on the data item being prompted for.
Some fields insist on some input and where possible a validity check is done on
the input value. Other fields can be left blank. Where possible an appropriate
default is displayed which can be entered by pressing {\tt <RETURN>}.

If a field has a default value, but you wish to enter an empty value to this
field, then type a {\tt <SPACE>}.

Once prompting has finished, all the fields are listed and the
administrator is asked whether anything needs to be modified. If the
answer is yes, all data is prompted for again (with defaults being the data
entered previously). If all data is correct, the administrator is prompted
to write the entry to the UDB.

\subsection{Deleting a user from the UDB}

The syntax for removing a user from the UDB is:

\begin{verbatim}
      % staradmin -d <username>
\end{verbatim}

If a given username does not exist, \staradmin\ returns an error message.
Otherwise the administrator is asked whether the appropriate UDB entry should
be removed.

Note that if the {\tt GRAVEYARD} variable in {\tt staradmin.conf} is set, the
UDB entry is moved to the directory defined there, rather than deleted.

\subsection{Modifiying a user entry in the UDB}

The syntax for modifying a user entry in the UDB is:

\begin{verbatim}
      % staradmin -m <username>
\end{verbatim}

If the given username does not exist, \staradmin\ returns an error message.
Otherwise the administrator is prompted to enter data for the user. Use the
same procedure as that described for adding a user in Section \ref{addinguser}.

\subsection{Viewing a user entry in the UDB}

The syntax for viewing a user entry in the UDB is:

\begin{verbatim}
      % staradmin -v <username>
\end{verbatim}

If the given username does not exist, \staradmin\ returns an error message.
Otherwise \staradmin\ lists the contents of the UDB entry. The administrator is
then asked if this UDB entry is to be modified. If the answer is {\tt 'y'}
then the procedure is as before in Section \ref{addinguser}.

\subsection{Updating the user summary lists}

The syntax for updating the user summary lists is:

\begin{verbatim}
      % staradmin -u
\end{verbatim}

The user summary files generated when executing this command depend on the
settings in {\tt staradmin.conf}.

\subsection{Cross-checking the UDB against the {\tt passwd} database}

The syntax for cross-checking the UDB against the {\tt passwd} database is:

\begin{verbatim}
      % staradmin -x
\end{verbatim}

\section{External Programs}

\subsection{{\tt mklabels}}

Currently only one example external program is configured to be run from
within \staradmin. This is the address sticky label post-processor. The
information in {\tt people.adr} is piped into the {\tt mklabels} program
to generate a printer ready file for spooling to a suitably stocked
printer. Note that the copy of {\tt mklabels} is only an example, and you
can specify an alternative program to be run at this point via {\tt
staradmin.conf}. This alternative external program could be a {\sf
FORTRAN} program.


\section{Other helpful tips}

\subsection{{\tt crontab} entry for the \staradmin\ cross-checker}

Here's a useful way of automatically letting {\tt star} know of any
discrepancies that may have crept into your UDB or {\tt passwd} database
in plenty of time to fix it before your {\tt usernames.lis} is copied to
RAL.

This works for a Digital Unix system and presumably for Solaris:

\begin{verbatim}
#
# Check UDB against passwd three days before usernames.lis is copied to RAL
# (usually on 14th of every month) and email results to star
#
52 0 11 * * /star/local/bin/staradmin -x | mailx -s "Staradmin -x" star
#
\end{verbatim}

\subsection{{\tt crontab} entry to update NIS {\tt aliases} map}
\label{cronalias}

Here's a {\tt crontab} entry for updating a NIS aliases map once a day.
It runs the example shell script {\tt mkpush\_aliases} (comes with the
\staradmin\ kit).

\begin{verbatim}
#
# Build and push the NIS aliases map once a day
#
32 1 * * * sh /star/local/staradmin/mkpush_aliases > /dev/null 2>&1
#
\end{verbatim}

%==============================================================================

\appendix
\newpage
\section{The {\tt EXTENDED} UDB and {\tt users.lis}}
\label{ifusers}

If you have a {\tt users.lis} file you would like to incorporate into the
UDB by choosing the {\bf\tt EXTENDED} option, then read this section very
carefully. The problem with {\tt users.lis} is that the sites that do have
such a file use wildly differing formats. UCL's format is currently the
only supported format for building an {\tt EXTENDED} UDB and regenerating
a {\tt users.lis}.

To use another site's {\tt users.lis} would require major hacking of the
scripts. Experience in beta-testing has shown that such hacking would
require much more time than a system administrator would spend manually
reformatting an existing {\tt users.lis} to the UCL format. Remember that
the building of the UDB need only be done once.

So the recommended method is to edit your {\tt users.lis} to the UCL
format.  The kit contains an example UCL format file ({\tt users.eg.UCL}).

Some sites may feel this is too much work. In that case, they should
select the {\tt BASE} UDB.

If you have a small number of users and you intend to manually enter the
{\tt EXTENDED} UDB information, then when selecting {\tt EXTENDED} in {\tt
staradmin.conf}, also point the environment variable {\tt
ADMIN\_USERS\_SRC} to an empty file.

The origin of a Starlink {\tt users.lis} is a common one, but over the
years the format has `evolved' at those sites who maintained one (this
tends to be the older sites). The information in these files also varies.
But basically they are of the same general format. There are at least two
sections of user data.

The first section occurs after a header containing summary information and
starts immediately after a line starting with two spaces and three {\tt
"+"}s. In this first section there is one 80-column line per user sorted
in alphabetical order keyed on surname. Some of the information in these
lines is also found in {\tt usernames.lis}. This section is relatively
easy to format to the UCL style. The most important thing is to line up
the {\bf columns} correctly. This section ends at a line consisting of all
{\tt "*"}s.

The second section contains further information about a particular user
and starts immediately after a line of {\tt "*"}s. The entries for each
user are sorted by account number. Each entry is also three lines long.
This section is usually the most difficult bit to maintain.

There may be a third section that contains postal addresses and user
breakdowns by site. The information in these sections, if they exist, has
moved to other files under the \staradmin\ scheme and will be ignored.

To copy the UCL format of {\tt users.lis} then you must be precise, the
first item in the following list being very, very important, and
including:

\begin{itemize}
\item !!!! Columns must line up {\bf exactly} in {\bf both} sections !!!!
      \newline {\tt buildudb} splits on columns, {\it not} whitespace.
\item Strip all tabs
\item First section starts after a line beginning {\tt "\hspace{2em}+++"} (note
      \underline{two} spaces before first {\tt +}).
\item No blank lines between the first section and the line of {\tt"*"}s
\item Second section entries are three lines long, and must be separated by
      a \underline{blank} line (i.e. no spaces or tabs)
\item In the second section, firstnames and title must be separated by a
      comma
\item In the second section, room number and telephone number must be
      separated by a comma
\item Remove any data in a third section or separate from second section
      with a long line of {\tt "*"}s.
\end{itemize}


Experience has shown that the second section of {\tt users.lis} can cause
the most problems. Here's an example of a UCL {\tt users.lis} entry (80
columns wide):

\begin{small}
\begin{verbatim}
--------------------------------------------------------------------------------
185     FISH            Adrian C, Dr            AFISH                   UCLL0185
                UCL     E24, x7147                      Indef             Jun 84
                Site Manager                            Sup: Adrian Fish
--------------------------------------------------------------------------------
\end{verbatim}
\end{small}

Here's a breakdown of the elements of each line:

{\bf Line 1}

\begin{center}
\begin{tabular}{ll}
{\tt 185}          & Application form number\\
{\tt FISH}         & Surname (uppercase)\\
{\tt Adrian C, Dr} & Firstnames and title, separated by a comma (v.imp)\\
{\tt AFISH}        & Username (uppercase); required by {\tt buildudb}\\
{\tt UCLL0185}     & Full account code formed from 3-letter Starlink \\
                   & site-code followed by 1-letter Starlink site code, \\
                   & then the application form number \\
\end{tabular}
\end{center}

{\bf Line 2}

\begin{center}
\begin{tabular}{ll}
{\tt UCL}  & Location code (from {\tt /star/admin/location}) \\
{\tt E24, x7147} & Room Number and Telephone (max. 28 chars) \\
{\tt Indef}  &  Account expiry date \\
{\tt Jun 84}  & Date account was added to the system \\
\end{tabular}
\end{center}

{\bf Line 3}

\begin{center}
\begin{tabular}{ll}
{\tt Site Manager}  & User's plan. Make sure this is not long enough to \\
                    & invade the {\tt Sup:} field \\
{\tt Sup: Adrian Fish} & {\bf Optional field} - contains PostGrad's\\
                       & supervisor if applicable. Whole field, \\
                       & (including Sup:) should be left {\bf blank} \\
                       &  if no supervisor \\
\end{tabular}
\end{center}

Lining up the columns is essential. Use the example UCL copy distributed
with the \staradmin\ kit ({\tt users.eg.UCL}) as a template.

\section{Summary Lists}
\label{summarylists}

Here is a list of all the files that \staradmin\ is able to generate
automatically.

\begin{itemize}
\item {\Large\tt usernames.lis}

Required by Starlink. This file is the primary source of information for the
Starlink wide user summaries. Contains six fields of information.

\item {\Large\tt users.lis}

This file supplements the information contained in {\tt usernames.lis}. It can
be used as a source for an alternative {\tt finger} scheme.

\item {\Large\tt people.adr}

This file contains users and their postal addresses. This file is used as
input for a labels processor. The reason this format is maintained is for
compatibility with existing sticky label processing programs.

\item {\Large\tt labels.lis}

Output sticky labels formatted names and addresses. Produced by a program
external to \staradmin\ that takes {\tt people.adr} as its input.

\item {\Large\tt astropersons.lis}

By popular demand, this file contains user data in a form acceptable for
inclusion in the RGO distributed {\tt astropersons.lis}. It is assumed that the
compilers of the list will add the appropriate `last updated' code to this
file.

\item {\Large\tt summary.lis}

Contains a breakdown of users based on category and location.

\item {\Large\tt phone.book}

This file is a handy list of users, usernames, room numbers and phone numbers.
Some sites may already access this information via {\tt finger} and the {\tt
passwd} GECOS field. This file is provided by \staradmin\ as an alternative
for sites that may not do this already, or do not want to use {\tt finger} for
security reasons.

Only primary users are included in the phone book and then only if they have a
non-empty {\tt Phone:} field in their UDB file.

\item {\Large\tt mail.aliases}

This file makes use of the results of two operations by \staradmin. The first
action of \staradmin\ is to generate a set of mailing lists based on UDB data
(as defined in {\tt starmail.conf}). Include instructions for these mailing
lists are placed manually in
the file {\tt Aliases.header}. The second action of \staradmin\ is to copy the
contents of {\tt Aliases.header} to {\tt mail.aliases} and append a list of
mail aliases for each user in the UDB based on their real name.

\end{itemize}

\section{Example of a UDB file}

Here is an example listing of the contents of a UDB file:

\begin{verbatim}
    Username:    afish
    Surname:     Fish
    Firstname:   Adrian
    Category:    t
    Location:    UCL
    Email:       afish@star.ucl.ac.uk
    Firstnames:  Adrian C
    Title:       Dr
    Status:      N
    Funding:
    Account:     UCLL0185
    Added:       Jun 84
    Expiry:      Indef
    Plan:        Site Manager
    Supervisor:
    Phone:       x7147
    Room:        E24
    Created:     24-Aug-1995 13:04
    Modified:    25-Aug-1995 09:30
\end{verbatim}

Each line consists of a keyword followed by some data.
Note that there are some keywords that may not be used at your site.
If {\tt UDBLEVEL} is set to {\tt BASE} in {\tt staradmin.conf} then
the administrator only ever sees the top {\bf six} entries when using
\staradmin.

The other keywords are used if you select the {\tt EXTENDED} UDB.
The reason for including all these extra keywords even if the {\tt BASE}
level is selected, is to protect any data that may have been entered
previously at the {\tt EXTENDED} level, but the administrator has for some
reason gone back to the {\tt BASE} level.

The file {\tt starudb.conf} contains a complete list of keywords used in the
UDB and defines various properties for each keyword. This file should
{\bf not} be modified.

\subsection{The keywords explained}

The following brief explanation of each keyword is also available when
manipulating a UDB entry interactively with \staradmin\ by typing {\tt "?"} at
any prompt.

\begin{tabular}{ll}
{\tt Username}    & Login id \\
{\tt Surname}     & User's family name \\
{\tt Firstname}   & User's firstname, usually the familiar name \\
{\tt Category}    & User category, one of {\tt r t o f a u *} \\
{\tt Location}    & Unique location code ({\tt /star/admin/location}) \\
{\tt Email}       & Email address for this user \\
{\tt Firstnames}  & Proper firstname and initials (c.f. Firstname)\\
{\tt Title}       & e.g. Dr, Prof, Ms, Mr \\
{\tt Status}      & Whether primary on-site, primary off-site, secondary or RUG\\
{\tt Funding}     & If user is externally funded, value is {\tt E}. Useful for AO bids \\
{\tt Account}     & Account code as it appears on Starlink Application Form \\
{\tt Added}       & Date user was added to system (month and year) \\
{\tt Expiry}      & Date this account will expire (Indef acceptable) \\
{\tt Plan}        & User's main research interests \\
{\tt Supervisor}  & PostGrads supervisor's name. \\
{\tt Phone}       & Phone number (max 28 chars) \\
{\tt Room}        & Office number \\
{\tt Created}     & Used by \staradmin. \\
{\tt Modified}    & Used by \staradmin. \\
\end{tabular}

\end{document}
