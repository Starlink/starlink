\documentstyle[11pt]{article}
\pagestyle{myheadings}

%------------------------------------------------------------------------------
\newcommand{\stardoccategory}  {Starlink System Note}
\newcommand{\stardocinitials}  {SSN}
\newcommand{\stardocnumber}    {40.8}
\newcommand{\stardocauthors}   {K.\ Shortridge \& H.\ Meyerdierks}
\newcommand{\stardocdate}      {2 January 1992}
\newcommand{\stardoctitle}     {Figaro Release Notes\\
                                \bigskip
                                Standard and National Figaro 3.0-3}
%------------------------------------------------------------------------------

\newcommand{\stardocname}{\stardocinitials /\stardocnumber}
\renewcommand{\_}{{\tt\char'137}}     % re-centres the underscore
\markright{\stardocname}
\setlength{\textwidth}{160mm}
\setlength{\textheight}{230mm}
\setlength{\topmargin}{-2mm}
\setlength{\oddsidemargin}{0mm}
\setlength{\evensidemargin}{0mm}
\setlength{\parindent}{0mm}
\setlength{\parskip}{\medskipamount}
\setlength{\unitlength}{1mm}

%------------------------------------------------------------------------------
% Add any \newcommand or \newenvironment commands here
%------------------------------------------------------------------------------

\begin{document}
\thispagestyle{empty}
SCIENCE \& ENGINEERING RESEARCH COUNCIL \hfill \stardocname\\
RUTHERFORD APPLETON LABORATORY\\
{\large\bf Starlink Project\\}
{\large\bf \stardoccategory\ \stardocnumber}
\begin{flushright}
\stardocauthors\\
\stardocdate
\end{flushright}
\vspace{-4mm}
\rule{\textwidth}{0.5mm}
\vspace{5mm}
\begin{center}
{\Large\bf \stardoctitle}
\end{center}
\vspace{5mm}

%------------------------------------------------------------------------------
%  Add this part if you want a table of contents
\setlength{\parskip}{0mm}
\tableofcontents
\setlength{\parskip}{\medskipamount}
\markright{\stardocname}
%------------------------------------------------------------------------------

\newpage
\section{About this document}

The main body of this document is formed by the release notes as
issued by Keith Shortridge (AAO) with Figaro 3.0. Some additional
information on the actual release of (Standard) Figaro within Starlink
should however be included. Also some release notes relating to the
Figaro national directory and to National Figaro (updates to Standard
Figaro) should be given.

\begin{itemize}

\item Section \ref{AAOnotes} contains the original Figaro Release Notes ---
Version 3.0.

\item Section \ref{Starnotes} describes the actual implementation at Starlink
sites.

\item Section \ref{natfig} describes National Figaro 3.0-3, which includes the
ICL monolith for Figaro.

\item Section \ref{develop} contains notes for users who want to develop their
own Figaro applications, especially if they want to compile their own monolith.

\item Section \ref{update} describes how to update from version 3.0-2 to 3.0-3.
(3.0-2 means Standard Figaro 3.0 and National Figaro 3.0-2. Standard Figaro
remained unchanged while National Figaro 3.0-1 and 3.0-2 were released.)

\end{itemize}

\section{Figaro release notes---version 3.0}
\label{AAOnotes}

\begin{quote}\em

This section was originally in RUNOFF format and was issued as SSN/40.7,  dated
28 May 1991. It has been converted to \LaTeX\ for this issue of the document.
The original format style of SSN/40.7 has been retained within the constraints
of \LaTeX\ formatting, but no changes have been made to the text within this
section, except for the removal of the first paragraph concerning `Change
bars'.

\end{quote}

\subsection{Introduction}

This note describes how to install the Figaro data reduction system. It is
designed to accompany a release tape containing the various Figaro directories
in VMS Backup format. The tape contains two save sets. The first is called
FIGARO3.0, the second is called STARBITS.BCK. FIGARO3.0 contains all the files
that are specific to Figaro; STARBITS.BCK contains those parts of Starlink
software that are used by Figaro and can be supplied in this way (that is, it
does not include anything that is in any proprietary). Starlink sites can
ignore the second save set, while non-Starlink sites may find they need some
of the files it contains.

You can choose to install Figaro at one of three levels, depending on the use
that will be made of the system and the disk space available.

\begin{enumerate}

\item If you only want to run Figaro as it stands, you can copy from tape
only those files actually needed for the execution of the Figaro commands,
together with the various documentation files and help libraries.

\item If you do not need to change any of Figaro, but want users to be able to
develop their own programs that run as local extensions to Figaro, you need to
copy the development directory and the various libraries as well as the files
needed for actual execution.

\item If you want the whole system available in source form, you can copy the
entire contents of the release tape onto disk.
\end{enumerate}

The detailed descriptions of how to install each of these levels follow. Each
level is described separately, in what is intended to be a self-contained way.
If an old version of Figaro is being replaced by the new version, then the
simplest thing is just to completely delete the old version before installing
the new one. Of course, this means that any local changes made to the system
will be lost, which is one reason that any such changes should be made in a
local directory, as discussed in more detail later.

\subsection{Using this installation guide}

This guide is quite complex. This is because Figaro has tended more and more to
make use of standard Starlink software, which means that procedures for
installing it now differ considerably between Starlink and non-Starlink sites.
The following may help you navigate through it:

Read the introductory section on Version 3.0.

Starlink site considerations:
\begin{description}
	\item Are you a Starlink site?
\end{description}
If you are then:

Read the section on HDS.

Otherwise (you are not a Starlink site) then:
\begin{description}
	\item Do you have the NAG library?
	\begin{description}
		\item If not then:
		\begin{description}
			\item You will need it to link and develop Figaro
programs. Read the section on `NAG'
		\end{description}
	\end{description}
	\item Do you already have Starlink's version of SGS/GKS?
	\begin{description}
		\item If not then:
		\begin{description}
			\item Do you want to use the old version of PGPLOT
instead?
			\begin{description}
				\item If so then:
				\begin{description}
					\item Think again. The GKS version is
the standard one now for Figaro. However, if you must use the old PGPLOT, read
the section on `Using the old PGPLOT'.
				\end{description}
				\item Otherwise (using GKS):
                         	\begin{description}
					\item Get it from Starlink, and
install it according to the release notes that come with it. The section in
these notes on SGS/GKS may help a little.
				\end{description}
			\end{description}
		\end{description}
	\end{description}
	\item Do you already have the latest versions of HDS, PGPLOT/GKS etc?
	\begin{description}
		\item If not then:
		\begin{description}
			\item Read the section on `Starlink Software at
non-Starlink sites', and install the software supplied in the second save set
on the tape.
		\end{description}
	\end{description}
\end{description}

End starlink site considerations.

System version considerations:
\begin{description}
	\item Do you want to install the minimum (execute only) version of
Figaro?
	\begin{description}
		\item If so, then:
		\begin{description}
			\item Follow the instructions in the section
`Installing the running version'
		\end{description}
	\end{description}
	\item Alternatively, do you want to install a system that will allow
Figaro programs to be developed, but does not have all the source files?
	\begin{description}
		\item If so then:
		\begin{description}
			\item Follow the instructions in the section
`Installing a development system'
		\end{description}
	\end{description}
	\item Alternatively, do you want to install the entire system?
	\begin{description}
		\item If so then:
		\begin{description}
			\item Follow the instructions in the section
`Installing the entire system'
		\end{description}
	\end{description}
\end{description}
End system version considerations.

That covers all the bits you NEED to read to install Figaro. All the other
sections are for background information, and should only be needed if you have
trouble with the system. It would be a good idea to read the section on
logical names used by Figaro, and check that they are all defined properly on
your system.

\subsection{Version 3.0}

All Figaro users should
read FIGARO\_PROG\_S:CHANGES.MEM to see what is new in this release.
Their system managers should try to encourage them to do so.

Figaro version 3.0 is the first Figaro release in which all the major
applications have been re-coded to use the DSA subroutine library and so will
support both the traditional Figaro data file format (known generally as `DST'
format, and with the usual file extension of `.DST') and Starlink's new NDF
format (whose files generally have the extension `.SDF'). More details of how
this works can be found in the documentation supplied with Figaro 3.0, in
the file FIGARO\_PROG\_S:FIGARO.DOC under the topic VERSION 3.0.

To the installer of the system, the only question posed by this new facility is
that of the correct setting for the logical name FIGARO\_FORMATS which controls
the default file format(s) used. This can be set most conveniently in the file
FIGARO\_PROG\_L:FIGARO.COM after the installation is completed. Leaving it
undefined leads to roughly the same behaviour as obtained by earlier versions
of Figaro.

Generally speaking, you should:

\begin{enumerate}

\item Leave it undefined if you do not expect to have to deal with NDF format
files at all. Figaro will still be able to deal with such files if it
encounters them, but the .SDF extension will always have to be specified
explicitly. New files will always be produced in DST format. You get the same
effect by defining FIGARO\_FORMATS as ``DST''.

\item Define it as ``NDF'' if you do not expect to have to deal with DST
format files at all. Figaro will still be able to deal with such files if it
encounters them, but the .DST extension will always have to be specified
explicitly. New files will always be produced in NDF format.

\item Define it as ``DST,NDF'' if you expect to have to handle a mixture of
the two file formats, but expect most use to be made of DST format files. Now
Figaro will look for files of either extension if no explicit file extension
is specified, but will give precedence (after a warning message in cases of
ambiguity) to .DST files. New files will be produced in DST format.

\item Define it as ``NDF,DST'' if you expect to have to handle a mixture of
the two file formats, but expect most use to be made of NDF format files. Now
Figaro will look for files of either extension if no explicit file extension
is specified, but will give precedence (after a warning message in cases of
ambiguity) to .SDF files. New files will be produced in NDF format.

\end{enumerate}

Note that users can define FIGARO\_FORMATS for themselves, either directly from
DCL or in their personal FIGARO\_PROG\_U:FIGARO.COM files.

The only other change that relates directly to installation of Figaro 3.0 is
the fact that Figaro 3.0 now uses Starlink's GNS library to deal with graphics
device names for GKS. Sites that are already using this will not have to
do anything new, but sites that are not will have to install it (there is a
copy included in the second save set of the release tape), and should note
that the section on SGS/GKS in these notes has changed significantly (being now
much simpler) from the Figaro 2.4 version.

Other changes worth noting are the support added to many applications
for data quality and error arrays, the new ability to specify parameters as
lists taken from files using an `@' syntax, which allows an application to
run repeatedly without the overheads of image activation, and some of the new
and modified applications; IMAGE has been changed significantly, some
optimal spectral extraction routines have been included, etc.

\subsection{Version 2}

This section is repeated from the original version 2.1 release notes. It is
included mainly for sentimental reasons.

A number of changes have been made to Figaro in the move from version 1 to
version 2, and whoever is responsible for Figaro at a particular site should be
aware of them, in order to be able to fend off the complaints from the users,
very few of whom can ever be relied on to read documentation for themselves.
What they ought to be reading is the section in the main Figaro documentation
(FIGARO.DOC) headed VERSION\_2, or getting the same information by typing HELP
FIGARO VERSION\_2.

Briefly, the major changes are as follows:

\begin{itemize}
\item The actual disk format used has changed. The used are unchanged, but the
actual disk files used are now HDS files, accessed through Starlink's HDS
(Hierarchical Data System). This means that Version 2 Figaro programs cannot
read Version 1 files (and vice versa), and so all existing files must be
converted. Anyone who gets the message `file not created by this data system'
when trying to read a file is trying to read a file in the wrong format.

Version 1 files may be converted to version 2 format by the utility program
DTA2HDS. If anybody needs to send version 2 files to a site still running
version 1 of Figaro, they should convert them to version 1 format first. The
utility program HDS2DTA converts version 2 format files to version 1 format.
Note that sites still on version 1 will NOT have this program available. (In
emergencies, they could be sent a copy of FIGARO\_PROG\_S:HDS2DTA.EXE, which is
a self-contained program, but they should not be sent the source and expected
to be able to re-link it themselves, because its linking is somewhat complex.)

Note that anyone just using the standard Figaro programs should notice no
difference in the way they run, except for this need to convert data files.
People writing their own need to be aware of some other changes.

\item The format of the parameter files (the .PAR files) has changed. Anyone
who has written their own Figaro programs will have to reprocess their
connection files using the CREPAR utility.

\item The directory structure used for Figaro development has changed. The
[FIGARO.PICT] directory has gone, being replaced by [FIGARO.DEV] and its
assorted sub-directories. The logical name FIGARO\_PROG\_D is being phased out,
being replaced by FIGARO\_DEV (both should point to [FIGARO.DEV]). The operation
of the linking procedure FIG.COM has changed very slightly; it now looks at the
respective dates of source and object files (if any) before deciding whether or
not to recompile, and its use of the debugger has improved marginally.

\item A new system, Callable Figaro, is now available for use as an alternative
to coding sequences of Figaro commands as DCL procedures. Users should be
encouraged to try this, and should be pointed towards the command `HELP FIGARO
CALLABLE'.

\end{itemize}

\subsection{Packages used}

Anyone installing Figaro will probably find it useful to know what
external subroutine packages (as opposed to those written specifically
for it) Figaro uses.

The version of Figaro as supplied uses the NAG mathematical subroutine library.
An earlier version used the IMSL package, and the source files for the earlier
versions of the Figaro routines concerned are included---although the IMSL
package is not. These versions are now rather out of date, and will not be
included in the next release. NAG and IMSL are both commercial packages that
have to be obtained separately.

Following the release of version 2.4, Figaro became very heavily dependent on
the GKS (Graphics Kernel System) package. GKS 7.2, which is what Figaro uses,
is an international graphics standard. The version of GKS 7.2 that is intended
to be used is that produced by Rutherford Appleton Lab in England (the home of
Starlink). This is not included in the Figaro release tape, and any site that
does not have it should contact Starlink at RAL directly for it.  Although it
can be supplied free to academic establishments, it is not supposed to be given
away freely to just anyone on things like this Figaro  tape. Starlink also
provide a package called SGS (Simple Graphics System) which provides a simpler
interface to GKS (which is notoriously difficult to use directly), and a
package called GNS which is used to simplify the somewhat tortuous way nice
friendly device names (like IKON1) are  turned into the identifiers needed by
GKS. These release notes assume that GKS, GNS and SGS are already installed,
although SGS and GNS are provided  on the release tape in the directories
[FIGARO.STARLINK.SGS] and [FIGARO.STARLINK.GNS] (second save set), for sites
that do not have them.

There are other implementations of GKS 7.2, and in principle Figaro  should
work with them; this has not been tried, so far as is known; the main problem
is that a system like Figaro that uses separate images tends to have to start
up GKS afresh for each command, and this normally clears the display screen.
The RAL version prevents this, but it is using a non-standard extension to do
so.

Despite the emphasis placed on SGS/GKS in the previous paragraph, very few
Figaro applications in fact call SGS/GKS routines directly. The imaging
routines FINDSP and OVERPF both call GKS directly, and SOFT and HARD  call as
SGS device name routine. Most Figaro applications do all their line graphics
through calls to a graphics subroutine package developed at  Caltech, called
PGPLOT. PGPLOT as used at Caltech was based on the low level  graphics package
GRPCKG. In 1987, Dave Terrett from Starlink, and Tim Pearson at Caltech (the
original author of PGPLOT), collaborated to produce  a version of PGPLOT that
used GKS 7.2 instead of GRPCKG. It is that version of PGPLOT that is now used
by Figaro, and with which all the  routines have been linked. This GKS version
of PGPLOT is included on the Figaro release tape, in the directory
[FIGARO.STARLINK.PGPLOT] (second save  set), for the benefit of non-Starlink
sites which do not already have it.

A discussion of what is involved in using a non-GKS version of PGPLOT can be
found in a later section of these notes.

The imaging routines in Figaro use an image display package called  TVPCKG.
This originated at Caltech, where it only supported a Grinnell GR270.  A new
version of TVPCKG was produced at AAO, which only supported the  ARGS displays
available at AAO, and this was distributed first with Figaro 2.3. Figaro 3.0 is
released with a version of TVPCKG produced at the Australian National
University, which can handle a number of different  displays, including the
Grinnell, ARGS and IKON. It also has support for a VaxStation, and for a
Macintosh II running the MacImager program (available from AAO---but with no
guarantees---to anyone interested). It can also handle a RAMTEK and an AED
display, but the routines it uses for this are dummied in the version included
in this release. The TVPCKG supplied with 3.0 has not yet been tested with a
Grinnell, so there may be some problems  associated with it on that device. Any
version of TVPCKG could be linked with the 3.0 applications, except that it
will have to supply versions of two new routines: TVSIZE and TVCSIZE, which
return the display size and the character size for the device. The ARGS
routines use an early  Starlink package (ARGSLIB) for some of their
communication with the ARGS. This Starlink software is supplied on the Figaro
release tape in [FIGARO.STARLINK.ARGS] (second save set), for the benefit of
those sites that do no already have it.

Two Figaro programs, STARIN and STAROUT, use the Starlink interim environment
routines. This Starlink software is supplied on the Figaro release  tape in the
second save set  in [FIGARO.STARLINK.INTERIM].

Many routines use the VMS screen management routines (the SMG\$ package).  If
these routines are to work properly on a non-DEC terminal, a capabilities file
for the terminal will be needed. The RTL manual  describes how to prepare such
a file (section 3.10 of the VMS 4.0 manual), and this is relatively
straightforward---and worth doing anyway, if you have such terminals. If this
proves difficult, the SMG usage is controlled by the graphics dialogue routines
(the GKD routines). It is possible to set them up for any particular terminal
type so that they do not use SMG. See the GKD package for details.

Access to disk data files by all Figaro programs is through a set of routines
called the DTA\_ package. Originally a stand-alone package, for version 2 of
Figaro this was re-written to use Starlink's HDS (Hierarchical Data System)
package for the actual disk access. A version of HDS is included on the Figaro
release tape, for users who do not have this package, in [FIGARO.STARLINK.HDS]
(second save set). This is a straight copy of the standard Starlink version.
HDS is the one package used by Figaro where the version used under VMS version
4 will not work under VMS version 5. For this reason, a second copy of the HDS
shareable image is included on the release tape. This is a pre-release copy of
the latest version, and works under VMS 5, but it is a pre-release copy and so
is something of an interim measure. The source for this version is not
included. If anyone has trouble with this,  they should contact the AAO for
help. The old version of the DTA\_ package is still included on the release
tape and in principle Figaro programs can be linked with either version.
However, the old version is only supplied for use with the format conversion
routines and should not normally be used.

Tape I/O is mainly performed using the Caltech tape I/O package MTPCKG. One
routine, WIFITS, makes use of a subroutine package called FIT---which is
supplied as part of Figaro---and for version 2.4 this was re-coded to use
Starlink's TIO routines; these provide better diagnostics when a tape runs off
the end of the reel, which has proven a regular occurrence with WIFITS!  Again,
for non-Starlink sites that do not have this package, it is  supplied with this
release in [FIGARO.STARLINK.TAPEIO].

\subsection{Starlink software at non-Starlink sites}

Anyone installing Figaro at a Starlink site need not read this section.  You
will already have all the files installed and logical names defined.

All the Starlink software needed to run Figaro (with the exception of GKS and
NAG---see the section on GKS for further details) has been included  in the
second save set on the release tape (STARBITS.BCK). When the  release was
prepared, these were put into the directory [FIGARO.STARLINK...].  following
description assumes that you do not have any of this software already on your
machine; if you do, you may want to modify the following procedures slightly.

As with the main Figaro system, you can choose to install 1) only those files
needed to run Figaro (essentially the shareable images), 2) the files needed to
run Figaro plus those needed to develop new Figaro programs (the shareable
images and the linker control files), or 3) all the files provided.

\begin{enumerate}

\item Decide where you want the files to go. If you just want the files  needed
to run Figaro, or those plus those needed for Figaro development, these few
files are most easily kept in a single directory, which has been called
disk:[STARBITS] in this description. If you want all the files, then it is best
to maintain the original directory structure, and in this description they are
shown put into a structure disk:[STARBITS...]. If you don't like that, you
should substitute whatever you prefer. It doesn't really matter where things
go, so long as the logical names point to them properly. The files will take
just under  1,000 blocks for the running files, just over 1,200 blocks for the
running and development files, and just under 5,000 blocks for all the files.

\item Create the base directory to be used and set it as the default directory.
You will need some privilege to do this, since it involves writing into the
[000000] directory of the disk.

\begin{verbatim}
      $ CREATE/DIRECTORY disk:[STARBITS]
      $ SET DEF disk:[STARBITS]
\end{verbatim}

\item Select the magnetic tape to be used for the backup. The following
description uses the logical name `tape'. You should substitute the  correct
name. Allocate the drive, physically mount the tape on the drive, then use the
MOUNT command to mount the tape as a foreign tape:

\begin{verbatim}
      $ ALLOCATE tape
      $ MOUNT/FOR tape
\end{verbatim}

\item Backup the required files from the tape. The details of the backup
depend on just which files you want.

\begin{enumerate}

\item For the running files only:

\begin{verbatim}
      $ BACKUP/LOG tape:STARBITS.BCK/SELECT=(-
              [FIGARO.STARLINK.ARGS]ARGSSHAR.EXE,-
              [FIGARO.STARLINK.PGPLOT]GRPSHR.EXE,-
              [FIGARO.STARLINK.INTERIM]INTERIM.EXE) *
\end{verbatim}

\item For the running files and the development files:

\begin{verbatim}
      $ BACKUP/LOG tape:STARBITS.BCK/SELECT=(-
              [FIGARO.STARLINK.ARGS]ARGSSHAR.EXE,-
              [FIGARO.STARLINK.PGPLOT]GRPSHR.EXE,-
              [FIGARO.STARLINK.INTERIM]INTERIM.EXE, -
              [FIGARO.STARLINK.PGPLOT]GRPSHR.OLB,-
              [FIGARO.STARLINK.INTERIM]INTERIM.OLB,-
              [FIGARO.STARLINK.SGS]SGSLINK.COM,-
              [FIGARO.STARLINK.SGS]SGS.OLB,-
              [FIGARO.STARLINK.TAPEIO]TAPEIO.OLB,-
              [FIGARO.STARLINK.ARGS]ARGSOPT.OPT)
              *
\end{verbatim}

\item For all the files:

\begin{verbatim}
      $ BACKUP/LOG tape:STARBITS.BCK/SELECT=[FIGARO.STARLINK...] -
                                            [STARBITS...]
\end{verbatim}

\end{enumerate}

\item Now, define the logical names needed. If all users are to be able to use
this software, these definitions should all have a /SYSTEM added, and should
also be added to the system startup file. The precise definitions depend on
which files have been backed up.

\begin{enumerate}

\item For the running files only:

\begin{verbatim}
      $ DEFINE ARGSSHAR disk:[STARBITS]ARGSSHAR.EXE
      $ DEFINE GRPSHR   disk:[STARBITS]GRPSHR.EXE
      $ DEFINE INTERIM  disk:[STARBITS]INTERIM.EXE
\end{verbatim}

\item For the running and development files only:

\begin{verbatim}
      $ DEFINE ARGSSHAR    disk:[STARBITS]ARGSSHAR.EXE
      $ DEFINE GRPSHR      disk:[STARBITS]GRPSHR.EXE
      $ DEFINE INTERIM     disk:[STARBITS]INTERIM.EXE
      $ DEFINE ARGSOPT     disk:[STARBITS]ARGSOPT.OPT
      $ DEFINE INTERIM_DIR disk:[STARBITS]
      $ DEFINE LIBDIR      disk:[STARBITS]
      $ DEFINE SGS_DIR     disk:[STARBITS]
      $ DEFINE PGPLOT_DIR  disk:[STARBITS]
\end{verbatim}

\item For the whole system:

\begin{verbatim}
      $ DEFINE ARGSSHAR    disk:[STARBITS.ARGS]ARGSSHAR.EXE
      $ DEFINE GRPSHR      disk:[STARBITS.PGPLOT]GRPSHR.EXE
      $ DEFINE INTERIM     disk:[STARBITS.INTERIM]INTERIM.EXE
      $ DEFINE ARGSOPT     disk:[STARBITS.ARGS]ARGSOPT.OPT
      $ DEFINE INTERIM_DIR disk:[STARBITS.INTERIM]
      $ DEFINE LIBDIR      disk:[STARBITS.TAPEIO]
      $ DEFINE SGS_DIR     disk:[STARBITS.SGS]
      $ DEFINE PGPLOT_DIR  disk:[STARBITS.PGPLOT]
\end{verbatim}

\end{enumerate}

\item Dismount and deallocate the tape:

\begin{verbatim}
      $ DISM tape
      $ DEALL tape
\end{verbatim}

\end{enumerate}

\subsection{HDS}

This note is intended to explain a slight inconsistency between the HDS  used
by Figaro 3.0 and the Standard Starlink HDS, that has come about because of an
awkward lack of synchronisation between various releases.

Figaro 3.0 has been linked with a version of HDS that is later than the version
distributed to Starlink sites at the time this note is being written (April
91). The main difference is that this later version supports the rather
obscure HDS routine DAT\_WHERE (which allows a routine to bypass HDS in order
to get direct access to data array in an HDS file---not something that should
be done lightly, but which can be useful where efficiency is the only
consideration). This routine is necessary to support the equally obscure DSA
routine DSA\_DATA\_LOCATION (to which the same comments apply). None of the
Figaro applications released with Figaro 3.0 uses either of these routines and
it is not envisaged that standard Figaro applications will ever use them, but
DSA\_DATA\_LOCATION is documented in the Figaro Programmer's Guide and is used
by other (non-Figaro) software that uses DSA (the AAO ADAM Data Recording Task,
DRT, used at Hawaii and at AAO is the only example I know of), so it was felt
that it should be supported by the Figaro 3.0 release.

To make it clear which version of HDS is used, AAO has always used the logical
name HDS\_SHR to indicate the shareable image built from this later version,
and so the Figaro 3.0 applications expect this to be defined. The startup
command procedure, FIGARO\_PROG\_S:FIGARO.COM defines this to point to the
image FIGARO\_PROG\_S:HDS\_SHR.EXE released with Figaro 3.0. When this version
is released by Starlink, this file will no longer be needed, but the  logical
name can be allowed to remain. Eventually, Starlink will release even later
version, with bug-fixes that should be used by Figaro. At this point, it may
become necessary to re-link the Figaro applications, but by then we may well be
on to Figaro 3.1. Note that HDS\_SHR is `pure' HDS---it  does not include the
additional CMP\_ routines and the like that are included in the Starlink HDS
releases.

\subsection{SGS/GKS/GNS}

This section should only be of interest to non-Starlink sites.

Unfortunately, it is not possible to supply Starlink's GKS on the standard
Figaro  release  tape. Any  academic  institution  that is involved in
astronomical research may get a copy  of  Starlink's  GKS, but they have to
apply to Starlink for this. Usually, Starlink will approve this quite
straightforwardly, and a copy can then be supplied. The person to contact is
the Starlink Software Librarian, Mike Lawden, at Starlink, Rutherford Appleton
Laboratory,  Chilton,  DIDCOT,  Oxon OX11 0QX, UK. Telephone:   0235-21900
X5114;  electronic  mail: UK.AC.RL.STAR::MDL on the JANET network.

When you get GKS, you should get SGS along with it.  SGS is not a proprietary
package like GKS, but it sits on top of GKS and provides a simpler interface to
it.  Also you should get GNS,  which  is  another non-proprietary  package
which provides a simple way of using friendly device names for the devices
supported by GKS.  Without GNS,  GKS  and SGS  need  a complex set of logical
names defining in order to get the same effect.  With GNS, a relatively simple
text file serves the  same purpose.

A standard set of GKS/GNS/SGS logical names need to be defined. These are
usually defined in terms of the main directories containing the GKS and GNS
files, which should be be given the logical names GKS\_DIR and GNS\_DIR. The
full list, as taken from the AAO startup file, is:

\begin{verbatim}
      $ DEFINE/SYSTEM GKSSHARE   GKS_DIR:GKSSHARE.EXE
      $ DEFINE/SYSTEM FPLSHARE   GKS_DIR:FPLSHARE.EXE
      $ DEFINE/SYSTEM GKS_EMF    GKS_DIR:GKSEMF.DAM
      $ DEFINE/SYSTEM GKS_WDT    GKS_DIR:GKSWDT.DAM
      $ DEFINE/SYSTEM GKS_FONTS  GKS_DIR:GKSDBS.DAM
      $ DEFINE/SYSTEM GKS_PAR    GKS_DIR:GKS.PAR
      $ DEFINE/SYSTEM GKS_EPAR   GKS_DIR:GKSE.PAR
      $ DEFINE/SYSTEM GNS_GKSNAMES GNS_DIR:GKSNAMES.DAT
      $ DEFINE/SYSTEM GNS_GKSDEVICES GNS_DIR:GNSDEVICES.DAT
\end{verbatim}

Normally, any site that has installed GKS properly will not need to worry
about these definitions. If you want to check, the files that Figaro needs to
find in GKS\_DIR are FPLSHARE.EXE, GKSLIBG.OLB and GKSSHARE.EXE, and in
GNS\_DIR are GKSNAMES.DAT and GNSDEVICES.DAT. The two files GKSNAMES.DAT and
GNSDEVICES.DAT are supplied by Starlink and include details of all the devices
supported by the standard Starlink GKS, so can be left unchanged.  However,
most sites do edit the GKSNAMES.DAT file to exclude devices that they do not
have---otherwise they will show up in response to commands such as  `SOFT
OPTIONS' (which prints out all the available devices) and this  can be
confusing. The GNS documentation (Starlink's User Note SUN 57) describes the
format of this file in detail.

\subsection{Using a non-GKS version of PGPLOT}

If necessary, it should be possible to use a non-GKS version of PGPLOT. This is
discouraged, because it is a deviation from the usual way of doing things, but
it may be necessary at some sites.  For example, some sites may well have
devices that their local version of PGPLOT supports and which GKS does not. It
is probably worth trying to find out if there is a GKS driver for the device
since new ones are being developed all the time (Starlink will be able to help
there).

In principle, since PGPLOT is always used through the shareable image GRPSHR
(the name comes from GRPCKG, the lower level of the original PGPLOT system), it
should be possible just to change the definition of the logical name GRPSHR so
that it points to the non-GKS shareable image, and everything should work
perfectly.

In practice, there are two problems. It is important that the transfer vectors
for the two shareable images have the PGPLOT routines in the same  order, and
that the shareable images are generally compatible.  This  is generally
controlled by a version ID for the shareable image, which must match that used
when the program was linked. It seems that the  Starlink shareable image has a
later version ID than the original PGPLOT version, and for this reason sites
wanting to use the original PGPLOT will probably have to relink the various
routines that use line graphics.

The other problem concerns SOFT and HARD. To make things easy for the user,
these have always intervened slightly and have processed the device
specification supplied to make sure it conforms to the PGPLOT  requirements.
Now, the two versions of PGPLOT use quite different device specifications. For
example, to specify an ARGS, the old version used the specification `/AR' (and
the ARGS in question was determined by the ARGS command), while the new version
uses ARGS1 or ARGS2 (or whatever the SGS logical names are). This means that
the old version would react to the command ``SOFT AR'' by adding the `/' that
it  was sure it needed, which the new one reacts to ``SOFT AR'' by checking
that AR is a properly defined SGS logical name and  rejecting it because it is
not.

Since all SOFT and HARD actually do, in the end, is set the Figaro user
variables SOFT and HARD respectively, the following options are possible:

\begin{enumerate}

\item Ignore SOFT and HARD entirely, and set the graphics devices using LET.
For example:

\begin{verbatim}
      $ LET SOFT="/AR"
\end{verbatim}

\item Resurrect the SOFT and HARD versions used by Figaro 2.3. The old .EXE
images can be put into the local directory and the local FIGARO.COM used to
force their use in preference to the new  versions. Most sites using the old
PGPLOT version will probably already have Figaro 2.3; if not, the code  for
SOFT and HARD could be hacked to get the old effect, or the old source could be
obtained from AAO.


\item Use the new SOFT and HARD versions, but use the FORCE and NODRAW
keywords to make them accept the old format specifications. For example:

\begin{verbatim}
      $ SOFT /AR FORCE NODRAW
\end{verbatim}

\item To get the same effect without the explicit use of the keywords,  a minor
change to the code is needed to change the defaults for DRAW and FORCE to true
and false respectively. Such modified versions of SOFT and HARD could be put in
the local Figaro directory and used instead of the  standard versions.

\end{enumerate}

\subsection{NAG}

Any site that wants to relink Figaro routines, or to write routines  that need
a numerical subroutine library, should have a copy of the NAG (Numerical
Algorithms Group) subroutine library.  It's a relatively cheap  package, and
certainly worth having anyway. NAG is based at NAG Ltd, Mayfield House, 256
Banbury Road, Oxford OX2 7DE, UK, (Tel 0865-511245, Telex 83354 NAG UK G), but
have agents in most countries. All that Figaro requires is that the logical
name NAG\_LIB should point at the NAG.OLB object library. This should be the
double precision Fortran version of the library. If you don't have NAG, then
defining NAG\_LIB so that it points at a dummy library will stop the link
procedure (FIGARO\_DEV:FIG.COM) falling over, but that won't  help if you
actually want to use the routines.

\subsection{Installing the running version}

The following procedure will install just those files needed to run Figaro ---
the executable images and those files that they use. All of these are in the
directory backed up as [FIGARO.FIGARO] on the tape.  This directory  also
contains the documentation files. The procedure described here creates a top
level directory called [FIGARO] on a disk that you specify, and puts all the
Figaro files in that directory. If you do not want that main directory to be at
top level you will have to modify the procedure accordingly.

\begin{enumerate}

\item What follows assumes that you have a copy of Starlink's SGS and GKS  7.2
systems. If you do not, these will have to be obtained and installed first.
For more details, see the section on SGS/GKS in these notes. It also assumes
that you have the various other Starlink shareable images needed  by the
running system.  If you do not, these can be extracted from the second save set
on the tape.  See the section on `Starlink software at non-Starlink sites' in
these notes.

\item Decide which disk is to be used for the directory.  The following
description uses the logical name `disk'. You should substitute the correct
name. The files will take just over 29,000 blocks.

\item Create a directory called [FIGARO] on that disk, and set it as the
default directory. You will need some privilege to do this, since it involves
writing into the [000000] directory of the disk.

\begin{verbatim}
      $ CREATE/DIRECTORY disk:[FIGARO]
      $ SET DEF disk:[FIGARO]
\end{verbatim}

\item Select the magnetic tape to be used for the backup. The following
description uses the logical name `tape'. You should substitute the  correct
name. Allocate the drive, physically mount the tape on the drive, then use the
MOUNT command to mount the tape as a foreign tape.

\begin{verbatim}
      $ ALLOCATE tape
      $ MOUNT/FOR tape
\end{verbatim}

\item Backup the files from the [FIGARO.FIGARO] directory on the tape into
the FIGARO directory on the disk.

\begin{verbatim}
      $ BACKUP/LOG/SELECT=[FIGARO.FIGARO]*.* tape:FIGARO3.0 *
\end{verbatim}

(The last file is ZOOM.EXE---you can hit \verb+^Y+ once that has been copied.)

\item To run Figaro, a number of logical names need defining. Some sites,
particularly Starlink sites, will have these defined already. And the command
FIGARO need defining. You should read the section on `Logical names' in these
notes and make sure that all those listed as required by the running  system
are defined suitably. If you are a Starlink site, then you should already have
the necessary logical names defined---they have not changed from previous
releases. If you are not a Starlink site, then if you have already installed
GKS and those Starlink files required as described in the earlier section, then
all you should need to define is the following logical name:

\begin{verbatim}
      $ DEFINE FIGARO_PROG_S disk:[FIGARO]
\end{verbatim}
and the following symbol:
\begin{verbatim}
      $ FIGARO:==@FIGARO_PROG_S:FIGARO
\end{verbatim}

If all users are to be able to run Figaro, the logical name definition  should
be made /SYSTEM and included in the system startup file (SYSTARTUP.COM).  The
symbol should be defined in the common login file (SYSLOGIN.COM).

Note that if you have only installed the running system, then your main  Figaro
directory, the one given the logical name FIGARO\_PROG\_S, is disk:[FIGARO] and
not disk:[FIGARO.FIGARO] as it is for the more comprehensive installations.

\item Dismount and deallocate the tape:

\begin{verbatim}
      $ DISM tape
      $ DEALL tape
\end{verbatim}

\item You may want to edit the file FIGARO.COM in the directory FIGARO\_PROG\_S
to suit your particular installation, but in principle it should be possible to
make most local modifications through  a local version,
FIGARO\_PROG\_L:FIGARO.COM, as described later in the section on `Local Figaro
directories'.

\item Giving the command FIGARO should now start up Figaro.

\end{enumerate}

\subsection{Producing a minimum system}

Sites that are very short of disk space, and begrudge Figaro all of its  29,000
blocks, may want to try to reduce the disk usage by deleting some files they
will not need. The following are candidates:

\begin{enumerate}

\item The printable versions of the documentation can be printed off and then
deleted. Any of the .MEM and .DOC files can be dispensed with, but the largest
and probably least needed files are COMMANDS.DOC (about 2,200 blocks) and
FIGARO.DOC (about 700 blocks). These are just printable versions of the on-line
help libraries COMMANDS.HLB and FIGARO.HLB.

\item The files for any commands that are not required can be deleted.  If it
is really certain that nobody will need to use files in, for example, the
peculiar format handled by RJKM, then all the RJKM.* files may  be deleted
without affecting the system.  Note, however, that  comparatively little space
is saved by this sort of surgery.

\item If nobody is going to use the Callable Figaro  system,  then over  9,000
blocks  may  be  saved by deleting the shareable image BIGFIG.EXE.  However, it
is hoped that  users  will  be encouraged to use the Callable Figaro system,
and it would be a pity to delete it just for a few thousand blocks.  A better
ploy would be to use the ever-increasing size of systems like Figaro as an
argument  with  the  powers  that  be  for  the purchase of more disk space.

\end{enumerate}

\subsection{Installing a development system}

The following procedure will install a development Figaro system, including
the  object  libraries and development procedure files, but excluding the
source files.  The directory structure  on  tape  has  a main  directory
called [FIGARO] and a number of sub-directories.  The procedure described here
creates a top level directory called [FIGARO] on  a disk that you specify, and
creates a sub-directory called [.DEV] that contains the necessary development
procedures, one called [.LIBS] that contains the object libraries, one called
[.FIGARO] that contains the main system files and one called [.INCLUDES]  that
contains  some files  that  may  need to be INCLUDEd by Figaro applications
code.  If you do not want the main directory to be at top level you will have
to modify the procedure accordingly.

\begin{enumerate}

\item What follows assumes that you have a copy of  Starlink's  SGS and  GKS
7.2  systems.  If you do not, these will have to be obtained and installed
first. For  more  details,  see  the section  on SGS/GKS in these notes.  It
also assumes that you have the various other Starlink shareable  images  needed
by the  running  system. If you do not, these can be extracted from the second
save set on the tape. See  the  section  on `Starlink  Software  at
non-Starlink  sites' in these notes. You will also need the NAG library and
should  already  have the  logical  name NAG\_LIB defined.  See the section on
`NAG' for more details.

\item Decide which disk is to  be  used  for  the  directory. The following
description  uses  the  logical  name `disk'.  You should substitute the
correct name.  The files will take just over 38,000 blocks.

\item During the backup, a top level directory called [FIGARO] will be  created
on disk:.  You should make sure that you have the privilege to do this.

\item Select the magnetic tape to be  used  for  the  backup. The following
description  uses  the  logical  name `tape'.  You should substitute the
correct  name. Allocate  the  drive, physically  mount  the  tape on the
drive, then use the MOUNT command to mount the tape as a foreign tape.

\begin{verbatim}
      $ ALLOCATE tape
      $ MOUNT/FOR tape
\end{verbatim}

\item Backup all the required directories from the  tape  into  the FIGARO
directory on the disk.

\begin{verbatim}
      $ BACKUP/LOG tape:FIGARO3.0/SELECT=([FIGARO.FIGARO]*.*,-
              [FIGARO.DEV]*.*,[FIGARO.INCLUDES]*.*,[FIGARO.LIBS]*.*, -
              [FIGARO.OLD.PGPLOT]GRPCKG.OLB)  disk:[*...]
\end{verbatim}

(The last file is GRPCKG.OLB---you can hit \verb+^Y+ once it has been copied.)

\item To run Figaro, a number of logical names need defining.  Some sites,
particularly  Starlink sites, will have these defined already.  The command
FIGARO also needs defining.  You should read  the  section on `Logical names'
in these notes and make sure that all those listed as required by the running
system and  for Figaro development are defined suitably.  If you are a Starlink
site, then you should already have  the  necessary logical  names  defined ---
they have not changed from previous releases.  If you are not a Starlink site,
then if  you  have already  installed  GKS, and those Starlink files required
as described in the earlier section, then all you should need to define is the
following logical name:

\begin{verbatim}
      $ DEFINE FIGARO_PROG_S disk:[FIGARO.FIGARO]
\end{verbatim}

and the following symbol:

\begin{verbatim}
      $ FIGARO:==@FIGARO_PROG_S:FIGARO
\end{verbatim}

If all users are to be able to run Figaro, the  logical  name definition
should be made /SYSTEM and included in the system startup file (SYSTARTUP.COM).
The symbol should  be  defined in the common login file (SYSLOGIN.COM).

\item The  following  logical  names are used during Figaro development, and
also need to be defined.  Again, unless only a  select  number  of  users  are
going  to  develop  Figaro software,  these  definitions  should  also  be
/SYSTEM  and included in the startup file.  In addition, the symbol CREPAR
needs to be defined.

\begin{verbatim}
      $ DEFINE DTASRC      disk:[FIGARO.INCLUDES]
      $ DEFINE DYN_SOURCE  disk:[FIGARO.INCLUDES]
      $ DEFINE FIGARO_DEV  disk:[FIGARO.DEV]
      $ DEFINE FIGARO_LIBS disk:[FIGARO.LIBS]
      $ DEFINE FIGARO_WORK SYS$SCRATCH  ! or any scratch directory
      $ DEFINE PGPLOTDIR   disk:[FIGARO.OLD.PGPLOT]

      $ CREPAR:==$FIGARO_DEV:CREPAR
\end{verbatim}

\item Dismount and deallocate the tape:

\begin{verbatim}
      $ DISM tape
      $ DEALL tape
\end{verbatim}

\item You may want to edit the file  FIGARO.COM  in  the  directory
FIGARO\_PROG\_S  to  suit  your particular installation, but in principle  it
should  be  possible  to make most local modifications   through
a   local  version, FIGARO\_PROG\_L:FIGARO.COM, as described later in
the  section on `Local Figaro directories'.

\item Giving the command FIGARO should now start up Figaro.

\end{enumerate}

\subsection{Installing the entire system}

The following procedure will install the entire Figaro system, including the
source files. The directory structure on tape has a main directory called
[FIGARO] and a number of  sub-directories. The procedure described here creates
a top level directory called [FIGARO] on a disk that you specify, and puts all
the Figaro files in that directory and its sub-directories. If you do not want
that main directory to be at top level you will have to modify the procedure
accordingly.

\begin{enumerate}

\item What follows assumes that you have a copy of  Starlink's  SGS and  GKS
7.2  systems.  If you do not, these will have to be obtained and installed
first. For  more  details,  see  the section  on SGS/GKS in these notes.  It
also assumes that you have the various other Starlink shareable  images  needed
by the  running  system. If you do not, these can be extracted from the second
save set on the tape. See  the  section  on `Starlink Software at non-Starlink
sites' in these notes.

\item Decide which disk is to  be  used  for  the  directory. The following
description  uses  the  logical  name `disk'.  You should substitute the
correct name.  The files will take just under 74,000 blocks.

\item During the backup, a top level directory called [FIGARO] will be  created
on disk:.  You should make sure that you have the privilege to do this.

\item Select the magnetic tape to be  used  for  the  backup. The following
description  uses  the  logical  name `tape'.  You should substitute the
correct  name. Allocate  the  drive, physically  mount  the  tape on the drive,
then use the MOUNT command to mount the tape as a foreign tape.

\begin{verbatim}
      $ ALLOCATE tape
      $ MOUNT/FOR tape
\end{verbatim}

\item Backup all the files from the tape into the FIGARO  directory on the
disk.

\begin{verbatim}
      $ BACKUP/LOG tape:FIGARO3.0 disk:[*...]

      $ DISM tape
      $ DEALL tape
\end{verbatim}

\item To run Figaro, a number of logical names need defining.  Some sites,
particularly  Starlink sites, will have these defined already.  The command
FIGARO also needs defining.  You should read  the  section on `Logical names'
in these notes and make sure that all those listed as required by the running
system and  for Figaro development are defined suitably.  If you are a Starlink
site, then you should already have  the  necessary logical  names  defined ---
they have not changed from previous releases.  If you are not a Starlink site,
then if  you  have already  installed  GKS, and those Starlink files required
as described in the earlier section, then all you should need to define is the
following logical name:

\begin{verbatim}
      $ DEFINE FIGARO_PROG_S disk:[FIGARO.FIGARO]
\end{verbatim}

and the following symbol:

\begin{verbatim}
      $ FIGARO:==@FIGARO_PROG_S:FIGARO
\end{verbatim}

If all users are to be able to run Figaro, the  logical  name definition
should be made /SYSTEM and included in the system startup file (SYSTARTUP.COM).
The symbol should  be  defined in the common login file (SYSLOGIN.COM).

\item The  following  logical  names are used during Figaro
development, and also need to be defined.  Again, unless only a  select  number
of  users  are  going  to  develop  Figaro software,  these  definitions
should  also  be  /SYSTEM  and included in the startup file.  In addition, the
symbol CREPAR needs to be defined.

\begin{verbatim}
      $ DEFINE DTASRC        disk:[FIGARO.DTA]
      $ DEFINE DYN_SOURCE    disk:[FIGARO.DYN]
      $ DEFINE FIGARO_DEV    disk:[FIGARO.DEV]
      $ DEFINE FIGARO_LIBS   disk:[FIGARO.LIBS]
      $ DEFINE FIGARO_WORK   SYS$SCRATCH  ! or any scratch directory
      $ DEFINE FIGARO_SOURCE disk:[DEV.SOURCE]
      $ DEFINE FIGARO_CALL   disk:[DEV.CALL]
      $ DEFINE PGPLOTDIR     disk:[FIGARO.OLD.PGPLOT]

      $ CREPAR:==$FIGARO_DEV:CREPAR
\end{verbatim}

\item Giving the command FIGARO should now start up Figaro.

\item You may want to edit the file  FIGARO.COM  in  the  directory
FIGARO\_PROG\_S  to  suit  your particular installation, but in principle  it
should  be  possible  to make most local modifications through a local
version, FIGARO\_PROG\_L:FIGARO.COM, as described later in  the  section on
`Local Figaro directories'.

\end{enumerate}
\subsection{Logical names and symbols used by the Figaro running system}

The following list contains all the logical names needed  to  run Figaro
programs.  These were collected for Figaro 2.4 as the result of an experiment
in which the system logical name table was disabled,  no process  logical names
were defined originally, and logical names were then defined one by one until
all Figaro programs ran.  This seemed  a more  comprehensive  way  of  going
about  things than just trying to produce a list of logical names believed to
be  used  by  the  system. So,  to  run  Figaro,  all  these  logical  names
should  be defined. Ideally, they should be defined in the system tables,
unless Figaro is used  only  by one or two people at a particular site.
Unfortunately, it was not possible to find an unused VAX that could be used to
repeat the  experiment  for  Figaro  3.0, so the list here has been edited to
reflect the changes to the system but has not been tested as thorough- ly as
was the list produced for Figaro 2.4.

\begin{description}

\item [FIGARO\_PROG\_S] Directory. The main Figaro directory. Normally, this
will be disk:\-[FIGARO.\-FIGARO]. If only the running system has been
installed, this should be disk:\-[FIG\-ARO].

\item [ARGSSHAR] Shareable image. The shareable image used by the Starlink
ARGS package. See the section on Starlink files.

\item [GKS\_DIR] Directory. The directory holding the GKS files. Not
strictly referenced by Figaro, but usually defined as part of the GKS
installation and a convenient base to define some of the other GKS names
on.

\item [GKSSHARE] Shareable image. The GKS shareable image. Normally this is
GKS\_\-DIR:\-GKS\-SHARE.EXE.

\item [GNS\_GKSNAMES] Text file. This is the list of GKS devices
available on the system and the names to be used to refer to them.

\item [GNS\_GKSDEVICES] Binary file. This is compiled from other files
supplied with GNS (see the GNS documentation), to provide details of
the devices supported by GKS.

\item [FPLSHARE] Shareable image. This is another image used by
GKS. Normally, this is GKS\_\-DIR:\-FPLSHARE.EXE.

\item [HDS\_SHR] Shareable image. This is the shareable image used
by HDS, and should be FIG\-ARO\_\-PROG\_S:HDS\_SHR.EXE. (But see the
section on HDS in these notes.)

\item [GRPSHR] Shareable image. This is the shareable image for the GKS
version of PGPLOT, and should be GRPSHR.EXE in the standard GKS PGPLOT
directory. This is normally given the logical name PGPLOT\_DIR, so GRPSHR
is usually defined as PGPLOT\_DIR:\-GRPSHR.EXE.

\item [INTERIM] Shareable image. This is the shareable image used by the
Starlink interim environment, and should be INTERIM.EXE in whatever
directory is appropriate.

\item [TVPSHR] Shareable image. This is the shareable image used by TVPCKG, and
is supplied with Figaro. It should be defined as FIGARO\_PROG\_S:TVPCKG.EXE.

\item [FGRSHR] Shareable image. This is the shareable image containing the main
Figaro libraries, and should be defined as FIGARO\_PROG\_S:FGRSHR.EXE.

\item [BIGFIG] Shareable image. This is the shareable image used by Callable
Figaro. It should be defined as FIGARO\_PROG\_S:BIGFIG.EXE.

\end{description}

Note that of these, any Starlink site should have all bar FIGARO\_PROG\_S,
TVPCKG, FGRSHR, and BIGFIG defined quite independently of Figaro, and other
sites should have defined the others in  the process of installing GKS and the
other Starlink software. In addition, FGRSHR, BIGFIG and TVPSHR are defined by
the Figaro start-up procedure, FIGARO.COM, so may not need to be defined
independently for most purposes. However, if anyone is going to use Callable
Figaro programs without first running FIGARO.COM, they will need all these
defined if their Callable Figaro programs are to run.

In addition, the assorted SGS/GKS logical names listed in the `SGS/GKS' section
need to be defined, but this should have happened as part of the SGS/GKS
installation.

Just for completeness, the logical names SYS\$SYSTEM, SYS\$SHARE, SYS\$LIBRARY
and SYS\$SCRATCH were also turned up as being used by Figaro in the experiment
of running with no system logical names defined. However, any  VAX which does
not have these defined is a very oddly setup VAX indeed. The only name that
needs comment is SYS\$SCRATCH. Figaro uses this as the directory in which it
writes the VARS.DST and BVARS.DST files. Normally, VMS systems define this as
the user's login directory; this may not be appropriate, and system managers
may prefer to redefine it to indicate some  less permanent location.

The symbol FIGARO should be defined as

\begin{verbatim}
      $ FIGARO :== @FIGARO_PROG_S:FIGARO
\end{verbatim}

So, in summary, Starlink sites and sites that have already installed the
additional Starlink packages, should find that the only logical names and
symbols needed specifically to run Figaro are the logical name FIGARO\_PROG\_S
and the symbol FIGARO.

System managers should also read the section on the use of National and Local
Figaro directories, since most sites will want to define the logical  name
FIGARO\_PROG\_L to point to a directory containing local Figaro modifications.

\subsection{Logical names used in Figaro development}

If users are going to be developing their own programs that interact with
Figaro at all, then some of the logical names used by the system will have to
be defined. At the very least, if users are writing the minimal one-off
programs that just use the DTA\_ package to access Figaro data files, they will
need to be able to find the DTA.OPT file in the directory [FIGARO.LIBS], and
since in any case this file refers to that directory as FIGARO\_LIBS, this
logical name should be defined.

If users are going to use the main linking procedure FIG.COM in the development
directory, then all the logical names that it uses will have to be defined. A
full list, excluding those logical names listed as required by the running
system, almost all of which are also needed for development, is as follows:

\begin{description}

\item [FIGARO\_DEV] Directory. Contains the files used for Figaro
development. Should be disk:\-[FIGARO.DEV].

\item [FIGARO\_LIBS] Directory. Contains the various Figaro object libraries.
Should be disk:\-[FIGARO.LIBS].

\item [DYN\_SOURCE] Directory. Contains the include file DYNAMIC\_MEMORY.INC
used by a number of applications. Should be disk:[FIGARO.INCLUDES] if
only the development files have been installed. If the full system is
installed, should be disk:[FIGARO.DYN].

\item [DTASRC] Directory. Contains the include files for the DTA\_
package, used by a number of older applications. Should be
disk:[FIGARO.INCLUDES] if only the development files have been installed. If
the full system is installed, should be disk:[FIGARO.DTA].

\item [SGS\_DIR] Directory. Contains the SGS files, in particular
SGSLINK.COM. Should be the standard directory used for SGS.

\item [NAG\_LIB] Object library. This should point to the .OLB file that
contains a double precision version of the NAG routines.

\item [INTERIM\_DIR] Directory. This should be the directory that contains the
files used by the Starlink Interim library.

\item [PGPLOT\_DIR] Directory. This should be the directory that contains the
files used by the GKS version of PGPLOT.

\item [PGPLOTDIR] Directory. This should be the directory that contains the
files used by the old version of PGPLOT, that is [FIGARO.OLD.PGPLOT].

\item [LIBDIR] Directory. This should be the directory that contains the
standard Starlink object libraries, in particular TAPEIO.OLB.

\item [GKS\_DIR] Directory. Contains the GKS files.

\item [ARGSOPT] Linker .OPT file. This should point to the file, ARGSOPT.OPT,
that controls linking with the Starlink ARGS shareable image.

\item [FIGARO\_WORK] Directory. Used by the Callable Figaro linking procedure.
This should be any scratch directory that the user can write into.

\item [FIGARO\_SOURCE] Directory. The directory containing the source of the
various Figaro applications. This is not in fact used other than by some  of
the automatic documentation utilities, but it is often convenient to have it
defined. It should be defined as disk:[FIGARO.DEV.SOURCE].

\item [FIGARO\_CALL] Directory. The directory used for developing the Callable
Figaro system. This should not normally be needed by any user, but it is
sometimes convenient to have it defined. A later version of Figaro is intended
to have provision for user-written extensions to Callable Figaro, and this will
be needed then.

\end{description}

Most of these will have been defined independently of Figaro by Starlink sites
or sites that have installed the various Starlink packages being used. The
names that are specific to Figaro are DTASRC, PGPLOTDIR, FIGARO\_CALL,
FIGARO\_DEV, FIGARO\_LIBS, FIGARO\_\-SOURCE, FIGARO\_WORK and  DYN\_SOURCE,
which can be defined using:

\begin{verbatim}
      $ DEFINE DTASRC        disk:[FIGARO.INCLUDES]
                                     or disk:[FIGARO.DEV]
      $ DEFINE FIGARO_CALL   disk:[FIGARO.DEV.CALL]
      $ DEFINE FIGARO_DEV    disk:[FIGARO.DEV]
      $ DEFINE FIGARO_LIBS   disk:[FIGARO.LIBS]
      $ DEFINE FIGARO_SOURCE disk:[FIGARO.DEV.SOURCE]
      $ DEFINE FIGARO_WORK   SYS$SCRATCH or anywhere more suitable.
      $ DEFINE DYN_SOURCE    disk:[FIGARO.INCLUDES]
                                     or disk:[FIGARO.DYN]
      $ DEFINE PGPLOTDIR     disk:[FIGARO.OLD.PGPLOT]
\end{verbatim}

In addition, the command CREPAR needs to be defined in order to process .CON
files. This should be:

\begin{verbatim}
      $ CREPAR :== $FIGARO_DEV:CREPAR
\end{verbatim}

and the command FIGLINK may need defining for use with Callable Figaro. This is
defined by FIGARO.COM, but should perhaps be defined separately so that users
do not have to run FIGARO.COM in order to use Callable Figaro.

\begin{verbatim}
      $ FIGLINK:==@FIGARO_PROG_S:FIGLINK
\end{verbatim}

\subsection{Figaro directories}

The Figaro sub-directories are used in the following ways:

\begin{description}

\item [CNV] The source of the CNV library. These routines perform fast
conversion between arrays of different types, and can optionally allow for and
propagate specified `bad values' used to flag data elements.

\item [DEV] This is the main Figaro development directory. It is usually given
the logical name FIGARO\_DEV. This directory contains a number of command
procedures and utility programs connected with the maintenance of Figaro,
described more fully in the next section. It also contains the following
sub-directories:

\begin{description}

\item [.CALL] Contains those files that are particular to the Callable Figaro
system, including the command procedure FIGBUILD.COM which is used to re-link
the shareable image used by Callable Figaro.

\item [.DOCS] Contains the files from which the Figaro documentation is
created.

\item [.SOURCE] Contains the source files for the Figaro programs, and their
connection files.

\end{description}

\item [DSA] Contains the source of the DSA\_ Data Structure Access routines.

\item [DSK] Contains the source of the DSK\_ routines. These are versions of
PGPLOT routines that are used by the Figaro program SPLOT to produce `build'
files.

\item [DTA] Contains the source for the DTA\_ routines. These are the
routines used to access the data structures held in Figaro files. Also
contains DTA.RNO, which is the main documentation for these routines. This
directory contains the version of these routines that uses HDS.

\item [DUT] Contains the source of the DUT\_ routines. These are the
routines used by IMAGE to perform the scaling and reshaping of the images.

\item [DYN] Contains the source of the DYN\_ routines. Also includes the
file `DYNAMIC\_MEMORY\-.INC' which is included by many of the applications
that have been rewritten to use DSA. This directory should have the
logical name DYN\_SOURCE, so that this include file can be picked up.

%***** Marker *****

\item [FIG] Contains the source of the various FIG\_ routines. FIG\_ is
a general prefix used by routines that are called directly by Figaro
applications but are specific to Figaro.

\item [FIGARO] Contains the files actually needed to run Figaro. These
are the .EXE images, the .PAR parameter files, the .INF on-line help
files, the .TAB tabulated data files, and a few other odds and ends. This
directory should be given the logical name FIGARO\_PROG\_S.

\item [FIT] Contains the source for the FIT\_ routines. This is a package of
routines designed to facilitate the writing of FITS tapes.

\item [GEN] Contains the source for the GEN\_ routines. These are
routines of `general' utility, although some of them may look a trifle
specialised.

\item [GKD] Contains the source for the GKD routines. These are routines used
to control dialogues with the user in an interactive graphics environment.

\item [HDS] This contains the version of HDS used to link Figaro 3.0, since
this is a later version than that released as the standard Starlink version.

\item [ICH] Contains a set of string handling routines---the ICH\_ routines.
(There are also some routines called PCH\_, but these are not used by Figaro.)

\item [INCLUDES] Contains include files that may be needed by Figaro
applications. These are normally to be found in the directories devoted to
the various packages, but are also available in this directory for use if
just the development system has been installed. This directory can be
deleted if the full system is installed, but very little space is saved by
doing so.

\item [JT] Contains the source of the assortment of routines written at
Caltech by John Tonry that are used by various Figaro applications.

\item [LIBS] Contains most of the object libraries used by Figaro. It
should be given the logical name FIGARO\_LIBS.

\item [MEM] Contains the source of the MEM\_ routines used by the DUT\_
routines to obtain dynamically allocated memory.

\item [NAG\_FIX] Contains the source of routines that emulate discontinued NAG
routines still required by some Figaro applications.

\item [OLD] Contains packages that are part of old versions of Figaro, but are
still required for some purposes.

\begin {description}

\item [.DTA1] Contains the source for the old version of the DTA\_ routines,
which did not use HDS. These are the source  for  the  object
library FIGARO\_LIBS:DTA1.OLB, and are only now used by the format conversion
routines DTA2HDS and HDS2DTA.

\item [.PGPLOT] Contains the source for the old PGPLOT graphics package.
This is a very old version, but contains some routines that are used (for
the moment) by VAPLOT, and so is still needed for that program.

\end{description}

\item [PAR] Contains the source for the PAR routines. These are the routines
used by Figaro to communicate with the user, and to handle parameter
evaluation. Note that these have been extensively revised for version 2 of
Figaro.

\item [STARLINK] Contains packages that are normally available at
Starlink sites, but may not be installed at other sites. This directory
and its various sub-directories are to be found in the second save set
on the Figaro release tape, and so should not appear at Starlink sites.
Packages that have some proprietary aspects, such as GKS or NAG are not
included.

\begin{description}

\item [.ARGS] Contains the various files needed by the Starlink ARGS
package, such as ARGS\-OPT\-.OPT and ARGSSHAR.EXE.

\item [.GNS] Contains the various files needed by Starlink's GNS
routines.

\item [.INTERIM] Contains the files used by the Starlink interim library,
such as the shareable image INTERIM.EXE.

\item [.PGPLOT...] Contains the source of the GKS version of PGPLOT.

\item [.SGS] Contains the files used by SGS, although these are of little
use without GKS.

\item [.TAPEIO] Contains the files used by the TIO\_ routines.

\end {description}

\item [TAPES] contains the source of the tape I/O package used by Figaro,
Caltech's MTPCKG.

\item [TVPCKG] contains the files needed to create the ANU version of the
image display package, TVPCKG.

\begin{description}

\item [.AED] contains the code specific to the AED display. In the
version released with Figaro 3.0 this consists only of dummy subroutines.

\item [.ARGS] contains those additional files, needed by TVPCKG for the ARGS,
that are not part of the normal Starlink package.

\item [.ANU] contains the basic root code for the package. This
includes the code for the Grinnell.

\item [.IKON] contains the code specific to the IKON display.

\item [.MAC] contains the code specific to the MAC II display. (The MAC
II needs to be running the MacImager program, which is not included, but
can be obtained from AAO.)

\item [.RAMTEK] contains the code specific to the RAMTEK display. In the
version released with Figaro 3.0 this consists only of dummy subroutines.

\item [.TEST] contains the source of a test program TVTEST.

\item [.UIS] contains the code specific to a VaxStation, and also includes
the code for the separate process that services the TVPCKG display requests.

\end{description}

\item [VARS] contains the source of the VAR\_ routines, used to access the
Figaro user variables.

\end{description}

\subsection{Files in the FIGARO\_DEV directory ([FIGARO.DEV])}

The following list covers most of the files found in the Figaro development
directory.

\begin{description}

\item CABUJY .FOR and .EXE files. This is a utility program that
converts a .TAB file for a standard star from AB magnitudes to either
milli- or micro-Janskys. Most Figaro standard files are created in AB
magnitude versions, then converted to Jansky units by CABUJY. CABUJY only
converts the numbers; the headings and SET commands have to be edited by
hand.

\item COMCOM .FOR and .EXE files. A utility used for preparing Figaro
releases. This reads the COMMANDS.DAT file and generates files
containing `@FIG' and `@REL' commands for each Figaro command, enabling the
re-compiling and re-linking to be automated.

\item COMDOC .FOR and .EXE files. A utility used to prepare the
COMMANDS.DOC file. It reads the COMMANDS.DAT file and generates
COMMANDS.DOC from the source and connection files for each command.

\item COMHELP .PAS and .EXE files.  A utility used to prepare the
COMMANDS.HLB library.  It reads the COMMANDS.DAT file and generates a help
file from the source and connection files for each command which can then be
turned into the help library.

\item COMLIST .FOR and .EXE.  One of the pre-processor programs used to
process the basic documentation file FIGARODOCS:FIGARO.RNP by the command
procedures PHELP and FHELP.

\item COMMANDS.DAT The master file that contains the names of all the Figaro
commands, the subroutines that service them, and other assorted information.
Many procedures connected with Figaro development make use of this file.

\item COMMENT.FOR and .EXE.  A utility program that produces  crude
documentation of a package by extracting the comments from the start
of its subroutines.

\item CREPAR.EXE The utility that processes .CON files to produce .PAR and
.INF files.  The source is in [FIGARO.PAR].

\item DEVICES.DAT A devices file for use with GKDCREATE. This one is purely
illustrative; all devices used at AAO can be handled by the GKD routines
without any explicit control being needed.

\item DOCUPD.COM The command procedure that performs a full update on the
Figaro documentation prior to a new release.

\item FGRSHR.COM The command procedure used to generate the FGRSHR shareable
image that contains most of the Figaro standard libraries - including DSA.

\item FHELP.COM The command procedure that  processes FIGARO.RNP to
produce the on-line help library FIGARO.HLB.

\item FIG.COM The command procedure used to relink Figaro application
programs. This is the main file in this directory that will be used by
individual users developing their own Figaro programs.

\item FIGSHR.COM The command procedure used to create a new version of the
FIGSHR shareable image---the one that contains the common Figaro
libraries, GEN\_, PAR\_, VAR\_, DTA\_, and ICH\_. This is now more or
less replaced by FGRSHR, which also contains DSA\_.

\item GKDCREATE.EXE Utility program that creates the GKD system control file
in FIGARO\_\-PROG\_S from DEVICES.DAT.

\item GMAIN .FOR and .EXE. Utility program that generates the main program
for a Figaro application. Run as part of FIG.COM.

\item PHELP.COM Command procedure to create a new version of the printable
help file FIGARO.DOC from the basic file FIGARO.RNP

\item REL.COM Command procedure used to release---ie copy over to the main
Figaro directory---the modified files for a Figaro application.

\item RNOMOD .FOR and EXE. One of the pre-processor programs used to
process the basic documentation file FIGARODOCS:FIGARO.RNP by the command
procedures PHELP and FHELP.

\end{description}

\subsection{Figaro documentation}

All Figaro documentation files are to be found in the main Figaro
directory, FIGARO\_PROG\_S. There is one file that is used as an on-line help
file (using the VMS `HELP' facility); all the others are text files that
may be printed off. The text files can be deleted if necessary, to save
space. They can all be regenerated from  the  original  files using
the  command  procedure FIGARO\_DEV:DOCUPD, when required. The files are:

\begin{description}

\item [BUGS.MEM] Describes any known Figaro bugs and problems (this is,
of course, a small subset of all the actual bugs in Figaro).

\item [CHANGES.MEM] A description of the changes made to Figaro routines
since the last release. This is a list of functional changes rather than a
detailed list of every change made to every file. It is aimed at the
user rather than  the  support programmer.

\item [COMMANDS.DOC] Contains details of all the Figaro commands, although
in a rather slap-happy form; it lists the connection file for each
command, together with the first set of comments from the program that
services that command. This file is generated from the list of commands
held in COMMANDS.DAT  in  the  development  directory, together with the
connection files and program source files, by the utility routine COMDOC.

\item [COMMANDS.HLB] An on-line help library containing details (sketchy,
in some cases) of all the individual Figaro commands. It is generated
by the COMHELP program in the development directory as part of the
DOCUPD.COM documentation update procedure.

\item [DOCINTRO.MEM] This is an introductory page that should precede
COMMANDS.DOC. It explains why COMMANDS.DOC is such a tatty piece of
documentation.

\item [DTA.DOC] This is the manual for the DTA\_ routines---the package
used to access the Figaro disk data files. It will only be needed by
programmers wanting to write Figaro programs. It is produced by the command
procedure DTADOC from the file DTA.RNO in the [FIGARO.DTA] directory.

\item [ECH\_RED.MEM] Describes the processing of AAO echelle data.

\item [FIGARO.HLB] The main on-line help library. This is produced by the
command procedure FHELP.COM from the file FIGARO.RNP, in the development
directory.

\item [FIGARO.DOC] This is a printable version of the on-line help
library, produced by the command procedure PHELP.COM from the file
FIGARO.RNP, in the development directory.  This is the main Figaro
reference document.

\item [FIGBASIC.MEM] This is a beginner's introduction to Figaro.  It
concentrates on describing the way parameters are specified in commands.

\item [FIGPROG.MEM] The Figaro Programmer's guide.  This explains how to
write simple programs that can read Figaro data files, as well as how to
write programs that fit in properly with Figaro in terms of use of parameters,
user-interface, etc.

\item [FORMAT.MEM] A description of the Figaro data format. This is a
description of the data objects in Figaro files and of the overall structure
of these files, rather than a `what byte goes where' detailed description of
the disk file layout. This  is  really  just  an  appendix to the
Programmer's Guide.

\item [PAR.DOC] This is not a manual, but  a  file containing  a
formal description of the PAR\_ routines. As such, it is really an
appendix to the programmer's guide.

\item [VAR.DOC] Like PAR.DOC, this is not a manual, but a file containing a
formal description of the VAR\_ routines.  It too should be regarded as
an appendix to the programmer's guide.

\end{description}

There are also a couple of documents that describe individual applications.
GAUSS.MEM describes GAUSS, and ABLINE.MEM describes ABLINE.

\subsection{Use of national-, local- and user- Figaro directories}

Figaro uses the directory called FIGARO\_PROG\_S as its main directory,
but it also checks to see if there are directories called FIGARO\_PROG\_N,
FIGARO\_PROG\_L and FIGARO\_PROG\_U. These are intended to be used as
national (not strictly specific to a nation, whatever that is, but specific to
a larger group of machines that just one site; for example to Starlink as a
whole, or to the AAO systems), as local (ie specific to a particular
installation) and user (specific to a particular user) directories.  The
idea is that if local changes are made to Figaro, these should be done
without changing the files in [FIGARO..], but should be made in a local
directory and new images re-linked into that directory. A file called
FIGARO.COM in the local directory can then be used to redefine the Figaro
command in question.

It is recommended that all Figaro sites have a local directory, and the
logical name FIGARO\_\-PROG\_L defined. If nothing else, a NEWS.TXT in the
local directory will be output in preference to the one in the standard
system directory, and this is a file that is often changed by local sites. In
principle, it should not now be necessary to change any of the files in the
directories released as part of Figaro, and such changes are strongly
discouraged; they are very hard to keep track of. Almost all changes needed
to Figaro can be made by using the local directory to override the standard
files. In particular, the use of logical name search paths can be used to
make the linking procedures search local directories first, before the
standard directories.  It's usually tempting to change the released files,
because it's usually much easier in the short term; it's usually regretted
when the next version comes along. You might even consider write-protecting
all the Figaro directories, as an aid to self-control.

An example may help. Suppose a local change has to be made to the `SPLOT'
command. The procedure is as follows:

\begin{enumerate}

\item Assuming a local directory exists, and has been given the logical
name FIGARO\_PROG\_L, first copy the source of the Figaro program into it.

\begin{verbatim}
      $ SET DEF FIGARO_PROG_L
      $ COPY FIGARO_SOURCE:SPLOT.FOR *
\end{verbatim}

\item Edit SPLOT.FOR, making the necessary changes, and recompile it.

\item If you are adding a new command, or making changes that require that
you modify the entry in COMMANDS.DAT for the command, then you will need a
modified version in your directory to drive the linking procedure.  The
easiest thing is just to copy the one from the development directory:

\begin{verbatim}
      $ COPY FIGARO_DEV:COMMANDS.DAT *
\end{verbatim}

and then edit it. The file FIGARO\_DEV:COMMANDS.DAT is used by the linking
procedure if it cannot find a suitable entry in a COMMANDS.DAT file in the
default directory.

\item Link the new version of SPLOT using the FIG.COM file in the
development directory.

\begin{verbatim}
      $ @FIGARO_DEV:FIG SPLOT
\end{verbatim}

This creates a file SPLOT.EXE in the local directory.

\item You now need a FIGARO.COM file in the local directory to redefine
the SPLOT command so that it uses the new version. It should contain the line

\begin{verbatim}
      $ SPLOT:==$FIGARO_PROG_L:SPLOT
\end{verbatim}

\end{enumerate}

If changes have to be made to routines in one of the subroutine libraries, you
can either produce a local version of FIG.COM which uses a new version of the
library in the local directory, or you can put the new library in the local
directory and make FIGARO\_LIBS a search path, for example:

\begin{verbatim}
      $ DEFINE FIGARO_LIBS FIGARO_PROG_L,disk:[FIGARO.LIBS]
\end{verbatim}

It might be better to use a sub-directory of the local directory, but that
depends on how many changes you are making. Earlier versions of these
notes have mentioned the possibility of changing the standard library, but
search paths make that unnecessary as well as undesirable. The important thing
is to be able to tell what changes have been made, since any new Figaro
release will probably over-write all the files in the [FIGARO...] directories.

\section{Starlink's release---version 3.0-3}
\label{Starnotes}

\subsection{Standard Figaro}

The ``entire system'' no longer includes the source files for the non-HDS
version of DTA, {\em i.e.\/} the contents of the directory {\tt
[FIGARO.\-OLD.DTA]}. But the library {\tt [FIGARO.LIBS]DTA1.OLB} remains.

The entire system also no longer includes the non-GKS version of PGPLOT, {\em
i.e.\/} the contents of the directory {\tt [FIGARO.\-OLD.PGPLOT]}. Thus the
non-GKS {\tt GRPCKG} and {\tt GRPSHR} are no longer available.

The entire system no longer includes the source files of HDS, ({\em i.e.\/}
contents of {\tt [FIGARO.HDS]} since this has been superseded by Starlink's HDS
4.0. However, the shareable image remains in {\tt
FIGARO\_\-PROG\_S:\-HDS\_SHR.EXE}, if only for sentimental reasons.

Standard Figaro no longer contains the ICL monolith ({\em i.e.\/} files {\tt
[FIGARO.\-MONOLITH.\-FIGARO]\-FIGARO.*}. Instead National Figaro contains a
complete monolith with all the updates to applications included.

The entire system is about 65000 VAX blocks in size. It is recommended only for
the nodes where major software development in the Figaro software environment
is going on ({\em e.g.\/} CAVAD, REVAD and the associated overseas
observatories) and for RLVAD as the central Starlink Software Library. Other
Starlink sites have either the ``development system'' or the ``extended
development system''.

Compared to the entire system, the extended development system lacks the source
code of the Figaro subroutine libraries. This should cause no trouble since
only in very rare circumstances would the libraries be modified locally. All
applications source code is available in the extended development system. When
bits of library source code are needed for reference, this can always be
accessed at RLVAD. The size of the extended development system is about 50000
VAX blocks.

The development system lacks also the applications' source code. Its size is
40000 VAX blocks. This is intended for nodes where little or no Figaro
programming is going on. Programme development is still possible, since all
object libraries are available. But the occasional need for a template
application source must be satisfied by referring to RLVAD.

When linking Figaro applications it is advisable to include a linker option
which increases the linker parameter ISD\_MAX. Otherwise executable files may
become excessively large in certain circumstances. For this reason the link
procedure {\tt [FIGARO.\-DEV]\-FIG.COM} is superseded by a corresponding
procedure in National Figaro.

The HDS package as delivered with Figaro from AAO is obsolete. It is superseded
by Starlink's HDS 4.0, which also introduces a new data format. The pre-3.0-3
Figaro routines would be unable to read the new data format.

The linker options file {\tt DTA.OPT} had to be changed so that HDS\_IMAGE
would be used instead of HDS\_SHR. The updated options file is in National
Figaro.

The subtle interplay between DTA, HDS and EMS made changes to 24 DTA-routines
necessary, because DTA quite happily mixed DTA and HDS error status variables.
Also DAT\_GET currently fails when a string is to be got and the memory storage
supplied is too short for the HDS object. DTA has to take care of this problem
now in the routine DTA\_RDVAR. {\tt [FIGARO.\-LIBS]\-DTA.OLB} is superseded by
a corresponding file in National Figaro.

HDS 4.0 produces a plethora of error messages. Before, these errors were not
reported by HDS but only indicated through the status variable. DTA then took
care of reporting when necessary. In order to suppress HDS' messages, the
Figaro fixed part {\tt MAIN.FOR} must call ERR\_MARK, ERR\_ANNUL and ERR\_RLSE.
To this end {\tt [FIGARO.DEV]GMAIN.FOR} and {\tt .EXE} are superseded by
corresponding files in National Figaro.

This works fine as long as an application is not linked with {\tt FGRSHR} {\it
and} PGPLOT (shareable GKS version). In order to suppress HDS' messages also
for PGPLOTting routines, {\tt [FIGARO.\-DEV]\-FIG.COM} needs a further change.
Before, if an application had to be linked with shareable GKS-PGPLOT, SGS was
included in the {\tt LINK} command via {\tt @SGS\_DIR:SGSLINK}. This has been
replaced by the actual contents of that file, but excluding EMSSHARE/SHARE.

The shareable images {\tt FIGSHR, FGRSHR, BIGFIG} (in that order) have been
relinked with the HDS 4.0 shareable image. The new {\tt .EXE} files supersede
the old ones in {\tt FIGARO\_\-PROG\_S}.

The corresponding libraries are {\tt BIGFIG.OLB} (superseded in {\tt
FIGARO\_\-PROG\_S}), {\tt FIGSHR.OLB} and {\tt FGRSHR.OLB}. The latter two are
stored in the National Figaro directory tree, but the old files remain in
Standard Figaro for simplicity.

The Callable Figaro shareable image {\tt BIGFIG} had to be rebuilt after GRPSHR
no longer allowed calls to GRETEXT and GRSETCOL, which are called by the Figaro
applications TIPPEX and ABLINE respectively. These calls were replaced by calls
to PGETXT and PGSCI respectively. However, these changes enter only into a
transient version of {\tt FIGARO.OLB}, which was used only to build {\tt
BIGFIG}. The released application sources remain unfixed.

{\tt BIGFIG} also had to be linked with PSX.

In order for Callable Figaro to supress error reports by HDS and other
packages, the routine FIGARO\_CALL (source file {\tt FIGAROCALL.FOR}) was
changed to call ERR\_MARK, ERR\_ANNUL, ERR\_RLSE in a similar manner to {\tt
MAIN.FOR}. However, these changes enter only into a transient version of {\tt
CALL.OLB}, shich was used only to build {\tt BIGFIG}. The released source
remains unchanged.

Users will have to relink their Callable Figaro applications. {\tt
[FIGARO.\-DEV.\-CALL]\-FIGCALL} would need relinking, but was not relinked. It
is regarded as a template source rather than an executable provided
ready-to-use.

Subsequently all Figaro applications were relinked with FGRSHR and the new
versions supersede the old ones in {\tt FIGARO\_\-PROG\_S}. This includes those
Standard Figaro applications which are actually superseded by updates ({\em
e.g.\/} bug fixes) in National Figaro. Users also will have to relink their
Figaro applications.

{\tt DTA2HDS, DTAORHDS, HDS2DTA} are not Figaro applications in the usual
sense. They form links between the HDS data format and the---largely unused---
non-HDS data format. These have also been relinked and superseded in {\tt
FIGARO\_\-PROG\_S}. However, they have not been tested.

In the entire system a small number of executables are spread over source code
directories. These have not been checked for the necessity of relinking them.
Thus they may not run any more. Since the file names are typically {\tt
TEST.EXE} or {\tt SCRAP.EXE}, they are considered as irrelevant.

\subsection{National Figaro}
\label{natfig}

Mostly, National Figaro provides a (small) number of debugged applications and
some additional applications. This includes an updated (complete) {\tt
FIGARO\_DEV:COMMANDS.DAT}. The updated set of applications (from National and
Standard Figaro) are also available as an ICL monolith.

The logical name and symbol definitions for Figaro are described in Section
\ref{names}.

National Figaro is stored in the {\tt [FIGPACK.\-NFIGARO]} directory tree. It
is currently on equal terms with three other Starlink-released Figaro-based
packages: NDPROGS, TAURUS and TWODSPEC. Consequently there are four
sub-directories {\tt [FIGPACK]\-NFIGARO.DIR, NDPROGS.DIR,\linebreak TAURUS.DIR,
TWODSPEC.DIR}. Since each package has its own startup file {\tt FIGARO.COM} and
all have to be executed when the command FIGARO is given, the logical name {\tt
FIGARO\_\-PROG\_N} is a search list of five directories. The first {\tt
FIGARO.COM} found therein is {\tt [FIGPACK]\-FIGARO.COM}. It is the ``master''
file and executes the other four explicitly.

The directory structure {\tt [FIGPACK.NFIGARO...]} is the same as {\tt
[FIGARO...]}, except that empty directories are not present. Standard Figaro's
{\tt [FIGARO.MONOLITH...]} contain only a \LaTeX\ document on ICL Figaro. ICL
Figaro itself is in {\tt [FIGPACK.\-NFIGARO.\-MONOLITH.\-FIGARO]}, and a number
of files used for generating the monolith are in {\tt
[FIGPACK.\-NFIGARO.\-MONOLITH]}.

To install ICL Figaro, {\tt SSC:ADAM\_PACKAGES.ICL} must include the following
package startup definition:

\begin{verbatim}
      HIDDEN PROC FIGARO
         LOAD FIG_DIR:FIGARO.PRC
         LOAD FIG_DIR:NFIGARO
      END PROC
\end{verbatim}

The second {\tt LOAD} was introduced with National Figaro 3.0-1, when
application updates were compiled into a separate monolith. It is obsolete now,
but might prove useful for future releases. To avoid any error messages at ICL
Figaro startup, National Figaro provides an empty {\tt NFIGARO.ICL}.

Callable Figaro exists only in Standard Figaro, thus National Figaro's updates
are not included in Callable Figaro.

Irrespective of the Standard Figaro system (development, extended development,
or entire), National Figaro includes all its source files.

Some added applications are not Figaro proper: NDFBAD is an A-task in the DCL
version of National Figaro. It is included in the ICL version. VWSOPEN,
VWSCLOSE, XOPEN, XCLOSE are DCL command procedures rather than programmes
written in Fortran.

When DCL Figaro is started up, the help libraries are sorted such that {\tt
NFIGARO.HLB} is searched before {\tt FIGARO.HLB} and {\tt COMMANDS.HLB} and
that all these are searched before any other libraries. This is achieved in
National Figaro's {\tt FIGARO.COM} with Malcolm Currie's ADD\-HELP\-TO\-VMS.

National Figaro provides server processes to run TVPCKG on a VAXstation which
may be either a VWS or an Xwindows workstation. The server is started with the
Figaro command VWSOPEN or XOPEN. It must be stopped before log off with
VWSCLOSE or XCLOSE. When running ICL Figaro, a premature call of VWSCLOSE or
XCLOSE may cause a TVPCKG-based application to enter an endless loop. This can
be avoided by calling VWSOPEN or XOPEN again before calling the TVPCKG-based
application. Once an application has entered the endless loop, {\tt <CTRL-C>}
forces an interrupt and one can {\tt KILL FIG\_DIR:FIGARO} once the {\tt ICL>}
prompt appears. Then call VWSOPEN or XOPEN and call the application again.

The FIGARO command does not define any logical names or symbols for Figaro
programme development, except that it defines the symbol FIGDEV. FIGDEV defines
the logical names necessary for programme development.


\subsection{Figaro software development}
\label{develop}

After Figaro has been started up with the command FIGARO, the command FIGDEV
defines a number of logical names and symbols used for programme development in
the Figaro software environment. See Section \ref{names} for details.

Most notable is the symbol FIG, i.e. the command procedure {\tt
FIGARO\_DEV:FIG.COM}, which is used to compile and link a Figaro application
into a stand-alone image ({\em i.e.\/} a DCL Figaro application). The original
{\tt FIG.COM} from Standard Figaro has been superseded by National Figaro. This
is necessary to link with the correct version of HDS. Also the linker parameter
ISD\_MAX is doubled to 192 in order to avoid excessively large executables.

A users' documentation on writing Figaro applications is the ``FIGARO
Programmer's Guide'', dated January 1990, by Keith Shortridge. Once the source
files {\tt <source>.FOR, <applic>.CON} exist in the working directory, they can
be compiled with the commands FIG and CREPAR respectively. If {\tt <source>}
differs from {\tt <applic>}, a file {\tt COMMANDS.DAT} is necessary to relate
the two. This file may also contain Figaro link qualifiers which signal special
requirements to the link procedure FIG.

It is also possible to turn a number of private Figaro applications into a
monolith to be run from ICL. Since some of the utilities one would use for
this, make specific assumptions, you have to rename files from {\tt FIGARO.*}
to something else. There exists no ready-to-use procedure like FIG for the
assembly of a private Figaro-ish monolith. The following description may,
however, be helpful.

Proceed as follows:

\begin{itemize}

\item Set your working directory (default directory) to where your monolith
will be stored ({\tt <mono\_dir>}).

\item Compile all your applications and subroutines needed for the monolith and
put them into an object library ({\tt <mono\_lib>.OLB}) in the working
directory.

\item Compile all the application connections files into {\tt <applic>.PAR}
files in the working directory.

\item You also need a file {\tt COMMANDS.DAT} that lists all the applications
you want the monolith to include and recognise as commands. Ideally this should
be a proper {\tt COMMANDS.DAT}, {\em i.e.\/} list application name, Fortran
source file name, and Figaro link qualifiers. The link qualifiers denote
special link requirements for the FIG link procedure. But {\em e.g.\/} a PARQ
qualifier---if necessary at all---would also be necessary to produce a
correct monolith interface file.

\item You need a Fortran source file for the monolith routine itself. You can
copy {\tt FIGPACK\_\-DISK:\-[FIGPACK.\-NFIGARO.\-MONOLITH.\-FIGARO]\-FIGARO.FOR}
to {\tt <monolith>.FOR}, but you\linebreak have to change the {\tt SUBROUTINE}
statement according to your monolith's name. It is necessary that your monolith
have a name different from ``FIGARO'' and that the subroutine has that very
name.

\item Set up for Figaro and ADAM software development and modify two logical
names according to present needs.

\begin{verbatim}
      $ FIGARO
      $ FIGDEV
      $ ADAMSTART
      $ ADAMDEV
      $ DEFINE FIGARO_PARS SYS$DISK:[]
      $ DEFINE FIGARO_DEV SYS$DISK:[]
\end{verbatim}

\item With the command list {\tt COMMANDS.DAT}, look for the binary interface
files {\tt <applic>.PAR} and convert these into one monolith interface file.
Once renamed to {\tt <monolith>}, compile it.

\begin{verbatim}
      $ RUN FIGPACK_DISK:[FIGPACK.NFIGARO.MONOLITH]CREIFL.EXE
      $ RENAME FIGARO.IFL <monolith>.IFL
      $ COMPIFL <monolith>
\end{verbatim}

\item With the command list {\tt COMMANDS.DAT}, create an IF-THEN-ELSE-IF block
with subroutine calls, which is to be included when the monolith routine is
compiled. At the same time an ICL procedure to start up the monolith is
created. Once this is renamed to {\tt <monolith>} it will have to be edited. It
contains numerous references to FIGARO rather than {\tt <monolith>}. For each
application there are two lines. The DEFINE command is vital, but the directory
specification {\tt FIG\_DIR} is wrong and has to be changed to {\tt
<mono\_dir>} so that your {\tt <monolith>.EXE} will be found. The DEFHELP
command may indicate a help library that should be consulted when you type
``HELP {\tt <applic>}'' at the {\tt ICL>} prompt. If you have no such help
library, the DEFHELP commands should be removed from {\tt <monolith>.PRC}

\begin{verbatim}
      $ RUN FIGPACK_DISK:[FIGPACK.NFIGARO.MONOLITH]GFIGMON.EXE
      $ RENAME FIGARO.PRC <monolith>.PRC
      $ EDIT <monolith>.PRC
\end{verbatim}

\item The crucial link
command may seem trivial now, but you have put considerable work into
assembling {\tt <mono\_lib>}. And a lot of clever linker arrangements were put
into the options file.

\begin{verbatim}
      $ FORTRAN <monolith>
      $ MLINK <monolith>,<mono_lib>/LIB,-
          FIGPACK_DISK:[FIGPACK.NFIGARO.MONOLITH]LINKMONO/OPT
\end{verbatim}

\item Now test the result:

\begin{verbatim}
      $ ADAMSTART
      $ ICL
      ICL> LOAD <mono_dir><monolith>.PRC
      ICL> <applic>
\end{verbatim}

\item For regular use, it is convenient to have an abbreviation for the {\tt
LOAD} command. That could be defined in your ICL-login file: In your login
file you would define the logial name ICL\_LOGIN:

\begin{verbatim}
      $ DEFINE/JOB ICL_LOGIN <disk><directory><my_ICL_login>.ICL
\end{verbatim}

This ICL procedure is executed each time you start ICL. In that ICL procedure
you would include a DEFSTRING command:

\begin{verbatim}
      DEFSTRING <monolith> LOAD <mono_dir><monolith>.PRC
\end{verbatim}

Once you started ICL, you just type {\tt <monolith>} to initialise your monoith
and {\tt <applic>} to run the application. You may also want to include the
ADAMSTART in your (DCL-) login file.

\end{itemize}

Obviously, this procedure is too complicated to perform in the testing and
debugging phase of programme development. But the same source code can be
debugged with the DCL Figaro software environment.

\subsection{Logical names and symbols}
\label{names}

The way in which logical names are set up, especially for development of Figaro
applications, has been changed. This is because linking Figaro applications has
now to take care of modifications made to shareable images, object libraries
and options files.

The following logical names should be in the system table:

\begin{verbatim}
      FIGARO_DISK    (device with [FIGARO] directory)
      FIGPACK_DISK   (device with [FIGPACK] directory)
      FIG_DIR        FIGARO_DISK:[FIGARO.MONOLITH.FIGARO],-
                     FIGPACK_DISK:[FIGPACK.NFIGARO.MONOLITH.FIGARO]
      FIGARO_PROG_S  FIGARO_DISK:[FIGARO]
      FIGARO_PROG_N  FIGPACK_DISK:[FIGPACK],-
                     FIGPACK_DISK:[FIGPACK.NFIGARO.FIGARO],-
                     FIGPACK_DISK:[FIGPACK.NDPROGS.NDPROGS],-
                     FIGPACK_DISK:[FIGPACK.TAURUS.TAURUS],-
                     FIGPACK_DISK:[FIGPACK.TWODSPEC.TWODSPEC]
      FIGARO_PROG_L  (the local Figaro directory)
      BIGFIG         FIGARO_PROG_S:BIGFIG.EXE
      FIGSHR         FIGARO_PROG_S:FIGSHR.EXE
      FGRSHR         FIGARO_PROG_S:FGRSHR.EXE
      TVPSHR         FIGARO_PROG_S:TVPSHR.EXE
\end{verbatim}

The following symbols should be defined in {\tt SSC:LOGIN.COM}:

\begin{verbatim}
      FIGARO  :== @FIGARO_PROG_S:FIGARO.COM
      FIGLINK :== @FIGARO_PROG_S:FIGLINK.COM
\end{verbatim}

The command FIGARO defines two logical names:

\begin{verbatim}
      PLT$UPDATE  IMMEDIATE
      SYS$INPUT   TT/USER_MODE
\end{verbatim}

Apart from some 240 applications, the command FIGARO defines the following
symbols:

\begin{verbatim}
      FIGLINK     :== @FIGARO_PROG_S:FIGLINK.COM
      FIGDEV      :== @FIGARO_PROG_N:FIGDEV.COM
      FIGARO_MODE :== 'F$MODE()'
      PPRINT      :== PRINT/NOFEED/NOHEAD
\end{verbatim}

The command FIGDEV defines the following logical names in the process table:

\begin{verbatim}
      FIGARO_DEV    FIGPACK_DISK:[FIGPACK.NFIGARO.DEV],-
                    FIGARO_DISK:[FIGARO.DEV]
      FIGARO_LIBS   FIGPACK_DISK:[FIGPACK.NFIGARO.LIBS],-
                    FIGARO_DISK:[FIGARO.LIBS]
      FIGARO_SOURCE FIGPACK_DISK:[FIGPACK.NFIGARO.DEV.SOURCE],-
                    FIGARO_DISK:[FIGARO.DEV.SOURCE]
      FIGARO_PARS   SYS$DISK:[],-
                    FIGARO_PROG_U,FIGARO_PROG_L,-
                    FIGARO_PROG_N,FIGARO_PROG_S
      FIGARO_WORK   SYS$DISK:[]
      FIGARO_CALL   FIGARO_DISK:[FIGARO.DEV.CALL]
      DTASRC        FIGARO_DISK:[FIGARO.DTA],-
                    FIGARO_DISK:[FIGARO.INCLUDES]
      DYN_SOURCE    FIGARO_DISK:[FIGARO.DYN],-
                    FIGARO_DISK:[FIGARO.INCLUDES]
\end{verbatim}

The command FIGDEV defines the following symbols:

\begin{verbatim}
      GMAIN   :== $FIGARO_DEV:GMAIN.EXE
      FIG     :== @FIGARO_DEV:FIG.COM
      FIGLINK :== @FIGARO_PROG_S:FIGLINK.COM
      CREPAR  :== $FIGARO_DEV:CREPAR.EXE
\end{verbatim}

\section{Update from 3.0-2 to 3.0-3}
\label{update}

This section describes how to proceed from an existing Standard
Figaro 3.0 and National Figaro 3.0-2 to Figaro 3.0-3.

\begin{itemize}

\item Make sure that the logical names in the system table are defined (see
Section \ref{names}).

\item Reduce your Standard Figaro installation to the ``entire system'':

\begin{verbatim}
      $ DELETE FIGARO_DISK:[FIGARO.MONOLITH.FIGARO]FIGARO.*;*,-
         [FIGARO.OLD.PGPLOT]*.*;*,-
         [FIGARO.OLD.DTA1]*.*;*,-
         [FIGARO.HDS]*.*;*
\end{verbatim}

\item If you do not need the sources of the Figaro libraries, reduce your
Standard Figaro installation to the ``extended development system'' (NOTE: do
{\it not} delete the contents of directories {\tt [FIGARO.\-DEV],
[FIGARO.\-FIGARO], [FIGARO.\-INCLUDES], [FIGARO.\-LIBS],
[FI\-GARO.\-MONOLITH]}):

\begin{verbatim}
      $ DELETE FIGARO_DISK:[FIGARO.CNV]*.*;*,-
         [FIGARO.DSA]*.*;*,[FIGARO.DSK]*.*;*,-
         [FIGARO.DTA]*.*;*,[FIGARO.DUT]*.*;*,-
         [FIGARO.DYN]*.*;*,[FIGARO.FIG]*.*;*,-
         [FIGARO.FIT]*.*;*,[FIGARO.GEN]*.*;*,-
         [FIGARO.GKD]*.*;*,[FIGARO.ICH]*.*;*,-
         [FIGARO.JT]*.*;*,[FIGARO.MEM]*.*;*,-
         [FIGARO.NAG_FIX]*.*;*,[FIGARO.PAR]*.*;*,-
         [FIGARO.TAPES]*.*;*,[FIGARO.VARS]*.*;*
      $ DELETE FIGARO_DISK:[FIGARO.TVPCKG...]*.*;*/EXCLUDE=.DIR
\end{verbatim}

\item If you do not need the sources of the Figaro applications, reduce your
Standard Figaro installation to the ``development system'' (NOTE: do {\it not}
delete the contents of the directory {\tt [FIGARO.\-DEV]} itself):

\begin{verbatim}
      $ DELETE FIGARO_DISK:[FIGARO.DEV.*]*.*;*
\end{verbatim}

\item Update the Standard Figaro files in {\tt FIGARO\_DISK:[FIGARO.FIGARO]}:

\begin{verbatim}
      $ DELETE FIGARO_DISK:[FIGARO.FIGARO]*.OLB;*,*.COM;*,*.EXE;*
      $ BACKUP <save_set_1> FIGARO_DISK:[FIGARO.FIGARO]
\end{verbatim}

\item Update National Figaro:

\begin{verbatim}
      $ DELETE FIGPACK_DISK:[FIGPACK.NFIGARO...]*.*;*/EXCLUDE=.DIR
      $ BACKUP <save_set_2> FIGPACK_DISK:[FIGPACK...]
\end{verbatim}

\end{itemize}

\end{document}
