\documentstyle[11pt]{article}
\pagestyle{myheadings}

%------------------------------------------------------------------------------
\newcommand{\stardoccategory}  {Starlink System Note}
\newcommand{\stardocinitials}  {SSN}
\newcommand{\stardocnumber}    {14.1}
\newcommand{\stardocauthors}   {P N Daly}
\newcommand{\stardocdate}      {27 October, 1992}
\newcommand{\stardoctitle}     {Installing the CGS4DR V1.6--0 Software}
%------------------------------------------------------------------------------

\newcommand{\stardocname}{\stardocinitials /\stardocnumber}
\renewcommand{\_}{{\tt\char'137}}     % re-centres the underscore
\markright{\stardocname}
\setlength{\textwidth}{160mm}
\setlength{\textheight}{230mm}
\setlength{\topmargin}{-2mm}
\setlength{\oddsidemargin}{0mm}
\setlength{\evensidemargin}{0mm}
\setlength{\parindent}{0mm}
\setlength{\parskip}{\medskipamount}
\setlength{\unitlength}{1mm}

%------------------------------------------------------------------------------
% Add any \newcommand or \newenvironment commands here
%------------------------------------------------------------------------------

\begin{document}
\thispagestyle{empty}
SCIENCE \& ENGINEERING RESEARCH COUNCIL \hfill \stardocname\\
RUTHERFORD APPLETON LABORATORY\\
{\large\bf Starlink Project\\}
{\large\bf \stardoccategory\ \stardocnumber}
\begin{flushright}
\stardocauthors\footnote{{\em Present Address:}
 {\sf Joint Astronomy Centre, 660 N A`oh\={o}k\={u} Place, University Park,
  Hilo HI 96720, USA.}}\\
\stardocdate
\end{flushright}
\vspace{-4mm}
\rule{\textwidth}{0.5mm}
\vspace{5mm}
\begin{center}
{\Large\bf \stardoctitle}
\end{center}
\vspace{5mm}

%------------------------------------------------------------------------------
%  Add this part if you want a table of contents
  \setlength{\parskip}{0mm}
  \tableofcontents
  \setlength{\parskip}{\medskipamount}
  \markright{\stardocname}
%------------------------------------------------------------------------------

\section{Requirements}

\subsection{Software Revision Levels}

CGS4DR relies heavily upon ADAM and Figaro. The release described in this
document requires, as a minimum, {\tt ADAM v2.0-1}, {\tt FIGARO v3.0-5} and
{\tt VMS v5.5-1.} If your software revision levels are below these, you
{\em must} upgrade before installing CGS4DR.

Note also that CGS4DR comes shipped with a private version of SMSICL and
{\em will not} work with the {\sl Starlink} released version. Manager's
need not take any action on this, the system will choose the correct version.
It is hoped that {\sl Starlink} will make this version of ICL the default
in the near future.

\subsection{Preferred Hardware}

Ideally, the software should run on a single user VAXstation having
at least 24Mb of main memory and a colour display but it will run on a
VT100-compatible terminal and display to any supported PGPLOT graphics device.

\subsection{Disk Space}

Once installed, the software will occupy $\sim$ 9500 blocks of disk space
(without the source code and libraries as distributed by default).

\subsection{Required Process Quotas}

CGS4DR runs as four (or more) subprocesses and may require
some hefty quotas. A typical UAF entry might look like this:

\begin{center}
\begin{tabular}{|l r|l r|l r|}
\hline
\multicolumn{6}{|c|}{\sl UAF Quotas for CGS4DR} \\
\hline
Maxjobs     & 0      & Fillm    & 200  & Bytlm    & 50000 \\
Maxacctjobs & 0      & Shrfillm & 0    & Pbytlm   & 0 \\
Maxdetach   & 0      & BIOlm    & 100  & JTquota  & 4096 \\
Prclm       & 100    & DIOlm    & 100  & WSdef    & 1024 \\
Prio        & 4      & ASTlm    & 500  & WSquo    & 2048 \\
Queprio     & 0      & TQElm    & 10   & WSextent & 4096 \\
CPU         & (none) & Enqlm    & 100  & Pgflquo  & 75000 \\
\hline
\end{tabular}
\end{center}

\subsection{Reserved Nodenames}

Note that any nodenames beginning with the characters {\tt IRT} are assumed
to be part of the UKIRT cluster and will invoke CGS4DR in `online' mode.
In general, users do not want to do this so you should rename any offending
nodes before installing the software. This may be the first time a software
product has necessitated a DECnet change!

\section{Installing the Software}

\subsection{Unpacking the Software}
The software will usually be distributed via a `Starlink Software Change'
notice. The top-level directory of the software must be called
{\tt [CGS4DR\_DEV]} to maintain compatability with the Hawai`i version.

{\bf NB: Delete all previous copies of the CGS4DR system}.

Unpack the software into a suitable directory:

\begin{verbatim}
  $  BACKUP/LOG CGS4DR_V160.BCK/SAVE DISK$STAR:[*...]*.*.*/OWNER=ORIGINAL
\end{verbatim}

Then, define some system-wide logical names (and put them in
{\tt LSSC:STARTUP.COM}):

\begin{verbatim}
  $  DEFINE/SYSTEM/TRANSLATION=(T,C)   CGS4DR_ROOT     DISK$STAR:[CGS4DR_DEV.]
  $  DEFINE/SYSTEM/TRANSLATION=(T,C)   CGS4DR_DEVROOT  DISK$STAR:[CGS4DR_DEV.]
  $  DEFINE/SYSTEM                     CGS4DR_COM      CGS4DR_ROOT:[000000]
\end{verbatim}

If {\tt DISK\$STAR} is a concealed device remember to translate it first as VMS
does not like logical names defining concealed devices on concealed devices.
You will also need to place CGS4DR definitions in {\tt LSSC:LOGIN.COM}:

\begin{verbatim}
  $ @CGS4DR_COM:SYLOGIN
\end{verbatim}

\subsection{PostScript, Canon, LN03 and Inkjet Hardcopy}

All hardcopy printing is enabled via the {\tt PRINT\_HARDCOPY} item (with the
exception of the inkjet harcopy facility).
This option requires a print command, a graphics file and a queue name.
For PostScript, Canon and LN03 laser printers the system will set defaults
as follows:

\begin{center}
\begin{tabular}{|l|c|c|c|}
\hline
\ \ \ &  {\bf Print Command} & {\bf Graphics File} & {\bf Queue Name} \\
\hline
{\bf Canon} & {\sf print/delete/nonotify/noflag/passall} & {\sf canon.dat} & {\sf sys\$laser} \\
\hline
{\bf LN03} & {\sf print/delete/nonotify/noflag/passall} & {\sf ln03.plt} & {\sf sys\$laser} \\
\hline
{\bf PostScript } & {\sf print/delete/nonotify} & {\sf gks\_72.ps} & {\sf sys\$postscript} \\
\hline
\end{tabular}
\end{center}

These defaults can be altered to suit your site by two methods:

\begin{itemize}
\item If you know a little about ICL you can add further defaults or modify the
above queues. The file to edit is {\tt CGS4DR\_ROOT:[PRC]CGS4\_REDUCTION.ICL}
and the procedure to edit is called {\sf hardcopy\_entry}. If you don't feel
comfortable doing this, send me an e-mail and I'll include further options in
a future release.
\item If your generic postscript queue is called {\tt SYS\_POSTSCRIPT},
you can {\tt \$ DEFINE/SYSTEM SYS\$POSTSCRIPT SYS\_POSTSCRIPT} and leave the
defaults in the software alone. This applies to {\tt SYS\$LASER}, too.
\end{itemize}

If you have an inkjet hardcopy printer and have installed Clive Page's
IKONPAINT software (SUN/71) this is another option you can enable in the
menus. The file to edit is called {\tt CGS4DR\_ROOT:[SCT]CGS4\_REDUCTION.SCT}
and, note that, comments begin with an {\tt *} character so just un-comment
the line (by removing the asterisk):

\begin{verbatim}
  * option hardcopy_inkjet   switch  hardcopy_inkjet  hardcopy_inkjet
\end{verbatim}

Note that this facility is {\em not} guaranteed as the Joint Astronomy Centre
has no such plotter with which to test the software.

\subsection{Old Noticeboard Files}

Prior to testing the installation, delete any files called
{\tt CRED4\_NOTICEBOARD.NBD} in your (current) default directory.
Do {\em not} delete that file from within the CGS4DR tree!

The reason for deleting the file is that a new one will be created which
is backwards compatible with previous versions of the system. If there is
such a file already in your (current) default directory and you do {\em not}
delete it, the software will fail
to load properly and you will not be able to run CGS4DR.

\subsection{Testing the Installation}

No astronomical data is distributed with the software but it can be tested
with the following keystrokes. First, fire up the software using:

\begin{verbatim}
  $ CGS4DR  OFFLINE  SYS$SCRATCH:[SYSTEM]  POSTSCRIPT_L
\end{verbatim}

where {\tt SYS\$SCRATCH:[SYSTEM]} is a scratch directory. If you have a colour
workstation, you can replace {\tt POSTSCRIPT\_L} with {\tt XWINDOWS} or
something similar\footnote{The benefit of using {\tt XWINDOWS} or {\tt VWS}
is that the software will load a colour test image upon completion of the
startup. This test image is a simple colour table ramp.}.
If the CGS4DR software starts without error the display will show:

\begin{verbatim}
+ SMS CGS4 Data Reduction System V1.6-0 ------+
|setup           reduce          engineer     |
+---------------------------------------------+




-------------------------------------------------------------------------------
               WELCOME TO THE CGS4 DATA REDUCTION SYSTEM V1.6-0

                     Please wait while the system loads



Loading CGS4_EXE:CRED4 into CRED4
Loading CGS4_EXE:RED4 into RED4
Loading CGS4_EXE:ENG4 into ENG4
Loading CGS4_EXE:P4 into P4
Enter the  `setup'  menu to set data reduction parameters
Enter the  `reduce' menu to start automatic data reduction
-------------------------------------------------------------------------------


Use ENTER to select or PF1 to return to previous menu.      (Do not press PF4!)
\end{verbatim}

Enter any menu by using the {\tt ENTER} key on the auxiliary keypad.
Instructions on how to get out of the menu are given in reverse video at the
bottom of the screen. As a general rule, if you are in a menu containing
further menus you  can use PF1 whereas for an action, use PF4 (abort). Scanning
through menus in  this way should produce no errors.

Enter the {\tt REDUCE} menu. The following text will appear:

\begin{verbatim}
* Automatic data reduction will NOT be started *

To start automatic data reduction, use pause_continue_reduction
from within the `reduce' menu or set PAUSE_ON_ENTRY to FALSE
Data reduction is currently STOPPED.
The data reduction queue is empty.
\end{verbatim}

This tells you that the DR system is not running (sic) and that the queue is
empty (as it should be). Now position the cursor over the
{\tt CANCEL\_ALL\_QENTRIES} item using the (down) arrow
key and press {\tt ENTER}. The queue will be deleted; it should report that it
was already empty.

Using the arrow keys position the cursor over {\tt PAUSE\_CONTINUE\_REDUCTION}
and press {\tt ENTER}. The message is:

\begin{verbatim}
* Automatic data reduction started *
* The bad pixel mask # will be used *
* No linearisation will be carried out *
\end{verbatim}

Exit the {\tt REDUCE} menu (using PF1)  and enter the {\tt ENGINEER} menu.
Enter the {\tt SOFTWARE\_TESTS} menu and select {\tt TASK\_STATUS}. Select
{\tt ALL} (for all tasks) and PF1 out of this menu thus setting the action
going. The software will report on the status of four tasks: CRED4, RED4,
ENG4 and P4. The output will be similar to (for each task):

\begin{verbatim}
The CRED4 task is idle
CRED4 Task : CGS4 Data Reduction Software V1.6-0
Created on : 16 September 1992
Created by : Phil Daly, JACH::PND
CRED4 Task : The uncached CRED4 task is OK
\end{verbatim}

All four tasks shoule be {\tt IDLE} and {\tt OK}. Enter the {\tt TASK\_RESET}
menu and select {\tt ALL} and PF1 out to start the action going. The software
will reset all tasks and produce a display (for each task) of the form:

\begin{verbatim}
Resetting CRED4 task
\end{verbatim}

Finally, select the {\tt LIST\_NOTICEBOARD} item and press {\tt ENTER}. The
action will start straight away and report the contents of the DR noticeboard.
It is important that no error is reported during this phase as that might
imply a corrupt noticeboard. Many items are output to the scrolling region
of the SMS screen.

To exit the software use PF1 until asked :

\begin{verbatim}
Kill tasks and exit from SMS? /N/
\end{verbatim}

Answer {\tt Y} and wait. The system will respond with:

\begin{verbatim}
Do you mean to exit from SMS? (y/n)
\end{verbatim}

Stroking just the {\tt Y} key is enough to exit at this stage.

If there were no errors reported during the above, you have successfully
installed CGS4DR.

\subsection{Post the CGS4DR News Item}

Once the installation has been completed sucessfully, you should post the
news item contained in {\tt CGS4DR\_COM:CGS4DR\_V160.NEWS.}

\section{Running CGS4DR from a Sun Workstation}

CGS4DR does {\em not} run under the Unix environment at the present time
({\em i.e.} \stardocdate). That fact will not deter users from saying
something like, `... but I have seen Joe Bloggs running it on a Sun'.

CGS4DR {\em can} run on a Sun workstation when that workstation acts {\em
purely as an X-terminal}. The following is the JACH recipe to enable such a
facility --- it probably will not work in the UK if TCP/IP names cannot  be
resolved (but I have no way of telling!). It also relies on the DEC and  Sun
keyboard mapping utilities provided by AAO which should have been  distributed
by {\sl Starlink}. Your Sun {\em must} be running OpenWindows. In the  example
that follows, the Sun workstation has a name of kaua.jach.hawaii.edu and a
prompt of {\sf \%} whereas the VAX has a nodename of KAMAKA and a prompt of
{\sf \$}.

First, login to the Sun and let OpenWindows fire up.
Before starting a session, you will need to allow security access to
your Sun from a VAXen server. This is done by placing a suitable entry
into the .rhosts file in your home directory such as:

\begin{verbatim}
kamaka.jach.hawaii.edu pnd
\end{verbatim}

You may require similar security options on your VAXen.

Now create a SunOS window on the Sun using MB3 and the {\tt PROGRAMS} option.
The item to look for is {\tt SHELL ...} or similar.

Mapping a Sun keyboard to a DECterm is performed using the distributed
{\sf dk} and {\sf sk} images. If you do not have these, consult your
system manager. The definitions should be something like:

\begin{quote}
{\scriptsize \tt
  alias dk 'xmodmap /usr/bin/deckeymap;xset  r 28;xset  r 29;xset  r 30;xset  r 105'

  alias sk 'xmodmap /usr/bin/Sunkeymap;xset -r 28;xset -r 29;xset -r 30;xset -r 105'
}
\end{quote}

Clearly, the command {\tt dk} is designed to map a DECterm keyboard and
{\tt sk} will unmap it back to a Sun keyboard. For a Sun keyboard mapped as a
DECterm, the top row of keys on the auxiliary keypad become the arrow
keys required by SMS. The mapping is:

\begin{center}
\begin{tabular}{|l|c|c|c|c|}
\hline
\multicolumn{5}{|c|}{\sl DECterm / Sun Keyboard Mapping} \\
\hline
{\bf Sun } & {\sf Pause}   & {\sf PrSc}      & {\sf Scroll}    & {\sf NumLock} \\
\hline
{\bf DEC } & {\sf UpArrow} & {\sf DownArrow} & {\sf LeftArrow} & {\sf RightArrow} \\
\hline
\end{tabular}
\end{center}

Click on the window you have just created and do the following:

\begin{verbatim}
  %  xhost  kamaka.jach.hawaii.edu
  %  dk
  %  telnet  kamaka
\end{verbatim}

Login to the VAX as usual and set the display back to your Sun workstation:

\begin{verbatim}
  $  set  display/create/node=kaua.jach.hawaii.edu/transport=tcpip
  $  create/terminal/detach
  $  logout
\end{verbatim}

A terminal window should have appeared on your Sun before logging out. You
do not need the original Sun login window now, either, so you can logout
of that if you so desire. Give the new DECterm input focus and type:

\begin{verbatim}
  $  set display/create/node=kaua.jach.hawaii.edu/transport=tcpip
  $  define/job  decw$display  'f$trnlnm("DECW$DISPLAY","LNM$PROCESS_TABLE")'
  $  xmake  xwindows -title "CGS4 Data Reduction"  -geometry 1024x768 -col 240
  $  cgs4dr  offline  disk$ukirtdata:[cgs4_data.19920420]  xwindows
  .
  . (do reduction on data in directory disk$ukirtdata:[cgs4_data.19920420])
  .
  $  xdestroy  xwindows
  $  set  display/delete
  $  deassign/job  decw$display
  $  logout
\end{verbatim}

Note that the DECterm and X-window need not be destroyed if you require them
at some point in the future.

\end{document}
