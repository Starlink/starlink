\documentstyle {article}
\markright{SGP/13.3}
\setlength{\textwidth}{153mm}
\setlength{\textheight}{220mm}
\setlength{\oddsidemargin}{3mm}
\setlength{\evensidemargin}{3mm}
\pagestyle{myheadings}

\begin{document}
\thispagestyle{plain}
\noindent
SCIENCE \& ENGINEERING RESEARCH COUNCIL \hfill SGP/13.3\\
RUTHERFORD APPLETON LABORATORY\\
{\large\bf Starlink Project\\}
{\large\bf Starlink General Paper 13.3}
\begin{flushright}
R J Dickens, R A E Fosbury, P T Wallace\\
26 March 1987
\end{flushright}
\vspace{-4mm}
\rule{\textwidth}{0.5mm}
\vspace{10mm}
\begin{center}
{\Large\bf Starlink Applications Software Development}
\end{center}
\vspace{10mm}
\section {INTRODUCTION}
The purpose of this paper\footnote{This is a minor revision by M D Lawden of a
paper first issued on 7th November 1980. No attempt has been made to bring it
fully up-to-date.} is to announce to the astronomical community the
present plans for the development of applications software within Starlink.
The paper follows widespread consultation with astronomers---through a survey
carried out in December 1979, visits to many university departments and
establishments during the first half of 1980 and a workshop held in June 1980.
In some areas we can be more specific than others and no doubt plans and
priorities will undergo major revision as time passes.
We believe that widespread and up-to-date knowledge of the intentions and
progress of work on applications is essential if we are effectively to harness
the effort and expertise available amongst UK astronomers and to maintain
uniformity of style and high standards everywhere.
Comments on these proposals and continual feedback as Starlink develops will
therefore be greatly appreciated.

By `Starlink software' we mean generally useful software, at some
level or other centrally supported, and conforming to Starlink conventions.
In most cases (certainly for image processing programs) Starlink software will
be designed for use in the Starlink software environment; ie.\ using the
Starlink subroutine interfaces (which have been implemented---see reference
1), the adopted Starlink graphics package (GKS) and the Starlink Command
Language (SCL), both yet to be implemented.
Individual users are, of course, free to adopt whatever procedures they prefer
in order to achieve their immediate astronomical objectives.
However, the use of ad hoc techniques in place of recognised Starlink methods
will seriously compromise ease of use and uniformity, and will preclude
centralised support.
It should also be pointed out that the Starlink Command Language will not be
usable without the subroutine interfaces.
Use of the latter (which essentially define how programs should handle
interaction and do much of their bulk input/output) will be straightforward and
is the first essential step for compatibility with Starlink.
All prospective programmers are strongly urged to study reference 1 which
describes the subroutines.
This may be obtained from any site manager who will also be able to offer
advice.

It is Starlink's objective to respond to the real needs of the astronomical
community with well-designed and properly supported applications software,
presented in a uniform style and with ease of use combined with flexibility.
\section {THE MATURE SYSTEM}
\subsection {Overview}
The main components of the fully-fledged system are illustrated in figure 1.
The DEC VAX computer system will be run under the standard VMS operating system
and associated utilities, augmented by extensive subroutine libraries
(eg.\ NAG), a graphics package (GKS) and the Starlink software environment
(SSE).
Other program development tools (for example language preprocessors) may also
be supported as Starlink standards.
Operating within this environment will be the ensemble of applications software,
a central feature of which will be the programs for standard manipulations of
n-dimensional images, including the instrument-independent components of the
major data reduction and analysis applications, for example spectroscopy,
surface and stellar photometry, and polarimetry.
Data of great variety will be `accepted' from a wide range of astronomical
instruments and converted into the standard internal image format for access
via the Starlink subroutines.
Another group of programs will perform specialised pre-processing of data from
various instruments (eg.\ echelle spectrographs, CCD cameras, radio telescopes).
There will be a variety of astronomical utility programs---astrometry,
starfield overlay plotting, earth velocity corrections, etc.  Many Starlink
programs will require  standard information---wavelength tables, star
catalogues, etc.---and this will be provided.
Access to such information through a centrally maintained and supported database
might be Starlink's first step in what promises to be a major growth area which
may eventually encompass very large databases (for example observing archives)
which would be stored centrally and accessed through the network links.
\subsection {Progress to maturity}
In many biological organisms, gaining full maturity occupies a considerable
fraction of the total lifespan.
This is also true of many complex computer systems and it it expected that
Starlink will be no exception.
We outline below the various stages (which are, in various degrees, overlapped)
envisaged in the development of Starlink applications so that the effort to be
allocated to each area can be seen in perspective.
However, the timescale is long and strategies will inevitably change in the
light of experience.
The various development stages are given below: details of the different
categories of application may be found in section 3.
\begin{enumerate}
\item Initial implementation and development of various existing data reduction
packages on an ad hoc basis.
This is necessary while system software (environment, graphics, special drivers,
etc.) is under development; it provides experience with the VAX and provides
facilities useful to the astronomer almost immediately.
We are currently in this first stage of development and many packages are
productively running on Starlink VAXs.
Of course, ad hoc development on Starlink will continue indefinitely and
Starlink-endorsed implementation of stand-alone systems (eg.\ IDL, TVS) will go
on for another year at least.
Distribution and support of certain existing ad hoc packages will be undertaken
by Starlink as an interim measure in especially urgent cases (for example the
reduction of IPCS data).
\item Now that the interfaces to parameters and bulk data have been implemented,
we are entering the second stage of development in producing within Starlink a
few supported primitive data manipulation programs.
This will enable us to test the software interfaces, demonstrate how they work
and gain experience with applications we understand relatively well.
This phase will last perhaps three months.
During this period some basic astronomical utilities and subroutines will be
provided; work has already begun on the suite of Starlink programs that will
read data tapes in FITS format (reference 2) and various foreign formats
(IPCS, SDRSYS, etc.), creating images in Starlink internal format.
\item This may be followed by an implementation of an existing 1D/2D package
using the Starlink environment, further testing the software interfaces while
providing a useful facility at low cost.
\newpage
\vspace{110mm}
\begin{center}
STICK FIGURE 1 HERE
\end{center}
\newpage
\item The Starlink 1D/2D basic data reduction package will be implemented, in
many cases by adopting or refining an already running program and abandoning
others which no longer have unique roles.
As this process (which it is hoped will be underway by early 1981) progresses,
a Starlink system of uniform appearance to the user will emerge and the scars
left by the earlier ad hoc phase will fade.
A similar process will occur within the utility area.
\item We now enter the long term development stage where facilities tailored to
particular instruments (including measuring machines), more advanced image
processing, and specialised data analysis packages appear.
During this period, continual development, refinement and rationalisation will
take place.
This is the longest and most important phase in the development of Starlink
applications software.
\item We expect an increasing sophistication in Starlink as effort becomes
available to take advantage of opportunities provided in a variety of fields.
For example the need to access standard catalogues should lead to
experimentation with database techniques.
More advanced and rapid communications technology may allow access to large
central databases, as well as to other computers (eg.\ ICF, DAP) and perhaps to
data gathering facilities (AAT, LPO).
\end{enumerate}
There is expected to be increasing international contact during the development
of Starlink.
There will be great benefits if a high level of software standardisation is
achieved between Starlink and other groups.
\section {CATEGORIES OF APPLICATIONS SOFTWARE}
In this section we give details of the broad classifications of applications
software shown in figure 1.
\subsection {Data acceptance and preliminary reduction}
The programs for data acceptance into the standard internal image format range
from simple tape reading programs to difficult and complex preliminary
reduction tasks such as the extraction of echelle orders from microdensitometer
scans of a photographic plate.
The data acceptance programs will in general be specific to an individual
instrument.
We expect that Starlink will become less and less directly concerned with these
special-to-instrument problems since, as telescope instrumentation computers
become more powerful, a greater proportion of the data leaving the observatory
will have had its own peculiar instrumental distortions removed, either in real
time or after local preliminary reduction.

Starlink has already decided to adopt FITS (Flexible Image Transport System ---
see reference 2) as a standard data format, both on tape and as a model for
internal representation on disc (reference 1).
We strongly urge users to adopt a FITS representation of image data whenever
possible, even if not using Starlink software.

The chief problem in converting non-FITS data into Starlink internal form will
be the choosing of suitable FITS keywords to identify the various pieces of
header information associated with the different data sources.
It will be necessary to liaise with leading FITS users elsewhere in the world
to try and agree on the form of optional keywords in cases where the
information is likely to have a wide applicability.

We are aware that not all astronomical data are best expressed as numbers
regularly distributed throughout an n-dimensional data array.
Another frequently encountered form is a list of events or objects, each having
an associated set of parameters.
An example is an X-ray `image' from the Einstein observatory which consists of
a list of photon detections, each event being tagged with an x,y position, an
energy, and an arrival time.
This {\em can} be represented as a 4096 x 4096 x 32 x t array (where t is the
experiment duration divided by the time resolution), but this would usually be
grossly inefficient since the array would be almost empty.
Another example is the type of output generated by some of the plate-scanning
machines---a list of objects, each with a position, brightness and shape.
A simpler case is the string of photon arrival time from the AAT
ultra-high-speed photometer.
Although FITS is in principle capable of handling these data, it may not always
be the most convenient or efficient format to use.

Some important data sources are listed below with remarks on the required
acceptance programs:
\begin{description}
\begin{description}
\item [IPCS]: Image Photon Counting Systems are currently operating at the AAT
and SAAO, and the Boksenberg travelling machine also operates in various
locations.
Each of these systems has its own slightly different data format evolving
slowly with time.
In addition, the AAO also exports IPCS data in FITS or its own SDRSYS format.
\item [IDS]: Most Image Dissector Scanner data analysed in the UK come from the
AAO in SDRSYS format.
In addition, Starlink must be able to accept data from the ESO IDS (in IHAP or
FITS format) and from Kitt Peak (also presumably in FITS).
\item [Other spectroscopic detectors]: Examples include the Las Campanas
intensified reticon (on Tandberg cassette) and the Mount Stromlo PCA (in FITS).
\end{description}
\end{description}
Aside from the relatively simple problem of reading the tapes, the particular
processes special to these detectors are those of coincidence correction,
flat-fielding, and correction for geometrical distortion.
The currently outstanding problem is that of the proper and efficient handling
of the geometric correction of the large 2D spectroscopic images produced by the
IPCS.
This is one of the first tasks of the spectroscopy SIG.
\begin{description}
\begin{description}
\item [IUE]: The reduction of data from the International Ultraviolet Explorer
satellite poses its own set of problems.
There are two stages in which Starlink may play a role.
The first is the geometric and photometric correction of the raw images and the
assigning of wavelength scales.
The NASA-supplied software is known to contain errors, making investigators
suspicious of the pre-calibrated data.
This software is now being completely rewritten by NASA and it may not be
necessary for Starlink to consider providing its own software for handling the
raw data; the IUE SIG will make its own recommendations on this matter.
The second problem is that of the extraction of the spectra from the corrected
echelle images.
Here there are already a number of competing schemes and, although this is a
problem common to 2D detectors used for spectroscopy, there is no doubt that the
IUE problem is especially difficult.
\item [Echelle extraction]: The problem of extracting spectra from the image
produced by echelle or echelette devices is not restricted to IUE.
The AAT RGO spectrograph has an echelette mode and the Cassegrain spectrograph
on the La Palma 2.5m will also include this option.
Two different approaches to this problem have been tried.
The first is to assign the extraction problem to the detector (this could in
principle be done with the IPCS) or the plate measuring machine (eg.\ the PDS)
and control the scan to read the relevant data directly.
The second is to accept a complete 2D image and, like IUE, leave the computer
to do all the work.
It is likely that the second method will become the norm.
\item [Plate scanning machines]: The plate scanners in most general use are the
PDS microdensitometer (at RGO and AAO), the Cosmos machine (ROE) and the APM
(Cambridge).
All of these machines have a very large potential data output rate when used to
digitise plates onto magnetic tape and on Cosmos and APM it is usual instead to
preprocess the data in special hardware, outputting lists of parameterised
objects as discussed above.
If the machines do produce pixel data, there is the relatively minor problem of
assembling them into a Starlink image, plus the much more difficult problems of
defect removal and correction for the emulsion characteristics.
\item [Taurus]: This is a scanning, imaging Fabry-Perot device which, in
conjunction with a suitable 2D detector (namely the IPCS), produces data cubes
that can be rearranged into 3D arrays with two spatial coordinates and one
wavelength coordinate.
The onerous rearrangement process is already being performed on a Starlink VAX.
The resulting monochromatic images, as individuals, present no special
processing difficulties, but there are formidable and unsolved problems of
storage and display for the 3D arrays themselves.
\item [Other non-photographic detectors]: The IPCS is already used as an area
photometer, both as part of Taurus and directly with broad and narrow filters.
In the near future there are going to be several more 2D electronic detectors
available to the UK community.
They will each have their own particular problems of flat-fielding and defect
removal (eg.\ cosmic ray event removal in CCD pictures).
Quite soon we expect one of the staple diets of Starlink to be CCD images in
FITS format.
\item [Surface polarimetry]: The Durham polarimeter uses the McMullan
electronographic camera to produce sets of matched pictures which are scanned
with a PDS.
The preprocessing is concerned largely with field matching and defect removal.
\item [Synthesis radio telescopes]: Most operating synthesis telescopes have
their own tailored systems for handling data in the Fourier plane, calibrating
and performing the 2D transforms.
However, once the maps are made Starlink clearly has a role to play in
interactive analysis.
If the data are accepted after the map making stage then the problems are
probably not severe.
\item [Conventional photometers]: With the exception of the high speed
photometry strings already discussed, the data output rate from single-object
instruments is not very high, posing few problems for Starlink.
\item [X-ray instruments]: We have already mentioned the particular data format
produced by X-ray imaging instruments; just how Starlink will handle this format
has yet to be decided.
It is clear, however, that many non X-ray specialists will want to process data
from, for example, the Einstein satellite, and no doubt Starlink facilities will
be developed to allow this to be done.
\item [The space telescope]: We expect the data from the ST to pose no
particular problems for Starlink either in terms of quantity or data acceptance
procedures.
Because of the extreme astronomical interest of the data, however, ST-specific
software will have very high priority.
\end{description}
\end{description}
The above list of data sources covers most of the things astronomers think they
want to do now.
It is clear, however, that with the rapid development of detectors and
improvements in observing techniques, these requirements will evolve.
The organisation of the software development effort must be sufficiently
flexible to respond to this evolution.
\subsection {Data reduction, manipulation and analysis}
We assume that the data are residing on disc in the standard internal image
format, having perhaps been partially calibrated either by the local instrument
software or by the Starlink data acceptance facilities.
The next stage consists of instrument independent data reduction and analysis
software.
It is not our intention in this paper to list detailed specifications for a
complete list of programs, but rather to point out those areas where the
Starlink applications staff and the special interest groups (SIGs) will have to
concentrate their early efforts.

There is a set of primitive operations that will be required for almost all
types of 1-D and 2-D data reduction.
These are fairly obvious and will be the first application programs to be
written by Starlink applications staff.
They will include such operations as:
\begin{description}
\begin{description}
\item [image-image arithmetic]
\item [image-scalar arithmetic]
\item [smoothing and other filtering]
\item [image descriptor editing]
\item [pixel editing]
\item [image display and plotting]
\item [trim, compress, expand]
\item [background/continuum estimation]
\item [function fitting]
\item [coordinate transformations]
\end{description}
\end{description}
Because of their frequent use, these primitives will have to be written to
operate in a general and efficient manner.

The higher level applications will address specific data reduction and analysis
problems, frequently identified by the SIGs.
Many of these programs will be adaptations of existing software (at least for
the time being) and the documentation Starlink has been furnished with on these
systems (see Appendix II) will be a rich source of information.
These existing systems encapsulate an enormous quantity of expertise and
experience which Starlink intends to exploit.
Aside from those operations which are commonplace in existing systems, there is
an urgent need for new approaches to some problems, especially where the power
of the VAX can offer a quantum jump in performance.
An example is the problem of the geometric correction and wavelength calibration
of 2D spectroscopic images.
At present, the individual spectra in the set are calibrated one by one,
usually involving an unacceptable amount of interactive terminal time as well
as producing less than optimum results.
This approach arose because most of the current spectroscopic data reduction
systems have grown out of an initial requirement to handle the 1D data produced
by the first generation of digital detectors.
Starlink presents the opportunity to handle 2D reductions properly.
Unfortunately, Starlink has comparatively little documentation on 2D photometry
packages for the general user; advice on this area will come from the 2D
photometry SIG and the image processing and enhancement SIG.

The set of software for the astronomical analysis of reduced data is essentially
open ended and we expect much of it to be written by astronomers with an
immediate need to extract the scientifically interesting information and write
papers.
If the whole data reduction process has been carried through using Starlink,
there is clearly an incentive for the astronomer to write the analysis programs
within the Starlink environment.
Some of the existing packages do offer some data analysis software, but this is
obviously an area in which Starlink can rapidly accumulate a pool of readily
available astronomical knowledge.
With the difficulties of data access, interaction and display largely removed
by the Starlink environment, the writing of such analysis software will in many
cases be considerably simplified.
An example of such an analysis program might be the fitting of a synthetic
spectrum to observed data.
We would also like to see experiments with new techniques---for example
nonlinear restoration---which might eventually be incorporated into Starlink's
standard repertoire of data reduction processes.
The reports of the SIGs will be the place to find details of proposed data
analysis applications.
\subsection {Utilities}
Observing astronomers rely heavily on using astronomical utility programs both
to prepare for observing runs and to interpret the resulting data.
Again, Starlink offers the first real opportunity for the community to share a
standard, optimal and completely trustworthy set of programs, fully exploiting
the Starlink VAX and peripherals.
Here are some examples:
\begin{description}
\begin{description}
\item [celestial coordinate conversions]
\item [star overlay plotting]
\item [topocentric position and velocity corrections]
\item [time conversions]
\item [predictions of celestial phenomena]
\item [various grades of astrometry]
\item [guide probe and telescope offset predictions]
\end{description}
\end{description}
Aside from these astronomical utilities there will be an extensive set of
subroutine libraries, offering mathematical (NAG is already available),
statistical, graphical and astronomical functions.
\subsection {Database}
It is clear that the community would benefit greatly through having organised
access to a large body of astronomical data and catalogue information.
The Starlink network gives an opportunity to provide centralised maintenance of
a generally available database.
However, the magnitude of the task is daunting and it is apparent that the
existing Starlink organisation is too small on its own to embark on an ambitious
project in this area.
We can no doubt start in a small way by offering some star catalogues and
lists of spectral line wavelengths but anything more demands almost a separate
organisation working in close collaboration with Starlink.
Here are some of the things that may be provided without recourse to especially
expensive techniques:
\begin{description}
\begin{description}
\item [catalogues of stars---positional and photometric]
\item [catalogues of non-stellar objects]
\item [catalogues of radio sources]
\item [catalogues of X-ray sources]
\item [IRAS catalogue]
\item [AAT plate log]
\item [UK Schmidt plate log]
\item [solar system ephemerides]
\item [standard star spectra]
\item [the solar spectrum]
\item [atomic data---wavelengths, collision strengths, f-values,
photoionisation cross-sections]
\end{description}
\end{description}
Formal database management techniques would be required to handle the larger
items:
\begin{description}
\begin{description}
\item [AAT data archives]
\item [data archives from space experiments---IUE, ST, etc.]
\item [literature abstracts]
\end{description}
\end{description}
The list could easily be extended and this area could soak up as much effort as
was made available.
We think, however, that Starlink cannot give the problem of database very high
priority until its primary function of data reduction and analysis is well under
control.
\section {MANAGEMENT \& DISTRIBUTION OF APPLICATIONS PROGRAMMERS}
Many application programs will be written by astronomers and research
assistants at levels ranging from provision of an algorithm (expressed, for
convenience and precision, in a programming language) to implementation of a
fully working and documented package.
To coordinate this varied and extensive effort over the country there is a
Head of Applications who will lead a group of three of four programmers working
full time under contract to Starlink.
The applications programmers, as well as developing Starlink software
themselves, will also adapt and refine programs and documentation produced by
Starlink users, and will be responsible for formal support of all Starlink
applications software.
The head is based at RAL and the group members will, in the first instance, be
located at university nodes.
The Starlink Project Scientist monitors and advises on astronomical
requirements.
Additional Starlink effort will be available from site managers.
\subsection {Starlink-employed programmers}
It is anticipated that RAL will place extramural development contracts with the
appropriate university to employ the programmer.
This means that whilst his work is for Starlink, the programmer has a strong
duty to the university department, reporting to the department head in the same
way that a Starlink site manager does.
This work should therefore be appropriate to the needs of astronomers using the
node if this is possible.
At the same time he will be working within an overall programme laid down by the
head of applications who will be attempting a nationwide coordination.
Note that it is hoped to allocate work in such a way that the applications
required by Starlink's plans are also those required by the astronomers close to
the programmer.
This will bring about the close and continual programmer-astronomer interaction
that is essential for producing both relevant and high quality software and
which might be lacking in a centralised software effort.
However, local requirements must not be allowed to draw the programmer too far
away from the needs of the astronomical community at large.
For example, it is often desirable for a programmer to share in a data reduction
effort in order more acutely to appreciate the astronomer's needs, but his role
must not become that of a research assistant.
\subsection {Site managers}
In many ways the site manager's role as an applications programmer will be
similar to the above, except that his primary responsibility lies with the
management of the node.
It is, however, hoped that he will be available for some applications
programming.
\subsection {Astronomers}
Most astronomers who write programs need answers to specific problems and would
normally get on and construct a program to do just what they need and no more.
In the context of reduction and processing of data from instruments widely used
by the community this approach is grossly inefficient, giving rise to many
programs doing essentially similar operations, scattered around the country and
often unusable by other than the originator.
Part of Starlink's objective is to avoid this waste of effort by coordinating
much of the software common to many requirements into a few universal and
centrally supported packages.
To this end, astronomers are encouraged to write their new programs in, or
convert existing ones to, the Starlink environment.
This process must be sufficiently straightforward to encourage the astronomer
to take this, rather than the traditional direct approach.
The latter may in any case give inferior results and involve more effort.
We are therefore anxious to give every help to the astronomer at whatever level
he needs it, from implementing an algorithm he suggests to helping him, where
necessary, to install a complete package within Starlink.

Astronomers should not hesitate to approach Starlink staff when considering
writing (or having their own programming assistants write) software for which a
general demand can clearly be demonstrated.
Obviously, the level of assistance an astronomer can expect to receive from
Starlink is a function of his willingness to share his work with other users of
the network; those who wish to restrict use of their programs are perfectly
entitled to do so but must expect only limited assistance.
All astronomers who contribute software to Starlink will be encouraged to
discuss fully their plans before beginning work, to incorporate generality
where possible, to work within the Starlink environment, and to conform to
high standards of coding and documentation (see section 5.2).
Some will produce work that will be accepted for support by Starlink as it
stands; they will receive full credit for this (see section 5.1).
Many will produce work that needs a little tidying up by Starlink applications
programmers; this will be acceptable although they may receive less credit.
Really scrappy work will be rewritten by Starlink staff and the astronomer will
be credited only with the algorithm if it is sufficiently original and
ingenious.
\subsection {The role of the special interest groups}
After the implementation of a kernel of generally required software, the
priorities for further applications development must be set by the astronomical
community.
How is the head of applications, who is responsible for detailed direction of
the software effort, to be kept aware of current developments and trends?
It is the role of the project scientist to retain an overview of the needs of
the community and to identify the currently interesting sources of data.
It is clear, however, that detailed proposals for new software must come from
small groups of individuals who are active in fields of special interest.
Following the experience of the SERC Interactive Computing Facility, a number of
special interest groups are being set up; their brief is to propose to the
head of Starlink plans for the development of software within their fields.
The Starlink management can then coordinate these proposals and, in particular,
identify the regions of overlap to reduce the overall effort.
Assessment of the relative priorities of the proposals from the various
SIGs will be made by Starlink after full consultations with the SIGs concerned.
Formal ratification of major priority or resource allocation decisions will be
the responsibility of the SAG---the Scientific Advisory Group (since replaced
by SUC---the Starlink User Committee).
The current membership of the SIGs is stored in file ADMINDIR:SIG.LIS.
\section {DIFFICULTIES}
During the development of Starlink applications software a wide range of
difficulties will have to be faced.
In this section we comment on those which were discussed at the workshop and
which have not already been touched on.
\subsection {Incentives}
What incentives can be given to astronomer/programmers to contribute software to
Starlink?
The answer to this question will depend very much on the individual concerned;
some will be satisfied with the knowledge that they have provided the community
with reliable, widely used and readily supported software, whereas others will
demand full acknowledgment and publicity for their efforts.

Obviously, original software involved in specific projects will receive
acknowledgment in published work resulting from its use, the originator most
likely being a co-author.
It is also feasible to publish software within Starlink for widespread
distribution, where it is of general interest.
It should be noted that the Working Group on Coordination of Astronomical
Software (WGCAS) has written to major astronomical journals with a strong
recommendation that papers describing new software of general interest be
treated alike with astronomical papers.

It is hoped that those astronomers who provide consistently good software and
who take the trouble to document their work conscientiously will become well
known and respected for this.
In addition, such individuals will be exceptionally well placed to take
advantage of any funding Starlink may have at its disposal for Software
development projects.
\subsection {Support}
There are likely to be several levels of software support, depending on the
source and quality of the software.

At the lowest level, Starlink will advertise and distribute a program and offer
general guidance to potential users.
To qualify for even this minimal level of support, a program will have to be
(supposedly) bug-free, of general use, reasonably well coded (including adequate
comments), and be provided with user documentation.
Also, it must not interfere with the normal running of a Starlink node, avoiding
extravagant or eccentric use of disc space, memory, cpu or peripherals, and must
be runnable, without special privileges, on standard Starlink equipment.
The author will be expected to give detailed advice to users, to fix bugs, and
to maintain compatibility between his programs and the changing environment
(operating system updates, new peripherals, etc.).

The highest level of support will free the author completely from support
commitments, all user education, bug fixing and program enhancements being
handled by Starlink staff.
This level will require coding of a high standard, full conformity with Starlink
software conventions, and excellent program documentation as well as complete
and well written user documentation.

At the lower support levels the author will ultimately be responsible for the
correctness of his programs.
At the higher levels Starlink assume this responsibility and this may
necessitate a commissioning or refereeing process in cases where widespread and
frequent use will not itself constitute an adequate test.
\subsection {Direction}
Who provides direction and how close should it be?
The SIGs provide input to Starlink, enabling priorities to be set, which then
become the responsibility of the head of applications to implement.
(Overall direction must be provided centrally to avoid duplication and to
ensure uniformity of style.)
He will then select the best programmer for the job and allocate the work.
Progress is then monitored periodically at a level adequate for maintaining
standards and the overall schedule.
The programmer will be expected to seek advice from the astronomers at his node.
Frequent reviews of progress will be made within the applications group in
consultation with users.
\subsection {Standards}
Very few astronomers (and not all programmers) have had formal education in
programming techniques, even fewer in documentation techniques and virtually
none so far in the Starlink environment.
In attempting to provide standard software, Starlink has a unique opportunity to
develop a high standard in its programs and documentation.
It is therefore hoped to arrange various seminars aimed at providing recommended
standards to follow and including examples of the use of the Starlink
environment.

How are the standards of coding and documentation defined and imposed?
These are exceptionally difficult problems; it will be hard enough to reach
agreement on standards among, and impose those standards on, Starlink-employed
programmers, let alone astronomer/programmers not under the direct control of
Starlink.

The software community has become increasingly aware over the last two decades
of the crucial importance of high standards of coding and documentation.
The maintenance phase of a program's lifetime is now known to involve several
times the labour that was expended on the original implementation.
Time and time again poorly structured and inadequately documented programs have
to be rewritten from scratch; a costly exercise and likely to produce a new
crop of bugs.
Fortunately, there is a remarkable degree of consensus in the computing
community on what constitute adequate standards (see for example reference 3).
Starlink's own standards, which will be set down in a document in due course
(SGP/16), will be based partly on these commonly accepted standards and partly
on the recommendations of its own applications programmers who will after all
be responsible for support.
Software support is a difficult and tedious job and the astronomical community
must be willing to make an effort to ease the burden.
Starlink intends to provide a limited programme of education for
astronomer/programmers and may implement software aids---questionnaire
programs to produce pro forma documentation for example.

User documentation is of paramount importance; the authors of user manuals must
be prepared to accommodate all standards of user, not just the initiated few.
Written documentation must be consistent with online prompts etc., and must
conform to some uniform house style.
Every effort must be made to ensure that new users can begin to operate an
application suite profitably within a couple of hours and without skilled aid.

Some internal refereeing system will be tried in order to maintain uniform and
high standards.
One particular experiment in user documentation that has proved very successful
elsewhere is strongly recommended; it is to have an astronomer NOT closely
acquainted with programming in charge of user documentation at each site.
If there arise several such `tame' astronomers distributed through the network
they can pool their ideas and produce one common manual of uniform style, to be
used by everyone (even though some nodes will have different specialities).
\subsection {Obsolescence}
Almost all Starlink software will change with time and it will frequently be
necessary to alter the appearance, to the user, of a facility.
This will be a constant source of friction between the user community and the
Starlink management, especially on those rare occasions when an old established
facility has to be dropped in favour of a new one.
The magnitude and difficulty of the software support problem are, however, so
great that it is not practicable to continue supporting superseded programs;
Starlink will have to take a hard line, simplifying and streamlining the system
as a whole.
Naturally, major decisions in this area will only be made after adequate
discussions between Starlink and the astronomical community (largely through
the SIGs) have taken place.
\section {REFERENCES}
\begin{enumerate}
\item SGP/7. `Specification of Starlink Primitive Subroutine Interfaces'.
\item Wells, D.C., \& Greisen, E.W.  `FITS --- A Flexible Image Transport
System'.
International workshop on image processing in astronomy, Trieste, Italy, 1979.
\item Kernighan, B.W., \& Plauger, P.J.  `The Elements of Programming Style'.
(2nd ed.)  1978, McGraw-Hill.
\end{enumerate}
\appendix
\section {Manuals describing other systems}
\begin{tabbing}
Users Guide888\=8\kill
IHAP\>European Southern Observatory\\
TVS\>Cerro Tololo / University of Vienna\\
RICHFLD\>Kitt Peak National Observatory\\
GIPSY\>Kapteyn Laboratory, Groningen\\
SDRSYS\>Anglo-Australian Observatory\\
RGODR\>Royal Greenwich Observatory / University College London\\
SPICA\>University College London\\
DUSTERS\>University of Durham\\
IDL\>Research Systems Inc., Denver\\
SOL\>Laboratory for Atmospheric \& Space Physics, University of Colorado\\
CDCA\>Observatoire de Nice\\
Users Guide\>Mullard Radio Astronomy\\
ESP\>Royal Observatory, Edinburgh\\
Users Guide\>NRAO Very Large Array
\end{tabbing}
\end{document}
