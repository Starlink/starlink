\documentstyle[11pt]{article}
\pagestyle{headings}

% -----------------------------------------------------------------------------
% ? Document identification
\newcommand{\stardoccategory}  {Starlink General Paper}
\newcommand{\stardocinitials}  {SGP}
\newcommand{\stardocsource}    {sgp48.1}
\newcommand{\stardocnumber}    {48.1}
\newcommand{\stardocauthors}   {R.F. Warren-Smith}
\newcommand{\stardocdate}      {4th April 1996}
\newcommand{\stardoctitle}     {Starlink Software Priorities for 1996}
% ? End of document identification
% -----------------------------------------------------------------------------

\newcommand{\stardocname}{\stardocinitials /\stardocnumber}
\markright{\stardocname}
\setlength{\textwidth}{160mm}
\setlength{\textheight}{230mm}
\setlength{\topmargin}{-2mm}
\setlength{\oddsidemargin}{0mm}
\setlength{\evensidemargin}{0mm}
\setlength{\parindent}{0mm}
\setlength{\parskip}{\medskipamount}
\setlength{\unitlength}{1mm}

% -----------------------------------------------------------------------------
%  Hypertext definitions.
%  ======================
%  These are used by the LaTeX2HTML translator in conjunction with star2html.

%  Comment.sty: version 2.0, 19 June 1992
%  Selectively in/exclude pieces of text.
%
%  Author
%    Victor Eijkhout                                      <eijkhout@cs.utk.edu>
%    Department of Computer Science
%    University Tennessee at Knoxville
%    104 Ayres Hall
%    Knoxville, TN 37996
%    USA

%  Do not remove the %\begin{rawtex} and %\end{rawtex} lines (used by
%  star2html to signify raw TeX that latex2html cannot process).
%\begin{rawtex}
\makeatletter
\def\makeinnocent#1{\catcode`#1=12 }
\def\csarg#1#2{\expandafter#1\csname#2\endcsname}

\def\ThrowAwayComment#1{\begingroup
    \def\CurrentComment{#1}%
    \let\do\makeinnocent \dospecials
    \makeinnocent\^^L% and whatever other special cases
    \endlinechar`\^^M \catcode`\^^M=12 \xComment}
{\catcode`\^^M=12 \endlinechar=-1 %
 \gdef\xComment#1^^M{\def\test{#1}
      \csarg\ifx{PlainEnd\CurrentComment Test}\test
          \let\html@next\endgroup
      \else \csarg\ifx{LaLaEnd\CurrentComment Test}\test
            \edef\html@next{\endgroup\noexpand\end{\CurrentComment}}
      \else \let\html@next\xComment
      \fi \fi \html@next}
}
\makeatother

\def\includecomment
 #1{\expandafter\def\csname#1\endcsname{}%
    \expandafter\def\csname end#1\endcsname{}}
\def\excludecomment
 #1{\expandafter\def\csname#1\endcsname{\ThrowAwayComment{#1}}%
    {\escapechar=-1\relax
     \csarg\xdef{PlainEnd#1Test}{\string\\end#1}%
     \csarg\xdef{LaLaEnd#1Test}{\string\\end\string\{#1\string\}}%
    }}

%  Define environments that ignore their contents.
\excludecomment{comment}
\excludecomment{rawhtml}
\excludecomment{htmlonly}
%\end{rawtex}

%  Hypertext commands etc. This is a condensed version of the html.sty
%  file supplied with LaTeX2HTML by: Nikos Drakos <nikos@cbl.leeds.ac.uk> &
%  Jelle van Zeijl <jvzeijl@isou17.estec.esa.nl>. The LaTeX2HTML documentation
%  should be consulted about all commands (and the environments defined above)
%  except \xref and \xlabel which are Starlink specific.

\newcommand{\htmladdnormallinkfoot}[2]{#1\footnote{#2}}
\newcommand{\htmladdnormallink}[2]{#1}
\newcommand{\htmladdimg}[1]{}
\newenvironment{latexonly}{}{}
\newcommand{\hyperref}[4]{#2\ref{#4}#3}
\newcommand{\htmlref}[2]{#1}
\newcommand{\htmlimage}[1]{}
\newcommand{\htmladdtonavigation}[1]{}

%  Starlink cross-references and labels.
\newcommand{\xref}[3]{#1}
\newcommand{\xlabel}[1]{}

%  LaTeX2HTML symbol.
\newcommand{\latextohtml}{{\bf LaTeX}{2}{\tt{HTML}}}

%  Define command to re-centre underscore for Latex and leave as normal
%  for HTML (severe problems with \_ in tabbing environments and \_\_
%  generally otherwise).
\newcommand{\latex}[1]{#1}
\newcommand{\setunderscore}{\renewcommand{\_}{{\tt\symbol{95}}}}
\latex{\setunderscore}

%  Redefine the \tableofcontents command. This procrastination is necessary
%  to stop the automatic creation of a second table of contents page
%  by latex2html.
\newcommand{\latexonlytoc}[0]{\tableofcontents}

% -----------------------------------------------------------------------------
%  Debugging.
%  =========
%  Remove % on the following to debug links in the HTML version using Latex.

% \newcommand{\hotlink}[2]{\fbox{\begin{tabular}[t]{@{}c@{}}#1\\\hline{\footnotesize #2}\end{tabular}}}
% \renewcommand{\htmladdnormallinkfoot}[2]{\hotlink{#1}{#2}}
% \renewcommand{\htmladdnormallink}[2]{\hotlink{#1}{#2}}
% \renewcommand{\hyperref}[4]{\hotlink{#1}{\S\ref{#4}}}
% \renewcommand{\htmlref}[2]{\hotlink{#1}{\S\ref{#2}}}
% \renewcommand{\xref}[3]{\hotlink{#1}{#2 -- #3}}
% -----------------------------------------------------------------------------
% ? Document specific \newcommand or \newenvironment commands.

\newcommand{\proj}[4]{\subsubsection{\label{#2:#3}#1}#4}

% ? End of document specific commands
% -----------------------------------------------------------------------------
%  Title Page.
%  ===========
\renewcommand{\thepage}{\roman{page}}
\begin{document}
\thispagestyle{empty}

%  Latex document header.
%  ======================
\begin{latexonly}
   CCLRC / {\sc Rutherford Appleton Laboratory} \hfill {\bf \stardocname}\\
   {\large Particle Physics \& Astronomy Research Council}\\
   {\large Starlink Project\\}
   {\large \stardoccategory\ \stardocnumber}
   \begin{flushright}
   \stardocauthors\\
   \stardocdate
   \end{flushright}
   \vspace{-4mm}
   \rule{\textwidth}{0.5mm}
   \vspace{5mm}
   \begin{center}
   {\Large\bf \stardoctitle}
   \end{center}
   \vspace{5mm}

% ? Heading for abstract if used.
   \vspace{10mm}
   \begin{center}
      {\Large\bf Abstract}
   \end{center}
% ? End of heading for abstract.
\end{latexonly}

%  HTML documentation header.
%  ==========================
\begin{htmlonly}
   \xlabel{}
   \begin{rawhtml} <H1> \end{rawhtml}
      \stardoctitle
   \begin{rawhtml} </H1> \end{rawhtml}

% ? Add picture here if required.
% ? End of picture

   \begin{rawhtml} <P> <I> \end{rawhtml}
   \stardoccategory \stardocnumber \\
   \stardocauthors \\
   \stardocdate
   \begin{rawhtml} </I> </P> <H3> \end{rawhtml}
      \htmladdnormallink{CCLRC}{http://www.cclrc.ac.uk} /
      \htmladdnormallink{Rutherford Appleton Laboratory}
                        {http://www.cclrc.ac.uk/ral} \\
      \htmladdnormallink{Particle Physics \& Astronomy Research Council}
                        {http://www.pparc.ac.uk} \\
   \begin{rawhtml} </H3> <H2> \end{rawhtml}
      \htmladdnormallink{Starlink Project}{http://www.starlink.ac.uk/}
   \begin{rawhtml} </H2> \end{rawhtml}
   \htmladdnormallink{\htmladdimg{source.gif} Retrieve hardcopy}
      {http://www.starlink.ac.uk/cgi-bin/hcserver?\stardocsource}\\

%  HTML document table of contents.
%  ================================
%  Add table of contents header and a navigation button to return to this
%  point in the document (this should always go before the abstract \section).
  \label{stardoccontents}
  \begin{rawhtml}
    <HR>
    <H2>Contents</H2>
  \end{rawhtml}
  \renewcommand{\latexonlytoc}[0]{}
  \htmladdtonavigation{\htmlref{\htmladdimg{contents_motif.gif}}
        {stardoccontents}}

% ? New section for abstract if used.
  \section{\xlabel{abstract}Abstract}
% ? End of new section for abstract

\end{htmlonly}

% -----------------------------------------------------------------------------
% ? Document Abstract. (if used)
%  ==================
This document presents an overview of Starlink's software priorities
for 1996. As background, it also gives details of all the software
projects that were proposed and considered, including the written
proposals provided by the seven Starlink Software Strategy Groups.
% ? End of document abstract
% -----------------------------------------------------------------------------
% ? Latex document Table of Contents (if used).
%  ===========================================
 \newpage
 \begin{latexonly}
   \vspace{30mm}
   \setlength{\parskip}{0mm}
   \latexonlytoc
   \setlength{\parskip}{\medskipamount}
   \markright{\stardocname}
 \end{latexonly}
% ? End of Latex document table of contents
% -----------------------------------------------------------------------------
\newpage
\renewcommand{\thepage}{\arabic{page}}
\setcounter{page}{1}
\section{\xlabel{introduction}INTRODUCTION}

This document gives details of Starlink's software priorities for
1996, together with supporting background information. It is based
closely on a paper approved by the Starlink Panel at its meeting in
November 1995.

For an overview, readers should consult \S\ref{priorities} which gives
a prioritised list of the software projects that were considered for
attention during 1996. This list takes into account the expected rate
of progress during 1996 (based on current manpower estimates) and
splits the projects into those likely, and those not likely, to be
tackled during the year.

The priority list also makes reference to the project descriptions,
which appear in \S\ref{details}. These (generally one-paragraph)
descriptions are intended to outline the likely scope of each project.

As further background, the Appendices to this document contain copies
of the written input received from the seven Starlink Software
Strategy Groups (SSGs) who were invited, amongst others, to propose
projects for consideration. The SSGs also serve a more general role in
identifying a coherent longer-term software strategy within their
subject areas. The interested reader is referred to their input for
information about the trends and requirements that have affected the
detailed choice of software projects for 1996.

\section{\xlabel{priority_list}\label{priorities}OVERALL PRIORITY LIST FOR 1996}

The software projects which have been identified for possible
attention by Starlink during 1996 are listed below, where they are
shown in priority order within each of the broad subject areas covered
in the Appendices to this document.

\begin{itemize}
\item Those projects marhed ``yes'' are those which we would expect to
complete within the year.

\item Projects marked ``?'' are those whose completion (or even
commencement) is uncertain within the year. Such projects are likely
to yield results only if progress on earlier projects is more rapid
than anticipated.

\item Projects marked ``no'' are those which we expect not to be able to
carry out this year.
\end{itemize}

Note that these expectations are based on anticipated levels of
manpower as of November 1995. Should unexpectedly good progress be
made, we would continue to tackle new projects according to their
priority order within this list. Alternatively, should unexpected
delays occur or manpower shortages be experienced, the lower priority
projects would be the first to be omitted, although we would hope to
avoid abandoning any projects that had already been started.

\newcommand{\pry}[3]{{\bf #1 (\ref{#2:#3})} & {\bf yes} \\}
\newcommand{\prq}[3]{#1 (\ref{#2:#3}) & ? \\}
\newcommand{\prn}[3]{#1 (\ref{#2:#3}) & no \\}

\begin{htmlonly}
\renewcommand{\pry}[3]{\item (yes) #1 (see \ref{#2:#3})}
\renewcommand{\prq}[3]{\item (?) #1 (see \ref{#2:#3}) \\}
\renewcommand{\prn}[3]{\item (no) #1 (see \ref{#2:#3}) \\}
\begin{center}
\begin{description}
\item[Spectroscopy:]
\begin{itemize}
\pry{Transparent Foreign Data Access for Figaro}{spec}{figaro}
\pry{Echelle Data Reduction}{spec}{echelle}
\pry{IUE Final Archive Access}{spec}{IUE}
\prq{WWW Communication}{spec}{WWW}
\prq{Documentation Improvements}{spec}{docs}
\prn{Fibre Spectroscopy Data Reduction}{spec}{fibre}
\end{itemize}

\item[Image Processing:]
\begin{itemize}
\pry{ISO Data Access}{ip}{ISO}
\pry{Handling Large Datasets}{ip}{largedata}
\pry{Graphical Image Manipulation Tool}{ip}{GUI}
\prq{Imaging Polarimetry Data Reduction}{ip}{polar}
\prn{Advice on Format Conversion Implications}{ip}{convert}
\prn{Upgrades to PISA}{ip}{pisa}
\end{itemize}

\item[Theory \& Statistical Analysis:]
\begin{itemize}
\pry{Advertise Computer Algebra Software}{theory}{algebra}
\pry{Enhanced User Support For Visualisation Software}{theory}{visual}
\pry{Ease the Transition to Fortran~90}{theory}{F90}
\end{itemize}

\item[Information Services \& Databases:]
\begin{itemize}
\pry{A UK Astronomy ftp Archive}{info}{archive}
\pry{A Gripe Command}{info}{gripe}
\pry{Observing Preparation Software}{info}{prepare}
\prq{Makefiles}{info}{makefiles}
\prn{Automatic Recording of Software Usage}{info}{usage}
\prn{Enhancement of Catalogue Handling Applications}{info}{CURSA}
\prn{Conversion of VMS/SCAR Catalogues}{info}{catalogues}
\prn{Upgrades to HTML Facilities}{info}{HTML}
\prn{Update CHART Documentation}{info}{chart}
\end{itemize}

\item[Graphics \& Infrastructure:]
\begin{itemize}
\pry{Astrometric Coordinate Systems}{infra}{astrom}
\pry{Integration with IRAF}{ip}{IRAF}
\pry{Removal of NAG dependence}{infra}{NAG}
\pry{Investigate New Command Language Possibilities}{infra}{cl}
\pry{Simplified Interface to N-Dimensional Data}{infra}{IMG}
\pry{Port of Starlink Software to Linux}{infra}{PC}
\pry{Library Interfaces for the C Language}{infra}{C}
\prq{ADAM Message System Upgrades}{infra}{ADAM}
\prn{Graphics Evaluation}{infra}{graphics}
\prn{Assist with SCUBA Software}{infra}{SCUBA}
\prn{Facilitate Modification of Standard Applications}{infra}{modify}
\prn{Experimental Port to Non-SPARC Solaris}{infra}{solaris}
\end{itemize}

\item[Radio, Mm \& Sub-Mm Astronomy:]
\begin{itemize}
\pry{Fix Problems with SPECX}{radio}{SPECX}
\pry{Importing Datasets into IRAS90}{radio}{IRAS90}
\pry{2-D Gaussian Fitting Program}{radio}{gauss}
\pry{Export Copies of SPECX and JCMTDR}{radio}{export}
\pry{Distribution of the DIFMAP Package}{radio}{difmap}
\pry{UNIX Version of the FLUXES Program}{radio}{FLUXES}
\pry{AIPS Cookbook On-Line}{radio}{AIPScook}
\pry{Fix Figaro/KAPPA Parameter Problem}{radio}{param}
\prq{Export SPECX Data to Visualisation Software}{radio}{visual}
\prn{UNIX Port of IRAS MEM Software}{radio}{MEM}
\prn{RA, Dec Pointing Corrections in JCMTDR}{radio}{pointing}
\end{itemize}

\item[X-Ray Astronomy:]
\begin{itemize}
\pry{Maintaining FTOOLS/XSELECT Software}{xray}{FTOOLS}
\pry{Sky Map/Catalogue Plotting Software}{xray}{skymap}
\prq{Improvements to GUI Tools}{xray}{GUI}
\end{itemize}

\item[Documentation:]
\begin{itemize}
\pry{Cookbooks}{docs}{cookbooks}
\end{itemize}
\end{description}
\end{center}
\end{htmlonly}

Because of unavoidable delays in distributing items of software, the
results of some of these projects will inevitably not become available
before early 1997, although the software work itself is scheduled for
completion by mid-November 1996 and many items should be available
before then. Users with a need for a particular item are invited to
contact Starlink Software Support (ussc@star.rl.ac.uk) in the first
instance, as it is often possible to make ``pre-release'' versions of
software available to users who are willing to report back on their
experiences with it.

\begin{latexonly}
{\scriptsize
\begin{center}
\begin{tabular}{|l|c|}
\hline
{\normalsize \bf Project} & {\normalsize \bf This Year?} \\
\hline\hline
\pry{Transparent Foreign Data Access for Figaro}{spec}{figaro}
\pry{Echelle Data Reduction}{spec}{echelle}
\pry{IUE Final Archive Access}{spec}{IUE}
\prq{WWW Communication}{spec}{WWW}
\prq{Documentation Improvements}{spec}{docs}
\prn{Fibre Spectroscopy Data Reduction}{spec}{fibre}
\hline
\pry{ISO Data Access}{ip}{ISO}
\pry{Handling Large Datasets}{ip}{largedata}
\pry{Graphical Image Manipulation Tool}{ip}{GUI}
\prq{Imaging Polarimetry Data Reduction}{ip}{polar}
\prn{Advice on Format Conversion Implications}{ip}{convert}
\prn{Upgrades to PISA}{ip}{pisa}
\hline
\pry{Advertise Computer Algebra Software}{theory}{algebra}
\pry{Enhanced User Support For Visualisation Software}{theory}{visual}
\pry{Ease the Transition to Fortran~90}{theory}{F90}
\hline
\pry{A UK Astronomy ftp Archive}{info}{archive}
\pry{A Gripe Command}{info}{gripe}
\pry{Observing Preparation Software}{info}{prepare}
\prq{Makefiles}{info}{makefiles}
\prn{Automatic Recording of Software Usage}{info}{usage}
\prn{Enhancement of Catalogue Handling Applications}{info}{CURSA}
\prn{Conversion of VMS/SCAR Catalogues}{info}{catalogues}
\prn{Upgrades to HTML Facilities}{info}{HTML}
\prn{Update CHART Documentation}{info}{chart}
\hline
\pry{Astrometric Coordinate Systems}{infra}{astrom}
\pry{Integration with IRAF}{ip}{IRAF}
\pry{Removal of NAG dependence}{infra}{NAG}
\pry{Investigate New Command Language Possibilities}{infra}{cl}
\pry{Simplified Interface to N-Dimensional Data}{infra}{IMG}
\pry{Port of Starlink Software to Linux}{infra}{PC}
\pry{Library Interfaces for the C Language}{infra}{C}
\prq{ADAM Message System Upgrades}{infra}{ADAM}
\prn{Graphics Evaluation}{infra}{graphics}
\prn{Assist with SCUBA Software}{infra}{SCUBA}
\prn{Facilitate Modification of Standard Applications}{infra}{modify}
\prn{Experimental Port to Non-SPARC Solaris}{infra}{solaris}
\hline
\pry{Fix Problems with SPECX}{radio}{SPECX}
\pry{Importing Datasets into IRAS90}{radio}{IRAS90}
\pry{2-D Gaussian Fitting Program}{radio}{gauss}
\pry{Export Copies of SPECX and JCMTDR}{radio}{export}
\pry{Distribution of the DIFMAP Package}{radio}{difmap}
\pry{UNIX Version of the FLUXES Program}{radio}{FLUXES}
\pry{AIPS Cookbook On-Line}{radio}{AIPScook}
\pry{Fix Figaro/KAPPA Parameter Problem}{radio}{param}
\prq{Export SPECX Data to Visualisation Software}{radio}{visual}
\prn{UNIX Port of IRAS MEM Software}{radio}{MEM}
\prn{RA, Dec Pointing Corrections in JCMTDR}{radio}{pointing}
\hline
\pry{Maintaining FTOOLS/XSELECT Software}{xray}{FTOOLS}
\pry{Sky Map/Catalogue Plotting Software}{xray}{skymap}
\prq{Improvements to GUI Tools}{xray}{GUI}
\hline
\pry{Cookbooks}{docs}{cookbooks}
\hline
\end{tabular}
\end{center}}
\end{latexonly}

\section{\xlabel{brief_details_of_projects}\label{details}BRIEF DETAILS OF PROJECTS PROPOSED}

This section give details of the software projects which were
identified and considered as possible candidates for attention by
Starlink's software staff during 1996. The information on which these
proposals are based draws on discussions with users (and points raised
at site visits, {\em etc.}), the Software Strategy Group proposals (to
be found in the Appendices to this document), requests received from
users and active developers, recommendations from the ADAM Review
(\xref{SGP/45}{sgp45}{}), the Software Survey Results and the Starlink
Project's own knowledge and expertise.

For each project, a short description is given of the extent of the
work that would be carried out, together with a brief rationale where
necessary. Note that inclusion of a project in this section does not
imply that it will receive attention, but merely that it has been
considered. The list of projects prioritised for completion during
1996 is given in \S\ref{priorities}.

Projects are described here in no particular order. In many cases,
more complete descriptions can be found in the accompanying Software
Strategy Group input (see the Appendices).

\subsection{SPECTROSCOPY}

\proj{Transparent Foreign Data Access for Figaro}{spec}{figaro}
{We would adapt the Figaro package to allow it to use the facilities
recently introduced as part of the NDF library for accessing a range
of foreign format data files ``on-the-fly''. This change is also a
prerequisite if Figaro users are to benefit from improved
interoperability with other software systems such as IRAF
(\S\ref{infra:IRAF}).

The work would involve adding an alternative data access layer within
the software and modifying some applications. As a side effect this
would also improve data format compatibility with other Starlink
software. An outline plan has been agreed with the package's author
(Keith Shortridge, AAO).}

\proj{Echelle Data Reduction}{spec}{echelle}
{We would conduct a ``bug blitz'' on the Echomop Echelle reduction
package (initiated by a mail-shot to users) and place the package on
an elevated level of maintenance for one year to eliminate outstanding
problems and improve its ease of use. This work would complement
improvements to documentation and efficiency that have already been
implemented.}

\proj{IUE Final Archive Access}{spec}{IUE}
{We would investigate the problems associated with making IUE final
archive data accessible to Starlink users. This would probably involve
at least the provision of a utility for reading FITS binary tables
holding Final Archive data. If it turns out that further reduction of
the data is also required, then we would carry out a study to identify
suitable data reduction paths and publicise the results (we would not
expect to develop further software as part of this project but may
identify new work for discussion).}

\proj{Spectropolarimetry Data Reduction}{spec}{polar}
{We would release the POLMAP data reduction package. Further
development of spectropolarimetry data reduction might then be best
incorporated into the similar proposal for imaging work
(\S\ref{ip:polar}). We would therefore address these as a single
project.}

\proj{ISO Data Access}{spec}{ISO}
{We would complete work on providing software for accessing
spectroscopic data from ISO as soon as the required data format
specification becomes available.  This work would be addressed as part
of the similar project for handling ISO imaging data
(\S\ref{ip:ISO}).}

\proj{Release of Portable CGS4DR}{spec}{CGS4DR}
{The release of portable CGS4DR with a graphical user interface has
been completed as part of the 1995 programme of work. We believe that
the problems with the interim (command line) version
\htmlref{identified by the Spectroscopy SSG}{specSSG:CGS4DR} have also
been fixed.}

\proj{Fibre Spectroscopy Data Reduction}{spec}{fibre}
{We would carry out a survey to identify requirements, preferences and
available software for reducing fibre spectroscopy data and make
suitable recommendations to guide users -- perhaps in the form of a
cookbook if that seems appropriate.}

\proj{Interoperability with Other Software}{spec}{IRAF}
{This is addressed by projects listed under other headings (see
\S\S\ref{spec:figaro}, \ref{ip:IDL}, \ref{infra:IRAF}, and
\ref{infra:cl}), for which we note the \htmlref{Spectroscopy SSG's
support}{specSSG:IRAF}.}

\proj{Documentation Improvements}{spec}{docs}
{We would investigate the feasibility of including trial data and
demonstration software into documentation by means of a suitable pilot
project. We note the suggestion that the Echomop package might provide
a suitable example.}

\proj{WWW Communication}{spec}{WWW}
{We would set up appropriate links from the Starlink home page to
encourage communication with the Spectroscopy SSG pages. This would
probably be done as part of a wider development of WWW information
related to software.}

\proj{Port of Starlink Software to Linux}{spec}{PC}
{We note the \htmlref{spectroscopy SSG's strong
encouragement}{specSSG:PC} for the Starlink Software Collection to be
made available on PCs running the Linux operating system. This project
is described elsewhere (see \S\ref{infra:PC}).}


\subsection{IMAGE PROCESSING}

\proj{Astrometric Support for Applications}{ip}{astrom}
{We note the \htmlref{Image Processing SSG's high priority
request}{ipSSG:astrom} for facilities that allow applications software
to handle accurate astrometric information associated with images, and
their proposal that a suitable library be developed to support this. A
project to address this is described elsewhere (see
\S\ref{infra:astrom}).}

\proj{Handling Large Datasets}{ip}{largedata}
{We would conduct a study into the problems of handling image datasets
which are substantially larger than currently processed. We would
publicise recommendations to users on choice of software for this
purpose and to programmers on techniques for handling such data. We
would also identify any further software development work that may be
required (but would not expect to implement it as part of this
project).}

\proj{Graphical Image Manipulation Tool}{ip}{GUI}
{We would implement a graphical tool for performing common interactive
astronomical tasks on images in a readily-accessible and efficient
way. Although SAOIMAGE is perhaps the closest existing facility for
comparison purposes, we
would not expect to overlap much with what it provides, but to
concentrate instead on accessing new astronomical functionality in a
similar way. Such a tool might best be developed by extending image
{\em widgets} that have already been developed elsewhere in a
compatible manner to our own (Tcl/Tk based) approach, and we would
explore the possibility for international collaboration.

As part of the user survey we would need to conduct in order to design
such a tool, we would also survey the need for similar graphical
interfaces for other purposes.}

\proj{Integration with IRAF}{ip}{IRAF}
{We note the \htmlref{Image Processing SSG's support}{ipSSG:IRAF} for
a system that allows Starlink software to be run from the IRAF command
langage. A project to address this is described elsewhere
(\S\ref{infra:IRAF}).}

\proj{Imaging Polarimetry Data Reduction}{ip}{polar}
{We would investigate the requirements for reducing two-channel
imaging polarimetry data from the instruments available to UK
astronomers (incorporating the similar requirements of
spectropolarimetry work as far as possible -- \S\ref{spec:polar}) and
provide suitable data reduction software. According to current
information, disparate software initiatives associated with several
different instruments may already be underway and we would investigate
the scope for collaboration and coordination of this work.}

\proj{Transparent Foreign Data Access for Figaro}{ip}{figaro}
{We note the \htmlref{Image Processing SSG's support}{ipSSG:figaro}
for Figaro to be able to access foreign format data files. A project
to address this is described elsewhere (\S\ref{spec:figaro}).}

\proj{Interoperability with IDL}{ip}{IDL}
{The possibility of providing an interface between existing Starlink
applications and IDL as part of the overall interoperability programme
would probably best be examined as part of investigations into new
command languages described elsewhere (\S\ref{infra:cl}).

Although IDL is a commercial system, it has many attractions as a
command language with the additional advantage of already being in
wide use. At present, this would appear to make it the most promising
system for prototyping a new command language for Starlink wide use.}

\proj{ISO Data Access}{ip}{ISO}
{We would complete work on providing software for accessing imaging
data from ISO (along with spectroscopic data -- \S\ref{spec:ISO}) as
soon as the required data format specification becomes available.}

\proj{Advice on Format Conversion Implications}{ip}{convert}
{We would provide documentation on the limitations of the most
commonly-encountered astronomical data formats and the implications
for converting data between those formats. Advice on how to avoid
unnecessary data loss would be included. We would also investigate
ways of warning users when data loss may be occurring.}

\proj{CCD Reduction Cookbook}{ip}{CCDcook}
{We would develop a cookbook describing in easy-to-follow terms
precisely how to reduce CCD data. We would propose to address this
project as part of a more general survey of cookbook requirements
(\S\ref{docs:cookbooks}).}

\proj{Photometry Cookbook}{ip}{photcook}
{We would develop a cookbook describing precisely how to reduce
imaging data to obtain photometric results. We would propose to
address this project as part of a more general survey of cookbook
requirements (\S\ref{docs:cookbooks}).}

\proj{Starlink Software on PCs}{ip}{PC}
{We note the \htmlref{low priority given by the Image Processing
SSG}{ipSSG:PC} to the project to make the Starlink Software Collection
available on PCs (see \S\ref{infra:PC}).}

\proj{Upgrades to PISA}{ip}{pisa}
{We receive frequent requests to upgrade the PISA object analysis
package to use floating point data instead of the INTEGER$*$2 format
it currently uses. This is a difficult project what would take a
disproportionate amount of work, but it may be possible to convert to
INTEGER$*$4 format more easily (a trial has already been done) and
achieve most of the benefits.

We would investigate the usefulness of this approach further and
produce a new release if it seems workable.}

\subsection{THEORY AND STATISTICAL ANALYSIS}

\proj{Advertise Computer Algebra Software}{theory}{algebra}
{We would convert the results of last year's survey of computer
algebra software into a brief guide to this field to help new
prospective users. As part of this, we would ensure that the existing
central MAPLE service is well-publicised and monitor its use with a
view to deciding whether to continue this service.}

\proj{Enhanced User Support For Visualisation Software}{theory}{visual}
{Following the distribution of products produced last year to support
visualisation of theory datasets, we would provide an enhanced level
of user support for one year. This would initially consist of helping
prospective users to assess the {\em IBM Data Explorer} software, and
would then go on to include extending the facilities already provided
in the light of experience with using this new system. The intention
would be to ease the path of new users into the visualisation field,
and might consist largely of supplying expertise.}

\proj{UK Archive of Theory Software}{theory}{mirror}
{We would include the requirement to provide a UK ftp archive of
theory software in the more general project described elsewhere (see
\S\ref{info:archive}) to find a solution to the problem of slow network access
to international astronomy software, documentation and data.}

\proj{Ease the Transition to Fortran~90}{theory}{F90}
{With the planned distribution of Fortran~90 compilers to Starlink
sites in response to increasing user interest, some compatibility
problems might be anticipated due to the simultaneous use of both
Fortran~77 and Fortran~90.  At present, we know of only one definite
(but fairly minor) difficulty, but others may arise, particularly when
linking programs that use code compiled with both compilers. If it
becomes necessary to re-build subroutine libraries with Fortran~90
compilers, considerable work could potentially be involved.

The purpose of this project would be to perform the minimum necessary
to ease the transition. The amount of work involved is clearly very
uncertain, but it would not at this stage involve re-building existing
software with new compilers unless this proved unavoidable, nor would
it involve re-design of software ({\em e.g.}\ library interfaces) for
more convenient Fortan~90 usage (although these remain possibilities
for future work).}

\subsection{\label{info}INFORMATION SERVICES AND DATABASES}

\proj{A UK Astronomy ftp Archive}{info}{archive}
{There have been requests from many areas of UK astronomy for a
facility whereby a local (UK) copy of important astronomical software,
documentation and data could be kept to overcome the problems of slow
network access to international ftp sites.

We would investigate the provision of such a facility, initially
aiming to ``mirror'' the IRAF and AIPS packages within the UK, but
also to provide a service for other less well-centralised information
sources that are important to UK astronomy. (Note that this would
inevitably also depend on the availability of hardware, especially
disk space.)}

\proj{Makefiles}{info}{makefiles}
{We would document the {\em makefile} system used for distributing
Starlink software and investigate ways of automating the generation of
makefiles. This could probably be done relatively easily using
standard UNIX tools (but note that there is actually no need for
submitted software to come with makefiles, as these can always be
written by Starlink staff if necessary).}

\proj{Removal of Calls to the NAG Library}{info}{NAG}
{We note the \htmlref{Information Services and Databases SSG's
support}{infoSSG:NAG} for the project to remove calls to the NAG
library from applications to facilitate their international
distribution. This is described elsewhere (\S\ref{infra:NAG}).}

\proj{Automatic Recording of Software Usage}{info}{usage}
{We would develop a system for automatically recording usage of (any)
software at Starlink sites and for producing statistics suitable for
guiding policy decisions and assisting with software support
work. After investigating possible mechanisms, part of this work would
involve preparing guidelines for the operation of such a system (for
approval by the Starlink Panel) to protect the privacy of individuals
and to address any other concerns raised.}

\proj{Cookbook/Tutorial Documentation Survey}{info}{cookbook}
{We note \htmlref{the SSG's advocacy}{infoSSG:cookbook} of this form
of documentation, but note that the main problem at present lies in
identifying which subjects would best be served by cookbooks. To
address this problem, we propose to conduct a small user survey, as
described elsewhere (\S\ref{docs:cookbooks}).}

\proj{Port of Starlink Software to Linux}{info}{PC}
{We note the \htmlref{SSG's encouragement}{infoSSG:PC} for the
Starlink Software Collection to be made available on PCs running the
Linux operating system (see \S\ref{infra:PC}).}

\proj{Enhancement of Catalogue Handling Applications}{info}{CURSA}
{We would allocate a fixed amount of time to be spent on enhancements
to the CURSA package of catalogue handling applications and the CAT
library that supports it, and to promoting this software amongst a
wider audience. These enhancements would be driven by user requests,
of which we already have rather more than can reasonably be
accommodated. Nevertheless, a useful degree of enhancement to this
package should be possible this year.}

\proj{Conversion of VMS/SCAR Catalogues}{info}{catalogues}
{We would investigate usage of the astronomical catalogues available
on the old VAX/VMS STADAT machine at RAL and identify which (if any)
are worth making available in a more accessible form. For those
identified, we would convert to FITS table format for compatibulity
with the CURSA software (above). We would also implememt a new way of
making these catalogues available to users, a WWW interface being a
likely possibility.}

\proj{Upgrades to HTML Facilities}{info}{HTML}
{It is possible that during the next year significant improvements to
the HTML language may become available to allow the use of ({\em
e.g.}) tables in hypertext documents. If this occurs, we would provide
suitable upgrades to our facilities for hypertext documentation.}

\proj{Update CHART Documentation}{info}{chart}
{We would update the documentation for the CHART package, as it is not
currently consistent with the UNIX version of the software.}

\proj{A Gripe Command}{info}{gripe}
{We would implement a ``gripe'' command ({\em e.g.}\ as in AIPS) to
make it simple for users to feed back complaints about software to its
author (and Starlink). The ability to run this from within any
existing package may be required.}

\proj{Observing Preparation Software}{info}{prepare}
{We would investigate the software available for preparing to make
observations ({\em e.g.}\ rising and setting times, phase of moon,
finding charts, standard stars, {\em etc.}) and provide a collection
of suitable applications with documentation on their use (this would
probably include the OBSERVE program already distributed by Starlink). We would
not expect to write much, if any, of this software ourselves.}

\subsection{\label{infra}GRAPHICS AND INFRASTRUCTURE}

\proj{Port of Starlink Software to Linux}{infra}{PC}
{We would continue our port of the Starlink Software Collection to
Linux by moving on to port applications packages. Given the evidence
from SSGs and discussions with users, it appears that this is seen as a
high priority mainly in areas with relatively light machine
resource requirements, and as less important in areas such as image
processing, where demands are typically heavier. We would therefore
concentrate on spectroscopy applications as the first priority, but
would nevertheless expect to have most applications running on Linux
by the end of the year.

The infrastructure part of this port is already essentially complete,
although we might have to return to parts of it if new problems are
revealed when applications start to use it.}

\proj{Removal of NAG dependence}{infra}{NAG}
{We would continue the project, which is already well advanced, to
remove the dependence of Starlink applications on the NAG
library. This would then allow free distribution via ftp and the WWW.}

\proj{Integration with IRAF}{infra}{IRAF}
{We would develop our current prototype system for running Starlink
applications from the IRAF command language to the point of releasing
a production system for general use. This would include at least one
Starlink applications package accessible as an IRAF package. We would
make Figaro the highest priority package for inclusion, but since this
also depends on work to allow Figaro to access IRAF datasets (see
\S\ref{spec:figaro}) it might not be realistic to complete all of this in one
year. In that event, we might need to substitute an alternative
package in the first instance.}

\proj{Training Courses}{infra}{training}
{We note that the \htmlref{Graphics and Infrastructure SSG does not
favour}{infraSSG:training} Starlink running training courses on how to
use software.}

\proj{Library Interfaces for the C Language}{infra}{C}
{We would implement C~language interfaces to Starlink subroutine
libraries, to provide an alternative calling interface to Fortran~77,
and would update the documentation to describe them. This would open
the way to developing applications in C by the end of the year.}

\proj{Astrometric Coordinate Systems}{infra}{astrom}
{We would develop a subroutine library, based on the existing SLALIB
library, that provides the functions required to allow applications
properly to handle accurate astrometric information associated with
astronomical datasets. Facilities for plotting astrometric grids would
be included. The library would be designed for use independently of
other Starlink software as far as possible and accuracy would be
substantially superior to existing systems. Access to specific data
formats would be performed separately from the main library interface
to facilitate its re-use in other environments.}

\proj{Graphics Evaluation}{infra}{graphics}
{We would conduct an investigation into modern 3-D graphics libraries
and related visualisation products with a view to identifying a
successor to GKS. If a suitable product can be identified, we would
provide a trial release for evaluation by potential users before
considering what further integration (if any) is needed with other
astronomical software.}

\proj{Experimental Port to Non-SPARC Solaris}{infra}{solaris}
{We would conduct a trial port of key items of Starlink software to
run on a version of the Solaris operating system on non-SPARC
hardware. This would be an experiment intended to identify any
problems that might arise, and would not necessarily result in a
usable product at this stage.}

\proj{Investigate New Command Language Possibilities}{infra}{cl}
{We would investigate ways of integrating state-of-the art command
language facilities with existing applications software as a move
towards greater interoperability between the software systems in use
in the UK. This would involve an evaluation of the options available
(Tcl/Tk, IDL, Khoros, AVS, Java, perl {\em etc.}) and the development
of a prototype system, comparable to the work already done on IRAF, as
a test of its feasibility.}

\proj{Facilitate Modification of Standard Applications}{infra}{modify}
{We would investigate the provision of facilities to make it far
easier to locate copies of the source code of Starlink applications,
to modify those copies for personal use, and to re-build a modified
copy of the application.}

\proj{Simplified Interface to N-Dimensional Data}{infra}{IMG}
{We would complete the project, which is already well-advanced, to
provide a much simplified programming interface to astronomical
datasets with between 1 and 3 dimensions. Simple access to header
information would also be included.

The value of this development has recently been enhanced by the
addition of ``on-the-fly'' format conversion to the NDF library, which
would give such an interface simultaneous access a range of common
astronomical data formats with no additional effort on the user's
part.}

\proj{ADAM Message System Upgrades}{infra}{ADAM}
{We would implement changes to the ADAM message system to allow (old)
VMS based ADAM systems to communicate with newer UNIX systems. This
facility is required at UKIRT to allow data reduction facilities to be
migrated from VMS to UNIX platforms.}

\proj{Assist with SCUBA Software}{infra}{SCUBA}
{We would provide assistance to resolve any problems experienced by
the SCUBA project once its real-time software goes into service. (This
will be the first demanding use of a real-time UNIX ADAM system.)}

\proj{C~shell Cookbook}{infra}{cshell}
{We would provide a book of common C~shell techniques, with examples
from astronomy, to help users write their own UNIX shell scripts for
data processing. We would propose to address this project as part of a
more general survey of cookbook requirements
(\S\ref{docs:cookbooks}).}

\subsection{RADIO, MM \& SUB-MM ASTRONOMY}

\proj{Fix Problems with SPECX}{radio}{SPECX}
{We are already collaborating with JACH staff to solve the problems in
SPECX that have been reported and which appear to arise from
inappropriate use of the HDS library. Under new support arrangements
which have been agreed with all the parties concerned, the necessary
changes will be carried out by JACH staff, with Starlink providing any
necessary programming advice and implementing the re-release to
Starlink sites.}

\proj{Importing Datasets into IRAS90}{radio}{IRAS90}
{A new application in KAPPA will shortly be released which we believe
will address this problem (which is primarily one of handling
astrometric information correctly). We would monitor the situation and
make further improvements if necessary.}

\proj{2-D Gaussian Fitting Program}{radio}{gauss}
{We would provide a facility for performing the required 2-D
multiple-component Gaussian fits to JCMT data. This would very
probably be possible by modification or interfacing to existing
software.}

\proj{Export Copies of SPECX and JCMTDR}{radio}{export}
{We would provide easily-accessed versions of SPECX and JCMTDR
(packaged with other required software), most probably based on the
new WWW ``Software Store'' once it is populated with the necessary
software items.}

\proj{Distribution of the DIFMAP Package}{radio}{difmap}
{We would survey users to determine the level of interest in this
package and, if appropriate, arrange to distribute it to Starlink
sites.}

\proj{UNIX Version of the FLUXES Program}{radio}{FLUXES}
{We would provide a UNIX version of this program, or a suitable
alternative if appropriate.}

\proj{AIPS Cookbook On-Line}{radio}{AIPScook}
{We would provide an on-line (WWW) version of the AIPS cookbook.}

\proj{Fix Figaro/KAPPA Parameter Problem}{radio}{param}
{We believe that the problem which arose was temporary and no longer
exists, but would investigate and provide a solution if necessary.}

\proj{Export SPECX Data to Visualisation Software}{radio}{visual}
{We would provide software and documentation to allow SPECX data to be
transferred easily to commercial visualisation systems.}

\proj{UNIX Port of IRAS MEM Software}{radio}{MEM}
{We would investigate the work required and likely usage of this
(originally VMS specific) product on UNIX and provide a port if
appropriate.}

\proj{RA, Dec Pointing Corrections in JCMTDR}{radio}{pointing}
{We expect this minor upgrade may be implemented by ROE staff, but
would perform the work ourselves if this is not the case.}

\proj{Importing ISO Data}{radio}{ISO}
{This requirement would be covered by projects described elsewhere
(see \S\ref{spec:ISO} and \S\ref{ip:ISO}).}

\subsection{X-RAY ASTRONOMY}

\proj{Maintaining FTOOLS/XSELECT Software}{xray}{FTOOLS}
{We would take on the task of building and distributing this software
for installation at Starlink sites and provide a suitable ``contact
person'' to help with problems.}

\proj{Sky Map/Catalogue Plotting Software}{xray}{skymap}
{We would provide a software package to cover the general area of
plotting astronomical catalogue data on top of image data, with the
ability to select catalogue entries and specify various plotting
options and coordinate systems. The solution would be designed for use
both inside and outside X-ray astronomy, and would probably depend on
new astrometric capabilities described elsewhere
(\S\ref{infra:astrom}).

Investigation would be required to define the requirements of this
project more closely. There are many possibilities, and the eventual
scope is likely to be determined primarily by the amount of time
allocated to the project. This year, we would aim to provide an
initial, simple but effective solution that could be re-assessed for
possible further development in future if it proves successful.}

\proj{Port of Starlink Software to Linux}{xray}{PC}
{We note the relatively low priority given to the port of Starlink
software to Linux (\S\ref{infra:PC}) given \htmlref{by the X-ray
SSG}{xraySSG:PC}.}

\proj{Further Development of Tcl/Tk Facilities}{xray}{cl}
{We also note \htmlref{the SSG's support}{xraySSG:cl} for further work
to develop scripting capabilities based on Tcl/Tk. We would propose to
consider this as part of a project to examine command languages in
general (see \S\ref{infra:cl}).}

\proj{Improvements to GUI Tools}{xray}{GUI}
{We note the \htmlref{SSG's support}{xraySSG:GUI} for the provision of
further graphical interfaces to software. In the absence of specific
proposals, we would respond to this by providing any necessary small
enhancements to the GUI tools we developed during 1995. The experience
of developing a graphical interface for ASTERIX (being undertaken by
non-Starlink staff in Birmingham) is likely to be particularly
valuable in identifying possible improvements.}

\subsection{DOCUMENTATION}

\proj{Cookbooks}{docs}{cookbooks}
{We have found it difficult to persuade SSGs to propose good subjects
for which cookbooks might be written. Where they have done so, they
have tended to give them low priority, which does not tally well with
the results of the last software survey which clearly identified them
as a high priority.

To address this, we would conduct a simple survey of users to obtain
suggestions for the areas we should cover (providing a few examples)
and to identify which would be most useful. Depending on the outcome,
we would then investigate the best way of gathering the necessary
information and producing a cookbook for (say) the most popular half
dozen suggestions.

We would include the cookbook suggestions noted elsewhere in this
exercise in order to test their relevance to the user community.}

\newpage
\appendix
\section{\xlabel{spectroscopy_ssg_input}SPECTROSCOPY SSG INPUT}

\subsection{Membership}

\begin{description}
\item[Members:]\mbox{}\\
Dave Wonnacott (secretary, MSSL)\\
Keith Smith (chair, UCL)\\
Michael Lemke (Cambridge)\\
Robert Rolleston (Armagh Observatory)\\
Suzanne Ramsay Howat (ROE)\\
Tim Harries (St Andrews)\\
Tim Naylor (Keele)

\item[Also present:]\mbox{}\\
David Berry (Manchester)\\
Horst Meyerdierks (ROE)\\
Malcolm Currie (RAL)\\
Rodney Warren-Smith (RAL)

\item[Apologies:]\mbox{}\\
Bahram Mobasher (ICSTM)\\
Phil Puxley (ROE)

\item[Meeting date:]Wednesday 6th September 1995

\item[Venue:]University College London
\end{description}

\subsection{Specific Recommendations}

This part of the report documents the recommendations of the
Spectroscopy Software Strategy Group (SSSG) in order of priority,
without specific consideration for their manpower requirements.

\subsubsection{Migration of FIGARO to standard NDF library}

\begin{description}
\item[Recommendation:]
The spectroscopic reduction package FIGARO should be converted to make
use of Starlink's standard NDF library, in order to make it consistent
with the rest of the SSC and to ensure its long-term compatibility
with Starlink's objective of providing transparent access to multiple
data formats.
\end{description}

The Starlink Software Survey identified FIGARO as {\bf the} most
important scientific package in the SSC.  User feedback suggests that
this situation is essentially unchanged despite the increasing use of
external packages such as IRAF.  In the previous year's proposal, the
SSSG discussed the feasibility of converting the entire FIGARO package
to make use of the standard NDF library, primarily as a means of
providing transparent file-format conversion and for consistency with
the rest of the SSC. At that time, the manpower requirements for such
an undertaking were considered to be outweighed by the perceived
demand for such a facility.

In view of Starlink's new, long-term strategy concerning
inter-operability of its software with external packages, the SSSG
felt it appropriate to re-examine this issue.  It agreed that the new
strategy reinforces the value of transparent file-format conversion
(`on the fly'), especially for a package such as FIGARO which, by its
nature, contains numerous components (`applications') which are used
in quite diverse contexts. The SSSG was also particularly concerned
with the long-term security of FIGARO in view of its continued use of
an obsolete NDF access library (which requires constant updating
parallel with modifications to the standard NDF library).

The SSSG therefore considers it timely to address this outstanding
problem, and accordingly recommends that a phased programme of
migration of FIGARO applications to use the standard NDF library be
instituted. A revised (but as yet unqualified) estimate of the
manpower requirements for this conversion is considerably more
optimistic than last year's: approximately 2 programmer-months would
be needed for a new basic library interface, with the average FIGARO
application requiring a small overhead of typically a fraction of a
day.  After the initial library interface is developed, it is expected
that individual applications could be migrated in a user-transparent
fashion (on a `most-important-first' basis), thereby mitigating the
impact of this project on Starlink's broader spectroscopy-related
programme.

\subsubsection{Echelle Data Reduction}

\begin{description}
\item[Recommendation:]
On the basis of a report submitted to the SSSG, ECHOMOP was identified
as providing the preferred framework for a comprehensive echelle
reduction facility.  It is recommended that new releases of this
software should be placed at an elevated level of maintenance support
in order to promote its development.
\end{description}

In last year's proposal, the SSSG identified echelle data reductions
as an outstanding area for which the existing software provision is
inadequate.  This problem is expected to become more acute with the
advent of ever-larger CCD detectors (e.g., EEV9 on UES/WHT) sampling
more fully the free-spectral range of new and existing echelle
spectrographs ({\em e.g.}\ HIRES on the Keck; HROS on Gemini).  The
processing of large (in size and number) echelle datasets is {\bf
still} the single most conspicuous area in which software improvements
and automation could ease the post-observing burden of astronomical
spectroscopists.

The SSSG has now received a provisional report from Starlink outlining
the major echelle reduction packages currently available to the
international astronomical community.  This report identifies the
DOECSLIT package within IRAF, and the ECHOMOP package within the SSC
as the two leading echelle reduction facilities.  There are several
advantages and disadvantages associated with each of these packages:
in brief, DOECSLIT is a stable product accessed via the IRAF command
language (cl) familiar to and popular with increasing numbers of
astronomers; ECHOMOP on the other hand, provides a more comprehensive
set of echelle reduction tasks with full provision for error
propagation, albeit hidden behind a somewhat unfriendly
user-interface.  In light of the fact that the author of DOECSLIT has
no immediate plans to improve his software, the SSSG agreed that
ECHOMOP represents a valuable component of the
echelle-spectroscopist's arsenal deserving further support and
development by Starlink.

It was felt that the most responsive approach to dealing with the
various faults identified in ECHOMOP would be to place its next
release in an elevated status of maintenance support.  Improvements to
the software should then be guided by the responsible programmer
actively soliciting input from all users ({\em e.g.}\ via an automated
`gripe' command or e-mail shots).  Bugs requiring only a few
programmer-days effort to fix would be resolved with a quick
turnaround.  Problems needing more than 2 programmer-weeks effort to
address demand a more appropriate, co-ordinated strategy: they should
be collated to build a prioritised list of user requirements, which
can then be reviewed for action at the next SSSG meeting.

User feedback obtained to date indicates considerable dislike of the existing
ECHOMOP interface, and it is expected that demand for an improvement
in this area may prove a user priority.  Options discussed in this
regard by the SSSG included (a) the provision of a GUI (perhaps with a
facility for compiling reduction scripts), and (b) the incorporation
of ECHOMOP within the IRAF cl.  There was some feeling that the latter
option might safeguard the long-term competitiveness of this package
and promote its use in the broader community, but should not be
undertaken at the expense of ECHOMOP's capacity to handle error
propagation. Both options have significant manpower implications and
of course would be contingent on there being sufficient user demand.

\subsubsection{IUE Final Archive Data}

\begin{description}
\item[Recommendation:]
A facility for reading IUEFA fully extracted spectra should be
developed in order to ensure access to this valuable data resource as
and when it becomes available. In the event that {\bf high-resolution}
IUEFA data products are produced in calibrated-image format only, a
suitable data reduction facility should be made available. The
relative merits of existing packages in the SSC and elsewhere as
suitable reduction engines in this context will need to be assessed.
\end{description}

The IUE Final Archive Project (IUEFA) will ultimately provide a
valuable data resource for UK ultraviolet astronomers working in a
broad diversity of fields.

The SSSG understands that it is now probable that the IUEFA will
produce only low-resolution spectra in {\bf fully extracted} form;
high-resolution data will be available as photometrically calibrated
and geometrically re-sampled images, albeit of considerably higher
quality than the existing, standard IUESIPS product.  In any event,
the development of a suitable FITS binary table reader tailored to the
IUEFA format will be essential in order to allow fully extracted IUEFA
spectra to be imported into the Starlink environment for manipulation
and analysis using tools familiar to Starlink users ({\em i.e.}\ the
SSC).  This would represent a fairly self-contained task requiring
perhaps 1-2 programmer-months, and should be scheduled so as to
coincide with the IUEFA's first release of data to the UK community.

In the event that IUEFA {\bf high-resolution} products are only
available as photometrically calibrated images, a suitable reduction
facility will be required at the time of their release.  Within the
SSC, IUEDR and ECHOMOP have been identified as potential reduction
engines for these data -- in essentially their existing forms --
although both would require retro-fitted file-reading facilities
capable of recognising and handling header information unique to IUE
data.  The relative merits of these packages in this context need to
be carefully assessed before any concrete proposals on this matter can
be advanced, bearing in consideration the possibility that some
appropriate tailored package in, for example, the IRAF environment may
also become available.  Contingent on a final decision being made on
the nature of IUEFA high-resolution data products, it is recommended
that Starlink prepare a brief report outlining the relative merits of
these options and the associated manpower requirements for
consideration of the SSSG.

\subsubsection{Spectropolarimetric Data Reduction/Analysis}

\begin{description}
\item[Recommendation:]
The spectropolarimetric data {\bf reduction} package TSP should be
updated to handle data from all instruments normally available to UK
astronomers.  The spectropolarimetric data {\bf analysis} package
POLMAP should be released as part of the SSC as soon as possible.
\end{description}

In the previous proposal, spectropolarimetry was identified as a
growth area in UK astronomy with the increasing availability of
dedicated instruments at several observatories (e.g., ISIS polarimeter
on the WHT; RGO spectrograph polarimetry module on the AAT).  The SSSG
felt that forseable demand for a spectropolarimetric data reduction
facility could be accommodated within the TSP package, provided that
it is updated to handle data from the many instruments currently
available to the UK community.  Furthermore, a new spectropolarimetric
{\bf analysis} package, POLMAP, has been submitted to Starlink for
inclusion in the SSC, but is still awaiting release.  It is strongly
recommended that this is given immediate attention; the exposure of
this software to wider use will promote its development and allow a
better assessment of the software needs of the spectropolarimetry
community to be made.

\subsection{General Discussions}

In addition to the firm recommendations made above, several other
topics were discussed by the SSSG for which clear decisions were
reached, but do not merit the same high priority as those listed
above.  In some cases, the implications of these recommendations are
broader than the specific remit of this SSG.  They are listed below in
no particular order of priority.

\subsubsection{ISO Data Reader/Converter}

Although the ISO Project has yet to finalize its format for data
products, a provisional specification has been released on CD-ROM.
Consistent with the prioritised recommendation presented in last
year's proposal, Malcolm Currie was actioned by the SSSG to
investigate this provisional format with a view to using it as the
basis for the development of a tailored FITS reader.  As the launch of
ISO is imminent (November 1995 at time of writing), he was urged to
seek a final specification from the ISO Project at the earliest
possible date.

\subsubsection{\label{specSSG:CGS4DR}Portable CGS4DR}

The SSSG discussed the issue of the port of CGS4DR to Unix systems,
following on from the recommendation made in last year's proposal that
this be carried out as soon as possible.  It is understood that an
interim release of Portable CGS4DR is now available, although several
critical bugs were reported at the meeting which preclude its use on
data part-reduced at the telescope.  The SSSG expressed support for
the implementation of a GUI-version of CGS4DR as a collaborative
project between JAC and Starlink, and encouraged those involved to
ensure a rapid release to the user community.

\subsubsection{Multi-Object (Fibre) Spectroscopy}

Multi-object (fibre) spectroscopy is, like spectropolarimetry, an
expanding field in UK astronomy with several new instruments
commissioned over the last year (WYFFOS at the WHT; 2dF/FLAIR at the
AAT).  In relation to this, Malcolm Currie presented a report to the
SSSG concerning the provision within the SSC (and other sources of
software) for the reduction of multi-object spectra.  He emphasized
that while there are no dedicated multi-object spectroscopy reduction
packages, per se, within the SSC, it is in principle possible to
combine elements from several different packages to achieve the same
effect. He also pointed out that there are several instrumentation
groups currently in the process of developing their own
instrument-specific software (e.g., an IRAF-based package for WYFFOS).
He drew attention to the DOFIBER package within IRAF for which a
detailed user guide is available.  His recommendation to the SSSG was
that any significant manpower effort in this area undertaken by
Starlink should either substantially improve on existing software or
take the form of documentation outlining how the reduction of
multi-object spectra can be achieved by combining various components
of the SSC and other packages.

\subsubsection{\label{specSSG:IRAF}Inter-operability of Starlink and Other-Party Software}

The SSSG expressed support for the concept of inter-operability of
Starlink and other-party software (e.g., IRAF) both verbally, and
through its recommendation that the widely-used FIGARO package be made
consistent with the rest of the SSC in its use of a standard NDF
library.  The SSSG recognized that this represents an important factor
in promoting the use of Starlink software outside the UK community.

\subsubsection{Documentation}

The SSSG discussed the issue of documentation, particularly in
relation to complex software such as ECHOMOP.  Support was expressed
for on-line tutorial-format documentation which incorporates {\bf real}
data with which the new user can experiment under guidance. It was
felt that this would be valuable in providing a {\bf first-hand}
experience of the use of a package via exposure to a concrete example
with `worked solutions'.

\subsubsection{Communications}

In order to advertise the activities and recommendations of the SSSG,
its Secretary has prepared, and continues to maintain, an appropriate
collection of WWW pages which are accessible via the internet.  A
`user comments' facility has been implemented in order to encourage
input from the community outside the usual formal channels.  The SSSG
recommends that Starlink advertise the availability of this facility
and encourages users to take up the opportunity to influence the SSSG
through this channel.  The provision of hyperlinks from appropriate
Starlink WWW pages is requested.

The SSSG WWW page can be accessed at:
\begin{quote}
\begin{verbatim}
http://mssly1.mssl.ucl.ac.uk/~ddxw/sssg/sssg_hp.html
\end{verbatim}
\end{quote}

\subsubsection{\label{specSSG:PC}Linux port}

The SSSG has received positive feedback from users concerning the
prospective port of the SSC to Linux, especially in relation to
spectroscopic reduction and analysis software.  Now that the NAG
routines have been removed from the SSC, the SSSG strongly encourages
Starlink to continue with the Linux port.

\subsection{Glossary}

\begin{quote}
\begin{tabular}{l@{--}l}
AAT & Anglo-Australian Observatory \\
cl & command language (IRAF) \\
CGS4DR & An infrared data (CGS4) reduction package \\
DOECSLIT & echelle reduction package in IRAF \\
DOFIBER & multi-object spectrum reduction package in IRAF \\
ECHOMOP & An echelle data reduction package \\
FIGARO & A spectroscopic data reduction package \\
FITS & Flexile Image Transport System \\
GUI & Graphical User Interface \\
IRAF & Image Reduction and Analysis Facility \\
ISO & Infrared Space Observatory \\
IUE & International Ultraviolet Explorer \\
IUEFA & International Ultraviolet Explorer Final Archive \\
Linux & Unix operating system for PCs \\
NAG & Numerical Algorithms Group \\
NDF & N-dimensional Data Format \\
POLMAP & A spectropolarimetry reduction and analysis package \\
SSC & Starlink Software Collection \\
SSG & Software Strategy Group \\
SSSG & Spectroscopy Software Strategy Group \\
TSP & Time Series and Polarimetry reduction package \\
WHT & William Herschel Telescope \\
WWW & The World Wide Web \\
WYFFOS & Wide Field Faint Object Spectrograph \\
2dF & the 2-degree Field (at the AAT)
\end{tabular}
\end{quote}

\newpage
\section{\xlabel{image_processing_ssg_input}IMAGE PROCESSING SSG INPUT}

\subsection{Membership}

\begin{description}
\item[Members:]\mbox{}\\
Barbara Bromage (UCLAN)\\
Carole Mundell (Jodrell Bank)\\
Dave Carter* (RGO)\\
Helen Walker (RAL ISO Support)\\
Myf Bryce* (Manchester, Chair)\\
Nial Tanvir (IOA)\\
Nigel Metcalfe (Durham)\\
Phil James (LJMU)\\
Steve Phillipps (Bristol)\\
Tim Gledhill (Herts)

\item[Also present:]\mbox{}\\
David Berry (Manchester)\\
John Palmer** (Manchester)\\
Peter Draper (Durham)\\
Rodney Warren-Smith (RAL)

\item[*] Not at meeting
\item[**] Deputizing as chair for M. Bryce

\item[Meeting date:]October 25, 1995

\item[Venue:]Manchster Department of Physics and Astronomy
\end{description}

\subsection{Recommendations and Priorities}

The Group has recommended that Starlink undertake the following tasks.
Four priority groupings are suggested, with group 1 being the most
urgent.  The discussions of the Group are appended in following
sections to indicate the reasoning behind the decisions.

\subsubsection{\label{ipSSG:astrom}PRIORITY 1}

\begin{description}
\item[Astrometry]\mbox{}\\
Produce an astrometry `library', probably based on or layered onto
SLALIB, along with appropriate NDF handler routines. Existing
applications should be able to use these with only minor adjustments.

\item[Large data sets]\mbox{}\\
A study should be undertaken within the next year to assess the impact
of the increasing size of data sets on the usability of Starlink
software. In the interim a `guide to the processing of large data
sets' should be produced.

\item[GUIs]\mbox{}\\
It is felt that GUIs provide a useful method for constructing
easy-to-use front ends for specific tasks, integrating a number of
packages or routines. Over the coming year, Starlink should commission
a study to find out which tasks could most benefit from such
treatment.

Starlink should utilize the work it has done so far on GUIs to produce
a data inspecton tool with a `look and feel' along the lines of
SAOIMAGE/XIMTOOL, but with much enhanced astonomical
functionality. Features that are at present partially supplied by
KAPPA INSPECT and IRAF IMEXAMINE could be integrated and extended. New
features such as the display of spectra extracted from a slice `into'
a three dimensional data-cube could also be incorporated.

\end{description}

\subsubsection{\label{ipSSG:IRAF}PRIORITY 2}

\begin{description}
\item[Integration of Starlink software and IRAF]\mbox{}\\
It was felt that the ability of Starlink packages to appear as IRAF
`tasks' was important. Work should continue on the infrastructure
required to fulfill this aim and, within 1 year, FIGARO should be the
first package integrated.  Present efforts to persuade the IRAF team
to adopt the NDF standard as an alternative data format should
continue.

\item[Polarimetry Support]\mbox{}\\
Suitable polarimetry data reduction software should be made available
to the Starlink community as soon as possible.

\item[\label{ipSSG:figaro}Figaro - NDF support]\mbox{}\\
FIGARO routines should be upgraded to use standard NDF access
routines.

\end{description}

\subsubsection{PRIORITY 3}

\begin{description}
\item[Interface between Starlink Software and IDL]\mbox{}\\
A suitable Starlink interface into the IDL environment should be
produced.

\item[ISO data]\mbox{}\\
A small amount of effort may be needed to interface Starlink packages
and specific software written by the ISO technical experts.

\item[Data Format Conversion]\mbox{}\\
The CONVERT documentation should be enhanced to detail the loss of
information that may occur as a consequence of data format
conversion. Additionally, the CONVERT routines should be altered to
warn users of such a possibility prior to the conversion taking place.

\item[Cookbooks]\mbox{}\\
The production of 2 cookbooks is recommended; 1, ``A guide to the
basic reduction and calibration of CCD data''; 2, ``Performing
photometry: a guide to the available packages''

\end{description}

\subsubsection{PRIORITY 4}

\begin{description}
\item[PC software port]
\end{description}

\subsection{Discussion}

\subsubsection{ISO Data}

The group heard that test image data was still not available. It is
understood that Starlink packages will be able to read the ISO binary
FITS files, unlike IRAF which will not do so at present.

A number of problems associated with processing ISO data were
discussed:

\begin{itemize}
\item ISO data will include `stacks' of relatively raw image data.
\item There will be a need to correct for instrument drift.
\item There will probably not be a good model of detector response until
some way through the mission.
\item Astrometry information will be contained in a file separate from
the image header.
\end{itemize}

It is expected that data sets will start to arrive around March-May
1996.  Bearing in mind that firstly, the ISO data flow and therefore
data processing software will have a limited lifetime and secondly,
ISO specific software is being/will be written by the ISO instrument
technicians, it was felt that any Starlink effort should be aimed at
interfacing existing Starlink packages (possibly CCDPACK) with the
software written by the ISO instrument technicians. The level of
Starlink effort is anticipated to be of the order of 1 man-month over
the combing year.

\subsubsection{Large Data Sets}

It was generally felt that the average size of data sets has been
increasing and will continue to increase over the next few years. This
is true across all the wavelength ranges. Some doubt was expressed as
to how these large data sets could be processed efficiently. It was
recommended that Starlink commission a study into this phenomenon
during the coming year. In addition it was suggested that a guide to
processing large data sets be produced by Starlink to clarify the ways
in which present software should be used to handle such data.

\subsubsection{GUIs}

The Group did not consider that it was either necessary or desirable
to produce a `standard' Starlink GUI. It was felt, however, that the
use of GUIs to aid the performance of specific tasks was valuable and
therefore it was recommended that Starlink continue it's present GUI
infrastructure work. Further, the Group recommended that over the
coming year Starlink should commission a survey into the possible
needs of users vis a viz GUI fronted tasks.

It was accepted that users are keen to be able to use data processing
tools across the wavelength range, and it was felt that GUIs should
aid this ideal.

In the case of the `image processing community' it was thought that
there was an identifiable need for an enhanced interactive data
visualisation tool. The popularity of a tool such as SAOIMAGE was
based on the fact that it was fast and responsive to use and performed
a small subset of tasks in an intuitive, user-friendly manner. The
Group therefore suggests that Starlink should attempt to produce such
an enhanced tool over the next year.

\subsubsection{Starlink/non Starlink software integration}

The Group felt that it is increasingly important for Starlink to
enhances the integration of Starlink software with other software used
by the global astronomical community. Not only would this reduce
effort in `reinventing the wheel' but could well result in an
increased interest internationally in Starlink software.

\begin{itemize}
\item{\bf IRAF}
The efforts being made to allow Starlink packages to appear as tasks
within IRAF were welcomed. It was accepted that an effort equivalent
to 6 man-months would be required initially to provide a 'production
system' to allow such integration.  A subsequent expenditure of around
1 man-month per package was considered feasible. (See also Data format
section below).It was agreed that as a first step the FIGARO package
should be made available as such a task.

\item{\bf IDL}
It was noted that there was a large community of IDL users. It was
therefore important to allow access between the IDL environment and
Starlink software. The group agreed that it was therefore desirable
that Starlink should work on an IDL interface of some kind.
\end{itemize}

\subsubsection{Data formats}

It was reported that there has been progress towards persuading the
IRAF team to make available routines that would enable NDF structured
data to be accessed from IRAF. The Group considered this a highly
desirable development and encouraged Starlink to continue along this
path.

It was noted that when converting between data formats it is often the
case that header information is `lost' as a natural consequence of
differences in the way the data is stored. It was recommended that the
documentation for the CONVERT package should be enhanced to detail the
potential losses that can occur when using CONVERT routines. It was
also requested that if possible the CONVERT routines should be
modified to provide an `on screen' warning before such a loss occurs.

\subsubsection{\label{ipSSG:PC}PC software port}

The Group did not feel that the port of Starlink software to LINUX on
PCs was a particularly pressing need. It was thought that at present
there would only be a limited number of high-end PCs available at
Starlink sites and that the introduction of `yet another operating
system' would in fact be a retrograde step.

\subsubsection{Astrometry}

It was recognized that this was potentially a large project. It was
however felt that it was very important to be able to handle
astrometry data in a unified way.  It was suggested that a way forward
was to produce an software library, probably based around or layered
onto SLALIB, which with an appropriate NDF handler could be utilized
by existing applications with only minor adjustments. It was
anticipated that such a library could be in place by the end of the
year for an expenditure of approximately 9 man-months.  The Group
recommends that such an approach be adopted.

\subsubsection{Polarimetry}

Dual-beam polarimetry instruments are operating or soon will be
operating at all major observatories. There is no Starlink software in
place to handle such data. TSP can provide rough-and-ready processing,
but lacks the ability to follow errors, plot vector data, handle
circular polarization etc.

The Group recommended that suitable reduction software be made
available as soon as possible. This will probably entail liazing with
the instrument specialists at the observatories (UKIRT and/or
AAO). Existing software there should be adopted or adapted if
possible. It was felt that Starlink has a wealth of Polarimetry
expertise and should be able to provide a competitive package.

\subsubsection{Visualization tools}

A survey of visualization tools has been undertaken at the request of
the Theory SSG. The Group felt that nothing further specific should be
undertaken for the next year until the recommendations of the survey
had been made available and some experience of the tools had been
gathered.

\subsubsection{Cookbooks}

The Group recommended that Starlink should provide the following two
'cookbooks' over the coming year:
\begin{itemize}
\item A guide to CCD reduction.
\item A guide to Photometry - with particular emphasis on the use of
appropriate packages.
\end{itemize}

\subsubsection{Figaro}

The Group agreed that it was desirable that Figaro be upgraded to
utilize standard NDF access routines.

\newpage
\section{\xlabel{theory_and_statistical_analysis_ssg_input}THEORY AND STATISTICAL ANALYSIS SSG INPUT}

\subsection{Membership}

\begin{description}
\item[Members:]\mbox{}\\
Geraint Lewis (secretary, gfl@ast.cam.ac.uk)\\
Huw Lloyd (hml@staru1.livjm.ac.uk)\\
Melvyn Davies (chair, mbd@ast.cam.ac.uk)\\
Neil Francis (N.Francis@astro.cf.ac.uk)\\
Steinn Sigurdsson (steinn@ast.cam.ac.uk)

\item[Also present:]\mbox{}\\
David Berry (dsb@ast.man.ac.uk)\\
Rodney Warren-Smith (rfws@star.rl.ac.uk)

\item[Apologies:]\mbox{}\\
Ian Bonnell (IoA)\\
Matthew Brown (QMW)

\item[Meeting date:]9th November 1995

\item[Venue:]IoA, Cambridge
\end{description}

\subsection{Reports}

The group considered two reports produced by Starlink personel in
response to the last SSG meeting.

\subsubsection{Visualization Tools}

Since the last meeting, David Berry (Manchester) has been
investigating various visualisation software packages and he reported
his findings to the group. In summary, David felt that the package
produced by IBM -- DX -- seemed the most promising for astronomical
applications. IBM are willing to send the software to users for free
for a 60-day trial period.  David, and others, had done some software
development for use with DX, as well as writing some documentation
that would become publicly available shortly. The group felt it
important that the whole astronomical community hear about the
availability of DX -- possibly in the next Starlink bulletin. They
also felt it important that starlink provide user support during the
60-day trial period.

\subsubsection{Computer Algebra Software}

The group studied the report by Martin Clayton (UCL) on computer
algebra software. Martin has concluded that most Starlink sites that
have a need for such software have found a local solution, though one
member of the group recounted having used the version of Maple on
STADAT. We felt that it would be a sound idea to further advertise the
availability of Maple at RAL, and see if the current, extremely low,
rate of usage increases over the coming year. If it didn't the group
felt it would be reasonable to discontinue this service.

\subsection{Requests for the Year Ahead}

The group identified three key areas where Starlink resources should
be applied:

\begin{enumerate}
\item In order that Starlink users be able to successfully use
the DX visualisation software during the 60 trial period, it was felt
that David Berry, or another, equally expert, Starlink applications
programmer be available to help users during December/January.
Explicitly, they might be called upon to produce small pieces of code
to transform users' data files into the formats readable by the
software.

\item In a variety of contexts, all members of the group related problems
with anonymous ftp transfer of important software available publicly
from sites chiefly in the US. Given the current (lack of) speed in
communications to North America at most times, the idea of a more
local archive was welcomed. Such a system, presumedly located at RAL,
could contain software such as CLOUDY, and various hyrodynamical codes
currently available from the Illinois Supercomputing Center, for
example.  We feel that the community should be asked for further
suggestions of software packages important to them.  We understand
that other SSGs have also suggested such an archive, for observational
packages such as IRAF and AIPS; it would seem logical to combine the
efforts for both the observational and theoretical communities.

\item All members of the group welcomed the introduction of fortran 90
to all Starlink nodes. Some concern was expressed relating to the
compatibility of existing software -- for example NAG routines -- with
the new fortran.  The group identified as a key need for the coming
year that resources within Starlink be used to ensure as smooth a
transition as possible to Fortran~90 within the community.

\end{enumerate}

\newpage
\section{\xlabel{information_services_and_databases_ssg_input}INFORMATION SERVICES AND DATABASES SSG INPUT}

\subsection{Strategy}

\subsubsection{Information Services}

The use of WWW has continued to accelerate since the first meeting of
this SSG and there is no reason to expect this growth to slow. The
number of WWW browsers has grown from about two 2 years ago to around
a dozen now. Current HTML standards have failed to stay ahead of the
developers with a resulting proliferation of extensions to the mark up
language.  The next year or two should see the more popular and well
thought out extensions being incorporated into the standard, which
should enable Starlink to construct WWW documents containing tables
and other complex formatting constructs already used in the paper
documentation.

There is still no clear choice for the best mark up language for Starlink
documentation. Ideally a form which can be used to generate both HTML
and paper based documents would be used, but at the moment we use a
language oriented towards paper based documents with an HTML convertor.
Within the next 5 years we can anticipate further developments in mark
up languages but in the mean time we should proceed with making Starlink
documentation WWW based using the current compromise.

Developments in WWW on a 5 year time scale are much harder to predict.
They will almost certainly include next generation document languages,
such as VRML and other interpreted interactive documents (such as Java).
The applicability of these languages to the astronomical community is
impossible to anticipate at this time. Many of what seem to be outlandish
developments now (eg VRML) make significant demands on hardware which
we may have to cater for.

\subsubsection{Software \& Data Distribution}

The distribution of Starlink software via the WWW is a positive step
to making the Project's efforts more visible and more widely used (but
see the recommendations for a couple of caveats on the way this is
achieved).

The use of CDROM as a medium for data distribution has expanded
significantly in the last 5 years, with the Digital Sky Survey leading
the way with about 100 CDROMs. There are a number of other large surveys in
progress, including the ESO Southern Sky survey, the INT Wide Field
survey and many more. The ever increasing data volume of these surveys
means the data will most probably be accessed at the site where the
survey data is reduced, via a WWW interface such as Skyview. While
such interfaces make adequate provision for interactive use, more
automated work (such as extraction of images for every object in a
catalogue) will become more difficult without standardisation of the
query protocol and provision of tools to interface between the user's
object catalogue and the WWW browser being used to access the
database.

The increasing use of such databases may mean that network bandwidths
outside Super-Janet become a problem. In this case a UK based mirror
site or sites may be advantageous for the more commonly used
databases.  Perhaps the natural successor to STADAT would be such a
mirror site.

\subsubsection{Database Software}

There is a consensus within the SSG that there is (and will continue
to be) a need for table handling software, and that the CAT package
developed by Starlink is the most suitable package which can meet this
need. CAT is sufficiently divorced from the rest of Starlink to have
international appeal and is the only FITS based DBMS around.

\subsection{Recommendations to Starlink}

\begin{enumerate}

\item Concern was expressed that the construction of the Makefile required
for Starlink software distribution was now becoming so complex that
there was a danger of discouraging software submission to Starlink.
Care should be taken that the additional infrastructure for
maintaining the WWW software archive should not further complicate
matters.

The project is encouraged to provide,

\begin{itemize}
\item a brief document describing the purpose of the standard Makefile,
including a diagram describing the different states of a package and
how the targets result in a change of state.

\item a Makefile 'wizard' of some sort, which could write the Makefile
given the essential information which comprises perhaps 5% of the text
in such files.
\end{itemize}

\item \label{infoSSG:NAG}The SSG fully endorses Starlink's efforts to
make its software more accessible to the international community and
encourages it to take those steps necessary to further this aim, in
particular the provision of an alternative to the NAG library.

\item The SSG agreed that instrumentation of software to gauge usage would
be a useful thing to implement, with the proviso that it should be
completely robust and that users should have to explicitly register
themselves to take part. In addition use should be made of
questionnaires targeted towards particular subject areas.

\item \label{infoSSG:cookbook}The SSG agreed that the emphasis should
be placed on cookbooks and tutorials rather than (say) taught courses,
but recognised the fact that such documents are best written by users
rather than implementors.  Persuading the former is more difficult
than ordering the latter!  (A radical thought; perhaps they could be
paid).

\item \label{infoSSG:PC}There was considerable interest among the
members of the SSG on the Linux operating system for PCs. The SSG
encourages the Project to continue its work on porting the SSC to that
platform.

\item Continuation of work on the CAT library and associated
applications.

\begin{itemize}
\item Continue and expand the publicisation of CAT, especially in the
US via such media as the Legacy magazine.

\item Implement the list of work items on CAT/CURSA, down to roughly
item \verb/#/15 on the list produced by ACD (dated 17/7/95)

\item Support conversion and distribution of selected catalogues from
the old SCAR catalogue set.

\end{itemize}
\end{enumerate}

\newpage
\section{\xlabel{graphics_and_infrastructure_ssg_input}GRAPHICS AND INFRASTRUCTURE SSG INPUT}

\subsection{Membership}

\begin{description}
\item[Members:]\mbox{}\\
Andrew Newsam\\
Barry Kellett\\
Carl Shaw\\
David Shone\\
Karl Glazebrook\\
Paul Alexander (chair)\\
Richard Bower

\item[Also Present:]\mbox{}\\
Alan Chipperfield\\
Brian McIlwrath\\
David Terrett\\
Malcolm Currie\\
Peter Draper\\
Rodney Warren-Smith

\item[Meeting date:]Tuesday 20th June 1995

\item[Venue:]Cavendish Laboratory, Cambridge
\end{description}

\subsection{Preamble}

This was the second meeting of the Graphics and Infrastructure SSG.
The discussion during focussed on short and long term issues (as we
had been specifically requested to do).  One immediate consequence of
this discussion was that we are of the opinion that it is important
for the planning exercise for the longer term form that STARLINK
software should take must occur in parallel with the shorter term
objectives.  We therefore present our discussion in this form as two
parallel streams and hope that time and effort will be found for our
recommendations under both headings to be implemented.

We noted the considerable progress already made by the STARLINK team
in implementing the full list of recommendations we had made at our
last meeting and congratulate them on this.  Our recommendations are
therefore framed on the assumption that certain work which is near
completion will indeed be completed in the near future.

\subsection{Recommendations for short term objectives}

\subsubsection{Support for Linux Platform}

\begin{description}
\item[Recommendation:] The port of the STARLINK software collection to
Linux should continue and that this work be given significant
priority.
\end{description}

We noted from RW-S that number of SSGs and the STARLINK panel were of
mixed opinion as to the value of this port, principally the objection
appears to be that the hardware on which Linux runs will not be
adequate for large data reduction programmes.  Our discussion at the
last meeting and this emphasized the importance of this port for two
reasons:

\begin{enumerate}
\item Although it is unlikely that STARLINK will provide and support
directly Intel hardware, such equipment is cheap and available to a
very large number of astronomers via other sources (private puchases,
university support etc.).  This makes it a very important platform
therefore for general use.

\item The port is not only a port to a new platform but also to using
the GNU gcc and g77/f2c compilers.  In this sense it is very much a
generic port and should be a valuable stepping stone to support for
the STARLINK collection on almost any UNIX platform.
\end{enumerate}

We note the related work on removing the dependence of NAG from the
software collection is also central to making software distribution to
any site a far easier task.

\subsubsection{\label{infraSSG:training}Training and tutorial support}

\begin{description}
\item[Recommendation:] We do not beleive any useful role would be served in
the area of graphics and infrastructure by STARLINK running tutorial
courses.  The effort required to do so would detract from higher
priority items.
\end{description}

We were asked by RW-S to consider this issue.  There was no support at
all for the proposal in the area of graphics and infrastructure, one
specific session of a course on high to write STARLINK was discussed
but found no favour.  It is difficult to see how such a course could
be run as a substantial facility for hands-on sessions would be
required if the course were to be of any value.

The SSG noted that the situation may be different in specific
application areas where the tutorials would be much more focussed on
specific astronomical or instrumental considerations; this is however
the concern of other SSGs.

\subsubsection{Bindings for C}

\begin{description}
\item[Recommendation:] A C-binding should be provided for all routines in the
STARLINK collection.
\end{description}

A growing proportion of scientists do not learn f77 (or f90).  It is
now not appropriate for a major enterprise such as the STARLINK to
effectively demand that applications are coded in FORTRAN.  It is
particularly odd that even libraries written in C are provided only
with a F77 binding in the software collection.  Having both a C and
F77 binding allows programmers to choose the best language for a
particular task and should be regarded as a high-priority item.

\subsubsection{Coordinate systems}

\begin{description}
\item[Recommendation:] A new library should be developped or SLALIB expanded to
provide generic support for coordinate systems.
\end{description}

A large number of application programs support (or should support)
directly coordinate systems, but there does not exist an accessible
library of appropriate routines.  Such a library could be produced
largely by extraction of routines from IRAS90.  The committee supports
this aim and believes it would be a valuable tool provided:

\begin{quote}
The support was at a level of a high-quality product; there is nothing
worse than an astrometris routine which fails, for example for large
fields or near the pole of its coordinate system.  We would not like
to see a library aimed at satisfying a short term need, but one which
had the quality of SLALIB.
\end{quote}

We urge STARLINK therefore to provide support at a high level and with
clearly documented accuracy levels and to achive this to aim to
support a limited number of the more common projects properly rather
than aim for complete coverage.

Any aspects of this library that can be separated from the underlying
data structures should be and provided independent of the rest of the
STARLINK software library in much the same way as SLALIB achieves.

\subsubsection{Graphics evaluation}

\begin{description}
\item[Recommendation:] STARLINK should undertake a detailed investigation of
the suitability of various modern graphics libraries such as PEX and
OpenGL with intention of switching to support these in preference to
GKS.  Simulataneously STARLINK to evaluate specific visualization
packages.
\end{description}

The need for improved suport for graphics, especially 3D, has a clear
scientific base:

\begin{quote}
Considerable amounts of data being collected from all telescopes are
in the forms of 3 or more dimensional data (spectral-line images for
example).  Traditional 2D graphics (PGPLOT and GKS etc.)  are not
capable of providing adequate visualization leading to poorer analysis
and interpretation of data that would otherwise be possible and wasted
astronmer time.
\end{quote}

There is also a technical issue:

\begin{quote}
The support for GKS under X windows is poor and it is a declining
standard.  If STARLINK is to continue to support graphical
applications it is essential that they move to a well supported
standard.
\end{quote}

We considered in detail whether a visualization tool (such as AVS, IDL
etc.) would be sufficient by itself.  Although we consider the
provision of such a tool to be a very high priority item we also
believe that there will be a continuing need to incorporate graphics
directly into application programs so that astronomers can interacts
with their data during the data processing stage and have support for
some of the esoteric requirements of astrometric projections etc.

At the present time there are really only two high-quality options,
PEX or OpenGL.  The former is defined by the X consortium and has a
free implementation distributed with X11R6 although for performance a
hardware specific implementation would be required.  OpenGL on the
other hand is a specification which is CopyRight Silicon Graphic Ins
(SGI).  OpenGL appears to be however a more widely used and superior
product.  There are some legitimate free implementations of OpenGL
appearing and these should be considered.  This situation is not much
different to the original situation for GKS however which was adopted
by STARLINK.  Our recommendation is that a thorough analysis of the
benefits and costs of these two emerging standards (and other
comparable products) should be undertaken by STARLINK and a report
produced.

In brief support is required for at least:

\begin{quote}
3D graphics, display list and immediate mode rendering, tools
suitable for the visualization of scientific data (cell arrays,
support for voxel objects would be an advantage).  Consideration
should also be given to ease of programming and workstation support.

A final issue is the cost implications of supporting either of these
tools.  Do we need PEX lisences for all X terminals?  Can PEX and
OpenGL both render directly to X (we believe OpenGL can).
\end{quote}

\subsubsection{Support for Solaris on non SPARC hardware}

\begin{description}
\item[Recommendation:] STARLINK should investigate, and hopefully implement
a port of the software collection to non-SPARC solaris platforms.
\end{description}

Solaris is one of the few variants of UNIX which is becoming widely
available on other hardware (Intel and PowerPC most notably). We note
the support for the new PowerPC range which IBM are to release in the
near future.  PowerPC 604 based systems have a very substantial
price/performance advantage (if the advertising is to be believed)
over almost all other processor ranges (SPARC, ALPHA, MIPS etc.).  If
STARLINK were to provide software support on such platforms it could
prove potentially useful when considering hardware purchases as it
would not imply support for additional operating systems.  We urge
STARLINK to investigate how easy it is to support the same operating
system on multiple hardware architectures and therefore gain valuable
information which will be of use in formulating decisions about
hardware purchases.

\subsection{Considerations of long term objectives}

We had a considerable discussion concerning the future direction (over
a five-year plus time span) of astronomical software supplied by
STARLINK.

Perhaps most surprising was the wide agreement on the most profitable
way forward. In brief the points of agreement were:

\begin{enumerate}
\item STARLINK should move towards developping a high-level (4GL) language for
support of astronomical data reduction.  In many ways this is is just
a binding for the existing software tools to an appropriate language.
In this respect we do not consider ICL to have a long term role as
many excellent alternatives exist which STARLINK would not have to
support directly.  Some possibilities (which may not be exclusive
choices) are:

\begin{description}
\item[tcl] --- provide a high-level scripting like approach
\item[idl]  --- incorporate astronomical ideas into the idl language
although this is of course proprietry software.
\item[java] --- the new network aware, platform indenpendent byte-interpreted/
or compiled, object oriented language (see below).
\item[perl] --- originally designed as the ultimate UNIX scripting tool
the latest perl release is designed to be easily extensible or be used
as an embedded intepreter (c.f. tcl).
\end{description}

One or more of these might be combined; indeed tcl/tk will have a very
close link to java.

Other possibilities exist but were not considered in detail, see item
3.

\item Emphasis should be on development of the underlying analysis code
and tools for constructing user interfaces should be taken largely
from other sources (e.g.tcl/tk).

\item Code and applications should develop in such a way that the binding to
the high-level interface to form the 4GL should as far as possible be
independent of the chosen high-level environment.  This would enable
the code to be re-used in many environments and would ensure greater
longevity for the system as computer science developments in user
interface design are vertain to occur over this period.

\item At present the most exciting route appears to be java/tcl/tk with
tcl/tk forming the scripting part and user interface design while java
provides an excellent object oriented base for a 4GL language.  The
astronomical tools would be added as modules to java and tcl/tk (which
will have direct support for java).  Astronomers would then interact
on one of a number of levels:

\begin{enumerate}
\item Use finished applications probably with GUI front ends.

\item Extend existing applications and combine them with system tools
and write scripts to combine other applications/aspects of the 4GL.
This would be done in a good scripting language, probably tcl.  Such
an approach would be accessible to the vast majority of astronomers.
The aim would be that most users would not need to do more than this
degree of coding.  This implies that there would need to be direct
support for data processing tasks in the high-level scripting
environment together with access to advanced graphics (in addition to
Tk) for data visualization.

\item Write new applications in an object oriented framework using the
lower-level binding of the 4GL say to java.  This would be the level
now undertaken by application programmers.  The language would be
efficient and optimized for algorithm development and implementation.

\item Extend the language by adding facilities in a low-level language of
the programmers choice (c, c++, f77...).  The need for this level of
support will decrease quickly after the initial implementation.  While
this pattern of use seems the best way forward there are other good
alternatives to the two languages tcl and java:

\begin{itemize}
\item tcl $-->$ perl?
\item java $-->$ use a lower level as of now C, C++, f90...
\end{itemize}

\item Finally we note the unanimous enthusiasm for java --- this is almost
certainly due to the fact that it is perhaps the first language which
has been developped from the bottom up to support those issues that
any one involved in software development knows are very important:

\begin{enumerate}
\item Object oriented including automatic memory management!
\item Network aware
\item Multi threading (multi tasking)
\item Platform independent
\item Support for various user interface possibilities (including
WWW).
\end{enumerate}

No doubt over the time scale envisaged competitors to java will
appear, but at the time of writing it appears as a very exciting new
development.

\end{enumerate}
\end{enumerate}

\subsection{Glossary of Non-STARLINK terminology}

\begin{description}
\item[AIPS]The Astronomical Image Processing System used in Radio Astronomy.
\item[IRAF]The US optical astronomers equivalent of AIPS.
\item[X86]The "X86" family or processors used in IBM compatible PCs, mostly
the 486DX and Pentium chips.
\item[PowerPC]The new chip from IBM/Apple; could be a very important platform.
\item[SPARC]The Sun chip used in its SPARC workstations.
\item[ALPHA]The new DEC chip --- currently the market leader in terms of
power, but Sun. IBM, HP, SGI are all pushing hard.
\item[SGI]Silicon Graphics --- probably make the best graphics workstations,
used extensively in visualization.
\item[IDL, PV-WAVE, AVS]Commonly used commercial visualization applications
\item[Khoros, apE]Good freely available visualization applications
\item[Motif]The now standard "look" and feel for X-windows (also a
programming library).  This will be available on ALL flavours of UNIX
workstations and is similar in appearance to WINDOWS
\item[WINDOWS]The Microsoft windowing system on a PC
\item[WINDOWS NT]The 32-bit Microsoft windowing system on a PC
\item[Tcl]Tool Control Language.  An excellent scripting language which
can be embedded in other applications (c.f. ICL).  In wide use and may
well become a "stanadrd".  It is free although now is supported by
SUN, but with a bried to be freely available on all systems (UNIX, PC,
MAC).
\item[Tk]A Motif like windowing system linked to tcl (from the same author).
This is a high-quality product again free and arguably far superior
(technically) to Motif/Windows!
\item[PERL]scripting language with (in V5) object-oriented approach, modules
for networking, tk; cn be used as an embedded interpretor.
\item[PEX, PHIGS, OpenGL]State of the art graphics packages -- see above
\item[GNU]Free software which is widely distributed under a license which
prohibits commercial exploitation.  Good quality stuff and often
superior to equivalent standard UNIX utilities.
\item[Linux]A free UNIX-like cloned operating system for the X86 architecture.
Uses GNU software extensively, but also supports X-windows etc.  A
very complete system.
\item[JAVA]The new network aware object-oriented programming language released
by Sun.
\item[g77]The GNU fortran compiler.
\item[f2c]The GNU fortran to C converter.
\item[C$++$]An object-oriented development of C.
\item[F90]The new FORTRAN standard.  It has many object-oriented
features.
\item[API]Application programmer interface.  The common term for the
subroutine/function calls used to access the facilities of a
programming library.
\end{description}

\newpage
\section{\xlabel{radio_mm_and_sub-mm_astronomy_ssg_input}RADIO, MM \& SUB-MM ASTRONOMY}

\subsection{Membership}

\begin{description}
\item[Members:]\mbox{}\\
A.G.Gibb (Kent)\\
D. Ward-Thompson (chair, ROE)\\
D.J.Titterington (secretary, MRAO)\\
J.F.Lightfoot (ROE)\\
J.L.Osborne (Durham)\\
K.J.Richardson (QMW)\\
M.A.Garrett (Jodrell Bank)\\
P.J.Warner (MRAO)

\item[Also present:]\mbox{}\\
H.Meyerdierks (ROE)\\
J.S.Richer (MRAO)\\
R.F.Warren-Smith (RAL)\\
R.Padman (MRAO)

\item[Apologies:]\mbox{}\\
D.L.Shone (Jodrell Bank)\\
D.S.Berry (Manchester)\\
G.H.Macdonald (Kent)\\
J.W.Palmer (Manchester)

\item[Date]16th August 1995

\item[Venue:]MRAO, Cambridge
\end{description}

\subsection{Summary}

This report represents the views of the members of the Radio, MM and
Submm SSG, as expressed at its second meeting, August 1995. The
document is divided into three sections: Firstly, the progress made by
Starlink since the last meeting is reviewed, and in particular the
progress in implementing the recommendations of the first meeting are
discussed:-- At the time of writing, 4 out of 10 items on last year's
wish-list have been discharged, with 3 more planned for later this
year. Secondly, the longer term needs of the community are discussed
both in terms of the developments in instrumentation that are in the
pipeline for the next few years, and in terms of attempting to predict
how we might expect software and data reduction techniques to evolve
away from current practices in the longer term. The most important
first steps along this road were felt to be related to exploring the
usage of data visualisation techniques. Finally, recommendations are
given on general matters concerned with the organisation of software
development and the SSGs, and a detailed software wish-list for the
next 12 months is provided at the end of the document, which has been
concatenated this year, at the request of Starlink. The wish-list
contains 13 items, of which 4 have been carried forward from last
year's list, and 3 of those 4 are already in Starlink's software plan
for completion in 1995. Since some items require relatively little
effort, and there are only 9 new items, the wish-list is felt to be in
line with the number of users represented by this SSG ($\sim$
200-300).

\subsection{Progress since last meeting}

The SSG reminds Starlink that the radio/mm/submm community still
represents around one-sixth of the UK astronomical community (see
report of last year's radio/submm SSG meeting), and so should be
provided with roughly one staff-year per year of applications
programmer effort, for as long as the current Starlink staffing levels
are maintained. Bearing this in mind, we review Starlink's progress in
the last 12 months, dealing with the detailed wish-list, item by item.

In summary, out of last year's total of 10 wish-list items: 4 have
been completed; 3 are planned for release by the end of 1995; 2 have
not been attempted; and 1 was cancelled (AIPS++ workshop). Provided
the goal of releasing the 3 planned items is met, the SSG felt that
this is a satisfactory result, but would prefer to see all of the
items on this year's list carried out. Furthermore, one of the
released items (SPECX) still contains a major bug, which should be
fixed at the earliest opportunity.

The SSG also decided that it should represent at least the
long-wavelength part of the ISO community. These users would currently
be dispersed among several SSG's, and their voices may not be heard in
unison by Starlink. This was felt to be within the remit of this SSG,
since ISO will work out to a wavelength of 200$\mu$m, which is in the
submm part of the spectrum. The validity of this move is backed by
SGP44, which clearly states that ``SSGs and the subjects they cover
may be modified'' (SGP 44.1 section 3.9).

\subsubsection{Last year's radio wish-list}

\begin{enumerate}
\item {\bf AIPS support.} Starlink reported that the responsibility for AIPS
support was given to David Berry (Manchester), but this has now passed
to John W. Palmer (Manchester), although the change was not publicised
widely. JWP is preparing an installation guide and will also be
available to provide user support in the near future. The SSG felt
that Starlink had satisfactorily met the requirements of this
item. However, ongoing support for AIPS remains the radio community's
top priority in the foreseeable future, and the SSG requests Starlink
to advertise John Palmer's role to those sites running AIPS.

\item {\bf AIPS++ awareness.} Starlink reported that RFWS attended the AIPS++
Steering Committee Review (December 1994), and reported on the
reorganisation of the project that followed the review. Progress
continues to be slow, with a full release not now expected before the
year 2000. AIPS++ development is now being targeted towards supporting
particular instruments, however, and some of the software that is
being developed for the Green Bank Telescope (e.g. for on-the-fly
single dish mapping) may be of interest to the UK community on a
shorter timescale. The SSG felt that Starlink had satisfactorily met
the requirements of this item.

\item {\bf AIPS$++$ Workshop.} This workshop did not take place due to the late
running of the AIPS$++$ project. The SSG felt that maintaining a
watching brief on AIPS$++$ progress was all that was needed from
Starlink at this stage.

\item {\bf DIFMAP.} Starlink reported that the author of the Caltech
VLBI DIFMAP mapping package had agreed to write a short article for
the next Starlink Newsletter, after which it would assess likely usage
of this package.  Unfortunately, development of DIFMAP appears to be
effectively frozen at present, although this may change if further
funding is found. The SSG felt that it was regrettable that this
simple item had taken so long to complete, but if the outcome of the
response to DIFMAP proved favourable, then it would request Starlink
to distribute this to interested sites.

\item {\bf IRAS90.} Starlink reported that this item, which requested
improvements in the ability to read data into IRAS90, had not yet been
dealt with, although it was in the plan for completion by mid-November
1995.  The SSG hoped that this target would be met, and would meantime
keep this item on the wish-list.

\end{enumerate}

\subsubsection{Last year's mm/submm wish-list}

\begin{enumerate}
\item {\bf GSD file reading.} Starlink reported that this request had been
satisfactorily met, in that Unix versions of SPECX and JCMTDR now
include code for reading and writing GSD format data files. The SSG
agreed, with the reservations outlined in the next paragraph.

\item {\bf Unix SPECX.} Starlink reported that SPECX had been ported to Unix,
and was now supported under Solaris and OSF. The SSG agreed that the
port had been completed, but felt that at least one major bug still
exists in all Unix versions of SPECX, associated with the handling of
HDS data files in Unix: open work files can be corrupted in some
circumstances if an untidy exit to the program is made. In addition,
access to files containing large data cubes can be extremely slow and
could be significantly shortened. These two matters have been
transferred to this year's wish-list.

\item {\bf Exportable versions of SPECX and JCMTDR.} Starlink reported that
the packaging of JCMT software to provide standalone versions of SPECX
and JCMTDR, which can be made available to non-UK observers to run at
their own institutions, is also being addressed. Starlink is
developing a WWW-based software distribution system which will allow
users to select software items by using an interactive form; they will
then receive the specified items and all the necessary dependent
libraries packaged as a distribution kit, in a single network file
transfer.  This system is now running in test mode, with an expected
release date of mid-November. Before JCMTDR can be packaged in this
way, an alternative must be found for a NAg library routine which
cannot be included in the distribution for copyright reasons. The SSG
felt that Starlink appeared to be making good progress in this
direction, and looked forward to the release of this facility.

\item {\bf  Gaussian fitting.} Starlink reported that the provision of
a general two-dimensional gaussian fitting procedure for JCMT data
reduction has not been addressed, and is not in this year's software
plan. The SSG regretted Starlink's decision to omit this item from its
plans; and transferred this to the current year's wish-list, with the
recommendation that the best place to include this facility may now be
as a variation of the fitting procedure within the PISA
package. Alternatively, an imitation of the AIPS routine JMFIT may be
quicker to write.

\item {\bf Extract\_Integrations.} Starlink reported that the provision of
a Unix replacement for the NOD2 routine Extract\_Integrations has not
been addressed, and is not in this year's software plan. The SSG
regretted Starlink's decision to omit this item from its plans;
however, since the user request for this item came from the submm
polarimetry community, and since the JCMT polarimeter will be
superseded by SCUBA next year, it was not felt that any software
effort should now be expended in this direction.

\end{enumerate}

\subsection{Forward look}

\subsubsection{Instrumentation plans for the next three years}

The SSG reviewed the instrumentation plans of UK and other sites in
order to identify the software requirements for the next few years.

\begin{description}
\item[Radio Astronomy]\mbox{}\\[-3.0ex]
\begin{description}
\item[VLBI]\mbox{}\\
Several new developments are expected within the radio community which
will lead to a significant increase in the amount of data to be
reduced, particularly in the area of VLBI. Upgrades to the European
VLBI Network (EVN/JIVE), including a new correlator at Dwingeloo,
receiver upgrades elsewhere, and a new recording system, will provide
incoming data rates approaching 1 Gbit/sec, with observers having to
deal eventually with raw datasets of the order of 10 Gbytes per
observation. This will clearly impact on the data reduction
requirements of the community.  The VLBA network is now operating in
production mode and is in use by UK observers; there is also the
prospect in the next few years of satellite VLBI experiments, and of
mm-band VLBI becoming available to the UK.

\item[Jodrell]\mbox{}\\
The main priority is to equip MERLIN with 15~GHz receivers.  There is
also a move towards frequency flexible operations with the forthcoming
upgrade of the E-systems dishes to a VLA-type carousel system. Other
MERLIN telescopes will be upgraded over the next few years. Once
complete it should be possible to change between observing bands on
timescales of minutes. The chief data reduction package remains AIPS.

\item[MRAO]\mbox{}\\
The prospects for the next 3 years include the CAT+, VSA and COAST
instruments at Cambridge, to add to the continuing observation
programmes of the Ryle Telescope. The software is in part locally
written, but also uses AIPS.

\item[Other sites]\mbox{}\\
Data from the 151MHz Southern Sky Survey of the Mauritius Radio
Telescope will begin to arrive for reduction at Durham. Whilst the
telescope-specific software will be handled elsewhere, the software
requirements of this group focus on the need for comparisons of
all-sky survey data from different telescopes at different
wavelengths.

\item[Summary]\mbox{}\\
All of these developments point to a large increase in the amount of
data to be processed, and this will no doubt give rise to problems of
its own, but the expectation is that most of the reduction will be
done using the existing software packages.  Telescope sites will
continue to develop local software for initial calibration and
interference checking, but will then rely on AIPS and/or the Caltech
VLBI package for subsequent processing. It is unlikely that AIPS++
will be adopted to any significant extent during this period; indeed,
one of the effects of the difficulties suffered by the AIPS++ project
is that support at NRAO for Classic AIPS is now once again very
strong, and its active life has almost certainly been extended until
the end of the century. However, new reduction algorithms are unlikely
to be added to Classic AIPS at this stage, and a watching brief on the
AIPS++ project should be maintained. Development of the Caltech VLBI
package, however, appears to be frozen at present, although this may
begin again if funds become available.

\end{description}

\item[MM/Submm Astronomy]\mbox{}\\
In the millimetre/submillimetre wavebands the major developments
chiefly revolve around the expected delivery of 3 new receivers for
JCMT -- SCUBA, RxB3 \& RxW.

\begin{description}
\item[SCUBA]\mbox{}\\
The installation of the SCUBA receiver at JCMT is now expected to take
place early in 1996. The software for SCUBA is under development at
ROE and will be installed in Hawaii at the same time. Initial
calibration will be carried out at the telescope, with raw data being
stored in Starlink NDF file format. Software for subsequent off-line
reduction will be in the form of an ADAM A-task, with reduced data and
error information being exported either in NDF format or as FITS files
for import to other packages. RFWS asked that Starlink be kept
up-to-date with the progress of SCUBA software development, so that
potential problems with interfacing to other Starlink software can be
identified at an early stage. A standalone version of the off-line
reduction package will be needed before long. In the longer term, an
implementation of the DBMEM package for SCUBA is likely to be a
requirement. Depending on the other committments of its author (JSR)
at that time, programming support from Starlink may be required.

\item[RxB3]\mbox{}\\
The new Canadian receiver for the JCMT, RxB3, is also due for delivery
in early 1996. This will be a new B-band receiver to replace the
current RxB3i. This receiver will, however, just produce data in a
format suitable for reduction by SPECX.

\item[RxW]\mbox{}\\
The third receiver due for delivery to JCMT in 1996 will be RxW, which
is being built at MRAO. This is a dual C-band and D-band receiver,
which is capable of producing higher sensitivity data than any
previously available to the UK community at these frequencies. The
data from this receiver will also be compatible with SPECX.

\item[Kent]\mbox{}\\
In the longer term, the University of Kent are building a (non-common
user) receiver for the 800/900 GHz band which will be mounted on the
Italian TIRGO telescope on the Gornergrat and on UKIRT. This receiver
should produce SPECX format data. However, completion of this
instrument is not expected for around 3 years. The arrangements for
observing time will be along similar lines to that used for RxG at
JCMT.

\item[SCUBA-POL]\mbox{}\\
There are also plans for adding a polarimeter to SCUBA to provide a
continuum polarimetry capability, with software development being
partly funded by ROE, QMW and the Japanese. However, there may be a
need for a small amount of Starlink input into this in the
future. This instrument is expected in around 2 years.

\item[Heterodyne arrays]\mbox{}\\
On a 3-5 year timescale, heterodyne array receivers, with a much
increased data collection rate, will start to become available to UK
astronomers. At this point enhancements to SPECX will be required, and
other packages should be kept under review. In particular, techniques
for dealing with large amounts of non-regularly spaced spectral line
data may become available elsewhere: e.g. the IRAM data reduction
package CLASS, and eventually, AIPS++.

\item[Summary]\mbox{}\\
The needs of the mm/sub-mm community over the next three years are
likely to be satisfied largely by continued development of the
existing packages. However, Starlink should remain aware of progress
being made at non-UK telescopes.

\end{description}
\end{description}

\subsubsection{The longer term view of software}

Against the background of Starlink General Paper SGP42, the SSG was
invited to envisage how it might like to see the software situation in
5 years from now. By way of introduction, RFWS reviewed the progress
that has been made since the first round of SSG meetings in setting
out the Starlink software strategy for the next 3 years, as described
in SGP42. One aspect of the new strategy relevant to this SSG is that
Starlink will be concentrating software development effort in
relatively narrow areas in future, while continuing to provide
maintenance support for software developed within the UK. Starlink are
however intending to provide a broader level of software distribution
for UK astronomy, including more items from non-Starlink and non-UK
sources. The current levels of manpower, in terms of support
programmers, are expected to be maintained over the next 12 months.

However, looking a little further forward into the future, the
committee spent some time to see if anything could be said at this
stage about how data reduction techniques in this field might evolve
in the longer term. Current software is based on `old technology'
using monolithic packages often developed for specialised jobs,
sometimes with overlapping functionality, but largely independent of
each other. User interfaces are command line or menu driven, although
the use of GUIs is increasing; graphical output is supported by calls
to underlying subroutine libraries.

Two developments that are likely to become more important to users in
this field are the emerging data visualisation products, such as AVS
and IDL, and the new scripting languages, such as Tcl/Tk.  Experience
with the visualisation tools is limited at present, but they would
seem to lend themselves well to the analysis and understanding of the
data structures that are produced in this area -- e.g. data cubes from
submm spectral line mapping. It is possible that as these tools are
brought into use, new ways of presenting and thinking about the data
will follow.  Starlink are evaluating some of the current commercial
products and expect to make some recommendations later this year. Some
thought should be given on how best to export data from Starlink
software (e.g. SPECX) to these packages.

The new scripting languages, such as Tcl/Tk and PERL, are already
becoming well known and are in daily use by some of the more adept
users within the community. There was some discussion about how they
can be applied to data reduction techniques; for instance by providing
batch processing scripts for standard reduction procedures. Such
scripts are already available (e.g. the AIPS pipeline for reduction of
Merlin data) and are well liked by naive users; however, reservations
about this were expressed.  It was agreed in the end that the ability
to set up procedures for off-line processing and the ability to
interact closely with the data are both essential requirements in data
reduction.

Another area of continuing interest to this SSG, which can involve the
use of scripting languages, is that of observation
scheduling. Flexible scheduling tools for SCUBA are being developed
and will be made available to observers in due course.

Database software is also important to members of this community
(emphasizing the overlap in the interests of the various SSGs), and
those involved in survey work will continue to need good database
tools, providing efficient ways of accessing and analysing catalogue
data.

The final conclusion of this part of the discussion was that new data
processing techniques are likely to be adopted by workers in this
field, but that the requirements that can be foreseen at this stage
are likely to be met by the existing and the emerging technology.  It
was decided that the first steps in the new directions would best be
served by exploring data visualisation techniques.

\subsection{Priorities for the next 12 months}

We have split our recommendations for priorities for the next 12
months into general recommendations, and a detailed software
wish-list, which is a prioritised list of the specific action items
that emerged from discussions.  This year the wish-list has been
concatenated into a single list for both radio and mm/submm
astronomy. The committee feel that this is a relatively modest list
for such a wide community, and hope that all of these items will be
addressed by Starlink within the next twelve months, especially as
many of them concern software vital to new receivers. Starred items
are carried forward from last year.

\subsubsection{General Recommendations}

The following general recommendations emerged from the discussion;
these are general points concerned with the organisation of software
development and the SSGs.

\begin{description}
\item[Software management:]\mbox{}\\
Some concern was expressed at how the responsibility for managing some
of the main software items (SPECX and JCMTDR) has devolved.  In the
case of SPECX, new developments are done by the author (RP) as time
and other committments allow.  Preparation of the Starlink Unix
distribution has been done in the past by HM, based at ROE; but
maintenance of the `active' VMS version at the telescope, and of the
ports to non-supported platforms (including SunOS) is now done by Remo
Tilanus in Hawaii.  The committee would like to see clarification of
the arrangements for management of these important packages, with
Starlink taking responsibility for the coordination of effort, and
communication between the various different parties.

\item[Consultation and Communication:]\mbox{}\\
For the SSGs to function effectively, input must be canvassed from as
wide a user base as possible. Starlink should take the initiative by
advertising the meetings and the membership of committees, and by
issuing news items and posts to the uk.org.starlink.announce
newsgroup, with requests for input before each meeting. Similarly
Starlink should advertise the availability of the reports when they
are posted on the WWW after the meetings. Documents should be
available over the network in both HTML and compressed PostScript
formats. The idea of a newsgroup, or Forum conference, devoted to the
SSGs was discussed, but wasn't thought worthwhile unless interest is
shown by users.

\item[Documentation:]\mbox{}\\
The SSG supports the progress made by Starlink in preparing on-line
documentation for the Starlink Software Collection.  Cookbooks for
JCMTDR and SPECX (already available) and the SCUBA software (required
by early next year) should be included in the on-line documentation
set.  A UK mirror site for the AIPS Cookbook WWW pages should be set
up and advertised.

\item[Software support:]\mbox{}\\
Support for all currently available software should be continued. The
most important non-Starlink item to this SSG is AIPS. Support for the
installation and use of AIPS within the UK should be continued by
Starlink. Details of the installation guide and other material in
preparation by the support programmer should be distributed to all
AIPS sites when they are available. The position of John Palmer as
AIPS support person should be advertised. In addition, the watching
brief on AIPS++ should be maintained.

\end{description}

\subsubsection{Wish-list for the next 12 months}

\begin{enumerate}
\item The SPECX bug concerned with the use of HDS files under Unix must be
fixed as soon as possible. The corruption of open work files after an
untidy exit from the program is causing major problems.

\item The SPECX problem of slow access when opening files containing large
data cubes should be addressed. According to Starlink this is not a
difficult problem to overcome.

\item Modifications are still required to allow the importing of
general datasets into IRAS90. In addition the PREPARE application
should be modified to accept any NDF file.

\item Provision of a general two-dimensional, multiple component,
gaussian fitting routine is still required by JCMT users. The best
place to provide this may now be as an addition to the pisafit command
within the PISA package.

\item Standalone versions of SPECX and JCMTDR are needed as soon
as possible. In the case of JCMTDR this requires finding alternatives
to the use of a NAG routine.

\item Starlink should include a `request for user interest' with the
next Newsletter's item about the Caltech DIFMAP package, and if there
is sufficient response, distribute it to interested sites.

\item A Unix version of the VMS FLUXES program, which uses the RGO
ephemeris data, should be made available. This program is required by
all continuum users of JCMT, and is currently run by users remotely in
Hawaii.

\item The AIPS cookbook should be included as part of the on-line
documentation set.

\item The recent `upgrade' to the parameter handling system of KAPPA and
Figaro should be `fixed' to the level of user-friendliness in the
prompt defaults which existed previously.

\item A suitable procedure should be found for exporting data from
SPECX to commercial data visualisation packages. This would be the
first step towards the long-term future.

\item The amount of effort required for the conversion to Unix of the
program for performing maximum entropy deconvolution of IRAS data
should be explored. If this is a matter of only 1-2 weeks' work, then
it should be undertaken. This routine is still preferable to the IPAC
alternative, which is to use the MCM method of Aumann et al (1990),
and it was a facility lost in the move to Unix. The Chairman will
ascertain its likely usage.

\item The ability to specify pointing corrections within JCMTDR in
terms of RA, Dec as well as azimuth and elevation should be
provided. This will also be required by SCUBA.

\item Some small level of support for importing ISO data into Starlink
format may be required when ISO data starts to be available in late
1995 or early 1996.

\end{enumerate}
{\small *Item carried over from last year's list.}

\newpage
\section{\xlabel{x-ray_astronomy_ssg_input}X-RAY ASTRONOMY SSG INPUT}

\subsection{Membership}

\begin{description}
\item[Members:]\mbox{}\\
Andrew Norton (OU)\\
Anne Sansom (UCLAN)\\
Bob Vallance (Birmingham)\\
Coel Hellier (Keele)\\
Colin Barber (Leicester)\\
Hongchao Pan (Oxford)\\
Jon Mittaz (MSSL)\\
Michael Merrifield (Southampton)\\
Mike Watson (chair, Leicester)\\
Roderick Johnstone (Cambridge)

\item[Also Present:]\mbox{}\\
Rodney Warren-Smith (RAL)

\item[Meeting date:]18th August 1995

\item[Venue:]Department of Physics and Astronomy, Leicester University
\end{description}

\subsection{Overview}

For the last 5 years or more ROSAT has been the dominant project for a
large fraction of the UK X-ray Astronomy community. Science data
analysis provision for ROSAT in the UK has focused on the ASTERIX
package, supported primarily by Birmingham University but with some
Leicester University involvement. Initially ASTERIX was all that was
available in the UK for ROSAT data analysis, but the situation has
changed markedly over the last couple of years as "foreign" packages
such as IRAF/PROS and FTOOLS have matured and gained currency in
several UK institutions.

We are currently seeing the trend away from ASTERIX continuing and
interest in alternate packages seems certain to increase over the next
few years. There are currently three operational EUV/X-ray missions:
ROSAT, ASCA and EUVE. Only ROSAT has a direct UK involvement, and only
ROSAT data reduction is supported by ASTERIX, but there is substantial
UK interest in the other two missions, and hence in the data analysis
packages required to use the data (FTOOLS for ASCA and IRAF for EUVE).

The next year will see the launch of two new X-ray missions: XTE and
SAX.  Within the next 5 years these will be joined by Spectrum
X-gamma, AXAF, Astro-E and XMM. There is UK involvement in Jet-X on
Spectrum X-gamma, in AXAF, and in XMM, but only in the case of Jet-X
is the UK in a position to influence strongly the science data
analysis planning. FTOOLS seems certain to become the most important
package over the next 5 years since XTE, SAX, many of the Spectrum
X-gamma instruments, Astro-E and very possibly XMM will adopt an
FTOOLS-based approach. AXAF software is likely to adopt a hybrid
approach underpinned by IRAF/PROS developments but with some degree of
compatibility with FTOOLS.

Jet-X software developments are also adopting a hybrid approach, with
FTOOLS and ASTERIX both taking a role, and much of the software effort
devoted to ASTERIX at present is going into making the system fully
FITS-compatible. This will effectively convert ASTERIX into a set of
FTOOLS, and allow it to be used for the post-reduction analysis of
data from a wide variety of missions.

\subsection{Future Strategy}

Given this background it is clear that the most important software
developments for UK X-ray astronomy are likely to occur outside this
country, and in particular at NASA's GSFC. The general feeling of the
X-ray Astronomy SSG is that the UK cannot hope to compete in this area
and that there is little point in continuing to develop ASTERIX as the
'alternative' UK system with a full range of functionality, and
interfaces to all relevant future data sources. This does not, of
course, mean that ASTERIX should immediately abandoned. The UK
investment in ASTERIX has been large over the last decade and ASTERIX
itself has many features liked by users (some of which are absent or
poorly presented in other packages). The question for the UK X-ray
astronomy community is thus whether a cut-down version of ASTERIX
which would preserve (and in the future develop) the best parts of the
current system is a sensible option, and if so whether sufficent
resources can be found to make this viable. No clear consensus on this
issue emerged from the meeting of the X-ray Astronomy SSG, but the
majority of the members are not supportive of further substantial
development. The future of ASTERIX is, nevertheless, still being
actively examined at Birmingham. We plan another meeting of the X-ray
Astronomy SSG within the next 3 months to discuss this matter

My personal\footnote{Mike Watson} footnote to this is simply to note
that one can foresee an entirely different response from the UK
community if the present generous level of support for X-ray astronomy
software packages from agencies like NASA were to be drastically
reduced, but until this happens I believe that we are taking the only
sensible approach.

The role that Starlink can, or should, play for the X-ray astronomy
community over the coming years is thus less than clear, and the
formal requests we make to Starlink, which are of rather general
nature, reflect this uncertainty. In essence it was felt that the
X-ray community had too many uncertainties to make long term strategic
decisions about software: flexibility was the fundamental priority!

\subsection{Formal requests to Starlink}

The X-ray SSG makes two formal requests to Starlink:

\begin{enumerate}
\item Taking responsibility for the routine task of maintaining/installing
the GSFC FTOOLS/XSELECT software on various platforms. The precise way
this should work may require further discussion.

\item To investigate the provision of a software package to produce sky
maps/overlays of catalogued sources, possibly incorporating some of
the well-developed display features (e.g. incorporated in ROSAT
Standard Analysis Software. Important points to be considered include:

\begin{itemize}
\item which input catalogues should be supported (and in which formats)?
(should this include pixel data such as the digital sky survey, APM
data, NED extracts etc etc.)

\item specification of astrometry information

\item choice of display environment (several options supported?)
\end{itemize}

\end{enumerate}

The X-ray SSG also considered the value to the X-ray community of
various general software development areas. Its general views on these
are summarised here:

\begin{enumerate}
\item {\label{xraySSG:PC}\bf LINUX.} LINUX development activity was
not high on X-ray SSG's priorities.

\item {\label{xraySSG:cl}\bf TcL/Tk.} Further work in this area to
provide useful scripting tools was seen as being important.

\item {\label{xraySSG:GUI}\bf GUIs.} The X-ray SSG recognized the clear
need further to develop GUI interfaces to applications packages to
provide a greater degree of commonality in user interfaces. Such
activities are already underway for ASTERIX.

\end{enumerate}

\end{document}
