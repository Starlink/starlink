\documentstyle{article} 
\pagestyle{myheadings}

%------------------------------------------------------------------------------
\newcommand{\stardoccategory}  {Starlink General Paper}
\newcommand{\stardocinitials}  {SGP}
\newcommand{\stardocnumber}    {25.10}
\newcommand{\stardocauthors}   {J C Sherman, A V Roberts, C A Clayton, M D Lawden}
\newcommand{\stardocdate}      {19 September 1991}
\newcommand{\stardoctitle}     {Starlink Site Manager's Guide --- Major Nodes}
%------------------------------------------------------------------------------

\newcommand{\stardocname}{\stardocinitials /\stardocnumber}
\renewcommand{\_}{{\tt\char'137}}     % re-centres the underscore
\markright{\stardocname}
\setlength{\textwidth}{160mm}
\setlength{\textheight}{230mm}
\setlength{\topmargin}{-2mm}
\setlength{\oddsidemargin}{0mm}
\setlength{\evensidemargin}{0mm}
\setlength{\parindent}{0mm}
\setlength{\parskip}{\medskipamount}
\setlength{\unitlength}{1mm}

\begin{document}
\thispagestyle{empty}
SCIENCE \& ENGINEERING RESEARCH COUNCIL \hfill \stardocname\\
RUTHERFORD APPLETON LABORATORY\\
{\large\bf Starlink Project\\}
{\large\bf \stardoccategory\ \stardocnumber}
\begin{flushright}
\stardocauthors\\
\stardocdate
\end{flushright}
\vspace{-4mm}
\rule{\textwidth}{0.5mm}
\vspace{5mm}
\begin{center}
{\Large\bf \stardoctitle}
\end{center}
\vspace{5mm}

\large

\setlength{\parskip}{0mm}
\tableofcontents
\setlength{\parskip}{\medskipamount}
\markright{\stardocname}

\newpage

\section {Introduction}

This paper is intended primarily for the managers of Starlink's Major Nodes,
all but 2 of which (RAL and ROE) are supported by Starlink via Site Contracts or
(Cambridge) via a contract which is similar to a Site Contract.
Managers should make themselves familiar with the terms of the corresponding
contract --- copies are available from the University or from Starlink.

A companion paper (SGP/37) is available for managers of Minor Nodes.
You should already know which category your node falls in but, if not, refer to
ADMINDIR:WHOSWHO.

This paper will help you understand the Starlink Project and your role in it.
It assumes you are aware of the documentation vendors provide for managers
and operators of their computers, so it contains only the extra information
you need for Starlink purposes.
It contains many references to other Starlink documents; you need to become
familiar with these to carry out your job properly.
Also, you should visit the Project Starlink site at RAL to see how that site is
run and to get to know the Project staff.
You can propose such a visit at any time --- don't wait to be asked.

There are a number of people you can turn to for help.
These include the other site managers and Project staff with particular areas
of expertise.
Their names, addresses (network and postal) and functions are specified in
ADMINDIR:WHOSWHO.LIS.
In general, use the VMS MAIL utility (SUN/36) to communicate with people as
your message can then be dealt with at a convenient time and you have a printed
record.

\section {Site Manager's Responsibilities}

A site manager's responsibilities are many and varied.
The following list includes those duties which are common to all Major Nodes;
individual site managers may be asked to perform additional duties from time to
time.
\begin{itemize}
\item Implement Starlink policy.
This means both project-wide policy which is relevant to your site, and local
policy recommended by your Area Management Committee (AMC).
\item Liaise with Starlink central management, especially with regard to
hardware and software changes.
\item Liaise with field service organisations, such as DEC and other
maintainers of equipment and software, and deal with equipment faults.
\item Operate the computer and associated equipment to provide a satisfactory
day-to-day service to users, both on and off site.
This includes providing advice and assistance to users of Starlink application
packages, and answering questions on operating systems and layered products.
You should be the first person users turn to when they have problems, although
you may need assistance to answer some questions.
\item Install licensed software releases and the Starlink Software Collection.
The Starlink Software Changes for VMS (SSCs) and Unix Starlink Software Changes
(USSCs) issued by RAL should be implemented as soon as possible.
\item Encourage users who are developing software to do so according to Starlink
programming standards (SGP/16) and within the ADAM environment (SG/4, SUN/101).
\item Control the use of the node and ensure, via the VMS security system and
AUDITOR on VAXs, that only accredited users use the system.
Register new users.
\item Maintain usage and configuration records and up-to-date inventory lists.
\item Maintain on-line equipment inventories locally.
\item Check maintenance schedules maintained on-line at RLVAD.
\item Act as secretary to the Area Management Committee.
\item Keep the managers of any Minor Nodes in your Area informed of any
decisions made or news obtained at Site Manager's Meetings which may be
relevant to them.
\item Keep up-to-date with relevant conferences.
\item Help manage the finances of the node.
\item Maintain the on-line information summaries requested by the Starlink
Documentation Librarian.
\item Distribute documentation to users, and maintain a filing cabinet
containing copies of Starlink papers for users.
Distribute the Starlink Bulletin.
\item Keep appropriate non-users at your site informed about Starlink.
Give them things like the Starlink Bulletin and the Starlink User Guide.
\end{itemize}
In carrying out these duties you will sometimes need advice or information
from Project staff at RAL.
Concerning the allowed use of the node and registration of users, consult
Gordon Bromage.
For maintenance, inventories or the progress of orders for new hardware or
software, consult Andrea Roberts.
For site contract renewals, contract policy or hardware plans, contact John
Sherman.
For documentation, contact Mike Lawden.
For the Starlink Software Collection, contact Martin Bly.
For applications software development, consult Rodney Warren-Smith.
For graphics and networking, consult David Terrett.
For the ADAM software environment, consult Peter Allan.

\section {Administration}

The hardware configuration of your node should not be altered in any significant
way, even temporarily, without the approval of the Network Manager.
Furthermore, additions to your node which are proposed by users and which are
not funded by Starlink (e.g.\ those funded by SERC block grant or by ad hoc SERC
or UGC grants) must be discussed with yourself and with the Starlink Project
well before the proposal is finalised.
The advantages of early discussion with the Project are noted in SGP/39.
In brief they are:
\begin{itemize}
\item The feasibility of the proposal and its relation to Starlink's plans can
be reviewed.
\item Short-term and long-term maintenance can be discussed.
\item In the case of SERC grants, subsequent review by the Starlink User
Committee and/or Theory and Computation Panel will be simplified.
\end{itemize}
You must provide Starlink management with the following information in August
of each year:
\begin{enumerate}
\item An Annual Report, prepared in collaboration with your AMC.
This should contain contributions from any Minor Nodes in your Catchment Area.
\item Hardware Configuration Diagrams of your computer installation, including
Ethernet configuration, and a note of the proprietary software licensed for
each machine.
\item A Hardware Inventory of the equipment installed at your site in the
approved format.
(The inventory list must conform to the approved format since it will become
part of your site contract.)
\end{enumerate}
Examples of the sort of material required can be obtained from David Rawlinson.
Items 2 and 3 should also be sent once a month, even if there have been no
changes --- in this case only the revision dates on the documents you send need
be modified.
It is important that RAL has accurate and up-to-date records of the equipment
at your node or your node may be neglected when hardware upgrades are planned.

All site managers are entitled to reimbursement of their expenses when they
travel on Starlink business (e.g.\ to attend Site Managers meetings, Special
Interest Group meetings, or ad hoc meetings called by the Project Management).
The RGO, ROE and RAL managers should all claim through their local
establishment.
The managers of the University sites should submit a claim to the University,
consistent with the University's normal regulations.
Subsequently, the University can claim reimbursement from Starlink by including
the expenses on one of the quarterly site contract invoices.
These arrangements are noted in the site contract document, which also notes
that prior approval must have been obtained from Starlink before any claim
can be paid.
Prior approval has been given for all travel and subsistence expenses in
connection with several ``standard" purposes, this approval being in the
form of a letter from John Sherman dated 28 July 1989.
If the letter, with its list of ``standard" purposes, has been mislaid, please
ask for another copy.
Prior approval must still be obtained, case by case, for any travel and
subsistence expenses in connection with visits which are not in the ``standard"
list, or where there is some doubt.

Members of Starlink Remote User Groups are entitled to reimbursement of travel
expenses when they visit their nearest Major Node to use Starlink facilities.
They should complete RAL claim form N3(V) which you must check and initial
before forwarding it to Andrea Roberts.
RAL imposes strict rules about who can be paid travelling expenses, so it is
worth contacting the Project beforehand in dubious or unusually expensive cases
(e.g.\ where overnight accommodation will be required or where several trips
per day are claimed).

\section {Meetings}

There are four series of meetings which you must attend:
\begin{description}
\begin{description}
\item [SMM --- Site Managers' Meeting:]
This enables Major Node site managers and relevant Project staff to discuss
problems, report developments, and agree on common standards and procedures.
Meetings are held twice a year, once at RAL and once at one of the other
Starlink sites.
\item [AGM --- Annual General Meeting:]
This meeting is held once a year at RAL (Cosener's House).
It is similar to an SMM, but also includes contract Application Programmers and
all managers of Minor Nodes.
It is therefore considerably larger and lasts longer than SMMs.
\item [AMC --- Area Management Committee:]
Most Major Nodes are the centre of an AMC which must meet at least twice a year.
It advises the site manager on resource allocation, and informs him of long term
user needs.
Its membership includes the site manager (who should be secretary) and the most
active users.
One or more representatives of Starlink central management must attend each
meeting; meetings without representatives of Starlink central management can
be held, of course, but these are not formal AMC meetings.
AMCs are explained in more detail in SGP/8.  
\item [SLUG --- Starlink Local User Group:]
AMC meetings are preceded by SLUG meetings which are open to all users.
Minutes of both AMC and SLUG meetings should be taken and sent to the Project
Management at RAL.
\end{description}
\end{description}
The dates of all Starlink meetings are usually known well in advance so clashes
with other commitments can be avoided.
 
There are three other series of meetings which you need to know about:
\begin{description}
\begin{description}
\item [SUC --- Starlink Users' Committee:]
This is the SERC committee which oversees the Starlink Project.
It reports directly to the APS board and meets 3 times a year, usually in
London.
Its membership is listed in ADMINDIR:WHOSWHO.
The Project Management always attends.
\item [SIG --- Special Interest Groups:]
These are user groups which consider software development in specific
application areas.
The meetings tend to be held where most of the SIG's work is being done.
You may be asked to be a member of a SIG if you have special expertise,
or volunteer if you believe you can contribute.
The membership of current SIGs is listed in ADMINDIR:SIG.LIS; see also SGP/14.
When a SIG meeting is held at your site, attend as an observer even if you are
not a member.
\item [HAG --- Hardware Advisory Group:]
This group advises the Project on future hardware purchases.
Its membership is listed in ADMINDIR:HAG.LIS.
\end{description}
\end{description}

\section {Operating Standards}

There are a number of practical aspects to consider when setting up and
operating a Starlink computer, some of which are described below.
The initial setting up will be done by an experienced manager, so this
section does not describe the set up procedure in detail.
Make the most of Site Managers' Meetings to find out what ingenious but
doubtless undocumented procedures are used at other sites.

Advice on setting up a Starlink SUN-based System running Unix can be found in
SSN/7.
Some notes for VAX-based systems are included below.

\subsection {Disk assignments}

Select one of your disk drives to be the System disk.
Put your users on one or more User disks with labels USER1, USER2, etc.
User disks should be under quota control, but the System disk normally should
not.
If quotas have to be used on the System disk, it is essential that the [SYSTEM]
account has ample space.
Usually, quotas on the System disk can be avoided by making the User disks and
the System disk different devices.
 
Disk space is always at a premium, so start your users off with a small quota
(say 2000 blocks, equivalent to 1 Mbyte) and give more only to those who really
need it.
If users request more space temporarily, remember to reclaim it at the
earliest opportunity.
Encourage users to minimise their usage of disk space and make sure they
understand the use of PURGE, BACKUP and SET FILE/TRUNCATE.
De-fragment your disk space on a regular basis using the RABBIT7 software
supplied for this purpose.

Most sites have one or more SCRATCH disks which are always mounted, the
algorithm used to delete old files from these disks varying from site to site.
Alan Lotts, the manager at Durham, can supply examples of the algorithms in use.

Starlink software and data can be distributed over many disks and these can be
chosen as you wish, with the restriction that directories [STARLOCAL] and
[STARLHOLD] must be stored on the same disk.
The structure of Starlink software is described in section 8.
The disk storing the core directory [STARLINK] is referred to
by the logical name STARDISK, and the disk storing the local software
(directories [STARLOCAL] and [STARLHOLD]) is referred to by the logical name
LSTARDISK.
These two logical names should be assigned to disks of your choice,
or sometimes to the same disk, in SYS\$MANAGER:SYSTARTUP.COM.
Other software and data can be stored on any disk or disks.
As explained in section 8, there is no need to store optional software items
that are not used at your site.
You can copy missing items over the network from RAL when required.
 
Logical names for standard software are defined in [STARLINK]STARTUP.COM, while
logical names for local and optional software and data should be defined by
you in [STARLOCAL]STARTUP.COM.

\subsection {Disk backup}

The following recommendations for disk backup are based on the responses to a
questionnaire which was sent to all site managers, most of whom have many years
experience of backup procedures.
The responses demonstrated good agreement among most site managers.
The recommendations should not be treated as rigid rules to be applied at every
site --- there is room for some local variation.
Nevertheless, any site which deviates substantially from them should consider
why everyone else is doing something different.
 
The purpose of backup is to protect your computer against accidental file
deletion, the most common causes of accidental deletion being user error and
disk failure, in that order.
Since the whole purpose of backup is to provide security, the approach adopted
by experienced managers, and in the following recommendations, is to avoid
unnecessary risk.
 
\begin{itemize}
\item Full backups should be run every 4 to 6 weeks.
\item Normal use of the system should be prohibited during full backups
and the /VERIFY switch should be used.
\item Where possible, full backups should be made out of prime time, for
example early morning, evening or weekend.
\item Incremental backups should be run on user disks at least once a week
(most sites run incrementals once a day).
The frequency of incrementals depends on local factors such as the number and
type of tape drives available.
Incremental backups of the system disk need not be done as frequently as for
user disks.
\item High capacity cartridge drives, e.g.\ Exabyte or TK70, should be used to
enable backups to be run overnight where possible, and thus avoid using the
system in prime time.
\item Normal use of the system should be allowed during incremental runs
which should not use the /VERIFY switch.
\item Don't eliminate too much from the backup.
The wrong choice here will not save much time, but will cause a large amount of
extra work during any restore.
Files which could be excluded are catalogues which are available on the
Database machine at RAL (but do you really need them locally in the first
place?).
\item Consider storing your backup media away from the machine room to reduce
the possibility of both the online data and the backups being destroyed by the
same disaster.
\item Be very careful when you brief a stand-in to act as manager in your
absence.
Stand-ins are a major cause of damage to system files.
Please ask the stand-in to call RAL or another site manager if he or she gets
stuck.
A stand-in who presses on with his or her fingers crossed when things don't
go quite as the manager had led him or her to expect can do a surprising
amount of damage, especially when the stand-in is armed with numerous
privileges.
\item If you have not yet retrieved files from full or incremental backups,
use a scratch disk to make an end-to-end test of your procedures.
\end{itemize}
Command procedures for backup are available from Peter Allan, RAL, or from
Alan Lotts, Durham.

\subsection {Special usernames}

A number of special usernames, such as FIELD and NETUSR, will be created during
the initial setting up of a Starlink site.
In this section we mention those that are of particular relevance to Starlink.

OPER is used when acting as site manager; ask your users to send any operational
MAIL, e.g.\ requests to mount a disk, to OPER.
All Starlink Software Changes (SSCs) are sent to OPER from RAL.
You can contact other site managers by sending MAIL messages to OPER at their
nodes and they will address MAIL for you to OPER.
OPER is the primary way of communicating between sites and should be monitored
regularly; a deputy must be instructed to do this if you are away, even for
a day.

STAR is used for maintaining Starlink software.
Give it the same privileges as OPER and UIC=\-[STARLINK,\-STAR].
Set its default protection to (RE,RWED,RE,RE) and its default directory to
[STAR].
Create directory [STAR.TEMP] for receiving and implementing SSCs.

SYSTEM is used for VMS work.

You will probably also have a personal username.
Try to use either OPER or your personal username when sending MAIL messages
since the use of other usernames can cause confusion over which reply address
to use.
 
Every manager of a Major Node is permitted a username on the Project node at
RAL.
The username should be the initials of the manager and should be changed when
a new person takes over.
The password should never be divulged to users.
Managers also have usernames on the database machine, STADAT; for more detail,
see SUN/30.

\subsection {Use of other nodes}

Users are not normally expected to have usernames at more than one node.
DECnet allows access to files on other nodes, the NETSHOW command gives
information about other nodes, and TALK can be used to another node (SUN/36).
Together with MAIL, these facilities allow most legitimate use of another node
without the need for a username there.
 
You should install a captive account with the username STARLINK to provide
access to the network for visiting users from other nodes.
For more details, see Appendix A.

\subsection {User authorisation file}

Each accredited Starlink application form will lead to the generation of a new
username which implies a new entry in the UAF.
The following standards should be applied to UAF fields.
\begin{description}
\begin{description}
\item [USERNAME:] The initials of the user (e.g.\ MDL).
\item [OWNER:] The name of the user (e.g.\ MIKE LAWDEN).
\item [ACCOUNT:] Assign the 8 characters of this field as follows:
\begin{itemize}
\item 1-3 defines the user's home location.
Use the 3-letter codes given in Appendix B if the user is based at a Starlink
site, or invent your own codes for users based elsewhere.
\item 4 defines your site; use the 1-letter codes given in Appendix B.
\item 5-8 defines the sequence number of the application form.
\item An example account number is:
\begin{quote}
CARR0020
\end{quote}
which is interpreted as:
\begin{description}
\begin{description}
\item [CAR:] Cardiff (home base)
\item [R:] RAL (Starlink site)
\item [0020:] 20th application form for RAL
\end{description}
\end{description}
\end{itemize}
\end{description}
\end{description}

\subsection {Passwords}

To protect a site from hackers, all passwords should be difficult to guess
(passwords found in a dictionary should not be used since they can ``guessed''
by hacking software) and should be changed periodically.
The defaults provided by VMS should be used: 6 characters changed at
6 month intervals for non-privileged users, and 8 characters changed at 1 month
intervals for privileged users (like OPER).

\subsection {Process names}

Process names can be set by the command:
\begin{quote}
{\tt \$ SET PROCESS/NAME="xxx"}
\end{quote}
If none is set, VMS uses the username.
Encourage your users to set their process name to their personal name in their
LOGIN.COM file since this enables the owner of a particular process to be
identified.
Process names which are offensive or silly are not allowed; stop such processes
on sight.

\subsection {Magnetic tapes}

Starlink recommends the following standard for labelling magnetic tapes.
The label should be of the form:
\begin{quote}
SSSnnn
\end{quote}
where: `SSS' is the 3-letter site code (see Appendix B) and
`nnn' is a sequence number (000\ldots999).

\subsection {Games}

Computer games are not allowed.
Any node where games are played will be regarded by the Project Management as
undermining the Project's efforts to obtain and retain funding for Starlink and
the blatant waste of resources will be taken into account when considering
upgrades!

\subsection {Maintenance schedules}

It is the responsibility of site managers to make sure that their Starlink
maintained equipment is on an appropriate maintenance contract.

The maintenance schedules that are administered through RAL are stored 
online at RLVAD, and are available to all managers.
The files are kept in LSTARDISK:\-[STARLOCAL.\-ADMIN.\-MAINTENANCE.\-decnetname]
for each site, and have obvious names, such as DEC\_HW.LIS for DEC hardware.
These files should be checked on a regular basis, especially if any schedule
amendments have been requested for your site. 

The schedule lists stored online are exact copies of the details that each 
maintenance company has for each site.
{\em If it is not online, we do not have any record of it for maintenance
purposes.}
For amendments, additions or deletions required by your site, please send full
details of the changes to Andrea Roberts on RLVAD::AVR, stating schedule numbers
where known.
A quick way to find your schedule numbers is to look in 
LSTARDISK:[STARLOCAL.ADMIN.MAINTENANCE.SCHEDULES] where you will find 
listings for all sites.
The amendments will not be put online until confirmation of change has been
received from the maintenance company, but there is a directory called
AMENDMENTS.DIR in LSTARDISK:[STARLOCAL.ADMIN.MAINTENANCE] that will hold the
latest amendment requests.

It is vital that these maintenance schedules are kept up to date.
Failure to do so will cause substantial delays (months, not weeks) in repairing
faults.

\section {VAX Notes}

All site managers are required to read and contribute to the following 
VAXnotes conferences (SUN/44, SSN/51) on a regular basis:
\begin{description}
\begin{description}

\item [RLVAD::SYSTEMS\_MANAGEMENT] ---
This is {\bf the} most important conference for site managers.
It contains invaluable advice and instructions from other managers and Project
staff which is not documented anywhere else (e.g.\ the availability of new
versions of system software).
All managers should read this conference at least once a week, preferably daily.

\item [RLVAD::NEWS] ---
Managers should post to this conference any NEWS items they wish to broadcast
via the Starlink-wide NEWS system (SUN/51).
Similarly, managers should review this conference daily, taking any notes
posted by other managers and inserting them in their local NEWS system.

\item [CAVAD::FAULTS] ---
{\bf All} hardware faults (however small or uninteresting) should be logged in
this conference in order to help identify systematic problems and design faults.

\item [RLVAD::NETWORKS] ---
New Starlink DECnet addresses and general network information are posted here.
When new DECnet nodes are added at a site, the network manager for the relevant 
DECnet area (normally the manager of the nearest Major Node) should post a new
list of DECnet addresses for that area in this conferences.
Other managers should then use that announcement to update their own DECnet
databases.
See the NETWORKS conference itself for more information.
 
\item [RLVAD::SSC\_RELEASE] ---
SSC release announcements are posted here.
See Appendix C for details of how to implement software changes.

\item [RLVAD::USSC\_RELEASE] ---
USSC release announcements are posted here.

\item [RLVAD::BUG\_REPORTS] ---
All bug reports on Starlink software are posted here.

\item [RLVAD::CONFERENCES] ---
This should be consulted regularly to alert managers to any new conferences
in which they should participate.
Other conferences that may be of interest to managers include VAXSTATIONS,
DECSTATIONS, HARDWARE and UNIX\_MANAGEMENT.
See the CONFERENCES conference for more details.

\end{description}
\end{description}

Some sites set up local conferences of a more general and chatty nature, e.g.\
a GOSSIP conference.
These are OK provided they do not get out of hand.
Here are some guidelines for conference management:
\begin{itemize}
\item There is a recognised set of strictly Starlink-related and Starlink-wide
conferences which Starlink organises and is responsible for.
They include the ones listed above.
\item When the user community at a site or in an area wishes to set up a
conference devoted to another topic, Starlink will have no objections {\em as
long as the AMC agrees to accept responsibility for the conference.}
AMCs may wish to restrict a given conference to local users, or some other
subset, but are at liberty to create conferences accessible to the whole of
Starlink if they believe this is appropriate.
\item Each conference will be operated by a single named individual who will
be a moderator of the conference, but who may wish to delegate some of the
work to one or more other moderators.
\item It is recommended that moderators apply commonsense restrictions on
style and content, such as discouraging discussion of religion, politics or sex.
They should take into account both the resources used (such as disk space and
network traffic) and the impression of the node concerned and of Starlink in
general that might be given to outsiders.
Serious topics of no direct relevance to Starlink activities are probably an
unwise use of the VAX Notes facility, whatever the level of resources.
Light-hearted contributions are not discouraged, but should avoid excessive
length.
\item Starlink will apply its own guidelines to the `official' conferences,
and also will recommend the AMCs to apply them to their conferences, but
without insisting they do so.
\end{itemize}

\section {User Registration}

Anyone who wants to use Starlink facilities must complete the form `APPLICATION
FOR USE OF STARLINK' (obtainable from the Starlink Documentation Librarian) and
return it to the manager of the appropriate site.
Each form should cover one person and one username only.
Shared usernames should be avoided unless absolutely essential as they are a
security risk.
If you have them, make sure that their use is properly controlled.
 
In deciding on appropriate usage of your node, be guided by the following
priorities (highest at the top):
\begin{enumerate}
\item Astronomical data reduction that can only be done interactively.
\item Other astronomical data reduction.
\item Other astronomical research work that can only be done interactively.
\item Other astronomical research work.
\item Other APSB research work.
\end{enumerate}
If there is any doubt whether or not work falls within Starlink's terms of
reference, consult the Project Scientist, Gordon Bromage.
A difficult question is where to draw the dividing line between, for example,
routine telemetry processing associated with the operation of a particular
satellite or instrument, which is clearly not Starlink work, and analysis of
calibrated scientific data, which clearly is.

Site managers can provisionally accredit new users themselves, provided they 
are reasonably sure the work to be done is high on the preceding priority list.
Dubious cases should be referred at once to the Project Scientist who reviews
all applications in due course.
The Project Scientist may not allow an application if the work does not meet
Starlink criteria and has asked for usernames to be withdrawn when they have
been granted incorrectly.
Copies of all completed application forms should be sent to the Documentation
Librarian at RAL.

As a user's work evolves, his or her reason for using Starlink, which was
legitimate initially, may become less legitimate.
If the work has changed substantially, ask the user to fill in a new
Application Form and forward a copy to RAL.
If it is not clear that the new work is suitable for Starlink, refer the case
to the Project Scientist.
 
New users of your site who are also new to Starlink should be given a `New User
Documentation Pack'.
(If they have previously used other sites, just give them any local
documentation you think appropriate.)
Its contents are at your discretion, but should include the Starlink User Guide
(SUG), enclosed in the specially made loose leaf binder, and the DEC
manual `Introduction to VAX/VMS'.
Supplies can be obtained from the Documentation Librarian, who can also supply
an example of the `New User Documentation Pack' provided at RAL.

In the rare cases where it is essential (see section 5.4), a user may have a
username on more than one node.
Such a user should specify a `Primary Node', responsible for supplying him
or her with Starlink documentation.
His or her other nodes are termed `Secondary Nodes'.
The distinction between Primary and Secondary nodes helps to avoid double
counting of users.

Every six months, Starlink issues a newsletter called the Starlink Bulletin.
Copies are sent to every site and you must distribute a copy to all
your primary users, and to any non-users who you think may be interested.
(Your secondary users will receive a copy via their primary node.)
The Bulletin is the primary mechanism by which Starlink informs its users of
Project status, policies and plans and it costs a lot of money, so please
distribute it promptly.

Apart from the New User Documentation Pack and the Starlink Bulletin, document
distribution to users is at your discretion.
RAL sends every Starlink site a small number of paper copies of every
project-wide document.
You should maintain a central filing cabinet of these documents containing
copies available for users to take away.
This is particularly important for documents containing diagrams and for the
many documents prepared using \LaTeX.
The latter are awkward to read from normal terminals, and expensive to reproduce
from the \LaTeX\ source.
 
When a user leaves permanently (say for more than 3 months), remove his or her
username and files from the system and delete references to him or her in the
on-line information summaries.
This should be done within one month of departure.

\section {Starlink Software}

Many different aspects of the Starlink software are described in the Starlink
documentation.
The following documents provide general background to Starlink software
development:
\begin{description}
\begin{description}
\item [SGP/4:] Starlink C programming standards
\item [SGP/13:] Starlink applications software development
\item [SGP/16:] Starlink applications programming standards
\item [SGP/21:] Starlink software distribution policy
\item [SUN/1:] The Starlink software collection
\end{description}
\end{description}
The following documents deal with specific details of the management of the
Starlink Software Collection:
\begin{description}
\begin{description}
\item [SGP/19:] Starlink software submission
\item [SGP/20:] Starlink software management
\item [SSN/15:] Starlink software installation
\item [SSN/41:] Starlink software changes
\end{description}
\end{description}
You should read all these papers to gain a proper appreciation of Starlink
software and how it is managed.
Some of the most crucial concepts and facts are summarised in section 7 of the
Starlink User Guide.
 
The top level structure of Starlink software is:
\begin{description}
\begin{description}
\item [Standard set:] comprises [STARLINK], ADAM, FIGARO, and KAPPA, and must
be installed at every Starlink site.
\item [Secure set:] comprises [STARSEC].
This holds software that could represent a security risk and which should not
be given away with the Standard set.
The items in the set are optional.
Secure items are identified by the key `*S*' in the Starlink Software Index.
\item [Option set:] comprises optional items installed outside the [STARLINK]
directory.
If you have difficulty storing all the Starlink software at your site, you may
delete (or decide not to install) optional items which are not needed by your
users, e.g.\ ASTERIX.
If you get a request for an optional item which you have not installed at your
site, you can copy the latest version from RAL by logging in to RLVAD, making a
save set of the appropriate directories, and copying this over the network.
You will also need to puzzle out which symbols and logical names need to be
defined before the software will run.
Optional items are identified by the key `*O*' in the Starlink Software Index.
\item [Restricted set:] comprises items which are only installed on STADAT
(Starlink's central data and software facility at RAL).
Restricted items are identified by the key `*R*' in the Starlink Software Index.
\item [Local set:] comprises [STARLOCAL] and [STARLHOLD].
Their contents are site dependent.
\item [Data:] comprises such things as star catalogues.
\end{description}
\end{description}
Proprietary (licensed) software can be found in the Option set, and the
Restricted set.
Clearly, such software must NOT be given away.
SGP/21 contains a complete list of proprietary software; an online list is in
ADMINDIR:SSI.LIS.

A Starlink objective is that every Starlink site should have an
up-to-date copy of the required parts of the Starlink Software Collection
installed on its system.
SSN/15 tells you how to install the Collection from scratch.
Once this has been done, you will get a series of Starlink Software Change
(SSC) notices which tell you how to update your copy of the Collection (SGP/20,
SSN/41).
These are sent to an appropriate username at your site (usually OPER).
Print them out, then file and implement them as quickly as possible in order to
keep your site in step with the other Starlink sites. 
The technique for implementing them is described in Appendix C.
You should copy the updates within a month of release since they are only kept
on-line for that length of time.
We do archive the SSC save sets for sending to overseas sites (AAO etc).
In dire emergencies SSC save sets may be recovered for you to copy, but there
may well be a long delay before this is done.

A complete set of SSCs is stored in ADMINDIR:SSCLIS.TLB and SSCCOM.TLB.
Each release document is stored as a separate module in these libraries, with
names like SSCnnn where `nnn' is the number of the release.
This is a useful source of information about specific software items.
The Starlink Software Index (ADMINDIR:SSI.LIS) identifies all the SSCs which
have affected each item, so if you are having trouble with an item, look
through its associated SSCs and you may find the answer to your problem.

You must not distribute any part of the Collection to anyone else without
permission from Starlink management (SGP/21).
Contact the Software Librarian, Martin Bly, in the first instance.
In particular, proprietary software such as the NAG libraries must not be
distributed.
If you do receive permission to distribute Starlink software to another site,
give the Librarian the following details about the site:
\begin{itemize}
\item The full postal address, network address, telephone and telex number.
\item What you sent, when you sent it, tape type (and, if half-inch, the
density used).
\item The name of a contact at the site.
\item What support you intend to give. (Will you answer queries from users?
Will you supply updates?)
\end{itemize}
GKS may be distributed to non-profit-making astronomical research centres
for use in connection with other Starlink software.
This involves a registration procedure and must be done by the Project.
 
If you, or a user, have trouble with a software item, send details to the
Software Librarian (RLVAD::STAR).
Do not just make a local fix and forget about it; inform the Librarian and
get the correction properly distributed through the official release mechanism.
When reporting bugs, please include sufficient detail to make the bug
repeatable --- a bug which cannot be repeated cannot, in general, be fixed.
Details of reported bugs can be found in the conference RLVAD::BUG\_REPORTS.

Don't let your version of Starlink software become chronically different from
RAL's.
This has caused major problems in the past.
Software developed on another node has been submitted for release and did not
work at RAL because it relied on some local modification.
Also, software which was released by RAL sometimes didn't work at another
site because of local peculiarities.
Be warned: if you allow your site to get into this kind of mess you may
find that new or improved software doesn't run at your site.

\section {On-line Information Summaries}

The Starlink User Guide explains how information is organised in Starlink and
how it may be accessed.
This section describes your role in providing the part of this information
called `On-line Information Summaries'.
It does not consider those summaries distributed with the Starlink Software
Collection as these are maintained by RAL.

Starlink never has as many resources as it would like, so it is important that
the resources it does have are managed efficiently.
This can only be done if up-to-date, accurate and consistent information is
kept on users, documentation, software and Project management.
Such information changes quite rapidly.
Furthermore, it is often the most recent changes that are of most interest.
One way to provide this information is to maintain a set of on-line text
files containing information summaries.
Information in these files should never be allowed to get more than a day or
two out of date.
They are referred to many times in Starlink documentation, particularly the
Starlink User Guide, so users will expect them to be up-to-date.
If a person discovers that information in one of these files is wrong or
missing, he will rapidly lose faith in the file and will not use it.
Conversely, if he finds it is up-to-date, accurate and consistent, he will
regard it as a valuable reference.
Thus, the effort required to set up and maintain these files is only
worthwhile if the job is done properly.

RAL maintains a large number of these information summaries; they are listed in
RLVAD::\-LDOCS\-DIR:\-ONLINE.\-LIS.
Every one of them has arisen in response to a specific need.
They incorporate many years of practical experience during which the following
principles have been found to be useful:
\begin{itemize}
\item The information density should be as {\em high} as possible and
(provided the
requirements of the file are met) the amount of information stored should as
{\em low} as possible.
This reduces bulk and speeds searching.
\item The format should be consistent --- similar types of information should be
stored in the same format.
\item Each file should have a standard heading format which should include the
current date on the left and the file name on the right.
The current date should be the date it was last edited, not the date you printed
it out.
\item They should be easy to display and print.
The maximum line length should be 80 characters.
This just fills the screen on most terminals and a printed listing can be
trimmed to A4 size.
\item They should be simple, quick and easy to change.
This means using text files produced by a text editor.
Do not use text processors such as Runoff or \LaTeX.
Do not use the underscore character or tabs.
\item Anticipate your reader's needs and do as much work for him as possible
(don't set exercises) --- e.g.\ put names in alphabetical order, align phone
numbers in columns, count the number of things in different categories, etc.
\item Similar files at different sites should have the same format and type of
content.
This helps people find information at other sites because they are familiar with
how it is stored.
\end{itemize}
Some of these principles are conflicting, but the Project believes the formats
and contents recommended below are a good practical compromise.
The formats are not always specified in detail below, so we suggest you model
your own files on the equivalent RAL files.
The major effort needed is in the initial setting up.
Once this has been done, maintenance should take only a few percent of your
time.

On-line information on people is subject in the UK to the Data Protection Act.
This demands that such information be registered and places restrictions on its
use.
Starlink's files are believed to be covered {\em at SERC sites only}\, by the
general SERC registration, and we have given full details of the relevant files
(see RLVAD::LADMINDIR:DPA.LIS) to the appropriate SERC contact.
However, this does not cover Universities.
The manager of every Starlink site at a University should, therefore, inform
the Data Protection officer at that site of the existence and contents of any
files held on his Starlink computer(s) which contain information about specific
people.
Your DP officer will probably have a form for you to fill in.
File RLVAD::LADMINDIR:DPA.LIS contains the details that I submitted to our DP
officer at RAL with respect to our Starlink computers.
This may help you prepare your own submission.
You may also have local files that you should add to the list.
Users of Starlink facilities are also responsible for registering their
own files containing information on other people with their DP registrar.
However, I do not believe this involves the Site Manager.
Any queries to do with the Data Protection Act should be sent to Mike Lawden.

\subsection {Mandatory files}

The following files are {\em mandatory} at your site:
\begin{description}
\begin{description}
\item [LADMINDIR:] \hfill
\begin{description}
\item [USERNAMES.LIS] \hfill
\item [USERS.LIS] \hfill
\item [AMC.LIS] \hfill
\item [INVENTORY.LIS] \hfill
\end{description}
\end{description}
\end{description}
{\em The} {\bf USERNAMES} {\em file is the most important single information
file that you maintain.}
It is used every month by RAL to compile a list of all current Starlink users
which is released to all sites as ADMINDIR:USERNAMES.\-LIS.
It makes your users `visible' to the rest of the Starlink community, and enables
RAL to keep accurate records of the Starlink user population.
It is the primary source of information for people who wish to know usernames
for the purpose of sending mail messages.
The format of your file must be identical to the RAL file and you must not use
tabs.
The format required is as follows:
\begin{itemize}
\item Column 1: User code:
\begin{description}
\begin{description}
\item [r] --- UK-resident research astronomer actively processing data
 (includes post-graduate students).
\item [t] --- UK-resident scientific and technical staff supporting research
 astronomy (includes site managers!).
\item [o] --- Other UK-resident users (e.g.\ secretaries and e-mail-only users).
\item [f] --- Non-resident users (abroad for at least half the current academic
 year).
\end{description}
\end{description}
You should only specify a code for your primary users.
\item Column 2: Type an `*' if the user is a secondary user.
Thus, if column 1 contains a code, column 2 should be blank, and if column 2
contains an `*', column 1 should be blank.
\item Columns 3-26: Username(s).
If a user uses more than one username, put the principal one first (this will
most likely be used for e-mail) and separate the usernames with a `/' character.
For example: UNAME1/UNAME2/UNAME3.
\item Columns 27-49: Surname.
\item Columns 50-80: First name or familiar name.
\end{itemize}
Your users are not properly registered until their names appear in this file.
It is most important that you give high priority to keeping this file accurate
and up-to-date.
When a user leaves, be sure to remove his or her name from this file.
Every month, Mike Lawden merges all the local USERNAMES files to produce a
new version of ADMINDIR:\-USERNAMES.\-LIS.
This usually throws up a few inconsistencies between different site's files.
A `Username queries' message is sent to the sites involved to try to resolve
differences.
Please read and act on the instructions or requests in this message promptly.

The {\bf USERS} file records information about the registered users who
currently have access to your computer.
The information required is:
\begin{description}
\begin{description}
\item [Form number:] Use the serial number of the application form.
\item [Account:] see section 5.5.
\item [Username]
\item [Name:] This should include the name by which he or she is familiarly
known, and the title (Prof, Dr, Ms, Rev, Lord, \ldots).
\item [Work Address:] this should be as concise as possible, e.g.\ `RAL',
`MSSL', `Oxford U - Th Phys'.
\item [Work phone number:] including extension.
\item [Usage:] This should say, very briefly, what a user is doing on the
computer.
`IUE data analysis' is about the level required.
\item [Registration date:] this is the date when the user was first registered
for this username.
Use the code `xy' where `x' is the year 198x or 199x and `y' is the month number
expressed as a hexadecimal digit; thus `01' represents January 1990 and `5C'
represents December 1985 or December 1995.
The serial number should resolve any ambiguity between the 80s and 90s.
\item [Revision date:] this is the date when the information on this user was
last amended.
If the information has not been revised, leave blank.
\item [Termination date:] This is the date on which a user is expected to stop
using the computer.
Most users won't know this, in which case specify `Indef'.
\end{description}
\end{description}
The RAL approach is to divide the file into the following sections:
\begin{enumerate}
\item Registered users: Summary
\item Registered users: Detailed information
\item Off site users.
\end{enumerate}
The `Summary' section serves as an index to the `Detailed information' section
and may be all that is required for many purposes.
Each entry in the `Detailed information' section should be checked at least once
a year by contacting the person directly.
This should be done a year after the `Revision date', unless the `Termination
date' comes first in which case the check should be done then.
This way, the monitoring process is spread over the year and does not become an
intolerable burden; you just check a few users every month.

The {\bf AMC} file lists the current members of your Area Management Committee.

The {\bf INVENTORY} file specifies the hardware components that exist at your
site.
It must be in a format which will allow it to become part of your site contract.

\subsection {Recommended files}

The following files are {\em recommended} at your site:
\begin{description}
\begin{description}
\item [LADMINDIR:] \hfill
\begin{description}
\item [PEOPLE.ADR] \hfill
\item [SSI.LIS] \hfill
\item [SURGERY.LIS] \hfill
\item [CHARTS.DIR] \hfill
\item [NUDP.LIS] \hfill
\item [DPA.LIS] \hfill
\end{description}
\item [LDOCSDIR:] \hfill
\begin{description}
\item [DOCS.LIS] \hfill
\item [BUGS.LIS] \hfill
\end{description}
\end{description}
\end{description}
The {\bf PEOPLE} file stores the postal addresses of people you may wish to contact.
You may use this for any purpose you wish, but the following groups of people
should be on the list:
\begin{itemize}
\item Active registered users of your site.
\item Non-users to whom you wish to send information about Starlink.
\end{itemize}
Do not use more than 40 characters per line, then the address can be printed
on a sticky label.
Program COMSAL can be used to prepare a compressed listing of this file (LSN/31
PRO).
Program DISLIS can be used to print names on selected address lists (LSN/12
PRO).
Program STICKY can be used to print addresses on sticky labels for selected
address lists (LSN/28 PRO).
These local programs can be adapted to your requirements.

The {\bf SSI} file specifies details of local software that you think should be
generally known at your site.
The format should be modelled on ADMINDIR:SSI.LIS, leaving out those parts
dealing with software releases.

The {\bf SURGERY} file records any modifications you have made at your node to
the standard Starlink Software Collection.
For instance, you may decide to save space by deleting some source code or some
items.

The {\bf CHARTS} directory contains summaries of usage of the machine.
The recommended procedure for maintaining these files is documented in LSN/29
(PRO).
They are a vast improvement on the indigestible accounts listing.

The {\bf NUDP} file specifies the documentation that should be given to a new
user on initial registration.

The {\bf DPA} file specifies the names and contents of the files that contain
personal information and are therefore governed by the Data Protection Act (UK).

The {\bf DOCS} file lists every current local Starlink document which has been
released and is still valid at your site, in order of type and serial number.

The {\bf BUGS} file documents known bugs in local software.

\section {References}

\begin{description}
\begin{description}
\item [LSN/12 (PRO)]: DISLIS --- Produce distribution lists from an address list
\item [LSN/28 (PRO)]: STICKY --- Produce address labels from an address list
\item [LSN/29 (PRO)]: ACCOUNTS --- A machine accounting system
\item [LSN/31 (PRO)]: COMSAL --- Compress an address list
\item [SG/4]: ADAM ---- The Starlink software environment
\item [SGP/4]: Starlink C programming standard
\item [SGP/8]: Starlink local management
\item [SGP/13]: Starlink applications software development
\item [SGP/14]: Starlink special interest groups
\item [SGP/16]: Starlink applications programming standards
\item [SGP/19]: Starlink software submission
\item [SGP/20]: Starlink software management
\item [SGP/21]: Starlink software distribution policy
\item [SGP/37]: Starlink site manager's guide --- minor nodes
\item [SGP/39]: Upgrades to Starlink hardware and software
\item [SSN/7]:  Administering Starlink Sun systems
\item [SSN/15]: Starlink software installation
\item [SSN/41]: Starlink software changes
\item [SSN/51]: VAXnotes
\item [SUG]: Starlink User Guide
\item [SUN/1]: Starlink software --- An introduction
\item [SUN/30]: Using the Starlink database MicroVAX
\item [SUN/36]: Starlink networking
\item [SUN/44]: VAXnotes --- An introduction
\item [SUN/51]: Submission of NEWS items
\item [SUN/101]: Introduction to ADAM programming
\item [DEC]: Introduction to VMS
\end{description}
\end{description}

\appendix

\section {Captive Account For Network Access}

\subsection {Introduction}

Astronomers travel around a great deal but like to keep in touch with their home
institutions.
Often this means that they want to be able to login to a Starlink VAX where they
happen to be visiting and use a network connection (either DECnet or JANET) to
access their `home' machine to read MAIL, contact their colleagues, etc.
Most users do not have accounts on more than one Starlink machine and hence find
that they cannot login when they happen to be at a remote site.
In the past they would ask the site manager to allow them use of a `visitor'
account, but this is often not possible at odd hours of the day or weekends.
In addition, the use of such accounts needs careful management if it is not to
become a security risk.
This appendix (based on LSN/2 (UCL) by R N Hook) describes a simple solution
to this problem involving the use of a captive account.
It should be installed at all Starlink sites.

\subsection {Setting up the captive account}

The aim is to set up an identical secure account at each node which will allow
access to other machines only, i.e.\ deny access to DCL locally.
Making an account `captive' is a convenient way of limiting what a user can do
in this manner.
Details of the procedure for setting up such a secure account are given in the
`Guide to VAX/VMS System Security' Section 5.8.1.
The main thing to be avoided is the possibility of the user breaking out of the
captive procedure and returning to DCL.
The following listing of the UAF entry for the STARLINK username at UCL shows
one way such an account can be configured:

\normalsize

\begin{verbatim}
Username: STARLINK                         Owner:  NETWORK ACCESS ACCOUNT
Account:  UCLLNET                          UIC:    [355,1] ([STARLINK])
CLI:      DCL                              Tables: 
Default:  SYS$SYSDEVICE:[NETWORK]
LGICMD:   SYS$MANAGER:STAR_LOGIN.COM
Login Flags:  Disctly Lockpwd Captive Diswelcome Disnewmail Dismail Disreport
Primary days:   Mon Tue Wed Thu Fri Sat Sun
Secondary days:                            
Primary   000000000011111111112222  Secondary 000000000011111111112222
Day Hours 012345678901234567890123  Day Hours 012345678901234567890123
Network:  -----  No access  ------            -----  No access  ------
Batch:    -----  No access  ------            -----  No access  ------
Local:    ##### Full access ######            ##### Full access ######
Dialup:   -----  No access  ------            -----  No access  ------
Remote:   -----  No access  ------            -----  No access  ------
Expiration:            (none)    Pwdminimum:  6   Login Fails:     0
Pwdlifetime:        300 00:00    Pwdchange:             (none) 
Last Login: 15-SEP-1986 15:31 (interactive), 17-JUN-1986 15:45 (non-interactive)
Maxjobs:         0  Fillm:        30  Bytlm:         8192
Maxacctjobs:     0  Shrfillm:      0  Pbytlm:           0
Maxdetach:       0  BIOlm:        18  JTquota:       1024
Prclm:           0  DIOlm:        18  WSdef:          250
Prio:            4  ASTlm:        15  WSquo:          400
Queprio:         0  TQElm:        10  WSextent:      4096
CPU:        (none)  Enqlm:        30  Pgflquo:      20480
Authorized Privileges: 
  TMPMBX NETMBX
Default Privileges: 
  TMPMBX NETMBX
\end{verbatim}

\large

Note the use of all the FLAGS and the non-standard login file.
The username STARLINK for the account was agreed by the site managers and
should be used.
It is important to allow only LOCAL access to prevent people using these
accounts to hop around the country from machine to machine.
The password can be null if desired.

\subsection {The login command procedure}

The login procedure for this account must allow the user to access a selection
of sites around the network if it is to be useful.
The DCL command SET HOST is used with the optional qualifier /X29 to allow
access to all sites on DECnet and JANET.
It was also felt that some form of HELP facility would be of value to
inexperienced users of this username.
The procedure used at UCL is reproduced below but could doubtless be tailored
and improved for other sites:

\normalsize

\begin{quote}
\begin{verbatim}
$ ! Login procedure for NETWORK access captive account
$ !
$ ! Richard Hook, UCL, September 1986
$ ! 
$ ! Make this as secure as possible
$ !
$ set on
$ set nocontrol_y
$ on error then logout/brief
$ !
$ ! Welcome user and explain how this works
$ !
$ write sys$output " "
$ write sys$output "Welcome to NETWORK access account"
$ write sys$output "(Type HELP for some explanation.)"
$ write sys$output " "
$ !
$ ! Prompt the user
$ !
$ Prompt:
$ read/prompt="Node? "  sys$command nodename
$ !
$ ! Check for HELP request
$ !
$ if (f$locate(nodename,"HELP") .nes. 0) .and. -
     (f$locate(nodename,"help") .nes. 0) then goto SetHost
$    type sys$input


      This NETWORK facility allows a visitor to call another machine using
      the SET HOST command but prevents him from using DCL commands.

      In reply to the prompt Node? reply either with the name of the DECnet
      (Starlink) node name (e.g. RLVAD,REVAD etc) or with <number>/X where
      <number> is the DTE address of the Janet machine you wish to call.

$    goto Prompt
$ !
$ ! Use SET HOST to get network connection
$ !
$ SetHost:
$ !
$ !  filter out any attempt to set host to the starlink database system
$ !  (STADAT 19.7 = 19463)
$ !
$ nodename = f$edit(nodename,"COLLAPSE,UPCASE,UNCOMMENT")
$ nodename = f$extract(0,f$locate(":",nodename),nodename)
$1$:
$ if f$extract(0,1,nodename).nes."_" then goto 2$
$     nodename = f$extract(1,9999,nodename)
$     goto 1$
$2$:
$ if nodename.eqs."STADAT" .or. nodename.eq.19463 then goto no_access
$!
$ deassign sys$input
$ assign sys$command sys$input
$ set host 'nodename'
$ !
$ ! Throw the user out
$ !
$ logout/brief
$no_access:
$ write sys$output "Access to STADAT is not allowed"
$ logout/brief
\end{verbatim}
\end{quote}

\large

\subsection {Using the account for JANET access}

One problem with such an approach is that the normal mnemonics for the JANET
sites are unavailable because the CBS is not being used.
The PSI provides a facility to use logical names in a similar way to mnemonics,
rather than having to find out and remember the long DTE numbers.
Such logical names must be entered into the PSI\$DTE\_TABLE system logical name
table.
It is not sensible or syntactically possible to transfer all the NRS mnemonics
to this table, but it is desirable that all the commonly needed sites are
entered.
For example, a privileged user may enter the London PSS gateway address as
follows:
\begin{quote}
{\tt \$ ASSIGN/TABLE=PSI\$DTE\_TABLE 40000040 PSS}
\end{quote}
It is desirable that a standard minimum set of such mnemonics be common to all
Starlink sites.
Ask Alan Lotts for details.

\newpage

\section {Short Codes For Starlink Sites}

The 1-letter and 3-letter short codes for Starlink sites are:
\begin{tabbing}
XXXXX\=1-letterXX\=3-letterXX\=\kill
\>1-letter\>3-letter\>Site\\
\\  
\>A\>ARM\>Armagh\\
\>Q\>BEL\>Belfast\\
\>B\>BIR\>Birmingham\\
\>C\>CAM\>Cambridge: IOA/RGO\\
\>F\>CAR\>Cardiff\\
\>D\>DUR\>Durham\\
\>V\>HAT\>Hatfield\\
\>W\>IMP\>Imperial College of Science, Technology \& Medicine\\
\>J\>JOD\>Jodrell Bank\\
\>K\>KEE\>Keele\\
\>N\>KEN\>Kent\\
\>I\>LEI\>Leicester\\
\>M\>MAN\>Manchester\\
\>O\>OXF\>Oxford\\
\>P\>PRE\>Preston\\
\>U\>QMW\>Queen Mary \& Westfield College\\
\>H\>RAL\>RAL (Astrophysics cluster)\\
\>R\>PRO\>RAL (Project cluster)\\
\>Z\>CEN\>RAL (STADAT)\\
\>E\>ROE\>Royal Observatory Edinburgh\\
\>S\>SOU\>Southampton\\
\>T\>STA\>St Andrews\\
\>X\>SUS\>Sussex\\
\>L\>UCL\>University College London\\
\end{tabbing}

\newpage

\section{Installing Starlink Software Changes}

The directory containing the files for release SSCn is stored at RAL as a
compressed save set in DISK\$SOFTDEV:\-[STAR.\-TEMP]\-SSCn.\-BCK.
Copy this to your local site and install the release.
There may be an optional save set called SSCnX.BCK in the same directory.
This need not be copied if you do not want to install an optional item.
However, you should {\em always} copy SSCn.BCK and run the command procedure
SSCn.COM that it contains.
Also, you should install updates in their numerical order since information
files are updated in every release.

\begin{description}

\item [COPY]: Copy SSCn.BCK and any optional save sets from RAL.
If on DECnet, store the command:
\begin{verbatim}
    $ COPY RLVAD::DISK$SOFTDEV:[STAR.TEMP]SSCn.BCK disk:[STAR.TEMP]*.*
\end{verbatim}
in file CPYSSC.COM, then submit it to a batch queue:
\begin{verbatim}
    $ SUBMIT CPYSSC
\end{verbatim}
If not on DECnet, use the TRANSFER command:
\begin{verbatim}
    $ TRANSFER/CODE=FAST/USERNAME=NETUSR,NETUSR -
         UK.AC.RL.STAR::DISK$SOFTDEV:[STAR.TEMP]SSCn.BCK -
         disk:[STAR.TEMP]SSCn.BCK
\end{verbatim}

\item [RECREATE]: Recreate the [STAR.TEMP.SSCn] directory:
\begin{verbatim}
    $ SET DEF [STAR.TEMP]
    $ LZDCM  SSCn.BCK SSCn.BCK
    $ PURGE
    $ BACKUP SSCn.BCK/SAVE [*...]
\end{verbatim}
Notice that you must de-compress the copied version of the save-set using
LZDCM before using BACKUP to recreate the directory.
People of a nervous disposition may use the qualifiers /LOG and /VERIFY in the
BACKUP command if this makes them feel happier.

\item [IMPLEMENT]: Follow the implementation instructions in SSCn.LIS (which you
should already have available) to install the release.

\item [CLEAN-UP]: Carry out any actions of the CLEAN-UP phase (SSN/41) that you
consider necessary.
\end{description}

\rule{\textwidth}{0.3mm}
\begin{quote}
N.B.\ YOU MUST COPY THE SSCn.BCK FILE FROM RAL WITHIN A MONTH OF NOTIFICATION
OF A RELEASE AS THE FILE WILL DISAPPEAR WITHOUT NOTICE AFTER THIS TIME.
\end{quote}
\rule{\textwidth}{0.3mm}

\end{document}
