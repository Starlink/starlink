\documentstyle{article}
\pagestyle{myheadings}

%------------------------------------------------------------------------------
\newcommand{\stardoccategory}  {Starlink General Paper}
\newcommand{\stardocinitials}  {SGP}
\newcommand{\stardocnumber}    {3.1}
\newcommand{\stardocauthors}   {P M Allan}
\newcommand{\stardocdate}      {21 June 1990}
\newcommand{\stardoctitle}     {Network Futures}
%------------------------------------------------------------------------------

\newcommand{\stardocname}{\stardocinitials /\stardocnumber}
\markright{\stardocname}
\setlength{\textwidth}{160mm}
\setlength{\textheight}{240mm}
\setlength{\topmargin}{-5mm}
\setlength{\oddsidemargin}{0mm}
\setlength{\evensidemargin}{0mm}
\setlength{\parindent}{0mm}
\setlength{\parskip}{\medskipamount}
\setlength{\unitlength}{1mm}

\begin{document}
\thispagestyle{empty}
SCIENCE \& ENGINEERING RESEARCH COUNCIL \hfill \stardocname\\
RUTHERFORD APPLETON LABORATORY\\
{\large\bf Starlink Project\\}
{\large\bf \stardoccategory\ \stardocnumber}
\begin{flushright}
\stardocauthors\\
\stardocdate
\end{flushright}
\vspace{-4mm}
\rule{\textwidth}{0.5mm}
\vspace{5mm}
\begin{center}
{\Large\bf \stardoctitle}
\end{center}
\vspace{5mm}

\section{Introduction}

Over the history of Starlink, there have been few changes in the way that we
have provided network access, but the rate of change is likely to increase in
the near future. The purpose of this document is to outline the directions in
which things are proceeding so that you can plan for them now.

The intended audience is Starlink node managers. A general familiarity with
general network concepts and DECnet is assumed.

\section{History}

The original Starlink network consisted of six VAX-11/780 computers that were
connected via private leased lines in a star topology (hence our name). The hub of the
network was the VAX at RAL. Each VAX had a single DECnet DDCMP circuit that
connected it to the VAX at RAL and this provided connections between all six
machines. There was no Coloured Books software at the time. A disadvantage of
this arrangement was that if the VAX at RAL went down, the network was dead.
Around 1982, use of the private lines was discontinued and the physical network
connections were provided by using SERCnet, the forerunner of our present
JANET. Unlike the previous point to point connection, this was a packet
switched network. This required the use of PSI software so that we could run
DECnet DLM circuits over the network. We also purchased the
Coloured Books software to provide access to the many computers attached to
SERCnet that did not use DECnet. The situation between 1983 and 1987 remained
fairly stable and a typical VAX set up is shown in
figure~\ref{early-configuration}.

\begin{figure}[htbp]
\begin{center}
\begin{picture}(50,70)
  \thicklines

  \put(0,20) {\framebox(40,50)}
  \put(20,75) {\makebox(0,0){VAX-11/780}}
  \put(0,30) {\line(1,0){40}}
  \put(0,40) {\line(1,0){40}}
  \put(0,50) {\line(1,0){40}}
  \put(20,50) {\line(0,1){20}}
  \put(0,20) {\makebox(40,10){DUP11}}
  \put(0,30) {\makebox(40,10){PSI level 2}}
  \put(0,40) {\makebox(40,10){PSI level 3}}
  \put(0,50) {\makebox(20,20){DECnet}}
  \put(20,50) {\makebox(20,20){\shortstack{Coloured\\Books}}}
  \put(20,10) {\line(0,1){10}}
  \put(10,0) {\framebox(20,10){Modem}}
  \put(30,5) {\vector(1,0){10}}
  \put(41,5) {\makebox(0,0)[l]{JANET}}

\end{picture}
\caption[]{Network Configuration 1984-1988}
\label{early-configuration}
\end{center}
\end{figure}

In 1987 we began to see VAXclusters being set up, which required the addition
of ethernet interfaces to our machines. At first these were just to provide a
connection between the CPUs that formed the cluster, but later on these
private ethernets were joined to larger site wide ethernets. For example the
ethernet at RAL has some hundreds of VAXs connected to it and the Manchester
LAN spreads out as far as Jodrell Bank, although admittedly not at
10Mbits/sec. The situation was now as shown in figure~\ref{vaxcluster} and
little had changed as far as off site networking was concerned.

\begin{figure}[htbp]
\begin{center}

\begin{picture}(140,80)
\thicklines

\put(10,10)
{
  \begin{picture}(30,70)
    \put(0,20) {\framebox(30,50)}
    \put(15,75) {\makebox(0,0){MicroVAX II}}
    \put(0,30) {\line(1,0){30}}
    \put(0,40) {\line(1,0){30}}
    \put(0,20) {\makebox(30,10){DEQNA}}
    \put(0,30) {\makebox(30,10){Ethernet}}
    \put(0,40) {\makebox(30,30){DECnet}}
    \put(15,10) {\line(0,1){10}}
    \put(5,0) {\framebox(20,10){H4000}}
  \end{picture}
}

\put(60,10)
{
  \begin{picture}(60,70)
    \put(0,20) {\framebox(60,50)}
    \put(30,75) {\makebox(0,0){VAX-11/780}}
    \put(0,30) {\line(1,0){60}}
    \put(0,40) {\line(1,0){60}}
    \put(30,50) {\line(1,0){30}}
    \put(30,20) {\line(0,1){30}}
    \put(40,50) {\line(0,1){20}}
    \put(0,20) {\makebox(30,10){DELUA}}
    \put(0,30) {\makebox(30,10){Ethernet}}
    \put(30,20) {\makebox(30,10){DUP11}}
    \put(30,30) {\makebox(30,10){PSI level 2}}
    \put(30,40) {\makebox(30,10){PSI level 3}}
    \put(0,50) {\makebox(40,20){DECnet}}
    \put(40,50) {\makebox(20,20){\shortstack{Coloured\\Books}}}
    \put(15,10) {\line(0,1){10}}
    \put(5,0) {\framebox(20,10){H4000}}
    \put(45,15) {\line(0,1){5}}
    \put(35,5) {\framebox(20,10){Modem}}
    \put(55,10) {\vector(1,0){10}}
    \put(66,10) {\makebox(0,0)[l]{JANET}}
  \end{picture}
}

\put(0,10) {\line(1,0){120}}
\put(0,8) {\line(1,0){120}}
\put(121,9) {\makebox(0,0)[l]{Ethernet}}

\end{picture}

\caption[]{VAXcluster Configuration circa 1988}
\label{vaxcluster}

\end{center}
\end{figure}

The cluster satellites were DECnet end nodes, so they could communicate with
all other Starlink machines by routing DECnet traffic through the boot node
that had the synchronous interface, however, they did not use the Coloured
Books software, simply due to the cost of providing it on all machines in a
cluster. This is the stage that most nodes are currently at. They either have a
single machine configured as shown in figure~\ref{early-configuration} or a
cluster as shown in figure~\ref{vaxcluster}.

It is worth noting that until 1989, Starlink was the only project allowed to
run DECnet over the JANET network. There is an important political point here.
The Joint Network Team, which is responsible for the running of JANET, do not
like us running DECnet over JANET. There are two reasons for this. Firstly, it
is part of their remit to encourage open networking, which they take to mean
not using manufacturer specific protocols. Secondly, DECnet is less efficient
in its use of network bandwidth than are the Coloured Books protocols. Against
this, DECnet provides many functions that we require that Coloured Books do not
provide. For example, our use of VAXnotes would be severely hampered if we
could not use DECnet as it would mean logging onto the machine that hosted a
conference to read anything in that conference. Also utilities such as NETSHOW
and TALK would be impossible to provide in their current form using Coloured
Books. It was only because we were already using the functionality of DECnet
that Coloured Books does not provide that we were allowed to run DECnet over
JANET at all. We have a long standing commitment to give up using DECnet over
JANET as soon as Coloured Books provides all the functionality that we require,
however clearly this will never happen. In fact the JNT seem to have recognised
this as the HEP community have been allowed to run DECnet over JANET since
1989. As for the future of networking, this clearly means a migration to the
use of OSI protocols. Fortunately for us, DECnet will be one of the first
protocols to move to OSI compliance. This is what DECnet phase~V is all about.
At present it is not clear when, if ever, Coloured Books will be OSI compliant.

The future involves new hardware such as dedicated X.25 router boxes (DEMSAs
and the like), Spider gateways, Pink Book software and new acronyms like NSAP.
For three sites (RAL, Manchester and ROE), the future has already arrived, and
DECnet/OSI is looming over the horizon. The rest of this document is about the
new types of hardware and software, when each type is appropriate and how to
configure it if and when you get it.

\section{PSI and PSI-access}

It is probably best to start off with PSI and PSI-access as we have used PSI
for many years now. In the beginning, when Starlink used it own private leased
lines for the physical connection between sites, the connection between sites
were DDCMP circuits. These provided point to point connections which, as
mentioned above, suffered from RAL being a single point of failure. It is
possible to set up redundant circuits, much as we have now with our DLM
circuits, but with DDCMP this costs a lot of real money in leasing the lines as
opposed to costing a bit of virtual money in extra network bandwidth. A packet
switched network such as JANET is much more flexible as it parcels the data up
into `packets' which are sent out onto the network. The network can dynamically
route the data packets from source to destination by the best route, so there
is no single point of failure, nor even a single point of congestion. This is
rather like the way the telephone network functions. The software that allows
us to use a packet switched network is PSI. In the figures, PSI is shown as
being divided into level~2 and level~3. The importance of this division will
become clearer when Pink Book is considered. Essentially level 3 formats the
packets and level 2 sends them out onto the network and receives them at the
other end.

PSI provides a programming interface on top of which other network programs can
sit. This is what the Coloured Books software does. When used over a PSDN,
DECnet also uses PSI, although the interaction between the two is more complex
than figure~\ref{early-configuration} implies. In the simple example shown in
figure~\ref{early-configuration}, PSI provides the route by which the Coloured
Books software can send data out onto JANET. This is what is known as native
mode. It is also possible to set up PSI so that it can pass data between JANET
and another VAX running the PSI-access software. Note that this is not routing
in the DECnet sense. The machine running the PSI software is called the
connector node and the machine running PSI-access is called the access system.
The machine running the PSI-access software presents the same programming
interface to software that sits on top of it as does PSI, but it does not have
the level~2 functions that let it talk to a synchronous interface such as a
DUP-11. Instead the connection between the access system and the connector node
is provided by DECnet. A possible two node VAXcluster with Coloured Books on
each machine is shown in figure~\ref{two-cbs}. Both machines have Coloured
Books software installed and although one machine accesses the network directly
whereas the other one accesses the network via the first machine, this is
transparent to the user. Note that no Starlink cluster has been configured like
this.

\begin{figure}[htbp]
\begin{center}

\begin{picture}(140,80)
\thicklines

\put(0,10)
{
  \begin{picture}(60,70)
    \put(0,20) {\framebox(60,50)}
    \put(30,75) {\makebox(0,0){MicroVAX II}}
    \put(0,30) {\line(1,0){30}}
    \put(0,40) {\line(1,0){60}}
    \put(30,50) {\line(1,0){30}}
    \put(30,20) {\line(0,1){30}}
    \put(40,50) {\line(0,1){20}}
    \put(0,20) {\makebox(30,10){DELQA}}
    \put(0,30) {\makebox(30,10){Ethernet}}
    \put(0,40) {\makebox(30,30){DECnet}}
    \put(30,40) {\makebox(30,10){PSI level 3}}
    \put(40,50) {\makebox(20,20){\shortstack{Coloured\\Books}}}
    \put(15,10) {\line(0,1){10}}
    \put(5,0) {\framebox(20,10){H4000}}
  \end{picture}
}

\put(70,10)
{
  \begin{picture}(60,70)
    \put(0,20) {\framebox(60,50)}
    \put(30,75) {\makebox(0,0){VAX-11/780}}
    \put(0,30) {\line(1,0){60}}
    \put(0,40) {\line(1,0){60}}
    \put(30,50) {\line(1,0){30}}
    \put(30,20) {\line(0,1){30}}
    \put(40,50) {\line(0,1){20}}
    \put(0,20) {\makebox(30,10){DELUA}}
    \put(0,30) {\makebox(30,10){Ethernet}}
    \put(30,20) {\makebox(30,10){DUP11}}
    \put(30,30) {\makebox(30,10){PSI level 2}}
    \put(30,40) {\makebox(30,10){PSI level 3}}
    \put(0,50) {\makebox(40,20){DECnet}}
    \put(40,50) {\makebox(20,20){\shortstack{Coloured\\Books}}}
    \put(15,10) {\line(0,1){10}}
    \put(5,0) {\framebox(20,10){H4000}}
    \put(45,15) {\line(0,1){5}}
    \put(35,5) {\framebox(20,10){Modem}}
    \put(55,10) {\vector(1,0){5}}
    \put(66,10) {\makebox(0,0)[l]{JANET}}
  \end{picture}
}

\put(0,10) {\line(1,0){120}}
\put(0,8) {\line(1,0){120}}
\put(121,9) {\makebox(0,0)[l]{Ethernet}}

\end{picture}

\caption[]{VAXcluster with CBS on both nodes}
\label{two-cbs}

\end{center}
\end{figure}

To set up PSI-access, an entry is made
in the X25-ACCESS module of the NCP database to tell the access node which
machine is the connector node. The connector node knows about the access
system by defining an appropriate destination in its X25-SERVER module.
This is fully described in the PSI documentation and is usually performed
by the PSI{\tt\_}SET{\tt\_}UP.COM command procedure when the software is
installed.

So to summarize, PSI lets a VAX communicate with a WAN via
its own synchronous interface whereas PSI-access allows a VAX to
communicate with a WAN via its DECnet connection to another machine and it
is this other machine that connects directly to the WAN. While this DECnet
connection will usually be made via ethernet, this is not necessary. For
example, we could have evolved to a situation in which we still ran DDCMP
circuits over our old leased lines and access to JANET and the outside
world was provided by having PSI on a VAX at RAL and PSI-access on
machines at other sites. It is left as an exercise to the reader to
consider the pros and cons of this hypothetical configuration.

To summarize the summary, you use PSI when you have a synchronous interface in
your microVAX and you use PSI-access when you don't.

\section{Pink Book}

Pink Book is also known as LLC2. I think the name Pink Book is confusing as
unlike Grey Book, Blue Book and Yellow Book (these are what make up the
Coloured Books), it does not provide any new types of service at the user
level. It simply provides the same services, but now over ethernet. LLC2 is
actually an extra component of PSI, and is best described by looking at
figure~\ref{llc2}. You will see that LLC2 is an alternative to PSI's normal
level 2 functionality. Whereas PSI level 2 normally talks to a synchronous
interface such as a DSV-11, LLC2 talks to the ethernet interface. This lets us
use X.25 protocols over the ethernet. For example, you can use POST to send
mail to any machine on the ethernet that uses Coloured Books software. While
this can be useful, the main use of LLC2 within Starlink is likely to be the
ability to access JANET via an X.25 {\bf--} ethernet gateway such as the Spider
gateway 200 which is currently being installed on several campuses. The
rationale behind this is that most machines need ethernet interfaces for other
purposes, so using them to provide access to JANET means that individual
machines no longer need a synchronous interface. Spider gateways need to be
able to translate NRS names to DTE addresses, but they have a limited memory
size and can only hold about 200 addresses. In practice, they need to have a
second machine available with much more storage capacity on disk that can
provide storage for the whole of the NRS table.

{\bf N.B.} You cannot set up DLM circuits over an LLC2 line. This means that
you cannot connect to the Starlink DECnet just using PSI and LLC2. There has to
be a synchronous interface somewhere in your system, even if it is just part of
a DEMSA.

\begin{figure}[htbp]
\begin{center}

\begin{picture}(100,70)
\thicklines
  \put(20,20) {\dashbox{2.5}(70,40)}
  \put(55,65) {\makebox(0,0){VAX PSI}}
  \put(45,45) {\framebox(20,10){Level 3}}
  \put(30,25) {\framebox(20,10){LLC 2}}
  \put(40,35) {\line(1,1){10}}
  \put(60,25) {\framebox(20,10){Level 2}}
  \put(70,35) {\line(-1,1){10}}
  \put(40,25) {\line(0,-1){20}}
  \put(40,5) {\vector(-1,0){10}}
  \put(29,5) {\makebox(0,0)[r]{Ethernet}}
  \put(70,25) {\line(0,-1){20}}
  \put(70,5) {\vector(1,0){10}}
  \put(81,5) {\makebox(0,0)[l]{PSDN}}
\end{picture}

\caption[]{VAX PSI and LLC2}
\label{llc2}

\end{center}
\end{figure}

\begin{figure}[htbp]
\begin{center}

\begin{picture}(140,80)
\thicklines

\put(30,10)
{
  \begin{picture}(60,70)
    \put(0,20) {\framebox(60,50)}
    \put(30,75) {\makebox(0,0){VAX-11/780}}
    \put(0,30) {\line(1,0){60}}
    \put(0,40) {\line(1,0){60}}
    \put(20,50) {\line(1,0){40}}
    \put(40,20) {\line(0,1){10}}
    \put(20,30) {\line(0,1){20}}
    \put(40,30) {\line(0,1){10}}
    \put(40,50) {\line(0,1){20}}
    \put(0,20) {\makebox(40,10){DELUA}}
    \put(40,20) {\makebox(20,10){DUP11}}
    \put(0,30) {\makebox(20,10){Ethernet}}
    \put(20,30) {\makebox(20,10){LLC2}}
    \put(40,30) {\makebox(20,10){PSI level 2}}
    \put(20,40) {\makebox(40,10){PSI level 3}}
    \put(0,50) {\makebox(40,20){DECnet}}
    \put(40,50) {\makebox(20,20){\shortstack{Coloured\\Books}}}
    \put(15,10) {\line(0,1){10}}
    \put(5,0) {\framebox(20,10){H4000}}
    \put(45,15) {\line(0,1){5}}
    \put(35,5) {\framebox(20,10){Modem}}
    \put(55,10) {\vector(1,0){5}}
    \put(66,10) {\makebox(0,0)[l]{JANET}}
  \end{picture}
}

\put(0,10) {\line(1,0){120}}
\put(0,8) {\line(1,0){120}}
\put(121,9) {\makebox(0,0)[l]{Ethernet}}

\end{picture}

\caption[]{VAX with PSI and LLC2}
\label{everything}

\end{center}
\end{figure}

\section{PSI-access versus LLC2}

Both PSI-access and LLC2 manage to send data out via an ethernet connection
rather than a synchronous interface, so there is often some confusion as to how
the two differ. The most important point is that the two cannot talk to each
other. PSI-access is a DEC proprietory product and will only work between DEC
equipment. LLC2 is a more open protocol. If you are running PSI-access, then
the only thing that you can talk to is a PSI connector node (a VAX or a DEMSA).
If you are running LLC2, then you can only talk to another machine that is
running LLC2. In practice, you would use PSI-access to directly connect to
another part of your own system. The word `system' here is deliberately vague,
but probably consists of the VAX computers and DEMSAs that you control. LLC2
would be used to directly connect to a much wider range of computers, not all
of them made by DEC. By setting up the appropriate routing, both PSI-access and
LLC2 can be used to contact computers further afield, but as mentioned above,
you cannot run DECnet circuits over LLC2.

\section{X.25 routers}

An X.25 router is a dedicated machine that provides a connection between an
ethernet and a WAN. The one currently in use at a few Starlink sites is the
DEMSA. This machine provides the functionality of a microVAX CPU, some memory,
an ethernet interface and two DSV-11s. It runs the X25router software, not VMS,
however, in practice it looks just like a DECnet node and a PSI connector node.
Most of its operations can be controlled via NCP. For a site where there are
several machines that require access to JANET via means other than a Spider
gateway, a DEMSA can be an economical solution as it saves the cost of a
synchronous interface for each machine. Since LLC2 does not support DECnet DLM
circuits, it cannot be used to provide a DECnet connection and another  method
must be found. At Manchester a DEMSA was purchased to be shared jointly between
Astronomy, Jodrell Bank and High Energy Physics. This choice was made because
both Astronomy and Jodrell Bank were having unibus based machines replaced with
Q-bus based machines, meaning that our old DUP-11s were redundant. The cost of
buying Q-bus synchronous interfaces was about the same as our share of buying a
DEMSA. A DEMSA has the additional benefit that it takes the level~2  DECnet
routing off our Vaxs, thereby reducing the load on the CPU slightly. In our
case, there was also the additional consideration that we wanted to upgrade our
network line speed from 9.6K baud (or 7.2K baud in the case of Jodrell Bank) to
64K baud. While we managed to persuade the MCC to give us one fast line, the
prospect of getting three was distinctly small. By purchasing a DEMSA, we only
need a single connection point to JANET. DEMSAs have further advantages in that
they never need to go down for disk backups and since they are inherently
simple machines, they should very rarely break down.

There are various hardware and software configurations available. The hardware
is a DEMSA or a DEMSB. These are also known as DEC microservers. These come
bundled with either the DECrouter 2000, X25router 2000, X25portal 2000 or
DECnet/SNA gateway software. Whereas a DEMSA has four synchronous interfaces
that run at up to 256K baud, a DEMSB has only a single one that runs at up to
64K baud (19.2K baud over X.25). A DEMSB can not be used as a DECnet level 2
router and it supports only one DLM circuit. The packaging of the software
will be changed when DECnet moves to phase~V. Both the DEMSA and DEMSB are
DECnet phase~V compatible.

\section{DECnet phase V}

This is also known as DECnet/OSI as it will use OSI protocols. This means that
it will be able to communicate with any other computer that runs network
software based on OSI protocols, not just those that run DECnet. DECnet/OSI is
a major change to DECnet and will mean a change in the way that we use it. At
present a DECnet node (a VAX or a dedicated router) can be an end node, a
routing node (level 1 router) or an area router (level 2 router). With
DECnet/OSI there will be end systems or intermediate systems. Initially it will
not be possible to use a VAX as an area router, and indeed it may never be
possible. Since area routing will be a much more CPU intensive operation than
it is now, it should be moved off VAXs in any case. A DEMSA will make a
perfectly good DECnet area router, but in phase~V, it will not be possible to
use a DEMSA both as a DECnet node and as a PSI connector node, as can be done
at present. Essentially, this is due to a DEMSA not having enough memory to
support both functions, but DEC will also be packaging the software differently.
If you use a DEMSA to provide both your DECnet connection and your PSI
connection to JANET at present, then you may need to purchase a second DEMSA
when we move to DECnet/OSI. Another possibility is to use your current DEMSA to
provide your DECnet connection and to use a Spider gateway, if your site has
one, to provide your PSI (and hence Coloured Books) connection.

\section{What does {\em my} site need?}

If you are at a small
university site and have the only system that needs a DECnet connection to
the Starlink/SPAN/HEP network, then a synchronous interface, such as a
DSV-11, in your VAX is still appropriate. This will allow you to connect to
both the Starlink/SPAN/HEP DECnet network and to JANET at up 64K baud. If
there are several machines that require a DECnet connection, then a DEMSA
may be cost effective, but remember that you will need two DEMSAs (or more
likely, some equivalent hardware) when we move to DECnet/OSI.

A synchronous interface in your microVAX will always be sufficient to provide
all of your network connectivity, although it is not necessarily the most
cost effective solution. LLC2 software plus access to a Spider gateway is
never sufficient by itself as you cannot run DECnet (phase~IV or phase~V)
this way.

There are three  main configurations that seem likely to be in
use in the future. A small site will have a synchronous interface in a
microVAX. Larger sites will have a DEMSA plus a Spider gateway or two
DEMSAs. At present, many University sites are getting Spider gateways, so
as long as your University has a campus wide ethernet, you will be able
to use this to provide a fast PSI connection on to JANET. While we are
still at DECnet phase IV, the software required in each of the above
cases is

\begin{itemize}
\item PSI on the microVAX
\item PSI-access and PSI plus LLC2 on the microVAX and X25router on the DEMSA
\item PSI-access on the microVAX and X25router on the two DEMSAs.
\end{itemize}

As mentioned above, the packaging of the software will be changed with
DECnet/OSI. If you have two DEMSA (in preparation for DECnet/OSI), but no
Spider gateway, then you do not need PSI plus LLC2 to communicate around
Starlink and the wide world, however, you may still want them to access other
non-DECnet computers on your ethernet.

\section{A Case Study}

To try to see how all the components fit together, let us consider an actual
site, the Department of Astronomy at Manchester University. This site currently
has one of the more complex arrangements of network software and it may be
instructive to see how this came about. Prior to the end of 1989, the site was
as shown in figure~\ref{vaxcluster}. The cluster consisted of a VAX-11/750 and
several microVAXs. The 750 had a DUP-11 as its synchronous interface. This was
connected to the MCC CPSE and thence to the JPSE via a pair of 9600 baud
modems. The software on the 750 consisted of PSI, Coloured Books and Red Book.
The University purchased LLC2 for the 750 as part of a campus wide movement
towards using the Spider gateways and away from each system having its own
connection to the CPSE. It is worth recalling that the University has a
commitment to provide us with X.25 access for the Coloured Books software, but
no requirement to provide us with DECnet facilities. When the 750 was replaced
by a dual host microVAX 3400 system, no new synchronous interface was purchased
as we had purchased a DEMSA instead. The network software on one of the
microVAX 3400s now consists of PSI-access, PSI plus LLC2 and Coloured Books.
The DEMSA plus DECnet provides us with off site access to the Starlink DECnet.
PSI-access is necessary to provide X.25 access for the microVAX 3400 to JANET
via the DEMSA and PSI plus LLC2 is necessary to provide X.25 access to machines
on our local ethernet and to machines on JANET if we go via the Spider gateway.
At present, nearly all of the network traffic to and from JANET goes through
the DEMSA and the Spider gateway is only used for test purposes. This is
because the original configuration of the Spider gateways did not allow the
registration of more than 200 site names. This restriction has now been
overcome, however, it is still more convenient to use the DEMSA than the Spider
gateway.

When we make the transition to DECnet/OSI, the DEMSA will not be able to
handle both DECnet and X.25 protocols. The way that this situation will
probably be handled will be to use the DEMSA for DECnet traffic as at present
and to use the Spider gateways for X.25 traffic. At that point the PSI-access
software would be redundant.

\section{The Colour Books authorization file}

One of the things that causes most difficulty with different network
configurations is the registering of sites in the CBS network authorization
database (NAD). The main question is whether you are going through a Spider
gateway or not. In simplest case of both the local and remote sites having a
configuration like figure~\ref{early-configuration} or \ref{vaxcluster}, then
an entry in your NAD will be something like

\begin{verbatim}
    ADD SITE UK.AC.LEICESTER.STARLINK JANET "" -
            "FTP=000021213000.FTP", -
            "JTMP=000021213000.JTMP", -
            "MAIL=000021213000.FTP.MAIL", -
            "X29=000021213000"-
            /ALIAS = (-
            "UK.AC.LE.STAR"-
            )-
            /FAST_SELECT-
            /NOREV_CHARGE-
            /NORESTRICTED-
            /TRUSTED-
            /AUTO_REVERSE-
            /WINDOW_SIZE=4-
            /PACKET_SIZE=256
\end{verbatim}

This means that the Starlink node at Leicester should be contacted at the JANET
address (DTE) of 000021213000. In this case, the network name given on the
first line is JANET, however this name can be anything chosen by the system
manager. All that is necessary is that it agrees with one of the network names
defined in the NCP X25-PROTOCOL module on your system.

If the remote system is running PSI-access and does not have a direct network
connection, then all that is necessary is to add the appropriate subaddress to
the DTE. In the case of Manchester, the microVAX is accessed through the DEMSA,
so the NAD entry is this.

\begin{verbatim}
    ADD SITE UK.AC.MANCHESTER.ASTRONOMY.STARLINK JANET "" -
            "FTP=00001010900111.FTP", -
            "JTMP=00001010900111.JTMP", -
            "MAIL=00001010900111.FTP.MAIL", -
            "X29=00001010900111"-
            /ALIAS = (-
            "UK.AC.MAN.AST.STAR"-
            )-
            /FAST_SELECT-
            /NOREV_CHARGE-
            /NORESTRICTED-
            /TRUSTED-
            /AUTO_REVERSE-
            /WINDOW_SIZE=4-
            /PACKET_SIZE=256
\end{verbatim}

It is not absolutely necessary to have a subaddress when using PSI-access as a
particular machine may be configured to field calls with no subaddress, but it
is more usual to have the subaddress present.

Things start to get more complicated when a Spider gateway is involved and it
depends on whether the gateway is at your site or the remote site. The
appropriate entry in the NAD to make a call from Manchester to RAL out through
the Spider gateway at Manchester is

\begin{verbatim}
    ADD SITE UK.AC.RUTHERFORD.STARLINK CONS "08003900309D7E" -
            "FTP=503453431433351450441451523350", -
            "JTMP=505053431433351450441451523350", -
            "MAIL=503953431433351450441451523350", -
            "X29=505653431433351450441451523350"-
            /ALIAS = (-
            "UK.AC.RL.STAR"-
            )-
            /FAST_SELECT-
            /NOREV_CHARGE-
            /NORESTRICTED-
            /TRUSTED-
            /AUTO_REVERSE-
            /WINDOW_SIZE=4-
            /PACKET_SIZE=256
\end{verbatim}

There are several new points here. Firstly the network name is now CONS, not
JANET. This is the network name at Manchester for making a call using LLC2
rather than using PSI-access. Secondly the first line now contains an explicit
primary service, which is the ethernet address of the Spider gateway. This is
also known as the MAC address. Actually the final two hexadecimal digits are
not part of the ethernet address, but are the local service access point
(LSAP). These are usually 7E (decimal 126) by convention. Note that when
setting up the LLC2 software, it is quite easy to get an LSAP of zero instead
of 7E. You should watch out for this. Finally the facility expressions have
become more complicated. These are actually an ASCII like coding of the
facility and the site name. The codes are the ASCII code for a character, less
32. The first pair of digits are always 50 and indicate a local NSAP. The next
pair are the code for the letter B for blue book (FTP), G for grey book (mail),
R for red book (JTMP) or X for X29. Subsequent pairs of digits are the codes
for (in this case) UK.AC.RL.STAR. 33 is the code for A, 34 for B, etc. and 14
is the code for a dot. This local NSAP is sent to the Spider gateway which
forwards the call onto the appropriate place. Although this seems complicated
at first, it does have the advantage that you do not need to know the addresses
of remote sites to be able to call them as the gateway knows the addresses. If
an address changes, only the gateway has to be updated, not your NAD.

Secondly there is the type of entry needed when you are making a call using
normal PSI or PSI-access at your end, but there is a Spider gateway at the
remote end. In this case the entry in the NAD is of the following type. The
network name is now JANET, since that is how we are sending out the call.
The first part of the facility expression contains the JANET address of the
Spider gateway at the remote site and the last part of the facility expression
is the NSAP of the computer that you are trying to contact.

\begin{verbatim}
    ADD SITE UK.AC.CAMBRIDGE.WEST.ASTRONOMY JANET "" -
            "FTP=000008020300.ISO.388261100000829030020277", -
            "JTMP=000008020300.ISO.388261100000829030020677", -
            "MAIL=000008020300.ISO.388261100000829030020377", -
            "TS29=00000802030005.ISO.388261100000829030020077", -
            "X29=000008020300"-
            /ALIAS = (-
            "UK.AC.CAM.WEST.AST"-
            )-
            /FAST_SELECT-
            /NOREV_CHARGE-
            /NORESTRICTED-
            /TRUSTED-
            /AUTO_REVERSE-
            /WINDOW_SIZE=4-
            /PACKET_SIZE=256
\end{verbatim}

Finally there is the case of contacting a machine on your local ethernet.
Whether or not you are using the Spider gateway to contact sites on JANET, you
clearly want to keep calls to other local machines on the ethernet and not send
them out onto JANET, only to have them come back onto the ethernet. An example
of contacting another machine on the Manchester campus LAN is

\begin{verbatim}
    ADD SITE UK.AC.MANCHESTER-COMPUTING-CENTRE.VMSFE CONS "AA0004000CE87E" -
            "FTP=388261100001000003030102", -
            "JTMP=388261100001000003030104", -
            "MAIL=388261100001000003030103", -
            "X29=388261100001000003030101"-
            /ALIAS = (-
            "UK.AC.MCC.VMSFE"-
            )-
            /FAST_SELECT-
            /NOREV_CHARGE-
            /NORESTRICTED-
            /TRUSTED-
            /AUTO_REVERSE-
            /WINDOW_SIZE=4-
            /PACKET_SIZE=256
\end{verbatim}

The fact that the other machine is also a VAX is not particularly important.
This time the primary address is the ethernet address of the machine that you
are contacting. If the remote machine is a VAX, this can be either the physical
address (the AA-00- one) or the hardware address (the 08-00- one). The facility
expression now contains the real NSAP of the remote computer. At Manchester the
last two digits are a code to indicate the type of facility, however, this is
not in use at all sites. Cambridge, at least, has a different suffix to
indicate the type of facility at present.


\appendix
\section{Acronyms}
\begin{quote}
\begin{itemize}
\item[CPSE] Campus Packet Switch Exchange
\item[DDCMP] DIGITAL Data Communications Message Protocol
\item[DLM] Data Link Mapping
\item[ISO] International Standards Organisation
\item[JANET] Joint Academic Network
\item[JPSE] JANET Packet Switch Exchange
\item[MAC] Medium Access Control
\item[LAN] Local Area Network (often synonymous with ethernet)
\item[LLC] Logical Link Control
\item[LSAP] Local Service Access Point (an ISO address)
\item[NSAP] Network Service Access Point (an ISO address)
\item[OSI] Open System Interconnection
\item[PSDN] Packet Switched Data Network
\item[PSI] Packetnet System Interface
\item[WAN] Wide Area Network
\end{itemize}
\end{quote}

\end{document}
