\documentstyle[11pt]{article}
\pagestyle{myheadings}

%------------------------------------------------------------------------------
\newcommand{\stardoccategory}  {Starlink General Paper}
\newcommand{\stardocinitials}  {SGP}
\newcommand{\stardocnumber}    {8.9}
\newcommand{\stardocauthors}   {G E Bromage}
\newcommand{\stardocdate}      {11 February 1992}
\newcommand{\stardoctitle}     {Starlink Local Management}
%------------------------------------------------------------------------------

\newcommand{\stardocname}{\stardocinitials /\stardocnumber}
\renewcommand{\_}{{\tt\char'137}}     % re-centres the underscore
\markright{\stardocname}
\setlength{\textwidth}{160mm}
\setlength{\textheight}{230mm}
\setlength{\topmargin}{-2mm}
\setlength{\oddsidemargin}{0mm}
\setlength{\evensidemargin}{0mm}
\setlength{\parindent}{0mm}
\setlength{\parskip}{\medskipamount}
\setlength{\unitlength}{1mm}

\begin{document}
\thispagestyle{empty}
SCIENCE \& ENGINEERING RESEARCH COUNCIL \hfill \stardocname\\
RUTHERFORD APPLETON LABORATORY\\
{\large\bf Starlink Project\\}
{\large\bf \stardoccategory\ \stardocnumber}
\begin{flushright}
\stardocauthors\\
\stardocdate
\end{flushright}
\vspace{-4mm}
\rule{\textwidth}{0.5mm}
\vspace{5mm}
\begin{center}
{\Large\bf \stardoctitle}
\end{center}
\vspace{5mm}

\section {Area Management Committees}

{\bf Starlink UK Catchment Areas}, each monitored by an {\bf Area Management
Committee (AMC)}, have been defined to include the nodes and groups shown in the
table below for the 1991/92 Reporting Year.

New {\bf Remote User Groups (RUGs)} are normally added to an AMC area based on
geographical proximity, so that, for example, a potential RUG at Newcastle
would be added to the North East Area.
Remote users from abroad are registered at a Node and with an AMC area chosen
to be most appropriate for scientific or administrative reasons, e.g.\ La Palma
users to the East-Anglia Area because of the RGO connection, and VILSPA Madrid
users to the South West Area because of the RAL IUE responsibilities.

\begin{table}[h]
\begin{center}
\begin{tabular}{|l|l|l|l|l|} \hline
{\bf Area and AMC name} & {\bf Major nodes} & {\bf Minor nodes} &
{\bf RUGs (UK)} \\
\hline
\hline
Scotland & Edinburgh/ROE & St Andrews & Glasgow \\
\hline
Northern Ireland & QUB/Armagh & - & - \\
\hline
North East & Durham & - & - \\
\hline
North West & Jodrell Bank & Keele & Leeds \\
& Manchester & Lancs (Preston) & Bradford \\
\hline
Midlands West & Birmingham & - & Aberystwyth \\
\hline
Midlands East & Leicester & - & - \\
\hline
East Anglia & Cambridge/RGO & Hatfield Poly & - \\
\hline
South West & Oxford & Cardiff & Bristol \\
& RAL & Southampton & - \\
\hline
London \& South East & UCL & QMW London & MSSL (Dorking) \\
& & ICSTM London & - \\
& & Sussex & - \\
& & Kent (Canterbury) & - \\
\hline
\end{tabular}
\end{center}
\end{table}
The full current membership of each AMC should be stored in the file
LADMINDIR:AMC.LIS stored at each Starlink site.
The names of the current AMC areas and their chairmen are listed in the
file ADMINDIR:WHOSWHO.LIS.

\section{Terms of Reference of AMCs}

The function of an AMC is to oversee the operation of its respective area
and report to the Starlink Project.
The AMC terms of reference are as follows:
\begin{itemize}
\item to ensure the implementation of Starlink policies concerning the operation
of all the sites in its Area, including Minor Nodes and Remote User Groups.
\item to monitor all aspects of the service to the users, including resources,
availability, documentation and usage, paying special attention to the needs of
remote users.
\item to resolve any conflicting demands on resources and to deal with user
recommendations or complaints which cannot be dealt with by the Site Managers.
\item to make recommendations on any relevant matters of policy or operation to
the Starlink management (e.g.\ provision of hardware is a central
responsibility, but Starlink will be strongly guided by recommendations from
the AMC and the user community).
\end{itemize}

\section{Starlink Local User Groups}

Starlink encourages AMCs to set up Starlink Local User Groups (SLUGs) at each
Starlink site as a forum for discussion of local Starlink matters.
Such SLUGs should hold meetings that are open to all registered Starlink users
at that site, and should encourage the active participation of users in the
decision making process.
The SLUG chairmen should attend the AMC meetings and/or send reports to the AMC.

\section {Site Managers}

Site Managers play a very important but difficult role within Starlink.
They report formally to their respective employers (SERC establishment,
university or polytechnic), but at the same time have responsibility for the
implementation of Starlink policies at their site.
This divided loyalty needs to be recognised by AMCs (and users).
All day-to-day aspects of the operation of the sites are the responsibility of
the Site Managers, guided as necessary by their Department Head/AMC/SLUG.

Site Managers must ensure that only accredited users use the system.
Subject always to subsequent confirmation by the Starlink Project Scientist,
Site Managers can themselves immediately install new users whose requirements
clearly fall within the guidelines set out below (section 5).
Doubtful cases must be referred to the Project Scientist first.
The duties of Site Managers are described in more detail in the Site
Manager's Guides (SGP/25 and SGP/37).

\section {Starlink Policies}

Starlink is provided by SERC as a research facility for all accredited UK
astrophysicists and astronomers, and is primarily intended for the interactive
reduction and analysis of observational data.
Other kinds of astronomical work are not excluded, for example theoretical
computations running in batch mode, but such work is to be run with lower
priority.
The use of any Starlink resources for teaching purposes, for computer games, or
for any commercial activity is expressly forbidden.
In particular, Starlink software must not be used for commercial gain or passed
on to third parties without the consent of the Project --- for more details
on Starlink software distribution policy see SGP/21.

The order of priority for usage of all Starlink resources (equipment, software
and manpower) is as follows:
\begin{enumerate}
\item Astronomical data reduction that can only be done interactively;
\item Other astronomical data reduction;
\item Other astronomical research work that can only be done interactively;
\item Other astronomical research work;
\item Other APSB research work.
\end{enumerate}
In practice, work at the lowest priority, i.e.\ non-astronomical APSB work, is
very rarely run and has never been run at many sites.
The word `astronomical' in the context of Starlink usage is interpreted as
including studies in certain areas of solar system research, namely:
\begin{itemize}
\item the study of the sun as a star;
\item the study of planets, comets, asteroids, and interplanetary space using
telescopes that also study non-solar-system objects (e.g.\ IUE, IRAS, AAO).
\end{itemize}
Other solar system research is only allowed under category (5) above.

Category (3) above includes relevant interactive display and analysis of the
results of theoretical modelling work.
It is recognised that data produced by the use of supercomputers are in many
ways analogous to those obtained from observational facilities.
Where appropriate and feasible, therefore, the use of Starlink by theoretical
astronomers interactively analysing such `theory data' is encouraged.

All the Standard Starlink software must be available at all sites on devices
that are as similar as possible to each other and which present essentially the
same aspect to users everywhere.
A selection of appropriate Optional Software Items should also be installed,
although what is appropriate will vary from site to site.
Site Managers may store infrequently used standard software offline at sites
where disc space is limited.
Such software must be available at short notice if required.

The majority of Starlink applications software comes from astronomer users
themselves, and users are strongly encouraged to develop such software to
Starlink standards and within the Starlink ADAM environment (see SUN/94 and
SG/4).
The Starlink programming standards are specified in SGP/16.

\section{AMC Membership and Meetings}

All AMCs must meet at least once a year, and most meet twice yearly.
The chairman and members of each AMC must be accredited by the Starlink Project
Scientist.
The committee should consist of representatives of major user groups, Minor
Nodes and RUGs, local management, and any relevant local computing committees,
if appropriate.
Starlink recommends that representatives of the local user group (SLUG) and/or
of research students are members of the AMC.
A Major Node Site Manager will act as the secretary of his or her AMC.
Broadly speaking, AMCs should represent the most active users and not
necessarily the most senior personnel in each community.

Starlink management must be informed in advance of each meeting and will
arrange for a representative to attend.
Expenses will be paid (UK travel $+$ day subsistence costs) for travelling
members attending AMC meetings.
Minutes of AMC meetings must be kept.
Each AMC must send an Annual Report to the Starlink Project Scientist in August
each year for consideration by the Starlink Project and the Starlink Users'
Committee.

\section {References}

\begin{description}
\begin{description}
\item [SGP/16]: Starlink applications programming standards.
\item [SGP/21]: Starlink software distribution policy.
\item [SGP/25]: Starlink site manager's guide --- Major nodes.
\item [SGP/37]: Starlink site manager's guide --- Minor nodes.
\item [SUN/94]: ADAM --- Starlink software environment.
\item [SG/4]:   ADAM --- The Starlink software environment.
\end{description}
\end{description}
\end{document}

\end{document}
***

\end{document}
