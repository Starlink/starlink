\documentstyle{article}
\pagestyle{myheadings}

%------------------------------------------------------------------------------
\newcommand{\stardoccategory}  {ADAM Portability Project Note}
\newcommand{\stardocinitials}  {Starlink/APPN}
\newcommand{\stardocnumber}    {26.0}
\newcommand{\stardocauthors}   {B D Kelly}
\newcommand{\stardocdate}      {13 April 1994}
\newcommand{\stardoctitle}     {The Unix ADAM communication libraries}
%------------------------------------------------------------------------------

\newcommand{\stardocname}{\stardocinitials /\stardocnumber}
\markright{\stardocname}
\setlength{\textwidth}{160mm}
\setlength{\textheight}{230mm} % changed from 240
\setlength{\topmargin}{-2mm}   % changed from -5
\setlength{\oddsidemargin}{0mm}
\setlength{\evensidemargin}{0mm}
\setlength{\parindent}{0mm}
\setlength{\parskip}{\medskipamount}
\setlength{\unitlength}{1mm}

%------------------------------------------------------------------------------
% Add any \newcommand or \newenvironment commands here
%------------------------------------------------------------------------------

\begin{document}
\thispagestyle{empty}
SCIENCE \& ENGINEERING RESEARCH COUNCIL \hfill \stardocname\\
ROYAL OBSERVATORY EDINBURGH\\
{\large\bf Computing Section\\}
{\large\bf \stardoccategory\ \stardocnumber}
\begin{flushright}
\stardocauthors\\
\stardocdate
\end{flushright}
\vspace{-4mm}
\rule{\textwidth}{0.5mm}
\vspace{5mm}
\begin{center}
{\Large\bf \stardoctitle}
\end{center}
\vspace{5mm}

%------------------------------------------------------------------------------
%  Add this part if you want a table of contents
%  \setlength{\parskip}{0mm}
%  \tableofcontents
%  \setlength{\parskip}{\medskipamount}
%  \markright{\stardocname}
%------------------------------------------------------------------------------

\section {Summary}

The libraries providing the implementation of Adam intertask
communication under Unix are summarised.


\section {Introduction}

This document provides an overview of the libraries which have been
implemented to provide Adam intertask communication under Unix. The
libraries are described individually in more detail in other documents
(APPN/25 for atimer, exh and msp). Conventions common to the
implementation of the C code are also described.

The libraries are

\begin{itemize}
\item atimer - allows a library to specify a function to be called after a delay
\item exh - allows a library to specify a shutdown routine
\item msp - provides intertask communication primitives
\item ams - C-callable Adam message system
\item fams - FORTRAN-callable Adam message system
\item MESSYS - FORTRAN-callable Adam version 2 compatible message system
\end{itemize}


\section {Conventions}

The code for atimer, exh, msp and ams is all ANSI C. The source code for
each library is stored in a single file (eg exh.c) and the function
declarations are held in two include files, for example

\begin{itemize}
\item exh.h - the callable interface to exh
\item exh\_static.h - local functions declared static
\end{itemize}

These two files are generated by combining a convention about the layout
of function headers with a program called "headgen" which reads through C
source code looking for function headers and writing them to an output
file. The convention has been chosen that "/*+" at the start of a line
marks the beginning of a routine which is part of the callable interface,
whereas "/*=" marks a private routine. The declaration is terminated by
")" as the first character on a line.

An example of using headgen on exh.c would be

\begin{verbatim}
...> headgen.a exh.c exh.h "/*+"
...> headgen.a exh.c exh\_static.h "/*="
\end{verbatim}

\section {task exit handler, exh}

exh is C-callable and written in C.

The "exit handler" is actually a signal handler responding to

\begin{itemize}
\item SIGHUP
\item SIGINT
\item SIGQUIT
\item SIGKILL
\item SIGTERM
\item SIGTRAP
\item SIGABRT
\item SIGEMT
\item SIGFPE
\end{itemize}

and it is assumed that adam tasks will normally exit as a result of one of
these.

\section {timer facility, atimer}

atimer is C-callable and written in C.

The atimer application interface consists of two routines, allowing a
caller to specify an application routine to be called after a time
interval specified in milliseconds. The caller also has to specify a
timer identifier which can be used to cancel a previously initiated
timer. If the timed interval is allowed to complete, the specified
application function is called and is passed the timer identifier as its
sole argument.

Note that the given timer identifier has to be unique within the process
(cf the VMS \$setimr facility).


\section {message system primitives msp}

msp is C-callable and written in C.

msp provides the low-level adam intertask communication facility. As
such, it is intended to support the higher levels of intertask
communication which implement the adam protocols. The result is that msp
must not be called as a separate facility within adam tasks.

\section {adam message system ams}

ams is C-callable and written in C. It provides the standard Adam
communication functions to

\begin{itemize}
\item open a path to another task
\item receive a message
\item send a message
\item send a reply to a message
\item get a reply associated with a specific message identifier
\end{itemize}

\section {FORTRAN-callable adam message system fams}

fams is FORTRAN-callable and written in C using the CNF and F77
facilities (SGP/5). It is implemented as a wrap-around to ams. The arguments
to the fams routines are straightforward FORTRAN versions of the ams
arguments.

\section {V2 compatible adam message system MESSYS}

MESSYS is FORTRAN-callable and written in FORTRAN. It is implemented as a
wrap-around to fams and is intended to allow the existing Adam version 2
libraries written in FORTRAN to operate unchanged. The aim should be for
FORTRAN code to be moved to the fams interface as convenient, and subject
to fams becoming available under VMS.

The difference exists between fams and MESSYS because the opportunity has
been taken to rectify certain anomalies in the software layering implicit
in the MESSYS call interface. These anomalies arose because MESSYS was
originally conceived as purely an implementation layer inside the "ADAM\_"
library. It is still believed that no application code calls any MESSYS
routines directly, however, certain Adam libraries do.

\section {Conclusion}

The libraries summarised here have been implemented under SunOS and are
believed to be portable to Solaris and osf/1. The ams and fams libraries
provide a "clean" call interface for C and FORTRAN respectively. The
MESSYS library provides compatibility with earlier versions of the Adam
system.

\end {document}
