\chapter{\xlabel{pol2_overview}POL-2 Overview}
\label{sec:pol2}
\section{\xlabel{pol2}The instrument}

POL-2 is an 

\ref{Friberg}


The Submillimetre Common User Bolometer Array-2 (SCUBA-2) is
ancillary to POL-2. 
10,000-pixel bolometer camera. It has two arrays operating
simultaneously to map the sky in the atmospheric windows of 450 and
850$\mu$m. 


\subsection{Polarization}

euqations go here

\subsection*{Half wave plate}


\section{\xlabel{obs_modes}Observing modes}
\label{sec:mmodes}

One observing mode is offered for POL-2 \textsc{POL2DAISY}. 


\section{The raw data}
\label{sec:rawdata}
A normal science observation will follow the following sequence.

\begin{enumerate}\itemsep-0.2em
\item Flat-field
\item Science scans
\item Flat-field
\end{enumerate}

The \param{POL2INBEAM} keyword in the FITS header may be used to
identify if POL-2 is in the beam (see
\cref{Section}{sec:fitsheader}{Headers and file structure}).  When you
access raw from the \htmladdnormallink{Science
  Archive}{http://www3.cadc-ccda.hia-iha.nrc-cnrc.gc.ca/jcmt/} you
will get all of the files listed above. Later when you reduce your
data using the map-maker you must include all the science files
\emph{and} the first flat-field.  The final flat-field is not
currently used.

Shown below is a list of the raw files for a single sub-array (in this
case s8a) for a short calibration observation. The first and last
scans are the flat-field observations,which occur after the shutter
opens to the sky at the start of the observation and closes at the end
(note the identical file size); all of the scans in between are
science.


\begin{terminalv}
% ls -lh /jcmtdata/raw/scuba2/s8a/20131227/00034
\end{terminalv}

\begin{terminalv}
-rw-r--r-- 1 jcmtarch jcmt 8.0M Dec 27 03:00 s8a20131227_00034_0001.sdf
-rw-r--r-- 1 jcmtarch jcmt  22M Dec 27 03:00 s8a20131227_00034_0002.sdf
-rw-r--r-- 1 jcmtarch jcmt  22M Dec 27 03:01 s8a20131227_00034_0003.sdf
-rw-r--r-- 1 jcmtarch jcmt  22M Dec 27 03:02 s8a20131227_00034_0004.sdf
-rw-r--r-- 1 jcmtarch jcmt  22M Dec 27 03:02 s8a20131227_00034_0005.sdf
-rw-r--r-- 1 jcmtarch jcmt 6.8M Dec 27 03:02 s8a20131227_00034_0006.sdf
-rw-r--r-- 1 jcmtarch jcmt 8.0M Dec 27 03:03 s8a20131227_00034_0007.sdf
\end{terminalv}

The SCUBA-2 data acquisition (DA) system writes out a data file every
30 seconds; each of which contains 22\,MB of data. The only exception
is the final science scan which will usually be smaller (6.8\,MB in
the example above), typically requiring less than 30 seconds of data
to complete the observation.

\textbf{Note:} All of these files are written out eight times, once
for each of the eight sub-arrays.

The main data arrays of each file are cubes, with the first two
dimensions enumerating bolometer columns and rows within a sub-array,
and the third time slices (sampled at roughly 200\,Hz).

A standardised file naming scheme is used in which each file name starts
with the sub-array name, followed by the UT date of the observation in
the format \texttt{yyyymmdd}, followed by a five-digit observation
number, followed by the scub-scan number. The name ends with the standard
suffix \texttt{.sdf} used by all Starlink data files. For instance, the files
listed above hold data from the s8a sub-array for observation 34 taken on
27th December 2013.

\subsubsection*{Units}

Raw POL-2 data come in uncalibrated units. The first calibration
step is to scale the raw data to units of picowatts (pW)
by applying the flat-field solution. This step is performed internally
by the map-maker but can be done manually when examining the raw
data---see \cref{Section}{sec:concat}{Concatenate \& apply a
  flat-field}.

The second step is to scale the resulting map by the flux conversion factor
(FCF) to give units of janskys. When running the
\textsc{Orac-dr} pipeline this is done automatically.
However it is good to check that the FCF value applied to your data is sensible.
Checking FCF's must be done manually, instructions for this is
given in \cref{Section}{sec:own_fcf}{Determining your own Flux conversion
  factors}.


