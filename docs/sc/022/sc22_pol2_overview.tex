\chapter{\xlabel{pol2_overview}POL-2 Overview}
\label{sec:pol2}
\section{\xlabel{pol2}The instrument}

The POL-2 instrument is a linear polarimetry module for the
Submillimetre Common User Bolometer Array-2 (SCUBA-2), a 10,000
bolomtere camera on the JCMT \cite{Friberg}\cite{Bastien2011}.
POL-2 in itself is not a detector - thus requiring SCUBA-2
and its detectors for operation. SCUBA-2 operates
simultaneously at both 850 and \SI{450}{\micro\metre}. The POL-2
instrument is currently commissioned at \SI{850}{\micro\metre} only.


\subsection*{Polarization}

In polarimetric terms light is conventionally 
described by the four Stokes parameters: I, Q, U and V.


I is the total intensity, Q, is the radiation linearly polarized in the
direction parallel or perpendicular to the reference plane. U, is the
radiation linearly polarized in the directions 45$^{\circ }$ to the
reference plane, and V, is the circularly polarized radiation.

POL-2 is designed to charecterise linear polarisation. 
The V parameter, consequently, is not discussed further
with the focus on I, Q and U.

The linear Polarized intensity can be described as:

\begin{equation}
P = \sqrt{Q^{2}+U^{2}}
\end{equation}

where

\begin{equation}
Q = Q_{m} - I . IP_{q}
\end{equation}
\begin{equation}
U = U_{m} - I . IP_{u}
\end{equation}

where Q$_{m}$ and U$_{m}$ are the measured values of Q and U.
I is the astronomical total intensity.
IP is the instrument polarization. The IP affects both Q
(IP$_{q}$) and U (IP$_{u}$).


\subsection*{How POL-2 works}

POL-2 is located in front of the window to the SCUBA-2 instrument
(as is seen in Figure \ref{fig:pol2sc2}), and covers the full
field of view of SCUBA-2. The POL-2 polarimeter
utilises three optical components
that cover the full field of SCUBA-2:

\begin{enumerate}\itemsep-0.2em
\item a wire-grid polarizer used as a calibrator (only included in the
beam for test purposes)
\item a Half-Wave Plate
\item a second wire-grid polarizer used as an analyzer
\end{enumerate}


these components can be seen in Figure \ref{fig:pol2components}.
Rotating the HWP rotates any linearly polarised component of incoming
radiation. The rotating linearly polarised component is
transmitted or reflected by the grid, causing a modulation in the transmitted
intensity.

The radiation passing through the polarimeter is detected by
SCUBA-2. The detected intensity (I$_{detected}$) is a
combination of \emph{both} the unpolarised intensity (I$_{unpolarised}$)
and the linearly polarised intensity (P). This detected intensity can be
described by:


\begin{equation}
I_{detected} = \frac{I_{unpolarised}}{2}+ PI\cdot\left(\frac{1+\cos(4\phi)}{2} \right)
\end{equation}

where $\phi$ is the angle of the HWP.

\begin{figure}[t!]
\begin{center}
\includegraphics[width=0.45\linewidth]{pol2-out-of-beam.png}
\includegraphics[width=0.45\linewidth]{pol2-in-beam.png}
\label{fig:pol2sc2}
\caption [POL-2 mounted on SCUBA-2]{
  \small POL-2 mounted on the front of SCUBA-2.
  The left image shows the SCUBA-2
  window. The right image shows the components of POL-2 inserted
  in front of the SCUBA-2 window: the calibrator grid, rotating HWP
  and the analyzer grid. The calibrator grid is only inserted
  for test purposes.
}
\end{center}
\end{figure}




\subsection*{The Half-Wave Plate:}

As described in the POL-2 commissioning document
the HWP is constructed from five individual synthetic sapphire layers approximately 0.9
mm thick and 200 mm in diameter. The transmission properties of sapphire
are generally good at the SCUBA-2 wavelengths but are dependent on the thickness and ambient
temperature. The total effective transmission of the HWP integrated across the 850
and \SI{450}{\micro\metre} filter bands are about 86\% and 57\% respectively
(Savini et al. 2009 - insert full reference).


Rotation of the HWP results in the rotation of any linear polarised component of incoming radiation.



\begin{figure}[t!]
\begin{center}
\includegraphics[width=0.7\linewidth]{pol2-three-components.png}
\label{fig:pol2components}
\caption [POL-2 components]{
  \small The three blades that combine to form POL-2 are partially
  extended showing the two wire grids and the achromatic HWP.
  The two wire grids are the calibrator grid and the analyzer grid.
  The rotating HWP is located between these two fixed grids.
  The calibrator grid is only inserted for test purposes.
  Stiffeners can be seen on all three blades. The one for the HWP
  is particularly thick. Their purpose is to reduce vibrations while
  the HWP spins.
}
\end{center}
\end{figure}

The waveplate is typically rotated at 2Hz, providing a fast modulation of any linear polarization by 8Hz.
The data acquisition rate is ~175Hz, yielding 20 samples per cycle.



\section{Instrument Polarization}

At the angular resolution of JCMT, planets such as Uranus should appear as unpolarised point sources.
In practice, however, POL-2 observations of such sources exhibit a measurable level of polarisation - albeit typically less
than 1.5\% at 850 microns. This is evidence that some part of the incoming astronomical radiation
is being partially polarised by one or more of the components of the telescope/POL-2/SCUBA-2
that are in the light path. This polarisation is referred to as "Instrumental Polarisation" (IP).

In order to establish the true Q and U from an astronomical source, it is necessary to correct
for this effect. For the case of a low degree of polarisation in the incoming radiation and a low
degree of IP, the following is a good approximation for correcting the measurement for the effects of the IP:

\begin{equation}
Q = Q_{m} - I. ip_{q}
\end{equation}

\begin{equation}
U = U_{m} - I. ip_{u}
\end{equation}

where $Q_{m}$ and $U_{m}$ are the measured values for a single bolometer sample at
some point on the sky. Q and U are the true (corrected) values, I is the astronomical
total intensity at the same point on the sky (i.e. the total intensity after removal of
the sky and electronic backgrounds) and $ip_{q}$  and $ip_{u}$  are factors that may vary
slowly with focal plane position and/or azimuth and elevation.

IP correction of a POL-2 map therefore requires the use of a total intensity map of the same area
of the sky to be available. This total intensity map is referred to as the \emhp{IP reference map}.

Where flat mirrors or surfaces will produce a small, constant polarization across the beam,
curved mirrors and other structures (for example the secondary mirror supports) will
produce more complex polarization effects - and these may distort the beam shape.
Side-lobes can often show up with strong (typically 10-20\%) polarization but these
effects are usually far from the main-beam. Typical calculations of antenna patterns
for symmetrical Cassegrain antennas have not predicted strong polarization in the main beam.

The JCMT IP footprint is stronger than expected (though typically less than 1.5\% of
the total intensity), and has the following distinctive features:

\begin{enumerate}\itemsep-0.2em
\item The polarisation intensity is elevation dependent
\item There is ellipticity of the beam and it is elevation dependent
\item The beam is elongated in the horizontal direction.
\end{enumerate}

The dominant source of IP at the JCMT is the Goretex membrane, used as a wind blind.
This membrane introducing both losses and polarization. Due to SCUBA-2's location
on the Nasmyth platform this polarization effect is elevation dependent.


\section{\xlabel{obs_modes}Observing mode}
\label{sec:mmodes}

The standard POL-2 observing mode, POLCV\_DAISY, is a “scan and spin” mode,
in which the telescope is moving continuously in a Daisy-type pattern while the
HWP spins.

The POLCV\_DAISY scan mode is similar to the established Daisy scan mode
routinely used for non-polarimetric SCUBA-2 observations of point-like or compact
sources. However it is slightly altered to allow for a slower telescope scanning
speed.


\begin{figure}[t!]
\begin{center}
\includegraphics[width=0.9\linewidth]{scan_pattern_daisy_comparison.png}
\label{fig:scancompsrison}
\caption [Scan Pattern Comparison]{Left: Scan pattern from a typical SCUBA-2 CV_Daisy
observation. Right: Scan pattern from a POL-2 Daisy. The standard POLCV\_DAISY scan
parameters are given in Table~\ref{tab:scanpar}
  \small
}
\end{center}
\end{figure}




The telescope must scan slowly enough to obtain sufficient data at each
point on the sky to allow good $Q$ and $U$ values to be determined. The current
commissioned scan pattern has a size of 200\arcs{} and a scan speed of
8\si{\arcsecond}/s. The data reduction splits the data stream into short
segments and determines a pair of $Q$ and $U$ values from each segment.

The length of each such data segment is the time it takes the telescope to traverse
a pixel in the generated map. With the current scanning parameters this is 0.5 and
0.25 seconds for 850 and \SI{450}{\micro\metre}, respectively. The modulation
generated by any polarisation is 8 Hz at the current HWP rotation speed
(2 Hz).

The standard POLCV\_DAISY scan parameters are given in Table~\ref{tab:scanpar}.

\begin{table}[h!]
\begin{center}
\begin{threeparttable}
\begin{tabular}{r|l}
\hline
Parameter & Value\\
\hline
Half-wave plate rotation frequency& \SI{2}{Hz}\\
 Antenna scanning speed & 8\si{\arcsecond}/s\\
 $R_0$ (map pattern radius)\tnote{\textdagger}
& 133\arcs{}\\
 $R_t$ (turn radius) & 99\arcs{}\\
 $R_a$ (nominal avoidance radius) & 77\arcs\\
\hline
\end{tabular}
\begin{tablenotes}
\small \textdagger Note this radius is \emph{not} the size of the resulting map
\end{tablenotes}
\end{threeparttable}
\caption{The scan parameters used in the POLCV\_DAISY mode.\label{tab:scanpar}}
\end{center}
\end{table}


\begin{figure}[t!]
\begin{center}
\includegraphics[width=0.6\linewidth]{POLCV_DAISY_schematic_detailed.png}
\label{fig:scandetail}
\caption [Detail of POL-2 Scan Pattern]{Detail of POLCV\_DAISY. $R_{0}$ is the map pattern radius, $R_{t}$ the turn radius, and $R_{a}$ is the nominal avoidance radius. For more details see Table \ref{tab:scanpar}.
  \small
}
\end{center}
\end{figure}


\section{The raw data}
\label{sec:rawdata}
SCUBA-2 is the detector for POL-2 and as such the raw data format of POL-2 data is
the same as a typical SCUBA-2 observation. The sequence for both observations is:

\begin{enumerate}\itemsep-0.2em
\item Dark noise
\item Flat-field
\item Science scans
\item Flat-field
\end{enumerate}

<<<<<<< 8ee5bbbfbe34a3a8fc3d0b20aa99041d1b6923b7
The \param{SEQ\_TYPE} keyword in the FITS header may be used to identify the nature of each scan.
When you access raw data from the \htmladdnormallink{Science
=======
The \param{SEQ\_TYPE}\ keyword in the FITS header may be used to identify the nature of each scan (see
\cref{Section}{sec:fitsheader}{Headers and file structure}).  When you
access raw from the \htmladdnormallink{Science
>>>>>>> SC/22: Escape underscore to prevent most of the text appearing in tt.
  Archive}{http://www3.cadc-ccda.hia-iha.nrc-cnrc.gc.ca/jcmt/} you
will get all of the files listed above.


Critically the \param{INBEAM} keyword in the FITS header may be used to
identify if POL-2 is in the beam, and hence differentiate between SCUBA-2
and POL-2 observations.


\begin{tip}
Use the  \Kappa\ command fitslist to see all FITS headers in a particular
NDF. To obtain a specific header simply use the command fitsval:
\begin{terminalv}
% fitsval s8a20160112_00056_0001.sdf INBEAM
pol
\end{terminalv}
\end{tip}

Shown below is an incomplete list of the raw files for a single sub-array (in this
case s8a) for a short POL-2 observation. The first and last
scans are the flat-field observations,which occur after the shutter
opens to the sky at the start of the observation and closes at the end
(note the identical file size); all of the scans in between are
science.


\begin{terminalv}
% ls -lh /jcmtdata/raw/scuba2/s8a/20160112/00056
\end{terminalv}

\begin{terminalv}
-rw-r--r-- 1 jcmtarch jcmt 5.6M Jan 12  2016 s8a20160112_00056_0001.sdf
-rw-r--r-- 1 jcmtarch jcmt 7.9M Jan 12  2016 s8a20160112_00056_0002.sdf
-rw-r--r-- 1 jcmtarch jcmt  25M Jan 12  2016 s8a20160112_00056_0003.sdf
-rw-r--r-- 1 jcmtarch jcmt  25M Jan 12  2016 s8a20160112_00056_0004.sdf
-rw-r--r-- 1 jcmtarch jcmt  25M Jan 12  2016 s8a20160112_00056_0005.sdf
...
-rw-r--r-- 1 jcmtarch jcmt  25M Jan 12  2016 s8a20160112_00056_0025.sdf
-rw-r--r-- 1 jcmtarch jcmt  25M Jan 12  2016 s8a20160112_00056_0026.sdf
-rw-r--r-- 1 jcmtarch jcmt  25M Jan 12  2016 s8a20160112_00056_0027.sdf
-rw-r--r-- 1 jcmtarch jcmt  22M Jan 12  2016 s8a20160112_00056_0028.sdf
-rw-r--r-- 1 jcmtarch jcmt 7.9M Jan 12  2016 s8a20160112_00056_0029.sdf
\end{terminalv}

The SCUBA-2 data acquisition (DA) system writes out a data file every
30 seconds; each of which contains 22\,MB of data. The only exception
is the final science scan which will usually be smaller (6.8\,MB in
the example above), typically requiring less than 30 seconds of data
to complete the observation.

\textbf{Note:} All of these files are written out eight times, once
for each of the eight sub-arrays. It should also be noted that the POL-2 instrument has not
been released from commissioning at 450 microns.

The main data array in each NDF is a cube, with the first two
dimensions corresponding to bolometer columns and rows within a sub-array,
and the third dimension corresponding to time slice index (sampled at roughly 200\,Hz).

A standardised file naming scheme is used in which each file name starts
with the sub-array name, followed by the UT date of the observation in
the format \texttt{yyyymmdd}, followed by a five-digit observation
number, followed by the scub-scan number. The name ends with the standard
suffix \texttt{.sdf} used by all Starlink NDF data files. For instance, the files
listed above hold data from the s8a sub-array for observation 34 taken on
12th January 2016.




\subsubsection*{Units/Calibration}

Raw POL-2 data come in uncalibrated units. The first calibration
step is to scale the raw data to units of picowatts (pW)
by applying the flat-field solution. This step is performed internally
by the SMURF command calcqu - used to calculate I, Q and U time-streams from
the raw data - but can be done manually when examining the raw
data.

If the purpose of a POL-2 observation is to measure the percentage
polarizations or vector angles within a source/region of interest then
the data may remain in pW. On the other hand, if the purpose is to
measure the absolute polarized intensities then a value for the flux
conversion factor (FCF) is required.

The resulting map may have the FCF applied to give units of janskys. As is
recommended with SCUBA-2 observing it is good to check that the FCF value
applied to your data is sensible (and must be done manually).





