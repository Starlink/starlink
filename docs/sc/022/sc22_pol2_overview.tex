\chapter{\xlabel{pol2_overview}POL-2 Overview}
\label{sec:pol2}
\section{\xlabel{pol2}The instrument}

The POL-2 instrument is a linear polarimiter for the 
Submillimetre Common User Bolometer Array-2 (SCUBA-2), a 10,000 
bolomtere camera on the JCMT \cite{Friberg}\cite{Bastien2011}. SCUBA-2 operates
simultaneously at both 850 and \SI{450}{\micro\metre}. The POL-2
instrument is currently commissioned at \SI{850}{\micro\metre} only.


\subsection*{Polarization}

A quick flick through any text book on optics will yield a section
on the scattering of light and its description. Conventionally light
is described in terms of the four Stokes parameters: I, Q, U and V.

%\begin{equation}
%\left( \begin{array}{c}
%I\\
%Q \\
%U\\
%V  \end{array} \right)
%\end{equation}

I is the total intensity, Q, is the radiation linearly polarized in the
direction parallel or perpendicular to the reference plane. U, is the
radiation linearly polarized in the directions 45$^{\circ }$ to the
reference plane, and V, is the circularly polarized radiation. 

POL-2 only handles linear polrization. As a result we do not use
the V parameter in this document from this point and only focus
on I, Q and U.

The linear Polarized intensity can be described as:

\begin{equation}
PI = \sqrt{Q^{2}+U^{2}}
\end{equation}

where 

\begin{equation}
Q = Q_{m} - I . IP_{q} 
\end{equation}
\begin{equation}
U = U_{m} - I . IP_{u}
\end{equation}

where Q$_{m}$ and U$_{m}$ are the measured values of Q and U.
I is the astronomical total intensity.
IP is the instrument polarization. The IP affects both Q 
(IP$_{q}$) and U (IP$_{u}$).


\subsection*{How POL-2 works}

POL-2 sits in front of the window to the SCUBA-2 instrument
as is seen in Figure \ref{fig:pol2sc2}. POL-2 covers the full
field of view of SCUBA-2.

POL-2 in itself is not a detector - thus requiring SCUBA-2 
and its detectors for operation. The POL-2 polarimeter
utilizes three optical components
that cover the full field of SCUBA-2: 

\begin{enumerate}\itemsep-0.2em
\item a wire-grid polarizer used as calibrator
\item Half-Wave Plate
\item a second wire-grid polarizer used as an analyzer
\end{enumerate}


these components can be seen in Figure \ref{fig:pol2components}.
Rotating the HWP rotates any linear polarised component of incoming
radiation. The rotating linear polarised component is
transmitted or reflected by the grid causing a modulation in the transmitted
intensity. 

The total radiation passing through the polarimeter is detected by
SCUBA-2. The polarimeter detects an intensity (I$_{detected}$) that is a
combination of \emph{both} the unploarized intensity (I$_{unpolarised}$)
and the linear polarised intensity (PI). This detected intensity can be
described by:


\begin{equation}
I_{detected} = \frac{I_{unpolarised}}{2}+ PI\cdot\left(\frac{1+\cos(4\phi)}{2} \right)
\end{equation}

where $\phi$ is the angle of the HWP.

\begin{figure}[t!]
\begin{center}
\includegraphics[width=0.4\linewidth]{pol2-out-of-beam.png}
\includegraphics[width=0.4\linewidth]{pol2-in-beam.png}
\label{fig:pol2sc2}
\caption [POL-2 mounted on SCUBA-2]{
  \small POL-2 mounted on the front of SCUBA-2. 
  The left image shows the SCUBA-2
  window. The right image shows the components of POL-2 inserted
  in front of the SCUBA-2 window: the calibrator grid, rotating HWP
  and the analyzer grid. The calibrator grid is only inserted
  for test purposes.
}
\end{center}
\end{figure}




\subsection*{The Half Wave Plate:}

As described in the POL-2 commissioning document 
the HWP is made out of five individual sapphire layers approximately 0.9
mm thick and 200 mm in diameter. The transmission of sapphire is good at
\SI{850}{\micro\metre} and absorbs somewhat at \SI{450}{\micro\metre}, depending on the thickness and ambient
temperature. The total transmission of the HWP integrated across the 850
and \SI{450}{\micro\metre} filter bands are about 86\% and 57\% respectively 
(Savini et al. 2009 - insert full reference).




Rotating the HWP rotates any linear polarised component of incoming radiation.






\begin{figure}[t!]
\begin{center}
\includegraphics[width=0.6\linewidth]{pol2-three-components.png}
\label{fig:pol2components}
\caption [POL-2 components]{
  \small The three blades that combine to form POL-2 are partially
  extended showing the two wire grids and the achromatic HWP.
  The two wire grids are the calibrator grid and the analyzer grid
  the rotating HWP is located between these two fixed grids.
  The calibrator grid is only inserted for test purposes.
  Stiffeners can be seen on all three blades. The one for the HWP 
  is particularly thick. Their purpose is to reduce vibrations while
  the HWP spins.
}
\end{center}
\end{figure}





\section{Instrument Polarization}

\begin{enumerate}\itemsep-0.2em
\item The polarisation intensity is elevation dependent
\item There is ellipticity of the beam and it is elevation dependent
\item The beam is elongated in the horizontal direction.
\end{enumerate}


\section{\xlabel{obs_modes}Observing mode}
\label{sec:mmodes}

The standard POL-2 observing mode, POLCV\_DAISY, is a “scan and spin” mode,
in which the telescope is moving continuously in a Daisy-type pattern while the
HWP spins.

The POLCV\_DAISY scan mode is similar to the established Daisy scan mode
routinely used for non-polarimetric SCUBA-2 observations of point-like or compact
sources. However it is slightly altered to allow for a slower telescope scanning
speed.


\begin{figure}[t!]
\begin{center}
\includegraphics[width=0.6\linewidth]{scan_pattern_daisy_comparison.png}
\label{fig:scancompsrison}
\caption [Scan Pattern Comparison]{
  \small
}
\end{center}
\end{figure}




The telescope must scan slowly enough to obtain sufficient data at each 
point on the sky that good $Q$ and $U$ values may be determined. The current 
commissioned scan pattern has a size of 200\arcs{} and a scan speed of 
8\si{\arcsecond}/s. The data reduction determines each pair of $Q$ and $U$ 
segments of the data stream. 


The length of the data segment is the time it takes the telescope to traverse
a pixel in the generated map. With the current scanning parameters this is 0.5 and 
0.25 seconds for 850 and \SI{450}{\micro\metre}, respectively. The modulation 
generated by any polarisation is 8 Hz at the current wave plate rotation speed 
(2 Hz). 

The standard POLCV\_DAISY scan parameters are given in Table~\ref{tab:scanpar}.

\begin{table}[h!]
\begin{center}
\begin{threeparttable}
\begin{tabular}{r|l}
\hline
Parameter & Value\\
\hline
Half-wave plate rotation frequency& \SI{2}{Hz}\\
 Antenna scanning speed & 8\si{\arcsecond}/s\\
 $R_0$ (map pattern radius)\tnote{\textdagger}
& 133\arcs{}\\
 $R_t$ (turn radius) & 99\arcs{}\\
 $R_a$ (nominal avoidance radius) & 77\arcs\\
\hline
\end{tabular}
\begin{tablenotes}
\small \textdagger Note this radius is \emph{not} the size of the resulting map
\end{tablenotes}
\end{threeparttable}
\caption{The scan parameters used in the POLCV\_DAISY mode.\label{tab:scanpar}}
\end{center}
\end{table}


\begin{figure}[t!]
\begin{center}
\includegraphics[width=0.6\linewidth]{POLCV_DAISY_schematic_detailed.png}
\label{fig:scandetail}
\caption [Detail of POL-2 Scan Pattern]{
  \small
}
\end{center}
\end{figure}


\section{The raw data}
\label{sec:rawdata}
A normal science observation will follow the following sequence.

\begin{enumerate}\itemsep-0.2em
\item Flat-field
\item Science scans
\item Flat-field
\end{enumerate}

The \param{POL2INBEAM} keyword in the FITS header may be used to
identify if POL-2 is in the beam (see
\cref{Section}{sec:fitsheader}{Headers and file structure}).  When you
access raw from the \htmladdnormallink{Science
  Archive}{http://www3.cadc-ccda.hia-iha.nrc-cnrc.gc.ca/jcmt/} you
will get all of the files listed above. Later when you reduce your
data using the map-maker you must include all the science files
\emph{and} the first flat-field.  The final flat-field is not
currently used.

Shown below is a list of the raw files for a single sub-array (in this
case s8a) for a short calibration observation. The first and last
scans are the flat-field observations,which occur after the shutter
opens to the sky at the start of the observation and closes at the end
(note the identical file size); all of the scans in between are
science.


\begin{terminalv}
% ls -lh /jcmtdata/raw/scuba2/s8a/20131227/00034
\end{terminalv}

\begin{terminalv}
-rw-r--r-- 1 jcmtarch jcmt 8.0M Dec 27 03:00 s8a20131227_00034_0001.sdf
-rw-r--r-- 1 jcmtarch jcmt  22M Dec 27 03:00 s8a20131227_00034_0002.sdf
-rw-r--r-- 1 jcmtarch jcmt  22M Dec 27 03:01 s8a20131227_00034_0003.sdf
-rw-r--r-- 1 jcmtarch jcmt  22M Dec 27 03:02 s8a20131227_00034_0004.sdf
-rw-r--r-- 1 jcmtarch jcmt  22M Dec 27 03:02 s8a20131227_00034_0005.sdf
-rw-r--r-- 1 jcmtarch jcmt 6.8M Dec 27 03:02 s8a20131227_00034_0006.sdf
-rw-r--r-- 1 jcmtarch jcmt 8.0M Dec 27 03:03 s8a20131227_00034_0007.sdf
\end{terminalv}

The SCUBA-2 data acquisition (DA) system writes out a data file every
30 seconds; each of which contains 22\,MB of data. The only exception
is the final science scan which will usually be smaller (6.8\,MB in
the example above), typically requiring less than 30 seconds of data
to complete the observation.

\textbf{Note:} All of these files are written out eight times, once
for each of the eight sub-arrays.

The main data arrays of each file are cubes, with the first two
dimensions enumerating bolometer columns and rows within a sub-array,
and the third time slices (sampled at roughly 200\,Hz).

A standardised file naming scheme is used in which each file name starts
with the sub-array name, followed by the UT date of the observation in
the format \texttt{yyyymmdd}, followed by a five-digit observation
number, followed by the scub-scan number. The name ends with the standard
suffix \texttt{.sdf} used by all Starlink data files. For instance, the files
listed above hold data from the s8a sub-array for observation 34 taken on
27th December 2013.




\subsubsection*{Units/Calibration}

Raw POL-2 data come in uncalibrated units. The first calibration
step is to scale the raw data to units of picowatts (pW)
by applying the flat-field solution. This step is performed internally
by the map-maker but can be done manually when examining the raw
data.

If the desired output of a POL-2 observation is to get a handle on the relative
polarized fraction of a source/region of interest then the data may remain in
pW.

If the desired output of a PO-2 observation is the absolute polarized intensity
of a region then a handle on the flux conversion factor (FCF) is required.

The resulting map may have the FCF applied to give units of janskys. As is 
recommended with SCUBA-2 observing it is good to check that the FCF value 
applied to your data is sensible (and must be done manually).

