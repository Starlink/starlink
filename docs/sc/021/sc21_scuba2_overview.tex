\chapter{\xlabel{scuba2_overview}SCUBA-2 Overview}
\label{sec:s2}
\section{\xlabel{scuba2}The instrument}


The Submillimetre Common User Bolometer Array-2 (SCUBA-2) is a
10,000-pixel bolometer camera. It has two arrays operating
simultaneously to map the sky in the atmospheric windows of 450 and
850$\mu$m.

\subsubsection*{How it works}
The SCUBA-2 bolometers are integrated arrays of superconducting
transition edge sensors (TESs) with a characteristic transition
temperature, $T_c$. In addition, each TES is ringed with a resistive
heater which can compensate for changes in sky power. The SCUBA-2
focal plane is kept at a base temperature slightly below $T_c$,
however a voltage is applied across each TES resistance to position
the bolometer at the transition temperature. From this point, any
increase of temperature on the bolometers (e.g. from an astronomical
signal) will increase the TES resistance and heat it up. This causes a
drop in current and therefore a drop in temperature making the system
self-regulating.

For properly performing bolometers, the change in current through the
TES is proportional to the change in resistance, with the response
calibrated using flat-field observations (described below). This
changing current generates a magnetic field which is amplified by a
chain of superconducting quantum interference devices (SQUIDs). This
induces a feedback current which is proportional to the current
flowing through the TES, and it is this feedback current that is
recorded during data acquisition.


\subsubsection*{Setups}

Before science data can be taken the system must be optimised. These
`setups' are performed after slewing to the azimuth of the source,
where the SQUID, TES and heater biases are set to pre-determined
nominal values, in order to position the bolometers in the middle of
the transition range.

\subsubsection*{Flat-field}

 The shutter then opens onto the sky, and
as it does so the gradual increase in sky power hitting the array is
compensated for by a decrease in the resistive heater power via a
servo loop designed to keep the TES output constant. This acts to keep
the bolometers positioned at the centre of the transition range and is
known as \textbf{heater tracking}.

The responsivity of the bolometers will change slightly between the
dark and the sky; therefore, once the shutter is fully open a fast
\textbf{flat-field} observation is carried out to recalibrate them.
\textbf{A flat-field measures the responsivity of each bolometer to
changing sky power}. It does this by utilising the resistance heaters
which are ramped up and down around the nominal value. The change in
current through the TES is then recorded for each bolometer giving a
measure of its responsivity. The flat field solution is then the
inverse linear gradient of the current as a function of heater power.

At this point bolometers with responsivities above or below a
threshold limit are rejected, along with bolometers that display a
non-linear response or have a poor S/N. A second flat-field is
performed at the end of an observation so bolometers whose
responsivity has changed over the course of the observation can be
flagged.

For full details of the array setup and operation see Holland et al.
(2013) \cite{s2main}.

\begin{figure}[t!]
\begin{center}
\includegraphics[width=0.8\linewidth]{sc21_arrays}
\label{fig:arrays}
\caption[The physical layout of the arrays at each wavelength]{
  \small The layout of the arrays at 850$\mu$m (left) and
  450$\mu$m (right). The labels denote the name assigned to each
  sub-array. Raw data files are generated separately for each sub-array
  and must be co-added. This figure was made by running
 \wcsmosaic\ on a raw sub-scan from each sub-array.
}
\end{center}
\end{figure}

\section{\xlabel{obs_modes}Observing modes}
\label{sec:mmodes}

Two observing modes are offered for SCUBA-2---\textsc{daisy} and
\textsc{pong}. As the bulk of \mbox{SCUBA-2} observing involves
large area mapping, both observing modes are scan patterns. Your
choice depends on the size of the area you wish to map, where you
would like your integration time concentrated and the degree of
extended emission you wish to recover.


\begin{aligndesc}

\item[\textbf{PONG}] A \textsc{pong} map is the scan strategy for
  covering a large area. The default options allow for three
  sizes---900\,arcsec, 1800\,arcsec and 3600\,arcsec. A single
  \textsc{pong} map is a square of these dimensions and the telescope
  fills in the square by bouncing off the edge of the area. To ensure
  an even sky background it is recommended a minimum of three, but
  preferably more than five, \textsc{pong} maps are included in a
  single observation with a rotation introduced between each one. In
  this way a circular pattern is built up, (see the lower right-hand
  panel of \cref{Figure}{fig:scan}{graphic below}), with a diameter
  equal to your requested map size.

  To recover large-scale extended structure you are advised to use
  larger \textsc{pong} maps which scan at a higher rate. This option
  is preferable to tiling multiple smaller maps. Ultimately it is the
  size of the SCUBA-2 field-of-view that determines the sensitivity to
  large-scale structure.

\item[\textbf{DAISY}] \textsc{daisy} maps are the option for
  point-like or compact sources ($<$3~arcmin) by maximising the
  exposure time on the centre of the image. The telescope moves at a
  constant velocity in a `spirograph' pattern that has the advantage
  of keeping the source on the array throughout the observation. This
  is shown in the top panel of \cref{Figure}{fig:scan}{the figure
    below}.

\end{aligndesc}

\subsubsection*{Why these patterns?}

SCUBA-2 removes atmospheric noise in the data-processing
stage (Holland et al. 2013) \cite{s2main}. The power spectrum
of data taken by SCUBA-2 has a $1/f$ noise curve at lower frequencies. To
ensure astronomical signals are far away from this $1/f$ noise, fast
scanning speeds are required.

In order to disentangle source structure from other
slowly varying signals (e.g. extinction, sky noise, $1/f$ noise), the
scan pattern must pass across each region of the map from different
directions and at different times. The scan patterns themselves, along
with the associated parameters (velocity and scan-spacing), have been
designed and optimised to meet both these criteria.

\starfig{sc21_wayne_scan}{[t!]}{width=0.98\linewidth}{fig:scan}{%
  Illustration of the SCUBA-2 observing patterns}{%
  From left: telescope track over a single rotation of the pattern;
  telescope track after multiple rotations of the pattern; resulting
  exposure time map. The scan pattern for your observation can be
  visualised in this way with \topcat\ using the output from
  \jcmtstate.  See \cref{section}{sec:scan}{Displaying scan patterns} for
  more details.  Figure modified from Holland et al. (2013).
}

