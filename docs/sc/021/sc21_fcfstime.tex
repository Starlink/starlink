\chapter{\xlabel{fcfstime}FCFs by Time of Night}
\label{app:fcfstime}

Figure \ref{fig:FCFsTimeOfNight} shows the Beam (Peak) FCFs at 
450 and 850 microns as a function of UT time for observations of the 
primary calibrator, Uranus along with secondary calibrators CRL 2688, and CRL 618.  
The Peak FCF is larger in the evening and morning primarily because thermal 
deformations of the dish dilute the flux from the main beam into the secondary 
(error) component (See Mairs et al 2021~\cite{mairs21}). There is no significant 
change to the arcsecond FCFs in the early evening or late morning.

\begin{figure}
\begin{center}
\includegraphics[width=0.47\linewidth]{sc21-FCFsTimeOfNight-450} \hspace{0.02\linewidth}
\includegraphics[width=0.47\linewidth]{sc21-FCFsTimeOfNight-850}
\caption[FCFs Time of Night]{The Normalized Peak FCFs at 450 (left) and 850 microns (right) 
as a function of observation time. All FCFs are derived using the primary calibrator Uranus and 
secondary calibrator CRL 2688. The vertical lines mark the beginning and end of the “stable” observation 
period from 07:00–17:00 (UTC). The horizontal (dotted) lines show the FCF uncertainties derived for the 
stable observation period around a value of 1.0 (horizontal, solid line). 
Data are coloured according to season. \label{fig:FCFsTimeOfNight}}
\end{center}
\end{figure}

Figure \ref{fig:FCFsTimeOfNightFits} show the Peak FCF trends in detail for evening and morning 
observations of Uranus and CRL 2688. The data are bootstrap-fitted with linear functions and “rs” 
indicates the Spearman-Rank correlation of the fit.

\begin{figure}
\begin{center}
\includegraphics[width=0.47\linewidth]{sc21-FCFsTimeOfNightFitsE-450} \hspace{0.02\linewidth}
\includegraphics[width=0.47\linewidth]{sc21-FCFsTimeOfNightFitsE-850}\\
\includegraphics[width=0.47\linewidth]{sc21-FCFsTimeOfNightFitsM-450} \hspace{0.02\linewidth}
\includegraphics[width=0.47\linewidth]{sc21-FCFsTimeOfNightFitsM-850}
\caption[FCFs Time of Night Fits]{Linear fits to the Peak FCFs at 450 microns (left) and 850 microns (right)
 during the evening (top) and morning (bottom) when dish distortions caused by thermal gradients affect the 
 beam shape. The distributions during the stable part of the night (07:00-17:00 UTC) show no significant 
 slope and therefore require no corrections to the default FCFs shown in Appendix \ref{app:fcfs}. \label{fig:FCFsTimeOfNightFits}}
\end{center}
\end{figure}


The Peak FCFs DECREASE in the evening as the ambient temperature cools and the dish settles while 
the Peak FCFs INCREASE in the morning as the ambient temperature warms and the dish becomes 
unstable to thermal gradients. Table \ref{tab:FCFsTimeOfNight} summarises the rate of change. 
As of Starlink release 2021A, ORAC-DR \textbf{does not} apply these corrections by default. If
you wish to apply these corrections, the FCFs must be modified manually (see Section \ref{subsec:ApplyingFCF}).


\begin{table}[h!]
\begin{center}
\begin{tabular}{|c|c|c|}
 \hline
 \multicolumn{1}{|c|}{Wavelength} &
 \multicolumn{1}{c|}{Time Range (UTC)} &
 \multicolumn{1}{c|}{Peak FCF Correction (\% hr$^{-1}$)}
 \\ \hline
$450~\mu$m & 03:00-07:00 & $9.1\pm0.5$ \\
$450~\mu$m & 17:00-22:00 & $7.2\pm0.6$ \\
\hline
$850~\mu$m & 03:00-07:00 & $3.2\pm0.1$ \\
$850~\mu$m & 17:00-22:00 & $3.1\pm0.1$ \\ \hline
\end{tabular}
\end{center}
\label{tab:FCFsTimeOfNight}
\end{table}

