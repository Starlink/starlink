\chapter{\xlabel{matchedfilter}SCUBA-2 Matched Filter}
\label{app:mf}

In order to optimally find sources that are the size of the telescope
beam, and suppress this residual large-scale noise, the \picard\
recipe \drrecipe{SCUBA2\_MATCHED\_FILTER} may be used. If there were
no large-scale noise in the map, the filtered signal map would be
calculated as follows:

\begin{equation}
  {\cal{M}} = \frac{[M(x,y)/\sigma^2(x,y)] \otimes P(x,y)}
  {[1/\sigma^2(x,y)] \otimes [P^2(x,y)]},
\end{equation}

where $M(x,y)$ and $\sigma(x,y)$ are the signal and RMS noise maps
respectively produced by \smurf, and $P(x,y)$ is a map of the
PSF. Here \(\otimes\)
denotes the 2-dimensional cross-correlation operator. Similarly, the
variance map would be calculated as

\begin{equation}
  {\cal{N}}^2 = \frac{1}{[1/\sigma^2(x,y)] \otimes [P^2(x,y)]}.
\end{equation}

This operation is equivalent to calculating the maximum-likelihood fit
of the PSF centered over every pixel in the map, taking into account
the noise. Presently $P(x,y)$ is simply modelled as an ideal Gaussian
with a FWHM set to the diffraction limit of the telescope.

However, since there is large-scale (and therefore correlated from
pixel to pixel) noise, the recipe also has an additional step. It
first smooths the map by cross-correlating with a larger Gaussian
kernel to estimate the background, and then subtracts it from the
image. The same operation is also applied to the PSF to estimate the
effective shape of a point-source in this background-subtracted map.

Before running \textsc{Picard}, a simple parameters file called
\file{smooth.ini} may be created.
\begin{terminalv}
[SCUBA2_MATCHED_FILTER]
SMOOTH_FWHM = 15
\end{terminalv}
%
where \param{SMOOTH\_FWHM~=~15} indicates that the background should
be estimated by first smoothing the map and PSF with a 15~arcsec FWHM
Gaussian. The recipe is then executed as follows:
%
\begin{terminalv}
% picard -recpars smooth.ini SCUBA2_MATCHED_FILTER map.sdf
\end{terminalv}
%
The output of this operation is a smoothed image called
\file{map\_mf.sdf}. By default, the recipe automatically normalizes
the output such that the peak flux densities of point sources are
conserved. Note that the accuracy of this normalization depends on how
closely the real PSF matches the 7.5~arcsec and 14~arcsec full-width
at half-maximum (FWHM) Gaussian shapes assumed at 450$\mu$m and
850$\mu$m, respectively (an explicit PSF can also be supplied using
the \param{PSF\_MATCHFILTER} recipe parameter).



