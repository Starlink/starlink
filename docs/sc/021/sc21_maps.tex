\chapter{\xlabel{maps}Reducing Your Data}
\label{sec:maps}

This chapter describes how to run the map-maker and what to look out
for during processing. We also discuss reducing your data using the
\oracdr\ science pipeline and why you might want to chose this option.

As discussed in \cref{Chapter}{sec:dimm}{The
Dynamic Iterative Map-Maker}, all of the settings for the map-maker
are stored in configuration files. In this chapter we use
the default configuration file, \file{dimmconfig.lis} to reduce CRL~2688,
one of SCUBA-2's secondary calibrators. For an
overview of the specialised configuration files available see
\cref{Section}{sec:config}{this section}.



\section{\xlabel{sciencepl}Using the science pipeline}

You can also reduce your data using the \oracdr\ science pipeline on a
local computer. There are advantages to running the map-maker using
the pipeline. You can feed the pipeline observations of multiple
sources rather than feed in a single source at a time. The pipeline
will recognise the different sources and make a separate map for each,
whereas the map-maker would make a single large map that would include
all your sources (no matter how widely spaced!).


The pipeline will produce calibrated maps; by default these are
calibrated using the standard FCFs, although you can specify your
own if you wish.

Running the science pipeline is very straightforward and can be as
simple as the example below. Here a list is made of all your raw data,
the pipeline is then initiated, finally the reduction is started with
instructions to loop through all data listed in the supplied file and
the filename in question is given.

\begin{terminalv}
% ls s8*.sdf > myfiles.lis
% oracdr_scuba2_850
% oracdr -loop file -files myfiles.lis
\end{terminalv}

\textbf{For a more in-depth discussion on running the pipeline and a
  discussion of the various outputs see \cref{Chapter}{sec:pipe}{The
    SCUBA-2 Pipeline} or \pipelinesun.}



