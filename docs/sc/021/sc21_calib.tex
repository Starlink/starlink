\chapter{\xlabel{calib}SCUBA-2 data calibration}
\label{app:cal}

\section{\xlabel{fcf}Flux conversion factors (FCF)}
\label{app:fcf}

Primary and secondary calibrator observations have been reduced using
the specifically designed \file{dimmconfig\_bright\_compact.lis}.  The
maps produced from this are then analysed using tailor-made \picard\
recipes. For instructions on applying the FCFs to your map see
\cref{Section}{sec:cmult}{this page}.

A map reduced by the map-maker has units of pW. To calibrate the data
into units of janskys (Jy), a set of bright, point-source objects with
well-known flux densities are observed regularly to provide a flux
conversion factor (FCF). The data (in pW) can be multiplied by this
FCF to obtain a calibrated map. The FCF can also be used to assess the
relative performance of the instrument from night to night. The noise
equivalent flux density (NEFD) is a measure of the instrument
sensitivity, and while not discussed here, is also produced by the
\textsc{Picard} recipe shown here. For calibration of primary and
secondary calibrators, the FCFs and NEFDs have been calculated as
follows:

\begin{enumerate}
\item The \textsc{Picard} recipe \drrecipe{SCUBA2\_FCFNEFD} takes the
  reduced map, crops it, and runs background removal. Surface-fitting
  parameters are changeable in the \textsc{Picard} parameter file.

\item It then runs the \Kappa\ \beamfit\ task on the specified point
  source. The \task{beamfit} task will estimate the peak
  (uncalibrated) flux density and the FWHM. The integrated flux
  density within a given aperture (30-arcsec radius default) is
  calculated using \photom\ \autophotom. Flux densities for
  calibrators such as Uranus, Mars, CRL~618, CRL~2688 and HL~Tau are
  already known to \picard. To derive an FCF for other sources of
  known flux densities, the fluxes can be added to the parameter file
  with the source name (in upper case, spaces
  removed): \param{FLUX\_450.MYSRC~=~0.050}
  and \param{FLUX\_850.MYSRC~=~0.005} (where the values are in Jy),
  for example.

\item Three different FCF values are calculated, two of which are
  described below.

  \begin{enumerate}

  \item \textbf{The arcsecond FCF}
    \begin{equation}
      \label{eq:fcf_arcsec}
      \mathrm{FCF_{arcsec}} = \frac{S_{\mathrm{tot}}}{P_{\mathrm{int}} \times
        A_{\mathrm{pix}}}
    \end{equation}

    where $S_{\mathrm{tot}}$ is the total flux density of the
    calibrator, $P_{\mathrm{int}}$ is the integrated sum of the source
    in the map (in pW) and $A_{\mathrm{pix}}$ is the pixel area in
    arcsec$^2$, producing an FCF in Jy/arcsec$^2$/pW.

  \item\textbf{The beam FCF}
    \begin{equation}
      \label{eq:fcf_beam}
      \mathrm{FCF_{beam}} = \frac{S_{\mathrm{peak}}}{P_{\mathrm{peak}}}
    \end{equation}
    producing an FCF in units of Jy/beam/pW.
  \end{enumerate}

\end{enumerate}

The measured peak signal here is derived from the Gaussian fit of
\task{beamfit}. The peak value is susceptible to pointing and focus
errors, and we have found this number to be somewhat unreliable,
particularly at 450$\mu$m.


\section{\xlabel{extinction}Extinction correction}

Analysis of the SCUBA-2 secondary calibrators has allowed calculation
of the transmission relationships for the SCUBA-2 450\,$\mu$m and
850\,$\mu$m pass-bands to be determined. Full details of the analysis
and on-sky calibration methods of SCUBA-2 can be found in Dempsey et
al.\ (2013)~\cite{dempsey12}\cite{dempsey-spie}.

Archibald et al. (2002)\,\cite{archibald} describes how the Caltech
Submillimeter Observatory (CSO) 225\,GHz opacity, $\tau_{225}$,
relates to SCUBA opacity terms in each band, $\tau_{450}$ and
$\tau_{850}$. The JCMT water-vapour radiometer (WVM) uses the 183\,GHz
water line to calculate the precipitable water vapour (PWV) along the
line-of-sight of the telescope. This PWV is then input into an
atmospheric model to calculate the zenith opacity at 225\,GHz
($\tau_{225}$). This allows ease of comparison with the adjacent CSO
225\,GHz tipping radiometer. The opacities have been as:

\begin{equation}
\tau_{450} = 26.0 \times (\tau_{225} - 0.012);
\end{equation}
and
\begin{equation}
\tau_{850} = 4.6 \times (\tau_{225} - 0.0043).
\end{equation}

The SCUBA-2 filter characteristics are described in detail
\htmladdnormallinkfoot{on the JCMT
  website}{http://www.eaobservatory.org/jcmt/instrumentation/continuum/scuba-2/filters/}.

The extinction correction parameters that scale from $\tau_{225}$ to
the relevant filter have been added to the map-maker code. You can
override these values by setting \param{ext.taurelation.filtname} in
your map-maker config files to the two coefficients `($a$,$b$)' that
you want to use (where \param{filtname} is the name of the
filter). The defaults are listed in
\file{\$SMURF\_DIR/smurf\_extinction.def}.

% It is worth noting that if an individual science map and
% corresponding calibrator observation has already been reduced with
% the old factors (and your source and calibrator are at about the
% same airmass and if the tau did not change appreciably), any errors
% in extinction correction should cancel out in the calibration.


