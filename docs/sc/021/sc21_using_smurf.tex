\chapter{\xlabel{manual}Running \task{makemap} Outside the Pipeline}
\label{sec:manual}

The previous chapter described how to make maps using the SCUBA-2 Pipeline.
Users - particularly new users - are encourage to use the pipeline for
their map-making. However, greater control of the map-making
process is available by running the \makemap\ command directly, rather
than from within the pipeline. This chapter describes how to do this, but
should be seen as ``advanced usage''.

\begin{quote}
\emph{
Note, when \makemap\ is run manually (rather than from the pipeline) the
resulting image will be in units of pW, not Jy. Each pixel value
in the map represents the weighted mean value of the bolometer samples
(in pW) that fall within the pixel, after removal of the noise components
described in \cref{Section}{sec:models}{The individual models}. Each
weight is the reciprocal of the variance associated with the bolometer.
Thus each pixel value can be thought of as the astronomical power
received by a typical bolometer at each point on the sky.
\cref{Section}{sec:cmult}{Flux conversion factors} describes how to
convert a map from units of pW to Jy.
}
\end{quote}

\section{Running \texttt{makemap}}

Before running \makemap\ directly, you need to ensure that the Starlink
environment has been initialised and the \smurf\ package started (see
\cref{Section}{sec:starinit}{Initialising Starlink} and
\cref{Section}{sec:packinit}{KAPPA and SMURF for data processing}).

To run \makemap, you  need to supply values for the following
command-line parameters\footnote{Note the distinction between
``command-line parameters'' (also known as ``ADAM'' parameters) that are
supplied on the \texttt{makemap} command line, and ``configuration parameters''
that are specified within a configuration file. Values for all
\emph{configuration} parameters are obtained using a single \emph{command-line}
parameter called \texttt{CONFIG}.}:

\begin{itemize}

\item \texttt{IN} - a list of input NDFs containing raw SCUBA-2 data (see
\cref{Section}{sec:rawdata}{The raw data} and \cref{Section}{sec:ndf}
{Data formats}). There are many ways in which the list of files can be supplied,
as described in Section ``\xref{Specifying Groups of Objects}{sun95}{se_groups}''
in ``\xref{SUN/95}{sun95}{}. The easiest is to create a simple text
file containing the names of the raw data files -- one per line --- and
then supply the name of the text file, preceded by an up-caret character
(\,\texttt{\^{}}\,), as the value for parameter \texttt{IN}. Note, the names of
the raw data files can contain wild-cards such as ``$*$'' and ``\%''.

\item \texttt{OUT} - the name of the NDF in which to store the final
map. The supplied file name should either have a file type of
``\texttt{.sdf}'', or no file type at all (in which case \texttt{.sdf}
will be appended to the supplied value). Any existing file with the same
name will be over-written.

\item \texttt{CONFIG} - a string that specifies new values for one or more
configuration parameters. These new values are used in place of the
\xref{default values}{sun258}{PAR_FULL} described in
\xref{SUN/258}{sun258}{}. Any configuration parameter not specified by
the \texttt{CONFIG} string retains its default value. As with files
names, there are many ways in which the group of parameter values can be
specified, and the same documentation should be consulted for details
(\xref{SUN/95}{sun95}{se_groups}). Again, the easiest way is to list the
parameter values in a simple text file with one parameter setting (a
``<name>=<value>'' string) on each line. See \cref{Section}{sec:config}
{Specialised configuration files} for a list of the pre-defined
configuration files that come with \smurf. Note, \texttt{CONFIG}
can be set to the special value ``\texttt{def}'' to force \makemap\ to
use the default values for all parameters. It is possible to create a map
using just default parameter values, but it will not usually be a very
good map. Advice on which parameters to edit can be found in
\cref{Section}{sec:tweak}{Tweaking the configuration file}.

\end{itemize}

If any of the above parameters are \emph{not} given on the \texttt{makemap}
command line, prompts will be issued and the user invited to supply values
for the missing parameters.

So an example command line would be:
\begin{terminalv}
% makemap in=^myfiles.lis out=850_crl2688 \
          config=^$STARLINK_DIR/share/smurf/dimmconfig_bright_compact.lis
\end{terminalv}

Here, the file \texttt{myfiles.lis} contains a list of the raw data
files to be included in the map, and could for instance look like this:

\begin{terminalv}
% cat myfiles.lis
/jcmtdata/raw/scuba2/s8a/20120720/00030/*
/jcmtdata/raw/scuba2/s8b/20120720/00030/*
/jcmtdata/raw/scuba2/s8c/20120720/00030/*
/jcmtdata/raw/scuba2/s8d/20120720/00030/*
\end{terminalv}

This uses all available data for all four 850\,$\mu$m sub-arrays, for
observation 30 taken on 20th July 2012\footnote{The input files should all be
for a single waveband and a single observation --- do not mix files from
different wavebands and/or observations.}.

The file \brightcompact\ is one of the pre-defined configuration files
supplied with \smurf. It contains configuration parameter values that are
optimised for creating maps from bright compact sources.

\begin{tip}
  An up-caret (\,\texttt{\^{}}\,) is required any time you are reading
  in a group text file in \starlink. For the map-maker this includes
  the configuration file (a group of configuration parameters) and a list
  of your input files (e.g. \texttt{in=\^{}\,myfiles.lis}).
\end{tip}

After the \makemap\ command completes, the final map will be left in file
\texttt{850\_crl2688.sdf} and can be displayed using \gaia:

\begin{terminalv}
% gaia 850_crl2688
\end{terminalv}

There are many other command-line parameters that can be supplied when
running \makemap, but only the three listed above are mandatory. All the
others assume default values if not supplied on the command-line. For a
full list of all available command-line parameters, their functions and
default values, see the description of \makemap\ within the \textsc{smurf}
document, \xref{Starlink User Note 258}{sun258}{}.

You can supply values for any of these extra parameter on the command
line. For instance, one of the command-line parameters that is usually
defaulted is \texttt{PIXSIZE}, which specifies the pixel size for the final
map, in arc-seconds. The default pixel sizes, defined as one quarter of the
Airy disk rounded up to the nearest half arc-second, are:

\begin{itemize}
\item 2\,arcsec at 450$\mu$m
\item 4\,arcsec at 850$\mu$m
\end{itemize}

So for instance, the above command-line could be modified as follows to
produce a map with 3 arc-second pixels:

\begin{terminalv}
% makemap in=^myfiles.lis out=850_crl2688 pixsize=3 \
          config=^$STARLINK_DIR/share/smurf_bright_compact.lis
\end{terminalv}

Note, if parameters are specified by name on the command-line, as
in these examples, then the order in which they are specified is
insignificant. Also, parameter names are case-insensitive.

\begin{tip}
  Map-maker not finding your raw data files from a path? Check you have
  escaped or protected any shell meta-characters in your `in' value,
  for instance by putting double quotes around it or escaping
  wild-cards using a backslash (\emph{e.g.} \texttt{in=\textbackslash*.sdf} or just
  \texttt{in=\textbackslash*}) Note, this issue applies only to wild-cards
  included directly on the command line --- there is no need to escape or
  protect wild cards within a text file.
\end{tip}

\section{Interpreting the screen output from \task{makemap}}

Whilst \makemap\ runs, it outputs information continuously to the screen
about what it is doing. The amount of information displayed can be
controlled using the \xref{\texttt{MSG\_FILTER}}{sun258}{SEC_MSG} command-line
parameter. It defaults to ``\texttt{normal}'', but if you want more
information you could set it to (say) ``\texttt{verbose}'':

\begin{terminalv}
% makemap in=^myfiles.lis out=850_crl2688 msg_filter=verb \
          config=^$STARLINK_DIR/share/smurf_bright_compact.lis
\end{terminalv}

Note - unambiguous abbreviations may be used for many command line
parameters --- so ``\texttt{verb}'' is acceptable instead of
``\texttt{verbose}''. Setting \texttt{MSG\_FILTER} to ``\texttt{quiet}''
will suppress all screen output.

\begin{tip}
  Map-maker generates a lot of screen output! The main things to check
  are that the ``\texttt{NORMALIZED MAP CHANGE}'' value (the mean value,
  not the max value) decreases nicely towards your requested
  \xparam{MAPTOL}{maptol} value as each iteration is completed, and that
  the ``\texttt{Total samples available for map:}'' value does not fall
  too low (you should usually be looking for values above 50\% of
  maximum). If neither of these two items look problematic, it is usually
  safe to pay less attention to the other screen output.
\end{tip}

The following shows the screen output generated by \makemap\ if
\texttt{MSG\_FILTER} is left set to its default value of
``\texttt{normal}''. Explanatory comments, which are not actually
part of the output generated by \makemap, are included in \emph{emphasised
type}:

\begin{terminalv}
% makemap in=^myfiles.lis out=850_crl2688 \
          config=^$STARLINK_DIR/share/smurf_bright_compact.lis

Out of 32 input files, 4 were darks, 8 were fast flats and 20 were science
Processing data from instrument 'SCUBA-2' for object 'CRL2688' from the
following observation  :
  20120720 #30 scan
\end{terminalv}

\emph{The output starts by reporting information about the input data files,
including the number that contain on-sky bolometer values (``science''
data), the astronomical object and the SCUBA-2 observation number.}

~
\begin{terminalv}

MAKEMAP: Map-maker will use no more than 92586 MiB of memory

   Projection parameters used:
      CRPIX1 = 0
      CRPIX2 = 0
      CRVAL1 = 315.578333333333 ( RA = 21:02:18.800 )
      CRVAL2 = 36.6938055555556 ( Dec = 36:41:37.70 )
      CDELT1 = -0.00111111111111111 ( -4 arcsec )
      CDELT2 = 0.00111111111111111 ( 4 arcsec )
      CROTA2 = 0

   Output map pixel bounds: ( -132:122, -126:129 )

   Output map WCS bounds:
        Right ascension: 21:01:38.318 -> 21:03:03.280
        Declination: 36:33:07.19 -> 36:50:11.70
\end{terminalv}

\emph{Next comes information about the output map. The world coordinate
system is described by means of the equivalent \htmladdnormallink{FITS-WCS}
{http://fits.gsfc.nasa.gov/fits_wcs.html} keywords\footnote{In fact, NDF
data structures do not use FITS-WCS to describe WCS --- instead they use the
\htmladdnormallink{AST}{http://www.starlink.ac.uk/ast} library, which
provides a much more flexible scheme for handling WCS.}}

\emph{The reported pixel bounds of the output map refer to a pixel coordinate
system in which the source is centred at position (0,0). Note, the definition of
pixel coordinates within the NDF format allows the origin of pixel
coordinates to be at any nominated position within the array --- it does
not have to be at the bottom left corner as in FITS. \texttt{makemap}
chooses to put the pixel origin at the specified source position. }

\emph{Finally, the bounds of the map are given in the celestial coordinate
system specified by the \xref{\texttt{SYSTEM}}{sun258}{MAKEMAP} command-line
parameter.  This parameter defaults to ``\texttt{tracking}'', which causes
the map to be created in the celestial coordinate system in which the
observation parameters were originally defined. It may instead be set to a
specific coordinate system (e.g. ``galactic'', ``icrs'', etc.) to force the
map to be made in that system.}

~
\begin{terminalv}
smf_iteratemap: will down-sample data to match angular scale of 4 arcsec
smf_iteratemap: Iterate to convergence (max 40)
smf_iteratemap: stop when mean normalized map change < 0.05
\end{terminalv}

\emph{The data is down-sampled so that the on-sky distance between
adjacent samples is roughly equal to the pixel size (4 arc-seconds in
this case). This saves memory and computing time without adversely
affecting the final map. The degree of down-sampling can be controlled
using the \xparam{DOWNSAMPSCALE}{downsampscale} configuration parameter.}

\emph{Next come information about the stopping criteria for the iterative
map-making algorithm. In this case, iterations will stop when 40 iterations are
completed, or the normalised change in the map between iterations reduces
to less that 0.05. These values are specified by configuration parameters
\xparam{NUMITER}{numiter} and \xparam{MAPTOL}{maptol} (see \cref{Section}
{sec:converge}{Stopping criteria}).}

~
\begin{terminalv}
smf_iteratemap: provided data are in 1 continuous chunks, the largest of which
has 5957 samples (153.729 s)
smf_iteratemap: map-making requires 1626 MiB (map=28 MiB model calc=1598 MiB)
smf_iteratemap: Continuous chunk 1 / 1 =========
\end{terminalv}

\emph{In almost all cases, the raw data files constituting a SCUBA-2
observation will correspond to a single continuous stream of data
samples, taken at roughly 200 Hz. Sometimes however, this may not be the
case\footnote{A common cause of this is if some sub-scans are omitted
from the list of input files supplied to \task{makemap}.}, so the user is
now told how many chunks of data were found, and how long they were.
Something may be wrong if the input data contains any breaks.}

\emph{Next comes information about the amount of memory needed to make
the map. If insufficient memory is available to process all the input
data together, it will be split into chunks, and a separate map made from
each chunk. These maps are later co-added to form the final output map.
This can often result in a poorer map --- see \latexhtml{the box
describing \emph{Data Chunking} on Page~\pageref{box:chunk})~}
{\htmlref{Data Chunking)}{box:chunk}}.}

\emph{Finally, a loop is entered to process each chunk in turn, and the
user is told which chunk is currently being processed. In this
case there  is only one chunk (which is good).}

~
\begin{terminalv}
smf_iteratemap: Iteration 1 / 40 ---------------
--- Size of the entire data array ------------------------------------------
bolos  : 5120
tslices: 5957(2.6 min)
Total samples: 30499840
\begin{terminalv}
--- Quality flagging statistics --------------------------------------------
 BADDA:   10805998 (35.43%),        1814 bolos
BADBOL:   10865568 (35.62%),        1824 bolos
DCJUMP:      19631 ( 0.06%),
  STAT:      71680 ( 0.24%),          14 tslices
 NOISE:      41699 ( 0.14%),           7 bolos
Total samples available for map:   19586411, 64.22% of max (3287.97 bolos)
\end{terminalv}

\emph{Now we have a number of statistics describing the cleaned data
prior to the first iteration\footnote{Setting
\texttt{MSG\_FILTER=verbose} on the \texttt{makemap} command line will
generate more information about the cleaning process.}.}

\emph{Each sub-array contains a grid of $32\times40$ bolometers, making 5120
bolometers in total over all four sub-arrays. The number of time slices in
the concatenated data after down-sampling is reported. In this case, 5957
time slices over 2.6 minutes equates to a down-sampled frequency of about
38 Hz. The total number of samples is the product of the number of
bolometers and the number of time slices. Each of the following lines
indicates the percentage of the data that has been flagged as unusable for
various reasons:}

\begin{itemize}
\item \emph{BADDA - flagged as inoperative during data acquisition.}
\item \emph{DCJUMP - flagged as bad because of a sudden step change in
the base-line.}
\item \emph{STAT - flagged as bad because the telescope was stationary
(or at least moving too slowly).}
\item \emph{NOISE - flagged as bad because they were too noisy.}
\end{itemize}

\emph{The BADBOL item gives the fraction of bolometers that have been
flagged as entirely bad for any of these reasons, and from which no data
will be used.}

~
\begin{terminalv}
smf_iteratemap: Calculate time-stream model components
smf_iteratemap: Rebin residual to estimate MAP
smf_iteratemap: Calculate ast
\end{terminalv}

\emph{Indicates the models that are being calculated. More information
about each individual model is displayed if you set
\texttt{MSG\_FILTER=verbose} on the \texttt{makemap} command line.}

~
\begin{terminalv}
--- Quality flagging statistics --------------------------------------------
 BADDA:   10805998 (35.43%),        1814 bolos  ,change          0 (+0.00%)
BADBOL:   11008536 (36.09%),        1848 bolos  ,change     142968 (+1.32%)
DCJUMP:      19631 ( 0.06%),                    ,change          0 (+0.00%)
  STAT:      71680 ( 0.24%),          14 tslices,change          0 (+0.00%)
   COM:     372786 ( 1.22%),                    ,change     372786 (+0.00%)
 NOISE:      41699 ( 0.14%),           7 bolos  ,change          0 (+0.00%)
Total samples available for map:   19214687, 63.00% of max (3225.56 bolos)
     Change from last report:    -371724, -1.90% of previous
smf_iteratemap: Will calculate chi^2 next iteration
smf_iteratemap: *** NORMALIZED MAP CHANGE: 2.22778 (mean) 60.5292 (max)
\end{terminalv}

\emph{More statistics are displayed once the first iteration is completed
and the first estimate of the science map has been generated. Data
samples may be flagged as bad within the iterative stage for various
reasons, and so these statistics may be different to those shown prior to
the start of the iterative stage. Most significantly, the \texttt{COM:}
line shows the percentage of data that has been rejected because the time
stream did not resemble the common-mode signal closely enough. The
``\texttt{NORMALIZED MAP CHANGE}'' values are of no real significance on the
first iteration since there is no previous map with which to compare the
new map (in fact the numerical values are generated by comparing the new map
with a map full of zeros).}

~
\begin{terminalv}
smf_iteratemap: Iteration 2 / 40 ---------------
smf_iteratemap: Calculate time-stream model components
smf_calcmodel_noi: Calculating a NOI variance for each box of 581 samples
using variance of neighbouring residuals.
smf_iteratemap: Rebin residual to estimate MAP
smf_iteratemap: Calculate ast
--- Quality flagging statistics --------------------------------------------
 BADDA:   10805998 (35.43%),        1814 bolos  ,change          0 (+0.00%)
BADBOL:   11038321 (36.19%),        1853 bolos  ,change      29785 (+0.27%)
 SPIKE:         36 ( 0.00%),                    ,change         36 (+0.00%)
DCJUMP:      19631 ( 0.06%),                    ,change          0 (+0.00%)
  STAT:      71680 ( 0.24%),          14 tslices,change          0 (+0.00%)
   COM:     393125 ( 1.29%),                    ,change      20339 (+5.46%)
 NOISE:      41699 ( 0.14%),           7 bolos  ,change          0 (+0.00%)
Total samples available for map:   19194401, 62.93% of max (3222.16 bolos)
     Change from last report:     -20286, -0.11% of previous
smf_iteratemap: *** CHISQUARED = 0.997314089857974
smf_iteratemap: *** NORMALIZED MAP CHANGE: 1.78691 (mean) 8.09056 (max)

\end{terminalv}

\emph{The next iteration starts, and proceeds in much the same way as the
first iteration. The main difference is that the NOI model is generated
at the start of the second iteration. This model measures the variance
within each bolometer time-stream. It is used to weight the bolometer
samples when forming the mean sample value in each map
pixel\footnote{Bolometers are given equal weight in the map created at
the end of the first iteration since the NOI model has not yet been
calculated at that point.}. NOI is calculated from the residuals left after
removal of all other models, and is only calculated once --- subsequent
iterations re-use the same NOI values.}

\emph{The \texttt{SPIKE:} item that has appeared in the quality flagging
statistics record the number of samples that have been rejected as
transient spikes in the time-series. This flagging is done by comparing
the bolometer values that fall in each map pixel and flagging any that
appear to be statistical out-liers - see configuration parameter
\xparam{AST.MAPSPIKE}{ast.mapspike}.}

\emph{The mean ``\texttt{NORMALIZED MAP CHANGE}'' value should drop with each
subsequent iteration (note, the \emph{max} normalised map change value
can usually be ignored as it records the maximum normalised change in any
single map pixel and is thus highly subject to random variations).}

~
\begin{terminalv}
smf_iteratemap: Iteration 3 / 40 ---------------
smf_iteratemap: Calculate time-stream model components
smf_iteratemap: Rebin residual to estimate MAP
smf_iteratemap: Calculate ast
--- Quality flagging statistics --------------------------------------------
 BADDA:   10805998 (35.43%),        1814 bolos  ,change          0 (+0.00%)
BADBOL:   11050235 (36.23%),        1855 bolos  ,change      11914 (+0.11%)
 SPIKE:         36 ( 0.00%),                    ,change          0 (+0.00%)
DCJUMP:      19631 ( 0.06%),                    ,change          0 (+0.00%)
  STAT:      71680 ( 0.24%),          14 tslices,change          0 (+0.00%)
   COM:     401600 ( 1.32%),                    ,change       8475 (+2.16%)
 NOISE:      41699 ( 0.14%),           7 bolos  ,change          0 (+0.00%)
Total samples available for map:   19185947, 62.91% of max (3220.74 bolos)
     Change from last report:      -8454, -0.04% of previous
smf_iteratemap: *** CHISQUARED = 0.976599086972009
smf_iteratemap: *** change: -0.0207150028859653
smf_iteratemap: *** NORMALIZED MAP CHANGE: 0.305 (mean) 1.24407 (max)
\end{terminalv}

\emph{The mean normalised map change continues to drop with the third
iteration, as expected. The total number of samples going into the map is
dropping with each iteration, but only very slowly. This is mainly due to
the increased number of samples being flagged by the COM model, but is at
an acceptably small level.}

~
\begin{terminalv}
smf_iteratemap: Iteration 4 / 40 ---------------
smf_iteratemap: Calculate time-stream model components
smf_iteratemap: Rebin residual to estimate MAP
smf_iteratemap: Calculate ast
--- Quality flagging statistics --------------------------------------------
 BADDA:   10805998 (35.43%),        1814 bolos  ,change          0 (+0.00%)
BADBOL:   11056192 (36.25%),        1856 bolos  ,change       5957 (+0.05%)
 SPIKE:         36 ( 0.00%),                    ,change          0 (+0.00%)
DCJUMP:      19631 ( 0.06%),                    ,change          0 (+0.00%)
  STAT:      71680 ( 0.24%),          14 tslices,change          0 (+0.00%)
   COM:     409883 ( 1.34%),                    ,change       8283 (+2.06%)
 NOISE:      41699 ( 0.14%),           7 bolos  ,change          0 (+0.00%)
Total samples available for map:   19177684, 62.88% of max (3219.35 bolos)
     Change from last report:      -8263, -0.04% of previous
smf_iteratemap: *** CHISQUARED = 0.96404406922676
smf_iteratemap: *** change: -0.0125550177452489
smf_iteratemap: *** NORMALIZED MAP CHANGE: 0.0401588 (mean) 0.151877 (max)
\end{terminalv}

\emph{After the fourth iteration the mean normalised map change has
dropped to 0.0401588, which is below the value of 0.05 provided for the
\xparam{MAPTOL}{maptol} parameter by the \brightcompact\
configuration file\footnote{In fact, 0.05 is the default \texttt{maptol}
value, which is left unchanged by \texttt{dimmconfig\_bright\_compact.lis}.}.
Consequently, \texttt{makemap} considers the map to have converged. However,
in view of the fact that the \brightcompact\ configuration file uses
``AST masking'' (see \cref{Section}{sec:astmask}{AST masking}), it is
necessary to do one final iteration in order to assign correct values to
the pixels that lie outside the source mask. When AST masking is being
used, the reported normalised map change values only include pixels that
are within the source mask.}

~
\begin{terminalv}
smf_iteratemap: Iteration 5 / 40 ---------------
smf_iteratemap: Calculate time-stream model components
smf_iteratemap: Rebin residual to estimate MAP
smf_iteratemap: Calculate ast
--- Quality flagging statistics --------------------------------------------
 BADDA:   10805998 (35.43%),        1814 bolos  ,change          0 (+0.00%)
BADBOL:   11068106 (36.29%),        1858 bolos  ,change      11914 (+0.11%)
 SPIKE:         36 ( 0.00%),                    ,change          0 (+0.00%)
DCJUMP:      19631 ( 0.06%),                    ,change          0 (+0.00%)
  STAT:      71680 ( 0.24%),          14 tslices,change          0 (+0.00%)
   COM:     414727 ( 1.36%),                    ,change       4844 (+1.18%)
 NOISE:      41699 ( 0.14%),           7 bolos  ,change          0 (+0.00%)
Total samples available for map:   19172853, 62.86% of max (3218.54 bolos)
     Change from last report:      -4831, -0.03% of previous
smf_iteratemap: *** CHISQUARED = 0.964032127175568
smf_iteratemap: *** change: -1.19420511923707e-05
smf_iteratemap: *** NORMALIZED MAP CHANGE: 0.0157131 (mean) 0.103135 (max)
smf_iteratemap: ****** Completed in 5 iterations
smf_iteratemap: ****** Solution CONVERGED
Setting 24282 map pixels bad because they contain fewer than 4 samples (=0.01
of the mean samples per pixel).
Total samples available from all chunks: 19172853 (3218.54 bolos)
\end{terminalv}

\emph{After the extra iteration required by AST masking has been
performed, the final output map is created. It is always advisable to
check the final number of samples available for the map, as a low value
will cause your map to have high noise levels. In this case, 62.86\% of
the samples are available for the map, which is quite acceptable --- only
a couple of percent of the samples have been rejected by the map-making,
mostly flagged by the COM model. The bulk of the bad samples (35.43\%)
were rejected during data acquisition due to dead bolometers \emph{etc}.}

\emph{Note, the \xparam{NUMITER}{numiter} parameter is set to -40 by
\brightcompact, meaning that no more than 40 iterations will be
performed. In this particular case we only needed 4 iteration (plus a
mandatory extra iteration) to achieve
our requested \xparam{MAPTOL}{maptol} value. But it is possible for some
observations --- particularly observations of extended sources --- to
require more than 40 iterations to converge to a \texttt{maptol} of 0.05.
In such cases the screen output will end with a message saying that the
solution ``FAILED TO CONVERGE''\footnote{However a map will still be
created, but should be used with caution.}. In such cases, you \emph{could}
simply increase \texttt{numiter}, but you may also want to investigate
the cause of the slow convergence using one or more of the techniques
described in \cref{Chapter}{sec:raw}{Chapter 9}, as it may indicate
some issue with the raw data.}

\emph{Map pixels that receive a very small number of samples are automatically
set bad. These are usually the pixels around the periphery of the
observation, and will have very unreliable variance estimates. The
threshold is determined by the \xparam{HITSLIMIT}{hitslimit} parameter,
which defaults to 1 percent of the mean number of hits per pixel,
corresponding to 4 samples per pixel in this case.}

\emph{The observation used in this particular case was a short
observation of a calibrator, lasting only 2.6 minutes. Consequently there
was no need to divide the data up into multiple chunks in order to fit it
into memory.  However, for very long observations, or for shorter
observations when using a computer with less than the recommended amount
of memory (see \cref{Section}{sec:computing}{Chapter 1}), it may be
necessary to process the raw data in multiple chunks. If this happens, a
map is created from each chunk in turn, and all these maps are then added
together to form the final map.  The final message ``\texttt{Total
samples available from all chunks: 19172853 (3218.54 bolos)}'' indicates
the total number of samples (and equivalent number of bolometers) used
from all chunks. In this case there was only one chunk, so these values
are equal to the numbers reported at the end of the last iteration.}





