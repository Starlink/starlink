\documentstyle[11pt]{article} 
\pagestyle{myheadings}

%------------------------------------------------------------------------------
\newcommand{\stardoccategory}  {Starlink User Note}
\newcommand{\stardocinitials}  {SUN}
\newcommand{\stardocnumber}    {132.1}
\newcommand{\stardocauthors}   {J.\ F.\ Lightfoot \& P.\ A.\ Harrison}
\newcommand{\stardocdate}      {13 October 1992}
\newcommand{\stardoctitle}     {JCMTDR --- Figaro Applications for Reducing
JCMT Data \\[2ex] Version 1.0}
%------------------------------------------------------------------------------

\newcommand{\stardocname}{\stardocinitials /\stardocnumber}
\renewcommand{\_}{{\tt\char'137}}     % re-centres the underscore
\markright{\stardocname}
\setlength{\textwidth}{160mm}
\setlength{\textheight}{230mm}
\setlength{\topmargin}{-2mm}
\setlength{\oddsidemargin}{0mm}
\setlength{\evensidemargin}{0mm}
\setlength{\parindent}{0mm}
\setlength{\parskip}{\medskipamount}
\setlength{\unitlength}{1mm}

%------------------------------------------------------------------------------
% Add any \newcommand or \newenvironment commands here
%------------------------------------------------------------------------------

\begin{document}
\thispagestyle{empty}
SCIENCE \& ENGINEERING RESEARCH COUNCIL \hfill \stardocname\\
RUTHERFORD APPLETON LABORATORY\\
{\large\bf Starlink Project\\}
{\large\bf \stardoccategory\ \stardocnumber}
\begin{flushright}
\stardocauthors\\
\stardocdate
\end{flushright}
\vspace{-4mm}
\rule{\textwidth}{0.5mm}
\vspace{5mm}
\begin{center}
{\Large\bf \stardoctitle}
\end{center}
\vspace{5mm}

%------------------------------------------------------------------------------
%  Add this part if you want a table of contents
\setlength{\parskip}{0mm}
\tableofcontents
\setlength{\parskip}{\medskipamount}
\markright{\stardocname}
%------------------------------------------------------------------------------

\newpage

\section{Introduction}

JCMTDR is a package of Figaro applications that is intended to replace
NOD2 for the reduction of continuum mapping data obtained with UKT14
on the James Clerk Maxwell Telescope.

Two versions have been released; one that runs in DCL, the other an
ADAM monolith. The DCL commands are made available by typing:

\begin{verbatim}
      $ JCMTDR
\end{verbatim}

The monolith is loaded by typing:

\begin{verbatim}
      $ ADAMSTART
      $ ICL
      ICL> JCMTDR
\end{verbatim}

Help information for both versions can be obtained by typing HELP
JCMTDR.

\section {The Applications}

\subsection{AE2RD1}

This application will transform JCMT map data into a tangent  plane
image centred on a specified RA, Dec. The rebinning is performed by
convolving the input data with a truncated Bessel function. It is
similar but not identical to the NOD2 CONVERT function. 

The width of the Bessel function used is such that it would just be
fully  sampled by data points at the spacing of the pixels of the
output  map. The pixel spacing of the output map is set to be the
minimum  of the nominal pixel spacings of the input maps. To minimise
edge  effects the Bessel function is truncated at a radius of 7 pixels 
from its centre, and apodised over its outer third by a cosine 
function.

In Fourier terms the method approximates to Fourier transforming the
input dataset(s), multiplying the transform by a cylindrical top hat
(the F.T. of the Bessel function), then transforming back into image
space. 

The application can read in up to 10 separate input datasets. The
output map will be large enough to cover all the input data  points
and the map centre can be specified by the user (the  default is the
map centre of the first input dataset). Each input  dataset can be
assigned a weight relative to that of the first.  Output pixels
further than one pixel spacing from any input data  point will be
flagged as a bad value. Input pixels flagged as bad will be ignored.
Input error arrays are not propagated.

The convolution assumes that all pixels outside the measured  area are
zero. This will lower the brightness of objects that are  less than
the radius of the convolution function (7 output pixels)  from the map
edge, though the effect is only a few percent unless  the object is
very close (2 or 3 pixels) to the edge. 

\newpage    % ***** PAGE BREAK *****

The parameters used are:

\begin{small}
\begin{verbatim}
         PARAMETER  INFILE
            TYPE    CHAR
            NAME    IN(file)
            PROMPT  "(INfile) The name of a file containing JCMT map data"
            TEXT   ---------------------------------------------------
                   The name of a file containing the map data. Giving
                   END as the name of the file will finish the input 
                   sequence.
                   ---------------------------------------------------

         PARAMETER  OUTPUT
            TYPE    FILE
            NAME    OUT(PUT)
            OPTIONS OUTPUT
            PROMPT  "(OUTput) The name of the file to contain the output map"

         PARAMETER B1950 
            TYPE    KEY 
            NAME    B1(950) 
            PROMPT "(B1950) make output coordinates B1950 equinox"
            TEXT   ---------------------------------------------------
                   By default the application will transform the data 
                   into a map whose centre and tangent plane offsets
                   are in the FK5 system with Julian 2000 equinox.
                   Setting B1950 to YES will give results in the FK4
                   system with Besselian 1950 equinox.
                   ---------------------------------------------------

         PARAMETER WEIGHT
            TYPE   FLOAT
            PROMPT "(WEIGHT) weight of this dataset relative to the first"
            TEXT   ---------------------------------------------------
                   The first dataset will be coadded into the output
                   data array with unit weight. Subsequent datasets
                   can be coadded in with specified weights relative
                   to the first.
                   ---------------------------------------------------

         PARAMETER RA_CENTRE
            TYPE   CHAR
            PROMPT "(RA_centre) The RA of the centre of the output map"
            TEXT   ---------------------------------------------------
                   The positions of the output map pixels will be 
                   given in terms of a tangent plane offset in arcsec 
                   from a map centre specified by the user. This parameter
                   specifies the RA of the centre, it must be input in 
                   hh mm ss.s format.
                   ---------------------------------------------------

         PARAMETER DEC_CENTRE
            TYPE   CHAR
            PROMPT "(DEC_centre) The Dec. of the centre of the output map"
            TEXT   ---------------------------------------------------
                   The positions of the output map pixels will be 
                   given in terms of a tangent plane offset in arcsec 
                   from a map centre specified by the user. This parameter
                   specifies the declination of the centre, it must be 
                   input in (sign)dd mm ss.s format.
                   ---------------------------------------------------
\end{verbatim}
\end{small}

There is a problem with this method if the mapped area, or part of it,
was not sampled closely enough to fully sample the telescope beam.
Users should treat the resampled output for those  areas with
suspicion. In such cases it would be better to use AE2RD2, which uses
a NAG interpolation routine to perform the rebinning.


\subsubsection {Pointing Corrections}

If required, AE2RD1 will correct the nominal offsets of the  input map
pixels for pointing errors. To make this happen you should use the LET
command to specify the pointing correction in the .PCORR structure in
each input map file (note that using LET to create an array requires
the last element of the array to be created first). AE2RD1 will look
for a structure of the following form:

\begin{small}
\begin{verbatim}
      .PCORR
         .LST   (ncorr) lst1, lst2, .... lstn
         .D_RA  (ncorr) d_ra1, d_ra2, .... d_ran
         .D_DEC (ncorr) d_dec1, d_dec2, .... d_decn
\end{verbatim}
\end{small}

where \verb+ncorr+ is the number of points at which the correction has
been specified. The three arrays specify the magnitude of the
correction in RA and Dec as a function of LST. The LST must be
specified in radians, the pointing corrections in arcsec. The
correction for each map point is calculated by linear interpolation
between the LSTs that bracket the LST of the measurement. If the
measurement lies outside the described  range, the correction will be
that with the nearest LST. The corrections are {\em added} to the RA
and Dec of the pixels.

\subsection{AE2RD2}

This application performs the same function as AE2RD1 but rebins each
input dataset individually using NAG interpolation routines (E01SAF
and E01SBF) before coadding them into the result. Errors associated
with the input datasets are not propagated.

\subsection{JCMTEXTC}

This application corrects JCMT data for the effect of  atmospheric
extinction. The airmass at which each map pixel was  observed is
calculated, and multiplied by the estimated zenith  atmospheric
extinction at the time of observation to give the  extinction optical
depth along the line of sight. Multiplying the data point by the
exponential of the optical depth gives  the value that would have been
measured above the atmosphere. 

The zenith optical depth is assumed to vary linearly with  time
between the values given by the parameters TAU and ENDTAU.

The parameters used are:

\begin{small}
\begin{verbatim}
         PARAMETER  INPUT
            TYPE    FILE
            NAME    IN(PUT)
            OPTIONS INPUT
            PROMPT  "(INput) file containing map data to be corrected"

         PARAMETER  OUTPUT
            TYPE    FILE
            NAME    OUT(PUT)
            OPTIONS OUTPUT
            PROMPT  "(OUTput) file to contain the corrected data"

         PARAMETER  TAU
            TYPE    FLOAT
            NAME    TAU
            OPTIONS REQUIRED
            PROMPT  "(TAU) initial value of the atmospheric extinction"
            TEXT    --------------------------------------------------
                    The value of the zenith atmospheric extinction at
                    the start of the map observation.
                    --------------------------------------------------

         PARAMETER  ENDTAU
            TYPE    FLOAT
            NAME    ENDTAU
            OPTIONS REQUIRED
            PROMPT  "(ENDTAU) final value of the atmospheric extinction"
            TEXT    --------------------------------------------------
                    The value of the zenith atmospheric extinction at
                    the end of the map observation. The zenith extinction
                    for each pixel will be calculated assuming a linear 
                    gradient with time between the start and end values.
                    --------------------------------------------------
\end{verbatim}
\end{small}

\subsection{MAKEMAP}

This takes a JCMT data file in GSD format and converts it to a Figaro
file. It is recommended that the JOB logical name FIGARO\_FORMATS be
set  to `NDF' before running the application so that the output file
will be a Starlink NDF, readable by both KAPPA and Figaro. The command
has the following parameters:

\begin{small}
\begin{verbatim}
         PARAMETER  GSDFILE
            TYPE    CHAR
            NAME    GSD(FILE)
            PROMPT  "(GSDfile) GSD file containing JCMT data"
            TEXT    --------------------------------------------------
                    This file has the default extension .DAT which
                    should not be specified in the command line
                    --------------------------------------------------

         PARAMETER  OUTPUT
            TYPE    FILE
            NAME    OUT(PUT)
            OPTIONS OUTPUT
            PROMPT  "(OUTput) name of resulting Figaro data file"
            TEXT    --------------------------------------------------
                    This file will have the extension .DST unless the
                    logical name FIGARO_FORMATS is set to NDF, in which
                    case it will be a Starlink NDF with file extension
                    .SDF.
                    --------------------------------------------------

         PARAMETER  CHANNEL
            TYPE    INT
            NAME    CHAN(NEL)
            OPTIONS NOPROMPT
            PROMPT  "(CHANnel) backend channel to be used"
            TEXT    --------------------------------------------------
                    In a multi backend channel system the channel
                    number that should be used to make the map has to
                    be specified. The default is channel 1.
                    --------------------------------------------------

         PARAMETER  TEL_BEAM
            TYPE    CHAR
            NAME    TEL(_BEAM)
            PROMPT  "(TEL_beam) beam that telescope was in during map"
            TEXT    --------------------------------------------------
                    The beam towards which the telescope was pointing
                    while the map was made. The choice is L, M or R.
                    The majority of maps are made with the telescope
                    in the middle (M) beam, in fact this version of the
                    JCMTDR package won't work unless this is the case.
                    --------------------------------------------------

         PARAMETER  POS_BEAM
            TYPE    CHAR
            NAME    POS(_BEAM)
            PROMPT  "(POS_beam) positive beam of chopper"
            TEXT    --------------------------------------------------
                    The positive beam of the secondary chopper while
                    the map was made. This should be L or R.
                    --------------------------------------------------
\end{verbatim}
\end{small}

The output from MAKEMAP is a Figaro file whose main data array is a
2-d map made up of from the measured data arranged according to its
cell offset from the observation map centre. The x and y axes of the
data are the pixel offsets along the cell axes from the map centre. If
the data were obtained pixel by pixel rather than by scanning then it
is possible that not all the pixels in the mesh were observed. Such
pixels will be set to a `bad' value.

The application creates a JCMT-specific section in the file in which
are stored parameters relating to the telescope, the environment and
the way in which the map was made. 

If MAKEMAP fails to find important items in the GSD file, they will be 
prompted for. If all else fails you can create, delete, rename or set
individual items in the datafile using the Figaro CROBJ, DELOBJ, 
RENOBJ or LET commands (see section~\ref{sec-data} for a description
of the data structures used).

\subsection {MAP2MEM}

This application takes a JCMT map file and converts the data into a
format suitable for further processing (deconvolution and resampling)
by the  DBMEM package written by J.S.Richer. The parameters required
are:

\begin{small}
\begin{verbatim}
         PARAMETER  INPUT
            TYPE    FILE
            NAME    IN(PUT)
            OPTIONS INPUT
            PROMPT  "(INput) The name of the file containing JCMT map data"

         PARAMETER  OUTPUT
            TYPE    CHAR
            NAME    OUT(PUT)
            PROMPT  "(OUTput) The rootname of the DBMEM file to contain the map"
            TEXT   ---------------------------------------------------
                   The map will be output to 2 files; the header file
                   <rootname>.DBH, and the data file which will
                   either be <rootname>.MEM or .BIN depending on the
                   value of parameter BINARY.
                   ---------------------------------------------------

         PARAMETER B1950 
            TYPE    KEY 
            NAME    B1(950) 
            PROMPT "(B1950) make output coordinates B1950 equinox"
            TEXT   ---------------------------------------------------
                   By default the application will transform the data 
                   into a map whose centre and tangent plane offsets
                   are in the FK5 system with Julian 2000 equinox.
                   Setting B1950 to YES will give results in the FK4
                   system with Besselian 1950 equinox.
                   ---------------------------------------------------

         PARAMETER BINARY
            TYPE   KEY
            NAME   BINARY
            PROMPT "(BINARY) output data to unformatted file"
            TEXT   ---------------------------------------------------
                   YES will result in the output data being written to
                   file <rootname>.BIN, NO will give ASCII numbers in 
                   <rootname>.MEM.
                   ---------------------------------------------------

         PARAMETER RA_CENTRE
            TYPE   CHAR
            PROMPT "(RA_centre) The RA of the centre of the output map"
            TEXT   ---------------------------------------------------
                   The positions of the output map pixels will be 
                   given in terms of a tangent plane offset in arcsec 
                   from a map centre specified by the user. This parameter
                   specifies the RA of the centre, it must be input in 
                   hh mm ss.s format.
                   ---------------------------------------------------

         PARAMETER DEC_CENTRE
            TYPE   CHAR
            PROMPT "(DEC_centre) The Dec. of the centre of the output map"
            TEXT   ---------------------------------------------------
                   The positions of the output map pixels will be 
                   given in terms of a tangent plane offset in arcsec 
                   from a map centre specified by the user. This parameter
                   specifies the declination of the centre, it must be 
                   input in (sign)dd mm ss.s format.
                   ---------------------------------------------------

         PARAMETER NOISE
            TYPE   FLOAT
            PROMPT "(NOISE) Estimate of noise on each map point"
            TEXT   ---------------------------------------------------
                   An estimate of the noise on each measured data point.
                   If the input data already has an associated error
                   array then those numbers will be used.
                   ---------------------------------------------------
\end{verbatim}
\end{small}

\subsection{MAP2TS}

This application converts an image into a `time spectrum' with the map
pixels sorted in order of increasing LST of observation. The LST is
stored as  the X axis. It is useful to have the data in this form if
you want to spot and correct variations in the data that are functions
of time. The data will have to be transformed back into map format
using TS2MAP before being input to other JCMTDR applications.

The application has the following parameters:

\begin{small}
\begin{verbatim}
         PARAMETER  INPUT
            TYPE    FILE
            NAME    IN(PUT)
            OPTIONS INPUT
            PROMPT  "(INput) file containing map data"

         PARAMETER  OUTPUT
            TYPE    FILE
            NAME    OUT(PUT)
            OPTIONS OUTPUT
            PROMPT  "(OUTput) "file to contain time-sorted data"
\end{verbatim}
\end{small}

\subsection{RESTORE}

This application deconvolves the chopped-beam from a JCMT dual  beam
map of a source in a manner similar to that described by  Emerson,
Klein \& Haslam (Astron.Astrophys., 76, 92). The algorithm  is exactly
the same as that used in NOD2; first the spatial  frequencies to which
the chop is insensitive are set to zero,  then the chop is
deconvolved. The map is deconvolved scan by scan.

The source will come out negative if the positive beam of the chopper
was specified wrongly in MAKEMAP. This can be fixed by using LET to
set {\em file}.MORE.JCMT.MAP.POS\_BEAM to the correct value and
re-running RESTORE. Similarly, the chop throw  is read from {\em
file}.MORE.\-JCMT.\-MAP.CHOP\_THRW and can be changed if you  believe that
the nominal value was inaccurate.

If there are any bad pixels in the input data they will be ignored,
effectively treated as zero in the restoration process. If the pixel
is in the source then this is {\em not what you want to happen}. The
pixel should be set to something nearer to its real value by using one
of the Figaro data editing applications. Similarly  the process
carries the implicit assumption that all data points outside the raw
map would have been zero had they been measured. Bad results will be
obtained if the raw map does not extend right off the source in the
scan direction. Lastly, the zero level in the raw map must be set to 0
before RESTORE is run, often the raw data has a DC offset.

The parameters used are:

\begin{small}
\begin{verbatim}
         PARAMETER  INPUT
            TYPE    FILE
            NAME    IN(PUT)
            OPTIONS INPUT
            PROMPT  "(INput) file containing map data"

         PARAMETER  OUTPUT
            TYPE    FILE
            NAME    OUT(PUT)
            OPTIONS OUTPUT
            PROMPT  "(OUTput) file to contain restored map"

         PARAMETER  UNBAL
            TYPE    FLOAT
            NAME    UNBAL
            OPTIONS NOPROMPT
            PROMPT  "(UNBAL) amplitude (L beam) / amplitude (R beam)"
            TEXT    ------------------------------------------------------
                    The ratio of the amplitude of the left hand chop beam
                    and that of the right hand chop beam. For the JCMT this
                    will normally be 1.
                    ------------------------------------------------------
\end{verbatim}
\end{small}

\newpage    % ***** PAGE BREAK *****

\subsection{TS2MAP}

Performs the inverse operation of MAP2TS. The parameters required are:

\begin{small}
\begin{verbatim}
         PARAMETER  INPUT
            TYPE    FILE
            NAME    IN(PUT)
            OPTIONS INPUT
            PROMPT  "(INput) Name of file containing time-sorted data"

         PARAMETER  OUTPUT
            TYPE    FILE
            NAME    OUT(PUT)
            OPTIONS OUTPUT
            PROMPT  "(OUTput) Name of file to contain map data"
\end{verbatim}
\end{small}

\section{GSD Utilities}

Two utilities are provided to allow examination and  modification of
the raw GSD data files. They are available as DCL symbols and should
be prefaced by a \$ when run from ICL.

\subsection{GSD\_PRINT}

This reads the contents of a GSD file and writes them to an ASCII file
suitable for printing or examining with an editor.

\subsection{GSD\_FORMAT}

This is used to display individual, named items in the GSD file and
can modify their values. The program operates in a loop which can be
exited by responding to a prompt with CTRL-C.

\section{Datafile Formats}
\label{sec-data}

The applications can access data files either in `old Figaro' or
Starlink  NDF formats. It is recommended that the NDF format be used
as this will allow most other Starlink packages ({\em e.g.} KAPPA and
PHOTOM) to access the data as well. The IRCAM data reduction package
should also be able to access the main data arrays, though other parts
of the data structure will be inaccessible.

To make these applications (and the rest of Figaro) access Starlink
NDF files the user should define the logical name FIGARO\_FORMATS to
be {\tt  "NDF"}. If the applications are being run as an ADAM monolith from
ICL then the logical name should be defined /JOB. NDF files have the
extension {\tt .SDF} while `old Figaro' files have the extension {\tt
.DST}.

A list of all the structures that might be found in a JCMT datafile
before the rebinning stage follows (the example is in Starlink NDF
format).  The actual items and  structures present in a given file and
the order in which they appear will depend on the stage that has been
reached in the reduction process. The temporary structures used by the
MAP2TS and TS2MAP applications  have been left out.

\newpage    % ***** PAGE BREAK *****

JCMT Datafile structure before rebinning.

\begin{small}
\begin{verbatim}
! the name of the file.
OBS_IFD_0207
! `bad' pixels may be present.
  .BAD_PIXEL     Logical 1
! the main data array holding the data values.
  .DATA_ARRAY[77,33]  Float 4.99 5.002 5.01 4.998 4.973 4.952 4.969
                                         .... 5.013 5.016 5.016 5.023
! the x-axis structure. Before it has been re-binned to an RA, dec grid
! the x-axis values will be offsets along the x cell axis. After
! rebinning the offsets will be true RA offsets from the map centre.
  .AXIS[1]       Axis
    .DATA_ARRAY[77]  Float -190 -185 -180 -175 -170 -165 -160 -155 -150
                                     .... 160 165 170 175 180 185 190
    .UNITS[32]   Char    ARCSEC
    .LABEL[32]   Char    X offset
! the y-axis structure.
  .AXIS[2]       Axis
    .DATA_ARRAY[33]  Float -80 -75 -70 -65 -60 -55 -50 -45 -40 -35 -30
                                      .... 40 45 50 55 60 65 70 75 80
    .UNITS[32]   Char    ARCSEC
    .LABEL[32]   Char    Y offset
  .MORE          Ext
! the extension structure that will hold the JCMT-specific information
! used by the JCMTDR package.
    .JCMT        Ext_jcmt
! the array holding the local sidereal time at which each raw map measurement
! was made (radians).
      .LST       Ndf
        .DATA_ARRAY[77,33]  Double 1.429 1.429 1.429 1.429 1.429 1.429
                                             .... 1.64 1.64 1.64 1.64
      .OBS       Struc
        .OBJECT[64]  Char OMC1             ! object name
        .OBS_1[32]  Char GS_RMP            ! observer's name(s)
        .OBS_2[32]  Char KKP               !      "        "
        .PROJECT[32]  Char mu35            ! PATT number
        .NOBS    Int     207               ! observation number
      .TEL       Struc
        .NAME[32]  Char  JCMT              ! telescope name
        .FRONTEND[32]  Char UKT14          ! front-end receiver
        .BACKEND[32]  Char IFD             ! back-end
        .LONG    Double  3.57              ! telescope longitude (radians)
        .LAT     Double  0.346             ! latitude (radians)
        .HEIGHT   Double 4092.1            ! height above sea-level (metres)
      .MAP       Struc
        .CENTRE_CRD[32]  Char RB           ! coordinate system of map centre
        .EPOCH   Double  1950              ! epoch of coords
        .RACEN   Double  1.452             ! RA (radians)
        .DECCEN   Double -0.09437          ! Dec (radians)
        .LOCAL_CRD[32]  Char AZ            ! coordinate system for local offsets
        .V2Y     Float   0                 ! local vertical to cell y axis
                                           !    (anticlockwise, radians)
        .X2Y     Float   1.571             ! cell x to cell y (anticlockwise,
                                           !    radians)
        .CELL_X   Float  5                 ! size of x cell
        .CELL_X_UNIT[10]  Char ARCSEC
        .CELL_Y   Float  5                 ! size of y cell
        .CELL_Y_UNIT[10]  Char ARCSEC
        .ON_FLY   Int    255               ! flag T if data taken on-the-fly
        .INT_TIME   Float 1                ! integration time per point
        .SCAN_DIR[16]  Char HORIZONTAL     ! direction of scans in data array
        .MJD_START   Double 48571.5        ! modified Julian day of observation
        .CHOP_THRW    Float   50           ! chop-throw (arcsec)
        .CHOP_PA    Float   1.571          ! chopper position angle (radians)
        .CHOP_CRD[32]    Char LO           ! chopper coordinate system
        .TEL_BEAM    Char M                ! beam at which telescope was 
                                           !    pointing during map
        .POS_BEAM    Char R                ! +ve beam of chopper
        .FREQ    Float   273               ! chop-frequency (Hz)
      .ENVRNMNT   Struc
        .AMB_TEMP   Float 276.5            ! environment stuff
        .PRESSURE   Float 625.0
        .REL_HUMID   Float 0.1802
      .PCORR     Struc
        .LST
        .D_RA
        .D_DEC
! the structure generated by the JCMTEXTC application to store the airmass
! at which each pixel was measured.
      .AIRMASS   Struc
        .DATA_ARRAY[77,33]  Float 1.106 1.106 1.106 1.106 1.106 1.106
                                               .... 1.124 1.124 1.124
! the structure generated by the JCMTEXTC application to store the zenith
! opacity used to extinction correct each pixel.
      .TAU       Struc
        .DATA_ARRAY[77,33]  Float 0.254 0.254 0.254 0.254 0.254 0.254
                                               .... 0.258 0.258 0.258
  .TITLE[32]     Char    OMC1
\end{verbatim}
\end{small}

\newpage    % ***** PAGE BREAK *****

A fully reduced and rebinned data file should look something like the
following:

\begin{small}
\begin{verbatim}
TEST
  .BAD_PIXEL     Logical 1
  .DATA_ARRAY[90,57]  Float -1.701E+38 -1.701E+38 -1.701E+38 -1.701E+38
                                .... -1.701E+38 -1.701E+38 -1.701E+38
  .AXIS[1]       Axis
    .DATA_ARRAY[90]  Float 225 220 215 210 205 200 195 190 185 180 175
                                        .... -200 -205 -210 -215 -220
    .UNITS[32]   Char    ARCSEC
    .LABEL[32]   Char    RA
  .AXIS[2]       Axis
    .DATA_ARRAY[57]  Float -110 -105 -100 -95 -90 -85 -80 -75 -70 -65
                                     .... 140 145 150 155 160 165 170
    .UNITS[32]   Char    ARCSEC
    .LABEL[32]   Char    DEC
  .TITLE[32]     Char    OMC1
! the more structure now contains an array of FITS items that hold the names
! and weights of the files input to the rebinning application, and the method
! of rebinning used.
  .MORE          Ext
    .FITS[80,9]  Char    FILE_1  = 'OBS_IFD_0207'       / Name of input
\end{verbatim}
\end{small}

\end{document}
