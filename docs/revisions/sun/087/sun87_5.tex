\documentstyle[11pt]{article} 
\pagestyle{myheadings}

%------------------------------------------------------------------------------
\newcommand{\stardoccategory}  {Starlink User Note}
\newcommand{\stardocinitials}  {SUN}
\newcommand{\stardocnumber}    {87.5}
\newcommand{\stardocauthors}   {P.\,T.\,Wallace \\ A.\,J.\,J.\,Broderick}
\newcommand{\stardocdate}      {9 May 1994}
\newcommand{\stardoctitle}     {JPL --- Solar System Ephemeris (v2.0)}
%------------------------------------------------------------------------------

\newcommand{\stardocname}{\stardocinitials /\stardocnumber}
\renewcommand{\_}{{\tt\char'137}}     % re-centres the underscore
\markright{\stardocname}
\setlength{\textwidth}{160mm}
\setlength{\textheight}{230mm}
\setlength{\topmargin}{-2mm}
\setlength{\oddsidemargin}{0mm}
\setlength{\evensidemargin}{0mm}
\setlength{\parindent}{0mm}
\setlength{\parskip}{\medskipamount}
\setlength{\unitlength}{1mm}

%------------------------------------------------------------------------------
% Add any \newcommand or \newenvironment commands here
%------------------------------------------------------------------------------

\begin{document}
\thispagestyle{empty}
DRAL / {\sc Rutherford Appleton Laboratory} \hfill \stardocname\\
{\large Particle Physics \& Astronomy Research Council}\\
{\large \bf Starlink Project}\\
{\large\bf \stardoccategory\ \stardocnumber}
\begin{flushright}
\stardocauthors\\
\stardocdate
\end{flushright}
\vspace{-4mm}
\rule{\textwidth}{0.5mm}
\vspace{5mm}
\begin{center}
{\Large\bf \stardoctitle}
\end{center}
\vspace{5mm}
\setlength{\parskip}{0mm}
\tableofcontents
\setlength{\parskip}{\medskipamount}
\markright{\stardocname}

\section{Introduction}
The Jet Propulsion Laboratory has for many years distributed
definitive solar system ephemeris tapes,
containing files of ephemeris data plus Fortran programs to read them.
The ephemerides come from direct numerical integrations of the equations
of motion of the large bodies in the solar system, subject to fitting to a
variety of observations.  For distribution they are expressed in
terms of Chebyshev polynomials.

The JPL ephemeris distributed by Starlink is the DE200/LE200, the one
most consistent with the latest IAU resolutions, for the range
1960 to 2025.
DE200/LE200 is described in detail
in Chapter~5 of the {\it Explanatory Supplement to the
Astronomical Almanac}\/ (ed.\ P.\,K.\,Seidelmann,
University Science Books, 1992).
It provides extremely accurate position and velocity data for any of the nine
planets, the Moon, and the solar system and Earth-Moon barycentres (also
nutation components) for any given epoch in the range of the ephemeris.

{\bf Users of the old (VMS-only) version
should note that this new (Unix and VMS) version differs
in important respects.  Existing applications will need
to be changed before they can be linked against the new
library, and existing executable programs are not
compatible with the new format of ephemeris file.  For further
details, see Section~6.3.}

\section{Package Contents}
The Starlink JPL package consists of the following:
\begin{itemize}
\item the JPL DE200/LE200 ephemeris
in a machine-independent text format;
\item an executable program which builds the operational ephemeris
for the machine concerned, starting with the machine-independent
text version supplied by JPL;
\item the ephemeris as a direct-access binary file, suitable for
the particular machine concerned;
\item TESTEPH -- an executable program which tests the built ephemeris for
accuracy.
\item a text file containing test data required
by the TESTEPH program.
\item an object library which includes three user-callable routines:
\begin{itemize}
\item PLEPH -- an easy-to-use subroutine which for a given date
returns the coordinates of one body or centre with respect to another;
\item STATE -- a more comprehensive subroutine which returns more
than one set of coordinates at once;
\item CONST -- a subroutine for enquiring the constants (masses {\it etc.})
used in generating the ephemeris.
\end{itemize}
\item SHORTEPH -- an executable program which extracts a specified
subset from the ephemeris (to save space);
\item JPLPLEPH, JPLSTATE, JPLCONST -- executable programs which demonstrate
the use of the user-callable routines;
\item Fortran source code for all of the above programs;
\item an \verb|mk| script and a \verb|makefile| for building the
system on Unix platforms;
\item various \verb|.COM| and \verb|.BAT| files for building the
system on other platforms;
\item SOURCE\_JPL -- the original JPL Fortran source, for information.
\item This document.
\end{itemize}

\section{A Simple Example}
It should be noted that the package includes
no self-contained utility programs which
interpolate the ephemeris directly, apart from some
simple examples.  Instead, the
facility is supplied as a set of subroutines which can be
called by the user's own application code.

For example, to obtain the position and velocity of
the Earth, with respect to the Solar System Barycentre,
at JD~2439105.7117 (TDB):
\begin{enumerate}
\item Write a Fortran program which includes the statements:
\begin{verbatim}
         DOUBLE PRECISION PV(6)
         LOGICAL OK
             :
         CALL PLEPH( 39105.2117D0, 3, 12, PV, OK )
\end{verbatim}
Details of the PLEPH arguments are given in Section~4.1.
\item Compile and link.
\begin{tabbing}
xxxxxxxxxx \= xxxxxxxxxxxxxxxxxxxxxxxxxxxxxxxxxxxxxxxxxxxxxxxx \= \kill
\> \verb|f77 myprog.f -L/star/lib `jpl_link` -o myprog| \> (Unix) \\ \\
\> \verb|FORTRAN myprog| \> (VMS) \\
\> \verb|LINK myprog,JPLDIR:JPL/L|
\end{tabbing}
\item Specify the ephemeris file:
\begin{tabbing}
xxxxxxxxxx \= xxxxxxxxxxxxxxxxxxxxxxxxxxxxxxxxxxxxxxxxxxxxxxxx \= \kill
\> \verb|ln /star/etc/jpl/jpleph.dat JPLEPH| \> (Unix) \\ \\
\> \verb|DEFINE JPLEPH JPLDIR:JPLEPH.DAT| \> (VMS)
\end{tabbing}
\item Run the program.
\end{enumerate}
(You may wish to remove link files, deassign logical names
{\it etc.}\ at this point.)

\section{Reading the Ephemeris}
The fundamental purpose of the package, and the one capability
of interest to most users, is to look up the position of
a planet for a given moment in time.
There is a choice of two routines to accomplish this:
\begin{itemize}
\item PLEPH, which is extremely simple to use, and gives
the coordinates of a single body relative to a specified
origin.
\item STATE, which is more flexible, and suitable for
cases where the coordinates
of several bodies are required at once.
\end{itemize}
The routines are described later, following these
important {\it caveats}\/:
\begin{itemize}
\item Both PLEPH and STATE
require as their time argument an epoch in a dynamical
timescale -- what used to be called
{\it Ephemeris Time}\/ -- rather than the UTC of everyday life,
and this can be a trap for the unwary.
PLEPH and STATE accept times in the
Barycentric Dynamical Time (TDB) timescale, which
at the time of writing is displaced from UTC by about
1~minute.  For a given UTC you can get the offset
to TDB by calling the SLALIB routines sla\_DTT and sla\_RCC
(Starlink User Note~67).  The offset to Terrestrial Time TT
returned by sla\_DTT dominates, and in practice the
call to sla\_RCC can be
dispensed with.  In fact, the JPL documentation is vague
about which of these dynamical timescales is the correct
form to use.  The difference between them
never exceeds about 1.3~milliseconds.
\item It is up to the caller to allow for {\it planetary
aberration}\/ (the effect of light-time).  A
common example is where the geocentric position of a planet is being
requested, for a given moment in time.  In such a case the
following procedure is necessary:
\begin{enumerate}
\item Call PLEPH to get the planet's coordinates
at the time of interest.
\item Calculate $\sqrt{x^{2}+y^{2}+z^{2}}$,
the distance to the planet at that time,
and hence the light-time.
\item Call PLEPH again for the time of interest minus the
light-time.
\item For extra accuracy, iterate from step 2.
\end{enumerate}
\end{itemize}

\subsection{The PLEPH Subroutine}
PLEPH reads and interpolates a single item --
usually the position and velocity
of one body relative to another --
from the JPL ephemeris file on disc.
The call is:
\begin{verbatim}
      CALL PLEPH (TDB, NP, NC, R, OK)
\end{verbatim}
with arguments as follows:

Given:
\begin{tabbing}
xxxx \= xxxxxxx \= xxxxxxxxxxxxxxxxx \=
xxxxxxxxxxxxxxxxxxxxxxxxxxxxxxxxxxxxxxx \kill
\> \verb|TDB| \> double precision \>
             \parbox[t]{24em}{The epoch:  the Barycentric
                              Dynamical Time
                              (TDB) expressed as a
                              Modified Julian Date (JD$-2400000.5$)} \\ \\
\> \verb|NP| \> integer \>
             \parbox[t]{24em}{The body or reference point whose
                              coordinates are required
                              (see the list, below)} \\ \\
\> \verb|NC| \> integer \>
             \parbox[t]{24em}{The origin of the coordinate system -- the
                              same scheme as for \verb|NP|}
\end{tabbing}
Returned:
\begin{tabbing}
xxxx \= xxxxxxxxx \= xxxxxxxxxxxxxxxxxxx \=
xxxxxxxxxxxxxxxxxxxxxxxxxxxxxxxxxxxxxxx \kill
\> \verb|R(6)| \> double precision \>
              \parbox[t]{24em}{An array to receive position and
                               velocity:
                               $x, y, z, \dot{x}, \dot{y}, \dot{z}$} \\ \\
\> \verb|OK| \> logical \>
              \parbox[t]{24em}{Status:  .TRUE. indicates success.
                               .FALSE. indicates TDB out of range or illegal
                               \verb|NP|, \verb|NC| value.}
\end{tabbing}
The \verb|NP| and \verb|NC| arguments are each as follows:
\begin{tabbing}
xxxxxxxxxxxx \= xxxxxxxxxxxxxx \= \kill
\> 1 \' Mercury  \\
\> 2 \' Venus    \\
\> 3 \' Earth    \\
\> 4 \' Mars     \\
\> 5 \' Jupiter  \\
\> 6 \' Saturn   \\
\> 7 \' Uranus   \\
\> 8 \' Neptune  \\
\> 9 \' Pluto    \\
\> 10 \' Moon     \\
\> 11 \' Sun      \\
\> 12 \' SSB      \\
\> 13 \' EMB
\end{tabbing}
In each case, the position and velocity of the point
specified by \verb|NP| relative to
the point specified by \verb|NC| are put into \verb|R(1-6)|.

For example, the following call will obtain the position and velocity of the
Earth relative to the Solar System barycentre at 0 hours TDB on
30th~April~1986:
\begin{verbatim}
      CALL PLEPH ( 46550D0, 3, 12, R, OK )
\end{verbatim}

PLEPH can return results in either AU and AU/day or km and km/sec.
The default is AU and AU/day; to select the other units, include the following
Fortran statements:
\begin{verbatim}
      LOGICAL KM
      COMMON /STCOMM/ KM
      SAVE STCOMM
          :
          :
      KM=.TRUE.
\end{verbatim}
Calls to PLEPH will then return results in km and km/sec.

PLEPH can also be used to read the JPL values for the nutation angles.
To do this, use \verb|NP|=14 and \verb|NC|=0.  The units in this case are
radians and radians/day;  with \verb|KM=.TRUE.| the units
are radians and radians/sec.  ({\it n.b.}\ As a rule,
the Starlink SLALIB routine sla\_NUTC is a
more convenient way to calculate the nutation -- see Starlink User
Note~67.)

\subsection{The STATE Subroutine}
The basic interpolating routine is STATE, which offers more
flexibility and in some cases
greater efficiency than the PLEPH routine just described.
In particular, STATE allows the coordinates of several bodies to be
determined at once.

The call is:
\begin{verbatim}
      CALL STATE (JD, LIST, PV, NUT, OK)
\end{verbatim}
with arguments as follows.

Given:
\begin{tabbing}
xxxx \= xxxxxxxxx \= xxxxxxxxxxxxxxx \=
xxxxxxxxxxxxxxxxxxxxxxxxxxxxxxxxxxxxxxx \kill
\> \verb|JD(2)| \> double precision \>
             \parbox[t]{24em}{An
         array specifying the
         instant of time for which the ephemeris lookup is to be
         done, in the TDB timescale.  The time is the sum
         \verb|JD(1)+JD(2)|.  A variety of different combinations of the
         two elements is possible.  The simplest is to put the
         whole Julian Date into \verb|JD(1)|
         and to set \verb|JD(2)| to zero.
         For optimum precision, set \verb|JD(1)| to the Julian Date at
         the previous 0~hours TDB (so \verb|JD(1)| will be \verb|24xxxxx.5D0|)
         and \verb|JD(2)| to the fraction of a day from 0~hours to the
         required time.  Sometimes it may prove convenient to
         set \verb|JED(1)| to some fixed epoch and
         to set \verb|JED(2)| to the
         interval between then and the epoch concerned.} \\ \\
\> \verb|LIST(12)| \> integer \>
             \parbox[t]{24em}{An array specifying what
         interpolation is wanted for each of the bodies (or
         reference points) on the file:}
\end{tabbing}
\begin{tabbing}
xxxxxxxxxxxxxxxxxxxxxxxxxxxxxxxxxxxxxxx \= xxxx \= xx \= x \= \kill
         \> \verb|LIST(I)| \' = \>\> 0: \'\> no lookup for body \verb|I| \\
                \>    \' = \>\> 1: \'\> position only \\
                \>    \' = \>\> 2: \'\> position and velocity \\ \\
         \> where     \' \\ \\
         \> \verb|I|  \' = \>\> 1: \'\> Mercury \\
                \>    \' = \>\> 2: \'\> Venus   \\
                \>    \' = \>\> 3: \'\> Earth-Moon Barycentre \\
                \>    \' = \>\> 4: \'\> Mars    \\
                \>    \' = \>\> 5: \'\> Jupiter \\
                \>    \' = \>\> 6: \'\> Saturn  \\
                \>    \' = \>\> 7: \'\> Uranus  \\
                \>    \' = \>\> 8: \'\> Neptune \\
                \>    \' = \>\> 9: \'\> Pluto   \\
                \>    \' = \>\> 10: \'\> Moon (geocentric) \\
                \>    \' = \>\> 11: \'\> Nutations
\end{tabbing}
\pagebreak
Returned:
\begin{tabbing}
xxxx \= xxxxxxxxx \= xxxxxxxxxxxxxxx \=
xxxxxxxxxxxxxxxxxxxxxxxxxxxxxxxxxxxxxxx \kill
\> \verb|PV|(6,11) \> double precision \>
             \parbox[t]{24em}{An
         array to receive
         the requested quantities.  The body
         specified by \verb|LIST(I)| will have its state in the array
         starting at \verb|PV(1,I)|.  (On any given call, only those
         elements in \verb|PV| which have been requested are set; the
         rest of the \verb|PV| array is untouched.)  The order of
         components starting in \verb|PV(1,I)|
         is $x, y, z, \dot{x}, \dot{y}, \dot{z}$.
         The Moon state is always geocentric;
         the other nine states are either heliocentric or
         solar-system-barycentric, depending on the setting of
         \verb|COMMON| flags (see below).} \\ \\
\> \verb|NUT(4)| \> double precision \>
             \parbox[t]{24em}{An
         array to receive the
         nutations and rates, depending on
         the setting of \verb|LIST(11)|.
         The order of quantities in \verb|NUT| is
         $d\psi, d\epsilon, d\dot{\psi}, d\dot{\epsilon}$.} \\ \\
\> \verb|OK| \> logical \>
             \parbox[t]{24em}{Status:  .TRUE. indicates success.}
\end{tabbing}
                     
Other information is given in the labelled \verb|COMMON|
block \verb|/STCOMM/|:
\begin{tabbing}
xxxx \= xxxxxxxxx \= xxxxxxxxxxxxxxx \=
xxxxxxxxxxxxxxxxxxxxxxxxxxxxxxxxxxxxxxx \kill
\> \verb|KM| \> logical \>
             \parbox[t]{24em}{A flag defining the physical units of the output
 states. For \verb|KM=.FALSE.| (the default),
 the units are AU and AU/day; for \verb|KM=.TRUE.| the units are
 km and km/sec.
 KM also determines the time unit for the nutation rates; the
 angle unit is always radians.} \\ \\
\> \verb|BARY| \> logical \>
             \parbox[t]{24em}{A flag defining the origin of the results.
Only the nine planets are affected.  \verb|BARY=.FALSE.| (the default)
means the coordinates are heliocentric;  setting \verb|BARY=.TRUE.|
selects the Solar System Barycentre as the origin.} \\ \\
\> \verb|PVSUN(6)| \> double precision \>
             \parbox[t]{24em}{A 6-element
         double precision array containing the
         barycentric position and velocity of the Sun.}
\end{tabbing}

\section{Other Facilities}

\subsection{How to Obtain the Constants of the Ephemeris}
Should it be necessary to examine the constants that were used to generate a
particular JPL ephemeris --
planetary masses, conversion factors, range of dates
covered, {\it etc.}\ -- this can be done by means of the subroutine CONST.
The call is as follows:
\begin{verbatim}
      CALL CONST (NAM, VAL, SSS, N)
\end{verbatim}
and returns the following:

\begin{tabbing}
xxxx \= xxxxxxxxx \= xxxxxxxxxxxxxxx \=
xxxxxxxxxxxxxxxxxxxxxxxxxxxxxxxxxxxxxxx \kill
\> \verb|NAM()| \> character*6 \>
             \parbox[t]{24em}{An array to receive the names of all
                              the constants.} \\ \\
\> \verb|VAL()| \> double precision \>
              \parbox[t]{24em}{An array to receive the
                               values of all the constants.} \\ \\
\> \verb|SSS(3)| \> double precision \>
              \parbox[t]{24em}{An array to receive the
                               start and stop epochs (JD) and
                               the step size.} \\ \\
\> \verb|N| \> integer \>
              \parbox[t]{24em}{The number of entries written into
                               the \verb|NAM|
                               and \verb|VAL| arrays.}
\end{tabbing}

The arrays \verb|NAM| and \verb|VAL|
must have sufficiently large dimensions to accommodate
all the constants in the ephemeris file.  A value of 400 is
more than adequate and likely to remain so;  the actual value
for the present ephemeris is returned as argument \verb|N|.

\subsection{Producing a Short Ephemeris File}
If disk space is at a premium, the SHORTEPH program
can be used to extract a subset of the ephemeris file.  Such
a file can be as short as required, depending on the
length of the interval of interest.  Access times are unaffected
because the ephemeris file is direct-access rather than sequential.

To run SHORTEPH under Unix:
\begin{verbatim}
      ln /star/etc/jpl/jpleph.dat JPLEPH
      ln jplephs.dat JPLEPHS
      /star/etc/jpl/shorteph
\end{verbatim}
to produce the short ephemeris file \verb|jplephs.dat|.
Under VMS, assign the logical name JPLEPH to
JPLDIR:JPLEPH.DAT and the logical name JPLEPHS to the file to be
written, and then run JPLDIR:SHORTEPH.

The program prompts for the calendar date limits in the
(fixed) format \verb|YYYYMMDD,YYYYMMDD|.  A start or end
date of all zeroes will retain the start or end date
of the existing ephemeris.
                                                                        
\subsection{Testing the Ephemeris}
The TESTEPH program compares values interpolated from
the ephemeris with several hundred known correct values read from a file
of test data.  It reports the number of results that were
exactly right, how many lay within a fractional
tolerance (currently set to $10^{-15}$)
and how many exceeded the tolerance.

The Starlink software-distribution
procedures automatically run TESTEPH after building the
direct-access form of the ephemeris, and there is
normally no need for users to repeat the test.  However, for
reassurance that the ephemeris is functioning correctly,
TESTEPH may be run as follows.  Under Unix:
\begin{verbatim}
      ln /star/etc/jpl/jpleph.dat JPLEPH
      ln /star/etc/jpl/testephinput.dat TESTEPHINPUT
      /star/etc/jpl/testeph
\end{verbatim}
Under VMS, assign the logical name
JPLEPH to JPLDIR:JPLEPH.DAT
and the logical name TESTEPHINPUT to JPLDIR:TESTEPHINPUT.DAT,
then run JPLDIR:TESTEPH.

On some platforms
all the results pass the test, while on others a handful of
results lie just outside the tolerance:  it depends on
the floating-point format.

\section{Further Information}

\subsection{Coordinate System}
The positions and velocities produced by JPL DE200/LE200 ephemeris
are referred to the J2000 mean equatorial triad (IAU~1976,
as used for the FK5 star catalogue).
The 6-vectors have elements $( x, y, z, \dot{x}, \dot{y}, \dot{z} )$
in a
right-handed Cartesian coordinate system with the $x$-axis aligned to the first
point of Aries and the $z$-axis to the north celestial pole.

\subsection{The Fortran Routines}
The routines supplied are, as far as possible, the
untouched JPL Fortran source, {\it circa}\/~1990.  However,
certain changes had to be made, as follows:
\begin{itemize}
\item The JPL-supplied PLEPH routine is not used;  a revised form of the
old Starlink PLEPH is used instead.  The latter was developed
before JPL distributed its own version, and accepts epochs
in the form of Modified Julian Dates rather than the full
Julian Dates required by the JPL
version (the new Starlink PLEPH accepts both forms).
\item Some COMMON blocks have been re-ordered to put longer
items (double precision) before shorter items (integers).
This avoids data alignment problems reported by some
compilers.
\item The 1990 JPL routines used an
``asterisk dummy argument'' error-handling
mechanism, forcing applications to violate Starlink
rules (see Starlink General Paper~16).  A mechanism based on
a logical status return has been substituted.
\item In the EPHOPN routine for certain platforms, the record
size specification units
has had to be changed from words to bytes.  In addition, in some
versions the OPEN statement has been changed to avoid the need
for write permission on the ephemeris file.
\end{itemize}
The original 1990 JPL source is included, for information, in the
Starlink distributions.  On Unix platforms, the source is in
file \verb|source_jpl.txt| within the tar file
\verb|/star/jpl/jpl_source.tar|.  Under VMS the source is in
file JPLDIR:SOURCE\_JPL.TXT.

Since work began on the Starlink release of this software, JPL
have issued new versions.  These address many of the problems
encountered during the construction of the Starlink release,
and in some cases similar changes have been introduced -- for
example COMMON blocks have been re-ordered, and
the JPL version of PLEPH now has a logical status
return.  The Starlink release is believed to be fully
compatible with the latest JPL product, insofar as it can
read the same ephemerides and delivers identical results.
However, there are small differences, such as the calling
sequence for the STATE subroutine, and the operating
instructions.

\subsection{Differences with respect to Previous Releases}
The Starlink release described by this
document differs in important respects from earlier versions.
Here is a list of the changes of significance to existing users:
\begin{itemize}
\item Earlier releases were for VAX/VMS only; the latest release
is for several Unix platforms in addition to VMS.
\item Earlier versions included a tape ephemeris (VAX-only) covering
1800-2050 and a disc ephemeris covering 1960-2000.  In addition,
the full 1800-2050 range was available on disc on the central
STADAT machine. The present
version includes a disc ephemeris covering 1960-2025, the full range
available from the tape received from JPL.  A DE200/LE200 tape covering a
greater range can be requested from JPL if need be, as can
different ephemerides.
\item The PLEPH subroutine has an extra argument compared with
the earlier version (a status flag).  Also, to improve compatibility
with the JPL-supplied version,
the epoch can be expressed as a JD rather than an MJD.
\item The STATE subroutine was formerly called XSTATE, and now has an
extra argument (a status flag).
\item The CONST subroutine was formerly called XCONST.
\item The earlier version opened files by I/O unit number;  the
new version has names such as JPLEPH (hardwired by JPL).  As a
consequence, the operating instructions have changed.
\end{itemize}

\section{Acknowledgements}
The DE200/LE200 ephemeris was produced jointly by the Jet Propulsion Laboratory,
the US Naval Observatory and the US Naval Surface Weapons Center.
The tape used in this release came {\it via}\/ R.\,Harrison
of Jodrell~Bank.  Earlier versions of this package
benefited from correspondence with
C.\,Hohenkerk of RGO and E.\,M.\,Standish of JPL.
\pagebreak
\appendix
\section{Appendix: Example Programs}
Three example application programs are supplied: JPLPLEPH, JPLSTATE
and JPLCONST.
On Unix systems, the source and executables for the
example programs are in the directory \verb|/star/etc/jpl|.
Under VMS, the
source is held in the
text library JPLDIR:SOURCE.TLB and the executables reside in
JPLDIR.
\subsection{Example using PLEPH}
The executable program JPLPLEPH interpolates the ephemeris for a
particular (hardwired) epoch, to demonstrate
the use of the subroutine PLEPH. Under
Unix, it can be run as follows:
\begin{verbatim}
      ln /star/etc/jpl/jpleph.dat JPLEPH
      /star/etc/jpl/jplpleph
\end{verbatim}
Under VMS, assign the logical name JPLEPH to
the ephemeris file JPLDIR:JPLEPH.DAT
and run the program JPLDIR:JPLPLEPH.

JPLPLEPH calls the subroutine PLEPH to interpolate from the ephemeris file the
heliocentric position and velocity of the Earth for the instant
TDB~=~2451624.5.  It produces the following output (depending
on machine precision):

\begin{verbatim}
       -0.9960816789895932E+00
       -0.1092611522464465E-01
       -0.4738690663777706E-02
       -0.7521474026186866E-04
       -0.1583616804758534E-01
       -0.6866074103775283E-02
\end{verbatim}
The Fortran source of JPLPLEPH follows.
   
\begin{verbatim}
      PROGRAM JPLPLEPH
*+
*  - - - - - - - - -
*   J P L P L E P H
*  - - - - - - - - -
*
*  An example program using the Starlink subroutine PLEPH to
*  interpolate the JPL planetary ephemeris.  The heliocentric
*  Earth position and velocity for an instant near the vernal
*  equinox in 2000AD are determined.
*
*  Input is from the binary direct-access form of the JPL ephemeris
*  (as output by the EPHDSK program).  The file is called JPLEPH.
*
*  P.T.Wallace   Starlink   22 April 1994
*-

      IMPLICIT NONE

*  PLEPH arguments
      DOUBLE PRECISION TDB
      INTEGER NP,NC
      DOUBLE PRECISION PV(6)
      LOGICAL OK



*  TDB = Julian Date 2451624.5
      TDB=51624D0

*  Body is the Earth
      NP=3

*  Origin is the Sun
      NC=11

*  Read and interpolate the ephemeris
      CALL PLEPH(TDB,NP,NC,PV,OK)

*  List the resulting position and velocity
      IF (OK) WRITE (*,'(6(1X,E24.16/))') PV

      END
\end{verbatim}
\
\subsection{Example using STATE}
The program JPLSTATE interpolates the ephemeris for a
particular (hardwired) epoch, to demonstrate
the use of the subroutine STATE.  Under Unix, it can be run as
follows:
\begin{verbatim}
      ln /star/etc/jpl/jpleph.dat JPLEPH
      /star/etc/jpl/jplstate
\end{verbatim}
Under VMS, assign the logical name JPLEPH to the
ephemeris file JPLDIR:JPLEPH.DAT
and run the program JPLDIR:JPLSTATE.

JPLSTATE calls the subroutine STATE to interpolate from the ephemeris file the
heliocentric position and velocity of the Earth for the instant
TDB~=~2451624.5. The following output is produced (depending on
machine precision):

\begin{verbatim}
       -0.9961121408482240E+00
       -0.1093240728275330E-01
       -0.4738623145919944E-02
       -0.7419740229401869E-04
       -0.1584284216910741E-01
       -0.6868683708183170E-02
\end{verbatim}
The Fortran source for JPLSTATE follows.

\begin{verbatim}
      PROGRAM JPLSTATE
*+
*  - - - - - - - - -
*   J P L S T A T E
*  - - - - - - - - -
*
*  Example program using the subroutine STATE, in the JPL Ephemeris
*  package, to interpolate the planetary ephemeris.  The
*  heliocentric position and velocity of the Earth-Moon
*  barycentre for a given epoch near the vernal equinox of
*  2000AD are determined.
*
*  Input is from the binary direct-access form of the JPL ephemeris
*  (as output by the EPHDSK program).
*
*  P.T.Wallace   Starlink   25 April 1994
*-

      IMPLICIT NONE

*  Julian date for ephemeris interpolation (see STATE source)
      DOUBLE PRECISION DJ(2)

*  Flags indicating which values are required
      INTEGER LIST(12)

*  Array to receive ephemeris data (1,2 and 4-11 not used)
      DOUBLE PRECISION PV(6,11)

*  Array to receive nutation (not used)
      DOUBLE PRECISION DNUT(4)

      INTEGER I
      LOGICAL OK



*  Set up LIST array to interpolate for Earth only
      DO I=1,12
         LIST(I)=0
      END DO
      LIST(3)=2

*  Put Julian Date 2451624.5 in DJ(1) and DJ(2).
      DJ(1)=2451624.5D0
      DJ(2)=0D0

*  Interpolate the ephemeris and print results
      CALL STATE(DJ,LIST,PV,DNUT,OK)
      IF (OK) WRITE (*,'(6(1X,E24.16/))') (PV(I,3),I=1,6)

      END
\end{verbatim}

\subsection{Example using CONST}
The program JPLCONST demonstrates the use of the CONST subroutine to obtain the
constants of the ephemeris.  Under Unix, it can be run as follows:
\begin{verbatim}
      ln /star/etc/jpl/jpleph.dat JPLEPH
      /star/etc/jpl/jplconst
\end{verbatim}
Under VMS, assign the logical name JPLEPH to
the ephemeris file JPLDIR:JPLEPH.DAT
and run the program JPLDIR:JPLCONST.

JPLCONST calls the CONST subroutine and then
lists all the constants of the ephemeris.  (The listing is too long
to be included here.)
See also the Fortran source of the PLEPH subroutine, which shows
how the data output by CONST can be searched for named constants.
The Fortran source of JPLCONST follows.

\begin{verbatim}
      PROGRAM JPLCONST
*+
*  - - - - - - - - -
*   J P L C O N S T
*  - - - - - - - - -
*
*  Example program using the JPL routine CONST.
*
*  The program calls CONST, which reads from the JPL Ephemeris
*  file in binary direct-access form (as output by the program
*  EPHDSK).
*
*  P.T.Wallace   Starlink   18 April 1994
*-

      IMPLICIT NONE

*  Array for data names
      CHARACTER*6 NAM(400)

*  Arrays for data values
      DOUBLE PRECISION VAL(400)
      DOUBLE PRECISION SS(3)

*  Number of data values and counter
      INTEGER N,I



*  Pick up the information
      CALL CONST(NAM,VAL,SS,N)

* Print the results
      DO I=1,N
         WRITE(*,'(1X,I3,4X,A,4X,D20.14)')  I,NAM(I),VAL(I)
      END DO

      DO I=1,3
         PRINT *,SS(I)
      END DO

      END
\end{verbatim}

\newpage
\section{Appendix: Documentation Provided by JPL}
The JPL-supplied
document ``The JPL Export Planetary Ephemeris'' is reprinted below,
{\it verbatim}\/ apart from small formatting changes.  Please
note that some of the instructions are incorrect with respect to the
Starlink implementation, for example the argument lists of certain
subroutines.

\vspace{10mm}

%%%%%%%%%%%%%%%%%%%%%%%%%%%%%%%%%%%%%%%%%%%%%%%%%%%%%%%%%%%%%%%%%%%%%%%%%%
%                                                                        %
%  JPL TeX with small modifications by Starlink for LaTeX compatibility  %
%                                                                        %
%%%%%%%%%%%%%%%%%%%%%%%%%%%%%%%%%%%%%%%%%%%%%%%%%%%%%%%%%%%%%%%%%%%%%%%%%%

%\magnification=1095                                   % Starlink
%\hsize=6.5truein\vsize=8.9truein                      % Starlink
%\hoffset=.0truein\voffset=.7truein                    % Starlink
\parindent=0pt\parskip=8pt
\baselineskip=11.pt plus.5pt minus.5pt
\lineskiplimit=1pt\lineskip=1pt
\font\tt=cmtt10
\font\csc=cmcsc10
\font\bld=cmb10
\newcount\inum
\def\vp{\vskip\parskip}
\def\pf{\par\filbreak}
\def\el{\hfill\break}
\def\itm#1{\leavevmode\hangindent.75truein\hangafter=1
\advance\inum by 1
\hbox to2em{\rm\hfil\number\inum. }#1\vadjust{\kern5pt}\el}
\def\ul#1{$\underline{\hbox{\sl\strut#1}}$}
\def\hdg#1{\vskip4pt plus2pt\leavevmode\kern-1em \ul{#1}\par}
\centerline{\bf The JPL Export Planetary Ephemeris}
\vskip2pt
\hdg{Introduction}
The Jet Propulsion Laboratory is engaged in the production of
high-precision planetary and lunar ephemerides. Requests
for the ephemerides are frequently received
from interested users at institutions throughout the world.
In an effort to fulfill these requests in as efficient and
useful a fashion as possible, a machine-independent form of the
ephemerides and supporting software has been prepared.
This document describes its use.
 
The purpose of the export package is to provide users a convenient
means of establishing a file of ephemeris data on direct-access
mass storage and then, in subsequent programs,
calling a subroutine to obtain cartesian
coordinates of planetary position and velocity at any epoch covered
by the file.
 
Many users have the earlier edition of the export ephemeris
package. That set was created in 1976, before Fortran~77 was
the available standard.
The major differences between the current export package and
the older version are that (1)~the new software is
written entirely in standard Fortran~77, requiring no programming changes
whatever, and (2) the resulting files are written on disk as
direct-access files, yielding considerably improved read-access time.
Because of these changes,
the revised software structure and ephemeris
file formats described in this document are {\it not\/} compatible
with the earlier formats and cannot be used with them.
Anyone wishing to use future versions of the JPL ephemerides
will have to adapt his software to use the routines described
herein.
 
Although several users have had immediate success with the export
package and subsequent application of the ephemerides, the complete
absence of errors cannot be guaranteed. If you have difficulty
or discover errors in either the software or ephemeris file,
we would appreciate your notifying us at once. And any suggestions
for the improvement of this document are most welcome.
 
\hdg{The Export Tape}
The standard form of ephemeris-file transmittal is {\sl via\/}
a magnetic tape having two files. The first file contains
all necessary software for creating, testing, and using the file
on the user's computer. The card images are written on tape as a sequence
of blocked 80-character {\csc ascii} records. There are 24 records
(1920 bytes)
per block, unless requested otherwise by the user.
 
The second file on the tape contains the ephemeris information,
also encoded as a series of 1920-byte {\csc ascii} records.
It is the function of one of the programs ({\csc ephdsk}) in the software
set to decode the images on this file and produce the direct-access
file on mass storage.
 
This decoding of the export file can be
time-consuming. Because of the frequency of requests for ephemerides
on certain machines, binary-written tapes that avoid the
decoding process can be provided for {\csc vax, cdc cyber, modcomp},
Honeywell, Univac, and IBM System/360 computers.
The export cover letter will state the type of tape sent to you.

\pagebreak                                          % Starlink
\hdg{Getting Started}
Your first task is to extract and compile the thirteen program units
in file~1 of the magnetic tape. There are three main programs
({\csc ephdsk, shorteph, testeph}),
nine subroutines ({\csc const, ctoj, ephopn, interp, jtoc, pleph,
rci, split, state}), and one block data subprogram ({\csc comdat}).
The start and end of each are clearly indicated
by comment cards. All are written in {\csc ansi}-standard Fortran~77
and should compile without difficulty. When compilation
is complete you are ready
to create the direct-access ephemeris file.
(There is a fourteenth set of cards in the
software file, intended
as input to the ephemeris testing program. Its use is described later.)
 
{\bld Note: Some Fortran compilers have extensions or features that do not
follow ANSI standards. See Table~4 at the end of this document
for remarks concerning possible incompatibilities.}
 
% \settabs 8 \columns                                    % Starlink
\hdg{Creating Your File on Mass Storage}
Link the main program {\csc ephdsk} with the subroutines {\csc ctoj, ephopn,
jtoc, rci}, and {\csc comdat}. The program expects to read the tape on a
file having the logical name {\csc ephtap} and will write the output on the
logical file {\csc jpleph}. (These names are logical names, in that there
are no terminal periods in the Fortran {\tt OPEN} statements; the user
may wish to assign them {\sl via} operating system control
statements to files whose actual names are different.)
 
Position the tape at the start of file~2.
The density should be set to whatever is indicated
on the tape reel. The record size you must specify to the
executive language of your operating system is 80 bytes. The
block size is 1920 bytes (unless stated otherwise in the export
cover letter).
 
Run the program.
You will be prompted for two inputs.
The first is for a character specifying the format of file~2
({\tt A} for {\csc ascii} or {\tt B} for Binary). The format of your
file is stated in the export cover letter.
 
The second prompt is for an optional shortened time span of the
output ephemeris file. Ephemerides sent by JPL often cover
60 years or more and may exceed your needs.
You will be asked for new starting and ending integer
calendar date limits, in the form
% Starlink mods
%\vp
%\+&{\tt YYYYMMDD{\rm,}YYYYMMDD}\cr
\begin{verbatim}
        YYYYMMDD,YYYYMMDD
\end{verbatim}
% End of Starlink mods
 
If you wish to retain either the original starting or ending epoch,
enter a 0 in the desired field. To retain the full span
of the input file, enter only a {\tt <return>}.
 
On some smaller computers this file-creation
step can be time-consuming, requiring possibly five minutes or
more per decade of the ephemeris span. Bear in mind that this
operation is done only once, after which the export tape
can be set aside or archived.
 
\hdg{Testing the Ephemeris File}
When {\csc ephdsk} has completed execution, the direct-access file
{\csc jpleph} will be ready for use. It is strongly advisable, though
not mandatory, to test the integrity of the software
and of the file on mass storage.
One element of the software file is the main program {\csc testeph}.
That program should be compiled and linked with the elements
{\csc const, ephopn, interp, pleph, split, state}, and {\csc
comdat}.
 
When {\csc testeph} is run, it calls for no keyboard inputs; however,
it expects a set of input cards to be in the file {\csc testephinput}.
They are located in file~1 of the export tape following the last
Fortran deck. They are bounded by the initial and terminal
comment cards of the form
% Starlink mods
% \vp
% \+&{\tt Test~output~from~JPL~PLANETARY~EPHEMERIS:~DE-xxx~+~LE-xxx}\cr
% \+&{$\cdots$}\cr
% \+&{\tt End~of~Test~output~from~JPL~PLANETARY~EPHEMERIS:~DE-xxx~+~LE-xxx}\cr
\begin{verbatim}
        Test output from JPL PLANETARY EPHEMERIS: DE-xxx + LE-xxx
        ...
        End of Test output from JPL PLANETARY EPHEMERIS: DE-xxx + LE-xxx
\end{verbatim}
% End of Starlink mods
 
There are approximately 600 cards in the test deck. The entire deck, {\it
including the two comment cards}, must be copied to the  logical file
{\csc testephinput} before {\csc testeph} is run. The program compares
the names and values of all constants on the file with the
corresponding quantities produced by the originating software at
JPL and put in the test deck. It also interpolates the ephemeris
at various points and compares the results with JPL-supplied values
in the test deck.
 
Because the values in the test deck are encoded representations of
double-precision numbers, comparison in some cases will
not be exact but will match only to within a fractional tolerance.
That tolerance
is pre-coded as $10~{-15}$ in the {\csc testeph} variable
{\tt CFRAC}. Any comparison exceeding this value will be
printed on the output medium.
 
The interpolation times in the test deck are evenly spaced
and cover the full range of the {\it original\/} export file sent from JPL.
If you elected to shorten the ephemeris using the optional calendar
date inputs described above, {\csc testeph}
will ignore points lying outside
the range of your shortened file.
 
\hdg{Using the Ephemeris File}
When the ephemeris file has been satisfactorily established
on mass storage it can be accessed by the export software.
The two subroutines of direct interest to the user are
(1) {\csc pleph}, which provides interpolated
cartesian planetary positions and
velocities, and (2) {\csc const}, which returns the
names and values of initial conditions and other astronomical
constants employed in the creation of the ephemeris. The
names and meanings of the quantities output by {\csc const}
are given in Table~1.
 
{\csc pleph} is the most convenient subroutine for ephemeris access.
It supplies the interpolated position and velocity ($x,\>y,\>z,
\>\dot x,\>\dot y,\>\dot z$)
of any target reference point with respect to an arbitrary center.
Both the target and center points can be any of the nine
planets, the Sun, the Moon, the Earth-Moon barycenter, or the
solar-system barycenter. It also provides interpolated nutations
in longitude and obliquity ($\psi,\>\epsilon,\>\dot\psi,
\>\dot\epsilon$), if requested.
 
{\csc state}, one of the subroutines used by {\csc pleph},
may be used directly to obtain interpolated ephemeris quantities.
It differs from {\csc pleph} in that all planets, the Moon, and
nutations may be requested in a single call; however, its disadvantage
is that only heliocentric or solar-system-barycentric states are output;
users must do their own vector addition and scaling if other
reference points are desired. For simplicity the use of {\csc pleph}
for the retrieval of ephemeris quantities is recommended.
 
All programs and subroutines are extensively documented in their
listings; for completeness, the instructions for {\csc pleph}
and {\csc const} are reproduced in Tables~2 and~3.
 
Any program using {\csc pleph} will need to include
{\csc pleph, ephopn, interp, split, state}, and {\csc comdat}
in the linking process. The linking requirements for
programs using {\csc const} are {\csc const, ephopn}, and
{\csc comdat}.
 
The output from {\csc pleph \rm and \csc state}
has default units of {\csc au} and {\csc au}/day. If km and km/sec
are desired, the user must include the following statements
in his main program {\it before\/} the first call to {\csc state}
or {\csc pleph}:
% Starlink mods
%\tt
%\vp
%\+&LOGICAL KM\cr
%\+&COMMON/STCOMM/KM\cr
%\+&$\cdots$\cr
%\+&KM=.TRUE.\cr
%
%\rm
\begin{verbatim}
        LOGICAL KM
        COMMON/STCOMM/KM
        ...
        KM=.TRUE.
\end{verbatim}
% End of Starlink mods
The default units for nutations are radians and radians/day. If
{\tt KM=.TRUE.} as in the above prescription, nutations have units
of radians and radians/second.

\newpage                                             % Starlink
\hdg{Ephemeris File Size}
Intervals spanned by JPL ephemerides are segmented into contiguous
32-day granules (64-day granules for DE-102),
each granule being represented by a
record of coefficients occupying approximately 7200 bytes.
There are also two
header records on all files. A 40-year file will occupy about
3.3~megabytes for all ephemerides except DE-102, for which the
space required is approximately 1.65~megabytes.
% \eject                                                    Starlink
\hdg{Producing a Short Ephemeris}
Because ephemeris files are direct-access, the actual size of a given
file residing on disk has no influence on read-access times.
However, at some institutions mass storage space is at a premium, and
an abbreviated file may be desirable if only a short span of ephemeris
epochs is needed. The program {\csc shorteph} will produce a new
ephemeris containing only those data records needed to include
a range specified by the user. ({\csc shorteph} is intended to
extract and copy a portion of a file that is already on mass storage.
It does not apply to the case of the original reading and
conversion of the export tape.)
 
To use {\csc shorteph}, first compile and link it with {\csc
ctoj, ephopn, jtoc, \rm and \csc comdat}. The input ephemeris
file is expected on logical file {\csc jpleph}; the shortened output file
will be written into the logical name {\csc jplephs}.
 
You will be asked for new starting and ending integer
calendar date limits, in the form
% Starlink mods
%\vp
%\+&{\tt YYYYMMDD{\rm,}YYYYMMDD}\cr
\begin{verbatim}
        YYYYMMDD,YYYYMMDD
\end{verbatim}
% End of Starlink mods
If you wish to retain either the original starting or ending epoch,
enter a 0 in the desired field.
 
\hdg{Ephemeris Coordinates}
If the DE-\# of an ephemeris is less than 200 (except for DE-102),
its coordinate
system is with respect to the equator and equinox of 1950.0, as
defined by the FK4 stellar catalog reference system. If the number
is 200 or greater, the coordinates are with respect to the mean
equator and equinox of J2000, coinciding with the origin of the FK5.
 
DE-102, the ``long ephemeris,'' has an origin which is closer
to the B1950.0 dynamical equinox than to the FK4; however, the
difference is only about $0.\kern-3pt''4$, negligible for
most applications.
 
\hdg{Machine Particulars}
There seems to be no uniform set of Input/Output standards in use by
the computer industry, particularly with regard to reading
binary tapes. Some observations and suggestions are given in Table~4 of
this document.
 
\hdg{Ephemeris Information}
Technical documentation for JPL ephemerides is given in
Newhall {\it et al.}~1983, {\it Astron. Astrophys. \bf 125},
150--167.
For questions or further information concerning the export
ephemeris, please contact
$$\hbox{\vtop{\hbox{E.~Myles Standish, Jr.}%
\hbox{JPL, 301-150}\hbox{Pasadena, CA\ \ 91109-8099}%
\kern4pt
\hbox{(818) 354-3959 (commercial)}%
\hbox{\phantom{(818)} 792-3959 (FTS)}}\hbox to.7truein{}
\vtop{\hbox{X~X~(Skip)~Newhall}\hbox{JPL, 238-332}%
\hbox{Pasadena, CA\ \ 91109-8099}%
\kern4pt
\hbox{(818) 354-0000 (commercial)}%
\hbox{\phantom{(818)} 792-0000 (FTS)}}}$$
%\bye                                                    Starlink

\newpage                                               % Starlink

%%%%%%%%%%%%%%%%%%%%%%%%%%%%%%%%%%%%%%%%%%%%%%%%%%%%%%%%%%%%%%%%%%%%%%%%%%

%\magnification=1095                                   % Starlink
%\hsize=6.5truein\vsize=8.9truein                      % Starlink
%\hoffset=.0truein\voffset=.7truein                    % Starlink
\parindent=0pt\parskip=8pt
\baselineskip=11.pt plus.5pt minus.5pt
\lineskiplimit=1pt\lineskip=1pt
\font\tt=cmtt10
\font\csc=cmcsc10
\newcount\inum
\def\vp{\vskip\parskip}
\def\pf{\par\filbreak}
\def\el{\hfill\break}
\def\itm#1{\leavevmode\hangindent.75truein\hangafter=1
\advance\inum by 1
\hbox to2em{\rm\hfil\number\inum. }#1\vadjust{\kern5pt}\el}
\def\ul#1{$\underline{\hbox{\sl\strut#1}}$}
\def\hdg#1{\vskip4pt\leavevmode\kern-1em \ul{#1}\par}
% \pageno=6                                             % Starlink
\centerline{\bf Table 2. PLEPH Calling Sequence}
\vskip8pt
% Starlink mods
%{\tt
%C++++++++++++++++++++++++++\el
%C\el
%\phantom{C}~~~~~SUBROUTINE~PLEPH(JD,TARG,CENT,RRD,*)\el
%C\el
%C++++++++++++++++++++++++++\el
%C\el
%C~~~~~THIS~SUBROUTINE~READS~THE~JPL~PLANETARY~EPHEMERIS\el
%C~~~~~AND~GIVES~THE~POSITION~AND~VELOCITY~OF~THE~POINT~'TARG'\el
%C~~~~~WITH~RESPECT~TO~'CENT'.\el
%C\el
%C~~~~~CALLING~SEQUENCE~PARAMETERS:\el
%C\el
%C~~~~~~~JD~=~D.P.~JULIAN~EPHEMERIS~DATE~AT~WHICH~INTERPOLATION\el
%C~~~~~~~~~~~~IS~WANTED.\el
%C\el
%C~~~~~TARG~=~INTEGER~NUMBER~OF~'TARGET'~POINT.\el
%C\el
%C~~~~~CENT~=~INTEGER~NUMBER~OF~CENTER~POINT.\el
%C\el
%C~~~~~~~~~~~~THE~NUMBERING~CONVENTION~FOR~'TARG'~AND~'CENT'~IS:\el
%C\el
%C~~~~~~~~~~~~~~~~1~=~MERCURY~~~~~~~~~~~8~=~NEPTUNE\el
%C~~~~~~~~~~~~~~~~2~=~VENUS~~~~~~~~~~~~~9~=~PLUTO\el
%C~~~~~~~~~~~~~~~~3~=~EARTH~~~~~~~~~~~~10~=~MOON\el
%C~~~~~~~~~~~~~~~~4~=~MARS~~~~~~~~~~~~~11~=~SUN\el
%C~~~~~~~~~~~~~~~~5~=~JUPITER~~~~~~~~~~12~=~SOLAR-SYSTEM~BARYCENTER\el
%C~~~~~~~~~~~~~~~~6~=~SATURN~~~~~~~~~~~13~=~EARTH-MOON~BARYCENTER\el
%C~~~~~~~~~~~~~~~~7~=~URANUS~~~~~~~~~~~14~=~NUTATIONS~(LONGITUDE~AND~OBLIQ)\el
%C~~~~~~~~~~~~~~~~~~~~~~~~~~~~15~=~LIBRATIONS,~IF~ON~EPH~FILE\el
%C\el
%C~~~~~~~~~~~~~(IF~NUTATIONS~ARE~WANTED,~SET~TARG~=~14.~FOR~LIBRATIONS,\el
%C~~~~~~~~~~~~~~SET~TARG~=~15.~'CENT'~WILL~BE~IGNORED~ON~EITHER~CALL.)\el
%C\el
%C~~~~~~RRD~=~OUTPUT~6-WORD~D.P.~ARRAY~CONTAINING~POSITION~AND~VELOCITY\el
%C~~~~~~~~~~~~OF~POINT~'TARG'~RELATIVE~TO~'CENT'.~THE~UNITS~ARE~AU~AND\el
%C~~~~~~~~~~~~AU/DAY.~FOR~LIBRATIONS~THE~UNITS~ARE~RADIANS~AND~RADIANS\el
%C~~~~~~~~~~~~PER~DAY.~IN~THE~CASE~OF~NUTATIONS~THE~FIRST~FOUR~WORDS~OF\el
%C~~~~~~~~~~~~RRD~WILL~BE~SET~TO~NUTATIONS~AND~RATES,~HAVING~UNITS~OF\el
%C~~~~~~~~~~~~RADIANS~AND~RADIANS/DAY.\el
%C\el
%C~~~~~~~~~~~~NOTE:~IN~MANY~CASES~THE~USER~WILL~NEED~ONLY~POSITION\el
%C~~~~~~~~~~~~~~~~~~VALUES~FOR~EPHEMERIDES~OR~NUTATIONS.~FOR\el
%C~~~~~~~~~~~~~~~~~~POSITION-ONLY~OUTPUT,~THE~INTEGER~VARIABLE~'IPV'\el
%C~~~~~~~~~~~~~~~~~~IN~THE~COMMON~AREA~/PLECOM/~SHOULD~BE~SET~=~1\el
%C~~~~~~~~~~~~~~~~~~BEFORE~THE~NEXT~CALL~TO~PLEPH.~(ITS~DEFAULT\el
%C~~~~~~~~~~~~~~~~~~VALUE~IS~2,~WHICH~RETURNS~BOTH~POSITIONS~AND\el
%C~~~~~~~~~~~~~~~~~~RATES.)\el
%C\el
%C~~~~~~~~*~=~STATEMENT~\#~IN~CASE~OF~ERROR~RETURN~(EPOCH~OUT~OF~RANGE\el
%C~~~~~~~~~~~~OR~REQUEST~FOR~NUTATIONS~OR~LIBRATIONS~WHEN~NOT~ON~FILE).
%}
\begin{verbatim}
C++++++++++++++++++++++++++
C
      SUBROUTINE PLEPH(JD,TARG,CENT,RRD,*)
C
C++++++++++++++++++++++++++
C
C     THIS SUBROUTINE READS THE JPL PLANETARY EPHEMERIS
C     AND GIVES THE POSITION AND VELOCITY OF THE POINT 'TARG'
C     WITH RESPECT TO 'CENT'.
C
C     CALLING SEQUENCE PARAMETERS:
C
C       JD = D.P. JULIAN EPHEMERIS DATE AT WHICH INTERPOLATION
C            IS WANTED.
C
C     TARG = INTEGER NUMBER OF 'TARGET' POINT.
C
C     CENT = INTEGER NUMBER OF CENTER POINT.
C
C            THE NUMBERING CONVENTION FOR 'TARG' AND 'CENT' IS:
C
C                1 = MERCURY           8 = NEPTUNE
C                2 = VENUS             9 = PLUTO
C                3 = EARTH            10 = MOON
C                4 = MARS             11 = SUN
C                5 = JUPITER          12 = SOLAR-SYSTEM BARYCENTER
C                6 = SATURN           13 = EARTH-MOON BARYCENTER
C                7 = URANUS           14 = NUTATIONS (LONGITUDE AND OBLIQ)
C                            15 = LIBRATIONS, IF ON EPH FILE
C
C             (IF NUTATIONS ARE WANTED, SET TARG = 14. FOR LIBRATIONS,
C              SET TARG = 15. 'CENT' WILL BE IGNORED ON EITHER CALL.)
C
C      RRD = OUTPUT 6-WORD D.P. ARRAY CONTAINING POSITION AND VELOCITY
C            OF POINT 'TARG' RELATIVE TO 'CENT'. THE UNITS ARE AU AND
C            AU/DAY. FOR LIBRATIONS THE UNITS ARE RADIANS AND RADIANS
C            PER DAY. IN THE CASE OF NUTATIONS THE FIRST FOUR WORDS OF
C            RRD WILL BE SET TO NUTATIONS AND RATES, HAVING UNITS OF
C            RADIANS AND RADIANS/DAY.
C
C            NOTE: IN MANY CASES THE USER WILL NEED ONLY POSITION
C                  VALUES FOR EPHEMERIDES OR NUTATIONS. FOR
C                  POSITION-ONLY OUTPUT, THE INTEGER VARIABLE 'IPV'
C                  IN THE COMMON AREA /PLECOM/ SHOULD BE SET = 1
C                  BEFORE THE NEXT CALL TO PLEPH. (ITS DEFAULT
C                  VALUE IS 2, WHICH RETURNS BOTH POSITIONS AND
C                  RATES.)
C
C        * = STATEMENT \# IN CASE OF ERROR RETURN (EPOCH OUT OF RANGE
C            OR REQUEST FOR NUTATIONS OR LIBRATIONS WHEN NOT ON FILE).
\end{verbatim}
% \bye
% End of Starlink mods

\newpage                                               % Starlink

%%%%%%%%%%%%%%%%%%%%%%%%%%%%%%%%%%%%%%%%%%%%%%%%%%%%%%%%%%%%%%%%%%%%%%%%%%

%\magnification=1095                                   % Starlink
%\hsize=6.5truein\vsize=8.9truein                      % Starlink
%\hoffset=.0truein\voffset=.7truein                    % Starlink
\parindent=0pt\parskip=8pt
\baselineskip=11.pt plus.5pt minus.5pt
\lineskiplimit=1pt\lineskip=1pt
\font\tt=cmtt10
\font\csc=cmcsc10
\newcount\inum
\def\vp{\vskip\parskip}
\def\pf{\par\filbreak}
\def\el{\hfill\break}
\def\itm#1{\leavevmode\hangindent.75truein\hangafter=1
\advance\inum by 1
\hbox to2em{\rm\hfil\number\inum. }#1\vadjust{\kern5pt}\el}
\def\ul#1{$\underline{\hbox{\sl\strut#1}}$}
\def\hdg#1{\vskip4pt\leavevmode\kern-1em \ul{#1}\par}
% \pageno=7
\centerline{\bf Table 3. CONST Calling Sequence}
\vskip8pt
% Starlink mods
%{\tt
%C+++++++++++++++++++++++++++++\el
%C\el
%\phantom{C}~~~~~SUBROUTINE~CONST(NAM,VAL,SSS,N)\el
%C\el
%C+++++++++++++++++++++++++++++\el
%C\el
%C~~~~~THIS~SUBROUTINE~OBTAINS~THE~CONSTANTS~FROM~THE~EPHEMERIS~FILE\el
%C\el
%C~~~~~CALLING~SEQEUNCE~PARAMETERS~(ALL~OUTPUT):\el
%C\el
%C~~~~~~~NAM~=~CHARACTER*6~ARRAY~OF~CONSTANT~NAMES\el
%C\el
%C~~~~~~~VAL~=~D.P.~ARRAY~OF~VALUES~OF~CONSTANTS\el
%C\el
%C~~~~~~~SSS~=~D.P.~JD~START,~JD~STOP,~STEP~OF~EPHEMERIS\el
%C\el
%C~~~~~~~~~N~=~INTEGER~NUMBER~OF~ENTRIES~IN~'NAM'~AND~'VAL'~ARRAYS\el
%C\el
%C~~~~~THE~ARRAYS~'NAM'~AND~'VAL'~MUST~HAVE~SUFFICIENTLY~LARGE\el
%C~~~~~DIMESNIONS~TO~ACCOMMODATE~ALL~ENTRIES~ON~THE~FILE.~THE\el
%C~~~~~VALUE~400~IS~A~SAFE~UPPER~LIMIT~FOR~THIS~DIMENSION.
%}
\begin{verbatim}
C+++++++++++++++++++++++++++++
C
      SUBROUTINE CONST(NAM,VAL,SSS,N)
C
C+++++++++++++++++++++++++++++
C
C     THIS SUBROUTINE OBTAINS THE CONSTANTS FROM THE EPHEMERIS FILE
C
C     CALLING SEQEUNCE PARAMETERS (ALL OUTPUT):
C
C       NAM = CHARACTER*6 ARRAY OF CONSTANT NAMES
C
C       VAL = D.P. ARRAY OF VALUES OF CONSTANTS
C
C       SSS = D.P. JD START, JD STOP, STEP OF EPHEMERIS
C
C         N = INTEGER NUMBER OF ENTRIES IN 'NAM' AND 'VAL' ARRAYS
C
C     THE ARRAYS 'NAM' AND 'VAL' MUST HAVE SUFFICIENTLY LARGE
C     DIMESNIONS TO ACCOMMODATE ALL ENTRIES ON THE FILE. THE
C     VALUE 400 IS A SAFE UPPER LIMIT FOR THIS DIMENSION.
\end{verbatim}
% \bye
% End of Starlink mods

\newpage                                               % Starlink

%%%%%%%%%%%%%%%%%%%%%%%%%%%%%%%%%%%%%%%%%%%%%%%%%%%%%%%%%%%%%%%%%%%%%%%%%%

%\magnification=1095                                   % Starlink
%\hsize=6.5truein\vsize=8.9truein                      % Starlink
%\hoffset=.0truein\voffset=.7truein                    % Starlink
\parindent=0pt\parskip=8pt
\baselineskip=11.pt plus.5pt minus.5pt
\lineskiplimit=1pt\lineskip=1pt
\font\tt=cmtt10
\font\csc=cmcsc10
\newcount\inum
\def\vp{\vskip\parskip}
\def\pf{\par\filbreak}
\def\el{\hfill\break}
\def\itm#1{\leavevmode\hangindent.75truein\hangafter=1
\advance\inum by 1
\hbox to2em{\rm\hfil\number\inum. }#1\vadjust{\kern5pt}\el}
\def\ul#1{$\underline{\hbox{\sl\strut#1}}$}
\def\hdg#1{\vskip4pt\leavevmode\kern-1em \ul{#1}\par}
% \pageno=8                                             Starlink
\centerline{\bf Table 4. Comments on Software Use}
\vskip8pt
\hdg{File Sizes}
According to ANSI standard Fortran, the {\tt RECL} parameter in the
Fortran {\tt OPEN} statement for direct-access
unformatted files defines the size of each record in 4-byte words.
The software included in this package follows that standard.
However, some Fortran compilers (notably those from IBM
and Language Systems for Macintosh) interpret that parameter as a
byte count. If that is the case with your system, then you will have
to make a single change to the subroutine {\csc ephopn}.
The affected part is self-explanatory and is reproduced here:
% Starlink mods
%{\tt
%\vp
%\settabs 24 \columns
%\+&C~~~~~~~THE~FOLLOWING~STATEMENTS~DEFINE~THE~LENGTH~OF~THE~RECORDS\cr
%\+&C~~~~~~~IN~THE~DIRECT-ACCESS~FILE.~IF~THE~'RECL'~SPECIFICATION~IN\cr
%\+&C~~~~~~~THE~FORTRAN~COMPILER~ON~YOUR~MACHINE~EXPECTS~THE~LENGTH~TO\cr
%\+&C~~~~~~~BE~IN~WORDS,~LEAVE~THE~STATEMENTS~AS~THEY~ARE.~IF~THE~LENGTH\cr
%\+&C~~~~~~~IS~EXPECTED~IN~BYTES,~REMOVE~THE~'C'~FROM~COLUMN~1~IN~THE\cr
%\+&C~~~~~~~SECOND~STATEMENT.\cr
%\+&C\cr
%\+&~~~~~~IRECSZ=IBSZ\cr
%\+&C~~~~~IRECSZ=IRECSZ*4\cr}
\begin{verbatim}
        C       THE FOLLOWING STATEMENTS DEFINE THE LENGTH OF THE RECORDS
        C       IN THE DIRECT-ACCESS FILE. IF THE 'RECL' SPECIFICATION IN
        C       THE FORTRAN COMPILER ON YOUR MACHINE EXPECTS THE LENGTH TO
        C       BE IN WORDS, LEAVE THE STATEMENTS AS THEY ARE. IF THE LENGTH
        C       IS EXPECTED IN BYTES, REMOVE THE 'C' FROM COLUMN 1 IN THE
        C       SECOND STATEMENT.
        C
              IRECSZ=IBSZ
        C     IRECSZ=IRECSZ*4
\end{verbatim}
% End of Starlink mods
\hdg{VAX Operation}
If you wish to read a binary-format tape on a VAX, you will want to
make a change to the program {\csc ephdsk}. The {\tt OPEN} statement
for the binary tape is:
% Starlink mods
%\vp
%{\tt
%\+&C~~~~~~~~~~~~~~~~~~~~~~~~'OPEN'~STATEMENT~FOR~BINARY-FORMAT~TAPE\cr
%\+&~~~~~~~~OPEN(9,\cr
%\+&~~~~~*~~~~~~~FILE='EPHTAP',\cr
%\+&~~~~~*~~~~~~~ACCESS='SEQUENTIAL',\cr
%\+&~~~~~*~~~~~~~FORM='UNFORMATTED',\cr
%\+&C~~~~*~~~~~~~RECORDTYPE='VARIABLE',~!~REMOVE~'C'~IN~COL~1~FOR~VAX~I/O\cr
%\+&~~~~~*~~~~~~~STATUS='OLD')\cr}
\begin{verbatim}
   C                        'OPEN' STATEMENT FOR BINARY-FORMAT TAPE
           OPEN(9,
        *       FILE='EPHTAP',
        *       ACCESS='SEQUENTIAL',
        *       FORM='UNFORMATTED',
   C    *       RECORDTYPE='VARIABLE', ! REMOVE 'C' IN COL 1 FOR VAX I/O
        *       STATUS='OLD')
\end{verbatim}
% End of Starlink mods
Remove the {\tt C} as indicated in the listing.
 
The relevant statements used in conjunction with the program are:
% Starlink mods
%\vp
%{\tt
%\+&\$~ALLOCATE \char123 tape unit\char125 :\cr
%\+&\$~MOUNT/FOREIGN/DEN=$\cdots$/BLOCKSIZE=8192
%\char123 tape unit\char125 :\cr
%\+&\$~DEFINE EPHTAP \char123 tape unit\char125 :\cr}
%\vp
%{\tt
%\+&\$~ALLOCATE \char123 tape unit\char125 :\cr
%\+&\$~MOUNT/FOREIGN/DEN=$\cdots$/BLOCKSIZE=8192
%\char123 tape unit\char125 :\cr
%\+&\$~DEFINE EPHTAP \char123 tape unit\char125 :\cr}
\begin{verbatim}
$ ALLOCATE {tape unit}:
$ MOUNT/FOREIGN/DEN=.../BLOCKSIZE=8192 {tape unit}:
$ DEFINE EPHTAP {tape unit}:
\end{verbatim}
% End of Starlink mods
where the dots following the {\tt DEN} specification denote the correct
density for your tape.
%\bye                                                 Starlink

%%%%%%%%%%%%%%%%%%%%%%%%%%%%%%%%%%%%%%%%%%%%%%%%%%%%%%%%%%%%%%%%%%%%%%%%%%

\end{document}
