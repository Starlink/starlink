\documentstyle[11pt]{article} 
\pagestyle{myheadings}

%------------------------------------------------------------------------------
\newcommand{\stardoccategory}  {Starlink User Note}
\newcommand{\stardocinitials}  {SUN}
\newcommand{\stardocnumber}    {15.5}
\newcommand{\stardocauthors}   {D L Terrett}
\newcommand{\stardocdate}      {23 September 1993}
\newcommand{\stardoctitle}     {PGPLOT --- Graphics Subroutine Library}
%------------------------------------------------------------------------------

\newcommand{\stardocname}{\stardocinitials /\stardocnumber}
\renewcommand{\_}{{\tt\char'137}}     % re-centres the underscore
\markright{\stardocname}
\setlength{\textwidth}{160mm}
\setlength{\textheight}{230mm}
\setlength{\topmargin}{-2mm}
\setlength{\oddsidemargin}{0mm}
\setlength{\evensidemargin}{0mm}
\setlength{\parindent}{0mm}
\setlength{\parskip}{\medskipamount}
\setlength{\unitlength}{1mm}

%------------------------------------------------------------------------------
% Add any \newcommand or \newenvironment commands here
%------------------------------------------------------------------------------

\begin{document}
\thispagestyle{empty}
SCIENCE \& ENGINEERING RESEARCH COUNCIL \hfill \stardocname\\
RUTHERFORD APPLETON LABORATORY\\
{\large\bf Starlink Project\\}
{\large\bf \stardoccategory\ \stardocnumber}
\begin{flushright}
\stardocauthors\\
\stardocdate
\end{flushright}
\vspace{-4mm}
\rule{\textwidth}{0.5mm}
\vspace{5mm}
\begin{center}
{\Large\bf \stardoctitle}
\end{center}
\vspace{5mm}

%------------------------------------------------------------------------------
%  Add this part if you want a table of contents
%  \setlength{\parskip}{0mm}
%  \tableofcontents
%  \setlength{\parskip}{\medskipamount}
%  \markright{\stardocname}
%------------------------------------------------------------------------------

\section{Introduction}

PGPLOT is a high level graphics package for plotting X {\em versus} Y plots,
functions, histograms, bar charts, contour maps and grey-scale images. Complete
diagrams can be produced with a minimal number of subroutine calls, but control
over colour, lines-style, character font, etc.\ is available if required. The
package was written (by Dr T J Pearson of the Caltech astronomy department)
with astronomical applications in mind and has become a {\it de facto} standard
for graphics in astronomy world wide.

The package exists in two version: the original version which uses a low
level graphics package known as GRPCKG which was also written at Caltech,
and a version developed by Starlink, in collaboration with Dr Pearson, which
uses GKS. The two versions have identical subroutine interfaces and
applications can be moved from one version to the other simply by re-linking.
It is the GKS version that is distributed and supported by Starlink.

The current release is based on PGPLOT Version 4.9G.

The main source of information on using PGPLOT is the PGPLOT manual obtainable
from your site manager or the Starlink software librarian, which describes the
original GRPCKG version. The majority of the manual applies equally to both
versions but when using the GKS version the following sections should be
ignored and the information in this note used instead:
\begin{itemize}
\item Section 1.3 --- Linking with PGPLOT
\item Section 1.4 --- Graphics Devices
\item Section 2.11 --- Compiling and Running the Program
\item Appendix C --- Installation Instructions
\item Appendix D --- Supported Devices
\item Appendix E --- Writing a Device Handler
\end{itemize}

\section{Using the GKS version}

There are two ways in which the GKS version of PGPLOT can be used:
     
\begin{enumerate}
\item As a self contained graphics package where {\em all} graphics, including
opening and closing the workstation, is done with PGPLOT. Programs written in
this way can be run with other implementations of PGPLOT. Such programs
are linked with the command:
\begin{quote}\tt
\% f77 {\em prog}.f -L/star/lib `pgplot\_link`
\end{quote}
on UNIX, or:
\begin{quote}\tt
\$ LINK {\em prog},PGPLOT\_DIR:PGPLOT\_LINK/OPT
\end{quote}
on VMS.

\item To plot a picture in the current viewport of an already open GKS
workstation (see section~\ref{viewport}). VMS Programs using PGPLOT in this 
way must
be linked with the PGPLOT object module library ({\tt PGPLOT\_DIR:GRPCKG/LIB})
and the following:
\begin{quote}\tt
GNS\_DIR:GNS/LIB, PSX\_DIR:PSX\_LINK/OPT, -\\
CNF\_DIR:CNF\_LINK/OPT, GKS\_DIR:GKSLINK/OPT 
\end{quote}
\end{enumerate}

\section{PGPLOT on Starlink}

On Starlink systems, PGPLOT programs will use the GKS version of the library by
default but the GRPCKG version may also have been installed, either because it
supports some graphics device that the GKS version does not, or because it is
needed by software which contains calls to obsolete GRPCKG routines. On VMS the
version being used both when linking and running a program depends on the
definition of the logical names GRPSHR or PGPLOT\_IMAGE depending on exactly how
the program was linked. For the GKS version the definitions
will be:
\begin{quote}\tt
"GRPSHR" = "PGPLOT\_DIR:GRPSHR.EXE"\\
"PGPLOT\_IMAGE" = "PGPLOT\_DIR:GRPSHR.EXE"
\end{quote}
To use the GRPCKG version the logical name must be re-defined to point the
directory containing the GRPCKG library (ask your system manager if you
don't know where this is).

\section{Graphics Devices}\label{GraphicsDevices}

Any graphics device supported by GKS can be used with PGPLOT and device
names are translated using the graphics name service described in SUN/57
section 2. Device names containing a `{\tt /}' character (such as UNIX file
names) must be surrounded by quote (`{\tt"}') characters.

If a question mark is typed in response to the prompt from {\tt PGBEG}, a
list of those workstation names defined on your system will be listed on the
terminal. 

On some hard copy devices the output from a PGPLOT program is a file and
some further action (such as printing the file) is required to produce a plot.
If you are unfamiliar with a particular device, consult SUN/83.

If one of the metafile workstations is selected (not available on VMS), 
the metafile can be tailored
for a particular real workstation type by appending {\tt /TARGET={\em 
workstation}} to the device specification. The default target is the
monochrome A4 Postscript workstation. The resulting metafile can be played
back on any workstation but will be tailored with respect to such things as
resolution, number of colour indices, etc. for the selected target.

The device name syntax described in the PGPLOT manual is also
supported; when using this form of device name, the device type is specified
using a GNS workstation name.

The PGPLOT cursor routines cannot be implemented on devices which do not have a
keyboard. For all such devices Starlink has implemented a GKS workstation which
combines the graphics device with a VT style terminal and these
workstations should be used instead. The relevant devices and the workstation
types of the combination workstations are listed in Table~\ref{vt-dev}.
\begin{table}
\caption{GKS workstation types}\label{vt-dev}
\[\begin{tabular}{|l|c|c|} \hline
&display only &display $+$ terminal \\
\hline
Ikon & 3200 & 3202 \\
Ikon Overlay & 3201 & 3203 \\
ARGS & 160 & 162 \\
ARGS Overlay & 161 &  163 \\
\hline \end{tabular}\]
\end{table}

\section{Using PGPLOT with ADAM}

To link an ADAM program with PGPLOT add:
\begin{quote}\tt
	`pgplot\_link\_adam`
\end{quote}
on UNIX  or:
\begin{quote}\tt
	PGPLOT\_DIR:PGPLOT\_LINK\_ADAM/OPT 
\end{quote}
on VMS to the {\tt alink} command.

{\tt PGBEG} (or {\tt PGBEG}) should not be called with a '?' as the device
name as this causes it to prompt the user for a device name
bypassing the ADAM message system.

The {\tt PGPAGE} (or {\tt PGPAPER}) routines does not behave as expected
when called from an ADAM task. {\tt PGPAGE} does not detect that
the task is an interactive job and so clears the screen without
prompting on the terminal

\section{Plotting in the current viewport}\label{viewport}

PGPLOT can be used to plot a picture in the current viewport on an already open
GKS workstation. When used in this way, the second argument to {\tt PGBEG}
(normally the workstation name) is a GKS workstation identifier (encoded as a
character string). PGPLOT then behaves as if the region of the display surface
defined by the current viewport is a complete workstation. When {\tt PGEND} is
called the workstation is not closed but the state of GKS is restored to what
it was at the time that {\tt PGBEG} was called. 

PGPLOT assumes that it has exclusive control over the GKS and so the only
graphics calls allowed between {\tt PGBEG} and {\tt PGEND} are PGPLOT
routines and GKS inquiry routines. 

The following simple example is a subroutine that uses PGPLOT to draw an X, Y
plot in an SGS zone.
\begin{small}
\begin{verbatim}
            SUBROUTINE XYPLOT (IZONE, X, Y, N, XLO, XHI, YLO, YHI, ISTAT)    
      *++                                                                    
      *                                                                      
      *   XYPLOT   Draw X,Y plot in an SGS zone                              
      *                                                                      
      *   Description:                                                       
      *                                                                      
      *      Uses PGPLOT to draw an X,Y plot of the real arrays X & Y in the 
      *      region of the display surface defined be the specified SGS zone 
      *                                                                      
      *   Input arguments:                                                   
      *                                                                      
      *      IZONE    INTEGER         SGS zone identifier                    
      *      X        REAL(N)         X values of data points                
      *      Y        REAL(N)         Y   "    "    "    "                   
      *      N        INTEGER         Number of data points                  
      *      XLO      REAL            Lower X axis limit                     
      *      XHI      REAL            Higher X  "    "                       
      *      YLO      REAL            Lower Y  "     "                       
      *      YHI      REAL            Higher Y  "    "                       
      *                                                                      
      *   Output arguments:                                                  
      *                                                                      
      *      ISTAT    INTEGER         SGS status                             
      *                                                                      
      *   Side effects:                                                      
      *                                                                      
      *      The specified SGS zone is selected.                             
      *++                                                                    
            IMPLICIT NONE                                                    
            INTEGER  IZONE, N, ISTAT                                         
            REAL     X(N), Y(N), XLO, XHI, YLO, YHI                          
                                                                             
            CHARACTER*10 WKID                                                
            INTEGER IWKID                                                    
                                                                             
                                                                             
      *  Select the specified SGS zone                                       
            CALL SGS_SELZ(IZONE, ISTAT)                                      
            IF (ISTAT.EQ.0) THEN                                             
                                                                             
      *     Inquire the GKS workstation identifier of the current zone       
               CALL SGS_ICURW(IWKID)                                         
                                                                             
      *     Encode workstation id as a character string                      
               WRITE(UNIT=WKID, FMT='(I10)') IWKID                           
                                                                             
      *     Open PGPLOT                                                      
               CALL PGBEG(0, WKID, 1, 1)                                     
                                                                             
      *     Define axis limits                                               
               CALL PGENV(XLO, XHI, YLO, YHI, 0, 0)                          
                                                                             
      *     Plot the data                                                    
               CALL PGPT(N, X, Y, 2)                                         
                                                                             
      *     Close down PGPLOT                                                
               CALL PGEND                                                    
            END IF                                                           
            END                                                              
\end{verbatim}                                                         
\end{small}

Because other plotting packages may have plotted on the same physical device,
there are some restrictions when using PGPLOT in this way:

\begin{itemize}
\item {\tt PGSIZE} cannot be used.

\item {\tt PGPAGE} will never clear the screen. If the display surface has
been divided into sub-pictures {\tt PGPAGE} will move to the next sub-picture
in the usual way.

\item On devices with fixed colour tables, the default PGPLOT colour table
will not be set up by {\tt PGBEG} unless the display surface is empty.

\end{itemize}

\section{Example Programs}

The directory {\tt /star/starlink/lib/pgplot/examples} ({\tt 
[STARLINK.LIB.PGPLOT.EXAMPLES]} on VMS) contains the source of a number 
of example programs which demonstrate most of the features of PGPLOT. 
On UNIX they can be run with the command:
\begin{quote}\tt
/star/bin/examples/pgplot/pgdemo{\em n}
\end{quote}
where {\em n} is between 1 and 9 provided that they have been installed.
(They may not have been in order to save disk space).

On VMS they can be run
with the command:
\begin{quote}\tt
\$ RUN STARDISK:[STARLINK.LIB.PGPLOT.EXAMPLES]PGDEMO{\em n}
\end{quote}

\section{Other Differences}
In general the two versions produce identical results when run on the same
device but the following differences should be noted:
\begin{itemize}
\item The GKS version of {\tt PGPAGE} cannot be used to make the display surface
larger than the default size provided by {\tt PGBEG}.
\item The GKS version of {\tt PGBEG} clears the display surface immediately
instead of waiting until the first vector is plotted.
\end{itemize}

\section{Support}

PGPLOT is Starlink supported software and bugs should be reported through the
usual channels and not by contacting Dr Pearson directly. Problems with the GKS
specific code will be dealt with by Starlink but all changes to the code which
is common to the two versions of PGPLOT must be made in collaboration with Dr
Pearson.

The mixing of calls to PGPLOT and GKS routines is not supported except as
described in section~\ref{viewport}, and neither
version supports the calling of GRPCKG routines directly. Existing programs
that call GRPCKG should be re-written to call the equivalent PGPLOT routines.

\section{Restrictions}

The current version contains the following bugs and restrictions. They will
be removed in a future release.
\begin{itemize}

\item On VMS The metafile workstation cannot be used. 

\item {\tt PGBOX} (and {\tt PGENV}): If the axis range is small compared with 
the absolute values of the endpoints, an integer overflow can occur.

\item {\tt PGBOX} (and {\tt PGENV}): Logarithmic axes are not labelled 
very well when they span only one or two decades; only the decade markers 
are labelled.

\item The environment variable {\tt PLOT\_IDENT} puts the user name on only 
the last page; it should put it on all pages.

\end{itemize}

\section{New Features}
The following new features are introduced with this release:

\begin{itemize}
\item New routines

The following new routines have been introduced with since the PGPLOT manual
was last revised:

\[\begin{tabular}{llllll}
PGARRO &PGCIRC &PGERRB &PGPIXL &PGPNTS &PGQAH \\
PGQCS &PGQPOS &PGQVSZ &PGSAH &PGSAVE &PGSCRN \\
PGUNSA &PGVECT &PGWEDG
\end{tabular}\]

A new manual describing the new routines will be available
in due course; descriptions for the new routines from Appendix A
of the new manual are attached.

\item The background and foreground colours (colour indexes 0 and 1) can
be set by defining the environment variables {\tt PGPLOT\_BACKGROUND} and
{\tt PGPLOT\_FOREGROUND} (logical names on VMS) to be one of the colour names
accepted by {\tt PGSCRN}.

\item On workstations with high quality thick lines (such as PostScript), 
thick lines are drawn by the workstation which reduces the size of plot files.

\item On monochrome workstations, lines and characters are always drawn with
either color index 1 or 0. Provided that the colours of colour index 1 are 
not changed from the defaults this
avoids having lines and characters ``dithered'' to simulate grey levels
and being almost invisible on laser printers.

\item Using {\tt /APPEND} when opening a workstation that does not support
being opened without clearing the display surface no longer generates an
error message.

\item The following bugs have been fixed:
\begin{itemize}
\item {\tt PGQCOL} now returns the correct value for the minimum available
colour when {\tt /APPEND} is used when opening the workstation.

\item Device names containing a {\tt /} can now be specified.

\end{itemize}
\item Old PGPLOT programs may use subroutine names longer than six characters. These
names have been replaced by name that conform to the Fortran 77 standard but the
old names are still available. It is advisable to change the names as
given in the following table.

\[\begin{tabular}{|l|l||l|l|}\hline
New Name &Old Name& New Name &Old Name \\\hline\hline
PGPAGE &PGPAPER &PGPT &PGPOINT \\
PGBEG  &PGBEGIN &PGPTXT &PGPTEXT \\
PGCURS &PGCURSE &PGSVP &PGVPORT \\
PGLAB  &PGLABEL &PGVSIZ &PGVSIZE \\
PGMTXT &PGMTEXT &PGVSTD &PGVSTAND \\
PGNCUR &PGNCURSE &PGSWIN &PGWINDOW \\
PGPAP  &PGPAPER & & \\\hline
\end{tabular}\]

\end{itemize}
\newpage
\begin{small}
\begin{verbatim}
PGARRO -- draw an arrow

      SUBROUTINE PGARRO (X1, Y1, X2, Y2)
      REAL X1, Y1, X2, Y2

Draw an arrow from the point with world-coordinates (X1,Y1) to
(X2,Y2). The size of the arrowhead at (X2,Y2) is determined by
the current character size set by routine PGSCH. The default size
is 1/40th of the smaller of the width or height of the view surface.
The appearance of the arrowhead (shape and solid or open) is
controlled by routine PGSAH.

Arguments:
 X1, Y1 (input)  : world coordinates of the tail of the arrow.
 X2, Y2 (input)  : world coordinates of the head of the arrow.


PGCIRC -- draw a filled or outline circle

      SUBROUTINE PGCIRC (XCENT, YCENT, RADIUS)
      REAL XCENT, YCENT, RADIUS

Draw a circle. The action of this routine depends
on the setting of the Fill-Area Style attribute. If Fill-Area Style
is SOLID (the default), the interior of the circle is solid-filled
using the current Color Index. If Fill-Area Style is HOLLOW, the
outline of the circle is drawn using the current line attributes
(color index, line-style, and line-width).

Arguments:
 XCENT  (input)  : world x-coordinate of the center of the circle.
 YCENT  (input)  : world y-coordinate of the center of the circle.
 RADIUS (input)  : radius of circle (world coordinates).


PGERRB -- horizontal error bar

      SUBROUTINE PGERRB (DIR, N, X, Y, E, T)
      INTEGER DIR, N
      REAL X(*), Y(*), E(*)
      REAL T
 
Plot error bars in the direction specified by DIR.
This routine draws an error bar only; to mark the data point at
the start of the error bar, an additional call to PGPT is required.
 
Arguments:
 DIR    (input)  : direction to plot the error bar relative to
                   the data point.  DIR is 1 for +X; 2 for +Y;
                   3 for -X; and 4 for -Y;
 N      (input)  : number of error bars to plot.
 X      (input)  : world x-coordinates of the data.
 Y      (input)  : world y-coordinates of the data.
 E      (input)  : value of error bar distance to be added to the
                   data position in world coordinates.
 T      (input)  : length of terminals to be drawn at the ends
                   of the error bar, as a multiple of the default
                   length; if T = 0.0, no terminals will be drawn.
 
Note: the dimension of arrays X, Y, and E must be greater
than or equal to N. If N is 1, X, Y, and E may be scalar
variables, or expressions.

 
PGPIXL -- draw pixels

      SUBROUTINE PGPIXL (IA, IDIM, JDIM, I1, I2, J1, J2, 
     1                   X1, X2, Y1, Y2)
      INTEGER IDIM, JDIM, I1, I2, J1, J2
      INTEGER IA(IDIM,JDIM)
      REAL    X1, X2, Y1, Y2

Draw lots of solid-filled (tiny) rectangles alligned with the
coordinate axes. Best performance is achieved when output is
directed to a pixel-oriented device and the rectangles coincide
with the pixels on the device. In other cases, pixel output is
emulated.

The subsection of the array IA defined by indices (I1:I2, J1:J2)
is mapped onto world-coordinate rectangle defined by X1, X2, Y1
and Y2. This rectangle is divided int (I2 - I1 + 1) * (J2 - J1 + 1)
small rectangles. Each of these small rectangles is solid-filled
with the color index specified by the corresponding element of 
IA.

On most devices, the output region is "opaque", i.e., it obscures
all graphical elements previously drawn in the region. But on
devices that do not have erase capability, the background shade
is "transparent" and allows previously-drawn graphics to show
through.

Arguments:
 IA     (input)  : the array to be plotted.
 IDIM   (input)  : the first dimension of array A.
 JDIM   (input)  : the second dimension of array A.
 I1, I2 (input)  : the inclusive range of the first index
                   (I) to be plotted.
 J1, J2 (input)  : the inclusive range of the second
                   index (J) to be plotted.
 X1, Y1 (input)  : world coordinates of one corner of the output
                   region
 X2, Y2 (input)  : world coordinates of the opposite corner of the
                   output region


PGPNTS -- draw one or more graph markers

      SUBROUTINE PGPNTS (N, X, Y, SYMBOL, NS)
      INTEGER N, NS
      REAL X(*), Y(*)
      INTEGER SYMBOL(*)
 
Draw Graph Markers. Unlike PGPT, this routine can draw a different
symbol at each point. The markers
are drawn using the current values of attributes color-index,
line-width, and character-height (character-font applies if the symbol
number is >31).  If the point to be marked lies outside the window,
no marker is drawn.  The "pen position" is changed to
(XPTS(N),YPTS(N)) in world coordinates (if N > 0).
 
Arguments:
 N      (input)  : number of points to mark.
 X      (input)  : world x-coordinate of the points.
 Y      (input)  : world y-coordinate of the points.
 SYMBOL (input)  : code number of the symbol to be plotted at each
                   point (see PGPT).
 NS     (input)  : number of values in the SYMBOL array.  If NS <= N,
                   then the first NS points are drawn using the value
                   of SYMBOL(I) at (X(I), Y(I)) and SYMBOL(1) for all
                   the values of (X(I), Y(I)) where I > NS.
 
Note: the dimension of arrays X and Y must be greater than or equal
to N and the dimension of the array SYMBOL must be greater than or
equal to NS.  If N is 1, X and Y may be scalars (constants or
variables).  If NS is 1, then SYMBOL may be a scalar.  If N is
less than 1, nothing is drawn.


PGQAH -- inquire arrow-head style

      SUBROUTINE PGQAH (FS, ANGLE, VENT)
      INTEGER  FS
      REAL ANGLE, VENT

Query the style to be used for arrowheads drawn with routine PGARRO.

Argument:
 FS     (output) : FS = 1 => filled; FS = 2 => outline.
 ANGLE  (output) : the acute angle of the arrow point, in degrees.
 VENT   (output) : the fraction of the triangular arrow-head that
                   is cut away from the back.


PGQCS -- Inquire character height in a variety of units.

      SUBROUTINE PGQCS(UNITS, XCH, YCH)
      INTEGER UNITS
      REAL XCH, YCH

Return the current PGPLOT character height in a variety of units.
This routine provides facilities that are not available via PGQCH.
Use PGQCS if the character height is required in units other than
those used in PGSCH.

Arguments:
 UNITS  (input)  : Used to specify the units of the output value:
                   UNITS = 0 : normalized device coordinates
                   UNITS = 1 : inches
                   UNITS = 2 : millimeters
                   UNITS = 3 : pixels
                   Other values give an error message, and are
                   treated as 0.
 XCH    (output) : The character height for vertically written text.
 YCH    (output) : The character height for horizontally written text.
                   NB. XCH=YCH if UNITS=1 or UNITS=2.


PGQPOS -- return current pen position

      SUBROUTINE PGQPOS (X, Y)
      REAL X, Y
 
Primitive routine to return the "pen" position in world
coordinates (X,Y).
 
Arguments:
 X      (output)  : world x-coordinate of the previous pen position.
 Y      (output)  : world y-coordinate of the previous pen position.


Module: PGQVSZ -- Find the window defined by the full view surface

      SUBROUTINE PGQVSZ (UNITS, X1, X2, Y1, Y2)
      INTEGER UNITS
      REAL X1, X2, Y1, Y2

Return the window, in a variety of units, defined by the full
device view surface (0 -> 1 in normalized device coordinates).

Input:
  UNITS    0,1,2,3 for output in normalized device coords,
           inches, mm, or absolute device units (dots)
Output
  X1,X2    X window
  Y1,Y2    Y window


PGSAH -- set arrow-head style

      SUBROUTINE PGSAH (FS, ANGLE, VENT)
      INTEGER  FS
      REAL ANGLE, VENT

Set the style to be used for arrowheads drawn with routine PGARRO.

Argument:
 FS     (input)  : FS = 1 => filled; FS = 2 => outline.
                   Other values are treated as 2. Default 1.
 ANGLE  (input)  : the acute angle of the arrow point, in degrees;
                   angles in the range 20.0 to 90.0 give reasonable
                   results. Default 45.0.
 VENT   (input)  : the fraction of the triangular arrow-head that
                   is cut away from the back. 0.0 gives a triangular
                   wedge arrow-head; 1.0 gives an open >. Values 0.3


PGSAVE -- save PGPLOT attributes

      SUBROUTINE PGSAVE

This routine saves the current PGPLOT attributes in a private storage
area. They can be restored by calling PGUNSA (unsave). Attributes
saved are: character font, character height, color index, fill-area
style, line style, line width, pen position, arrow-head style.
Color representation is not saved.

Calls to PGSAVE and PGUNSA should always be paired. Up to 20 copies
of the attributes may be saved. PGUNSA always retrieves the last-saved
values (first-in first-out stack).

Arguments: none


PGSCRN -- set color representation by name

      SUBROUTINE PGSCRN(CI, NAME, IER)
      INTEGER CI
      CHARACTER*(*) NAME
      INTEGER IER

Set color representation: i.e., define the color to be
associated with a color index.  Ignored for devices which do not
support variable color or intensity.  This is an alternative to
routine PGSCR. The color representation is defined by name instead
of (R,G,B) components.

Color names are defined in an external file which is read the first
time that PGSCRN is called. The name of the external file is
found as follows:
1. if environment variable (logical name) PGPLOT_RGB is defined,
   its value is used as the file name;
2. otherwise, if environment variable PGPLOT_DIR is defined, a
   file "rgb.txt" in the directory named by this environment
   variable is used;
3. otherwise, file "rgb.txt" in the current directory is used.

If all of these fail to find a file, an error is reported and
the routine does nothing.

Each line of the file
defines one color, with four blank- or tab-separated fields per
line. The first three fields are the R, G, B components, which
are integers in the range 0 (zero intensity) to 255 (maximum
intensity). The fourth field is the color name. The color name
may include embedded blanks. Example:

255   0   0 red
255 105 180 hot pink
255 255 255 white
  0   0   0 black

Arguments:
 CI     (input)  : the color index to be defined, in the range 0-max.
                   If the color index greater than the device
                   maximum is specified, the call is ignored. Color
                   index 0 applies to the background color.
 NAME   (input)  : the name of the color to be associated with
                   this color index. This name must be in the
                   external file. The names are not case-sensitive.
                   If the color is not listed in the file, the
                   color representation is not changed.
 IER    (output) : returns 0 if the routine was successful, 1
                   if an error occurred (either the external file
                   could not be read, or the requested color was
                   not defined in the file).


PGUNSA -- restore PGPLOT attributes

      ENTRY PGUNSA

This routine restores the PGPLOT attributes saved in the last call to
PGSAVE. See PGSAVE.

Arguments: none


PGVECT -- vector map of a 2D data array, with blanking

      SUBROUTINE PGVECT (A, B, IDIM, JDIM, I1, I2, J1, J2, C, NC, TR,
     1                   BLANK)
      INTEGER IDIM, JDIM, I1, I2, J1, J2, NC
      REAL    A(IDIM,JDIM), B(IDIM, JDIM), TR(6), BLANK, C

Draw a vector map of two arrays.  This routine is the similar to
PGCONB as that array elements that have the "magic value" defined by
the argument BLANK are ignored, making gaps in the vector map.  The
routine may be useful for data measured on most but not all of the
points of a grid. Vectors are displayed as arrows; the style of the
arrowhead can be set with routine PGSAH, and the the size of the
arrowhead is determined by the current character size, set by PGSCH.

Arguments:
 A      (input)  : horizontal component data array.
 B      (input)  : vertical component data array.
 IDIM   (input)  : first dimension of A and B.
 JDIM   (input)  : second dimension of A and B.
 I1,I2  (input)  : range of first index to be mapped (inclusive).
 J1,J2  (input)  : range of second index to be mapped (inclusive).
 C      (input)  : scale factor for vector lengths, if 0.0, C will be
                   set so that the longest vector is equal to the
                   smaller of TR(2)+TR(3) and TR(5)+TR(6).
 NC     (input)  : vector positioning code.
                   <0 vector head positioned on coordinates
                   >0 vector base positioned on coordinates
                   =0 vector centered on the coordinates
 TR     (input)  : array defining a transformation between the I,J
                   grid of the array and the world coordinates. The
                   world coordinates of the array point A(I,J) are
                   given by:
                     X = TR(1) + TR(2)*I + TR(3)*J
                     Y = TR(4) + TR(5)*I + TR(6)*J
                   Usually TR(3) and TR(5) are zero - unless the
                   coordinate transformation involves a rotation
                   or shear.
 BLANK   (input) : elements of arrays A or B that are exactly equal to
                   this value are ignored (blanked).


PGWEDG -- Anotate a gray-scale plot with a wedge.

      SUBROUTINE PGWEDG(SIDE, DISP, WIDTH, FG, BG, LABEL)
      CHARACTER *(*) SIDE,LABEL
      REAL DISP, WIDTH, FG, BG

Plot an anotated grey-scale wedge parallel to a given axis of the
the current viewport.

Arguments:
 SIDE   (input)  : The first character must be one of the characters
                   'B', 'L', 'T', or 'R' signifying the Bottom, Left,
                   Top, or Right edge of the viewport.
 DISP   (input)  : the displacement of the wedge from the specified
                   edge of the viewport, measured outwards from the
                   viewport in units of the character height. Use a
                   negative value to write inside the viewport, a
                   positive value to write outside.
 WIDTH  (input)  : The total width of the wedge including anotation,
                   in units of the character height.
 FG     (input)  : The value which is to appear with shade
                   1 ("foreground"). Use the values of FG and BG
                   that were sent to PGGRAY.
 BG     (input)  : the value which is to appear with shade
                   0 ("background").
 LABEL  (input)  : Optional units label. If no label is required
                   use ' '.
\end{verbatim}
\end{small}
\end{document}
