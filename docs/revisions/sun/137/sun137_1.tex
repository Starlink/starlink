\documentstyle[11pt]{article} 
\pagestyle{myheadings}

%------------------------------------------------------------------------------
\newcommand{\stardoccategory}  {Starlink User Note}
\newcommand{\stardocinitials}  {SUN}
\newcommand{\stardocnumber}    {137.1}
\newcommand{\stardocauthors}   {Paul Harrison \\
                                Paul Rees}
\newcommand{\stardocdate}      {4 March 1993}
\newcommand{\stardoctitle}     {PONGO\\[2ex]
   A Set of Applications for Interactive Data Plotting \\[2.5ex]
   Version 1.0} 
%------------------------------------------------------------------------------

\newcommand{\stardocname}{\stardocinitials /\stardocnumber}
\renewcommand{\_}{{\tt\char'137}}     % re-centres the underscore
\markright{\stardocname}
\setlength{\textwidth}{160mm}
\setlength{\textheight}{230mm}
\setlength{\topmargin}{-2mm}
\setlength{\oddsidemargin}{0mm}
\setlength{\evensidemargin}{0mm}
\setlength{\parindent}{0mm}
\setlength{\parskip}{\medskipamount}
\setlength{\unitlength}{1mm}

%------------------------------------------------------------------------------
% Add any \newcommand or \newenvironment commands here
% Definition of 'starheadings' page style 
% Note the use of ##1 for parameter of \def\sectionmark inside the
% \def\ps@starheadings.

% To simplify error messages.
\showboxdepth=0

\raggedbottom
\sloppy         
\def\eg{{\em e.g.\ }}
\def\ie{{\em i.e.\ }}
\def\etc{{\em etc.}}

\newcommand{\halfpfig}[1]{
\setlength{\unitlength}{1in}
\begin{picture}(5.0,5.0)
\put(0,5.0){\special{include #1}}
\typeout{#1 inserted on page \arabic{page}}
\end{picture}
}

% Column label definitions.
\def\xcol{{\sf XCOL}}
\def\excol{{\sf EXCOL}}
\def\ycol{{\sf YCOL}}
\def\eycol{{\sf EYCOL}}
\def\zcol{{\sf ZCOL}}
\def\symcol{{\sf SYMCOL}}
\def\labcol{{\sf LABCOL}}

% Types for various entities.
\def\pnam#1{{\tt #1}}
\def\cnam#1{{\tt #1}}


%+
%  Name:
%     LAYOUT.TEX

%  Purpose:
%     Define Latex commands for laying out documentation produced by PROLAT.

%  Language:
%     Latex

%  Type of Module:
%     Data file for use by the PROLAT application.

%  Description:
%     This file defines Latex commands which allow routine documentation
%     produced by the SST application PROLAT to be processed by Latex. The
%     contents of this file should be included in the source presented to
%     Latex in front of any output from PROLAT. By default, this is done
%     automatically by PROLAT.

%  Notes:
%     The definitions in this file should be included explicitly in any file
%     which requires them. The \include directive should not be used, as it
%     may not then be possible to process the resulting document with Latex
%     at a later date if changes to this definitions file become necessary.

%  Authors:
%     RFWS: R.F. Warren-Smith (STARLINK)

%  History:
%     10-SEP-1990 (RFWS):
%        Original version.
%     10-SEP-1990 (RFWS):
%        Added the implementation status section.
%     12-SEP-1990 (RFWS):
%        Added support for the usage section and adjusted various spacings.
%     10-DEC-1991 (RFWS):
%        Refer to font files in lower case for UNIX compatibility.
%     {enter_further_changes_here}

%  Bugs:
%     {note_any_bugs_here}

%-

%  Define length variables.
\newlength{\sstbannerlength}
\newlength{\sstcaptionlength}

%  Define a \tt font of the required size.
\font\ssttt=cmtt10 scaled 1095

%  Define a command to produce a routine header, including its name,
%  a purpose description and the rest of the routine's documentation.
\newcommand{\sstroutine}[3]{
   \goodbreak
   \rule{\textwidth}{0.5mm}
   \vspace{-7ex}
   \newline
   \settowidth{\sstbannerlength}{{\Large {\bf #1}}}
   \setlength{\sstcaptionlength}{\textwidth}
   \addtolength{\sstbannerlength}{0.5em}
   \addtolength{\sstcaptionlength}{-2.0\sstbannerlength}
   \addtolength{\sstcaptionlength}{-4.45pt}
   \parbox[t]{\sstbannerlength}{\flushleft{\Large {\bf #1}}}
   \parbox[t]{\sstcaptionlength}{\center{\Large #2}}
   \parbox[t]{\sstbannerlength}{\flushright{\Large {\bf #1}}}
   \begin{description}
      #3
   \end{description}
}

%  Format the description section.
\newcommand{\sstdescription}[1]{\item[Description:] #1}

%  Format the usage section.
\newcommand{\sstusage}[1]{\item[Usage:] \mbox{} \\[1.3ex] {\ssttt #1}}

%  Format the invocation section.
\newcommand{\sstinvocation}[1]{\item[Invocation:]\hspace{0.4em}{\tt #1}}

%  Format the arguments section.
\newcommand{\sstarguments}[1]{
   \item[Arguments:] \mbox{} \\
   \vspace{-3.5ex}
   \begin{description}
      #1
   \end{description}
}

%  Format the returned value section (for a function).
\newcommand{\sstreturnedvalue}[1]{
   \item[Returned Value:] \mbox{} \\
   \vspace{-3.5ex}
   \begin{description}
      #1
   \end{description}
}

%  Format the parameters section (for an application).
\newcommand{\sstparameters}[1]{
   \item[Parameters:] \mbox{} \\
   \vspace{-3.5ex}
   \begin{description}
      #1
   \end{description}
}

%  Format the examples section.
\newcommand{\sstexamples}[1]{
   \item[Examples:] \mbox{} \\
   \vspace{-3.5ex}
   \begin{description}
      #1
   \end{description}
}

%  Define the format of a subsection in a normal section.
\newcommand{\sstsubsection}[1]{\item[{#1}] \mbox{} \\}

%  Define the format of a subsection in the examples section.
\newcommand{\sstexamplesubsection}[1]{\item[{\ssttt #1}] \mbox{} \\}

%  Format the notes section.
\newcommand{\sstnotes}[1]{\item[Notes:] \mbox{} \\[1.3ex] #1}

%  Provide a general-purpose format for additional (DIY) sections.
\newcommand{\sstdiytopic}[2]{\item[{\hspace{-0.35em}#1\hspace{-0.35em}:}] \mbox{} \\[1.3ex] #2}

%  Format the implementation status section.
\newcommand{\sstimplementationstatus}[1]{
   \item[{Implementation Status:}] \mbox{} \\[1.3ex] #1}

%  Format the bugs section.
\newcommand{\sstbugs}[1]{\item[Bugs:] #1}

%  Format a list of items while in paragraph mode.
\newcommand{\sstitemlist}[1]{
  \mbox{} \\
  \vspace{-3.5ex}
  \begin{itemize}
     #1
  \end{itemize}
}

%  Define the format of an item.
\newcommand{\sstitem}{\item}

%  End of LAYOUT.TEX layout definitions.
%.

%------------------------------------------------------------------------------

\begin{document}
\thispagestyle{empty}
SCIENCE \& ENGINEERING RESEARCH COUNCIL \hfill \stardocname\\
RUTHERFORD APPLETON LABORATORY\\
{\large\bf Starlink Project\\}
{\large\bf \stardoccategory\ \stardocnumber}
\begin{flushright}
\stardocauthors\\
\stardocdate
\end{flushright}
\vspace{-4mm}
\rule{\textwidth}{0.5mm}
\vspace{5mm}
\begin{center}
{\Large\bf \stardoctitle}
\end{center}
\vspace{5mm}

\begin{figure}[hb]
\setlength{\unitlength}{1in}
\begin{picture}(5.0,5.0)
\put(0.05,5.3){\special{include sun137_cover.ps-tex}}
\end{picture}
\end{figure}

\newpage

%------------------------------------------------------------------------------
%  Add this part if you want a table of contents
\setlength{\parskip}{0mm}
\tableofcontents
\setlength{\parskip}{\medskipamount}
\markright{\stardocname}
%------------------------------------------------------------------------------
\newpage

\section{Introduction}

PONGO is a set of applications for interactively plotting data.
These applications are written to behave in a similar way to MONGO (SUN/64),
but using PGPLOT (SUN/15) as their plotting package.
PONGO offers greater functionality than MONGO and is integrated within
the STARLINK ADAM environment, meaning that the benefits of ICL (SG/5) and
AGI (SUN/48) can be fully exploited.

The advantages of using PONGO are:

\begin{itemize}
\item Data in PONGO are read from a text file using the \cnam{READF}
command.
Features of this command are:

\begin{itemize}
\item the ability to read files which contain character strings as well as 
numeric values;
\item support for comment lines and column headings as an aid to remembering 
what the file contains;
\item error trapping during file reading;
\item user specification of column delimiters, allowing \LaTeX\ and \TeX\
format tables to be read;
\item selective reading of the data.
\end{itemize}

\item Complicated mathematical manipulations can be performed on the data using 
Fortran-like statements to define the required transformation.
\item Specialized extra data columns are provided, \ie \labcol, \symcol\ and
\zcol.
\item Many interactive cursor functions are provided.
\item Error ellipses can be drawn.
\item Vector plots can be drawn.
\item Simple statistical analysis of the data is available.
\item The data can be resampled.
\item User-specified functions defined by Fortran-like statements can be drawn.
\item Plots of astronomical positional data can be made in one of several
geometries.
\item The data in the XCOL and YCOL data areas can be read as positions of 
Right Ascension and Declination respectively, in the format

\begin{quote}
\begin{verbatim}
HH:MM:SS.SSS   +DD:MM:SS.SSS
\end{verbatim}
\end{quote}
\end{itemize}

Because the parameters PONGO uses for its commands are often similar to the
arguments of the equivalent PGPLOT subroutines, it is useful to read the PGPLOT
manual in conjunction with this user guide if you are not already familiar with
PGPLOT.
In several cases the full descriptions of the parameters are given only in the
PGPLOT manual.


\section{A Tutorial Example}

The files connected with this example can be found in the directory
\verb+PONGO_EXAMPLES+.

In order to start PONGO you must first start ADAM and ICL.
This may be done by typing at the DCL prompt

\begin{quote}
\begin{verbatim}
$ ADAM PONGO
\end{verbatim}
\end{quote}

Note that this does not actually load any PONGO application: this will not
occur until the first PONGO command is issued, at which point there may be a 
brief delay.

In any PONGO session the first action is to open the plotting device.
This is done by typing

\begin{quote}
\begin{verbatim}
ICL> BEGPLOT MG100
\end{verbatim}
\end{quote}

Where MG100 should be replaced by the name of the plotting device you are
intending to use.
Note that once you have begun a PONGO session with the \cnam{BEGPLOT} command,
the \verb+ICL>+ prompt changes to \verb+PONGO>+.
This is important because the only commands you can use with success at the
\verb+ICL>+ prompt are \cnam{BEGPLOT} and \cnam{DEVICE}, all other commands
should be typed at the \verb+PONGO>+ prompt.
The data may then be read using the command

\begin{quote}
\begin{verbatim}
PONGO> READF DATA=PONGO_EXAMPLES:TUTORIAL.DAT XCOL=1 YCOL=3 ALL
\end{verbatim}
\end{quote}

Plotting limits are set up using the range of the data by the command

\begin{quote}
\begin{verbatim}
PONGO> DLIMITS
\end{verbatim}
\end{quote}

Axes for the plot may be drawn using 

\begin{quote}
\begin{verbatim}
PONGO> BOXFRAME
\end{verbatim}
\end{quote}

and finally the points may be plotted as asterisk-like symbols and the axes
labelled with

\begin{quote}
\begin{verbatim}
PONGO> POINTS 3
PONGO> LABEL 'X axis' 'Y axis' 'PLOT TITLE'
\end{verbatim}
\end{quote}

Note that the data values are remembered by PONGO and the plot you have
just created may be erased and recreated by typing

\begin{quote}
\begin{verbatim}
PONGO> ADVANCE
PONGO> BOXFRAME
PONGO> POINTS
PONGO> LABEL
\end{verbatim}
\end{quote}

To close a device and end a PONGO plot the command \cnam{ENDPLOT} should be
used. 
This is important if you are going to switch to another set of applications
such as KAPPA, otherwise the plotting device characteristics will be
inaccessible to the second package.

More extensive examples are referred to in \S\ref{exam_sect}.


\section{Classified List of Commands}

This section presents a list of the available PONGO commands, classified into 
several broad categories: commands which begin and end a PONGO session, 
commands for manipulating data, commands which control plotting, plotting
commands, and commands which perform simple statistics on the data.
Not all the commands given are individual applications, many are synonyms for
other applications with specific parameters provided for convenience.
Detailed descriptions of these commands are given in Appendix \ref{defn_sect}.
The parts of the command names outside parentheses define the minimum
abbreviation for that application.


\subsection{Commands which begin and end PONGO}

\small
\begin {quote}
\begin {description} 
\item [BEGP(LOT)] -- Open a plotting device.
\item [DEVICE] -- Open a plotting device (equivalent to \cnam{BEGPLOT}).
\item [ENDP(LOT)] -- Close down the current plotting device.
\end {description}
\end {quote}
\normalsize


\subsection{Commands for plotting}

\small
\begin {quote}
\begin {description} 
\item [ADV(ANCE)] -- Clear the graphics screen (equivalent to \cnam{CLEAR
SCREEN}).
\item [ANN(OTATE)] -- Annotate the plotted data.
\item [ARC] -- Draw an arc of an ellipse.
\item [BIN] -- Plot a histogram using previously binned data (equivalent to
\cnam{PLOTHIST B}).
\item [BOX(FRAME)] -- Draw a frame and axes on the plot.
\item [CONN(ECT)] -- Draw straight lines between the data points (equivalent to
\cnam{GPOINTS C}).
\item [CURSE] -- Use the cursor to get interactive input.
\item [DRAW] -- Draw a line from the current pen position the specified point
(equivalent to \cnam{PRIM D}).
\item [ELLIPSES] -- Draw error ellipses.
\item [ERASE] -- Clear the graphics screen (equivalent to \cnam{CLEAR SCREEN}).
\item [ERRORBAR] -- Draw error bars on the plotted data.
\item [ERRX] -- Draw symmetric error bars in the X direction (equivalent to 
\cnam{ERRORBAR X}). 
\item [ERRY] -- Draw symmetric error bars in the Y direction (equivalent to
\cnam{ERRORBAR Y}).
\item [GPOINTS] -- Plot points or lines between the data.
\item [GRID] -- Draw a coordinate grid at specified intervals.
\item [GT\_CIRCLE] -- Draw a great circle between two points.
\item [HIST(OGRAM)] -- Bin and plot a histogram of the data (equivalent to 
\cnam{PLOTHIST H}).
\item [LAB(EL)] -- Draw the axis labels and title on the plot.
\item [MARK] -- Draw a point mark at the specified position (equivalent to 
\cnam{PRIM K}).
\item [MTEXT] -- Draw a text string relative to the viewport (ICL hidden
procedure using \cnam{WTEXT}).
\item [PLOTF(UN)] -- Plot a given function.
\item [PLOTH(IST)] -- Plot a histogram of the data.
\item [POI(NTS)] -- Draw a point mark at each of the data points (equivalent to
\cnam{GPOINTS P}).
\item [PRIM] -- Perform primitive plotting functions.
\item [PTEXT] -- Draw a text string at the specified position and angle (ICL
hidden procedure using \cnam{WTEXT}).
\item [PVECT] -- Draw proper motion vectors.
\item [RADIATE] -- Draw a line from the given point to the first NP data points
(ICL hidden procedure using \cnam{GETPOINT}, \cnam{MOVE} and \cnam{DRAW}).
\item [SIZE(PLOT)] -- Draw point marks of differing sizes at each of the data
points (equivalent to \cnam{GPOINTS S}).
\item [TEXT] -- Draw a text string on the plot at the specified position 
(equivalent to \cnam{WTEXT S}).
\item [VECT] -- Draw vectors from each data point.
\item [WTEXT] -- Draw a text string on the plot.
\item [XERR] -- Draw symmetric error bars in the X direction (equivalent to
\cnam{ERRORBAR X}). 
\item [YERR] -- Draw symmetric error bars in the Y direction (equivalent to
\cnam{ERRORBAR Y}). 
\end {description}
\end {quote}
\normalsize


\subsection{Commands which control plotting}

\small
\begin {quote}
\begin {description} 
\item [CHANGE] -- Change plotting attributes.
\item [CLEAR] -- Clear plotting attributes.
\item [DLIM(ITS)] -- Set the world coordinate limits using the data range
(equivalent to \cnam{WORLD DATA}).
\item [EXPA(ND)] -- Set the character height (equivalent to \cnam{CHANGE
CHEIGHT=}).
\item [FONT] -- Set the text font (equivalent to \cnam{CHANGE FONT=}).
\item [INQ(UIRE)] -- Display PONGO status information.
\item [LIM(ITS)] -- Set the world coordinate limits (equivalent to 
\cnam{WORLD GIVEN}).
\item [LT(YPE)] -- Set the line style (equivalent to \cnam{CHANGE LINESTY=}).
\item [LWE(IGHT)] -- Set the line width (equivalent to \cnam{CHANGE LINEWID=}).
\item [MOVE] -- Set the current pen position (equivalent to \cnam{PRIM M}).
\item [PALET(TE)] -- Change the plotting pen colours.
\item [PAPER] -- Change the size and aspect ratio of the plotting surface.
\item [PEN] -- Set the current pen (equivalent to \cnam{CHANGE COLOUR=}).
\item [RESETP(ONGO)] -- Reset the state of PONGO (ICL hidden procedure using 
\cnam{SETGLOBAL}, \cnam{CLEAR}, \cnam{VIEWPORT}, \cnam{WORLD} and 
\cnam{CHANGE}).
\item [SHOWP(ONGO)] -- Show the PONGO global parameters (ICL hidden procedure
using \cnam{GETGLOBAL}).
\item [VIEWPORT] -- Set the viewport for the current plotting device.
\item [VPORT] -- Set the viewport using normalised device
coordinates (equivalent to \cnam{VIEWPORT NDC}).
\item [VP\_BH] -- Set the viewport to the bottom half of the plotting
surface (equivalent to \cnam{VIEWPORT NDC 0.0833 0.917 0.05 0.45}). 
\item [VP\_BL] -- Set the viewport to the bottom-left quarter of the 
plotting surface (equivalent to \cnam{VIEWPORT NDC 0.0417 0.459 0.05 0.45}). 
\item [VP\_BR] -- Set the viewport to the bottom-right quarter of the 
plotting surface (equivalent to \cnam{VIEWPORT NDC 0.5417 0.959 0.05 0.45}). 
\item [VP\_TH] -- Set the viewport to the top half of the plotting
surface (equivalent to \cnam{VIEWPORT NDC 0.0833 0.917 0.55 0.95}). 
\item [VP\_TL] -- Set the viewport to the top-left quarter of the 
plotting surface (equivalent to \cnam{VIEWPORT NDC 0.0417 0.459 0.55 0.95}). 
\item [VP\_TR] -- Set the viewport to the top-right quarter of the 
plotting surface (equivalent to \cnam{VIEWPORT NDC 0.5417 0.959 0.55 0.95}). 
\item [VSIZE] -- Set the viewport using its physical size in inches
(equivalent to \cnam{VIEWPORT INCHES}). 
\item [VSTAND] -- Set the standard viewport (equivalent to 
\cnam{VIEWPORT STANDARD}).
\item [WNAD] -- Adjust the viewport so that the X and Y scales are the same
(equivalent to \cnam{VIEWPORT ADJUST}).
\item [WORLD] -- Set the world coordinates for the plot.
\end {description}
\end {quote}
\normalsize


\subsection{Commands for manipulating data}

\small
\begin {quote}
\begin {description} 
\item [AVEDAT] -- Average the data in the XCOL and YCOL areas.
\item [CCMATH] -- Perform inter-column maths.
\item [CLOG] -- Take the logarithm of a column.
\item [DATA] -- Specify the data file name (equivalent to \cnam{SETGLOBAL
PONGO\_DATA}).
\item [DEGTOR] -- Convert the specified data area from degrees to radians
(ICL hidden procedure using \cnam{CCMATH}).
\item [EXC(OLUMN)] -- Specify the column containing the X-axis error data
(equivalent to \cnam{SETGLOBAL PONGO\_EXCOL}).
\item [EYC(OLUMN)] -- Specify the column containing the Y-axis error data
(equivalent to \cnam{SETGLOBAL PONGO\_EYCOL}). 
\item [GETP(OINT)] -- Retrieve information for a plotted data point.
\item [LABC(OLUMN)] -- Specify the column used for data labels (equivalent to
\cnam{SETGLOBAL PONGO\_LABCOL}).
\item [PCOL(UMN)] -- Specify the column used for symbol codes (equivalent
to \cnam{SETGLOBAL PONGO\_SYMCOL}).
\item [PTINFO] -- Get the coordinates of a specified data point (ICL hidden
procedure using \cnam{GETPOINT}).
\item [READ(F)] -- Read from a formatted data file.
\item [RTODEG] -- Convert the specified data area from radians to degrees (ICL
hidden procedure using \cnam{CCMATH}).
\item [SYMC(OLUMN)] -- Specify the column used for symbol codes (equivalent to
\cnam{SETGLOBAL PONGO\_SYMCOL}). 
\item [WRITE(I)] -- Write information to an output file.
\item [XC(OLUMN)] -- Specify the column containing the X-axis data (equivalent
to \cnam{SETGLOBAL PONGO\_XCOL}).
\item [XLIN(EAR)] -- Put 1\ldots N into the XCOL data area (equivalent to 
\cnam{CCMATH X=INDEX}).
\item [XLOG(ARITHM)] -- Take the logarithm of the X-axis data (equivalent to
\cnam{CLOG X}).
\item [XOFF(SET)] -- Add a constant offset to the X-axis data (ICL hidden
procedure using \cnam{CCMATH}).
\item [XSCALE] -- Multiply the values in the XCOL and EXCOL data areas
by a constant (ICL hidden procedure using \cnam{CCMATH}).
\item [YC(OLUMN)] -- Specify the column containing the Y-axis data (equivalent
to \cnam{SETGLOBAL PONGO\_YCOL}).
\item [YLIN(EAR)] -- Put 1\ldots N into the YCOL data area (equivalent to 
\cnam{CCMATH Y=INDEX}).
\item [YLOG(ARITHM)] -- Take the logarithm of the Y-axis data (equivalent to
\cnam{CLOG Y}).
\item [YOFF(SET)] -- Add a constant offset to the Y-axis data (ICL hidden
procedure using CCMATH).
\item [YSCALE] -- Multiply the values in the YCOL and EYCOL data areas 
by a constant (ICL hidden procedure using \cnam{CCMATH}).
\item [ZC(OLUMN)] -- Specify the column containing the Z-axis data (equivalent
to \cnam{SETGLOBAL PONGO\_ZCOL}).
\item [ZSCALE] -- Multiply the values in the ZCOL data area by a constant (ICL
hidden procedure using \cnam{CCMATH}).
\end {description}
\end {quote}
\normalsize


\subsection{Commands for performing simple statistics}

\small
\begin {quote}
\begin {description} 
\item [FITC(URVE)] -- Fit a curve to the data.
\item [FITL(INE)] -- Fit a straight line to the data.
\end {description}
\end {quote}
\normalsize


\section{Data File Formats} \label{data_sect}

The simplest form of input file for PONGO is a text file with the data in
columns separated by spaces, as in the file \verb+PONGO_EXAMPLES:TUTORIAL.DAT+.
However, PONGO allows a considerable level of fine control over the data used
for plotting from a particular file by providing for the use of column
delimiters, column labels and comments in data files.


\subsection{Column delimiters}

The default column delimiter is a space character, although the \cnam{READF}
command does have the ability to use other delimiters by setting the
\pnam{DELIM} parameter. 
It is possible for more than one delimiter character to be used, \eg using
\verb+&\+ would be a good way to read a table that was in \TeX\ format 
(\verb+&+ for a \LaTeX\ tabular format table). 
A null string for the \pnam{DELIM} parameter has the same effect as a single
space.


\subsection{Column labels}

It is possible to give each column in a file a symbolic name that can be used
to reference the column when reading the file and can be automatically
transferred to the appropriate axis label on the plot. 
To do this, the first line of the file should be of the form

\begin{quote}
\begin{verbatim}
!$label 1$label 2$label 3$
\end{verbatim}
\end{quote}

where there are as many labels, each delimited by a \verb+$+, as there are
columns. 
Care should be taken to ensure that there is no leading white space in the
column labels, although it is permissible for the column labels to contain
white space elsewhere. 
Any padding that is required to make the column labels line up with the data
columns should be achieved with multiple dollar signs,
\eg

\begin{quote}
\begin{verbatim}
!$RA$$$$Declination$
\end{verbatim}
\end{quote}

When specifying the column on the command line, \eg \cnam{YCOL Dec}, it is
permissible to abbreviate the string to a minimum match.
However, {\em the match is case-sensitive.}


\subsection{Comments}

Comments may be placed in the data file by prefixing the line with one of the
standard comment characters. 
There are two comment characters allowed in the data file, specified by the
parameters \pnam{HARDCOMM} and \pnam{SOFTCOM} for the command \cnam{READF}. 
These comment characters must occur in the first column of a line
to be recognised as comment characters. 
The main purpose of comment characters is to document data files and to 
comment out unwanted lines of data. 
The existence of two comment characters provides the ability to selectively 
read data subsets from files.
Blank lines are ignored in data files.

Although not recommended practice, the documentation of data files can be done
without the use of comment characters: because PONGO will reject any line in 
which the required numerical column cannot be interpreted as a valid number,
it does not matter whether a comment character is put at the start of the line
or not.


\subsection{Astronomical coordinates}

It is possible to read data stored in an astronomical coordinate format (\ie
\verb+hh:mm:ss.sss+ and \verb+dd:mm:ss.sss+) into PONGO. 
The data is stored internally in radians: in this conversion any data read
into \xcol\ is assumed to be a Right Ascension (\ie of the first form above)
and any data read into \ycol\ is assumed to be a Declination (\ie of the second
form above).


\section{Internal Data Areas}

There are seven internal data areas into which data may be read.
These are as follows:

\begin{quote}
\begin{description}
\item [\xcol] -- This area is principally intended to hold X-axis data.
However, when using the command \cnam{PLOTHIST H}, this area is intended to
hold the data before they are binned to draw the histogram.

When using the command \cnam{ERRORBAR X} with the symmetric option turned
off, this area contains the position of one of the ends of the error bar.

\item [\ycol] -- This area is intended to hold Y-axis data.

When using the command \cnam{ERRORBAR Y} with the symmetric option turned
off, this area contains the position of one of the ends of the error bar.

\item [\excol] -- This area is principally intended to hold errors for the
X-axis data values. 
These errors are generally assumed to be symmetric about the data point
by the command \cnam{ERRORBAR}, but with the symmetric option turned off
\excol\ holds the position of the opposite end of the error bar to
\xcol.

The \excol\ area is assumed to contain the standard error of the \xcol\ data by
the command \cnam{ELLIPSES}.

The \excol\ area is also used to hold the X-axis component of the vector offset
from the data point by the command \cnam{VECT}, and the proper motion in Right
Ascension by the command \cnam{PVECT}.

When using the command \cnam{ARC}, the \excol\ data are assumed to contain the
magnitude of the semi-major axis of the ellipse.

The command \cnam{AVEDAT} will write the standard deviation of the average 
it calculated for each bin into the \excol\ data area.

\item [\eycol] -- Similar to \excol, but applying to the Y-axis data values.

The \eycol\ area is used to hold the Y-axis component of the vector offset
from the data point by the command \cnam{VECT}, and the proper motion in 
Declination by the command \cnam{PVECT}.

When using the command \cnam{ARC}, the \eycol\ data are assumed to contain the
magnitude of the semi-minor axis of the ellipse.

\item [\zcol] -- Currently, PONGO does not have the ability to draw three
dimensional plots.
As a result, the \zcol\ data area does not contain Z-axis data in the
conventional sense.
However, there are a number of commands that use the values stored in the
\zcol\ area. 

The command \cnam{SIZEPLOT} (a synonym of \cnam{GPOINTS S}) uses the values
stored in \zcol\ to scale the size of the marker symbols plotted for each data
point. 

The commands \cnam{VECT} and \cnam{PVECT} can optionally use the values in
\zcol\ to scale the length of the vectors they draw.

The command \cnam{ELLIPSES} assumes that \zcol\ contains the values of the
normalised covariances between the X- and Y-axis errors.

\item [\labcol] This area is used for storing a character string associated
with each of the data points. 
The strings may be plotted interactively using the \cnam{ANNOTATE} or
\cnam{CURSE} commands.

\item [\symcol] This area is used to store a PGPLOT symbol number (of type
\verb+INTEGER+) associated with each of the data points. 

% Note that this symbol number is not restricted to the 32 specified as 
% markers in the PGPLOT manual, any of the Hershey character numbers can 
% be used as well.
\end{description}
\end{quote}


\section{Projections} \label{proj_sect}

PONGO is capable of plotting astronomical coordinate data in one of several
`projections' or, more strictly, in one of several geometries (not all of the
`projections' are true projections onto a plane, but are geometries that have
other properties -- \eg the equal area property where areas are preserved in
the transformation from the celestial sphere to a two dimensional plot).
{\em Table \ref{proj_tab}} lists the geometries available in PONGO.

All the main point plotting commands will work in any of the available 
geometries.
They assume that the data in the \xcol\ and \ycol\ areas are stored internally
in radians and refer to the longitude and latitude respectively of the point to
be plotted.
Where the position values are to be entered on the command line, the latitude
and longitude are normally given in degrees.

In addition to the point plotting commands, there are also three commands whose
only use is when plotting in a particular projection type:

\begin {quote}
\begin{description}
\item [GRID] -- Plot a coordinate grid in the current projection. 
The default parameters of this command are to draw a grid over the whole sky
with a longitude line every 30 degrees and a latitude line every 10 degrees.
\item [GT\_CIRCLE] -- Draw a great circle between two specified points on
the celestial sphere.
\item [PVECT] -- Draw proper motion vectors from each plotted point as small
great circle arcs.
\end{description}
\end {quote}

\begin{table}[t]
\centering
\begin{tabular}{|l|p{0.7\textwidth}|} \hline
& \\
Name     & Description \\
& \\ \hline
& \\
NONE     & No projection is used. \\
AITOFF   & The Aitoff projection (has an equal area property). \\
ARC      & Move an equal distance on the tangent plane as a great circle on the
celestial sphere. \\
GLS      & Right Ascension intervals are multiplied by $\cos\delta$ (has an
equal area property). \\
MERCATOR & The Mercator projection. \\
SIN      & Drop a perpendicular onto the tangent plane. \\
STG      & The Stereographic projection. \\
TAN      & Projection onto the tangent plane through the centre of the
celestial sphere. \\
& \\ \hline
\end{tabular}
\caption{Projection geometries available in PONGO} \label{proj_tab}
\end{table}


\section{Using PONGO With Other ADAM Applications}

Using graphics between different applications packages is often difficult
because once one package has finished plotting and completed execution all
information concerning the contents of the plot is lost.
This problem may be overcome by storing relevant graphical information for the
plot (\eg the graphics device, the size and position of the plot on the display
surface, the coordinate limits of the data) in a file and providing facilities
to store and retrieve the information in this database.
This allows different applications to be used to display images, draw contours
and  annotate, for example, on the same plot without each application losing
access to the plot dimensions.
One such facility is provided by Starlink and is called AGI -- Applications
Graphics Interface (SUN/48).
As its name implies, this facility provides a means by which any application
can store plotting information for later use and retrieve plotting information
stored by previous applications.
To use AGI to good effect requires familiarity with some nomenclature.

AGI divides the plotting surface up into {\bf pictures}, where a picture
represents a rectangular area on the display surface.
AGI stores coordinate information for each picture it creates on a graphics
device in its database.
AGI pictures are independent of the graphics package used to plot the data
(where supported; \ie SGS, IDI or PGPLOT), but they do depend explicitly upon 
the graphics device used to plot the data.
AGI stores each picture in its database, grouped by the graphics device (or 
{\bf workstation} in Starlink graphics terminology) upon which the picture was
created, and each is stored sequentially in the order in which they were
created.
AGI pictures are given a name which indicates their use:
a {\bf BASE} picture refers to the whole plotting surface and the name is
reserved only for this use;
a {\bf FRAME} picture refers to a picture which contains other pictures;
and a {\bf DATA} picture refers to a picture which contains a plot of the data.
AGI also stores a label for each picture: a short character string which
can be used by an application to uniquely identify a picture.
Finally, AGI can store a one-line comment for each picture in the database to 
describe its use.

PONGO uses the definitions of {\bf world coordinates}, {\bf window},
{\bf device coordinates} and {\bf viewport} used by the PGPLOT graphics 
package (SUN/15) upon which it is based.
Here, a PONGO plot represents a window of given size in the data or world
coordinates the user is working in.
The data are plotted within these world coordinate limits in a rectangular
area of the workstation surface.
This area is called the viewport and its size and shape are defined in terms of
the device coordinates.
The data are plotted exactly within this viewport and so plot annotation and
labelling will normally extend beyond the viewport limits.
AGI pictures refer directly to PONGO viewports.
Clearly, not all viewports used by PONGO will be associated with data windows
and refer to plotted data; many will just define regions of the workstation
surface in which further viewports will be defined.
These viewports are directly analogous to the AGI FRAME pictures and are
stored as such by AGI.
Those viewports which are used for plotting data within are directly analogous
to AGI DATA pictures and again are stored as such by AGI.

By default, when \cnam{BEGPLOT} (or \cnam{DEVICE}) is used to begin a new 
PONGO plotting session; \eg

\begin{quote}
\begin{verbatim}
ICL> BEGPLOT IKON_VT
\end{verbatim}
\end{quote}

PONGO will open the device and use the AGI BASE picture for that device, unless
there is a more recent picture.
In the case of a more recent picture, a FRAME picture will be created with the
dimensions of the picture current on entry.
In either case, the workstation will also be cleared.
When a PONGO plot ends, using \cnam{ENDPLOT}, the AGI picture current on entry
is again made current.

There are two further modes in which a PONGO plot may be begun: assert that
PONGO use the BASE picture for plotting (the \cnam{BEGPLOT BASE} parameter); 
and assert that PONGO use the last DATA picture in the current FRAME, without
clearing the graphics device (the \cnam{BEGPLOT OVERLAY} parameter).
With the \cnam{BEGPLOT BASE} parameter you can force PONGO to make the AGI BASE
picture current on entry; \ie if you specifically want to use the whole display
surface for plotting.
The latter mode (\cnam{BEGPLOT OVERLAY}) is designed for PONGO to be used for 
plotting with other applications packages in a coordinated way.
Here, PONGO will search for the last DATA picture in the AGI database and use
the associated coordinates to define the PGPLOT viewport.
The workstation will not be cleared on entry.
This means that anything plotted with PONGO will be on the same axes and to 
the same scale as the existing plot.

As an example, PONGO can be used to annotate images displayed using KAPPA.
First, KAPPA will be used to set the image display device and to clear the AGI
database for the device (\ie a tidy start):

\begin{quote}
\begin{verbatim}
ICL> KAPPA
ICL> IDSET IKON_VT
ICL> IDCLEAR
\end{verbatim}
\end{quote}

It is necessary to first define the viewport the image is to be plotted within
using PONGO in order to assure that there is enough room around the image for
the annotation (remember, the annotation of PONGO graphs normally lies outside
the viewport):

\begin{quote}
\begin{verbatim}
ICL> PONGO
PONGO> BEGPLOT IKON_VT
PONGO> VSTAND
PONGO> ENDPLOT
\end{verbatim}
\end{quote}

Then use KAPPA to display the image:

\begin{quote}
\begin{verbatim}
ICL> PICLIST PICNUM=3
ICL> DISPLAY PONGO_EXAMPLES:DOR MODE=PERC PERCENTILES=[3,99.9]
ICL> LUTHEAT
\end{verbatim}
\end{quote}

Here, KAPPA PICLIST is used to select the last defined AGI DATA picture, which
is a DATA picture created by PONGO with enough space around it to annotate the
plot.
Note that PICLIST lists a table of pictures in the AGI database.
These include the BASE picture, referring to the entire plotting surface of the
graphics device; the PONGO FRAME picture, defined on entry and again referring 
to the entire plotting surface; and finally the PONGO DATA picture, referring
the area of the plotting surface within which the image is displayed.

Now PONGO may be used to annotate the plot:

\begin{quote}
\begin{verbatim}
ICL> BEGPLOT IKON_VT OVERLAY
PONGO> BOXFRAME BCINST BCINST
PONGO> LABEL "X-axis pixel" "Y-axis pixel" "KAPPA image"
PONGO> ENDPLOT
\end{verbatim}
\end{quote}

The image has been displayed by KAPPA and then the axes drawn and annotated
using PONGO.
In this case, the annotation represents the pixel number in each of the axes
of the image.
A more detailed example of the interaction between KAPPA and PONGO, via AGI, is
provided in \S\ref{exam_sect}.


\section{Examples} \label{exam_sect}

These examples are intended to show some of the features of PONGO.
The ICL procedure scripts for these examples can be found in the directory 
\verb+PONGO_EXAMPLES+.
Each procedure will prompt for a graphics device and should be invoked from the 
\verb+ICL>+ prompt using the ICL command \verb+LOAD+, \eg

\begin {quote}
\begin{verbatim}
PONGO> LOAD PONGO_EXAMPLES:PPDOTDIAG
\end{verbatim}
\end {quote}

and their execution can be followed if the ICL command \verb+SET TRACE+ has
previously been executed.

Note that some of the examples involve the colour representation of lines, 
which may be difficult to see on a greyscale device (\eg on a monochrome
Postscript device or a monochrome image display).
It is recommended that a colour image display device is used to run the
example procedures.

\newpage

\subsection{SPECTRUM.ICL}

This procedure produces {\em Figure \ref{spec_fig}}, a plot of a low resolution
IUE spectrum extracted by IUEDR (SUN/37) and written using the IUEDR command 
\verb+OUTSPEC SPECTYPE=2+.
The output file was subsequently edited to make the file label lines PONGO
comments and to add a line of PONGO column labels (see \S\ref{data_sect}) 
at the beginning of the file.
IUEDR indicates bad or missing data in an output spectrum by attributing zero
fluxes to the affected wavelength samples.
These can be detected and discarded using the \pnam{SELCOND} parameter of 
the \cnam{READF} command: \eg

\begin{quote}
\begin{verbatim}
PONGO> READF PONGO_EXAMPLES:SWP3196.LAP SELCOND="Flux > 0.0" NOALL
\end{verbatim}
\end{quote}

In this example PONGO draws a polyline across all missing data flagged by
IUEDR.

\begin{figure}[htbp]
\centering
\halfpfig{sun137_fig1.ps-tex}
\caption{SPECTRUM.ICL} \label{spec_fig}
\end{figure}

\newpage


\subsection{ERRORS.ICL}

The procedure \verb+ERRORS.ICL+ was used to plot {\em Figure \ref{error_fig}}.
This example demonstrates plotting simple error bars using PONGO
(the \cnam{ERRORBAR} command) and also performing simple statistics on the data
(the \cnam{FITLINE} and \cnam{FITCURVE} commands).
Note that a summary of the statistics is reported for each fit to the data.

\begin{figure}[htbp]
\centering
\halfpfig{sun137_fig2.ps-tex}
\caption{ERRORS.ICL} \label{error_fig}
\end{figure}

\newpage


\subsection{HISTOGRAM.ICL}

The procedure \verb+HISTOGRAM.ICL+ was used to plot {\em Figure \ref{hist_fig}},
which illustrates the plotting of histograms with PONGO (the \cnam{PLOTHIST} 
command).
This procedure also illustrates how the drawing of the box around the plot 
can be controlled using the \cnam{BOXFRAME} command.

\begin{figure}[htbp]
\centering
\halfpfig{sun137_fig3.ps-tex}
\caption{HISTOGRAM.ICL} \label{hist_fig}
\end{figure}

\newpage


\subsection{PPDOTDIAG.ICL}

This procedure produces {\em Figure \ref{ppdot_fig}}, a period versus period
derivative diagram for pulsars.
Note the use of a column within the data file to set the symbol number of each
plotted point, and the use of the \verb+PONGO_NDATA+ global parameter for
making a title containing the number of points that have been read in.

\begin{figure}[htbp]
\centering
\halfpfig{sun137_fig4.ps-tex}
\caption{PPDOTDIAG.ICL} \label{ppdot_fig}
\end{figure}

\newpage


\subsection{ELLIPSES.ICL} 

The procedure \verb+ELLIPSES.ICL+ was used to plot {\em Figure \ref{ellip_fig}},
which illustrates the use of the \cnam{ELLIPSES} command for plotting error
ellipses.

\begin{figure}[htbp]
\centering
\halfpfig{sun137_fig5.ps-tex}
\caption{ELLIPSES.ICL} \label{ellip_fig}
\end{figure}

\newpage


\subsection{PROJECTIONS.ICL}

This procedure illustrates some of the different `geometries' available in
PONGO.
It plots four different views of the distribution of a selection of the known
pulsars in Right Ascension and Declination to produce {\em Figure
\ref{proj_fig}}.

\begin{figure}[htbp]
\centering
\halfpfig{sun137_fig6.ps-tex}
\caption{PROJECTIONS.ICL} \label{proj_fig}
\end{figure}

\newpage


\subsection{Interactive input}

After executing \verb+PROJECTIONS.ICL+ it is a good idea to gain some
experience with the \cnam{CURSE} application.
This can be done using the procedure \verb+INTERACTIVE.ICL+.
The procedure will plot a graph and then invoke the \cnam{CURSE} application,
resulting in the following instructions being printed on the screen

\begin{quote}
\begin{verbatim}
CURSE cursor options:
   Q - Quit.
   D - Draw to the cursor position.
   E - Expand the plotting limits.
   G - Calculate the gradient.
   L - Label the plot.
   M - Mark a point.
   O - Left justified annotation.
   P - Right justified annotation.
   S - Shrink the plotting limits.
   V - Move to the cursor position.
   X - Get the cursor position.
   Z - Zoom the plotting limits.
\end{verbatim}
\end{quote}

where the letter signifies the key that is to be pressed on the keyboard to
achieve the desired effect. 

If you have examined the file \verb+INTERACTIVE.ICL+ you will have seen that a
label has been read in for each of the data points. 
It is now possible for you to use these labels by moving the cursor close to a
plotted point and pressing \verb+O+ or \verb+P+.
When you do this, a label for the point nearest the cursor will be written at
the cursor position.
Do this for several points and then press \verb+Q+ to exit from the
\cnam{CURSE} command.
PONGO will have remembered the particular labelling that that you have
performed and will then use the command \cnam{WRITEI LABLST} to write the
PONGO commands required to recreate these labels to a specified file.
(You can use the procedure \verb+INTERACTIVE.ICL+ to explore the other
functions of the \cnam{CURSE} application.)

The \zcol\ area has been filled with the distances to each of the
pulsars.
These values can be used to plot points whose sizes are inversely proportional
to distance (\ie the closer pulsars are larger) by first taking the reciprocal
of the \zcol\ area with the \cnam{CCMATH} command:

\begin{quote}
\begin{verbatim}
PONGO> ADVANCE
PONGO> GRID
PONGO> CCMATH Z=1/(Z+0.1)
\end{verbatim}
\end{quote}

and then the points plotted with the command 

\begin{quote}
\begin{verbatim}
PONGO> SIZEPLOT
\end{verbatim}
\end{quote}

Note here that a small positive value has been added to the distance data
(\ie {\tt Z+0.1}) to avoid very large symbols for nearby pulsars.


\subsection{AGI.ICL}

This procedure should be invoked from the \verb+ICL>+ prompt {\em not}
from the \verb+PONGO>+ prompt (the example procedure will prompt for the name
of an image display device). 
It illustrates the interaction between the KAPPA package (SUN/95) and PONGO 
via the AGI graphics database.
Here, the image display has been done using the KAPPA \cnam{DISPLAY} command
and PONGO has been used for all the line drawing.
The IKONPAINT utility (SUN/71) may be used to obtain a colour hard-copy of this
illustration. 


\section{Writing PONGO Procedures}

PONGO procedures are ICL procedures (SG/5) which make use of PONGO commands.
ICL is a very powerful, programmable command language which can be used to
interact between PONGO and any other ADAM package by means of ICL variables and
the ADAM parameter system.
To become familiar with using ICL to write PONGO procedures it is recommended 
that the example procedures provided in the directory \verb+PONGO_EXAMPLES+ are 
examined in some detail.
These procedures illustrate many features of both ICL and PONGO, with comments
to explain some of the more complex commands (any line beginning {\bf \{ } in 
these scripts is a comment).
More detailed information concerning ICL is provided in SG/5.


\section{Panic Section}

\subsection{Getting help}

When in ICL, on-line help on PONGO may be examined using the command

\begin{quote}
\begin{verbatim}
ICL> HELP PONGO
\end{verbatim}
\end{quote}

This will provide a brief description of the package and how to begin and end a 
PONGO plotting session.
Once the PONGO commands have been made available within ICL, \ie by typing

\begin{quote}
\begin{verbatim}
$ ADAM PONGO
\end{verbatim}
\end{quote}

at the DCL prompt or

\begin{quote}
\begin{verbatim}
ICL> PONGO
\end{verbatim}
\end{quote}

at the \verb+ICL>+ prompt, on-line help on any PONGO command may be examined
using the command

\begin{quote}
\begin{verbatim}
ICL> HELP <PONGO command>
\end{verbatim}
\end{quote}

The command

\begin{quote}
\begin{verbatim}
ICL> HELP PONGO
\end{verbatim}
\end{quote}

will produce a brief introduction to PONGO, and the command 

\begin{quote}
\begin{verbatim}
ICL> HELP INTRODUCTION
\end{verbatim}
\end{quote}

will produce a more detailed introduction to PONGO and how to get started.
These help commands are also available at the \verb+PONGO>+ prompt.
A classified list of PONGO applications (\cnam{HELP CLASSIFIED}) is provided
within the help system and help is also provided on running the PONGO examples
(\cnam{HELP EXAMPLES}). 

If difficulties are encountered with PONGO and the on-line help system does not
reveal the cause, it is recommended that the relevant sections of this document
are read, or re-read.
It may be that this document does not provide enough detail concerning the
behaviour of the PGPLOT graphics package: for detailed information regarding 
PGPLOT, the PGPLOT manual should be consulted (available as a Miscellaneous User
Document, MUD, at all Starlink sites).
As a last resort, the problem may be regarded as a software bug and reported to
the STARLINK Software Librarian (RLVAD::STAR).


\subsection{\cnam{BEGPLOT} or \cnam{DEVICE} does not work}

If the error message

\begin{quote}
\begin{verbatim}
TOOFEWPARS   Not enough procedure parameters
\end{verbatim}
\end{quote}

is returned after the \cnam{BEGPLOT} or \cnam{ENDPLOT} commands have been
invoked, the reason is that another application package or ICL procedure has
switched parameter checking on within ICL (SG/5).
The problem is simply overcome by typing the command

\begin{quote}
\begin{verbatim}
ICL> SET NOCHECKPARS
\end{verbatim}
\end{quote}

The PONGO commands will then work correctly.


\subsection{\cnam{READF} cannot find a data file}
                   
If \cnam{READF} fails to find a file referred to by a logical name, \eg

\begin{quote}
\begin{verbatim}
PONGO> READF MYSCRATCH:MYFILE.DAT XCOL=1 YCOL=3
!! FIO_ASSOC/DATA FIO-I-FILNF, file not found
!  Error opening the file.
!  READF: Data could not be read.
\end{verbatim}
\end{quote}

then either the data file does not exist or the logical name has been defined
incorrectly.
It is of particular importance to define all logical names to be used within
PONGO (or any ADAM package run from ICL) as job logical names, \eg

\begin{quote}
\begin{verbatim}
$ DEFINE /JOB MYSCRATCH DISK$SCRATCH:[JBLOGGS]
\end{verbatim}
\end{quote}

This is because all ADAM applications run from ICL are run within subprocesses,
which do not inherit process logical names.


\subsection{\cnam{READF} cannot find a column label}

If a column label is used with the \cnam{READF} command instead of the column
number (see \S\ref{data_sect}) and \cnam{READF} fails, \eg

\begin{quote}
\begin{verbatim}
PONGO> READF MYFILE.DAT XCOL=1 YCOL=3 ZCOL="ZCOL"
File: DISK$SCRATCH:[JBLOGGS.PLOTS]MYFILE.DAT;1
XCOL - 1 is column number 1.
YCOL - 3 is column number 3.
!! ZCOL incorrectly specified as ZCOL.
!  READF: Data could not be read.
\end{verbatim}
\end{quote}

then either the data file does not have column labels or the incorrect
column label has been used.
It is of particular significance here that the column labels are {\em 
case-sensitive}.
In fact, JBLOGGS found that the column label for his \zcol\ data was
\pnam{Zcol}:

\begin{quote}
\begin{verbatim}
PONGO> READF MYFILE.DAT XCOL=1 YCOL=3 ZCOL="Zcol"
File: DISK$SCRATCH:[JBLOGGS.PLOTS]MYFILE.DAT;1
XCOL - 1 is column number 1.
YCOL - 3 is column number 3.
ZCOL - Zcol is column number 5.
EXCOL - 0 is column number 0.
EYCOL - 0 is column number 0.
LABCOL - 0 is column number 0.
SYMCOL - 0 is column number 0.
42 data points read.
PONGO>
\end{verbatim}
\end{quote}


\subsection{Strange behaviour of \cnam{DLIMITS}}

If PONGO does not seem to be doing something correctly, it is often
because there are unwanted data in the error columns.
This can come about when a file that does not contain errors is read (the
appropriate parameters to \cnam{READF} have been set to zero), following one
where there have been errors.
Here, it should be noted that setting a column parameter in \cnam{READF} to
zero means that no data will be read from the file into that particular area,
it does not clear the data values.
This is a feature rather than a bug, because it allows X and Y data from two
different files to be plotted together, as long as each file contains the same
number of points (the number of points read from the second file is
assumed to be the number required).
Although advantageous, it is possible for the internal data areas to get into a
mess with this arrangement.
If this seems to have happened then the command 

\begin{quote}
\begin{verbatim}
PONGO> CLEAR DATA
\end{verbatim}
\end{quote}

should be used to clear all the data areas.


\subsection{Error messages from \cnam{SHOWPONGO}}

The commands \cnam{SHOWPONGO} may sometimes deliver the error messages

\begin{quote}
\begin{verbatim}
!! Object 'PONGO_<param>' not found.
!! DAT_FIND: Error finding a named component in an HDS structure.
\end{verbatim}
\end{quote}

when invoked (\verb+<param>+ refers to a PONGO parameter name in this
context).
The reason is simply that no global parameter of that name currently exists.
This can either be because PONGO has not been used before (or not a lot),
or because the ADAM global parameter file (\verb+ADAM_USER:GLOBAL.SDF+) has
been deleted since the last PONGO session.
Although the error message look alarming, they are harmless.
To stop these error messages being output, the command \cnam{RESETPONGO}
can be executed. 
This command will assign a default value to each of the PONGO global parameters
except \pnam{PONGO\_DATA}, the data file name used by \cnam{READF}, and
\cnam{SHOWPONGO} will be subsequently more informative.


\subsection{AGI problems}

The AGI database is kept in a separate HDS file (SG/4) for each machine on your
Local Area VAXcluster in the directory \verb+ADAM_USER+.
The command \cnam{BEGPLOT} opens the AGI database and reads database
information relevant to the graphics device being used.
During a PONGO plotting session, the database file is held open to update the
database as the plotting session proceeds.
At the end of plotting, when \cnam{ENDPLOT} is executed, the update of the 
database is completed and the database file is closed.
If another non-PONGO application which uses AGI is run before the current PONGO
plotting session has been ended (\ie using \cnam{ENDPLOT}), it will be unable
to access the AGI database and will subsequently fail.
In the event of this happening, using the PONGO \cnam{ENDPLOT} command will
restore the correct behaviour of the non-PONGO application.

What is more important, in order to use AGI successfully, is the use of
\cnam{ENDPLOT} before exiting ICL.
Because PONGO keeps the AGI database file open throughout a plotting session it
is possible to exit ICL without having first executed \cnam{ENDPLOT}.
If this is done, it is likely that the AGI database will not have been fully
updated before being closed, with the result that the next time AGI is
used it will behave inconsistently.
PONGO crashing during a plotting session (hopefully, a rare event) can also
result in a corrupted AGI database.
There is no solution to this problem other than to delete the database file and
start again.
The AGI database files are kept in the directory \verb+ADAM_USER+ and have
names of the form \verb+AGI_<machine>.SDF+, where \verb+<machine>+ is the 
name of the machine on which AGI is being used.
This file may be deleted at the \verb+ICL>+ prompt, {\em not} the \verb+PONGO>+
prompt; \eg

\begin{quote}
\begin{verbatim}
ICL> $ DELETE ADAM_USER:AGI_RLSTAR.SDF;*
\end{verbatim}
\end{quote}

The \cnam{BEGPLOT} command may then be used to begin a new PONGO plot,
creating a new AGI database as a result.


\subsection{RESET peculiarity}

When using the ADAM parameter system (SG/4) \pnam{RESET} qualifier to set
parameters back to their default values (this facility would be particularly
useful for \cnam{BOXFRAME}, for example), there will be no effect on parameters
which get their values from global parameters.
A future release of ADAM may alter the behaviour of \pnam{RESET} so that it
does reset parameters to their defaults in all cases.


\section{References}

\begin{sloppypar}
\begin {trivlist} \item[]
\begin {tabular}{p{15em}lp{22em}}
Bailey, J.A. \& Chipperfield, A.J. & 1993 & SG/5 --- ICL -- The Interactive
Command Language for ADAM.\\
Currie, M.J. & 1992 & SUN/95 --- KAPPA -- Kernel Application Package.\\
Eaton, N. & 1992 & SUN/48 --- AGI -- Applications Graphics Interface.\\
Giddings, J. \& Rees, P. & 1989 & SUN/37 --- IUEDR -- IUE Data Reduction
package.\\
Lawden, M.D. \& Hartley, K.F. & 1992 & SG/4 --- ADAM -- The Starlink Software
Environment.\\
Page, C.G. \& Mellor, G.R. & 1992 & SUN/71 --- IKONPAINT -- Ikon and GWM
window to Inkjet Hard-copy.\\
Terrett, D.L. & 1987 & SUN/64 --- MONGO -- Interactive Plotting Program.\\
Terrett, D.L. & 1991 & SUN/15 --- PGPLOT -- Graphics Subroutine Library.\\
\end {tabular}
\end {trivlist}
\end{sloppypar}

%-----------------------------------------------------------------------------
%
%                        R E F E R E N C E
%
%-----------------------------------------------------------------------------
\appendix

\newpage
\section{Alphabetical List of Commands}

In the following list the parts of the command names outside parentheses
define the minimum abbreviation for that application.

\small
\begin {quote}
\begin {description} 
\item [ADV(ANCE)] -- Clear the graphics screen (equivalent to \cnam{CLEAR
SCREEN}).
\item [ANN(OTATE)] -- Annotate the plotted data.
\item [ARC] -- Draw an arc of an ellipse.
\item [AVEDAT] -- Average the data in the XCOL and YCOL areas.
\item [BEGP(LOT)] -- Open a plotting device.
\item [BIN] -- Plot a histogram using previously binned data (equivalent to
\cnam{PLOTHIST B}).
\item [BOX(FRAME)] -- Draw a frame and axes on the plot.
\item [CCMATH] -- Perform inter-column maths.
\item [CHANGE] -- Change plotting attributes.
\item [CLEAR] -- Clear plotting attributes.
\item [CLOG] -- Take the logarithm of a column.
\item [CONN(ECT)] -- Draw straight lines between the data points (equivalent to
\cnam{GPOINTS C}).
\item [CURSE] -- Use the cursor to get interactive input.
\item [DATA] -- Specify the data file name (equivalent to \cnam{SETGLOBAL
PONGO\_DATA}).
\item [DEGTOR] -- Convert the specified data area from degrees to radians
(ICL hidden procedure using \cnam{CCMATH}).
\item [DEVICE] -- Open a plotting device (equivalent to \cnam{BEGPLOT}).
\item [DLIM(ITS)] -- Set the world coordinate limits using the data range
(equivalent to \cnam{WORLD DATA}).
\item [DRAW] -- Draw a line from the current pen position the specified point
(equivalent to \cnam{PRIM D}).
\item [ELLIPSES] -- Draw error ellipses.
\item [ENDP(LOT)] -- Close down the current plotting device.
\item [ERASE] -- Clear the graphics screen (equivalent to \cnam{CLEAR SCREEN}).
\item [ERRORBAR] -- Draw error bars on the plotted data.
\item [ERRX] -- Draw symmetric error bars in the X direction (equivalent to 
\cnam{ERRORBAR X}). 
\item [ERRY] -- Draw symmetric error bars in the Y direction (equivalent to
\cnam{ERRORBAR Y}).
\item [EXC(OLUMN)] -- Specify the column containing the X-axis error data
(equivalent to \cnam{SETGLOBAL PONGO\_EXCOL}).
\item [EXPA(ND)] -- Set the character height (equivalent to \cnam{CHANGE
CHEIGHT=}).
\item [EYC(OLUMN)] -- Specify the column containing the Y-axis error data
(equivalent to \cnam{SETGLOBAL PONGO\_EYCOL}). 
\item [FITC(URVE)] -- Fit a curve to the data.
\item [FITL(INE)] -- Fit a straight line to the data.
\item [FONT] -- Set the text font (equivalent to \cnam{CHANGE FONT=}).
\item [GETP(OINT)] -- Retrieve information for a plotted data point.
\item [GPOINTS] -- Plot points or lines between the data.
\item [GRID] -- Draw a coordinate grid at specified intervals.
\item [GT\_CIRCLE] -- Draw a great circle between two points.
\item [HIST(OGRAM)] -- Bin and plot a histogram of the data (equivalent to 
\cnam{PLOTHIST H}).
\item [INQ(UIRE)] -- Display PONGO status information.
\item [LABC(OLUMN)] -- Specify the column used for data labels (equivalent to
\cnam{SETGLOBAL PONGO\_LABCOL}).
\item [LAB(EL)] -- Draw the axis labels and title on the plot.
\item [LIM(ITS)] -- Set the world coordinate limits (equivalent to 
\cnam{WORLD GIVEN}).
\item [LT(YPE)] -- Set the line style (equivalent to \cnam{CHANGE LINESTY=}).
\item [LWE(IGHT)] -- Set the line width (equivalent to \cnam{CHANGE LINEWID=}).
\item [MARK] -- Draw a point mark at the specified position (equivalent to 
\cnam{PRIM K}).
\item [MOVE] -- Set the current pen position (equivalent to \cnam{PRIM M}).
\item [MTEXT] -- Draw a text string relative to the viewport (ICL hidden
procedure using \cnam{WTEXT}).
\item [PALET(TE)] -- Change the plotting pen colours.
\item [PAPER] -- Change the size and aspect ratio of the plotting surface.
\item [PCOL(UMN)] -- Specify the column used for symbol codes (equivalent
to \cnam{SETGLOBAL PONGO\_SYMCOL}).
\item [PEN] -- Set the current pen (equivalent to \cnam{CHANGE COLOUR=}).
\item [PLOTF(UN)] -- Plot a given function.
\item [PLOTH(IST)] -- Plot a histogram of the data.
\item [POI(NTS)] -- Draw a point mark at each of the data points (equivalent to
\cnam{GPOINTS P}).
\item [PRIM] -- Perform primitive plotting functions.
\item [PTEXT] -- Draw a text string at the specified position and angle (ICL
hidden procedure using \cnam{WTEXT}).
\item [PTINFO] -- Get the coordinates of a specified data point (ICL hidden
procedure using \cnam{GETPOINT}).
\item [PVECT] -- Draw proper motion vectors.
\item [RADIATE] -- Draw a line from the given point to the first NP data points
(ICL hidden procedure using \cnam{GETPOINT}, \cnam{MOVE} and \cnam{DRAW}).
\item [READ(F)] -- Read from a formatted data file.
\item [RESETP(ONGO)] -- Reset the state of PONGO (ICL hidden procedure using 
\cnam{SETGLOBAL}, \cnam{CLEAR}, \cnam{VIEWPORT}, \cnam{WORLD} and 
\cnam{CHANGE}).
\item [RTODEG] -- Convert the specified data area from radians to degrees (ICL
hidden procedure using \cnam{CCMATH}).
\item [SHOWP(ONGO)] -- Show the PONGO global parameters (ICL hidden procedure
using \cnam{GETGLOBAL}).
\item [SIZE(PLOT)] -- Draw point marks of differing sizes at each of the data
points (equivalent to \cnam{GPOINTS S}).
\item [SYMC(OLUMN)] -- Specify the column used for symbol codes (equivalent to
\cnam{SETGLOBAL PONGO\_SYMCOL}). 
\item [TEXT] -- Draw a text string on the plot at the specified position 
(equivalent to \cnam{WTEXT S}).
\item [VECT] -- Draw vectors from each data point.
\item [VIEWPORT] -- Set the viewport for the current plotting device.
\item [VPORT] -- Set the viewport using normalised device
coordinates (equivalent to \cnam{VIEWPORT NDC}).
\item [VP\_BH] -- Set the viewport to the bottom half of the plotting
surface (equivalent to \cnam{VIEWPORT NDC 0.0833 0.917 0.05 0.45}). 
\item [VP\_BL] -- Set the viewport to the bottom-left quarter of the 
plotting surface (equivalent to \cnam{VIEWPORT NDC 0.0417 0.459 0.05 0.45}). 
\item [VP\_BR] -- Set the viewport to the bottom-right quarter of the 
plotting surface (equivalent to \cnam{VIEWPORT NDC 0.5417 0.959 0.05 0.45}). 
\item [VP\_TH] -- Set the viewport to the top half of the plotting
surface (equivalent to \cnam{VIEWPORT NDC 0.0833 0.917 0.55 0.95}). 
\item [VP\_TL] -- Set the viewport to the top-left quarter of the 
plotting surface (equivalent to \cnam{VIEWPORT NDC 0.0417 0.459 0.55 0.95}). 
\item [VP\_TR] -- Set the viewport to the top-right quarter of the 
plotting surface (equivalent to \cnam{VIEWPORT NDC 0.5417 0.959 0.55 0.95}). 
\item [VSIZE] -- Set the viewport using its physical size in inches
(equivalent to \cnam{VIEWPORT INCHES}). 
\item [VSTAND] -- Set the standard viewport (equivalent to 
\cnam{VIEWPORT STANDARD}).
\item [WNAD] -- Adjust the viewport so that the X and Y scales are the same
(equivalent to \cnam{VIEWPORT ADJUST}).
\item [WORLD] -- Set the world coordinates for the plot.
\item [WRITE(I)] -- Write information to an output file.
\item [WTEXT] -- Draw a text string on the plot.
\item [XC(OLUMN)] -- Specify the column containing the X-axis data (equivalent
to \cnam{SETGLOBAL PONGO\_XCOL}).
\item [XERR] -- Draw symmetric error bars in the X direction (equivalent to
\cnam{ERRORBAR X}). 
\item [XLIN(EAR)] -- Put 1\ldots N into the XCOL data area (equivalent to 
\cnam{CCMATH X=INDEX}).
\item [XLOG(ARITHM)] -- Take the logarithm of the X-axis data (equivalent to
\cnam{CLOG X}).
\item [XOFF(SET)] -- Add a constant offset to the X-axis data (ICL hidden
procedure using \cnam{CCMATH}).
\item [XSCALE] -- Multiply the values in the XCOL and EXCOL data areas
by a constant (ICL hidden procedure using \cnam{CCMATH}).
\item [YC(OLUMN)] -- Specify the column containing the Y-axis data (equivalent
to \cnam{SETGLOBAL PONGO\_YCOL}).
\item [YERR] -- Draw symmetric error bars in the Y direction (equivalent to
\cnam{ERRORBAR Y}). 
\item [YLIN(EAR)] -- Put 1\ldots N into the YCOL data area (equivalent to 
\cnam{CCMATH Y=INDEX}).
\item [YLOG(ARITHM)] -- Take the logarithm of the Y-axis data (equivalent to
\cnam{CLOG Y}).
\item [YOFF(SET)] -- Add a constant offset to the Y-axis data (ICL hidden
procedure using CCMATH).
\item [YSCALE] -- Multiply the values in the YCOL and EYCOL data areas 
by a constant (ICL hidden procedure using \cnam{CCMATH}).
\item [ZC(OLUMN)] -- Specify the column containing the Z-axis data (equivalent
to \cnam{SETGLOBAL PONGO\_ZCOL}).
\item [ZSCALE] -- Multiply the values in the ZCOL data area by a constant (ICL
hidden procedure using \cnam{CCMATH}).
\end {description}
\end {quote}
\normalsize

\newpage
\section{PONGO Command Definitions} \label{defn_sect}

This section gives detailed descriptions for each of the PONGO commands.
The commands are listed in alphabetical order.
A description of the command is followed by a description of each of the 
command parameters and its action.
The parameter descriptions include default behaviour; \ie what value is taken
if a parameter value is not given on the command line.
For simple cases, this behaviour is described in {\bf[} {\em square brackets}
{\bf]}; for more complicated behaviour, a full description is provided.
A empty set of square brackets, \ie {\bf []}, indicates that a parameter value 
{\em must} be specified on the command line.

\small
\begin{sloppypar}
\sstroutine{
   ADVANCE
}{
   Clear the graphics screen
}{
   \sstdescription{
      The plotting surface is cleared.

      This command is a synonym for CLEAR SCREEN.
   }
}
\sstroutine{
   ANNOTATE
}{
   Annotate the plotted data
}{
   \sstdescription{
      Each of the points on the plot is labeled with the appropriate
      internal label (if it has been read from the data file). If no
      parameters are specified, the default action is for the label to
      be written with a zero offset in X and an offset of approximately
      one character height in Y.
   }
   \sstparameters{
      \sstsubsection{
         XOFF = \_REAL (Read and Write)
      }{
         The X coordinate offset of the string relative to each data
         point. The application will use the value 0.0 (i.e. no offset)
         unless a value is given on the command line.
         [0.0]
      }
      \sstsubsection{
         YOFF = \_REAL (Read and Write)
      }{
         The Y coordinate offset of the string relative to each data
         point.

         The application will prompt with a value of about 1/40th of
         the height of the viewport unless a value is given on the
         command line.

         [1/40th of the viewport height.]
      }
      \sstsubsection{
         JUSTIFICATION = \_REAL (Read and Write)
      }{
         The justification about the point specified by XOFF and YOFF
         relative to each data point (in the range 0.0 to 1.0).  Here,
         0.0 means left justify the text relative to the data point,
         1.0 means right justify the text relative to the data point,
         0.5 means centre the string on the data point, other values
         will give intermediate justifications.

         If no value is specified on the command line, the current
         value is used. The current value is initially set to 0.0.
      }
      \sstsubsection{
         PROJECTION = \_CHAR (Read)
      }{
         The projection that has been used to plot the data. This is
         explained in more detail in the section on projections. Allowed
         values: {\tt "}NONE{\tt "}, {\tt "}TAN{\tt "}, {\tt "}SIN{\tt "}, {\tt "}ARC{\tt "}, {\tt "}GLS{\tt "}, {\tt "}AITOFF{\tt "},
         {\tt "}MERCATOR{\tt "}, and {\tt "}STG{\tt "}.

         This parameter is not specified on the command line. The value
         of the global parameter PONGO\_PROJECTN is used. If
         PONGO\_PROJECTN is not defined, the default value {\tt "}NONE{\tt "} is
         used.
      }
      \sstsubsection{
         RACENTRE = \_CHAR (Read)
      }{
         The centre of the projection in RA (i.e. the angle must be
         specified as hh:mm:ss.sss). This parameter is only required for
         PROJECTION values other than {\tt "}NONE{\tt "}.

         This parameter is not specified on the command line. The value
         of the global parameter PONGO\_RACENTRE is used. If
         PONGO\_RACENTRE is not defined, the default value {\tt "}0{\tt "} is used.
      }
      \sstsubsection{
         DECCENTRE = \_CHAR (Read)
      }{
         The centre of the projection in declination (i.e. the angle
         must be specified as dd:mm:ss.sss). This parameter is only
         required for PROJECTION values other than {\tt "}NONE{\tt "}.

         This parameter is not specified on the command line. The value
         of the global parameter PONGO\_DECCENTRE is used. If
         PONGO\_DECCENTRE is not defined, the default value {\tt "}0{\tt "} is used.
      }
   }
}
\sstroutine{
   ARC
}{
   Draw an arc of an ellipse
}{
   \sstdescription{
      A specified arc of an ellipse is drawn from the position angles of
      the start and end of the arc, the semi axes, the position of the
      centre and the rotation of the axes. If no parameters are
      specified then whole ellipses are drawn from the data stored in
      the following data areas:

      \sstitemlist{

         \sstitem
            XCOL -- X centre,

         \sstitem
            YCOL -- Y centre,

         \sstitem
            EXCOL -- semi-major axis,

         \sstitem
            EYCOL -- semi-minor axis.
      }
   }
   \sstparameters{
      \sstsubsection{
         A = \_REAL (Read and Write)
      }{
         The semi-major axis of the ellipse.

         If no value is specified on the command line, the current
         value is used.  If there is no current value, a default value
         1.0 is used.
      }
      \sstsubsection{
         B = \_REAL (Read and Write)
      }{
         The semi-minor axis of the ellipse.

         If no value is specified on the command line, the current
         value is used.  If there is no current value, a default value
         1.0 is used.
      }
      \sstsubsection{
         X0 = \_DOUBLE (Read and Write)
      }{
         The X coordinate of the centre of the ellipse.

         If no value is specified on the command line, the current
         value is used. If there is no current value, a default value
         0.0 is used.
      }
      \sstsubsection{
         Y0 = \_DOUBLE (Read and Write)
      }{
         The Y coordinate of the centre of the ellipse.

         If no value is specified on the command line, the current
         value is used. If there is no current value, a default value
         0.0 is used.
      }
      \sstsubsection{
         PASTART = \_REAL (Read and Write)
      }{
         The position angle of the start of the arc (degrees).

         If no value is specified on the command line, the current
         value is used. If there is no current value, a default value
         0.0 is used.
      }
      \sstsubsection{
         PAEND = \_REAL (Read and Write)
      }{
         The position angle of the end of the arc (degrees).

         If no value is specified on the command line, the current
         value is used. If there is no current value, a default value
         360.0 is used.
      }
      \sstsubsection{
         ROTATION = \_REAL (Read and Write)
      }{
         The angle that the major axis makes with the horizontal
         (degrees anti-clockwise).

         If no value is specified on the command line, the current
         value is used. If there is no current value, a default value
         0.0 is used.
      }
      \sstsubsection{
         FROMDATA = \_LOGICAL (Read)
      }{
         If TRUE, the command will use the data already loaded to draw
         whole ellipses, with positions and sizes specified as above.
         [FALSE]
      }
      \sstsubsection{
         PROJECTION = \_CHAR (Read)
      }{
         The geometry that is to be used to plot the arc.  This is
         explained in detail in the section on projections.  Allowed
         values: {\tt "}NONE{\tt "}, {\tt "}TAN{\tt "}, {\tt "}SIN{\tt "}, {\tt "}ARC{\tt "}, {\tt "}GLS{\tt "}, {\tt "}AITOFF{\tt "},
         {\tt "}MERCATOR{\tt "} and {\tt "}STG{\tt "}.

         This parameter is not specified on the command line. If no
         value is specified on the command line, the current value is
         used. If there is no current value, a default value {\tt "}NONE{\tt "} is
         used.
      }
      \sstsubsection{
         RACENTRE = \_CHAR (Read)
      }{
         The centre of the projection in RA (i.e. the angle must be
         specified as hh:mm:ss.sss). This parameter is only required for
         PROJECTION values other than {\tt "}NONE{\tt "}.

         This parameter is not specified on the command line. The value
         of the global parameter PONGO\_RACENTRE is used. If
         PONGO\_RACENTRE is not defined, the default value {\tt "}0{\tt "} is used.
      }
      \sstsubsection{
         DECCENTRE = \_CHAR (Read)
      }{
         The centre of the projection in declination (i.e. the angle
         must be specified as dd:mm:ss.sss). This parameter is only
         required for PROJECTION values other than {\tt "}NONE{\tt "}.

         This parameter is not specified on the command line. The value
         of the global parameter PONGO\_DECCENTRE is used. If
         PONGO\_DECCENTRE is not defined, the default value {\tt "}0{\tt "} is used.
      }
      \sstsubsection{
         ERSCALE = \_REAL (Read and Write)
      }{
         The factor used to scale values in the EXCOL and EYCOL data
         areas. This allows the ellipse axes lengths to be scaled,
         changing the sizes of ellipses produced using the FROMFILE
         parameter.

         This parameter is not specified on the command line. The value
         of the global parameter PONGO\_ERSCALE is used. If
         PONGO\_ERSCALE is not defined, the default value 1.0 is used.
      }
   }
   \sstexamples{
      \sstexamplesubsection{
         PONGO$>$ ARC 1 1 0 0
      }{
         will draw a circle of radius 1 (world coordinates), assuming the
         PASTART and PAEND parameters have their default values (0.0
         and 360.0 degrees respectively).
      }
   }
}
\sstroutine{
   AVEDAT
}{
   Average the data in the XCOL and YCOL areas
}{
   \sstdescription{
      Rebin the XCOL and YCOL data, averaging the data in each sample,
      and puts the result back into the XCOL and YCOL areas. The
      standard deviations of the averages are put into the EXCOL and
      EYCOL areas.  There are two ways in which the averaging may be
      done:

      \sstitemlist{

         \sstitem
           the data may be split into N equally sized bins over the X
             range, and the values in each bin averaged;

         \sstitem
           the data may be averaged in groups of N data.
      }
   }
   \sstparameters{
      \sstsubsection{
         ACTION = \_CHAR (Read)
      }{
         The type of binning be used for the averaging. If {\tt "}X{\tt "}, the
         data are divided into NBIN bins over the X range.  If {\tt "}N{\tt "},
         bins of varying widths with each containing NBIN data points
         are formed.

         [The value will be prompted for. It has the default {\tt "}X{\tt "}.]
      }
      \sstsubsection{
         NBIN = \_INTEGER (Read)
      }{
         Depending upon the value of ACTION, either the number of bins
         (ACTION={\tt "}X{\tt "}), or the number of points per bin (ACTION={\tt "}N{\tt "}).

         [The value will be prompted for. It has the default 10.]
      }
      \sstsubsection{
         XMIN = \_REAL (Read)
      }{
         The minimum X value to be used in the average.

         [The value of the global parameter PONGO\_XMIN is used. If
         PONGO\_XMIN is not defined, the default value 0.0 is used.]
      }
      \sstsubsection{
         XMAX = \_REAL (Read)
      }{
         The maximum X value to be used in the average.

         [The value of the global parameter PONGO\_XMAX is used. If
         PONGO\_XMAX is not defined, the default value 1.0 is used.]
      }
   }
}
\sstroutine{
   BEGPLOT
}{
   Open a plotting device
}{
   \sstdescription{
      Set up a device for subsequent PONGO plotting commands. This
      application can be used with parameters which allow plotting onto
      an AGI picture created by a different package (e.g. KAPPA).

      Successful execution will display information about the current
      picture from the AGI database. In OVERLAY mode, BEGPLOT will give
      information about the package that created the picture; in BASE
      mode, BEGPLOT will state that a new base picture has been
      created.

      This command is an ICL hidden procedure which uses the
      undocumented PONGO application BEGPONGO.
   }
   \sstparameters{
      \sstsubsection{
         DEVICE = DEVICE (Read and Write)
      }{
         The name of the device to be used for plotting.  The names of
         the currently available devices can be found using the INQUIRE
         DEVICE command.

         The value of the global parameter GRAPHICS\_DEVICE is used
         unless a value is specified on the command line. If
         GRAPHICS\_DEVICE is not defined and no value is specified on
         the command line, the value will be prompted for.
      }
      \sstsubsection{
         BASE = \_LOGICAL (Read and Write)
      }{
         If TRUE, the plotting device will be cleared and the whole of
         its plotting surface used. If FALSE, the current picture will
         be used and a PGPLOT viewport created inside it. If the value
         of OVERLAY is TRUE, the value of BASE is ignored.  Once set,
         this parameter will retain its value in subsequent invocations
         of BEGPLOT.
         [FALSE]
      }
      \sstsubsection{
         OVERLAY = \_LOGICAL (Read)
      }{
         If TRUE, the PGPLOT viewport created will exactly overlay
         the current DATA picture. This is useful for drawing axis
         labels using BOXFRAME on an image that has been displayed by
         another package (e.g. KAPPA DISPLAY).
         [FALSE]
      }
      \sstsubsection{
         XMIN = \_REAL (Write)
      }{
         The left hand edge of the world coordinate limits. Its value
         is set by the application from the values of the BASE and
         OVERLAY parameters. The result is written to the global
         parameter PONGO\_XMIN.
      }
      \sstsubsection{
         XMAX = \_REAL (Write)
      }{
         The right hand edge of the world coordinate limits.  Its
         value is set by the application from the values of the BASE
         and OVERLAY parameters. The result is written to the global
         parameter PONGO\_XMAX.
      }
      \sstsubsection{
         YMIN = \_REAL (Write)
      }{
         The lower edge of the world coordinate limits. Its value is
         set by the application from the values of the BASE and OVERLAY
         parameters. The result is written to the global parameter
         PONGO\_YMIN.
      }
      \sstsubsection{
         YMAX = \_REAL (Write)
      }{
         The upper edge of the world coordinate limits. Its value is
         set by the application from the values of the BASE and OVERLAY
         parameters. The result is written to the global parameter
         PONGO\_YMAX.
      }
      \sstsubsection{
         CHEIGHT = \_REAL (Write)
      }{
         The character height scaling factor. A value of 1.0 implies a
         nominal character height of 1/40th the viewport height. The
         value is set by the application from the value of the BASE
         parameter. The result is written to the global parameter
         PONGO\_CHEIGHT.
      }
   }
}
\sstroutine{
   BIN
}{
   Plot a histogram using previously binned data
}{
   \sstdescription{
      The data in the XCOL and YCOL data areas are used to plot a
      histogram. The data in the XCOL area specify the bin edges, while
      the data in the YCOL area specify the respective frequency for
      each bin.

      This command is a synonym for PLOTHIST B.
   }
   \sstparameters{
      \sstsubsection{
         AUTOSCALE = \_LOGICAL (Read and Write)
      }{
         If TRUE, PGPLOT auto-scaling is used to determine the plotting
         limits. If FALSE, the limits defined by the bins of the
         histogram determine the plotting limits. Here, the plotting
         limits must previously have been set using the LIMITS
         application and the plot frame drawn using BOXFRAME. Setting
         NOAUTOSCALE can be used to draw more than one histogram on the
         same plot.

         If no value is specified on the command line, the current value
         is used. The current value is initially set to TRUE.
      }
      \sstsubsection{
         CENTRE = \_LOGICAL (Read)
      }{
         This parameter specifies whether the values in the XCOL data
         area denote the centre of each bin (when TRUE) or its lower
         edge (when FALSE).
         [FALSE]
      }
   }
}
\sstroutine{
   BOXFRAME
}{
   Draw a frame and axes on the plot
}{
   \sstdescription{
      Draw a frame, axes and tick-marks on a plot. The application has
      great flexibility in the type of axis labelling that can be
      produced -- it is essentially an interface to the PGPLOT routine
      PGBOX.
   }
   \sstparameters{
      \sstsubsection{
         XOPT = \_CHAR (Read and Write)
      }{
         A string that controls the style of the X-axis labelling and
         tick-marks. It consists of a series of letters, which have the
         meanings shown below (reproduced from the PGPLOT manual):

         \sstitemlist{

            \sstitem
               {\tt "}A{\tt "} -- Draw an axis (the X axis is the horizontal line
               Y=0, the Y axis is the vertical line X=0).

            \sstitem
               {\tt "}B{\tt "} -- Draw the bottom (X) or left (Y) edge of the frame.

            \sstitem
               {\tt "}C{\tt "} -- Draw the top (X) or right (Y) edge of the frame.

            \sstitem
               {\tt "}G{\tt "} -- Draw a grid of vertical (X) or horizontal (Y)
               lines.

            \sstitem
               {\tt "}I{\tt "} -- Invert the tick-marks (i.e. draw them outside the
               viewport instead of inside).

            \sstitem
               {\tt "}L{\tt "} -- Label the axis logarithmically (see below).

            \sstitem
               {\tt "}N{\tt "} -- Write numeric labels in the conventional location
               below the viewport (X) or to the left of the viewport (Y).

            \sstitem
               {\tt "}P{\tt "} -- Extend (project) major tick-marks outside the box
               (ignored if option I is specified).

            \sstitem
               {\tt "}M{\tt "} -- Write numeric labels in the unconventional location
               above the viewport (X) or to the right of the viewport (Y).

            \sstitem
               {\tt "}T{\tt "} -- Draw major tick-marks at the major coordinate
               interval.

            \sstitem
               {\tt "}S{\tt "} -- Draw minor tick-marks (sub-ticks).

            \sstitem
               {\tt "}V{\tt "} -- Orient numeric labels vertically (this is only
               applicable to Y -- the default is to write Y-axis labels
               parallel to the axis).

         }
         [The global parameter PONGO\_XOPT is used. If PONGO\_XOPT is not
         defined, the default value {\tt "}BCNST{\tt "} is used.]
      }
      \sstsubsection{
         YOPT = \_CHAR (Read and Write)
      }{
         A string that controls the style of the Y-axis labelling and
         tick-marks. It consists of a series of letters, as given for
         the parameter XOPT.

         [The global parameter PONGO\_YOPT is used.  If PONGO\_YOPT is not
         defined, the default value {\tt "}BCNST{\tt "} is used.]
      }
      \sstsubsection{
         XTIC = \_REAL (Read and Write)
      }{
         The major tick-mark interval on the X-axis. If XTIC is set to
         0.0, PGPLOT makes a sensible choice.

         If the value is not specified on the command line, the current
         value is used. The current value is initially set to 0.0.
      }
      \sstsubsection{
         YTIC = \_REAL (Read and Write)
      }{
         The major tick-mark interval on the Y-axis. If YTIC is set to
         0.0, PGPLOT makes a sensible choice.

         If the value is not specified on the command line, the current
         value is used. The current value is initially set to 0.0.
      }
      \sstsubsection{
         NXSUB = \_INTEGER (Read and Write)
      }{
         The number of minor tick-marks between each major tick-mark on
         the X-axis. If NXSUB is set to 0, PGPLOT makes a sensible
         choice.

         If the value is not specified on the command line, the current
         value is used. The current value is initially set to 0.
      }
      \sstsubsection{
         NYSUB = \_INTEGER (Read and Write)
      }{
         The number of minor tick-marks between each major tick-mark on
         the Y-axis. If NYSUB is set to 0, PGPLOT makes a sensible
         choice.

         If the value is not specified on the command line, the current
         value is used. The current value is initially set to 0.
      }
   }
}
\sstroutine{
   CCMATH
}{
   Perform inter-column maths
}{
   \sstdescription{
      Perform inter-column maths (using TRANSFORM, SUN/61). The
      expressions and functions recognised have Fortran types and
      syntax.  Any construct that is legal in TRANSFORM is legal in
      this subroutine, with the additional function INDEX for filling
      the array with an increasing sequence of integers. See SUN/61 for
      further details.

      The names used for the data areas are as follows:

      \sstitemlist{

         \sstitem
            {\tt "}X{\tt "} -- the XCOL data area,

         \sstitem
            {\tt "}Y{\tt "} -- the YCOL data area,

         \sstitem
            {\tt "}Z{\tt "} -- the ZCOL data area,

         \sstitem
            {\tt "}EX{\tt "} -- the EXCOL error area,

         \sstitem
            {\tt "}EY{\tt "} -- the EYCOL error area.
      }
   }
   \sstparameters{
      \sstsubsection{
         X = \_CHAR (Read)
      }{
         The transformation to perform on the contents of the XCOL data
         area.

         [{\tt "}X -- i.e. will cause the contents of the data area to remain
         unchanged.]
      }
      \sstsubsection{
         Y = \_CHAR (Read)
      }{
         The transformation to perform on contents of the YCOL data
         area.

         [{\tt "}Y{\tt "} -- i.e. will cause the contents of the data area to remain
         unchanged.]
      }
      \sstsubsection{
         Z = \_CHAR (Read)
      }{
         The transformation to perform on contents of the ZCOL data
         area.

         [{\tt "}Z{\tt "} -- i.e. will cause the contents of the data area to remain
         unchanged.]
      }
      \sstsubsection{
         EX = \_CHAR (Read)
      }{
         The transformation to perform on contents of the EXCOL data
         area.

         [{\tt "}EX{\tt "} -- i.e. will cause the contents of the data area to
         remain unchanged.]
      }
      \sstsubsection{
         EY = \_CHAR (Read)
      }{
         The transformation to perform on contents of the EYCOL data
         area.

         [{\tt "}EY{\tt "} -- i.e. will cause the contents of the data area to
         remain unchanged.]
      }
   }
   \sstexamples{
      \sstexamplesubsection{
         PONGO$>$ CCMATH X=2$*$Y
      }{
         will fill each element of the XCOL data area with twice the
         corresponding element of the YCOL data area.
      }
   }
   \sstnotes{
      \sstitemlist{

         \sstitem
         More than one array may be manipulated with a single command.

         \sstitem
         The INDEX function cannot be combined with any other function.
      }
   }
}
\sstroutine{
   CHANGE
}{
   Change plotting attributes
}{
   \sstdescription{
      Change the PGPLOT plotting attributes: e.g. line style, pen
      colour etc. Several of the attributes can be changed at the same
      time. Each of the parameters is remembered from the last
      invocation of CHANGE: after BEGPLOT has been run, a single
      invocation of CHANGE can be used to reset the plotting attributes
      to their values the last time PONGO was used.
   }
   \sstparameters{
      \sstsubsection{
         COLOUR = \_INTEGER (Read and Write)
      }{
         The pen number (colour index) PGPLOT uses for plotting. The
         value should be between 1 and 255 (note that pen zero is
         reserved for the background colour and should not be changed).
         Usually the first 16 pens are predefined to have the following
         colours:

         \sstitemlist{

            \sstitem
               0 -- background,

            \sstitem
               1 -- foreground (default),

            \sstitem
               2 -- red,

            \sstitem
               3 -- green,

            \sstitem
               4 -- blue,

            \sstitem
               5 -- cyan,

            \sstitem
               6 -- magenta,

            \sstitem
               7 -- yellow,

            \sstitem
               8 -- red $+$ yellow (orange),

            \sstitem
               9 -- green $+$ yellow,

            \sstitem
               10 -- green $+$ cyan,

            \sstitem
               11 -- blue $+$ cyan,

            \sstitem
               12 -- blue $+$ magenta,

            \sstitem
               13 -- red $+$ magenta,

            \sstitem
               14 -- dark grey,

            \sstitem
               15 -- light grey.

         }
         It is possible to change the colour representation of any of
         the pen colour indices using the PALETTE application.

         If the value is not specified on the command line, the current
         value is used. The current value is initially set to 1 (i.e.
         foreground).
      }
      \sstsubsection{
         CHEIGHT = \_REAL (Read and Write)
      }{
         The character height scaling. This parameter scales the
         default character height and also alters the size of the tick
         marks and symbols that PGPLOT plots. The default character
         height in PGPLOT is about 1/40 of the viewport height.

         [The value of the global parameter PONGO\_CHEIGHT is used. If
         PONGO\_CHEIGHT is not defined, the default value 1.0 is used.]
      }
      \sstsubsection{
         FONT = \_INTEGER (Read and Write)
      }{
         The font used by PGPLOT. The styles for each font are as
         follows:

         \sstitemlist{

            \sstitem
               1 -- single-stroke font (default),

            \sstitem
               2 -- roman font,

            \sstitem
               3 -- italic font,

            \sstitem
               4 -- script font.

         }
         If the value is not specified on the command line, the current
         value is used. The current value is initially set to 1 (i.e.
         single-stroke font).
      }
      \sstsubsection{
         FILLSTY = \_INTEGER (Read and Write)
      }{
         The fill style used by PGPLOT. The fill styles are as follows:

         \sstitemlist{

            \sstitem
               1 -- solid fill,

            \sstitem
               2 -- hollow fill.

         }
         If the value is not specified on the command line, the current
         value is used. The current value is initially set to 1 (i.e.
         solid fill).
      }
      \sstsubsection{
         LINESTY = \_INTEGER (Read and Write)
      }{
         The line style used by PGPLOT.  The line style may be one of
         the following:

         \sstitemlist{

            \sstitem
               1 -- full line (default),

            \sstitem
               2 -- dashed,

            \sstitem
               3 -- dot-dash-dot-dash,

            \sstitem
               4 -- dotted,

            \sstitem
               5 -- dash-dot-dot-dot.

         }
         If the value is not specified on the command line, the current
         value is used. The current value is initially set to 1 (i.e.
         full line).
      }
      \sstsubsection{
         LINEWID = \_INTEGER (Read and Write)
      }{
         The line width scaling. This parameter scales the default line
         width.

         If the value is not specified on the command line, the current
         value is used. The current value is initially set to 1.
      }
   }
   \sstexamples{
      \sstexamplesubsection{
         PONGO$>$ CHANGE RESET
      }{
         will reset the plotting attributes to their default values.
      }
   }
}
\sstroutine{
   CLEAR
}{
   Clear plotting attributes
}{
   \sstdescription{
      Clear or reset various PONGO plotting attributes. Several or all
      of these plotting attributes can be specified on the command line.
   }
   \sstparameters{
      \sstsubsection{
         SCREEN = \_LOGICAL (Read)
      }{
         If TRUE, the plotting surface will be cleared.
         [FALSE]
      }
      \sstsubsection{
         DATA = \_LOGICAL (Read)
      }{
         If TRUE, the data arrays will be cleared. This can be useful
         if there is still unwanted data left in some of the data
         areas: for example, the READF command does not automatically
         clear columns that it does not read in. Even if the
         appropriate READF parameters have been set to 0 to inhibit
         reading error columns from the second file, the values from
         the first file will remain in the data areas. This behaviour is
         designed to allow columns to be read from separate files and
         combined in one plot. However, it can cause the automatic axis
         limit finding routines to appear to fail because of data left
         in error columns. This most commonly occurs when a file not
         containing error values is read immediately after one that
         does.
         [FALSE]
      }
      \sstsubsection{
         LIMITS = \_LOGICAL (Read)
      }{
         If TRUE, the data limits will be cleared and the data
         re-examined in order to determine the axis limits. This
         command is often necessary if a complicated function has been
         performed on the data using the CCMATH command, because the
         current PONGO parameters may no longer be applicable to the
         data.
         [FALSE]
      }
      \sstsubsection{
         LABLST = \_LOGICAL (Read)
      }{
         If TRUE, the internal label list generated by the CURSE
         command will be cleared.
         [FALSE]
      }
      \sstsubsection{
         AGI = \_LOGICAL (Read)
      }{
         If TRUE, the AGI database is cleared for the current device.
         [FALSE]
      }
      \sstsubsection{
         ERSCALE = \_REAL (Read)
      }{
         The scale factor by which the errors are to be multiplied.

         The value of the global parameter PONGO\_ERSCALE is used. If
         PONGO\_ERSCALE is not defined, the default value 1.0 is used.
      }
   }
}
\sstroutine{
   CLOG
}{
   Take the logarithm of a column
}{
   \sstdescription{
      Take the base 10 logarithm of a data column.  This application
      should be used to take the logarithm of the data columns in
      preference to doing it with CCMATH, because it can deal with the
      associated error values consistently. It automatically adds the
      {\tt "}L{\tt "} option to the PONGO\_XOPT or PONGO\_YOPT global parameters, as
      appropriate.
   }
   \sstparameters{
      \sstsubsection{
         ACTION = \_CHAR (Read)
      }{
         The data column to transform. It should be one of the
         following:

         \sstitemlist{

            \sstitem
               {\tt "}X{\tt "} -- XCOL

            \sstitem
               {\tt "}Y{\tt "} -- YCOL

            \sstitem
               {\tt "}Z{\tt "} -- ZCOL

         }
         [The value will be prompted for.]
      }
      \sstsubsection{
         XOPT = \_CHAR (Write)
      }{
         The PGPLOT X-axis options string.  The global parameter will be
         updated to include the PGPLOT {\tt "}L{\tt "} axis option at the start.
         This option means that logarithmic style axis labels and tick
         marks will be plotted. The READF command will automatically
         remove any {\tt "}L{\tt "} characters at the start of this string since it
         assumes that they have been put there by CLOG, and fresh data
         read in will not be logarithmic. If data are naturally
         logarithmic, the {\tt "}L{\tt "} should be placed other than at the start
         of the string to make an {\tt "}L{\tt "} option that will not be modified
         by PONGO.

         It is not intended that this parameter be set by the user when
         CLOG is executed.

         The value will be written to the global parameter PONGO\_XOPT.
      }
      \sstsubsection{
         YOPT = \_CHAR (Write)
      }{
         The PGPLOT Y-axis options string. Its action is similar to the
         XOPT parameter.
         It is not intended that this parameter be set by the user when
         CLOG is executed.

         The value will be written to the global parameter PONGO\_YOPT.
      }
   }
}
\sstroutine{
   CONNECT
}{
   Draw straight lines between the data points
}{
   \sstdescription{
      Straight line segments are drawn between the data points in the
      XCOL and YCOL data areas.

      This command is a synonym for GPOINTS C.
   }
   \sstparameters{
      \sstsubsection{
         PROJECTION = \_CHAR (Read)
      }{
         Specifies the geometry that is to be used to plot the data.
         This is explained in more detail in the section on
         projections.  Allowed values: {\tt "}NONE{\tt "}, {\tt "}TAN{\tt "}, {\tt "}SIN{\tt "}, {\tt "}ARC{\tt "},
         {\tt "}GLS{\tt "}, {\tt "}AITOFF{\tt "}, {\tt "}MERCATOR{\tt "} and {\tt "}STG{\tt "}.

         This parameter is not specified on the command line. The value
         of the global parameter PONGO\_PROJECTN is used. If
         PONGO\_PROJECTN is not defined, the default value {\tt "}NONE{\tt "} is
         used.
      }
      \sstsubsection{
         RACENTRE = \_CHAR (Read)
      }{
         The centre of the projection in RA (i.e. the angle must be
         specified as hh:mm:ss.sss). This parameter is only required for
         PROJECTION values other than {\tt "}NONE{\tt "}.

         This parameter is not specified on the command line. The value
         of the global parameter PONGO\_RACENTRE is used. If
         PONGO\_RACENTRE is not defined, the default value {\tt "}0{\tt "} is used.
      }
      \sstsubsection{
         DECCENTRE = \_CHAR (Read)
      }{
         The centre of the projection in declination (i.e. the angle
         must be specified as dd:mm:ss.sss). This parameter is only
         required for PROJECTION values other than {\tt "}NONE{\tt "}.

         This parameter is not specified on the command line. The value
         of the global parameter PONGO\_DECCENTRE is used. If
         PONGO\_DECCENTRE is not defined, the default value {\tt "}0{\tt "} is used.
      }
   }
}
\sstroutine{
   CURSE
}{
   Use the cursor to get interactive input
}{
   \sstdescription{
      Display the cursor and perform various interactive tasks. These
      tasks are as follows:

      \sstitemlist{

         \sstitem
            Q -- QUIT 
            The application ends.

         \sstitem
            D -- DRAW
            A line is drawn from the PGPLOT {\tt "}current position{\tt "} to  the
            cursor position. This position is also written to the
            internal label list.

         \sstitem
            E -- EXPAND
            The plotting limits are expanded by a factor of 2
            about the position of the cursor. No other action is taken.
            This allows the screen to be cleared and the graph to be
            re-plotted without having to set the limits explicitly.

         \sstitem
            G -- GRADIENT
            Draw a line between two consecutive cursor hits and report
            the gradient of the line.

         \sstitem
            L -- LABEL
            Write a label directly onto the plot. The application
            uses up to two points which define the angle at which the
            label is to be drawn. Once {\tt "}L{\tt "} has been pressed, the user
            is given the option to add another point in the standard
            PGPLOT fashion (c.f. the PGPLOT routine PGNCURSE); i.e.

         \sstitemlist{

            \sstitem
                  A -- add a point

            \sstitem
                  D -- delete a point

            \sstitem
                  X -- finish
         }
            On pressing the {\tt "}X{\tt "} key, a label is prompted for. If only
            one point is supplied, the label is plotted horizontally.

         \sstitem
            M -- MARK
            Mark the current cursor position with the current symbol
            type.

         \sstitem
            O -- ANNOTATE
            The label for the nearest data point is written with its
            right hand end at the position defined by the cursor.  The
            information to create this label is stored in an internal
            table. This information can be written out into file
            suitable for including in an ICL command file by using the
            WRITEI application, e.g.

               PONGO$>$ WRITEI LABLST

         \sstitem
            P -- ANNOTATE
            The label for the nearest data point is written with its
            left hand end at the position defined by the cursor.  The
            information to create this label is stored in an internal
            table. This information can be written out into file
            suitable for including in an ICL command file by using the
            WRITEI application, e.g.

               PONGO$>$ WRITEI LABLST

         \sstitem
            S -- SHRINK
            The plotting limits are set so as to zoom out by a factor
            of 2 about the position of the cursor. No other action is
            taken. This allows the screen to be cleared and the graph
            to be re-plotted without having to set the limits
            explicitly.

         \sstitem
            V -- MOVE
            The PGPLOT {\tt "}current position{\tt "} will be set to the cursor
            position. This position is also written to the internal
            label list.

         \sstitem
            X -- POSITION
            The current position of the cursor in world coordinates is
            written to the terminal and the XCURSE and YCURSE
            parameters.

         \sstitem
            Z -- ZOOM
            The limits for a zoomed version of the current plot are
            set. The application will prompt for two points which
            define the bottom left corner and the top right corner of
            the new graph surface.

         \sstitemlist{

            \sstitem
                  A -- add a point

            \sstitem
                  D -- delete a point

            \sstitem
                  X -- finish
         }
      }
   }
   \sstparameters{
      \sstsubsection{
         SYMBOL = \_INTEGER (Read)
      }{
         The symbol number used in the MARK option.

         If the value is not specified on the command line, the current
         value is used. The current value is initially set to 1.
      }
      \sstsubsection{
         LABEL = \_CHAR (Read)
      }{
         The label to be written to the screen with the LABEL option.

         [The value is prompted for.]
      }
      \sstsubsection{
         JUSTIFICATION = \_REAL (Read and Write)
      }{
         The justification about the point (in the range 0.0 to 1.0).
         Here, 0.0 means left justify the text relative to the data
         point, 1.0 means right justify the text relative to the data
         point, 0.5 means centre the string on the data point, other
         values will give intermediate justifications.

         If the value is not specified on the command line, the current
         value is used. The current value is initially set to 0.5 (i.e.
         centre the text).
      }
      \sstsubsection{
         XCURSE = \_REAL (Write)
      }{
         The X-axis position of the last graphics cursor hit when using
         the {\tt "}X{\tt "} option.

         The value is written to the PONGO\_XCURSE global parameter.
      }
      \sstsubsection{
         YCURSE = \_REAL (Write)
      }{
         The Y-axis position of the last graphics cursor hit when using
         the {\tt "}X{\tt "} option.

         The value is written to the PONGO\_YCURSE global parameter.
      }
      \sstsubsection{
         XMIN = \_REAL (Write)
      }{
         The left hand edge of the world coordinate limits. The value
         is set by the application in the zooming options. It is not
         intended that the value be specified on the command line.

         The value is written to the global parameter PONGO\_XMIN.
      }
      \sstsubsection{
         XMAX = \_REAL (Write)
      }{
         The right hand edge of the world coordinate limits. The value
         is set by the application in the zooming options. It is not
         intended that the value be specified on the command line.

         The value is written to the global parameter PONGO\_XMAX.
      }
      \sstsubsection{
         YMIN = \_REAL (Write)
      }{
         The lower edge of the world coordinate limits. The value is
         set by the application in the zooming options. It is not
         intended that the value be specified on the command line.

         The value is written to the global parameter PONGO\_YMIN.
      }
      \sstsubsection{
         YMAX = \_REAL (Write)
      }{
         The upper edge of the world coordinate limits. The value is
         set by the application in the zooming options. It is not
         intended that the value be specified on the command line.

         The value is written to the global parameter PONGO\_YMAX.
      }
      \sstsubsection{
         PROJECTION = \_CHAR (Read)
      }{
         The projection that has been used to plot the data.  This is
         explained in more detail in the section on projections.
         Allowed values: {\tt "}NONE{\tt "}, {\tt "}TAN{\tt "}, {\tt "}SIN{\tt "}, {\tt "}ARC{\tt "}, {\tt "}GLS{\tt "}, {\tt "}AITOFF{\tt "},
         {\tt "}MERCATOR{\tt "} and {\tt "}STG{\tt "}.

         This parameter is not specified on the command line. The value
         of the global parameter PONGO\_PROJECTN is used. If
         PONGO\_PROJECTN is not defined, the default value {\tt "}NONE{\tt "} is
         used.
      }
      \sstsubsection{
         RACENTRE = \_CHAR (Read)
      }{
         The centre of the projection in RA (i.e. the angle must be
         specified as hh:mm:ss.sss). This parameter is only required for
         PROJECTION values other than {\tt "}NONE{\tt "}.

         This parameter is not specified on the command line. The value
         of the global parameter PONGO\_RACENTRE is used. If
         PONGO\_RACENTRE is not defined, the default value {\tt "}0{\tt "} is used.
      }
      \sstsubsection{
         DECCENTRE = \_CHAR (Read)
      }{
         The centre of the projection in declination (i.e. the angle
         must be specified as dd:mm:ss.sss). This parameter is only
         required for PROJECTION values other than {\tt "}NONE{\tt "}.

         This parameter is not specified on the command line. The value
         of the global parameter PONGO\_DECCENTRE is used. If
         PONGO\_DECCENTRE is not defined, the default value {\tt "}0{\tt "} is used.
      }
   }
}
\sstroutine{
   DATA
}{
   Specify the data file name
}{
   \sstdescription{
      The file name for the input data is set.

      This command is a synonym for SETGLOBAL PONGO\_DATA.
   }
   \sstparameters{
      \sstsubsection{
         DATA = FILENAME (Write)
      }{
         The name of the formatted data file. []
      }
   }
}
\sstroutine{
   DEGTOR
}{
   Convert the specified data area from degrees to radians
}{
   \sstdescription{
      The values in the specified data area are converted from degrees
      to radians.

      This command is an ICL hidden procedure which uses the CCMATH
      application.
   }
   \sstparameters{
      \sstsubsection{
         COLUMN = \_CHAR (Read and Write)
      }{
         The column to be converted from degrees to radians. []
      }
   }
}
\sstroutine{
   DEVICE
}{
   Open a plotting device
}{
   \sstdescription{
      Set up a device for subsequent PONGO plotting commands. This
      application can be used with parameters which allow plotting onto
      an AGI picture created by a different package (e.g. KAPPA).

      Successful execution will display information about the current
      picture from the AGI database. In OVERLAY mode, DEVICE will give
      information about the package that created the picture; in BASE
      mode, DEVICE will state that a new base picture has been
      created.

      This command is a synonym for BEGPLOT.
   }
   \sstparameters{
      \sstsubsection{
         DEVICE = DEVICE (Read and Write)
      }{
         The name of the device to be used for plotting.  The names of
         the currently available devices can be found using the INQUIRE
         DEVICE command.

         The value of the global parameter GRAPHICS\_DEVICE is used
         unless a value is specified on the command line. If
         GRAPHICS\_DEVICE is not defined and no value is specified on
         the command line, the value will be prompted for.
      }
      \sstsubsection{
         BASE = \_LOGICAL (Read and Write)
      }{
         If TRUE, the plotting device will be cleared and the whole of
         its plotting surface used. If FALSE, the current picture will
         be used and a PGPLOT viewport created inside it. If the value
         of OVERLAY is TRUE, the value of BASE is ignored.  Once set,
         this parameter will retain its value in subsequent invocations
         of DEVICE.
         [FALSE]
      }
      \sstsubsection{
         OVERLAY = \_LOGICAL (Read)
      }{
         If TRUE, the PGPLOT viewport created will exactly overlay
         the current DATA picture. This is useful for drawing axis
         labels using BOXFRAME on an image that has been displayed by
         another package (e.g. KAPPA DISPLAY).
         [FALSE]
      }
      \sstsubsection{
         XMIN = \_REAL (Write)
      }{
         The left hand edge of the world coordinate limits. Its value
         is set by the application from the values of the BASE and
         OVERLAY parameters. The result is written to the global
         parameter PONGO\_XMIN.
      }
      \sstsubsection{
         XMAX = \_REAL (Write)
      }{
         The right hand edge of the world coordinate limits.  Its
         value is set by the application from the values of the BASE
         and OVERLAY parameters. The result is written to the global
         parameter PONGO\_XMAX.
      }
      \sstsubsection{
         YMIN = \_REAL (Write)
      }{
         The lower edge of the world coordinate limits. Its value is
         set by the application from the values of the BASE and OVERLAY
         parameters. The result is written to the global parameter
         PONGO\_YMIN.
      }
      \sstsubsection{
         YMAX = \_REAL (Write)
      }{
         The upper edge of the world coordinate limits. Its value is
         set by the application from the values of the BASE and OVERLAY
         parameters. The result is written to the global parameter
         PONGO\_YMAX.
      }
      \sstsubsection{
         CHEIGHT = \_REAL (Write)
      }{
         The character height scaling factor. A value of 1.0 implies a
         nominal character height of 1/40th the viewport height. The
         value is set by the application from the value of the BASE
         parameter. The result is written to the global parameter
         PONGO\_CHEIGHT.
      }
   }
}
\sstroutine{
   DLIMITS
}{
   Set the world coordinate limits using the data range
}{
   \sstdescription{
      The ranges of the data given in the XCOL and YCOL data areas are
      used to set the world coordinate limits for plotting. A small
      border is also added to the data limits when calculating the world
      coordinate limits.

      This command is a synonym for WORLD DATA.
   }
   \sstparameters{
      \sstsubsection{
         XMIN = \_REAL (Read and Write)
      }{
         The world coordinate of the left-hand edge of the plot.

         The application will determine the value.
      }
      \sstsubsection{
         XMAX = \_REAL (Read and Write)
      }{
         The world coordinate of the right-hand edge of the plot.

         The application will determine the value.
      }
      \sstsubsection{
         YMIN = \_REAL (Read and Write)
      }{
         The world coordinate of the lower edge of the plot.

         The application will determine the value.
      }
      \sstsubsection{
         YMAX = \_REAL (Read and Write)
      }{
         The world coordinate of the upper edge of the plot.

         The application will determine the value.
      }
      \sstsubsection{
         PROJECTION = \_CHAR (Read and Write)
      }{
         The geometry to be used for plotting the data.  This is
         explained in more detail in the section on projections.
         Allowed values: {\tt "}NONE{\tt "}, {\tt "}TAN{\tt "}, {\tt "}SIN{\tt "}, {\tt "}ARC{\tt "}, {\tt "}GLS{\tt "}, {\tt "}AITOFF{\tt "},
         {\tt "}MERCATOR{\tt "} and {\tt "}STG{\tt "}.

         [The value of the global parameter PONGO\_PROJECTN is used. If
         PONGO\_PROJECTN is not defined, the default value {\tt "}NONE{\tt "} is
         used.]
      }
      \sstsubsection{
         RACENTRE = \_CHAR (Read and Write)
      }{
         The centre of the projection in RA (i.e. the angle must be
         specified as hh:mm:ss.sss). This parameter is only required for
         PROJECTION values other than {\tt "}NONE{\tt "}.

         [The value of the global parameter PONGO\_RACENTRE is used. If
         PONGO\_RACENTRE is not defined, the default value {\tt "}0{\tt "} is used.]
      }
      \sstsubsection{
         DECCENTRE = \_CHAR (Read and Write)
      }{
         The centre of the projection in declination (i.e. the angle
         must be specified as dd:mm:ss.sss). This parameter is only
         required for PROJECTION values other than {\tt "}NONE{\tt "}.

         [The value of the global parameter PONGO\_DECCENTRE is used. If
         PONGO\_DECCENTRE is not defined, the default value {\tt "}0{\tt "} is used.]
      }
   }
}
\sstroutine{
   DRAW
}{
   Draw a line from the current pen position to the specified point
}{
   \sstdescription{
      A straight line is drawn from the current pen position to the
      specified point.

      This command is a synonym for PRIM D.
   }
   \sstparameters{
      \sstsubsection{
         X = \_REAL (Read and Write)
      }{
         The X coordinate of the point.

         If no value is specified on the command line, the value is
         prompted for.
      }
      \sstsubsection{
         Y = \_REAL (Read and Write)
      }{
         The Y coordinate of the point.

         If no value is specified on the command line, the value is
         prompted for.
      }
   }
}
\sstroutine{
   ELLIPSES
}{
   Draw error ellipses
}{
   \sstdescription{
      Draw error ellipses at each of the data points using values in
      the EXCOL and EYCOL error data areas, and the ZCOL data values as
      the normalized covariance. Depending upon the value of the AXES
      parameter, the major and minor axes of the ellipses will also be
      drawn.

      The size of the ellipses depends upon the parameter ERSCALE which
      should be set before the data are read in so that the WORLD
      application can calculate the viewing area properly. This allows
      the ellipse size to be varied so that it can be drawn for
      different confidence levels.

      \begin{center}
      \begin{tabular}{|c|c|} \hline
      & \\
      ERSCALE & Confidence \\
              & level \\ \hline
      & \\
      1.00 & 46\% \\
      2.30 & 68.3\% \\
      4.61 & 90\% \\
      9.21 & 99\% \\ \hline
      \end{tabular}
      \end{center}
   }
   \sstparameters{
      \sstsubsection{
         AXES = \_LOGICAL (Read and Write)
      }{
         If TRUE, the axes of the ellipses will be drawn.

         If the value is not specified on the command line, the current
         value is used.  The current value is initially set to FALSE.
      }
      \sstsubsection{
         ERSCALE = \_REAL (Read and Write)
      }{
         Scale the error ellipses as described above. If the command
         WORLD DATA (which automatically sets the graph limits) is to
         work properly, the global parameter PONGO\_ERSCALE should have
         been set by READF. However, if this facility is not required,
         it is perfectly acceptable to set ERSCALE when invoking
         ELLIPSES.

         [The value of the global parameter PONGO\_ERSCALE is used. If
         PONGO\_ERSCALE is not defined, the default value 1.0 is used.]
      }
   }
}
\sstroutine{
   ENDPLOT
}{
   Close down the current plotting device
}{
   \sstdescription{
      Close down the current plotting device, storing the current
      picture description in the AGI database. The position and world
      coordinate limits of the plot will be stored in the AGI database,
      along with any comment text.

      ENDPLOT must be executed before using a plotting application from
      another package (e.g. KAPPA) -- failure to do so will result in
      an AGI error and may corrupt the AGI database.

      This command is an ICL hidden procedure which uses the
      undocumented PONGO application ENDPONGO.
   }
   \sstparameters{
      \sstsubsection{
         COMMENT = \_CHAR (Read and Write)
      }{
         A comment for the AGI database entry for the plot that has
         just been completed. Any comment will be prefixed with the
         string {\tt "}PONGO:{\tt "}.
         [{\tt "}Final viewport{\tt "}]
      }
      \sstsubsection{
         DEVICE = DEVICE (Read)
      }{
         The name of the current plotting device.

         The value of the global parameter GRAPHICS\_DEVICE. If
         GRAPHICS\_DEVICE is not defined, the current value is used. If
         the current value is not defined, the value is prompted for.
      }
   }
}
\sstroutine{
   ERASE
}{
   Clear the graphics screen
}{
   \sstdescription{
      The plotting surface is cleared.

      This command is a synonym for CLEAR SCREEN.
   }
}
\sstroutine{
   ERRORBAR
}{
   Draw error bars on the plotted data
}{
   \sstdescription{
      Draw error bars in the X or Y directions, either treating the
      values in the EXCOL and EYCOL data areas as symmetric errors
      about the point, or as an upper limit with the XCOL or YCOL data
      area holding the other limit.

      PONGO will plot error bars correctly even after logarithms of the
      data have been taken for the symmetric option, as long as the CLOG
      application has been used to perform the transformation (as
      opposed to CCMATH). For the non-symmetric case, the CCMATH
      application should be used to take the logarithms of the data
      in the EXCOL and EYCOL data areas.
   }
   \sstparameters{
      \sstsubsection{
         ACTION = \_CHAR (Read)
      }{
         {\tt "}X{\tt "} or {\tt "}Y{\tt "} depending upon which set of error bars is to be
         drawn.

         [The value is prompted for.]
      }
      \sstsubsection{
         ERTERM = \_REAL (Read and Write)
      }{
         The length of the terminals on the error bars: a multiple of
         the default length.

         If the value is not specified on the command line, the current
         value is used. The current value is initially set to 1.0.
      }
      \sstsubsection{
         SYMERR = \_LOGICAL (Read and Write)
      }{
         If TRUE, the values in the error data areas represent a
         symmetric error about the values in the data columns. If
         FALSE, the data columns represent the lower limits, and the
         error columns represent the upper limits.
         [TRUE]
      }
   }
}
\sstroutine{
   ERRX
}{
   Draw symmetric error bars in the X direction
}{
   \sstdescription{
      Draw symmetric error bars in the X direction.

      This command is a synonym for ERRORBAR X.
   }
   \sstparameters{
      \sstsubsection{
         ERTERM = \_REAL (Read and Write)
      }{
         The length of the terminals on the error bars: a multiple of
         the default length.

         If no value is specified on the command line, the current
         value is used. The current value is initially set to 1.0.
      }
   }
}
\sstroutine{
   ERRY
}{
   Draw symmetric error bars in the Y direction
}{
   \sstdescription{
      Draw symmetric error bars in the Y direction.

      This command is a synonym for ERRORBAR Y.
   }
   \sstparameters{
      \sstsubsection{
         ERTERM = \_REAL (Read and Write)
      }{
         The length of the terminals on the error bars: a multiple of
         the default length.

         If no value is specified on the command line, the current
         value is used. The current value is initially set to 1.0.
      }
   }
}
\sstroutine{
   EXCOLUMN
}{
   Specify the column containing the X-axis error data
}{
   \sstdescription{
      Specify the column in the data file from which the errors on the
      X-axis data are to be read.

      This command is a synonym for SETGLOBAL PONGO\_EXCOL.
   }
   \sstparameters{
      \sstsubsection{
         EXCOL = \_CHAR (Write)
      }{
         The column number (counting from 1), or the symbolic name of a
         column, from which the X-axis error data are read by the READF
         command. The value {\tt "}0{\tt "} means {\tt "}do not read these data{\tt "}. []
      }
   }
}
\sstroutine{
   EXPAND
}{
   Set the character height
}{
   \sstdescription{
      Change the character height scaling. This command scales the
      default character height and also alters the size of the tick
      marks and symbols that PONGO plots. The default character height
      is about 1/40 of the viewport height.

      This command is a synonym for CHANGE CHEIGHT=.
   }
   \sstparameters{
      \sstsubsection{
         CHEIGHT = \_REAL (Read and Write)
      }{
         The character height scaling. This parameter scales the
         default character height and also alters the size of the tick
         marks and symbols that PGPLOT plots. The default character
         height in PGPLOT is about 1/40 of the viewport height. []
      }
   }
}
\sstroutine{
   EYCOLUMN
}{
   Specify the column containing the Y-axis error data
}{
   \sstdescription{
      Specify the column in the data file from which the errors on the
      Y-axis data are to be read.

      This command is a synonym for SETGLOBAL PONGO\_EYCOL.
   }
   \sstparameters{
      \sstsubsection{
         EYCOL = \_CHAR (Write)
      }{
         The column number (counting from 1), or the symbolic name of a
         column, from which the Y-axis error data are read by the READF
         command. The value {\tt "}0{\tt "} means {\tt "}do not read these data{\tt "}. []
      }
   }
}
\sstroutine{
   FITCURVE
}{
   Fit a curve to the data
}{
   \sstdescription{
      If the y-axis data are available and the WEIGHT parameter is TRUE,
      this application will perform a weighted least squares fit to a
      Chebyshev polynomial of specified order. If the y-axis error data
      are not available or WEIGHT is FALSE, an unweighted fit is
      performed. The best fit polynomial is plotted. The resultant fit
      parameters are displayed and written to the parameter CHEBCOEF.

      This application uses the NAG subroutines E02ADF and E02AEF.
   }
   \sstparameters{
      \sstsubsection{
         ACTION = \_CHAR (Read and Write)
      }{
         Type of curve to be fitted. Currently this action must be
         {\tt "}POLY{\tt "}.

         If the parameter is not specified on the command line, it will
         be prompted for.
      }
      \sstsubsection{
         NPOLY = \_INTEGER (Read and Write)
      }{
         The order of the polynomial.

         If the parameter is not specified on the command line, it
         will be prompted for.
      }
      \sstsubsection{
         COLOUR = \_INTEGER (Read and Write)
      }{
         The colour index used when plotting the fitted curve.

         If no value is specified on the command line, the current value
         is used. If there is no current value, a default of 2 (i.e.
         red) will be used.
      }
      \sstsubsection{
         WEIGHT = \_LOGICAL (Read and Write)
      }{
         Whether the fit is to use the y-axis error data in the EYCOL
         data area, if available. If no error data are available, the
         fit will always be unweighted.

         If the value is not specified on the command line, the current
         value is used. If there is no current value, a default value of
         TRUE is used.
      }
      \sstsubsection{
         CHEBCOEF[10] = \_DOUBLE (Write)
      }{
         If ACTION is {\tt "}POLY{\tt "}, the polynomial coefficients resulting
         from the fit are written to this parameter.

         The value of this parameter is written to the global parameter
         PONGO\_CHEBCOEF.
      }
      \sstsubsection{
         XMIN = \_REAL (Read)
      }{
         The minimum X value to be used in the fit.

         The value of the global parameter PONGO\_XMIN is used. If
         PONGO\_XMIN is not defined, the default value 0.0 is used.
      }
      \sstsubsection{
         XMAX = \_REAL (Read)
      }{
         The maximum X value to be used in the fit.

         The value of the global parameter PONGO\_XMAX is used. If
         PONGO\_XMAX is not defined, the default value 1.0 is used.
      }
      \sstsubsection{
         YMIN = \_REAL (Read)
      }{
         If ACTION is {\tt "}SPLINE{\tt "}, the minimum Y value to be used in the
         fit. If ACTION is {\tt "}POLY{\tt "}, this parameter has no effect.

         The value of the global parameter PONGO\_YMIN is used. If
         PONGO\_YMIN is not defined, the default value 0.0 is used.
      }
      \sstsubsection{
         YMAX = \_REAL (Read)
      }{
         If ACTION is {\tt "}SPLINE{\tt "}, the maximum Y value to be used in the
         fit. If ACTION is {\tt "}POLY{\tt "}, this parameter has no effect.

         The value of the global parameter PONGO\_YMAX is used. If
         PONGO\_YMAX is not defined, the default value 1.0 is used.
      }
   }
}
\sstroutine{
   FITLINE
}{
   Fit a straight line to the data
}{
   \sstdescription{
      If the y-axis error data are available and the WEIGHT parameter is
      TRUE, this application will perform a weighted least squares fit
      to a straight line for the data over the range delimited by the
      XMIN and XMAX parameters. If the y-axis error data are not
      available or WEIGHT is FALSE, an unweighted fit is performed. The
      best fit straight line is plotted. The resultant fit parameters
      are displayed along with some simple statistics for the data
      (these statistics are also weighted in the case of a weighted
      fit).
   }
   \sstparameters{
      \sstsubsection{
         COLOUR = \_INTEGER (Read and Write)
      }{
         The colour index used when plotting the fitted line.

         If the value is not specified on the command line, the current
         value is used. If there is no current value, a default value of
         2 (i.e. red) is used.
      }
      \sstsubsection{
         WEIGHT =\_LOGICAL (Read and Write)
      }{
         Whether the fit is to use the y-axis error data in the EYCOL
         data area, if available. If no error data are available, the
         fit will always be unweighted.

         If the value is not specified on the command line, the current
         value is used. If there is no current value, a default value of
         TRUE is used.
      }
      \sstsubsection{
         XMIN = \_REAL (Read)
      }{
         The minimum X value to be used in the fit.

         The value of the global parameter PONGO\_XMIN is used. If
         PONGO\_XMIN is not defined, the default value 0.0 is used.
      }
      \sstsubsection{
         XMAX = \_REAL (Read)
      }{
         The maximum X value to be used in the fit.

         The value of the global parameter PONGO\_XMAX is used. If
         PONGO\_XMAX is not defined, the default value 1.0 is used.
      }
   }
}
\sstroutine{
   FONT
}{
   Set the text font
}{
   \sstdescription{
      Change the text font.

      This command is a synonym for CHANGE FONT=.
   }
   \sstparameters{
      \sstsubsection{
         FONT = \_INTEGER (Read and Write)
      }{
         The font used by PGPLOT. The styles for each font are as
         follows:

         \sstitemlist{

            \sstitem
               1 -- single-stroke font,

            \sstitem
               2 -- roman font,

            \sstitem
               3 -- italic font,

            \sstitem
               4 -- script font.

         }
         []
      }
   }
}
\sstroutine{
   GETPOINT
}{
   Retrieve information for a plotted data point
}{
   \sstdescription{
      Return the attributes of a plotted data point to ICL variables.

      This application has been written to aid the implementation of
      ICL procedures. Because it is only possible to make enquiries
      about a single point per invocation, any ICL procedure built
      around GETPOINT will work slowly if a large number of data are
      involved. For such cases it may be better to consider writing a
      customized PONGO application.
   }
   \sstparameters{
      \sstsubsection{
         ACTION = \_CHAR (Read)
      }{
         The method of specifying the data point in question. If {\tt "}N{\tt "},
         interpret the VALUE parameter as specifying the index number
         of that point. If {\tt "}C{\tt "},  the VALUE parameter is used to try to
         match the LABCOL entry for a point.

         [The value is prompted for.]
      }
      \sstsubsection{
         VALUE = \_CHAR (Read)
      }{
         The value to be used in the search for the data point.
         Depending upon the value of the ACTION parameter, this may
         either be an integer specifying the index number of the point
         in the data area, or a case-sensitive minimum match string for
         a label column entry in the data area.

         [The value is prompted for.]
      }
      \sstsubsection{
         X = \_REAL (Write)
      }{
         The returned value of the X coordinate of the selected point.
      }
      \sstsubsection{
         Y = \_REAL (Write)
      }{
         The returned value of the Y coordinate of the selected point.
      }
      \sstsubsection{
         Z = \_REAL (Write)
      }{
         The returned value of the Z coordinate of the selected point.
      }
      \sstsubsection{
         EX = \_REAL (Write)
      }{
         The returned value of the X coordinate error of the selected
         point.
      }
      \sstsubsection{
         EY = \_REAL (Write)
      }{
         The returned value of the Y coordinate error of the selected
         point.
      }
      \sstsubsection{
         SYMBOL = \_INTEGER (Write)
      }{
         The returned value of the symbol column of the selected point.
      }
      \sstsubsection{
         LABEL = \_CHAR (Write)
      }{
         The returned value of the label column of the selected point.
      }
   }
   \sstexamples{
      \sstexamplesubsection{
         PONGO$>$ GETPOINT C {\tt '}3C45{\tt '} X=(XP) Y=(YP)
      }{
         This will return the X and Y coordinates of the data point
         that has the label {\tt '}3C45{\tt '}, if it exists, to the ICL variables
         XP and YP.
      }
   }
}
\sstroutine{
   GPOINTS
}{
   Plot points or lines between the data
}{
   \sstdescription{
      General plotting application. This application can be used simply
      to plot a symbol at the position of each point, to plot a symbol
      whose size depends upon the values in the ZCOL data area, or to
      draw lines connecting the data points.
   }
   \sstparameters{
      \sstsubsection{
         ACTION = \_CHAR (Read)
      }{
         The type of plot to produce. This can be {\tt "}C{\tt "}, {\tt "}P{\tt "} or {\tt "}S{\tt "},
         with the following meanings:

         \sstitemlist{

            \sstitem
            {\tt "}C{\tt "} (connect) -- This action simply draws straight line
            segments between the data points.

            \sstitem
            {\tt "}P{\tt "} (points) -- Draw a symbol at each of the data points. The
            symbol type that is used to mark each point is determined in
            one of 3 ways:

            \sstitemlist{

               \sstitem
            If no SYMBOL parameter is supplied on the command line,
            and no symbol numbers have been read into the symbol data
            area by READF, the point style will be set by the current
            value of SYMBOL.

               \sstitem
            If no SYMBOL parameter is supplied on the command line,
            and values have been read into the symbol data area by
            READF, the symbol number for each point will determine
            the style of the point plotted.

               \sstitem
            If SYMBOL is specified on the command line, it will
            override each of the above options. The same specified
            symbol will be used to mark all points.

            }
         The value of the symbol index should be an integer which
         refers to the standard PGPLOT symbols.

            \sstitem
            {\tt "}S{\tt "} (sizeplot) -- This action uses the values stored in the
            ZCOL data area to determine the size of the plotted symbol.
            The value of each entry in the ZCOL data area is effectively
            used as an argument to a CHANGE CHEIGHT command before each
            point is plotted. The SCALE parameter can be used to make
            these values cover a reasonable range by multiplying the Z
            data values.

         }
         [The value is prompted for.]
      }
      \sstsubsection{
         SYMBOL = \_INTEGER (Read and Write)
      }{
         The PGPLOT symbol number that is used to mark the data points.

         If a value is specified on the command line, it will be used
         for plotting symbols for all the data. If not value is
         specified on the command line, the application attempts to use
         the SYMCOL data for its symbols. If no symbol values have been
         read into the SYMCOL data area, the current value is used for
         all the data. The current value is initially set to 1.
      }
      \sstsubsection{
         SCALE = \_REAL (Read and Write)
      }{
         The scale factor used to multiply the ZCOL data values to get
         a reasonable range of symbol sizes when ACTION={\tt "}S{\tt "}.

         If no value is specified on the command line, the current
         value is used. The current value is initially set to 1.0.
      }
      \sstsubsection{
         PROJECTION = \_CHAR (Read)
      }{
         Specifies the geometry that is to be used to plot the data.
         This is explained in more detail in the section on
         projections.  Allowed values: {\tt "}NONE{\tt "}, {\tt "}TAN{\tt "}, {\tt "}SIN{\tt "}, {\tt "}ARC{\tt "},
         {\tt "}GLS{\tt "}, {\tt "}AITOFF{\tt "}, {\tt "}MERCATOR{\tt "} and {\tt "}STG{\tt "}.

         This parameter is not specified on the command line. The value
         of the global parameter PONGO\_PROJECTN is used. If
         PONGO\_PROJECTN is not defined, the default value {\tt "}NONE{\tt "} is
         used.
      }
      \sstsubsection{
         RACENTRE = \_CHAR (Read)
      }{
         The centre of the projection in RA (i.e. the angle must be
         specified as hh:mm:ss.sss). This parameter is only required for
         PROJECTION values other than {\tt "}NONE{\tt "}.

         This parameter is not specified on the command line. The value
         of the global parameter PONGO\_RACENTRE is used. If
         PONGO\_RACENTRE is not defined, the default value {\tt "}0{\tt "} is used.
      }
      \sstsubsection{
         DECCENTRE = \_CHAR (Read)
      }{
         The centre of the projection in declination (i.e. the angle
         must be specified as dd:mm:ss.sss). This parameter is only
         required for PROJECTION values other than {\tt "}NONE{\tt "}.

         This parameter is not specified on the command line. The value
         of the global parameter PONGO\_DECCENTRE is used. If
         PONGO\_DECCENTRE is not defined, the default value {\tt "}0{\tt "} is used.
      }
   }
}
\sstroutine{
   GRID
}{
   Draw a coordinate grid at specified intervals
}{
   \sstdescription{
      Draw a grid in the current projection at user specified intervals
      in spherical coordinates. The intervals, start and end values
      should all be specified in degrees. The defaults for the grid
      separations normally produce desirable effects for all sky plots.
      In specifying the grid intervals it is sometimes necessary to
      take account of rounding errors that might occur, and to bear in
      mind that in some geometries a single point on the celestial
      sphere maps onto two points on the projected coordinates -- some
      care is needed to ensure that the whole grid is drawn.
   }
   \sstparameters{
      \sstsubsection{
         PHIMIN = \_DOUBLE (Read and Write)
      }{
         The start longitude in degrees for the coordinate grid.

         If no value is specified on the command line, the current
         value is used. The current value is initially set to 0.0.
      }
      \sstsubsection{
         PHIMAX = \_DOUBLE (Read and Write)
      }{
         The end longitude in degrees for the coordinate grid.

         If no value is specified on the command line, the current
         value is used. The current value is initially set to 360.0.
      }
      \sstsubsection{
         PHISTEP = \_DOUBLE (Read and Write)
      }{
         The spacing between longitude grid lines in degrees.

         If no value is specified on the command line, the current
         value is used. The current value is initially set to 30.0.
      }
      \sstsubsection{
         THEMIN = \_DOUBLE (Read and Write)
      }{
         The start latitude in degrees for the coordinate grid.

         If no value is specified on the command line, the current
         value is used. The current value is initially set to -90.0.
      }
      \sstsubsection{
         THEMAX = \_DOUBLE (Read and Write)
      }{
         The end latitude in degrees for the coordinate grid.

         If no value is specified on the command line, the current
         value is used. The current value is initially set to 90.0.
      }
      \sstsubsection{
         THESTEP = \_DOUBLE (Read and Write)
      }{
         The spacing between latitude grid lines in degrees.

         If no value is specified on the command line, the current
         value is used. The current value is initially set to 10.0.
      }
      \sstsubsection{
         PROJECTION = \_LITERAL (Read)
      }{
         The geometry to be used to plot the grid.  This is explained
         in more detail in the section on projections.  Allowed values:
         {\tt "}NONE{\tt "}, {\tt "}TAN{\tt "}, {\tt "}SIN{\tt "}, {\tt "}ARC{\tt "}, {\tt "}GLS{\tt "}, {\tt "}AITOFF{\tt "}, {\tt "}MERCATOR{\tt "} and
         {\tt "}STG{\tt "}.

         This parameter is not specified on the command line. The value
         of the global parameter PONGO\_PROJECTN is used. If
         PONGO\_PROJECTN is not defined, the default value {\tt "}NONE{\tt "} is
         used.
      }
      \sstsubsection{
         RACENTRE = \_LITERAL (Read)
      }{
         The centre of the projection in RA (i.e. the angle must be
         specified as hh:mm:ss.sss). This parameter is only required for
         PROJECTION values other than {\tt "}NONE{\tt "}.

          This parameter is not specified on the command line. The
          value of the global parameter PONGO\_RACENTRE is used. If
          PONGO\_RACENTRE is not defined, the default value {\tt "}0{\tt "} is used.
      }
      \sstsubsection{
         DECCENTRE = \_LITERAL (Read)
      }{
         The centre of the projection in declination (i.e. the angle
         must be specified as dd:mm:ss.sss). This parameter is only
         required for PROJECTION values other than {\tt "}NONE{\tt "}.

         This parameter is not specified on the command line. The value
         of the global parameter PONGO\_DECCENTRE is used. If
         PONGO\_DECCENTRE is not defined, the default value {\tt "}0{\tt "} is used.
      }
   }
   \sstnotes{
      \sstitemlist{

         \sstitem
         It is sometimes necessary to specify the grid intervals in a
         manner which avoids rounding errors to obtain the last grid line.
         e.g 9.9999 instead of 10.
      }
   }
}
\sstroutine{
   GT\_CIRCLE
}{
   Draw a great circle between two points
}{
   \sstdescription{
      Draw a great circle between two points in current projection. The
      great circle is specified by giving the coordinates in degrees of
      two points on the celestial sphere. Either the small or large
      great circle may be drawn.
   }
   \sstparameters{
      \sstsubsection{
         PHISTART = \_DOUBLE (Read and Write)
      }{
         The longitude of the start of the great circle in degrees.

         If no value is specified on the command line, the current
         value is used. The current value is initially set to 0.0.
      }
      \sstsubsection{
         THESTART = \_DOUBLE (Read and Write)
      }{
         The latitude of the start of the great circle in degrees.

         If no value is specified on the command line, the current
         value is used. The current value is initially set to 0.0
      }
      \sstsubsection{
         PHIEND = \_DOUBLE (Read and Write)
      }{
         The longitude of the end of the great circle in degrees.

         If no value is specified on the command line, the current
         value is used. The current value is initially set to 0.0.
      }
      \sstsubsection{
         THEEND = \_DOUBLE (Read and Write)
      }{
         The latitude of the end of the great circle in degrees.

         If no value is specified on the command line, the current
         value is used. The current value is initially set to 0.0.
      }
      \sstsubsection{
         ACUTE = \_LOGICAL (Read and Write)
      }{
         If TRUE, the smaller great circle arc is drawn between the
         given points.

         If no value is specified on the command line, the current
         value is used. The current value is initially set to TRUE.
      }
      \sstsubsection{
         PROJECTION = \_CHAR (Read)
      }{
         The geometry to be used to plot the great circle.  This is
         explained in more detail in the section on projections.
         Allowed values: {\tt "}NONE{\tt "}, {\tt "}TAN{\tt "}, {\tt "}SIN{\tt "}, {\tt "}ARC{\tt "}, {\tt "}GLS{\tt "}, {\tt "}AITOFF{\tt "},
         {\tt "}MERCATOR{\tt "} and {\tt "}STG{\tt "}.

         This parameter is not specified on the command line. The value
         of the global parameter PONGO\_PROJECTN is used. If
         PONGO\_PROJECTN is not defined, the default value {\tt "}NONE{\tt "} is
         used.
      }
      \sstsubsection{
         RACENTRE = \_CHAR (Read)
      }{
         The centre of the projection in RA (i.e. the angle must be
         specified as hh:mm:ss.sss). This parameter is only required for
         PROJECTION values other than {\tt "}NONE{\tt "}.

         This parameter is not specified on the command line. The value
         of the global parameter PONGO\_RACENTRE is used. If
         PONGO\_RACENTRE is not defined, the default value {\tt "}0{\tt "} is used.
      }
      \sstsubsection{
         DECCENTRE = \_CHAR (Read)
      }{
         The centre of the projection in declination (i.e. the angle
         must be specified as dd:mm:ss.sss). This parameter is only
         required for PROJECTION values other than {\tt "}NONE{\tt "}.

         This parameter is not specified on the command line. The value
         of the global parameter PONGO\_DECCENTRE is used. If
         PONGO\_DECCENTRE is not defined, the default value {\tt "}0{\tt "} is used.
      }
   }
}
\sstroutine{
   HISTOGRAM
}{
   Bin and plot a histogram of the data
}{
   \sstdescription{
      The data in the XCOL data area are binned and plotted as a
      histogram.  It is possible to plot several histograms with
      different bin sizes from the same data in XCOL because the data
      are unaffected by HISTOGRAM.

      This command is a synonym for PLOTHIST H.
   }
   \sstparameters{
      \sstsubsection{
         BINMIN = \_REAL (Read and Write)
      }{
         The lower limit of the binning.

         If no value is specified on the command line, the current
         value is used. If there is no current value, the value of the
         global parameter PONGO\_XMIN is used.
      }
      \sstsubsection{
         BINMAX = \_REAL (Read and Write)
      }{
         The upper limit of the binning.

         If no value is specified on the command line, the current
         value is used. If there is no current value, the value of the
         global parameter PONGO\_XMAX.
      }
      \sstsubsection{
         NBIN = \_INTEGER (Read and Write)
      }{
         The number of equally sized bins to be drawn between the
         limits of the histogram.

         If no value is specified on the command line, the current
         value is used. The current value is initially set to 10.
      }
      \sstsubsection{
         AUTOSCALE = \_LOGICAL (Read and Write)
      }{
         If TRUE, PGPLOT auto-scaling is used to determine the plotting
         limits. If FALSE, the limits defined by the bins of the
         histogram determine the plotting limits. Here, the plotting
         limits must previously have been set using the LIMITS
         application and the plot frame drawn using BOXFRAME. Setting
         NOAUTOSCALE can be used to draw more than one histogram on the
         same plot.

         If no value is specified on the command line, the current
         value is used. The current value is initially set to TRUE.
      }
   }
}
\sstroutine{
   INQUIRE
}{
   Display PONGO status information
}{
   \sstdescription{
       Display information about the status of PONGO and the data which
       have been read in. The options are:

      \sstitemlist{

         \sstitem
             PGPLOT -- Display the current font, character height colour
             etc.

         \sstitem
             LIMITS -- Display the data limits and the PGPLOT world
             coordinate limits.

         \sstitem
             COLUMNS -- Display the column names from the data file, if
             they have been set up appropriately.

         \sstitem
             DEVICES -- Display the available graphics devices.

         \sstitem
             DATA -- Show the data that has been read in.

      }
      More than one of these options can be specified on the command line
      at any one time.

      The DATA option uses additional parameters to allow scrolling.
   }
   \sstparameters{
      \sstsubsection{
         PGPLOT = \_LOGICAL (Read)
      }{
         Display the current PGPLOT plotting attributes.
         [FALSE]
      }
      \sstsubsection{
         LIMITS = \_LOGICAL (Read)
      }{
         Display the data limits and the current PGPLOT viewport and
         world coordinate limits.
         [FALSE]
      }
      \sstsubsection{
         COLUMNS = \_LOGICAL (Read)
      }{
         Display the data file column headings (if available).
         [FALSE]
      }
      \sstsubsection{
         DEVICES = \_LOGICAL (Read)
      }{
         Display the plotting devices available.
         [FALSE]
      }
      \sstsubsection{
         DATA = \_LOGICAL (Read)
      }{
         Display the contents of all data areas in a formatted form.
         [FALSE]
      }
      \sstsubsection{
         PAGE = \_INTEGER (Read and Write)
      }{
         The page length of the terminal (in the range 1 to 100). It is
         used to stop information scrolling off the top of the screen.
         The parameter will be prompted for at the end of each screen:
         a null response to the prompt will terminate the listing.

         If no value is specified on the command line, the current
         value is used. The current value is initially set to 24.
      }
      \sstsubsection{
         FROM = \_INTEGER (Read and Write)
      }{
         The number of the first item to be displayed.

         If no value is specified on the command line, the current
         value is used. The current value is initially set to 0
         (implying the start of the list).
      }
      \sstsubsection{
         TO = \_INTEGER (Read and Write)
      }{
         The number of the last item to be displayed.

         If no value is specified on the command line, the current
         value is used. The current value is initially set to 0
         (implying the end of the list).
      }
   }
}
\sstroutine{
   LABCOLUMN
}{
   Specify the column used for data labels
}{
   \sstdescription{
      Specify the column in the data file from which the data labels are
      to be read.

      This command is a synonym for SETGLOBAL PONGO\_LABCOL.
   }
   \sstparameters{
      \sstsubsection{
         LABCOL = \_CHAR (Read and Write)
      }{
         The column number (counting from 1), or the symbolic name of a
         column, from which the symbolic name for each data point is
         read by the READF command. The value {\tt "}0{\tt "} means {\tt "}do not read
         these data{\tt "}. []
      }
   }
}
\sstroutine{
   LABEL
}{
   Draw the axis labels and title on the plot
}{
   \sstdescription{
      Draw the axis labels and the title on the plot. If the COLUMNS
      parameter is specified, the axis labels are obtained from the
      column labels read from the data file.
   }
   \sstparameters{
      \sstsubsection{
         XLABEL = \_CHAR (Read and Write)
      }{
         The X-axis label.

         If no value is specified on the command line, then if COLUMNS
         is TRUE the value is taken from the column heading in the
         data file, otherwise the value of the global parameter
         PONGO\_XLABEL is used. If PONGO\_XLABEL is not defined, the
         value {\tt "} {\tt "} is used.
      }
      \sstsubsection{
         YLABEL = \_CHAR (Read and Write)
      }{
         The Y-axis label.

         If no value is specified on the command line, then if COLUMNS
         is TRUE the value is taken from the column heading in the
         data file, otherwise the value of the global parameter
         PONGO\_YLABEL is used. If PONGO\_YLABEL is not defined, the
         value {\tt "} {\tt "} is used.
      }
      \sstsubsection{
         TITLE = \_CHAR (Read and Write)
      }{
         The plot title.

         [The value of the global parameter PONGO\_TITLE is used. If
         PONGO\_TITLE is not defined, the value {\tt "} {\tt "} is used.]
      }
      \sstsubsection{
         COLUMNS = \_LOGICAL (Read)
      }{
         If TRUE, the values of the X and Y labels will be obtained from
         the column headings in the data file.
         [FALSE]
      }
   }
}
\sstroutine{
   LIMITS
}{
   Set the world coordinate limits
}{
   \sstdescription{
      The world coordinate limits are set from the parameters given
      on the command line.

      This command is a synonym for WORLD GIVEN.
   }
   \sstparameters{
      \sstsubsection{
         XMIN = \_REAL (Read and Write)
      }{
         The world coordinate of the left-hand edge of the plot.

         [The value of the global parameter PONGO\_XMIN is used.]
      }
      \sstsubsection{
         XMAX = \_REAL (Read and Write)
      }{
         The world coordinate of the right-hand edge of the plot.

         [The value of the global parameter PONGO\_XMAX is used.]
      }
      \sstsubsection{
         YMIN = \_REAL (Read and Write)
      }{
         The world coordinate of the lower edge of the plot.

         [The value of the global parameter PONGO\_YMIN is used.]
      }
      \sstsubsection{
         YMAX = \_REAL (Read and Write)
      }{
         The world coordinate of the upper edge of the plot.

         [The value of the global parameter PONGO\_YMAX is used.]
      }
      \sstsubsection{
         PROJECTION = \_CHAR (Read and Write)
      }{
         The geometry to be used for plotting the data.  This is
         explained in more detail in the section on projections.
         Allowed values: {\tt "}NONE{\tt "}, {\tt "}TAN{\tt "}, {\tt "}SIN{\tt "}, {\tt "}ARC{\tt "}, {\tt "}GLS{\tt "}, {\tt "}AITOFF{\tt "},
         {\tt "}MERCATOR{\tt "} and {\tt "}STG{\tt "}.

         [The value of the global parameter PONGO\_PROJECTN is used. If
         PONGO\_PROJECTN is not defined, the default value {\tt "}NONE{\tt "} is
         used.]
      }
      \sstsubsection{
         RACENTRE = \_CHAR (Read and Write)
      }{
         The centre of the projection in RA (i.e. the angle must be
         specified as hh:mm:ss.sss). This parameter is only required for
         PROJECTION values other than {\tt "}NONE{\tt "}.

         [The value of the global parameter PONGO\_RACENTRE is used. If
         PONGO\_RACENTRE is not defined, the default value {\tt "}0{\tt "} is used.]
      }
      \sstsubsection{
         DECCENTRE = \_CHAR (Read and Write)
      }{
         The centre of the projection in declination (i.e. the angle
         must be specified as dd:mm:ss.sss). This parameter is only
         required for PROJECTION values other than {\tt "}NONE{\tt "}.

         [The value of the global parameter PONGO\_DECCENTRE is used. If
         PONGO\_DECCENTRE is not defined, the default value {\tt "}0{\tt "} is used.]
      }
   }
}
\sstroutine{
   LTYPE
}{
   Set the line style
}{
   \sstdescription{
      Change the line style.

      This command is a synonym for CHANGE LINESTY=.
   }
   \sstparameters{
      \sstsubsection{
         LINESTY = \_INTEGER (Read and Write)
      }{
         The line style used by PGPLOT.  The line style may be one of
         the following:

         \sstitemlist{

            \sstitem
               1 -- full line,

            \sstitem
               2 -- dashed,

            \sstitem
               3 -- dot-dash-dot-dash,

            \sstitem
               4 -- dotted,

            \sstitem
               5 -- dash-dot-dot-dot.

         }
         []
      }
   }
}
\sstroutine{
   LWEIGHT
}{
   Set the line width
}{
   \sstdescription{
      Change the line width scaling (the normal line width is 1.0).

      This command is a synonym for CHANGE LINEWID=.
   }
   \sstdiytopic{
      ADAM Parameters
   }{
      LINEWID = \_INTEGER (Read and Write)
         The line width scaling. This parameter scales the default line
         width. []
   }
}
\sstroutine{
   MARK
}{
   Draw a point mark at the specified position
}{
   \sstdescription{
      Draw a point mark at a specified position.

      This command is a synonym for PRIM K.
   }
   \sstparameters{
      \sstsubsection{
         X = \_REAL (Read and Write)
      }{
         The X coordinate of the point.

         [The value is prompted for.]
      }
      \sstsubsection{
         Y = \_REAL (Read and Write)
      }{
         The Y coordinate of the point.

         [The value is prompted for.]
      }
      \sstsubsection{
         SYMBOL = \_INTEGER (Read and Write)
      }{
         The PGPLOT symbol number for drawing the point mark.

         If no value is given on the command line, the current value is
         used. The current value is initially set to 1.
      }
      \sstsubsection{
         PROJECTION = \_CHAR (Read)
      }{
         The geometry that is to be used for plotting. This is
         explained in more detail in the section on projections.
         Allowed values: {\tt "}NONE{\tt "}, {\tt "}TAN{\tt "}, {\tt "}SIN{\tt "}, {\tt "}ARC{\tt "}, {\tt "}GLS{\tt "}, {\tt "}AITOFF{\tt "},
         {\tt "}MERCATOR{\tt "} and {\tt "}STG{\tt "}.

         This parameter is not specified on the command line. The value
         of the global parameter PONGO\_PROJECTN is used. If
         PONGO\_PROJECTN is not defined, the default value {\tt "}NONE{\tt "} is
         used.
      }
      \sstsubsection{
         RACENTRE = \_CHAR (Read)
      }{
         The centre of the projection in RA (i.e. the angle must be
         specified as hh:mm:ss.sss). This parameter is only required for
         PROJECTION values other than {\tt "}NONE{\tt "}.

         This parameter is not specified on the command line. The value
         of the global parameter PONGO\_RACENTRE is used. If
         PONGO\_RACENTRE is not defined, the default value {\tt "}0{\tt "} is used.
      }
      \sstsubsection{
         DECCENTRE = \_CHAR (Read)
      }{
         The centre of the projection in declination (i.e. the angle
         must be specified as dd:mm:ss.sss). This parameter is only
         required for PROJECTION values other than {\tt "}NONE{\tt "}.

         This parameter is not specified on the command line. The value
         of the global parameter PONGO\_DECENTRE is used. If
         PONGO\_DECCENTRE is not defined, the default value {\tt "}0{\tt "} is used.
      }
   }
   \sstnotes{
      When using non-planar coordinates, the coordinates should be
      given in degrees.
   }
}
\sstroutine{
   MOVE
}{
   Set the current pen position
}{
   \sstdescription{
      The current pen position is set to the given coordinates.

      This command is a synonym for PRIM M.
   }
   \sstparameters{
      \sstsubsection{
         X = \_REAL (Read and Write)
      }{
         The X coordinate of the point.

         If no value is specified on the command line, the value is
         prompted for.
      }
      \sstsubsection{
         Y = \_REAL (Read and Write)
      }{
         The Y coordinate of the point.

         If no value is specified on the command line, the value is
         prompted for.
      }
   }
}
\sstroutine{
   MTEXT
}{
   Draw a text string relative to the viewport
}{
   \sstdescription{
      Draw a text string on the plot at a position specified relative
      to the viewport. The command uses the PGPLOT routine PGMTEXT.

      This command is a an ICL hidden procedure which uses the WTEXT
      application.
   }
   \sstparameters{
      \sstsubsection{
         SIDE = \_CHAR (Read and Write)
      }{
         The side of the viewport where the text is to plotted. This
         may be one of the following:
         \sstitemlist{

            \sstitem
               {\tt "}T{\tt "} -- The top edge.

            \sstitem
               {\tt "}B{\tt "} -- The bottom edge.

            \sstitem
               {\tt "}L{\tt "} -- The left-hand edge.

            \sstitem
               {\tt "}R{\tt "} -- The right-hand edge.

            \sstitem
               {\tt "}LV{\tt "} -- The left-hand edge, but with the string written
               vertically.

            \sstitem
               {\tt "}RV{\tt "} -- The right-hand edge, but with the string written
               vertically.

         }
         []
      }
      \sstsubsection{
         XPOS = \_REAL (Read and Write)
      }{
         The number of character heights from the viewport where the
         text is to be plotted (negative values are allowed). []
      }
      \sstsubsection{
         YPOS = \_REAL (Read and Write)
      }{
         The fraction along the edge where the text is to be plotted. []
      }
      \sstsubsection{
         JUSTIFICATION = \_REAL (Read and Write)
      }{
         The justification about the specified point (in the range 0.0
         to 1.0).  Here, 0.0 means left justify the text relative to
         the data point, 1.0 means right justify the text relative to
         the data point, 0.5 means centre the string on the data point,
         other values will give intermediate justifications. []
      }
      \sstsubsection{
         TEXT = \_CHAR (Read and Write)
      }{
         The text string to be plotted. This may include any of the
         PGPLOT control sequences for producing special characters. []
      }
   }
}
\sstroutine{
   PALETTE
}{
   Change the plotting pen colours
}{
   \sstdescription{
      The colour representation for a given colour index is updated.
   }
   \sstparameters{
      \sstsubsection{
         COLOUR = \_INTEGER (Read and Write)
      }{
         The colour index, i.e. pen, to be updated (in the range 0 to
         255).

         If no value is specified on the command line, the current
         value is used. The current value is initially set to 1.
      }
      \sstsubsection{
         RED = \_REAL (Read and Write)
      }{
         The red intensity (in the range 0.0 to 1.0).

         If no value is specified on the command line, the current
         value is used. The current value is initially set to 1.0.
      }
      \sstsubsection{
         GREEN = \_REAL (Read and Write)
      }{
         The green intensity (in the range 0.0 to 1.0).

         If no value is specified on the command line, the current
         value is used. The current value is initially set to 1.0.
      }
      \sstsubsection{
         BLUE = \_REAL (Read and Write)
      }{
         The blue intensity (in the range 0.0 to 1.0).

         If no value is specified on the command line, the current
         value is used. The current value is initially set to 1.0.
      }
   }
}
\sstroutine{
   PAPER
}{
   Change the size and aspect ratio of the plotting surface
}{
   \sstdescription{
      The width of the plotting surface and its aspect ratio (i.e.
      height/width) are modified.
   }
   \sstparameters{
      \sstsubsection{
         WIDTH = \_REAL (Read and Write)
      }{
         The width of the plotting surface in inches. If the specified
         width is 0.0, the maximum possible width for the device is
         used.

         [The value is prompted for.]
      }
      \sstsubsection{
         ASPECT = \_REAL (Read and Write)
      }{
         The aspect ratio of the plotting surface: i.e. height/width.

         [The value is prompted for.]
      }
   }
}
\sstroutine{
   PCOLUMN
}{
   Specify the column used for symbol codes
}{
   \sstdescription{
      Specify the column in the data file from which the symbol codes
      are to be read.

      This command is a synonym for SETGLOBAL PONGO\_SYMCOL.
   }
   \sstparameters{
      \sstsubsection{
         SYMCOL = \_CHAR (Read and Write)
      }{
         The column number (counting from 1), or the symbolic name of a
         column, from which the PGPLOT symbol code for each data point
         is read by the READF command. The value {\tt "}0{\tt "} means {\tt "}do not read
         these data{\tt "}. []
      }
   }
}
\sstroutine{
   PEN
}{
   Set the current pen
}{
   \sstdescription{
      Set the current pen used for plotting.

      This command is a synonym for CHANGE COLOUR=.
   }
   \sstparameters{
      \sstsubsection{
         COLOUR = \_INTEGER (Read and Write)
      }{
         The pen number (colour index) PGPLOT uses for plotting. The
         value should be between 0 and 255. Usually the first 16 pens
         are predefined to have the following colours:

         \sstitemlist{

            \sstitem
               0 -- background,

            \sstitem
               1 -- foreground (default),

            \sstitem
               2 -- red,

            \sstitem
               3 -- green,

            \sstitem
               4 -- blue,

            \sstitem
               5 -- cyan,

            \sstitem
               6 -- magenta,

            \sstitem
               7 -- yellow,

            \sstitem
               8 -- red $+$ yellow (orange),

            \sstitem
               9 -- green $+$ yellow,

            \sstitem
               10 -- green $+$ cyan,

            \sstitem
               11 -- blue $+$ cyan,

            \sstitem
               12 -- blue $+$ magenta,

            \sstitem
               13 -- red $+$ magenta,

            \sstitem
               14 -- dark grey,

            \sstitem
               15 -- light grey.

         }
         It is possible to change the colour representation of any of
         the pen colour indices using the PALETTE application. []
      }
   }
}
\sstroutine{
   PLOTFUN
}{
   Plot a given function
}{
   \sstdescription{
      Plot a function specified on the command line by a Fortran-like
      expression, or through the parameters resulting from a previous
      fit (using a polynomial or spline).
   }
   \sstparameters{
      \sstsubsection{
         ACTION = \_CHAR (Read)
      }{
         The type of function to be plotted. This must be one of the
         following:
         \sstitemlist{

            \sstitem
               {\tt "}FUNC{\tt "} -- Use a Fortran-like expression to define the
               function.

            \sstitem
               {\tt "}POLY{\tt "} -- Use a set of polynomial coefficients to define
               the function.

            \sstitem
               {\tt "}SPLINE{\tt "} -- Use a set of spline coefficients from the file
               SPLINEFILE to define the function.

         }
         [The value is prompted for.]
      }
      \sstsubsection{
         EXPRESSION = \_CHAR (Read)
      }{
         The Fortran-like expression to be plotted, in terms of X.

         [The value is prompted for.]
      }
      \sstsubsection{
         XMIN = \_REAL (Read)
      }{
         The value of X from which the function is plotted.

         [The value of the global parameter PONGO\_XMIN is used.]
      }
      \sstsubsection{
         XMAX = \_REAL (Read)
      }{
         The value of X to which the function is plotted.

         [The value of the global parameter PONGO\_XMAX is used.]
      }
      \sstsubsection{
         NPOLY = \_INTEGER (Read and Write)
      }{
         The order of the polynomial: used when ACTION is {\tt "}POLY{\tt "}.

         [The value of the global parameter PONGO\_NPOLY is used.]
      }
      \sstsubsection{
         POLYCOEF = \_DOUBLE (Read)
      }{
         A list of polynomial coefficients: used when ACTION is {\tt "}POLY{\tt "}.

         [The value of the global parameter POLY\_POLYCOEF is used.]
      }
      \sstsubsection{
         SPLINEFILE = FILENAME (Read)
      }{
         The name of the file containing the coefficients and knot
         positions from a previous spline fit -- used when ACTION is
         {\tt "}SPLINE{\tt "}.

         [The value of the global parameter PONGO\_SPLINEF is used. If
         PONGO\_SPLINEF is not defined, the value is prompted for.]
      }
   }
}
\sstroutine{
   PLOTHIST
}{
   Plot a histogram of the data
}{
   \sstdescription{
      This application has two modes:
      \sstitemlist{

         \sstitem
         Bin the data in the XCOL data area and plot the result.

         \sstitem
         Plot data that have already been binned and provided in the XCOL
         and YCOL data areas.
      }
   }
   \sstparameters{
      \sstsubsection{
         ACTION = \_CHAR (Read)
      }{
         The mode of PLOTHIST as described above:

         \sstitemlist{

            \sstitem
            {\tt "}H{\tt "} -- If the data in the XCOL data area are not binned, they
            can be binned and then plotted.  It is possible to plot
            several histograms with different bin sizes from the same data
            in XCOL because the data are unaffected by PLOTHIST.

            \sstitem
            {\tt "}B{\tt "} -- If the data have already been binned, this mode will
            plot a histogram using the XCOL and YCOL data areas. The XCOL
            data area should specify the bin edges and the YCOL data area
            should contain their respective frequencies.

         }
         [The value is prompted for.]
      }
      \sstsubsection{
         BINMIN = \_REAL (Read and Write)
      }{
         When ACTION is {\tt "}H{\tt "}, this parameter specifies the lower limit
         of the binning.

         If no value is specified on the command line, the current
         value is used. If there is no current value, the value of the
         global parameter PONGO\_XMIN is used.
      }
      \sstsubsection{
         BINMAX = \_REAL (Read and Write)
      }{
         When ACTION is {\tt "}H{\tt "}, this parameter specifies the upper limit of
         the binning.

         If no value is specified on the command line, the current
         value is used. If there is no current value, the value of the
         global parameter PONGO\_XMAX.
      }
      \sstsubsection{
         NBIN = \_INTEGER (Read and Write)
      }{
         When ACTION is {\tt "}H{\tt "}, this parameter specifies the number of
         equally sized bins to be drawn between the limits of the
         histogram.

         If no value is specified on the command line, the current
         value is used. The current value is initially set to 10.
      }
      \sstsubsection{
         AUTOSCALE = \_LOGICAL (Read and Write)
      }{
         If TRUE, PGPLOT auto-scaling is used to determine the plotting
         limits. If FALSE, the limits defined by the bins of the
         histogram determine the plotting limits. Here, the plotting
         limits must previously have been set using the LIMITS
         application and the plot frame drawn using BOXFRAME. Setting
         NOAUTOSCALE can be used to draw more than one histogram on the
         same plot.

         If no value is specified on the command line, the current
         value is used. The current value is initially set to TRUE.
      }
      \sstsubsection{
         CENTRE = \_LOGICAL (Read)
      }{
         When ACTION is {\tt "}B{\tt "}, this parameter specifies whether the
         values in the XCOL data area denote the centre of each bin
         (when TRUE) or its lower edge (when FALSE).
         [FALSE]
      }
   }
}
\sstroutine{
   POINTS
}{
   Draw a point mark at each of the data points
}{
   \sstdescription{
      Draw a symbol at each of the data points.  The symbol type
      that is used to mark each point is determined in one of three ways:

      \sstitemlist{

         \sstitem
            If no SYMBOL parameter is supplied on the command line, and
            no symbol numbers have been read into the symbol data area by
            READF, the point style will be set by the current value of
            SYMBOL.

         \sstitem
            If no SYMBOL parameter is supplied on the command line, and
            values have been read into the symbol data area by READF, the
            symbol number for each point will determine the style of the
            point plotted.

         \sstitem
            If SYMBOL is specified on the command line, it will override
            each of the above options. The same specified symbol will be
            used to mark all points.

      }
      The value of the symbol index should be an integer which refers
      to the standard PGPLOT symbols.

      This command is a synonym for GPOINTS P.
   }
   \sstparameters{
      \sstsubsection{
         SYMBOL = \_INTEGER (Read and Write)
      }{
         The PGPLOT symbol number that is used to mark the data points.

         If a value is specified on the command line, it will be used
         for plotting symbols for all the data. If no value is
         specified on the command line, the application attempts to use
         the SYMCOL data for its symbols. If no symbol values have been
         read into the SYMCOL data area, the current value is used for
         all the data. The current value is initially set to 1.
      }
      \sstsubsection{
         SCALE = \_REAL (Read and Write)
      }{
         This parameter is not used with this invocation of GPOINTS.

         If no value is specified on the command line, the current
         value is used. The current value is initially set to 1.0.
      }
      \sstsubsection{
         PROJECTION = \_CHAR (Read)
      }{
         Specifies the geometry that is to be used to plot the data.
         This is explained in more detail in the section on
         projections.  Allowed values: {\tt "}NONE{\tt "}, {\tt "}TAN{\tt "}, {\tt "}SIN{\tt "}, {\tt "}ARC{\tt "},
         {\tt "}GLS{\tt "}, {\tt "}AITOFF{\tt "}, {\tt "}MERCATOR{\tt "} and {\tt "}STG{\tt "}.

         This parameter is not specified on the command line. The value
         of the global parameter PONGO\_PROJECTN is used. If
         PONGO\_PROJECTN is not defined, the default value {\tt "}NONE{\tt "} is
         used.
      }
      \sstsubsection{
         RACENTRE = \_CHAR (Read)
      }{
         The centre of the projection in RA (i.e. the angle must be
         specified as hh:mm:ss.sss). This parameter is only required for
         PROJECTION values other than {\tt "}NONE{\tt "}.

         This parameter is not specified on the command line. The value
         of the global parameter PONGO\_RACENTRE is used. If
         PONGO\_RACENTRE is not defined, the default value {\tt "}0{\tt "} is used.
      }
      \sstsubsection{
         DECCENTRE = \_CHAR (Read)
      }{
         The centre of the projection in declination (i.e. the angle
         must be specified as dd:mm:ss.sss). This parameter is only
         required for PROJECTION values other than {\tt "}NONE{\tt "}.

         This parameter is not specified on the command line. The value
         of the global parameter PONGO\_DECCENTRE is used. If
         PONGO\_DECCENTRE is not defined, the default value {\tt "}0{\tt "} is used.
      }
   }
}
\sstroutine{
   PRIM
}{
   Perform primitive plotting functions
}{
   \sstdescription{
      Perform the primitive plotting functions: move to, draw line to,
      and mark.
   }
   \sstparameters{
      \sstsubsection{
         ACTION = \_CHAR (Read and Write)
      }{
         Type of primitive function. This may be one of the following:

         \sstitemlist{

            \sstitem
            {\tt "}M{\tt "} -- move to,

            \sstitem
            {\tt "}D{\tt "} -- draw line to,

            \sstitem
            {\tt "}K{\tt "} -- mark.

         }
         [The value is prompted for.]
      }
      \sstsubsection{
         X = \_REAL (Read and Write)
      }{
         X coordinate of point.

         [The value is prompted for.]
      }
      \sstsubsection{
         Y = \_REAL (Read and Write)
      }{
         Y coordinate of point.

         [The value is prompted for.]
      }
      \sstsubsection{
         SYMBOL = \_INTEGER (Read and Write)
      }{
         PGPLOT symbol number for drawing the point mark.

         If no value is specified on the command line, the current
         value is used. The current value is initially set to 1.
      }
      \sstsubsection{
         PROJECTION = \_CHAR (Read)
      }{
         The geometry that is to be used for plotting. This is
         explained in more detail in the section on projections.
         Allowed values: {\tt "}NONE{\tt "}, {\tt "}TAN{\tt "}, {\tt "}SIN{\tt "}, {\tt "}ARC{\tt "}, {\tt "}GLS{\tt "}, {\tt "}AITOFF{\tt "},
         {\tt "}MERCATOR{\tt "} and {\tt "}STG{\tt "}.

         This parameter is not specified on the command line. The value
         of the global parameter PONGO\_PROJECTN is used. If
         PONGO\_PROJECTN is not defined, the default value {\tt "}NONE{\tt "} is
         used.
      }
      \sstsubsection{
         RACENTRE = \_CHAR (Read)
      }{
         The centre of the projection in RA (i.e. the angle must be
         specified as hh:mm:ss.sss). This parameter is only required for
         PROJECTION values other than {\tt "}NONE{\tt "}.

         This parameter is not specified on the command line. The value
         of the global parameter PONGO\_RACENTRE is used. If
         PONGO\_RACENTRE is not defined, the default value {\tt "}0{\tt "} is used.
      }
      \sstsubsection{
         DECCENTRE = \_CHAR (Read)
      }{
         The centre of the projection in declination (i.e. the angle
         must be specified as dd:mm:ss.sss). This parameter is only
         required for PROJECTION values other than {\tt "}NONE{\tt "}.

         This parameter is not specified on the command line. The value
         of the global parameter PONGO\_DECENTRE is used. If
         PONGO\_DECCENTRE is not defined, the default value {\tt "}0{\tt "} is used.
      }
   }
   \sstnotes{
      Currently, only the mark function (ACTION={\tt "}K{\tt "}) uses non-planar
      coordinate geometries. When using non-planar coordinates, the
      coordinates should be given in degrees.
   }
}
\sstroutine{
   PTEXT
}{
   Draw a text string at the specified position and angle
}{
   \sstdescription{
      Draw a text string on the plot at the specified position and
      angle. The command uses the PGPLOT routine PGPTEXT.

      This command is a an ICL hidden procedure which uses the WTEXT
      application.
   }
   \sstparameters{
      \sstsubsection{
         XPOS = \_REAL (Read and Write)
      }{
         The X coordinate of the text. []
      }
      \sstsubsection{
         YPOS = \_REAL (Read and Write)
      }{
         The Y coordinate of the text. []
      }
      \sstsubsection{
         ANGLE = \_REAL (Read and Write)
      }{
         The angle relative to the horizontal at which the text string
         is to be plotted. []
      }
      \sstsubsection{
         JUSTIFICATION = \_REAL (Read and Write)
      }{
         The justification about the specified point (in the range 0.0
         to 1.0).  Here, 0.0 means left justify the text relative to
         the data point, 1.0 means right justify the text relative to
         the data point, 0.5 means centre the string on the data point,
         other values will give intermediate justifications. []
      }
      \sstsubsection{
         TEXT = \_CHAR (Read and Write)
      }{
         The text string to be plotted. This may include any of the
         PGPLOT control sequences for producing special characters. []
      }
   }
}
\sstroutine{
   PTINFO
}{
   Get the coordinates of a specified data point
}{
   \sstdescription{
      The value of a label in the LABCOL data area is given and the
      resulting (x,y) coordinates associated with the label are
      printed.

      This command is an ICL hidden procedure which uses the GETPOINT
      application.
   }
   \sstparameters{
      \sstsubsection{
         LABEL = \_CHAR (Read)
      }{
         The label associated with the data point. []
      }
   }
}
\sstroutine{
   PVECT
}{
   Draw proper motion vectors
}{
   \sstdescription{
      Draw proper motion vectors on a projection of the celestial
      sphere. The XCOL and YCOL data areas are assumed to contain
      positions in radians, the EXCOL and EYCOL data areas are assumed
      to contain the proper motions in radians per year. It is possible
      to use the ERSCALE parameter to multiply the the proper motion so
      that it is correct for a given number of years. (The proper
      motion in RA is assumed to be \$$\backslash$dot\{$\backslash$alpha\}$\backslash$cos$\backslash$delta\$.)
   }
   \sstparameters{
      \sstsubsection{
         ERSCALE = \_REAL (Read and Write)
      }{
         The scale factor for multiplying the vectors.

         [The value of the global parameter PONGO\_ERSCALE is used. If
         PONGO\_ERSCALE is not defined, the default value 1.0 is used.]
      }
      \sstsubsection{
         ZMULT = \_LOGICAL (Read)
      }{
         If TRUE, the ZCOL values are additionally used to multiply the
         vectors.
         [FALSE]
      }
      \sstsubsection{
         PROJECTION = \_CHAR (Read)
      }{
         The geometry used to plot the data.  This is explained in more
         detail in the section on projections.  Allowed values: {\tt "}NONE{\tt "},
         {\tt "}TAN{\tt "}, {\tt "}SIN{\tt "}, {\tt "}ARC{\tt "}, {\tt "}GLS{\tt "}, {\tt "}AITOFF{\tt "}, {\tt "}MERCATOR{\tt "} and {\tt "}STG{\tt "}.

         This parameter is not specified on the command line. The value
         of the global parameter PONGO\_PROJECTN is used. If
         PONGO\_PROJECTN is not defined, the default value {\tt "}NONE{\tt "} is
         used.
      }
      \sstsubsection{
         RACENTRE = \_CHAR (Read)
      }{
         The centre of the projection in RA (i.e. the angle must be
         specified as hh:mm:ss.sss). This parameter is only required for
         PROJECTION values other than {\tt "}NONE{\tt "}.

         This parameter is not specified on the command line. The value
         of the global parameter PONGO\_RACENTRE is used. If
         PONGO\_RACENTRE is not defined, the default value {\tt "}0{\tt "} is used.
      }
      \sstsubsection{
         DECCENTRE = \_CHAR (Read)
      }{
         The centre of the projection in declination (i.e. the angle
         must be specified as dd:mm:ss.sss). This parameter is only
         required for PROJECTION values other than {\tt "}NONE{\tt "}.

         This parameter is not specified on the command line. The value
         of the global parameter PONGO\_DECCENTRE is used. If
         PONGO\_DECCENTRE is not defined, the default value {\tt "}0{\tt "} is used.
      }
   }
}
\sstroutine{
   RADIATE
}{
   Draw a line from the given point to the first NP data points
}{
   \sstdescription{
      A line is drawn from the given radiant, position (X,Y), to the
      first NP data points in the XCOL and YCOL data areas.

      This command is an ICL hidden procedure which uses successive
      calls to the GETPOINT, MOVE and DRAW applications.
   }
   \sstparameters{
      \sstsubsection{
         X = \_REAL (Read)
      }{
         The X-axis position of the radiant. []
      }
      \sstsubsection{
         Y = \_REAL (Read)
      }{
         The Y-axis position of the radiant. []
      }
      \sstsubsection{
         NP = \_INTEGER (Read)
      }{
         The number of data in the XCOL and YCOL data areas to use,
         starting from the beginning of the data. []
      }
   }
}
\sstroutine{
   READF
}{
   Read from a formatted data file
}{
   \sstdescription{
      Read input data from a formatted data file. READF attempts to
      read data from columns in an sequential formatted file in the
      most flexible manner possible. It is possible to specify the
      following:

      \sstitemlist{

         \sstitem
            What comment delimiter characters are used within the data
            file -- if used, the comment delimiter must be the first
            character of a line in the data file.

         \sstitem
            What the column delimiters are (more than one character is
            possible).

         \sstitem
            Symbolic names for each of the data columns.

      }
      The application is intended to be very robust: if a read error
      occurs within a line, READF will report an error and attempt to
      continue.

      The application has many parameters for controlling how data are
      read, but the default values of these parameters are sufficient
      for reading most data files.
   }
   \sstparameters{
      \sstsubsection{
         DATA = FILENAME (Read and Write)
      }{
         The name of the formatted data file.

         If the value is not specified on the command line, the value
         of the global parameter PONGO\_DATA is used. If PONGO\_DATA is
         not defined, the current value is used. If the current value
         is not defined, the value is prompted for.
      }
      \sstsubsection{
         HARDCOM = \_CHAR (Read and Write)
      }{
         A character used to indicate a comment line in the data file.
         The character must appear in the first column of a comment.

         If the value is not specified on the command line, the current
         value is used. The current value is initially set to {\tt "}!{\tt "}.
      }
      \sstsubsection{
         SOFTCOM = \_CHAR (Read and Write)
      }{
         A character to indicate a comment line in the data file. This
         parameter allows a second character to be used as a comment
         delimiter.  The character must appear in the first column of a
         comment.

         If the value is not specified on the command line, the current
         value is used. The current value is initially set to {\tt "}!{\tt "}.
      }
      \sstsubsection{
         XCOL = \_CHAR (Read and Write)
      }{
         The column number (counting from 1), or the symbolic name of a
         column, from which the X-axis data are read. The value {\tt "}0{\tt "}
         means {\tt "}do not read these data{\tt "}.

         If the value is not specified on the command line, the value
         of the global parameter PONGO\_XCOL is used. If PONGO\_XCOL is
         not defined, the current value is used. The current value is
         initially set to {\tt "}0{\tt "}.
      }
      \sstsubsection{
         YCOL = \_CHAR (Read and Write)
      }{
         The column number (counting from 1), or the symbolic name of a
         column, from which the Y-axis data are read. The value {\tt "}0{\tt "}
         means {\tt "}do not read these data{\tt "}.

         If the value is not specified on the command line, the value
         of the global parameter PONGO\_YCOL is used. If PONGO\_YCOL is
         not defined, the current value is used. The current value is
         initially set to {\tt "}0{\tt "}.
      }
      \sstsubsection{
         ZCOL = \_CHAR (Read and Write)
      }{
         The column number (counting from 1), or the symbolic name of a
         column, from which the Z-axis data are read. The value {\tt "}0{\tt "}
         means {\tt "}do not read these data{\tt "}.

         If the value is not specified on the command line, the value
         of the global parameter PONGO\_ZCOL is used. If PONGO\_ZCOL is
         not defined, the current value is used. The current value is
         initially set to {\tt "}0{\tt "}.
      }
      \sstsubsection{
         EXCOL = \_CHAR (Read and Write)
      }{
         The column number (counting from 1), or the symbolic name of a
         column, from which the X-axis error data are read. The value
         {\tt "}0{\tt "} means {\tt "}do not read these data{\tt "}.

         If the value is not specified on the command line, the value
         of the global parameter PONGO\_EXCOL is used. If PONGO\_EXCOL
         is not defined, the current value is used. The current value
         is initially set to {\tt "}0{\tt "}.
      }
      \sstsubsection{
         EYCOL = \_CHAR (Read and Write)
      }{
         The column number (counting from 1), or the symbolic name of a
         column, from which the Y-axis error data are read. The value
         {\tt "}0{\tt "} means {\tt "}do not read these data{\tt "}.

         If the value is not specified on the command line, the value
         of the global parameter PONGO\_EYCOL is used.  If PONGO\_EYCOL
         is not defined, the current value is used. The current value
         is initially set to {\tt "}0{\tt "}.
      }
      \sstsubsection{
         LABCOL = \_CHAR (Read and Write)
      }{
         The column number (counting from 1), or the symbolic name of a
         column, from which the symbolic name for each data point is
         read. The value {\tt "}0{\tt "} means {\tt "}do not read these data{\tt "}.

         If the value is not specified on the command line, the value
         of the global parameter PONGO\_LABCOL is used. If PONGO\_LABCOL
         is not defined, the current value is used. The current value
         is initially set to {\tt "}0{\tt "}.
      }
      \sstsubsection{
         SYMCOL = \_CHAR (Read and Write)
      }{
         The column number (counting from 1), or the symbolic name of a
         column, from which the PGPLOT symbol code for each data point
         is read. The value {\tt "}0{\tt "} means {\tt "}do not read these data{\tt "}.

         If the value is not specified on the command line, the value
         of the global parameter PONGO\_SYMCOL is used. If PONGO\_SYMCOL
         is not defined, the current value is used. The current value
         is initially set to {\tt "}0{\tt "}.
      }
      \sstsubsection{
         DELIM = \_CHAR (Read and Write)
      }{
         The character string interpreted as a column delimiter when
         reading the data file. For example, this can be used to read
         LATEX format tables by setting DELIM={\tt "}\&{\tt "}.

         If the value is not specified on the command line, the current
         value is used. The current value is initially set to {\tt "} {\tt "}.
      }
      \sstsubsection{
         FROM = \_INTEGER (Read and Write)
      }{
         The first line of data to be read from the data file. The value
         0 defaults to the beginning of the file.
         [0]
      }
      \sstsubsection{
         TO = \_INTEGER (Read and Write)
      }{
         The last line of data to be read from the data file. The value
         0 defaults to the end of the file.
         [0]
      }
      \sstsubsection{
         SELCOND = \_CHAR (Read and Write)
      }{
         A condition (or criterion) upon which to select values from
         the data file.  This condition has the form

         \begin{quote}
            [SELECT\_COL] [COND] [VAL1\{,VAL2, ...\}]
         \end{quote}

         where

         \begin{quote}
         \sstitemlist{

            \sstitem
               SELECT\_COL is the data area used for the selection
               test. This can be specified either by column number
               (counting from 1) or by the symbolic name of a column.
               There is no restriction on which column is used for
               selection, i.e. it does not have to be one of the columns
               from which data are being read.

            \sstitem
               COND is the selection criterion. It may be one of the
               following:

            \sstitemlist{

               \sstitem {\tt "}={\tt "} -- equals;
               \sstitem {\tt "}\#{\tt "} -- not equal;
               \sstitem {\tt "}$>${\tt "} -- greater than;
               \sstitem {\tt "}$<${\tt "} -- less than;
               \sstitem {\tt "}CE{\tt "} -- equal to string;
               \sstitem {\tt "}C\#{\tt "} -- not equal to string;
               \sstitem {\tt "}RA{\tt "} -- in the range VAL1 to VAL2;
               \sstitem {\tt "}LI{\tt "} -- select if in the following list of values;
               \sstitem {\tt "}EX{\tt "} -- exclude if in the following list of values;
               \sstitem {\tt "}IN{\tt "} -- select if the substring is contained within the
               value;
               \sstitem {\tt "}A$>${\tt "} -- absolute value greater than;
               \sstitem {\tt "}A$<${\tt "} -- absolute value less than.
            }

            \sstitem
               VAL1\{,VAL2, ...\} the value (or values) against which the
               selection is made.

         }
         \end{quote}

         Note that there must be white space around the selection
         criterion. A value of {\tt "}0{\tt "} means {\tt "}read everything{\tt "}.

         [The value is prompted for.]
      }
      \sstsubsection{
         XOPT = \_CHAR (Read and Write)
      }{
         A string that controls the style of the X-axis labelling and
         tick marks. It consists of a series of letters, which are
         described fully in the documentation for the BOXFRAME
         application.

         READF updates the value of the global parameters PONGO\_XOPT.
         The application will automatically remove any {\tt "}L{\tt "} characters
         at the start of the options string, because it is assumed
         that they have been inserted by the CLOG application -- any
         new data will not have had logarithms taken.  If data are
         given in logarithmic form, the {\tt "}L{\tt "} character should be
         inserted into the options strings anywhere except at the
         start.

         [The value of the global parameter PONGO\_XOPT is used. If
         PONGO\_XOPT is not defined, the default value {\tt "}BCNST{\tt "} is used.]
      }
      \sstsubsection{
         YOPT = \_CHAR (Read and Write)
      }{
         A string that controls the style of the Y-axis labelling and
         tick marks. It consists of a series of letters, which are
         described fully in the documentation for the BOXFRAME
         application.

         READF updates the value of the global parameters PONGO\_YOPT.
         The application will automatically remove any {\tt "}L{\tt "} characters
         at the start of the options string, because it is assumed
         that they have been inserted by the CLOG application -- any
         new data will not have had logarithms taken.  If data are
         given in logarithmic form, the {\tt "}L{\tt "} character should be
         inserted into the options strings anywhere except at the
         start.

         [The value of the global parameter PONGO\_YOPT is used. If
         PONGO\_XOPT is not defined, the default value {\tt "}BCNST{\tt "} is used.]
      }
      \sstsubsection{
         ERSCALE = \_REAL (Read)
      }{
         The scale factor to be applied to the EXCOL and EYCOL data.

         [The value of the global parameter PONGO\_ERSCALE is used. If
         PONGO\_ERSCALE is not defined, the default value 1.0 is used.]
      }
      \sstsubsection{
         ADD = \_LOGICAL (Read)
      }{
         If FALSE, the data values already held will be cleared before
         reading new data; if TRUE, the data read will be appended to
         the existing data.
         [FALSE]
      }
      \sstsubsection{
         ALL = \_LOGICAL (Read and Write)
      }{
         If TRUE, the whole data file will be read; if FALSE, a
         selection condition will be prompted for.

         If the value is not specified on the command line, the current
         value is used. The current value is initially set to TRUE.
      }
      \sstsubsection{
         QUICK = \_LOGICAL (Read and Write)
      }{
         If TRUE, a {\tt "}quick mode{\tt "} read is performed. This mode can only
         be used on files which exclusively contain numeric data. This
         parameter can over-ride the action of the LABCOL and SELCOND
         parameters.

         If the value is not specified on the command line, the current
         value is used. The current value is initially set to FALSE.
      }
      \sstsubsection{
         NDATA = \_INTEGER (Write)
      }{
         The number of data read from the data file.
         [0]
      }
   }
}
\sstroutine{
   RESETPONGO
}{
   Reset the state of PONGO.
}{
   \sstdescription{
      The state of PONGO (global parameter values, AGI and PGPLOT) is returned
      to a default state: all global parameters are reset to their default
      values, the AGI database for the graphics device is cleared,  and the
      viewport is reset to the standard PGPLOT viewport.  The global parameter
      values for the PONGO data file (PONGO\_DATA) and the Chebyshev polynomial
      coefficients (PONGO\_CHEBCOEF) are not reset.

      This command is an ICL hidden procedure using the SETGLOBAL command and
      the CHANGE, CLEAR, VIEWPORT and WORLD applications.
   }
}
\sstroutine{
   RTODEG
}{
   Convert the specified data area from radians to degrees
}{
   \sstdescription{
      The values in the specified data area are converted from radians
      to degrees.

      This command is an ICL hidden procedure which uses the CCMATH
      application.
   }
   \sstparameters{
      \sstsubsection{
         COLUMN = \_CHAR (Read and Write)
      }{
         The column to be converted from radians to degrees. []
      }
   }
}
\sstroutine{
   SHOWPONGO
}{
   Show the PONGO global parameters
}{
   \sstdescription{
      The PONGO global parameters are listed with their current values
      using the ICL command GETGLOBAL. It does not list the Chebyshev
      polynomial coefficients, PONGO\_CHEBCOEF.

      This command is an ICL hidden procedure using the GETGLOBAL command.
   }
}
\sstroutine{
   SIZEPLOT
}{
   Draw point marks of differing sizes at each of the data points
}{
   \sstdescription{
      Draw symbols of differing sizes at each of the data points.
      This application uses the values stored in the ZCOL data area to
      determine the size of each plotted symbol; i.e. the value of each
      entry in the ZCOL data area is effectively used as an argument to
      a CHANGE CHEIGHT command before each point is plotted. The SCALE
      parameter can be used to make these values cover a reasonable
      range by multiplying the Z data values.

      This command is a synonym for GPOINTS S.
   }
   \sstparameters{
      \sstsubsection{
         SYMBOL = \_INTEGER (Read and Write)
      }{
         The PGPLOT symbol number that is used to mark the data points.

         If a value is specified on the command line, it will be used
         for plotting symbols for all the data. If no value is
         specified on the command line, the application attempts to use
         the SYMCOL data for its symbols. If no symbol values have been
         read into the SYMCOL data area, the current value is used for
         all the data. The current value is initially set to 1.
      }
      \sstsubsection{
         SCALE = \_REAL (Read and Write)
      }{
         The scale factor used to multiply the ZCOL data values to get
         a reasonable range of symbol sizes.

         If the value is not specified on the command line, the current
         value is used. The current value is initially set to 1.0.
      }
      \sstsubsection{
         PROJECTION = \_CHAR (Read)
      }{
         Specifies the geometry that is to be used to plot the data.
         This is explained in more detail in the section on
         projections.  Allowed values: {\tt "}NONE{\tt "}, {\tt "}TAN{\tt "}, {\tt "}SIN{\tt "}, {\tt "}ARC{\tt "},
         {\tt "}GLS{\tt "}, {\tt "}AITOFF{\tt "}, {\tt "}MERCATOR{\tt "} and {\tt "}STG{\tt "}.

         This parameter is not specified on the command line. The value
         of the global parameter PONGO\_PROJECTN is used. If
         PONGO\_PROJECTN is not defined, the default value {\tt "}NONE{\tt "} is
         used.
      }
      \sstsubsection{
         RACENTRE = \_CHAR (Read)
      }{
         The centre of the projection in RA (i.e. the angle must be
         specified as hh:mm:ss.sss). This parameter is only required for
         PROJECTION values other than {\tt "}NONE{\tt "}.

         This parameter is not specified on the command line. The value
         of the global parameter PONGO\_RACENTRE is used. If
         PONGO\_RACENTRE is not defined, the default value {\tt "}0{\tt "} is used.
      }
      \sstsubsection{
         DECCENTRE = \_CHAR (Read)
      }{
         The centre of the projection in declination (i.e. the angle
         must be specified as dd:mm:ss.sss). This parameter is only
         required for PROJECTION values other than {\tt "}NONE{\tt "}.

         This parameter is not specified on the command line. The value
         of the global parameter PONGO\_DECCENTRE is used. If
         PONGO\_DECCENTRE is not defined, the default value {\tt "}0{\tt "} is used.
      }
   }
}
\sstroutine{
   SYMCOLUMN
}{
   Specify the column used for symbol codes
}{
   \sstdescription{
      Specify the column in the data file from which the symbol codes
      are to be read.

      This command is a synonym for SETGLOBAL PONGO\_SYMCOL.
   }
   \sstparameters{
      \sstsubsection{
         SYMCOL = \_CHAR (Read and Write)
      }{
         The column number (counting from 1), or the symbolic name of a
         column, from which the PGPLOT symbol code for each data point
         is read by the READF command. The value {\tt "}0{\tt "} means {\tt "}do not read
         these data{\tt "}. []
      }
   }
}
\sstroutine{
   TEXT
}{
   Draw a text string on the plot at the specified position
}{
   \sstdescription{
      Draw a text string on the plot at coordinates (x,y), using
      the PGPLOT routine PGTEXT.

      This command is a synonym for WTEXT S.
   }
   \sstparameters{
      \sstsubsection{
         XPOS = \_REAL (Read and Write)
      }{
         The X coordinate of the text.

         [The value is prompted for.]
      }
      \sstsubsection{
         YPOS = \_REAL (Read and Write)
      }{
         The Y coordinate of the text.

         [The value is prompted for.]
      }
      \sstsubsection{
         TEXT = \_CHAR (Read and Write)
      }{
         The text string to be plotted. This may include any of the
         PGPLOT control sequences for producing special characters.

         [The value is prompted for.]
      }
      \sstsubsection{
         JUSTIFICATION = \_REAL (Read and Write)
      }{
         The justification about the specified point (in the range 0.0
         to 1.0).  Here, 0.0 means left justify the text relative to
         the data point, 1.0 means right justify the text relative to
         the data point, 0.5 means centre the string on the data point,
         other values will give intermediate justifications.

         If no value is specified on the command line, the current
         value is used. The current value is initially set to 0.0.
      }
   }
}
\sstroutine{
   VECT
}{
   Draw vectors from each data point
}{
   \sstdescription{
      Use the values in the EXCOL and EYCOL data areas as signed
      offsets in X and Y to plot vectors from the data points. These
      vectors are scaled using the error scale parameter ERSCALE.
      Individual vectors may also be scaled using the contents of the
      ZCOL data area.
   }
   \sstparameters{
      \sstsubsection{
         ERSCALE = \_REAL (Read and Write)
      }{
         Factor for scaling all vectors.

         [The value of the global parameter PONGO\_ERSCALE is used. If
         PONGO\_ERSCALE is not defined, the default value 1.0 is used.]
      }
      \sstsubsection{
         ZMULT = \_LOGICAL (Read)
      }{
         If TRUE, the values of the ZCOL data area are used as
         multipliers for the vectors. This parameter may only be
         specified on the command line.
         [FALSE]
      }
   }
}
\sstroutine{
   VIEWPORT
}{
   Set the viewport for the current plotting device
}{
   \sstdescription{
      Control the PGPLOT viewport. The viewport is the region of the
      plotting surface through which the graph is seen.
   }
   \sstparameters{
      \sstsubsection{
         ACTION = \_CHAR (Read)
      }{
         The method used to set the viewport. It may be one of the
         following:

         \sstitemlist{

            \sstitem
               {\tt "}STANDARD{\tt "} -- The viewport is set to the standard PGPLOT
               viewport.

            \sstitem
               {\tt "}ADJUST{\tt "} -- The viewport is adjusted so that the scales
               along the X and Y axes are the same number of world
               coordinate units per unit length. The newly created
               viewport fits within the old viewport.

            \sstitem
               {\tt "}NDC{\tt "} -- The viewport is set by specifying its extent in
               the X and Y directions in terms of normalised device
               coordinates (i.e. coordinates that run from 0 to 1 along
               the horizontal and vertical directions).

            \sstitem
               {\tt "}INCHES{\tt "} -- The viewport is set by specifying its extent
               in the X and Y directions in terms of inches.

         }
         [The value is prompted for.]
      }
      \sstsubsection{
         XVPMIN = \_REAL (Read and Write)
      }{
         The left hand side of the viewport.

         If the value is not specified on the command line, the current
         value is used. The current value is initially set to 0.0.
      }
      \sstsubsection{
         XVPMAX = \_REAL (Read and Write)
      }{
         The right hand side of the viewport.

         If the value is not specified on the command line, the current
         value is used. The current value is initially set to 1.0.
      }
      \sstsubsection{
         YVPMIN = \_REAL (Read and Write)
      }{
         The lower side of the viewport.

         If the value is not specified on the command line, the current
         value is used. The current value is initially set to 0.0.
      }
      \sstsubsection{
         YVPMAX = \_REAL (Read and Write)
      }{
         The upper side of the viewport.

         If the value is not specified on the command line, the current
         value is used. The current value is initially set to 1.0.
      }
      \sstsubsection{
         XMIN = \_REAL (Read)
      }{
         The left hand edge of the world coordinate system.

         [The value of the global parameter PONGO\_XMIN is used. If
         PONGO\_XMIN is not defined, the default value 0.0 is used.]
      }
      \sstsubsection{
         XMAX = \_REAL (Read)
      }{
         The right hand edge of the world coordinate system.

         [The value of the global parameter PONGO\_XMAX is used. If
         PONGO\_XMAX is not defined, the default value 1.0 is used.]
      }
      \sstsubsection{
         YMIN = \_REAL (Read)
      }{
         The lower edge of the world coordinate system.

         [The value of the global parameter PONGO\_YMIN is used. If
         PONGO\_YMIN is not defined, the default value 0.0 is used.]
      }
      \sstsubsection{
         YMAX = \_REAL (Read)
      }{
         The upper edge of the world coordinate system.

         [The value of the global parameter PONGO\_YMAX is used. If
         PONGO\_YMIN is not defined, the default value 1.0 is used.]
      }
   }
}
\sstroutine{
   VPORT
}{
   Set the viewport using normalised device coordinates
}{
   \sstdescription{
      The viewport is set by specifying its extent in the X and Y
      directions in terms of normalised device coordinates (i.e.
      coordinates that run from 0 to 1 along the horizontal and
      vertical directions).

      This command is a synonym for VIEWPORT NDC.
   }
   \sstparameters{
      \sstsubsection{
         XVPMIN = \_REAL (Read and Write)
      }{
         The left hand side of the viewport.

         If the value is not specified on the command line, the current
         value is used. The current value is initially set to 0.0.
      }
      \sstsubsection{
         XVPMAX = \_REAL (Read and Write)
      }{
         The right hand side of the viewport.

         If the value is not specified on the command line, the current
         value is used. The current value is initially set to 1.0.
      }
      \sstsubsection{
         YVPMIN = \_REAL (Read and Write)
      }{
         The lower side of the viewport.

         If the value is not specified on the command line, the current
         value is used. The current value is initially set to 0.0.
      }
      \sstsubsection{
         YVPMAX = \_REAL (Read and Write)
      }{
         The upper side of the viewport.

         If the value is not specified on the command line, the current
         value is used. The current value is initially set to 1.0.
      }
      \sstsubsection{
         XMIN = \_REAL (Read)
      }{
         The left hand edge of the world coordinate system.

         The value of the global parameter PONGO\_XMIN is used. If
         PONGO\_XMIN is not defined, the default value 0.0 is used.
      }
      \sstsubsection{
         XMAX = \_REAL (Read)
      }{
         The right hand edge of the world coordinate system.

         The value of the global parameter PONGO\_XMAX is used. If
         PONGO\_XMAX is not defined, the default value 1.0 is used.
      }
      \sstsubsection{
         YMIN = \_REAL (Read)
      }{
         The lower edge of the world coordinate system.

         The value of the global parameter PONGO\_YMIN is used. If
         PONGO\_YMIN is not defined, the default value 0.0 is used.
      }
      \sstsubsection{
         YMAX = \_REAL (Read)
      }{
         The upper edge of the world coordinate system.

         The value of the global parameter PONGO\_YMAX is used. If
         PONGO\_YMIN is not defined, the default value 1.0 is used.
      }
   }
}
\sstroutine{
   VP\_BH
}{
   Set the viewport to the bottom half of the plotting surface
}{
   \sstdescription{
      Set the viewport to the bottom half of the plotting surface.

      This command is a synonym for VIEWPORT NDC 0.0833 0.917 0.05 0.45.
   }
   \sstparameters{
      \sstsubsection{
         XMIN = \_REAL (Read)
      }{
         The left hand edge of the world coordinate system.

         The value of the global parameter PONGO\_XMIN is used. If
         PONGO\_XMIN is not defined, the default value 0.0 is used.
      }
      \sstsubsection{
         XMAX = \_REAL (Read)
      }{
         The right hand edge of the world coordinate system.

         The value of the global parameter PONGO\_XMAX is used. If
         PONGO\_XMAX is not defined, the default value 1.0 is used.
      }
      \sstsubsection{
         YMIN = \_REAL (Read)
      }{
         The lower edge of the world coordinate system.

         The value of the global parameter PONGO\_YMIN is used. If
         PONGO\_YMIN is not defined, the default value 0.0 is used.
      }
      \sstsubsection{
         YMAX = \_REAL (Read)
      }{
         The upper edge of the world coordinate system.

         The value of the global parameter PONGO\_YMAX is used. If
         PONGO\_YMIN is not defined, the default value 1.0 is used.
      }
   }
}
\sstroutine{
   VP\_BL
}{
   Set the viewport to the bottom-left quarter of the plotting
   surface
}{
   \sstdescription{
      Set the viewport to the bottom left quarter of the plotting
      surface.

      This command is a synonym for VIEWPORT NDC 0.0417 0.459 0.05 0.45.
   }
   \sstparameters{
      \sstsubsection{
         XMIN = \_REAL (Read)
      }{
         The left hand edge of the world coordinate system.

         The value of the global parameter PONGO\_XMIN is used. If
         PONGO\_XMIN is not defined, the default value 0.0 is used.
      }
      \sstsubsection{
         XMAX = \_REAL (Read)
      }{
         The right hand edge of the world coordinate system.

         The value of the global parameter PONGO\_XMAX is used. If
         PONGO\_XMAX is not defined, the default value 1.0 is used.
      }
      \sstsubsection{
         YMIN = \_REAL (Read)
      }{
         The lower edge of the world coordinate system.

         The value of the global parameter PONGO\_YMIN is used. If
         PONGO\_YMIN is not defined, the default value 0.0 is used.
      }
      \sstsubsection{
         YMAX = \_REAL (Read)
      }{
         The upper edge of the world coordinate system.

         The value of the global parameter PONGO\_YMAX is used. If
         PONGO\_YMIN is not defined, the default value 1.0 is used.
      }
   }
}
\sstroutine{
   VP\_BR
}{
   Set the viewport to the bottom-right quarter of the plotting
   surface
}{
   \sstdescription{
      Set the viewport to the bottom right quarter of the plotting
      surface.

      This command is a synonym for VIEWPORT NDC 0.5417 0.959 0.05 0.45.
   }
   \sstparameters{
      \sstsubsection{
         XMIN = \_REAL (Read)
      }{
         The left hand edge of the world coordinate system.

         The value of the global parameter PONGO\_XMIN is used. If
         PONGO\_XMIN is not defined, the default value 0.0 is used.
      }
      \sstsubsection{
         XMAX = \_REAL (Read)
      }{
         The right hand edge of the world coordinate system.

         The value of the global parameter PONGO\_XMAX is used. If
         PONGO\_XMAX is not defined, the default value 1.0 is used.
      }
      \sstsubsection{
         YMIN = \_REAL (Read)
      }{
         The lower edge of the world coordinate system.

         The value of the global parameter PONGO\_YMIN is used. If
         PONGO\_YMIN is not defined, the default value 0.0 is used.
      }
      \sstsubsection{
         YMAX = \_REAL (Read)
      }{
         The upper edge of the world coordinate system.

         The value of the global parameter PONGO\_YMAX is used. If
         PONGO\_YMIN is not defined, the default value 1.0 is used.
      }
   }
}
\sstroutine{
   VP\_TH
}{
   Set the viewport to the top half of the plotting surface
}{
   \sstdescription{
      Set the viewport to the top half of the plotting surface.

      This command is a synonym for VIEWPORT NDC 0.0833 0.917 0.55 0.95.
   }
   \sstparameters{
      \sstsubsection{
         XMIN = \_REAL (Read)
      }{
         The left hand edge of the world coordinate system.

         The value of the global parameter PONGO\_XMIN is used. If
         PONGO\_XMIN is not defined, the default value 0.0 is used.
      }
      \sstsubsection{
         XMAX = \_REAL (Read)
      }{
         The right hand edge of the world coordinate system.

         The value of the global parameter PONGO\_XMAX is used. If
         PONGO\_XMAX is not defined, the default value 1.0 is used.
      }
      \sstsubsection{
         YMIN = \_REAL (Read)
      }{
         The lower edge of the world coordinate system.

         The value of the global parameter PONGO\_YMIN is used. If
         PONGO\_YMIN is not defined, the default value 0.0 is used.
      }
      \sstsubsection{
         YMAX = \_REAL (Read)
      }{
         The upper edge of the world coordinate system.

         The value of the global parameter PONGO\_YMAX is used. If
         PONGO\_YMIN is not defined, the default value 1.0 is used.
      }
   }
}
\sstroutine{
   VP\_TL
}{
   Set the viewport to the top-left quarter of the plotting surface
}{
   \sstdescription{
      Set the viewport to the top left quarter of the plotting surface.

      This command is a synonym for VIEWPORT NDC 0.0417 0.459 0.55 0.95.
   }
   \sstparameters{
      \sstsubsection{
         XMIN = \_REAL (Read)
      }{
         The left hand edge of the world coordinate system.

         The value of the global parameter PONGO\_XMIN is used. If
         PONGO\_XMIN is not defined, the default value 0.0 is used.
      }
      \sstsubsection{
         XMAX = \_REAL (Read)
      }{
         The right hand edge of the world coordinate system.

         The value of the global parameter PONGO\_XMAX is used. If
         PONGO\_XMAX is not defined, the default value 1.0 is used.
      }
      \sstsubsection{
         YMIN = \_REAL (Read)
      }{
         The lower edge of the world coordinate system.

         The value of the global parameter PONGO\_YMIN is used. If
         PONGO\_YMIN is not defined, the default value 0.0 is used.
      }
      \sstsubsection{
         YMAX = \_REAL (Read)
      }{
         The upper edge of the world coordinate system.

         The value of the global parameter PONGO\_YMAX is used. If
         PONGO\_YMIN is not defined, the default value 1.0 is used.
      }
   }
}
\sstroutine{
   VP\_TR
}{
   Set the viewport to the top-right quarter of the plotting surface
}{
   \sstdescription{
      Set the viewport to the top right quarter of the plotting surface.

      This command is a synonym for VIEWPORT NDC 0.5417 0.959 0.55 0.95.
   }
   \sstparameters{
      \sstsubsection{
         XMIN = \_REAL (Read)
      }{
         The left hand edge of the world coordinate system.

         The value of the global parameter PONGO\_XMIN is used. If
         PONGO\_XMIN is not defined, the default value 0.0 is used.
      }
      \sstsubsection{
         XMAX = \_REAL (Read)
      }{
         The right hand edge of the world coordinate system.

         The value of the global parameter PONGO\_XMAX is used. If
         PONGO\_XMAX is not defined, the default value 1.0 is used.
      }
      \sstsubsection{
         YMIN = \_REAL (Read)
      }{
         The lower edge of the world coordinate system.

         The value of the global parameter PONGO\_YMIN is used. If
         PONGO\_YMIN is not defined, the default value 0.0 is used.
      }
      \sstsubsection{
         YMAX = \_REAL (Read)
      }{
         The upper edge of the world coordinate system.

         The value of the global parameter PONGO\_YMAX is used. If
         PONGO\_YMIN is not defined, the default value 1.0 is used.
      }
   }
}
\sstroutine{
   VSIZE
}{
   Set the viewport using its physical size in inches
}{
   \sstdescription{
      The viewport is set by specifying its minimum and maximum extents
      in the X and Y directions in terms of inches.

      This command is a synonym for VIEWPORT INCHES.
   }
   \sstparameters{
      \sstsubsection{
         XVPMIN = \_REAL (Read and Write)
      }{
         The left hand side of the viewport.

         If the value is not specified on the command line, the current
         value is used. The current value is initially set to 0.0.
      }
      \sstsubsection{
         XVPMAX = \_REAL (Read and Write)
      }{
         The right hand side of the viewport.

         If the value is not specified on the command line, the current
         value is used. The current value is initially set to 1.0.
      }
      \sstsubsection{
         YVPMIN = \_REAL (Read and Write)
      }{
         The lower side of the viewport.

         If the value is not specified on the command line, the current
         value is used. The current value is initially set to 0.0.
      }
      \sstsubsection{
         YVPMAX = \_REAL (Read and Write)
      }{
         The upper side of the viewport.

         If the value is not specified on the command line, the current
         value is used. The current value is initially set to 1.0.
      }
      \sstsubsection{
         XMIN = \_REAL (Read)
      }{
         The left hand edge of the world coordinate system.

         The value of the global parameter PONGO\_XMIN is used. If
         PONGO\_XMIN is not defined, the default value 0.0 is used.
      }
      \sstsubsection{
         XMAX = \_REAL (Read)
      }{
         The right hand edge of the world coordinate system.

         The value of the global parameter PONGO\_XMAX is used. If
         PONGO\_XMAX is not defined, the default value 1.0 is used.
      }
      \sstsubsection{
         YMIN = \_REAL (Read)
      }{
         The lower edge of the world coordinate system.

         The value of the global parameter PONGO\_YMIN is used. If
         PONGO\_YMIN is not defined, the default value 0.0 is used.
      }
      \sstsubsection{
         YMAX = \_REAL (Read)
      }{
         The upper edge of the world coordinate system.

         The value of the global parameter PONGO\_YMAX is used. If
         PONGO\_YMIN is not defined, the default value 1.0 is used.
      }
   }
}
\sstroutine{
   VSTAND
}{
   Set the standard viewport
}{
   \sstdescription{
      The viewport is set to the standard PGPLOT viewport.

      This command is a synonym for VIEWPORT STANDARD.
   }
   \sstparameters{
      \sstsubsection{
         XMIN = \_REAL (Read)
      }{
         The left hand edge of the world coordinate system.

         The value of the global parameter PONGO\_XMIN is used. If
         PONGO\_XMIN is not defined, the default value 0.0 is used.
      }
      \sstsubsection{
         XMAX = \_REAL (Read)
      }{
         The right hand edge of the world coordinate system.

         The value of the global parameter PONGO\_XMAX is used. If
         PONGO\_XMAX is not defined, the default value 1.0 is used.
      }
      \sstsubsection{
         YMIN = \_REAL (Read)
      }{
         The lower edge of the world coordinate system.

         The value of the global parameter PONGO\_YMIN is used. If
         PONGO\_YMIN is not defined, the default value 0.0 is used.
      }
      \sstsubsection{
         YMAX = \_REAL (Read)
      }{
         The upper edge of the world coordinate system.

         The value of the global parameter PONGO\_YMAX is used. If
         PONGO\_YMIN is not defined, the default value 1.0 is used.
      }
   }
}
\sstroutine{
   WNAD
}{
   Adjust the viewport so that the X and Y scales are the same
}{
   \sstdescription{
      The viewport is adjusted so that the scales along the X and Y
      axes are the same number of world coordinate units per unit
      length. The newly created viewport fits within the old viewport.

      This command is a synonym for VIEWPORT ADJUST.
   }
   \sstparameters{
      \sstsubsection{
         XVPMIN = \_REAL (Read and Write)
      }{
         The left hand side of the viewport.

         If the value is not specified on the command line, the current
         value is used. The current value is initially set to 0.0.
      }
      \sstsubsection{
         XVPMAX = \_REAL (Read and Write)
      }{
         The right hand side of the viewport.

         If the value is not specified on the command line, the current
         value is used. The current value is initially set to 1.0.
      }
      \sstsubsection{
         YVPMIN = \_REAL (Read and Write)
      }{
         The lower side of the viewport.

         If the value is not specified on the command line, the current
         value is used. The current value is initially set to 0.0.
      }
      \sstsubsection{
         YVPMAX = \_REAL (Read and Write)
      }{
         The upper side of the viewport.

         If the value is not specified on the command line, the current
         value is used. The current value is initially set to 1.0.
      }
      \sstsubsection{
         XMIN = \_REAL (Read)
      }{
         The left hand edge of the world coordinate system.

         The value of the global parameter PONGO\_XMIN is used. If
         PONGO\_XMIN is not defined, the default value 0.0 is used.
      }
      \sstsubsection{
         XMAX = \_REAL (Read)
      }{
         The right hand edge of the world coordinate system.

         The value of the global parameter PONGO\_XMAX is used. If
         PONGO\_XMAX is not defined, the default value 1.0 is used.
      }
      \sstsubsection{
         YMIN = \_REAL (Read)
      }{
         The lower edge of the world coordinate system.

         The value of the global parameter PONGO\_YMIN is used. If
         PONGO\_YMIN is not defined, the default value 0.0 is used.
      }
      \sstsubsection{
         YMAX = \_REAL (Read)
      }{
         The upper edge of the world coordinate system.

         The value of the global parameter PONGO\_YMAX is used. If
         PONGO\_YMIN is not defined, the default value 1.0 is used.
      }
   }
}
\sstroutine{
   WORLD
}{
   Set the world coordinates for the plot
}{
   \sstdescription{
      Set up the world coordinate limits for the plot. The world
      coordinate system is the one in which the data are plotted. It
      is possible to specify the limits explicitly, or to have them
      calculated from the range of the data that have been read, or to
      recall the limits from a previous plot using AGI (SUN/48).
   }
   \sstparameters{
      \sstsubsection{
         ACTION = \_CHAR (Read and Write)
      }{
         The way in which the world coordinate limits are to be
         determined:
         \sstitemlist{

            \sstitem
               {\tt "}DATA{\tt "} -- The limits are calculated from the data limits,
               with a small border added.

            \sstitem
               {\tt "}DATA0{\tt "} -- The limits are calculated as for DATA, but the
               origin is included as one of the data points.

            \sstitem
               {\tt "}GIVEN{\tt "} -- The limits specified on the command line are
               used.

            \sstitem
               {\tt "}RECALL{\tt "} -- The coordinates are recalled from a previous
               plot using the AGI database.

         }
         [The value is prompted for.]
      }
      \sstsubsection{
         XMIN = \_REAL (Read and Write)
      }{
         The world coordinate of the left-hand edge of the plot.

         The application will determine the value if ACTION is one of
         {\tt "}DATA{\tt "}, {\tt "}DATA0{\tt "} or {\tt "}RECALL{\tt "}. If ACTION={\tt "}GIVEN{\tt "} and no value is
         specified on the command line, the value of the global
         parameter PONGO\_XMIN is used.
      }
      \sstsubsection{
         XMAX = \_REAL (Read and Write)
      }{
         The world coordinate of the right-hand edge of the plot.

         The application will determine the value if ACTION is one of
         {\tt "}DATA{\tt "}, {\tt "}DATA0{\tt "} or {\tt "}RECALL{\tt "}. If ACTION={\tt "}GIVEN{\tt "} and no value is
         specified on the command line, the value of the global
         parameter PONGO\_XMAX is used.
      }
      \sstsubsection{
         YMIN = \_REAL (Read and Write)
      }{
         The world coordinate of the lower edge of the plot.

         The application will determine the value if ACTION is one of
         {\tt "}DATA{\tt "}, {\tt "}DATA0{\tt "} or {\tt "}RECALL{\tt "}. If ACTION={\tt "}GIVEN{\tt "} and no value is
         specified on the command line, the value of the global
         parameter PONGO\_YMIN is used.
      }
      \sstsubsection{
         YMAX = \_REAL (Read and Write)
      }{
         The world coordinate of the upper edge of the plot.

         The application will determine the value if ACTION is one of
         {\tt "}DATA{\tt "}, {\tt "}DATA0{\tt "} or {\tt "}RECALL{\tt "}. If ACTION={\tt "}GIVEN{\tt "} and no value is
         specified on the command line, the value of the global
         parameter PONGO\_YMAX is used.
      }
      \sstsubsection{
         PICLAB = \_CHAR (Read and Write)
      }{
         The AGI label of the picture to be recalled.

         [If ACTION={\tt "}RECALL{\tt "}, the value is prompted for.]
      }
      \sstsubsection{
         PROJECTION = \_CHAR (Read and Write)
      }{
         The geometry to be used for plotting the data.  This is
         explained in more detail in the section on projections.
         Allowed values: {\tt "}NONE{\tt "}, {\tt "}TAN{\tt "}, {\tt "}SIN{\tt "}, {\tt "}ARC{\tt "}, {\tt "}GLS{\tt "}, {\tt "}AITOFF{\tt "},
         {\tt "}MERCATOR{\tt "} and {\tt "}STG{\tt "}.

         [The value of the global parameter PONGO\_PROJECTN is used. If
         PONGO\_PROJECTN is not defined, the default value {\tt "}NONE{\tt "} is
         used.]
      }
      \sstsubsection{
         RACENTRE = \_CHAR (Read and Write)
      }{
         The centre of the projection in RA (i.e. the angle must be
         specified as hh:mm:ss.sss). This parameter is only required for
         PROJECTION values other than {\tt "}NONE{\tt "}.

         [The value of the global parameter PONGO\_RACENTRE is used. If
         PONGO\_RACENTRE is not defined, the default value {\tt "}0{\tt "} is used.]
      }
      \sstsubsection{
         DECCENTRE = \_CHAR (Read and Write)
      }{
         The centre of the projection in declination (i.e. the angle
         must be specified as dd:mm:ss.sss). This parameter is only
         required for PROJECTION values other than {\tt "}NONE{\tt "}.

         [The value of the global parameter PONGO\_DECCENTRE is used. If
         PONGO\_DECCENTRE is not defined, the default value {\tt "}0{\tt "} is used.]
      }
   }
}
\sstroutine{
   WRITEI
}{
   Write information to an output file
}{
   \sstdescription{
      Write specified information concerning the current data-set to an
      output file.
   }
   \sstparameters{
      \sstsubsection{
         ACTION = \_CHAR (Read and Write)
      }{
         The type of information to be written. This may be one of the
         following:
         \sstitemlist{

            \sstitem
               {\tt "}LABLST{\tt "} -- Write the internal list of labels out.

            \sstitem
               {\tt "}DATA{\tt "} -- Write out selected data.

            \sstitem
               {\tt "}AGIPIC{\tt "} -- Write the label, name and comment for the
               current AGI picture to the AGI database.

         }
         [The value is prompted for.]
      }
      \sstsubsection{
         FILE = FILENAME (Read and Write)
      }{
         The name of the output file to be written.

         [The value is prompted for.]
      }
      \sstsubsection{
         FORMAT = \_CHAR (Read and Write)
      }{
         The Fortran FORMAT to be used.

         If the value is not specified on the command line, the current
         value is used. The current value is initially set to {\tt "}G25.16{\tt "}.
      }
      \sstsubsection{
         AGINAME = \_CHAR (Read and Write)
      }{
         The AGI name for the current picture. This may be one of the
         following:
         \sstitemlist{

            \sstitem
               {\tt "}DATA{\tt "} -- Used to indicate that the AGI picture contains
               the representation of data in some graphical form (i.e. a
               graph).

            \sstitem
               {\tt "}FRAME{\tt "} -- Used to indicate that the AGI picture contains
               a group of other plots (i.e. several {\tt "}DATA{\tt "} pictures).

         }
         [{\tt "}DATA{\tt "}]
      }
      \sstsubsection{
         AGICOMMENT = \_CHAR (Read and Write)
      }{
         The AGI comment for the current picture.

         If the value is not specified on the command line, the current
         value is used. The current value is set to {\tt "}User viewport{\tt "}.
      }
      \sstsubsection{
         AGILABEL = \_CHAR (Read and Write)
      }{
         The AGI label for the current picture.

         [The value is prompted for.]
      }
      \sstsubsection{
         X = \_LOGICAL (Read and Write)
      }{
         If TRUE, the XCOL data area will be output.
         [FALSE]
      }
      \sstsubsection{
         Y = \_LOGICAL (Read and Write)
      }{
         If TRUE, the YCOL data area will be output.
         [FALSE]
      }
      \sstsubsection{
         Z = \_LOGICAL (Read and Write)
      }{
         If TRUE, the ZCOL data area will be output.
         [FALSE]
      }
      \sstsubsection{
         EX = \_LOGICAL (Read and Write)
      }{
         If TRUE, the EXCOL data area will be output.
         [FALSE]
      }
      \sstsubsection{
         EY = \_LOGICAL (Read and Write)
      }{
         If TRUE, the EYCOL data area will be output.
         [FALSE]
      }
   }
}
\sstroutine{
   WTEXT
}{
   Draw a text string on the plot
}{
   \sstdescription{
      Draw a text string on the current plot at a given position,
      justification and orientation.
   }
   \sstparameters{
      \sstsubsection{
         ACTION = \_CHAR (Read and Write)
      }{
         The way in which the text string is to be written. It may be
         one of the following:
         \sstitemlist{

            \sstitem
               {\tt "}P{\tt "} -- Use PGPTEXT which allows the position,
               justification and angle of the text to be specified.

            \sstitem
               {\tt "}M{\tt "} -- Use PGMTEXT which allows the text to be written
               relative to the viewport.

            \sstitem
               {\tt "}S{\tt "} -- Use PGTEXT which allows only simple (x,y)
               positioning of the text.

         }
         [The value is prompted for.]
      }
      \sstsubsection{
         XPOS = \_REAL (Read and Write)
      }{
         If ACTION is {\tt "}P{\tt "} or {\tt "}S{\tt "}, the X coordinate of the text.  With
         the {\tt "}M{\tt "} action, this parameter specifies the number of
         character heights from the viewport where the text is to be
         plotted (negative values are allowed).

         [The value is prompted for.]
      }
      \sstsubsection{
         YPOS = \_REAL (Read and Write)
      }{
         If ACTION is {\tt "}P{\tt "} or {\tt "}S{\tt "}, the Y coordinate of the text.  With
         the {\tt "}M{\tt "} action, this parameter specifies the fraction along
         the edge where the text is to be plotted.

         [The value is prompted for.]
      }
      \sstsubsection{
         TEXT = \_CHAR (Read and Write)
      }{
         The text string to be plotted. This may include any of the
         PGPLOT control sequences for producing special characters.

         [The value is prompted for.]
      }
      \sstsubsection{
         SIDE = \_CHAR (Read and Write)
      }{
         If ACTION={\tt "}M{\tt "}, the side of the viewport where the text is to
         plotted. This may be one of the following:
         \sstitemlist{

            \sstitem
               {\tt "}T{\tt "} -- The top edge.

            \sstitem
               {\tt "}B{\tt "} -- The bottom edge.

            \sstitem
               {\tt "}L{\tt "} -- The left-hand edge.

            \sstitem
               {\tt "}R{\tt "} -- The right-hand edge.

            \sstitem
               {\tt "}LV{\tt "} -- The left-hand edge, but with the string written
               vertically.

            \sstitem
               {\tt "}RV{\tt "} -- The right-hand edge, but with the string written
               vertically.

         }
         If the value is not specified on the command line, the current
         value is used. The current value is initially set to {\tt "}T{\tt "}.
      }
      \sstsubsection{
         JUSTIFICATION = \_REAL (Read and Write)
      }{
         The justification about the specified point (in the range 0.0
         to 1.0).  Here, 0.0 means left justify the text relative to
         the data point, 1.0 means right justify the text relative to
         the data point, 0.5 means centre the string on the data point,
         other values will give intermediate justifications.

         If the value is not specified on the command line, the current
         value is used. The current value is initially set to 0.0.
      }
      \sstsubsection{
         ANGLE = \_REAL (Read and Write)
      }{
         If ACTION={\tt "}P{\tt "}, the angle relative to the horizontal at which
         the text string is to be plotted.

         If the value is not specified on the command line, the current
         value is used. The current value is initially set to 0.0.
      }
   }
}
\sstroutine{
   XCOLUMN
}{
   Specify the column containing the X-axis data
}{
   \sstdescription{
      Specify the column in the data file from which the X-axis data
      are to be read.

      This command is a synonym for SETGLOBAL PONGO\_XCOL.
   }
   \sstparameters{
      \sstsubsection{
         XCOL = \_CHAR (Read and Write)
      }{
         The column number (counting from 1), or the symbolic name of a
         column, from which the X-axis data are read by the READF
         command. The value {\tt "}0{\tt "} means {\tt "}do not read these data{\tt "}. []
      }
   }
}
\sstroutine{
   XERR
}{
   Draw symmetric error bars in the X direction
}{
   \sstdescription{
      Draw symmetric error bars in the X direction.

      This command is a synonym for ERRORBAR X.
   }
   \sstparameters{
      \sstsubsection{
         ERTERM = \_REAL (Read and Write)
      }{
         The length of the terminals on the error bars: a multiple of
         the default length.

         If the value is not specified on the command line, the current
         value is used. The current value is initially set to 1.0.
      }
   }
}
\sstroutine{
   XLINEAR
}{
   Put 1 ... N into the XCOL data area
}{
   \sstdescription{
      Put the row number (i.e. 1 to NDATA) into the XCOL data area.

      This command is a synonym for CCMATH X=INDEX.
   }
}
\sstroutine{
   XLOGARITHM
}{
   Take the logarithm of the X-axis data
}{
   \sstdescription{
      Take the base 10 logarithm of the X-axis data.

      This command is a synonym for CLOG X.
   }
}
\sstroutine{
   XOFFSET
}{
   Add a constant offset to the X-axis data
}{
   \sstdescription{
      Add a constant offset to the X-axis data.

      This command is an ICL hidden procedure which uses the CCMATH
      application.
   }
   \sstparameters{
      \sstsubsection{
         OFFSET = \_REAL (Read)
      }{
         The value of the X-axis offset. []
      }
   }
}
\sstroutine{
   XSCALE
}{
   Multiply the values in the XCOL and EXCOL data areas by a
   constant
}{
   \sstdescription{
      Multiply the values in the XCOL and EXCOL data areas by a
      constant.

      This command is an ICL hidden procedure which uses the CCMATH
      application.
   }
   \sstparameters{
      \sstsubsection{
         SCALE = \_REAL (Read)
      }{
         The constant by which the XCOL and EXCOL data areas are
         scaled. []
      }
   }
}
\sstroutine{
   YCOLUMN
}{
   Specify the column containing the Y-axis data
}{
   \sstdescription{
      Specify the column in the data file from which the Y-axis data
      are to be read.

      This command is a synonym for SETGLOBAL PONGO\_YCOL.
   }
   \sstparameters{
      \sstsubsection{
         YCOL = \_CHAR (Read and Write)
      }{
         The column number (counting from 1), or the symbolic name of a
         column, from which the Y-axis data are read. The value {\tt "}0{\tt "}
         means {\tt "}do not read these data{\tt "}. []
      }
   }
}
\sstroutine{
   YERR
}{
   Draw symmetric error bars in the Y direction
}{
   \sstdescription{
      Draw symmetric error bars in the Y direction.

      This command is a synonym for ERRORBAR Y.
   }
   \sstparameters{
      \sstsubsection{
         ERTERM = \_REAL (Read and Write)
      }{
         The length of the terminals on the error bars: a multiple of
         the default length.

         If no value is specified on the command line, the current
         value is used. The current value is initially set to 1.0.
      }
   }
}
\sstroutine{
   YLINEAR
}{
   Put 1 ... N into the YCOL data area
}{
   \sstdescription{
      Put the row number (i.e. 1 to NDATA) into the YCOL data area.

      This command is a synonym for CCMATH Y=INDEX.
   }
}
\sstroutine{
   YLOGARITHM
}{
   Take the logarithm of the Y-axis data
}{
   \sstdescription{
      Take the base 10 logarithm of the Y-axis data.

      This command is a synonym for CLOG Y.
   }
}
\sstroutine{
   YOFFSET
}{
   Add a constant offset to the Y-axis data
}{
   \sstdescription{
      Add a constant offset to the Y-axis data.

      This command is an ICL hidden procedure which uses the CCMATH
      application.
   }
   \sstparameters{
      \sstsubsection{
         YOFFSET = \_REAL (Read)
      }{
         The value of the Y-axis offset. []
      }
   }
}
\sstroutine{
   YSCALE
}{
   Multiply the values in the YCOL and EYCOL data areas by a
   constant
}{
   \sstdescription{
      Multiply the values in the YCOL and EYCOL data areas by a
      constant.

      This command is an ICL hidden procedure which uses the CCMATH
      application.
   }
   \sstparameters{
      \sstsubsection{
         SCALE = \_REAL (Read)
      }{
         The constant by which the YCOL and EYCOL data areas are
         scaled. []
      }
   }
}
\sstroutine{
   ZCOLUMN
}{
   Specify the column containing the Z-axis data
}{
   \sstdescription{
      Specify the column in the data file from which the Z-axis data
      are to be read.

      This command is a synonym for SETGLOBAL PONGO\_ZCOL.
   }
   \sstparameters{
      \sstsubsection{
         ZCOL = \_CHAR (Read and Write)
      }{
         The column number (counting from 1), or the symbolic name of a
         column, from which the Z-axis data are read. The value {\tt "}0{\tt "}
         means {\tt "}do not read these data{\tt "}. []
      }
   }
}
\sstroutine{
   ZSCALE
}{
   Multiply the values in the ZCOL data area by a constant
}{
   \sstdescription{
      Multiply the values in the ZCOL data area by a constant.

      This command is an ICL hidden procedure which uses the CCMATH
      application.
   }
   \sstparameters{
      \sstsubsection{
         SCALE = \_REAL (Read)
      }{
         The constant by which the ZCOL data area is scaled. []
      }
   }
}
\end{sloppypar}
\normalsize

\newpage
\section{PONGO and the MONGO Graphics Package}

PONGO has been designed to have broadly the same command interface as that
offered by the MONGO graphics package.
A number of aliases (ICL DEFSTRING definitions) have been set up for certain
commands so that PONGO will do more or less what is expected for the equivalent
MONGO command, but there are some substantial differences.
It might well be possible to do a better job of imitating MONGO using ICL
procedures; however, the aim is not to imitate the precise behaviour of
MONGO, but for existing users of MONGO to be met with a package that is not
totally unfamiliar to them when they begin using PONGO.
Ultimately, PONGO offers substantially more than MONGO, both by its flexibility
and by what it can do.

\small
\begin{table}
\begin{center}
\begin{tabular}{|l|p{0.65\textwidth}|}
\hline
& \\
Command    & Behavioural differences \\
& \\
\hline
& \\
BOX        & The optional arguments are different in the two cases. \\
CONNECT    & \\
DATA       & \\
DRAW       & \\
ERASE      & \\
ERRORBAR   & The way in which the errors are plotted is different.
\\
LTYPE      & The actual styles produced are different. \\
EXPAND     & In PONGO this will also alter sizes of tickmarks. \\
LWEIGHT    & \\
PCOLUMN    & The symbol numbers in the file refer to the standard
PGPLOT marker symbols. \\
PEN        & \\
POINTS     & PONGO has additional optional arguments. \\
XCOLUMN    & In PONGO a symbolic name can be used optionally for
the column description. \\
XLINEAR    & In PONGO no arguments can be given. The array is
always filled in an increasing integer sequence from one. It is
possible to perform any desired manipulation on the values in the
column using the \cnam{CCMATH} command. \\
XLOGARITHM & \\
YCOLUMN    & In PONGO a symbolic name can be used optionally for
the column description. \\
YLINEAR    & In PONGO no arguments can be given. The array is
always filled in an increasing integer sequence from one (see the description
of \cnam{XLINEAR}). \\
YLOGARITHM & \\
& \\ \hline
\end{tabular}
\end{center}
\caption{Commands which have nearly the same effect in MONGO and PONGO} 
\label{mongo_same_tab}
\end{table}
\normalsize

Having read the rest of this document it will have become clear that there
are a substantial number of differences between MONGO and PONGO: perhaps
most significantly, it is not possible to run a MONGO script file and expect it
to work.
The most important difference at the command level is that once \pnam{XCOL}
\etc\ have been set up, the data must be input explicitly using the
\cnam{READF} command, where MONGO would read the data in each time that a
command that needs them is executed.
{\em Table \ref{mongo_same_tab}} is a list of the commands that work as
expected (with the above proviso) and {\em Table \ref{mongo_diff_tab}} gives a
list of the closest equivalents to MONGO commands.

\small
\begin{table}
\begin{center}
\begin{tabular}{|l|l|p{0.5\textwidth}|}
\hline
& & \\
MONGO     & PONGO          & Behaviour \\
& & \\
\hline
& & \\
AXIS      & BOXFRAME       & Most effects achievable by can be  accomplished by
altering the parameters of the \cnam{BOXFRAME} command. \\
BADY      & READF          & Use the \pnam{COND} parameter to selectively 
read the file. \\
ECOLUMN   & EXCOLUMN       & The errors for the X and Y directions are read
separately in PONGO. \\
          & EYCOLUMN       & The errors for the X and Y directions are read
separately in PONGO. \\
GRID      & BOXFRAME       & A grid can be drawn by using \verb+G+ in 
\pnam{XOPT} (not to be confused with PONGO command \cnam{GRID}). \\
HARDCOPY  & & \\
HISTOGRAM & PLOTHIST       & Use the command \cnam{PLOTHIST B} (note that the
command \cnam{PLOTHIST H} will automatically bin unbinned data). \\
LABEL     & WTEXT          & PONGO is more flexible. \\
LINES     & READF          & Use the \pnam{FROM} and \pnam{TO} parameters. \\
PTYPE     & POINTS         & PONGO is restricted to the
standard PGPLOT symbols: although there are 32 of them, you cannot define your
own as you can in MONGO. \\
PUTLABEL  & WTEXT & \\
& & \\ \hline
\end{tabular}
\end{center}
\caption{PONGO command equivalents for certain MONGO commands} 
\label{mongo_diff_tab}
\end{table}

\normalsize

\end{document}
