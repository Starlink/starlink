\documentstyle{article} 
\pagestyle{myheadings}

%------------------------------------------------------------------------------
\newcommand{\stardoccategory}  {Starlink User Note}
\newcommand{\stardocinitials}  {SUN}
\newcommand{\stardocnumber}    {67.19}
\newcommand{\stardocauthors}   {P.\,T.\,Wallace}
\newcommand{\stardocdate}      {23rd March 1994}
\newcommand{\stardoctitle}     {SLALIB --- Positional Astronomy Library}
%------------------------------------------------------------------------------

\newcommand{\stardocname}{\stardocinitials /\stardocnumber}
\markright{\stardocname}
\setlength{\textwidth}{160mm}
\setlength{\textheight}{240mm}
\setlength{\topmargin}{-5mm}
\setlength{\oddsidemargin}{0mm}
\setlength{\evensidemargin}{0mm}
\setlength{\parindent}{0mm}
\setlength{\parskip}{\medskipamount}
\setlength{\unitlength}{1mm}

\begin{document}
\thispagestyle{empty}
SCIENCE \& ENGINEERING RESEARCH COUNCIL \hfill \stardocname\\
RUTHERFORD APPLETON LABORATORY\\
{\large\bf Starlink Project\\}
{\large\bf \stardoccategory\ \stardocnumber}
\begin{flushright}
\stardocauthors\\
\stardocdate
\end{flushright}
\vspace{-4mm}
\rule{\textwidth}{0.5mm}
\vspace{5mm}
\begin{center}
{\Large\bf \stardoctitle}
\end{center}
\vspace{5mm}
\setlength{\parskip}{0mm}
\tableofcontents
\setlength{\parskip}{\medskipamount}
\markright{\stardocname}

%------------------------------------------------------------------------------

\newcommand{\radec}     {$[\alpha,\delta\,]$}
\newcommand{\hadec}     {$[h,\delta\,]$}
\newcommand{\azel}      {$[Az,El\,]$}
\newcommand{\gal}       {$[l^{I\!I},b^{I\!I}]$}
\newcommand{\xy}        {$[x,y\,]$}
\newcommand{\xyz}       {$[x,y,z\,]$}
\newcommand{\xyzd}      {$[\dot{x},\dot{y},\dot{z}\,]$}
\newcommand{\xyzxyzd}   {$[x,y,z,\dot{x},\dot{y},\dot{z}\,]$}
\newcommand{\arcsec}    {$\hspace{-0.05em}\raisebox{-0.5ex}
                        {$^{'\hspace{-0.1em}'}$}
                        \hspace{-0.7em}.\hspace{-0.05em}$}
\newcommand{\tsec}      {\mbox{$^{\rm s}\!\!.$}}
\newcommand{\hms}[4]    {$#1^{\rm h}\,#2^{\rm m}\,#3\tsec#4$}
\newcommand{\dms}[4]    {$#1^{\circ}\,#2\raisebox{-0.5ex}{$^{'}$}\,#3\arcsec#4$}

\newcommand{\callhead}[1]{\goodbreak\vspace{\bigskipamount}{\large\bf#1}}
\newenvironment{callset}{\begin{list}{}{\setlength{\leftmargin}{2cm}
                             \setlength{\parsep}{\smallskipamount}}}{\end{list}}
\newcommand{\subp}[1]{\item\hspace{-1cm}#1\\}
\newcommand{\subq}[2]{\item\hspace{-1cm}#1\\\hspace*{-1cm}#2\\}
\newcommand{\name}[1]{\mbox{#1}}
\newcommand{\fortvar}[1]{\mbox{\em #1}}
\renewcommand{\_}{{\tt\char'137}}
\newcommand{\routine}[3]
{\hbadness=10000
  \vbox
  {
    \rule{\textwidth}{0.3mm}\\
    {\Large {\bf #1} \hfill #2 \hfill {\bf #1}}\\
    \setlength{\oldspacing}{\topsep}
    \setlength{\topsep}{0.3ex}
    \begin{description}
      #3
    \end{description}
    \setlength{\topsep}{\oldspacing}
  }
}
\newcommand{\action}[1]
{\item[ACTION]: #1}
\newcommand{\call}[1]
{\item[CALL]: \hspace{0.4em}{\tt #1}}
\newlength{\oldspacing}
\newcommand{\args}[2]
{
  \goodbreak
  \setlength{\oldspacing}{\topsep}
  \setlength{\topsep}{0.3ex}
  \begin{description}
  \item[#1]:\\[1.5ex]
    \begin{tabular}{p{7em}p{6em}p{26em}}
      #2
    \end{tabular}
  \end{description}
  \setlength{\topsep}{\oldspacing}
} 
\newcommand{\spec}[3]
{
  {\em {#1}} & {\bf \mbox{#2}} & {#3}
}
\newcommand{\specel}[2]
{
  \multicolumn{1}{c}{#1} & {} & {#2}
}
\newcommand{\anote}[1]
{
  \goodbreak
  \setlength{\oldspacing}{\topsep}
  \setlength{\topsep}{0.3ex}
  \begin{description}
    \item[NOTE]:
        #1
  \end{description}
  \setlength{\topsep}{\oldspacing}
}
\newcommand{\notes}[1]
{
  \goodbreak
  \setlength{\oldspacing}{\topsep}
  \setlength{\topsep}{0.3ex}
  \begin{description}
    \item[NOTES]:
        #1
  \end{description}
  \setlength{\topsep}{\oldspacing}
}
\newcommand{\aref}[1]
{
  \goodbreak
  \setlength{\oldspacing}{\topsep}
  \setlength{\topsep}{0.3ex}
  \begin{description}
    \item[REFERENCE]:
        #1
  \end{description}
  \setlength{\topsep}{\oldspacing}
}
\newcommand{\refs}[1]
{
  \goodbreak
  \setlength{\oldspacing}{\topsep}
  \setlength{\topsep}{0.3ex}
  \begin{description}
    \item[REFERENCES]:
        #1
  \end{description}
  \setlength{\topsep}{\oldspacing}
}
%------------------------------------------------------------------------------

\section{INTRODUCTION}
\subsection{Purpose}
SLALIB\footnote{The name isn't an acronym;
it just stands for {\it Subprogram Library~A}.}
is a library of routines
intended to make accurate and reliable positional-astronomy
applications easier to write.
Most SLALIB routines are concerned with astronomical position and time, but a
number have wider trigonometrical, numerical or general applications.
The applications ASTROM, COCO, RV and TPOINT
all make extensive use of the SLALIB
routines, as do a number of telescope control systems around the world.
The SLALIB versions currently in service are written in
Fortran~77 and run on VAX/VMS, several Unix platforms and PC.
A generic ANSI~C version is also available;  it is functionally
similar to the Fortran version upon which the present document
concentrates.

\subsection{Example Application}
Here is a simple example of an application program written
using SLALIB calls:

\begin{verbatim}
      PROGRAM FK4FK5
*
*  Read a B1950 position from I/O unit 5 and reply on I/O unit 6
*  with the J2000 equivalent.  Enter a period to quit.
*
      IMPLICIT NONE
      CHARACTER C*80,S
      INTEGER I,J,IHMSF(4),IDMSF(4)
      DOUBLE PRECISION R4,D4,R5,D5
      LOGICAL BAD

*   Loop until a period is entered
      C = ' '
      DO WHILE (C(:1).NE.'.')

*     Read h m s d ' "
         READ (5,'(A)') C
         IF (C(:1).NE.'.') THEN
            BAD = .TRUE.

*        Decode the RA
            I = 1
            CALL sla_DAFIN(C,I,R4,J)
            IF (J.EQ.0) THEN
               R4 = 15D0*R4

*           Decode the Dec
               CALL sla_DAFIN(C,I,D4,J)
               IF (J.EQ.0) THEN

*              FK4 to FK5
                  CALL sla_FK45Z(R4,D4,1950D0,R5,D5)

*              Format and output the result
                  CALL sla_DR2TF(2,R5,S,IHMSF)
                  CALL sla_DR2AF(1,D5,S,IDMSF)
                  WRITE (6,
     :       '(1X,I2.2,2I3.2,''.'',I2.2,2X,A,I2.2,2I3.2,''.'',I1)')
     :                                                     IHMSF,S,IDMSF
                  BAD = .FALSE.
               END IF
            END IF
            IF (BAD) WRITE (6,'(1X,''?'')')
         END IF
      END DO

      END
\end{verbatim}
In this example, SLALIB not only provides the complicated FK4 to
FK5 transformation but also
simplifies the tedious and error-prone tasks
of decoding and formatting angles
expressed as hours, minutes {\it etc}.  The
example incorporates range checking, and avoids the
notorious ``minus zero'' problem (an often-perpetrated bug where
declinations between $0^{\circ}$ and $-1^{\circ}$ lose their minus
sign).
With a little extra elaboration and a few more calls to SLALIB,
defaulting can be provided (enabling unused fields to
be replaced with commas to avoid retyping), proper motions
can be handled, different epochs can be specified, and
so on.  See the program COCO (SUN/56) for further ideas.

\subsection{Scope}
SLALIB contains 140 routines covering the following topics:
\begin{itemize}
\item String Decoding,
      Sexagesimal Conversions
\item Angles, Vectors \& Rotation Matrices
\item Calendars,
      Timescales
\item Precession \& Nutation
\item Proper Motion
\item FK4/5,
      Elliptic Aberration
\item Geocentric Coordinates
\item Apparent \& Observed Place
\item Refraction \& Air Mass
\item Ecliptic,
      Galactic,
      Supergalactic Coordinates
\item Ephemerides
\item Astrometry
\item Numerical Methods
\end{itemize}

\subsection{Objectives}
SLALIB was designed to give application programmers
a basic set of positional-astronomy tools which were
accurate and easy to use.  To this end, the library is:
\begin{itemize}
\item Readily available, including source code and documentation.
\item Supported and maintained.
\item Portable -- coded in standard languages and available for
multiple computers and operating systems.
\item Thoroughly commented, both for maintainability and to
assist those wishing to cannibalize the code.
\item Stable.
\item Trustworthy -- some care has gone into
testing SLALIB, both by comparison with published data and
by checks for internal consistency.
\item Rigorous -- corners are not cut,
even where the practical consequences would in most cases be
negligible.
\item Comprehensive, without including too many esoteric features
required only by specialists.
\item Practical -- almost all the routines have been written to
satisfy real needs encountered during the development of
real-life applications.
\item Environment-independent -- the package is
completely free of pauses, stops, I/O {\it etc}.
\item Self-contained -- SLALIB calls no other libraries.
\end{itemize}
A few {\it caveats}:
\begin{itemize}
\item SLALIB does not pretend to be canonical.  It is in essence
an anthology, and the adopted algorithms are liable
to change as more up-to-date ones become available.
\item The functions aren't orthogonal -- there are several
cases of different
routines doing similar things, and many examples where
sequences of SLALIB calls have simply been packaged, all to
make applications less trouble to write.
\item There are holes -- there is no support for orbital calculations,
for example, and no high-precision Solar-System predictions.
\item SLALIB is not homogeneous, though important subsets
(for example the FK4/FK5 routines) are.
\item The library is not foolproof.  You have to know what
you are trying to do ({\it e.g.}\ by reading textbooks on positional
astronomy), and it is the caller's responsibility to supply
sensible arguments (although enough internal validation is done to
avoid arithmetic errors).
\item Without being written in a wasteful
manner, SLALIB is nonetheless optimized for maintainability
rather than speed.  In addition, there are many places
where considerable simplification would be possible if some
specified amount of accuracy could be sacrificed;  such
compromises are left to the individual programmer and
are not allowed to limit SLALIB's value as a source
of comparison results.
\end{itemize}

\section{LINKING}
On VAX/VMS:
\begin{verse}
\verb|$ LINK progname,SLALIB_DIR:SLALIB/LIB|
\end{verse}

On Sun and DECstation:
\begin{verse}
\verb|% f77 progname.o -L/star/lib 'sla_link' -o progname.out|
\end{verse}
(The above assumes that all Starlink directories have been added to
the \verb|PATH| and \verb|LD_LIBRARY_PATH| environment variables
as described in SUN/118.)

\section{AVAILABLE VERSIONS}
\subsection{Fortran}
The Fortran versions of SLALIB use ANSI Fortran~77 with a few
commonplace extensions.  Three out of the 140 routines require
platform-specific techniques and exist in different versions.
SLALIB has been implemented on the following platforms:
VAX/VMS,
PC (MS-DOS and Microsoft Fortran),
DECstation (Ultrix),
DEC Alpha (OSF/1),
Sun/SunOS,
Sun/Solaris,
CONVEX,
Perkin-Elmer and
Fujitsu.  All but the last three are supported by Starlink
and/or the author.

\subsection{C}
An ANSI C version of SLALIB has been completed but had not at the
time of writing been officially released.
The functionality of the C version matches
that of the Fortran SLALIB, partly for the convenience of
existing users of the Fortran version, some of whom have
implemented C ``wrappers''.  The function names
cannot be the same as the Fortran versions because of potential
linking problems when
both forms of the library are present; the C routine which
is the equivalent of (for example) \verb|SLA_REFRO| is \verb|slaRefro|.
The types of arguments follow the Fortran version, except
that integers are \verb|int| rather than \verb|long|.
Argument passing is by value
(except for arrays and strings of course)
for given arguments and by pointer for returned arguments.

Further details of the C version of SLALIB will appear in due course.
The package follows normal Starlink practice as far as building
and linking are concerned.  For the time being, the definitive guide to
the calling sequences is the file \verb|slalib.h|.

\section{FURTHER DEVELOPMENT}

\subsection{Other Languages}
The homogeneity and ease of use of SLALIB could perhaps be improved
in the future
by turning to C++ and object-oriented techniques.  For example ``celestial
position'' could be a class and many of the transformations
could happen automatically.  This requires further study and
would almost certainly result in a complete redesign.
Similarly,
the impact of Fortran~90 has yet to be assessed.  Once compilers
become widely available, some internal recoding may be worthwhile
in order to simplify parts of the code.  However, as with C++,
a redesign of the
application interfaces will be needed if the capabilities of the
new language are to be exploited to the full.

\subsection{New Functions}
In a package like SLALIB it is difficult to know how far to go.  Is it
enough to provide the primitive operations, or should more
complicated functions be packaged?  Is it worth encroaching on
specialist areas, where individual experts have all written their
own software already?  To what extent should CPU efficiency be
an issue?  How
much support of different numerical precisions is required?  And
so on.

In practice, almost all the routines in SLALIB are there because they were
needed for some specific application, and this is likely to remain the
pattern for any enhancements in the future.
Suggestions for additional SLALIB routines should be addressed to the
author.

\section{ACKNOWLEDGMENTS}

SLALIB is descended from a package of routines written
for the AAO 16-bit minicomputers
in the mid-1970s.  The coming of the VAX
allowed a much more comprehensive and thorough package
to be designed for Starlink, especially important
at a time when the adoption
of the IAU 1976 resolutions meant that astronomers
would have to cope with a mixture of reference frames,
timescales and nomenclature.

Much of the preparatory work on SLALIB was done by Althea~Wilkinson of
Manchester University.  During its development,
Andrew~Murray, Catherine~Hohenkerk, Andrew~Sinclair, Bernard~Yallop
and Brian~Emerson of Her Majesty's Nautical Almanac Office were consulted
on many occasions; their advice was indispensable.
I am especially grateful to Catherine~Hohenkerk for supplying preprints of
papers, and test data.
A number of enhancements to SLALIB
were at the suggestion of
Russell~Owen, University of Washington,
Phil~Hill, St~Andrews University and
Bill~Vacca, JILA, Boulder.

The C version began as a hand-coded transcription
by Steve~Eaton (University of Leeds), was enhanced by John~Straede (AAO)
and Martin~Shepherd (Caltech), and was then revised, completed
and validated by the present author.  Additional comments came from
Bob~Payne (NRAO) and
Jeremy~Bailey (AAO).

\pagebreak

\section{SUMMARY OF CALLS}
The basic trigonometrical and numerical facilities are supplied in both single
and double precision versions.
Most of the more esoteric position and time routines use double precision
arguments only, even in cases where single precision would normally be adequate
in practice.
Certain routines with modest accuracy objectives are supplied in
single precision versions only.
In the calling sequences which follow, no attempt has been made
to distinguish between single and double precision argument names,
and frequently the same name is used on different occasions to
mean different things.
However, none of the routines uses a mixture of single and
double precision arguments;  each routine is either wholly
single precision or wholly double precision.

In the classified list, below,
{\it subroutine}\, subprograms are those whose names and argument lists
are preceded by `CALL', whereas {\it function}\, subprograms are
those beginning `R=' (when the result is REAL) or `D=' (when
the result is DOUBLE~PRECISION).

The list is, of course, merely for quick reference;  inexperienced
users {\bf must} refer to the detailed specifications given later.
In particular, {\bf don't guess} whether arguments are single or
double precision; the result could be a program that happens to
works on one sort of machine but not on another.

\callhead{String Decoding}
\begin{callset}
\subp{CALL sla\_INTIN (STRING, NSTRT, IRESLT, JFLAG)}
   Convert free-format string into integer
\subq{CALL sla\_FLOTIN (STRING, NSTRT, RESLT, JFLAG)}
     {CALL sla\_DFLTIN (STRING, NSTRT, DRESLT, JFLAG)}
   Convert free-format string into floating-point number
\subq{CALL sla\_AFIN (STRING, NSTRT, RESLT, JFLAG)}
     {CALL sla\_DAFIN (STRING, NSTRT, DRESLT, JFLAG)}
   Convert free-format string from deg,arcmin,arcsec to radians
\end{callset}

\callhead{Sexagesimal Conversions}
\begin{callset}
\subq{CALL sla\_CTF2D (IHOUR, IMIN, SEC, DAYS, J)}
     {CALL sla\_DTF2D (IHOUR, IMIN, SEC, DAYS, J)}
   Hours, minutes, seconds to days
\subq{CALL sla\_CD2TF (NDP, DAYS, SIGN, IHMSF)}
     {CALL sla\_DD2TF (NDP, DAYS, SIGN, IHMSF)}
   Days to hours, minutes, seconds
\subq{CALL sla\_CTF2R (IHOUR, IMIN, SEC, RAD, J)}
     {CALL sla\_DTF2R (IHOUR, IMIN, SEC, RAD, J)}
   Hours, minutes, seconds to radians
\subq{CALL sla\_CR2TF (NDP, ANGLE, SIGN, IHMSF)}
     {CALL sla\_DR2TF (NDP, ANGLE, SIGN, IHMSF)}
   Radians to hours, minutes, seconds
\subq{CALL sla\_CAF2R (IDEG, IAMIN, ASEC, RAD, J)}
     {CALL sla\_DAF2R (IDEG, IAMIN, ASEC, RAD, J)}
   Degrees, arcminutes, arcseconds to radians
\subq{CALL sla\_CR2AF (NDP, ANGLE, SIGN, IDMSF)}
     {CALL sla\_DR2AF (NDP, ANGLE, SIGN, IDMSF)}
   Radians to degrees, arcminutes, arcseconds
\end{callset}

\callhead{Angles, Vectors and Rotation Matrices}
\begin{callset}
\subq{R~=~sla\_RANGE (ANGLE)}
     {D~=~sla\_DRANGE (ANGLE)}
   Normalize angle into range $\pm\pi$
\subq{R~=~sla\_RANORM (ANGLE)}
     {D~=~sla\_DRANRM (ANGLE)}
   Normalize angle into range $0\!-\!2\pi$
\subq{CALL sla\_CS2C (A, B, V)}
     {CALL sla\_DCS2C (A, B, V)}
   Spherical coordinates to \xyz
\subq{CALL sla\_CC2S (V, A, B)}
     {CALL sla\_DCC2S (V, A, B)}
   \xyz\ to spherical coordinates
\subq{R~=~sla\_VDV (VA, VB)}
     {D~=~sla\_DVDV (VA, VB)}
   Scalar product of two 3-vectors
\subq{R~=~sla\_VXV (VA, VB, VC)}
     {D~=~sla\_DVXV (VA, VB, VC)}
   Vector product of two 3-vectors
\subq{CALL sla\_VN (V, UV, VM)}
     {CALL sla\_DVN (V, UV, VM)}
   Normalize a 3-vector also giving the modulus
\subq{R~=~sla\_SEP (A1, B1, A2, B2)}
     {D~=~sla\_DSEP (A1, B1, A2, B2)}
   Angle between two points on a sphere
\subq{R~=~sla\_BEAR (A1, B1, A2, B2)}
     {D~=~sla\_DBEAR (A1, B1, A2, B2)}
   Direction of one point on a sphere seen from another
\subq{CALL sla\_EULER (ORDER, PHI, THETA, PSI, RMAT)}
     {CALL sla\_DEULER (ORDER, PHI, THETA, PSI, RMAT)}
   Form rotation matrix from three Euler angles
\subq{CALL sla\_AV2M (AXIS, ANGLE, RMAT)}
     {CALL sla\_DAV2M (AXIS, ANGLE, RMAT)}
   Form rotation matrix from axial vector
\subq{CALL sla\_M2AV (RMAT, AXVEC)}
     {CALL sla\_DM2AV (RMAT, AXVEC)}
   Determine axial vector from rotation matrix
\subq{CALL sla\_MXV (RM, VA, VB)}
     {CALL sla\_DMXV (DM, VA, VB)}
   Rotate vector forwards
\subq{CALL sla\_IMXV (RM, VA, VB)}
     {CALL sla\_DIMXV (DM, VA, VB)}
   Rotate vector backwards
\subq{CALL sla\_MXM (A, B, C)}
     {CALL sla\_DMXM (A, B, C)}
   Product of two 3x3 matrices
\subq{CALL sla\_CS2C6 (A, B, R, AD, BD, RD, V)}
     {CALL sla\_DS2C6 (A, B, R, AD, BD, RD, V)}
   Conversion of position and velocity in spherical
     coordinates to Cartesian coordinates
\subq{CALL sla\_CC62S (V, A, B, R, AD, BD, RD)}
     {CALL sla\_DC62S (V, A, B, R, AD, BD, RD)}
   Conversion of position and velocity in Cartesian
     coordinates to spherical coordinates
\end{callset}

\callhead{Calendars}
\begin{callset}
\subp{CALL sla\_CLDJ (IY, IM, ID, DJM, J)}
   Gregorian Calendar to Modified Julian Date
\subp{CALL sla\_CALDJ (IY, IM, ID, DJM, J)}
   Gregorian Calendar to Modified Julian Date,
     permitting century default
\subp{CALL sla\_DJCAL (NDP, DJM, IYMDF, J)}
   Modified Julian Date to Gregorian Calendar,
     in a form convenient for formatted output
\subp{CALL sla\_DJCL (DJM, IY, IM, ID, FD, J)}
   Modified Julian Date to Gregorian Year, Month, Day, Fraction
\subp{CALL sla\_CALYD (IY, IM, ID, NY, ND, J)}
   Calendar to year and day in year
\subp{D~=~sla\_EPB (DATE)}
   Modified Julian Date to Besselian Epoch
\subp{D~=~sla\_EPB2D (EPB)}
   Besselian Epoch to Modified Julian Date
\subp{D~=~sla\_EPJ (DATE)}
   Modified Julian Date to Julian Epoch
\subp{D~=~sla\_EPJ2D (EPJ)}
   Julian Epoch to Modified Julian Date
\end{callset}


\callhead{Timescales}
\begin{callset}
\subp{D~=~sla\_GMST (UT1)}
   Conversion from Universal Time to sidereal time
\subp{D~=~sla\_EQEQX (DATE)}
   Equation of the equinoxes
\subp{D~=~sla\_DAT (DJU)}
   Offset of Atomic Time from UT: TAI$-$UTC
\subp{D~=~sla\_DTT (DJU)}
   Offset of Terrestrial Time TT from UT: TT$-$UTC
\subp{D~=~sla\_RCC (TDB, UT1, WL, U, V)}
   Relativistic clock correction: TDB$-$TT
\end{callset}

\callhead{Precession and Nutation}
\begin{callset}
\subp{CALL sla\_NUT (DATE, RMATN)}
   Nutation matrix
\subp{CALL sla\_NUTC (DATE, DPSI, DEPS, EPS0)}
   Longitude and obliquity components of nutation, and
     mean obliquity
\subp{CALL sla\_PREC (EP0, EP1, RMATP)}
   Precession matrix
\subp{CALL sla\_PRENUT (EPOCH, DATE, RMATPN)}
   Combined precession/nutation matrix
\subp{CALL sla\_PREBN (BEP0, BEP1, RMATP)}
   Precession matrix, old system
\subp{CALL sla\_PRECES (SYSTEM, EP0, EP1, RA, DC)}
   Precession, in either the old or the new system
\end{callset}

\callhead{Proper Motion}
\begin{callset}
\subp{CALL sla\_PM (R0, D0, PR, PD, PX, RV, EP0, EP1, R1, D1)}
   Adjust for proper motion
\end{callset}

\callhead{FK4/5 Conversions}
\begin{callset}
\subp{CALL sla\_FK425 (\vtop
                       {\hbox{R1950, D1950, DR1950, DD1950, P1950, V1950,}
                        \hbox{R2000, D2000, DR2000, DD2000, P2000, V2000)}}}
   Convert B1950.0 FK4 star data to J2000.0 FK5
\subp{CALL sla\_FK45Z (R1950, D1950, EPOCH, R2000, D2000)}
   Convert B1950.0 FK4 position to J2000.0 FK5 assuming zero
     proper motion in an inertial frame and no parallax
\subp{CALL sla\_FK524 (\vtop
                       {\hbox{R2000, D2000, DR2000, DD2000, P2000, V2000,}
                        \hbox{R1950, D1950, DR1950, DD1950, P1950, V1950)}}}
   Convert J2000.0 FK5 star data to B1950.0 FK4
\subp{CALL sla\_FK54Z (R2000, D2000, BEPOCH,
               R1950, D1950, DR1950, DD1950)}
   Convert J2000.0 FK5 position to B1950.0 FK4 assuming zero
     proper motion and no parallax
\subp{CALL sla\_DBJIN (STRING, NSTRT, DRESLT, J1, J2)}
   Like sla\_DFLTIN but with extensions to accept leading `B' and `J'
\subp{CALL sla\_KBJ (JB, E, K, J)}
   Select epoch prefix `B' or `J'
\subp{D~=~sla\_EPCO (K0, K, E)}
   Convert an epoch into the appropriate form -- `B' or `J'
\end{callset}


\callhead{Elliptic Aberration}
\begin{callset}
\subp{CALL sla\_ETRMS (EP, EV)}
   E-terms
\subp{CALL sla\_SUBET (RC, DC, EQ, RM, DM)}
   Remove the E-terms
\subp{CALL sla\_ADDET (RM, DM, EQ, RC, DC)}
   Add the E-terms
\end{callset}

\callhead{Geocentric Coordinates}
\begin{callset}
\subp{CALL sla\_OBS (NUMBER, ID, NAME, WLONG, PHI, HEIGHT)}
   Interrogate list of observatory parameters
\subp{CALL sla\_GEOC (P, H, R, Z)}
   Convert geodetic position to geocentric
\subp{CALL sla\_PVOBS (P, H, STL, PV)}
   Position and velocity of observatory
\end{callset}

\callhead{Apparent and Observed Place}
\begin{callset}
\subp{CALL sla\_MAP (RM, DM, PR, PD, PX, RV, EQ, DATE, RA, DA)}
   Mean place to geocentric apparent place
\subp{CALL sla\_MAPPA (EQ, DATE, AMPRMS)}
   Precompute mean to apparent parameters
\subp{CALL sla\_MAPQK (RM, DM, PR, PD, PX, RV, AMPRMS, RA, DA)}
   Mean to apparent using precomputed parameters
\subp{CALL sla\_MAPQKZ (RM, DM, AMPRMS, RA, DA)}
   Mean to apparent using precomputed parameters, for zero proper
     motion, parallax and radial velocity
\subp{CALL sla\_AMP (RA, DA, DATE, EQ, RM, DM)}
   Geocentric apparent place to mean place
\subp{CALL sla\_AMPQK (RA, DA, AOPRMS, RM, DM)}
   Apparent to mean using precomputed parameters
\subp{CALL sla\_AOP (\vtop
                      {\hbox{RAP, DAP, UTC, DUT, ELONGM, PHIM, HM, XP, YP,}
                       \hbox{TDK, PMB, RH, WL, TLR, AOB, ZOB, HOB, DOB, ROB)}}}
   Apparent place to observed place
\subp{CALL sla\_AOPPA (\vtop
                        {\hbox{UTC, DUT, ELONGM, PHIM, HM, XP, YP,}
                         \hbox{TDK, PMB, RH, WL, TLR, AOPRMS)}}}
   Precompute apparent to observed parameters
\subp{CALL sla\_AOPPAT (UTC, AOPRMS)}
   Update sidereal time in apparent to observed parameters
\subp{CALL sla\_AOPQK (RAP, DAP, AOPRMS, AOB, ZOB, HOB, DOB, ROB)}
   Apparent to observed using precomputed parameters
\subp{CALL sla\_OAP (\vtop
                     {\hbox{TYPE, OB1, OB2, UTC, DUT, ELONGM, PHIM, HM, XP, YP,}
                      \hbox{TDK, PMB, RH, WL, TLR, RAP, DAP)}}}
   Observed to apparent
\subp{CALL sla\_OAPQK (TYPE, OB1, OB2, AOPRMS, RA, DA)}
   Observed to apparent using precomputed parameters
\subp{D~=~sla\_ZD (HA, DEC, PHI)}
   \hadec\ to zenith distance
\subp{D~=~sla\_PA (HA, DEC, PHI)}
   \hadec\ to parallactic angle
\end{callset}

\callhead{Refraction and Air Mass}
\begin{callset}
\subp{CALL sla\_REFRO (ZOBS, HM, TDK, PMB, RH, WL, PHI, TLR, EPS, REF)}
   Change in zenith distance due to refraction
\subp{CALL sla\_REFCO (HM, TDK, PMB, RH, WL, PHI, TLR, EPS, REFA, REFB)}
   Constants for simple refraction model
\subp{CALL sla\_REFZ (ZU, REFA, REFB, ZR)}
   Unrefracted to refracted ZD, simple model
\subp{CALL sla\_REFV (VU, REFA, REFB, VR)}
   Unrefracted to refracted \azel\ vector, simple model
\subp{D~=~sla\_AIRMAS (ZD)}
   Air mass
\end{callset}

\callhead{Ecliptic Coordinates}
\begin{callset}
\subp{CALL sla\_ECMAT (DATE, RMAT)}
   Equatorial to ecliptic rotation matrix
\subp{CALL sla\_EQECL (DR, DD, DATE, DL, DB)}
   J2000.0 `FK5' to ecliptic coordinates
\subp{CALL sla\_ECLEQ (DL, DB, DATE, DR, DD)}
   Ecliptic coordinates to J2000.0 `FK5'
\end{callset}

\callhead{Galactic Coordinates}
\begin{callset}
\subp{CALL sla\_EG50 (DR, DD, DL, DB)}
   B1950.0 `FK4' to galactic
\subp{CALL sla\_GE50 (DL, DB, DR, DD)}
   Galactic to B1950.0 `FK4'
\subp{CALL sla\_EQGAL (DR, DD, DL, DB)}
   J2000.0 `FK5' to galactic
\subp{CALL sla\_GALEQ (DL, DB, DR, DD)}
   Galactic to J2000.0 `FK5'
\end{callset}

\callhead{Supergalactic Coordinates}
\begin{callset}
\subp{CALL sla\_GALSUP (DL, DB, DSL, DSB)}
   Galactic to supergalactic
\subp{CALL sla\_SUPGAL (DSL, DSB, DL, DB)}
   Supergalactic to galactic
\end{callset}

\callhead{Ephemerides}
\begin{callset}
\subp{CALL sla\_EARTH (IY, ID, FD, POSVEL)}
   Approximate heliocentric position and velocity of the Earth
\subp{CALL sla\_EVP (DATE, DEQX, DVB, DPB, DVH, DPH)}
   Barycentric and heliocentric velocity and position of the Earth
\subp{CALL sla\_MOON (IY,ID,FD,POSVEL)}
   Approximate geocentric position and velocity of the Moon
\subp{R~=~sla\_RVEROT (PHI, RA, DA, ST)}
   Velocity component due to rotation of the Earth
\subp{CALL sla\_ECOR (RM, DM, IY, ID, FD, RV, TL)}
   Components of velocity and light time due to Earth orbital motion
\subp{R~=~sla\_RVLSRD (R2000, D2000)}
   Velocity component due to solar motion wrt dynamical LSR
\subp{R~=~sla\_RVLSRK (R2000, D2000)}
   Velocity component due to solar motion wrt kinematical LSR
\subp{R~=~sla\_RVGALC (R2000, D2000)}
   Velocity component due to rotation of the Galaxy
\subp{R~=~sla\_RVLG (R2000, D2000)}
   Velocity component due to rotation and translation of the
   Galaxy, relative to the mean motion of the local group
\end{callset}

\callhead{Astrometry}
\begin{callset}
\subq{CALL sla\_S2TP (RA, DEC, RAZ, DECZ, XI, ETA, J)}
     {CALL sla\_DS2TP (RA, DEC, RAZ, DECZ, XI, ETA, J)}
   Transform spherical coordinates into tangent plane
\subq{CALL sla\_DTP2S (XI, ETA, RAZ, DECZ, RA, DEC)}
     {CALL sla\_TP2S (XI, ETA, RAZ, DECZ, RA, DEC)}
   Transform tangent plane coordinates into spherical
\subp{CALL sla\_PCD (DISCO,X,Y)}
   Apply pincushion/barrel distortion
\subp{CALL sla\_UNPCD (DISCO,X,Y)}
   Remove pincushion/barrel distortion
\subp{CALL sla\_FITXY (ITYPE,NP,XYE,XYM,COEFFS,J)}
   Fit a linear model to relate two sets of \xy\ coordinates
\subp{CALL sla\_PXY (NP,XYE,XYM,COEFFS,XYP,XRMS,YRMS,RRMS)}
   Compute predicted coordinates and residuals
\subp{CALL sla\_INVF (FWDS,BKWDS,J)}
   Invert a linear model
\subp{CALL sla\_XY2XY (X1,Y1,COEFFS,X2,Y2)}
   Transform one \xy\
\subp{CALL sla\_DCMPF (COEFFS,XZ,YZ,XS,YS,PERP,ORIENT)}
   Decompose a linear fit into scales {\it etc.}
\end{callset}

\callhead{Numerical Methods}
\begin{callset}
\subq{CALL sla\_SMAT (N, A, Y, D, JF, IW)}
     {CALL sla\_DMAT (N, A, Y, D, JF, IW)}
   Matrix inversion and solution of simultaneous equations
\subp{CALL sla\_SVD (M, N, MP, NP, A, W, V, WORK, JSTAT)}
   Singular value decomposition of a matrix
\subp{CALL sla\_SVDSOL (M, N, MP, NP, B, U, W, V, WORK, X)}
   Solution from given vector plus SVD
\subp{CALL sla\_SVDCOV (N, NP, NC, W, V, WORK, CVM)}
   Covariance matrix from SVD
\subp{R~=~sla\_RANDOM (SEED)}
   Generate pseudo-random real number in the range {$0 \leq x < 1$}
\subp{R~=~sla\_GRESID (S)}
   Generate pseudo-random normal deviate ($\equiv$ `Gaussian residual')
\end{callset}

\callhead{Real-time}
\begin{callset}
\subp{CALL sla\_WAIT (DELAY)}
    Interval wait
\end{callset}

\pagebreak

\section{SUBPROGRAM SPECIFICATIONS}
%-----------------------------------------------------------------------
\routine{SLA\_ADDET}{Add E-terms of Aberration}
{
 \action{Add the E-terms (elliptic component of annual aberration) to a
  pre IAU 1976 mean place to conform to the old catalogue convention.}
 \call{CALL sla\_ADDET (RM, DM, EQ, RC, DC)}
}
\args{GIVEN}
{
 \spec{RM,DM}{D}{\radec\ without E-terms (radians)} \\
 \spec{EQ}{D}{Besselian epoch of mean equator and equinox}
}
\args{RETURNED}
{
 \spec{RC,DC}{D}{\radec\ with E-terms included (radians)}
}
\anote{Most star positions from pre-1984 optical catalogues (or
       obtained by astrometry with respect to such stars) have the
       E-terms built-in.  If it is necessary to convert a formal mean
       place (for example a pulsar timing position) to one
       consistent with such a star catalogue, then the \radec\
       should be adjusted using this routine.}
\aref{{\it Explanatory Supplement to the Astronomical Ephemeris},
 section 2D, page 48.}
%-----------------------------------------------------------------------
\routine{SLA\_AFIN}{Sexagesimal character string to angle}
{
 \action{Decode a free-format sexagesimal string (degrees, arcminutes,
         arcseconds) into a single precision floating point
         number (radians).}
 \call{CALL sla\_AFIN (STRING, NSTRT, RESLT, JF)}
}
\args{GIVEN}
{
 \spec{STRING}{C*(*)}{string containing deg, arcmin, arcsec fields} \\
 \spec{NSTRT}{I}{pointer to start of decode (beginning of STRING = 1)}
}
\args{RETURNED}
{
 \spec{NSTRT}{I}{advanced past the decoded angle} \\
 \spec{RESLT}{R}{angle in radians} \\
 \spec{JF}{I}{status:} \\
 \spec{}{}{\hspace{1.5em}   0 = OK} \\
 \spec{}{}{\hspace{0.7em} $+1$ = default, RESLT unchanged (note 2)} \\
 \spec{}{}{\hspace{0.7em} $-1$ = bad degrees (note 3)} \\
 \spec{}{}{\hspace{0.7em} $-2$ = bad arcminutes (note 3)} \\
 \spec{}{}{\hspace{0.7em} $-3$ = bad arcseconds (note 3)} \\
}
\goodbreak
\setlength{\oldspacing}{\topsep}
\setlength{\topsep}{0.3ex}
\begin{description}
 \item [EXAMPLE]: \\ [1.5ex]
  \begin{tabular}{p{9em}p{15em}p{15em}}
   {\it argument} & {\it before} & {\it after} \\ \\
   STRING & $'$\verb*|-57 17 44.806  12 34 56.7|$'$ & unchanged \\
   NSTRT & 1 & 16 ({\it i.e.}\ pointing to 12...) \\
   RESLT & - & $-1.00000$ \\
   JF & - & 0
  \end{tabular}
 \item A further call to sla\_AFIN, without adjustment of NSTRT, will
       decode the second angle, \dms{12}{34}{56}{7}.
\end{description}
\setlength{\topsep}{\oldspacing}
\notes
{
 \begin{enumerate}
  \item The first three ``fields'' in STRING are degrees, arcminutes,
   arcseconds, separated by spaces or commas.  The degrees field
   may be signed, but not the others.  The decoding is carried
   out by the sla\_DFLTIN routine and is free-format.
  \item Successive fields may be absent, defaulting to zero.  For
   zero status, the only combinations allowed are degrees alone,
   degrees and arcminutes, and all three fields present.  If all
   three fields are omitted, a status of +1 is returned and RESLT is
   unchanged.  In all other cases RESLT is changed.
  \item Range checking:
   \begin{itemize}
    \item The degrees field is not range checked.  However, it is
     expected to be integral unless the other two fields are absent.
    \item The arcminutes field is expected to be 0-59, and integral if
     the arcseconds field is present.  If the arcseconds field
     is absent, the arcminutes is expected to be 0-59.9999...
    \item The arcseconds field is expected to be 0-59.9999...
    \item Decoding continues even when a check has failed.  Under these
     circumstances the field takes the supplied value, defaulting to
     zero, and the result RESLT is computed and returned.
   \end{itemize}
   \item Further fields after the three expected ones are not treated as
    an error.  The pointer NSTRT is left in the correct state for
    further decoding with the present routine or with sla\_DFLTIN
    {\it etc}.  See the example, above.
   \item If STRING contains hours, minutes, seconds instead of
    degrees {\it etc},
    or if the required units are turns (or days) instead of radians,
    the result RESLT should be multiplied as follows: \\ [1.5ex]
    \begin{tabular}{p{7em}p{7em}p{20em}}
    {\it for STRING} & {\it to obtain} & {\it multiply RESLT by} \\ \\
    ${\circ}$~~\raisebox{-0.7ex}{$'$}~~\raisebox{-0.7ex}{$''$}
     & radians & $1.0$ \\
    ${\circ}$~~\raisebox{-0.7ex}{$'$}~~\raisebox{-0.7ex}{$''$}
     & turns & $1/{2 \pi} = 0.1591549430918953358$ \\
    h m s & radians & $15.0$ \\
    h m s & days & $15/{2\pi} = 2.3873241463784300365$
   \end{tabular}
 \end{enumerate}
}
%-----------------------------------------------------------------------
\routine{SLA\_AIRMAS}{Air Mass}
{
 \action{Air mass at given zenith distance (double precision).}
 \call{D~=~sla\_AIRMAS (ZD)}
}
\args{GIVEN}
{
 \spec{ZD}{D}{observed zenith distance (radians)}
}
\args{RETURNED}
{
 \spec{sla\_AIRMAS}{D}{air mass (1 at zenith)}
}
\notes
{
 \begin{enumerate}
  \item The {\it observed}\, zenith distance referred to above means
        ``as affected by refraction''.
  \item The routine uses Hardie's (1962) polynomial fit to Bemporad's
        data for the relative air mass, $X$, in units of thickness at the
        zenith as tabulated by Schoenberg (1929). This is adequate for all
        normal needs as it is accurate to better than
        0.1\% up to $X = 6.8$ and better than 1\% up to $X = 10$.
        Bemporad's tabulated values are unlikely to be trustworthy
        to such accuracy 
        because of variations in density, pressure and other  
        conditions in the atmosphere from those assumed in his work.
  \item The sign of the ZD is ignored.
  \item At zenith distances greater than about $\zeta = 87^{\circ}$ the
        air mass is held constant to avoid arithmetic overflows.
 \end{enumerate}
}
\refs
{
 \begin{enumerate}
  \item Hardie, R.H., 1962, in {\it Astronomical Techniques}\,
        ed. W.A.\ Hiltner, University of Chicago Press, p180.
  \item Schoenberg, E., 1929, Hdb.\ d.\ Ap.,
        Berlin, Julius Springer, 2, 268.
 \end{enumerate}
}
%-----------------------------------------------------------------------
\routine{SLA\_AMP}{Apparent to Mean}
{
 \action{Convert star \radec\ from geocentric apparent to
         mean place (post IAU 1976).}
 \call{CALL sla\_AMP (RA, DA, DATE, EQ, RM, DM)}
}
\args{GIVEN}
{
 \spec{RA,DA}{D}{apparent \radec\ (radians)} \\
 \spec{DATE}{D}{TDB for apparent place (JD$-$2400000.5)} \\
 \spec{EQ}{D}{equinox:  Julian epoch of mean place}
}
\args{RETURNED}
{
 \spec{RM,DM}{D}{mean \radec\ (radians)}
}
\notes
{
 \begin{enumerate}
  \item The distinction between the required TDB and TT is
        always negligible.  Moreover, for all but the most
        critical applications UTC is adequate.
  \item The accuracy is limited by the routine sla\_EVP, called
        by sla\_MAPPA, which computes the Earth position and
        velocity using the methods of Stumpff.  The maximum
        error is about 0.3~milliarcsecond.
  \item Iterative techniques are used for the aberration and
        light deflection corrections so that the routines
        sla\_AMP (or sla\_AMPQK) and sla\_MAP (or sla\_MAPQK) are
        accurate inverses;  even at the edge of the Sun's disc
        the discrepancy is only about 1~nanoarcsecond.
  \item Where multiple apparent places are to be converted to
        mean places, for a fixed date and equinox, it is more
        efficient to use the sla\_MAPPA routine to compute the
        required parameters once, followed by one call to
        sla\_AMPQK per star.
 \end{enumerate}
}
\refs
{
 \begin{enumerate}
  \item 1984 {\it Astronomical Almanac}, pp B39-B41.
  \item Lederle \& Schwan, 1984.\ {\it Astr.Astrophys.}\ {\bf 134}, 1-6.
 \end{enumerate}
}
%-----------------------------------------------------------------------
\routine{SLA\_AMPQK}{Quick Apparent to Mean}
{
 \action{Convert star \radec\ from geocentric apparent to mean place
         (post IAU 1976).  Use of this routine is appropriate when
         efficiency is important and where many star positions are
         all to be transformed for one epoch and equinox.  The
         star-independent parameters can be obtained by calling
         the sla\_MAPPA routine.}
 \call{CALL sla\_AMPQK (RA, DA, AMPRMS, RM, DM)}
}
\args{GIVEN}
{
 \spec{RA,DA}{D}{apparent \radec\ (radians)} \\
 \spec{AMPRMS}{D(21)}{star-independent mean-to-apparent parameters:} \\
 \specel   {(1)}     {time interval for proper motion (Julian years)} \\
 \specel   {(2-4)}   {barycentric position of the Earth (AU)} \\
 \specel   {(5-7)}   {heliocentric direction of the Earth (unit vector)} \\
 \specel   {(8)}     {(gravitational radius of
                      Sun)$\times 2 / $(Sun-Earth distance)} \\
 \specel   {(9-11)}  {\mbox{\boldmath $v$}: barycentric Earth
                                               velocity in units of c} \\
 \specel   {(12)}    {$\sqrt{1-\mid\mbox{\boldmath $v$}\mid^2}$} \\
 \specel   {(13-21)} {precession/nutation $3\times3$ matrix}
}
\args{RETURNED}
{
 \spec{RM,DM}{D}{mean \radec\ (radians)}
}
\notes
{
 \begin{enumerate}
  \item The accuracy is limited by the routine sla\_EVP, called
        by sla\_MAPPA, which computes the Earth position and
        velocity using the methods of Stumpff.  The maximum
        error is about 0.3~milliarcsecond.
  \item Iterative techniques are used for the aberration and
        light deflection corrections so that the routines
        sla\_AMP (or sla\_AMPQK) and sla\_MAP (or sla\_MAPQK) are
        accurate inverses;  even at the edge of the Sun's disc
        the discrepancy is only about 1~nanoarcsecond.
 \end{enumerate}
}
\refs
{
 \begin{enumerate}
  \item 1984 {\it Astronomical Almanac}, pp B39-B41.
  \item Lederle \& Schwan, 1984.\ {\it Astr.Astrophys.}\ {\bf 134}, 1-6.
 \end{enumerate}
}
%-----------------------------------------------------------------------
\routine{SLA\_AOP}{Apparent to Observed}
{
 \action{Apparent to observed place, for optical sources distant from
         the solar system.}
 \call{CALL sla\_AOP (\vtop{
         \hbox{RAP, DAP, DATE, DUT, ELONGM, PHIM, HM, XP, YP,}
         \hbox{TDK, PMB, RH, WL, TLR, AOB, ZOB, HOB, DOB, ROB)}}}
}
\args{GIVEN}
{
 \spec{RAP,DAP}{D}{geocentric apparent \radec\ (radians)} \\
 \spec{DATE}{D}{UTC date/time (Modified Julian Date, JD$-$2400000.5)} \\
 \spec{DUT}{D}{$\Delta$UT:  UT1$-$UTC (UTC seconds)} \\
 \spec{ELONGM}{D}{mean longitude of the observer (radians, east +ve)} \\
 \spec{PHIM}{D}{mean geodetic latitude of the observer (radians)} \\
 \spec{HM}{D}{observer's height above sea level (metres)} \\
 \spec{XP,YP}{D}{polar motion \xy\ coordinates (radians)} \\
 \spec{TDK}{D}{local ambient temperature (degrees K; std=273.155D0)} \\
 \spec{PMB}{D}{local atmospheric pressure (mB; std=1013.25D0)} \\
 \spec{RH}{D}{local relative humidity (in the range 0D0-1D0)} \\
 \spec{WL}{D}{effective wavelength (micron, {\it e.g.}\ 0.55D0)} \\
 \spec{TLR}{D}{tropospheric lapse rate (degrees K per metre,
                                              {\it e.g.}\ 0.0065D0)}
}
\args{RETURNED}
{
 \spec{AOB}{D}{observed azimuth (radians: N=0, E=$90^{\circ}$)} \\
 \spec{ZOB}{D}{observed zenith distance (radians)} \\
 \spec{HOB}{D}{observed Hour Angle (radians)} \\
 \spec{DOB}{D}{observed $\delta$ (radians)} \\
 \spec{ROB}{D}{observed $\alpha$ (radians)}
}
\notes
{
 \begin{enumerate}
  \item The DATE argument is UTC expressed as an MJD.  This is,
        strictly speaking, wrong, because of leap seconds.  However,
        as long as the $\Delta$UT and the UTC are consistent there
        are no difficulties, except during a leap second.  In this
        case, the start of the 61st second of the final minute should
        begin a new MJD day and the old pre-leap $\Delta$UT should
        continue to be used.  As the 61st second completes, the MJD
        should revert to the start of the day as, simultaneously,
        the $\Delta$UT changes by one second to its post-leap new value.
  \item The $\Delta$UT (UT1$-$UTC) is tabulated in IERS circulars and
        elsewhere.  It increases by exactly one second at the end of
        each UTC leap second, introduced in order to keep $\Delta$UT
        within $\pm0$\tsec9.
  \item This routine returns zenith distance rather than elevation
        in order to reflect the fact that no allowance is made for
        depression of the horizon.
  \item The accuracy of the result is limited by the corrections for
        refraction.  Providing the meteorological parameters are
        known accurately and there are no gross local effects, the
        predicted azimuth and elevation should be within about
        0.1~arcsec for $\zeta<70^{\circ}$.  Even
        at a topocentric zenith distance of
        $90^{\circ}$, the accuracy in elevation should be better than
        1~arcmin;  useful results are available for a further
        $3^{\circ}$, beyond which the sla\_REFRO routine returns a
        fixed value of the refraction.  The complementary
        routines sla\_AOP (or sla\_AOPQK) and sla\_OAP (or sla\_OAPQK)
        are self-consistent to better than 1~microarcsecond all over
        the celestial sphere.
  \item It is advisable to take great care with units, as even
        unlikely values of the input parameters are accepted and
        processed in accordance with the models used.
  \item {\it Apparent}\, \radec\ means the geocentric apparent right ascension
        and declination, which is obtained from a catalogue mean place
        by allowing for space motion, parallax, precession, nutation,
        annual aberration, and the Sun's gravitational lens effect.  For
        star positions in the FK5 system ({\it i.e.}\ J2000), these effects can
        be applied by means of the sla\_MAP {\it etc.}\ routines.  Starting from
        other mean place systems, additional transformations will be
        needed;  for example, FK4 ({\it i.e.}\ B1950) mean places would first
        have to be converted to FK5, which can be done with the
        sla\_FK425 {\it etc.}\ routines.
  \item {\it Observed}\, \azel\ means the position that would be seen by a
        perfect theodolite located at the observer.  This is obtained
        from the geocentric apparent \radec\ by allowing for Earth
        orientation and diurnal aberration, rotating from equator
        to horizon coordinates, and then adjusting for refraction.
        The \hadec\ is obtained by rotating back into equatorial
        coordinates, using the geodetic latitude corrected for polar
        motion, and is the position that would be seen by a perfect
        equatorial located at the observer and with its polar axis
        aligned to the Earth's axis of rotation ({\it n.b.}\ not to the
        refracted pole).  Finally, the $\alpha$ is obtained by subtracting
        the {\it h}\, from the local apparent ST.
  \item To predict the required setting of a real telescope, the
        observed place produced by this routine would have to be
        adjusted for the tilt of the azimuth or polar axis of the
        mounting (with appropriate corrections for mount flexures),
        for non-perpendicularity between the mounting axes, for the
        position of the rotator axis and the pointing axis relative
        to it, for tube flexure, for gear and encoder errors, and
        finally for encoder zero points.  Some telescopes would, of
        course, exhibit other properties which would need to be
        accounted for at the appropriate point in the sequence.
  \item This routine takes time to execute, due mainly to the
        rigorous integration used to evaluate the refraction.
        For processing multiple stars for one location and time,
        call sla\_AOPPA once followed by one call per star to sla\_AOPQK.
        Where a range of times within a limited period of a few hours
        is involved, and the highest precision is not required, call
        sla\_AOPPA once, followed by a call to sla\_AOPPAT each time the
        time changes, followed by one call per star to sla\_AOPQK.
  \item The DATE argument is UTC expressed as an MJD.  This is,
        strictly speaking, wrong, because of leap seconds.  However,
        as long as the $\Delta$UT and the UTC are consistent there
        are no difficulties, except during a leap second.  In this
        case, the start of the 61st second of the final minute should
        begin a new MJD day and the old pre-leap $\Delta$UT should
        continue to be used.  As the 61st second completes, the MJD
        should revert to the start of the day as, simultaneously,
        the $\Delta$UT changes by one second to its post-leap new value.
  \item The $\Delta$UT (UT1$-$UTC) is tabulated in IERS circulars and
        elsewhere.  It increases by exactly one second at the end of
        each UTC leap second, introduced in order to keep $\Delta$UT
        within $\pm0$\tsec9.
  \item IMPORTANT -- TAKE CARE WITH THE LONGITUDE SIGN CONVENTION.  The
        longitude required by the present routine is {\bf east-positive},
        in accordance with geographical convention (and right-handed).
        In particular, note that the longitudes returned by the
        sla\_OBS routine are west-positive, following astronomical
        usage, and must be reversed in sign before use in the present
        routine.
  \item The polar coordinates XP,YP can be obtained from IERS
        circulars and equivalent publications.  The
        maximum amplitude is about 0.3~arcseconds.  If XP,YP values
        are unavailable, use XP=YP=0D0.  See page B60 of the 1988
        {\it Astronomical Almanac}\, for a definition of the two angles.
  \item The height above sea level of the observing station, HM,
        can be obtained from the {\it Astronomical Almanac}\, (Section J
        in the 1988 edition), or via the routine sla\_OBS.  If P,
        the pressure in millibars, is available, an adequate
        estimate of HM can be obtained from the following expression:
        \begin{quote}
         \verb|HM=-8149.9415D0*LOG(P/1013.25D0)|
        \end{quote}
        (See {\it Astrophysical Quantities}, C.W.Allen, 3rd~edition,
        \S52.)  Similarly, if the pressure P is not known,
        it can be estimated from the height of the observing
        station, HM as follows:
        \begin{quote}
         \verb|P=1013.25D0*EXP(-HM/8149.9415D0)|
        \end{quote}
        Note, however, that the refraction is proportional to the
        pressure and that an accurate P value is important for
        precise work.
 \end{enumerate}
}
%-----------------------------------------------------------------------
\routine{SLA\_AOPPA}{Appt-to-Obs Parameters}
{
 \action{Pre-compute the set of apparent to observed place parameters
         required by sla\_AOPQK and sla\_OAPQK.}
 \call{CALL sla\_AOPPA (\vtop{
          \hbox{DATE, DUT, ELONGM, PHIM, HM, XP, YP,}
          \hbox{TDK, PMB, RH, WL, TLR, AOPRMS)}}}
}
\args{GIVEN}
{
 \spec{DATE}{D}{UTC date/time (Modified Julian Date, JD$-$2400000.5)} \\
 \spec{DUT}{D}{$\Delta$UT:  UT1$-$UTC (UTC seconds)} \\
 \spec{ELONGM}{D}{mean longitude of the observer (radians, east +ve)} \\
 \spec{PHIM}{D}{mean geodetic latitude of the observer (radians)} \\
 \spec{HM}{D}{observer's height above sea level (metres)} \\
 \spec{XP,YP}{D}{polar motion \xy\ coordinates (radians)} \\
 \spec{TDK}{D}{local ambient temperature (degrees K; std=273.155D0)} \\
 \spec{PMB}{D}{local atmospheric pressure (mB; std=1013.25D0)} \\
 \spec{RH}{D}{local relative humidity (in the range 0D0-1D0)} \\
 \spec{WL}{D}{effective wavelength (micron, {\it e.g.}\ 0.55D0)} \\
 \spec{TLR}{D}{tropospheric lapse rate (degrees K per metre,
                                              {\it e.g.}\ 0.0065D0)}
}
\args{RETURNED}
{
 \spec{AOPRMS}{D(14)}{star-independent apparent-to-observed parameters:} \\
 \specel   {(1)}     {geodetic latitude (radians)} \\
 \specel   {(2,3)}   {sine and cosine of geodetic latitude} \\
 \specel   {(4)}     {magnitude of diurnal aberration vector} \\
 \specel   {(5)}     {height (HM)} \\
 \specel   {(6)}     {ambient temperature (TDK)} \\
 \specel   {(7)}     {pressure (PMB)} \\
 \specel   {(8)}     {relative humidity (RH)} \\
 \specel   {(9)}     {wavelength (WL)} \\
 \specel   {(10)}    {lapse rate (TLR)} \\
 \specel   {(11,12)} {refraction constants A and B (radians)} \\
 \specel   {(13)}    {longitude + eqn of equinoxes +
                       ``sidereal $\Delta$UT'' (radians)} \\
 \specel   {(14)}    {local apparent sidereal time (radians)}
}
\notes
{
 \begin{enumerate}
  \item It is advisable to take great care with units, as even
        unlikely values of the input parameters are accepted and
        processed in accordance with the models used.
  \item The DATE argument is UTC expressed as an MJD.  This is,
        strictly speaking, wrong, because of leap seconds.  However,
        as long as the $\Delta$UT and the UTC are consistent there
        are no difficulties, except during a leap second.  In this
        case, the start of the 61st second of the final minute should
        begin a new MJD day and the old pre-leap $\Delta$UT should
        continue to be used.  As the 61st second completes, the MJD
        should revert to the start of the day as, simultaneously,
        the $\Delta$UT changes by one second to its post-leap new value.
  \item The $\Delta$UT (UT1$-$UTC) is tabulated in IERS circulars and
        elsewhere.  It increases by exactly one second at the end of
        each UTC leap second, introduced in order to keep $\Delta$UT
        within $\pm0$\tsec9.  The ``sidereal $\Delta$UT'' which forms
        part of AOPRMS(13) is the same quantity, but converted from solar
        to sidereal seconds and expressed in radians.
  \item IMPORTANT -- TAKE CARE WITH THE LONGITUDE SIGN CONVENTION.  The
        longitude required by the present routine is {\bf east-positive},
        in accordance with geographical convention (and right-handed).
        In particular, note that the longitudes returned by the
        sla\_OBS routine are west-positive, following astronomical
        usage, and must be reversed in sign before use in the present
        routine.
  \item The polar coordinates XP,YP can be obtained from IERS
        circulars and equivalent publications.  The
        maximum amplitude is about 0.3~arcseconds.  If XP,YP values
        are unavailable, use XP=YP=0D0.  See page B60 of the 1988
        {\it Astronomical Almanac}\, for a definition of the two angles.
  \item The height above sea level of the observing station, HM,
        can be obtained from the {\it Astronomical Almanac}\, (Section J
        in the 1988 edition), or via the routine sla\_OBS.  If P,
        the pressure in millibars, is available, an adequate
        estimate of HM can be obtained from the following expression:
        \begin{quote}
         \verb|HM=-8149.9415D0*LOG(P/1013.25D0)|
        \end{quote}
        (See {\it Astrophysical Quantities}, C.W.Allen, 3rd~edition,
        \S52.)  Similarly, if the pressure P is not known,
        it can be estimated from the height of the observing
        station, HM as follows:
        \begin{quote}
         \verb|P=1013.25D0*EXP(-HM/8149.9415D0)|
        \end{quote}
        Note, however, that the refraction is proportional to the
        pressure and that an accurate P value is important for
        precise work.
 \end{enumerate}
}
%-----------------------------------------------------------------------
\routine{SLA\_AOPPAT}{Update Appt-to-Obs Parameters}
{
 \action{Recompute the sidereal time in the apparent to observed place
         star-independent parameter block.}
 \call{CALL sla\_AOPPAT (DATE, AOPRMS)}
}
\args{GIVEN}
{
 \spec{DATE}{D}{UTC date/time (Modified Julian Date, JD$-$2400000.5)} \\
 \spec{AOPRMS}{D(14)}{star-independent apparent-to-observed parameters:} \\
 \specel{(1-12)}{not required} \\
 \specel{(13)}{longitude + eqn of equinoxes +
               ``sidereal $\Delta$UT'' (radians)} \\
 \specel{(14)}{not required}
}
\args{RETURNED}
{
 \spec{AOPRMS}{D(14)}{star-independent apparent-to-observed parameters:} \\
 \specel{(1-13)}{not changed} \\
 \specel{(14)}{local apparent sidereal time (radians)}
}
\anote{For more information, see sla\_AOPPA.}
%-----------------------------------------------------------------------
\routine{SLA\_AOPQK}{Quick Appt-to-Observed}
{
 \action{Quick apparent to observed place.}
 \call{CALL sla\_AOPQK (RAP, DAP, AOPRMS, AOB, ZOB, HOB, DOB, ROB)}
}
\args{GIVEN}
{
 \spec{RAP,DAP}{D}{geocentric apparent \radec\ (radians)} \\
 \spec{AOPRMS}{D(14)}{star-independent apparent-to-observed parameters:} \\
 \specel{(1)}{geodetic latitude (radians)} \\
 \specel{(2,3)}{sine and cosine of geodetic latitude} \\
 \specel{(4)}{magnitude of diurnal aberration vector} \\
 \specel{(5)}{height (metres)} \\
 \specel{(6)}{ambient temperature (degrees K)} \\
 \specel{(7)}{pressure (mB)} \\
 \specel{(8)}{relative humidity ($0-1$)} \\
 \specel{(9)}{wavelength (micron)} \\
 \specel{(10)}{lapse rate (degrees K per metre)} \\
 \specel{(11,12)}{refraction constants A and B (radians)} \\
 \specel{(13)}{longitude + eqn of equinoxes +
               ``sidereal $\Delta$UT'' (radians)} \\
 \specel{(14)}{local apparent sidereal time (radians)}
}
\args{RETURNED}
{
 \spec{AOB}{D}{observed azimuth (radians: N=0, E=$90^{\circ}$)} \\
 \spec{ZOB}{D}{observed zenith distance (radians)} \\
 \spec{HOB}{D}{observed Hour Angle (radians)} \\
 \spec{DOB}{D}{observed Declination (radians)} \\
 \spec{ROB}{D}{observed Right Ascension (radians)}
}
\notes
{
 \begin{enumerate}
  \item This routine returns zenith distance rather than elevation
        in order to reflect the fact that no allowance is made for
        depression of the horizon.
  \item The accuracy of the result is limited by the corrections for
        refraction.  Providing the meteorological parameters are
        known accurately and there are no gross local effects, the
        predicted azimuth and elevation should be within about
        0.1~arcsec for $\zeta<70^{\circ}$.  Even
        at a topocentric zenith distance of
        $90^{\circ}$, the accuracy in elevation should be better than
        1~arcmin;  useful results are available for a further
        $3^{\circ}$, beyond which the sla\_REFRO routine returns a
        fixed value of the refraction.  The complementary
        routines sla\_AOP (or sla\_AOPQK) and sla\_OAP (or sla\_OAPQK)
        are self-consistent to better than 1~microarcsecond all over
        the celestial sphere.
  \item It is advisable to take great care with units, as even
        unlikely values of the input parameters are accepted and
        processed in accordance with the models used.
  \item {\it Apparent}\, \radec\ means the geocentric apparent right ascension
        and declination, which is obtained from a catalogue mean place
        by allowing for space motion, parallax, precession, nutation,
        annual aberration, and the Sun's gravitational lens effect.  For
        star positions in the FK5 system ({\it i.e.}\ J2000), these effects can
        be applied by means of the sla\_MAP {\it etc.}\ routines.  Starting from
        other mean place systems, additional transformations will be
        needed;  for example, FK4 ({\it i.e.}\ B1950) mean places would first
        have to be converted to FK5, which can be done with the
        sla\_FK425 {\it etc.}\ routines.
  \item {\it Observed}\, \azel\ means the position that would be seen by a
        perfect theodolite located at the observer.  This is obtained
        from the geocentric apparent \radec\ by allowing for Earth
        orientation and diurnal aberration, rotating from equator
        to horizon coordinates, and then adjusting for refraction.
        The \hadec\ is obtained by rotating back into equatorial
        coordinates, using the geodetic latitude corrected for polar
        motion, and is the position that would be seen by a perfect
        equatorial located at the observer and with its polar axis
        aligned to the Earth's axis of rotation ({\it n.b.}\ not to the
        refracted pole).  Finally, the $\alpha$ is obtained by subtracting
        the {\it h}\, from the local apparent ST.
  \item To predict the required setting of a real telescope, the
        observed place produced by this routine would have to be
        adjusted for the tilt of the azimuth or polar axis of the
        mounting (with appropriate corrections for mount flexures),
        for non-perpendicularity between the mounting axes, for the
        position of the rotator axis and the pointing axis relative
        to it, for tube flexure, for gear and encoder errors, and
        finally for encoder zero points.  Some telescopes would, of
        course, exhibit other properties which would need to be
        accounted for at the appropriate point in the sequence.
  \item The star-independent apparent-to-observed-place parameters
        in AOPRMS may be computed by means of the sla\_AOPPA routine.
        If nothing has changed significantly except the time, the
        sla\_AOPPAT routine may be used to perform the requisite
        partial recomputation of AOPRMS.
  \item The  ``sidereal $\Delta$UT'' which forms part of AOPRMS(13)
        is UT1$-$UTC converted from solar to
        sidereal seconds and expressed in radians.

 \end{enumerate}
}
%-----------------------------------------------------------------------
\routine{SLA\_AV2M}{Rotation Matrix from Axial Vector}
{
 \action{Form the rotation matrix corresponding to a given axial vector
         (single precision).}
 \call{CALL sla\_AV2M (AXVEC, RMAT)}
}
\args{GIVEN}
{
 \spec{AXVEC}{R(3)}{axial vector (radians)}
}
\args{RETURNED}
{
 \spec{RMAT}{R(3,3)}{rotation matrix}
}
\notes
{
 \begin{enumerate}
  \item A rotation matrix describes a rotation about some arbitrary axis.
        The axis is called the {\it Euler axis}, and the angle through which the
        reference frame rotates is called the Euler angle.  The axial
        vector supplied to this routine has the same direction as the
        Euler axis, and its magnitude is the Euler angle in radians.
  \item If AXVEC is null, the unit matrix is returned.
  \item The reference frame rotates clockwise as seen looking along
        the axial vector from the origin.
 \end{enumerate}
}
%-----------------------------------------------------------------------
\routine{SLA\_BEAR}{Direction Between Points on a Sphere}
{
 \action{Returns the bearing (position angle) of one point on a
         sphere seen from another (single precision).}
 \call{R~=~sla\_BEAR (A1, B1, A2, B2)}
}
\args{GIVEN}
{
 \spec{A1,B1}{R}{spherical coordinates of one point} \\
 \spec{A2,B2}{R}{spherical coordinates of the other point}
}
\args{RETURNED}
{
 \spec{sla\_BEAR}{R}{bearing from first point to second}
}
\notes
{
 \begin{enumerate}
 \item The spherical coordinates are \radec,
       $[\lambda,\phi]$ {\it etc.}, in radians.
 \item The result is the bearing (position angle), in radians,
       of point [A2,B2] as seen
       from point [A1,B1].  It is in the range $\pm \pi$.  If [A2,B2]
       is due east of [A1,B1] the reulst is $+\pi/2$. Zero is returned
       if the two points are coincident.
 \end{enumerate}
}
%-----------------------------------------------------------------------
\routine{SLA\_CAF2R}{Deg,Arcmin,Arcsec to Radians}
{
 \action{Convert degrees, arcminutes, arcseconds to radians
         (single precision).}
 \call{CALL sla\_CAF2R (IDEG, IAMIN, ASEC, RAD, J)}
}
\args{GIVEN}
{
 \spec{IDEG}{I}{degrees} \\
 \spec{IAMIN}{I}{arcminutes} \\
 \spec{ASEC}{R}{arcseconds}
}
\args{RETURNED}
{
 \spec{RAD}{R}{angle in radians} \\
 \spec{J}{I}{status:} \\
 \spec{}{}{\hspace{1.5em} 1 = IDEG outside range 0$-$359} \\
 \spec{}{}{\hspace{1.5em} 2 = IAMIN outside range 0$-$59} \\
 \spec{}{}{\hspace{1.5em} 3 = ASEC outside range 0$-$59.999$\cdots$}
}
\notes
{
 \begin{enumerate}
  \item The result is computed even if any of the range checks fail.
  \item The sign must be dealt with outside this routine.
 \end{enumerate}
}
%-----------------------------------------------------------------------
\routine{SLA\_CALDJ}{Calendar Date to MJD}
{
 \action{Gregorian Calendar to Modified Julian Date, with century default.}
 \call{CALL sla\_CALDJ (IY, IM, ID, DJM, J)}
}
\args{GIVEN}
{
 \spec{IY,IM,ID}{I}{year, month, day in Gregorian calendar}
}
\args{RETURNED}
{
 \spec{DJM}{D}{modified Julian Date (JD$-$2400000.5) for $0^{\rm h}$} \\
 \spec{J}{I}{status:} \\
 \spec{}{}{\hspace{1.5em} 0 = OK} \\
 \spec{}{}{\hspace{1.5em} 1 = bad year   (MJD not computed)} \\
 \spec{}{}{\hspace{1.5em} 2 = bad month  (MJD not computed)} \\
 \spec{}{}{\hspace{1.5em} 3 = bad day    (MJD computed)} \\
}
\notes
{
 \begin{enumerate}
  \item This routine supports the {\it century default}\, feature.
        Acceptable years are:
        \begin{itemize}
         \item 00-49, interpreted as 2000-2049,
         \item 50-99, interpreted as 1950-1999, and
         \item 100 upwards, interpreted literally.
        \end{itemize}
        For 1-100AD use the routine sla\_CLDJ instead.
  \item For year $n$BC use IY = $-(n-1)$.
  \item When an invalid year or month is supplied (status J~=~1~or~2)
        the MJD is {\bf not} computed.  When an invalid day is supplied
        (status J~=~3) the MJD {\bf is} computed.
 \end{enumerate}
}
%-----------------------------------------------------------------------
\routine{SLA\_CALYD}{Calendar to Year, Day}
{
 \action{Gregorian calendar to year and day-in-year.}
 \call{CALL sla\_CALYD (IY, IM, ID, NY, ND, J)}
}
\args{GIVEN}
{
 \spec{IY,IM,ID}{I}{year, month, day in Gregorian calendar}
}
\args{RETURNED}
{
 \spec{NY}{I}{year AD (usually the same as IY)} \\
 \spec{ND}{I}{day in year (1 = January 1st)} \\
 \spec{J}{I}{status:} \\
 \spec{}{}{\hspace{0.7em} $-$1
 = OK, but outside range 1900~Mar~1 - 2100~Feb~28} \\
 \spec{}{}{\hspace{1.5em}  0 = OK, and within above range} \\
 \spec{}{}{\hspace{1.5em}  2 = bad month} \\
 \spec{}{}{\hspace{1.5em}  3 = bad day}
}
\notes
{
 \begin{enumerate}
  \item This routine supports the {\it century default}\, feature.
        Acceptable years are:
        \begin{itemize}
         \item 00-49, interpreted as 2000-2049,
         \item 50-99, interpreted as 1950-1999, and
         \item 100 upwards, interpreted literally.
        \end{itemize}
  \item For year $n$BC use IY = $-(n-1)$.
  \item When an invalid month is supplied (status J~=~1~or~2)
        the day is {\bf not} computed.  When an invalid day is
        supplied (status J~=~3) the year and day {\bf are} computed.
  \item J=0 means that the date is within the range where the Gregorian
        rule concerning century leap years can be neglected.
 \end{enumerate}
}
%-----------------------------------------------------------------------
\routine{SLA\_CC2S}{Cartesian to Spherical}
{
 \action{Cartesian coordinates to spherical coordinates (single precision).}
 \call{CALL sla\_CC2S (V, A, B)}
}
\args{GIVEN}
{
 \spec{V}{R(3)}{\xyz\ vector}
}
\args{RETURNED}
{
 \spec{A,B}{R}{spherical coordinates in radians}
}
\notes
{
 \begin{enumerate}
  \item The spherical coordinates are longitude (+ve anticlockwise
        looking from the +ve latitude pole) and latitude.  The
        Cartesian coordinates are right handed, with the {\it x}-axis
        at zero longitude and latitude, and the {\it z}-axis at the
        +ve latitude pole.
  \item If V is null, zero A and B are returned.
  \item At either pole, zero A is returned.
 \end{enumerate}
}
%-----------------------------------------------------------------------
\routine{SLA\_CC62S}{Cartesian 6-Vector to Spherical}
{
 \action{Conversion of position \& velocity in Cartesian coordinates
         to spherical coordinates (single precision).}
 \call{CALL sla\_CC62S (V, A, B, R, AD, BD, RD)}
}
\args{GIVEN}
{
 \spec{V}{R(6)}{\xyzxyzd}
}
\args{RETURNED}
{
 \spec{A}{R}{longitude (radians) -- for example $\alpha$} \\
 \spec{B}{R}{latitude (radians) -- for example $\delta$} \\
 \spec{R}{R}{radial coordinate} \\
 \spec{AD}{R}{longitude derivative (radians per unit time)} \\
 \spec{BD}{R}{latitude derivative (radians per unit time)} \\
 \spec{RD}{R}{radial derivative}
}
%-----------------------------------------------------------------------
\routine{SLA\_CD2TF}{Days to Hour,Min,Sec}
{
 \action{Convert an interval in days to hours, minutes, seconds
         (single precision).}
 \call{CALL sla\_CD2TF (NDP, DAYS, SIGN, IHMSF)}
}
\args{GIVEN}
{
 \spec{NDP}{I}{number of decimal places of seconds} \\
 \spec{DAYS}{R}{interval in days}
}
\args{RETURNED}
{
 \spec{SIGN}{C}{`+' or `$-$'} \\
 \spec{IHMSF}{I(4)}{hours, minutes, seconds, fraction}
}
\notes
{
 \begin{enumerate}
  \item NDP less than zero is interpreted as zero.
  \item The largest useful value for NDP is determined by the size of
        DAYS, the format of REAL floating-point numbers on the target
        machine, and the risk of overflowing IHMSF(4).  For example,
        on the VAX, for DAYS up to 1.0, the available floating-point
        precision corresponds roughly to NDP=3.  This is well below
        the ultimate limit of NDP=9 set by the capacity of the 32-bit
        integer IHMSF(4).
  \item The absolute value of DAYS may exceed 1.0.  In cases where it
        does not, it is up to the caller to test for and handle the
        case where DAYS is very nearly 1.0 and rounds up to 24~hours,
        by testing for IHMSF(1)=24 and setting IHMSF(1-4) to zero.
\end{enumerate}
}
%-----------------------------------------------------------------------
\routine{SLA\_CLDJ}{Calendar to MJD}
{
 \action{Gregorian Calendar to Modified Julian Date.}
 \call{CALL sla\_CLDJ (IY, IM, ID, DJM, J)}
}
\args{GIVEN}
{
 \spec{IY,IM,ID}{I}{year, month, day in Gregorian calendar}
}
\args{RETURNED}
{
 \spec{DJM}{D}{modified Julian Date (JD$-$2400000.5) for $0^{\rm h}$} \\
 \spec{J}{I}{status:} \\
 \spec{}{}{\hspace{1.5em} 0 = OK} \\
 \spec{}{}{\hspace{1.5em} 1 = bad year} \\
 \spec{}{}{\hspace{1.5em} 2 = bad month} \\
 \spec{}{}{\hspace{1.5em} 3 = bad day}
}
\notes
{
 \begin{enumerate}
  \item When an invalid year or month is supplied (status J~=~1~or~2)
        the MJD is {\bf not} computed.  When an invalid day is supplied
        (status J~=~3) the MJD {\bf is} computed.
  \item The year must be $-$4699 ({\it i.e.}\ 4700BC) or later.
        For year $n$BC use IY = $-(n-1)$.
  \item An alternative to the present routine is sla\_CALDJ, which
        accepts a year with the century missing.
 \end{enumerate}
}
\aref{The algorithm is derived from that of Hatcher,
      {\it Q.\,Jl.\,R.\,astr.\,Soc.}\ (1984) {\bf 25}, 53-55.}
%-----------------------------------------------------------------------
\routine{SLA\_CR2AF}{Radians to Deg,Arcmin,Arcsec}
{
 \action{Convert an angle in radians to degrees, arcminutes,
         arcseconds (single precision).}
 \call{CALL sla\_CR2AF (NDP, ANGLE, SIGN, IDMSF)}
}
\args{GIVEN}
{
 \spec{NDP}{I}{number of decimal places of arcseconds} \\
 \spec{ANGLE}{R}{angle in radians}
}
\args{RETURNED}
{
 \spec{SIGN}{C}{`+' or `$-$'} \\
 \spec{IDMSF}{I(4)}{degrees, arcminutes, arcseconds, fraction}
}
\notes
{
 \begin{enumerate}
  \item NDP less than zero is interpreted as zero.
  \item The largest useful value for NDP is determined by the size of
        ANGLE, the format of REAL floating-point numbers on the target
        machine, and the risk of overflowing IDMSF(4).  For example,
        on the VAX, for ANGLE up to $2\pi$, the available floating-point
        precision corresponds roughly to NDP=3.  This is well below
        the ultimate limit of NDP=9 set by the capacity of the 32-bit
        integer IHMSF(4).
  \item The absolute value of ANGLE may exceed $2\pi$.  In cases where it
        does not, it is up to the caller to test for and handle the
        case where ANGLE is very nearly $2\pi$ and rounds up to $360^{\circ}$,
        by testing for IDMSF(1)=360 and setting IDMSF(1-4) to zero.
 \end{enumerate}
}
%-----------------------------------------------------------------------
\routine{SLA\_CR2TF}{Radians to Hour,Min,Sec}
{
 \action{Convert an angle in radians to hours, minutes, seconds
         (single precision).}
 \call{CALL sla\_CR2TF (NDP, ANGLE, SIGN, IHMSF)}
}
\args{GIVEN}
{
 \spec{NDP}{I}{number of decimal places of seconds} \\
 \spec{ANGLE}{R}{angle in radians}
}
\args{RETURNED}
{
 \spec{SIGN}{C}{`+' or `$-$'} \\
 \spec{IHMSF}{I(4)}{hours, minutes, seconds, fraction}
}
\notes
{
 \begin{enumerate}
  \item NDP less than zero is interpreted as zero.
  \item The largest useful value for NDP is determined by the size of
        ANGLE, the format of REAL floating-point numbers on the target
        machine, and the risk of overflowing IHMSF(4).  For example,
        on the VAX, for ANGLE up to $2\pi$, the available floating-point
        precision corresponds roughly to NDP=3.  This is well below
        the ultimate limit of NDP=9 set by the capacity of the 32-bit
        integer IHMSF(4).
  \item The absolute value of ANGLE may exceed $2\pi$.  In cases where it
        does not, it is up to the caller to test for and handle the
        case where ANGLE is very nearly $2\pi$ and rounds up to 24~hours,
        by testing for IHMSF(1)=24 and setting IHMSF(1-4) to zero.
\end{enumerate}
}
%-----------------------------------------------------------------------
\routine{SLA\_CS2C}{Spherical to Cartesian}
{
 \action{Spherical coordinates to Cartesian coordinates (single precision).}
 \call{CALL sla\_CS2C (A, B, V)}
}
\args{GIVEN}
{
 \spec{A,B}{R}{spherical coordinates in radians: \radec\ {\it etc.}}
}
\args{RETURNED}
{
 \spec{V}{R(3)}{\xyz\ unit vector}
}
\anote{The spherical coordinates are longitude (+ve anticlockwise
       looking from the +ve latitude pole) and latitude.  The
       Cartesian coordinates are right handed, with the {\it x}-axis
       at zero longitude and latitude, and the {\it z}-axis at the
       +ve latitude pole.}
%-----------------------------------------------------------------------
\routine{SLA\_CS2C6}{Spherical Pos/Vel to Cartesian}
{
 \action{Conversion of position \& velocity in spherical coordinates
         to Cartesian coordinates (single precision).}
 \call{CALL sla\_CS2C6 (A, B, R, AD, BD, RD, V)}
}
\args{GIVEN}
{
 \spec{A}{R}{longitude (radians) -- for example $\alpha$} \\
 \spec{B}{R}{latitude (radians) -- for example $\delta$} \\
 \spec{R}{R}{radial coordinate} \\
 \spec{AD}{R}{longitude derivative (radians per unit time)} \\
 \spec{BD}{R}{latitude derivative (radians per unit time)} \\
 \spec{RD}{R}{radial derivative}
}
\args{RETURNED}
{
 \spec{V}{R(6)}{\xyzxyzd}
}
%-----------------------------------------------------------------------
\routine{SLA\_CTF2D}{Hour,Min,Sec to Days}
{
 \action{Convert hours, minutes, seconds to days (single precision).}
 \call{CALL sla\_CTF2D (IHOUR, IMIN, SEC, DAYS, J)}
}
\args{GIVEN}
{
 \spec{IHOUR}{I}{hours} \\
 \spec{IMIN}{I}{minutes} \\
 \spec{SEC}{R}{seconds}
}
\args{RETURNED}
{
 \spec{DAYS}{R}{interval in days} \\
 \spec{J}{I}{status:} \\
 \spec{}{}{\hspace{1.5em} 0 = OK} \\
 \spec{}{}{\hspace{1.5em} 1 = IHOUR outside range 0-23} \\
 \spec{}{}{\hspace{1.5em} 2 = IMIN outside range 0-59} \\
 \spec{}{}{\hspace{1.5em} 3 = SEC outside range 0-59.999$\cdots$}
}
\notes
{
 \begin{enumerate}
  \item The result is computed even if any of the range checks fail.
  \item The sign must be dealt with outside this routine.
 \end{enumerate}
}
%-----------------------------------------------------------------------
\routine{SLA\_CTF2R}{Hour,Min,Sec to Radians}
{
 \action{Convert hours, minutes, seconds to radians (single precision).}
 \call{CALL sla\_CTF2R (IHOUR, IMIN, SEC, RAD, J)}
}
\args{GIVEN}
{
 \spec{IHOUR}{I}{hours} \\
 \spec{IMIN}{I}{minutes} \\
 \spec{SEC}{R}{seconds}
}
\args{RETURNED}
{
 \spec{RAD}{R}{angle in radians} \\
 \spec{J}{I}{status:} \\
 \spec{}{}{\hspace{1.5em} 0 = OK} \\
 \spec{}{}{\hspace{1.5em} 1 = IHOUR outside range 0-23} \\
 \spec{}{}{\hspace{1.5em} 2 = IMIN outside range 0-59} \\
 \spec{}{}{\hspace{1.5em} 3 = SEC outside range 0-59.999$\cdots$}
}
\notes
{
 \begin{enumerate}
  \item The result is computed even if any of the range checks fail.
  \item The sign must be dealt with outside this routine.
 \end{enumerate}
}
%-----------------------------------------------------------------------
\routine{SLA\_DAF2R}{Deg,Arcmin,Arcsec to Radians}
{
 \action{Convert degrees, arcminutes, arcseconds to radians
  (double precision).}
 \call{CALL sla\_DAF2R (IDEG, IAMIN, ASEC, RAD, J)}
}
\args{GIVEN}
{
 \spec{IDEG}{I}{degrees} \\
 \spec{IAMIN}{I}{arcminutes} \\
 \spec{ASEC}{D}{arcseconds}
}
\args{RETURNED}
{
 \spec{RAD}{D}{angle in radians} \\
 \spec{J}{I}{status:} \\
 \spec{}{}{\hspace{1.5em} 1 = IDEG outside range 0$-$359} \\
 \spec{}{}{\hspace{1.5em} 2 = IAMIN outside range 0$-$59} \\
 \spec{}{}{\hspace{1.5em} 3 = ASEC outside range 0$-$59.999$\cdots$}
}
\notes
{
 \begin{enumerate}
  \item The result is computed even if any of the range checks fail.
  \item The sign must be dealt with outside this routine.
 \end{enumerate}
}
%-----------------------------------------------------------------------
\routine{SLA\_DAFIN}{Sexagesimal character string to angle}
{
 \action{Decode a free-format sexagesimal string (degrees, arcminutes,
         arcseconds) into a double precision floating point
         number (radians).}
 \call{CALL sla\_DAFIN (STRING, NSTRT, DRESLT, JF)}
}
\args{GIVEN}
{
 \spec{STRING}{C*(*)}{string containing deg, arcmin, arcsec fields} \\
 \spec{NSTRT}{I}{pointer to start of decode (beginning of STRING = 1)}
}
\args{RETURNED}
{
 \spec{NSTRT}{I}{advanced past the decoded angle} \\
 \spec{DRESLT}{D}{angle in radians} \\
 \spec{JF}{I}{status:} \\
 \spec{}{}{\hspace{1.5em}   0 = OK} \\
 \spec{}{}{\hspace{0.7em} $+1$ = default, DRESLT unchanged (note 2)} \\
 \spec{}{}{\hspace{0.7em} $-1$ = bad degrees (note 3)} \\
 \spec{}{}{\hspace{0.7em} $-2$ = bad arcminutes (note 3)} \\
 \spec{}{}{\hspace{0.7em} $-3$ = bad arcseconds (note 3)} \\
}
\goodbreak
\setlength{\oldspacing}{\topsep}
\setlength{\topsep}{0.3ex}
\begin{description}
 \item [EXAMPLE]: \\ [1.5ex]
  \begin{tabular}{p{9em}p{15em}p{15em}}
   {\it argument} & {\it before} & {\it after} \\ \\
   STRING & $'$\verb*|-57 17 44.806  12 34 56.7|$'$ & unchanged \\
   NSTRT & 1 & 16 ({\it i.e.}\ pointing to 12...) \\
   RESLT & - & $-1.00000$\verb|D0| \\
   JF & - & 0
  \end{tabular}
 \item A further call to sla\_DAFIN, without adjustment of NSTRT, will
       decode the second angle, \dms{12}{34}{56}{7}.
\end{description}
\setlength{\topsep}{\oldspacing}
\notes
{
 \begin{enumerate}
  \item The first three ``fields'' in STRING are degrees, arcminutes,
   arcseconds, separated by spaces or commas.  The degrees field
   may be signed, but not the others.  The decoding is carried
   out by the sla\_DFLTIN routine and is free-format.
  \item Successive fields may be absent, defaulting to zero.  For
   zero status, the only combinations allowed are degrees alone,
   degrees and arcminutes, and all three fields present.  If all
   three fields are omitted, a status of +1 is returned and DRESLT is
   unchanged.  In all other cases DRESLT is changed.
  \item Range checking:
   \begin{itemize}
    \item The degrees field is not range checked.  However, it is
     expected to be integral unless the other two fields are absent.
    \item The arcminutes field is expected to be 0-59, and integral if
     the arcseconds field is present.  If the arcseconds field
     is absent, the arcminutes is expected to be 0-59.9999...
    \item The arcseconds field is expected to be 0-59.9999...
    \item Decoding continues even when a check has failed.  Under these
     circumstances the field takes the supplied value, defaulting to
     zero, and the result DRESLT is computed and returned.
   \end{itemize}
   \item Further fields after the three expected ones are not treated as
    an error.  The pointer NSTRT is left in the correct state for
    further decoding with the present routine or with sla\_DFLTIN
    {\it etc}.  See the example, above.
   \item If STRING contains hours, minutes, seconds instead of
    degrees {\it etc},
    or if the required units are turns (or days) instead of radians,
    the result DRESLT should be multiplied as follows: \\ [1.5ex]
    \begin{tabular}{p{7em}p{7em}p{20em}}
    {\it for STRING} & {\it to obtain} & {\it multiply DRESLT by} \\ \\
    ${\circ}$~~\raisebox{-0.7ex}{$'$}~~\raisebox{-0.7ex}{$''$}
     & radians & $1.0$\verb|D0| \\
    ${\circ}$~~\raisebox{-0.7ex}{$'$}~~\raisebox{-0.7ex}{$''$}
     & turns & $1/{2 \pi} = 0.1591549430918953358$\verb|D0| \\
    h m s & radians & $15.0$\verb|D0| \\
    h m s & days & $15/{2\pi} = 2.3873241463784300365$\verb|D0|
   \end{tabular}
 \end{enumerate}
}
%------------------------------------------------------------------------------
\routine{SLA\_DAT}{TAI$-$UTC}
{
 \action{Increment to be applied to Coordinated Universal Time UTC to give
         International Atomic Time TAI.}
 \call{D~=~sla\_DAT (UTC)}
}
\args{GIVEN}
{
 \spec{UTC}{D}{UTC date as a modified JD (JD$-$2400000.5)}
}
\args{RETURNED}
{
 \spec{sla\_DAT}{D}{TAI$-$UTC in seconds}
}
\notes
{
 \begin{enumerate}
 \item Prior to 1972 January 1 a fixed value of 10~seconds is returned.
 \item This routine has to be updated on each occasion that a
       leap second is announced, and programs using it relinked.
       Refer to the program source code for information on when the
       most recent leap second was added.
 \end{enumerate}
}
%-----------------------------------------------------------------------
\routine{SLA\_DAV2M}{Rotation Matrix from Axial Vector}
{
 \action{Form the rotation matrix corresponding to a given axial vector
         (double precision).}
 \call{CALL sla\_DAV2M (AXVEC, RMAT)}
}
\args{GIVEN}
{
 \spec{AXVEC}{D(3)}{axial vector (radians)}
}
\args{RETURNED}
{
 \spec{RMAT}{D(3,3)}{rotation matrix}
}
\notes
{
 \begin{enumerate}
  \item A rotation matrix describes a rotation about some arbitrary axis.
        The axis is called the {\it Euler axis}, and the angle through which the
        reference frame rotates is called the {\it Euler angle}.  The axial
        vector supplied to this routine has the same direction as the
        Euler axis, and its magnitude is the Euler angle in radians.
  \item If AXVEC is null, the unit matrix is returned.
  \item The reference frame rotates clockwise as seen looking along
        the axial vector from the origin.
 \end{enumerate}
}
%-----------------------------------------------------------------------
\routine{SLA\_DBEAR}{Direction Between Points on a Sphere}
{
 \action{Returns the bearing (position angle) of one point on a
         sphere relative to another (double precision).}
 \call{D~=~sla\_DBEAR (A1, B1, A2, B2)}
}
\args{GIVEN}
{
 \spec{A1,B1}{D}{spherical coordinates of one point} \\
 \spec{A2,B2}{D}{spherical coordinates of the other point}
}
\args{RETURNED}
{
 \spec{sla\_DBEAR}{D}{bearing from first point to second}
}
\notes
{
 \begin{enumerate}
 \item The spherical coordinates are \radec,
       $[\lambda,\phi]$ {\it etc.}, in radians.
 \item The result is the bearing (position angle), in radians,
       of point [A2,B2] as seen
       from point [A1,B1].  It is in the range $\pm \pi$.  If [A2,B2]
       is due east of [A1,B1] the reulst is $+\pi/2$. Zero is returned
       if the two points are coincident.
 \end{enumerate}
}
%-----------------------------------------------------------------------
\routine{SLA\_DBJIN}{Decode String to B/J Epoch (DP)}
{
 \action{Decode a character string into a DOUBLE PRECISION number,
         with special provision for Besselian and Julian epochs.
         The string syntax is as for sla\_DFLTIN, prefixed by
         an optional `B' or `J'.}
 \call{CALL sla\_DBJIN (STRING, NSTRT, DRESLT, J1, J2)}
}
\args{GIVEN}
{
 \spec{STRING}{C}{string containing field to be decoded} \\
 \spec{NSTRT}{I}{pointer to first character of field in string}
}
\args{RETURNED}
{
 \spec{NSTRT}{I}{incremented past the decoded field} \\
 \spec{DRESLT}{D}{result} \\
 \spec{J1}{I}{DFLTIN status:} \\
 \spec{}{}{\hspace{0.7em} $-$1 = $-$OK} \\
 \spec{}{}{\hspace{1.5em}   0 = +OK} \\
 \spec{}{}{\hspace{1.5em}   1 = null field} \\
 \spec{}{}{\hspace{1.5em}   2 = error} \\
 \spec{J2}{I}{syntax flag:} \\
 \spec{}{}{\hspace{1.5em}   0 = normal DFLTIN syntax} \\
 \spec{}{}{\hspace{1.5em}   1 = `B' or `b'} \\
 \spec{}{}{\hspace{1.5em}   2 = `J' or `j'}
}
\notes
{
 \begin{enumerate}
  \item The purpose of the syntax extensions is to help cope with mixed
        FK4 and FK5 data, allowing fields such as `B1950' or `J2000'
        to be decoded.
  \item In addition to the syntax accepted by sla\_DFLTIN,
        the following two extensions are recognized by sla\_DBJIN:
        \begin{enumerate}
         \item A valid non-null field preceded by the character `B'
               (or `b') is accepted.
         \item A valid non-null field preceded by the character `J'
               (or `j') is accepted.
         \end{enumerate}
  \item The calling program is told of the `B' or `J' through an
        supplementary status argument.  The rest of
        the arguments are as for sla\_DFLTIN.
 \end{enumerate}
}
%-----------------------------------------------------------------------
\routine{SLA\_DC62S}{Cartesian 6-Vector to Spherical}
{
 \action{Conversion of position \& velocity in Cartesian coordinates
         to spherical coordinates (double precision).}
 \call{CALL sla\_DC62S (V, A, B, R, AD, BD, RD)}
}
\args{GIVEN}
{
 \spec{V}{D(6)}{\xyzxyzd}
}
\args{RETURNED}
{
 \spec{A}{D}{longitude (radians)} \\
 \spec{B}{D}{latitude (radians)} \\
 \spec{R}{D}{radial coordinate} \\
 \spec{AD}{D}{longitude derivative (radians per unit time)} \\
 \spec{BD}{D}{latitude derivative (radians per unit time)} \\
 \spec{RD}{D}{radial derivative}
}
%-----------------------------------------------------------------------
\routine{SLA\_DCC2S}{Cartesian to Spherical}
{
 \action{Cartesian coordinates to spherical coordinates (double precision).}
 \call{CALL sla\_DCC2S (V, A, B)}
}
\args{GIVEN}
{
 \spec{V}{D(3)}{\xyz\ vector}
}
\args{RETURNED}
{
 \spec{A,B}{D}{spherical coordinates in radians}
}
\notes
{
 \begin{enumerate}
  \item The spherical coordinates are longitude (+ve anticlockwise
        looking from the +ve latitude pole) and latitude.  The
        Cartesian coordinates are right handed, with the {\it x}-axis
        at zero longitude and latitude, and the {\it z}-axis at the
        +ve latitude pole.
  \item If V is null, zero A and B are returned.
  \item At either pole, zero A is returned.
 \end{enumerate}
}
%-----------------------------------------------------------------------
\routine{SLA\_DCMPF}{Interpret Linear Fit}
{
 \action{Decompose an \xy\ linear fit into its constituent parameters:
         zero points, scales, nonperpendicularity and orientation.}
 \call{CALL sla\_DCMPF (COEFFS,XZ,YZ,XS,YS,PERP,ORIENT)}
}
\args{GIVEN}
{
 \spec{COEFFS}{D(6)}{transformation coefficients (see note)}
}
\args{RETURNED}
{
 \spec{XZ}{D}{{\it x} zero point} \\
 \spec{YZ}{D}{{\it y} zero point} \\
 \spec{XS}{D}{{\it x} scale} \\
 \spec{YS}{D}{{\it y} scale} \\
 \spec{PERP}{D}{nonperpendicularity (radians)} \\
 \spec{ORIENT}{D}{orientation (radians)}
}
\notes
{
 \begin{enumerate}
  \item The model relates two sets of \xy\ coordinates as follows.
        Naming the six elements of COEFFS $a,b,c,d,e$ \& $f$,
        the model transforms coordinates $[x_{1},y_{1}\,]$ into coordinates
        $[x_{2},y_{2}\,]$ as follows:
        \begin{verse}
         $x_{2} = a + bx_{1} + cy_{1}$ \\
         $y_{2} = d + ex_{1} + fy_{1}$
        \end{verse}
        The sla\_DCMPF routine decomposes this transformation
        into four steps:
        \begin{enumerate}
        \item Zero points:
              \begin{verse}
               $x' = x_{1} + {\rm XZ}$ \\
               $y' = y_{1} + {\rm YZ}$
              \end{verse}
        \item Scales:
              \begin{verse}
               $x'' = x' {\rm XS}$ \\
               $y'' = y' {\rm YS}$
              \end{verse}
        \item Nonperpendicularity:
              \begin{verse}
               $x''' = + x'' \cos {\rm PERP} + y'' \sin {\rm PERP}$ \\
               $y''' = + x'' \sin {\rm PERP} + y'' \cos {\rm PERP}$
              \end{verse}
        \item Orientation:
              \begin{verse}
               $x_{2} = + x''' \cos {\rm ORIENT} +
                          y''' \sin {\rm ORIENT}$ \\
               $y_{2} = - x''' \sin {\rm ORIENT} +
                          y''' \cos {\rm ORIENT}$
              \end{verse}
        \end{enumerate}
  \item See also sla\_FITXY, sla\_PXY, sla\_INVF, sla\_XY2XY.
 \end{enumerate}
}
%-----------------------------------------------------------------------
\routine{SLA\_DCS2C}{Spherical to Cartesian}
{
 \action{Spherical coordinates to Cartesian coordinates (double precision).}
 \call{CALL sla\_DCS2C (A, B, V)}
}
\args{GIVEN}
{
 \spec{A,B}{D}{spherical coordinates in radians: \radec\ {\it etc.}}
}
\args{RETURNED}
{
 \spec{V}{D(3)}{\xyz\ unit vector}
}
\anote{The spherical coordinates are longitude (+ve anticlockwise
       looking from the +ve latitude pole) and latitude.  The
       Cartesian coordinates are right handed, with the {\it x}-axis
       at zero longitude and latitude, and the {\it z}-axis at the
       +ve latitude pole.}
%-----------------------------------------------------------------------
\routine{SLA\_DD2TF}{Days to Hour,Min,Sec}
{
 \action{Convert an interval in days into hours, minutes, seconds
         (double precision).}
 \call{CALL sla\_DD2TF (NDP, DAYS, SIGN, IHMSF)}
}
\args{GIVEN}
{
 \spec{NDP}{I}{number of decimal places of seconds} \\
 \spec{DAYS}{D}{interval in days}
}
\args{RETURNED}
{
 \spec{SIGN}{C}{`+' or `$-$'} \\
 \spec{IHMSF}{I(4)}{hours, minutes, seconds, fraction}
}
\notes
{
 \begin{enumerate}
  \item NDP less than zero is interpreted as zero.
  \item The largest useful value for NDP is determined by the size
        of DAYS, the format of DOUBLE PRECISION floating-point numbers
        on the target machine, and the risk of overflowing IHMSF(4).
        For example, on the VAX, for DAYS up to 1D0, the available
        floating-point precision corresponds roughly to NDP=12.  However,
        the practical limit is NDP=9, set by the capacity of the 32-bit
        integer IHMSF(4).
  \item The absolute value of DAYS may exceed 1D0.  In cases where it
        does not, it is up to the caller to test for and handle the
        case where DAYS is very nearly 1D0 and rounds up to 24~hours,
        by testing for IHMSF(1)=24 and setting IHMSF(1-4) to zero.
\end{enumerate}
}
%-----------------------------------------------------------------------
\routine{SLA\_DEULER}{Euler Angles to Rotation Matrix}
{
 \action{Form a rotation matrix from the Euler angles -- three
         successive rotations about specified Cartesian axes
         (double precision).}
 \call{CALL sla\_DEULER (ORDER, PHI, THETA, PSI, RMAT)}
}
\args{GIVEN}
{
 \spec{ORDER}{C}{specifies about which axes the rotations occur} \\
 \spec{PHI}{D}{1st rotation (radians)} \\
 \spec{THETA}{D}{2nd rotation (radians)} \\
 \spec{PSI}{D}{3rd rotation (radians)}
}
\args{RETURNED}
{
 \spec{RMAT}{D(3,3)}{rotation matrix}
}
\notes
{
 \begin{enumerate}
 \item A rotation is positive when the reference frame rotates
       anticlockwise as seen looking towards the origin from the
       positive region of the specified axis.
 \item The characters of ORDER define which axes the three successive
       rotations are about.  A typical value is `ZXZ', indicating that
       RMAT is to become the direction cosine matrix corresponding to
       rotations of the reference frame through PHI radians about the
       old {\it z}-axis, followed by THETA radians about the resulting
       {\it x}-axis,
       then PSI radians about the resulting {\it z}-axis.
 \item The axis names can be any of the following, in any order or
       combination:  X, Y, Z, uppercase or lowercase, 1, 2, 3.  Normal
       axis labelling/numbering conventions apply;  the {\it xyz} ($\equiv123$)
       triad is right-handed.  Thus, the `ZXZ' example given above
       could be written `zxz' or `313' (or even `ZxZ' or `3xZ').  ORDER
       is terminated by length or by the first unrecognized character.
       Fewer than three rotations are acceptable, in which case the later
       angle arguments are ignored.  Zero rotations produces a unit RMAT.
 \end{enumerate}
}
%-----------------------------------------------------------------------
\routine{SLA\_DFLTIN}{Decode a Double Precision Number}
{
 \action{Convert free-format input into double precision floating point.}
 \call{CALL sla\_DFLTIN (STRING, NSTRT, DRESLT, JFLAG)}
}
\args{GIVEN}
{
 \spec{STRING}{C}{string containing number to be decoded} \\
 \spec{NSTRT}{I}{pointer to where decoding is to commence} \\
 \spec{DRESLT}{D}{current value of result}
}
\args{RETURNED}
{
 \spec{NSTRT}{I}{advanced to next number} \\
 \spec{DRESLT}{D}{result} \\
 \spec{JFLAG}{I}{status: $-$1 = $-$OK, 0 = +OK, 1 = null result, 2 = error}
}
\notes
{
 \begin{enumerate}
 \item The reason sla\_DFLTIN has separate `OK' status values
       for + and $-$ is to enable minus zero to be detected.
       This is of crucial importance
       when decoding mixed-radix numbers.  For example, an angle
       expressed as degrees, arcminutes and arcseconds may have a
       leading minus sign but a zero degrees field.
 \item A TAB is interpreted as a space, and lowercase characters are
       interpreted as uppercase.  {\it n.b.}\ The test for TAB is
       {\bf VAX-specific} but may also work on other computers
       which use the ASCII character set.
 \item The basic format is the sequence of fields $\pm n.n x \pm n$,
       where $\pm$ is a sign
       character `+' or `$-$', $n$ means a string of decimal digits,
       `.' is a decimal point, and $x$, which indicates an exponent,
       means `D' or `E'.  Various combinations of these fields can be
       omitted, and embedded blanks are permissible in certain places.
 \item Spaces:
       \begin{itemize}
       \item Leading spaces are ignored.
       \item Embedded spaces are allowed only after +, $-$, D or E,
             and after the decimal point if the first sequence of
             digits is absent.
       \item Trailing spaces are ignored;  the first signifies
             end of decoding and subsequent ones are skipped.
       \end{itemize}
 \item Delimiters:
       \begin{itemize}
       \item Any character other than +,$-$,0-9,.,D,E or space may be
             used to signal the end of the number and terminate decoding.
       \item Comma is recognized by sla\_DFLTIN as a special case; it
             is skipped, leaving the pointer on the next character.  See
             13, below.
       \item Decoding will in all cases terminate if end of string
             is reached.
       \end{itemize}
 \item Both signs are optional.  The default is +.
 \item The mantissa $n.n$ defaults to unity.
 \item The exponent $x\!\pm\!n$ defaults to `D0'.
 \item The strings of decimal digits may be of any length.
 \item The decimal point is optional for whole numbers.
 \item A {\it null result}\, occurs when the string of characters
       being decoded does not begin with +,$-$,0-9,.,D or E, or
       consists entirely of spaces.  When this condition is
       detected, JFLAG is set to 1 and DRESLT is left untouched.
 \item NSTRT = 1 for the first character in the string.
 \item On return from sla\_DFLTIN, NSTRT is set ready for the next
       decode -- following trailing blanks and any comma.  If a
       delimiter other than comma is being used, NSTRT must be
       incremented before the next call to sla\_DFLTIN, otherwise
       all subsequent calls will return a null result.
 \item Errors (JFLAG=2) occur when:
       \begin{itemize}
       \item a +, $-$, D or E is left unsatisfied; or
       \item the decimal point is present without at least
             one decimal digit before or after it; or
       \item an exponent more than 100 has been presented.
       \end{itemize}
 \item When an error has been detected, NSTRT is left
       pointing to the character following the last
       one used before the error came to light.  This
       may be after the point at which a more sophisticated
       program could have detected the error.  For example,
       sla\_DFLTIN does not detect that `1D999' is unacceptable
       (on the VAX) until the entire number has been decoded.
 \item Certain highly unlikely combinations of mantissa and
       exponent can cause arithmetic faults during the
       decode, in some cases despite the fact that they
       together could be construed as a valid number.
 \item Decoding is left to right, one pass.
 \item See also sla\_FLOTIN and sla\_INTIN.
 \end{enumerate}
}
%-----------------------------------------------------------------------
\routine{SLA\_DIMXV}{Apply 3D Reverse Rotation}
{
 \action{Multiply a 3-vector by the inverse of a rotation
         matrix (double precision).}
 \call{CALL sla\_DIMXV (DM, VA, VB)}
}
\args{GIVEN}
{
 \spec{DM}{D(3,3)}{rotation matrix} \\
 \spec{VA}{D(3)}{vector to be rotated}
}
\args{RETURNED}
{
 \spec{VB}{D(3)}{result vector}
}
\notes
{
 \begin{enumerate}
  \item This routine performs the operation:
        \begin{verse}
         {\bf b} = {\bf M}$^{T}\cdot${\bf a}
        \end{verse}
        where {\bf a} and {\bf b} are the 3-vectors VA and VB
        respectively, and  {\bf M} is the $3\times3$ matrix DM.
  \item The main function of this routine is apply an inverse
        rotation;  under these circumstances, ${\bf \rm M}$ is
        {\it orthogonal}, with its inverse the same as its transpose.
  \item To comply with the ANSI Fortran 77 standard, VA and VB must
        {\bf not} be the same array.  The routine is, in fact, coded
        so as to work properly on the VAX and many other systems even
        if this rule is violated, something that is {\bf not}, however,
        recommended.
 \end{enumerate}
}
%-----------------------------------------------------------------------
\routine{SLA\_DJCAL}{MJD to Gregorian for Output}
{
 \action{Modified Julian Date to Gregorian Calendar Date, expressed
         in a form convenient for formatting messages (namely
         rounded to a specified precision, and with the fields
         stored in a single array).}
 \call{CALL sla\_DJCAL (NDP, DJM, IYMDF, J)}
}
\args{GIVEN}
{
 \spec{NDP}{I}{number of decimal places of days in fraction} \\
 \spec{DJM}{D}{modified Julian Date (JD$-$2400000.5)}
}
\args{RETURNED}
{
 \spec{IYMDF}{I(4)}{year, month, day, fraction in Gregorian calendar} \\
 \spec{J}{I}{status:  nonzero = out of range}
}
\notes
{
 \begin{enumerate}
  \item Any date after 4701BC March 1 is accepted.
  \item NDP should be 4 or less to avoid overflow on machines which
        use 32-bit integers.
 \end{enumerate}
}
\aref{The algorithm is derived from that of Hatcher,
      {\it Q.\,Jl.\,R.\,astr.\,Soc.}\ (1984) {\bf 25}, 53-55.}
%-----------------------------------------------------------------------
\routine{SLA\_DJCL}{MJD to Year,Month,Day,Frac}
{
 \action{Modified Julian Date to Gregorian year, month, day,
         and fraction of a day.}
 \call{CALL sla\_DJCL (DJM, IY, IM, ID, FD, J)}
}
\args{GIVEN}
{
 \spec{DJM}{D}{modified Julian Date (JD$-$2400000.5)}
}
\args{RETURNED}
{
 \spec{IY}{I}{year AD} \\
 \spec{IM}{I}{month} \\
 \spec{ID}{I}{day} \\
 \spec{FD}{D}{fraction of day} \\
 \spec{J}{I}{status:} \\
 \spec{}{}{\hspace{1.5em} 0 = OK} \\
 \spec{}{}{\hspace{0.7em} $-$1 = unacceptable date (before 4701BC March 1)}
}
\aref{The algorithm is derived from that of Hatcher,
      {\it Q.\,Jl.\,R.\,astr.\,Soc.}\ (1984) {\bf 25}, 53-55.}
%-----------------------------------------------------------------------
\routine{SLA\_DM2AV}{Rotation Matrix to Axial Vector}
{
 \action{From a rotation matrix, determine the corresponding axial vector
        (double precision).}
 \call{CALL sla\_DM2AV (RMAT, AXVEC)}
}
\args{GIVEN}
{
 \spec{RMAT}{D(3,3)}{rotation matrix}
}
\args{RETURNED}
{
 \spec{AXVEC}{D(3)}{axial vector (radians)}
}
\notes
{
 \begin{enumerate}
  \item A rotation matrix describes a rotation about some arbitrary axis.
        The axis is called the {\it Euler axis}, and the angle through
        which the reference frame rotates is called the {\it Euler angle}.
        The {\it axial vector}\, returned by this routine has the same
        direction as the Euler axis, and its magnitude is the Euler angle
        in radians.
  \item The magnitude and direction of the axial vector can be separated
        by means of the routine sla\_DVN.
  \item The reference frame rotates clockwise as seen looking along
        the axial vector from the origin.
  \item If RMAT is null, so is the result.
 \end{enumerate}
}
%-----------------------------------------------------------------------
\routine{SLA\_DMAT}{Solve Simultaneous Equations}
{
 \action{Matrix inversion and solution of simultaneous equations
         (double precision).}
 \call{CALL sla\_DMAT (N, A, Y, D, JF, IW)}
}
\args{GIVEN}
{
 \spec{N}{I}{number of unknowns} \\
 \spec{A}{D(N,N)}{matrix} \\
 \spec{Y}{D(N)}{vector}
}
\args{RETURNED}
{
 \spec{A}{D(N,N)}{matrix inverse} \\
 \spec{Y}{D(N)}{solution} \\
 \spec{D}{D}{determinant} \\
 \spec{JF}{I}{singularity flag: 0=OK} \\
 \spec{IW}{I(N)}{workspace}
}
\notes
{
 \begin{enumerate}
  \item For the set of $n$ simultaneous linear equations in $n$ unknowns:
        \begin{verse}
         {\bf A}$\cdot${\bf y} = {\bf x}
        \end{verse}
        where:
        \begin{itemize}
         \item {\bf A} is a non-singular $n \times n$ matrix,
         \item {\bf y} is the vector of $n$ unknowns, and
         \item {\bf x} is the known vector,
        \end{itemize}
        sla\_DMAT computes:
        \begin{itemize}
         \item the inverse of matrix {\bf A},
         \item the determinant of matrix {\bf A}, and
         \item the vector of $n$ unknowns {\bf y}.
        \end{itemize}
        Argument N is the order $n$, A (given) is the matrix {\bf A},
        Y (given) is the vector {\bf x} and Y (returned)
        is the vector {\bf y}.
        The argument A (returned) is the inverse matrix {\bf A}$^{-1}$,
        and D is {\it det}({\bf A}).
  \item JF is the singularity flag.  If the matrix is non-singular,
        JF=0 is returned.  If the matrix is singular, JF=$-$1
        and D=0D0 are returned.  In the latter case, the contents
        of array A on return are undefined.
  \item The algorithm is Gaussian elimination with partial pivoting.
        This method is very fast;  some much slower algorithms can give
        better accuracy, but only by a small factor.
  \item This routine replaces the obsolete sla\_DMATRX.
 \end{enumerate}
}
%-----------------------------------------------------------------------
\routine{SLA\_DMXM}{Multiply $3\times3$ Matrices}
{
 \action{Product of two $3\times3$ matrices (double precision).}
 \call{CALL sla\_DMXM (A, B, C)}
}
\args{GIVEN}
{
 \spec{A}{D(3,3)}{matrix {\bf A}} \\
 \spec{B}{D(3,3)}{matrix {\bf B}}
}
\args{RETURNED}
{
 \spec{C}{D(3,3)}{matrix result: {\bf A}$\cdot${\bf B}}
}
\anote{To comply with the ANSI Fortran 77 standard, A, B and C must
       be different arrays.  The routine is, in fact, coded
       so as to work properly on the VAX and many other systems even
       if this rule is violated, something that is {\bf not}, however,
       recommended.}
%-----------------------------------------------------------------------
\routine{SLA\_DMXV}{Apply 3D Rotation}
{
 \action{Multiply a 3-vector by a rotation matrix (double precision).}
 \call{CALL sla\_DMXV (DM, VA, VB)}
}
\args{GIVEN}
{
 \spec{DM}{D(3,3)}{rotation matrix} \\
 \spec{VA}{D(3)}{vector to be rotated}
}
\args{RETURNED}
{
 \spec{VB}{D(3)}{result vector}
}
\notes
{
 \begin{enumerate}
  \item This routine performs the operation:
        \begin{verse}
           {\bf b} = {\bf M}$\cdot${\bf a}
        \end{verse}
        where {\bf a} and {\bf b} are the 3-vectors VA and VB
        respectively, and {\bf M} is the $3\times3$ matrix DM.
  \item The main function of this routine is apply a
        rotation;  under these circumstances, {\bf M} is a
        {\it proper real orthogonal}\, matrix.
  \item To comply with the ANSI Fortran 77 standard, VA and VB must
        {\bf not} be the same array.  The routine is, in fact, coded
        so as to work properly on the VAX and many other systems even
        if this rule is violated, something that is {\bf not}, however,
        recommended.
 \end{enumerate}
}
%------------------------------------------------------------------------------
\routine{SLA\_DR2AF}{Radians to Deg,Min,Sec,Frac}
{
 \action{Convert an angle in radians to degrees, arcminutes, arcseconds,
         fraction (double precision).}
 \call{CALL sla\_DR2AF (NDP, ANGLE, SIGN, IDMSF)}
}
\args{GIVEN}
{
 \spec{NDP}{I}{number of decimal places of arcseconds} \\
 \spec{ANGLE}{D}{angle in radians}
}
\args{RETURNED}
{
 \spec{SIGN}{C}{`+' or `$-$'} \\
 \spec{IDMSF}{I(4)}{degrees, arcminutes, arcseconds, fraction}
}
\notes
{
 \begin{enumerate}
  \item NDP less than zero is interpreted as zero.
  \item The largest useful value for NDP is determined by the size
        of ANGLE, the format of DOUBLE~PRECISION floating-point numbers
        on the target machine, and the risk of overflowing IDMSF(4).
        For example, on the VAX, for ANGLE up to $2\pi$, the available
        floating-point precision corresponds roughly to NDP=12.  However,
        the practical limit is NDP=9, set by the capacity of the 32-bit
        integer IDMSF(4).
  \item The absolute value of ANGLE may exceed $2\pi$.  In cases where it
        does not, it is up to the caller to test for and handle the
        case where ANGLE is very nearly $2\pi$ and rounds up to $360^{\circ}$,
        by testing for IDMSF(1)=360 and setting IDMSF(1-4) to zero.
 \end{enumerate}
}
%-----------------------------------------------------------------------
\routine{SLA\_DR2TF}{Radians to Hour,Min,Sec,Frac}
{
 \action{Convert an angle in radians to hours, minutes, seconds,
         fraction (double precision).}
 \call{CALL sla\_DR2TF (NDP, ANGLE, SIGN, IHMSF)}
}
\args{GIVEN}
{
 \spec{NDP}{I}{number of decimal places of seconds} \\
 \spec{ANGLE}{D}{angle in radians}
}
\args{RETURNED}
{
 \spec{SIGN}{C}{`+' or `$-$'} \\
 \spec{IHMSF}{I(4)}{hours, minutes, seconds, fraction}
}
\notes
{
 \begin{enumerate}
  \item NDP less than zero is interpreted as zero.
  \item The largest useful value for NDP is determined by the size
        of ANGLE, the format of DOUBLE PRECISION floating-point numbers
        on the target machine, and the risk of overflowing IHMSF(4).
        For example, on the VAX, for ANGLE up to $2\pi$, the available
        floating-point precision corresponds roughly to NDP=12.  However,
        the practical limit is NDP=9, set by the capacity of the 32-bit
        integer IHMSF(4).
  \item The absolute value of ANGLE may exceed $2\pi$.  In cases where it
        does not, it is up to the caller to test for and handle the
        case where ANGLE is very nearly $2\pi$ and rounds up to 24~hours,
        by testing for IHMSF(1)=24 and setting IHMSF(1-4) to zero.
\end{enumerate}
}
%-----------------------------------------------------------------------
\routine{SLA\_DRANGE}{Put Angle into Range $\pm\pi$}
{
 \action{Normalize an angle into the range $\pm\pi$ (double precision).}
 \call{D~=~sla\_DRANGE (ANGLE)}
}
\args{GIVEN}
{
 \spec{ANGLE}{D}{angle in radians}
}
\args{RETURNED}
{
 \spec{sla\_DRANGE}{D}{ANGLE expressed in the range $\pm\pi$.}
}
%-----------------------------------------------------------------------
\routine{SLA\_DRANRM}{Put Angle into Range $0\!-\!2\pi$}
{
 \action{Normalize an angle into the range $0\!-\!2\pi$
         (double precision).}
 \call{D~=~sla\_DRANRM (ANGLE)}
}
\args{GIVEN}
{
 \spec{ANGLE}{D}{angle in radians}
}
\args{RETURNED}
{
 \spec{sla\_DRANRM}{D}{ANGLE expressed in the range $0\!-\!2\pi$}
}
%-----------------------------------------------------------------------
\routine{SLA\_DS2C6}{Spherical Pos/Vel to Cartesian}
{
 \action{Conversion of position \& velocity in spherical coordinates
         to Cartesian coordinates (double precision).}
 \call{CALL sla\_DS2C6 (A, B, R, AD, BD, RD, V)}
}
\args{GIVEN}
{
 \spec{A}{D}{longitude (radians) -- for example $\alpha$} \\
 \spec{B}{D}{latitude (radians) -- for example $\delta$} \\
 \spec{R}{D}{radial coordinate} \\
 \spec{AD}{D}{longitude derivative (radians per unit time)} \\
 \spec{BD}{D}{latitude derivative (radians per unit time)} \\
 \spec{RD}{D}{radial derivative}
}
\args{RETURNED}
{
 \spec{V}{D(6)}{\xyzxyzd}
}
%-----------------------------------------------------------------------
\routine{SLA\_DS2TP}{Spherical to Tangent Plane}
{
 \action{Projection of spherical coordinates onto the tangent plane
         (double precision).}
 \call{CALL sla\_DS2TP (RA, DEC, RAZ, DECZ, XI, ETA, J)}
}
\args{GIVEN}
{
 \spec{RA,DEC}{D}{spherical coordinates of point to be projected (radians)} \\
 \spec{RAZ,DECZ}{D}{spherical coordinates of tangent point (radians)}
}
\args{RETURNED}
{
 \spec{XI,ETA}{D}{rectangular coordinates on tangent plane (radians)} \\
 \spec{J}{I}{status:} \\
 \spec{}{}{\hspace{1.5em} 0 = OK, star on tangent plane} \\
 \spec{}{}{\hspace{1.5em} 1 = error, star too far from axis} \\
 \spec{}{}{\hspace{1.5em} 2 = error, antistar too far from axis} \\
 \spec{}{}{\hspace{1.5em} 3 = error, antistar on tangent plane}
}
\anote{This projection is called the {\it gnomonic}\, projection;  the
       \xy\ coordinates are called {\it standard coordinates}.  The latter
       are in units of the distance from the tangent plane to the projection
       point, {\it i.e.}\ radians near the origin.}
%-----------------------------------------------------------------------
\routine{SLA\_DSEP}{Angle Between 2 Points on Sphere}
{
 \action{Angle between two points on a sphere (double precision).}
 \call{D~=~sla\_DSEP (A1, B1, A2, B2)}
}
\args{GIVEN}
{
 \spec{A1,B1}{D}{spherical coordinates of one point (radians)} \\
 \spec{A2,B2}{D}{spherical coordinates of the other point (radians)}
}
\args{RETURNED}
{
 \spec{sla\_DSEP}{D}{angle between [A1,B1] and [A2,B2] in radians}
}
\notes
{
 \begin{enumerate}
  \item The spherical coordinates are right ascension and declination,
  longitude and latitude, {\it etc.}, in radians.
  \item The result is always positive.
 \end{enumerate}
}
%-----------------------------------------------------------------------
\routine{SLA\_DTF2D}{Hour,Min,Sec to Days}
{
 \action{Convert hours, minutes, seconds to days (double precision).}
 \call{CALL sla\_DTF2D (IHOUR, IMIN, SEC, DAYS, J)}
}
\args{GIVEN}
{
 \spec{IHOUR}{I}{hours} \\
 \spec{IMIN}{I}{minutes} \\
 \spec{SEC}{D}{seconds}
}
\args{RETURNED}
{
 \spec{DAYS}{D}{interval in days} \\
 \spec{J}{I}{status:} \\
 \spec{}{}{\hspace{1.5em} 0 = OK} \\
 \spec{}{}{\hspace{1.5em} 1 = IHOUR outside range 0-23} \\
 \spec{}{}{\hspace{1.5em} 2 = IMIN outside range 0-59} \\
 \spec{}{}{\hspace{1.5em} 3 = SEC outside range 0-59.999$\cdots$}
}
\notes
{
 \begin{enumerate}
  \item The result is computed even if any of the range checks fail.
  \item The sign must be dealt with outside this routine.
 \end{enumerate}
}
%-----------------------------------------------------------------------
\routine{SLA\_DTF2R}{Hour,Min,Sec to Radians}
{
 \action{Convert hours, minutes, seconds to radians (double precision).}
 \call{CALL sla\_DTF2R (IHOUR, IMIN, SEC, RAD, J)}
}
\args{GIVEN}
{
 \spec{IHOUR}{I}{hours} \\
 \spec{IMIN}{I}{minutes} \\
 \spec{SEC}{D}{seconds}
}
\args{RETURNED}
{
 \spec{RAD}{D}{angle in radians} \\
 \spec{J}{I}{status:} \\
 \spec{}{}{\hspace{1.5em} 0 = OK} \\
 \spec{}{}{\hspace{1.5em} 1 = IHOUR outside range 0-23} \\
 \spec{}{}{\hspace{1.5em} 2 = IMIN outside range 0-59} \\
 \spec{}{}{\hspace{1.5em} 3 = SEC outside range 0-59.999$\cdots$}
}
\notes
{
 \begin{enumerate}
  \item The result is computed even if any of the range checks fail.
  \item The sign must be dealt with outside this routine.
 \end{enumerate}
}
%-----------------------------------------------------------------------
\routine{SLA\_DTP2S}{Tangent Plane to Spherical}
{
 \action{Transform tangent plane coordinates into spherical
         coordinates (double precision)}
 \call{CALL sla\_DTP2S (XI, ETA, RAZ, DECZ, RA, DEC)}
}
\args{GIVEN}
{
 \spec{XI,ETA}{D}{tangent plane rectangular coordinates (radians)} \\
 \spec{RAZ,DECZ}{D}{spherical coordinates of tangent point (radians)}
}
\args{RETURNED}
{
 \spec{RA,DEC}{D}{spherical coordinates (radians)}
}
\anote{The tangent plane projection is called the {\it gnomonic}\,
       projection;  the \xy\ coordinates are called {\it standard coordinates}.
       The latter are in units of the distance from the tangent plane to the
       projection point, {\it i.e.}\ radians near the origin.}
%-----------------------------------------------------------------------
\routine{SLA\_DTT}{TT minus UTC}
{
 \action{Compute $\Delta$TT, the increment to be applied to
         Coordinated Universal Time UTC to give
         Terrestrial Time TT.}
 \call{D~=~sla\_DTT (DJU)}
}
\args{GIVEN}
{
 \spec{DJU}{D}{UTC date as a modified JD (JD$-$2400000.5)}
}
\args{RETURNED}
{
 \spec{sla\_DTT}{D}{TT$-$UTC in seconds}
}
\notes
{
 \begin{enumerate}
  \item Pre 1972 January 1 a fixed value of 10 + ET$-$TAI is returned.
  \item TT is one interpretation of the defunct timescale
        {\it Ephemeris Time}, ET.
 \end{enumerate}
}
%-----------------------------------------------------------------------
\routine{SLA\_DVDV}{Scalar Product}
{
 \action{Scalar product of two 3-vectors (double precision).}
 \call{D~=~sla\_DVDV (VA, VB)}
}
\args{GIVEN}
{
 \spec{VA}{D(3)}{first vector} \\
 \spec{VB}{D(3)}{second vector}
}
\args{RETURNED}
{
 \spec{sla\_DVDV}{D}{scalar product VA.VB}
}
%-----------------------------------------------------------------------
\routine{SLA\_DVN}{Normalize Vector}
{
 \action{Normalize a 3-vector, also giving the modulus (double precision).}
 \call{CALL sla\_DVN (V, UV, VM)}
}
\args{GIVEN}
{
 \spec{V}{D(3)}{vector}
}
\args{RETURNED}
{
 \spec{UV}{D(3)}{unit vector in direction of V} \\
 \spec{VM}{D}{modulus of V}
}
\anote{If the modulus of V is zero, UV is set to zero as well.}
%-----------------------------------------------------------------------
\routine{SLA\_DVXV}{Vector Product}
{
 \action{Vector product of two 3-vectors (double precision).}
 \call{CALL sla\_DVXV (VA, VB, VC)}
}
\args{GIVEN}
{
 \spec{VA}{D(3)}{first vector} \\
 \spec{VB}{D(3)}{second vector}
}
\args{RETURNED}
{
 \spec{VC}{D(3)}{vector product VA$\times$VB}
}
%-----------------------------------------------------------------------
\routine{SLA\_EARTH}{Approx Earth Pos/Vel}
{
 \action{Approximate heliocentric position and velocity of the Earth.}
 \call{CALL sla\_EARTH (IY, ID, FD, POSVEL)}
}
\args{GIVEN}
{
 \spec{IY}{I}{year AD} \\
 \spec{ID}{I}{day in year (1 = Jan 1st)} \\
 \spec{FD}{R}{fraction of day}
}
\args{RETURNED}
{
 \spec{POSVEL}{R(6)}{Earth \xyzxyzd\ (au, au~s$^{-1}$)}
}
\notes
{
 \begin{enumerate}
  \item The date and time is TDB (formerly ET) in the Gregorian
        calendar, and is interpreted in a manner which is valid
        between 1900~March~1 and 2100~February~28.
  \item The Earth heliocentric 6-vector is referred to the
        FK4 mean equator and equinox
        of date.
  \item Maximum/RMS errors 1950-2050:
        \begin{itemize}
         \item 13/5~$\times10^{-5}$~AU = 19200/7600~km in position
         \item 47/26~$\times10^{-10}$~AU~s$^{-1}$ =
               0.0070/0.0039~km~s$^{-1}$ in speed
        \end{itemize}
  \item More accurate results are obtainable with the routine sla\_EVP.
 \end{enumerate}
}
%-----------------------------------------------------------------------
\routine{SLA\_ECLEQ}{Ecliptic to Equatorial}
{
 \action{Transformation from ecliptic longitude and latitude to
         J2000.0 \radec.}
 \call{CALL sla\_ECLEQ (DL, DB, DATE, DR, DD)}
}
\args{GIVEN}
{
 \spec{DL,DB}{D}{ecliptic longitude and latitude
                          (mean of date, IAU 1980 theory, radians)} \\
 \spec{DATE}{D}{TDB (formerly ET) as Modified Julian Date
                                             (JD$-$2400000.5)}
}
\args{RETURNED}
{
 \spec{DR,DD}{D}{J2000.0 mean \radec\ (radians)}
}
%-----------------------------------------------------------------------
\routine{SLA\_ECMAT}{Form $\alpha,\delta\rightarrow\lambda,\beta$ Matrix}
{
 \action{Form the equatorial to ecliptic rotation matrix (IAU 1980 theory).}
 \call{CALL sla\_ECMAT (DATE, RMAT)}
}
\args{GIVEN}
{
 \spec{DATE}{D}{TDB (formerly ET) as Modified Julian Date
                                           (JD$-$2400000.5)}
}
\args{RETURNED}
{
 \spec{RMAT}{D(3,3)}{rotation matrix}
}
\notes
{
 \begin{enumerate}
  \item RMAT is matrix {\bf M} in the expression
        {\bf v}$_{ecl}$~=~{\bf M}$\cdot${\bf v}$_{equ}$.
  \item The equator, equinox and ecliptic are mean of date.
 \end{enumerate}
}
\aref{Murray, C.A., {\it Vectorial Astrometry}, section 4.3.}
%-----------------------------------------------------------------------
\routine{SLA\_ECOR}{RV \& Time Corrns to Sun}
{
 \action{Component of Earth orbit velocity and heliocentric
         light time in a given direction.}
 \call{CALL sla\_ECOR (RM, DM, IY, ID, FD, RV, TL)}
}
\args{GIVEN}
{
 \spec{RM,DM}{R}{mean \radec\ of date (radians)} \\
 \spec{IY}{I}{year AD} \\
 \spec{ID}{I}{day in year (1 = Jan 1st)} \\
 \spec{FD}{R}{fraction of day}
}
\args{RETURNED}
{
 \spec{RV}{R}{component of Earth orbital velocity (km~s$^{-1}$)} \\
 \spec{TL}{R}{component of heliocentric light time (s)}
}
\notes
{
 \begin{enumerate}
  \item The date and time is TDB (formerly ET) in the Gregorian
        calendar, and is interpreted in a manner which is valid
        between 1900~March~1 and 2100~February~28.
  \item Sign convention:
        \begin{itemize}
         \item The velocity component is +ve when the
               Earth is receding from
               the given point on the sky.
         \item The light time component is +ve
               when the Earth lies between the Sun and
               the given point on the sky.
        \end{itemize}
 \item Accuracy:
       \begin{itemize}
        \item The velocity component is usually within 0.004~km~s$^{-1}$
              of the correct value and is never in error by more than
              0.007~km~s$^{-1}$.
        \item The error in light time correction is about 0.03~s at worst,
              but is usually better than 0.01~s.
       \end{itemize}
       For applications requiring higher accuracy, see the sla\_EVP routine.
 \end{enumerate}
}
%-----------------------------------------------------------------------
\routine{SLA\_EG50}{B1950 $\alpha,\delta$ to Galactic}
{
 \action{Transformation from B1950.0 FK4 equatorial coordinates to
         IAU 1958 galactic coordinates.}
 \call{CALL sla\_EG50 (DR, DD, DL, DB)}
}
\args{GIVEN}
{
 \spec{DR,DD}{D}{B1950.0 \radec\ (radians)}
}
\args{RETURNED}
{
 \spec{DL,DB}{D}{galactic longitude and latitude \gal\ (radians)}
}
\anote{The equatorial coordinates are B1950.0 FK4.  Use the
       routine sla\_EQGAL if conversion from J2000.0 FK5 coordinates
       is required.}
\aref{Blaauw {\it et al.}, 1960, {\it Mon.Not.R.astr.Soc.},
      {\bf 121}, 123.}
%-----------------------------------------------------------------------
\routine{SLA\_EPB}{MJD to Besselian Epoch}
{
 \action{Conversion of Modified Julian Date to Besselian Epoch.}
 \call{D~=~sla\_EPB (DATE)}
}
\args{GIVEN}
{
 \spec{DATE}{D}{Modified Julian Date (JD$-$2400000.5)}
}
\args{RETURNED}
{
 \spec{sla\_EPB}{D}{Besselian Epoch}
}
\aref{Lieske, J.H., 1979, {\it Astr.Astrophys.}\ {\bf 73}, 282.}
%-----------------------------------------------------------------------
\routine{SLA\_EPB2D}{Besselian Epoch to MJD}
{
 \action{Conversion of Besselian Epoch to Modified Julian Date.}
 \call{D~=~sla\_EPB2D (EPB)}
}
\args{GIVEN}
{
 \spec{EPB}{D}{Besselian Epoch}
}
\args{RETURNED}
{
 \spec{sla\_EPB2D}{D}{Modified Julian Date (JD$-$2400000.5)}
}
\aref{Lieske, J.H., 1979. {\it Astr.Astrophys.}\ {\bf 73}, 282.}
%-----------------------------------------------------------------------
\routine{SLA\_EPCO}{Convert Epoch to B or J}
{
 \action{Convert an epoch to Besselian or Julian to match another one.}
 \call{D~=~sla\_EPCO (K0, K, E)}

}
\args{GIVEN}
{
 \spec{K0}{C}{form of result:  `B'=Besselian, `J'=Julian} \\
 \spec{K}{C}{form of given epoch:  `B' or `J'} \\
 \spec{E}{D}{epoch}
}
\args{RETURNED}
{
 \spec{sla\_EPCO}{D}{the given epoch converted as necessary}
}
\notes
{
 \begin{enumerate}
  \item The result is always either equal to or very close to
        the given epoch E.  The routine is required only in
        applications where punctilious treatment of heterogeneous
        mixtures of star positions is necessary.
  \item K0 and K are not validated.  They are interpreted as follows:
        \begin{itemize}
         \item If K0 and K are the same, the result is E.
         \item If K0 is `B' and K isn't, the conversion is J to B.
         \item In all other cases, the conversion is B to J.
        \end{itemize}
 \end{enumerate}
}
%-----------------------------------------------------------------------
\routine{SLA\_EPJ}{MJD to Julian Epoch}
{
 \action{Convert Modified Julian Date to Julian Epoch.}
 \call{D~=~sla\_EPJ (DATE)}
}
\args{GIVEN}
{
 \spec{DATE}{D}{Modified Julian Date (JD$-$2400000.5)}
}
\args{RETURNED}
{
 \spec{sla\_EPJ}{D}{Julian Epoch}
}
\aref{Lieske, J.H., 1979.\ {\it Astr.Astrophys.}, {\bf 73}, 282.}
%-----------------------------------------------------------------------
\routine{SLA\_EPJ2D}{Julian Epoch to MJD}
{
 \action{Convert Julian Epoch to Modified Julian Date.}
 \call{D~=~sla\_EPJ2D (EPJ)}
}
\args{GIVEN}
{
 \spec{EPJ}{D}{Julian Epoch}
}
\args{RETURNED}
{
 \spec{sla\_EPJ2D}{D}{Modified Julian Date (JD$-$2400000.5)}
}
\aref{Lieske, J.H., 1979.\ {\it Astr.Astrophys.}, {\bf 73}, 282.}
%-----------------------------------------------------------------------
\routine{SLA\_EQECL}{J2000 $\alpha,\delta$ to Ecliptic}
{
 \action{Transformation from J2000.0 equatorial coordinates to
         ecliptic longitude and latitude.}
 \call{CALL sla\_EQECL (DR, DD, DATE, DL, DB)}
}
\args{GIVEN}
{
 \spec{DR,DD}{D}{J2000.0 mean \radec\ (radians)} \\
 \spec{DATE}{D}{TDB (formerly ET) as Modified Julian Date (JD$-$2400000.5)}
}
\args{RETURNED}
{
 \spec{DL,DB}{D}{ecliptic longitude and latitude
                        (mean of date, IAU 1980 theory, radians)}
}
%-----------------------------------------------------------------------
\routine{SLA\_EQEQX}{Equation of the Equinoxes}
{
 \action{Equation of the equinoxes.}
 \call{D~=~sla\_EQEQX (DATE)}
}
\args{GIVEN}
{
 \spec{DATE}{D}{TDB (formerly ET) as Modified Julian Date (JD$-$2400000.5)}
}
\args{RETURNED}
{
 \spec{sla\_EQEQX}{D}{The equation of the equinoxes (radians)}
}
\anote{The equation of the equinoxes is added to a {\it mean}\,
       sidereal time to give the {\it apparent}\, sidereal time.}
%-----------------------------------------------------------------------
\routine{SLA\_EQGAL}{J2000 $\alpha,\delta$ to Galactic}
{
 \action{Transformation from J2000.0 FK5 equatorial coordinates to
 IAU 1958 galactic coordinates.}
 \call{CALL sla\_EQGAL (DR, DD, DL, DB)}
}
\args{GIVEN}
{
 \spec{DR,DD}{D}{J2000.0 \radec\ (radians)}
}
\args{RETURNED}
{
 \spec{DL,DB}{D}{galactic longitude and latitude \gal\ (radians)}
}
\anote{The equatorial coordinates are J2000.0 FK5.  Use the routine
       sla\_EG50 if conversion from B1950.0 FK4 coordinates is required.}
\aref{Blaauw {\it et al.}, 1960, {\it Mon.Not.R.astr.Soc.},
      {\bf 121}, 123.}
%-----------------------------------------------------------------------
\routine{SLA\_ETRMS}{E-terms of Aberration}
{
 \action{Compute the E-terms vector -- the part of the annual
         aberration which arises from the eccentricity of the
         Earth's orbit.}
 \call{CALL sla\_ETRMS (EP, EV)}
}
\args{GIVEN}
{
 \spec{EP}{D}{Besselian epoch}
}
\args{RETURNED}
{
 \spec{EV}{D(3)}{E-terms as $[\Delta x, \Delta y, \Delta z\,]$}
}
\anote{Note the use of the J2000 aberration constant (20.49552 arcsec).
       This is a reflection of the fact that the E-terms embodied in
       existing star catalogues were computed from a variety of
       aberration constants.  Rather than adopting one of the old
       constants the latest value is used here.}
\refs
{
 \begin{enumerate}
  \item Smith, C.A.\ {\it et al.}, 1989.  {\it Astr.J.}\ {\bf 97}, 265.
  \item Yallop, B.D.\ {\it et al.}, 1989.  {\it Astr.J.}\ {\bf 97}, 274.
 \end{enumerate}
}
%-----------------------------------------------------------------------
\routine{SLA\_EULER}{Rotation Matrix from Euler Angles}
{
 \action{Form a rotation matrix from the Euler angles -- three
         successive rotations about specified Cartesian axes
         (single precision).}
 \call{CALL sla\_EULER (ORDER, PHI, THETA, PSI, RMAT)}
}
\args{GIVEN}
{
 \spec{ORDER}{C*(*)}{specifies about which axes the rotations occur} \\
 \spec{PHI}{R}{1st rotation (radians)} \\
 \spec{THETA}{R}{2nd rotation (radians)} \\
 \spec{PSI}{R}{3rd rotation (radians)}
}
\args{RETURNED}
{
 \spec{RMAT}{R(3,3)}{rotation matrix}
}
\notes
{
 \begin{enumerate}
  \item A rotation is positive when the reference frame rotates
        anticlockwise as seen looking towards the origin from the
        positive region of the specified axis.
  \item The characters of ORDER define which axes the three successive
        rotations are about.  A typical value is `ZXZ', indicating that
        RMAT is to become the direction cosine matrix corresponding to
        rotations of the reference frame through PHI radians about the
        old {\it z}-axis, followed by THETA radians about the resulting
        {\it x}-axis,
        then PSI radians about the resulting {\it z}-axis.  In detail:
        \begin{itemize}
         \item The axis names can be any of the following, in any order or
               combination:  X, Y, Z, uppercase or lowercase, 1, 2, 3.  Normal
               axis labelling/numbering conventions apply;
               the {\it xyz} ($\equiv123$)
               triad is right-handed.  Thus, the `ZXZ' example given above
               could be written `zxz' or `313' (or even `ZxZ' or `3xZ').
         \item ORDER is terminated by length or by the first unrecognized
               character.
         \item Fewer than three rotations are acceptable, in which case
               the later angle arguments are ignored.
        \end{itemize}
  \item Zero rotations produces a unit RMAT.
 \end{enumerate}
}
%-----------------------------------------------------------------------
\routine{SLA\_EVP}{Earth Position \& Velocity}
{
 \action{Barycentric and heliocentric velocity and position of the Earth.}
 \call{CALL sla\_EVP (DATE, DEQX, DVB, DPB, DVH, DPH)}
}
\args{GIVEN}
{
 \spec{DATE}{D}{TDB (formerly ET) as a Modified Julian Date
                                        (JD$-$2400000.5)} \\
 \spec{DEQX}{D}{Julian Epoch ({\it e.g.}\ 2000D0) of mean equator and
                equinox of the vectors returned.  If DEQX~$<0$,
                  all vectors are referred to the mean equator and
                  equinox (FK5) of date DATE.}
}
\args{RETURNED}
{
 \spec{DVB}{D(3)}{barycentric \xyzd, au~s$^{-1}$} \\
 \spec{DPB}{D(3)}{barycentric \xyz, au} \\
 \spec{DVH}{D(3)}{heliocentric \xyzd, au~s$^{-1}$} \\
 \spec{DPH}{D(3)}{heliocentric \xyz, au}
}
\notes
{
 \begin{enumerate}
  \item This routine is used when accuracy is more important
        than CPU time, yet the extra complication of reading a
        pre-computed ephemeris is not justified.  The maximum
        deviations from the JPL~DE96 ephemeris are as follows:
        \begin{itemize}
         \item velocity (barycentric or heliocentric): 420~mm~s$^{-1}$
         \item position (barycentric): 6900~km
         \item position (heliocentric): 1600~km
        \end{itemize}
  \item The routine is an adaption of the BARVEL and BARCOR
        subroutines of P.Stumpff, which are described in
        {\it Astr.Astrophys.Suppl.Ser.}\ {\bf 41}, 1-8 (1980).
        Most of the changes are merely cosmetic and do not affect
        the results at all.  However, some adjustments have been
        made so as to give results that refer to the new (IAU 1976
        `FK5') equinox and precession, although the differences these
        changes make relative to the results from Stumpff's original
        `FK4' version are smaller than the inherent accuracy of the
        algorithm.  One minor shortcoming in the original routines
        that has {\bf not} been corrected is that slightly better
        numerical accuracy could be achieved if the various polynomial
        evaluations were to be so arranged that the smallest terms were
        computed first.  Note also that one of Stumpff's precession
        constants differs by 0.001~arcsec from the value given in the
        {\it Explanatory Supplement}.
 \end{enumerate}
}
%-----------------------------------------------------------------------
\routine{SLA\_FITXY}{Fit Linear Model to Two \xy\ Sets}
{
 \action{Fit a linear model to relate two sets of \xy\ coordinates.}
 \call{CALL sla\_FITXY (ITYPE,NP,XYE,XYM,COEFFS,J)}
}
\args{GIVEN}
{
 \spec{ITYPE}{I}{type of model: 4 or 6 (note 1)} \\
 \spec{NP}{I}{number of samples (note 2)} \\
 \spec{XYE}{D(2,NP)}{expected \xy\ for each sample} \\
 \spec{XYM}{D(2,NP)}{measured \xy\ for each sample}
}
\args{RETURNED}
{
 \spec{COEFFS}{D(6)}{coefficients of model (note 3)} \\
 \spec{J}{I}{status:} \\
 \spec{}{}{\hspace{1.5em} 0 = OK} \\
 \spec{}{}{\hspace{0.7em} $-$1 = illegal ITYPE} \\
 \spec{}{}{\hspace{0.7em} $-$2 = insufficient data} \\
 \spec{}{}{\hspace{0.7em} $-$3 = singular solution}
}
\notes
{
 \begin{enumerate}
  \item ITYPE, which must be either 4 or 6, selects the type of model
        fitted.  Both allowed ITYPE values produce a model COEFFS which
        consists of six coefficients, namely the zero points and, for
        each of XE and YE, the coefficient of XM and YM.  For ITYPE=6,
        all six coefficients are independent, modelling squash and shear
        as well as origin, scale, and orientation.  However, ITYPE=4
        selects the {\it solid body rotation}\, option;  the model COEFFS
        still consists of the same six coefficients, but now two of
        them are used twice (appropriately signed).  Origin, scale
        and orientation are still modelled, but not squash or shear --
        the units of X and Y have to be the same.
  \item For NC=4, NP must be at least 2.  For NC=6, NP must be at
        least 3.
  \item The model is returned in the array COEFFS.  Naming the
        six elements of COEFFS $a,b,c,d,e$ \& $f$,
        the model transforms {\it measured}\, coordinates
        $[x_{m},y_{m}\,]$ into {\it expected}\, coordinates
        $[x_{e},y_{e}\,]$ as follows:
        \begin{verse}
         $x_{e} = a + bx_{m} + cy_{m}$ \\
         $y_{e} = d + ex_{m} + fy_{m}$
        \end{verse}
        For the {\it solid body rotation}\, option (ITYPE=4), the
        magnitudes of $b$ and $f$, and of $c$ and $e$, are equal.  The
        signs of these coefficients depend on whether there is a
        sign reversal between $[x_{e},y_{e}]$ and $[x_{m},y_{m}]$;
        fits are performed
        with and without a sign reversal and the best one chosen.
  \item Error status values J=$-$1 and $-$2 leave COEFFS unchanged;
        if J=$-$3 COEFFS may have been changed.
  \item See also sla\_PXY, sla\_INVF, sla\_XY2XY, sla\_DCMPF.
 \end{enumerate}
}
%-----------------------------------------------------------------------
\routine{SLA\_FK425}{FK4 to FK5}
{
 \action{Convert B1950.0 FK4 star data to J2000.0 FK5.
         This routine converts stars from the old, Bessel-Newcomb, FK4
         system to the new, IAU~1976, FK5, Fricke system.  The precepts
         of Smith~{\it et~al.}\ (see reference~1) are followed,
         using the implementation
         by Yallop~{\it et~al.}\ (reference~2) of a matrix method
         due to Standish.
         Kinoshita's development of Andoyer's post-Newcomb precession is
         used.  The numerical constants from Seidelmann~{\it et~al.}\
         (reference~3) are used canonically.}
 \call{CALL sla\_FK425 (\vtop{
         \hbox{R1950,D1950,DR1950,DD1950,P1950,V1950,}
         \hbox{R2000,D2000,DR2000,DD2000,P2000,V2000)}}}
}
\args{GIVEN}
{
 \spec{R1950}{D}{B1950.0 $\alpha$ (radians)} \\
 \spec{D1950}{D}{B1950.0 $\delta$ (radians)} \\
 \spec{DR1950}{D}{B1950.0 proper motion in $\alpha$
                              (radians per tropical year)} \\
 \spec{DD1950}{D}{B1950.0 proper motion in $\delta$
                              (radians per tropical year)} \\
 \spec{P1950}{D}{B1950.0 parallax (arcsec)} \\
 \spec{V1950}{D}{B1950.0 radial velocity (km~s$^{-1}$, +ve = moving away)}
}
\args{RETURNED}
{
 \spec{R2000}{D}{J2000.0 $\alpha$ (radians)} \\
 \spec{D2000}{D}{J2000.0 $\delta$ (radians)} \\
 \spec{DR2000}{D}{J2000.0 proper motion in $\alpha$
                              (radians per Julian year)} \\
 \spec{DD2000}{D}{J2000.0 proper motion in $\delta$
                              (radians per Julian year)} \\
 \spec{P2000}{D}{J2000.0 parallax (arcsec)} \\
 \spec{V2000}{D}{J2000.0 radial velocity (km~s$^{-1}$, +ve = moving away)}
}
\notes
{
 \begin{enumerate}
  \item The $\alpha$ proper motions are $\dot{\alpha}$ rather than
        $\dot{\alpha}\cos\delta$, and are per year rather than per century.
  \item Conversion from Besselian epoch 1950.0 to Julian epoch
        2000.0 only is provided for.  Conversions involving other
        epochs will require use of the appropriate precession,
        proper motion, and E-terms routines before and/or after FK425
        is called.
  \item In the FK4 catalogue the proper motions of stars within
        $10^{\circ}$ of the poles do not include the {\it differential
        E-terms}\, effect and should, strictly speaking, be handled
        in a different manner from stars outside these regions.
        However, given the general lack of homogeneity of the star
        data available for routine astrometry, the difficulties of
        handling positions that may have been determined from
        astrometric fields spanning the polar and non-polar regions,
        the likelihood that the differential E-terms effect was not
        taken into account when allowing for proper motion in past
        astrometry, and the undesirability of a discontinuity in
        the algorithm, the decision has been made in this routine to
        include the effect of differential E-terms on the proper
        motions for all stars, whether polar or not.  At epoch J2000,
        and measuring on the sky rather than in terms of $\Delta\alpha$,
        the errors resulting from this simplification are less than
        1~milliarcsecond in position and 1~milliarcsecond per
        century in proper motion.
  \item See also sla\_FK45Z, sla\_FK524, sla\_FK54Z.
 \end{enumerate}
}
\refs
{
 \begin{enumerate}
  \item Smith, C.A.\ {\it et al.}, 1989.\  {\it Astr.J.}\ {\bf 97}, 265.
  \item Yallop, B.D.\ {\it et al.}, 1989.\ {\it Astr.J.}\ {\bf 97}, 274.
  \item Seidelmann, P.K.\ (ed), 1992.  {\it Explanatory
        Supplement to the Astronomical Almanac}, ISBN~0-935702-68-7.
 \end{enumerate}
}
%-----------------------------------------------------------------------
\routine{SLA\_FK45Z}{FK4 to FK5, no P.M. or Parallax}
{
 \action{Convert B1950.0 FK4 star data to J2000.0 FK5 assuming zero
         proper motion in an inertial frame.
         This routine converts stars from the old, Bessel-Newcomb, FK4
         system to the new, IAU~1976, FK5, Fricke system, in such a
         way that the FK5 proper motion is zero.  Because such a star
         has, in general, a non-zero proper motion in the FK4 system,
         the routine requires the epoch at which the position in the
         FK4 system was determined.  The method is from appendix~2 of
         reference~1, but using the constants of reference~4.}
 \call{CALL sla\_FK45Z (R1950,D1950,BEPOCH,R2000,D2000)}
}
\args{GIVEN}
{
 \spec{R1950}{D}{B1950.0 FK4 $\alpha$ at epoch BEPOCH (radians)} \\
 \spec{D1950}{D}{B1950.0 FK4 $\delta$ at epoch BEPOCH (radians)} \\
 \spec{BEPOCH}{D}{Besselian epoch ({\it e.g.}\ 1979.3D0)}
}
\args{RETURNED}
{
 \spec{R2000}{D}{J2000.0 FK5 $\alpha$ (radians)} \\
 \spec{D2000}{D}{J2000.0 FK5 $\delta$ (radians)}
}
\notes
{
 \begin{enumerate}
  \item The epoch BEPOCH is strictly speaking Besselian, but
        if a Julian epoch is supplied the result will be
        affected only to a negligible extent.
  \item Conversion from Besselian epoch 1950.0 to Julian epoch
        2000.0 only is provided for.  Conversions involving other
        epochs will require use of the appropriate precession,
        proper motion, and E-terms routines before and/or
        after FK425 is called.
  \item In the FK4 catalogue the proper motions of stars within
        $10^{\circ}$ of the poles do not include the {\it differential
        E-terms}\, effect and should, strictly speaking, be handled
        in a different manner from stars outside these regions.
        However, given the general lack of homogeneity of the star
        data available for routine astrometry, the difficulties of
        handling positions that may have been determined from
        astrometric fields spanning the polar and non-polar regions,
        the likelihood that the differential E-terms effect was not
        taken into account when allowing for proper motion in past
        astrometry, and the undesirability of a discontinuity in
        the algorithm, the decision has been made in this routine to
        include the effect of differential E-terms on the proper
        motions for all stars, whether polar or not.  At epoch 2000,
        and measuring on the sky rather than in terms of $\Delta\alpha$,
        the errors resulting from this simplification are less than
        1~milliarcsecond in position and 1~milliarcsecond per
        century in proper motion.
  \item See also sla\_FK425, sla\_FK524, sla\_FK54Z.
 \end{enumerate}
}
\refs
{
 \begin{enumerate}
  \item Aoki, S., {\it et al.}, 1983.\ {\it Astr.Astrophys.}, {\bf 128}, 263.
  \item Smith, C.A.\ {\it et al.}, 1989.\  {\it Astr.J.}\ {\bf 97}, 265.
  \item Yallop, B.D.\ {\it et al.}, 1989.\ {\it Astr.J.}\ {\bf 97}, 274.
  \item Seidelmann, P.K.\ (ed), 1992.  {\it Explanatory
        Supplement to the Astronomical Almanac}, ISBN~0-935702-68-7.
 \end{enumerate}
}
%-----------------------------------------------------------------------
\routine{SLA\_FK524}{FK5 to FK4}
{
 \action{Convert J2000.0 FK5 star data to B1950.0 FK4.
         This routine converts stars from the new, IAU~1976, FK5, Fricke
         system, to the old, Bessel-Newcomb, FK4 system.
         The precepts of Smith~{\it et~al.}\ (reference~1) are followed,
         using the implementation by Yallop~{\it et~al.}\ (reference~2)
         of a matrix method due to Standish.  Kinoshita's development of
         Andoyer's post-Newcomb precession is used.  The numerical
         constants from Seidelmann~{\it et~al.}\ (reference~3) are
         used canonically.}
 \call{CALL sla\_FK524 (\vtop{
         \hbox{R2000,D2000,DR2000,DD2000,P2000,V2000,}
         \hbox{R1950,D1950,DR1950,DD1950,P1950,V1950)}}}
}
\args{GIVEN}
{
 \spec{R2000}{D}{J2000.0 $\alpha$ (radians)} \\
 \spec{D2000}{D}{J2000.0 $\delta$ (radians)} \\
 \spec{DR2000}{D}{J2000.0 proper motion in $\alpha$
                              (radians per Julian year)} \\
 \spec{DD2000}{D}{J2000.0 proper motion in $\delta$
                              (radians per Julian year)} \\
 \spec{P2000}{D}{J2000.0 parallax (arcsec)} \\
 \spec{V2000}{D}{J2000 radial velocity (km~s$^{-1}$, +ve = moving away)}
}
\args{RETURNED}
{
 \spec{R1950}{D}{B1950.0 $\alpha$ (radians)} \\
 \spec{D1950}{D}{B1950.0 $\delta$ (radians)} \\
 \spec{DR1950}{D}{B1950.0 proper motion in $\alpha$
                              (radians per tropical year)} \\
 \spec{DD1950}{D}{B1950.0 proper motion in $\delta$
                              (radians per tropical year)} \\
 \spec{P1950}{D}{B1950.0 parallax (arcsec)} \\
 \spec{V1950}{D}{radial velocity (km~s$^{-1}$, +ve = moving away)}
}
\notes
{
 \begin{enumerate}
  \item The $\alpha$ proper motions are $\dot{\alpha}$ rather than
        $\dot{\alpha}\cos\delta$, and are per year rather than per century.
  \item Note that conversion from Julian epoch 2000.0 to Besselian
        epoch 1950.0 only is provided for.  Conversions involving
        other epochs will require use of the appropriate precession,
        proper motion, and E-terms routines before and/or after
        FK524 is called.
  \item In the FK4 catalogue the proper motions of stars within
        $10^{\circ}$ of the poles do not include the {\it differential
        E-terms}\, effect and should, strictly speaking, be handled
        in a different manner from stars outside these regions.
        However, given the general lack of homogeneity of the star
        data available for routine astrometry, the difficulties of
        handling positions that may have been determined from
        astrometric fields spanning the polar and non-polar regions,
        the likelihood that the differential E-terms effect was not
        taken into account when allowing for proper motion in past
        astrometry, and the undesirability of a discontinuity in
        the algorithm, the decision has been made in this routine to
        include the effect of differential E-terms on the proper
        motions for all stars, whether polar or not.  At epoch 2000,
        and measuring on the sky rather than in terms of $\Delta\alpha$,
        the errors resulting from this simplification are less than
        1~milliarcsecond in position and 1~milliarcsecond per
        century in proper motion.
  \item See also sla\_FK425, sla\_FK45Z, sla\_FK54Z.
 \end{enumerate}
}
\refs
{
 \begin{enumerate}
  \item Smith, C.A.\ {\it et al.}, 1989.\  {\it Astr.J.}\ {\bf 97}, 265.
  \item Yallop, B.D.\ {\it et al.}, 1989.\ {\it Astr.J.}\ {\bf 97}, 274.
  \item Seidelmann, P.K.\ (ed), 1992.  {\it Explanatory
        Supplement to the Astronomical Almanac}, ISBN~0-935702-68-7.
 \end{enumerate}
}
%-----------------------------------------------------------------------
\routine{SLA\_FK54Z}{FK5 to FK4, no P.M. or Parallax}
{
 \action{Convert a J2000.0 FK5 star position to B1950.0 FK4 assuming
         zero proper motion and parallax.
         This routine converts star positions from the new, IAU~1976,
         FK5, Fricke system to the old, Bessel-Newcomb, FK4 system.}
 \call{CALL sla\_FK54Z (R2000,D2000,BEPOCH,R1950,D1950,DR1950,DD1950)}
}
\args{GIVEN}
{
 \spec{R2000}{D}{J2000.0 FK5 $\alpha$ (radians)} \\
 \spec{D2000}{D}{J2000.0 FK5 $\delta$ (radians)} \\
 \spec{BEPOCH}{D}{Besselian epoch ({\it e.g.}\ 1950D0)}
}
\args{RETURNED}
{
 \spec{R1950}{D}{B1950.0 FK4 $\alpha$ at epoch BEPOCH (radians)} \\
 \spec{D1950}{D}{B1950.0 FK4 $\delta$ at epoch BEPOCH (radians)} \\
 \spec{DR1950}{D}{B1950.0 FK4 proper motion in $\alpha$
                              (radians per tropical year)} \\
 \spec{DD1950}{D}{B1950.0 FK4 proper motion in $\delta$
                              (radians per tropical year)}
}
\notes
{
 \begin{enumerate}
  \item The $\alpha$ proper motions are $\dot{\alpha}$ rather than
        $\dot{\alpha}\cos\delta$, and are per year rather than per century.
  \item Conversion from Julian epoch 2000.0 to Besselian epoch 1950.0
       only is provided for.  Conversions involving other epochs will
       require use of the appropriate precession routines before and
       after this routine is called.
  \item Unlike in the sla\_FK524 routine, the FK5 proper motions, the
       parallax and the radial velocity are presumed zero.
  \item It is the intention that FK5 should be a close approximation
       to an inertial frame, so that distant objects have zero proper
       motion;  such objects have (in general) non-zero proper motion
       in FK4, and this routine returns those {\it fictitious proper
       motions}.
  \item The position returned by this routine is in the B1950
       reference frame but at Besselian epoch BEPOCH.  For
       comparison with catalogues the BEPOCH argument will
       frequently be 1950D0.
  \item See also sla\_FK425, sla\_FK45Z, sla\_FK524.
 \end{enumerate}
}
%-----------------------------------------------------------------------
\routine{SLA\_FLOTIN}{Decode a Real Number}
{
 \action{Convert free-format input into single precision floating point.}
 \call{CALL sla\_FLOTIN (STRING, NSTRT, RESLT, JFLAG)}
}
\args{GIVEN}
{
 \spec{STRING}{C}{string containing number to be decoded} \\
 \spec{NSTRT}{I}{pointer to where decoding is to commence} \\
 \spec{RESLT}{R}{current value of result}
}
\args{RETURNED}
{
 \spec{NSTRT}{I}{advanced to next number} \\
 \spec{RESLT}{R}{result} \\
 \spec{JFLAG}{I}{status: $-$1 = $-$OK, 0 = +OK, 1 = null result, 2 = error}
}
\notes
{
 \begin{enumerate}
 \item The reason sla\_FLOTIN has separate `OK' status values
       for + and $-$ is to enable minus zero to be detected.
       This is of crucial importance
       when decoding mixed-radix numbers.  For example, an angle
       expressed as degrees, arcminutes and arcseconds may have a
       leading minus sign but a zero degrees field.
 \item A TAB is interpreted as a space, and lowercase characters are
       interpreted as uppercase.  {\it n.b.}\ The test for TAB is
       {\bf VAX-specific} but may also work on other computers
       which use the ASCII character set.
 \item The basic format is the sequence of fields $\pm n.n x \pm n$,
       where $\pm$ is a sign
       character `+' or `$-$', $n$ means a string of decimal digits,
       `.' is a decimal point, and $x$, which indicates an exponent,
       means `D' or `E'.  Various combinations of these fields can be
       omitted, and embedded blanks are permissible in certain places.
 \item Spaces:
       \begin{itemize}
       \item Leading spaces are ignored.
       \item Embedded spaces are allowed only after +, $-$, D or E,
             and after the decimal point if the first sequence of
             digits is absent.
       \item Trailing spaces are ignored;  the first signifies
             end of decoding and subsequent ones are skipped.
       \end{itemize}
 \item Delimiters:
       \begin{itemize}
       \item Any character other than +,$-$,0-9,.,D,E or space may be
             used to signal the end of the number and terminate decoding.
       \item Comma is recognized by sla\_FLOTIN as a special case; it
             is skipped, leaving the pointer on the next character.  See
             13, below.
       \item Decoding will in all cases terminate if end of string
             is reached.
       \end{itemize}
 \item Both signs are optional.  The default is +.
 \item The mantissa $n.n$ defaults to unity.
 \item The exponent $x\!\pm\!n$ defaults to `E0'.
 \item The strings of decimal digits may be of any length.
 \item The decimal point is optional for whole numbers.
 \item A {\it null result}\, occurs when the string of characters
       being decoded does not begin with +,$-$,0-9,.,D or E, or
       consists entirely of spaces.  When this condition is
       detected, JFLAG is set to 1 and RESLT is left untouched.
 \item NSTRT = 1 for the first character in the string.
 \item On return from sla\_FLOTIN, NSTRT is set ready for the next
       decode -- following trailing blanks and any comma.  If a
       delimiter other than comma is being used, NSTRT must be
       incremented before the next call to sla\_FLOTIN, otherwise
       all subsequent calls will return a null result.
 \item Errors (JFLAG=2) occur when:
       \begin{itemize}
       \item a +, $-$, D or E is left unsatisfied; or
       \item the decimal point is present without at least
             one decimal digit before or after it; or
       \item an exponent more than 100 has been presented.
       \end{itemize}
 \item When an error has been detected, NSTRT is left
       pointing to the character following the last
       one used before the error came to light.  This
       may be after the point at which a more sophisticated
       program could have detected the error.  For example,
       sla\_FLOTIN does not detect that `1E999' is unacceptable
       (on the VAX) until the entire number has been decoded.
 \item Certain highly unlikely combinations of mantissa and
       exponent can cause arithmetic faults during the
       decode, in some cases despite the fact that they
       together could be construed as a valid number.
 \item Decoding is left to right, one pass.
 \item See also sla\_DFLTIN and sla\_INTIN.
 \end{enumerate}
}
%-----------------------------------------------------------------------
\routine{SLA\_GALEQ}{Galactic to J2000 $\alpha,\delta$}
{
 \action{Transformation from IAU 1958 galactic coordinates
         to J2000.0 FK5 equatorial coordinates.}
 \call{CALL sla\_GALEQ (DL, DB, DR, DD)}
}
\args{GIVEN}
{
 \spec{DL,DB}{D}{galactic longitude and latitude \gal}
}
\args{RETURNED}
{
 \spec{DR,DD}{D}{J2000.0 \radec}
}
\notes
{
 \begin{enumerate}
  \item All arguments are in radians.
  \item The equatorial coordinates are J2000.0 FK5.  Use the routine
        sla\_GE50 if conversion to B1950.0 FK4 coordinates is
                           required.
 \end{enumerate}
}
%-----------------------------------------------------------------------
\routine{SLA\_GALSUP}{Galactic to Supergalactic}
{
 \action{Transformation from IAU 1958 galactic coordinates to
         de Vaucouleurs supergalactic coordinates.}
 \call{CALL sla\_GALSUP (DL, DB, DSL, DSB)}
}
\args{GIVEN}
{
 \spec{DL,DB}{D}{galactic longitude and latitude \gal\ (radians)}
}
\args{RETURNED}
{
 \spec{DSL,DSB}{D}{supergalactic longitude and latitude (radians)}
}
\refs
{
 \begin{enumerate}
  \item de Vaucouleurs, de Vaucouleurs, \& Corwin, {\it Second Reference
    Catalogue of Bright Galaxies}, U.Texas, p8.
  \item Systems \& Applied Sciences Corp., documentation for the
        machine-readable version of the above catalogue,
        Contract NAS 5-26490.
 \end{enumerate}
 (These two references give different values for the galactic
 longitude of the supergalactic origin.  Both are wrong;  the
 correct value is $l^{I\!I}=137.37$.)
}
%-----------------------------------------------------------------------
\routine{SLA\_GE50}{B1950 $\alpha,\delta$ to Galactic}
{
 \action{Transformation from IAU 1958 galactic coordinates to
         B1950.0 FK4 equatorial coordinates.}
 \call{CALL sla\_GE50 (DL, DB, DR, DD)}
}
\args{GIVEN}
{
 \spec{DL,DB}{D}{galactic longitude and latitude \gal}
}
\args{RETURNED}
{
 \spec{DR,DD}{D}{B1950.0 \radec}
}
\notes
{
 \begin{enumerate}
  \item All arguments are in radians.
  \item The equatorial coordinates are B1950.0 FK4.  Use the
        routine sla\_GALEQ if conversion to J2000.0 FK5 coordinates
        is required.
 \end{enumerate}
}
\aref{Blaauw {\it et al.}, 1960, {\it Mon.Not.R.astr.Soc.},
      {\bf 121}, 123.}
%-----------------------------------------------------------------------
\routine{SLA\_GEOC}{Geodetic to Geocentric}
{
 \action{Convert geodetic position to geocentric.}
 \call{CALL sla\_GEOC (P, H, R, Z)}
}
\args{GIVEN}
{
 \spec{P}{D}{latitude (geodetic, radians)} \\
 \spec{H}{D}{height above reference spheroid (geodetic, metres)}
}
\args{RETURNED}
{
 \spec{R}{D}{distance from Earth axis (AU)} \\
 \spec{Z}{D}{distance from plane of Earth equator (AU)}
}
\notes
{
 \begin{enumerate}
  \item Geocentric latitude can be obtained by evaluating {\tt ATAN2(Z,R)}.
  \item IAU 1976 constants are used.
 \end{enumerate}
}
\aref{Green, R.M., 1985.\ {\it Spherical Astronomy}, Cambridge U.P., p98.}
%-----------------------------------------------------------------------
\routine{SLA\_GMST}{UT to GMST}
{
 \action{Conversion from universal time UT1 to Greenwich Mean
         sidereal time.}
 \call{D~=~sla\_GMST (UT1)}
}
\args{GIVEN}
{
 \spec{UT1}{D}{universal time (strictly UT1) expressed as
                 modified Julian Date (JD$-$2400000.5)}
}
\args{RETURNED}
{
 \spec{sla\_GMST}{D}{Greenwich mean sidereal time (radians)}
}
\anote{The IAU~1982 expression (see page S15 of the 1984 {\it Astronomical
       Almanac})\, is used, but rearranged to reduce rounding errors.  This
       expression is always described as giving the GMST at $0^{\rm h}$UT;
       in fact, it gives the difference between the
       GMST and the UT, which happens to equal the GMST (modulo
       24~hours) at $0^{\rm h}$UT each day.  In sla\_GMST, the
       fractional UT is used directly as the argument for the
       canonical formula, and the fractional part of the UT is
       added separately;  note that the factor $1.0027379\cdots$ does
       not appear.}
%-----------------------------------------------------------------------
\routine{SLA\_GRESID}{Gaussian Residual}
{
 \action{Generate pseudo-random normal deviate or {\it Gaussian residual}.}
 \call{R~=~sla\_GRESID (S)}
}
\args{GIVEN}
{
 \spec{S}{R}{standard deviation}
}
\notes
{
 \begin{enumerate}
  \item The results of many calls to this routine will be
        normally distributed with mean zero and standard deviation S.
  \item The Box-Muller algorithm is used.
  \item The implementation is {\bf VAX-dependent}.
 \end{enumerate}
}
\aref{Ahrens \& Dieter, 1972.\ {\it Comm.A.C.M.}\ {\bf 15}, 873.}
%-----------------------------------------------------------------------
\routine{SLA\_IMXV}{Apply 3D Reverse Rotation}
{
 \action{Multiply a 3-vector by the inverse of a rotation
         matrix (single precision).}
 \call{CALL sla\_IMXV (RM, VA, VB)}
}
\args{GIVEN}
{
 \spec{RM}{R(3,3)}{rotation matrix} \\
 \spec{VA}{R(3)}{vector to be rotated}
}
\args{RETURNED}
{
 \spec{VB}{R(3)}{result vector}
}
\notes
{
 \begin{enumerate}
  \item This routine performs the operation:
        \begin{verse}
         {\bf b} = {\bf M}$^{T}\cdot${\bf a}
        \end{verse}
        where {\bf a} and {\bf b} are the 3-vectors VA and VB
        respectively, and {\bf M} is the $3\times3$ matrix RM.
  \item The main function of this routine is apply an inverse
        rotation;  under these circumstances, ${\bf M}$ is
        {\it orthogonal}, with its inverse the same as its transpose.
  \item To comply with the ANSI Fortran 77 standard, VA and VB must
        {\bf not} be the same array.  The routine is, in fact, coded
        so as to work properly on the VAX and many other systems even
        if this rule is violated, something that is {\bf not}, however,
        recommended.
 \end{enumerate}
}
%-----------------------------------------------------------------------
\routine{SLA\_INTIN}{Decode an Integer Number}
{
 \action{Convert free-format input into an integer.}
 \call{CALL sla\_INTIN (STRING, NSTRT, IRESLT, JFLAG)}
}
\args{GIVEN}
{
 \spec{STRING}{C}{string containing number to be decoded} \\
 \spec{NSTRT}{I}{pointer to where decoding is to commence} \\
 \spec{IRESLT}{I}{current value of result}
}
\args{RETURNED}
{
 \spec{NSTRT}{I}{advanced to next number} \\
 \spec{IRESLT}{I}{result} \\
 \spec{JFLAG}{I}{status: $-$1 = $-$OK, 0 = +OK, 1 = null result, 2 = error}
}
\notes
{
 \begin{enumerate}
 \item The reason sla\_INTIN has separate `OK' status values
       for + and $-$ is to enable minus zero to be detected.
       This is of crucial importance
       when decoding mixed-radix numbers.  For example, an angle
       expressed as degrees, arcminutes and arcseconds may have a
       leading minus sign but a zero degrees field.
 \item A TAB is interpreted as a space. {\it n.b.}\ The test for TAB is
       {\bf VAX-specific} but may also work on other computers
       which use the ASCII character set.
 \item The basic format is the sequence of fields $\pm n$,
       where $\pm$ is a sign
       character `+' or `$-$', and $n$ means a string of decimal digits.
 \item Spaces:
       \begin{itemize}
       \item Leading spaces are ignored.
       \item Spaces between the sign and the number are allowed.
       \item Trailing spaces are ignored;  the first signifies
             end of decoding and subsequent ones are skipped.
       \end{itemize}
 \item Delimiters:
       \begin{itemize}
       \item Any character other than +,$-$,0-9 or space may be
             used to signal the end of the number and terminate decoding.
       \item Comma is recognized by sla\_INTIN as a special case; it
             is skipped, leaving the pointer on the next character.  See
             9, below.
       \item Decoding will in all cases terminate if end of string
             is reached.
       \end{itemize}
 \item The sign is optional.  The default is +.
 \item A {\it null result}\, occurs when the string of characters
       being decoded does not begin with +,$-$ or 0-9, or
       consists entirely of spaces.  When this condition is
       detected, JFLAG is set to 1 and IRESLT is left untouched.
 \item NSTRT = 1 for the first character in the string.
 \item On return from sla\_INTIN, NSTRT is set ready for the next
       decode -- following trailing blanks and any comma.  If a
       delimiter other than comma is being used, NSTRT must be
       incremented before the next call to sla\_INTIN, otherwise
       all subsequent calls will return a null result.
 \item Errors (JFLAG=2) occur when:
       \begin{itemize}
       \item there is a + or $-$ but no number; or
       \item the number is greater than $2^{31}-1$.
       \end{itemize}
 \item When an error has been detected, NSTRT is left
       pointing to the character following the last
       one used before the error came to light.
 \item See also sla\_FLOTIN and sla\_DFLTIN.
 \end{enumerate}
}
%-----------------------------------------------------------------------
\routine{SLA\_INVF}{Invert Linear Model}
{
 \action{Invert a linear model of the type produced by the
         sla\_FITXY routine.}
 \call{CALL sla\_INVF (FWDS,BKWDS,J)}
}
\args{GIVEN}
{
 \spec{FWDS}{D(6)}{model coefficients}
}
\args{RETURNED}
{
 \spec{BKWDS}{D(6)}{inverse model} \\
 \spec{J}{I}{status:  0 = OK, $-$1 = no inverse}
}
\notes
{
 \begin{enumerate}
  \item The models relate two sets of \xy\ coordinates as follows.
        Naming the six elements of FWDS $a,b,c,d,e$ \& $f$,
        where two sets of coordinates $[x_{1},y_{1}]$ and
        $[x_{2},y_{2}\,]$ are related thus:
        \begin{verse}
         $x_{2} = a + bx_{1} + cy_{1}$ \\
         $y_{2} = d + ex_{1} + fy_{1}$
        \end{verse}
        the present routine generates a new set of coefficients
        $p,q,r,s,t$ \& $u$ (the array BKWDS) such that:
        \begin{verse}
         $x_{1} = p + qx_{2} + ry_{2}$ \\
         $y_{1} = s + tx_{2} + uy_{2}$
        \end{verse}
  \item Two successive calls to this routine will deliver a set
        of coefficients equal to the starting values.
  \item To comply with the ANSI Fortran 77 standard, FWDS and BKWDS must
        {\bf not} be the same array.  The routine is, in fact, coded
        so as to work properly on the VAX and many other systems even
        if this rule is violated, something that is {\bf not}, however,
        recommended.
  \item See also sla\_FITXY, sla\_PXY, sla\_XY2XY, sla\_DCMPF.
 \end{enumerate}
}
%-----------------------------------------------------------------------
\routine{SLA\_KBJ}{Select Epoch Prefix}
{
 \action{Select epoch prefix `B' or `J'.}
 \call{CALL sla\_KBJ (JB, E, K, J)}
}
\args{GIVEN}
{
 \spec{JB}{I}{sla\_DBJIN prefix status:  0=none, 1=`B', 2=`J'} \\
 \spec{E}{D}{epoch -- Besselian or Julian}
}
\args{RETURNED}
{
 \spec{K}{C}{`B' or `J'} \\
 \spec{J}{I}{status:  0=OK}
}
\anote{The routine is mainly intended for use in conjunction with the
       sla\_DBJIN routine.  If the value of JB indicates that an explicit
       B or J prefix was detected by sla\_DBJIN, a `B' or `J'
       is returned to match.  If JB indicates that no explicit B or J
       was supplied, the choice is made on the basis of the epoch
       itself;  B is assumed for E $<1984$, otherwise J.}
%-----------------------------------------------------------------------
\routine{SLA\_M2AV}{Rotation Matrix to Axial Vector}
{
 \action{From a rotation matrix, determine the corresponding axial vector
        (single precision).}
 \call{CALL sla\_M2AV (RMAT, AXVEC)}
}
\args{GIVEN}
{
 \spec{RMAT}{R(3,3)}{rotation matrix}
}
\args{RETURNED}
{
 \spec{AXVEC}{R(3)}{axial vector (radians)}
}
\notes
{
 \begin{enumerate}
  \item A rotation matrix describes a rotation about some arbitrary axis.
        The axis is called the {\it Euler axis}, and the angle through
        which the reference frame rotates is called the {\it Euler angle}.
        The {\it axial vector}\, returned by this routine has the same
        direction as the Euler axis, and its magnitude is the Euler angle
        in radians.
  \item The magnitude and direction of the axial vector can be separated
        by means of the routine sla\_VN.
  \item The reference frame rotates clockwise as seen looking along
        the axial vector from the origin.
  \item If RMAT is null, so is the result.
 \end{enumerate}
}
%-----------------------------------------------------------------------
\routine{SLA\_MAP}{Mean to Apparent}
{
 \action{Transform star \radec\ from mean place to geocentric apparent.
         The reference frames and timescales used are post IAU~1976.}
 \call{CALL sla\_MAP (RM, DM, PR, PD, PX, RV, EQ, DATE, RA, DA)}
}
\args{GIVEN}
{
 \spec{RM,DM}{D}{mean \radec\ (radians)} \\
 \spec{PR,PD}{D}{proper motions:  \radec\ changes per Julian year} \\
 \spec{PX}{D}{parallax (arcsec)} \\
 \spec{RV}{D}{radial velocity (km~s$^{-1}$, +ve if receding)} \\
 \spec{EQ}{D}{epoch and equinox of star data (Julian)} \\
 \spec{DATE}{D}{TDB for apparent place (JD$-$2400000.5)}
}
\args{RETURNED}
{
 \spec{RA,DA}{D}{apparent \radec\ (radians)}
}
\notes
{
 \begin{enumerate}
  \item EQ is the Julian epoch specifying both the reference
        frame and the epoch of the position -- usually 2000.
        For positions where the epoch and equinox are
        different, use the routine sla\_PM to apply proper
        motion corrections before using this routine.
  \item The distinction between the required TDB and TT is
        always negligible.  Moreover, for all but the most
        critical applications UTC is adequate.
  \item The $\alpha$ proper motions are $\dot{\alpha}$ rather than
        $\dot{\alpha}\cos\delta$, and are per year rather than per century.
  \item This routine may be wasteful for some applications
        because it recomputes the Earth position/velocity and
        the precession/nutation matrix each time, and because
        it allows for parallax and proper motion.  Where
        multiple transformations are to be carried out for one
        epoch, a faster method is to call the sla\_MAPPA routine
        once and then either the sla\_MAPQK routine (which includes
        parallax and proper motion) or sla\_MAPQKZ (which assumes
        zero parallax and proper motion).
 \end{enumerate}
}
\refs
{
 \begin{enumerate}
  \item 1984 {\it Astronomical Almanac}, pp B39-B41.
  \item Lederle \& Schwan, 1984.\ {\it Astr.Astrophys.}\ {\bf 134}, 1-6.
 \end{enumerate}
}
%-----------------------------------------------------------------------
\routine{SLA\_MAPPA}{Mean to Apparent Parameters}
{
 \action{Compute star-independent parameters in preparation for
         conversions between mean place and geocentric apparent place.
         The parameters produced by this routine are required in the
         parallax, light deflection, aberration, and precession/nutation
         parts of the mean/apparent transformations.
         The reference frames and timescales used are post IAU~1976.}
 \call{CALL sla\_MAPPA (EQ, DATE, AMPRMS)}
}
\args{GIVEN}
{
 \spec{EQ}{D}{epoch of mean equinox to be used (Julian)} \\
 \spec{DATE}{D}{TDB (JD$-$2400000.5)}
}
\args{RETURNED}
{
 \spec{AMPRMS}{D(21)}{star-independent mean-to-apparent parameters:} \\
 \specel   {(1)}     {time interval for proper motion (Julian years)} \\
 \specel   {(2-4)}   {barycentric position of the Earth (AU)} \\
 \specel   {(5-7)}   {heliocentric direction of the Earth (unit vector)} \\
 \specel   {(8)}     {(gravitational radius of
                      Sun)$\times 2 / $(Sun-Earth distance)} \\
 \specel   {(9-11)}  {\mbox{\boldmath $v$}: barycentric Earth
                                               velocity in units of c} \\
 \specel   {(12)}    {$\sqrt{1-\mid\mbox{\boldmath $v$}\mid^2}$} \\
 \specel   {(13-21)} {precession/nutation $3\times3$ matrix}
}
\notes
{
 \begin{enumerate}
  \item For DATE, the distinction between the required TDB and TT
        is always negligible.  Moreover, for all but the most
        critical applications UTC is adequate.
  \item The accuracy of the routines using the parameters AMPRMS is
        limited by the routine sla\_EVP, used here to compute the
        Earth position and velocity by the methods of Stumpff.
        The maximum error in the resulting aberration corrections is
        about 0.3 milliarcsecond.
  \item The vectors AMPRMS(2-4) and AMPRMS(5-7) are referred to
        the mean equinox and equator of epoch EQ.
  \item The parameters produced by this routine are used by
        sla\_MAPQK and sla\_MAPQKZ.
 \end{enumerate}
}
\refs
{
 \begin{enumerate}
  \item 1984 {\it Astronomical Almanac}, pp B39-B41.
  \item Lederle \& Schwan, 1984.\ {\it Astr.Astrophys.}\ {\bf 134}, 1-6.
 \end{enumerate}
}
%-----------------------------------------------------------------------
\routine{SLA\_MAPQK}{Quick Mean to Apparent}
{
 \action{Quick mean to apparent place:  transform a star \radec\ from
         mean place to geocentric apparent place, given the
         star-independent parameters.  The reference frames and
         timescales used are post IAU 1976.}
 \call{CALL sla\_MAPQK (RM, DM, PR, PD, PX, RV, AMPRMS, RA, DA)}
}
\args{GIVEN}
{
 \spec{RM,DM}{D}{mean \radec\ (radians)} \\
 \spec{PR,PD}{D}{proper motions:  \radec\ changes per Julian year} \\
 \spec{PX}{D}{parallax (arcsec)} \\
 \spec{RV}{D}{radial velocity (km~s$^{-1}$, +ve if receding)} \\
 \spec{AMPRMS}{D(21)}{star-independent mean-to-apparent parameters:} \\
 \specel   {(1)}     {time interval for proper motion (Julian years)} \\
 \specel   {(2-4)}   {barycentric position of the Earth (AU)} \\
 \specel   {(5-7)}   {heliocentric direction of the Earth (unit vector)} \\
 \specel   {(8)}     {(gravitational radius of
                      Sun)$\times 2 / $(Sun-Earth distance)} \\
 \specel   {(9-11)}  {\mbox{\boldmath $v$}: barycentric Earth
                                               velocity in units of c} \\
 \specel   {(12)}    {$\sqrt{1-\mid\mbox{\boldmath $v$}\mid^2}$} \\
 \specel   {(13-21)} {precession/nutation $3\times3$ matrix}
}
\args{RETURNED}
{
 \spec{RA,DA}{D }{apparent \radec\ (radians)}
}
\notes
{
 \begin{enumerate}
  \item Use of this routine is appropriate when efficiency is important
        and where many star positions, all referred to the same equator
        and equinox, are to be transformed for one epoch.  The
        star-independent parameters can be obtained by calling the
        sla\_MAPPA routine.
  \item If the parallax and proper motions are zero the sla\_MAPQKZ
        routine can be used instead.
  \item The vectors AMPRMS(2-4) and AMPRMS(5-7) are referred to
        the mean equinox and equator of epoch EQ.
  \item Within about 300~arcsec of the centre of the Sun the
        gravitational deflection term is set to zero to avoid
        overflow.  Otherwise, no account is taken of the
        impossibility of observing stars which lie behind
        the Sun.
 \end{enumerate}
}
\refs
{
 \begin{enumerate}
  \item 1984 {\it Astronomical Almanac}, pp B39-B41.
  \item Lederle \& Schwan, 1984.\ {\it Astr.Astrophys.}\ {\bf 134}, 1-6.
 \end{enumerate}
}
%-----------------------------------------------------------------------
\routine{SLA\_MAPQKZ}{Quick Mean-Appt, no PM {\it etc.}}
{
 \action{Quick mean to apparent place:  transform a star \radec\ from
         mean place to geocentric apparent place, given the
         star-independent parameters, and assuming zero parallax
         and proper motion.
         The reference frames and timescales used are post IAU~1976.}
 \call{CALL sla\_MAPQKZ (RM, DM, AMPRMS, RA, DA)}
}
\args{GIVEN}
{
 \spec{RM,DM}{D}{mean \radec\ (radians)} \\
 \spec{AMPRMS}{D(21)}{star-independent mean-to-apparent parameters:} \\
 \specel   {(1)}     {time interval for proper motion (Julian years)} \\
 \specel   {(2-4)}   {barycentric position of the Earth (AU)} \\
 \specel   {(5-7)}   {heliocentric direction of the Earth (unit vector)} \\
 \specel   {(8)}     {(gravitational radius of
                      Sun)$\times 2 / $(Sun-Earth distance)} \\
 \specel   {(9-11)}  {\mbox{\boldmath $v$}: barycentric Earth
                                               velocity in units of c} \\
 \specel   {(12)}    {$\sqrt{1-\mid\mbox{\boldmath $v$}\mid^2}$} \\
 \specel   {(13-21)} {precession/nutation $3\times3$ matrix}
}
\args{RETURNED}
{
 \spec{RA,DA}{D}{apparent \radec\ (radians)}
}
\notes
{
 \begin{enumerate}
  \item Use of this routine is appropriate when efficiency is important
        and where many star positions, all with parallax and proper
        motion either zero or already allowed for, and all referred to
        the same equator and equinox, are to be transformed for one
        epoch.  The star-independent parameters can be obtained by
        calling the sla\_MAPPA routine.
  \item The corresponding routine for the case of non-zero parallax
        and proper motion is sla\_MAPQK.
  \item The vectors AMPRMS(2-4) and AMPRMS(5-7) are referred to the
        mean equinox and equator of epoch EQ.
  \item Within about 300~arcsec of the centre of the Sun the
        gravitational deflection term is set to zero to avoid
        overflow.  Otherwise, no account is taken of the
        impossibility of observing stars which lie behind
        the Sun.
 \end{enumerate}
}
\refs
{
 \begin{enumerate}
  \item 1984 {\it Astronomical Almanac}, pp B39-B41.
  \item Lederle \& Schwan, 1984.\ {\it Astr.Astrophys.}\ {\bf 134}, 1-6.
 \end{enumerate}
}
%-----------------------------------------------------------------------
\routine{SLA\_MOON}{Approx Moon Pos/Vel}
{
 \action{Approximate geocentric position and velocity of the Moon.}
 \call{CALL sla\_MOON (IY, ID, FD, POSVEL)}
}
\args{GIVEN}
{
 \spec{IY}{I}{year AD} \\
 \spec{ID}{I}{day in year (1 = Jan 1st)} \\
 \spec{FD}{R }{fraction of day}
}
\args{RETURNED}
{
 \spec{POSVEL}{R(6)}{Moon \xyzxyzd\ (au, au~s$^{-1}$)}
}
\notes
{
 \begin{enumerate}
  \item The date and time is TDB (formerly ET) in the Gregorian
        calendar, and is interpreted in a manner which is valid
        between 1900 March 1 and 2100 February 28.
  \item The Moon 6-vector is Moon centre relative to Earth centre,
        mean equator and equinox of date.
  \item The position is accurate to better than 0.5~arcminute
        in direction and $10^{-5}$ au in distance.  The velocity
        is accurate to better than 0.5~arcsec per hour in direction
        and 4 m~s$^{-1}$ in distance.  Note that the distance accuracy is
        comparatively poor because this routine is merely
        intended for use in computing topocentric direction.
 \end{enumerate}
}
\aref{Meeus, {\it l'Astronomie}, June 1984, p348. (The present
      routine is only a partial implementation of the original
      Meeus algorithm, which offers somewhat higher accuracy in direction
      and much higher accuracy in distance, or rather in horizontal
      parallax, when fully implemented.)}
%-----------------------------------------------------------------------
\routine{SLA\_MXM}{Multiply $3\times3$ Matrices}
{
 \action{Product of two $3\times3$ matrices (single precision).}
 \call{CALL sla\_MXM (A, B, C)}
}
\args{GIVEN}
{
 \spec{A}{R(3,3)}{matrix {\bf A}} \\
 \spec{B}{R(3,3)}{matrix {\bf B}}
}
\args{RETURNED}
{
 \spec{C}{R(3,3)}{matrix result: {\bf A}$\cdot${\bf B}}
}
\anote{To comply with the ANSI Fortran 77 standard, A, B and C must
       be different arrays.  The routine is, in fact, coded
       so as to work properly on the VAX and many other systems even
       if this rule is violated, something that is {\bf not}, however,
       recommended.}
%-----------------------------------------------------------------------
\routine{SLA\_MXV}{Apply 3D Rotation}
{
 \action{Multiply a 3-vector by a rotation matrix (single precision).}
 \call{CALL sla\_MXV (RM, VA, VB)}
}
\args{GIVEN}
{
 \spec{RM}{R(3,3)}{rotation matrix} \\
 \spec{VA}{R(3)}{vector to be rotated}
}
\args{RETURNED}
{
 \spec{VB}{R(3)}{result vector}
}
\notes
{
 \begin{enumerate}
  \item This routine performs the operation:
        \begin{verse}
         {\bf b} = {\bf M}$\cdot${\bf a}
        \end{verse}
        where {\bf a} and {\bf b} are the 3-vectors VA and VB
        respectively, and {\bf M} is the $3\times3$ matrix RM.
  \item The main function of this routine is apply a
        rotation;  under these circumstances, ${\bf M}$ is a
        {\it proper real orthogonal}\, matrix.
  \item To comply with the ANSI Fortran 77 standard, VA and VB must
        {\bf not} be the same array.  The routine is, in fact, coded
        so as to work properly on the VAX and many other systems even
        if this rule is violated, something that is {\bf not}, however,
        recommended.
 \end{enumerate}
}
%------------------------------------------------------------------------------
\routine{SLA\_NUT}{Nutation Matrix}
{
 \action{Form the matrix of nutation (IAU 1980 theory) for a given date.}
 \call{CALL sla\_NUT (DATE, RMATN)}
}
\args{GIVEN}
{
 \spec{DATE}{D}{TDB (formerly ET) as Modified Julian Date
                                          (JD$-$2400000.5)}
}
\args{RETURNED}
{
 \spec{RMATN}{D(3,3)}{nutation matrix}
}
\anote{The matrix is in the sense:
       \begin{verse}
        {\bf v}$_{true}$ =  {\bf M}$\cdot${\bf v}$_{mean}$
       \end{verse}
       where {\bf v}$_{true}$ is the star vector relative to the
       true equator and equinox of date, {\bf M} is the
       $3\times3$ matrix RMATN and
       {\bf v}$_{mean}$ is the star vector relative to the
       mean equator and equinox of date.}
\refs
{
 \begin{enumerate}
  \item Final report of the IAU Working Group on Nutation,
        chairman P.K.Seidelmann, 1980.
  \item Kaplan, G.H., 1981.\ {\it USNO circular no.\ 163}, pA3-6.
 \end{enumerate}
}
%-----------------------------------------------------------------------
\routine{SLA\_NUTC}{Nutation Components}
{
 \action{Nutation (IAU 1980 theory):  longitude \& obliquity
         components, and mean obliquity.}
 \call{CALL sla\_NUTC (DATE, DPSI, DEPS, EPS0)}
}
\args{GIVEN}
{
 \spec{DATE}{D}{TDB (formerly ET) as Modified Julian Date
                                           (JD$-$2400000.5)}
}
\args{RETURNED}
{
 \spec{DPSI,DEPS}{D}{nutation in longitude and obliquity (radians)} \\
 \spec{EPS0}{D}{mean obliquity (radians)}
}
\refs
{
 \begin{enumerate}
  \item Final report of the IAU Working Group on Nutation,
        chairman P.K.Seidelmann, 1980.
  \item Kaplan, G.H., 1981.\ {\it USNO circular no.\ 163}, pA3-6.
 \end{enumerate}
}
%------------------------------------------------------------------------------
\routine{SLA\_OAP}{Observed to Apparent}
{
 \action{Observed to apparent place.}
 \call{CALL sla\_OAP (\vtop{
         \hbox{TYPE, OB1, OB2, DATE, DUT, ELONGM, PHIM,}
         \hbox{HM, XP, YP, TDK, PMB, RH, WL, TLR, RAP, DAP)}}}
}
\args{GIVEN}
{
 \spec{TYPE}{C*(*)}{type of coordinates -- `R', `H' or `A' (see below)} \\
 \spec{OB1}{D}{observed Az, HA or RA (radians; Az is N=0, E=$90^{\circ}$)} \\
 \spec{OB2}{D}{observed zenith distance or $\delta$ (radians)} \\
 \spec{DATE}{D }{UTC date/time (Modified Julian Date, JD$-$2400000.5)} \\
 \spec{DUT}{D}{$\Delta$UT:  UT1$-$UTC (UTC seconds)} \\
 \spec{ELONGM}{D}{mean longitude of the observer (radians, east +ve)} \\
 \spec{PHIM}{D}{mean geodetic latitude of the observer (radians)} \\
 \spec{HM}{D}{observer's height above sea level (metres)} \\
 \spec{XP,YP}{D}{polar motion \xy\ coordinates (radians)} \\
 \spec{TDK}{D}{local ambient temperature (degrees K; std=273.155D0)} \\
 \spec{PMB}{D}{local atmospheric pressure (mB; std=1013.25D0)} \\
 \spec{RH}{D}{local relative humidity (in the range 0D0-1D0)} \\
 \spec{WL}{D}{effective wavelength (micron, {\it e.g.}\ 0.55D0)} \\
 \spec{TLR}{D}{tropospheric lapse rate (degrees K per metre,
                                                {\it e.g.}\ 0.0065D0)}
}
\args{RETURNED}
{
 \spec{RAP,DAP}{D}{geocentric apparent \radec}
}
\notes
{
 \begin{enumerate}
  \item Only the first character of the TYPE argument is significant.
        `R' or `r' indicates that OBS1 and OBS2 are the observed Right
        Ascension and Declination;  `H' or `h' indicates that they are
        Hour Angle (west +ve) and Declination; anything else (`A' or
        `a' is recommended) indicates that OBS1 and OBS2 are Azimuth
        (north zero, east is $90^{\circ}$) and Zenith Distance.  (Zenith
        distance is used rather than elevation in order to reflect the
        fact that no allowance is made for depression of the horizon.)
  \item The accuracy of the result is limited by the corrections for
        refraction.  Providing the meteorological parameters are
        known accurately and there are no gross local effects, the
        predicted azimuth and elevation should be within about
        0.1~arcsec for $\zeta<70^{\circ}$.  Even
        at a topocentric zenith distance of
        $90^{\circ}$, the accuracy in elevation should be better than
        1~arcmin;  useful results are available for a further
        $3^{\circ}$, beyond which the sla\_REFRO routine returns a
        fixed value of the refraction.  The complementary
        routines sla\_AOP (or sla\_AOPQK) and sla\_OAP (or sla\_OAPQK)
        are self-consistent to better than 1~microarcsecond all over
        the celestial sphere.
  \item It is advisable to take great care with units, as even
        unlikely values of the input parameters are accepted and
        processed in accordance with the models used.
  \item {\it Observed}\, \azel\ means the position that would be seen by a
        perfect theodolite located at the observer.  This is
        related to the observed \hadec\ via the standard rotation, using
        the geodetic latitude (corrected for polar motion), while the
        observed HA and RA are related simply through the local
        apparent ST.  {\it Observed}\, \radec\ or \hadec\ thus means the
        position that would be seen by a perfect equatorial located
        at the observer and with its polar axis aligned to the
        Earth's axis of rotation ({\it n.b.}\ not to the refracted pole).
        By removing from the observed place the effects of
        atmospheric refraction and diurnal aberration, the
        geocentric apparent \radec\ is obtained.
  \item Frequently, {\it mean}\, rather than {\it apparent}\,
        \radec\ will be required,
        in which case further transformations will be necessary.  The
        sla\_AMP {\it etc.}\ routines will convert
        the apparent \radec\ produced
        by the present routine into an FK5 J2000 mean place, by
        allowing for the Sun's gravitational lens effect, annual
        aberration, nutation and precession.  Should FK4 B1950
        coordinates be needed, the routines sla\_FK524 {\it etc.}\ will also
        need to be applied.
  \item To convert to apparent \radec\ the coordinates read from a
        real telescope, corrections would have to be applied for
        encoder zero points, gear and encoder errors, tube flexure,
        the position of the rotator axis and the pointing axis
        relative to it, non-perpendicularity between the mounting
        axes, and finally for the tilt of the azimuth or polar axis
        of the mounting (with appropriate corrections for mount
        flexures).  Some telescopes would, of course, exhibit other
        properties which would need to be accounted for at the
        appropriate point in the sequence.
  \item The star-independent apparent-to-observed-place parameters
        in AOPRMS may be computed by means of the sla\_AOPPA routine.
        If nothing has changed significantly except the time, the
        sla\_AOPPAT routine may be used to perform the requisite
        partial recomputation of AOPRMS.
  \item The DATE argument is UTC expressed as an MJD.  This is,
        strictly speaking, wrong, because of leap seconds.  However,
        as long as the $\Delta$UT and the UTC are consistent there
        are no difficulties, except during a leap second.  In this
        case, the start of the 61st second of the final minute should
        begin a new MJD day and the old pre-leap $\Delta$UT should
        continue to be used.  As the 61st second completes, the MJD
        should revert to the start of the day as, simultaneously,
        the $\Delta$UT changes by one second to its post-leap new value.
  \item The $\Delta$UT (UT1$-$UTC) is tabulated in IERS circulars and
        elsewhere.  It increases by exactly one second at the end of
        each UTC leap second, introduced in order to keep $\Delta$UT
        within $\pm0$\tsec9.
  \item IMPORTANT -- TAKE CARE WITH THE LONGITUDE SIGN CONVENTION.  The
        longitude required by the present routine is {\bf east-positive},
        in accordance with geographical convention (and right-handed).
        In particular, note that the longitudes returned by the
        sla\_OBS routine are west-positive, following astronomical
        usage, and must be reversed in sign before use in the present
        routine.
  \item The polar coordinates XP,YP can be obtained from IERS
        circulars and equivalent publications.  The
        maximum amplitude is about 0.3~arcseconds.  If XP,YP values
        are unavailable, use XP=YP=0D0.  See page B60 of the 1988
        {\it Astronomical Almanac}\, for a definition of the two angles.
  \item The height above sea level of the observing station, HM,
        can be obtained from the {\it Astronomical Almanac}\, (Section J
        in the 1988 edition), or via the routine sla\_OBS.  If P,
        the pressure in millibars, is available, an adequate
        estimate of HM can be obtained from the following expression:
        \begin{quote}
         \verb|HM=-8149.9415D0*LOG(P/1013.25D0)|
        \end{quote}
        (See {\it Astrophysical Quantities}, C.W.Allen, 3rd~edition,
        \S52.)  Similarly, if the pressure P is not known,
        it can be estimated from the height of the observing
        station, HM as follows:
        \begin{quote}
         \verb|P=1013.25D0*EXP(-HM/8149.9415D0)|
        \end{quote}
        Note, however, that the refraction is proportional to the
        pressure and that an accurate P value is important for
        precise work.
 \end{enumerate}
}
%-----------------------------------------------------------------------
\routine{SLA\_OAPQK}{Quick Observed to Apparent}
{
 \action{Quick observed to apparent place.}
 \call{CALL sla\_OAPQK (TYPE, OB1, OB2, AOPRMS, RAP, DAP)}
}
\args{GIVEN}
{
 \spec{TYPE}{C*(*)}{type of coordinates -- `R', `H' or `A' (see below)} \\
 \spec{OB1}{D}{observed Az, HA or RA (radians; Az is N=0, E=$90^{\circ}$)} \\
 \spec{OB2}{D}{observed zenith distance or $\delta$ (radians)} \\
 \spec{AOPRMS}{D(14)}{star-independent apparent-to-observed parameters:} \\
 \specel   {(1)}     {geodetic latitude (radians)} \\
 \specel   {(2,3)}   {sine and cosine of geodetic latitude} \\
 \specel   {(4)}     {magnitude of diurnal aberration vector} \\
 \specel   {(5)}     {height (HM)} \\
 \specel   {(6)}     {ambient temperature (TDK)} \\
 \specel   {(7)}     {pressure (PMB)} \\
 \specel   {(8)}     {relative humidity (RH)} \\
 \specel   {(9)}     {wavelength (WL)} \\
 \specel   {(10)}    {lapse rate (TLR)} \\
 \specel   {(11,12)} {refraction constants A and B (radians)} \\
 \specel   {(13)}    {longitude + eqn of equinoxes +
                       ``sidereal $\Delta$UT'' (radians)} \\
 \specel   {(14)}    {local apparent sidereal time (radians)}
}
\args{RETURNED}
{
 \spec{RAP,DAP}{D}{geocentric apparent \radec}
}
\notes
{
 \begin{enumerate}
  \item Only the first character of the TYPE argument is significant.
        `R' or `r' indicates that OBS1 and OBS2 are the observed Right
        Ascension and Declination;  `H' or `h' indicates that they are
        Hour Angle (west +ve) and Declination; anything else (`A' or
        `a' is recommended) indicates that OBS1 and OBS2 are Azimuth
        (north zero, east is $90^{\circ}$) and Zenith Distance.  (Zenith
        distance is used rather than elevation in order to reflect the
        fact that no allowance is made for depression of the horizon.)
  \item The accuracy of the result is limited by the corrections for
        refraction.  Providing the meteorological parameters are
        known accurately and there are no gross local effects, the
        predicted azimuth and elevation should be within about
        0.1~arcsec for $\zeta<70^{\circ}$.  Even
        at a topocentric zenith distance of
        $90^{\circ}$, the accuracy in elevation should be better than
        1~arcmin;  useful results are available for a further
        $3^{\circ}$, beyond which the sla\_REFRO routine returns a
        fixed value of the refraction.  The complementary
        routines sla\_AOP (or sla\_AOPQK) and sla\_OAP (or sla\_OAPQK)
        are self-consistent to better than 1~microarcsecond all over
        the celestial sphere.
  \item It is advisable to take great care with units, as even
        unlikely values of the input parameters are accepted and
        processed in accordance with the models used.
  \item {\it Observed}\, \azel\ means the position that would be seen by a
        perfect theodolite located at the observer.  This is
        related to the observed \hadec\ via the standard rotation, using
        the geodetic latitude (corrected for polar motion), while the
        observed HA and RA are related simply through the local
        apparent ST.  {\it Observed}\, \radec\ or \hadec\ thus means the
        position that would be seen by a perfect equatorial located
        at the observer and with its polar axis aligned to the
        Earth's axis of rotation ({\it n.b.}\ not to the refracted pole).
        By removing from the observed place the effects of
        atmospheric refraction and diurnal aberration, the
        geocentric apparent \radec\ is obtained.
  \item Frequently, {\it mean}\, rather than {\it apparent}\,
        \radec\ will be required,
        in which case further transformations will be necessary.  The
        sla\_AMP {\it etc.}\ routines will convert
        the apparent \radec\ produced
        by the present routine into an FK5 J2000 mean place, by
        allowing for the Sun's gravitational lens effect, annual
        aberration, nutation and precession.  Should FK4 B1950
        coordinates be needed, the routines sla\_FK524 {\it etc.}\ will also
        need to be applied.
  \item To convert to apparent \radec\ the coordinates read from a
        real telescope, corrections would have to be applied for
        encoder zero points, gear and encoder errors, tube flexure,
        the position of the rotator axis and the pointing axis
        relative to it, non-perpendicularity between the mounting
        axes, and finally for the tilt of the azimuth or polar axis
        of the mounting (with appropriate corrections for mount
        flexures).  Some telescopes would, of course, exhibit other
        properties which would need to be accounted for at the
        appropriate point in the sequence.
  \item The star-independent apparent-to-observed-place parameters
        in AOPRMS may be computed by means of the sla\_AOPPA routine.
        If nothing has changed significantly except the time, the
        sla\_AOPPAT routine may be used to perform the requisite
        partial recomputation of AOPRMS.
 \end{enumerate}
}
%-----------------------------------------------------------------------
\routine{SLA\_OBS}{Observatory Parameters}
{
 \action{Look up an entry in a standard list of
         groundbased observing stations parameters.}
 \call{CALL sla\_OBS (N, C, NAME, W, P, H)}
}
\args{GIVEN}
{
 \spec{N}{I}{number specifying observing station}
}
\args{GIVEN or RETURNED}
{
 \spec{C}{C*(*)}{identifier specifying observing station}
}
\args{RETURNED}
{
 \spec{NAME}{C*(*)}{name of specified observing station} \\
 \spec{W}{D}{longitude (radians, west +ve)} \\
 \spec{P}{D}{geodetic latitude (radians, north +ve)} \\
 \spec{H}{D}{height above sea level (metres)}
}
\notes
{
 \begin{enumerate}
  \item Station identifiers C may be up to 10 characters long,
        and station names NAME may be up to 40 characters long.
  \item C and N are {\it alternative}\, ways of specifying the observing
        station.  The C option, which is the most generally useful,
        may be selected by specifying an N value of zero or less.
        If N is 1 or more, the parameters of the Nth station
        in the currently supported list are interrogated, and
        the station identifier C is returned as well as NAME, W,
        P and H.
  \item If the station parameters are not available, either because
        the station identifier C is not recognized, or because an
        N value greater than the number of stations supported is
        given, a name of `?' is returned and W, P and H are left in
        their current states.
  \item Programs can obtain a list of all currently supported
        stations by calling the routine repeatedly, with N=1,2,3...
        When NAME=`?' is seen, the list of stations has been
        exhausted.  The stations at the time of writing are listed
        below.
  \item Station numbers, identifiers, names and other details are
        subject to change and should not be hardwired into
        application programs.
  \item All station identifiers C are uppercase only;  lower case
        characters must be converted to uppercase by the calling
        program.  The station names returned may contain both upper-
        and lowercase.  All characters up to the first space are
        checked;  thus an abbreviated ID will return the parameters
        for the first station in the list which matches the
        abbreviation supplied, and no station in the list will ever
        contain embedded spaces.  C must not have leading spaces.
  \item IMPORTANT -- BEWARE OF THE LONGITUDE SIGN CONVENTION.  The
        longitude returned by sla\_OBS is
        {\bf west-positive} in accordance with astronomical
        usage.  However, this sign convention is left-handed and is
        the opposite of the one used by geographers; elsewhere in
        SLALIB the preferable east-positive convention is used.  In
        particular, note that for use in sla\_AOP, sla\_AOPPA and
        sla\_OAP the sign of the longitude must be reversed.
  \item Users are urged to inform the author of any improvements
        they would like to see made.  For example:
        \begin{itemize}
         \item typographical corrections
         \item more accurate parameters
         \item better station identifiers or names
         \item additional stations
        \end{itemize}
 \end{enumerate}
Stations supported by sla\_OBS at the time of writing:
\begin{tabbing}
xxxxxxxxxxxxxxxxx \= \kill
{\it ID} \> {\it NAME} \\ \\
AAT        \> Anglo-Australian 3.9m Telescope \\
ANU2.3     \> Siding Spring 2.3 metre \\
APO3.5     \> Apache Point 3.5m \\
ARECIBO    \> Arecibo 1000 foot \\
BLOEMF     \> Bloemfontein 1.52 metre \\
BOSQALEGRE \> Bosque Alegre 1.54 metre \\
CAMB1MILE  \> Cambridge 1 mile \\
CAMB5KM    \> Cambridge 5km \\
CATALINA61 \> Catalina 61 inch \\
DUNLAP74   \> David Dunlap 74 inch \\
DUPONT     \> Du Pont 2.5m Telescope, Las Campanas \\
EFFELSBERG \> Effelsberg 100 metre \\
ESO3.6     \> ESO 3.6 metre \\
ESONTT     \> ESO 3.5 metre NTT \\
FLAGSTF61  \> USNO 61 inch astrograph, Flagstaff \\
GBVA140    \> Greenbank 140 foot \\
GBVA300    \> Greenbank 300 foot \\
HARVARD    \> Harvard College Observatory 1.55m \\
HPROV1.52  \> Haute Provence 1.52 metre \\
HPROV1.93  \> Haute Provence 1.93 metre \\
JCMT       \> JCMT 15 metre \\
JODRELL1   \> Jodrell Bank 250 foot \\
KOTTAMIA   \> Kottamia 74 inch \\
KPNO158    \> Kitt Peak 158 inch \\
KPNO36FT   \> Kitt Peak 36 foot \\
KPNO84     \> Kitt Peak 84 inch \\
KPNO90     \> Kitt Peak 90 inch \\
LICK120    \> Lick 120 inch \\
LOWELL72   \> Perkins 72 inch, Lowell \\
LPO1       \> Jacobus Kapteyn 1m Telescope \\
LPO2.5     \> Isaac Newton 2.5m Telescope \\
LPO4.2     \> William Herschel 4.2m Telescope \\
MAUNAK88   \> Mauna Kea 88 inch \\
MCDONLD2.1 \> McDonald 2.1 metre \\
MCDONLD2.7 \> McDonald 2.7 metre \\
MMT        \> MMT, Mt Hopkins \\
MTEKAR     \> Mt Ekar 1.82 metre \\
MTHOP1.5   \> Mt Hopkins 1.5 metre \\
MTLEMMON60 \> Mt Lemmon 60 inch \\
NOBEYAMA   \> Nobeyama 45 metre \\
OKAYAMA    \> Okayama 1.88 metre \\
PALOMAR200 \> Palomar 200 inch \\
PALOMAR60  \> Palomar 60 inch \\
PARKES     \> Parkes 64 metre \\
QUEBEC1.6  \> Quebec 1.6 metre \\
SAAO74     \> Sutherland 74 inch \\
SANPM83    \> San Pedro Martir 83 inch \\
ST.ANDREWS \> St Andrews \\
STEWARD90  \> Steward 90 inch \\
STROMLO74  \> Mount Stromlo 74 inch \\
SUGARGROVE \> Sugar Grove 150 foot \\
TAUTNBG    \> Tautenburg 2 metre \\
TIDBINBLA  \> Tidbinbilla 64 metre \\
TOLOLO1.5M \> Cerro Tololo 1.5 metre \\
TOLOLO4M   \> Cerro Tololo 4 metre \\
UKIRT      \> UK Infra Red Telescope \\
USSR6      \> USSR 6 metre \\
USSR600    \> USSR 600 foot \\
VICBC      \> Victoria B.C. 1.85 metre \\
VLA        \> Very Large Array
\end{tabbing}
}
%-----------------------------------------------------------------------
\routine{SLA\_PA}{$h,\delta$ to Parallactic Angle}
{
 \action{Hour angle and declination to parallactic angle
         (double precision).}
 \call{D~=~sla\_PA (HA, DEC, PHI)}
}
\args{GIVEN}
{
 \spec{HA}{D}{hour angle in radians (geocentric apparent)} \\
 \spec{HA}{D}{declination in radians (geocentric apparent)} \\
 \spec{HA}{D}{latitude in radians (geodetic)}
}
\args{RETURNED}
{
 \spec{sla\_PA}{D}{parallactic angle (radians, in the range $\pm \pi$)}
}
\notes
{
 \begin{enumerate}
  \item The parallactic angle at a point in the sky is the position
        angle of the vertical, {\it i.e.}\ the angle between the direction to
        the pole and to the zenith.  In precise applications care must
        be taken only to use geocentric apparent \hadec\ and to consider
        separately the effects of atmospheric refraction and telescope
        mount errors.
  \item At the pole a zero result is returned.
 \end{enumerate}
}
%------------------------------------------------------------------------------
\routine{SLA\_PCD}{Apply Radial Distortion}
{
 \action{Apply pincushion/barrel distortion to a tangent-plane \xy.}
 \call{CALL sla\_PCD (DISCO,X,Y)}
}
\args{GIVEN}
{
 \spec{DISCO}{D}{pincushion/barrel distortion coefficient} \\
 \spec{X,Y}{D}{tangent-plane \xy}
}
\args{RETURNED}
{
 \spec{X,Y}{D}{distorted \xy}
}
\notes
{
 \begin{enumerate}
  \item The distortion is of the form $\rho = r (1 + c r^{2})$, where $r$ is
        the radial distance from the tangent point, $c$ is the DISCO
        argument, and $\rho$ is the radial distance in the presence of
        the distortion.
  \item For {\it pincushion}\, distortion, C is +ve;  for
        {\it barrel}\, distortion, C is $-$ve.
  \item For X,Y in units of one projection radius (in the case of
        a photographic plate, the focal length), the following
        DISCO values apply:

        \vspace{2ex}

        \hspace{5em}
        \begin{tabular}{|l|c|} \hline
         Geometry & DISCO \\ \hline \hline
         astrograph & 0.0 \\ \hline
         Schmidt & $-$0.3333 \\ \hline
         AAT PF doublet & +147.069 \\ \hline
         AAT PF triplet & +178.585 \\ \hline
         AAT f/8 & +21.20 \\ \hline
         JKT f/8 & +14.6 \\ \hline
        \end{tabular}

        \vspace{2ex}

  \item There is a companion routine, sla\_UNPCD, which performs
        an approximately inverse operation.
 \end{enumerate}
}
%------------------------------------------------------------------------------
\routine{SLA\_PM}{Proper Motion}
{
 \action{Apply corrections for proper motion to a star \radec.}
 \call{CALL sla\_PM (R0, D0, PR, PD, PX, RV, EP0, EP1, R1, D1)}
}
\args{GIVEN}
{
 \spec{R0,D0}{D}{\radec\ at epoch EP0 (radians)} \\
 \spec{PR,PD}{D}{proper motions:  rate of change of \radec\
                      (radians per year)} \\
 \spec{PX}{D}{parallax (arcsec)} \\
 \spec{RV}{D}{radial velocity (km~s$^{-1}$, +ve if receding)} \\
 \spec{EP0}{D}{start epoch in years ({\it e.g.}\ Julian epoch)} \\
 \spec{EP1}{D}{end epoch in years (same system as EP0)}
}
\args{RETURNED}
{
 \spec{R1,D1}{D}{\radec\ at epoch EP1 (radians)}
}
\anote{The $\alpha$ proper motions are $\dot{\alpha}$ rather than
       $\dot{\alpha}\cos\delta$, and are in the same coordinate
       system as R0,D0.}
\refs
{
 \begin{enumerate}
  \item 1984 {\it Astronomical Almanac}, pp B39-B41.
  \item Lederle \& Schwan, 1984.\ {\it Astr. Astrophys.}\ {\bf 134}, 1-6.
 \end{enumerate}
}
%-----------------------------------------------------------------------
\routine{SLA\_PREBN}{FK4 Precession Matrix}
{
 \action{Generate the matrix of precession between two epochs,
         using the old, pre IAU~1976, Bessel-Newcomb model, in
         Andoyer's formulation.}
 \call{CALL sla\_PREBN (BEP0, BEP1, RMATP)}
}
\args{GIVEN}
{
 \spec{BEP0}{D}{beginning Besselian epoch} \\
 \spec{BEP1}{D}{ending Besselian epoch}
}
\args{RETURNED}
{
 \spec{RMATP}{D(3,3)}{precession matrix}
}
\anote{The matrix is in the sense:
       \begin{verse}
        {\bf v}$_{1}$ =  {\bf M}$\cdot${\bf v}$_{0}$
       \end{verse}
       where {\bf v}$_{1}$ is the star vector relative to the
       mean equator and equinox of epoch BEP1, {\bf M} is the
       $3\times3$ matrix RMATP and
       {\bf v}$_{0}$ is the star vector relative to the
       mean equator and equinox of epoch BEP0.}
\aref{Smith {\it et al.}, 1989.\ {\it Astr.J.}\ {\bf 97}, 269.}
%-----------------------------------------------------------------------
\routine{SLA\_PREC}{FK5 Precession Matrix}
{
 \action{Form the matrix of precession between two epochs (IAU 1976, FK5).}
 \call{CALL sla\_PREC (EP0, EP1, RMATP)}
}
\args{GIVEN}
{
 \spec{EP0}{D}{beginning epoch} \\
 \spec{EP1}{D}{ending epoch}
}
\args{RETURNED}
{
 \spec{RMATP}{D(3,3)}{precession matrix}
}
\notes
{
 \begin{enumerate}
  \item The epochs are TDB Julian epochs.
  \item The matrix is in the sense:
        \begin{verse}
         {\bf v}$_{1}$ =  {\bf M}$\cdot${\bf v}$_{0}$
        \end{verse}
        where {\bf v}$_{1}$ is the star vector relative to the
        mean equator and equinox of epoch EP1, {\bf M} is the
        $3\times3$ matrix RMATP and
        {\bf v}$_{0}$ is the star vector relative to the
        mean equator and equinox of epoch EP0.
   \item Though the matrix method itself is rigorous, the
         precession angles are expressed through canonical
         polynomials which are valid only for a limited time
         span.  The absolute accuracy is better than 1 arcsec
         for a few hundred years either side of 2000, but
         beyond this the errors rapidly increase.
 \end{enumerate}
}
\refs
{
 \begin{enumerate}
  \item Lieske, J.H., 1979.\ {\it Astr.Astrophys.}\ {\it 73}, 282;
        equations 6 \& 7, p283.
  \item Kaplan, G.H., 1981.\ {\it USNO circular no.\ 163}, pA2.
 \end{enumerate}
}
%-----------------------------------------------------------------------
\routine{SLA\_PRECES}{Precession}
{
 \action{Precession -- either FK4 (Bessel-Newcomb, pre~IAU~1976) or
         FK5 (Fricke, post~IAU~1976) as required.}
 \call{CALL sla\_PRECES (SYSTEM, EP0, EP1, RA, DC)}
}
\args{GIVEN}
{
 \spec{SYSTEM}{C}{precession to be applied: `FK4' or `FK5'} \\
 \spec{EP0,EP1}{D}{starting and ending epoch} \\
 \spec{RA,DC}{D}{\radec, mean equator \& equinox of epoch EP0}
}
\args{RETURNED}
{
 \spec{RA,DC}{D}{\radec, mean equator \& equinox of epoch EP1}
}
\notes
{
 \begin{enumerate}
  \item Lowercase characters in SYSTEM are acceptable.
  \item The epochs are Besselian if SYSTEM=`FK4' and Julian if `FK5'.
        For example, to precess coordinates in the old system from
        equinox 1900.0 to 1950.0 the call would be:
        \begin{quote}
         {\tt CALL sla\_PRECES ('FK4', 1900D0, 1950D0, RA, DC)}
        \end{quote}
  \item This routine will {\bf NOT} correctly convert between the old and
        the new systems -- for example conversion from B1950 to J2000.
        For these purposes see sla\_FK425, sla\_FK524, sla\_FK45Z and
        sla\_FK54Z.
  \item If an invalid SYSTEM is supplied, values of $-$99D0,$-$99D0 are
        returned for both RA and DC.
 \end{enumerate}
}
%-----------------------------------------------------------------------
\routine{SLA\_PRENUT}{Precession/Nutation Matrix}
{
 \action{Form the matrix of precession and nutation (IAU~1976, FK5).}
 \call{CALL sla\_PRENUT (EPOCH, DATE, RMATPN)}
}
\args{GIVEN}
{
 \spec{EPOCH}{D}{Julian Epoch for mean coordinates} \\
 \spec{DATE}{D}{Modified Julian Date (JD$-$2400000.5)
                       for true coordinates}
}
\args{RETURNED}
{
 \spec{RMATPN}{D(3,3)}{combined precession/nutation matrix}
}
\notes
{
 \begin{enumerate}
  \item The epoch and date are TDB.
  \item The matrix is in the sense:
        \begin{verse}
         {\bf v}$_{true}$ =  {\bf M}$\cdot${\bf v}$_{mean}$
        \end{verse}
        where {\bf v}$_{true}$ is the star vector relative to the
        true equator and equinox of epoch DATE, {\bf M} is the
        $3\times3$ matrix RMATPN and
        {\bf v}$_{mean}$ is the star vector relative to the
        mean equator and equinox of epoch EPOCH.
 \end{enumerate}
}
%-----------------------------------------------------------------------
\routine{SLA\_PVOBS}{Observatory Position \& Velocity}
{
 \action{Position and velocity of an observing station.}
 \call{CALL sla\_PVOBS (P, H, STL, PV)}
}
\args{GIVEN}
{
 \spec{P}{D}{latitude (geodetic, radians)} \\
 \spec{H}{D}{height above reference spheroid (geodetic, metres)} \\
 \spec{STL}{D}{local apparent sidereal time (radians)}
}
\args{RETURNED}
{
 \spec{PV}{D(6)}{\xyzxyzd\ (AU, AU~s$^{-1}$, geocentric apparent)}
}
\anote{IAU 1976 constants are used.}
%-----------------------------------------------------------------------
\routine{SLA\_PXY}{Apply Linear Model}
{
 \action{Given arrays of {\it expected}\, and {\it measured}\,
         \xy\ coordinates, and a
         linear model relating them (as produced by sla\_FITXY), compute
         the array of {\it predicted}\, coordinates and the RMS residuals.}
 \call{CALL sla\_PXY (NP,XYE,XYM,COEFFS,XYP,XRMS,YRMS,RRMS)}
}
\args{GIVEN}
{
 \spec{NP}{I}{number of samples} \\
 \spec{XYE}{D(2,NP)}{expected \xy\ for each sample} \\
 \spec{XYM}{D(2,NP)}{measured \xy\ for each sample} \\
 \spec{COEFFS}{D(6)}{coefficients of model (see below)}
}
\args{RETURNED}
{
 \spec{XYP}{D(2,NP)}{predicted \xy\ for each sample} \\
 \spec{XRMS}{D}{RMS in X} \\
 \spec{YRMS}{D}{RMS in Y} \\
 \spec{RRMS}{D }{total RMS (vector sum of XRMS and YRMS)}
}
\notes
{
 \begin{enumerate}
  \item The model is supplied in the array COEFFS.  Naming the
        six elements of COEFFS $a,b,c,d,e$ \& $f$,
        the model transforms {\it measured}\, coordinates
        $[x_{m},y_{m}\,]$ into {\it predicted}\, coordinates
        $[x_{p},y_{p}\,]$ as follows:
        \begin{verse}
         $x_{p} = a + bx_{m} + cy_{m}$ \\
         $y_{p} = d + ex_{m} + fy_{m}$
        \end{verse}
  \item The residuals are $(x_{p}-x_{e})$ and $(y_{p}-y_{e})$.
  \item If NP is less than or equal to zero, no coordinates are
        transformed, and the RMS residuals are all zero.
  \item See also sla\_FITXY, sla\_INVF, sla\_XY2XY, sla\_DCMPF
 \end{enumerate}
}
%-----------------------------------------------------------------------
\routine{SLA\_RANDOM}{Random Number}
{
 \action{Generate pseudo-random real number in the range $0 \leq x < 1$.}
 \call{R~=~sla\_RANDOM (SEED)}
}
\args{GIVEN}
{
 \spec{SEED}{R}{an arbitrary real number}
}
\args{RETURNED}
{
 \spec{SEED}{R}{a new arbitrary value} \\
 \spec{sla\_RANDOM}{R}{Pseudo-random real number $0 \leq x < 1$.}
}
\anote{The implementation is {\bf VAX-dependent}.}
%-----------------------------------------------------------------------
\routine{SLA\_RANGE}{Put Angle into Range $\pm\pi$}
{
 \action{Normalize an angle into the range $\pm\pi$ (single precision).}
 \call{R~=~sla\_RANGE (ANGLE)}
}
\args{GIVEN}
{
 \spec{ANGLE}{R}{angle in radians}
}
\args{RETURNED}
{
 \spec{sla\_RANGE}{R}{ANGLE expressed in the range $\pm\pi$.}
}
%-----------------------------------------------------------------------
\routine{SLA\_RANORM}{Put Angle into Range $0\!-\!2\pi$}
{
 \action{Normalize an angle into the range $0\!-\!2\pi$ (single precision).}
 \call{R~=~sla\_RANORM (ANGLE)}
}
\args{GIVEN}
{
 \spec{ANGLE}{R}{angle in radians}
}
\args{RETURNED}
{
 \spec{sla\_RANORM}{R}{ANGLE expressed in the range $0\!-\!2\pi$}
}
%-----------------------------------------------------------------------
\routine{SLA\_RCC}{Barycentric Coordinate Time}
{
 \call{D~=~sla\_RCC (TDB, UT1, WL, U, V)}
 \action{The relativistic clock correction TDB$-$TT, the
         difference between {\it proper time}\,
         on Earth and {\it coordinate time}\, in the solar system barycentric
         space-time frame of reference.  The proper time is TT;  the
         coordinate time is {\it an implementation}\, of TDB.}
}
\args{GIVEN}
{
 \spec{TDB}{D}{coordinate time (MJD: JD$-$2400000.5)} \\
 \spec{UT1}{D}{universal time (fraction of one day)} \\
 \spec{WL}{D}{clock longitude (radians west)} \\
 \spec{U}{D}{clock distance from Earth spin axis (km)} \\
 \spec{V}{D}{clock distance north of Earth equatorial plane (km)}
}
\args{RETURNED}
{
 \spec{sla\_RCC}{D}{TDB$-$TT (sec)}
}
\notes
{
 \begin{enumerate}
  \item TDB may be considered to
        be the coordinate time in the solar system barycentre frame of
        reference, and TT is the proper time given by clocks at mean sea
        level on the Earth.
  \item The result has a main (annual) sinusoidal term of amplitude
        approximately 1.66ms, plus planetary terms up to about
        20$\mu$s, and lunar and diurnal terms up to 2$\mu$s.  The
        variation arises from the transverse Doppler effect and the
        gravitational red-shift as the observer varies in speed and
        moves through different gravitational potentials.
  \item The argument TDB is, strictly, the barycentric coordinate time;
        however, the terrestrial proper time (TT) can in practice be used.
  \item The geocentric model is that of Fairhead \& Bretagnon (1990),
        in its full
        form.  It was supplied by Fairhead (private communication)
        as a Fortran subroutine.  A number of coding changes were made to
        this subroutine in order
        match the calling sequence of previous versions of the present
        routine, to comply with Starlink programming standards and to
        avoid compilation problems on certain machines.  Under
        VAX/VMS, the numerical results are essentially unaffected by the
        changes.  The topocentric model is from Moyer (1981) and Murray (1983).
        During the interval 1950-2050, the absolute accuracy of the
        geocentric model is better than $\pm3$~nanoseconds
        relative to direct numerical integrations using the JPL DE200/LE200
        solar system ephemeris.
  \item The IAU definition of TDB is that it must differ from TT only by
        periodic terms.  Though practical, this is an imprecise definition
        which ignores the existence of very long-period and secular effects
        in the dynamics of the solar system.  As a consequence, different
        implementations of TDB will, in general, differ in zero-point and
        will drift linearly relative to one other.
 \end{enumerate}
}
\refs
{
 \begin{enumerate}
  \item Fairhead, L.\ \& Bretagnon, P., 1990.\
        {\it Astr.Astrophys.}\ {\bf 229}, 240-247.
  \item Moyer, T.D., 1981.\ {\it Cel.Mech.}\ {\bf 23}, 33.
  \item Murray, C.A., 1983,\ {\it Vectorial Astrometry}, Adam Hilger.
 \end{enumerate}
}
%-----------------------------------------------------------------------
\routine{SLA\_REFCO}{Refraction Constants}
{
 \action{Determine the constants $a$ and $b$ in the
         atmospheric refraction model
         $\Delta \zeta = a \tan \zeta + b \tan^{3} \zeta$,
         where $\zeta$ is the {\it observed}\, zenith distance
         ({\it i.e.}\ affected by refraction) and $\Delta \zeta$ is
         what to add to $\zeta$ to give the {\it topocentric}\,
         ({\it i.e.\ in vacuo}) zenith distance.}
 \call{CALL sla\_REFCO (HM, TDK, PMB, RH, WL, PHI, TLR, EPS, REFA, REFB)}
}
\args{GIVEN}
{
 \spec{HM}{D}{height of the observer above sea level (metre)} \\
 \spec{TDK}{D}{ambient temperature at the observer (degrees K)} \\
 \spec{PMB}{D}{pressure at the observer (millibar)} \\
 \spec{RH}{D}{relative humidity at the observer (range 0-1)} \\
 \spec{WL}{D}{effective wavelength of the source (micrometre)} \\
 \spec{PHI}{D}{latitude of the observer (radian, astronomical)} \\
 \spec{TLR}{D}{temperature lapse rate in the troposphere
                                     (degrees K per metre)} \\
 \spec{EPS}{D}{precision required to terminate iteration (radian)}
}
\args{RETURNED}
{
 \spec{REFA}{D}{$\tan \zeta$ coefficient (radians)} \\
 \spec{REFB}{D}{$\tan^{3} \zeta$ coefficient (radians)}
}
\notes
{
 \begin{enumerate}
  \item Typical values for the TLR and EPS arguments might be 0.0065D0 and
        1D-10 respectively.
  \item The radio refraction is chosen by specifying WL $>100$~micrometres.
  \item The constants are such that the model agrees precisely with
        the full integration performed by the sla\_REFRO routine at
        zenith distances $\tan^{-1} 1$ and $\tan^{-1} 4$.
  \item Relative to the comprehensive refraction model used by this routine,
        the simple $\Delta \zeta = a \tan \zeta + b \tan^{3} \zeta$
        formula achieves
        0.5~arcsec accuracy for $\zeta<80^{\circ}$,
        0.01~arcsec accuracy for $\zeta<60^{\circ}$, and
        0.001~arcsec accuracy for $\zeta<45^{\circ}$.
 \end{enumerate}
}
%-----------------------------------------------------------------------
\routine{SLA\_REFRO}{Refraction}
{
 \action{Atmospheric refraction, for radio or optical wavelengths.}
 \call{CALL sla\_REFRO (ZOBS, HM, TDK, PMB, RH, WL, PHI, TLR, EPS, REF)}
}
\args{GIVEN}
{
 \spec{ZOBS}{D}{observed zenith distance of the source (radians)} \\
 \spec{HM}{D}{height of the observer above sea level (metre)} \\
 \spec{TDK}{D}{ambient temperature at the observer (degrees K)} \\
 \spec{PMB}{D}{pressure at the observer (millibar)} \\
 \spec{RH}{D}{relative humidity at the observer (range 0-1)} \\
 \spec{WL}{D}{effective wavelength of the source (micrometre)} \\
 \spec{PHI}{D}{latitude of the observer (radian, astronomical)} \\
 \spec{TLR}{D}{temperature lapse rate in the troposphere
                                     (degrees K per metre)} \\
 \spec{EPS}{D}{precision required to terminate iteration (radian)}
}
\args{RETURNED}
{
 \spec{REF}{D}{refraction: {\it in vacuo}\, ZD minus observed ZD (radians)}
}
\notes
{
 \begin{enumerate}
  \item Typical values for the TLR and EPS arguments might be 0.0065D0 and
        1D-10 respectively.
  \item This routine computes the refraction for zenith distances up to
        and a little beyond $90^{\circ}$ using the method of Hohenkerk and
        Sinclair ({\it NAO Technical Notes}\, 59 and 63).  The code is a
        slightly modified form of the AREF subroutine of C.Hohenkerk
        (HMNAO, September~1984), with extensions to support
        radio wavelengths.  Most of the modifications are cosmetic;
        in addition the angle arguments have been changed to radians,
        any value of ZOBS is allowed, and other values have been
        limited to safe values.
        The radio expressions were
        devised by A.T.Sinclair (RGO -- private communication), based on
        the Essen \& Froome refractivity formula adopted in Resolution 1
        of the 13th International Geodesy Association General Assembly
        ({\it Bulletin Geod\'esique}\, 1963 p390).
  \item The radio refraction is chosen by specifying WL $>100$~micrometres.
  \item Before use, the value of ZOBS is expressed in the range $\pm \pi$.
        If this ranged ZOBS is negative, the result REF is computed from its
        absolute value before being made negative to match.  In addition, if
        it has an absolute value greater than $93^{\circ}$, a fixed REF value
        equal to the result for ZOBS = $93^{\circ}$ is returned, appropriately
        signed.
  \item Fixed values of the water vapour exponent, height of tropopause, and
        height at which refraction is negligible are used.
  \end{enumerate}
}
%-----------------------------------------------------------------------
\routine{SLA\_REFV}{Apply Refraction to Vector}
{
 \action{Adjust an unrefracted Cartesian vector to include the effect of
         atmospheric refraction, using the simple
         $\Delta \zeta = a \tan \zeta + b \tan^{3} \zeta$ model.}
 \call{CALL sla\_REFV (VU, REFA, REFB, VR)}
}
\args{GIVEN}
{
 \spec{VU}{D}{unrefracted position of the source (\azel\ 3-vector)} \\
 \spec{REFA}{D}{$\tan \zeta$ coefficient (radians)} \\
 \spec{REFB}{D}{$\tan^{3} \zeta$ coefficient (radians)}
}
\args{RETURNED}
{
 \spec{VR}{D}{refracted position of the source (\azel\ 3-vector)}
}
\notes
{
 \begin{enumerate}
  \item Note that this routine applies the adjustment for refraction in
        the opposite sense to the usual one -- it takes an unrefracted
        ({\it in vacuo}) position and produces an observed (refracted)
        position, whereas the basic
        $\Delta \zeta = a \tan \zeta + b \tan^{3} \zeta$ 
        model strictly
        applies to the case where a refracted position is available and
        must be corrected for refraction.  This requires an inverted form of
        the refraction expression;  the algorithm used here is based on
        two iterations of the Newton-Raphson method.
  \item See also the routine sla\_REFZ, which performs the adjustment to
        the zenith distance rather than in Cartesian \azel\ coordinates.
        The results from sla\_REFZ are slightly more accurate (in a
        numerical sense) to those produced by the present routine, due to
        the various approximations used in the latter for simplicity and
        speed.
  \item This routine will not return useful answers at low and negative
        elevations.  To avoid arithmetic problems, vectors below a small
        positive elevation are returned unaltered.  The vector VU must be
        of unit length;  no check is made.
 \end{enumerate}
}
%-----------------------------------------------------------------------
\routine{SLA\_REFZ}{Apply Refraction to ZD}
{
 \action{Adjust an unrefracted zenith distance to include the effect of
         atmospheric refraction, using the simple
         $\Delta \zeta = a \tan \zeta + b \tan^{3} \zeta$ model.}
 \call{CALL sla\_REFZ (ZU, REFA, REFB, ZR)}
}
\args{GIVEN}
{
 \spec{ZU}{D}{unrefracted zenith distance of the source (radians)} \\
 \spec{REFA}{D}{$\tan \zeta$ coefficient (radians)} \\
 \spec{REFB}{D}{$\tan^{3} \zeta$ coefficient (radians)}
}
\args{RETURNED}
{
 \spec{ZR}{D}{refracted zenith distance (radians)}
}
\notes
{
 \begin{enumerate}
  \item Note that this routine applies the adjustment for refraction in
        the opposite sense to the usual one -- it takes an unrefracted
        ({\it in vacuo}) position and produces an observed (refracted)
        position, whereas the basic
        $\Delta \zeta = a \tan \zeta + b \tan^{3} \zeta$ 
        model strictly
        applies to the case where a refracted position is available and
        must be corrected for refraction.  This requires an inverted form of
        the refraction expression;  the algorithm used here is based on
        two iterations of the Newton-Raphson method.
        For numerical consistency with the
        refracted to unrefracted model, two iterations are used;  the error
        is less than $10^{-11}$~arcseconds at $80^{\circ}$ degrees ZD, and
        is still under 1~milliarcsecond at $88^{\circ}$.
  \item For the results to be useful, ZU should be in the range zero to
        $89^{\circ}$ (expressed in radians).  For ZU bigger than $89^{\circ}$
        or negative, ZR is set equal to ZU.
  \item See also the routine sla\_REFV, which performs the adjustment in
        Cartesian \azel\ coordinates.
 \end{enumerate}
}
%-----------------------------------------------------------------------
\routine{SLA\_RVEROT}{RV Corrn to Earth Centre}
{
 \action{Velocity component in a given direction due to Earth rotation.}
 \call{R~=~sla\_RVEROT (PHI, RA, DA, ST)}
}
\args{GIVEN}
{
 \spec{PHI}{R}{geodetic latitude of observing station (radians)} \\
 \spec{RA,DA}{R}{apparent \radec\ (radians)} \\
 \spec{ST}{R}{local apparent sidereal time (radians)}
}
\args{RETURNED}
{
 \spec{sla\_RVEROT}{R}{Component of Earth rotation in
                       direction RA,DA (km~s$^{-1}$)}
}
\notes
{
 \begin{enumerate}
  \item Sign convention: the result is positive when the observatory
        is receding from the given point on the sky.
  \item Accuracy: the simple algorithm used assumes a spherical Earth and
        an observing station at sea level;  for actual observing
        sites, the error is unlikely to be greater than 0.0005~km~s$^{-1}$.
        For applications requiring greater precision, see the routine
        sla\_PVOBS.
 \end{enumerate}
}
%-----------------------------------------------------------------------
\routine{SLA\_RVGALC}{RV Corrn to Galactic Centre}
{
 \action{Velocity component in a given direction due to the rotation
         of the Galaxy.}
 \call{R~=~sla\_RVGALC (R2000, D2000)}
}
\args{GIVEN}
{
 \spec{R2000,D2000}{R}{J2000.0 mean \radec\ (radians)}
}
\args{RETURNED}
{
 \spec{sla\_RVGALC}{R}{Component of dynamical LSR motion in direction
                       R2000,D2000 (km~s$^{-1}$)}
}
\notes
{
 \begin{enumerate}
  \item Sign convention: the result is positive when the LSR
        is receding from the given point on the sky.
  \item The Local Standard of Rest used here is a point in the
        vicinity of the Sun which is in a circular orbit around
        the Galactic centre.  Sometimes called the {\it dynamical}\/ LSR,
        it is not to be confused with a {\it kinematical}\/ LSR, which
        is the mean standard of rest of star catalogues or stellar
        populations.
  \item The dynamical LSR velocity due to Galactic rotation is assumed to
        be 220~km~s$^{-1}$ towards $l^{I\!I}=90^{\circ}$,
                                   $b^{I\!I}=0$.
 \end{enumerate}
\aref{Kerr \& Lynden-Bell (1986), MNRAS, 221, p1023.},
}
%-----------------------------------------------------------------------
\routine{SLA\_RVLG}{RV Corrn to Local Group}
{
 \action{Velocity component in a given direction due to the combination
         of the rotation of the Galaxy and the motion of the Galaxy
         relative to the mean motion of the local group.}
 \call{R~=~sla\_RVLG (R2000, D2000)}
}
\args{GIVEN}
{
 \spec{R2000,D2000}{R}{J2000.0 mean \radec\ (radians)}
}
\args{RETURNED}
{
 \spec{sla\_RVLG}{R}{Component of {\bf solar} ({\it n.b.})
                     motion in direction R2000,D2000 (km~s$^{-1}$)}
}
\anote{Sign convention: the result is positive when
       the Sun is receding from the given point on the sky.}
\aref{{\it IAU Trans.}\ 1976.\ {\bf 168}, p201.}
%-----------------------------------------------------------------------
\routine{SLA\_RVLSRD}{RV Corrn to Dynamical LSR}
{
 \action{Velocity component in a given direction due to the Sun's
         motion with respect to the ``dynamical'' Local Standard of Rest.}
 \call{R~=~sla\_RVLSR (R2000, D2000)}
}
\args{GIVEN}
{
 \spec{R2000,D2000}{R}{J2000.0 mean \radec\ (radians)}
}
\args{RETURNED}
{
 \spec{sla\_RVLSR}{R}{Component of solar motion in direction
                      R2000,D2000 (km~s$^{-1}$)}
}
\notes
{
 \begin{enumerate}
  \item Sign convention: the result is positive when
        the Sun is receding from the given point on the sky.
  \item The Local Standard of Rest used here is the {\it dynamical}\/ LSR,
        a point in the vicinity of the Sun which is in a circular
        orbit around the Galactic centre.  The Sun's motion with
        respect to the dynamical LSR is called the {\it peculiar}\/ solar
        motion.
  \item There is another type of LSR, called a {\it kinematical}\/ LSR.  A
        kinematical LSR is the mean standard of rest of specified star
        catalogues or stellar populations, and several slightly
        different kinematical LSRs are in use.  The Sun's motion with
        respect to an agreed kinematical LSR is known as the
        {\it standard}\/ solar motion.
        The dynamical LSR is seldom used by observational astronomers,
        who conventionally use a kinematical LSR such as the one implemented
        in the routine sla\_RVLSRK.
  \item The peculiar solar motion is from Delhaye (1965), in {\it Stars
        and Stellar Systems}, vol~5, p73:  in Galactic Cartesian
        coordinates (+9,+12,+7)~km~s$^{-1}$.
        This corresponds to about 16.6~km~s$^{-1}$
        towards Galactic coordinates $l^{I\!I}=53^{\circ},b^{I\!I}=+25^{\circ}$.
 \end{enumerate}
}
%-----------------------------------------------------------------------
\routine{SLA\_RVLSRK}{RV Corrn to Kinematical LSR}
{
 \action{Velocity component in a given direction due to the Sun's
         motion with respect to a kinematical Local Standard of Rest.}
 \call{R~=~sla\_RVLSR (R2000, D2000)}
}
\args{GIVEN}
{
 \spec{R2000,D2000}{R}{J2000.0 mean \radec\ (radians)}
}
\args{RETURNED}
{
 \spec{sla\_RVLSR}{R}{Component of solar motion in direction
                      R2000,D2000 (km~s$^{-1}$)}
}
\notes
{
 \begin{enumerate}
  \item Sign convention: the result is positive when
        the Sun is receding from the given point on the sky.
  \item The Local Standard of Rest used here is one of several
        {\it kinematical}\/ LSRs in common use.  A kinematical LSR is the
        mean standard of rest of specified star catalogues or stellar
        populations.  The Sun's motion with respect to a kinematical
        LSR is known as the {\it standard}\/ solar motion.
  \item There is another sort of LSR, seldom used by observational
        astronomers, called the {\it dynamical}\/ LSR.  This is a
        point in the vicinity of the Sun which is in a circular orbit
        around the Galactic centre.  The Sun's motion with respect to
        the dynamical LSR is called the {\it peculiar}\/ solar motion.  To
        obtain a radial velocity correction with respect to the
        dynamical LSR use the routine sla\_RVLSRD.
  \item The adopted standard solar motion is 20~km~s$^{-1}$
        towards $\alpha=18^{\rm h},\delta=+30^{\circ}$ (1900).
 \end{enumerate}
\refs
{
 \begin{enumerate}
  \item Delhaye (1965), in {\it Stars and Stellar Systems}, vol~5, p73.
  \item {\it Methods of Experimental Physics}\/ (ed Meeks), vol~12,
        part~C, sec~6.1.5.2, p281.
 \end{enumerate}
}
}
%-----------------------------------------------------------------------
\routine{SLA\_S2TP}{Spherical to Tangent Plane}
{
 \action{Projection of spherical coordinates onto the tangent plane
         (single precision).}
 \call{CALL sla\_S2TP (RA, DEC, RAZ, DECZ, XI, ETA, J)}
}
\args{GIVEN}
{
 \spec{RA,DEC}{R}{spherical coordinates of point to be projected (radians)} \\
 \spec{RAZ,DECZ}{R}{spherical coordinates of tangent point (radians)}
}
\args{RETURNED}
{
 \spec{XI,ETA}{R}{rectangular coordinates on tangent plane (radians)} \\
 \spec{J}{I}{status:} \\
 \spec{}{}{\hspace{1.5em} 0 = OK, star on tangent plane} \\
 \spec{}{}{\hspace{1.5em} 1 = error, star too far from axis} \\
 \spec{}{}{\hspace{1.5em} 2 = error, antistar too far from axis} \\
 \spec{}{}{\hspace{1.5em} 3 = error, antistar on tangent plane}
}
\anote{This projection is called the {\it gnomonic}\, projection;  the
       \xy\ coordinates are called {\it standard coordinates}.
       The latter are in units of the distance from the tangent plane
       to the projection point, {\it i.e.}\ radians near the origin.}
%-----------------------------------------------------------------------
\routine{SLA\_SEP}{Angle Between 2 Points on Sphere}
{
 \action{Angle between two points on a sphere (single precision).}
 \call{R~=~sla\_SEP (A1, B1, A2, B2)}
}
\args{GIVEN}
{
 \spec{A1,B1}{R}{spherical coordinates of one point (radians)} \\
 \spec{A2,B2}{R}{spherical coordinates of the other point (radians)}
}
\args{RETURNED}
{
 \spec{sla\_SEP}{R}{angle between [A1,B1] and [A2,B2] in radians}
}
\notes
{
 \begin{enumerate}
  \item The spherical coordinates are right ascension and declination,
  longitude and latitude, {\it etc.}\, in radians.
  \item The result is always positive.
 \end{enumerate}
}
%-----------------------------------------------------------------------
\routine{SLA\_SMAT}{Solve Simultaneous Equations}
{
 \action{Matrix inversion and solution of simultaneous equations
         (single precision).}
 \call{CALL sla\_SMAT (N, A, Y, D, JF, IW)}
}
\args{GIVEN}
{
 \spec{N}{I}{number of unknowns} \\
 \spec{A}{R(N,N)}{matrix} \\
 \spec{Y}{R(N)}{vector}
}
\args{RETURNED}
{
 \spec{A}{R(N,N)}{matrix inverse} \\
 \spec{Y}{R(N)}{solution} \\
 \spec{D}{R}{determinant} \\
 \spec{JF}{I}{singularity flag: 0=OK} \\
 \spec{IW}{I(N)}{workspace}
}
\notes
{
 \begin{enumerate}
  \item For the set of $n$ simultaneous linear equations in $n$ unknowns:
        \begin{verse}
         {\bf A}$\cdot${\bf y} = {\bf x}
        \end{verse}
        where:
        \begin{itemize}
         \item {\bf A} is a non-singular $n \times n$ matrix,
         \item {\bf y} is the vector of $n$ unknowns, and
         \item {\bf x} is the known vector,
        \end{itemize}
        sla\_SMAT computes:
        \begin{itemize}
         \item the inverse of matrix {\bf A},
         \item the determinant of matrix {\bf A}, and
         \item the vector of $n$ unknowns {\bf y}.
        \end{itemize}
        Argument N is the order $n$, A (given) is the matrix {\bf A},
        Y (given) is the vector {\bf x} and Y (returned)
        is the vector {\bf y}.
        The argument A (returned) is the inverse matrix {\bf A}$^{-1}$,
        and D is {\it det}({\bf A}).
  \item JF is the singularity flag.  If the matrix is non-singular,
        JF=0 is returned.  If the matrix is singular, JF=$-$1
        and D=0.0 are returned.  In the latter case, the contents
        of array A on return are undefined.
  \item The algorithm is Gaussian elimination with partial pivoting.
        This method is very fast;  some much slower algorithms can give
        better accuracy, but only by a small factor.
  \item This routine replaces the obsolete sla\_SMATRX.
 \end{enumerate}
}
%-----------------------------------------------------------------------
\routine{SLA\_SUBET}{Remove E-terms}
{
 \action{Remove the E-terms (elliptic component of annual aberration)
         from a pre IAU~1976 catalogue \radec\ to give a mean place.}
 \call{CALL sla\_SUBET (RC, DC, EQ, RM, DM)}
}
\args{GIVEN}
{
 \spec{RC,DC}{D}{\radec\ with E-terms included (radians)} \\
 \spec{EQ}{D}{Besselian epoch of mean equator and equinox}
}
\args{RETURNED}
{
 \spec{RM,DM}{D}{\radec\ without E-terms (radians)}
}
\anote{Most star positions from pre-1984 optical catalogues (or
       obtained by astrometry with respect to such stars) have the
       E-terms built-in.  This routine converts such a position to a
       formal mean place (allowing, for example, comparison with a
       pulsar timing position).}
\aref{{\it Explanatory Supplement to the Astronomical Ephemeris},
      section 2D, page 48.}
%-----------------------------------------------------------------------
\routine{SLA\_SUPGAL}{Supergalactic to Galactic}
{
 \action{Transformation from de Vaucouleurs supergalactic coordinates
         to IAU 1958 galactic coordinates.}
 \call{CALL sla\_GALSUP (DL, DB, DSL, DSB)}
}
\args{GIVEN}
{
 \spec{DSL,DSB}{D}{supergalactic longitude and latitude (radians)}
}
\args{RETURNED}
{
 \spec{DL,DB}{D}{galactic longitude and latitude \gal\ (radians)}
}
\refs
{
 \begin{enumerate}
  \item de Vaucouleurs, de Vaucouleurs, \& Corwin, {\it Second Reference
    Catalogue of Bright Galaxies}, U.Texas, p8.
  \item Systems \& Applied Sciences Corp., documentation for the
        machine-readable version of the above catalogue,
        Contract NAS 5-26490.
 \end{enumerate}
 (These two references give different values for the galactic
 longitude of the supergalactic origin.  Both are wrong;  the
 correct value is $l^{I\!I}=137.37$.)
}
%------------------------------------------------------------------------------
\routine{SLA\_SVD}{Singular Value Decomposition}
{
 \action{Singular value decomposition.
         This routine expresses a given matrix {\bf A} as the product of
         three matrices {\bf U}, {\bf W}, {\bf V}$^{T}$:
         \begin{tabbing}
         XXXXXX \= \kill
         \> {\bf A} = {\bf U} $\cdot$ {\bf W} $\cdot$ {\bf V}$^{T}$
         \end{tabbing}
         where:
         \begin{tabbing}
         XXXXXX \= XXXX \= \kill
         \> {\bf A} \> is any $m$ (rows) $\times n$ (columns) matrix,
                       where $m > n$ \\
         \> {\bf U} \> is an $m \times n$ column-orthogonal matrix \\
         \> {\bf W} \> is an $n \times n$ diagonal matrix with
                       $w_{ii} \geq 0$ \\
         \> {\bf V}$^{T}$ \> is the transpose of an $n \times n$
                             orthogonal matrix
\end{tabbing}
}
 \call{CALL sla\_SVD (M, N, MP, NP, A, W, V, WORK, JSTAT)}
}
\args{GIVEN}
{
 \spec{M,N}{I}{$m$, $n$, the numbers of rows and columns in matrix {\bf A}} \\
 \spec{MP,NP}{I}{physical dimensions of array containing matrix {\bf A}} \\
 \spec{A}{D(MP,NP)}{array containing $m \times n$ matrix {\bf A}}
}
\args{RETURNED}
{
 \spec{A}{D(MP,NP)}{array containing $m \times n$ column-orthogonal
                    matrix {\bf U}} \\
 \spec{W}{D(N)}{$n \times n$ diagonal matrix {\bf W}
               (diagonal elements only)} \\
 \spec{V}{D(NP,NP)}{array containing $n \times n$ orthogonal
                    matrix {\bf V} ({\it n.b.}\ not {\bf V}$^{T}$)} \\
 \spec{WORK}{D(N)}{workspace} \\
 \spec{JSTAT}{I}{0~=~OK, $-$1~=~array A wrong shape, $>$0~=~index of W
                 for which convergence failed (see note~3, below)}
}
\notes
{
 \begin{enumerate}
  \item M and N are the {\it logical}\, dimensions of the
        matrices and vectors concerned, which can be located in
        arrays of larger {\it physical}\, dimensions, given by MP and NP.
  \item V contains matrix V, not the transpose of matrix V.
  \item If the status JSTAT is greater than zero, this need not
        necessarily be treated as a failure.  It means that, due to
        chance properties of the matrix A, the QR transformation
        phase of the routine did not fully converge in a predefined
        number of iterations, something that very seldom occurs.
        When this condition does arise, it is possible that the
        elements of the diagonal matrix W have not been correctly
        found.  However, in practice the results are likely to
        be trustworthy.  Applications should report the condition
        as a warning, but then proceed normally.
 \end{enumerate}
}
\refs{The algorithm is an adaptation of the routine SVD in the {\it EISPACK}\,
      library (Garbow~{\it et~al.}\ 1977, {\it EISPACK Guide Extension},
      Springer Verlag), which is a FORTRAN~66 implementation of the Algol
      routine SVD of Wilkinson \& Reinsch 1971 ({\it Handbook for Automatic
      Computation}, vol~2, ed Bauer~{\it et~al.}, Springer Verlag).  These
      references give full details of the algorithm used here.  
      A good account of the use of SVD in least squares problems is given
      in {\it Numerical Recipes}\, (Press~{\it et~al.}\ 1986, Cambridge
      University Press), which includes another variant of the EISPACK code.}
%-----------------------------------------------------------------------
\routine{SLA\_SVDCOV}{Covariance Matrix from SVD}
{
 \action{From the {\bf W} and {\bf V} matrices from the SVD
         factorization of a matrix
         (as obtained from the sla\_SVD routine), obtain
         the covariance matrix.}
 \call{CALL sla\_SVDCOV (N, NP, NC, W, V, WORK, CVM)}
}
\args{GIVEN}
{
 \spec{N}{I}{$n$, the number of rows and columns in
             matrices {\bf W} and {\bf V}} \\
 \spec{NP}{I}{first dimension of array containing $n \times n$
              matrix {\bf V}} \\
 \spec{NC}{I}{first dimension of array CVM} \\
 \spec{W}{D(N)}{$n \times n$ diagonal matrix {\bf W}
                (diagonal elements only)} \\
 \spec{V}{D(NP,NP)}{array containing $n \times n$ orthogonal matrix {\bf V}}
}
\args{RETURNED}
{
 \spec{WORK}{D(N)}{workspace} \\
 \spec{CVM}{D(NC,NC)}{array to receive covariance matrix}
}
\aref{{\it Numerical Recipes}, section 14.3.}
%-----------------------------------------------------------------------
\routine{SLA\_SVDSOL}{Solution Vector from SVD}
{
 \action{From a given vector and the SVD of a matrix (as obtained from
         the sla\_SVD routine), obtain the solution vector.
         This routine solves the equation:
         \begin{tabbing}
         XXXXXX \= \kill
         \> {\bf A} $\cdot$ {\bf x} = {\bf b}
         \end{tabbing}
         where:
         \begin{tabbing}
         XXXXXX \= XXXX \= \kill
         \> {\bf A} \> is a given $m$ (rows) $\times n$ (columns)
                       matrix, where $m \geq n$ \\
         \> {\bf x} \> is the $n$-vector we wish to find, and \\
         \> {\bf b} \> is a given $m$-vector
         \end{tabbing}
         by means of the {\it Singular Value Decomposition}\, method (SVD).}
 \call{CALL sla\_SVDSOL (M, N, MP, NP, B, U, W, V, WORK, X)}
}
\args{GIVEN}
{
 \spec{M,N}{I}{$m$, $n$, the numbers of rows and columns in matrix {\bf A}} \\
 \spec{MP,NP}{I}{physical dimensions of array containing matrix {\bf A}} \\
 \spec{B}{D(M)}{known vector {bf b}} \\
 \spec{U}{D(MP,NP)}{array containing $m \times n$ matrix {\bf U}} \\
 \spec{W}{D(N)}{$n \times n$ diagonal matrix {\bf W}
                (diagonal elements only)} \\
 \spec{V}{D(NP,NP)}{array containing $n \times n$ orthogonal matrix {\bf V}}
}
\args{RETURNED}
{
 \spec{WORK}{D(N)}{workspace} \\
 \spec{X}{D(N)}{unknown vector {\bf x}}
}
\notes
{
 \begin{enumerate}
  \item In the Singular Value Decomposition method (SVD),
        the matrix {\bf A} is first factorized (for example by
        the routine sla\_SVD) into the following components:
        \begin{tabbing}
        XXXXXX \= \kill
        \> {\bf A} = {\bf U} $\cdot$ {\bf W} $\cdot$ {\bf V}$^{T}$
        \end{tabbing}
        where:
        \begin{tabbing}
        XXXXXX \= XXXX \= \kill
        \> {\bf A} \> is any $m$ (rows) $\times n$ (columns) matrix,
                      where $m > n$ \\
        \> {\bf U} \> is an $m \times n$ column-orthogonal matrix \\
        \> {\bf W} \> is an $n \times n$ diagonal matrix with
                      $w_{ii} \geq 0$ \\
        \> {\bf V}$^{T}$ \> is the transpose of an $n \times n$
                            orthogonal matrix
        \end{tabbing}
        Note that $m$ and $n$ are the {\it logical}\, dimensions of the
        matrices and vectors concerned, which can be located in
        arrays of larger {\it physical}\, dimensions MP and NP.
        The solution is then found from the expression:
        \begin{tabbing}
        XXXXXX \= \kill
        \> {\bf x} = {\bf V} $\cdot [diag(1/${\bf W}$_{j})]
           \cdot (${\bf U}$^{T} \cdot${\bf b})
        \end{tabbing}
  \item If matrix {\bf A} is square, and if the diagonal matrix {\bf W} is not
        altered, the method is equivalent to conventional solution
        of simultaneous equations.
  \item If $m > n$, the result is a least-squares fit.
  \item If the solution is poorly determined, this shows up in the
        SVD factorization as very small or zero {\bf W}$_{j}$ values.  Where
        a {\bf W}$_{j}$ value is small but non-zero it can be set to zero to
        avoid ill effects.  The present routine detects such zero
        {\bf W}$_{j}$ values and produces a sensible solution, with highly
        correlated terms kept under control rather than being allowed
        to elope to infinity, and with meaningful values for the
       other terms.
 \end{enumerate}
}
\aref{{\it Numerical Recipes}, section 2.9.}
%-----------------------------------------------------------------------
\routine{SLA\_TP2S}{Tangent Plane to Spherical}
{
 \action{Transform tangent plane coordinates into spherical
         coordinates (single precision)}
 \call{CALL sla\_TP2S (XI, ETA, RAZ, DECZ, RA, DEC)}
}
\args{GIVEN}
{
 \spec{XI,ETA}{R}{tangent plane rectangular coordinates (radians)} \\
 \spec{RAZ,DECZ}{R}{spherical coordinates of tangent point (radians)}
}
\args{RETURNED}
{
 \spec{RA,DEC}{R}{spherical coordinates (radians)}
}
\anote{The tangent plane projection is called the {\it gnomonic}\,
       projection;  the \xy\ coordinates are called {\it standard coordinates}.
       The latter are in units of the distance from the tangent plane to the
       projection point, {\it i.e.}\ radians near the origin.}
%-----------------------------------------------------------------------
\routine{SLA\_UNPCD}{Remove Radial Distortion}
{
 \action{Remove pincushion/barrel distortion from a distorted \xy\
         to give tangent-plane \xy.}
 \call{CALL sla\_UNPCD (DISCO,X,Y)}
}
\args{GIVEN}
{
 \spec{DISCO}{D}{pincushion/barrel distortion coefficient} \\
 \spec{X,Y}{D}{distorted \xy}
}
\args{RETURNED}
{
 \spec{X,Y}{D}{tangent-plane \xy}
}
\notes
{
 \begin{enumerate}
  \item The distortion is of the form $\rho = r (1 + c r^{2})$, where $r$ is
        the radial distance from the tangent point, $c$ is the DISCO
        argument, and $\rho$ is the radial distance in the presence of
        the distortion.
  \item For {\it pincushion}\, distortion, C is +ve;  for
        {\it barrel}\, distortion, C is $-$ve.
  \item For X,Y in units of one projection radius (in the case of
        a photographic plate, the focal length), the following
        DISCO values apply:

        \vspace{2ex}

        \hspace{5em}
        \begin{tabular}{|l|c|} \hline
         Geometry & DISCO \\ \hline \hline
         astrograph & 0.0 \\ \hline
         Schmidt & $-$0.3333 \\ \hline
         AAT PF doublet & +147.069 \\ \hline
         AAT PF triplet & +178.585 \\ \hline
         AAT f/8 & +21.20 \\ \hline
         JKT f/8 & +14.6 \\ \hline
        \end{tabular}

        \vspace{2ex}

  \item The present routine is an approximate inverse to the
        companion routine sla\_PCD, obtained from two iterations
        of Newton's method.  The mismatch between the sla\_PCD
        and sla\_UNPCD is negligible for astrometric applications;
        to reach 1~milliarcsec at the edge of the AAT triplet or
        Schmidt field would require field diameters of $2.4^{\circ}$
        and $42^{\circ}$ respectively.
 \end{enumerate}
}
%-----------------------------------------------------------------------
\routine{SLA\_VDV}{Scalar Product}
{
 \action{Scalar product of two 3-vectors (single precision).}
 \call{R~=~sla\_VDV (VA, VB)}
}
\args{GIVEN}
{
 \spec{VA}{R(3)}{first vector} \\
 \spec{VB}{R(3)}{second vector}
}
\args{RETURNED}
{
 \spec{sla\_VDV}{R}{scalar product VA.VB}
}
%-----------------------------------------------------------------------
\routine{SLA\_VN}{Normalize Vector}
{
 \action{Normalize a 3-vector, also giving the modulus (single precision).}
 \call{CALL sla\_VN (V, UV, VM)}
}
\args{GIVEN}
{
 \spec{V}{R(3)}{vector}
}
\args{RETURNED}
{
 \spec{UV}{R(3)}{unit vector in direction of V} \\
 \spec{VM}{R}{modulus of V}
}
\anote{If the modulus of V is zero, UV is set to zero as well.}
%-----------------------------------------------------------------------
\routine{SLA\_VXV}{Vector Product}
{
 \action{Vector product of two 3-vectors (single precision).}
 \call{CALL sla\_VXV (VA, VB, VC)}
}
\args{GIVEN}
{
 \spec{VA}{R(3)}{first vector} \\
 \spec{VB}{R(3)}{second vector}
}
\args{RETURNED}
{
 \spec{VC}{R(3)}{vector product VA$\times$VB}
}
%-----------------------------------------------------------------------
\routine{SLA\_WAIT}{Time Delay}
{
 \action{Wait for a specified interval.}
 \call{CALL sla\_WAIT (DELAY)}
}
\args{GIVEN}
{
 \spec{DELAY}{R}{delay in seconds}
}
\notes
{
 \begin{enumerate}
  \item The implementation is {\bf VAX-specific}.
  \item The delay actually requested is restricted to the range
        100ns-200s in the present implementation.
  \item There is no guarantee of accuracy, though on the VAX (as
        on many other computers) the program will certainly not
        resume execution {\it before}\, the stated interval has
        elapsed.
 \end{enumerate}
}
%-----------------------------------------------------------------------
\routine{SLA\_XY2XY}{Apply Linear Model to an \xy}
{
 \action{Transform one \xy\ into another using a linear model of the type
         produced by the sla\_FITXY routine.}
 \call{CALL sla\_XY2XY (X1,Y1,COEFFS,X2,Y2)}
}
\args{GIVEN}
{
 \spec{X1,Y1}{D}{\xy\ before transformation}
}
\args{RETURNED}
{
 \spec{COEFFS}{D(6)}{transformation coefficients (see note)} \\
 \spec{X2,Y2}{D}{\xy\ after transformation}
}
\notes
{
 \begin{enumerate}
  \item The model relates two sets of \xy\ coordinates as follows.
        Naming the six elements of COEFFS $a,b,c,d,e$ \& $f$,
        the present routine performs the transformation:
        \begin{verse}
          $x_{2} = a + bx_{1} + cy_{1}$ \\
          $y_{2} = d + ex_{1} + fy_{1}$
        \end{verse}
  \item See also sla\_FITXY, sla\_PXY, sla\_INVF, sla\_DCMPF.
 \end{enumerate}
}
%-----------------------------------------------------------------------
\routine{SLA\_ZD}{$h,\delta$ to Zenith Distance}
{
 \action{Hour angle and declination to zenith distance
         (double precision).}
 \call{D~=~sla\_ZD (HA, DEC, PHI)}
}
\args{GIVEN}
{
 \spec{HA}{D}{hour angle in radians} \\
 \spec{HA}{D}{declination in radians} \\
 \spec{HA}{D}{latitude in radians}
}
\args{RETURNED}
{
 \spec{sla\_ZD}{D}{zenith distance (radians, $0\!-\!\pi$)}
}
\notes
{
 \begin{enumerate}
  \item The latitude must be geodetic.  In critical applications,
        corrections for polar motion should be applied.
  \item In some applications it will be important to specify the
        correct type of hour angle and declination in order to
        produce the required type
        of zenith distance.  In particular, it may be
        important to distinguish between the zenith distance
        as affected by refraction, which would require the
        {\it observed}\, \hadec, and the zenith distance {\it in vacuo},
        which would require the {\it topocentric}\, \hadec.  If
        the effects of diurnal aberration can be neglected, the
        {\it apparent}\, \hadec\ may be used instead of the
        {\it topocentric}\, \hadec.
  \item No range checking of arguments is done.
  \item In applications which involve many zenith distance calculations,
        rather than calling the present routine it will be more
        efficient to use inline code, having previously computed fixed
        terms such as sine and cosine of latitude, and perhaps sine and
        cosine of declination.
 \end{enumerate}
}
\end{document}
