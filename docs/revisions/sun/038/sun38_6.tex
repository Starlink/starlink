\documentstyle[11pt]{article} 
\pagestyle{myheadings}

%------------------------------------------------------------------------------
\newcommand{\stardoccategory}  {Starlink User Note}
\newcommand{\stardocinitials}  {SUN}
\newcommand{\stardocnumber}    {38.6}
\newcommand{\stardocauthors}   {P M Allan \& M J Bly}
\newcommand{\stardocdate}      {20 April 1993}
\newcommand{\stardoctitle}     {DOCFIND --- Starlink document search}
%------------------------------------------------------------------------------

\newcommand{\stardocname}{\stardocinitials /\stardocnumber}
\renewcommand{\_}{{\tt\char'137}}     % re-centres the underscore
\markright{\stardocname}
\setlength{\textwidth}{160mm}
\setlength{\textheight}{230mm}
\setlength{\topmargin}{-2mm}
\setlength{\oddsidemargin}{0mm}
\setlength{\evensidemargin}{0mm}
\setlength{\parindent}{0mm}
\setlength{\parskip}{\medskipamount}
\setlength{\unitlength}{1mm}

%------------------------------------------------------------------------------
% Add any \newcommand or \newenvironment commands here
%------------------------------------------------------------------------------

\begin{document}
\thispagestyle{empty}
SCIENCE \& ENGINEERING RESEARCH COUNCIL \hfill \stardocname\\
RUTHERFORD APPLETON LABORATORY\\
{\large\bf Starlink Project\\}
{\large\bf \stardoccategory\ \stardocnumber}
\begin{flushright}
\stardocauthors\\
\stardocdate
\end{flushright}
\vspace{-4mm}
\rule{\textwidth}{0.5mm}
\vspace{5mm}
\begin{center}
{\Large\bf \stardoctitle}
\end{center}
\vspace{5mm}

%------------------------------------------------------------------------------
%  Add this part if you want a table of contents
%  \setlength{\parskip}{0mm}
%  \tableofcontents
%  \setlength{\parskip}{\medskipamount}
%  \markright{\stardocname}
%------------------------------------------------------------------------------

\section{Introduction}

One of the recurrent problems with Starlink is that as the volume of software
grows, so does the number of documents describing it. These include project
wide documents and documents which are local to other sites. It is often
extremely difficult to find out which document should be consulted about a
particular topic.

Starlink maintains a list of currently valid project wide documents called
`STARLINK DOCUMENTATION' in the file DOCSDIR:DOCS.LIS. This may be printed if
required. It is maintained at RAL by the Starlink Software Librarian and
updated at other sites by the Starlink software update process. The current
date is stated at the beginning of the file so you can see if your site is up
to date by comparing your site's file with the RAL file.

Program FIND has been developed to help users search this mass of
documentation. An alternative approach is to consult file DOCSDIR:SUBJECT.LIS.
This is a Key-Word index to Starlink documentation which does not rely only on
document titles. Once again, it is maintained up to date at RAL but other sites
may lag behind. The specified current date will tell all. DOCSDIR:ANALYSIS.LIS
is also centrally maintained. This is a list of the Starlink documents listed
by which Software item is the main subject of each document.

\section{Use}

DOCFIND is a procedure that searches the file DOCSDIR:SUBJECT.LIS for a given
key word and then runs the program FIND to search the file DOCSDIR:DOCS.LIS for
the same key word in the titles of the Starlink documents. This technique is
used since not all key words will appear in the titles of the documents.

To invoke DOCFIND simply type:
\begin{verbatim}
      $ DOCFIND [word]
\end{verbatim}
and answer the prompts. The parameter is optional and will be prompted for if
not present on the command line. Its value is the character string which is to
be sought within the subjects file and all the titles of documents in the
Index. If a null value is entered ({\it i.e.} \verb+<CR>+ only) then the
subjects file is not searched, but ALL the documents in the index are 
displayed. In either case, output to the terminal is held up at the end of 
each page, until \verb+<CR>+ is hit.

Having listed the relevant documents, it then prompts for the name of a
particular document:

\begin{itemize}
\item First it prompts for the document's number. If a null response is given,
the program stops. The generation number may be entered, but will be ignored;
the latest version will always be found.
\item If a valid number is found, it asks for the document type (SUN,LGP etc).
\item If the first character is an L, denoting a `local' document, it asks for
the relevant site name. The default is RAL, but your site manager may have
changed this to the abbreviation for your local site.
\end{itemize}

Assuming that a valid document number has been entered and a copy of it is
available on-line at this site, that document is displayed at the terminal.
After each page, the user may carry on to the next page by hitting \verb+<CR>+,
or stop by typing YES (or Y or T or TRUE). Finally, the user is offered the
opportunity of having a printed copy of the selected document.

\subsection{Use with \LaTeX files}

Originally, Starlink documentation was written using either the Digital RUNOFF
program, or the GEROFF program from RAL. Both of the programs are essentially
just text layout programs and produce a file that can be listed on an
alphanumeric terminal. More recently, however, most of the documentation has
been converted to \LaTeX format. This produces a vastly superior hard copy
product, but has the disadvantage that it is not easy to display on a
terminal. If a particular document is no longer available in RUNOFF or GEROFF
format, the DOCFIND program will not be able to find the formatted file to list.
Instead it will list the \LaTeX source file. This has many control sequences
that make the listing a bit indigestible, but if you can write documents using
\LaTeX you should be able to read most of it. It is not intended that you should
actually read long documents in this way; you should ask your site manager for
a hard copy version of the document. Where the \LaTeX listing is useful is as a
memory jogger. Perhaps you have forgotten a particular logical name and you
know it is in the SUN. The \LaTeX source should be sufficiently legible for you
to be able to find it.

\section{Notes}

Ideally, each site should obtain copies of all documents from all sites as soon
as they appear in the standard directories at any site. However, if for any
reason the one you require is not available, please consult your site manager.

Remember that because a document is not in the Index does not mean that it does
not exist, nor that there is no copy at your node.

The site manager may well have spare copies of the document sought, so you
should consult him before asking for a printed copy of what might be a very
long document. It is even possible that you have filed a copy of it yourself!
The number of pages in a document is specified in the index after the document
title. Papers which include diagrams (not stored on-line) are indicated in the
index by the key `D' in front of the title.

\end{document}
