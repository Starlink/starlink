\documentstyle[11pt]{article} 
\pagestyle{myheadings}

%------------------------------------------------------------------------------
\newcommand{\stardoccategory}  {Starlink User Note}
\newcommand{\stardocinitials}  {SUN}
\newcommand{\stardocnumber}    {88.4}
\newcommand{\stardocauthors}   {P C T Rees \\
                                D L Terrett}
\newcommand{\stardocdate}      {26 May 1992}
\newcommand{\stardoctitle}     {NCAR --- Graphics Utilities}
%------------------------------------------------------------------------------

\newcommand{\stardocname}{\stardocinitials /\stardocnumber}
\renewcommand{\_}{{\tt\char'137}}     % re-centres the underscore
\markright{\stardocname}
\setlength{\textwidth}{160mm}
\setlength{\textheight}{230mm}
\setlength{\topmargin}{-2mm}
\setlength{\oddsidemargin}{0mm}
\setlength{\evensidemargin}{0mm}
\setlength{\parindent}{0mm}
\setlength{\parskip}{\medskipamount}
\setlength{\unitlength}{1mm}

%------------------------------------------------------------------------------
% Add any \newcommand or \newenvironment commands here
%------------------------------------------------------------------------------

\begin{document}
\thispagestyle{empty}
SCIENCE \& ENGINEERING RESEARCH COUNCIL \hfill \stardocname\\
RUTHERFORD APPLETON LABORATORY\\
{\large\bf Starlink Project\\}
{\large\bf \stardoccategory\ \stardocnumber}
\begin{flushright}
\stardocauthors\\
\stardocdate
\end{flushright}
\vspace{-4mm}
\rule{\textwidth}{0.5mm}
\vspace{5mm}
\begin{center}
{\Large\bf \stardoctitle}
\end{center}
\vspace{5mm}

\begin {picture}(160,160)
\put(0,160){\special{include sun88.ps}}
\end {picture}

\newpage

%------------------------------------------------------------------------------
%  Add this part if you want a table of contents
  \setlength{\parskip}{0mm}
  \tableofcontents
  \setlength{\parskip}{\medskipamount}
  \markright{\stardocname}
%------------------------------------------------------------------------------

\newpage

\section{Introduction}

The National Center for Atmospheric Research in Boulder, Colorado has, for
many years, distributed a extensive suite of high level graphics
utilities.
When NCAR converted these utilities to run over the ISO standard GKS, Starlink
was used as a test site.
Starlink was subsequently given permission to release the NCAR utilities as
part of the Starlink Software Collection.
The NCAR utilities offer an alternative high level graphics package to PGPLOT 
(SUN/15) for use within Starlink applications.
The NCAR package is arguably more powerful than PGPLOT in functionality and 
flexibility, but it does require a little more time to master.
Starlink provides a set of extensions to the NCAR graphics utilities,
SNX (SUN/90), to make the use of NCAR more accessible to the beginner.
For lower level work, SGS (SUN/85) or GKS (SUN/83) should be used.

The utilities provided by NCAR are:

\begin{quote}
\begin{description}
\item [AUTOGRAPH] ---
 Draws and annotates curves or families of curves.
\item [CONRAN] --- 
 Contours irregularly spaced data labelling the contour lines.
\item [CONRAQ] --- 
 Like CONRAN, but faster because it has no labelling capability.
\item [CONRAS] --- 
 Like CONRAN, but slower because lines are smoothed and crowded lines are
 removed.
\item [CONREC] --- 
 Contours two dimensional arrays, labelling the contour lines.
\item [CONRECQCK] --- 
 Like CONREC, but faster because it has no labelling capability.
\item [CONRECSUPR] --- 
 Like CONREC, but slower because lines are smoothed and crowded lines are
 removed.
\item [DASHCHAR] --- 
 Software dashed line package with labelling capability.
\item [DASHLINE] --- 
 Like DASHCHAR, but faster because it has no labelling capability.
\item [DASHSMTH] --- 
 Like DASHCHAR, but slower because lines are smoothed.
\item [DASHSUPR] --- 
 Like DASHCHAR, but slower because lines are smoothed and crowded lines are
 removed.
\item [EZMAP] --- 
 Plots continental and/or national and US state boundaries in one of nine map
 projections.
\item [GRIDAL] --- 
 Package for drawing graph paper, axis, {\em etc.}
\item [HAFTON] --- 
 Draws halftone (greyscale) picture from a two dimensional array.
 ({\em N.B.}\ The GKS cell array function GCA should be used on devices with a
 true greyscale capability.)
\item [HSTGRM] --- 
 Plots histograms.
\item [ISOSRF] --- 
 Plots iso-values surfaces (with hidden lines removed) from a three
 dimensional array.
\item [ISOSRFHR] --- 
 Plots iso-values surfaces (with hidden lines removed) from a high resolution
 three dimensional array.
\item [PWRITX] --- 
 Plots high quality software characters.                      
\item [PWRITY] --- 
 Plots simple software characters.
\item [PWRZI] --- 
 Draws characters in three-space, for use with the ISOSRF utility.
\item [PWRZS] --- 
 Draws characters in three-space, for use with the SRFACE utility.
\item [PWRZT] --- 
 Draws characters in three-space, for use with the THREED utility.
\item [SRFACE] --- 
 Plots a three dimensional display of a surface (with hidden lines removed)
 from a two dimensional array.
\item [STRMLN] --- 
 Plots a representation of a vector flow of any field for which planar vector
 components are given on a regular rectangular lattice.
 \footnote{Note: because of portability problems this utility is not available
 on UNIX.}
\item [THREED] --- 
 Provides a three space line drawing capability.
\item [VELVCT] --- 
 Draws a two dimensional velocity field by drawing arrows from the data
 locations.
\end{description}
\end{quote}

The NCAR graphics utilities are released exactly as supplied by NCAR except
that the default behaviour of AUTOGRAPH and HSTGRM has been changed to not
clear the screen after each plot --- this is more appropriate for an
interactive environment. 

Because of a name conflict with a Fortran run-time library  routine on UNIX,
the NCAR routine FLUSH has been removed from the UNIX  release of the NCAR
library.
In order to get the same behaviour as a call to FLUSH when using NCAR, 
the call 

\begin{verbatim}
       CALL PLOTIT( 0, 0, 2 )
\end{verbatim}

may be used (see the SPPS description in the NCAR manual).


\section{Documentation}

A manual containing extensive descriptions of all the NCAR routines
is available as a Starlink Miscelaneous User Document (MUD) and can be 
obtained from your Starlink site manager. 


\section{Using the Utilities}

Before calling any of the utilities, GKS must be open and at least one
workstation open and active. 
This can be done with the following sequence of GKS calls:

\begin{verbatim}
      CALL GOPKS( LUERR )
      CALL GOPWK( IWKID, ICONID, IWTYPE )
      CALL GACWK( IWKID )
\end{verbatim}

where LUERR is a Fortran logical unit to which messages resulting from
GKS internal errors may be written (unit 6 is recommended on all operating
systems used by Starlink), IWKID is a workstation identifier (any integer you
care to choose) and IWTYPE and ICONID are the workstation type and connection
identifier respectively.

At the time of writing, the version of RAL GKS available on VAX/VMS
is 7.2, whereas the version of RAL GKS available on UNIX is the ISO standard
version 7.4. 
There are some minor differences between these two versions of GKS, mostly 
in the provision of additional arguments for certain routines.
As a result, the calling sequence of the routine GOPKS on UNIX has an extra
argument (the value of this argument is not used by RAL GKS
and it is recommended to be set to -1): {\em e.g.} 

\begin{verbatim}
      CALL GOPKS( LUERR, -1 )
\end{verbatim}

A more friendly user interface can be provided if, instead of asking the
user for a GKS workstation type and connection identifier, a GNS workstation
name is used ({\em i.e.} as used by SGS).
A GNS name can be translated to its GKS equivalent by the GNS (SUN/57) call

\begin{verbatim}
      CALL GNS_TNG( NAME, IWKTYP, ICONID, STATUS )
\end{verbatim}

where STATUS is the inherited status (returned SAI\_\_OK if the translation is 
successful). 
More information on workstation names can be found in SUN/57.

If you wish to do any plotting with SGS routines, SGS\_OPEN must be used to
open GKS and the workstation.
SGS makes the entire display surface of a workstation available for plotting,
which in general means that part of the normalized device co-ordinate unit
square will not be visible.
Since the default behaviour of the NCAR utilities is to use the whole of the
NDC unit square, using them becomes a little more complicated.
This is discussed further in the section on co-ordinate systems
({\em i.e.} \S\ref{co-ord_sect}).

When plotting is complete, GKS must be closed down with either:

\begin{verbatim}
      CALL GDAWK( IWKID )
      CALL GCLWK( IWKID )
      CALL GCLKS
\end{verbatim}

or

\begin{verbatim}
      CALL SGS_CLOSE
\end{verbatim}

as appropriate.


\section{Co-ordinate Systems} \label{co-ord_sect}

When two graphics packages are used from the same program, one package may
interfere with the correct operation of the other. 
When mixing the NCAR
utilities and SGS, both packages manipulate the GKS transformations and
so precautions must be taken to avoid interference.

The NCAR documentation refers to ``fractional co-ordinates'' which in
GKS terms are normalized device co-ordinates (NDC). 
To plot in NCAR fractional co-ordinates with GKS calls, all that is 
necessary is to select the GKS normalization transformation 0 using the 
routine GSELNT. 
To do locator or stroke input in fractional co-ordinates, the GKS normalization 
transformation 0 must be made the highest priority using the routine 
GSVPIP.

With the exception of the utilities that plot three-dimensional objects, a 
two-dimensional ``user co-ordinate'' system (which may be logarithmic in one or
both dimensions) is defined. 
A set of functions is provided by NCAR to convert points between user
co-ordinates and fractional co-ordinates.
These functions are:

\begin{quote}
\begin{description}
\item [CFUX( X )] ---
 Returns the X user co-ordinate corresponding to the fractional co-ordinate X.
\item [CFUY( Y )] ---
 Returns the Y user co-ordinate corresponding to the fractional co-ordinate Y.
\item [CUFX( X )] ---
 Returns the X fractional co-ordinate corresponding to the user co-ordinate X.
\item [CUFY( Y )] ---
 Returns the Y fractional co-ordinate corresponding to the user co-ordinate Y.
\end{description}
\end{quote}

The use of these functions is illustrated in the example program in Appendix 
\ref{exam_sect}.
These functions are only valid while the GKS transformations set up by the NCAR
utilities are still current.
When mixing NCAR calls with SGS or GKS calls it may be necessary to save the
normalization transformation immediately after the NCAR call (using the GKS
routine GQNT) and restore it before using NCAR again (using the GKS routines 
GSWN and GSVP).
Note that calling any SGS zone selection routine will always correctly restore
the state of SGS.

If you use SGS\_OPEN to open the GKS workstation, the entire NDC unit square
will not usually be mapped onto the display surface. 
This is an inescapable consequence of making the whole display surface
available for plotting and means that the area of NDC (or fractional
co-ordinates) used by the utilities must be changed from the default.
The method described here demonstrates how to make the utilities plot in the
current zone. 
When the current zone is the base zone, this is equivalent to plotting on the
whole workstation.

The first step is to inquire the NDC limits of the current zone (SGS always
uses normalization transformation number 1):

\begin{verbatim}
      REAL VIEWP( 4 )
      REAL WIND( 4 )

      CALL GQNT( 1, IERR, WIND, VIEWP )
\end{verbatim}

The zone limits in NDC are now stored in the array VIEWP. 
The next step depends on the utility being used (in future releases a more
uniform mechanism may be provided):

\begin{quote}
\begin{description}
\item [AUTOGRAPH] ---
\begin{verbatim}
      CALL AGSETP( 'GRAPH.', VIEWP, 4 )
\end{verbatim}

\item [CONRAN,CONRAS,CONRAQ] ---
\begin{verbatim}
      CALL CONOP1( 'SCA=PRI' )
\end{verbatim}

\item [EZMAP] ---
\begin{verbatim}
      CALL MAPPOS( VIEWP( 1 ), VIEWP( 2 ), VIEWP( 3 ), VIEWP( 4 ) )
\end{verbatim}

\item [HSTGRM] ---
\begin{verbatim}
      CALL HSTOPR( 'WIN', VIEWP, 4 )
\end{verbatim}

\item [THREED] ---
\begin{verbatim}
      CALL SET3( VIEWP( 1 ), VIEWP( 2 ), VIEWP( 3 ), VIEWP( 4 ), ...
\end{verbatim}
\end{description}
\end{quote}

The plotting area used by HAFTON, STRMLN and the CONREC family is controlled
by variables in common blocks (see the individual routine documentation for
details).
ISOSRF, ISOSRFHR, SRFACE and VELVCT always use the entire unit
square.


\section{Linking}

All the routines in the NCAR library have standard Fortran six character
names, so you must beware of your own routines having the same names as these
library routines.
If this happens, the linker will not report any error and
the subsequent aberrant behaviour of your program will not point
unambiguously to the source of the problem.
The names of all the routines
in the library are listed in Appendix \ref{list_sect}.


\subsection{Use with the VAX/VMS operating system}

Non-ADAM programs may be linked with the NCAR utilities by

\begin{quote}
\begin{verbatim}
$ LINK program, NCAR_DIR:NCAR_LINK/OPT, STAR_LINK/OPT
\end{verbatim}
\end{quote}

This command line will also link with SGS, if it is being used.

By default, the DASHCHAR dashed line drawing package is used. 
If one of the other line drawing packages (DASHLINE, DASHSMTH or DASHSUPR) 
is required, it must be included in the link command explicitly: {\em e.g.}

\begin{quote}
\begin{verbatim}
$ LINK program, NCAR_DIR:DASHSMTH, NCAR_DIR:NCAR_LINK/OPT, STAR_LINK/OPT
\end{verbatim}
\end{quote}

will use the DASHSMTH package.
Similarly, the default version of the CONREC family of contour routines is
CONREC which can be replaced by CONRECQCK or CONRECSUPR in the same way.

Compiling and linking ADAM applications is discussed in Appendix
\ref{adam_link}.


\subsection{Use with the UNIX operating system}

\begin{sloppypar}
Assuming all Starlink directories have been added to the environment variables
{\bf PATH} and {\bf LD\_LIBRARY\_PATH} (see SUN/118), then to link a non-ADAM
program with NCAR the command line would be
\end{sloppypar}

\begin {quote}
\begin {small}
\begin{verbatim}
% f77 program.o -L/star/lib `ncar_link` -o program.out
\end{verbatim}
\end {small}
\end {quote}

As described for the VAX/VMS operating system, the DASHCHAR dashed line drawing
package is used by default.
If one of the other line drawing packages (DASHLINE, DASHSMTH or DASHSUPR which
all reside in the directory {\bf /star/lib}) 
is required, it must be included in the link command explicitly: {\em e.g.}

\begin{quote}
\begin{verbatim}
% f77 program.o /star/lib/dashsmth.o -L/star/lib `ncar_link` \
       -o program.out
\end{verbatim}
\end{quote}

will use the DASHSMTH package.
Similarly, the default version of the CONREC family of contour routines is
CONREC which can be replaced by CONRECQCK or CONRECSUPR in the same way.

Compiling and linking ADAM applications is discussed in Appendix
\ref{adam_link}.


\section{Demonstration Programs}

A demonstration program exists for each of the NCAR utilities.
On VAX/VMS these reside in the text library 
STARDISK:[STARLINK.LIB.NCAR.TESTS]SOURCE.TLB and have the module names

\begin{quote}
\begin{tabbing}
TDASHLINxxxxx\=TDASHLINxxxxx\=TDASHLINxxxxx\=TDASHLINxxxxx\=\kill
TAUTGRPH\>TCNRCQCK\>TCNRCSMT\>TCNRCSPR\>TCONRAN\\
TCONRAQ\>TCONRAS\>TCONREC\>TDASHCHR\>TDASHLIN\\
TDASHSPR\>TDSHSMTH\>TEZMAP\>TGRIDAL\>THAFTON\\
THSTGRM\>TISOSRF\>TISSRFHR\>TPWRITX\>TPWRITY\\
TPWRZI\>TPWRZS\>TPWRZT\>TSRFACE\>TSTRMLN\\
TTHREED\>TVELVCT
\end{tabbing}
\end{quote}

Each example program may be extracted from the
library using the command

\begin{quote}
\begin{verbatim}
$ LIBRARY/EXTRACT=module_name/OUTPUT=file_name -
$_     STARDISK:[STARLINK.LIB.NCAR.TESTS]SOURCE/TEXT
\end{verbatim}
\end{quote}

which may then be compiled and linked in the usual way.
Note that some of the example programs require linking with non-default dashed
line and contour drawing routines in order to link without error --- the NCAR 
manual gives full details of these dependencies.

\begin{sloppypar}
On UNIX the example programs reside in the directory {\bf
/star/\-starlink/\-lib/\-ncar/\-examples}.
The example module names on UNIX are 
\end{sloppypar}

\begin{quote}
\begin{tabbing}
tdashlinXXXXX\=tdashlinXXXXX\=tdashlinXXXXX\=tdashlinXXXXX\=\kill
exampl\>tautog\>tcnqck\>tcnsmt\>tcnsup\\
tconan\>tconaq\>tconas\>tconre\>tdashc\\
tdashl\>tdashp\>tdashs\>tezmap\>tgrida\\
thafto\>thstgr\>tisohr\>tisosr\>tpwrtx\\
tpwry\>tpwrzi\>tpwrzs\>tpwrzt\>tsrfac\\
tthree\>tvelvc
\end{tabbing}
\end{quote}

This directory includes a makefile which can build each or all of the 
example programs, {\em e.g.}

\begin{quote}
\begin{verbatim}
% make -f /star/starlink/lib/ncar/examples/makefile exampl
\end{verbatim}
\end{quote}


\section{References}

SUN/15 ~~PGPLOT --- Graphics Subroutine Library.\\
SUN/57 ~~GNS --- Graphics Workstation Name Service.\\
SUN/83 ~~GKS --- Graphical Kernel Syatem (7.2).\\
SUN/85 ~~SGS --- Simple Graphics System.\\
SUN/90 ~~SNX --- Starlink Extensions to the NCAR Graphics Utilities.\\
SUN/118 ~~Starlink Software on UNIX.\\
SUN/144 ~~ADAM --- UNIX Version.\\


\appendix
\newpage


\section{Example Program} \label{exam_sect}

\small
\begin{verbatim}
      PROGRAM EXAMPL
*+
*   Simple program to illustrate the use of AUTOGRAPH with SGS:
*      a set of X,Y points is plotted with AUTOGRAPH and then the mean 
*      Y value marked with a horizontal line drawn with SGS.
*+

      INTEGER N
      PARAMETER( N = 10 )

      CHARACTER * 20 WS
      INTEGER I, IBASE, ISTAT
      REAL X( N ), Y( N ), WIND( 4 ), VIEWP( 4 )
      REAL TOTAL, AMEAN, ANDC, XST, XEN

      DATA X / 1, 2, 3, 4, 5, 6, 7, 8, 9, 10 /
      DATA Y / 1, 2, 4, 3, 5, 6, 4, 5, 6, 10 /

*  Get the workstation name.
      WRITE ( *, '( 1X, A )', ERR=999 ) 'Workstation'
      READ ( *, '( A )', ERR=999 ) WS

*  Open SGS.
      CALL SGS_OPEN( WS, IBASE, ISTAT )
      IF ( ISTAT .NE. 0 ) GO TO 999

*  Obtain the viewport limits and set the limits for AUTOGRAPH.
      CALL GQNT( 1, IERR, WIND, VIEWP )
      CALL AGSETP( 'GRAPH.', VIEWP, 4 )

*  Draw the graph.
      CALL EZXY( X, Y, N, 'Autograph example_$' )

*  Find the mean Y value.
      TOTAL = 0.0

      DO 10 I = 1, N
         TOTAL = TOTAL + Y( I )
 10   CONTINUE

      AMEAN = TOTAL / REAL( N )

*  Convert the end points of thhe line to fractional co-ordinates (i.e. NDC)
*  before restoring SGS state.
      ANDC = CUFY( AMEAN )
      XST = CUFX( 1.0 )
      XEN = CUFX( 10.0 )

*  Re-establish the SGS zone
      CALL SGS_SELZ( IBASE, ISTAT )

*  Set the world co-ordinates to match the NDC (i.e. fractional co-ordinates).
      CALL SGS_SW( VIEWP( 1 ), VIEWP( 2 ), VIEWP( 3 ), VIEWP( 4 ), 
     :             ISTAT )

*  Draw the line and close SGS.
      CALL SGS_LINE( XST, ANDC, XEN, ANDC )
      CALL SGS_CLOSE

 999  CONTINUE

      END
\end{verbatim}
\normalsize


\newpage

\section{NCAR Subroutine Names} \label{list_sect}

\scriptsize
\begin{quote}
\begin{tabbing}
AGAXISxxxxxx\=AGAXISxxxxxx\=AGAXISxxxxxx\=AGAXISxxxxxx\=AGAXISxxxxxx\=\kill
AGAXIS\>AGBACK\>AGBNCH\>AGCHAX\>AGCHCU\>AGCHIL\\
AGCHNL\>AGCTCS\>AGCTKO\>AGCURV\>AGDASH\>AGDFLT\\
AGDLCH\>AGDSHN\>AGEXAX\>AGEXUS\>AGEZSU\>AGFPBN\\
AGFTOL\>AGGETC\>AGGETF\>AGGETI\>AGGETP\>AGGTCH\\
AGINIT\>AGKURV\>AGLBLS\>AGMAXI\>AGMINI\>AGNUMB\\
AGPPID\>AGPWRT\>AGQURV\>AGRPCH\>AGRSTR\>AGSAVE\\
AGSCAN\>AGSETC\>AGSETF\>AGSETI\>AGSETP\>AGSRCH\\
AGSTCH\>AGSTUP\>AGUTOL\>ANOTAT\>BOUND\>CALCNT\\
CCHECK\>CFUX\>CFUY\>CFVLD\>CHKCYC\>CHSTR\\
CLGEN\>CLSET\>CLSGKS\>CMFX\>CMFY\>CMUX\\
CMUY\>CONBD\>CONBDN\>CONCAL\>CONCLD\>CONCLS\\
CONCOM\>CONDET\>CONDRW\>CONDSD\>CONECD\>CONGEN\\
CONINT\>CONLCM\>CONLIN\>CONLOC\>CONLOD\>CONOP1\\
CONOP2\>CONOP3\>CONOP4\>CONOT2\>CONOUT\>CONPDV\\
CONPMM\>CONPMS\>CONRAN\>CONRAQ\>CONRAS\>CONREC\\
CONREO\>CONSLD\>CONSSD\>CONSTP\>CONTLK\>CONTNG\\
CONTOR\>CONXCH\>CPFX\>CPFY\>CPUX\>CPUY\\
CTCELL\>CUFX\>CUFY\>CURVE\>CURVE3\>CURVED\\
DANDR\>DASHBD\>DASHDB\>DASHDC\>DCHECK\>DISPLA\\
DRAWI\>DRAWPV\>DRAWS\>DRAWT\>DRCNTR\>DRLINE\\
DRWSTR\>DRWVEC\>E9RIN\>ENCD\>ENTSR\>EPRIN\\
ERROF\>EXPAND\>EZCNTR\>EZHFTN\>EZISOS\>EZMXY\\
EZMY\>EZSRFC\>EZSTRM\>EZVEC\>EZXY\>EZY\\
FDUM\>FENCE3\>FILLIN\>FL2INT\>FLUSH\>FRAME\\
FRST3\>FRSTC\>FRSTD\>FRSTPT\>FRSTS\>GBYTES\\
GETHOL\>GETSET\>GETSI\>GETUSV\>GNEWPT\>GRAY
\end{tabbing}

\vspace {-6mm}

\begin{tabbing}
AGAXISxxxxxx\=AGAXISxxxxxx\=AGAXISxxxxxx\=AGAXISxxxxxx\=AGAXISxxxxxx\=\kill
GRID\>GRIDAL\>GRIDL\>GRIDT\>GTDGTS\>GTNUM\\
GTNUMB\>GTSIGN\>HAFTON\>HALFAX\>HFINIT\>HST1\\
HSTBKD\>HSTEXP\>HSTGRM\>HSTLST\>HSTMED\>HSTOPC\\
HSTOPL\>HSTOPR\>HSTSTR\>HTABLE\>I1MACH\>I8SAV\\
IDICTL\>IDIOT\>INIT3D\>INITZI\>INITZS\>INITZT\\
INTZI\>INTZS\>INTZT\>ISHIFT\>ISOSRB\>ISOSRF\\
KFMX\>KFMY\>KFPX\>KFPY\>KMPX\>KMPY\\
KPMX\>KPMY\>KUMX\>KUMY\>KUPX\>KUPY\\
KURV1S\>KURV2S\>LABMOD\>LASTD\>LINE\>LINE3\\
LINE3W\>LINED\>MAPBD\>MAPCEM\>MAPCHI\>MAPDRW\\
MAPEOS\>MAPFST\>MAPGRD\>MAPGTC\>MAPGTI\>MAPGTL\\
MAPGTR\>MAPINT\>MAPIO\>MAPIQ\>MAPIT\>MAPLBL\\
MAPLMB\>MAPLOT\>MAPPOS\>MAPROJ\>MAPRS\>MAPRST\\
MAPSAV\>MAPSET\>MAPSTC\>MAPSTI\>MAPSTL\>MAPSTR\\
MAPTRE\>MAPTRN\>MAPTRP\>MAPUSR\>MAPVEC\>MAPVP\\
MINMAX\>MKMSK\>MMASK\>MXMY\>NERRO\>OPNGKS\\
PERIM\>PERIM3\>PERIML\>PLOTIT\>POINT\>POINT3\\
POINTS\>PSYM3\>PWRIT\>PWRITX\>PWRITY\>PWRX\\
PWRXBD\>PWRY\>PWRYBD\>PWRYGT\>PWRYSO\>PWRZ\\
PWRZGI\>PWRZGS\>PWRZGT\>PWRZI\>PWRZOI\>PWRZOS\\
PWRZOT\>PWRZS\>PWRZT\>Q8QST4\>R1MACH\>REORD\\
RESET\>RETSR\>SBYTES\>SET\>SET3\>SET3D\\
SETER\>SETI\>SETR\>SETUSV\>SRFABD\>SRFACE\\
SRFGK\>STCNTR\>STLINE\>STRMLN\>SUPCON\>SUPMAP\\
THREBD\>TICK3\>TICK4\>TICK43\>TICKS\>TR32\\
TRN32I\>TRN32S\>TRN32T\>UERRBD\>UTILBD\>VECT3\\
VECTD\>VECTOR\>VELDAT\>VELVCT\>VELVEC\>WTSTR\\
XTCH\>ZEROSC\>ZLSET
\end{tabbing}
\end{quote}
\normalsize


\newpage

\section{Using NCAR within ADAM Applications} \label{adam_link}

\subsection{Use with the VAX/VMS operating system}

When developing ADAM applications, all logical names and symbols used 
during compiling and linking are defined by executing the procedures

\begin{quote}
\begin{verbatim}
$ ADAMSTART
$ ADAM_DEV
\end{verbatim}
\end{quote}

The application may then be compiled using the FORTRAN command and linked using
the ALINK and MLINK procedures, {\em e.g.}

\begin{quote}
\begin{verbatim}
$ ALINK application,NCAR_DIR:NCAR_LINK_ADAM/OPT
\end{verbatim}
\end{quote}


\subsection{Use with the UNIX operating system}

\begin{sloppypar}
Assuming all Starlink directories have been added to the environment variables
{\bf PATH} and {\bf LD\_LIBRARY\_PATH} (see SUN/118), then to link an ADAM
application with NCAR the command line would be
\end{sloppypar}

\begin{quote}
\begin{verbatim}
% alink application.o `ncar_link_adam`
\end{verbatim}
\end{quote}

SUN/144 gives further details of compiling and linking ADAM applications 
with the UNIX operating system.


\section {Portability}

\subsection {Overview}

This section discusses the portability of NCAR, including the coding 
standard adopted and a list of those Starlink packages which 
need to be ported to the target machine before a port of NCAR can proceed.


\subsection {Coding and porting prerequisites}

The standard of Fortran used for the coding of NCAR is totally
compliant with ANSI Fortran 77.
A small number of operating system-specific routines have also been written
in ANSI C.
To use NCAR on any computer system the operating system specific routines
must be modified to reflect the integer and floating point arithmetic used.

NCAR requires either GKS Vn. 7.2 or GKS Vn. 7.4 to be available.


\subsection {Operating system specific routines}

Several NCAR subroutines make use of operating system features which are
specific to the machine upon which they are implemented.
The names of these routines and their purpose are as follows:

\begin {quote}
\begin {description}
\item [I1MACH] ( I ) 
\subitem Define INTEGER machine dependent constants.
\indexspace
\item [R1MACH] ( I ) 
\subitem Define REAL machine dependent constants.
\indexspace
\item [NCAR.H] \hfill
\subitem C language versions of the following routines are also available
and may be used by a particular implementation: IAND, IOR, ISHIFT.
\footnote{The routines IAND and IOR
perform logical AND and OR operations between two given INTEGERS. 
They are  often provided by implementations of Fortran 77 as intrinsic
functions, but are not featured in the ANSI Standard.
C language versions are provided for use with Fortran implementations without
IAND and IOR intrinsic functions.}
These routines assume a Fortran/C language interface which is defined
within the C header file NCAR.H.
NCAR.H will require modification for use on new operating systems.
\end {description}
\end {quote}

\end{document}
