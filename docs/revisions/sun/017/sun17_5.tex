\documentstyle[11pt]{article} 
\pagestyle{myheadings}

%------------------------------------------------------------------------------
\newcommand{\stardoccategory}  {Starlink User Note}
\newcommand{\stardocinitials}  {SUN}
\newcommand{\stardocnumber}    {17.5}
\newcommand{\stardocauthors}   {J F Lightfoot \& R M Prestage}
\newcommand{\stardocdate}      {22 September 1992}
\newcommand{\stardoctitle}     {SPECX --- A Millimetre Wave Spectral Reduction 
				Package \\[2ex] 
				-- Version 6.2 --}
%------------------------------------------------------------------------------

\newcommand{\stardocname}{\stardocinitials /\stardocnumber}
\renewcommand{\_}{{\tt\char'137}}     % re-centres the underscore
\markright{\stardocname}
\setlength{\textwidth}{160mm}
\setlength{\textheight}{230mm}
\setlength{\topmargin}{-2mm}
\setlength{\oddsidemargin}{0mm}
\setlength{\evensidemargin}{0mm}
\setlength{\parindent}{0mm}
\setlength{\parskip}{\medskipamount}
\setlength{\unitlength}{1mm}

%------------------------------------------------------------------------------
% Add any \newcommand or \newenvironment commands here
%------------------------------------------------------------------------------

\begin{document}
\thispagestyle{empty}
SCIENCE \& ENGINEERING RESEARCH COUNCIL \hfill \stardocname\\
RUTHERFORD APPLETON LABORATORY\\
{\large\bf Starlink Project\\}
{\large\bf \stardoccategory\ \stardocnumber}
\begin{flushright}
\stardocauthors\\
\stardocdate
\end{flushright}
\vspace{-4mm}
\rule{\textwidth}{0.5mm}
\vspace{5mm}
\begin{center}
{\Large\bf \stardoctitle}
\end{center}
\vspace{5mm}

%------------------------------------------------------------------------------
%  Add this part if you want a table of contents
%  \setlength{\parskip}{0mm}
%  \tableofcontents
%  \setlength{\parskip}{\medskipamount}
%  \markright{\stardocname}
%------------------------------------------------------------------------------

\section {Introduction}

SPECX is a general mm and sub-mm wavelength data reduction system
written by Rachael Padman at MRAO. Although it may be used to
process spectra obtained from many different instruments,
Starlink users will find it particularly useful for the reduction
of data obtained with the James Clerk Maxwell Telescope. 
SPECX has it's own command line interpreter, and manipulates 
spectra on a pop-down stack. The package is designed to work on
a graphics terminal with dual-screen facility and VT emulation 
({\em e.g.} a Pericom MG100), but can be usefully run
on other graphics terminals. Hardcopy may be output on any printer
for which a GKS driver is provided by Starlink.

Some of the major features of SPECX are:
\begin{itemize}
\item
The ability to process up to eight spectra (quadrants) simultaneously:
these may have the same or different centre frequencies, resolutions, 
etc. This allows users to operate on different filter-banks, or
dual polarization data for example, in parallel. 
\item
The ability to automatically save the current status of the system 
after each command is executed. This means that if SPECX is
stopped (unintentionally or otherwise), when restarted it will
`remember' the previous state of the program.
\item
Commands for the listing and display of spectra on a graphics
terminal, with hardcopy on a variety of printers.
\item
Single and multiple scan arithmetic, scan averaging, {\em etc.}
\item
Commands to store and retrieve intermediate spectra in storage registers.
\item
Commands to perform the fitting and removal of polynomial, 
harmonic and gaussian baselines.
\item
Commands for filtering and editing spectra.
\item
Commands to determine important line parameters (peak intensity, width, 
etc).
\item
The ability to perform Fourier transform and power spectrum calculations.
\item
Procedures for the calibration of data.
\item
The ability to assemble a number of reduced individual spectra into a 
map file, and contour or greyscale any plane or planes of the resulting cube.
\item
The ability to write indirect command files.
\end{itemize}

A thorough description of the package is given in the ``SPECX Users' Manual''
which has been distributed as a Starlink Miscellaneous User Document (MUD).

SPECX is intended primarily for the analysis of JCMT data and can read
such data directly from the GSD files produced at the
telescope. However, there are now facilities for reading data from a
range of other important mm-wave telescopes, {\em e.g.} Onsala,
FCRAO, UMASS CO survey, IRAM-FITS, JCMT RxG, NRAO 8-beam (see 
the SPECX user manual for details). 
Spectra and maps can be output as FITS files, Starlink NDFs or ASCII files
for reading into other packages.


\section {Changes from Version 6.1}

The most important change is that the package now uses PGPLOT for its
plotting rather than Mongo. This means that cube planes can now be displayed
as false colour images as well as contour plots. In addition, the user now 
has control of the plotted line thickness so that better quality graphics
can be produced for publication.

\section {Getting Started}

SPECX is made available by typing:
\begin{verbatim}
      $ specxstart
\end{verbatim}
and started by typing:
\begin{verbatim}
      $ specx 
\end{verbatim}
The program will display the current version number and
a brief informational message, then the SPECX prompt ({\tt>>}) will appear.

\begin{verbatim}

                   -------------------------------------------
                           SPECX V6.2    7th-June-1992
                             Copyright (C) R.Padman
                   -------------------------------------------

        .
        .

      >> 
\end{verbatim}

Commands may then be entered to read and process data.
SPECX commands consist of one or more keywords, separated
by hyphens. Minimum matching is applied to each keyword
individually. A full list of all valid SPECX commands
may be found by entering `SHOW-COMMANDS', with a carriage return
in response to the resulting prompt. On-line help is available
using the command `HELP'.

A DCL command may be executed by proceeding it with a {\tt \$}, {\em e.g.}

\begin{verbatim}
      >> $dir
\end{verbatim}

The tab and blank characters, `,', `\verb+\+' and `;' are SPECX delimiters;
to pass these to DCL the entire string must be enclosed in single 
inverted commas, {\em e.g.}

\begin{verbatim}
      >> '$dir/size scan*.dat'
\end{verbatim}

Control-C may be 
used to interrupt commands in progress.
Control-C typed at the command level (when the
{\tt>>} prompt is displayed) will exit the program.

The EXIT command is the normal method of returning to DCL.

\section {SPECX files and directories}

SPECX uses and produces a variety of files. For graphics, 
intermediate plot files (called PLOT.nnn) are produced in the directory 
with logical name SPECXDIR. The SPECXSTART command assigns 
SPECXDIR to SYS\$LOGIN so that, 
by default, the PLOT.nnn files will appear in 
your login directory. This may be over-ridden if you define 
SPECXDIR as a process logical name. For example, typing

\begin{verbatim}
      $ define SPECXDIR disk$scratch:[user.data]
\end{verbatim}

after the SPECXSTART will cause plot files to appear in the named directory.

SPECX uses Starlink PGPLOT for graphics  and thus will produce data  files for
plots sent to some graphics devices ({\em e.g.} CANON.DAT for the Canon laser
printer, GKS\_72.PS for a Postscript printer). These are created in the current
default  directory, and are automatically deleted when the plot finishes.

There is an option (defaulted true) in SPECX which  allows dumping of the
current status of the program after each  command is executed. This is saved in
the file SPECX.DMP in the  current default directory. If this file does not
exist when SPECX  is started for the first time, it will be created. If it is 
subsequently deleted (or SPECX is started from a different  directory), the
package will restart with default initialisation; otherwise SPECX will be
re-started will all flags as previously  set, data files opened, and so on. 


\section {Examples}
New users should read the introductory chapters of the SPECX manual,
which give explanations and examples of basic techniques. Some sample
data and a demonstration command file are also available in the
[.EXAMPLE] sub-directory of SYS\_SPECX. These can be copied to the user's own
file space and tried out. The demonstration file is DEMO.SPX,
which the user should print out and work through by typing the commands
in from the terminal. It demonstrates the reading of JCMT data,
the fitting of baselines and gaussians to spectra, and shows how to make
and display a datacube. The file can also be run as an indirect command
file by an experienced user to check that the package is working correctly.

\section{Linking User-supplied Subroutines}
User-defined commands can be added to SPECX by making use of the
subroutines EXTRNL1 to EXTRNL10. For the Starlink version of SPECX 
the steps required are:

\begin{itemize}
\item
Compile your routine:
\begin{small}
\begin{verbatim}
$ specxstart
$ @sys_specx:specxdev ! to set up logical names for include files such as SPECX
$                     ! common blocks. These common blocks should always be
$                     ! the official ones referred to by the logical names, and
$                     ! not private copies.
$ fortran extrnl1

\end{verbatim}
\end{small}
\item
Set up the logical names required to link with the NDF library.
\begin{verbatim}
$ ndf_dev
\end{verbatim}

\item
Copy the file LINKPG\_NDF.COM from SYS\_SPECX. 
and edit it to link
in your own version of the appropriate routine. For example, the
line:
\begin{verbatim}
$ link /exe=specx_v6-2 -
\end{verbatim}
might be replaced by:
\begin{verbatim}
$ link /exe=specx_v6-2 extrnl1, -
\end{verbatim}
\item
Copy the CLD file SYS\_SPECX:SPECX62.CLD, and 
edit it to run your own version of SPECX\_V6-2.EXE. 
Redefine the SPECX command by typing:
\begin{verbatim}
$ set command specx62.cld
\end{verbatim}
\end{itemize}

\section{Acknowledgements}
STARLINK would like to acknowledge Rachael Padman for making
the SPECX package available for use, assisting in its installation,
and providing the example data.

\end{document}
