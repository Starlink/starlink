\chapter{The ADAM Libraries}
\label{C_overlibs}
\markboth{Programmers}{\stardocname}

ADAM provides subroutine libraries which make various facilities available
to application programmers.
Each library has a {\em Facility Name} which is a mnemonic used to identify it.
For example, the Facility Name for one of the libraries in the Starlink Data
System is `DAT'.
The ways in which these Facility Names impact upon the programmer are:
\begin{quote}
\begin{description}
\item [Routine Names] ---

Each routine in a library has a name of the form:

\begin{small}
\begin{verbatim}
    fac_name
\end{verbatim}
\end{small}

where `{\tt fac}' is the Facility Name, and `{\tt name}' is the specific routine
name.
By using this convention, name clashes between different libraries and
applications can be avoided.
It also becomes possible to identify the ADAM calls in a piece of code.

\item [Error Symbols] ---

The error codes which can be returned via the STATUS argument are given
symbolic names using Fortran PARAMETER statements.
These names are made available to a program by including a file with logical
name `fac\_ERR', as in:

\begin{small}
\begin{verbatim}
    INCLUDE 'DAT_ERR'
\end{verbatim}
\end{small}

which includes the symbolic names of error codes returned by the Data System
(DAT).
This file need only be included by routines which need to identify the precise
nature of an error.
The symbolic names have the form:

\begin{small}
\begin{verbatim}
    fac__error
\end{verbatim}
\end{small}

where `{\tt error}' is a mnemonic used to identify the error (note the use of
two `\_' (underline) characters).
The actual values of the error codes originate from the VMS MESSAGE utility,
and so form a coherent set with the status returns from VMS libraries.
A set of error codes which may be used by writers of non-ADAM facility
libraries is provided in the include file USER\_ERR.

\item [Parametric Constants] ---

These are symbolic names given to the constants associated with a library.
The most commonly needed constants are defined in a file with logical name
`SAE\_PAR'.
In the situation where a program requires the parametric constants for
other facilities, these must be included by:

\begin{small}
\begin{verbatim}
    INCLUDE 'fac_PAR'
\end{verbatim}
\end{small}

The form of these symbolic constants is identical to that used for error
codes, namely:

\begin{small}
\begin{verbatim}
    fac__const
\end{verbatim}
\end{small}

\end{description}
\end{quote}

This chapter provides an overview of the most important ADAM libraries.
They are grouped into systems.
Each system is described in more detail in a subsequent chapter except the
database system, which is mainly used in the SCAR applications package, and the
utilities, which are described in Starlink User Notes.

\section{Parameter system}
\markboth{Programmers}{\stardocname}

This provides the mechanism for controlling the action of ADAM programs:
\begin{quote}
\begin{description}

\item [PAR] \hfill [AED/15, APN/6]

 This provides high-level access to the parameter system.
 It makes it easy to input a small number of data values from the terminal and
 to specify the data objects, devices and so on which the program requires.

\end{description}
\end{quote}

\section{Data system}
\markboth{Programmers}{\stardocname}

This provides the dominant mechanism by which programs store and manipulate
data.
The data system as a whole is sometimes referred to colloquially as HDS, and
sometimes as NDF.
The following facilities are available, listed using a top-down approach:
\begin{quote}
\begin{description}

\item [NDF] \hfill [\xref{SUN/33}{sun33}{}, \xref{SGP/38}{sgp38}{}]

 This stands for Extensible $N$-dimensional Data Format, as described
 in Chapter \ref{C_HDS}.
 It is the standard Starlink format for storing data arrays.
 As such, it is expected to form the basis of most Starlink `image' processing
 applications, where `images' include arrays which have more or less than two
 dimensions.

\item [ARY] \hfill [\xref{SUN/11}{sun11}{}, \xref{SGP/38}{sgp38}{}]

 This is a set of routines for accessing Starlink ARRAY data structures built
 using HDS.
 The most likely reason for using these routines directly is to access
 ARRAY structures stored in NDF extensions.

\item [REF] \hfill [\xref{SUN/31}{sun31}{}]

 These routines allow you to store references to HDS data objects in special
 HDS reference objects, and allow locators to reference objects to be obtained.
 Their main uses are to maintain a catalogue of HDS objects and to avoid
 duplicating a large dataset.

\item [HDS] \hfill [\xref{SUN/92}{sun92}{}]

 The Hierarchical Data System is a flexible system for storing and retrieving
 data, and is of great importance to the whole of the Starlink Project.
 It is the basic data system in ADAM and is also used in many other software
 items.
 It is used very extensively in the implementation of ADAM itself and is also
 used  by application programmers.
 The routines themselves are mainly concerned with the highest level of the
 data system, including the {\em container file} which is the interface
 between the data system and the host computer's file system.

\item [CMP] \hfill [\xref{SUN/92}{sun92}{}]

 These routines simplify the coding needed to access {\em structure} components
 within HDS.

\item [DAT] \hfill [\xref{SUN/92}{sun92}{}, APN/7]

 These routines access and manipulate data objects, which may be
 {\em primitive}\/ (e.g.\ arrays of numbers), or {\em structured}\/
 (i.e.\ collections of other objects).

\end{description}
\end{quote}

\section{Message and Error systems}
\label{S_errmess}
\markboth{Programmers}{\stardocname}

These related systems provide the {\em only} mechanism by which text should be
sent to the command device (user terminal or batch log file).
Output will normally go to the command device via a user-interface.
The facilities involved are:
\begin{quote}
\begin{description}

\item [MSG] \hfill [\xref{SUN/104}{sun104}{}]

 Reports non-error information to the user.

\item [ERR] \hfill [\xref{SUN/104}{sun104}{}]

 Reports error messages to the user and environment.

\item [EMS] \hfill [\xref{SSN/4}{ssn4}{}]

 Constructs and stores error messages, but doesn't communicate them to the user.
 It is, therefore, of no interest to the application programmer, but of
 considerable interest to the programmer who is building new {\em systems}
 software.
 It is included here for completeness.

\end{description}
\end{quote}

\section{Graphics system}
\label{S_graphics}
\markboth{Programmers}{\stardocname}

This consists of several facilities offering different levels of control,
but all (except IDI) are based on the ISO standard Graphics Kernel System (GKS).

\begin{quote}
\begin{description}

\item [NCAR/SNX] \hfill [\xref{SUN/88}{sun88}{}, \xref{SUN/90}{sun90}{}, MUD/59]

 An extensive suite of high level graphics utilities originating from the
 National Center for Atmospheric Research in Boulder, Colorado.
 SNX contains some Starlink-produced extensions to NCAR.

\item [PGPLOT] \hfill [\xref{SUN/15}{sun15}{}, MUD/61]

 A high level package for plotting x-y plots, functions, histograms, bar charts,
 contour maps and grey-scale images.
 The version in use on Starlink uses GKS for its low-level graphics
 input/output.

\item [NAG graphics] \hfill [\xref{SUN/29}{sun29}{}]

 The NAG Graphics Library (formerly the {\em Graphical Supplement}) was
 originally seen as a way of plotting data associated with the numerical
 routines which make up the main NAG library.
 However, it can be used quite independently of the main library for plotting
 graphs, functions, and contours, though most people now prefer to use NCAR
 or PGPLOT.

\item [SGS] \hfill [\xref{SUN/85}{sun85}{}]

 A `simplified' graphics facility which provides most of the commonly needed
 basic graphics functions in a convenient form.
 The primary simplification over GKS is that, while several workstations can
 be open simultaneously, only one can be active at a given time.
 It has been designed so that calls to its routines can be freely interspersed
 with those of GKS.
 Specialised GKS functions are not reproduced in SGS.

\item [GKS] \hfill [\xref{SUN/83}{sun83}{}, MUD/27]

 The basic set of `implementation' routines.
 It is a fairly powerful low-level graphics facility, and is {\em the}
 international 2-dimensional graphics standard.

\item [IDI] \hfill [\xref{SUN/65}{sun65}{}]

 A standard for displaying astronomical data on an image display.
 It complements, rather than replaces, GKS, and should be used where intimate
 control of the image display device is required, and for functions which
 are outside the scope of GKS.

\item [AGI] \hfill [\xref{SUN/48}{sun48}{}]

 A database system which stores information about plots on a graphics device.
 This enables a program to relate to plots produced by other programs.

\item [GNS] \hfill [\xref{SUN/57}{sun57}{}]

 Almost every Starlink graphics program will need the name of a GKS or IDI
 device.
 GNS --- the Graphics Name Service --- allows the programmer to choose a device
 from a reasonably `friendly' series of names, which is then translated into the
 GKS `workstation type'.
 Unless you are opening GKS workstations directly or making specialised
 enquiries about devices, you are most unlikely to need to call GNS yourself.

\end{description}
\end{quote}

\section{Input/output systems}
\markboth{Programmers}{\stardocname}

In general, data storage should be handled through the data system.
However, there may be a need to handle files directly.
This is provided by the following closely related facilities:
\begin{quote}
\begin{description}

\item [FIO] \hfill [\xref{SUN/143}{sun143}{}]

 Handles sequential, formatted files for the production of reports, e.g.\ for
 subsequent listing.
 The form of carriage control may be specified when the file is created.

\item [RIO] \hfill [\xref{SUN/143}{sun143}{}]

 Handles unformatted, direct-access files.

\end{description}
\end{quote}
There may also be a need to access `foreign' magnetic tapes:
\begin{quote}
\begin{description}

\item [MAG] \hfill [APN/1]

 Handles tape positioning, and reading or writing data blocks and tape marks.
 It also includes facilities for keeping track of tape positions so that
 programs can provide a friendly user interface.

\end{description}
\end{quote}

\markboth{Programmers}{\stardocname}
\section{Database system}

These are provided mainly for access to astronomical catalogues:
\begin{quote}
\begin{description}

\item [CHI] \hfill [\xref{SUN/119}{sun119}{}]

 This is gradually replacing ADC as the preferred method of accessing
 relational databases, such as the usual astronomical catalogues or large
 tables of measurements of sets of objects.

\item [ADC] \hfill

 Provides access to the relational database files handled by the SCAR programs.
 It is implemented as part of the SCAR software item.

\end{description}
\end{quote}

\section{Utilities}
\markboth{Programmers}{\stardocname}

These differ from the libraries mentioned above in that they are not associated
with parameters.
They are libraries which enable ADAM programs to do various tasks in a
consistent way:
\begin{quote}
\begin{description}
\item [CHR] \hfill [\xref{SUN/40}{sun40}]

 Manipulation of character strings, encoding and decoding numeric fields etc.

\item [CNF] \hfill [SGP/5]

 Interchange of character strings between Fortran and C software modules.

\item [PRIMDAT] \hfill [\xref{SUN/39}{sun39}{}]

 Arithmetic, mathematical operations, type conversion, and inter-comparison of
 any of the  primitive data types supported by HDS.

\item [PSX] \hfill [\xref{SUN/121}{sun121}{}]

  An interface to the Posix portable operating system.

\item [SLALIB] \hfill [\xref{SUN/67}{sun67}{}]

 Routines mainly concerned with astronomical position and time.

\item [TRANSFORM] \hfill [\xref{SUN/61}{sun61}{}]

 A standard, flexible method for manipulating coordinate transformations,
 and for transferring information about them between programs.

\end{description}
\end{quote}
