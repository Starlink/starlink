\chapter{The Message and Error Systems}
\label{C_messerrs}
\markboth{Programmers}{\stardocname}

There is a general need for application programs to provide the user with
informative messages about:

\begin {itemize}
\item What they do --- for example, during long operations it is helpful if the
 user is kept informed of what a program is doing.
\item What results have been obtained --- for example, the notification of  the
 final results from a procedure, or of some intermediate results that would help
 the user respond to further prompts.
\item What errors have occurred --- for example, errors which lead to the user
 being prompted to provide more sensible input to a program, or fatal errors
 which cause an application to stop.
\end {itemize}

This chapter describes two subroutine libraries which can be used for this
purpose.
They are:

\begin{quote}
\begin{description}
\item [MSG] --- Message Reporting System.
\item [ERR] --- Error Reporting System.
\end{description}
\end{quote}

These are fully described in \xref{SUN/104}{sun104}{}.

\section {MSG --- Message reporting system}

The obvious way to produce messages in Fortran programs is by WRITE and PRINT
statements.
It is possible to construct the text of a message from various components,
including numbers, formatted in a CHARACTER variable using the internal WRITE
statement.
The resulting message may then be displayed.
However, it is sometimes difficult to format numerical output in its most
concise form.
To do this in-line each time a message is sent to the user would be very
inconvenient and justifies the provision of a dedicated set of subroutines.
These considerations led to the production Message Reporting System.

\subsection {Reporting messages}

The primary message reporting subroutine is MSG\_OUT.
It has a calling sequence of the form:

\begin{small}
\begin{verbatim}
      CALL MSG_OUT(PARAM, TEXT, STATUS)
\end{verbatim}
\end{small}

where PARAM is a character string giving the name of the message, TEXT is a
character string giving the message text, and STATUS is the integer
{\em global status} value.

It sends the message string, TEXT, to the standard output stream.
This will normally be the user's terminal, but is the log file for a batch job.
The maximum message length is 200 characters.
If it exceeds this, it is truncated with an ellipsis, {\em i.e.} `\ldots', but
no error results.

Here is an example of using MSG\_OUT:

\begin{small}
\begin{verbatim}
      CALL MSG_OUT('EXAMPLE_MSGOUT', 'An example of MSG_OUT.', STATUS)
\end{verbatim}
\end{small}

It is sometimes useful to add blank lines for clarity.
MSG\_BLANK does this:

\begin{small}
\begin{verbatim}
      CALL MSG_BLANK(STATUS)
\end{verbatim}
\end{small}

Messages can also be stored in a character variable, rather than being output.
MSG\_LOAD does this.

\subsection {Conditional message reporting}

It is sometimes useful to have varying levels of message output which may be
controlled by the user.
This is achieved by MSG\_OUTIF:

\begin{small}
\begin{verbatim}
      CALL MSG_OUTIF(MSG__NORM, 'MESS_NAME', 'A conditional message', STATUS)
\end{verbatim}
\end{small}

Here, the first argument is the `priority' associated with the message.
It has three possible values, which represent filter levels:
\begin {quote}
\begin {description}
\item [MSG\_\_QUIET] --- always output message, regardless of output filter
setting.
\item [MSG\_\_NORM] --- output message if current output filter is set to
 either MSG\_\_NORM or MSG\_\_VERB.
\item [MSG\_\_VERB] --- output message only if current output filter is set to
 MSG\_\_VERB.
\end {description}
\end {quote}
(I know this looks wrong on a first reading, but `quiet' messages are messages
that are output in `quiet' mode, and are therefore the loudest!)
The default is MSG\_\_NORM.
It may be modified by MSG\_IFSET, for example:

\begin{small}
\begin{verbatim}
      CALL MSG_IFSET(MSG__QUIET, STATUS)
\end{verbatim}
\end{small}

\subsection {Message tokens}

Applications often need to include the values of Fortran variables in output
messages.
This is done in the Message System using tokens embedded in the message text.
For example, a program which measures the intensity of an emission line in
a spectrum can output its result by:

\begin{small}
\begin{verbatim}
      CALL MSG_SETR('FLUX', FLUX)
      CALL MSG_OUT('EXAMPLE_RESULT',
     :             'Emission flux is ^FLUX (erg/cm2/A/s).', STATUS)
\end{verbatim}
\end{small}

Here, MSG\_SETR assigns the result, stored in the REAL variable FLUX, to the
message token named `FLUX'.
The token string is then included in the message text by prefixing it with the
up-arrow, `$^\wedge$', escape character.
The usual set of similar routines is provided to handle the other standard
data types.

An additional feature of the MSG\_SETx routines is that calls using an existing
token name will result in the value being appended to any previously assigned
token string.
Here is the previous example written to exploit this feature:

\begin{small}
\begin{verbatim}
      FUNITS = ' erg/cm2/A/s'
      CALL MSG_SETR('FLUX', FLUX)
      CALL MSG_SETC('FLUX', FUNITS)
      CALL MSG_OUT('EXAMPLE_RESULT',
     :             'Emission flux is ^FLUX.', STATUS)
\end{verbatim}
\end{small}

Note that repeated calls append values with no separator, and hence a leading
space is needed in the FUNITS string to separate the flux value and its units
in the expanded message.

Sometimes a specific format is required; for example, the message may form part
of a table.
MSG\_FMTx is a set of subroutines of the form:

\begin{small}
\begin{verbatim}
      CALL MSG_FMTx(TOKEN, FORMAT, VALUE)
\end{verbatim}
\end{small}

where FORMAT is a valid Fortran 77 format string which can be used to encode
the supplied value, VALUE.
As for MSG\_SETx, {\em x} corresponds to each of the five standard Fortran 77
data types --- {\em D}, {\em R}, {\em I}, {\em L} and {\em C}.

Sometimes it is necessary to include the message token escape character,
`$^\wedge$', literally in a message.
When the character is not the last in a message string, it can be included
literally by duplicating it.
When it is immediately followed by a blank, or is at the end of the MSG\_OUT
text, it is included literally.
Escape characters and token names will also be output literally if they
appear within the value assigned to a message token; {\em i.e.}\, message token
substitution is not recursive.

\subsection {Message parameters}

In calls to MSG\_OUT and MSG\_LOAD, the message name is the name of a message
parameter which is associated with the message text.
This name should be no more than 15 characters long and may be associated with
a message specified in the interface file.
When the message parameter is specified in the interface file, this text is
used in preference to that given in the argument list.

Here is an example of using MSG\_OUT:

\begin{small}
\begin{verbatim}
      CALL MSG_OUT('RD_TAPE', 'Reading tape.', STATUS)
\end{verbatim}
\end{small}

This will generate the message:

\begin{small}
\begin{verbatim}
    Reading tape.
\end{verbatim}
\end{small}

If the message parameter, `RD\_TAPE', is associated with a different text
string in the interface file, {\em e.g.}

\begin{small}
\begin{verbatim}
    message RD_TAPE
       text 'The program is currently reading the tape, please wait.'
    endmessage
\end{verbatim}
\end{small}

then the output message would be the one defined in the interface file:

\begin{small}
\begin{verbatim}
    The program is currently reading the tape, please wait.
\end{verbatim}
\end{small}

This enables ADAM applications to support foreign languages.

\subsection {Parameter references}

We may need to refer to a program parameter in a message.
There are two kinds of reference required:
\begin{itemize}
\item The {\em keyword}.
\item The  {\em name} of an object, device, or file.
\end{itemize}
We can include references of these kinds by prefixing parameter names with an
{\em escape character} of which there are three: `$\wedge$', `\%', `\$'.
To include a parameter's keyword, prefix its name with a `\%' character, as in:

\begin{small}
\begin{verbatim}
    CALL MSG_OUT('ET_RANGE', '%ET parameter is ignored', STATUS)
\end{verbatim}
\end{small}

If the keyword of parameter ET is `EXPOSURE\_TIME', the output generated by this
call is:

\begin{small}
\begin{verbatim}
    EXPOSURE_TIME parameter is ignored
\end{verbatim}
\end{small}

To include the name of an object, device, or file associated with a parameter,
prefix its name with a `\$' character, as in:

\begin{small}
\begin{verbatim}
    CALL MSG_OUT('CREATING', 'Creating $DATASET', STATUS)
\end{verbatim}
\end{small}

If parameter DATASET is associated with an object called SWP1234, this would
produce the output:

\begin{small}
\begin{verbatim}
    Creating SWP1234
\end{verbatim}
\end{small}

Sometimes a parameter name is contained in a variable.
In order to use it in a message, a message token can be associated with its
name.
For example:

\begin{small}
\begin{verbatim}
      CALL MSG_SETC('PAR', PNAME)
      CALL MSG_OUT('PAR_UNEXP', '%^PAR=0.0 unexpected', STATUS)
\end{verbatim}
\end{small}

In this example, the escape sequence `$^\wedge$' prefixes the token name,
`PAR'.
The sequence `$^\wedge$PAR' gets replaced by the contents of the character
string contained in the variable PNAME; this, being prefixed by `\%', then
gets replaced in the final message by the keyword associated with the
parameter.

\subsection {Getting the conditional output level}

Routine MSG\_IFSET sets the filter level for conditional message output.
Routine MSG\_IFGET also sets it, but gets its value from the parameter system:

\begin{small}
\begin{verbatim}
      CALL MSG_IFGET(PNAME, STATUS)
\end{verbatim}
\end{small}

where PNAME is the parameter name.
Always use the same name, MSG\_FILTER, for this purpose.
The three acceptable strings are:
\begin {quote}
\begin {description}
\item [QUIET] --- representing MSG\_\_QUIET;
\item [NORMAL] --- representing MSG\_\_NORM;
\item [VERBOSE] --- representing MSG\_\_VERB.
\end {description}
\end {quote}
Any other value will result in an error report and the status value being
set to MSG\_\_INVIF.

\section{ERR --- Error reporting system}

Although the Message System could be used for reporting errors,
there are a number of reasons why a separate facility should be provided:
\begin {itemize}
\item The Message System uses the inherited status scheme, so MSG\_OUT and
 MSG\_LOAD will not execute if STATUS has an  error value.
 Consequently, it cannot report information about an earlier error.
\item In a program consisting of many subroutine levels, each routine which
 has something informative to say about an error should be able to contribute
 to the information that the user receives.
 This includes:
\begin {itemize}
\item The routine which first detects the error, as this will probably
 have access to information which is hidden from higher level routines.
\item The chain of routines between the main program and the routine in
 which the error originated.
 Some of these will usually be able to report on the context in which the error
 occurred, and so add relevant information which is not available to routines
 at lower levels.
\end {itemize}
 This can lead to several error reports arising from a single failure.
\item It is not always necessary for an error report to reach the user.
 For example, a routine may decide that it can safely handle an error detected
 at a lower level without informing the user.
 In this case, error reports associated with the error should be discarded, and
 this can only happen if the output of error messages to the  user is deferred.
\end {itemize}
These considerations have led to the design and implementation of a set of
routines which form the Error Reporting System.

\subsection{Inherited status checking}

The recommended method of indicating when errors have occurred in Starlink
software is to use an integer status value in each subroutine argument list.
This inherited status argument, say STATUS, should always be the last argument
and every routine should check its value on entry.
The {\em ADAM Error Strategy} is as follows:
\begin {itemize}
\item The routine returns without action if STATUS is input with a value
 other than SAI\_\_OK.
\item The routine leaves STATUS unchanged if it completes successfully.
\item The routine sets STATUS to an appropriate error value and outputs an
 error message if it fails to complete successfully.
\end {itemize}
Note that it is often useful to use a status argument and inherited status
checking in routines which `cannot fail'.
This prevents them executing, possibly producing a run-time error, if their
arguments contain rubbish after a previous error.
Every piece of software that calls such a routine is then saved from making
an extra status check.
Furthermore, if the routine is later upgraded it may acquire the potential
to fail, and so a status argument will subsequently be required.
If a status argument is included initially, existing code which calls
the routine will not need changing.

\subsection {Reporting errors}

The routine used to report errors is ERR\_REP.
It has a calling sequence of the form:

\begin{small}
\begin{verbatim}
      CALL ERR_REP(PARAM, TEXT, STATUS)
\end{verbatim}
\end{small}

where PARAM is the error message name, TEXT is the error message text, and
STATUS is the inherited status.
These arguments are similar to those used in the Message System routine
MSG\_OUT.

The error message name, PARAM, should be a globally unique identifier for the
error report with the form:

\begin{small}
\begin{verbatim}
    routn_message
\end{verbatim}
\end{small}

for routines in an application, or:

\begin{small}
\begin{verbatim}
    fac_routn_message
\end{verbatim}
\end{small}

for routines in a subroutine library.
In the former case, {\tt routn} is the name of the application routine from
which ERR\_REP is being called, and {\tt message} is a sequence of characters
uniquely identifying the error report within that routine.
In the latter case, {\tt fac\_routn} is the full name of the routine from
which ERR\_REP is being called, and {\tt message} is a sequence of characters
unique within that routine.
These naming conventions are designed to ensure that each error report made
within a complete software system has a unique error name associated with it.

Here is a simple example of error reporting where part of the application code
detects an invalid value of some kind, sets STATUS, reports the error, and then
aborts:

\begin{small}
\begin{verbatim}
          IF (<invalid value>) THEN
             STATUS = SAI__ERROR
             CALL ERR_REP('ROUTN_BADV', 'Value is invalid.', STATUS)
             GO TO 999
          END IF
          ...
     999  CONTINUE
          END
\end{verbatim}
\end{small}

This sequence of three operations:
\begin {quote}
\begin {enumerate}
\item Set STATUS to an error value.
\item Report an error.
\item Abort.
\end {enumerate}
\end {quote}
is the standard response to an error condition and should be adopted by all
software which uses the Error System.

Note that ERR\_REP differs from MSG\_OUT in that ERR\_REP will execute
regardless of the input value of STATUS.
Although the Starlink convention is for routines not to execute if their
status argument indicates a previous error, the Error System routines
obviously cannot behave in this way if their purpose is to report these errors.

Message tokens can be used in ERR\_REP in the same way as in MSG\_OUT.

\subsection{When to report an error}

In the following example, part of an application makes a series of routine
calls:

\begin{small}
\begin{verbatim}
          CALL ROUTN1(A, B, STATUS)
          CALL ROUTN2(C, STATUS)
          CALL ROUTN3(T, Z, STATUS)

    *  Check the global status.
          IF (STATUS .NE. SAI__OK) GO TO 999
          .
    999   CONTINUE
          END
\end{verbatim}
\end{small}

Each routine uses the inherited status strategy and reports errors by calling
ERR\_REP.
If an error occurs within any of them, STATUS will be set to an error value
and inherited status checking by all subsequent routines will cause them not
to execute.
Thus, it becomes unnecessary to check for an error after each routine call,
and a single check at the end of the sequence of calls is all that is required
to handle correctly any error condition that may arise.
Because an error report will already have been made by the routine that failed,
it is usually sufficient simply to abort if an error arises in a sequence of
routine calls.

It is important to distinguish the case where a called subroutine sets STATUS
and makes its own error report, as above, from the case where STATUS is set
explicitly as a result of a directly detected error, as in the previous example.
If the error reporting strategy is to function correctly, then responsibility
for reporting the error must lie with the routine which modifies the status
argument.
The golden rule is therefore:

\begin {quote}
{\em If STATUS is explicitly set to an error value, then an accompanying
call to ERR\_REP {\bf must} be made.}
\end {quote}

Unless there are good documented reasons why this cannot be done,
routines which return a bad status value and do not make an accompanying
error report should be regarded as containing a bug\footnote{For historical
reasons there are still many routines in ADAM which set a status value without
making an accompanying error report --- these are gradually being corrected.
If such a routine is used before it has been corrected, then the strategy
outlined here is recommended.
It is advisable not to complicate new code by attempting to make an error
report on behalf of the faulty subroutine.
If appropriate, please tell the relevant support person about the problem.}.

\subsection {Setting and defining status values}

Normally, set status values to the global constants SAI\_\_OK and SAI\_\_ERROR.
However, when writing subroutine libraries it is useful to have a larger
number of error codes available and to define these in a separate include file.
The naming convention:

\begin{small}
\begin{verbatim}
    fac__ecode
\end{verbatim}
\end{small}

should be used for the names of error codes where {\tt fac} is the
three-character facility prefix and {\tt ecode} is up to five alphanumeric
characters of error code name.
{\em Note the double underscore used in this naming convention.}
The include file should be referred to by the name fac\_ERR, {\em e.g.}

\begin{small}
\begin{verbatim}
      INCLUDE 'SGS_ERR'
\end{verbatim}
\end{small}

where in this case the facility name is {\tt SGS} (Simple Graphics System).
These symbolic constants should be defined at the beginning of every
routine which requires them, prior to the declaration of any subroutine
arguments or local variables.

The purpose of error codes is to enable the status argument to indicate that an
error has occurred by having a value which is not equal to SAI\_\_OK.
By using a set of pre-defined error codes the calling routine is able to test
the returned status to distinguish between error conditions which may require
different action.
Generally, it is not necessary to define a large number of error codes which
would allow a unique value to be used every time an error report is made.
It is sufficient to be able to distinguish the important classes of error which
may occur.
Examples of existing software can be consulted as a guide in this matter.

Software from outside a package which defines a set of error codes may
use that package's codes to test for specific error conditions arising within
that package.
However, with the exception of the SAI\_\_ codes, it should {\em not} assign
these values to the status argument.
To do so could cause confusion about which package detected the error.

\subsection{The content of error messages}

The purpose of an error message is to be informative and it should therefore
provide as much relevant information about the context of the error as possible.
It should not be misleading or contain irrelevant information.
Particular care is necessary when reporting errors from routines which might
be called by a wide variety of software.
They should not make unjustified assumptions.
For example, in a routine that adds two arrays, the report:

\begin{small}
\begin{verbatim}
    !! Error adding two arrays.
\end{verbatim}
\end{small}

would be preferable to:

\begin{small}
\begin{verbatim}
    !! Error adding two images.
\end{verbatim}
\end{small}

if the same routine could be called to add two spectra!

Normally it is adequate to report an error when is first detected, followed by
a further report from the `top-level' routine.
Only include routine names in error reports if they appear in documentation.

\subsection {Deferred error reporting}

The Error System can defer the output of a message to the user, and this allows
the final delivery of error messages to be controlled.
This is done by the routines ERR\_MARK, ERR\_RLSE, ERR\_FLUSH and ERR\_ANNUL.
This section describes their functions and how they are used.

The purpose of deferred error reporting can be illustrated by the following
example.
Consider a routine, HELPER, which detects an error during execution.
It reports the error to its caller, giving as much contextual information as
it can.
It also returns an error status, enabling the caller to react appropriately.
However, what may be considered an `error' in HELPER, {\em e.g.} an `end of
file' condition, may be considered by the caller to be something that can be
handled without informing the user, {\em e.g.} by terminating its input.
Thus, although HELPER will always report the error, it is not always necessary
for the associated error message to reach the user.
Deferred error reporting enables programs to handle such errors internally.


Suppose HELPER reports an error.
At this point, the error message may, or may not, have been received by the
user --- this will depend on the environment and on whether the caller deferred
the error report.
HELPER should not ensure delivery of the message to the user; its
responsibility ends when it aborts, and responsibility for handling the error
condition passes to the caller.

Suppose HELPER is called by HELPED which defers error messages so it can decide
how to handle errors.
It does this by calling ERR\_MARK before HELPER.
This ensures that subsequent error messages are deferred and stored in an
`error table'.
It also starts a new `error context' which is independent of previous error
messages or tokens.
Routine ERR\_RLSE returns to the previous context, whereupon any messages in
the new error context are transferred to the previous context.
In this way, no existing error messages can be lost through deferral.
ERR\_MARK and ERR\_RLSE should always be called in matching pairs and can
be nested.

After deferring the error messages, HELPED can handle the error condition
in one of two ways:

\begin {itemize}
\item by calling ERR\_ANNUL, which `annuls' the error, deleting any deferred
 error messages in the current context and resetting STATUS to SAI\_\_OK.
 This causes the error condition to be ignored.
\item by calling ERR\_FLUSH, which `flushes out' the error, sending any
 deferred error messages in the current context to the user and
 resetting STATUS to SAI\_\_OK.
 This notifies the user that a problem has occurred, but allows the
 application to continue anyway.
\end{itemize}

When an application finally ends, the value of the status argument will
reflect whether or not it finished with an error condition.
At this point, any remaining error messages will be delivered automatically
to the user.

\subsection {Error message parameters}

In calls to ERR\_REP, the error name is the name of a
message parameter which is associated with the error message text.
Like the message parameters used in MSG\_OUT and MSG\_LOAD, those used in
ERR\_REP may be associated with an error message specified in the interface
file as well as in the argument list.
