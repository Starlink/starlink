\chapter{The Graphics System}
\label{C_graphics}
\markboth{Programmers}{\stardocname}

The Starlink graphics software was introduced in Chapter \ref{C_overlibs}.
A guide to its use is available in \xref{SUN/113}{sun113}{}.
The main packages are:

\begin{quote}
\begin{description}

\item [PGPLOT] --- is recommended for general use.
It provides an easy to use yet powerful way of plotting 2-d axes, contour maps,
grey-scale, colour images, and many other styles of plot.
It is designed to appeal to astronomers and is especially good at publication
quality laser printer output.
It cannot be used in conjunction with SGS or any other plotting package.

\item [NCAR/SNX] --- provides similar facilities to PGPLOT plus some additional
styles of plot (e.g.\ stream-line diagrams) and in general is more powerful
(for example, it can draw an X,Y plot with different scales on the right and
left hand axes) but is somewhat harder to use.
It also contains a 3-d drawing package.
You can call SGS and GKS by using SNX.

\item [NAG] --- graphics can also be used, but it is not encouraged for software
that is to become part of the Starlink software collection because its
availability on all hardware platforms is not guaranteed (particulary in its
single precision form) and non-Starlink sites may not have a licence to use it.
NCAR and NAG graphics do not have their own mechanisms for opening graphics
devices and SGS or GKS must be used to open a device before plotting can begin.

\item [SGS] --- should be used for line graphics which requires a style not
catered for by PGPLOT or NCAR.
This is ``lower level'' than PGPLOT or NCAR (there are no facilities for
drawing axes for example) but it does have excellent facilities for
manipulating transformations and ``organising'' the display surface.
SGS programs can also make direct calls to GKS (with a few restrictions),
so the full power of GKS is available if SGS's facilities are not adequate.

\item [GKS] --- should normally not be used because getting even simple things
done (e.g.\ opening a workstation) can be very long-winded and more easily
achieved with SGS.
Furthermore, achieving device independence is far from straight-forward; all
the necessary facilities are available but making proper use of them requires
an in-depth knowledge of GKS.
Again, this is better left to SGS or higher level packages.

\end{description}
\end{quote}

Three more specialised graphics facilities provided by ADAM are:

\begin{quote}
\begin{description}

\item [IDI] --- should be used in programs requiring intimate interaction with
an image display device.
This is not based on GKS and offers access to facilities that are outside the
GKS model of a graphics device.
Currently, the only devices supported are the Ikon image display (on VAXs
only) and X-windows.

\item [AGI] --- should be used where a suite of graphics programs are being
written, or where a program is to interact with graphics produced by another
package (e.g.\ KAPPA).
It provides facilities for sharing information about pictures such as extent
and world coordinates.

\item [GNS] --- enables a program to obtain information about graphics devices.

\end{description}
\end{quote}

Before proceeding, a word of warning which is applicable to all Starlink
graphics.
Do {\em not} examine the workstation type or name in order to control the
behaviour of a program.
For example, you are not allowed to say `the GKS workstation type is 3200,
therefore this is an Ikon image display, therefore colour is available'.
This information must be obtained by calling the appropriate GKS, SGS, or
\mbox{PGPLOT} inquiry routines;  in the unlikely event of the property you are
interested in not being available through such inquiries, then devise some
mechanism external to the program, or solicit information from the user.

\section{SGS --- Simple graphics system}
\label{S_sgsgks}
\markboth{Programmers}{\stardocname}

The following example graphics program uses SGS to draw a circle on a selected
graphics device:

\begin{small}
\begin{verbatim}
          SUBROUTINE CIRCLE(STATUS)
          IMPLICIT NONE
          INCLUDE 'SAE_PAR'
          INTEGER STATUS
          REAL RADIUS          ! Radius of circle
          INTEGER ZONE         ! SGS Zone
    *.........................................................
          IF (STATUS.NE.SAI__OK) RETURN
          CALL PAR_GET0R('RADIUS', RADIUS, STATUS)
          CALL SGS_ASSOC('DEVICE', 'WRITE', ZONE, STATUS)
          IF (STATUS.EQ.SAI__OK) THEN
              CALL SGS_CIRCL(0.5, 0.5, RADIUS)
          END IF
          CALL SGS_ANNUL(ZONE, STATUS)
          CALL SGS_DEACT(STATUS)
          END
\end{verbatim}
\end{small}

The statements concerned specifically with graphics will now be considered
individually.

\begin{small}
\begin{verbatim}
          CALL SGS_ASSOC('DEVICE', 'WRITE', ZONE, STATUS)
\end{verbatim}
\end{small}

This makes the association between the parameter DEVICE and an SGS zone on the
graphics device provided by the parameter system.
The zone identifier is returned to the program in the ZONE variable.
SGS and the zone are made ready for graphical output.
The second parameter is the access mode whose possible values are 'READ',
'WRITE', and 'UPDATE'.
The value `WRITE' will cause the screen to be cleared; the other values
cause the screen to be left as it is.

\begin{small}
\begin{verbatim}
          CALL SGS_CIRCL(0.5, 0.5, RADIUS)
\end{verbatim}
\end{small}

This draws a circle centered at (0.5,0.5) in world coordinates with radius
RADIUS.

\begin{small}
\begin{verbatim}
          CALL SGS_ANNUL(ZONE, STATUS)
\end{verbatim}
\end{small}

This closes the graphics workstation without cancelling the associated
parameter. The routine SGS\_CANCL is similar but {\em does}
cancel the parameter.

\begin{small}
\begin{verbatim}
          CALL SGS_DEACT(STATUS)
\end{verbatim}
\end{small}

This de-activates SGS.

A possible interface file for this program is:

\begin{small}
\begin{verbatim}
    interface CIRCLE
       parameter RADIUS
          type     '_REAL'
          position 1
          prompt   'Radius of Circle'
          ppath    'current'
          vpath    'prompt'
       endparameter
       parameter DEVICE
          ptype    'DEVICE'
          position 2
          prompt   'Graphics Device'
          ppath    'current'
          vpath    'prompt'
       endparameter
    endinterface
\end{verbatim}
\end{small}

After compiling and linking CIRCLE:

\begin{small}
\begin{verbatim}
    $ for circle
    $ alink circle
\end{verbatim}
\end{small}

we can run it from ICL.
In the following example, circles of different radii are plotted on different
devices (things like device names and print queues are likely to be different
for you):

\begin{small}
\begin{verbatim}
    ICL> define circle circle
    ICL> circle 0.3 graphon
    ICL> circle
    RADIUS - Radius of Circle /0.3/ > 0.2
    DEVICE - Graphics Device /@graphon/ > postscript_l
    ICL> $ print/q=sys_ps/form=post gks_72.ps
    ICL>
\end{verbatim}
\end{small}

With hard copy graphics devices, such as `postscript\_l', a file is created
which must then be sent to the device with a DCL PRINT command.

\section{IDI --- Image display interface}
\markboth{Programmers}{\stardocname}

IDI is a standard for displaying astronomical data on an image display.
It is introduced in \xref{SUN/65}{sun65}{}, and the routine specifications are
given in
`An image display interface for astronomical image processing' by D L Terrett,
et al, published in {\em Astron.\ Astrophys.\ Suppl.\ Ser.} {\bf 76},
{\em 263-304 (1988).}
It has been integrated with ADAM by the usual ASSOC, ANNUL, and CANCL routines
discussed in Chapter~\ref{C_parsys}.
An example of its use is contained in \xref{SUN/65}{sun65}{}.

IDI is intended for programs that need to manipulate images to a greater extent
than can be done with GKS and its offspring.
Thus, it does not supersede GKS, but offers features not available in GKS.
It is not as good as GKS for producing line plots or character annotation ---
it does have routines to draw lines and plot text, but these are primitive and
offer you little control over how the result will appear in terms of character
sizes, style of line widths etc.

Its major strength is its ability to perform many types of interaction using the
mouse.
Like GKS, it can display an image, move the cursor, and rotate the look-up
table.
However, it can also scroll and zoom, blink, and read back a representation of
the whole display, which can then be used to obtain a hardcopy.
IDI allows these functions to be programmed in a device independent way, so
a program can use any device for which IDI has been implemented.

It is not possible to mix GKS and IDI calls on the same display; the two
packages use completely different display models.
However, it is possible to run the packages one after the other.
For example, IDI could follow GKS and the display could be opened without
resetting it.
However, this is an undesirable approach since the results could be
unpredictable.
A better solution is to use AGI to mediate between the packages.

\section{AGI --- Applications graphics interface}
\markboth{Programmers}{\stardocname}

One of the recurring problems in graphics programming is that you would
like a program to know about graphics drawn by another program.
For instance, you display an image of the sky, then later you want to obtain
the co-ordinates of the stars within the image via the cursor.
There are two ways of doing this.
The first is to duplicate the display code in the cursor application, which is
wasteful and inflexible.
The second is to store information about {\em pictures} in a database that can
be accessed by graphics programs.
Immediately after a picture is created, information about its position and
extent can be added to the database.
This second approach is implemented by AGI which can store information about
plots on any graphics device.

The AGI graphics database is stored in file AGI\_USER:AGI\_$<$hostname$>$.SDF
which contains a data object named WORKSTATION comprising a set of components,
one set for each device used.
Each type of workstation used causes the generation of a WORKSTATION structure
in the database (which is specific to each user).
The PICTURE structure stores details of one or more `pictures' which are
visible on the workstation's display.
The AGI routines are used to create and manipulate these picture structures in
the database.
The AGI documentation (\xref{SUN/48}{sun48}{}) talks about `storing pictures'
in the database.
Don't be alarmed by this loose terminology; the picture itself isn't stored,
just the following information:
\begin{quote}
\begin{description}
\item [PNAME] ---
A generic name specifying the type of the picture.
Currently the following names are used:
\begin{itemize}
\item {\em BASE}\, indicates the base picture --- the picture that extends
 over the  whole of  the plotting area.
 This name is reserved and should not be used to name other pictures.
\item {\em DATA}\, indicates that the picture contains a representation of
 data in a graphical form, e.g.\ a greyscale plot or a histogram.
\item {\em FRAME}\, indicates collections of DATA pictures.
 For instance, in a contour plot, the DATA picture is the area where contours
 may  potentially be drawn, whereas the FRAME picture comprises the annotated
 axes and labels, the key, and the data area.
\end{itemize}
\item [COMENT] ---
 This is a one-line description of the picture.
\item [DEVICE] ---
 This array holds the natural device coordinates.
\item [NDC] ---
 This array holds the normalised device coordinates.
\item [WORLD] ---
 This array holds the world coordinates corresponding to the corners of the
 picture.
\end{description}
\end{quote}
This information enables AGI to relate a particular position on the display
surface to a particular position in world coordinates, and vice versa.
The PNAME and COMENT fields are used to describe a `picture', and the DEVICE,
NDC, and WORLD fields are used to define its `extent'.

Each time a picture is stored in the database for a particular workstation,
it is added to the end of the PICTURE array.
This array acts as a push-down stack; thus the `last' picture is the one added
most recently to the array, and this `precedes' the one added before it.
This order corresponds to the visual impression of new pictures obscuring
older pictures on the display screen.

The database always has a `current' picture and workstation associated with it.
Each workstation also has a `base' picture associated with it; this fills the
display area and always exists.
A new picture is always created inside the current picture.
The current picture for each workstation is remembered between sessions.

\paragraph{AGI routines:}\hfill
\markboth{Programmers}{\stardocname}

The `objects' manipulated by the AGI library are `pictures' in the database.
The basic set of routines in the library begin with the prefix `AGI\_'.
In the descriptions below, the routines will be identified by the characters
which follow `AGI\_'.
They fall naturally into the following groups:
\begin{quote}
\begin{description}

\item [Parameter routines] ---

These are the usual two routines used to integrate a subroutine library with
the ADAM parameter system, i.e.\ {\bf ASSOC} and {\bf CANCL}.
They correspond to the {\bf OPEN} and {\bf CLOSE} stand-alone routines.
The `picture' object is identified by a picture identifier `picid'.
The parameter with which the picture is associated is a graphics display
device,
normally called a workstation in GKS.

\item [Activity routines] ---

These are the three routines that actually {\em do} something, rather than just
navigate around the database.

{\bf SELP} changes the identity of the current picture --- most AGI routines
 simply use the current picture but don't change its identity.

{\bf NUPIC} creates a new picture in the database and selects this as the
new current picture.
The information that must be provided is:
\begin{itemize}
\item The position of the new picture within the current picture.
\item The `NAME', `COMENT', and `EXTENT' of the new picture.
\end{itemize}
{\bf PDEL} deletes all the pictures in the database for the current workstation.

\item [Inquiry routines] ---

These nine routines obtain information about pictures.
Their names (after the `AGI\_') all start with `I'.
One routine, {\bf IBASE}, identifies the base picture for the current
workstation.
The other eight all obtain information about the current picture.
Five of these, {\bf ICURP}, {\bf ILAB}, {\bf INAME}, {\bf ICOM}, and
{\bf IWOCO}, obtain the identifier, label, name, comment, and extent.
Two others indicate whether or not the picture, {\bf IPOBS}, or a set of
test points, {\bf ITOBS}, in the picture are obscured by a test picture.
The last, {\bf ISAMD}, says whether or not the picture is on the same
workstation as a test picture.

\item [Search routines] ---

These eight routines search the database for a picture with a given name, or
which satisfies certain criteria.
The routine names are of the form {\bf RCxy}, where the `x' specifies where to
start the search and the direction of search, and  `y' can be used to specify
that the picture must enclose a given test point.
\end{description}
\end{quote}

\paragraph{Interfacing with other graphics components:}\hfill
\markboth{Programmers}{\stardocname}

The basic routines do not assume the use of any particular graphics
library.
However, in order to make AGI easy to use it is desirable to integrate it
with one or more graphics libraries.
For instance, AGI and the graphics library need to be associated with the
same workstation, and the entity addressed by the graphics library needs
to be associated with the `picture' addressed by AGI.

This integration has been done for SGS, by the addition of four routines which
begin `AGS\_'.
Two of them, {\bf ACTIV} and {\bf DEACT}, are used to initialise and close down
SGS.
The other two are {\bf NZONE}, to create a new SGS zone corresponding to an AGI
picture, and {\bf SZONE}, to save the current SGS zone as a picture in the
database.

This code will open AGI and create an SGS zone corresponding to the current
picture:

\begin{small}
\begin{verbatim}
    CALL AGI_ASSOC('DEVICE', MODE, PICID, STATUS)
    CALL AGS_ACTIV(STATUS)
    CALL AGS_NZONE(ZONE, STATUS)
\end{verbatim}
\end{small}

The associated interface file will contain specifications like these:

\begin{small}
\begin{verbatim}
    parameter  DEVICE
      ptype    'DEVICE'
      default  IKON
      prompt   'Name of display device'
      ...
    endparameter
\end{verbatim}
\end{small}

This will cause a workstation to be associated with parameter DEVICE, and PICID
will point to the current `picture' in the database entry for this workstation.
SGS will be opened using the MODE to decide whether or not to erase the
existing display, and an SGS zone will be opened which will correspond to the
current picture in the database.
You can then use PICID to address the database using AGI and AGS routines,
and ZONE to address the display using SGS routines.

Similar routines exist to interface AGI with PGPLOT and IDI.
These are prefixed by `AGP\_' and `AGD\_' respectively.

Full details of the AGI routines are given in \xref{SUN/48}{sun48}{}, and a
comprehensive example of its use is given in \xref{SUN/95}{sun95}{}.
You can examine the coding of other examples by looking at the DISPLAY and
PHOTOM routines in the KAPPA source library KAPPA\_DIR:KAPPA.TLB.
\markboth{Programmers}{\stardocname}

\section{GNS --- Graphics workstation name service}
\markboth{Programmers}{\stardocname}

Almost any graphics program requires the user to identify which graphics
device is to be used.
When the GKS graphics package is used, graphics devices are identified by two
integers: a `workstation type' and a `connection identifier'.
No one can be expected to remember the workstation types of all available
graphics devices, so a package called GNS has been provided that translates
`friendly' and easy to remember names into their GKS equivalents.

Most high-level graphics libraries, such as SGS, will perform the necessary name
translation when a workstation is opened, so {\em unless you are writing
programs that open GKS workstations directly (by calling {\tt GOPWK})} you will
not need to call any GNS routines yourself.

The GNS library provides three sorts of service to the users and writers of
graphics subroutine libraries and programs:

\begin{itemize}
\item Translating workstation names to their GKS equivalents.
\item Generating a list of the names and types of graphics device available on
 a system.
\item Answering inquiries about the properties of a particular graphics device;
 for example, its category (pen plotter, image display etc.) or its VMS device
 name.
\end{itemize}
More information on GNS and descriptions of its routines are given in
\xref{SUN/57}{sun57}{}.

\paragraph{Demonstration program:}\hfill
\markboth{Programmers}{\stardocname}

A complete list of all the workstation names defined on your system can be
obtained by running the program {\tt GNS\_DIR:GNSRUN}.
After listing all the GKS names along with a short description of each
workstation, the program prompts for a workstation name.
If one of the names in the list (or any other valid workstation name) is
entered, a simple demonstration picture is plotted on the device selected.
The IDI devices are then listed and again a demonstration plot can be
generated by providing an appropriate device name.

If the program produces error messages you should report them to your Site
Manager.
However, you should not necessarily expect all workstations on the list to be
usable.
A device could, for example, be in use by someone else.
