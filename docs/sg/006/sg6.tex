\documentclass[twoside,11pt]{article}

% ? Specify used packages
% ? End of specify used packages

\pagestyle{myheadings}

%------------------------------------------------------------------------------
\newcommand{\stardoccategory}  {Starlink Guide}
\newcommand{\stardocinitials}  {SG}
\newcommand{\stardocnumber}    {6.3}
\newcommand{\stardocauthors}   {R.F. Warren-Smith}
\newcommand{\stardocdate}      {4th February 1994}
\newcommand{\stardoctitle}     {ADAM\\[0.75ex]
                                Programmer's\\
                                Facilities \& Documentation\\
                                Guide}
\newcommand{\stardocabstract}  {
This document contains a directory of the facilities available to the ADAM
programmer. All documented aspects of ADAM programming are covered, ranging from
introductory documentation, through applications programming to real-time and
system programming. Each facility is briefly described along with the type of
work for which it might be used. Most importantly, this guide shows at a glance
where to obtain the relevant documentation.

It is hoped to update this document relatively frequently, so that it serves as
a guide to the latest ADAM developments. Its format, which includes an overall
index, is designed to allow its use as a table of contents for a personal file
of programming documentation.
}


%------------------------------------------------------------------------------
\newcommand{\previousdate}     {7th August 1992}

\newcommand{\stardocname}{\stardocinitials /\stardocnumber}
\markright{\stardocname}
\setlength{\textwidth}{160mm}
\setlength{\textheight}{230mm}
\setlength{\topmargin}{-2mm}
\setlength{\oddsidemargin}{0mm}
\setlength{\evensidemargin}{0mm}
\setlength{\parindent}{0mm}
\setlength{\parskip}{\medskipamount}
\setlength{\unitlength}{1mm}

% -----------------------------------------------------------------------------
%  Hypertext definitions.
%  ======================
%  These are used by the LaTeX2HTML translator in conjunction with star2html.

%  Comment.sty: version 2.0, 19 June 1992
%  Selectively in/exclude pieces of text.
%
%  Author
%    Victor Eijkhout                                      <eijkhout@cs.utk.edu>
%    Department of Computer Science
%    University Tennessee at Knoxville
%    104 Ayres Hall
%    Knoxville, TN 37996
%    USA

%  Do not remove the %begin{latexonly} and %end{latexonly} lines (used by 
%  star2html to signify raw TeX that latex2html cannot process).
%begin{latexonly}
\makeatletter
\def\makeinnocent#1{\catcode`#1=12 }
\def\csarg#1#2{\expandafter#1\csname#2\endcsname}

\def\ThrowAwayComment#1{\begingroup
    \def\CurrentComment{#1}%
    \let\do\makeinnocent \dospecials
    \makeinnocent\^^L% and whatever other special cases
    \endlinechar`\^^M \catcode`\^^M=12 \xComment}
{\catcode`\^^M=12 \endlinechar=-1 %
 \gdef\xComment#1^^M{\def\test{#1}
      \csarg\ifx{PlainEnd\CurrentComment Test}\test
          \let\html@next\endgroup
      \else \csarg\ifx{LaLaEnd\CurrentComment Test}\test
            \edef\html@next{\endgroup\noexpand\end{\CurrentComment}}
      \else \let\html@next\xComment
      \fi \fi \html@next}
}
\makeatother

\def\includecomment
 #1{\expandafter\def\csname#1\endcsname{}%
    \expandafter\def\csname end#1\endcsname{}}
\def\excludecomment
 #1{\expandafter\def\csname#1\endcsname{\ThrowAwayComment{#1}}%
    {\escapechar=-1\relax
     \csarg\xdef{PlainEnd#1Test}{\string\\end#1}%
     \csarg\xdef{LaLaEnd#1Test}{\string\\end\string\{#1\string\}}%
    }}

%  Define environments that ignore their contents.
\excludecomment{comment}
\excludecomment{rawhtml}
\excludecomment{htmlonly}

%  Hypertext commands etc. This is a condensed version of the html.sty
%  file supplied with LaTeX2HTML by: Nikos Drakos <nikos@cbl.leeds.ac.uk> &
%  Jelle van Zeijl <jvzeijl@isou17.estec.esa.nl>. The LaTeX2HTML documentation
%  should be consulted about all commands (and the environments defined above)
%  except \xref and \xlabel which are Starlink specific.

\newcommand{\htmladdnormallinkfoot}[2]{#1\footnote{#2}}
\newcommand{\htmladdnormallink}[2]{#1}
\newcommand{\htmladdimg}[1]{}
\newenvironment{latexonly}{}{}
\newcommand{\hyperref}[4]{#2\ref{#4}#3}
\newcommand{\htmlref}[2]{#1}
\newcommand{\htmlimage}[1]{}
\newcommand{\htmladdtonavigation}[1]{}

% Define commands for HTML-only or LaTeX-only text.
\newcommand{\html}[1]{}
\newcommand{\latex}[1]{#1}

% Use latex2html 98.2.
\newcommand{\latexhtml}[2]{#1}

%  Starlink cross-references and labels.
\newcommand{\xref}[3]{#1}
\newcommand{\xlabel}[1]{}

%  LaTeX2HTML symbol.
\newcommand{\latextohtml}{{\bf LaTeX}{2}{\tt{HTML}}}

%  Define command to re-centre underscore for Latex and leave as normal
%  for HTML (severe problems with \_ in tabbing environments and \_\_
%  generally otherwise).
\newcommand{\setunderscore}{\renewcommand{\_}{{\tt\symbol{95}}}}
\latex{\setunderscore}

%  Redefine the \tableofcontents command. This procrastination is necessary 
%  to stop the automatic creation of a second table of contents page
%  by latex2html.
\newcommand{\latexonlytoc}[0]{\tableofcontents}

% -----------------------------------------------------------------------------
%  Debugging.
%  =========
%  Remove % on the following to debug links in the HTML version using Latex.

% \newcommand{\hotlink}[2]{\fbox{\begin{tabular}[t]{@{}c@{}}#1\\\hline{\footnotesize #2}\end{tabular}}}
% \renewcommand{\htmladdnormallinkfoot}[2]{\hotlink{#1}{#2}}
% \renewcommand{\htmladdnormallink}[2]{\hotlink{#1}{#2}}
% \renewcommand{\hyperref}[4]{\hotlink{#1}{\S\ref{#4}}}
% \renewcommand{\htmlref}[2]{\hotlink{#1}{\S\ref{#2}}}
% \renewcommand{\xref}[3]{\hotlink{#1}{#2 -- #3}}
%end{latexonly}
% -----------------------------------------------------------------------------
% Add any \newcommand or \newenvironment commands here

%\renewcommand{\thefootnote}{\fnsymbol{footnote}}
\newcommand{\changed}{\bf $^*$}
\newcommand{\change}[1]{{\bf #1}}
\newcommand{\newfac}{{\bf \footnotemark[2]}}

\newcounter{headcount}
\newcounter{doccount}[headcount]
\newcommand{\bighead}[1]{
   \stepcounter{headcount}

   \vspace{10mm}
   \makebox[15mm][l] {\Large \bf \theheadcount} {\Large \bf #1}
   \vspace{7mm}

}
\begin{htmlonly}
   \newcommand{\bighead}[1]{
      \stepcounter{headcount}

      \section{#1}
   }
\end{htmlonly}

\newcommand{\docitem}[4]{
   \stepcounter{doccount}
   \vbox{
      \rule{160mm}{0.5mm}\vspace{1mm}
      \makebox[15mm][l]{\large \theheadcount -\thedoccount} {\large \bf #1} \hfill {\large \bf #3}\\
      \makebox[15mm][l]{} {\large #2}\\[-1mm]
      \rule{160mm}{0.1mm}\\[2mm]
      \makebox[15mm][l]{} \parbox{133mm}{
         #4
      }
      \vspace{2mm}
   }

}

\begin{htmlonly}
\newcommand{\docitem}[4]{
   \stepcounter{doccount}
   \vbox{
      \begin{tabular}{p{18mm}p{112mm}p{70mm}} \hline\hline
      \multicolumn{1}{l}{\large \theheadcount -\thedoccount} &
      {\large \bf #1} & 
      \multicolumn{1}{r}{\large \bf #3} \\
      & {\large #2}\\ \hline
      & \multicolumn{2}{p90mm}{#4}
      \end{tabular}
      \vspace{2mm}
   }

}
\end{htmlonly}

%------------------------------------------------------------------------------
\renewcommand{\thepage}{\roman{page}}
\begin{document}

\thispagestyle{empty}

%  Latex document header.
%  ======================

\begin{latexonly}
   SCIENCE \& ENGINEERING RESEARCH COUNCIL \hfill \stardocname\\
   RUTHERFORD APPLETON LABORATORY\\
   {\large\bf Starlink Project\\}
   {\large\bf \stardoccategory\ \stardocnumber}
   \begin{flushright}
   \stardocauthors\\
   \stardocdate
   \end{flushright}
   \vspace{-4mm}
   \rule{\textwidth}{0.5mm}
   \vspace{5mm}
   \begin{center}
   {\Huge\bf  \stardoctitle \\ [3.5ex]}
   {\Large \bf  \stardocdate}
   \end{center}
   \vspace{20mm}

% ? Heading for abstract if used.
   \vspace{10mm}
   \begin{center}
      {\Large\bf Abstract}
   \end{center}
% ? End of heading for abstract.
\end{latexonly}

%  HTML documentation header.
%  ==========================
\begin{htmlonly}
   \xlabel{}
   \begin{rawhtml} <H1> \end{rawhtml}
      \stardoctitle
%      \stardocversion\\
%      \stardocmanual
   \begin{rawhtml} </H1> <HR> \end{rawhtml}

% ? Add picture here if required.
%  \htmladdimg{sg6_cover.gif}
% ? End of picture

   \begin{rawhtml} <P> <I> \end{rawhtml}
   \stardoccategory\ \stardocnumber \\
   \stardocauthors \\
   \stardocdate
   \begin{rawhtml} </I> </P> <H3> \end{rawhtml}
      \htmladdnormallink{CCLRC}{http://www.cclrc.ac.uk} /
      \htmladdnormallink{Rutherford Appleton Laboratory}
                        {http://www.cclrc.ac.uk/ral} \\
      \htmladdnormallink{Science \& Engineering Research Council}
                        {http://www.stfc.ac.uk} \\
   \begin{rawhtml} </H3> <H2> \end{rawhtml}
      \htmladdnormallink{Starlink Project}{http://www.starlink.ac.uk/}
   \begin{rawhtml} </H2> \end{rawhtml}
   \htmladdnormallink{\htmladdimg{source.gif} Retrieve hardcopy}
      {http://www.starlink.ac.uk/cgi-bin/hcserver?\stardocsource}\\

%  HTML document table of contents. 
%  ================================
%  Add table of contents header and a navigation button to return to this 
%  point in the document (this should always go before the abstract \section). 
  \label{stardoccontents}
  \begin{rawhtml} 
    <HR>
    <H2>Contents</H2>
  \end{rawhtml}
  \newcommand{\latexonlytoc}[0]{}
  \htmladdtonavigation{\htmlref{\htmladdimg{contents_motif.gif}}
        {stardoccontents}}

% ? New section for abstract if used.
  \section{\xlabel{abstract}Abstract}
% ? End of new section for abstract
\end{htmlonly}

% -----------------------------------------------------------------------------
% ? Document Abstract. (if used)
%   ==================
\stardocabstract

% ? End of document abstract
% -----------------------------------------------------------------------------
% ? Latex document Table of Contents (if used).
%  ===========================================
% ? End of Latex document table of contents
%------------------------------------------------------------------------------
\newpage
\thispagestyle{empty}
\begin{latexonly}
\vspace*{\fill}
\begin{center}
{\Large This page deliberately left blank}
\end{center}
\vspace*{\fill}
\end{latexonly}

% Start of Index.
% ==============
\newpage
\markright{ADAM Programmer's Index, \stardocdate}
\renewcommand{\thepage}{\roman{page}}
\setcounter{page}{1}

\begin{latexonly}
\null\vspace {5mm}
\begin {center}
\rule{80mm}{0.5mm} \\ [1ex]
{\Large\bf ADAM\\[0.75ex]
           Programmer's\\
           Facilities \& Documentation\\[0.5ex]
           Index}\\[2.5ex]
           {\bf \stardocdate} \\ [2ex]
\rule{80mm}{0.5mm}
\end{center}
\vspace{30mm}
\end{latexonly}

\html{\section{Documentation Index}}
{\Large \bf 1 Introductory Documentation}

\latex{\footnotetext[1]{The documentation for these facilities has
changed since the previous version of this guide (dated
\previousdate). Please ensure that you have a copy of the most recent
documents.}} 
\html{An asterisk indicates that the documentation for
these facilities has changed since the previous version of this guide
(dated \previousdate). Please ensure that you have a copy of the most
recent documents.}

\begin{tabular}{p{8mm}p{27mm}p{90mm}l}
1-1 & INTRO & {\em Introduction to ADAM Programming} & \xref{SUN/101}{sun101}{}\\
1-2\changed & ADAM & {\em The Starlink Software Environment} & \xref{SG/4}{sg4}{}\\
\end{tabular}

\latex{\vspace{2ex}}
{\Large \bf 2 General Applications Programming}

\begin{tabular}{p{8mm}p{27mm}p{90mm}l}
2-1\changed & PAR & {\em ADAM Parameter System Routines} & \xref{SUN/114}{sun114}{}\\
2-2 & NDF & {\em Extensible N-Dimensional Data Format} & \xref{SUN/33}{sun33}{}\\
2-3\changed & MSG \& ERR & {\em Message and Error Reporting} & \xref{SUN/104}{sun104}{}\\
2-4 & CHR & {\em Character Handling} & \xref{SUN/40}{sun40}{}\\
2-5\changed & PGPLOT & {\em Graphics Library} & \xref{SUN/15}{sun15}{}\\
2-6 & SGS & {\em Simple Graphics System} & \xref{SUN/85}{sun85}{}\\
2-7\changed & FIO \& RIO & {\em File Input/Output} & \xref{SUN/143}{sun143}{}\\
2-8 & MAG & {\em Magnetic Tape Handling} & \xref{SUN/2}{sun2}{}\\
2-9 & PRIMDAT & {\em Primitive Numerical Data Processing} & \xref{SUN/39}{sun39}{}\\
2-10 & ICL & {\em ADAM Command Language} & \xref{SG/5}{sg5}{}\\
2-11 & IFL & {\em ADAM Interface Module Reference Manual} & \xref{SUN/115}{sun115}{}\\
\end{tabular}

\latex{\newpage}
\latex{\vspace{2ex}}
{\Large \bf 3 More-specialised Facilities}

\footnotetext[1]{The documentation for these facilities has changed since the
previous version of this guide (dated \previousdate). Please ensure that you
have a copy of the most recent documents.}
\footnotetext[2]{New facility.}
\begin{tabular}{p{8mm}p{27mm}p{90mm}l}
3-1 & SLALIB & {\em Positional Astronomy and Time} & \xref{SUN/67}{sun67}{}\\
3-2 & NCAR & {\em Graphics Utilities} & \xref{SUN/88}{sun88}{}\\
3-3 & SNX & {\em Starlink Extensions to NCAR} & \xref{SUN/90}{sun90}{}\\
3-4\changed & AGI & {\em Applications Graphics Interface} & \xref{SUN/48}{sun48}{}\\
3-5 & PSX & {\em POSIX Interface Routines} & \xref{SUN/121}{sun121}{}\\
3-6\changed & REF & {\em References to HDS Objects} & \xref{SUN/31}{sun31}{}\\
3-7 & TRANSFORM & {\em Coordinate Transformation Facility} & \xref{SUN/61}{sun61}{}\\
3-8\changed & HELP & {\em Help Text Retrieval System} & \xref{SUN/124}{sun124}{}\\
\end{tabular}

\latex{\vspace{2ex}}
{\Large \bf 4 Lower-level Facilities}

\begin{tabular}{p{8mm}p{27mm}p{90mm}l}
4-1 & ARY & {\em Access to ARRAY Data Structures} & \xref{SUN/11}{sun11}{}\\
4-2 & HDS & {\em Hierarchical Data System} & \xref{SUN/92}{sun92}{}\\
4-3\changed & IDI & {\em Image Display Interface} & \xref{SUN/65}{sun65}{}\\
4-4 & GKS & {\em Graphical Kernel System} & \xref{SUN/83}{sun83}{}\\
4-5\newfac & GRP & {\em Routines for Managing Groups of Objects} & \xref{SUN/150}{sun150}{}\\
4-6 & GWM & {\em X Graphics Window Manager} & \xref{SUN/130}{sun130}{}\\
4-7\changed & GNS & {\em Graphics Workstation Name Service} & \xref{SUN/57}{sun57}{}\\
\end{tabular}

\latex{\vspace{2ex}}
{\Large \bf 5 Programming Tools}

\begin{tabular}{p{8mm}p{27mm}p{90mm}l}
5-1 & STARLSE & {\em Starlink Language Sensitive Editor} & \xref{SUN/105}{sun105}{}\\
5-2 & SST & {\em Simple Software Tools} & \xref{SUN/110}{sun110}{}\\
5-3\changed & GENERIC & {\em Utility for Compiling Generic Fortran} & \xref{SUN/7}{sun7}{}\\
\end{tabular}

\latex{\vspace{2ex}}
{\Large \bf 6 Instrumentation Programming}

\begin{tabular}{p{8mm}p{27mm}p{90mm}l}
6-1 & TASK & {\em Guide to Writing ADAM Instrumentation Tasks} & \xref{SUN/134}{sun134}{}\\
6-2\changed & NBS & {\em Noticeboard System} & \xref{SUN/77}{sun77}{}\\
6-3 & MSP & {\em Message System Primitive Routines} & \xref{SSN/2}{ssn2}{}\\
\end{tabular}

\latex{\vspace{2ex}}
{\Large \bf 7 System Programming}

\begin{tabular}{p{8mm}p{27mm}p{90mm}l}
7-1 & EMS & {\em Error Message Service} & \xref{SSN/4}{ssn4}{}\\
7-2 & CNF \& F77 & {\em Mixed Language Programming} & \xref{SGP/5}{sgp5}{}\\
\end{tabular}

\latex{\vspace{2ex}}
{\Large \bf 8 Standards and Conventions}

\begin{tabular}{p{8mm}p{27mm}p{90mm}l}
8-1 & FORTRAN & {\em Application Programming Standard} & \xref{SGP/16}{sgp16}{}\\
8-2 & C & {\em Starlink C Programming Standard} & \xref{SGP/4}{sgp4}{}\\
8-3 & DATA & {\em Starlink Standard Data Structures} & \xref{SGP/38}{sgp38}{}\\
8-4 & PACKAGES & {\em Organisation of ADAM Applications Packages} & \xref{SSN/64}{ssn64}{}\\
8-5 & LIBRARIES & {\em Conventions for Accessing Starlink Libraries} & \xref{SSN/8}{ssn8}{}\\
8-6 & UNIX & {\em Starlink Software Organisation on UNIX} & \xref{SSN/66}{ssn66}{}\\
\end{tabular}

% Start of Guide.
% ==============
\newpage
\renewcommand{\thepage}{\arabic{page}}
\setcounter{page}{1}
\markright{ADAM Programmer's Guide, \stardocdate}

\begin{latexonly}
\null\vspace {5mm}
\begin {center}
\rule{80mm}{0.5mm} \\ [1ex]
{\Large\bf ADAM\\[0.75ex]
           Programmer's\\
           Facilities \& Documentation\\[0.5ex]
           Guide}\\[2.5ex]
           {\bf \stardocdate} \\ [2ex]
\rule{80mm}{0.5mm}
\end{center}
\vspace{25mm}
\end{latexonly}

\bighead{Introductory Documentation}

\docitem{INTRO}{Introduction to ADAM Programming}{\xref{SUN/101.2}{sun101}{}}{ 
A tutorial introduction to writing ADAM programs. This document is intended for
newcomers to ADAM who already have a basic understanding of Fortran programming
but are unfamiliar with ADAM conventions. The use of the most important
facilities listed in this guide is illustrated, and a set of
example programs is provided.
}

\docitem{ADAM}{The Starlink Software Environment}{\xref{SG/4.2}{sgp4}{}}{
This document provides the most comprehensive overview of ADAM from the
perspective of a typical Starlink user, covering both the running of existing
applications and the writing of new ones. It contains essential introductory
information for those new to ADAM, as well as reference material for the more
experienced. Since it is a large document, it cannot be updated very frequently,
so reference should be made to the other more specialised documents in this
guide if up-to-the-minute information is needed on a particular topic.
}\mbox{}\\
\rule{160mm}{0.5mm}

\newpage
\bighead{General Applications Programming}

\docitem{PAR}{ADAM Parameter System Routines}{\xref{SUN/114.1}{sun114}{}}{
Provides access to ``{\em parameters\/}'' from within applications. These are
the values which provide external control over an application (broadly similar
to command line switches in UNIX, or command qualifiers in VMS). The PAR library
allows these values to be obtained from the command line, by prompting the user,
or by a variety of other mechanisms. \change{The latest version of PAR includes
a complete revision of the documentation and the addition of a new set of
routines which permit parameter values to be obtained subject to various
constraints.}
}
\docitem{NDF}{Extensible N-Dimensional Data Format}{\xref{SUN/33.3}{sun33}{}}{
The Extensible N-Dimensional Data Format (NDF) is the Starlink format for
storing bulk data in the form of $N$-dimensional arrays of numbers. It is
typically used for storing spectra, images and similar datasets with higher
dimensionality. The NDF format is based on the {\em Hierarchical Data System},
HDS (see \xref{SUN/92}{sun92}{}), and is extensible; not only does it provide a comprehensive
set of standard ancillary items to describe the data, it can also be extended
indefinitely to handle additional user-defined information of any type. SUN/33
describes the routines provided for accessing NDF data objects. It also
discusses all the important NDF concepts and includes a selection of simple
example applications.
}
\docitem{MSG \& ERR}{Message and Error Reporting}{\xref{SUN/104.5}{sun104}{}}{
These provide a means for sending simple messages to the user of an application,
together with a mechanism for reporting errors. They address many of the common
problems encountered when formatting and displaying text through their use of
``{\em message tokens\/}'', a concept which is also exploited by a number of
other ADAM libraries. Most importantly, SUN/104 explains the ideas underlying
the error handling scheme and ``{\em inherited status checking\/}'' used by the
bulk of Starlink software.
}
\docitem{CHR}{Character Handling}{\xref{SUN/40.3}{sun40}{}}{
This library provides a simple way to perform many commonly required operations
on Fortran character variables. These include: formatting and decoding numerical
values, case conversion, analysing lists, string searching and character
testing. This library is currently being upgraded to include a substantial
number of new routines.
}

% 2008 March 30 (MJC)
% Cannot think of a way to achieve the original paper documentation
% appearance when there is more than one document associated with the
% software item, and obtain the desired effect in the hypertext too.
% So the paragraph description is repeated in latex and html versions.
% Any modifications to the description must be made in both versions.
% It's not possible to wrap up just the differing \docitem lines.

\begin{latexonly}
\docitem{PGPLOT}{Graphics Library \hfill \xref{\large \bf SUN/15.5}{sun15}{}\\
                          \mbox{} \hfill \xref{\large \bf SUN113.2}{sun113}{}}
        {\em PGPLOT Manual\/}{
If you are unsure which high-level graphics library to use, then this is
probably the one for you. It caters for most simple graph-plotting needs in a
straightforward way (although it may not satisfy the more demanding user, for
whom NCAR may be preferable -- see \xref{SUN/88}{sun88}{}). The PGPLOT manual, entitled ``{\em
PGPLOT -- Graphics Subroutine Library\/}'', is available from Starlink site
managers in the {\em Miscellaneous User Documents\/} (MUD) series. SUN/15
describes the PGPLOT implementation on Starlink and shows how it integrates with
SGS (see SUN/85). Note that the {\em Applications Graphics Interface} AGI also
provides special routines for ease of use with PGPLOT (see
\xref{SUN/48}{sun48}{}).
}
\end{latexonly}
\begin{htmlonly}
\docitem{PGPLOT}{Graphics Library}{{\em PGPLOT Manual\/}\\
                                   \xref{SUN/15.5}{sun15}{}\\
                                   \xref{SUN113.2}{sun113}{}}{
If you are unsure which high-level graphics library to use, then this is
probably the one for you. It caters for most simple graph-plotting needs in a
straightforward way (although it may not satisfy the more demanding user, for
whom NCAR may be preferable -- see \xref{SUN/88}{sun88}{}). The PGPLOT manual, entitled ``{\em
PGPLOT -- Graphics Subroutine Library\/}'', is available from Starlink site
managers in the {\em Miscellaneous User Documents\/} (MUD) series. SUN/15
describes the PGPLOT implementation on Starlink and shows how it integrates with
SGS (see SUN/85). Note that the {\em Applications Graphics Interface} AGI also
provides special routines for ease of use with PGPLOT (see
\xref{SUN/48}{sun48}{}).
}
\end{htmlonly}
\begin{latexonly}
\docitem{SGS}{Simple Graphics System \hfill \xref{\large\bf SUN/113.2}{sun113}{}}{\xref{SUN/85.5}{sun85}{}}{
SGS provides an easy-to-use interface to the {\em Graphical Kernel System}, GKS
(see \xref{SUN/83}{sun83}{}). GKS is the underlying system which provides graphics ``{\em
device-independence\/}'' but it is too low-level for general use, so calls to
SGS routines may be used instead to perform simple plotting operations which do
not amount to complete graph drawing. SGS is particularly useful for managing
the layout of plotting surfaces which are to be filled with graphs drawn by
higher-level routines. SUN/85 describes the ``{\em stand-alone\/}'' version of
SGS, while SUN/113  describes additional routines which connect SGS to ADAM. The
{\em Applications Graphics Interface} AGI also provides special routines for
ease of use with SGS (see SUN/48).
}
\end{latexonly}
\begin{htmlonly}
\docitem{SGS}{Simple Graphics System}{\xref{SUN/85.5}{sun85}{}\\
                                      \xref{SUN/113.2}{sun113}{}}{
SGS provides an easy-to-use interface to the {\em Graphical Kernel System}, GKS
(see \xref{SUN/83}{sun83}{}). GKS is the underlying system which provides graphics ``{\em
device-independence\/}'' but it is too low-level for general use, so calls to
SGS routines may be used instead to perform simple plotting operations which do
not amount to complete graph drawing. SGS is particularly useful for managing
the layout of plotting surfaces which are to be filled with graphs drawn by
higher-level routines. SUN/85 describes the ``{\em stand-alone\/}'' version of
SGS, while SUN/113  describes additional routines which connect SGS to ADAM. The
{\em Applications Graphics Interface} AGI also provides special routines for
ease of use with SGS (see SUN/48).
}
\end{htmlonly}
\docitem{FIO \& RIO}{File Input/Output}{\xref{SUN/143}{sun143}{}}{
Provides a set of routines to facilitate the reading and writing of Fortran
files. These routines are not intended to replace the portable features of
standard Fortran~77, but to help integrate these features with ADAM. They also
hide some of the unavoidable, but non-portable, features of Fortran
input/output. \change{The latest version of FIO \& RIO includes a complete
revision of the documentation and new facilities to assist with error reporting
and the portable classification of I/O error status values. Some rationalisation
of routine naming has also occurred.}
}
\docitem{MAG}{Magnetic Tape Handling}{\xref{SUN/2}{sun2}{}}{
Provides routines for accessing magnetic tapes and controlling tape drive
operations {\em via\/} the ADAM parameter system. A feature is incorporated to
minimise tape movement between applications by keeping track of the absolute
tape position.
}
\docitem{PRIMDAT}{Primitive Numerical Data Processing}{\xref{SUN/39.1}{sun39}{}}{
Provides arithmetic and mathematical functions for processing primitive
numerical data (real, integer \& double precision values, {\em etc.}) including
the non-standard data types ({\em e.g.}\ unsigned byte) which are used in ADAM
but are not directly supported by Fortran~77. Routines are provided for both
scalar values and arrays, with optional recognition of ``{\em undefined\/}'' 
data and methods for handling numerical errors such as overflow. A set of
include files is also provided to define machine-dependent constants, including
the important ``{\em bad\/}'' values used to flag undefined data. The systematic
naming of files and routines in this library is designed to integrate with the
GENERIC pre-processor (see \xref{SUN/7}{sun7}{}).
}
\docitem{ICL}{ADAM Command Language}{\xref{SG/5.1}{sg5}{}}{
Although ICL is really a command language from which applications are run, the
writer of ADAM applications will still find it useful to refer to this 
document. It describes the ICL language, which includes a number of special
features for controlling and communicating with ADAM tasks.
}
\docitem{IFL}{ADAM Interface Module Reference Manual}{\xref{SUN/115.1}{sun115}{}}{
An ``{\em interface module\/}'' (or ``{\em interface file\/}'') acts as an
interface between an application and its user. It describes to the ADAM
parameter system (see \xref{SUN/114}{sun114}{}) how to obtain values and other information ({\em
e.g.}\ help text) for an application's parameters, and hence influences, in
broad terms, how the application behaves. This document provides reference
information describing in detail how to write an interface module.
}\mbox{}\\
\rule{160mm}{0.5mm}


\newpage
\bighead{More-specialised Facilities}

\docitem{SLALIB}{Positional Astronomy and Time}{\xref{SUN/67.17}{sun67}{}}{
SLALIB (the name just stands for ``{\em Subprogram Library A\/}'') contains a
large number of routines mainly concerned with positional astronomy and time.
Some also have wider trigonometrical, numerical or general significance, while
others are essentially miscellaneous. Its facilities include: string decoding;
sexagesimal conversions; handling of angles, vectors and rotation matrices;
calendar and timescale calculations; precession, nutation and proper-motion
calculations; celestial coordinate conversions and astrometric transformations.
}
\begin{latexonly}
\docitem{NCAR}{Graphics Utilities \hfill \xref{\large\bf SUN/88.4}{sun88}{}}
              {NCAR Manual}{
NCAR is a large and sophisticated set of high-level plotting routines developed
at the National Centre for Atmospheric Research at Boulder, Colorado. It can do
more things than most other plotting packages and is reasonably straightforward
to use. The ``{\em NCAR Manual\/}'' is a large document in the {\em
Miscellaneous User Documents\/} (MUD) series, available from Starlink site
managers. SUN/88 provides a concise summary of the NCAR facilities and describes
the Starlink release of this system.
}
\end{latexonly}
\begin{htmlonly}
\docitem{NCAR}{Graphics Utilities}{NCAR Manual\\
                                   \xref{SUN/88.4}{sun88}{}}{
NCAR is a large and sophisticated set of high-level plotting routines developed
at the National Centre for Atmospheric Research at Boulder, Colorado. It can do
more things than most other plotting packages and is reasonably straightforward
to use. The ``{\em NCAR Manual\/}'' is a large document in the {\em
Miscellaneous User Documents\/} (MUD) series, available from Starlink site
managers. SUN/88 provides a concise summary of the NCAR facilities and describes
the Starlink release of this system.
}
\end{htmlonly}
\docitem{SNX}{Starlink Extensions to NCAR}{\xref{SUN/90.8}{sun90}{}}{
Since NCAR uses GKS, calls to its routines may be interspersed with calls to
other GKS-based graphics libraries. To do this effectively, however, it is
usually necessary to know about such things as the NCAR coordinate system, or to
arrange for an NCAR plot to appear within a specified region of the display
surface. The SNX routines provide a bridge between NCAR and SGS which allows
this type of interaction.
}
\docitem{AGI}{Applications Graphics Interface}{\xref{SUN/48.5}{sun48}{}}{
AGI is a Graphics Database system which allows graphics applications to record
information about the plots they produce so that subsequent applications can
make use of it. For example, information about the coordinate system of a graph
might be recorded so that a subsequent application could use a cursor to read
off positions. AGI allows a considerable amount of information to be stored,
ranging from simple comments to non-linear coordinate transformations and
references to HDS objects. It also allows graphics devices to be divided into
separate ``{\em pictures\/}'' so that multiple plots may appear, each with their
own graphics information. AGI has special interfaces for ease of use with the
PGPLOT, SGS and IDI graphics libraries.
}
\docitem{PSX}{POSIX Interface Routines}{\xref{SUN/121.1}{sun121}{}}{
POSIX is an IEEE standard defining an interface to a ``{\em virtual operating
system\/}'' which applications may use to interact with their host machine in a
portable manner. For example, POSIX could be used to determine the current
``{\em user-name\/}'' in a manner which does not vary between machines.
Unfortunately, the POSIX standard currently only provides for calls from
programs written in C (the Fortran interface definition is not yet complete), so
the PSX library provides a solution by making some of the more useful POSIX
functions available in Fortran-callable form. It provides an ADAM-style
interface to POSIX, including inherited status checking and error reporting.
}
\docitem{REF}{References to HDS Objects}{\xref{SUN/31.4}{sun31}{}}{
It is sometimes useful to use HDS to store pointers to other HDS objects. For
instance, the {\em Applications Graphics Interface}, AGI, depends on this
ability in order to associate HDS data objects with graphical displays (see
\xref{SUN/48}{sun48}{}), without having to make a separate copy of the data object. The REF
library is provided to facilitate this data-object ``{\em referencing\/}'' 
process and the subsequent accessing of objects which have been referenced in
this way.
}
\docitem{TRANSFORM}{Coordinate Transformation Facility}{\xref{SUN/61.2}{sun61}{}}{
A specialised facility which allows mappings (or coordinate transformations) to
be formulated and stored as HDS objects. These transformations may later be
recovered and evaluated. This facility is used by the {\em Applications Graphics
Interface}, AGI, to implement world-to-data coordinate transformations (see
SUN/48). TRANSFORM also incorporates a facility for parsing and evaluating
Fortran-like arithmetic expressions, a feature which can sometimes be used to
good effect in applications software.
}
\docitem{HELP}{Help Text Retrieval System}{\xref{SUN/124.6}{sun124}{}}{
A subroutine library designed for implementing interactive hierarchical help
systems. It is modelled on the VAX/VMS help system but has a number of
additional features, including the ability for ``{\em help files\/}'' to refer
to other help files. This library will mainly be of interest to those writing
ADAM system software, but this does not preclude its use in applications which
wish to handle their own help information explicitly.
}\mbox{}\\
\rule{160mm}{0.5mm}


\newpage
\bighead{Lower-level Facilities}

\docitem{ARY}{Access to Array Data Structures}{\xref{SUN/11.2}{sun11}{}}{
The HDS-based ARRAY data structure is one of the building blocks from which the
Extensible N-Dimensional Data Format, or NDF (see SUN/33), is constructed. The
ARY library provides access to these structures and is used extensively by the
NDF access routines. ARRAY structures can also be useful in their own right as a
more sophisticated alternative to primitive HDS arrays for storing data. They
can be particularly valuable when used in NDF ``{\em extensions\/}'' to extend
the NDF definition by adding arrays of additional information.
}
\begin{latexonly}
\docitem{HDS}{Hierarchical Data System \hfill {\large\bf SUN/224.2}}{\xref{SUN/92.8}{sun92}{}}{
HDS is the low-level data system upon which other important data structures,
such as the NDF (see SUN/33), are built. This document describes routines for
accessing these structures at the lowest level. Calls to HDS routines may be
intermixed with those to other higher-level data access routines, and are also
invaluable for creating and accessing data structures of your own design (such
as NDF extensions). All data structures created by HDS are portable between the
machines on which HDS is implemented. SUN/92 describes the ``{\em
stand-alone\/}'' version of HDS, while SUN/227 describes additional routines which
connect HDS to ADAM.
}
\end{latexonly}
\begin{htmlonly}
\docitem{HDS}{Hierarchical Data System}{\xref{SUN/92.8}{sun92}{}\\
                                        \xref{SUN/224.2}{sun224}{}}{
HDS is the low-level data system upon which other important data structures,
such as the NDF (see SUN/33), are built. This document describes routines for
accessing these structures at the lowest level. Calls to HDS routines may be
intermixed with those to other higher-level data access routines, and are also
invaluable for creating and accessing data structures of your own design (such
as NDF extensions). All data structures created by HDS are portable between the
machines on which HDS is implemented. SUN/92 describes the ``{\em
stand-alone\/}'' version of HDS, while SUN/227 describes additional routines which
connect HDS to ADAM.
}
\end{htmlonly}
\begin{latexonly}
\docitem{IDI}{Image Display Interface \hfill \xref{\large\bf SUN/65.7}{sun65}{}}
{\em Astron. Astrophys. paper\/}{
IDI is a standard for the display and manipulation of data on image displays in
astronomy. Unlike the other graphics packages, IDI is not based on GKS, so calls
to IDI cannot be intermixed with other graphics calls. However, IDI has the
advantage of providing facilities for interactive manipulation of image displays
({\em e.g.}\ panning, zooming and blinking) which GKS lacks. The definitive
document on IDI is ``{\em An image display interface for astronomical image
processing\/}'' by Terrett {\em et al.}\ (1988), Astron. Astrophys. Suppl. Ser.,
{\bf 76,} 263-304. This paper describes the IDI routines in detail but without
reference to a specific implementation. SUN/65 describes the Starlink
implementation of IDI, which includes some additional routines to connect it
with ADAM. The {\em Applications Graphics Interface} AGI also contains special
routines for ease of use with IDI (see \xref{SUN/48}{sun48}{}).
}
\end{latexonly}
\begin{htmlonly}
\docitem{IDI}{Image Display Interface}{{\em Astron. Astrophys. paper\/}\\
                                       \xref{SUN/65.7}{sun65}{}}{
IDI is a standard for the display and manipulation of data on image displays in
astronomy. Unlike the other graphics packages, IDI is not based on GKS, so calls
to IDI cannot be intermixed with other graphics calls. However, IDI has the
advantage of providing facilities for interactive manipulation of image displays
({\em e.g.}\ panning, zooming and blinking) which GKS lacks. The definitive
document on IDI is ``{\em An image display interface for astronomical image
processing\/}'' by Terrett {\em et al.}\ (1988), Astron. Astrophys. Suppl. Ser.,
{\bf 76,} 263-304. This paper describes the IDI routines in detail but without
reference to a specific implementation. SUN/65 describes the Starlink
implementation of IDI, which includes some additional routines to connect it
with ADAM. The {\em Applications Graphics Interface} AGI also contains special
routines for ease of use with IDI (see \xref{SUN/48}{sun48}{}).
}
\end{htmlonly}
\begin{latexonly}
\docitem{GKS}{Graphical Kernel System \hfill \xref{\large\bf SUN/83.12}{sun83}{}\\
                              \mbox{} \hfill \xref{\large\bf SUN/113.2}{sun113}{}}
             {\em RAL GKS Manual\/}{
This is the low-level graphics system which underlies all the other ADAM
graphics libraries (except IDI) and which provides ``{\em device
independence\/}'', allowing these libraries to access a large range of graphics
devices. GKS is an international standard, but its concepts are rather abstract
and inconveniently low-level for most simple graphical work. SGS (see SUN/85)
provides a simpler interface to GKS, but direct GKS calls are often useful and
can, with care, be intermixed with calls to other graphics routines. The primary
document on GKS is the ``{\em RAL GKS Manual\/}'', available in the {\em
Miscellaneous User Documents\/} (MUD) series from Starlink site managers. In
addition, SUN/83 gives details of the Starlink GKS release and hardware
implementations, while SUN/113 describes additional routines which connect GKS
(and SGS) with ADAM.
}
\end{latexonly}
\begin{htmlonly}
\docitem{GKS}{Graphical Kernel System}{{\em RAL GKS Manual\/}\\
                                        \xref{SUN/83.12}{sun83}{}\\
                                        \xref{SUN/113.2}{sun113}{}}{
This is the low-level graphics system which underlies all the other ADAM
graphics libraries (except IDI) and which provides ``{\em device
independence\/}'', allowing these libraries to access a large range of graphics
devices. GKS is an international standard, but its concepts are rather abstract
and inconveniently low-level for most simple graphical work. SGS (see SUN/85)
provides a simpler interface to GKS, but direct GKS calls are often useful and
can, with care, be intermixed with calls to other graphics routines. The primary
document on GKS is the ``{\em RAL GKS Manual\/}'', available in the {\em
Miscellaneous User Documents\/} (MUD) series from Starlink site managers. In
addition, SUN/83 gives details of the Starlink GKS release and hardware
implementations, while SUN/113 describes additional routines which connect GKS
(and SGS) with ADAM.
}
\end{htmlonly}
\docitem{GRP}{Routines for Managing Groups of Objects}{\xref{SUN/150.3}{sun150}{}}{
GRP provides facilities for managing groups of character strings which are the
``names'' of objects such as data files, galaxies, or even sets of coordinates.
A typical use might be within a package of co-operating applications which are
to process multiple datasets specified by a list of file names. GRP allows such
applications to access these names very flexibly -- by accepting explicitly
typed values, by reading them from a file (indirection) or by editing an
existing list (modification). It also supports the passing of groups of objects
between separate applications. GRP is a configurable low-level facility and it
is expected that higher-level interfaces specialised to particular types of
object will be provided in future.
}
\docitem{GWM}{X Graphics Window Manager}{\xref{SUN/130.2}{sun130}{}}{
GWM provides facilities for creating and controlling ``{\em persistent\/}''
graphics windows in an X windows environment. The windows it creates are
intended for use as ``{\em virtual displays\/}'' for GKS- and IDI-based
graphics; unlike other X windows, they do not disappear when an application
terminates, so the plot remains visible and may be accessed by subsequent
applications. GWM allows control over graphics windows both from the command
line and {\em via\/} a Fortran programming interface, so that applications may
create and destroy graphics windows for use both by themselves and by other
applications.
}
\docitem{GNS}{Graphics Workstation Name Service}{\xref{SUN/57.7}{sun157}{}}{
An important feature of ADAM graphics is its ``{\em device independence\/}'',
which means that graphical applications can generally be used with a wide range
of graphics devices. GNS provides the services needed for managing the names
which refer to these devices. It allows a local database of easy-to-remember
names to be set up, and then provides applications with the information they
require about each named graphics device. GNS is used by all the high-level
graphics libraries, so direct calls to GNS routines will seldom be necessary.
}\mbox{}\\ \rule{160mm}{0.5mm}


\newpage
\bighead{Programming Tools}

\docitem{STARLSE}{Starlink Language Sensitive Editor}{\xref{SUN/105.5}{sun105}{}}{
STARLSE is based on the VAX Language Sensitive Editor LSE. It exploits the
extensibility of LSE to provide a Language Sensitive Editor with special
facilities for the ADAM programmer. These include definitions of the calling
sequences for most of the libraries which appear in this guide, together with
on-line help information. Standard prologue templates are also provided, with
information being filled in automatically by the editor in many cases. These
prologues may be processed into various forms of documentation using
applications from the SST package (see \xref{SUN/110}{sun110}{}).
}
\docitem{SST}{Simple Software Tools}{\xref{SUN/110.1}{sun110}{}}{
Simple tools for performing common operations on software, implemented as a
package of ADAM applications. The most important feature of this package is the
ability to process prologue information formatted using STARLSE (see SUN/105)
into documentation in the form of \LaTeX\ documents, help libraries, or LSE
subroutine-library definitions. The resulting subroutine-library definitions may
then be used with STARLSE.
}
\docitem{GENERIC}{Utility for Compiling Generic Fortran}{\xref{SUN/7.4}{sun7}{}}{
GENERIC is a utility which pre-processes ``{\em generic\/}'' Fortran code into
multiple versions which are capable of processing different data types (all the
primitive numeric types are supported together with character and logical
values). It works by detecting special tokens embedded in the input file.
GENERIC integrates with the PRIMDAT package (see \xref{SUN/39}{sun39}{}) which provides
consistently named routines for performing operations on primitive data.
}
\docitem{SPT}{Software Porting Tools}{\xref{SUN/111.2}{sun111}{}}{
A package of tools to assist when moving software between different hardware
platforms. At present there are two SPT tools available; a utility for moving
Fortran~77 software between VAX/VMS and UNIX systems (it performs those
unavoidable conversions which are needed in this situation, including the
translation of include file names) and a utility for defining ``soft links'' to
standard Starlink include files on UNIX systems.
}\mbox{}\\
\rule{160mm}{0.5mm}

\newpage
\bighead{Instrumentation Programming}

\docitem{TASK}{Guide to Writing ADAM Instrumentation Tasks}{\xref{SUN/134.2}{sun134}{}}{
An introductory guide to writing ADAM ``{\em real-time\/}'' instrumentation
tasks. This document includes a description of the ADAM tasking model and
develops real-time concepts through a series of examples. It also describes the
library of ADAM TASK routines.
}
\docitem{NBS}{Noticeboard System}{\xref{SUN/77.2}{sun77}{}}{
NBS provides a mechanism for the rapid transfer of large quantities of data
between processes running on the same processor. It utilises shared memory in
which a hierarchical data structure (a ``{\em noticeboard\/}'') may be built and
subsequently written and read by co-operating processes. Noticeboards are
typically used in data-acquisition environments, but this does not preclude
their use for other purposes. The data-structuring facilities of NBS are broadly
similar to those of HDS (see \xref{SUN/92}{sun92}{}), although noticeboards have a static
structure and are volatile.
}
\docitem{MSP}{Message System Primitive Routines}{\xref{SSN/2.1}{ssn2}{}}{
These routines provide the low-level message-passing services upon which
inter-task communication within ADAM depends. They will normally only be of use
to ADAM system programmers or those developing real-time ADAM software.
}\mbox{}\\
\rule{160mm}{0.5mm}

\newpage
\bighead{System Programming}

\docitem{EMS}{Error Message Service}{\xref{SSN/4.4}{ssn4}{}}{
EMS is a set of low-level routines which implement the facilities required by
the ERR and MSG routines (see \xref{SUN/104}{sun104}{}). This separate interface to the lower
levels is provided for those writing system software which does not require the
ability to interact with a user, but simply needs to report and manage error
messages. In particular, EMS should be used in preference to ERR/MSG by those
developing ADAM system software residing in, or beneath, the parameter system.
}
\docitem{CNF \& F77}{Mixed Language Programming}{\xref{SGP/5.1}{sgp5}{}}{
Provides advice, routines and macros to help you pass information between code
written in Fortran~77 and C (in both directions) in a machine-independent way.
}\mbox{}\\
\rule{160mm}{0.5mm}


\newpage
\bighead{Standards and Conventions}

\docitem{FORTRAN}{Application Programming Standard}{\xref{SGP/16.10}{sgp16}{}}{
Describes the Starlink Fortran programming standard, which exists to promote
portability and maintainability in Fortran~77 software. It specifies use of the
Fortran~77 language standard plus a small number of approved extensions. It also
contains a wealth of advice on good programming style and how to avoid common
pitfalls.
}
\docitem{C}{Starlink C Programming Standard}{\xref{SGP/4.1}{sgp4}{}}{
Written in a similar style to the Fortran standard (SGP/16), this programming
standard for the C language also exists to promote portability and
maintainability. Although C is not recommended as an applications language, it
is being used to an increasing extent for real-time and system software. C has a
far greater potential for danger than Fortran, especially given the variety of
compilers available. This document guides you through the C minefield to help
you produce good software which will run on a wide range of computer systems.
}
\docitem{DATA}{Starlink Standard Data Structures}{\xref{SGP/38.2}{sgp38}{}}{
An extensive document describing the philosophy behind the use of HDS for
building general-purpose data structures. It also gives a full (and gory)
description of the {\em Extensible N-Dimensional Data Format}, or NDF. The
detail in this document is only likely to be of interest to those developing
related data-access software. However, the general sections covering the use and
design of data structures would be valuable reading for anyone planning
extensive use of HDS.
}
\docitem{PACKAGES}{Organisation of ADAM Applications Packages}{\xref{SSN/64.2}{ssn64}{}}{
Describes how separate ADAM applications packages are organised for release. It
covers such things as the provision of standard ``{\em startup\/}'' files, the
structure of help libraries, introductory help information, {\em etc.} This is
essential reading for anyone planning to release a new package of ADAM
applications {\em via\/} Starlink.
}
\docitem{LIBRARIES}{Conventions for Accessing Starlink Libraries (VMS)}
{SSN/8.1}{
This document will mainly be of interest to those developing new subroutine
libraries. It covers the facilities which a library should provide, the use of
shareable libraries on VMS, and the naming scheme through which callers of the
library should be able to access it. This document also describes the
distinction between the ADAM and ``{\em stand-alone\/}'' versions of a library.
}
\docitem{UNIX}{Starlink Software Organisation on UNIX}{\xref{SSN/66.2}{ssn62}{}}{
Describes how Starlink software is distributed and installed on UNIX machines.
This also covers access to subroutine libraries and their ``{\em include
files\/}'' on UNIX systems. The final part of the document covers ``{\em make
files\/}'' and the methods used for the building of UNIX executables.
}\mbox{}\\
\rule{160mm}{0.5mm}

\end{document}
