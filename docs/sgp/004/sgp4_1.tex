\documentstyle{article} 
\pagestyle{myheadings}

%------------------------------------------------------------------------------
\newcommand{\stardoccategory}  {Starlink General Paper}
\newcommand{\stardocinitials}  {SGP}
\newcommand{\stardocnumber}    {4.1}
\newcommand{\stardocauthors}   {Jo Murray}
\newcommand{\stardocdate}      {9th May 1991}
\newcommand{\stardoctitle}     {Starlink C Programming Standard}
%------------------------------------------------------------------------------

\newcommand{\stardocname}{\stardocinitials /\stardocnumber}
\markright{\stardocname}
\setlength{\textwidth}{160mm}
\setlength{\textheight}{240mm}
\setlength{\topmargin}{-5mm}
\setlength{\oddsidemargin}{0mm}
\setlength{\evensidemargin}{0mm}
\setlength{\parindent}{0mm}
\setlength{\parskip}{\medskipamount}
\setlength{\unitlength}{1mm}
\newcommand{\ansi}{ \bf ANSI standard change: }

\begin{document}
\thispagestyle{empty}
SCIENCE \& ENGINEERING RESEARCH COUNCIL \hfill \stardocname\\
RUTHERFORD APPLETON LABORATORY\\
{\large\bf Starlink Project\\}
{\large\bf \stardoccategory\ \stardocnumber}
\begin{flushright}
\stardocauthors\\
\stardocdate
\end{flushright}
\vspace{-4mm}
\rule{\textwidth}{0.5mm}
\vspace{5mm}
\begin{center}
{\Large\bf \stardoctitle}
\end{center}
\vspace{5mm}
%\tableofcontents
\markright{\stardocname}

%------------------------------------------------------------------------------

\newcounter{sruleno}
\newcommand{\srule}[1]
      {\addtocounter{sruleno}{1}
      \goodbreak
      \rule[0.5ex]{\textwidth}{0.3mm}
      {\Large #1 \hfill {\thesruleno}}
      \rule[0.5ex]{\textwidth}{0.1mm} \\ }
\renewcommand{\_}{{\tt\char'137}}

%------------------------------------------------------------------------------

\section{Introduction}

C is  a general-purpose language designed around 1972 for use on the Unix 
operating system.
The decisions to make the language portable and use it to rewrite Unix have
caused it to become very popular, and many operating systems, compilers and 
general purpose applications have been coded using C.
{\sl ``The C programming language''} by Kernighan and Ritchie (second 
edition) is a good introductory textbook.

C combines the advantages of high-level languages such as Fortran in the
realm of control structures and portability with `machine-level' features
such as pointers, increment operators, bitfields and bitwise
operators. The language  is especially suited to system programming because
its data types and operators are closely related to the operations of most
computers. However Starlink does not recommend the use of C for
applications programming. There are several reasons for this, not least
that Fortran provides better facilities for the manipulation of
multi-dimensional data arrays and the dynamic declaration of array size. 

Additionally there is the problem of maintenance.
It is very easy to write  C code which is cryptic -- but the temptation to
write highly-condensed code should be avoided.
The fragment of code below is reproduced (courtesy of Keith Shortridge, AAO) 
merely to illustrate that C offers much more opportunity to be 
obscure than Fortran ever could.
\begin{verbatim}
      MainDeviceHdl = GetMainDevice();
      SaveCTabHdl = (**((**MainDeviceHdl).gdPMap)).pmTable;
      HandToHand(&SaveCTabHdl);
      BitClr(&((*SaveCTabHdl)->transIndex),0);
      for (I = 0; I < (*SaveCTabHdl)->ctSize; I++) {
         ((*SaveCTabHdl)->ctTable)[I].value = I;
      }
      (*SaveMapHandle)->pmTable = SaveCTabHdl;
\end{verbatim}
A common pursuit for C programmers is to write code and then collapse it
into the shortest possible form.
Whilst the experienced C programmer will recognise the patterns of code 
which result,  it is important 
that such structures are well commented for the benefit of the less
experienced.

The language has no facilities to perform input and output, and lacks 
Fortran's large set of intrinsic functions.
However all implementations are accompanied by a (more or less) standard library of functions 
to accomplish I/O and other tasks.
These standard functions are made available to  a program by specifying
the modules  which contain the required set of functions via an include file.
For example the STanDard I/O routines are contained in the file {\tt stdio.h},
which must be explicitly {\tt include}d in a program file if the functions 
contained therein are to be used.




The suggestions which follow are made with maintainability, portability and 
efficiency in mind.
Several of these recommendations are made because certain code constructs 
will have different effects on different C  implementations.
In some cases the ambiguity is resolved by the ANSI  standard.
However it may be some time before all C implementations meet the standard,
and in any case it is preferable to avoid code which suggests more than one
possibility to the human reader.



\section{Coding standard}

The advice which follows is grouped into four sections concerning 
language, design, quality and presentation respectively. The reader is 
referred to SGP/16 for general advice on Starlink coding standards.

\subsection{Language considerations}

\srule{Use the standard syntax for include files.}

Files should be included with the standard syntax, that is:
\begin{verbatim}
      #include <stdio.h>
\end{verbatim}
or 
\begin{verbatim}
      #include "filename"
\end{verbatim}

The VAX C syntax 
\begin{verbatim}
      #include stdio
\end{verbatim}
extracts the header file from a VMS text library, and this is non-portable.



\srule{Don't assume an order of evaluation as specified by brackets 
will be followed.}

Many C compilers assume that the binary operators 
\verb~+,*,&,^,|~
are both associative and commutative, (in the case where all variables have 
the same type).
Hence it may rearrange the expression
{\tt (a+b)+(c+d)} to {\tt (a+d)+(b+c)}.

This rearrangement may cause problems if, for example, the order 
specified was chosen to avoid the possibility of overflow.
In cases where such considerations apply
it is necessary to use assignment to temporary variables to force the 
desired evaluation order.


{\ansi This freedom to regroup expressions has been removed by the 
new ANSI standard, in 
recognition of the reality that floating-point arithmetic is not in general
associative and commutative. Hence this advice can be forgotten
when ANSI~C-compliant compilers are generally in use.}



\srule{Don't write code where the result depends on the order of evaluation 
of the various expressions}

The C compiler is free to choose the order in which elements of an expression 
are evaluated. 

In the example below you might guess that the value of the subscript {\tt 
i} will be that before {\tt i} is incremented.
\begin{verbatim}
      x[i]=i++;
\end{verbatim}

In fact, the value of {\tt i} used for the subscript will be compiler dependent.
Taking advantage of knowledge of the practice adopted in a particular 
implementation will result in ambiguous and non-portable code.

In general:
\begin{itemize}
\item don't perform maths on the left-hand side of an equation which
includes the same variable on both sides, for example: 
\begin{verbatim}
      i++ = i + j;
\end{verbatim}
\item and don't use variables from the right-hand side to form an address
on the left-hand side, for example: 
\begin{verbatim}
      *p = ++p + i;
\end{verbatim}

\end{itemize}


\srule{Don't use the remainder operator with negative integral operands.
}

The remainder operator \% gives the remainder when the first operand is
divided by the second. That is, {\tt a\%b} is the modulus of {\tt a} 
with respect to {\tt b}.
However, if either {\tt a} or {\tt b} is negative the result will vary according to the
way integer division is implemented on the particular machine.
Of course, division by zero must also be avoided.


\srule{Use braces when nesting {\tt if} structures.}

The {\tt else} part of an {\tt if-else} is optional. When an {\tt else}
follows more than one {\tt if} it is matched with the closest previous 
{\tt else}-less {\tt if}.

In the example below, the intention of the programmer is reflected by the 
indentation of the code, however the compiler will match the {\tt else}
with the second {\tt if} statement.
\begin{verbatim}
      if (n!=0)
         a=b/n;
         if(a>>3)
            printf(" a is greater than 3\n");
      else
         printf(" Cannot divide by zero\n");
\end{verbatim}
Such errors can easily be introduced to code when extra conditions are
added to existing control structures.
Adopting the practice of always including braces to delimit the blocks on
which control structures operate will make code more robust.
An example format is shown below.
\begin{verbatim}
      if (test) 
      {
      } 
      else 
      {
      }
\end{verbatim}
Of course, it remains important to indent code in consideration of the 
human reader.


\srule{Pass arguments in the recommended manner.}

In general, function arguments in C are passed by {\sl value}.
That is, the function is supplied with the value of the argument but does
not have access to the original. Therefore changes to the value of the argument 
made in the function do not affect the original value in the calling 
routine. 

The exception is arrays; when an array is passed to a function, the {\sl 
address\/} of the first array element is passed. In this case, the function 
has access to the original array, and changes made are thus apparent to the
calling routine.

This behaviour has implications for the way function arguments should be
passed. In particular, the programmer must consider whether or not 
an altered value 
for an argument should be returned to the calling routine -- that is, whether
it is an input or an output argument.
The following practises are recommended:

\begin{itemize}
\item {\sl Input\/} arguments which are scalar and numeric should be 
passed by value.
Since the output value of the argument is not to be passed to the calling
routine, passing by value is appropriate. For example:
\begin{verbatim}
      func (i);
      ...
      int func(i);
\end{verbatim}

\item {\sl Output\/} arguments which are scalar and numeric should be 
passed by address.
In this case, the altered value of the argument must be passed 
back to the calling
routine, so the function must be given access to the original. 
Hence the argument must be explicitly passed by address.
For example:
\begin{verbatim}
      func (&i);
      ...
      int func(i);
\end{verbatim}


\item If the  {\sl input\/} arguments is an array,
it will implicitly be passed by address. To avoid an input array being 
modified by the function, it should be declared as {\tt const}.
For example:
\begin{verbatim}
      float array[100];
      ...
      int func (array,100);
      const float array[100];
\end{verbatim}
Of course, declaring the array as {\tt const} prevents the function
modifying it, thus neither of the statements below could be used in {\tt 
func}:
\begin{verbatim}
      array[3] = 4;
      ++array[3];
\end{verbatim}

Note -- don't pass arrays explicitly by address;
although permitted by the VAX VMS compiler, passing arrays {\sl explicitly\/} 
by address is in violation of the ANSI standard.
For example, the code below is wrong:
\begin{verbatim}
      float array[100];
      ...
      func (&array,100);
\end{verbatim}


\item If an array is an  {\sl output\/} argument, it should simply
be passed by address. For example:
\begin{verbatim}
      float array[100];
      ...
      sub (array,100);
\end{verbatim}


\item If the  {\sl input\/} arguments is a structure,
it should be passed by address as this is generally more efficient
than making a copy of the structure for the function.
An explicit pointer is required.
To avoid an input structure being 
modified by the function, it should be declared as {\tt const}.

\item If a structure  is an  {\sl output\/} argument, it should simply
be passed (explicitly) by address. 

\end{itemize}




\srule{Use casts to indicate where (legal) type conversions are taking 
place.}

For example the function {\tt sqrt} expects a {\tt double} argument.
To find the square root of the integer {\tt n}, the cast below should be 
used:
\begin{verbatim}
      sqrt((double) n)
\end{verbatim}
This does not affect the value of {\tt n} but provides a value of type 
{\tt double} as the argument for the {\tt sqrt} function.


\srule{Pay careful attention to the type of the return value from 
library functions.}

For example the code fragment below is wrong because {\tt getchar} returns 
an {\tt int} not a {\tt char}.
\begin{verbatim}
      #include "stdio"
      char c;
      c = getchar();
\end{verbatim}
The correct way to use {\tt getchar} is indicated in the example below:
\begin{verbatim}
      #include "stdio"
      char c;
      int i;
      i = getchar();
      if ( i == EOF )
          /* end of file - do something about it */
      else
          c = (char)i;                                  
\end{verbatim}

\srule{Don't cast pointer types.}

Conversion of a pointer from one type to another can cause problems if the 
alignment requirements for the types in question are different.
If the resulting pointer is illegal, this can cause an error or the pointer 
may
be automatically adjusted to the nearest legal address.
Conversion back to the original pointer type may not recover the original
pointer.

\srule{Use a function pointer type that specifies the correct return 
type.}

Function pointers must be correctly typed.
The assumption that the default
type ({\tt int}) is sufficiently large to accommodate a pointer of any
type is not correct on all computers. 
For example,  the first function shown below will produce unpredictable
results on some computers. 
The `good' function illustrates the correct approach.
\begin{verbatim}
      /* Bad: */
      func( double a, double b )
      {
         static double c; /* a * b */
         c = a * b;
         return &c;
      }

      /* Good: */
      double *func( double a, double b )
      {
         static double c; /* a * b */
         c = a * b;
         return &c;
      }
\end{verbatim}

\srule{Use {\tt void *} for generic pointers, not {\tt char *.}}

The ANSI C standard replaces {\tt char~*} with {\tt void~*} as the 
correct type for a generic pointer.


\srule{Always test the pointer returned by {\tt malloc} {\it etc.} 
for equality with   the null pointer before attempting to use it.}

Ignoring the possibility that {\tt malloc} has failed 
can cause an access violation 
on a  VAX and mayhem on many other systems.

\srule{Don't use a null pointer for anything other than assignment or 
comparison.}

A null pointer does not point anywhere. The effect of misusing a
null pointer is unpredictable. 
Consider the code below:
\begin{verbatim}
      ptr = NULL;
      printf ("Location 0 contains %d\n", *ptr);
\end{verbatim}
Some C implementations permit the 
hardware location~0 to be read (and written), whilst others do not.
Hence the code above will execute on some computers but not on others.

\srule{Don't assume a zeroed variable will be interpreted as a null
pointer.}

The result of converting the integer constant 0 to a pointer type is a 
null pointer. However the value NULL (which is defined as 0 in  
\verb!<stdio.h>!) should be explicitly used. That is:  
\begin{verbatim}
      ptr = NULL;
\end{verbatim}
Note that a null pointer does not necessarily
result from the conversion of a floating-point zero. 

Note also that the conversion of an integer zero to a floating-point type 
cannot be assumed to be a floating-point zero.



\srule{Never take the address of a constant.}

For example:
\begin{verbatim}
       i = &2;
\end{verbatim} 
This is allowed by the VAX  compiler but few (if any) others.

 
\newpage
\subsection{Program design}

\srule{Avoid the more obscure parts of the language.}

Different C implementations are very likely to vary from each other 
in such areas.
Keeping to the well-known parts of the language will reduce a 
program's dependence on the implementation getting everything correct.

\srule{Restrict use of library functions to those which are
generally available.}

Avoiding the use of library functions which are only implemented 
on some platforms is an important aspect of program portability.

\srule{Keep functions as short and specific as possible.}

Adopting this practice leads to understandable code and re-usable 
functions. It also simplifies program testing and bug finding.


\srule{Use only one return statement in a function.}

Generally a single return at the end of a function is recommended.
However it is permissible to have an extra return after 
an initial status check if such a check is performed.

\srule{Initialise all automatic variables before use.}

If local variables are not set they may contain junk.

\srule{Don't assume a particular encoding system is in use.}

Each standard character in C must have a distinct, non-negative integer 
encoding.
{\bf However this need not be accomplished using the ASCII encoding system.}
Many C programs rely on the assumption that letters of the alphabet,
(or the digits 0--9) are coded contiguously as is the case with ASCII.
For example, the test below might be used to check if a character is 
a capital letter:
\begin{verbatim}
      c >= 'A' && c <= 'Z'
\end{verbatim}
This code will work with ASCII, but not with the EBCDIC character set.

The programmer must use the functions supplied with the standard 
header $<$ctype.h$>$.
This contains functions for testing and converting characters.
For example the function {\tt isupper(c)} is a portable replacement for the 
code  above.
Similarly, the test of whether a character is a decimal digit 
should be accomplished with the standard function  {\tt isdigit(c)} --
not with the test: 
\begin{verbatim}
      c >='0' && c<='9'\end{verbatim}

\srule{Use the correct method for error detection.}

Most library routines return an indication of failure via an external 
variable called {\tt errno}.
The obvious way to check for an error might appear to be:
\begin{quote}
    {\sl call library function}\\
    {\tt if (errno)}\\
    {\sl  \ \ \ \ make an error report}
\end{quote}
However there is no guarantee that the library routine will clear {\tt 
errno} in the absence of an error. Initializing {\tt errno} to zero
before  calling the routine does not solve the problem;
{\tt errno} may be set as
a side effect of the function executing without the function actually failing!

The correct approach is to test the value returned by the function
for an error indication
before examining {\tt errno} to find the nature of the error. For example:
\begin{verbatim}
      status = func(a,b);
      if(status)
      {
       (examine errno)
      }
\end{verbatim}
See SSN/4, Section~8,  for advice on making a suitable error report
compliant with Starlink error-reporting conventions.


\srule{When passing an array to a function pass the size of the array 
too.}

Functions which get passed an array (particularly a 
character string) should also be passed the size of the array.
The size should be used to ensure that the function does not overwrite the 
end of the array.
Failure to adopt this practice can lead to bugs which are very difficult to 
track down.
Note that you cannot use dummy arguments when declaring the size of a 
passed array -- only pre-defined constants can be used.

\srule{Don't use the {\tt gets} function.}

It is impossible to ensure that the string being
read doesn't over-run the input buffer and overwrite other areas of memory.



\srule{Finish a switch statement with a break statement.}

Adding a break statement at the end of a switch statement does not
affect the program control flow, but does mean that there
is less likelihood of introducing a bug if a further case is added
in the future. 
\begin{verbatim}
      switch (c) {
         case 'a':
         ...
         case 'b':
         ...
          break;
      }
\end{verbatim}

\srule{Use the standard method for dealing with variadic functions.}

Unlike Fortran, C does not require that the number and type of 
arguments match those passed to a function.
Many programmers rely on this property to implement functions 
which have a variable number of arguments.

The set of functions and macros defined in the {\it stdargs\/} definition 
module provides a portable method of accessing variable-length argument
lists.
C programs which predate the ANSI standard may contain functions which rely on 
details of how parameters are passed on the 
stack, and are thus system specific and should be avoided.
User-defined functions that do not use this method are not portable due to the 
different argument passing conventions of various machines.




\srule{Use only non-floating  loop variables.}

The effects of cumulative rounding errors in real numbers cannot be ignored; 
the precision to which non-integer numbers are held must be considered and 
explicitly treated in the code if necessary.


\srule{Avoid changing the current loop index and range within a 
{\tt for} loop.}

Unlike Fortran, C allows the loop index  and range to be changed 
from within the loop control structure.
For example it is permissible to alter both {\tt i} and {\tt n}
in the body of the {\tt for} loop below.
\begin{verbatim}
      for (i=0; i < n; i++)
      {                       
         ...
      }
\end{verbatim}          
However this freedom should not be used unnecessarily as it leads to 
confusing code.

\srule{Avoid the {\tt goto} statement.}

The use of {\tt goto} statements leads to programs with confusing control 
flow so these should not be used unnecessarily.

\srule{No unused variables or unreachable code.}

Obviously these should not be tolerated.

\srule{Adopt a particular order for arguments in functions.}

It is recommended that the following order is used for arguments in 
functions:
\begin{enumerate}
\item supplied and unchanged, 
\item supplied and changed, 
\item returned, and 
\item status return.
\end{enumerate}



\srule{Functions should return a status value indicating success or failure.}

As mentioned above, global  status should be the final item 
in a function argument  list.
It should have type `pointer to integer', and it is recommended that it is 
implemented as shown in the example below:
\begin{verbatim}
      int function fun (int p, int q, int *status)
      {      
         ...
       }
\end{verbatim}
This ensures that the status value is inherited from and can be
returned to the calling module so that appropriate action can be
taken if an error is detected.






\srule{Don't rely on the availability (or indeed the non-availability) 
of nestable comments.}

Some C implementations use the concept of `nestable comments' in which each 
occurrence of /*  must be balanced by a subsequent */.
Whilst this provides an easy way of commenting out large portions of code
without worrying about the comments in such code,  nestable comments are not 
standard and code using them will not be portable.

If absolutely necessary, large pieces of code can be commented out using the preprocessor's conditional 
commands.
For example:
\begin{verbatim}
      #if 0
       ...
      #endif
\end{verbatim}
Conversely, if  a system does not employ nestable comments, 
avoiding the use of the sequence /* inside comments will ensure that a
program is acceptable to an implementation which does.

\srule{Take care when using unsigned variable types.}

Assigning negative numbers to unsigned types can produce surprizing
results; expressions which mix signed and unsigned types should be
treated with great caution.

\srule{Always explicitly declare {\tt signed} or {\tt unsigned} when using 
characters to store  numerical ({\sl i.e.} non-character) data.}

This is important because 
the default {\tt char} type may be signed 
or unsigned depending on the implementation.


\srule{Don't assume more bits or bytes in a data type than the standard
 requires.}

Also don't write code which will fail if there are more bits or bytes present 
than the standard requires.


\srule{Don't write hex or octal constants which assume the word-length of a 
machine.}

For example use {\tt$\sim0$}, not {\tt 0Xffffffff}.


\srule{Don't make assumptions about the order or degree of packing of 
bit fields.}

Both the order and packing of bit fields are machine-dependent, so 
assumptions regarding them will result in non-portable code.

\srule{Don't use bit fields for storing integer values.}

This should only be done  if storage space is 
really at a premium. Arithmetic on plain chars and ints will usually be far
more efficient.

\srule{Take care with element order within structures}

Put the longest elements first to minimize `padding'.

\srule{Don't add any explicit packing to structures in order to achieve optimum 
alignment.}

What is optimal on one machine may not be on another. Rely 
on the advice in the previous item.
When using bit fields, use the portable {\tt int~:~0}
field to ensure re-alignment afterwards if necessary.



\srule{Don't use casts and {\tt strcmp} (or {\tt memcmp}) to compare the contents of 
structures for equality.}

The padding regions within the structures may 
contain junk, even if the declared fields are identical.

\srule{Use assignment to copy structures (not {\tt strcpy} or {\tt memcpy}).}

As indicated above, this approach avoids the problem of padding.

\srule{Use the {\tt sizeof} function explicitly where appropriate.}

For clarity and portability, it is recommended that the size of objects be
obtained using this function.
For example, use:
\begin{verbatim}
      pntr = malloc( sizeof( struct ABC ) );
\end{verbatim}
not:
\begin{verbatim}
      #define SIZEABC ...

      pntr = malloc( SIZEABC );
\end{verbatim}

\srule{Don't write C structures to files which must be read on another 
machine.}

Details of structure packing are machine-dependent and should not be allowed 
to appear in files which must be portable.


\srule{Beware using standard library pseudo-random number generators.}

The range of the numbers generated will be machine-dependant.
ANSI compilers limit the upper bound of the range of numbers generated to 
{\tt RAND\_MAX}. 
This number can be used to normalize the distribution, although
problems with the resolution of the resulting number distribution may remain.





\srule{Use the Starlink machine-independent macros for Fortran-to-C 
interfaces.}

These are documented in SGP/5 (in preparation).

\newpage
\subsection{Program quality}


\srule{Set out pre-processor directives correctly.}

Start all pre-processor directives on column~1 and do not have any white
space between the {\tt \#} and the keyword -- some compilers insist on 
these requirements.

\srule{Check on the general availability of the pre-processor directives
used.}

For example \verb~#debug, #eject, #list, #module, & #section~
are not available on every platform.

\srule{Restrict the contents of personal header files.}

Header files are a very convenient place to encapsulate function prototypes 
and package-dependant global constants, global variables and global structures.
Defining these items should be the main
purpose of such files. 
Header files must {\sl not\/} be used extensively for conditional
compilation.
It is recommended that the number of personal header files used should 
be kept to a minimum.

\srule{Limit use of preprocessor commands.}

The use of preprocessor commands should be limited to conditional 
statements and the inclusion of symbolic constants.
In the case of portable interfaces it is sometimes necessary to define
macros so that a single version of the main body of the C~code can be 
maintained,
with any architectural differences confined to header files.


\srule{Define macros only once.}

Don't rely on defining a macro at one point and then re-defining its
value at another point. Define any macro only once, otherwise later changes may
be made at the wrong point and may not be used. Also, if the second
definition of the macro is inadvertently omitted (perhaps an include 
file may have been left out) you 
may use the wrong value unknowingly, 
whereas if the macro is defined
only once, omitting this definition accidentally will probably produce 
a compiler error and the mistake will be spotted. 


\srule{Test if macros have already been defined before defining them.}

A header file which defines a macro  should test whether it has been
previously defined  so that the macro is
not interpreted twice if it has actually been included twice. For example:
\begin{verbatim}
      #ifndef header_read
      #define header_read 1

        ...

      #endif
\end{verbatim}
This is a requirement for ANSI C header files and should also be applied to 
personal files.

\srule{Don't use reserved words as names of preprocessor macros.}

The following identifiers are reserved and must not be used as program 
identifiers:
\begin{verbatim}
      auto, break, case, char, const, continue, default, do, double, 
      else, enum, extern, float, for, goto, if, int, long, register,
      return, short, signed, sizeof, static, struct, switch, typedef, 
      union, unsigned, void, volatile, while.
\end{verbatim}
Although  not forbidden by the standard it is bad programming practice
to use these words as preprocessor macro names.


\srule{Don't put directory names explicitly into {\tt \#include} statements.}

Rather than specifying directory names, use the
compiler options to search the relevant directories.

\srule{Declare global variables used in routines as {\tt extern}.}

The use of global variables in C can lead to confusion over which
variables are active in a particular program module --
just as the use of COMMON variables in Fortran can be a problem.
Try to declare  all such variables as externs in the routines which use them.
If this practice is adopted the reader can just look at a page of code 
and see a list of ALL the active variables in a module.
It is recommended that:
\begin{itemize}
\item The global variables used by a program are defined only once 
in a single source file  in which they are explicitly initialized.
For example:
\begin{verbatim}
      int counter = 0;
\end{verbatim}
\item In each function which uses an external variable defined elsewhere,
 use the storage class {\tt extern} and do not supply
an explicit initializer.
\begin{verbatim}
      extern int counter;
\end{verbatim}
\end{itemize}
Of course, the variable must be given the same type in all 
locations.
The {\tt lint} program which is usually available on Unix machines can be 
used to check multiple files for consistent declarations.

\srule{Parameterize constants.}

Always a good idea, this is especially worthwhile in C.
This is because the handling of overflow, divide check and 
other exceptions is not defined by the language.
For example, if the value of a decimal constant exceeds the largest integer 
representable in type long, (or that of an octal or hexadecimal is greater
than an unsigned long)
the result is unpredictable.
The solution is to use parameterized
machine-dependant constants so they can be changed when moving computers.
Such constants  are available via  standard header files.
ANSI~C demands the provision of the header files {\tt float.h} 
and {\tt limits.h} which contain information on the size, range {\it etc.} 
of the 
various types in a specific implementation. These standard 
header files should always be used, not personal equivalents.

Parameterizing personal constants is also recommended.
For example:
\begin{verbatim}
      #define MAXLINES 1000
      #define MAXWORDS 100000
\end{verbatim}
However this can be overdone, for example:
\begin{verbatim}
      #define ZERO 0
\end{verbatim}


\srule{When using true/false variables, create a `pseudo-type' BOOL and define
  constants TRUE and FALSE.}

For example:
\begin{verbatim}
      #define BOOL short
      #define TRUE 1
      #define FALSE 0

      BOOL flag;
  
      flag = TRUE;
      if (flag)  
      ...
\end{verbatim}

\srule{If a function has arguments declare them, if not use void.}

Function prototypes should {\sl always\/} be used.
For example
\begin{verbatim}
      int recheck(char c, int i);
\end{verbatim}
This is important because type conversion can take place when arguments are 
passed into functions.
In the absence of a function prototype,  {\tt char} and {\tt short} 
become {\tt int}, and {\tt float} becomes {\tt double}.

If a function has no arguments, this should be indicated by specifying void 
within the brackets thus:
\begin{verbatim}
      int fun (void);
\end{verbatim}

Whilst function prototypes declare the type, they  need not necessarily
declare the names of the parameters. For example the first function prototype
in this item could be abbreviated to:
\begin{verbatim}
      int recheck(char, int);
\end{verbatim}
However it is recommended that appropriate names are used as these help
the reader of the program.  

\srule{Declare as {\tt void} those functions which do not return values.}

For example:
\begin{verbatim}
      void fun (int p, int g);
\end{verbatim}


\srule{If the return value of a function is being discarded then 
cast the  value to void.}

For example, rather than:
\begin{verbatim}
      printf("Hello\n");
\end{verbatim}
use:
\begin{verbatim}
      (void)printf("Hello\n");
\end{verbatim}

\newpage
\subsection{Presentation}

\srule{Take care with program presentation.}

As with Fortran, C code should be indented to reflect the program structure.
The use of blank lines to improve the layout is also recommended.

\srule{Use a maximum of 79 characters per line.}

C imposes no limit on the maximum length of lines, although many 
implementations have a limit -- typically 100--500 characters.
Keeping to 79 characters not only ensures you won't run into a 
limit problem but 
makes the program easy to read on any terminal.

\srule{Choose names of identifiers and functions carefully}

Appropriate names are an important aid to understanding a program 
and should always be used.


\srule{Avoid using case-sensitivity to distinguish identifiers.}

For example, the use of both {\tt Sum} and {\tt sum} as variables is likely 
to confuse the reader of a program.
The practice of using lower-case for ordinary variables and upper-case
for symbolic constants is recommended.




\srule{Make sure comments are clearly laid out.}

It is recommended that comments are laid out as follows:
\begin{verbatim}
      /* This is not a very interesting set of comments but at
         least it is clear where it begins and ends. 
         Well I think so. 
      */
\end{verbatim}
Comments should be kept on separate lines from program code.
The only exception is during the declaration of variables, when an in-line
comment as shown below is acceptable.
\begin{verbatim}
      int i;        /* loop variable */
\end{verbatim}


\srule{Pointer names should be clearly identified.}

For example, the variable {\tt i} and the associated pointer 
might be declared thus:
\begin{verbatim}
      int i, *i_ptr;
\end{verbatim}



\srule{When using the {\tt ==} or {\tt !=} operators, compare expressions with
{\tt 0}.}

For example, use:
\begin{verbatim}
      if ( x != 0 ) 
\end{verbatim}
rather than:
\begin{verbatim}
      if (x)
\end{verbatim}
Exceptions are:
\begin{itemize}
\item  if {\tt x} is being used solely as a true/false flag, or
\item when comparing a pointer to the null pointer, or
\item in the termination condition   of a {\tt for} loop, for example:
\begin{verbatim}
      for ( i = 100; i; i--)
\end{verbatim}
\end{itemize}

\srule{Indicate `infinite'  loops clearly.}

It is recommended  that an infinite loop is represented by:
\begin{verbatim}
      for (;;)
\end{verbatim}

\srule{Do not include whitespace inside compound operators.}

The compound assignment operators in C 
\verb~(= -= *= /= %= <<= >>= &= ^= |=)~ 
should be considered to be single tokens. The characters can be separated by
whitespace (or even comments); however it is bad programming practice 
to do so.

\section{Acknowledgements}

Thanks to Peter Allan, Chris Clayton, Malcolm Currie, Brian McIlwrath, Paul 
Rees, Keith Shortridge, Dave Terrett, Pat Wallace and 
Rodney Warren-Smith for their help in compiling this document.

\end{document}
