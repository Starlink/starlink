\documentclass[twoside,11pt]{article}
\pagestyle{myheadings}

% -----------------------------------------------------------------------------
% ? Document identification
\newcommand{\stardoccategory}  {Starlink General Paper}
\newcommand{\stardocinitials}  {SGP}
\newcommand{\stardocsource}    {sgp\stardocnumber}
\newcommand{\stardocnumber}    {41.1}
\newcommand{\stardocauthors}   {P T Wallace}
\newcommand{\stardocdate}      {16 September 1994}
\newcommand{\stardoctitle}     {Local Management Arrangements}
\newcommand{\stardocabstract}  
{In July 1993, the Starlink Panel reviewed the existing local
management arrangements, based on the ``Area Management Committees''
(AMCs), and decided to introduce a number of reforms.  The new
arrangements, which are the subject of this paper,
concentrate the old AMC responsibilities
into one individual at each site called the Site Chairman, leaving
it up to that individual to set up the consultative machinery
appropriate to his or her community.  Sites normally opt
to set up committees much like the old LMCs or AMCs, but adapted
to suit local needs.\\
\\
A prospective Site Chairman proposes and agrees with the Starlink
Project Scientist local procedures for consulting
and representing users which are acceptable to both Starlink and
the Institution.  The effectiveness of these procedures is
kept under review; accreditation of the Site Chairman is
renewed annually.}

% ? End of document identification
% -----------------------------------------------------------------------------

\newcommand{\stardocname}{\stardocinitials /\stardocnumber}
\markright{\stardocname}
\setlength{\textwidth}{160mm}
\setlength{\textheight}{230mm}
\setlength{\topmargin}{-2mm}
\setlength{\oddsidemargin}{0mm}
\setlength{\evensidemargin}{0mm}
\setlength{\parindent}{0mm}
\setlength{\parskip}{\medskipamount}
\setlength{\unitlength}{1mm}

% -----------------------------------------------------------------------------
%  Hypertext definitions.
%  ======================
%  These are used by the LaTeX2HTML translator in conjunction with star2html.

%  Comment.sty: version 2.0, 19 June 1992
%  Selectively in/exclude pieces of text.
%
%  Author
%    Victor Eijkhout                                      <eijkhout@cs.utk.edu>
%    Department of Computer Science
%    University Tennessee at Knoxville
%    104 Ayres Hall
%    Knoxville, TN 37996
%    USA

%  Do not remove the %begin{latexonly} and %end{latexonly} lines (used by 
%  star2html to signify raw TeX that latex2html cannot process).
%begin{latexonly}
\makeatletter
\def\makeinnocent#1{\catcode`#1=12 }
\def\csarg#1#2{\expandafter#1\csname#2\endcsname}

\def\ThrowAwayComment#1{\begingroup
    \def\CurrentComment{#1}%
    \let\do\makeinnocent \dospecials
    \makeinnocent\^^L% and whatever other special cases
    \endlinechar`\^^M \catcode`\^^M=12 \xComment}
{\catcode`\^^M=12 \endlinechar=-1 %
 \gdef\xComment#1^^M{\def\test{#1}
      \csarg\ifx{PlainEnd\CurrentComment Test}\test
          \let\html@next\endgroup
      \else \csarg\ifx{LaLaEnd\CurrentComment Test}\test
            \edef\html@next{\endgroup\noexpand\end{\CurrentComment}}
      \else \let\html@next\xComment
      \fi \fi \html@next}
}
\makeatother

\def\includecomment
 #1{\expandafter\def\csname#1\endcsname{}%
    \expandafter\def\csname end#1\endcsname{}}
\def\excludecomment
 #1{\expandafter\def\csname#1\endcsname{\ThrowAwayComment{#1}}%
    {\escapechar=-1\relax
     \csarg\xdef{PlainEnd#1Test}{\string\\end#1}%
     \csarg\xdef{LaLaEnd#1Test}{\string\\end\string\{#1\string\}}%
    }}

%  Define environments that ignore their contents.
\excludecomment{comment}
\excludecomment{rawhtml}
\excludecomment{htmlonly}

%  Hypertext commands etc. This is a condensed version of the html.sty
%  file supplied with LaTeX2HTML by: Nikos Drakos <nikos@cbl.leeds.ac.uk> &
%  Jelle van Zeijl <jvzeijl@isou17.estec.esa.nl>. The LaTeX2HTML documentation
%  should be consulted about all commands (and the environments defined above)
%  except \xref and \xlabel which are Starlink specific.

\newcommand{\htmladdnormallinkfoot}[2]{#1\footnote{#2}}
\newcommand{\htmladdnormallink}[2]{#1}
\newcommand{\htmladdimg}[1]{}
\newenvironment{latexonly}{}{}
\newcommand{\hyperref}[4]{#2\ref{#4}#3}
\newcommand{\htmlref}[2]{#1}
\newcommand{\htmlimage}[1]{}
\newcommand{\htmladdtonavigation}[1]{}

% Define commands for HTML-only or LaTeX-only text.
\newcommand{\html}[1]{}
\newcommand{\latex}[1]{#1}

% Use latex2html 98.2.
\newcommand{\latexhtml}[2]{#1}

%  Starlink cross-references and labels.
\newcommand{\xref}[3]{#1}
\newcommand{\xlabel}[1]{}

%  LaTeX2HTML symbol.
\newcommand{\latextohtml}{{\bf LaTeX}{2}{\tt{HTML}}}

%  Define command to re-centre underscore for Latex and leave as normal
%  for HTML (severe problems with \_ in tabbing environments and \_\_
%  generally otherwise).
\newcommand{\setunderscore}{\renewcommand{\_}{{\tt\symbol{95}}}}
\latex{\setunderscore}

% -----------------------------------------------------------------------------
%  Debugging.
%  =========
%  Remove % from the following to debug links in the HTML version using Latex.

% \newcommand{\hotlink}[2]{\fbox{\begin{tabular}[t]{@{}c@{}}#1\\\hline{\footnotesize #2}\end{tabular}}}
% \renewcommand{\htmladdnormallinkfoot}[2]{\hotlink{#1}{#2}}
% \renewcommand{\htmladdnormallink}[2]{\hotlink{#1}{#2}}
% \renewcommand{\hyperref}[4]{\hotlink{#1}{\S\ref{#4}}}
% \renewcommand{\htmlref}[2]{\hotlink{#1}{\S\ref{#2}}}
% \renewcommand{\xref}[3]{\hotlink{#1}{#2 -- #3}}
%end{latexonly}
% -----------------------------------------------------------------------------
% ? Document-specific \newcommand or \newenvironment commands.
% ? End of document-specific commands
% -----------------------------------------------------------------------------
%  Title Page.
%  ===========
\renewcommand{\thepage}{\roman{page}}
\begin{document}
\thispagestyle{empty}

%  Latex document header.
%  ======================
\begin{latexonly}
   CCLRC / {\sc Rutherford Appleton Laboratory} \hfill {\bf \stardocname}\\
   {\large Particle Physics \& Astronomy Research Council}\\
   {\large Starlink Project\\}
   {\large \stardoccategory\ \stardocnumber}
   \begin{flushright}
   \stardocauthors\\
   \stardocdate
   \end{flushright}
   \vspace{-4mm}
   \rule{\textwidth}{0.5mm}
   \vspace{5mm}
   \begin{center}
   {\Large\bf \stardoctitle}
   \end{center}
   \vspace{5mm}

% ? Heading for abstract if used.
   \vspace{10mm}
   \begin{center}
      {\Large\bf Abstract}
   \end{center}
% ? End of heading for abstract.
\end{latexonly}

%  HTML documentation header.
%  ==========================
\begin{htmlonly}
   \xlabel{}
   \begin{rawhtml} <H1> \end{rawhtml}
      \stardoctitle
   \begin{rawhtml} </H1> \end{rawhtml}

% ? Add picture here if required.
% ? End of picture

   \begin{rawhtml} <P> <I> \end{rawhtml}
   \stardoccategory\ \stardocnumber \\
   \stardocauthors \\
   \stardocdate
   \begin{rawhtml} </I> </P> <H3> \end{rawhtml}
      \htmladdnormallink{CCLRC}{http://www.cclrc.ac.uk} /
      \htmladdnormallink{Rutherford Appleton Laboratory}
                        {http://www.cclrc.ac.uk/ral} \\
      \htmladdnormallink{Particle Physics \& Astronomy Research Council}
                        {http://www.pparc.ac.uk} \\
   \begin{rawhtml} </H3> <H2> \end{rawhtml}
      \htmladdnormallink{Starlink Project}{http://www.starlink.ac.uk/}
   \begin{rawhtml} </H2> \end{rawhtml}
   \htmladdnormallink{\htmladdimg{source.gif} Retrieve hardcopy}
      {http://www.starlink.ac.uk/cgi-bin/hcserver?\stardocsource}\\

%  HTML document table of contents. 
%  ================================
%  Add table of contents header and a navigation button to return to this 
%  point in the document (this should always go before the abstract \section). 
  \label{stardoccontents}
  \begin{rawhtml} 
    <HR>
    <H2>Contents</H2>
  \end{rawhtml}
  \htmladdtonavigation{\htmlref{\htmladdimg{contents_motif.gif}}
        {stardoccontents}}

% ? New section for abstract if used.
  \section{\xlabel{abstract}Abstract}
% ? End of new section for abstract

\end{htmlonly}

% -----------------------------------------------------------------------------
% ? Document Abstract. (if used)
%  ==================
\stardocabstract
% ? End of document abstract
% -----------------------------------------------------------------------------
% ? Latex document Table of Contents (if used).
%  ===========================================
% \newpage
\begin{latexonly}
   \setlength{\parskip}{0mm}
   \tableofcontents
   \setlength{\parskip}{\medskipamount}
   \markright{\stardocname}
\end{latexonly}
% ? End of Latex document table of contents
% -----------------------------------------------------------------------------
\newpage
\renewcommand{\thepage}{\arabic{page}}
\setcounter{page}{1}

\section{Introduction}

An important feature of Starlink policy has always been
an emphasis on local flexibility:
wherever possible, things are run by people at the sites and tailored
to users' requirements, rather than being dictated from the centre.
The arrangements for doing this have evolved as the Starlink
network has changed in size and character. The most recent reforms
replaced the previous {\it Area Management Committees}\/ with a
more flexible scheme which
restored the emphasis on individual Starlink sites.

The present arrangements are designed to:
\begin{itemize}
\item enable the sites to run smoothly;
\item give users confidence that the service is being run their way;
\item provide contact between the Project and the users;~~~and
\item avoid unproductive meetings and unnecessary paperwork.
\end{itemize}

\section{History}

During the early 1980s, each Starlink site had a ``Local Management
Committee'' (LMC) to oversee usage of the facility and to resolve any local
disputes.  The LMCs had significant delegated powers and responsibilities,
reducing the need for Starlink central management to be involved in
local issues and ensuring that the node was being run in a way which
suited the users concerned.  Each LMC included representatives of both the
local groups of astronomers and also of the ``Remote User Groups'' (RUGs)
served by the node.  Every LMC had to meet at least
twice a year and to submit an Annual
Report to the Starlink Scientific Advisory Group, the early equivalent
of the Starlink Users' Committee (SUC) and the present Starlink Panel (SP).

As new nodes were established, the LMC arrangements became less
satisfactory:  there was an uncomfortable growth in the number of meetings
being held and in the volume of annual report material submitted to SUC,
while small nodes which were obliged to share resources with larger ones
had no common forum for managing these activities.  To address these
problems, the Local Management Committees were replaced by nine Area
Management Committees, each responsible for Starlink activities in a
given geographical area.  The special needs of the users at each site
were looked after by informal ``Starlink Local User Groups'' (SLUGs).  In
most respects, the AMCs were run like the LMCs, meeting twice a year,
producing an annual report to SUC, and making recommendations for
hardware upgrades and replacements in their area.

From the late 1980s, the AMC arrangements in their turn began showing signs
of age.  This happened because even the smallest individual node had
become better-equipped so that there remained little or no need to share
resources with other nodes.  In addition, most of the few remaining RUGs
had acquired local computing facilities good enough to remove the need for
frequent remote use of a
Starlink node.  Though in many cases the AMCs said they were
happy with the existing arrangements, there were other instances where the
area groupings had little practical meaning, and the members of the AMC
had very little interest in each others' activities.
The need for change was amplified by
the recommendations of the 1991/92 Starlink Review, which placed heavy emphasis
on the individual Starlink node and which led, one way or another, to the
elimination of two of the AMCs' most important jobs:  the production of
hardware ``wish lists'' and the writing of annual reports.

In July 1993, the Project put before the Starlink Panel proposals
to allow sites more flexibility in their local management arrangements,
essentially by vesting the previous AMC powers in an individual at each
site called the ``Site Chairman'' and allowing that individual to
set up consultative machinery appropriate for the site concerned.
The AMCs were described in detail in {\it Starlink General Paper 8}.
Their terms of reference were as follows:
\begin{itemize}
\item to ensure the implementation of Starlink policies concerning the
      operation of all the sites in its Area, including Minor Nodes and
      Remote User Groups;
\item to monitor all aspects of the service to the users, including
      resources, availability, documentation and usage, paying special
      attention to the needs of remote users;
\item to resolve any conflicting demands on resources and to deal with
      user recommendations or complaints which cannot be dealt with by the
      Site Managers;
\item to make recommendations on any relevant matters of policy or
      operation to the Starlink management.
\end{itemize}
Under the new arrangements, the responsibilities of the
Site Chairman are similar;  they are listed in the Appendix.

\section{The R\^{o}le of the Site Manager}

The Site Managers are crucial because they are the principal
interface between Starlink management and the users.  To do this
challenging job effectively they need to be influential and respected,
despite their relatively junior rank.

They can, moreoever, be in a difficult position, because they report formally
to their respective employers, but at the same time have
responsibility for the implementation at their site of policies
defined by the Starlink Project.

The formal lines of responsibility are that the HEIs are contracted 
to carry out certain work for PPARC.  The details of the work
are specified by the Starlink Project on behalf of PPARC; a
key feature of the way the Starlink sites are run is that the
Site Managers take direct instructions from the Project.
However, the Site Managers are
employed by their local institutions, and are therefore formally under
the control of their Institution, at least for some aspects of their work.
Without tact and cooperation on the part of all concerned,
this split responsibility can
lead to friction and confusion, though historically this has been rare.
The Starlink Project tries to minimize
and forestall such problems through a policy of openness in
discussions and decision making.

The aim of Starlink management is to define a clear
framework of duties and responsibilities but to allow each
Site Manager some discretion in the way he or she provides a Starlink
service for the users at the site concerned.  The Site Managers'
activities are monitored through various local and national
mechanisms, including
the Local User Group meetings, the AMC meetings at present and
the meetings which are convened by
the Site Chairmen in the future, and the regular Starlink Site
Managers' Meetings.  This monitoring makes sure that all sites are,
as far as possible,
being run in accordance with Starlink policies and the wishes of the
users.

It cannot be stressed too much that Starlink's site management
arrangements assume the utmost cooperation and trust between
all concerned: the ``rules'' are too broad to support
a detailed demarcation of responsibilities and power.  Clearly
the Site Manager must carry out the duties
defined by Starlink and must run the Starlink Node in a manner
which meets with the approval of both Starlink management and the
users.  But he or she must also establish a working relationship
with local line management and is expected to collaborate closely
with the Site Chairman.  Similarly, the local
management and the Starlink Project must both be sensitive
to the difficult r\^{o}le played by the Site Manager.

\section{The Post-AMC Arrangements}

\subsection{Synopsis}

The changes which the Starlink Panel approved in 1993, with the intention of
establishing local management arrangements which were simpler than
the old AMC-based ones and more responsive to the needs and wishes of the
local communities, were these:
\begin{itemize}
\item The existing AMCs were to be formally dissolved.
\item At each site there would be one individual, agreed by the
      Starlink Project Scientist and the Institution,
      called the ``Site Chairman''.
\item The Site Chairman's terms of reference (see the Appendix) 
      would be similar to those of the old AMCs but applying only to his or
      her Starlink site.
\item The Site Chairman would be expected to set up suitable local
      arrangements to consult and represent users, for example:
      \begin{itemize}
      \item A committee resembling an AMC or LMC.
      \item An extended role for some existing committee.
      \item Plenary meetings of all users.
      \end{itemize}
\item With the assistance of the Starlink Site Manager(s), the Site
      Chairman would be required to arrange at least one meeting per year at
      which members of the Starlink central team could meet all local
      users who wish to attend.
\end{itemize}

\subsection{Accreditation of the Site Chairman}

The Starlink Project Scientist (PS) renews annually the accreditation
of each Site Chairman, by inviting the Institution to
nominate one or more candidates and, if necessary, by holding discussions
at the site.  To be accredited, a prospective Site Chairman has
to satisfy the PS that proper arrangements have been made (i)~to
collaborate with the Site Manager and (ii)~to consult, represent
and inform the users.  One way of doing this is to
set up a ``Local Management Committee'' with the Site Manager as secretary
and containing representatives from all major user groupings and any RUGs.
(In the unlikely event of the agreed arrangements not being
satisfactorily implemented or falling into disuse,
accreditation may be withdrawn.)

In some cases -- for example where more than one Institution
shares a Starlink node -- a rotating Chairmanship (with
changes every two years or so) may be proposed.

During any periods when for any reason a site does not have
an accredited Site Chairman, the Site Manager will act as Site Chairman.
This does not, of course, include temporary absences, when the
Site Chairman may nominate a deputy.

\subsection{User Meetings}

The annual user meeting required by Starlink could be based on a SLUG
meeting or be specially convened.  In some cases, neighbouring sites
might wish to call a joint meeting -- in order to foster collaboration,
or because the sites are part of a larger organization.  Such joint
meetings also help the Project by reducing the total number of
meetings, something that could also be achieved by sites coordinating
their meetings so as to minimize travel for Starlink representatives.

The Project will do its best to comply with any requests from
sites for specific presentations or for particular staff to attend.

User meetings should be advertised and organized so as to facilitate
attendance by Remote User Groups.

\section{Responsibilities}
The following sections describe in general terms
which matters are normally (i)~dealt with by the
Site Manager; (ii)~discussed at SLUG meetings; and
(iii)~decided by the Site Chairman, typically through an LMC:

\subsection{Site Manager}

The Site Manager:
\begin{itemize}
\item is formally responsible within the Institution to local line
      management, but for the performance of his or her
      duties as a Starlink Site Manager
      is regarded as part of the Starlink Project;
\item has day-to-day responsibility for
      running the node (as described in the {\it Site Manager's Guide},
      \xref{SGP/25}{sgp25}{}), and for spending Starlink consumables money;
\item is the Project's principal contact point at the site for such
      matters as planning and installing hardware and
      software upgrades {\it etc}\,;
\item is the users' principal (but not sole) contact point for support and
      advice on the use of Starlink
      and for Starlink-related matters of concern to both the
      Project and local managements;~~and
\item is required to keep the users and the Site Chairman
      properly informed and, where appropriate, to promote open
      discussion through forums such as SLUGs and local or national VAX~Notes
      conferences.
\end{itemize}

\subsection{SLUGs}

The Starlink Local User Groups are informally constituted and
simply comprise all the users of the site concerned.  A SLUG is:
\begin{itemize}
\item the main forum for comments and suggestions 
      about all aspects of the running of the node.
\item an important source of suggestions on what
      new hardware and software is needed and should be included
      in the annual bid to the Starlink Panel;
\end{itemize}
SLUGs may be chaired by a user, with the Site Manager acting
as secretary, or {\it vice versa}.  The Site Manager and
Site Chairman should see that SLUG meetings are held regularly
and properly advertised.  The Project recommends a minimum of two
SLUG meetings per year, plus
additional ones if there is a particular demand from users or if
important issues arise that users would like to discuss.

A great strength of the SLUG system is that no user need feel excluded from
decisions about the running of the node.  It is important, therefore,
that users see the Site Chairman and the local committee as representing
them and taking account of their recommendations.

\subsection{Local Committees}

To allow some diversity in local management arrangements without
complicating the wording of the Site Contracts, the
Project's delegated management responsibilities and powers are vested
in the Site Chairman as an individual.  However, in most
cases the Site Chairman will be expected to form a panel representing
local users and to consult the panel over major issues.  Such panels
would normally:
\begin{itemize}
\item monitor users' views (for example by having representatives
      attend SLUG meetings)
      and take them into account when making decisions;
\item advertise their meetings in advance so that users can make
      representations to them:
\item include the Site Manager (who will be prepared to act as secretary);
\item inform users what has been discussed and what decisions have been made;
\item decide the final form of the annual site bid, subject to the
      agreement of the Head of Department;
\item decide the siting of hardware at the site, subject to Starlink
      rules on accessibility; and
\item determine the allocation of the various system resources to
      the different groups of users, types of work {\it etc},
      within overall policies promulgated by Starlink.
\end{itemize}
Starlink Project staff will not automatically attend these meetings,
but will do so on request from the Site Chairman.  One pattern
favoured by some (but not all) sites is where the Committee meets
immediately after a SLUG meeting and a Project representative
attends both meetings.  There must be at least one meeting per
year where Project staff can meet users, including members of
Remote User Groups.

\pagebreak

\appendix
\section{Starlink Site Chairman -- Terms of Reference}

At each Starlink site, it is the responsibility of the Site Chairman:
\begin{enumerate}
\item to establish, monitor and maintain consultative machinery,
      to ensure that the views of all users of the Starlink site
      are taken into account in its operation;
\item to ensure the implementation of Starlink policies concerning the
      operation of his or her Starlink site;
\item to monitor all aspects of the service to the users, including
      resources, availability, documentation and usage, paying special
      attention to the needs of remote users;
\item to resolve any conflicting demands on resources and to deal with
      user recommendations or complaints which cannot be dealt with by the
      Site Manager;
\item to make recommendations on any relevant matters of policy or
      operation to the Starlink management.
\end{enumerate}
\end{document}
