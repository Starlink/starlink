\documentclass[twoside,11pt]{article}
\pagestyle{myheadings}

% ----------------------------------------------------
% Fixed part
\newcommand{\stardoccategory}  {Starlink General Paper}
\newcommand{\stardocinitials}  {SGP}
\newcommand{\stardocsource}    {sgp\stardocnumber}
% Variable part
\newcommand{\stardocnumber}    {50.3}
\newcommand{\stardocauthors}   {M D Lawden}
\newcommand{\stardocdate}      {20 November 1997}
\newcommand{\stardoctitle}     {Starlink Document Styles}

\newcommand{\stardocabstract}  {
This document describes the various styles which are recommended for Starlink
documents.
It also explains how to use the templates which are provided by
Starlink to help authors create documents in a standard style.
\par
This paper is concerned mainly with conveying the ``look and feel" of the
various styles of Starlink document rather than describing the technical
details of how to produce them.
Other Starlink papers give recommendations for the detailed
aspects of document production, design, layout, and typography.
\par
The only style that is likely to be used by most Starlink authors is the
{\em Standard}\, style, described in section~\ref{standard_style}.}

%                                                            Section 1 - End *
% ? End of Document summary **************************************************

\newcommand{\stardocname}{\stardocinitials /\stardocnumber}
\markright{\stardocname}

\setlength{\textwidth}{160mm}
\setlength{\textheight}{230mm}
\setlength{\topmargin}{-2mm}
\setlength{\oddsidemargin}{0mm}
\setlength{\evensidemargin}{0mm}
\setlength{\parindent}{0mm}
\setlength{\parskip}{\medskipamount}
\setlength{\unitlength}{1mm}

% -----------------------
%  Hypertext definitions.
%  ======================
%  These are used by the LaTeX2HTML translator in conjunction with star2html.

%  Comment.sty: version 2.0, 19 June 1992
%  Selectively in/exclude pieces of text.
%
%  Author
%    Victor Eijkhout                                      <eijkhout@cs.utk.edu>
%    Department of Computer Science
%    University Tennessee at Knoxville
%    104 Ayres Hall
%    Knoxville, TN 37996
%    USA

%  Do not remove the %begin{latexonly} and %end{latexonly} lines (used by 
%  star2html to signify raw TeX that latex2html cannot process).
%begin{latexonly}
\makeatletter
\def\makeinnocent#1{\catcode`#1=12 }
\def\csarg#1#2{\expandafter#1\csname#2\endcsname}

\def\ThrowAwayComment#1{\begingroup
    \def\CurrentComment{#1}%
    \let\do\makeinnocent \dospecials
    \makeinnocent\^^L% and whatever other special cases
    \endlinechar`\^^M \catcode`\^^M=12 \xComment}
{\catcode`\^^M=12 \endlinechar=-1 %
 \gdef\xComment#1^^M{\def\test{#1}
      \csarg\ifx{PlainEnd\CurrentComment Test}\test
          \let\html@next\endgroup
      \else \csarg\ifx{LaLaEnd\CurrentComment Test}\test
            \edef\html@next{\endgroup\noexpand\end{\CurrentComment}}
      \else \let\html@next\xComment
      \fi \fi \html@next}
}
\makeatother

\def\includecomment
 #1{\expandafter\def\csname#1\endcsname{}%
    \expandafter\def\csname end#1\endcsname{}}
\def\excludecomment
 #1{\expandafter\def\csname#1\endcsname{\ThrowAwayComment{#1}}%
    {\escapechar=-1\relax
     \csarg\xdef{PlainEnd#1Test}{\string\\end#1}%
     \csarg\xdef{LaLaEnd#1Test}{\string\\end\string\{#1\string\}}%
    }}

%  Define environments that ignore their contents.
\excludecomment{comment}
\excludecomment{rawhtml}
\excludecomment{htmlonly}

%  Hypertext commands etc. This is a condensed version of the html.sty
%  file supplied with LaTeX2HTML by: Nikos Drakos <nikos@cbl.leeds.ac.uk> &
%  Jelle van Zeijl <jvzeijl@isou17.estec.esa.nl>. The LaTeX2HTML documentation
%  should be consulted about all commands (and the environments defined above)
%  except \xref and \xlabel which are Starlink specific.

\newcommand{\htmladdnormallinkfoot}[2]{#1\footnote{#2}}
\newcommand{\htmladdnormallink}[2]{#1}
\newcommand{\htmladdimg}[1]{}
\newenvironment{latexonly}{}{}
\newcommand{\hyperref}[4]{#2\ref{#4}#3}
\newcommand{\htmlref}[2]{#1}
\newcommand{\htmlimage}[1]{}
\newcommand{\htmladdtonavigation}[1]{}

%  Starlink cross-references and labels.
\newcommand{\xref}[3]{#1}
\newcommand{\xlabel}[1]{}

%  LaTeX2HTML symbol.
\newcommand{\latextohtml}{{\bf LaTeX}{2}{\tt{HTML}}}

%  Define command to re-centre underscore for Latex and leave as normal
%  for HTML (severe problems with \_ in tabbing environments and \_\_
%  generally otherwise).
\newcommand{\latex}[1]{#1}
\newcommand{\setunderscore}{\renewcommand{\_}{{\tt\symbol{95}}}}
\latex{\setunderscore}

% -----------
%  Debugging.
%  =========
%  Remove % from the following to debug links in the HTML version using Latex.

% \newcommand{\hotlink}[2]{\fbox{\begin{tabular}[t]{@{}c@{}}#1\\\hline{\footnotesize #2}\end{tabular}}}
% \renewcommand{\htmladdnormallinkfoot}[2]{\hotlink{#1}{#2}}
% \renewcommand{\htmladdnormallink}[2]{\hotlink{#1}{#2}}
% \renewcommand{\hyperref}[4]{\hotlink{#1}{\S\ref{#4}}}
% \renewcommand{\htmlref}[2]{\hotlink{#1}{\S\ref{#2}}}
% \renewcommand{\xref}[3]{\hotlink{#1}{#2 -- #3}}
%end{latexonly}


% ? Document specific commands ***********************************************
%                                                          Section 2 - Begin *

% Put in this section any \newcommand or \newenvironment commands you need to 
% define.  If you don't need to do this, leave this section alone.
% (If you've no idea what these command are, proceed immediately to Section 3.

%                                                            Section 2 - End *
% ? End of Document specific commands ****************************************

%  Front Page.
%  ===========
\renewcommand{\thepage}{\roman{page}}

\begin{document}
\thispagestyle{empty}

% ? Latex document front page heading ****************************************
%                                                          Section 3 - Begin *

% If you haven't specified a [software-version], then delete the line below
% containing "\stardocversion".
% If you haven't specified a [manual-type], then delete the line below
% containing "\stardocmanual".

\begin{latexonly}
   CCLRC / {\sc Rutherford Appleton Laboratory} \hfill {\bf \stardocname}\\
   {\large Particle Physics \& Astronomy Research Council}\\
   {\large Starlink Project}\\
   {\large \stardoccategory\ \stardocnumber}
   \begin{flushright}
   \stardocauthors\\
   \stardocdate
   \end{flushright}
   \vspace{-4mm}
   \rule{\textwidth}{0.5mm}
   \vspace{5mm}
   \begin{center}
   {\Huge\bf  \stardoctitle \\ [2.5ex]}

   \end{center}
   \vspace{5mm}

%                                                            Section 3 - End *
% ? End of Latex document front page heading *********************************

% ? Heading for Abstract *****************************************************
%                                                          Section 4 - Begin *

% Delete this section if you haven't supplied an Abstract in Section 1.

   \vspace{10mm}
   \begin{center}
      {\Large\bf Abstract}
   \end{center}

%                                                            Section 4 - End *
% ? End of Heading for Abstract **********************************************

\end{latexonly}

%  HTML documentation header.
%  ==========================
\begin{htmlonly}
   \xlabel{}
   \begin{rawhtml} <H1> \end{rawhtml}
      \stardoctitle
   \begin{rawhtml} </H1> \end{rawhtml}

% ? Picture ******************************************************************
%                                                          Section 5 - Begin *

% If you want to add a picture for display on the front page of the HTML
% version of your document, put the reference to it here.
% If you don't want to do this, leave this section alone.

%                                                            Section 5 - End *
% ? End of Picture ***********************************************************

   \begin{rawhtml} <P> <I> \end{rawhtml}
   \stardoccategory\ \stardocnumber \\
   \stardocauthors \\
   \stardocdate
   \begin{rawhtml} </I> </P> <H3> \end{rawhtml}
      \htmladdnormallink{CCLRC}{http://www.cclrc.ac.uk} /
      \htmladdnormallink{Rutherford Appleton Laboratory}
                        {http://www.cclrc.ac.uk/ral} \\
      \htmladdnormallink{Particle Physics \& Astronomy Research Council}
                        {http://www.pparc.ac.uk} \\
   \begin{rawhtml} </H3> <H2> \end{rawhtml}
      \htmladdnormallink{Starlink Project}{http://www.starlink.ac.uk/}
   \begin{rawhtml} </H2> \end{rawhtml}
   \htmladdnormallink{\htmladdimg{source.gif} Retrieve hardcopy}
      {http://www.starlink.ac.uk/cgi-bin/hcserver?\stardocsource}\\

%  HTML document table of contents. 
%  ================================
%  Add table of contents header and a navigation button to return to this 
%  point in the document (this should always go before the abstract \section). 
  \label{stardoccontents}
  \begin{rawhtml} 
    <HR>
    <H2>Contents</H2>
  \end{rawhtml}
  \htmladdtonavigation{\htmlref{\htmladdimg{contents_motif.gif}}
        {stardoccontents}}

% ? Abstract section *********************************************************
%                                                          Section 6 - Begin *

% Delete this section if you haven't supplied an Abstract in Section 1.

  \section{\xlabel{abstract}Abstract}

%                                                            Section 6 - End *
% ? End of Abstract section **************************************************

\end{htmlonly}

% ? Abstract text ************************************************************
%                                                          Section 7 - Begin *

% Delete this section if you haven't supplied an Abstract in Section 1.

\stardocabstract

%                                                            Section 7 - End *
% ? End of Abstract text *****************************************************

% ? Latex document Table of Contents *****************************************
%                                                          Section 8 - Begin *

% Delete this section if you do not want a Table of Contents in the Latex
% version of your document.
% (The HTML version of your document will always have a Table of Contents.)

\newpage
\begin{latexonly}
   \setlength{\parskip}{0mm}
   \tableofcontents
   \setlength{\parskip}{\medskipamount}
   \markright{\stardocname}
\end{latexonly}

%                                                            Section 8 - End *
% ? End of Latex document Table of Contents **********************************

\newpage
\renewcommand{\thepage}{\arabic{page}}
\setcounter{page}{1}

% ? Document Text ************************************************************
%                                                          Section 9 - Begin *

% Replace the place-holder [text] with the text of your document

\section{\label{introduction}\xlabel{introduction}Introduction}

This document describes the various styles that are recommended for Starlink
documents.
(Miscellaneous User Documents (MUDs) are not regarded as Starlink documents
in this context because they originate outside of the Project, so Starlink has
no control over their style.)
These styles are:
\begin{quote}
\begin{description}
\item [Standard] --
This is the ``default" style which is used for most Starlink documents.
Other styles are used by the Starlink Document Librarian to either
``package" documents written in {\em Standard} style, or to produce special
documents.
\item [Package] --
This style is for Starlink User Notes (larger than about 30 pages) which are
associated with Starlink Software Packages.
It may also be used as the basis for other document categories ({\em e.g.}
Cookbooks) which need to be made more attractive to readers.
\item [Introduction] --
This style is for the three components of the {\em New User's Document Pack},
(\xref{SUG}{sug}{},
 \xref{SUN/1}{sun1}{},
 \xref{SUN/145}{sun145}{}).
\item [Brochure] --
This style is for
\xref{SGP/31}{sgp31}{},
the document which describes the Starlink Project for outsiders.
\item [Glossy] --
This style is for promotional handouts and displays for exhibitions, visitors,
and new users.
\item [Bulletin] --
This style is for Starlink's newsletter, the
{\em \htmladdnormallink{Starlink Bulletin}
{http://www.starlink.ac.uk/bulletin.html}}.
\end{description}
\end{quote}
This document is concerned with the overall ``look and feel" of Starlink
documents.
It doesn't give detailed guidance on document production, design, layout, and
typography.
For guidance on these matters, please consult the following documents:
\begin{quote}
\begin{description}
\item [\xref{SC/9}{sc9}{}] --
 illustrates the most common typographical structures used in Starlink
 documents, and shows how to create them.
\item [\xref{SGP/28}{sgp28}{}] --
 gives general advice on how to write documents for Starlink.
\item [\xref{SUN/9}{sun9}{}] --
 shows how to use \LaTeX\ to create a document on paper.
\item [\xref{SUN/199}{sun199}{}] -- 
 shows how to add hypertext facilities to your documents.
\end{description}
\end{quote}

\section{\label{styles}\xlabel{styles}Styles}

\subsection{\label{standard_style}\xlabel{standard_style}Standard style}

The {\em Standard}\, style is the one you should use by default.
The other styles are specialist styles that are usually used only by the
Starlink Document Librarian.

The {\em Standard}\, style is based on \LaTeX\ document templates stored in
directory {\tt /star/docs}.
The names of the files which hold the templates are based on the codes
used to describe Starlink's document series (see
\xref{SUG}{sug}{} Section 6.1).
For example, the file holding the template for Starlink User Notes (SUNs)
is called {\tt sun.tex}.

To use a template, copy it from {\tt /star/docs}, replace the variable fields
with your own values, and add the document text.
The variable fields are at the front of the template and are enclosed by
brackets, for example {\tt [author]}.

The first set of variables ({\tt [number].[version]}, {\tt [author]},
{\tt [date]}, {\tt [title]}) are the things that appear at the top of a
document to identify it.

The next two ({\tt [software-version]}, {\tt [manual-type]}) are mainly used
by Programmers who are writing User Manuals for their software.
Delete these two lines if they are not appropriate for your document.
If you delete {\tt [software-version]} and {\tt [manual-type]},
also delete any other lines which contain the strings
\verb+\stardocversion+ or \verb+\stardocmanual+, otherwise you will get
\LaTeX\ error messages.

All but the shortest documents should have an abstract.
Replace {\tt [Text of abstract]} with your text.
If you don't supply an Abstract, delete all the lines (indicated as
{\tt ...}) between the following pairs of headings:
\begin{quote}
\begin{verbatim}
% ? Heading for abstract if used.
...
% ? End of heading for abstract.

% ? New section for abstract if used.
...
% ? End of new section for abstract

% ? Document Abstract. (if used)
...
% ? End of document abstract
\end{verbatim}
\end{quote}

As an example, here is the top section of a recently issued Starlink document
(SUN/28.11) showing how the variables have been replaced by actual values:

\begin{quote}
\begin{verbatim}
\documentstyle[11pt]{article}
\pagestyle{myheadings}

% ---------------------------------------------------------------
% ? Document identification
% Fixed part
\newcommand{\stardoccategory}  {Starlink User Note}
\newcommand{\stardocinitials}  {SUN}
\newcommand{\stardocsource}    {sun\stardocnumber}
% Variable part
\newcommand{\stardocnumber}    {28.11}
\newcommand{\stardocauthors}   {M.\,J.\,Bly}
\newcommand{\stardocdate}      {22nd August 1996}
\newcommand{\stardoctitle}     {NAG \\[2ex]
                                Numerical and Graphics Subroutine
                                Libraries}
\newcommand{\stardocversion}   {Mk 16/4}
\newcommand{\stardocmanual}    {User's Manual}
\newcommand{\stardocabstract}  {
The NAG Fortran and Graphics libraries are highly regarded in the field
of Numerical Computation and Graphics.
Starlink provides implementations of both for its UK sites, at double
precision, with interfaces to the Starlink GKS and standard X11 graphics
subsystems.
Single precision versions of the Fortran Library may also be available.}
% ? End of document identification

% ---------------------------------------------------------------
...
\end{verbatim}
\end{quote}

A good way of acquiring typographical skills is to look at existing Starlink
documents for features that you want to use, and then look at the source
code in {\tt /star/docs} to see how it is done.
The file name tells you the code of the document it contains.

\subsection{\label{package_style}\xlabel{package_style}Package style}

The {\em Package}\, style is simply a {\em Standard}\, style document with a
characteristic cover and binding.
The front cover has a coloured outside in a characteristic style, and a standard
inside containing a copyright notice, acknowledgement, address, {\em etc.}
The back cover is blank.
The binding is ``Spiral bound."

If you write a {\em Package}\, document, write it in {\em Standard}\, style
and submit it to Starlink for distribution.
If appropriate, a cover will be added by the Starlink Document Librarian who
will also organise the binding and copying, so you need not worry about these
details.
If you wish, supply one or more sample images (such as screen shots) to
be added to the cover.
These should illustrate the functionality of the software being described.
They will help to identify the document for readers.

An example of a typical cover in this style is shown in the Samples section.

\subsection{\label{introduction_style}\xlabel{introduction_style}Introduction style}

This style is used for components of the {\em New User's Document Pack}.
These documents are written and produced by the Starlink Document Librarian, 
so they need not be described in detail here.
They are characterised by their small size (A5), black spiral binding, and
coloured cover containing a picture of our mascot, Captain Starlink, against
a background image of the Orion Nebula.
They also have similar text to {\em Package}\, documents on their inside front
cover.

An example of a typical cover in this style is shown in the Samples section.

\subsection{\label{brochure_style}\xlabel{brochure_style}Brochure style}

This style is used only for SGP/31, which is really a Brochure describing the
Starlink Project for outsiders.
It is similar to the {\em Introduction}\, style but has wrap round binding
instead of spiral, a picture of the Vela Supernova Remnant instead of Orion, and
a graphical logo instead of Captain Starlink.
It is produced by the Starlink Document Librarian, so it need not be described
further here.

The cover is shown in the Samples section.
 
\subsection{\label{glossy_style}\xlabel{glossy_style}Glossy style}

This style is a single-sheet, double-sided A4 colour glossy, laid out in three
columns with a characteristic header and border.

Documents in this style are produced by the Starlink Document Librarian on a
PC using Pagemaker, so they need not be described further here.

An example is shown in the Samples section.

\subsection{\label{bulletin_style}\xlabel{bulletin_style}Bulletin style}

This is an A4-sized booklet laid out in two columns.
It has a characteristic front page and back page and usually contains several
coloured figures.
The text is produced by the Starlink Document Librarian using \LaTeX\ and
Pagemaker, and the figures are added by the printer from masters copies.

An example of the front cover design is shown in the Samples section.

This style is likely to change after issue 19 (September 1997) because the
editor will change from Mike Lawden to Anne Charles.

\section{\label{references}\xlabel{references}References}

\begin{quote}
\begin{description}
\item [\xref{SC/9}{sc9}{}] -- \LaTeX: Cookbook.
\item [\xref{SGP/28}{sgp28}{}] -- How to write good documents for Starlink.
\item [\xref{SUG}{sug}{}] -- Starlink User's Guide.
\item [\xref{SUN/9}{sun9}{}] -- \LaTeX: Document preparation system.
\item [\xref{SUN/199}{sun199}{}] -- 
 STAR2HTML: Convert Starlink documents to hypertext,
User's Manual.
\end{description}
\end{quote}


\appendix

\section{\label{samples}\xlabel{samples}Samples}

The following pages show samples of the different styles of Starlink document.
Coloured covers are reproduced in black-and-white -- this is adequate for
conveying the graphic design.

The samples are shown in the following order:
\begin{enumerate}
\item {\em 
 \htmladdnormallink{Standard}
 {http://www.starlink.ac.uk/star/docs/sgp50.htx/sgp50f1.gif}}\, style --
 front page.
\item {\em
 \htmladdnormallink{Package}
 {http://www.starlink.ac.uk/star/docs/sgp50.htx/sgp50f2.gif}}\, style --
 outside front cover.
\item {\em 
 \htmladdnormallink{Package}
 {http://www.starlink.ac.uk/star/docs/sgp50.htx/sgp50f3.gif}}\, style --
 inside front cover.
\item {\em 
 \htmladdnormallink{Introduction}
 {http://www.starlink.ac.uk/star/docs/sgp50.htx/sgp50f4.gif}}\, style --
 outside front cover.
\item {\em 
 \htmladdnormallink{Brochure}
  {http://www.starlink.ac.uk/star/docs/sgp50.htx/sgp50f5.gif}}\, style --
 outside front cover.
\item {\em 
 \htmladdnormallink{Glossy}
  {http://www.starlink.ac.uk/star/docs/sgp50.htx/sgp50f6.gif}}\, style --
 front page.
\item {\em 
 \htmladdnormallink{Bulletin}
 {http://www.starlink.ac.uk/star/docs/sgp50.htx/sgp50f7.gif}}\, style -- 
 front page.
\end{enumerate}

\end{document}
