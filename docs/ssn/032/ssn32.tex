\documentclass[11pt]{article}
\pagestyle{myheadings}

% -----------------------------------------------------------------------------
% ? Document identification
\newcommand{\stardoccategory}  {Starlink System Note}
\newcommand{\stardocinitials}  {SSN}
\newcommand{\stardocnumber}    {32.1}
\newcommand{\stardocsource}    {ssn\stardocnumber}
\newcommand{\stardocauthors}   {Geoff Mellor}
\newcommand{\stardocdate}      {8 November 1995}
\newcommand{\stardoctitle}     {Custom Jumpstart Tutorial}
% ? End of document identification
% -----------------------------------------------------------------------------

\newcommand{\stardocname}{\stardocinitials /\stardocnumber}
\markright{\stardocname}
\setlength{\textwidth}{160mm}
\setlength{\textheight}{230mm}
\setlength{\topmargin}{-2mm}
\setlength{\oddsidemargin}{0mm}
\setlength{\evensidemargin}{0mm}
\setlength{\parindent}{0mm}
\setlength{\parskip}{\medskipamount}
\setlength{\unitlength}{1mm}

% -----------------------------------------------------------------------------
%  Hypertext definitions.
%  ======================
%  These are used by the LaTeX2HTML translator in conjunction with star2html.

%  Comment.sty: version 2.0, 19 June 1992
%  Selectively in/exclude pieces of text.
%
%  Author
%    Victor Eijkhout                                      <eijkhout@cs.utk.edu>
%    Department of Computer Science
%    University Tennessee at Knoxville
%    104 Ayres Hall
%    Knoxville, TN 37996
%    USA

%  Do not remove the %begin{latexonly} and %end{latexonly} lines (used by 
%  star2html to signify raw TeX that latex2html cannot process).
%begin{latexonly}
\makeatletter
\def\makeinnocent#1{\catcode`#1=12 }
\def\csarg#1#2{\expandafter#1\csname#2\endcsname}

\def\ThrowAwayComment#1{\begingroup
    \def\CurrentComment{#1}%
    \let\do\makeinnocent \dospecials
    \makeinnocent\^^L% and whatever other special cases
    \endlinechar`\^^M \catcode`\^^M=12 \xComment}
{\catcode`\^^M=12 \endlinechar=-1 %
 \gdef\xComment#1^^M{\def\test{#1}
      \csarg\ifx{PlainEnd\CurrentComment Test}\test
          \let\html@next\endgroup
      \else \csarg\ifx{LaLaEnd\CurrentComment Test}\test
            \edef\html@next{\endgroup\noexpand\end{\CurrentComment}}
      \else \let\html@next\xComment
      \fi \fi \html@next}
}
\makeatother

\def\includecomment
 #1{\expandafter\def\csname#1\endcsname{}%
    \expandafter\def\csname end#1\endcsname{}}
\def\excludecomment
 #1{\expandafter\def\csname#1\endcsname{\ThrowAwayComment{#1}}%
    {\escapechar=-1\relax
     \csarg\xdef{PlainEnd#1Test}{\string\\end#1}%
     \csarg\xdef{LaLaEnd#1Test}{\string\\end\string\{#1\string\}}%
    }}

%  Define environments that ignore their contents.
\excludecomment{comment}
\excludecomment{rawhtml}
\excludecomment{htmlonly}

%  Hypertext commands etc. This is a condensed version of the html.sty
%  file supplied with LaTeX2HTML by: Nikos Drakos <nikos@cbl.leeds.ac.uk> &
%  Jelle van Zeijl <jvzeijl@isou17.estec.esa.nl>. The LaTeX2HTML documentation
%  should be consulted about all commands (and the environments defined above)
%  except \xref and \xlabel which are Starlink specific.

\newcommand{\htmladdnormallinkfoot}[2]{#1\footnote{#2}}
\newcommand{\htmladdnormallink}[2]{#1}
\newcommand{\htmladdimg}[1]{}
\newenvironment{latexonly}{}{}
\newcommand{\hyperref}[4]{#2\ref{#4}#3}
\newcommand{\htmlref}[2]{#1}
\newcommand{\htmlimage}[1]{}
\newcommand{\htmladdtonavigation}[1]{}

% Define commands for HTML-only or LaTeX-only text.
\newcommand{\html}[1]{}
\newcommand{\latex}[1]{#1}

% Use latex2html 98.2.
\newcommand{\latexhtml}[2]{#1}

%  Starlink cross-references and labels.
\newcommand{\xref}[3]{#1}
\newcommand{\xlabel}[1]{}

%  LaTeX2HTML symbol.
\newcommand{\latextohtml}{{\bf LaTeX}{2}{\tt{HTML}}}

%  Define command to re-centre underscore for Latex and leave as normal
%  for HTML (severe problems with \_ in tabbing environments and \_\_
%  generally otherwise).
\newcommand{\setunderscore}{\renewcommand{\_}{{\tt\symbol{95}}}}
\latex{\setunderscore}

% -----------------------------------------------------------------------------
%  Debugging.
%  =========
%  Remove % from the following to debug links in the HTML version using Latex.

% \newcommand{\hotlink}[2]{\fbox{\begin{tabular}[t]{@{}c@{}}#1\\\hline{\footnotesize #2}\end{tabular}}}
% \renewcommand{\htmladdnormallinkfoot}[2]{\hotlink{#1}{#2}}
% \renewcommand{\htmladdnormallink}[2]{\hotlink{#1}{#2}}
% \renewcommand{\hyperref}[4]{\hotlink{#1}{\S\ref{#4}}}
% \renewcommand{\htmlref}[2]{\hotlink{#1}{\S\ref{#2}}}
% \renewcommand{\xref}[3]{\hotlink{#1}{#2 -- #3}}
%end{latexonly}
% -----------------------------------------------------------------------------
% ? Document-specific \newcommand or \newenvironment commands.
% ? End of document-specific commands
% -----------------------------------------------------------------------------
%  Title Page.
%  ===========
\renewcommand{\thepage}{\roman{page}}
\begin{document}
\thispagestyle{empty}

%  Latex document header.
%  ======================
\begin{latexonly}
   CCLRC / {\sc Rutherford Appleton Laboratory} \hfill {\bf \stardocname}\\
   {\large Particle Physics \& Astronomy Research Council}\\
   {\large Starlink Project\\}
   {\large \stardoccategory\ \stardocnumber}
   \begin{flushright}
   \stardocauthors\\
   \stardocdate
   \end{flushright}
   \vspace{-4mm}
   \rule{\textwidth}{0.5mm}
   \vspace{5mm}
   \begin{center}
   {\Large\bf \stardoctitle}
   \end{center}
   \vspace{5mm}

% ? Heading for abstract if used.
%  \vspace{10mm}
%  \begin{center}
%     {\Large\bf Abstract}
%  \end{center}
% ? End of heading for abstract.
\end{latexonly}

%  HTML documentation header.
%  ==========================
\begin{htmlonly}
   \xlabel{}
   \begin{rawhtml} <H1> \end{rawhtml}
      \stardoctitle
   \begin{rawhtml} </H1> \end{rawhtml}

% ? Add picture here if required.
% ? End of picture

   \begin{rawhtml} <P> <I> \end{rawhtml}
   \stardoccategory\ \stardocnumber \\
   \stardocauthors \\
   \stardocdate
   \begin{rawhtml} </I> </P> <H3> \end{rawhtml}
      \htmladdnormallink{CCLRC}{http://www.cclrc.ac.uk} /
      \htmladdnormallink{Rutherford Appleton Laboratory}
                        {http://www.cclrc.ac.uk/ral} \\
      \htmladdnormallink{Particle Physics \& Astronomy Research Council}
                        {http://www.pparc.ac.uk} \\
   \begin{rawhtml} </H3> <H2> \end{rawhtml}
      \htmladdnormallink{Starlink Project}{http://www.starlink.ac.uk/}
   \begin{rawhtml} </H2> \end{rawhtml}
   \htmladdnormallink{\htmladdimg{source.gif} Retrieve hardcopy}
      {http://www.starlink.ac.uk/cgi-bin/hcserver?\stardocsource}\\

%  HTML document table of contents. 
%  ================================
%  Add table of contents header and a navigation button to return to this 
%  point in the document (this should always go before the abstract \section). 
  \label{stardoccontents}
  \begin{rawhtml} 
    <HR>
    <H2>Contents</H2>
  \end{rawhtml}
  \htmladdtonavigation{\htmlref{\htmladdimg{contents_motif.gif}}
        {stardoccontents}}

% ? New section for abstract if used.
% \section{\xlabel{abstract}Abstract}
% ? End of new section for abstract

\end{htmlonly}

% -----------------------------------------------------------------------------
% ? Document Abstract. (if used)
%  ==================
% ? End of document abstract
% -----------------------------------------------------------------------------
% ? Latex document Table of Contents (if used).
%  ===========================================
% \newpage
\begin{latexonly}
   \large
   \setlength{\parskip}{0mm}
   \tableofcontents
   \setlength{\parskip}{\medskipamount}
   \markright{\stardocname}
\end{latexonly}
% ? End of Latex document table of contents
% -----------------------------------------------------------------------------
\newpage
\renewcommand{\thepage}{\arabic{page}}
%\setcounter{page}{1}

\large

\section {Introduction}


Custom Jumpstart is a facility provided by Sun
which allows (almost) hands free Solaris installation.  

It provides the following benefits
\begin{itemize}
\item Unattended Solaris installation 
\item Facilitates multiple installations 
\item Facilitates subsequent operating system upgrades 
\item Automates post-installation configuration tasks
\item Enables easy system reconfiguration 
\item Reduces the need to backup system partitions
\end{itemize}

It requires perhaps a day's work to configure initially
but the subsequent savings in adminstration time make it 
really worthwhile. Whilst of most value to the larger 
installations, a Site Manager with more than three Suns can benefit
from using this process.

This tutorial is not meant to be a comprehensive guide but
just details of how I configured my system to utilise Jumpstart.
I don't claim it to be the best method but it works for me.
Hopefully it will save you some configuration time.


\section{Overview}

The key stages in configuring a jumpstart system are the following:

\begin{itemize}
\item Set up automatic system configuration
\item Configuring the Servers
\item Set up configuration directory
\item Add the install clients
\end{itemize}

Jumpstart works in the following way:

When the client is booted (using the command {\tt boot net - install}),
it issues a {\tt RARP} call to determine its Internet address from  the
boot server. It then uses {\tt tftp} to obtain the jumpstart boot image
from the boot server. From the client, the jumpstart boot image then issues a
{\tt hostconfig} request for boot parameters which are returned by the
bootserver. The client then mounts the root partition from the boot server
and executes {\tt /kernel/unix}. The bootstrap is now complete. The
client searches for the configuration and install servers using bootparams 
information and
then mounts the configuration and installation directories. The client
then runs {\tt suninstall} and installs itself.

\section{Automatic System Configuration}

The automatic system configuration involves collecting and setting up
the information for a system to introduce it to the network. The following 
items of information are required:

\begin{itemize}
\item Locale (only needed for non-English computing environments)
\item Ethernet Address
\item Host name
\item IP address
\item Time Zone
\end{itemize}

It is necessary to add entries for the ethernet, hostname and ip address
of the new system to your network tables. Make sure that your network
netmasks database also contains the correct entry for the subnet on which
this system will reside.

Finally it is necessary to include an entry in the timezone network map
for each domain. Using the admintool timezone database; the domainname
should be entered in the {\tt Host Name} field and the {\tt Timezone}
should be {\tt GB-Eire}. Then it is necessary to modify the hosts
database to specify the server that you wish to supply the time to the
systems {\em e.g.}:

\begin{quote}
{\tt 143.210.36.51   ltsun1 timehost \# Sparcstation 2}
\end{quote}

Finally edit the {\tt /etc/nsswitch.conf} file on all NIS+ servers, to
include a {\tt timezone} entry:

\begin{quote}
{\tt timezone: nisplus}
\end{quote}

\section{Configuring the Servers}

There are three server functions required by jumpstart:

\begin{itemize}
\item {\bf Configuration Server} This server contains the customised
configuration files used during the automatic installation process.
\item {\bf Install Server} This server provides the Solaris distribution media
\item {\bf Boot Server} This server provides clients on the same subnet with 
the ability to boot and access the network.
\end{itemize}

All my clients are on the same subnet, so I used one machine to provide
all three functions. If you have clients on a separate subnet then you
will have to consult the Sun documentation to determine how to
configure a separate boot server.

It is possible to either serve the Solaris distribution from a CD drive
or to copy an image to disk and serve that.  Sun strongly recommend the
latter method as it greatly improves performance.  As I had sufficient
free space (approx 400Mb), I chose to create a disk image under the
directory {\tt /export/install}.


This is achieved by first mounting the CD and then typing:

\begin{verbatim}
      # cd /cdrom/sol_2_4_hw1194_sparc/s0
      # ./setup_install_server /export/install
\end{verbatim}

Then export this directory {\it read-only} with root access for the install
client.

\section{Setting up a configuration directory}

The configuration directory contains all the files required to customise
your client installation. It contains the following items:

\begin{itemize}
\item A {\it class profile} for each category of install clients
\item Possibly {\it begin} scripts which are executed before the
      class files (ie before the actual installation of the software)
\item Possibly {\it finish} scripts which are executed post-installation 
\item A {\tt rules} file specifying which of the above file should be
      executed for each client or category of clients.
\item The script {\tt check} which produces a file called {\tt rules.ok}
      from the {\tt rules} file. This is the actual rules file used 
      during the installation process.
\end{itemize}


First create a directory on your configuration server (e.g. {\tt
/export/jumpstart}) and export it as below:

\begin{quote}
{\tt share -F nfs -o ro,anon=0 /export/jumpstart}
\end{quote}

Then change directory to your Solaris distribution. In my case, it
is the disk image in {\tt /export/install}.  Next,
copy the template files in the {\tt auto\_install\_sample} directory, 
{\em  e.g.}:

\begin{quote}
{\tt cp -r auto\_install\_sample/* /export/jumpstart}
\end{quote}

This populates your configuration directory with a set of examples
which you can use to create your configuration files.


\subsection{The {\tt rules} file}

Begin first by creating the {\tt rules} file. This is a lookup table
consisting of one or more rules that define matches between system
attributes and the class profiles and scripts. It is possible to match
on a variety of attributes such as architecture, model, domainname {\em etc}.

As I have only a relatively small number of mainly individual machines,
I just simply match on the hostname. My example {\tt rules} file is
in Appendix \ref{app:rules}. There are five entries to each line. The
first entry is the keyword and the second is the keyword value. The
next three entries specify the begin script, the profile and the finish
script respectively.

\subsection{Begin scripts}

Begin scripts are mainly used to create derived profiles. My setup
is not complicated enough to require such facilities so I don't use
them. Hence the ``{\tt -}'' in the {\tt rules} file.


\subsection{The Class Profile}

The class profile defines how the Solaris system shall be installed on the
system. Essentially, it provides answers to  most of the questions normally
asked during an interactive upgrade. The profile is used to specify
whether an upgrade or full installation take place. It can also specify
such things as the system type, disk partitioning and the software group
to be installed. An example profile is listed in Appendix \ref{app:profile}.

The first keyword in the example is {\tt install\_type}. This is required in
every profile to determine the type of installation (either {\tt
initial\_install} or {\tt upgrade}). 

The next keywords relate to the {\tt system\_type}. As I am configuring a
server without diskless clients, I then define the {\tt num\_clients} and
{\tt client\_swap} as 0.

I chose to do explicit partitioning and define all of my filesystem sizes
as can be seen from the next series of commands. It is also possible to use
{\tt existing} or {\tt default} partitioning. 

On my server, I have chosen to install the full Solaris release ({\tt
SUNWcall}) but the installed software can be completely tailored with
{\tt cluster} and {\tt package} commands if required.

For further  details, on profiles, see the Sun documentation.

\subsection{Finish scripts}


Finish scripts are where jumpstart gets exciting.  They allow you to do
all those post installation tasks automatically.

In my example, in Appendix \ref{app:finish}, I create some standard
directories and then copy in my standard files. This makes use of the
fact that during the installation the Jumpstart directory is mounted
onto the directory specified by the variable {\tt SI\_CONFIG\_DIR}.
Thus it is possible to keep master versions of these files and use them
for each install.  I create a subdirectory of my jumpstart directory
({\tt /files}) and place the files there. Note, that during the
installation, the new root directory is mounted under {\tt /a} so the
destination of the copy commands are all realative to {\tt /a}.

Next I run a script to set the root password at install time (This
script is provided as an example file when you configure the jumpstart
configuration directory).

It is also posible to install the patches automatically. The mandatory
patches which exist on the Solaris 2.4  distribution (and thus on my
disk image) can be installed in the standard way\footnote{The Solaris
distribution is mounted under the directory {\tt /cdrom} regardless of
whether it is a disk image or the distribution CD.}.  Then I use the
finish script to install some additional patches too.  First I created
a subdirectory of the Solaris distribution directory called: 

\begin{quote}
{\tt /export/install/additional\_patches}
\end{quote}

which can then be referenced as {\tt /cdrom/additional\_patches} from
the installation.  Then I install them as if onto a diskless client
with the {\tt installpatch -R } option.

Finally there are a few things that cannot be done during the install
and need to be done once the system has been rebooted (eg printer
configuration). So, I copy a final configuration file into the {\tt
/etc/init.d} directory and link to it from the {\tt /etc/rc3.d}
directory.  Thus it gets run at startup time. The final line of the
file deletes the link and so it only gets run once.

\subsection{Final configuration file}

My example final configuration file is shown in Appendix \ref{app:config}.
The main set of commands in here configure my printer systems. Then I also
run the {\tt newaliases} command.

The only thing that I have not found it possible to do is to run the
{\tt nisclient} script and this is the only thing I have left to configure
after the automatic upgrade. 

\subsection{The {\tt check} script}

Once you have written your files, run the {\tt check} script in the 
configuration directory. It will check the validity of the {\tt rules}
file and the {\tt class} files. If all is in order, it will create
a {\tt rules.ok} file for use during the installation.


\section{Adding Install Clients}


Now it is only necessary to add the install clients to the install server.
If your client is a dataless client and you are running a name service,
first use the {\tt hostmanager} form in {\tt admintool} to enter details
for your system. Apart from the standard information, you are required to
specify the fileserver for the {\tt /usr} and {\tt /usr/kvm} filesystems
and also the O.S. release. Note that the {\tt Remote Install} button
should be set to {\tt disable} regardless of whether you are installing
over a network or not.

Then it is necessary to run the {\tt add\_install\_client} script as below:

\begin{quote}
{\tt ./add\_install\_client -c }{\it server:jumpstart\_dir  host\_name
architecture}
\end{quote}

{\em e.g.}:

\begin{quote}
{\tt ./add\_install\_client -c ltsun0:/export/jumpstart ltsun7 sun4m}
\end{quote}

This creates an entry in the bootparams file on the install server.

\section{Commencing Installation}

Now it should be just necessary to boot your client using the syntax:

\begin{quote}
{\tt boot net - install} 
\end{quote}

and the auto installation should just get going. Time not just for a
coffee, but lunch as well!


\section{Troubleshooting}

These are the two problems that I encountered:


\begin{itemize}

\item
If the client complains ``{\tt Timeout waiting for ARP/RARP packet ...}''
then it can't find a system that knows about it on the network. Verify
the client details (hostname, ethernet address) on the server. 

\item
Make sure that you export the Jumpstart directory {\it read-only} as it
says in the manual. This disk is normally read-write on my system and I
found out the hard way that it matters!  It is a known bug that
exporting the disk read-write can allow the disk image to become
slightly corrupted - this manifests itself in a failure of the install
process to produce a {\tt hostname.le0} file on the client.  The client
cannot then  configure its network devices. If you get to this point,
you can recreate your disk image or add a line to the finish script to
create this file with the hostname in during install.

\end{itemize}

\section{References}

The relevant Sun documentation that describes this process in detail
is the ``{\it SPARC: Installing Solaris Software}'' manual. The Sun
``{\it Network Administration Student Guide}'' provides some additional
detail and is more coherent than the basic manual.

Neither book provides much detail on the details of writing finish scripts
and most of the examples described in this tutorial were determined by
trial and error. I have just obtained a new book from Sun Express
called ``{\it Automating Solaris Installations: A Custom Jumpstart
Guide}'' which provides many examples (on a floppy disk too!). I have
not had chance to review this book yet but beware with some of the
listed examples as I have already seen reports of some of them not
working correctly.

\appendix

\newpage
\section{Example {\tt rules} file}
\label{app:rules}

\begin{small}
\begin{verbatim}
#
# Geoff's rules file
#
# v1.0 1/8/95
#
#############################################################################
#
# RULE_KEYWORD AND RULE_VALUE DESCRIPTIONS
#
#
# rule_keyword    rule_value Type       rule_value Description
# ------------    ---------------       ----------------------
#  any            minus sign (-)        always matches
#  arch           text                  system's architecture type
#  domainname     text                  system's domain name 
#  disksize       text range            system's disk size 
#                                          disk device name (text)
#                                          disk size (MBytes range) 
#  hostname       text                  system's host name
#  installed      text text             system's installed version of Solaris
#                                          disk device name (text)
#                                          OS release (text)
#  karch          text                  system's kernel architecture
#  memsize        range                 system's memory size (MBytes range)
#  model          'text'                system's model number
#  network        text                  system's IP address
#  totaldisk      range                 system's total disk size (MBytes range)
#
#############################################################################

# rule keywords and rule values   begin script  profile          finish script
# -----------------------------   ------------  -------          -------------

hostname ltsun7                   -             ltsun7-profile   server-finish

hostname darc                     -             darc-upgrade     -
 
hostname darc2                    -             darc2-profile    client-finish

hostname euve1                    -             euve1-profile    server-finish

\end{verbatim}
\end{small}

\newpage


\section{Example class profile}
\label{app:profile}

\begin{small}
\begin{verbatim}

#
# ltsun7-profile
#
# installation of Solaris 2.4  - 1/8/95
#
install_type    initial_install
system_type     server
num_clients     0
client_swap     0
#
#
partitioning    explicit
filesys         c0t3d0s0 50 /
filesys         c0t3d0s1 150 swap
filesys         c0t3d0s3 75 /var
filesys         c0t3d0s4 100 /opt
filesys         c0t3d0s5 250 /usr
filesys         c0t3d0s7 free /export/home
filesys         c0t1d0s2 all /soft1
#
cluster         SUNWCall
#

\end{verbatim}
\end{small}

\newpage

\section{Example finish file}
\label{app:finish}

\begin{small}
\begin{verbatim}

#!/bin/sh
#
# server-finish
#
# grm 1/8/95
#
# finish script for server
#
umask = 022
#
# first create a few directories
#
echo "create directories"
#
mkdir /a/alpha
mkdir /a/soft1 /a/soft2 /a/scratch /a/star /a/temp
mkdir /a/euve /a/rdac_ftp /a/data1 /a/ftp
ln -s /ftp/pub /a/pub
#
 # now copy a few files in
#
echo "copy files"
#
# common files
#
cp ${SI_CONFIG_DIR}/files/sendmail.cf /a/etc/mail/sendmail.cf
cp ${SI_CONFIG_DIR}/files/.??* /a
cp ${SI_CONFIG_DIR}/files/defaultrouter /a/etc/defaultrouter
cp ${SI_CONFIG_DIR}/files/inetd.conf   /a/etc/inetd.conf
cp ${SI_CONFIG_DIR}/files/resolv.conf /a/etc/resolv.conf
cp ${SI_CONFIG_DIR}/files/syslog.conf  /a/etc/syslog.conf
cp ${SI_CONFIG_DIR}/files/shells      /a/etc/shells
#
# server files
#
cp ${SI_CONFIG_DIR}/files/xdm-init     /a/etc/init.d/xdm
cp ${SI_CONFIG_DIR}/files/dfstab-xrcad1  /a/etc/dfs/dfstab
cp ${SI_CONFIG_DIR}/files/auto_master-cad /a/etc/auto_master
#
# replace the xdm config files:
#
rm -f /a/usr/openwin/lib/xdm/*
cp ${SI_CONFIG_DIR}/files/xdm/* /a/usr/openwin/lib/xdm
#
# now do some links
#
ln -s /etc/init.d/xdm /a/etc/rc3.d/S30xdm 
ln -s /usr/local/bin/perl /a/usr/bin/perl
ln -s /soft2/X11R5 /a/opt/X11R5
ln -s /soft2/X11R5/include/X11 /a/usr/include/X11
#set client password
#
echo "set client password"
#
${SI_CONFIG_DIR}/set_root_pw
#
# Unpack man pages
#
catman
#
# add some hosts in
#
cat ${SI_CONFIG_DIR}/files/hosts.main >> /a/etc/inet/hosts
#
# Patch installation
echo "install patches"
#
# The mandatory stuff:
#
cd /cdrom/Patches
./install_patches /a
#
#
# now the other patches for Solaris 2.4
# (this assumes that an extra directory
#  called additional_patches has been created
#  on the disk image of the installation CD
#  and the patches moved there)
#
cd /cdrom/additional_patches/101945-29
./installpatch -R /a . > /a/var/sadm/install_data/extra_patches.log 2>&1
#
cd /cdrom/additional_patches/101959-03
./installpatch -R /a . >> /a/var/sadm/install_data/extra_patches.log 2>&1
#
cd /cdrom/additional_patches/102216-01
./installpatch -R /a . >> /a/var/sadm/install_data/extra_patches.log 2>&1
#
cd /cdrom/additional_patches/102218-02
./installpatch -R /a . >> /a/var/sadm/install_data/extra_patches.log 2>&1
#
cd /cdrom/additional_patches/102922-02
./installpatch -R /a . >> /a/var/sadm/install_data/extra_patches.log 2>&1
#
cd /cdrom/additional_patches/102292-01
./installpatch -R /a . >> /a/var/sadm/install_data/extra_patches.log 2>&1
#
#
# final configuration file (for after reboot)
#
cp ${SI_CONFIG_DIR}/files/S99finish_install /a/etc/init.d/S99finish_install
ln -s /etc/init.d/S99finish_install /a/etc/rc3.d/S99finish_install
#

\end{verbatim}
\end{small}

\newpage

\section{Example final configuration file}
\label{app:config}

\begin{small}
\begin{verbatim}

#!/bin/sh
#
# /etc/init.d/S99finish_install
#
# grm 11/8/95
#
# This script is intended to be run once after system installation
#
# finishes off the install after reboot:
# 1)configures system printers for all systems
# 2)runs newaliases
#
#
echo "Running finish_install script"
#
# define PATH
#
PATH=/usr/bin:/bin:/usr/sbin:/usr/lib/nis; export PATH
#
# Identify print servers
#
echo "Configuring printers ..."
#
lpsystem -t s5 ltsun0
lpsystem -t s5 ltsun1
#
# configure and enable the printers
#
lpadmin -p star_citoh -s ltsun1 -T unknown -I any \
 -D "C.Itoh 5000      - Line printer                - Starlink Terminal
Room"
accept star_citoh
enable star_citoh
#
lpadmin -p star_canon -s ltsun1 -T unknown -I any \
 -D "Canon LBP-8II    - Text printer                - Starlink Terminal
Room"
accept star_canon
enable star_canon
#
lpadmin -p star_ps1 -s ltsun1 -T unknown -I any \
 -D "HP laserjet IIID - Postscript printer (duplex) - Starlink Terminal
Room"
accept star_ps1
enable star_ps1
#
# set default printer destination
#
lpadmin -d star_citoh
#
#
# now do newaliases
#
echo "Creating Sendmail DBM files ..."
#
newaliases
#
# now delete link to startup file
#
rm /etc/rc3.d/S99finish_install
#
# and reboot
#
shutdown -i6 -y -g0

\end{verbatim}
\end{small}

\end{document}
