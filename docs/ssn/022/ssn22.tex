\documentclass[11pt,nolof,noabs]{starlink}


%------------------------------------------------------------------------------
\stardoccategory    {Starlink System Note}
\stardocinitials    {SSN}
\stardocnumber      {22.3}
\stardocauthors     {P.\,T.\,Wallace \& M.\,J.\,Bly}
\stardocdate        {8 June 1992}
\stardoctitle       {Coordinate Systems for Pixel Arrays}
%------------------------------------------------------------------------------

%------------------------------------------------------------------------------
% Add any \providecommand or \newenvironment commands here

\setlength{\unitlength}{1mm}

%------------------------------------------------------------------------------

\begin{document}
\scfrontmatter

\section{Background}

Despite early attempts to establish conventions for pixel array coordinate
systems, several different ones are now in use. The matter was discussed at the
27-28 June 1983 Environment Review Workshop and those present were asked to
submit written recommendations for a new, definitive, convention to PTW. These
have now been collated, with the following outcome.

\section{General}

Pixel numbering is distinct from picture coordinates. The former involves pairs
of integers, while the latter involves pairs of real numbers. In both cases,
however, the first number is called the x-coordinate and the second the
y-coordinate;  x increases to the right and y increases upwards.

\section{Pixel Numbering}

The bottom left-hand pixel is numbered (1,1). The two integers specifying the
pixel are the same as the Fortran array indices, for arrays beginning (1,1).

\section{Picture coordinates}

Pixels are rectangular cells, not grid points. The bottom left-hand corner  of
the bottom left-hand pixel has picture coordinates (0.0,0.0).

\newpage
\section{Summary}

For a Fortran array DIMENSION A(I,J):

\vspace{10mm}

\begin{picture}(120,80)

\put(30,20){\line(1,0){75}}
\put(30,40){\line(1,0){65}}
\put(30,60){\line(1,0){35}}
\put(85,60){\line(1,0){35}}
\put(75,80){\line(1,0){45}}

\put(30,20){\line(0,1){50}}
\put(60,20){\line(0,1){45}}
\put(90,20){\line(0,1){25}}
\put(90,55){\line(0,1){25}}
\put(120,50){\line(0,1){30}}

\put(0,0){(0.0,0.0)}
\put(10,5){\vector(4,3){20}}

\put(45,30){\makebox(0,0){A(1,1)}}
\put(45,50){\makebox(0,0){A(1,2)}}
\put(75,30){\makebox(0,0){A(2,1)}}
\put(105,70){\makebox(0,0){A(I,J)}}

\end{picture}

\vspace{10mm}

Note that this conforms to the existing IDI output routines, but that  \emph{(i)}\/~the current GKS draw pixel array implementation is reversed in y and
\emph{(ii)}\/~ DAOPHOT's coordinate system differs by 0.5 pixels in each
direction.

\end{document}
