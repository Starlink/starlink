\documentclass[twoside,11pt]{article}

% ? Specify used packages
% \usepackage{graphicx}        %  Use this one for final production.
% \usepackage[draft]{graphicx} %  Use this one for drafting.
% ? End of specify used packages

\pagestyle{myheadings}

% -----------------------------------------------------------------------------
% ? Document identification
% Fixed part
\newcommand{\stardoccategory}  {Starlink System Note}
\newcommand{\stardocinitials}  {SSN}
\newcommand{\stardocsource}    {ssn\stardocnumber}
\newcommand{\stardoccopyright} 
{Copyright \copyright\ 2002 Council for the Central Laboratory of the Research Councils}

% Variable part - replace [xxx] as appropriate.
\newcommand{\stardocnumber}    {64.5}
\newcommand{\stardocauthors}   {A J Chipperfield}
\newcommand{\stardocdate}      {3 September 2002}
\newcommand{\stardoctitle}     {ADAM --- Organization of Application Packages}
\newcommand{\stardocabstract}  {
This document explains the standard way in which 
\xref{ADAM}{sg4}{} applications packages
are set up to run on Starlink machines. It covers the systems permitting
use of the applications direct from the shell and from the Interactive Command
Language (\xref{ICL}{sg5}{}) on Unix.
\par
This document is relevant to applications programmers, the Starlink Software
Librarian and Site Managers.
}
% ? End of document identification
% -----------------------------------------------------------------------------

% +
%  Name:
%     ssn.tex
%
%  Purpose:
%     Template for Starlink System Note (SSN) documents.
%     Refer to SUN/199
%
%  Authors:
%     AJC: A.J.Chipperfield (Starlink, RAL)
%     BLY: M.J.Bly (Starlink, RAL)
%     PWD: Peter W. Draper (Starlink, Durham University)
%
%  History:
%     17-JAN-1996 (AJC):
%        Original with hypertext macros, based on MDL plain originals.
%     16-JUN-1997 (BLY):
%        Adapted for LaTeX2e.
%     13-AUG-1998 (PWD):
%        Converted for use with LaTeX2HTML version 98.2 and
%        Star2HTML version 1.3.
%      1-FEB-2000 (AJC):
%        Add Copyright statement in LaTeX
%     {Add further history here}
%
% -

\newcommand{\stardocname}{\stardocinitials /\stardocnumber}
\markboth{\stardocname}{\stardocname}
\setlength{\textwidth}{160mm}
\setlength{\textheight}{230mm}
\setlength{\topmargin}{-2mm}
\setlength{\oddsidemargin}{0mm}
\setlength{\evensidemargin}{0mm}
\setlength{\parindent}{0mm}
\setlength{\parskip}{\medskipamount}
\setlength{\unitlength}{1mm}

% -----------------------------------------------------------------------------
%  Hypertext definitions.
%  ======================
%  These are used by the LaTeX2HTML translator in conjunction with star2html.

%  Comment.sty: version 2.0, 19 June 1992
%  Selectively in/exclude pieces of text.
%
%  Author
%    Victor Eijkhout                                      <eijkhout@cs.utk.edu>
%    Department of Computer Science
%    University Tennessee at Knoxville
%    104 Ayres Hall
%    Knoxville, TN 37996
%    USA

%  Do not remove the %begin{latexonly} and %end{latexonly} lines (used by 
%  LaTeX2HTML to signify text it shouldn't process).
%begin{latexonly}
\makeatletter
\def\makeinnocent#1{\catcode`#1=12 }
\def\csarg#1#2{\expandafter#1\csname#2\endcsname}

\def\ThrowAwayComment#1{\begingroup
    \def\CurrentComment{#1}%
    \let\do\makeinnocent \dospecials
    \makeinnocent\^^L% and whatever other special cases
    \endlinechar`\^^M \catcode`\^^M=12 \xComment}
{\catcode`\^^M=12 \endlinechar=-1 %
 \gdef\xComment#1^^M{\def\test{#1}
      \csarg\ifx{PlainEnd\CurrentComment Test}\test
          \let\html@next\endgroup
      \else \csarg\ifx{LaLaEnd\CurrentComment Test}\test
            \edef\html@next{\endgroup\noexpand\end{\CurrentComment}}
      \else \let\html@next\xComment
      \fi \fi \html@next}
}
\makeatother

\def\includecomment
 #1{\expandafter\def\csname#1\endcsname{}%
    \expandafter\def\csname end#1\endcsname{}}
\def\excludecomment
 #1{\expandafter\def\csname#1\endcsname{\ThrowAwayComment{#1}}%
    {\escapechar=-1\relax
     \csarg\xdef{PlainEnd#1Test}{\string\\end#1}%
     \csarg\xdef{LaLaEnd#1Test}{\string\\end\string\{#1\string\}}%
    }}

%  Define environments that ignore their contents.
\excludecomment{comment}
\excludecomment{rawhtml}
\excludecomment{htmlonly}

%  Hypertext commands etc. This is a condensed version of the html.sty
%  file supplied with LaTeX2HTML by: Nikos Drakos <nikos@cbl.leeds.ac.uk> &
%  Jelle van Zeijl <jvzeijl@isou17.estec.esa.nl>. The LaTeX2HTML documentation
%  should be consulted about all commands (and the environments defined above)
%  except \xref and \xlabel which are Starlink specific.

\newcommand{\htmladdnormallinkfoot}[2]{#1\footnote{#2}}
\newcommand{\htmladdnormallink}[2]{#1}
\newcommand{\htmladdimg}[1]{}
\newenvironment{latexonly}{}{}
\newcommand{\hyperref}[4]{#2\ref{#4}#3}
\newcommand{\htmlref}[2]{#1}
\newcommand{\htmlimage}[1]{}
\newcommand{\htmladdtonavigation}[1]{}
\newcommand{\HTMLcode}[2][]{}

% Define commands for HTML-only or LaTeX-only text.
\newcommand{\html}[1]{}
\newcommand{\latex}[1]{#1}

% Use latex2html 98.2.
\newcommand{\latexhtml}[2]{#1}

%  Starlink cross-references and labels.
\newcommand{\xref}[3]{#1}
\newcommand{\xlabel}[1]{}

%  LaTeX2HTML symbol.
\newcommand{\latextohtml}{\LaTeX2\texttt{HTML}}

%  Define command to re-centre underscore for Latex and leave as normal
%  for HTML (severe problems with \_ in tabbing environments and \_\_
%  generally otherwise).
\renewcommand{\_}{\texttt{\symbol{95}}}

% -----------------------------------------------------------------------------
%  Debugging.
%  =========
%  Remove % from the following to debug links in the HTML version using Latex.

% \newcommand{\hotlink}[2]{\fbox{\begin{tabular}[t]{@{}c@{}}#1\\\hline{\footnotesize #2}\end{tabular}}}
% \renewcommand{\htmladdnormallinkfoot}[2]{\hotlink{#1}{#2}}
% \renewcommand{\htmladdnormallink}[2]{\hotlink{#1}{#2}}
% \renewcommand{\hyperref}[4]{\hotlink{#1}{\S\ref{#4}}}
% \renewcommand{\htmlref}[2]{\hotlink{#1}{\S\ref{#2}}}
% \renewcommand{\xref}[3]{\hotlink{#1}{#2 -- #3}}
%end{latexonly}
% -----------------------------------------------------------------------------
% ? Document-specific \newcommand or \newenvironment commands.
\newcommand{\dash}{--}
\begin{htmlonly}
\renewcommand{\dash}{-}
\end{htmlonly}
% ? End of document-specific commands
% -----------------------------------------------------------------------------
%  Title Page.
%  ===========
\renewcommand{\thepage}{\roman{page}}
\begin{document}
\thispagestyle{empty}

%  Latex document header.
%  ======================
\begin{latexonly}
   CCLRC / \textsc{Rutherford Appleton Laboratory} \hfill \textbf{\stardocname}\\
   {\large Particle Physics \& Astronomy Research Council}\\
   {\large Starlink Project\\}
   {\large \stardoccategory\ \stardocnumber}
   \begin{flushright}
   \stardocauthors\\
   \stardocdate
   \end{flushright}
   \vspace{-4mm}
   \rule{\textwidth}{0.5mm}
   \vspace{5mm}
   \begin{center}
   {\Large\textbf{\stardoctitle}}
   \end{center}
   \vspace{5mm}

% ? Heading for abstract if used.
   \vspace{10mm}
   \begin{center}
      {\Large\textbf{Abstract}}
   \end{center}
% ? End of heading for abstract.
\end{latexonly}

%  HTML documentation header.
%  ==========================
\begin{htmlonly}
   \xlabel{}
   \begin{rawhtml} <H1> \end{rawhtml}
      \stardoctitle
   \begin{rawhtml} </H1> <HR> \end{rawhtml}

   \begin{rawhtml} <P> <I> \end{rawhtml}
   \stardoccategory\ \stardocnumber \\
   \stardocauthors \\
   \stardocdate
   \begin{rawhtml} </I> </P> <H3> \end{rawhtml}
      \htmladdnormallink{CCLRC / Rutherford Appleton Laboratory}
                        {http://www.cclrc.ac.uk} \\
      \htmladdnormallink{Particle Physics \& Astronomy Research Council}
                        {http://www.pparc.ac.uk} \\
   \begin{rawhtml} </H3> <H2> \end{rawhtml}
      \htmladdnormallink{Starlink Project}{http://www.starlink.ac.uk/}
   \begin{rawhtml} </H2> \end{rawhtml}
   \htmladdnormallink{\htmladdimg{source.gif} Retrieve hardcopy}
      {http://www.starlink.ac.uk/cgi-bin/hcserver?\stardocsource}\\

%  HTML document table of contents. 
%  ================================
%  Add table of contents header and a navigation button to return to this 
%  point in the document (this should always go before the abstract \section). 
  \label{stardoccontents}
  \begin{rawhtml} 
    <HR>
    <H2>Contents</H2>
  \end{rawhtml}
  \htmladdtonavigation{\htmlref{\htmladdimg{contents_motif.gif}}
        {stardoccontents}}

% ? New section for abstract if used.
  \section{\xlabel{abstract}Abstract}
% ? End of new section for abstract

\end{htmlonly}

% -----------------------------------------------------------------------------
% ? Document Abstract. (if used)
%  ==================
\stardocabstract
% ? End of document abstract

% -----------------------------------------------------------------------------
% ? Latex Copyright Statement
%  =========================
\begin{latexonly}
  \newpage
  \vspace*{\fill}
  \stardoccopyright
\end{latexonly}
% ? End of Latex copyright statement

% -----------------------------------------------------------------------------
% ? Latex document Table of Contents (if used).
%  ===========================================
\newpage
\begin{latexonly}
  \setlength{\parskip}{0mm}
  \tableofcontents
  \setlength{\parskip}{\medskipamount}
  \markboth{\stardocname}{\stardocname}
\end{latexonly}
% ? End of Latex document table of contents
% -----------------------------------------------------------------------------
\cleardoublepage
\renewcommand{\thepage}{\arabic{page}}
\setcounter{page}{1}

% ? Main text
\section{\xlabel{introduction}Introduction}
This document explains the standard way in which 
\xref{ADAM}{sg4}{} applications packages
are set up to run on Starlink machines. It covers the systems permitting
use of the applications direct from the Unix shell and from the 
\xref{Interactive Command Language (ICL)}{sg5}{}. 

Changes in this version of the document are listed in Section \ref{changes}.

When a new package is being developed, the way in which it will fit into the
scheme should be discussed with the Starlink Software Librarian and the Head 
of Applications.
\xref{KAPPA}{sun95}{}
provides a good example of how things are done and may be used as a template.
Deviations from the scheme will be permitted but compliance with it will ease
the task of setting up and maintaining the package.

In this document  \textit{package} means
`the name of the package (in upper case for environment variables)'.

\section{\xlabel{packages}Packages}
\subsection{\xlabel{general}General}
\label{pack_gen}
The term `package' in this document may be taken to mean a related group of 
ADAM applications. Packages are described as \textit{standard} if they are
installed at every Starlink site or \textit{option} if the are only installed
when requested by users.
Allowance is also
made for \textit{local}\/ packages which are non-Starlink packages set up for 
general use at particular sites.
It is assumed that packages may be run directly from the shell or from ICL;
where this is not the case, the inappropriate files are omitted from the
installation.

In most cases applications in the package will all be linked into a single 
monolith (see 
\xref{SUN/144}{sun144}{monoliths}) for efficient running from ICL and to save
disk space.
Packages could however be defined to consist of any mixture of task types 
and procedures \dash\ they could, in fact, include elements from other packages.
Because a package may well consist of applications written by a number of
different authors, the term \textit{Package Administrator} is used for the
person with overall responsibility for the package.

Executable files, shell scripts, ICL command files and compiled 
\xref{interface files}{sun115}{} 
associated with a package will usually be installed in a directory referred to
by environment variable \textit{package}\_DIR.
For most applications packages this directory is 
\texttt{/star/bin/\textit{package}}.
If the package can be run direct from the shell, \textit{package}\_DIR should
also contain links with the name of each application in a monolith pointing
to that monolith.

\textit{e.g.}\ \texttt{ \% ln \$KAPPA\_DIR/kappa\_pm \$KAPPA\_DIR/add }

Individual interface files will also be required in addition to the monolithic 
interface file in this case.

\subsection{\xlabel{package_help_libraries}Package Help Libraries}
Each package should provide a \textit{Package Help Library}.
This is a Starlink HELP library (see 
\xref{SUN/124}{sun124}{}).
The structure of the HELP library is at the Package Administrator's discretion
but at the very least it should contain an overall description of the package
with subtopics for each application.

The usual format is described below:
\begin{description}
\item[Level 0] A general description of the package.
\item[Level 1] A description of individual applications and other general 
topics such as `Getting Started'.
\item[Level 2] As required. Each application will normally have a 
subtopic `parameters' which has subtopics describing each parameter. 
These parameter subtopics may be referred to in parameter help
specifications in task interface files (see 
\xref{SUN/115}{sun115}{}).
\end{description}
Package Help Libraries will usually be installed in directory 
\texttt{/star/help/\textit{package}} and named \texttt{\textit{package}.shl}.
Environment variable \textit{package}\_HELP should be defined to point to the 
Package Help Library and this used in the relevant commands.

\subsection{\xlabel{shell_package_startup_scripts}Shell Package Startup Scripts}
Generally, to start up a package for use direct from the shell, the 
user will type:
\begin{quote}
\texttt{\% \textit{package}}
\end{quote}
The Starlink login procedures will have defined command, \textit{package}, which, 
when issued by the user, will cause commands in the \textit{Package Startup 
Script} to be obeyed. 

The Package Startup Script will define commands for 
running the individual applications and for obtaining help. 
It should also issue a `Welcome' message stating the package name and version.
Typically it will contain mainly alias commands of the form:
\begin{quote}
\texttt{\% alias 
\textit{application} \$\textit{package}\_DIR/\textit{application}}

\textit{e.g.}:

\texttt{\% alias add \$KAPPA\_DIR/add}
\end{quote}
(Note that here \$KAPPA\_DIR/add is a link to the KAPPA monolith as described
in Section \ref{pack_gen}).

The Package Startup Script should be provided by the Package Administrator and
be installed in \textit{package}\_DIR.


\subsection{\xlabel{icl_package_definition_files}ICL Package Definition Files}
Generally, to start up a package under 
\xref{ICL}{sg5}{}, the user will type:
\begin{quote}
\texttt{ICL> \textit{package}}
\end{quote}
This will cause ICL commands in the \textit{Package Definition Command File} to be
obeyed. All but the simplest packages will contain a Package Definition 
Command File which is a file containing ICL commands to:
\begin{enumerate}
\item Define the commands which the user will use to run the package.
\item Specify the source(s) of help information for the package.
\item Display information about the package.
\end{enumerate}
For example, the Package Definition Command File for KAPPA on Unix
could be something like:
\begin{quote}
\begin{verbatim}
{ KAPPA - Package Definition Command File}

{ Re-define the top-level help topic
DEFHELP KAPPA $KAPPA_HELP 0

{ Define the individual commands
DEFINE ADD $KAPPA_DIR/kappa_pm
DEFHELP ADD $KAPPA_HELP

DEFINE (APER)ADD $KAPPA_DIR/kappa_pm
DEFHELP APERADD $KAPPA_DIR/KAPPA

... etc...

PRINT
PRINT  " --    Initialised for KAPPA    --"
PRINT  " -- Version 0.8-SU, 1993 January --"
PRINT
\end{verbatim}
\end{quote}

The Package Definition Command File is controlled by the Package Administrator
and should be installed in \textit{package}\_DIR.

For very simple packages, a Package Definition Command File may not be
appropriate \dash\ the commands required to run such a package may be set up
directly by the overall system startup procedures.

\section{\xlabel{the_overall_system}The Overall System}
\subsection{\xlabel{the_adam_packages_help_library}The ADAM\_PACKAGES Help Library}
\label{adamhelp}
A Starlink HELP library giving general descriptions of all Starlink packages 
will be maintained in \texttt{/star/help/adam\_package.shl} by the Starlink 
Software Librarian.
The top-level topic  will give a list, with a single line
description, of all the \textit{standard} and \textit{option} packages and will be
displayed if the user types:
\begin{quote} \begin{verbatim}
ICL> HELP PACKAGES
\end{verbatim} \end{quote}
The second level will be subtopics for the individual packages.
Each subtopic will give a brief description of the package and describe how
to start using it. Before any packages have been started up by the user, the
command:
\begin{quote} 
\texttt{ICL> HELP \textit{package}}
\end{quote}
will display the appropriate subtopic.

Environment variable ADAM\_PACKAGES points to this help file.

\subsection{\xlabel{icl_startup_command_files}ICL Startup Command
Files\label{icl_startup_command_files}}
ICL uses environment variables to identify files containing 
commands which it will obey automatically before taking input from any file 
specified as a parameter of the ICL command, or prompting for input.
The environment variables, in the order they are accessed, are:

\begin{tabular}{ll}
ICL\_LOGIN\_SYS   & Intended for `system' login commands\\
ICL\_LOGIN\_LOCAL & Intended for local site login commands\\
ICL\_LOGIN      & Intended for user's login commands
\end{tabular}

For Starlink sites, ICL\_LOGIN\_SYS points to a file, controlled by the
Starlink Software Librarian which contains ICL commands which:
\begin{enumerate}
\item For all Starlink packages, define, for the ICL
help system, the relevant entry in the ADAM\-\_PACKAGES help library.
\item For each \textit{standard}\/ package, define an
\textit{ICL Package Startup Command} which, if issued by the user, will
start up the package for use with ICL (usually by 
\xref{LOAD}{sg5}{LOAD}ing the Package Definition
Command File).
\item For each \textit{option}\/ package, check whether the package is installed
at the site by checking for the existence of an appropriate file (usually the
Package Definition Command File).
If the package is installed, an ICL Package Startup Command is defined as for
\textit{standard} packages; if not, an ICL Package Startup Command is defined 
which, if issued, will inform the user politely that the package is not 
available at the site\footnote{As a temporary measure, this is also done for 
\textit{standard} packages which have not yet been set up to run from ICL.}.
\item Check if environment variable LADAM\-\_PACKAGES defines a file which
exists and, if it does, 
\xref{LOAD}{sg5}{LOAD} the file (see  Section \ref{ladampacks}).
\item Print a brief introductory message.
\end{enumerate}

For example, for KAPPA (a \textit{standard} package), on Unix it will contain:
\begin{quote}
\begin{verbatim}
{  Definitions for KAPPA }
DEFHELP KAPPA $ADAM_PACKAGES KAPPA
DEFSTRING KAPPA LOAD $KAPPA_DIR/kappa
\end{verbatim}
\end{quote}

For CCDPACK (an \textit{option} package), it contains:
\begin{quote}
\begin{verbatim}
{  CCDPACK  }
DEFHELP CCDPACK $ADAM_PACKAGES CCDPACK
IF FILE_EXISTS("$CCDPACK_DIR/ccdpack.icl") 
  DEFSTRING CCDPACK LOAD $CCDPACK_DIR/ccdpack
ELSE
  DEFSTRING CCDPACK NOTINSTALLED CCDPACK
ENDIF
\end{verbatim}
\end{quote}
Where NOTINSTALLED is a procedure which politely tells the user that the
package is not available.
Note that, because IF statements can only be included in ICL procedures (and not
directly in command files), the command file first defines a hidden procedure
containing the appropriate code for all \textit{option} packages and then obeys
it to define the required Package Startup Command.

\subsection{\xlabel{ladam_packages}LADAM\_PACKAGES}
\label{ladampacks}
This is an environment variable pointing to an ICL command file which,
if it exists, will be 
\xref{LOAD}{sg5}{LOAD}ed during the ICL\_LOGIN\_SYS sequence.
The file is intended to define both the Package Startup Command and the source 
of introductory help for \textit{local} packages.

LADAM\-\_PACKAGES will be controlled by the local Site Manager and 
may also contain any other site-specific login commands.

\subsection{\xlabel{starlink_login_actions}Starlink Login Actions}
\subsubsection{/star/etc/login}
The following environment variables are defined in \texttt{/star/etc/login}
which must be `source'ed in order to use any Starlink software.

\begin{tabular}{ll}
Variable name & Normal setting\\
ICL\_LOGIN\_SYS   & /star/etc/icl\_login\_sys.icl\\
ADAM\_PACKAGES & /star/help/adam\_packages.shl\\
\\
and for standard packages
\\
\textit{package}\_DIR & /star/bin/\textit{package}\\
\textit{package}\_HELP & /star/help/\textit{package}\\
\end{tabular}

For \textit{option} packages, the existence of an appropriate file within the
package is checked to determine whether or not the package is installed.
If it is, \textit{package}\_DIR and \textit{package}\_HELP for the package will
be defined.

\subsubsection{/star/etc/cshrc}
\textit{Shell Package Startup Commands} are defined in \texttt{/star/etc/cshrc}
which must be `source'ed in order to use any Starlink software.
For \textit{standard} packages a command is defined which will `source' 
the Package Startup Script to start up the package for use from the shell.
For \textit{option} packages, the existence of an appropriate file within the
package is checked to determine whether or not the package is installed.
If it is, a Startup Command is defined as for \textit{standard} 
packages;
if not, a command is defined which will politely tell the user that
the package has not been installed.

\section{\xlabel{the_effect}The Effect}
\label{effect}
The following example session shows the sort of effect it is hoped to achieve
by means of this scheme. It does not pretend to be an exact copy of what you
would see \dash\ in particular the help text is probably out of date.
\small
\begin{quote}
\begin{verbatim}
% icl

ICL (UNIX) Version 3.0  04/08/94

  - Type HELP package_name for help on specific Starlink packages
  -   or HELP PACKAGES for a list of all Starlink packages
  - Type HELP [command] for help on ICL and its commands

ICL> HELP PACKAGES 
  
PACKAGES 
  
   The following ADAM applications packages are available from Starlink: 
  
   Standard Packages: 
    CATAPP    -  Catalogue applications
    CONVERT   -  Data format conversion. 
    FIGARO    -  General data-reduction. 
    KAPPA     -  Image processing. 
    SST       -  Simple Software Tools 
    UTILITIES -  Miscellaneous useful tools
  
   Option Packages: 
    ASTERIX   -  X-ray data analysis. 
    CCDPACK   -  CCD data reduction 
    DAOPHOT   -  Stellar photometry. 

    ... etc...  
  
  Additional information available: 
  
  ASTERIX    CATAPP     CCDPACK    CONVERT    DAOPHOT    FIGARO     KAPPA 

    ... etc...

Topic? KAPPA 
  
KAPPA  

  KAPPA---the Kernel APplication PAckage---currently comprises applications for
  general image processing, many of which will function with data of 
  dimensionality other than two; data visualisation, with flexible control of 
  the location and size of pictures; and the manipulation of NDF components. 

     ... etc...

  To make the commands of KAPPA available, type: 
  
      ICL> KAPPA 
  
Topic? 
ICL> HELP KAPPA 
  
KAPPA  

  KAPPA---the Kernel APplication PAckage---currently comprises applications for
  general image processing, many of which will function with data of 
  dimensionality other than two; data visualisation, with flexible control of 
  the location and size of pictures; and the manipulation of NDF components. 

     ... etc...

  To make the commands of KAPPA available, type: 
  
      ICL> KAPPA 
  
Topic? 
ICL> KAPPA 
 
 --      Initialised for KAPPA      -- 
 --   Version 0.8-SU, 1993 January  -- 
 
 Type HELP KAPPA or KAPHELP for KAPPA help 
  
ICL> HELP KAPPA 
  
Help

   Welcome to the KAPPA online help system.  Here you can display
   details about a specific KAPPA command or more-general information
   such as what KAPPA can do and how to use it.  

   ... etc...

  Additional information available:

  ADD        APERADD    BLINK      BLOCK      CADD       CDIV       CENTROID
  Changes_to_KAPPA      CHPIX      Classified_commands   CLEANER    CMULT

   ... etc...
 
Topic? ADD 
 
ADD 
  
  Adds two NDF data structures. 
  
  Usage:

     ADD IN1 IN2 OUT

  Description: 
  
     The routine adds two NDF data structures pixel-by-pixel to produce 
     a new NDF. 
  
  Additional information available: 
  
  Parameters Examples    Notes      Authors    History 
  
ADD Subtopic? PARAMETERS
  
ADD 
  
  Parameters 
  
    For information on individual parameters, select from the list below: 
  
    Additional information available: 
  
    IN1        IN2        OUT        TITLE 
  
ADD Parameters Subtopic? IN1 
  
ADD 
  
  Parameters 
  
    IN1 
  
      IN1 = NDF (Read) 
         First NDF to be added. 
  
ADD Parameters Subtopic? 
ADD Subtopic? 
Topic? 
ICL> ASTERIX

      ASTERIX  is not installed at this site.
   If you really want it, contact your Site Manager.

ICL> EXIT 
%
\end{verbatim}
\end{quote}
\normalsize

\section{\xlabel{summary}Summary}
The Starlink Software Librarian maintains:
\begin{description}
\item[/star/etc/login] which defines environment variables required by
\textit{standard} and installed \textit{option} packages.
\item[/star/etc/cshrc] which defines Shell Package Startup Commands for 
\textit{standard} and installed \textit{option} packages to be run from the shell.
\item[ICL\_LOGIN\_SYS] an ICL command file, automatically LOADed at ICL startup,
which defines ICL Package Startup Commands and basic help for all Starlink 
packages. (In the case of non-installed \textit{option} packages, the Startup 
Command will inform the user than the package is not available.)
\item[ADAM\_PACKAGES] which is a Starlink HELP file giving basic information
about all available packages.
\end{description}
If required, Site Managers provide and maintain:
\begin{description}
\item[LADAM\_PACKAGES] analogous to ADAM\_PACKAGES but for \textit{local} packages.
\end{description}
Package Administrators provide and maintain:
\begin{description}
\item[Package Startup Script] a shell script defining commands to be used when 
running the package directly from the shell.
\item[Package Definition Command File] an ICL command file defining commands
and help topics to be used when running the package from ICL.
\item[Package Help File] a Starlink HELP file providing detailed help 
information about the package.
\end{description}

\begin{thebibliography}{9}
\bibitem{icl} J A Bailey, \textit{ICL --- The New ADAM Command Language - Users
Guide}, Joint Astronomy Centre.
\bibitem{kappa} M J Currie, \textit{KAPPA --- Kernel Application Package},
Starlink User Note 95.
\bibitem{sgp20} M D Lawden, \textit{Starlink Software Management}, Starlink
General Paper 20.
\bibitem{sun144} A J Chipperfield, \textit{ADAM --- Unix Version},
Starlink User Note 144.
\bibitem{sun124} P T Wallace, \textit{HELP --- Interactive Help System}
Starlink User Note 124.
\bibitem{ifls} A J Chipperfield, \textit{ADAM --- Interface File Reference Manual},
Starlink User Note 115.
\bibitem{ssn44} A J Chipperfield, \textit{ADAM --- Installation Guide}, Starlink
System Note 44.
\end{thebibliography}

\section{\xlabel{document_changes}Document Changes}
\label{changes}
This document has been reformatted and references to the VAX/VMS setup removed.
There are no material changes for the Unix systems.

\end{document}
