%
% Writing Catalogue and Image Servers for GAIA and CURSA.
%
% Copyright 2000  Starlink, CCLRC.
%
% A.C. Davenhall (Edinburgh), 11/3/00.
%

\documentclass[twoside,11pt]{article}

% ? Specify used packages
% \usepackage{graphicx}        %  Use this one for final production.
% \usepackage[draft]{graphicx} %  Use this one for drafting.
% ? End of specify used packages

\pagestyle{myheadings}

% -----------------------------------------------------------------------------
% ? Document identification
% Fixed part
\newcommand{\stardoccategory}  {Starlink System Note}
\newcommand{\stardocinitials}  {SSN}
\newcommand{\stardocsource}    {ssn\stardocnumber}
\newcommand{\stardoccopyright}
{Copyright \copyright\ 2000 Council for the Central Laboratory of the Research Councils}

% Variable part - replace [xxx] as appropriate.
\newcommand{\stardocnumber}    {75.1}
\newcommand{\stardocauthors}   {A.C.~Davenhall}
\newcommand{\stardocdate}      {26 July 2000}
\newcommand{\stardoctitle}     {Writing Catalogue and Image Servers for
GAIA and CURSA}
\newcommand{\stardocabstract}
{GAIA and CURSA can interrogate remote catalogues via the Internet to
return lists of objects which satisfy a given criterion.  GAIA can also
extract images of a specified region of sky from remote databases.
Similar facilities are also available in other packages, such as the
ESO {\it SkyCat}\/ image display tool and the Gemini observing tool.
This functionality is achieved by having servers running on the remote
systems which accept queries sent by GAIA \emph{etc}, interrogate their
local copies of the catalogues to find the data which satisfy the query
and return them.  The servers can communicate with GAIA \emph{etc.} because
the query is sent, and the results returned, in a standard format.  This
document describes how to write such servers.

\begin{latexonly}
\vspace{5mm}
\end{latexonly}

\begin{center}
{\bf Who Should Read this Document?}
\end{center}

This document is aimed at programmers who intend to write servers which
will provide catalogue and image data for GAIA \emph{etc}.}
% ? End of document identification
% -----------------------------------------------------------------------------

% +
%  Name:
%     ssn.tex
%
%  Purpose:
%     Template for Starlink System Note (SSN) documents.
%     Refer to SUN/199
%
%  Authors:
%     AJC: A.J.Chipperfield (Starlink, RAL)
%     BLY: M.J.Bly (Starlink, RAL)
%     PWD: Peter W. Draper (Starlink, Durham University)
%
%  History:
%     17-JAN-1996 (AJC):
%        Original with hypertext macros, based on MDL plain originals.
%     16-JUN-1997 (BLY):
%        Adapted for LaTeX2e.
%     13-AUG-1998 (PWD):
%        Converted for use with LaTeX2HTML version 98.2 and
%        Star2HTML version 1.3.
%      1-FEB-2000 (AJC):
%        Add Copyright statement in LaTeX
%     {Add further history here}
%
% -

\newcommand{\stardocname}{\stardocinitials /\stardocnumber}
\markboth{\stardocname}{\stardocname}
\setlength{\textwidth}{160mm}
\setlength{\textheight}{230mm}
\setlength{\topmargin}{-2mm}
\setlength{\oddsidemargin}{0mm}
\setlength{\evensidemargin}{0mm}
\setlength{\parindent}{0mm}
\setlength{\parskip}{\medskipamount}
\setlength{\unitlength}{1mm}

% -----------------------------------------------------------------------------
%  Hypertext definitions.
%  ======================
%  These are used by the LaTeX2HTML translator in conjunction with star2html.

%  Comment.sty: version 2.0, 19 June 1992
%  Selectively in/exclude pieces of text.
%
%  Author
%    Victor Eijkhout                                      <eijkhout@cs.utk.edu>
%    Department of Computer Science
%    University Tennessee at Knoxville
%    104 Ayres Hall
%    Knoxville, TN 37996
%    USA

%  Do not remove the %begin{latexonly} and %end{latexonly} lines (used by
%  LaTeX2HTML to signify text it shouldn't process).
%begin{latexonly}
\makeatletter
\def\makeinnocent#1{\catcode`#1=12 }
\def\csarg#1#2{\expandafter#1\csname#2\endcsname}

\def\ThrowAwayComment#1{\begingroup
    \def\CurrentComment{#1}%
    \let\do\makeinnocent \dospecials
    \makeinnocent\^^L% and whatever other special cases
    \endlinechar`\^^M \catcode`\^^M=12 \xComment}
{\catcode`\^^M=12 \endlinechar=-1 %
 \gdef\xComment#1^^M{\def\test{#1}
      \csarg\ifx{PlainEnd\CurrentComment Test}\test
          \let\html@next\endgroup
      \else \csarg\ifx{LaLaEnd\CurrentComment Test}\test
            \edef\html@next{\endgroup\noexpand\end{\CurrentComment}}
      \else \let\html@next\xComment
      \fi \fi \html@next}
}
\makeatother

\def\includecomment
 #1{\expandafter\def\csname#1\endcsname{}%
    \expandafter\def\csname end#1\endcsname{}}
\def\excludecomment
 #1{\expandafter\def\csname#1\endcsname{\ThrowAwayComment{#1}}%
    {\escapechar=-1\relax
     \csarg\xdef{PlainEnd#1Test}{\string\\end#1}%
     \csarg\xdef{LaLaEnd#1Test}{\string\\end\string\{#1\string\}}%
    }}

%  Define environments that ignore their contents.
\excludecomment{comment}
\excludecomment{rawhtml}
\excludecomment{htmlonly}

%  Hypertext commands etc. This is a condensed version of the html.sty
%  file supplied with LaTeX2HTML by: Nikos Drakos <nikos@cbl.leeds.ac.uk> &
%  Jelle van Zeijl <jvzeijl@isou17.estec.esa.nl>. The LaTeX2HTML documentation
%  should be consulted about all commands (and the environments defined above)
%  except \xref and \xlabel which are Starlink specific.

\newcommand{\htmladdnormallinkfoot}[2]{#1\footnote{#2}}
\newcommand{\htmladdnormallink}[2]{#1}
\newcommand{\htmladdimg}[1]{}
\newenvironment{latexonly}{}{}
\newcommand{\hyperref}[4]{#2\ref{#4}#3}
\newcommand{\htmlref}[2]{#1}
\newcommand{\htmlimage}[1]{}
\newcommand{\htmladdtonavigation}[1]{}
\newcommand{\HTMLcode}[2][]{}

% Define commands for HTML-only or LaTeX-only text.
\newcommand{\html}[1]{}
\newcommand{\latex}[1]{#1}

% Use latex2html 98.2.
\newcommand{\latexhtml}[2]{#1}

%  Starlink cross-references and labels.
\newcommand{\xref}[3]{#1}
\newcommand{\xlabel}[1]{}

%  LaTeX2HTML symbol.
\newcommand{\latextohtml}{\LaTeX2\texttt{HTML}}

%  Define command to re-centre underscore for Latex and leave as normal
%  for HTML (severe problems with \_ in tabbing environments and \_\_
%  generally otherwise).
\renewcommand{\_}{\texttt{\symbol{95}}}

% -----------------------------------------------------------------------------
%  Debugging.
%  =========
%  Remove % from the following to debug links in the HTML version using Latex.

% \newcommand{\hotlink}[2]{\fbox{\begin{tabular}[t]{@{}c@{}}#1\\\hline{\footnotesize #2}\end{tabular}}}
% \renewcommand{\htmladdnormallinkfoot}[2]{\hotlink{#1}{#2}}
% \renewcommand{\htmladdnormallink}[2]{\hotlink{#1}{#2}}
% \renewcommand{\hyperref}[4]{\hotlink{#1}{\S\ref{#4}}}
% \renewcommand{\htmlref}[2]{\hotlink{#1}{\S\ref{#2}}}
% \renewcommand{\xref}[3]{\hotlink{#1}{#2 -- #3}}
%end{latexonly}
% -----------------------------------------------------------------------------
% ? Document-specific \newcommand or \newenvironment commands.

% Define commands for displaying angles as sexagesimal hours and minutes
% or degrees and minutes.

\newcommand{\tmin}   {\mbox{$^{\rm m}\!\!.$}}
\newcommand{\hm}[3] {$#1^{\rm h}\,#2\tmin#3$}
\newcommand{\dm}[2] {$#1^{\circ}\,#2\raisebox{-0.5ex}{$^{'}$}$}
\newcommand{\arcmin} {\raisebox{-0.5ex}{$^{'}$} }

\newcommand{\arcsec} {$\hspace{-0.05em}\raisebox{-0.5ex}
                     {$^{'\hspace{-0.1em}'}$}
                     \hspace{-0.7em}.\hspace{-0.05em}$}
\newcommand{\tsec}   {\mbox{$^{\rm s}\!\!.$}}
\newcommand{\hms}[4] {$#1^{\rm h}\,#2^{\rm m}\,#3\tsec#4$}
\newcommand{\dms}[4] {$#1^{\circ}\,#2\raisebox{-0.5ex}{$^{'}$}\,#3\arcsec#4$}

% ? End of document-specific commands
% -----------------------------------------------------------------------------
%  Title Page.
%  ===========
\renewcommand{\thepage}{\roman{page}}
\begin{document}
\thispagestyle{empty}

%  Latex document header.
%  ======================
\begin{latexonly}
   CCLRC / \textsc{Rutherford Appleton Laboratory} \hfill \textbf{\stardocname}\\
   {\large Particle Physics \& Astronomy Research Council}\\
   {\large Starlink Project\\}
   {\large \stardoccategory\ \stardocnumber}
   \begin{flushright}
   \stardocauthors\\
   \stardocdate
   \end{flushright}
   \vspace{-4mm}
   \rule{\textwidth}{0.5mm}
   \vspace{5mm}
   \begin{center}
   {\Large\textbf{\stardoctitle}}
   \end{center}
   \vspace{5mm}

% ? Heading for abstract if used.
   \vspace{10mm}
   \begin{center}
      {\Large\textbf{Abstract}}
   \end{center}
% ? End of heading for abstract.
\end{latexonly}

%  HTML documentation header.
%  ==========================
\begin{htmlonly}
   \xlabel{}
   \begin{rawhtml} <H1> \end{rawhtml}
      \stardoctitle
   \begin{rawhtml} </H1> <HR> \end{rawhtml}

   \begin{rawhtml} <P> <I> \end{rawhtml}
   \stardoccategory\ \stardocnumber \\
   \stardocauthors \\
   \stardocdate
   \begin{rawhtml} </I> </P> <H3> \end{rawhtml}
      \htmladdnormallink{CCLRC / Rutherford Appleton Laboratory}
                        {http://www.cclrc.ac.uk} \\
      \htmladdnormallink{Particle Physics \& Astronomy Research Council}
                        {http://www.pparc.ac.uk} \\
   \begin{rawhtml} </H3> <H2> \end{rawhtml}
      \htmladdnormallink{Starlink Project}{http://www.starlink.ac.uk/}
   \begin{rawhtml} </H2> \end{rawhtml}
   \htmladdnormallink{\htmladdimg{source.gif} Retrieve hardcopy}
      {http://www.starlink.ac.uk/cgi-bin/hcserver?\stardocsource}\\

%  HTML document table of contents.
%  ================================
%  Add table of contents header and a navigation button to return to this
%  point in the document (this should always go before the abstract \section).
  \label{stardoccontents}
  \begin{rawhtml}
    <HR>
    <H2>Contents</H2>
  \end{rawhtml}
  \htmladdtonavigation{\htmlref{\htmladdimg{contents_motif.gif}}
        {stardoccontents}}

% ? New section for abstract if used.
  \section{\xlabel{abstract}Abstract}
% ? End of new section for abstract

\end{htmlonly}

% -----------------------------------------------------------------------------
% ? Document Abstract. (if used)
%  ==================
\stardocabstract
% ? End of document abstract

% -----------------------------------------------------------------------------
% ? Latex document Table of Contents (if used).
%  ===========================================
\newpage
\section*{Accessing this document}

A hypertext version of this document is available.  To access it on
Starlink systems type:

\begin{quote}
{\tt showme ~ ssn75}
\end{quote}

On non-Starlink systems access URL:

\begin{quote}
\htmladdnormallink{
{\tt http://www.starlink.ac.uk/docs/ssn75.htx/ssn75.html}}
{http://www.starlink.ac.uk/docs/ssn75.htx/ssn75.html}
\end{quote}

Paper copies can be obtained from the Starlink document librarian,
who can be contacted as follows.

Postal address: \\
\begin{tabular}{l}
The Document Librarian. Starlink Project, Rutherford Appleton Laboratory,
  Chilton, \\
DIDCOT, Oxfordshire, OX11 0QX, United Kingdom.                \\
\end{tabular}

% \vspace{3mm}

Electronic mail: {\tt starlink@jiscmail.ac.uk}

% \vspace{3mm}

Fax: \\
\begin{tabular}{lr}
from within the United Kingdom: &    01235-445-848 \\
from overseas:                  & +44-1235-445-848 \\
\end{tabular}


\section*{Obtaining assistance}

If you run into difficulties writing a server then I might be able to
offer advice or assistance.  I can be contacted as follows.

Postal address: \\
\begin{tabular}{l}
A.C.~Davenhall.  Institute for Astronomy, Royal Observatory, Blackford Hill, \\
Edinburgh, EH9 3HJ, United Kingdom.  \\
\end{tabular}

% \vspace{4mm}

Electronic mail: {\tt acd@roe.ac.uk}

% \vspace{4mm}

Fax: \\
\begin{tabular}{lr}
from within the United Kingdom: &    0131-668-8416 \\
from overseas:                  & +44-131-668-8416 \\
\end{tabular}

\section*{Revision history}

\begin{enumerate}

  \item 26th July 2000: Version 1. Original version (ACD).

\end{enumerate}

% ? Latex Copyright Statement

\vspace*{\fill}
\stardoccopyright

\cleardoublepage
\begin{latexonly}
  \setlength{\parskip}{0mm}
  \tableofcontents

%   \newpage
  \listoffigures
  \listoftables

  \setlength{\parskip}{\medskipamount}
  \markboth{\stardocname}{\stardocname}
\end{latexonly}
% ? End of Latex document table of contents
% -----------------------------------------------------------------------------
\cleardoublepage
\renewcommand{\thepage}{\arabic{page}}
\setcounter{page}{1}

\section{\xlabel{INTRO}\label{INTRO}Introduction}

\xref{GAIA}{sun214}{}\cite{SUN214} and \xref{CURSA}{sun190}{}\cite{SUN190}
can interrogate remote catalogues via the Internet to retrieve lists of
objects which satisfy a given criterion.  GAIA can also extract images
of a specified region of sky from remote databases.  Similar facilities
are also available in other packages, such as the ESO
\htmladdnormallinkfoot{{\it SkyCat}\/}{http://archive.eso.org/skycat/}
image display tool and the Gemini observing tool.  This functionality
is achieved by having servers on the remote systems which accept queries
sent by GAIA \emph{etc}, interrogate their local copies of the catalogues
to find the data which satisfy the query and return them to the client
which sent the query.  This method of working is an example of a
`client-server architecture', with GAIA, CURSA \emph{etc.} acting as the
client.  The server and client can communicate because the query is sent,
and the results returned, in a standard format.  This document describes
how to write such servers and also the formats of the queries and results.

The standard formats for the queries and returned data were developed by
Allan~Brighton and colleagues at ESO for use with their Astronomical
Catalogue Library (ACL).  This subroutine library provides the
functionality for a client to access a remote server, and is used by,
for example, {\it SkyCat}\/ and GAIA.  Consequently, in this document
the formats will be called the `ACL format' and a server which accepts
queries and returns data in these formats will be called an `ACL server'.

The communication between the client and the server uses the Hyper-Text
Transfer Protocol (HTTP)  developed as part of the World Wide Web.  The
servers are, strictly speaking, {\it gateways}\/ using the Common Gateway
Interface (CGI).  One way of thinking of the client is as a very
specialised Web browser.  A consequence of this approach is that if a
site is to host an ACL server it must also be running a Web server.  The
HTTP and CGI protocols are, of course, enormously flexible and the
ACL format is a set of additional rules and restrictions which sit `on
top' of them.

An ACL server is a CGI gateway, and CGI gateways can be provided in any
number of different ways.  However, the usual technique is to implement
the gateway as a Perl script and this approach will be adopted in this
document.  Usually (though not always) the Perl script will invoke a
special-purpose program or Database Management System (DBMS) to
interrogate the catalogue database.

This document is divided into two parts:

\begin{description}

  \item[{\rm Part I}] -- a tutorial example of creating a simple ACL
   server,

  \item[{\rm Part II}] -- reference material, mostly describing the
   ACL format.

\end{description}


\section{\xlabel{FURTHER}\label{FURTHER}Further Reading}

The ACL format is documented in Section~2 of Allan~Brighton's {\it
Astronomical Catalogue Library User Manual}\/\cite{BRIGHTON98} (and the
description in Part~II is largely based on this manual).  It is a
subset of a proposed general format for exchanging information between
remote astronomical information services which is being developed at the
Centre de Donn\'{e}es astronomiques de Strasbourg (CDS) and elsewhere.
The full proposal is described in the working document {\it Astronomical
Server URL}\, by M.~Albrecht {\it et al.}\cite{SERVERURL}.

There are numerous books describing the HTTP and CGI protocols; I
have found the {\it The HTML Source Book}\/ by Ian~Graham\cite{GRAHAM95}
to be useful.  Similarly, there are many books on Perl.  I have used
{\it Learning Perl}\/ by  Randall~Schwartz\cite{SCHWARTZ93} and {\it
Programming Perl}\/ by Larry~Wall and Randall~Schwartz\cite{WALL91} and
found them comprehensive, convenient and accessible.

% - Part I ------------------------------------------------------------
\cleardoublepage
\markboth{\stardocname}{\stardocname}
\part{Tutorial Example: Creating a Simple Server}
\markboth{\stardocname}{\stardocname}
\section{\xlabel{INTRO_T}\label{INTRO_T}Introduction}

This part of the document is a tutorial example which describes how to
create a simple ACL server.  Example files are provided which illustrate
the procedure.  Firstly, however, the basics of using an ACL server to
query a catalogue are reviewed.


\section{\xlabel{BASICS_T}\label{BASICS_T}Basics of Querying Remote
Catalogues}

The basics of querying a catalogue are that some criterion is specified,
all the rows in the catalogue are examined and those which satisfy the
criterion are returned as the list of selected rows.  Some criteria
might be:

\begin{itemize}

  \item stars whose Right Ascension lies in the range \hm{13}{30}{0} to
   \hm{14}{00}{0},

  \item stars brighter than $m_{v} = 13.0$,

  \item stars with a B-V colour greater than 0.2.

\end{itemize}

However, a very common sort of search on astronomical catalogues is the
so-called `circular area search' or `cone search'.  Virtually all
astronomical catalogues contain celestial coordinates, in practice
Right Ascension and Declination for some equinox and epoch.  In a circular
area search the central coordinates and angular radius are specified.
All the objects in the catalogue which are less than this angular radius
from the central coordinates are selected.  That is, the circular area
search finds all the objects in the catalogue within a given circular
patch of sky.

It is usual for ACL servers to a support a circular area search and
often this may be the only type of search provided.  The full form of
an ACL circular area search is slightly more general, with both an inner
and outer radius specified, so that objects inside an annulus rather than
a circle are selected.  (The `traditional' circular area search corresponds
to setting the inner radius to zero.)

The catalogue being accessed will doubtless have other columns as well
as the Right Ascension and Declination, and the ACL server may permit
`range' searches on some of these columns.  In a range search minimum and
maximum values are specified for a column and a row is selected if its
value for the column falls within the given range.  Any range searches
specified are combined with each other and with the circular area
search using a `logical and'.  That is, for example, the objects selected
would correspond to those which are both in a given area of sky and in
a given magnitude range.  Though this mechanism allows powerful queries
to be made it still provides only a subset of all the conceivable types of
queries.

Before starting work on an ACL server for a catalogue, you need to decide
two things.

\begin{enumerate}

  \item Are circular area searches to be supported (the answer is almost
   certainly yes)?

  \item On which, if any, additional columns are range searches to be
   supported?

\end{enumerate}

In order to implement the server you need to provide two things:

\begin{enumerate}

  \item the server itself,

  \item an entry for the server in an ACL `configuration file'.

\end{enumerate}

An ACL configuration file defines the list of catalogues which a client,
such as \xref{GAIA}{sun214}{} or {\it SkyCat}, currently knows about.  It
has an entry for each catalogue.  The entry specifies details of the
catalogue which the client needs to know: the URL of its server, the
types of queries that it supports, the name by which it is to be described
to the user \emph{etc}.


\section{\xlabel{OBTAIN_T}\label{OBTAIN_T}Obtaining Example Files}

Some simple examples of server scripts and a configuration file are
provided with this document.  Subsequent sections of the tutorial
will describe these files and explain how to install them.  You may
also find them useful as templates for developing your own servers
and configuration files.  The files can be obtained in two different ways.

\begin{itemize}

  \item On Starlink systems they should be present in directory:

  \begin{quote}
   {\tt /star/examples/ssn75}
  \end{quote}

  \item Copies can be obtained by anonymous ftp from Edinburgh.  The
   details are:

  \begin{tabular}{ll}
   Anonymous ftp to: & {\tt ftp.roe.ac.uk}    \\
   Directory:        & {\tt /pub/acd/misc}    \\
   File:             & {\tt ssn75\_examples.tar} \\
  \end{tabular}

   The file is an uncompressed tar archive (it is only 52 Kb in size).
   Set ftp to `binary' mode to retrieve it.

\end{itemize}

The important files used in the tutorial are:

\begin{description}

  \item[{\tt simpleserver.cgi}] a simple server script,

  \item[{\tt secondserver.cgi}] a second, slightly more complicated, server
   script,

  \item[{\tt genfield.c}] a program to generate the list of stars returned
   by the servers,

  \item[{\tt simpleconfig.cfg}] a configuration file including the two
   example servers.

  \item[{\tt checkcfg}] a script for checking configuration files.

\end{description}

File {\tt 0README.LIS} gives a complete list.  You will probably find it
useful to print out copies of the files and have them to hand as you work
through the examples.  Copies of the servers are installed at Edinburgh, so
the URLs given in the examples should work, though you will be accessing
the Edinburgh versions rather than your own copies.


\section{\xlabel{SERVER_T}\label{SERVER_T}Creating a Server}

An ACL server is a Perl script which accepts a query in a standard format,
searches the corresponding catalogue to find the objects which satisfy
the query and returns them to the remote client.  The server {\tt
simpleserver.cgi} supplied with this document is more-or-less the simplest
practical server, but it illustrates the structure that they usually have.
You will find it helpful to have a copy to hand as you read through this
section.  {\tt simpleserver.cgi} supports only circular area searches
in which a central Right Ascension and Declination and an angular radius
are specified.  Because it is provided as an example it generates and
returns an artificial list of stars, centred on, and scaled to fit, the
specified area, rather than searching a real catalogue.  The overall
structure of {\tt simpleserver.cgi} is:

\begin{quote}
obtain the query string \\
parse the query string to obtain the central coordinates and radius \\
generate the list of stars to fit in this field \\
\hspace*{3 mm} (a real server would search a catalogue instead) \\
return the results to the client
\end{quote}

The comments in the source code should make the detailed working obvious.
However, the following notes about the query string and results returned
might be useful.

\subsection{Query string}

\begin{enumerate}

  \item If a CGI gateway is implemented as a Perl script the query
   string appears in the variable \verb-$ENV{'QUERY_STRING'}-.
   In {\tt simpleserver.cgi} the query is copied into a local variable
   by the line:

  \begin{quote}
   \verb-$query = $ENV{'QUERY_STRING'};-
  \end{quote}

  \item In {\tt simpleserver.cgi} the query string has the format:

  \begin{quote}
   {\tt ra={\it xxx}\&dec={\it yyy}\&radius={\it zzz}}
  \end{quote}

   For example:

  \begin{quote}
   {\tt ra=10:30:00\&dec=-30:30.0\&radius=3}
  \end{quote}

   where {\tt ra=}, {\tt dec=} and {\tt radius=} have their obvious
   meanings.  When you set up the entry for the server in the configuration
   file (see Section~\ref{CONFIG_T}, below) you have considerable latitude
   over this format.  However, the one used by {\tt simpleserver.cgi} is a
   common one, and is as good as any.  {\tt simpleserver.cgi} supports
   only circular area searches; if additional types of search were
   supported then the query string would contain additional parameters.

  \item The formats and units required for the parameters are as follows:

  \begin{description}

    \item[Right Ascension] sexagesimal or decimal hours,

    \item[Declination] sexagesimal or decimal degrees,

    \item[Radius] decimal minutes of arc.

  \end{description}

   If a sexagesimal value is entered then a colon (`:') should be used
   as a separator.  The Right Ascension and Declination should be for
   equinox and epoch J2000.  The server {\it must}\/ accept values in
   these units and formats because typically the client will present
   the same recipe and information to the user when requesting input
   for any of the various catalogues available to it.

\end{enumerate}

\subsection{Results returned}

\begin{enumerate}

  \item The first line of information returned by the server should
   always be the MIME type.  In {\tt simpleserver.cgi} it is written by
   the line:

  \begin{quote}
   \verb+{  print "Content-type: text/plain\n\n\n";+
  \end{quote}

   The MIME type is not part of the table of results, but rather is
   used by the client or Web browser to interpret the format of the data
   which follows.

  \item The list of selected objects are returned to the client as a
   stream of ASCII characters.  The list is written in the Tab-Separated
   Table (TST) format (see Section~\ref{TST_R}).

  \item The Perl script should simply write the results to standard
   output (whence it will be automatically forwarded to the remote
   client).  In {\tt simpleserver.cgi} the lines:

  \begin{quote}
   \verb-$tst = `$queryExe $ra $dec $radius`;- \\
   \verb-print "$tst";-
  \end{quote}

   invoke program {\tt genfield} (variable \verb-$queryExe- has
   previously been set to contain the name and directory specification
   of the executable for {\tt genfield}) to generate the star list,
   copy the list the to variable \verb-$tst- and then write the contents
   of \verb-$tst- to standard output.  (Hint: Perl has several mechanisms
   for invoking processes and directing their output to standard output;
   I found the one described to be the most suitable for use in a CGI
   script.)

  \item The last line written to standard output should be:

  \begin{quote}
   \verb-print "[EOD]\n";-
  \end{quote}

   The string `{\tt [EOD]}' informs the client that the server has
   finished sending data.

\end{enumerate}

\subsection{Installing and testing the server}

The procedures for installing CGI scripts vary at different sites; your
system manager should be able to advise on arrangements at your site.
However, to install {\tt simpleserver.cgi} you need to follow at least
the following steps.

\begin{enumerate}

  \item Copy files {\tt simpleserver.cgi} and {\tt genfield.c} to a
   suitable directory (there are likely to be restrictions on which
   directories can contain CGI scripts; see your system manager).

  \item Compile program {\tt genfield.c} and name the executable {\tt
   genfield}.  {\tt genfield.c} is a standard (and simple) C program;
   any C compiler should be able to handle it.

  \item Edit file {\tt simpleserver.cgi}.  Locate the line:

  \begin{quote}
   \verb-$queryExe = "/star/examples/ssn75/genfield";-
  \end{quote}

   which is towards the top of the script and change it to correspond
   to whichever directory you have put the files in.

  \item Check with your system manager whether there are any other
   local requirements for running CGI scripts.

  \item You are now ready to test the server.  The tests are best conducted
   from a normal Web browser, such as {\tt netscape}.  Start the browser
   and enter a URL similar to:

  \begin{verbatim}
http://www.roe.ac.uk/~acd/cgi-bin/simpleserver.cgi?ra=10.0&dec=30.0&radius=3
  \end{verbatim}

   This string comprises the normal URL for the CGI script, followed
   by a `{\tt ?}', followed by the query.  This format is the normal
   syntax for invoking CGI scripts and passing queries to them.  To
   invoke your version of the server you would substitute the appropriate
   URL in the example above.  Typing in the example exactly as given
   should invoke a version of the server running in Edinburgh.  If
   all is working correctly the browser should display a table similar
   to the one in Figure~\ref{SIMPLE_TST}.  If the server fails then
   your Web server probably maintains error logs which might contain
   useful diagnostics; your system manager should know where these logs
   are kept.

\end{enumerate}

\begin{figure}[htbp]

\begin{quote}
\begin{verbatim}

Example Star Field.

#column-units:          DEGREES         DEGREES         Magnitudes
#column-types:   CHAR*8         DOUBLE  DOUBLE  REAL
#column-formats: A8     F12.6   F12.6   F6.2

Id      RA      DEC     mag
--      --      ---     ---
Star 1  150.026667      30.050000       0.580000
Star 2  149.968333      29.961667       0.120000
Star 3  149.983333      30.041667       1.640000
Star 4  149.993333      30.005000       2.210000
Star 5  150.000000      30.000000       1.700000
Star 6  150.006667      29.996667       1.790000
Star 7  149.983333      29.993333       3.380000
Star 8  150.015000      29.951667       4.130000
[EOD]
\end{verbatim}

\caption{The table of values returned by the ACL server {\tt
simpleserver.cgi}.
\label{SIMPLE_TST} }
\end{quote}

\end{figure}


\subsection{Modifying the server}

Other servers are likely to be similar to {\tt simpleserver.cgi}.  To
modify it to access your own catalogue you would replace the invocation
of program {\tt getfield} with an invocation of a program which searched
your catalogue or DBMS.

{\tt simpleserver.cgi} is more-or-less the simplest practical server,
and is provided as an example.  Additional useful features in a server
include: copying the queries to a log file (so that you can monitor
usage) and copying error messages to a second log file (as diagnostics
in case of misadventure).  Another server, {\tt secondserver.cgi}, which
incorporates these features, is supplied with this document.  To install
it, follow the same procedure as for {\tt simpleserver.cgi}, except that
the lines towards the top of the script which need to be modified are:

  \begin{quote}
    \verb+$queryExe = "/star/examples/ssn75/genfield";+ \\
    \verb+$logDir = "/star/examples/ssn75/examplelogs";+
  \end{quote}

When developing a server it is often useful to comment out the line:

\begin{quote}
\begin{verbatim}
     $query = $ENV{'QUERY_STRING'};
\end{verbatim}
\end{quote}

and un-comment the line:

\begin{quote}
\begin{verbatim}
#    $query = "ra=10:30:00&dec=-30:30.0&radius=3";
\end{verbatim}
\end{quote}

so that the script has a query `hard-wired'.  It can now be run directly
from the command line rather than be invoked via a Web browser.  This
trick often makes debugging scripts a {\it lot}\/ easier.

Both the query and error log files written by {\tt secondserver.cgi}
are simple text files.  Note, however, that file {\tt query.TXT}, which
is supplied with the examples, allows the query log to be accessed by
\xref{CURSA}{sun190}{}\cite{SUN190} as an STL (Small Text List) format
catalogue.  Once the query log becomes large you might find it more
convenient to examine it with the CURSA catalogue browser {\tt xcatview}
rather than Unix commands such as {\tt cat} and {\tt more}.


\section{\xlabel{CONFIG_T}\label{CONFIG_T}Creating a Configuration File}

The configuration file used by \xref{GAIA}{sun214}{} \emph{etc}. mediates
interaction between the client and server.  It is an ASCII text file
containing details for each of a list of catalogues.  GAIA (or whatever)
reads the file and the catalogues listed in it become the ones that GAIA
knows about.  The details supplied for each catalogue are things like:
its URL, the type of queries supported, the name that will be used to
describe it to users \emph{etc}.  The entry for a typical simple catalogue
looks something like:

\begin{verbatim}
serv_type:      catalog
long_name:      Simple example server.
short_name:     simple@roe
url:            http://www.roe.ac.uk/~acd/cgi-bin/simpleserver.cgi?ra=...
symbol:         mag circle 3
\end{verbatim}

This entry is taken from {\tt simpleconfig.cfg}, the example configuration
file supplied with this document, though the {\tt url} entry has been
truncated.  By convention configuration files have file type `{\tt .cfg}'.
The purposes of the various items are as follows.

\begin{description}

  \item[{\tt serv\_type:}] is the type of the server.  For a
   straightforward catalogue the value required is `{\tt catalog}'.
   Other values are possible, though you will probably rarely use them.

  \item[{\tt long\_name:}] a one-line name or short description of the
   catalogue.  It will be presented to the user to allow him to identify
   the catalogue.

  \item[{\tt short\_name:}] a short name for the catalogue.  Conventionally
   it has the form:

  \begin{quote}
   {\it catalogue}{\tt @}{\it institution}
  \end{quote}

   where {\it catalogue}\, is an abbreviation for the catalogue and {\it
   institution}\, a standardised abbreviation for the institution where
   the on-line version is located.  By convention {\it institution}\/ has
   three or four characters.

  \item[{\tt url:}] the URL used to access the server.  Following the
   usual conventions for a CGI gateway it consists of the URL corresponding
   to the script which constitutes the server, followed by a `{\tt ?}'
   and then a string defining the query passed to the server (see
   Section~\ref{QUERY_T}).

  \item[{\tt symbol:}] specifies how objects are to be plotted (see
   Section~\ref{PLOTTING_R}).

\end{description}

There are various other optional items which can be included.  They are
described in Section~\ref{CONFIG_R}.

\subsection{\label{QUERY_T}URL query}

The string appended to the server URL in the configuration file and which
defines the type of queries supported by the catalogue has a format
something like:

\begin{quote}
\verb-ra=%ra&dec=%dec&radius=%r2-
\end{quote}

It consists of simple characters and `tokens'.  The tokens start with
a `{\tt \%}' character.  When GAIA makes a query the tokens are replaced
with values which correspond to the individual query and the resulting
string is sent to the server.  For example, tokens in the above string
might be susbstituted to yield:

\begin{quote}
\begin{verbatim}
ra=10:30:00&dec=-30:30.0&radius=3
\end{verbatim}
\end{quote}

Obviously the format of the query string appended to the URL in the
configuration file must correspond to that expected by the server.
Various standard tokens can be included in the query string.  Some common
ones are:

\begin{description}

  \item[{\tt \%ra}] Right Ascension,

  \item[{\tt \%dec}] Declination,

  \item[{\tt \%r1}] inner radius,

  \item[{\tt \%r2}] outer radius,

  \item[{\tt \%n}] maximum number of objects to return.

\end{description}

For a complete list see Section~\ref{CONFIG_R}.

\subsection{Installing and testing the configuration file}

Installing and testing the configuration file should be quite
straightforward.  The servers \newline
{\tt simpleserver.cgi} and {\tt secondserver.cgi} should be installed
(see Section~\ref{SERVER_T}, above).  Then proceed as follows.

\begin{enumerate}

  \item Edit the configuration file and change the URLs for the servers
   {\tt simple@roe} and {\tt second@roe} to correspond to wherever
   you have installed the servers.

  \item A Perl script to check a configuration file for errors is
   included with the example files for this document.  To check that
   you have not introduced any errors whilst editing the example
   configuration file type:

  \begin{quote}
   {\tt /star/examples/ssn75/checkcfg ~ simpleconfig.cfg}
  \end{quote}

   (If the example files are not in their standard location on a Starlink
   system then obviously you need to alter the directory specification
   accordingly.  Also, on non-Starlink systems you might need to edit the
   first line of {\tt checkcfg} to correspond to wherever Perl is installed
   on your system.)  If the configuration file is valid then {\tt checkcfg}
   will report:

  \begin{quote}
   {\tt Configuration file parsed successfully.}
  \end{quote}

   Conversely, if it contains errors then messages describing the problems
   will be reported.  {\tt checkcfg} is described in
   Section~\ref{CHECKCFG_R}.

  \item Start GAIA and test the server.  To import the configuration
   file into GAIA click on the {\sf Data-Servers} button on the right
   hand side of the main menu bar and choose the {\sf Browse Catalog
   Directories} option.  A window showing the catalogues available should
   appear.  Click on the {\sf File} button at the left of its top menu
   bar and choose the {\sf Load Config file\ldots} option.  A window
   allowing you to select the required file should then appear.

   Once the configuration file has been loaded you can choose from
   amongst its catalogues and make selections in the normal fashion.
   If you make a selection from the simple example server a list of
   objects similar to Figure~\ref{SIMPLE_TST} should be returned.

\end{enumerate}

\subsection{Modifying the configuration file}

When you create a new server you need to create an entry for it in
your configuration file.  If the server is just a simple variation of
{\tt simpleserver.cgi} or {\tt secondserver.cgi} and only supports
circular area queries then just duplicate the entry for {\tt simple@roe}
and change {\tt long\_name}, {\tt short\_name} and {\tt url} to correspond
to your server.

Additional modifications can be made as required.  The following section
gives some examples and the options are documented in
Section~\ref{CONFIG_R}.  Remember that script {\tt checkcfg} (see
Section~\ref{CHECKCFG_R}) is available for checking configuration files.


\section{\xlabel{CARRYON_T}\label{CARRYON_T}Carrying On}

This section introduces a few additional features that are often
required in servers and configuration files.

\subsection{Multiple catalogue servers}

{\tt simpleserver.cgi} can access only a single catalogue.  Often you
may want to write a server which can access each of several catalogues.
In this case your configuration file must contain an entry for each
{\it catalogue}, not each server.  An additional parameter, whose value
identifies the catalogue, is added to the query.  The server parses this
parameter to identify the catalogue required.  You can decide the
syntax and values of this parameter, though the configuration file and
server must agree.

For example, suppose that you are writing a server which will provide
access to the SAO, PPM and Hipparcos astrometric catalogues.  You might
invent a parameter called `{\tt catalogue}' whose value identifies the
catalogue.  Your configuration file would then have three entries like:

{\small
\begin{verbatim}
serv_type:  catalog
long_name:  SAO (Smithsonian Astrophysical Observatory) catalog
short_name: sao@roe
url:        http://www.roe.ac.uk/~acd/cgi-bin/ast.cgi?catalogue=sao&ra=%ra&dec=...
symbol:     mag square 3

serv_type:  catalog
long_name:  PPM (Positions and Proper Motions) catalogue
short_name: ppm@roe
url:        http://www.roe.ac.uk/~acd/cgi-bin/ast.cgi?catalogue=ppm&ra=%ra&dec=...
symbol:     mag square 3

serv_type:  catalog
long_name:  Hipparcos catalogue
short_name: sao@roe
url:        http://www.roe.ac.uk/~acd/cgi-bin/ast.cgi?catalogue=hipparcos&ra=%ra&dec=...
symbol:     mag square 3
\end{verbatim}
}

The server would be written to parse the value of {\tt catalogue} and then
search the catalogue indicated.

\subsection{Providing range queries}

As well as selecting objects in a circular area of sky it is possible
to further restrict the objects selected to only those for which the
values of some column lie within a given range.  Suppose selections
from an optical catalogue were to be optionally restricted to also
lie within a specified range of magnitudes and the column of magnitudes
was named {\tt mag}.  To provide this facility for a catalogue its entry in
the configuration file should be include the keyword {\tt search\_cols}
and the query part of its {\tt url} keyword should include the token
{\tt \%cond}.

\begin{enumerate}

  \item The syntax of the {\tt search\_cols} keyword is:

  \begin{quote}
   {\tt search\_cols:} {\it column-name minimum-prompt maximum-prompt}
  \end{quote}

   For example:

  \begin{quote}
   {\tt search\_cols: mag "Bright limit" "Faint limit"}
  \end{quote}

   {\it column-name}\/ is the name of the column.  The client uses
   {\it minimum-prompt}\/ and {\it maximum-prompt}\/ as prompts when
   soliciting the extrema of the required range from the user.  (Note
   that because the example column is a magnitude the ``Bright limit"
   corresponds to the smallest numerical value and the ``Faint limit"
   to the largest.)

  \item The {\tt \%cond} token is added to the query part of the {\tt
   url} keyword to supply details of the range query.  Such a query
   string might look like:

  \begin{quote}
  \verb-ra=%ra&dec=%dec&radius=%r2&%cond-
  \end{quote}

   The syntax of the values substituted into the {\tt \%cond} string is:

  \begin{quote}
   {\it column-name}\/ {\tt =} {\it minimum-value}{\tt ,}{\it maximum-value}
  \end{quote}

   For example, if a range of first to second magnitude had been specified
   for column {\tt mag} then {\tt \%cond} would translate to:

  \begin{quote}
   {\tt mag=1.0,2.0}
  \end{quote}

   The server that parses the query must interpret this string and
   ensure that the range selection specified is applied.

\end{enumerate}

See Section~\ref{RANGE_R} for further details.

\subsection{Linked configuration files}

In addition to entries for individual catalogues, configuration files
can also contain entries for other configuration files.  Typically
when such an entry is chosen all the entries in the target configuration
file are loaded into the client.  In this way a tree (or rather a network,
because recursion is allowed) of entries can be built up.  Entries of
this type are referred to as `directories' (by analogy with an hierachical
file system).

For a directory entry the {\tt serv\_type} should be `{\tt directory}'
and the {\tt url} should be the URL of the destination configuration
file.  {\tt long\_name} and {\tt short\_name} have their usual meaning.
Other options are unlikely to be required.  An example might be:

\begin{verbatim}
serv_type:      directory
long_name:      ESO Catalogues
short_name:     catalogs@eso
url:            http://archive.eso.org/skycat/skycat2.0.cfg
\end{verbatim}

\subsection{Handling queries which return no results}

Sometimes users will submit a query which no objects in the catalogue
satisfy.  For example, the query might correspond to an empty patch of
sky.  In practice such queries are quite common.  Unfortunately, the ACL
format does not prescribe the action the server should take in this case.
However, the {\it recommended}\/ action is for the server to return an
empty TST table.  That is, it should return all the header information
for a TST generated from the catalogue being queried (see
Section~\ref{TST_R}), down to and including the list of column names
and the line of dashes anf tab characters which terminate the header,
but no table of values.


% - Part II -----------------------------------------------------------
\cleardoublepage
\markboth{\stardocname}{\stardocname}
\part{Reference Material}
\section{\xlabel{INTRO_R}\label{INTRO_R}Introduction}

This part of the document comprises reference material describing
the ACL configuration file and the Tab-Separated Table (TST) format.
Most of the material is adapted from Section~2 of Allan~Brighton's {\it
Astronomical Catalogue Library User Manual}\/\cite{BRIGHTON98}.


\section{\xlabel{CONFIG_R}\label{CONFIG_R}The Configuration File}

This section describes the format of the ACL configuration file.  When
you refer to it you might find it useful to have to hand a copy of
either {\tt simpleconfig.cfg} or some other configuration file.

An ACL configuration file mediates the interaction between
a client such as \xref{GAIA}{sun214}{} and a remote server.
The configuration file comprises a list of one or more databases,
giving details for each.  Usually each `database' will be a simple
astronomical catalogue.  However, other alternatives are possible:
archives, name servers, \emph{etc}.  Consequently, in this section the
generic term `database' is used to denote each entry.  Also, it is
individual databases, not servers, which are listed in the configuration
file: some servers might give access to more than one database.
The details supplied for each database are things like: its URL, the
type of queries supported, the name that will be used to describe it to
users \emph{etc}.  The client reads the configuration file and the
databases listed become the ones that it knows about.

By convention, configuration files have file-type `{\tt .cfg}'.
They are ASCII text files which may be created and modified with a text
editor.  Their basic syntax is as follows.

\begin{itemize}

  \item Blank lines are ignored.

  \item Lines beginning with `{\tt \#}' are considered to be comments
   and are ignored.

  \item Lines ending with `${\tt \backslash}$' (backslash) are continued
   on the next line.

  \item The description for each database comprises several `keyword:value'
   pairs.  Each keyword:value pair occupies a single line, unless continued
   over more than one line.  The individual keywords are described in
   Section~\ref{KEYWORDS_R}, below.  Most are optional.

  \item The first keyword of each database entry must be {\tt serv\_type}.

  \item Subsequent keywords of each database entry can occur in any
   order.

  \item Unrecognised keywords are ignored (to permit future extensions).

  \item Many of the optional keywords correspond to facilities that are not
   supported by most databases.

  \item Some of the keywords require a list of values.  All such lists
   have the same basic format as a Tcl\footnote{The Tcl scripting language
   is described by its author, John~Ousterhout, in his {\it Tcl and the Tk
   Toolkit}\/\cite{OUSTERHOUT94}.} list: usually a list of words
   or strings enclosed in double-quotes (`{\tt "}') and separated by one
   or more space characters.  Alternatively, curly brackets (`{\tt \{ \}}')
   may be used instead of double-quotes.

  \item If a keyword's value comprises more than one list then these
   lists are separated by a colon (`{\tt :}').

  \item If a keyword's value includes a variable reference (see
   Section~\ref{PLOTTING_R}, below) then the variable is expressed in
   Tcl format: in practice it will start with a dollar character (`{\tt
   \$}').

\end{itemize}

\subsection{\label{KEYWORDS_R}Keywords}

The entry for an individual database in a configuration file comprises
some of the following keywords.  The entry for the database must begin
with the {\tt serv\_type} keyword.  Other keywords can occur in any
order.  The keywords are optional unless otherwise indicated.  The
keywords are case-sensitive and must be specified entirely in lower case.

\begin{description}

  \item[{\tt serv\_type}] (mandatory) The type of database.  The
   options permitted are described in Section~\ref{SERVERS_R}, below.

  \item[{\tt long\_name}] (mandatory) A one-line description of the
   database.  Typically the client will display it to allow the user to
   identify the database.

  \item[{\tt short\_name}] (mandatory) A short name for the database.
   Conventionally it has the form:

  \begin{quote}
   {\it database}{\tt @}{\it institution}
  \end{quote}

   where {\it database}\, is an abbreviation for the database and {\it
   institution}\, a standardised abbreviation for the institution where the
   on-line version is located.  By convention {\it institution}\/ has three
   or four characters; some common values are listed in Table~\ref{REMINST}.

\begin{table}[htbp]

\begin{center}
\begin{tabular}{cl}
Abbreviation  &  Institution  \\ \hline
{\tt cadc} & Canadian Astronomy Data Centre, Dominion Astrophysical
    Observatory  \\
{\tt eso} &  European Southern Observatory, Garching bei M\"{u}nchen \\
{\tt lei} &  Department of Physics and Astronomy, University of Leicester \\
{\tt roe} &  Royal Observatory Edinburgh \\
\end{tabular}
\end{center}

\caption{Abbreviations for institutions hosting ACL servers
\label{REMINST} }

\end{table}

  \item[{\tt url}] (mandatory) The URL and query template to access the
   server.  See Section~\ref{QUERY_R}.

  \item[{\tt symbol}] Defines how objects in the returned table should
   be plotted.  See Section~\ref{PLOTTING_R}.

  \item[{\tt copyright}] A copyright notice for the client to display.

  \item[{\tt search\_cols}] A list of columns on which range searches
   are permitted.  See Section~\ref{RANGE_R}.

  \item[{\tt sort\_cols}] The columns on which the returned table is
   sorted.

  \item[{\tt sort\_order}] The order into which the returned table is
   sorted.  The permitted values are: {\tt increasing} and {\tt
   descreasing}.

  \item[{\tt show\_cols}] By default a client will display all the
   columns in the returned table.  If a {\tt show\_cols} list is specified
   then by default only the columns in the list will be displayed.  Also
   the order of the list defines the order in which the columns should be
   displayed.

  \item[{\tt id\_col}] The number of a column in the returned table
   which contains a unique identifier (or `name') for each object in
   the table.  By default the first column of a TST format table contains
   such an identifer (see Section~\ref{TST_R}).

  \item[{\tt ra\_col}] The number of a column in the returned table
   which contains the Right Ascension.  By default the second column of a
   TST format table contains the Right Ascension (see Section~\ref{TST_R}).

  \item[{\tt dec\_col}] The number of a column in the returned table
   which contains the Declination.  By default the third column of a
   TST format table contains Declination (see Section~\ref{TST_R}).

  \item[{\tt x\_col}] The number of a column in the returned table
   which contains $x$\/ image pixel coordinates.

  \item[{\tt y\_col}] The number of a column in the returned table
   which contains $y$\/ image pixel coordinates.

  \item[{\tt is\_tcs}] If this keyword is present it should be set to
   `1'.  Its presence indicates that the returned table either is in or
   should be converted to TCS format.  TCS format catalogues have fixed
   column names and rules for converting column names when catalogues
   are imported from other formats.

  \item[{\tt help}] A URL pointing to a Web page (or pages) describing
   the database.

  \item[{\tt backup1}] A reserve URL to be used to access the database
   in the case where the URL specified by the {\tt url} keyword does
   not respond.

  \item[{\tt backup2}] A second reserve URL to be used to access the
   database in the case where the URLs specified by the {\tt url} and
   {\tt backup1} keywords do not respond.

\end{description}

For keywords {\tt id\_col}, {\tt ra\_col}, {\tt dec\_col}, {\tt x\_col}
and {\tt y\_col} the number of a column is defined as its sequence
number in the list of column names which defines the columns in a TST
table (see Section~\ref{TST_R} and Figure~\ref{TSTEXAMP}).  The first
column is numbered zero.  This sequence number is sometimes referred
to as the `column index'.

\subsection{\label{SERVERS_R}Types of servers}

The types of database which may be specified for keyword {\tt serv\_type}
are as follows.  If an unrecognised type is specified it will be ignored
(to permit future extensions).

\begin{description}

  \item[{\tt catalog}] The database is a simple catalogue; this is the
   simplest and most common option.

  \item[{\tt archive}] The database is an archive.  This option is
   similar to `{\tt catalog}', but the table returned may contain
   special columns whose values are URLs giving access to `bulk data'
   (images, spectra, time-series \emph{etc}\/) for the selected objects.
   See Section~\ref{TSTSPCOL_R} for details of these special columns.

  \item[{\tt namesvr}] The database is a name server.  The only type
   of query permitted is to submit to the server a character string
   corresponding to the name of an astronomical object.  The server
   returns a TST catalogue with a single row and three columns.
   The row corresponds to the object named and the three columns are:
   identifier (name), Right Ascension and Declination.  See
   configuration file {\tt simpleconfig.cfg} for an example of a
   {\tt namesvr} entry.

  \item[{\tt imagesvr}] The database is an image server.  See
   Section~\ref{IMAGE_R}.

  \item[{\tt local}] The database is a local catalogue in the TST format.

  \item[{\tt directory}] The `database' is another configuration file.
   This facility allows a tree (or rather network, since recursion is
   permitted) of linked configuration files to be built up.  The {\tt url}
   keyword gives the URL of the destination configuration file.  The term
   `directory' comes from making an analogy with an hierachical file
   system.  See configuration file {\tt simpleconfig.cfg} for an example.

\end{description}

\subsection{\label{QUERY_R}URL and query specification}

The {\tt url} keyword prescribes how the client should access a remote
server to query a database.  Following the usual conventions for a CGI
gateway it consists of the URL corresponding to the script which
constitutes the server, followed by a `{\tt ?}' and then a string
defining the query passed to the server.  The query string consists of
simple characters and `tokens'.  The tokens start with a `{\tt \%}'
character.  When a client makes a query the tokens are replaced
with values which correspond to the individual query and the resulting
string is sent to the server.  An example query string for a server
which provides circular area searches on a single catalogue might be:

\begin{quote}
\verb-ra=%ra&dec=%dec&radius=%r2-
\end{quote}

which contains the tokens {\tt \%ra}, {\tt \%dec} and {\tt \%r2}.  The
client might substitute them to yield:

\begin{quote}
\begin{verbatim}
ra=10:30:00&dec=-30:30.0&radius=3
\end{verbatim}
\end{quote}

The format of the query string must correspond to that expected by the
server.  This restriction aside, considerable freedom is allowed in the
format of the query string.  See configuration file {\tt simpleconfig.cfg}
for examples.  The complete list of the permitted tokens is as follows.

\begin{description}

  \item[{\tt \%ra}] Right Ascension (see Section~\ref{CUF_R}).

  \item[{\tt \%dec}] Declination (see Section~\ref{CUF_R}).

  \item[{\tt \%x}] $x$\/ image pixel coordinate (see Section~\ref{CUF_R}).

  \item[{\tt \%y}] $y$\/ image pixel coordinate (see Section~\ref{CUF_R}).

  \item[{\tt \%r}] radius of a circular area search (see Section~\ref{CUF_R}).

  \item[{\tt \%r1}] inner radius (see Section~\ref{CUF_R}).

  \item[{\tt \%r2}] outer radius (see Section~\ref{CUF_R}).

  \item[{\tt \%w}] width of a rectangular region (see Section~\ref{CUF_R}).

  \item[{\tt \%h}] height of a rectangular region (see Section~\ref{CUF_R}).

  \item[{\tt \%m1}] Minimum magnitude.

  \item[{\tt \%m2}] Maximum magnitude.

  \item[{\tt \%m}] Maximum magnitude when no minimum is specified.

  \item[{\tt \%n}] Maximum number of objects (or rows) to return.

  \item[{\tt \%cols }] A list of the columns which the returned table
   is to contain.  If no value is supplied all the columns in the
   database will be returned.  The order of the list corresponds to the
   order in which the columns are returned.

  \item[{\tt \%id }] The name (or identifier) of an object submitted to
   a {\tt namesvr} (`name server') database.

  \item[{\tt \%mime-type}] The HTTP mime-type to be inserted in a HTTP
   {\tt get} command.

  \item[{\tt \%sort}] A list of columns on which the returned table is
   to be sorted.  If no value is specified the table is returned unsorted.

  \item[{\tt \%sortorder}] The order into which the returned table is
   sorted.  The permitted values are {\tt increasing} and {\tt decreasing}.

  \item[{\tt \%cond}] A string specifying one or more range searches.
   See Section~\ref{RANGE_R}.

\end{description}

\subsection{\label{CUF_R}Coordinates, units and formats}

By default databases are assumed to contain columns of Right Ascension
and Declination on which they can be searched.  However, some databases
do not contain such coordinates, but rather have Cartesian $x,y$\/
positions.  Typically such positions occur in catalogues generated
by detecting the objects present in a CCD direct image frame or a
digitised photographic plate.  The following rules apply.

\begin{enumerate}

  \item By default databases are assumed to be searchable on columns
   of Right Ascension and Declination.  Further, these columns are
   assumed to be called `{\tt ra}' and `{\tt dec}', respectively.
   Simply include the tokens {\tt \%ra} and {\tt \%dec} in the query.

  \item If a `positional' search (ie. one finding objects in either a
   circular area or a region bounded by meridians of Right Ascension and
   parallels of Declination) is to be made on columns of Right Ascension
   and Declination then the formats and units of the given values should
   be:

  \begin{description}

    \item[Right Ascension] sexagesimal or decimal hours,

    \item[Declination] sexagesimal or decimal degrees.

  \end{description}

   If a sexagesimal value is entered than a colon (`:') should be used
   as a separator.  The coordinates should be for equinox and epoch J2000.

  \item If a database is searchable on $x,y$\/ positions rather than
   celestial coordinates then the following keywords should be added to
   its entry in the configuration file:

  \begin{quote}
    \verb+ra_col:     -1+  \\
    \verb+dec_col:    -1+  \\
    \verb+x_col:      +(sequence number of column holding $x$\/ coordinate) \\
    \verb+y_col:      +(sequence number of column holding $y$\/ coordinate)
  \end{quote}

   Setting {\tt ra\_col} and {\tt dec\_col} to {\tt -1} indicates that
   columns of Right Ascension and Declination are not available.

  \item The units of the $x,y$\/ positions are simply `pixels', that is,
   dimensionless numbers.  Note however, that they are of type REAL rather
   than INTEGER; positions can be expressed to a fraction of a pixel.

  \item The units in which the radius, height and width of circular and
   `rectangular' positional searches are expressed also vary depending
   on whether the database contains celestial coordinates or $x,y$\/
   positions.  The tokens affected are:

  \begin{quote}
   {\tt \%r \%r1 \%r2 \%w \%h}
  \end{quote}

   The alternatives for their units are:

  \begin{description}

    \item[in a search on celestial coordinates:] decimal minutes of arc,

    \item[in a search on $x,y$\/ positions:] decimal pixels.

  \end{description}

\end{enumerate}

\subsection{\label{RANGE_R}Range queries}

To permit range queries on one or more columns of a database the keyword
{\tt search\_cols} should be included in the entry for the database in
the configuration file and the {\tt \%cond} keyword should be included
in the query part of its {\tt url} keyword.  These items have the
following syntax.

\begin{description}

  \item[{\tt search\_cols} keyword] A list of one or more column names
   together with text prompts for the minimum and maximum values of the
   range required.

  \begin{quote}
   {\tt search\_cols:} {\it column-name-1 minimum-prompt-1 maximum-prompt-1}
   {\tt :} \\
   {\it column-name-2 minimum-prompt-2 maximum-prompt-2} \ldots
  \end{quote}

   The client uses the minimum and maximum prompts when soliciting the
   extrema of the required range from the user.  An example might be

  \begin{quote}
   {\tt search\_cols: mag "Bright limit" "Faint limit" : ${\tt \backslash}$ \\
   b\_v "Minimum colour" "Maximum colour"}
  \end{quote}

  \item[{\tt \%cond} token] This token is included in the query part
   of the {\tt url} keyword and specifies any range search that is
   supplied.  The syntax of the values substututed into the {\tt \%cond}
   string is:

  \begin{quote}
   {\it column-name-1}\/ {\tt =} {\it minimum-value-1}{\tt ,}{\it
   maximum-value-1}{\tt \&}
   {\it column-name-2}\/ {\tt =} {\it minimum-value-2}{\tt ,}{\it
   maximum-value-2} \ldots
  \end{quote}

   When a search is specified {\it column-name-n}, {\it minimum-value-n}\/
   and  {\it maximum-value-n}\/ are substituted with respectively the column
   name, minimum value and maximum value.  The server should process this
   string and return only rows that lie within the given ranges.  An
   example of substituted query might be:

  \begin{quote}
   {\tt mag=1.0,2.0\&b\_v=0.1,0.3}
  \end{quote}

\end{description}

\subsection{\label{PLOTTING_R}Plotting directives}

The keyword {\tt symbol} prescribes how objects in the returned table
of values are to be displayed in finding charts, image overlays \emph{etc}.
Considerable flexibility is allowed in the way that objects are plotted.
In particular, the two common types of cases which are conventionally used
in atlases and finding charts are supported:

\begin{itemize}

  \item the size of the plotted symbol is scaled according to some
   property not directly related to the apparent size of the object.
   For optical star catalogues the magnitude is traditionally used,

  \item the size and orientation of the plotted symbol bears some
   relation to the apparent size and orientation of the image.  For
   example, galaxies are often plotted as ellipses with the outline of
   the ellipse approximating to some isophote.

\end{itemize}

The {\tt symbol} keyword only prescribes how objects should be plotted
in finding charts and image overlays; it is not appropriate for
scatter-plots where the two axes are not celestial coordinates or $x,y$\/
positions.  The syntax of the {\tt symbol} keyword is:

\begin{quote}
{\tt symbol:} {\it column-names symbol-info size-expr}\/ {\tt :} \ldots
\end{quote}

This triumavirate of items can be repeated an arbitrary number of times,
with each occurence being separated by a colon (`{\tt :}').  The meaning
of each of the three items is as follows.

\begin{description}

  \item[{\it column-names}] is a list of the names of columns in the
   returned table which appear as variables in {\it symbol-info}\/ or
   {\it size-expr}.  The names should be separated by spaces.  Column
   names containing embedded spaces or characters which Tcl\footnote{The
   Tcl scripting language is described by its author, John~Ousterhout, in
   his {\it Tcl and the Tk Toolkit}\/\cite{OUSTERHOUT94}.} interprets as
   special should be enclosed in curly brackets (`{\tt \{ \} }').
   Alternatively, names which contain spaces but not any characters
   which Tcl regards as special may be enclosed in double-quotes
   (`{\tt " "}').

  \item[{\it symbol-info}] prescribes the appearance of the plotting
   symbol.  See Section~\ref{SYMBOLINFO_R} for details.

  \item[{\it size-expr}] is an expression which is evaluated to give the
   size of each symbol.  This expression uses Tcl syntax, with some or
   all of the column names listed in {\it column-names}\/ appearing as
   variables.  The expression may optionally be followed by the units
   of the value computed.  The units permitted are:

  \begin{description}

    \item[{\tt image}] image pixels,

    \item[{\tt deg J2000}] degrees for equinox J2000,

    \item[{\tt deg B1950}] degrees for equinox B1950.

  \end{description}

\end{description}

\pagebreak
\begin{table}
\begin{center}
\begin{tabular}{lc}
Symbol & Scale and \\
       & rotate?   \\ \hline
{\tt square}   &   \\
{\tt circle}   &   \\
{\tt triangle} &   \\
{\tt cross}    &   \\
{\tt diamond}  &   \\
{\tt plus}     &  $\bullet$ \\
{\tt ellipse}  &  $\bullet$ \\
{\tt compass}  &  $\bullet$ \\
{\tt line}     &  $\bullet$ \\
{\tt arrow}    &  $\bullet$ \\
\end{tabular}
\end{center}

\begin{center}
\caption{Plotting symbols \label{PLOTTING_SYMBOLS} }
\end{center}

\end{table}

\begin{table}[htbp]

\begin{center}
\begin{tabular}{ll}
Colour   & Name \\ \hline
default  & (default)      \\
red      & {\tt red}      \\
green    & {\tt green}    \\
blue     & {\tt blue}     \\
cyan     & {\tt cyan}     \\
magenta  & {\tt magenta}  \\
yellow   & {\tt yellow}   \\
\end{tabular}
\end{center}

\begin{quote}
The default colour is the opposite of the plot background.  Usually
it will be black or white.
\end{quote}

\begin{center}
\caption{Colours for plot symbols recognised by CURSA \label{CURSA_COLOURS} }
\end{center}

\end{table}

\subsubsection{symbol-info \label{SYMBOLINFO_R}}

{\it symbol-info}\/ prescribes the appearance of the symbol plotted.
It is a list comprising the following items:

\begin{quote}
{\it symbol ~colour ~ratio ~angle ~label ~condition}
\end{quote}

{\it symbol}\/ is mandatory; the other items are optional.

\begin{description}

  \item[{\it symbol}] is the plotting symbol to be used.  The values
   permitted are listed in Table~\ref{PLOTTING_SYMBOLS}.  The symbols
   marked with a bullet (`$\bullet$') are typically scaled and rotated
   to correspond to the appearance of the image; see {\it ratio}\/ and
   {\it angle}, below.

  \item[{\it colour}] is the colour in which the symbol is plotted.  It may
   be any valid X colour name.  If {\it colour}\/ is omitted then GAIA
   and {\it SkyCat}\/ draw the symbol in black and white, which stand the
   best chance of being visible when overlaid on a colour image.  CURSA
   will recognise only the restricted set of colours listed in
   Table~\ref{CURSA_COLOURS}.

  \item[{\it ratio}\/ and {\it angle}] are respectively the width / height
   ratio of the object and rotation from north.
%  {\it angle} {\sf is measured clockwise? widdershins?, in degrees? radians?}.
   Both quantities may be given as expressions involving column names.

  \item[{\it label}] is the label for each object.  It may be fixed text,
   an expression involving column names or a mixture of the two.

  \item[{\it condition}] is a boolean expression involving column names
   whose value prescribes whether the object is plotted or not.  If the
   expression evaluates to {\tt true} (or 1) then the object is plotted;
   if it evaluates to {\tt false} (or 0) it is not.  By having a list of
   entries for the {\tt symbol} keyword, each with a different condition
   attached to the {\it symbol-info}\/ item different symbols can be
   plotted for each object, depending on the values of its column entries.

\end{description}

\subsubsection{Examples}

\begin{description}

  \item[{\tt symbol: "" circle 12}] ~ \\
   Plots objects as circles of constant size.

  \item[{\tt symbol: mag circle 15-\$mag}] ~ \\
   Plots objects as circles scaled according to magnitude (column {\tt
   mag}).

  \item[{\tt symbol: mag \{circle red\} 15-\$mag}] ~ \\
   Plots objects as red circles scaled according to magnitude.

  \item[{\tt symbol: \{a b pa\} \{ellipse red \$a/\$b \$pa\}
   \{\$a/3600.0 "deg J2000"\} }] ~ \\
   Plots objects as red ellipses.  The ratio of the ellipse is computed
   from the semi-major and semi-minor axes ({\tt a} and {\tt b}
   respectively).  The orientation is computed from position angle ({\tt
   pa}).  The image size is computed in degrees from the semi-major axis,
   which is assumed to be in seconds of arc.

  \item[{\tt symbol: \{mag rv\} \{circle red "" "" "" \$rv>0 \} 15-\$mag :}
   ${\tt \backslash}$] ~ \\
   {\tt \{mag rv\} \{circle blue "" "" "" \$rv<=0 \} 15-\$mag} \\
   Plot objects as circles scaled according to magnitude.  However,
   the colour of each circle depends on the radial velocity ({\tt rv}).
   Objects with a positive radial velocity are shown in red, those with
   a negative one in blue.

\end{description}

\subsubsection{Inserting private Tcl procedures}

If a standard Tcl expression does not allow you to calculate the symbol
size as you wish then you may be able to invoke your own Tcl procedure
to perform the calculation.  This facility is only available if you are
using {\it SkyCat}\/ as a client and is only likely to be useful with
your own catalogues.  An example might be:

\begin{quote}
{\tt symbol "rmag bmag" circle "[my\_plot\_proc \$rmag \$bmag]"}
\end{quote}

where you provide Tcl procedure {\tt my\_plot\_proc}.

\subsection{\label{CHECKCFG_R}Checking a configuration file}

A Perl script is available to check the validity of configuration files.
You can use it to check for errors in any configuration files that you
have created or modified.  It is included with the examples provided with
this document as file:

\begin{quote}
{\tt /star/examples/ssn75/checkcfg}
\end{quote}

You can either run it from the examples directory or copy it to a
convenient local directory.  On non-Starlink systems you may need to edit
the first line to correspond to wherever Perl is located on your system.

To use the script simply specify the name of the configuration file,
optionally preceded by the appropriate directory specification, on the
command line:

\begin{quote}
{\tt /star/examples/ssn75/checkcfg} ~ {\it configuration-file-name}
\end{quote}

For example, to check a copy of the example configuration file supplied
with this document (assuming that there is a copy in your current
directory):

\begin{quote}
{\tt /star/examples/ssn75/checkcfg ~ simpleconfig.cfg}
\end{quote}

If the configuration file is valid the {\tt checkcfg} will display the
message:

\begin{quote}
{\tt Configuration file parsed successfully.}
\end{quote}

Conversely, if there are problems with the configuration file then
explanatory error messages are displayed.  Usually the number of the
invalid line in the configuration file is included in the message.
Note that {\tt checkcfg} is mostly checking for syntax errors; it does
not, for example, check that any URLs specified are valid.


\section{\xlabel{TST_R}\label{TST_R}The Tab-Separated Table Format}

This section gives a brief description of the tab-separated table (TST)
format.  ACL servers should return the list of selected objects in this
format.  Various packages, including GAIA, CURSA and Starbase can also
read local files containing catalogues in this format.  There are
alternative descriptions of it in \xref{SUN/214}{sun214}{}\cite{SUN214},
the \htmladdnormallinkfoot{Starbase FAQ}
{http://cfa-www.harvard.edu/~john/starbase/FAQ.html}
and the {\it Astronomical Catalogue Library User Manual}\/\cite{BRIGHTON98}.

TST format files are text files.  They are usually generated by a remote
server in response to a query from a local client.  However, they could
equally well be local files created with a text editor.
Figure~\ref{TSTEXAMP} shows a simple example of a tab-separated table.
This example is available as file:

\begin{verse}
{\tt /star/examples/ssn75/simple.TAB}
\end{verse}

\begin{figure}[htbp]

% Ensure that this figure corresponds to the example data file.

\begin{verbatim}
Simple TST example; stellar photometry catalogue.

A.C. Davenhall (Edinburgh) 26/7/00.

Catalogue of U,B,V colours.
UBV photometry from Mount Pumpkin Observatory,
see Sage, Rosemary and Thyme (1988).

# Start of parameter definitions.
EQUINOX: J2000.0
EPOCH: J1996.35
# End of parameter definitions.

#column-units: <tab>Hours <tab>Degrees <tab>Magnitudes <tab>Magnitudes <tab>Magnitudes
#column-types: CHAR*6 <tab>DOUBLE <tab>DOUBLE <tab>REAL <tab>REAL <tab>REAL
#column-formats: A6 <tab>D13.6 <tab>D13.6 <tab>F6.2 <tab>F6.2 <tab>F6.2

Id<tab>ra<tab>dec<tab>V<tab>B_V<tab>U_B
--<tab>--<tab>---<tab>-<tab>---<tab>---
Obj. 1<tab> 5:09:08.7<tab> -8:45:15<tab>  4.27<tab>  -0.19<tab>  -0.90
Obj. 2<tab> 5:07:50.9<tab> -5:05:11<tab>  2.79<tab>  +0.13<tab>  +0.10
Obj. 3<tab> 5:01:26.3<tab> -7:10:26<tab>  4.81<tab>  -0.19<tab>  -0.74
Obj. 4<tab> 5:17:36.3<tab> -6:50:40<tab>  3.60<tab>  -0.11<tab>  -0.47
...
\end{verbatim}

\begin{quote}
\caption[A simple tab-separated table.]{A simple tab-separated table.
Note that in a tab-separated table the list of column names, sequences of
dashes and fields in the table are separated by tab characters.  In this
figure tab characters are indicated by `{\tt <tab>}'.  (Note that the
first column, {\tt Id}, is an object name and hence its units are left
blank.  Thus, the CURSA-specific {\tt column-units:} is separated from
the following {\tt <tab>} character by only one or more spaces; see
Section~\ref{CURSA_R} for details of the CURSA extensions.)
\label{TSTEXAMP} }
\end{quote}

\end{figure}

The description of the table and the table of values occupy the same
file and the description occurs at the start of the file.  Most of the
description shown in Figure~\ref{TSTEXAMP} is optional.

The first line of the description is a title.  Lines beginning with a
`{\tt \#}' are comments which are ignored.

Parameter definitions start with the parameter name, and a colon (`{\tt :}')
is appended to the end of the name to identify it as such.  The rest of
the line contains the parameter value.  The name and value are the only
information stored for each parameter.  The example contains two
parameters: {\tt EQUINOX} and {\tt EPOCH}.

Any remaining lines in the description (apart from the last two, which
immediately precede the table of values) are free text.

The only mandatory items in the description are the two lines immediately
before the table of values.  The first of these lines is the list of column
names.  Each name is separated by a single tab character (ASCII code nine;
strictly speaking the horizontal tab).  In the figure tab characters are
shown as `{\tt <tab>}'; obviously a real tab-separated table would contain
actual tab characters instead.  The name is the only mandatory information
stored for each column.  The TST format places few restrictions on the
column names: they can contain spaces, special and punctuation characters
\emph{etc}.  However, it is usually a prudent precaution to restrict
column names to contain only alphanumeric and underscore (`{\tt \_}')
characters and to make the first character alphabetic.  If these
precautions are observed then fewer problems are likely to occur if the
table is subsequently converted to another format or read by a variety
of different clients.  Remember that if you are writing a server which
returns a TST table via the Internet then you do not know which client
will be used to access it.

The line immediately after the list of column names indicates the end of
the description and the start of the table.  It consists solely of dashes
and tab characters.  By convention there are as many sequences of dashes
as there are column names, each sequence is separated by a single tab
character and each has the same number of dashes as there are characters
in the corresponding column name.

In the table of values each row occupies a single line.  Individual fields
are separated by a single tab character.  The fields occur in the same
order as the corresponding column names.

\subsection{\label{TSTSPCOL_R}Special columns of identifiers and
celestial coordinates}

The TST format has the following additional rules for special columns
containing identifiers and celestial coordinates.

\begin{enumerate}

  \item By default, the first three columns in the table are: an identifier
   (that is, an object name), Right Ascension and Declination.  These
   columns are usually called {\tt Id}, {\tt ra} and {\tt dec}
   respectively.

  \item If the first three columns of the table are not an identifier,
   Right Ascension and Declination then the TST should contain the
   following three parameters:

  \begin{quote}
   {\tt id\_col \\
   ra\_col \\
   dec\_col}
  \end{quote}

   The value of each parameter should be the number of the appropriate
   column or, if there is no such column, {\tt -1}.  The column number is the
   sequence number of the column in the list of column names, starting
   counting at zero.

  \item The Right Ascension and the Declination are either both in
   decimal degrees or the Right Ascension is in sexagesimal hours
   and the Declination is in sexagesimal degrees.  If sexagesimal
   notation is used then a colon (`{\tt :}') is used as the sexagesimal
   separator.

  \item The equinox and epoch are specified using parameters {\tt
   equinox} and {\tt epoch} respectively.  The default value if the
   equinox and epoch are not specified is J2000.0.

\end{enumerate}

\subsection{Conventions for tables returned by ACL servers}

In addition to the basic TST format the following additional conventions
apply to tables returned by ACL servers.

\begin{enumerate}

  \item Keywords specified for a database in its entry in the configuration
   file are equivalent to, and treated as, parameters in the TST.  Thus,
   {\tt id\_col}, {ra\_col} and {\tt dec\_col} are usually supplied as
   keywords in the configuration file rather than parameters in the TST
   returned by the server.  If a client writes a table of retrieved objects
   as a local file in the TST format it should include these keywords as
   parameters.

  \item Some types of database, particularly {\tt archives} (see
   Section~\ref{SERVERS_R}), may return tables containing columns called
   {\tt more} and {\tt preview}.  These columns are optional.  However, if
   present they should be used as follows.

  \begin{description}

    \item[{\tt more}] a URL giving more information on the object.  The
     destination URL will typically be a conventional page of HTML,
     suitable for display with a Web browser.

    \item[{\tt preview}] a URL pointing to an image of the object.

  \end{description}

\end{enumerate}

\subsection{\label{CURSA_R}CURSA extensions}

Some additional items can be added to the TST header information which
provide additional information about the catalogue and allow
\xref{CURSA}{sun190}{}\cite{SUN190} to interpret it more precisely.
These items begin with a `{\tt \#}' character and thus are TST comments.
Consequently a catalogue which contains them remains perfectly standard
and valid and can be processed with software other than CURSA.  The
items are illustrated in Tables~\ref{SIMPLE_TST} and ~\ref{TSTEXAMP}
and are as follows.

\begin{description}

  \item[{\tt \#column-units:}] is followed by a tab-separated list of
   units for the columns.

  \item[{\tt \#column-types:}] is followed by a tab-separated list of
   data types for the columns.  The permitted data types are listed in
   Table~\ref{DATA_TYPES}.

  \item[{\tt \#column-formats:}] is followed by a tab-separated list of
   display formats for the columns.  Fortran-like, FITS-compatible formats
   are used.  This convention facilitates converting tables between the
   TST and FITS tables formats without loss of information.

\end{description}

\begin{table}[htbp]

\begin{center}
\begin{tabular}{llc}
CURSA Data Type & Description      & Standard \\
                &                  & Fortran 77? \\ \hline
BYTE            & Signed byte      & No   \\
WORD            & Signed word      & No   \\
INTEGER         & Signed integer   & Yes  \\
REAL            & Single precision & Yes  \\
DOUBLE          & Double precision & Yes  \\
LOGICAL         & Logical          & Yes  \\
CHAR[$*n$]      & Character string & Yes  \\
\end{tabular}

\begin{quote}
$n$ is the number of elements in the character string; it is a positive
integer.
\end{quote}

\caption{CURSA data types \label{DATA_TYPES} }
\end{center}

\end{table}

In all cases the items are listed in the order in which they occur
in the table.  There is no tab character between {\tt \#column-units:},
{\tt \#column-types:} and {\tt \#column-formats:} and the following
value.  Values in the special TST columns of Right Ascension and
Declination (that is, those identified by the {\tt ra\_col} and {\tt
dec\_col} parameters) are always interpretted using the TST rules for
representing angles rather than the CURSA ones.

\subsection{\label{MIME_R}MIME type}

The first line of information returned by an ACL server should always
be the MIME\footnote{Multipurpose Internet Mail Extensions.  The MIME
protocol was originally developed to allow non-text data to be included
in e-mail messages and was subsequently adopted for use with HTTP.} type:

\begin{quote}
{\tt Content-type: text/plain}
\end{quote}

This line is strictly speaking not part of the TST table and will be
used by the client or Web browser to interpret the format of the data
which follow.


\section{\xlabel{IMAGE_R}\label{IMAGE_R}Image Servers}

Image servers differ from other types of ACL server in that they return
a direct image or `pixel map' of a region of sky rather than a catalogue
of objects.  The image is returned formatted as a
\htmladdnormallinkfoot{FITS}{http://fits.gsfc.nasa.gov/fits\_home.html}
file which may optionally be compressed.

\begin{enumerate}

  \item Image servers have a configuration file entry similar to those
   for other types of database.  The server type is {\tt imagesvr}.
   The region of sky required is specified by giving the central
   Right Ascension and Declination (or $x,y$\/ pixel position if
   celestial coordinates are not available), width and height.  The
   usual tokens (see Section~\ref{QUERY_R}):

  \begin{quote}
  {\tt ~\%ra ~\%dec ~\%x ~\%y ~\%w ~\%h}
  \end{quote}

   are included in a query string as part of a {\tt url} keyword in the
   usual fashion.  The tokens have their usual meaning and units.  An
   example configuration file entry for an image server is:

\begin{verbatim}
serv_type:      imagesvr
long_name:      Digitized Sky at ESO
short_name:     dss@eso
url:            http://archive.eso.org/dss/dss\
?ra=%ra&dec=%dec&mime-type=%mime-type&x=%w&y=%h
copyright:      Digitized Sky Survey (c) by AURA, provided online by ESO
\end{verbatim}

  \item The server parses the query and generates a FITS file corresponding
   to the region specified.  It first returns the MIME type, followed
   by the FITS file which optionally may be compressed.  The MIME types
   corresponding to the various types of compression are listed in
   Table~\ref{FITS_MIME}.  For example, the MIME type for a gzipped
   FITS file is:

  \begin{quote}
   {\tt Content-type: image/x-gfits}
  \end{quote}

\end{enumerate}

\begin{table}[htbp]

\begin{center}
\begin{tabular}{ll}
MIME type                & Type of compression   \\ \hline
{\tt image/fits}         & uncompressed          \\
{\tt image/x-hfits}      & H compressed          \\
{\tt image/x-gfits}      & gzipped               \\
{\tt image/x-gstarbase}  & gzipped (alternative) \\
{\tt image/x-cfits}      & UNIX compressed       \\
{\tt image/x-cstarbase}  & UNIX compressed (alternative)  \\
{\tt text/html}          & error message in HTML format   \\
\end{tabular}

\caption{MIME types for compressed FITS images \label{FITS_MIME} }
\end{center}

\end{table}

% \newpage
\addcontentsline{toc}{section}{Acknowledgements}
\section*{Acknowledgements}

The material in Part II is largely based on Section~2 of Allan~Brighton's
{\it Astronomical Catalogue Library User Manual}\/\cite{BRIGHTON98}.
Martin~Bly, Allan~Brighton, Peter~Draper and Mike~Read provided useful
comments on the draft version of the document.  Any mistakes, of course,
are my own.


% References ----------------------------------------------------------

% \section{References}

% \input{refs.tex}

\newpage
\addcontentsline{toc}{section}{References}
\begin{thebibliography}{99}

  \bibitem{SERVERURL} M.~Albrecht, M.~Barylak, D.~Durand, P.~Fernique,
   A.~Micol, F.~Ochsenbein, F.~Pasian, B.~Pirenne, D.~Ponz and
   M.~Wenger, 19 September 1996, {\it Astronomical Server URL}\,
   (Version 1.0).  See URL: \htmladdnormallink{
   {\tt http://vizier.u-strasbg.fr/doc/asu.html}}
   {http://vizier.u-strasbg.fr/doc/asu.html}

  \bibitem{BRIGHTON98} A.~Brighton, 16 January 1998, {\it Astronomical
   Catalog Library User Manual}, issue 3.1 (document number
   GEN-SPE-ESO-19400-0949), European Southern Observatory Very Large
   Telescope Data Management Division.

  \bibitem{SUN190} A.C.~Davenhall, 25 July 2000,
   \xref{SUN/190.8}{sun190}{}: {\it CURSA --- Catalogue and Table
    Manipulation Applications}, Starlink.

  \bibitem{SUN214} P.W.~Draper and N.~Gray, 27 January 2000,
   \xref{SUN/214.7}{sun214}{}: {\it GAIA --- Graphical Astronomy and
   Image Analysis Tool}, Starlink.

  \bibitem{GRAHAM95} I.S.~Graham, {\it The HTML Sourcebook}, 1995 (John
   Wiley and Sons: New York).

  \bibitem{OUSTERHOUT94} J.K.Ousterhout, 1994, {\it Tcl and the Tk
   Toolkit}\/ (Addison-Wesley: Reading, Massachusetts).

  \bibitem{SCHWARTZ93} R.L.~Schwartz, 1993, {\it Learning Perl}
   (O'Reilly and Associates Inc: Sebastopol, California).

  \bibitem{WALL91} L.~Wall and R.~L.~Schwartz, 1991, {\it Programming
   Perl}, (O'Reilly and Associates Inc: Sebastopol, California).

\end{thebibliography}

\typeout{  }
\typeout{*****************************************************}
\typeout{  }
\typeout{Reminder: run this document through Latex three times}
\typeout{to resolve the references.}
\typeout{  }
\typeout{*****************************************************}
\typeout{  }

\end{document}
