\documentclass[11pt]{article}
\pagestyle{myheadings}

% -----------------------------------------------------------------------------
\newcommand{\stardoccategory}  {Starlink System Note}
\newcommand{\stardocinitials}  {SSN}
\newcommand{\stardocnumber}    {30.2}
\newcommand{\stardocsource}    {ssn\stardocnumber}
\newcommand{\stardocauthors}   {T.\,M.\,Gledhill}
\newcommand{\stardocdate}      {13th June 1996}
\newcommand{\stardoctitle}     {SYSTUNE\\[1ex]
 Tools for SPARC/Solaris Performance Monitoring and Tuning\\[1ex]
V0.4}
% -----------------------------------------------------------------------------

\newcommand{\stardocname}{\stardocinitials /\stardocnumber}
\markright{\stardocname}
\setlength{\textwidth}{160mm}
\setlength{\textheight}{230mm}
\setlength{\topmargin}{-2mm}
\setlength{\oddsidemargin}{0mm}
\setlength{\evensidemargin}{0mm}
\setlength{\parindent}{0mm}
\setlength{\parskip}{\medskipamount}
\setlength{\unitlength}{1mm}

% -----------------------------------------------------------------------------
% Hypertext definitions.
% These are used by the LaTeX2HTML translator in conjuction with star2html.

% Comment.sty: version 2.0, 19 June 1992
% Selectively in/exclude pieces of text.
%
% Author
%    Victor Eijkhout                                      <eijkhout@cs.utk.edu>
%    Department of Computer Science
%    University Tennessee at Knoxville
%    104 Ayres Hall
%    Knoxville, TN 37996
%    USA

%  Do not remove the %begin{latexonly} and %end{latexonly} lines (used by 
%  star2html to signify raw TeX that latex2html cannot process).
%begin{latexonly}
\makeatletter
\def\makeinnocent#1{\catcode`#1=12 }
\def\csarg#1#2{\expandafter#1\csname#2\endcsname}

\def\ThrowAwayComment#1{\begingroup
    \def\CurrentComment{#1}%
    \let\do\makeinnocent \dospecials
    \makeinnocent\^^L% and whatever other special cases
    \endlinechar`\^^M \catcode`\^^M=12 \xComment}
{\catcode`\^^M=12 \endlinechar=-1 %
 \gdef\xComment#1^^M{\def\test{#1}
      \csarg\ifx{PlainEnd\CurrentComment Test}\test
          \let\html@next\endgroup
      \else \csarg\ifx{LaLaEnd\CurrentComment Test}\test
            \edef\html@next{\endgroup\noexpand\end{\CurrentComment}}
      \else \let\html@next\xComment
      \fi \fi \html@next}
}
\makeatother

\def\includecomment
 #1{\expandafter\def\csname#1\endcsname{}%
    \expandafter\def\csname end#1\endcsname{}}
\def\excludecomment
 #1{\expandafter\def\csname#1\endcsname{\ThrowAwayComment{#1}}%
    {\escapechar=-1\relax
     \csarg\xdef{PlainEnd#1Test}{\string\\end#1}%
     \csarg\xdef{LaLaEnd#1Test}{\string\\end\string\{#1\string\}}%
    }}

%  Define environments that ignore their contents.
\excludecomment{comment}
\excludecomment{rawhtml}
\excludecomment{htmlonly}

%  Hypertext commands etc. This is a condensed version of the html.sty
%  file supplied with LaTeX2HTML by: Nikos Drakos <nikos@cbl.leeds.ac.uk> &
%  Jelle van Zeijl <jvzeijl@isou17.estec.esa.nl>. The LaTeX2HTML documentation
%  should be consulted about all commands (and the environments defined above)
%  except \xref and \xlabel which are Starlink specific.

\newcommand{\htmladdnormallinkfoot}[2]{#1\footnote{#2}}
\newcommand{\htmladdnormallink}[2]{#1}
\newcommand{\htmladdimg}[1]{}
\newenvironment{latexonly}{}{}
\newcommand{\hyperref}[4]{#2\ref{#4}#3}
\newcommand{\htmlref}[2]{#1}
\newcommand{\htmlimage}[1]{}
\newcommand{\htmladdtonavigation}[1]{}

% Define commands for HTML-only or LaTeX-only text.
\newcommand{\html}[1]{}
\newcommand{\latex}[1]{#1}

% Use latex2html 98.2.
\newcommand{\latexhtml}[2]{#1}

% Starlink cross-references and labels.
\newcommand{\xref}[3]{#1}
\newcommand{\xlabel}[1]{}

%  LaTeX2HTML symbol.
\newcommand{\latextohtml}{{\bf LaTeX}{2}{\tt{HTML}}}

%  Define command to recentre underscore for Latex and leave as normal
%  for HTML (severe problems with \_ in tabbing environments and \_\_
%  generally otherwise).
\newcommand{\setunderscore}{\renewcommand{\_}{{\tt\symbol{95}}}}
\latex{\setunderscore}

% -----------------------------------------------------------------------------
%  Debugging.
%  =========
%  Un-comment the following to debug links in the HTML version using Latex.

% \newcommand{\hotlink}[2]{\fbox{\begin{tabular}[t]{@{}c@{}}#1\\\hline{\footnotesize #2}\end{tabular}}}
% \renewcommand{\htmladdnormallinkfoot}[2]{\hotlink{#1}{#2}}
% \renewcommand{\htmladdnormallink}[2]{\hotlink{#1}{#2}}
% \renewcommand{\hyperref}[4]{\hotlink{#1}{\S\ref{#4}}}
% \renewcommand{\htmlref}[2]{\hotlink{#1}{\S\ref{#2}}}
% \renewcommand{\xref}[3]{\hotlink{#1}{#2 -- #3}}
%end{latexonly}
% -----------------------------------------------------------------------------
% Add any document-specific \newcommand or \newenvironment commands here

% -----------------------------------------------------------------------------
%  Title Page.
%  ===========
\begin{document}
\thispagestyle{empty}

%  Latex document header.
\begin{latexonly}
   CCLRC / {\sc Rutherford Appleton Laboratory} \hfill {\bf \stardocname}\\
   {\large Particle Physics \& Astronomy Research Council}\\
   {\large Starlink Project\\}
   {\large \stardoccategory\ \stardocnumber}
   \begin{flushright}
   \stardocauthors\\
   \stardocdate
   \end{flushright}
   \vspace{-4mm}
   \rule{\textwidth}{0.5mm}
   \vspace{5mm}
   \begin{center}
   {\Large\bf \stardoctitle}
   \end{center}
   \vspace{5mm}

%  Add heading for abstract if used.
%   \vspace{10mm}
%   \begin{center}
%      {\Large\bf Description}
%   \end{center}
\end{latexonly}

%  HTML documentation header.
\begin{htmlonly}
   \xlabel{}
   \begin{rawhtml} <H1> \end{rawhtml}
      \stardoctitle
   \begin{rawhtml} </H1> \end{rawhtml}

%  Add picture here if required.

   \begin{rawhtml} <P> <I> \end{rawhtml}
   \stardoccategory\ \stardocnumber \\
   \stardocauthors \\
   \stardocdate
   \begin{rawhtml} </I> </P> <H3> \end{rawhtml}
      \htmladdnormallink{CCLRC}{http://www.cclrc.ac.uk} /
      \htmladdnormallink{Rutherford Appleton Laboratory}
                        {http://www.cclrc.ac.uk/ral} \\
      Particle Physics \& Astronomy Research Council \\
   \begin{rawhtml} </H3> <H2> \end{rawhtml}
      \htmladdnormallink{Starlink Project}{http://www.starlink.ac.uk/}
   \begin{rawhtml} </H2> \end{rawhtml}
   \htmladdnormallink{\htmladdimg{source.gif} Retrieve hardcopy}
      {http://www.starlink.ac.uk/cgi-bin/hcserver?\stardocsource}\\

% HTML document table of contents (if used). 
% ==========================================
% Add table of contents header and a navigation button to return 
% to this point in the document (this should always go before the
% abstract \section). This places the table of contents on the title
% page. Do not use this if you want the normal behaviour.
%   \label{stardoccontents}
%   \begin{rawhtml} 
%     <HR>
%     <H2>Contents</H2>
%   \end{rawhtml}
%   \htmladdtonavigation{\htmlref{\htmladdimg{contents_motif.gif}}
%                                            {stardoccontents}}

%  Start new section for abstract if used.
%  \section{\xlabel{abstract}Abstract}

\end{htmlonly}

% -----------------------------------------------------------------------------
%  Document Abstract. (if used)
%  ==================
% -----------------------------------------------------------------------------
%  Latex document Table of Contents. (if used)
%  ===========================================
%  Replace the \latexonlytoc command with \tableofcontents if you're
%  not only having a contents list on the title page.
\begin{latexonly}
   \setlength{\parskip}{0mm}
   \tableofcontents
   \setlength{\parskip}{\medskipamount}
   \markright{\stardocname}
\end{latexonly}
% -----------------------------------------------------------------------------

\newpage
\section{Introduction}

This document follows on from \xref{SSN/26}{ssn26}{} which provides an
introduction to performance and tuning issues on SPARC/Solaris systems. 
Here we describe a package of tools designed to help monitor performance 
on your system, implement some of the advice given in SSN/26 and provide 
tuning tips.

In order to monitor performance and perform tuning operations, the Solaris
system manager traditionally has to rely on tools such as {\tt{vmstat}}, 
{\tt{iostat}}, {\tt{sar}} and the like which are somewhat limited in their
scope and usefulness. Repeatedly running tools such as these and maybe
using perl or awk scripts to parse the results can lead to unacceptable
overheads and even adversely affect the performance data being recorded.

Using an interpreted language called SymbEL (Symbol Engine Language),
developed by Richard Pettit at Sun Microsystems, it is now possible to 
retrieve information directly from the SPARC kernel and to construct
home-made tools simply by writing a script. This places information 
resources that were only previously available to the system C programmer 
at the disposal of every system manager. A good number of tools have already
been written\footnote{The tools currently available are those distributed
with the SymbEL package and have been written by Richard Pettit and Adrian
Cockcroft}, some of which are sophisticated GUI based system monitors, and
they form the basis of this package.

At the moment the package is essentially a wrap-around for the SPARC tuning
rules and tools package written by Adrian Cockcroft and Richard Pettit. The
tools have been collected together using a GUI front end which also provides
access to documentation and online help. In the near future, additional tools
should become available (perhaps contributed by Starlink Managers) which
can be incorporated into the package.

{\bf{Note that this package is only relevant to SPARC systems running Solaris
2.x}}. It will not work on systems running SunOS 4.x (Solaris 1). In addition,
some tools require the Motif library which was only included with Solaris at
version 2.4. 

A detailed guide to the interpretation of results from the various tools
in this package is beyond the scope of this brief document -- there are
simply too many options to cover. The package is provided as a toolkit
to enable system managers to develop the performance monitoring utilities
that will be of most use to them. 

The following sections describe the package installation, its use, and
a little about how it works. The document is intended only as an
introduction and as a pointer to other sources of information.

\section{\label{installation}\xlabel{installation}Installation and Removal}

The package installs under the {\tt{/opt}} filesystem in {\tt{/opt/RICHPse}}
and currently requires $\sim2$MB of disk space. In order to install the
package, superuser privileges are required.\footnote{Generally, superuser
privileges are required to write to {\tt{/opt}} and for {\tt{pkgadd}} to
execute properly} Scripts are provided to install and remove the package
cleanly.

\subsection{Installation}

To install the package, proceed as follows:

\begin{enumerate}

\item Obtain the tar file {\tt{systune\_v0.4.tar}} from the Starlink
FTP server \htmladdnormallinkfoot{}
{ftp://starlink-ftp.rl.ac.uk/pub/tools/systune\_v0.4.tar}
and expand the contents into a temporary directory\ldots

\begin{verbatim}
   # cd /temp-directory
   # tar xvf systune_0.4.tar
\end{verbatim}

\ldots to create some files in {\tt{systune\_v0.4}}.

\item Move to the created directory and, as {\tt{root}}, execute the install
script.

\begin{verbatim}
   # cd systune_v0.4
   # ./install
\end{verbatim}

Some informational messages will be displayed and you will be prompted
to confirm installation under {\tt{/opt/RICHPse}}. You will be given
the option of installing the package `architecture neutral'. This just
means that SE2.5 will work across platforms with different releases of
Solaris.  If you are only running one release ({\em{e.g.}}, 2.5) then
say {\tt{`n'}}, otherwise say {\tt{`y'}}. You will also be asked whether
you want to start some monitoring tools at boot time by inserting
start-up scripts in {\tt{/etc/rc2.d}}. The recommendation is that you
decline this option until experience has been gained running the
monitors from a regular user account. The start-up scripts will still
be inserted as templates in {\tt{/etc/init.d}}. If you want to enable
them at a later date then uncomment the commented-out lines.

\item The installation procedure will leave behind a {\tt{remove}} script
to cleanly remove the installation. This should be kept.

\end{enumerate}

After installation has completed, the software will be
installed in the directory {\tt{/opt/RICHPse}}. The SysTune GUI can be
started with the command:

\begin{verbatim}
   > /opt/RICHPse/starlink/systune.se &
\end{verbatim}

or by defining an alias to this command. The GUI, and most of the tools
can be run from a regular user account although a few need to be run as
root to function properly. Details are given under the Help menu when you 
run up the GUI.

\subsection{Removal}

The package can be cleanly removed using the {\tt{remove}} script left
behind by the installation process. You should first terminate any
running tools or monitors.

\begin{enumerate}
\item As {\tt{root}}, execute the package removal script.
\begin{verbatim}
   # cd /temp-directory/systune_v0.4
   # ./remove
\end{verbatim}
\end{enumerate}


\section{\label{using_the_package}\xlabel{using_the_package}Using the Package}

Although the tools in this package can be invoked independently, they
are more conveniently accessed from the SysTune GUI which groups them
loosely under two headings: 

\begin{enumerate}

\item {\bf{Statistics:}} These tools retrieve system statistics to give a
real-time picture of how the system is performing. Examples would be looking
at disk access times, network collision rates, available swap space {\em{etc.}}

\item {\bf{Monitors:}} These tools typically run in the background and
periodically check the system for deviation from established performance
tolerences using a rules strategy. When a performance target is broken,
the manager is notified.

\end{enumerate}

The tools are summarised in Table~\ref{table1} and Table~\ref{table2} and are
further described in the online help accessible through the SysTune GUI,
which is intended to be self documenting to a large extent.

\begin{table}
\begin{center}
\begin{tabular}[t]{|l|l|l|l|l|}
\hline
\multicolumn{5}{|c|}{\bf Statistics Tools} \\
\hline \hline
\bf Networks & \bf CPU system & \bf Disks & \bf NFS & \multicolumn{1}{c|}{\bf Description} \\
\hline
collisions &&&& collision rates for ethernet interfaces \\
net        &&&& network statistics ({\tt netstat -i}) \\
netstatx   &&&& network retransmissions/collisions \\
\hline
& cpus     &&&  cpus and clock rates \\
& {\bf msacct} &&& microstate accounting \\
& nproc    &&& number of processes \\
& ps-ax    &&& {\tt ps-ax} command \\
& {\bf pwatch}   &&& watch a process \\
& uptime   &&& continuous uptime display \\
& cpu meter &&& CPU usage GUI \\
& multi meter &&& GUI showing CPU and VM system \\
& kview    &&& GUI to monitor kernel variables \\
\hline
&& iostat  &&  like {\tt iostat -D} \\
&& xio     && disk access statistics \\
&& xit     && GUI version of {\tt iostat} \\
&& dfstats && GUI version of {\tt df} \\
\hline
&&& {\bf nfsstat-m} & like {\tt nfsstat -m} \\
\hline
& \multicolumn{1}{c}{vmstat} &&& like {\tt vmstat} \\
& \multicolumn{1}{c}{mpvmstat} &&& {\tt vmstat} for multi-processors \\
\hline
\multicolumn{3}{|c}{infotool} && global system information GUI\\
\hline
\end{tabular}
\caption{A summary of the tools in the statistics grouping. Tools
in bold type require root privileges to function properly.}
\label{table1}
\end{center}
\end{table}

\begin{table}
\begin{tabular}[h]{|l|p{4.5in}|}
\hline
\multicolumn{2}{|c|}{\bf Monitor Tools} \\
\hline \hline
\multicolumn{1}{|c|}{\bf Monitor}& \multicolumn{1}{c|}{\bf Description} \\
\hline
quick tune     & not really a monitor -- performs a quick system tuning 
                  diagnostic to identify possible problems \\
ruletool       & probably the most useful tool at the moment -- a GUI based  
                 monitor that interactively checks a
                 system at specified intervals for a wide range of        
                 performance problems \\
virtual adrian & a command line version of {\em ruletool} -- can be started
                 at boot time and can log results to a file \\
calendar monitor & a monitor which logs its results as calendar appointments
                   using the OpenWindows calendar manager. Useful for  
                   collecting results from multiple machines \\
syslog monitor & a monitor which logs its results {\em via} the syslogd \\
io monitor      & a monitor version of {\tt iostat} -- monitors for slow 
                 disks \\
net monitor    & a monitor version of {\tt netstatx} -- monitors for a slow
                 network \\
{\bf nfs monitor}   &  a monitor version of {\tt nfsstat-m} -- monitors for slow
                 NFS client mount points \\
vm monitor    & a monitor version of {\tt vmstat} -- monitors for low RAM,  
                swap and CPU overloads \\
{\bf percollator} & Performance collator: Runs the performance rules, including
WWW server performance, and writes concise output to a log file or stdout. \\ 
\hline
\end{tabular}
\caption{A summary of the tools in the monitors grouping. Tools in bold type 
require root privileges to function properly.}
\label{table2}
\end{table}

\newpage
\subsection{Statistics Tools}

The statistics tools retrieve information from the kernel and currently 
display it in one of three ways:

\begin{enumerate}

\item Tools with GUIs, such as {\tt{infotool}},  use their own pop-up
text windows or graphics to display information. These tools can be
exited and otherwise controlled {\em{via}} their own menu bar options.
After the tool has started, the SysTune window can be used again.

\item Some tools, such as {\tt{ps-ax}}, display their output in the
SysTune text window. Whilst these tools are running, the SysTune window
is frozen until they complete their function.

\item Other tools, such as {\tt{iostat}} and {\tt{vmstat}}, run in the 
background and send their output to stdout. These tools loop until explicitly
terminated using the {\tt{stop}} button on the SysTune GUI. Whilst these
tools are executing other tools cannot be launched.

\end{enumerate}

\subsection{\label{montools}\xlabel{montools}Monitor Tools}

The monitor tools run continuously in the background and only output 
information when they detect a problem. They periodically retrieve
information from the kernel (such as disk access times and CPU loads) and
feed it into a series of rules which define acceptable ranges of
performance. When one of these rules is `broken', the manager is notified.

Each rule has a default definition which can be seen using the {\bf{rules}}
button on the SysTune GUI. These are the rules that appear in Appendix A
of {\em{Sun Performance and Tuning}}. The value of a rule can be changed
by defining a new current value as an environment variable. For example,
the disk service time beyond which slow disk problems are reported is, by
default, 50ms. This could be increased to 70ms if required:

\begin{verbatim}
   > setenv DISK_SVC_T_PROBLEM 70
\end{verbatim}

Obviously, the default tuning rules should not be changed without good 
reason and probably only after considerable experimentation. For the changes
to have effect, any running monitors would have to be restarted. In the 
current implementation, the SysTune GUI needs to be restarted as well.

The monitor tools use the colour coding scheme described in Appendix A of
{\em{Sun Performance and Tuning}} to categorise the nature and severity of 
a problem. This is summarised in Table~\ref{table3}. For a graphic
illustration of this colour coding scheme in action, start up the RuleTool
monitor on a fairly busy system.

\begin{table}
\begin{center}
\begin{tabular}[t]{|l|p{4.5in}|}
\hline
\multicolumn{2}{|c|}{\bf Monitor Colour Codes} \\
\hline \hline
\multicolumn{1}{|c|}{\bf Colour} & \multicolumn{1}{c|}{\bf Meaning} \\
\hline
blue     & under utilisation or imbalance of an expensive resource \\
white    & low usage or inactivity \\
green    & performance within specifications \\
amber    & warning level - possible problem \\
red      & problem level - something needs adjusting for good performance \\
black    & critical problem could cause application or system to hang \\
\hline
\end{tabular}
\caption{A summary of the colour codes used with the monitor tools, as
defined in Appendix A of {\em{Sun Performance and Tuning}}.}
\label{table3}
\end{center}
\end{table}
 
\subsubsection{Interactive Monitoring}

Most of the monitors use the same rules definitions but output their results
in different ways. The {\bf{RuleTool}} monitor is probably the most useful
interactive system monitor available. This is a GUI based tool which uses
the above mentioned colour-coding scheme to indicate performance problems.
By clicking on the coloured bars, the data relating to the notified problem
can be obtained. 

\subsubsection{Non-interactive Monitoring}

It may be more convenient to have a monitor run in the background and
log its results to a file. The virtual\_adrian, syslog monitor, and calendar
monitors all provide variations on this theme. 

The calendar monitor ({\tt{mon\_cm}}) checks the system over a 15 minute
period and if any rule gives an amber state or worse, the rule state is
entered as an appointment for that time period. The calendar to use can be
supplied as a command line option so that output from monitors on 
different hosts can be sent to a calendar manager on a central host.

In order to send output from these monitors to a file or supply other
options (such as the calendar for {\tt{mon\_cm}} to use) the monitors must be
started from the command line using the full path, since in the current
implementation these options cannot be provided if the monitors are
launched from the SysTune GUI. This is a limitation of the GUI library
used to implement the SysTune GUI. 

As an example, here is a command to start the {\tt{mon\_cm}} monitor on node
{\tt{client1}} with a monitoring interval of 30 minutes (1800 seconds) and
send the output to the calendar manager belonging to user {\tt{star}} on node 
{\tt{server1}}. A calendar file must already have been created for user 
{\tt{star}} on {\tt{server1}} using {\tt{cm}} and you must have permission to
write to it. 

\begin{verbatim}
   client1> /opt/RICHPse/examples/mon_cm.se 1800 star@server1 &
\end{verbatim}

For further information on available command line options, see the 
information on {\bf{Tools}} available through the SysTune HELP menu.

The percollator.se monitor is new to SE2.5. It applies the performance
rules, including new rules for WWW server performance, and logs its
results in a concise format. It is intended to be a long-term monitor 
allowing usage/performance trends to be analysed. Graphing tools which
will use the concise output are currently being developed.

\subsubsection{Starting Monitors at Boot Time}

The four monitors virtual\_adrian, syslog monitor, percollator and
calendar monitor can be started at boot time by adding run control
scripts in /etc/rc2.d at sequence number 90.  The appropriate scripts
are placed in {\tt{/etc/init.d}} by the package installation procedure.
The scripts are commented out unless the option to activate monitors at
boot time was chosen during the installation. The scripts are:

\begin{quote}
\begin{verbatim}
/etc/init.d/mon_cm
/etc/init.d/monlog
/etc/init.d/va_monitor
/etc/init.d/percol
\end{verbatim}
\end{quote}

For example, to start up the calendar monitor at boot time you would first
edit {\tt{/etc/init.d/\-mon\_cm}} to remove commented out lines (if any) and
then create a link to it in {\tt{/etc/rc2.d}} 

\begin{quote}
{\tt{ln -s /etc/init.d/mon\_cm /etc/rc2.d/S90mon\_cm}}
\end{quote}

\section{\label{symbel_lang}\xlabel{symbel_lang}The SymbEL Language}

This section describes in a little more detail the SymbEL language used
to implement the tools in this package. 

SymbEL is an interpreted language ({\em{i.e.}}, it doesn't need to be 
compiled) that has a syntax very similar to that of C. The beauty of SymbEL
is that it allows the user to extract information from the running SPARC
kernel simply by writing a script, in the same way that you could extract
information from a file by writing a Perl script. Previously, to retrieve
kernel information, you would have to write and compile a C program to use
the kernel {\tt{kvm}} library, or in SunOS 5.x the {\tt{kstat}} library

Although the SymbEL language is syntactically similar to C, a key difference
is the definition of {\bf{active}} and {\bf{inactive}} variables. When an
active variable is read, a kernel access occurs and data is retrieved from
the kernel into that variable. Usually, active variables are structures so
that the whole structure will be filled when the variable is read.

SymbEL provides special {\bf{language classes}} that are used to declare the
active property of a variable, that is, that it will receive information
from the  kernel. This property is signified in a SymbEL script by
preceeding the variable by the language class followed by a dollar ({\tt{\$}})
sign. There are currently four language classes; the ones that provide
access to the kstat and kvm data being the kstat and kvm language classes. 

This is probably best illustrated by an example. The following script
will report the number of processes currently running on the system. If
the SymbEL interpreter ({\tt{/opt/RICHPse/bin/se}}) is installed on the
system, then this script could be typed in and executed.

\begin{quote}
\begin{verbatim}
   #!/opt/RICHPse/bin/se
   #include <stdio.se>
   #include <kstat.se>
   main ()
   {
     ks_system_misc kstat$misc;
     printf("There are %d processes on the system \n",kstat$misc.nproc);
   }
\end{verbatim}
\end{quote}

{\bf{Programming Notes}}

\begin{enumerate}

\item The initial line flags that this script should be processed using
the SymbEL interpreter {\tt{se}}.

\item The {\tt{stdio.se}} and {\tt{kstat.se}} header files (located in 
{\tt{/opt/RICHPse/include}}) are included to provide definitions for the
{\tt{printf}} function and for the kstat variable types.

\item The statement {\tt{ks\_system\_misc kstat\$misc;}} declares the
variable {\tt{kstat\$misc}} to be of type {\tt{ks\_system\_misc}} (a structure
type defined in the header file {\tt{kstat.se}}). The form of the variable
name indicates that it is an {\em{active}} variable of the {\tt{kstat\$}}
language class.

\item The print statement prints out the number of processes on the system
simply by referencing the appropriate structure member. This automatically
causes the structure to be updated with information from the kernel. 

\end{enumerate}

For a more detailed introduction to the SymbEL language, see the user's
manual:

\begin{verbatim}
   /opt/RICHPse/doc/USERS.MANUAL
\end{verbatim}

or access it through the HELP button on the SysTune GUI.
 
\section{\label{gui}\xlabel{gui}The GUI extension}

The SymbEL language has an {\tt{attach}} statement which allows it to be
extended by attaching shared libraries, such as system libraries like
libc.so. User supplied libraries can also be attached and a GUI extensions
library is included with the SymbEL release. This is basically a widget
library that allows simple GUI interfaces to be constructed. Although
fairly easy to use, the library is at present fairly limited and has some
bugs.

For more information on the GUI extension, see the user's manual

\begin{verbatim}
   /opt/RICHPse/doc/GUI_MANUAL
\end{verbatim}

or access it through the HELP button of the SysTune GUI. Alternatively,
have a look at the SysTune code which is written using the GUI extension
library.

\section{\label{more_info}\xlabel{more_info}Further Information}

For a general overview of tuning issues on SPARC/Solaris systems, see
\xref{{\em{Configuring, monitoring and tuning SPARC/Solaris 
systems}, SSN/26}}{ssn26}{}.

Both this brief note and the above SSN refer to the more detailed treatment
of tuning issues to be found in {\em{Sun Performance and Tuning -- SPARC and
Solaris}} by Adrian Cockcroft (ISBN 0-13-149642-5).

Further tuning information is available on-line at various URLs. Since
this information is being updated all the time and the links may be
volatile, the URLs are not listed explicitly here. Instead, they can be
accessed through \htmladdnormallinkfoot{the Starlink System Tuning and
Configuration Page}{http://www.starlink.ac.uk/\~{}cac/sunspot/sun\_perf.html}

%===========================================================================

% \newpage
\appendix
\section{\label{changes}\xlabel{changes}Changes since V0.3}

This is a maintenance release for Solaris 2.5 platforms. 

\begin{enumerate}
\item Systune V0.4 includes the SE2.5 package which works correctly on
Solaris 2.5 platforms (and earlier).
\item The Systune GUI includes links to 2 contributed tools ({\tt{dfstats}}
and {\tt{kview}}) and to the
percollator monitor ({\tt{percollator}}).
\item Help information and this document have changed accordingly.
\end{enumerate}

\section{\label{future_releases}\xlabel{future_releases}Future Releases}

The next release should be a major revision to V1.0 and will provide full
functionality for the SysTune GUI. Graphing tools for analyzing performance
logs should also become available.

\end{document}
